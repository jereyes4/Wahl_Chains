\documentclass{article}
\usepackage[utf8]{inputenc}
\usepackage{longtable}
\usepackage{verbatim}
\usepackage{amsmath}
\usepackage{amssymb}
\usepackage{hyperref}
\usepackage{geometry}

\usepackage{listings}
\usepackage{xcolor}

\geometry{left=7em, bottom=7em, right=7em, tmargin=7em, headheight=7em}
%New colors defined below
\definecolor{codegreen}{rgb}{0,0.6,0}
\definecolor{codegray}{rgb}{0.5,0.5,0.5}
\definecolor{codepurple}{rgb}{0.58,0,0.82}
\definecolor{backcolour}{rgb}{0.95,0.95,0.92}

%Code listing style named "mystyle"

\lstdefinelanguage{config}{
  morekeywords={Try,Fix,Ign,Dis,Di1,Di2,Di3,Di4,Di5,Di6,Di7,Di8,Di9,global,print},
  morecomment=[l]{\#},
}
\lstdefinestyle{mystyle}{
  backgroundcolor=\color{backcolour},
  commentstyle=\color{codegreen},
  keywordstyle=\color{magenta},
  numberstyle=\tiny\color{codegray},
  stringstyle=\color{codegray},
  basicstyle=\ttfamily\footnotesize,
  breakatwhitespace=false,         
  breaklines=true,                 
  captionpos=b,                    
  keepspaces=true,                 
  numbers=left,                    
  numbersep=5pt,                  
  showspaces=false,                
  showstringspaces=false,
  showtabs=false,                  
  tabsize=2,
  inputencoding=utf8,
  extendedchars=true,
  literate={á}{{\'a}}1 {é}{{\'e}}1 {í}{{\'i}}1 {ó}{{\'o}}1 {ú}{{\'u}}1
}
\lstset{style=mystyle}


\setlength{\parindent}{0pt}
\title{Index of single and double Wahl singularities by blowing up extremal elliptic surfaces}
\author{\ }
\date{\today}

\newcommand{\C}{\mathbb{C}}

\begin{document}

\maketitle

\tableofcontents

%%%%%%%%%%%%%%%%%%%%%%%%%%%%%%%%%%%%%%%%%%%%%
\section{$I_9 + 3I_1$}
Fibration given by the pencil
\[F_{\lambda} = y^3 - zx^2 + z^2x + 3\lambda xyz.\]
The nine exceptionals are as follows:
\begin{itemize}
  \item $E_1$ - $E_4$ at $[0,0,1]$.
  \item $E_5$ - $E_8$ at $[1,0,0]$.
  \item $E_9$ at $[1,0,1]$.
\end{itemize}
Let $w$ be a primite third root of unity, then singular fibers are as follows:
\begin{itemize}
  \item $\lambda = \infty$: $I_9$ fiber given by $y$, $E_1$, $E_2$, $E_3$, $x$, $z$, $E_7$, $E_6$ $E_5$ in order.
  \item $\lambda = -1$: $I_1$ fiber called $F_3$ with node at $[-1,-1,1]$.
  \item $\lambda = -w$: $I_1$ fiber called $F_1$ with node at $[-1,-w^2,1]$.
  \item $\lambda = -w^2$: $I_1$ fiber called $F_2$ with node at $[-1,-w,1]$.
\end{itemize}
Extra curves:
\begin{itemize}
  \item $H = x-z$, a triple section that passes through all nodes of the $I_1$'s and through the intersection of $x$ and $z$.
  \item $K = x+z$, double section through $x \cap z$ and $[1,0,1]$.
  \item $T_i = y - w^{2i}x$, $i=1,2,3$, double section through $[-1,-w^{2i},1]$ and $[0,0,1]$.
  \item $S_i = y + w^{2i}z$, $i=1,2,3$, double section through $[-1,-w^{2i},1]$ and $[1,0,0]$.
  \item $R_i = 2y + w^{2i}(z-x)$, $i=1,2,3$, double section through $[-1,-w^{2i},1]$ and $[1,0,1]$.
\end{itemize}
Input:
\lstinputlisting[language=config]{../Tests/9111_alt.txt}
%\lstinputlisting[language=config]{../Tests/9111.txt}
Result:
%%\input{summary/Exp_9111_new}
%\usepackage{longtable}
\subsection{1 chain, $K^2 = 1$}
\begin{longtable}{|c|c|c|c|c|c|}
\hline
\multicolumn{6}{|c|}{1 chain, $K^2 = 1$}\\
\hline
$(n,a)$ & Length & Nef & $\mathbb Q$-ef & Obstruction 0 & Index\\
\hline
\endfirsthead

\hline
$(n,a)$ & Length & Nef & $\mathbb Q$-ef & Obstruction 0 & Index\\
\hline
\endhead
\hline
\endfoot

$(13, 8)$ & 5 & YES & YES & YES & 1\\
$(13, 9)$ & 6 & YES & YES & YES & 2\\
$(14, 5)$ & 6 & YES & YES & YES & 3\\
$(16, 5)$ & 7 & YES & YES & YES & 4\\
$(16, 9)$ & 6 & YES & YES & YES & 5\\
$(17, 7)$ & 6 & YES & YES & YES & 6\\
$(19, 5)$ & 7 & YES & YES & YES & 7\\
$(19, 8)$ & 6 & YES & YES & YES & 8\\
$(21, 5)$ & 8 & YES & YES & YES & 9\\
$(24, 5)$ & 8 & YES & YES & YES & 10\\
$(26, 7)$ & 7 & YES & YES & YES & 11\\
$(30, 7)$ & 8 & YES & YES & YES & 12\\
$(a; 1, 0, 0; 13)$ & 5 & YES & YES & YES & 13\\
$(j; 0, 0, 0; 8)$ & 5 & YES & YES & YES & 14\\
$(j; 0, 1, 0; 10)$ & 6 & YES & YES & YES & 15
\end{longtable}
\subsection{1 chain, $K^2 = 2$}
\begin{longtable}{|c|c|c|c|c|c|}
\hline
\multicolumn{6}{|c|}{1 chain, $K^2 = 2$}\\
\hline
$(n,a)$ & Length & Nef & $\mathbb Q$-ef & Obstruction 0 & Index\\
\hline
\endfirsthead

\hline
$(n,a)$ & Length & Nef & $\mathbb Q$-ef & Obstruction 0 & Index\\
\hline
\endhead
\hline
\endfoot

$(27, 22)$ & 8 & YES & YES & YES & 16\\
$(28, 23)$ & 8 & YES & YES & YES & 17\\
$(32, 25)$ & 8 & YES & YES & YES & 18\\
$(34, 25)$ & 8 & YES & YES & YES & 19\\
$(36, 25)$ & 8 & YES & YES & YES & 20\\
$(39, 25)$ & 8 & YES & YES & YES & 21\\
$(40, 23)$ & 9 & YES & YES & YES & 22\\
$(41, 22)$ & 10 & YES & YES & YES & 23\\
$(41, 26)$ & 8 & YES & YES & YES & 24\\
$(42, 19)$ & 9 & YES & YES & YES & 25\\
$(44, 27)$ & 8 & YES & YES & YES & 26\\
$(48, 17)$ & 9 & YES & YES & YES & 27\\
$(49, 18)$ & 8 & YES & YES & YES & 28\\
$(49, 20)$ & 9 & YES & YES & YES & 29\\
$(51, 20)$ & 9 & YES & YES & YES & 30\\
$(53, 19)$ & 9 & YES & YES & YES & 31\\
$(64, 17)$ & 10 & YES & YES & YES & 32\\
$(64, 23)$ & 9 & YES & YES & YES & 33\\
$(72, 13)$ & 12 & YES & YES & YES & 34\\
$(79, 14)$ & 11 & YES & YES & YES & 35\\
$(89, 17)$ & 12 & YES & YES & YES & 36\\
$(a; 3, 0, 1; 31)$ & 8 & YES & YES & YES & 37\\
$(c; 0, 3, 1; 23)$ & 8 & YES & YES & YES & 38\\
$(c; 0, 3, 2; 29)$ & 9 & YES & YES & YES & 39\\
$(d; 0, 0, 3; 22)$ & 8 & YES & YES & YES & 40\\
$(d; 0, 2, 2; 13)$ & 9 & YES & YES & YES & 41\\
$(i; 0, 3, 0; 18)$ & 8 & YES & YES & YES & 42
\end{longtable}
\subsection{1 chain, $K^2 = 3$}
\begin{longtable}{|c|c|c|c|c|c|}
\hline
\multicolumn{6}{|c|}{1 chain, $K^2 = 3$}\\
\hline
$(n,a)$ & Length & Nef & $\mathbb Q$-ef & Obstruction 0 & Index\\
\hline
\endfirsthead

\hline
$(n,a)$ & Length & Nef & $\mathbb Q$-ef & Obstruction 0 & Index\\
\hline
\endhead
\hline
\endfoot

$(71, 50)$ & 9 & YES & YES & NO(2) & 43\\
$(73, 62)$ & 11 & YES & YES & YES & 44\\
$(83, 59)$ & 11 & YES & YES & YES & 45\\
$(84, 59)$ & 10 & YES & YES & YES & 46\\
$(85, 61)$ & 11 & YES & YES & YES & 47\\
$(91, 66)$ & 10 & YES & YES & YES & 48\\
$(95, 43)$ & 11 & YES & YES & YES & 49\\
$(97, 56)$ & 10 & YES & YES & NO(2) & 50\\
$(101, 37)$ & 10 & YES & YES & NO(2) & 51\\
$(104, 45)$ & 11 & YES & YES & YES & 52\\
$(111, 32)$ & 13 & YES & YES & YES & 53\\
$(113, 32)$ & 13 & YES & YES & YES & 54\\
$(113, 48)$ & 11 & YES & YES & YES & 55\\
$(115, 52)$ & 11 & YES & YES & YES & 56\\
$(119, 74)$ & 11 & YES & YES & YES & 57\\
$(124, 35)$ & 12 & YES & YES & YES & 58\\
$(125, 46)$ & 12 & YES & YES & YES & 59\\
$(127, 48)$ & 11 & YES & YES & YES & 60\\
$(129, 59)$ & 12 & YES & YES & YES & 61\\
$(132, 47)$ & 12 & YES & YES & YES & 62\\
$(138, 49)$ & 12 & YES & YES & YES & 63\\
$(140, 53)$ & 11 & YES & YES & YES & 64\\
$(144, 61)$ & 11 & YES & YES & YES & 65\\
$(145, 42)$ & 12 & YES & YES & YES & 66\\
$(148, 65)$ & 11 & YES & YES & YES & 67\\
$(151, 53)$ & 12 & YES & YES & YES & 68\\
$(154, 57)$ & 12 & YES & YES & YES & 69\\
$(161, 51)$ & 13 & YES & YES & YES & 70\\
$(163, 43)$ & 12 & YES & YES & YES & 71\\
$(175, 62)$ & 12 & YES & YES & YES & 72\\
$(177, 47)$ & 12 & YES & YES & YES & 73\\
$(178, 47)$ & 12 & YES & YES & YES & 74\\
$(181, 65)$ & 12 & YES & YES & YES & 75\\
$(187, 42)$ & 13 & YES & YES & YES & 76\\
$(188, 59)$ & 13 & YES & YES & YES & 77\\
$(196, 45)$ & 13 & YES & YES & YES & 78\\
$(197, 61)$ & 13 & YES & YES & YES & 79\\
$(239, 32)$ & 17 & YES & YES & YES & 80\\
$(251, 46)$ & 15 & YES & YES & YES & 81\\
$(257, 40)$ & 15 & YES & YES & YES & 82\\
$(265, 41)$ & 16 & YES & YES & YES & 83\\
$(b; 0, 4, 1; 48)$ & 10 & YES & YES & YES & 84\\
$(g; 0, 4, 0; 39)$ & 10 & YES & YES & YES & 85
\end{longtable}
\subsection{1 chain, $K^2 = 4$}
\begin{longtable}{|c|c|c|c|c|c|}
\hline
\multicolumn{6}{|c|}{1 chain, $K^2 = 4$}\\
\hline
$(n,a)$ & Length & Nef & $\mathbb Q$-ef & Obstruction 0 & Index\\
\hline
\endfirsthead

\hline
$(n,a)$ & Length & Nef & $\mathbb Q$-ef & Obstruction 0 & Index\\
\hline
\endhead
\hline
\endfoot

$(251, 74)$ & 13 & YES & YES & YES & 86\\
$(289, 101)$ & 15 & YES & YES & YES & 87\\
$(294, 85)$ & 14 & YES & YES & YES & 88\\
$(311, 132)$ & 14 & YES & YES & YES & 89\\
$(336, 137)$ & 14 & YES & YES & YES & 90\\
$(337, 138)$ & 14 & YES & YES & YES & 91\\
$(392, 53)$ & 20 & YES & YES & YES & 92\\
$(404, 107)$ & 15 & YES & YES & YES & 93\\
$(563, 91)$ & 18 & YES & YES & YES & 94\\
$(870, 269)$ & 16 & YES & YES & YES & 95\\
$(945, 388)$ & 15 & YES & YES & YES & 96
\end{longtable}
\subsection{2 chains, $K^2 = 1$}
\begin{longtable}{|c|c|c|c|c|c|c|c|c|c|}
\hline
\multicolumn{10}{|c|}{2 chains, $K^2 = 1$}\\
\hline
$(n,a)$ & Length & $(n,a)$ & Length & GCD & Nef & $\mathbb Q$-ef & Obstruction 0 & WH & Index\\
\hline
\endfirsthead

\hline
$(n,a)$ & Length & $(n,a)$ & Length & GCD & Nef & $\mathbb Q$-ef & Obstruction 0 & WH & Index\\
\hline
\endhead
\hline
\endfoot

$(7, 3)$ & 4 & $(5, 1)$ & 4 & 1 & YES & YES & YES & NO & 97\\
$(7, 3)$ & 4 & $(5, 1)$ & 4 & 1 & YES & YES & YES & NO & 98\\
$(7, 3)$ & 4 & $(7, 2)$ & 4 & 7 & YES & YES & YES & NO & 99\\
$(7, 3)$ & 4 & $(7, 2)$ & 4 & 7 & YES & YES & YES & NO & 100\\
$(7, 3)$ & 4 & $(7, 2)$ & 4 & 7 & YES & YES & YES & NO & 101\\
$(7, 3)$ & 4 & $(7, 3)$ & 4 & 7 & YES & YES & YES & NO & 102\\
$(8, 3)$ & 4 & $(7, 3)$ & 4 & 1 & YES & YES & YES & NO & 103\\
$(8, 3)$ & 4 & $(7, 3)$ & 4 & 1 & YES & YES & YES & NO & 104\\
$(8, 3)$ & 4 & $(7, 3)$ & 4 & 1 & YES & YES & YES & NO & 105\\
$(9, 2)$ & 5 & $(4, 1)$ & 3 & 1 & YES & YES & YES & NO & 106\\
$(9, 2)$ & 5 & $(4, 1)$ & 3 & 1 & YES & YES & YES & NO & 107\\
$(9, 4)$ & 5 & $(4, 1)$ & 3 & 1 & YES & YES & YES & NO & 108\\
$(9, 4)$ & 5 & $(4, 1)$ & 3 & 1 & YES & YES & YES & NO & 109\\
$(9, 2)$ & 5 & $(5, 1)$ & 4 & 1 & YES & YES & YES & NO & 110\\
$(9, 2)$ & 5 & $(5, 1)$ & 4 & 1 & YES & YES & YES & NO & 111\\
$(9, 2)$ & 5 & $(5, 1)$ & 4 & 1 & YES & YES & YES & NO & 112\\
$(9, 4)$ & 5 & $(5, 2)$ & 3 & 1 & YES & YES & YES & NO & 113\\
$(9, 2)$ & 5 & $(7, 3)$ & 4 & 1 & YES & YES & YES & NO & 114\\
$(9, 2)$ & 5 & $(7, 3)$ & 4 & 1 & YES & YES & YES & NO & 115\\
$(9, 4)$ & 5 & $(7, 2)$ & 4 & 1 & YES & YES & YES & NO & 116\\
$(9, 4)$ & 5 & $(8, 3)$ & 4 & 1 & YES & YES & YES & 185 & 117\\
$(10, 3)$ & 5 & $(4, 1)$ & 3 & 2 & YES & YES & YES & 128 & 118\\
$(10, 3)$ & 5 & $(4, 1)$ & 3 & 2 & YES & YES & YES & NO & 119\\
$(10, 3)$ & 5 & $(5, 1)$ & 4 & 5 & YES & YES & YES & NO & 120\\
$(10, 3)$ & 5 & $(5, 1)$ & 4 & 5 & YES & YES & YES & NO & 121\\
$(10, 3)$ & 5 & $(5, 2)$ & 3 & 5 & YES & YES & YES & NO & 122\\
$(11, 2)$ & 6 & $(2, 1)$ & 1 & 1 & YES & YES & YES & NO & 123\\
$(11, 3)$ & 5 & $(2, 1)$ & 1 & 1 & YES & YES & YES & NO & 124\\
$(11, 3)$ & 5 & $(2, 1)$ & 1 & 1 & YES & YES & YES & NO & 125\\
$(11, 4)$ & 5 & $(2, 1)$ & 1 & 1 & YES & YES & YES & NO & 126\\
$(11, 5)$ & 6 & $(2, 1)$ & 1 & 1 & YES & YES & YES & NO & 127\\
$(11, 3)$ & 5 & $(3, 1)$ & 2 & 1 & YES & YES & YES & 118 & 128\\
$(11, 3)$ & 5 & $(3, 1)$ & 2 & 1 & YES & YES & YES & NO & 129\\
$(11, 4)$ & 5 & $(3, 1)$ & 2 & 1 & YES & YES & YES & NO & 130\\
$(11, 4)$ & 5 & $(3, 1)$ & 2 & 1 & YES & YES & YES & NO & 131\\
$(11, 4)$ & 5 & $(3, 1)$ & 2 & 1 & YES & YES & YES & NO & 132\\
$(11, 5)$ & 6 & $(3, 1)$ & 2 & 1 & YES & YES & YES & NO & 133\\
$(11, 5)$ & 6 & $(3, 1)$ & 2 & 1 & YES & YES & YES & NO & 134\\
$(11, 5)$ & 6 & $(3, 1)$ & 2 & 1 & YES & YES & YES & NO & 135\\
$(11, 3)$ & 5 & $(4, 1)$ & 3 & 1 & YES & YES & YES & NO & 136\\
$(11, 3)$ & 5 & $(4, 1)$ & 3 & 1 & YES & YES & YES & NO & 137\\
$(11, 3)$ & 5 & $(4, 1)$ & 3 & 1 & YES & YES & YES & NO & 138\\
$(11, 4)$ & 5 & $(4, 1)$ & 3 & 1 & YES & YES & YES & NO & 139\\
$(11, 4)$ & 5 & $(4, 1)$ & 3 & 1 & YES & YES & YES & NO & 140\\
$(11, 5)$ & 6 & $(4, 1)$ & 3 & 1 & YES & YES & YES & NO & 141\\
$(11, 5)$ & 6 & $(4, 1)$ & 3 & 1 & YES & YES & YES & NO & 142\\
$(11, 5)$ & 6 & $(4, 1)$ & 3 & 1 & YES & YES & YES & NO & 143\\
$(11, 2)$ & 6 & $(5, 1)$ & 4 & 1 & YES & YES & YES & NO & 144\\
$(11, 2)$ & 6 & $(5, 1)$ & 4 & 1 & YES & YES & YES & NO & 145\\
$(11, 2)$ & 6 & $(5, 1)$ & 4 & 1 & YES & YES & YES & NO & 146\\
$(11, 3)$ & 5 & $(5, 1)$ & 4 & 1 & YES & YES & YES & NO & 147\\
$(11, 3)$ & 5 & $(5, 1)$ & 4 & 1 & YES & YES & YES & NO & 148\\
$(11, 3)$ & 5 & $(5, 2)$ & 3 & 1 & YES & YES & YES & NO & 149\\
$(11, 3)$ & 5 & $(5, 2)$ & 3 & 1 & YES & YES & YES & NO & 150\\
$(11, 4)$ & 5 & $(5, 2)$ & 3 & 1 & YES & YES & YES & 174 & 151\\
$(11, 4)$ & 5 & $(5, 2)$ & 3 & 1 & YES & YES & YES & NO & 152\\
$(11, 5)$ & 6 & $(5, 1)$ & 4 & 1 & YES & YES & YES & NO & 153\\
$(11, 5)$ & 6 & $(5, 1)$ & 4 & 1 & YES & YES & YES & NO & 154\\
$(11, 5)$ & 6 & $(5, 2)$ & 3 & 1 & YES & YES & YES & NO & 155\\
$(11, 5)$ & 6 & $(5, 2)$ & 3 & 1 & YES & YES & YES & NO & 156\\
$(11, 5)$ & 6 & $(6, 1)$ & 5 & 1 & YES & YES & YES & NO & 157\\
$(11, 5)$ & 6 & $(6, 1)$ & 5 & 1 & YES & YES & YES & NO & 158\\
$(11, 5)$ & 6 & $(7, 3)$ & 4 & 1 & YES & YES & YES & 182 & 159\\
$(11, 4)$ & 5 & $(8, 3)$ & 4 & 1 & YES & YES & YES & NO & 160\\
$(11, 2)$ & 6 & $(9, 4)$ & 5 & 1 & YES & YES & YES & NO & 161\\
$(11, 5)$ & 6 & $(9, 4)$ & 5 & 1 & YES & YES & YES & NO & 162\\
$(11, 4)$ & 5 & $(11, 4)$ & 5 & 11 & YES & YES & YES & NO & 163\\
$(11, 5)$ & 6 & $(11, 5)$ & 6 & 11 & YES & YES & YES & NO & 164\\
$(12, 5)$ & 5 & $(3, 1)$ & 2 & 3 & YES & YES & YES & NO & 165\\
$(12, 5)$ & 5 & $(3, 1)$ & 2 & 3 & YES & YES & YES & NO & 166\\
$(12, 5)$ & 5 & $(3, 1)$ & 2 & 3 & YES & YES & YES & NO & 167\\
$(13, 3)$ & 6 & $(2, 1)$ & 1 & 1 & YES & YES & YES & NO & 168\\
$(13, 5)$ & 5 & $(2, 1)$ & 1 & 1 & YES & YES & YES & NO & 169\\
$(13, 4)$ & 6 & $(3, 1)$ & 2 & 1 & YES & YES & YES & NO & 170\\
$(13, 4)$ & 6 & $(3, 1)$ & 2 & 1 & YES & YES & YES & NO & 171\\
$(13, 5)$ & 5 & $(3, 1)$ & 2 & 1 & YES & YES & YES & NO & 172\\
$(13, 5)$ & 5 & $(3, 1)$ & 2 & 1 & YES & YES & YES & NO & 173\\
$(13, 5)$ & 5 & $(3, 1)$ & 2 & 1 & YES & YES & YES & 151 & 174\\
$(13, 3)$ & 6 & $(4, 1)$ & 3 & 1 & YES & YES & YES & NO & 175\\
$(13, 3)$ & 6 & $(4, 1)$ & 3 & 1 & YES & YES & YES & NO & 176\\
$(13, 3)$ & 6 & $(11, 3)$ & 5 & 1 & YES & YES & YES & NO & 177\\
$(14, 5)$ & 6 & $(3, 1)$ & 2 & 1 & NO & YES & YES & NO & 178\\
$(14, 5)$ & 6 & $(3, 1)$ & 2 & 1 & YES & YES & YES & NO & 179\\
$(14, 3)$ & 6 & $(5, 1)$ & 4 & 1 & NO & YES & YES & NO & 180\\
$(15, 4)$ & 6 & $(4, 1)$ & 3 & 1 & NO & YES & YES & NO & 181\\
$(16, 7)$ & 6 & $(2, 1)$ & 1 & 2 & YES & YES & YES & 159 & 182\\
$(16, 5)$ & 7 & $(3, 1)$ & 2 & 1 & YES & YES & YES & NO & 183\\
$(16, 5)$ & 7 & $(3, 1)$ & 2 & 1 & NO & YES & YES & NO & 184\\
$(16, 7)$ & 6 & $(3, 1)$ & 2 & 1 & YES & YES & YES & 117 & 185\\
$(16, 3)$ & 7 & $(5, 1)$ & 4 & 1 & NO & YES & YES & NO & 186\\
$(16, 3)$ & 7 & $(5, 1)$ & 4 & 1 & NO & YES & YES & NO & 187\\
$(16, 7)$ & 6 & $(5, 1)$ & 4 & 1 & YES & YES & YES & NO & 188\\
$(16, 7)$ & 6 & $(5, 1)$ & 4 & 1 & YES & YES & YES & NO & 189\\
$(16, 7)$ & 6 & $(5, 1)$ & 4 & 1 & YES & YES & YES & NO & 190\\
$(16, 5)$ & 7 & $(7, 1)$ & 6 & 1 & YES & YES & YES & NO & 191\\
$(16, 7)$ & 6 & $(7, 3)$ & 4 & 1 & YES & YES & YES & NO & 192\\
$(16, 7)$ & 6 & $(9, 4)$ & 5 & 1 & YES & YES & YES & NO & 193\\
$(16, 5)$ & 7 & $(13, 4)$ & 6 & 1 & YES & YES & YES & NO & 194\\
$(17, 7)$ & 6 & $(2, 1)$ & 1 & 1 & YES & YES & YES & NO & 195\\
$(17, 5)$ & 6 & $(3, 1)$ & 2 & 1 & NO & YES & YES & NO & 196\\
$(17, 4)$ & 7 & $(4, 1)$ & 3 & 1 & NO & YES & YES & NO & 197\\
$(17, 4)$ & 7 & $(4, 1)$ & 3 & 1 & NO & YES & YES & NO & 198\\
$(19, 4)$ & 7 & $(2, 1)$ & 1 & 1 & YES & YES & YES & NO & 199\\
$(19, 4)$ & 7 & $(2, 1)$ & 1 & 1 & YES & YES & YES & NO & 200\\
$(19, 8)$ & 6 & $(2, 1)$ & 1 & 1 & YES & YES & YES & NO & 201\\
$(19, 8)$ & 6 & $(2, 1)$ & 1 & 1 & NO & YES & YES & NO & 202\\
$(19, 5)$ & 7 & $(4, 1)$ & 3 & 1 & YES & YES & YES & NO & 203\\
$(19, 5)$ & 7 & $(7, 1)$ & 6 & 1 & YES & YES & YES & NO & 204\\
$(19, 4)$ & 7 & $(11, 2)$ & 6 & 1 & YES & YES & YES & NO & 205\\
$(19, 5)$ & 7 & $(11, 3)$ & 5 & 1 & YES & YES & YES & 215 & 206\\
$(20, 9)$ & 7 & $(2, 1)$ & 1 & 2 & NO & YES & YES & NO & 207\\
$(21, 8)$ & 6 & $(2, 1)$ & 1 & 1 & NO & YES & YES & NO & 208\\
$(21, 5)$ & 8 & $(4, 1)$ & 3 & 1 & YES & YES & YES & NO & 209\\
$(23, 10)$ & 7 & $(2, 1)$ & 1 & 1 & NO & YES & YES & NO & 210\\
$(24, 5)$ & 8 & $(5, 1)$ & 4 & 1 & YES & YES & YES & NO & 211\\
$(24, 5)$ & 8 & $(7, 1)$ & 6 & 1 & YES & YES & YES & NO & 212\\
$(24, 5)$ & 8 & $(19, 4)$ & 7 & 1 & YES & YES & YES & NO & 213\\
$(25, 9)$ & 7 & $(2, 1)$ & 1 & 1 & NO & YES & YES & NO & 214\\
$(26, 7)$ & 7 & $(4, 1)$ & 3 & 2 & YES & YES & YES & 206 & 215\\
$(a; 1, 0, 0; 13)$ & 5 & $(2, 1)$ & 1 & 1 & YES & YES & YES & NO & 216\\
$(a; 2, 0, 0; 17)$ & 6 & $(5, 1)$ & 4 & 1 & YES & YES & YES & NO & 217\\
$(c; 0, 0, 0; 4)$ & 4 & $(3, 1)$ & 2 & 1 & YES & YES & NO(3) & NO & 218\\
$(c; 0, 1, 0; 11)$ & 5 & $(2, 1)$ & 1 & 1 & YES & YES & YES & NO & 219\\
$(c; 0, 1, 1; 5)$ & 6 & $(2, 1)$ & 1 & 1 & YES & YES & YES & NO & 220\\
$(c; 0, 2, 0; 7)$ & 6 & $(2, 1)$ & 1 & 1 & YES & YES & YES & NO & 221\\
$(f; 0, 0, 0; 6)$ & 4 & $(4, 1)$ & 3 & 2 & YES & YES & YES & NO & 222\\
$(f; 0, 0, 0; 6)$ & 4 & $(5, 2)$ & 3 & 1 & YES & YES & YES & NO & 223\\
$(f; 0, 0, 0; 6)$ & 4 & $(7, 3)$ & 4 & 1 & YES & YES & YES & NO & 224\\
$(f; 0, 0, 0; 6)$ & 4 & $(9, 2)$ & 5 & 3 & YES & YES & YES & NO & 225\\
$(f; 0, 1, 0; 7)$ & 5 & $(2, 1)$ & 1 & 1 & YES & YES & YES & NO & 226\\
$(f; 0, 1, 0; 7)$ & 5 & $(3, 1)$ & 2 & 1 & YES & YES & YES & NO & 227\\
$(f; 0, 1, 0; 7)$ & 5 & $(4, 1)$ & 3 & 1 & YES & YES & YES & NO & 228\\
$(f; 0, 1, 0; 7)$ & 5 & $(5, 1)$ & 4 & 1 & YES & YES & YES & NO & 229\\
$(j; 0, 0, 0; 8)$ & 5 & $(3, 1)$ & 2 & 1 & YES & YES & YES & NO & 230\\
$(j; 0, 0, 0; 8)$ & 5 & $(5, 1)$ & 4 & 1 & YES & YES & YES & NO & 231
\end{longtable}
\subsection{2 chains, $K^2 = 2$}
\begin{longtable}{|c|c|c|c|c|c|c|c|c|c|}
\hline
\multicolumn{10}{|c|}{2 chains, $K^2 = 2$}\\
\hline
$(n,a)$ & Length & $(n,a)$ & Length & GCD & Nef & $\mathbb Q$-ef & Obstruction 0 & WH & Index\\
\hline
\endfirsthead

\hline
$(n,a)$ & Length & $(n,a)$ & Length & GCD & Nef & $\mathbb Q$-ef & Obstruction 0 & WH & Index\\
\hline
\endhead
\hline
\endfoot

$(10, 3)$ & 5 & $(8, 3)$ & 4 & 2 & YES & YES & NO(2) & NO & 232\\
$(11, 2)$ & 6 & $(8, 3)$ & 4 & 1 & YES & YES & YES & NO & 233\\
$(11, 2)$ & 6 & $(8, 3)$ & 4 & 1 & YES & YES & YES & NO & 234\\
$(11, 4)$ & 5 & $(9, 2)$ & 5 & 1 & YES & YES & YES & NO & 235\\
$(11, 2)$ & 6 & $(10, 3)$ & 5 & 1 & YES & YES & YES & NO & 236\\
$(11, 2)$ & 6 & $(10, 3)$ & 5 & 1 & YES & YES & YES & NO & 237\\
$(11, 4)$ & 5 & $(10, 3)$ & 5 & 1 & YES & YES & YES & NO & 238\\
$(11, 5)$ & 6 & $(10, 3)$ & 5 & 1 & YES & YES & YES & NO & 239\\
$(13, 4)$ & 6 & $(8, 3)$ & 4 & 1 & YES & YES & YES & NO & 240\\
$(13, 3)$ & 6 & $(9, 4)$ & 5 & 1 & YES & YES & YES & NO & 241\\
$(13, 3)$ & 6 & $(9, 4)$ & 5 & 1 & YES & YES & YES & NO & 242\\
$(13, 3)$ & 6 & $(9, 4)$ & 5 & 1 & YES & YES & YES & NO & 243\\
$(13, 4)$ & 6 & $(9, 2)$ & 5 & 1 & YES & YES & YES & NO & 244\\
$(13, 4)$ & 6 & $(9, 2)$ & 5 & 1 & YES & YES & YES & NO & 245\\
$(13, 5)$ & 5 & $(10, 3)$ & 5 & 1 & YES & YES & YES & NO & 246\\
$(13, 3)$ & 6 & $(11, 4)$ & 5 & 1 & YES & YES & YES & NO & 247\\
$(13, 3)$ & 6 & $(11, 4)$ & 5 & 1 & YES & YES & YES & NO & 248\\
$(13, 3)$ & 6 & $(11, 4)$ & 5 & 1 & YES & YES & YES & 396 & 249\\
$(13, 4)$ & 6 & $(11, 5)$ & 6 & 1 & YES & YES & YES & NO & 250\\
$(13, 4)$ & 6 & $(11, 5)$ & 6 & 1 & YES & YES & YES & NO & 251\\
$(13, 6)$ & 7 & $(11, 2)$ & 6 & 1 & YES & YES & YES & NO & 252\\
$(13, 6)$ & 7 & $(11, 4)$ & 5 & 1 & YES & YES & YES & NO & 253\\
$(13, 6)$ & 7 & $(11, 4)$ & 5 & 1 & YES & YES & YES & NO & 254\\
$(14, 5)$ & 6 & $(7, 2)$ & 4 & 7 & YES & YES & YES & NO & 255\\
$(14, 5)$ & 6 & $(9, 2)$ & 5 & 1 & YES & YES & YES & NO & 256\\
$(14, 5)$ & 6 & $(9, 2)$ & 5 & 1 & YES & YES & YES & NO & 257\\
$(14, 5)$ & 6 & $(9, 2)$ & 5 & 1 & YES & YES & YES & NO & 258\\
$(14, 5)$ & 6 & $(13, 4)$ & 6 & 1 & YES & YES & YES & NO & 259\\
$(14, 5)$ & 6 & $(13, 4)$ & 6 & 1 & YES & YES & YES & NO & 260\\
$(15, 4)$ & 6 & $(9, 2)$ & 5 & 3 & YES & YES & YES & NO & 261\\
$(15, 4)$ & 6 & $(9, 2)$ & 5 & 3 & YES & YES & YES & NO & 262\\
$(15, 7)$ & 8 & $(11, 2)$ & 6 & 1 & YES & YES & YES & NO & 263\\
$(15, 4)$ & 6 & $(13, 6)$ & 7 & 1 & YES & YES & YES & NO & 264\\
$(16, 5)$ & 7 & $(9, 2)$ & 5 & 1 & YES & YES & YES & NO & 265\\
$(16, 5)$ & 7 & $(9, 2)$ & 5 & 1 & YES & YES & YES & NO & 266\\
$(16, 5)$ & 7 & $(9, 2)$ & 5 & 1 & YES & YES & YES & NO & 267\\
$(16, 5)$ & 7 & $(11, 3)$ & 5 & 1 & YES & YES & YES & NO & 268\\
$(16, 5)$ & 7 & $(11, 3)$ & 5 & 1 & YES & YES & YES & NO & 269\\
$(16, 7)$ & 6 & $(11, 4)$ & 5 & 1 & YES & YES & YES & 312 & 270\\
$(16, 7)$ & 6 & $(13, 6)$ & 7 & 1 & YES & YES & YES & 367 & 271\\
$(16, 3)$ & 7 & $(14, 5)$ & 6 & 2 & YES & YES & YES & NO & 272\\
$(16, 3)$ & 7 & $(14, 5)$ & 6 & 2 & YES & YES & YES & NO & 273\\
$(17, 3)$ & 7 & $(3, 1)$ & 2 & 1 & YES & YES & YES & NO & 274\\
$(17, 7)$ & 6 & $(6, 1)$ & 5 & 1 & YES & YES & YES & NO & 275\\
$(17, 7)$ & 6 & $(6, 1)$ & 5 & 1 & YES & YES & YES & NO & 276\\
$(17, 4)$ & 7 & $(7, 3)$ & 4 & 1 & YES & YES & YES & NO & 277\\
$(17, 4)$ & 7 & $(7, 3)$ & 4 & 1 & YES & YES & YES & NO & 278\\
$(17, 4)$ & 7 & $(7, 3)$ & 4 & 1 & YES & YES & YES & NO & 279\\
$(17, 5)$ & 6 & $(7, 3)$ & 4 & 1 & YES & YES & NO(2) & NO & 280\\
$(17, 4)$ & 7 & $(8, 3)$ & 4 & 1 & YES & YES & YES & NO & 281\\
$(17, 4)$ & 7 & $(8, 3)$ & 4 & 1 & YES & YES & YES & NO & 282\\
$(17, 4)$ & 7 & $(8, 3)$ & 4 & 1 & YES & YES & YES & NO & 283\\
$(17, 5)$ & 6 & $(8, 3)$ & 4 & 1 & YES & YES & YES & NO & 284\\
$(17, 3)$ & 7 & $(11, 5)$ & 6 & 1 & YES & YES & YES & NO & 285\\
$(17, 3)$ & 7 & $(11, 5)$ & 6 & 1 & YES & YES & YES & NO & 286\\
$(17, 5)$ & 6 & $(11, 5)$ & 6 & 1 & YES & YES & YES & NO & 287\\
$(17, 7)$ & 6 & $(11, 5)$ & 6 & 1 & YES & YES & YES & NO & 288\\
$(17, 7)$ & 6 & $(11, 5)$ & 6 & 1 & YES & YES & YES & NO & 289\\
$(17, 6)$ & 7 & $(13, 3)$ & 6 & 1 & YES & YES & YES & NO & 290\\
$(17, 7)$ & 6 & $(13, 4)$ & 6 & 1 & YES & YES & YES & NO & 291\\
$(18, 7)$ & 6 & $(5, 1)$ & 4 & 1 & YES & YES & YES & NO & 292\\
$(18, 7)$ & 6 & $(5, 1)$ & 4 & 1 & YES & YES & YES & NO & 293\\
$(18, 7)$ & 6 & $(6, 1)$ & 5 & 6 & YES & YES & YES & NO & 294\\
$(18, 7)$ & 6 & $(6, 1)$ & 5 & 6 & YES & YES & YES & NO & 295\\
$(18, 7)$ & 6 & $(6, 1)$ & 5 & 6 & YES & YES & YES & NO & 296\\
$(18, 5)$ & 6 & $(7, 3)$ & 4 & 1 & YES & YES & NO(2) & NO & 297\\
$(18, 7)$ & 6 & $(9, 4)$ & 5 & 9 & YES & YES & YES & NO & 298\\
$(18, 7)$ & 6 & $(9, 4)$ & 5 & 9 & YES & YES & YES & NO & 299\\
$(18, 5)$ & 6 & $(11, 5)$ & 6 & 1 & YES & YES & YES & NO & 300\\
$(18, 7)$ & 6 & $(13, 6)$ & 7 & 1 & YES & YES & YES & 591 & 301\\
$(19, 3)$ & 8 & $(4, 1)$ & 3 & 1 & YES & YES & YES & NO & 302\\
$(19, 3)$ & 8 & $(4, 1)$ & 3 & 1 & YES & YES & YES & NO & 303\\
$(19, 4)$ & 7 & $(4, 1)$ & 3 & 1 & YES & YES & YES & NO & 304\\
$(19, 4)$ & 7 & $(4, 1)$ & 3 & 1 & YES & YES & YES & NO & 305\\
$(19, 6)$ & 8 & $(5, 1)$ & 4 & 1 & YES & YES & YES & NO & 306\\
$(19, 6)$ & 8 & $(7, 3)$ & 4 & 1 & YES & YES & YES & NO & 307\\
$(19, 6)$ & 8 & $(7, 3)$ & 4 & 1 & YES & YES & YES & NO & 308\\
$(19, 7)$ & 6 & $(7, 3)$ & 4 & 1 & YES & YES & YES & NO & 309\\
$(19, 7)$ & 6 & $(7, 3)$ & 4 & 1 & YES & YES & YES & NO & 310\\
$(19, 6)$ & 8 & $(8, 3)$ & 4 & 1 & YES & YES & YES & NO & 311\\
$(19, 7)$ & 6 & $(9, 4)$ & 5 & 1 & YES & YES & YES & 270 & 312\\
$(19, 8)$ & 6 & $(11, 4)$ & 5 & 1 & YES & YES & YES & NO & 313\\
$(19, 3)$ & 8 & $(13, 6)$ & 7 & 1 & YES & YES & YES & NO & 314\\
$(19, 4)$ & 7 & $(14, 3)$ & 6 & 1 & YES & YES & YES & NO & 315\\
$(19, 6)$ & 8 & $(14, 3)$ & 6 & 1 & YES & YES & YES & NO & 316\\
$(19, 6)$ & 8 & $(18, 5)$ & 6 & 1 & YES & YES & YES & NO & 317\\
$(20, 3)$ & 8 & $(4, 1)$ & 3 & 4 & YES & YES & YES & NO & 318\\
$(20, 3)$ & 8 & $(4, 1)$ & 3 & 4 & YES & YES & YES & NO & 319\\
$(20, 9)$ & 7 & $(4, 1)$ & 3 & 4 & YES & YES & YES & NO & 320\\
$(20, 9)$ & 7 & $(4, 1)$ & 3 & 4 & YES & YES & YES & NO & 321\\
$(20, 9)$ & 7 & $(4, 1)$ & 3 & 4 & YES & YES & YES & NO & 322\\
$(20, 9)$ & 7 & $(5, 2)$ & 3 & 5 & YES & YES & YES & NO & 323\\
$(20, 7)$ & 8 & $(7, 2)$ & 4 & 1 & YES & YES & YES & NO & 324\\
$(20, 9)$ & 7 & $(7, 3)$ & 4 & 1 & YES & YES & YES & NO & 325\\
$(20, 7)$ & 8 & $(9, 2)$ & 5 & 1 & YES & YES & YES & NO & 326\\
$(20, 7)$ & 8 & $(9, 2)$ & 5 & 1 & YES & YES & YES & NO & 327\\
$(20, 9)$ & 7 & $(13, 3)$ & 6 & 1 & YES & YES & YES & NO & 328\\
$(20, 9)$ & 7 & $(13, 6)$ & 7 & 1 & YES & YES & YES & NO & 329\\
$(20, 7)$ & 8 & $(15, 2)$ & 8 & 5 & YES & YES & YES & NO & 330\\
$(20, 9)$ & 7 & $(15, 7)$ & 8 & 5 & YES & YES & YES & 411 & 331\\
$(20, 3)$ & 8 & $(17, 6)$ & 7 & 1 & YES & YES & YES & NO & 332\\
$(20, 9)$ & 7 & $(17, 7)$ & 6 & 1 & YES & YES & YES & 422 & 333\\
$(20, 7)$ & 8 & $(19, 7)$ & 6 & 1 & YES & YES & YES & NO & 334\\
$(20, 3)$ & 8 & $(20, 3)$ & 8 & 20 & YES & YES & YES & NO & 335\\
$(21, 8)$ & 6 & $(2, 1)$ & 1 & 1 & YES & YES & NO(2) & NO & 336\\
$(21, 8)$ & 6 & $(7, 3)$ & 4 & 7 & YES & YES & YES & NO & 337\\
$(21, 8)$ & 6 & $(7, 3)$ & 4 & 7 & YES & YES & YES & NO & 338\\
$(21, 8)$ & 6 & $(9, 4)$ & 5 & 3 & YES & YES & YES & NO & 339\\
$(21, 5)$ & 8 & $(21, 4)$ & 8 & 21 & YES & YES & YES & NO & 340\\
$(22, 9)$ & 7 & $(5, 1)$ & 4 & 1 & YES & YES & YES & NO & 341\\
$(22, 9)$ & 7 & $(5, 1)$ & 4 & 1 & YES & YES & YES & NO & 342\\
$(23, 4)$ & 8 & $(3, 1)$ & 2 & 1 & YES & YES & YES & NO & 343\\
$(23, 4)$ & 8 & $(3, 1)$ & 2 & 1 & YES & YES & YES & NO & 344\\
$(23, 4)$ & 8 & $(3, 1)$ & 2 & 1 & YES & YES & YES & NO & 345\\
$(23, 4)$ & 8 & $(4, 1)$ & 3 & 1 & YES & YES & YES & NO & 346\\
$(23, 4)$ & 8 & $(4, 1)$ & 3 & 1 & YES & YES & YES & NO & 347\\
$(23, 4)$ & 8 & $(4, 1)$ & 3 & 1 & YES & YES & YES & NO & 348\\
$(23, 5)$ & 7 & $(4, 1)$ & 3 & 1 & YES & YES & YES & NO & 349\\
$(23, 5)$ & 7 & $(4, 1)$ & 3 & 1 & YES & YES & YES & NO & 350\\
$(23, 9)$ & 7 & $(4, 1)$ & 3 & 1 & YES & YES & YES & NO & 351\\
$(23, 9)$ & 7 & $(4, 1)$ & 3 & 1 & YES & YES & YES & NO & 352\\
$(23, 6)$ & 8 & $(5, 1)$ & 4 & 1 & YES & YES & YES & NO & 353\\
$(23, 6)$ & 8 & $(5, 1)$ & 4 & 1 & YES & YES & YES & NO & 354\\
$(23, 9)$ & 7 & $(5, 1)$ & 4 & 1 & YES & YES & YES & NO & 355\\
$(23, 9)$ & 7 & $(5, 1)$ & 4 & 1 & YES & YES & YES & NO & 356\\
$(23, 9)$ & 7 & $(5, 1)$ & 4 & 1 & YES & YES & YES & NO & 357\\
$(23, 6)$ & 8 & $(7, 3)$ & 4 & 1 & YES & YES & YES & NO & 358\\
$(23, 7)$ & 7 & $(7, 3)$ & 4 & 1 & YES & YES & NO(2) & NO & 359\\
$(23, 6)$ & 8 & $(8, 3)$ & 4 & 1 & YES & YES & YES & NO & 360\\
$(23, 4)$ & 8 & $(14, 5)$ & 6 & 1 & YES & YES & YES & NO & 361\\
$(23, 6)$ & 8 & $(15, 4)$ & 6 & 1 & YES & YES & YES & 492 & 362\\
$(23, 6)$ & 8 & $(17, 5)$ & 6 & 1 & YES & YES & YES & NO & 363\\
$(23, 4)$ & 8 & $(21, 5)$ & 8 & 1 & YES & YES & YES & NO & 364\\
$(24, 11)$ & 8 & $(4, 1)$ & 3 & 4 & YES & YES & YES & NO & 365\\
$(24, 11)$ & 8 & $(5, 2)$ & 3 & 1 & YES & YES & YES & NO & 366\\
$(24, 11)$ & 8 & $(7, 3)$ & 4 & 1 & YES & YES & YES & 271 & 367\\
$(24, 5)$ & 8 & $(9, 4)$ & 5 & 3 & YES & YES & YES & NO & 368\\
$(24, 5)$ & 8 & $(11, 4)$ & 5 & 1 & YES & YES & YES & NO & 369\\
$(24, 5)$ & 8 & $(21, 5)$ & 8 & 3 & YES & YES & YES & NO & 370\\
$(25, 9)$ & 7 & $(3, 1)$ & 2 & 1 & YES & YES & YES & NO & 371\\
$(25, 11)$ & 7 & $(3, 1)$ & 2 & 1 & YES & YES & YES & NO & 372\\
$(25, 9)$ & 7 & $(4, 1)$ & 3 & 1 & YES & YES & YES & NO & 373\\
$(25, 9)$ & 7 & $(4, 1)$ & 3 & 1 & YES & YES & YES & NO & 374\\
$(25, 9)$ & 7 & $(4, 1)$ & 3 & 1 & YES & YES & YES & NO & 375\\
$(25, 4)$ & 9 & $(5, 1)$ & 4 & 5 & YES & YES & YES & NO & 376\\
$(25, 4)$ & 9 & $(5, 1)$ & 4 & 5 & YES & YES & YES & NO & 377\\
$(25, 4)$ & 9 & $(5, 1)$ & 4 & 5 & YES & YES & YES & NO & 378\\
$(25, 8)$ & 10 & $(5, 1)$ & 4 & 5 & YES & YES & YES & NO & 379\\
$(25, 9)$ & 7 & $(5, 2)$ & 3 & 5 & YES & YES & YES & NO & 380\\
$(25, 6)$ & 9 & $(8, 3)$ & 4 & 1 & YES & YES & YES & NO & 381\\
$(25, 9)$ & 7 & $(13, 3)$ & 6 & 1 & YES & YES & YES & NO & 382\\
$(25, 8)$ & 10 & $(19, 6)$ & 8 & 1 & YES & YES & YES & 544 & 383\\
$(25, 4)$ & 9 & $(24, 5)$ & 8 & 1 & YES & YES & YES & NO & 384\\
$(26, 11)$ & 7 & $(3, 1)$ & 2 & 1 & YES & YES & NO(2) & NO & 385\\
$(26, 11)$ & 7 & $(12, 5)$ & 5 & 2 & YES & YES & NO(2) & 441 & 386\\
$(26, 5)$ & 9 & $(15, 4)$ & 6 & 1 & YES & YES & YES & NO & 387\\
$(26, 7)$ & 7 & $(15, 4)$ & 6 & 1 & YES & YES & YES & NO & 388\\
$(27, 5)$ & 8 & $(2, 1)$ & 1 & 1 & YES & YES & YES & NO & 389\\
$(27, 10)$ & 7 & $(3, 1)$ & 2 & 3 & YES & YES & YES & NO & 390\\
$(27, 10)$ & 7 & $(4, 1)$ & 3 & 1 & YES & YES & YES & NO & 391\\
$(27, 10)$ & 7 & $(4, 1)$ & 3 & 1 & YES & YES & YES & NO & 392\\
$(27, 10)$ & 7 & $(4, 1)$ & 3 & 1 & YES & YES & YES & NO & 393\\
$(27, 11)$ & 8 & $(4, 1)$ & 3 & 1 & YES & YES & YES & NO & 394\\
$(27, 11)$ & 8 & $(4, 1)$ & 3 & 1 & YES & YES & YES & NO & 395\\
$(27, 10)$ & 7 & $(5, 1)$ & 4 & 1 & YES & YES & YES & 249 & 396\\
$(27, 10)$ & 7 & $(5, 1)$ & 4 & 1 & YES & YES & YES & NO & 397\\
$(27, 11)$ & 8 & $(7, 2)$ & 4 & 1 & YES & YES & YES & NO & 398\\
$(27, 11)$ & 8 & $(9, 4)$ & 5 & 9 & YES & YES & YES & NO & 399\\
$(27, 5)$ & 8 & $(11, 2)$ & 6 & 1 & YES & YES & YES & NO & 400\\
$(27, 10)$ & 7 & $(17, 6)$ & 7 & 1 & YES & YES & YES & NO & 401\\
$(28, 13)$ & 9 & $(3, 1)$ & 2 & 1 & YES & YES & YES & NO & 402\\
$(28, 13)$ & 9 & $(3, 1)$ & 2 & 1 & YES & YES & YES & NO & 403\\
$(28, 13)$ & 9 & $(4, 1)$ & 3 & 4 & YES & YES & YES & NO & 404\\
$(28, 5)$ & 8 & $(5, 1)$ & 4 & 1 & YES & YES & YES & NO & 405\\
$(28, 5)$ & 8 & $(5, 1)$ & 4 & 1 & YES & YES & YES & NO & 406\\
$(28, 5)$ & 8 & $(5, 1)$ & 4 & 1 & YES & YES & YES & NO & 407\\
$(28, 11)$ & 8 & $(5, 2)$ & 3 & 1 & YES & YES & YES & NO & 408\\
$(28, 13)$ & 9 & $(5, 2)$ & 3 & 1 & YES & YES & YES & NO & 409\\
$(28, 13)$ & 9 & $(6, 1)$ & 5 & 2 & YES & YES & YES & NO & 410\\
$(28, 13)$ & 9 & $(9, 4)$ & 5 & 1 & YES & YES & YES & 331 & 411\\
$(28, 5)$ & 8 & $(11, 2)$ & 6 & 1 & YES & YES & YES & NO & 412\\
$(28, 5)$ & 8 & $(11, 5)$ & 6 & 1 & YES & YES & YES & NO & 413\\
$(28, 13)$ & 9 & $(11, 5)$ & 6 & 1 & YES & YES & YES & NO & 414\\
$(28, 5)$ & 8 & $(14, 5)$ & 6 & 14 & YES & YES & YES & NO & 415\\
$(29, 4)$ & 10 & $(2, 1)$ & 1 & 1 & YES & YES & YES & NO & 416\\
$(29, 12)$ & 7 & $(3, 1)$ & 2 & 1 & YES & YES & NO(2) & NO & 417\\
$(29, 9)$ & 8 & $(4, 1)$ & 3 & 1 & YES & YES & YES & NO & 418\\
$(29, 9)$ & 8 & $(4, 1)$ & 3 & 1 & YES & YES & YES & NO & 419\\
$(29, 9)$ & 8 & $(4, 1)$ & 3 & 1 & YES & YES & YES & NO & 420\\
$(29, 12)$ & 7 & $(7, 3)$ & 4 & 1 & YES & YES & NO(2) & NO & 421\\
$(29, 12)$ & 7 & $(11, 5)$ & 6 & 1 & YES & YES & YES & 333 & 422\\
$(29, 4)$ & 10 & $(13, 6)$ & 7 & 1 & YES & YES & YES & NO & 423\\
$(29, 13)$ & 8 & $(13, 6)$ & 7 & 1 & YES & YES & YES & 606 & 424\\
$(29, 11)$ & 7 & $(29, 11)$ & 7 & 29 & YES & YES & YES & NO & 425\\
$(30, 11)$ & 7 & $(3, 1)$ & 2 & 3 & YES & YES & YES & NO & 426\\
$(30, 11)$ & 7 & $(3, 1)$ & 2 & 3 & YES & YES & YES & NO & 427\\
$(30, 11)$ & 7 & $(5, 2)$ & 3 & 5 & YES & YES & YES & 490 & 428\\
$(31, 7)$ & 8 & $(2, 1)$ & 1 & 1 & YES & YES & YES & NO & 429\\
$(31, 7)$ & 8 & $(3, 1)$ & 2 & 1 & YES & YES & YES & NO & 430\\
$(31, 13)$ & 7 & $(3, 1)$ & 2 & 1 & YES & YES & YES & NO & 431\\
$(31, 13)$ & 7 & $(3, 1)$ & 2 & 1 & YES & YES & YES & NO & 432\\
$(31, 14)$ & 8 & $(3, 1)$ & 2 & 1 & YES & YES & YES & NO & 433\\
$(31, 14)$ & 8 & $(3, 1)$ & 2 & 1 & YES & YES & YES & NO & 434\\
$(31, 14)$ & 8 & $(3, 1)$ & 2 & 1 & YES & YES & YES & NO & 435\\
$(31, 7)$ & 8 & $(5, 1)$ & 4 & 1 & YES & YES & YES & NO & 436\\
$(31, 7)$ & 8 & $(5, 1)$ & 4 & 1 & YES & YES & YES & NO & 437\\
$(31, 9)$ & 8 & $(5, 2)$ & 3 & 1 & YES & YES & YES & NO & 438\\
$(31, 11)$ & 8 & $(5, 2)$ & 3 & 1 & YES & YES & YES & NO & 439\\
$(31, 11)$ & 8 & $(7, 2)$ & 4 & 1 & YES & YES & YES & NO & 440\\
$(31, 13)$ & 7 & $(7, 3)$ & 4 & 1 & YES & YES & NO(2) & 386 & 441\\
$(31, 6)$ & 10 & $(9, 4)$ & 5 & 1 & YES & YES & YES & NO & 442\\
$(31, 11)$ & 8 & $(9, 2)$ & 5 & 1 & YES & YES & YES & NO & 443\\
$(31, 6)$ & 10 & $(20, 3)$ & 8 & 1 & YES & YES & YES & NO & 444\\
$(31, 11)$ & 8 & $(20, 7)$ & 8 & 1 & YES & YES & YES & NO & 445\\
$(31, 7)$ & 8 & $(24, 5)$ & 8 & 1 & YES & YES & YES & NO & 446\\
$(31, 6)$ & 10 & $(28, 5)$ & 8 & 1 & YES & YES & YES & NO & 447\\
$(32, 13)$ & 9 & $(2, 1)$ & 1 & 2 & YES & YES & YES & NO & 448\\
$(32, 13)$ & 9 & $(3, 1)$ & 2 & 1 & YES & YES & YES & NO & 449\\
$(32, 13)$ & 9 & $(4, 1)$ & 3 & 4 & YES & YES & YES & NO & 450\\
$(32, 7)$ & 8 & $(5, 1)$ & 4 & 1 & YES & YES & YES & NO & 451\\
$(32, 7)$ & 8 & $(5, 1)$ & 4 & 1 & YES & YES & YES & NO & 452\\
$(32, 9)$ & 8 & $(5, 2)$ & 3 & 1 & YES & YES & YES & NO & 453\\
$(32, 9)$ & 8 & $(7, 2)$ & 4 & 1 & YES & YES & YES & NO & 454\\
$(32, 13)$ & 9 & $(7, 1)$ & 6 & 1 & YES & YES & YES & NO & 455\\
$(32, 13)$ & 9 & $(8, 1)$ & 7 & 8 & YES & YES & YES & NO & 456\\
$(32, 13)$ & 9 & $(8, 1)$ & 7 & 8 & YES & YES & YES & NO & 457\\
$(32, 7)$ & 8 & $(9, 2)$ & 5 & 1 & YES & YES & YES & NO & 458\\
$(32, 13)$ & 9 & $(12, 5)$ & 5 & 4 & YES & YES & YES & NO & 459\\
$(32, 9)$ & 8 & $(17, 5)$ & 6 & 1 & YES & YES & YES & NO & 460\\
$(32, 7)$ & 8 & $(21, 5)$ & 8 & 1 & YES & YES & YES & NO & 461\\
$(32, 13)$ & 9 & $(27, 11)$ & 8 & 1 & YES & YES & YES & NO & 462\\
$(32, 13)$ & 9 & $(32, 13)$ & 9 & 32 & YES & YES & YES & NO & 463\\
$(33, 13)$ & 9 & $(2, 1)$ & 1 & 1 & YES & YES & YES & NO & 464\\
$(33, 13)$ & 9 & $(2, 1)$ & 1 & 1 & YES & YES & YES & NO & 465\\
$(33, 13)$ & 9 & $(3, 1)$ & 2 & 3 & YES & YES & YES & NO & 466\\
$(33, 13)$ & 9 & $(3, 1)$ & 2 & 3 & YES & YES & YES & NO & 467\\
$(33, 10)$ & 8 & $(4, 1)$ & 3 & 1 & YES & YES & YES & NO & 468\\
$(33, 10)$ & 8 & $(4, 1)$ & 3 & 1 & YES & YES & YES & NO & 469\\
$(33, 10)$ & 8 & $(4, 1)$ & 3 & 1 & YES & YES & YES & NO & 470\\
$(33, 13)$ & 9 & $(4, 1)$ & 3 & 1 & YES & YES & YES & NO & 471\\
$(33, 13)$ & 9 & $(5, 1)$ & 4 & 1 & YES & YES & YES & NO & 472\\
$(33, 13)$ & 9 & $(5, 1)$ & 4 & 1 & YES & YES & YES & NO & 473\\
$(33, 14)$ & 8 & $(5, 1)$ & 4 & 1 & YES & YES & NO(2) & NO & 474\\
$(33, 13)$ & 9 & $(6, 1)$ & 5 & 3 & YES & YES & YES & NO & 475\\
$(33, 10)$ & 8 & $(7, 3)$ & 4 & 1 & YES & YES & YES & NO & 476\\
$(33, 13)$ & 9 & $(7, 1)$ & 6 & 1 & YES & YES & YES & NO & 477\\
$(33, 13)$ & 9 & $(8, 3)$ & 4 & 1 & YES & YES & YES & NO & 478\\
$(33, 13)$ & 9 & $(13, 5)$ & 5 & 1 & YES & YES & YES & NO & 479\\
$(33, 13)$ & 9 & $(18, 7)$ & 6 & 3 & YES & YES & YES & NO & 480\\
$(33, 14)$ & 8 & $(19, 8)$ & 6 & 1 & YES & YES & NO(2) & 587 & 481\\
$(33, 13)$ & 9 & $(23, 9)$ & 7 & 1 & YES & YES & YES & 619 & 482\\
$(33, 13)$ & 9 & $(33, 13)$ & 9 & 33 & YES & YES & YES & NO & 483\\
$(34, 9)$ & 8 & $(2, 1)$ & 1 & 2 & YES & YES & YES & NO & 484\\
$(34, 13)$ & 7 & $(2, 1)$ & 1 & 2 & YES & YES & YES & NO & 485\\
$(34, 13)$ & 7 & $(2, 1)$ & 1 & 2 & YES & YES & YES & NO & 486\\
$(34, 9)$ & 8 & $(3, 1)$ & 2 & 1 & YES & YES & YES & NO & 487\\
$(34, 9)$ & 8 & $(3, 1)$ & 2 & 1 & YES & YES & YES & NO & 488\\
$(34, 13)$ & 7 & $(3, 1)$ & 2 & 1 & YES & YES & YES & NO & 489\\
$(34, 13)$ & 7 & $(3, 1)$ & 2 & 1 & YES & YES & YES & 428 & 490\\
$(34, 7)$ & 10 & $(4, 1)$ & 3 & 2 & YES & YES & YES & NO & 491\\
$(34, 9)$ & 8 & $(4, 1)$ & 3 & 2 & YES & YES & YES & 362 & 492\\
$(34, 9)$ & 8 & $(4, 1)$ & 3 & 2 & YES & YES & YES & NO & 493\\
$(34, 9)$ & 8 & $(4, 1)$ & 3 & 2 & YES & YES & YES & NO & 494\\
$(34, 9)$ & 8 & $(5, 1)$ & 4 & 1 & YES & YES & YES & NO & 495\\
$(34, 9)$ & 8 & $(5, 1)$ & 4 & 1 & YES & YES & YES & NO & 496\\
$(34, 7)$ & 10 & $(14, 3)$ & 6 & 2 & YES & YES & YES & NO & 497\\
$(35, 13)$ & 8 & $(3, 1)$ & 2 & 1 & YES & YES & YES & NO & 498\\
$(35, 11)$ & 9 & $(4, 1)$ & 3 & 1 & YES & YES & YES & NO & 499\\
$(35, 16)$ & 9 & $(4, 1)$ & 3 & 1 & YES & YES & YES & NO & 500\\
$(35, 11)$ & 9 & $(5, 2)$ & 3 & 5 & YES & YES & YES & NO & 501\\
$(35, 16)$ & 9 & $(5, 1)$ & 4 & 5 & YES & YES & YES & NO & 502\\
$(35, 16)$ & 9 & $(5, 1)$ & 4 & 5 & YES & YES & YES & NO & 503\\
$(35, 16)$ & 9 & $(13, 6)$ & 7 & 1 & YES & YES & YES & 529 & 504\\
$(35, 13)$ & 8 & $(14, 5)$ & 6 & 7 & YES & YES & YES & NO & 505\\
$(35, 6)$ & 10 & $(19, 3)$ & 8 & 1 & YES & YES & YES & NO & 506\\
$(35, 16)$ & 9 & $(24, 11)$ & 8 & 1 & YES & YES & YES & NO & 507\\
$(35, 16)$ & 9 & $(35, 16)$ & 9 & 35 & YES & YES & YES & NO & 508\\
$(36, 11)$ & 8 & $(3, 1)$ & 2 & 3 & YES & YES & YES & NO & 509\\
$(36, 11)$ & 8 & $(3, 1)$ & 2 & 3 & YES & YES & YES & NO & 510\\
$(36, 13)$ & 8 & $(5, 1)$ & 4 & 1 & YES & YES & YES & NO & 511\\
$(36, 13)$ & 8 & $(5, 1)$ & 4 & 1 & YES & YES & YES & NO & 512\\
$(36, 13)$ & 8 & $(7, 3)$ & 4 & 1 & YES & YES & YES & NO & 513\\
$(36, 13)$ & 8 & $(14, 5)$ & 6 & 2 & YES & YES & YES & 534 & 514\\
$(37, 10)$ & 8 & $(2, 1)$ & 1 & 1 & YES & YES & YES & NO & 515\\
$(37, 14)$ & 8 & $(2, 1)$ & 1 & 1 & YES & YES & YES & NO & 516\\
$(37, 14)$ & 8 & $(2, 1)$ & 1 & 1 & YES & YES & YES & NO & 517\\
$(37, 17)$ & 9 & $(2, 1)$ & 1 & 1 & YES & YES & YES & NO & 518\\
$(37, 10)$ & 8 & $(3, 1)$ & 2 & 1 & YES & YES & YES & NO & 519\\
$(37, 10)$ & 8 & $(3, 1)$ & 2 & 1 & YES & YES & YES & NO & 520\\
$(37, 14)$ & 8 & $(3, 1)$ & 2 & 1 & YES & YES & YES & NO & 521\\
$(37, 14)$ & 8 & $(3, 1)$ & 2 & 1 & YES & YES & YES & NO & 522\\
$(37, 17)$ & 9 & $(3, 1)$ & 2 & 1 & YES & YES & YES & NO & 523\\
$(37, 13)$ & 9 & $(4, 1)$ & 3 & 1 & YES & YES & YES & NO & 524\\
$(37, 16)$ & 9 & $(6, 1)$ & 5 & 1 & YES & YES & YES & NO & 525\\
$(37, 16)$ & 9 & $(7, 1)$ & 6 & 1 & YES & YES & YES & NO & 526\\
$(37, 16)$ & 9 & $(7, 1)$ & 6 & 1 & YES & YES & YES & NO & 527\\
$(37, 17)$ & 9 & $(9, 4)$ & 5 & 1 & YES & YES & YES & 615 & 528\\
$(37, 17)$ & 9 & $(11, 5)$ & 6 & 1 & YES & YES & YES & 504 & 529\\
$(37, 13)$ & 9 & $(20, 7)$ & 8 & 1 & YES & YES & YES & NO & 530\\
$(37, 16)$ & 9 & $(30, 13)$ & 8 & 1 & YES & YES & YES & NO & 531\\
$(37, 16)$ & 9 & $(37, 16)$ & 9 & 37 & YES & YES & YES & NO & 532\\
$(39, 14)$ & 8 & $(2, 1)$ & 1 & 1 & YES & YES & YES & NO & 533\\
$(39, 14)$ & 8 & $(11, 4)$ & 5 & 1 & YES & YES & YES & 514 & 534\\
$(39, 14)$ & 8 & $(14, 5)$ & 6 & 1 & YES & YES & YES & NO & 535\\
$(40, 17)$ & 9 & $(2, 1)$ & 1 & 2 & YES & YES & YES & NO & 536\\
$(40, 17)$ & 9 & $(2, 1)$ & 1 & 2 & YES & YES & YES & NO & 537\\
$(40, 17)$ & 9 & $(3, 1)$ & 2 & 1 & YES & YES & YES & NO & 538\\
$(40, 17)$ & 9 & $(4, 1)$ & 3 & 4 & YES & YES & YES & NO & 539\\
$(40, 17)$ & 9 & $(5, 2)$ & 3 & 5 & YES & YES & YES & NO & 540\\
$(40, 17)$ & 9 & $(8, 1)$ & 7 & 8 & YES & YES & YES & NO & 541\\
$(40, 17)$ & 9 & $(8, 1)$ & 7 & 8 & YES & YES & YES & NO & 542\\
$(40, 17)$ & 9 & $(40, 17)$ & 9 & 40 & YES & YES & YES & NO & 543\\
$(41, 13)$ & 10 & $(3, 1)$ & 2 & 1 & YES & YES & YES & 383 & 544\\
$(41, 16)$ & 8 & $(3, 1)$ & 2 & 1 & YES & YES & YES & NO & 545\\
$(41, 16)$ & 8 & $(3, 1)$ & 2 & 1 & YES & YES & YES & NO & 546\\
$(41, 16)$ & 8 & $(3, 1)$ & 2 & 1 & YES & YES & YES & NO & 547\\
$(41, 13)$ & 10 & $(5, 1)$ & 4 & 1 & YES & YES & YES & NO & 548\\
$(41, 16)$ & 8 & $(5, 1)$ & 4 & 1 & YES & YES & YES & NO & 549\\
$(41, 15)$ & 8 & $(7, 2)$ & 4 & 1 & YES & YES & YES & NO & 550\\
$(41, 15)$ & 8 & $(7, 3)$ & 4 & 1 & YES & YES & YES & NO & 551\\
$(41, 19)$ & 10 & $(7, 1)$ & 6 & 1 & YES & YES & YES & NO & 552\\
$(41, 8)$ & 12 & $(9, 1)$ & 8 & 1 & YES & YES & YES & NO & 553\\
$(41, 19)$ & 10 & $(11, 5)$ & 6 & 1 & YES & YES & YES & NO & 554\\
$(41, 19)$ & 10 & $(13, 6)$ & 7 & 1 & YES & YES & YES & NO & 555\\
$(41, 8)$ & 12 & $(21, 4)$ & 8 & 1 & YES & YES & YES & NO & 556\\
$(41, 8)$ & 12 & $(41, 8)$ & 12 & 41 & YES & YES & YES & NO & 557\\
$(42, 13)$ & 9 & $(2, 1)$ & 1 & 2 & YES & YES & YES & NO & 558\\
$(42, 19)$ & 9 & $(2, 1)$ & 1 & 2 & YES & YES & YES & NO & 559\\
$(42, 19)$ & 9 & $(3, 1)$ & 2 & 3 & YES & YES & YES & NO & 560\\
$(42, 19)$ & 9 & $(6, 1)$ & 5 & 6 & YES & YES & YES & NO & 561\\
$(42, 19)$ & 9 & $(6, 1)$ & 5 & 6 & YES & YES & YES & NO & 562\\
$(42, 19)$ & 9 & $(9, 4)$ & 5 & 3 & YES & YES & YES & NO & 563\\
$(43, 19)$ & 9 & $(2, 1)$ & 1 & 1 & YES & YES & YES & NO & 564\\
$(43, 19)$ & 9 & $(2, 1)$ & 1 & 1 & YES & YES & YES & NO & 565\\
$(43, 20)$ & 10 & $(2, 1)$ & 1 & 1 & NO & YES & YES & NO & 566\\
$(43, 16)$ & 9 & $(3, 1)$ & 2 & 1 & YES & YES & YES & NO & 567\\
$(43, 16)$ & 9 & $(3, 1)$ & 2 & 1 & YES & YES & YES & NO & 568\\
$(43, 19)$ & 9 & $(3, 1)$ & 2 & 1 & YES & YES & YES & NO & 569\\
$(43, 16)$ & 9 & $(4, 1)$ & 3 & 1 & YES & YES & YES & NO & 570\\
$(43, 8)$ & 9 & $(5, 1)$ & 4 & 1 & NO & YES & YES & NO & 571\\
$(43, 16)$ & 9 & $(6, 1)$ & 5 & 1 & YES & YES & YES & NO & 572\\
$(43, 19)$ & 9 & $(7, 1)$ & 6 & 1 & YES & YES & YES & NO & 573\\
$(43, 19)$ & 9 & $(7, 1)$ & 6 & 1 & YES & YES & YES & NO & 574\\
$(43, 19)$ & 9 & $(7, 3)$ & 4 & 1 & YES & YES & YES & NO & 575\\
$(43, 16)$ & 9 & $(19, 7)$ & 6 & 1 & YES & YES & YES & NO & 576\\
$(43, 16)$ & 9 & $(35, 13)$ & 8 & 1 & YES & YES & YES & NO & 577\\
$(43, 19)$ & 9 & $(43, 19)$ & 9 & 43 & YES & YES & YES & NO & 578\\
$(44, 19)$ & 10 & $(2, 1)$ & 1 & 2 & YES & YES & YES & NO & 579\\
$(44, 19)$ & 10 & $(3, 1)$ & 2 & 1 & YES & YES & YES & NO & 580\\
$(44, 19)$ & 10 & $(9, 4)$ & 5 & 1 & YES & YES & YES & NO & 581\\
$(45, 14)$ & 9 & $(3, 1)$ & 2 & 3 & NO & YES & YES & NO & 582\\
$(45, 17)$ & 9 & $(3, 1)$ & 2 & 3 & YES & YES & YES & NO & 583\\
$(45, 13)$ & 10 & $(4, 1)$ & 3 & 1 & YES & YES & YES & NO & 584\\
$(45, 19)$ & 8 & $(5, 1)$ & 4 & 5 & YES & YES & NO(2) & NO & 585\\
$(45, 13)$ & 10 & $(7, 2)$ & 4 & 1 & YES & YES & YES & NO & 586\\
$(45, 19)$ & 8 & $(7, 3)$ & 4 & 1 & YES & YES & NO(2) & 481 & 587\\
$(45, 17)$ & 9 & $(8, 3)$ & 4 & 1 & YES & YES & YES & NO & 588\\
$(46, 21)$ & 10 & $(2, 1)$ & 1 & 2 & YES & YES & YES & NO & 589\\
$(46, 13)$ & 10 & $(3, 1)$ & 2 & 1 & YES & YES & YES & NO & 590\\
$(46, 21)$ & 10 & $(3, 1)$ & 2 & 1 & YES & YES & YES & 301 & 591\\
$(46, 13)$ & 10 & $(7, 2)$ & 4 & 1 & YES & YES & YES & NO & 592\\
$(46, 21)$ & 10 & $(13, 6)$ & 7 & 1 & YES & YES & YES & NO & 593\\
$(47, 20)$ & 10 & $(2, 1)$ & 1 & 1 & NO & YES & YES & NO & 594\\
$(47, 8)$ & 12 & $(7, 1)$ & 6 & 1 & YES & YES & YES & NO & 595\\
$(47, 8)$ & 12 & $(9, 1)$ & 8 & 1 & YES & YES & YES & NO & 596\\
$(47, 8)$ & 12 & $(35, 6)$ & 10 & 1 & YES & YES & YES & 656 & 597\\
$(47, 8)$ & 12 & $(47, 8)$ & 12 & 47 & YES & YES & YES & NO & 598\\
$(48, 17)$ & 9 & $(3, 1)$ & 2 & 3 & YES & YES & YES & NO & 599\\
$(48, 7)$ & 12 & $(4, 1)$ & 3 & 4 & YES & YES & YES & NO & 600\\
$(48, 17)$ & 9 & $(5, 1)$ & 4 & 1 & YES & YES & YES & NO & 601\\
$(48, 17)$ & 9 & $(17, 6)$ & 7 & 1 & YES & YES & YES & NO & 602\\
$(48, 17)$ & 9 & $(31, 11)$ & 8 & 1 & YES & YES & YES & NO & 603\\
$(49, 19)$ & 8 & $(2, 1)$ & 1 & 1 & YES & YES & NO(2) & NO & 604\\
$(49, 20)$ & 9 & $(2, 1)$ & 1 & 1 & YES & YES & YES & NO & 605\\
$(49, 22)$ & 9 & $(2, 1)$ & 1 & 1 & YES & YES & YES & 424 & 606\\
$(49, 8)$ & 13 & $(5, 1)$ & 4 & 1 & YES & YES & YES & NO & 607\\
$(49, 22)$ & 9 & $(11, 5)$ & 6 & 1 & YES & YES & YES & NO & 608\\
$(49, 9)$ & 10 & $(28, 5)$ & 8 & 7 & YES & YES & YES & NO & 609\\
$(50, 21)$ & 8 & $(2, 1)$ & 1 & 2 & NO & YES & YES & NO & 610\\
$(50, 9)$ & 10 & $(5, 1)$ & 4 & 5 & NO & YES & YES & NO & 611\\
$(51, 16)$ & 10 & $(2, 1)$ & 1 & 1 & YES & YES & YES & NO & 612\\
$(51, 20)$ & 9 & $(2, 1)$ & 1 & 1 & YES & YES & YES & NO & 613\\
$(51, 20)$ & 9 & $(2, 1)$ & 1 & 1 & YES & YES & YES & NO & 614\\
$(51, 23)$ & 9 & $(2, 1)$ & 1 & 1 & YES & YES & YES & 528 & 615\\
$(51, 11)$ & 9 & $(4, 1)$ & 3 & 1 & NO & YES & YES & NO & 616\\
$(51, 20)$ & 9 & $(5, 1)$ & 4 & 1 & YES & YES & YES & NO & 617\\
$(51, 20)$ & 9 & $(5, 1)$ & 4 & 1 & YES & YES & YES & NO & 618\\
$(51, 20)$ & 9 & $(5, 2)$ & 3 & 1 & YES & YES & YES & 482 & 619\\
$(52, 23)$ & 10 & $(2, 1)$ & 1 & 2 & NO & YES & YES & NO & 620\\
$(52, 19)$ & 9 & $(3, 1)$ & 2 & 1 & YES & YES & YES & NO & 621\\
$(52, 19)$ & 9 & $(11, 4)$ & 5 & 1 & YES & YES & YES & NO & 622\\
$(53, 19)$ & 9 & $(2, 1)$ & 1 & 1 & YES & YES & YES & NO & 623\\
$(53, 24)$ & 10 & $(2, 1)$ & 1 & 1 & NO & YES & YES & NO & 624\\
$(53, 19)$ & 9 & $(3, 1)$ & 2 & 1 & YES & YES & YES & NO & 625\\
$(53, 11)$ & 10 & $(4, 1)$ & 3 & 1 & YES & YES & YES & NO & 626\\
$(53, 19)$ & 9 & $(5, 1)$ & 4 & 1 & YES & YES & YES & NO & 627\\
$(53, 19)$ & 9 & $(25, 9)$ & 7 & 1 & YES & YES & YES & 641 & 628\\
$(54, 25)$ & 11 & $(2, 1)$ & 1 & 2 & NO & YES & YES & NO & 629\\
$(55, 23)$ & 9 & $(2, 1)$ & 1 & 1 & NO & YES & YES & NO & 630\\
$(55, 24)$ & 9 & $(2, 1)$ & 1 & 1 & YES & YES & YES & NO & 631\\
$(55, 16)$ & 9 & $(3, 1)$ & 2 & 1 & NO & YES & YES & NO & 632\\
$(56, 25)$ & 11 & $(2, 1)$ & 1 & 2 & NO & YES & YES & NO & 633\\
$(56, 15)$ & 9 & $(3, 1)$ & 2 & 1 & NO & YES & YES & NO & 634\\
$(57, 25)$ & 9 & $(2, 1)$ & 1 & 1 & YES & YES & YES & NO & 635\\
$(58, 13)$ & 11 & $(2, 1)$ & 1 & 2 & YES & YES & YES & NO & 636\\
$(58, 17)$ & 9 & $(3, 1)$ & 2 & 1 & NO & YES & YES & NO & 637\\
$(59, 13)$ & 11 & $(9, 2)$ & 5 & 1 & YES & YES & YES & NO & 638\\
$(61, 22)$ & 9 & $(2, 1)$ & 1 & 1 & NO & YES & YES & NO & 639\\
$(64, 17)$ & 10 & $(3, 1)$ & 2 & 1 & YES & YES & YES & NO & 640\\
$(64, 23)$ & 9 & $(14, 5)$ & 6 & 2 & YES & YES & YES & 628 & 641\\
$(65, 24)$ & 9 & $(2, 1)$ & 1 & 1 & YES & YES & YES & NO & 642\\
$(65, 24)$ & 9 & $(3, 1)$ & 2 & 1 & YES & YES & YES & NO & 643\\
$(66, 25)$ & 9 & $(2, 1)$ & 1 & 2 & NO & YES & YES & NO & 644\\
$(67, 16)$ & 11 & $(4, 1)$ & 3 & 1 & YES & YES & YES & NO & 645\\
$(67, 16)$ & 11 & $(21, 5)$ & 8 & 1 & YES & YES & YES & NO & 646\\
$(71, 22)$ & 10 & $(2, 1)$ & 1 & 1 & YES & YES & YES & NO & 647\\
$(71, 25)$ & 11 & $(2, 1)$ & 1 & 1 & NO & YES & YES & NO & 648\\
$(71, 27)$ & 9 & $(2, 1)$ & 1 & 1 & NO & YES & YES & NO & 649\\
$(71, 17)$ & 11 & $(4, 1)$ & 3 & 1 & YES & YES & YES & NO & 650\\
$(71, 19)$ & 10 & $(4, 1)$ & 3 & 1 & YES & YES & YES & NO & 651\\
$(71, 13)$ & 12 & $(6, 1)$ & 5 & 1 & YES & YES & YES & NO & 652\\
$(71, 19)$ & 10 & $(15, 4)$ & 6 & 1 & YES & YES & YES & NO & 653\\
$(72, 13)$ & 12 & $(5, 1)$ & 4 & 1 & YES & YES & YES & NO & 654\\
$(74, 29)$ & 10 & $(2, 1)$ & 1 & 2 & NO & YES & YES & NO & 655\\
$(76, 13)$ & 12 & $(6, 1)$ & 5 & 2 & YES & YES & YES & 597 & 656\\
$(77, 16)$ & 11 & $(5, 1)$ & 4 & 1 & YES & YES & YES & NO & 657\\
$(91, 19)$ & 11 & $(5, 1)$ & 4 & 1 & YES & YES & YES & NO & 658\\
$(91, 19)$ & 11 & $(24, 5)$ & 8 & 1 & YES & YES & YES & NO & 659\\
$(99, 17)$ & 12 & $(6, 1)$ & 5 & 3 & YES & YES & YES & NO & 660\\
$(101, 16)$ & 13 & $(6, 1)$ & 5 & 1 & YES & YES & YES & NO & 661\\
$(a; 1, 0, 0; 13)$ & 5 & $(16, 5)$ & 7 & 1 & YES & YES & YES & NO & 662\\
$(a; 3, 0, 0; 7)$ & 7 & $(3, 1)$ & 2 & 1 & YES & YES & YES & NO & 663\\
$(a; 3, 1, 0; 31)$ & 8 & $(3, 1)$ & 2 & 1 & YES & YES & YES & NO & 664\\
$(a; 4, 0, 0; 25)$ & 8 & $(4, 1)$ & 3 & 1 & YES & YES & YES & NO & 665\\
$(a; 4, 0, 1; 37)$ & 9 & $(7, 1)$ & 6 & 1 & YES & YES & YES & NO & 666\\
$(b; 0, 3, 0; 29)$ & 8 & $(2, 1)$ & 1 & 1 & YES & YES & YES & NO & 667\\
$(c; 0, 0, 0; 4)$ & 4 & $(13, 6)$ & 7 & 1 & YES & YES & YES & NO & 668\\
$(c; 0, 0, 0; 4)$ & 4 & $(17, 6)$ & 7 & 1 & YES & YES & YES & NO & 669\\
$(c; 0, 1, 0; 11)$ & 5 & $(14, 5)$ & 6 & 1 & YES & YES & YES & NO & 670\\
$(c; 0, 1, 0; 11)$ & 5 & $(19, 5)$ & 7 & 1 & YES & YES & YES & NO & 671\\
$(c; 0, 1, 1; 5)$ & 6 & $(13, 4)$ & 6 & 1 & YES & YES & YES & NO & 672\\
$(c; 0, 2, 0; 7)$ & 6 & $(8, 3)$ & 4 & 1 & YES & YES & YES & NO & 673\\
$(c; 0, 2, 1; 19)$ & 7 & $(11, 3)$ & 5 & 1 & YES & YES & YES & NO & 674\\
$(c; 0, 3, 0; 17)$ & 7 & $(4, 1)$ & 3 & 1 & YES & YES & YES & NO & 675\\
$(c; 0, 3, 2; 29)$ & 9 & $(6, 1)$ & 5 & 1 & YES & YES & YES & NO & 676\\
$(c; 0, 4, 0; 10)$ & 8 & $(4, 1)$ & 3 & 2 & YES & YES & YES & NO & 677\\
$(c; 0, 4, 1; 9)$ & 9 & $(4, 1)$ & 3 & 1 & YES & YES & YES & NO & 678\\
$(c; 0, 4, 1; 9)$ & 9 & $(7, 1)$ & 6 & 1 & YES & YES & YES & NO & 679\\
$(d; 0, 0, 4; 13)$ & 9 & $(4, 1)$ & 3 & 1 & YES & YES & YES & NO & 680\\
$(d; 0, 1, 0; 6)$ & 6 & $(4, 1)$ & 3 & 2 & YES & YES & YES & NO & 681\\
$(d; 0, 1, 0; 6)$ & 6 & $(7, 3)$ & 4 & 1 & YES & YES & YES & NO & 682\\
$(d; 0, 1, 2; 11)$ & 8 & $(5, 1)$ & 4 & 1 & YES & YES & YES & NO & 683\\
$(d; 0, 1, 3; 27)$ & 9 & $(2, 1)$ & 1 & 1 & YES & YES & YES & NO & 684\\
$(d; 0, 1, 3; 27)$ & 9 & $(5, 1)$ & 4 & 1 & YES & YES & YES & NO & 685\\
$(d; 0, 2, 0; 7)$ & 7 & $(4, 1)$ & 3 & 1 & YES & YES & YES & NO & 686\\
$(d; 0, 3, 1; 23)$ & 9 & $(7, 1)$ & 6 & 1 & YES & YES & YES & NO & 687\\
$(e; 0, 3, 0; 7)$ & 8 & $(2, 1)$ & 1 & 1 & YES & YES & YES & NO & 688\\
$(e; 0, 3, 0; 7)$ & 8 & $(6, 1)$ & 5 & 1 & YES & YES & YES & NO & 689\\
$(e; 2, 0, 0; 24)$ & 7 & $(2, 1)$ & 1 & 2 & YES & YES & NO(2) & NO & 690\\
$(f; 0, 0, 0; 6)$ & 4 & $(19, 5)$ & 7 & 1 & YES & YES & YES & NO & 691\\
$(f; 0, 1, 0; 7)$ & 5 & $(10, 3)$ & 5 & 1 & YES & YES & YES & NO & 692\\
$(f; 0, 1, 0; 7)$ & 5 & $(12, 5)$ & 5 & 1 & YES & YES & YES & NO & 693\\
$(f; 0, 1, 0; 7)$ & 5 & $(13, 5)$ & 5 & 1 & YES & YES & YES & NO & 694\\
$(f; 0, 2, 0; 8)$ & 6 & $(17, 3)$ & 7 & 1 & YES & YES & YES & NO & 695\\
$(i; 0, 0, 0; 9)$ & 5 & $(6, 1)$ & 5 & 3 & YES & YES & YES & NO & 696\\
$(i; 0, 0, 0; 9)$ & 5 & $(9, 4)$ & 5 & 9 & YES & YES & YES & NO & 697\\
$(i; 0, 0, 0; 9)$ & 5 & $(19, 4)$ & 7 & 1 & YES & YES & YES & NO & 698\\
$(i; 0, 3, 0; 18)$ & 8 & $(5, 1)$ & 4 & 1 & YES & YES & YES & NO & 699\\
$(j; 0, 0, 0; 8)$ & 5 & $(11, 5)$ & 6 & 1 & YES & YES & YES & NO & 700\\
$(j; 0, 4, 0; 16)$ & 9 & $(6, 1)$ & 5 & 2 & YES & YES & YES & NO & 701
\end{longtable}
\subsection{2 chains, $K^2 = 3$}
\begin{longtable}{|c|c|c|c|c|c|c|c|c|c|}
\hline
\multicolumn{10}{|c|}{2 chains, $K^2 = 3$}\\
\hline
$(n,a)$ & Length & $(n,a)$ & Length & GCD & Nef & $\mathbb Q$-ef & Obstruction 0 & WH & Index\\
\hline
\endfirsthead

\hline
$(n,a)$ & Length & $(n,a)$ & Length & GCD & Nef & $\mathbb Q$-ef & Obstruction 0 & WH & Index\\
\hline
\endhead
\hline
\endfoot

$(17, 3)$ & 7 & $(14, 5)$ & 6 & 1 & YES & YES & YES & NO & 702\\
$(19, 6)$ & 8 & $(19, 6)$ & 8 & 19 & YES & YES & YES & NO & 703\\
$(20, 7)$ & 8 & $(17, 3)$ & 7 & 1 & YES & YES & YES & NO & 704\\
$(20, 7)$ & 8 & $(17, 3)$ & 7 & 1 & YES & YES & YES & NO & 705\\
$(20, 9)$ & 7 & $(18, 7)$ & 6 & 2 & YES & YES & YES & NO & 706\\
$(23, 10)$ & 7 & $(15, 4)$ & 6 & 1 & YES & YES & YES & 759 & 707\\
$(23, 10)$ & 7 & $(15, 4)$ & 6 & 1 & YES & YES & YES & NO & 708\\
$(23, 10)$ & 7 & $(18, 7)$ & 6 & 1 & YES & YES & YES & NO & 709\\
$(23, 10)$ & 7 & $(18, 7)$ & 6 & 1 & YES & YES & YES & NO & 710\\
$(23, 7)$ & 7 & $(20, 7)$ & 8 & 1 & YES & YES & YES & NO & 711\\
$(24, 11)$ & 8 & $(19, 5)$ & 7 & 1 & YES & YES & YES & NO & 712\\
$(24, 11)$ & 8 & $(21, 5)$ & 8 & 3 & YES & YES & YES & NO & 713\\
$(24, 11)$ & 8 & $(23, 9)$ & 7 & 1 & YES & YES & YES & NO & 714\\
$(25, 7)$ & 7 & $(9, 4)$ & 5 & 1 & YES & YES & NO(2) & NO & 715\\
$(25, 7)$ & 7 & $(9, 4)$ & 5 & 1 & YES & YES & NO(2) & NO & 716\\
$(25, 9)$ & 7 & $(13, 6)$ & 7 & 1 & YES & YES & YES & NO & 717\\
$(27, 10)$ & 7 & $(11, 4)$ & 5 & 1 & YES & YES & YES & NO & 718\\
$(27, 10)$ & 7 & $(11, 5)$ & 6 & 1 & YES & YES & YES & NO & 719\\
$(27, 11)$ & 8 & $(13, 3)$ & 6 & 1 & YES & YES & YES & NO & 720\\
$(27, 11)$ & 8 & $(13, 3)$ & 6 & 1 & YES & YES & YES & NO & 721\\
$(27, 11)$ & 8 & $(13, 3)$ & 6 & 1 & YES & YES & YES & NO & 722\\
$(27, 11)$ & 8 & $(24, 11)$ & 8 & 3 & YES & YES & YES & NO & 723\\
$(28, 11)$ & 8 & $(13, 3)$ & 6 & 1 & YES & YES & YES & NO & 724\\
$(28, 11)$ & 8 & $(13, 3)$ & 6 & 1 & YES & YES & YES & NO & 725\\
$(28, 5)$ & 8 & $(17, 6)$ & 7 & 1 & YES & YES & YES & NO & 726\\
$(28, 5)$ & 8 & $(17, 6)$ & 7 & 1 & YES & YES & YES & NO & 727\\
$(29, 11)$ & 7 & $(11, 5)$ & 6 & 1 & YES & YES & YES & NO & 728\\
$(29, 9)$ & 8 & $(12, 5)$ & 5 & 1 & YES & YES & YES & NO & 729\\
$(29, 9)$ & 8 & $(12, 5)$ & 5 & 1 & YES & YES & YES & NO & 730\\
$(29, 13)$ & 8 & $(13, 4)$ & 6 & 1 & YES & YES & YES & NO & 731\\
$(29, 13)$ & 8 & $(13, 4)$ & 6 & 1 & YES & YES & YES & NO & 732\\
$(30, 13)$ & 8 & $(9, 4)$ & 5 & 3 & YES & YES & NO(2) & NO & 733\\
$(31, 14)$ & 8 & $(6, 1)$ & 5 & 1 & YES & YES & YES & NO & 734\\
$(31, 13)$ & 7 & $(13, 6)$ & 7 & 1 & YES & YES & YES & NO & 735\\
$(31, 14)$ & 8 & $(13, 5)$ & 5 & 1 & YES & YES & YES & NO & 736\\
$(31, 9)$ & 8 & $(17, 4)$ & 7 & 1 & YES & YES & YES & NO & 737\\
$(31, 9)$ & 8 & $(17, 4)$ & 7 & 1 & YES & YES & YES & NO & 738\\
$(31, 13)$ & 7 & $(19, 6)$ & 8 & 1 & YES & YES & YES & NO & 739\\
$(32, 13)$ & 9 & $(28, 11)$ & 8 & 4 & YES & YES & YES & NO & 740\\
$(33, 10)$ & 8 & $(9, 4)$ & 5 & 3 & YES & YES & YES & NO & 741\\
$(33, 10)$ & 8 & $(9, 4)$ & 5 & 3 & YES & YES & YES & NO & 742\\
$(33, 14)$ & 8 & $(16, 3)$ & 7 & 1 & YES & YES & YES & NO & 743\\
$(33, 13)$ & 9 & $(19, 3)$ & 8 & 1 & YES & YES & YES & NO & 744\\
$(33, 13)$ & 9 & $(27, 11)$ & 8 & 3 & YES & YES & YES & NO & 745\\
$(34, 9)$ & 8 & $(12, 5)$ & 5 & 2 & YES & YES & YES & NO & 746\\
$(34, 9)$ & 8 & $(12, 5)$ & 5 & 2 & YES & YES & YES & NO & 747\\
$(34, 15)$ & 8 & $(14, 5)$ & 6 & 2 & YES & YES & YES & NO & 748\\
$(34, 15)$ & 8 & $(27, 11)$ & 8 & 1 & YES & YES & YES & NO & 749\\
$(35, 13)$ & 8 & $(9, 4)$ & 5 & 1 & YES & YES & NO(2) & NO & 750\\
$(35, 13)$ & 8 & $(17, 3)$ & 7 & 1 & YES & YES & YES & NO & 751\\
$(35, 13)$ & 8 & $(17, 3)$ & 7 & 1 & YES & YES & YES & NO & 752\\
$(36, 7)$ & 11 & $(34, 5)$ & 10 & 2 & YES & YES & YES & NO & 753\\
$(36, 7)$ & 11 & $(36, 5)$ & 11 & 36 & YES & YES & YES & NO & 754\\
$(37, 17)$ & 9 & $(5, 1)$ & 4 & 1 & YES & YES & YES & NO & 755\\
$(37, 17)$ & 9 & $(7, 2)$ & 4 & 1 & YES & YES & YES & NO & 756\\
$(37, 10)$ & 8 & $(9, 4)$ & 5 & 1 & YES & YES & YES & NO & 757\\
$(37, 10)$ & 8 & $(9, 4)$ & 5 & 1 & YES & YES & YES & NO & 758\\
$(37, 10)$ & 8 & $(9, 4)$ & 5 & 1 & YES & YES & YES & 707 & 759\\
$(37, 13)$ & 9 & $(9, 4)$ & 5 & 1 & YES & YES & YES & NO & 760\\
$(37, 14)$ & 8 & $(9, 4)$ & 5 & 1 & YES & YES & NO(2) & NO & 761\\
$(37, 17)$ & 9 & $(9, 4)$ & 5 & 1 & YES & YES & YES & NO & 762\\
$(37, 10)$ & 8 & $(11, 4)$ & 5 & 1 & YES & YES & YES & NO & 763\\
$(37, 10)$ & 8 & $(11, 4)$ & 5 & 1 & YES & YES & YES & NO & 764\\
$(37, 17)$ & 9 & $(12, 5)$ & 5 & 1 & YES & YES & YES & NO & 765\\
$(37, 13)$ & 9 & $(13, 4)$ & 6 & 1 & YES & YES & YES & NO & 766\\
$(37, 10)$ & 8 & $(16, 5)$ & 7 & 1 & YES & YES & YES & NO & 767\\
$(37, 17)$ & 9 & $(16, 7)$ & 6 & 1 & YES & YES & YES & 863 & 768\\
$(38, 11)$ & 9 & $(11, 5)$ & 6 & 1 & YES & YES & YES & NO & 769\\
$(38, 11)$ & 9 & $(11, 5)$ & 6 & 1 & YES & YES & YES & NO & 770\\
$(38, 17)$ & 9 & $(13, 5)$ & 5 & 1 & YES & YES & YES & NO & 771\\
$(38, 17)$ & 9 & $(16, 3)$ & 7 & 2 & YES & YES & YES & NO & 772\\
$(39, 11)$ & 9 & $(5, 1)$ & 4 & 1 & YES & YES & YES & NO & 773\\
$(39, 14)$ & 8 & $(8, 3)$ & 4 & 1 & YES & YES & YES & NO & 774\\
$(39, 14)$ & 8 & $(20, 7)$ & 8 & 1 & YES & YES & YES & NO & 775\\
$(39, 16)$ & 8 & $(21, 5)$ & 8 & 3 & YES & YES & YES & NO & 776\\
$(40, 11)$ & 8 & $(11, 4)$ & 5 & 1 & YES & YES & YES & NO & 777\\
$(40, 11)$ & 8 & $(11, 4)$ & 5 & 1 & YES & YES & YES & NO & 778\\
$(40, 17)$ & 9 & $(11, 2)$ & 6 & 1 & YES & YES & YES & NO & 779\\
$(41, 16)$ & 8 & $(16, 5)$ & 7 & 1 & YES & YES & YES & NO & 780\\
$(41, 16)$ & 8 & $(20, 7)$ & 8 & 1 & YES & YES & YES & NO & 781\\
$(42, 19)$ & 9 & $(7, 2)$ & 4 & 7 & YES & YES & YES & NO & 782\\
$(42, 19)$ & 9 & $(8, 3)$ & 4 & 2 & YES & YES & YES & NO & 783\\
$(42, 11)$ & 9 & $(10, 3)$ & 5 & 2 & YES & YES & YES & NO & 784\\
$(42, 19)$ & 9 & $(11, 3)$ & 5 & 1 & YES & YES & YES & NO & 785\\
$(42, 19)$ & 9 & $(13, 4)$ & 6 & 1 & YES & YES & YES & NO & 786\\
$(42, 19)$ & 9 & $(14, 5)$ & 6 & 14 & YES & YES & YES & NO & 787\\
$(42, 19)$ & 9 & $(16, 3)$ & 7 & 2 & YES & YES & YES & NO & 788\\
$(43, 16)$ & 9 & $(5, 1)$ & 4 & 1 & YES & YES & YES & NO & 789\\
$(43, 18)$ & 8 & $(11, 5)$ & 6 & 1 & YES & YES & YES & NO & 790\\
$(44, 17)$ & 8 & $(9, 4)$ & 5 & 1 & YES & YES & YES & NO & 791\\
$(44, 17)$ & 8 & $(9, 4)$ & 5 & 1 & YES & YES & YES & NO & 792\\
$(45, 13)$ & 10 & $(6, 1)$ & 5 & 3 & YES & YES & YES & NO & 793\\
$(45, 17)$ & 9 & $(7, 2)$ & 4 & 1 & YES & YES & YES & NO & 794\\
$(45, 17)$ & 9 & $(7, 2)$ & 4 & 1 & YES & YES & YES & NO & 795\\
$(45, 13)$ & 10 & $(9, 2)$ & 5 & 9 & YES & YES & YES & NO & 796\\
$(45, 13)$ & 10 & $(9, 2)$ & 5 & 9 & YES & YES & YES & NO & 797\\
$(45, 17)$ & 9 & $(12, 5)$ & 5 & 3 & YES & YES & YES & NO & 798\\
$(45, 8)$ & 9 & $(16, 5)$ & 7 & 1 & YES & YES & YES & NO & 799\\
$(45, 13)$ & 10 & $(19, 6)$ & 8 & 1 & YES & YES & YES & NO & 800\\
$(45, 13)$ & 10 & $(22, 3)$ & 9 & 1 & YES & YES & YES & NO & 801\\
$(45, 19)$ & 8 & $(40, 17)$ & 9 & 5 & YES & YES & YES & 914 & 802\\
$(46, 13)$ & 10 & $(23, 7)$ & 7 & 23 & YES & YES & YES & NO & 803\\
$(47, 15)$ & 11 & $(7, 1)$ & 6 & 1 & YES & YES & YES & NO & 804\\
$(47, 15)$ & 11 & $(7, 1)$ & 6 & 1 & YES & YES & YES & NO & 805\\
$(47, 18)$ & 8 & $(11, 5)$ & 6 & 1 & YES & YES & YES & NO & 806\\
$(47, 22)$ & 11 & $(12, 5)$ & 5 & 1 & YES & YES & YES & NO & 807\\
$(48, 17)$ & 9 & $(5, 1)$ & 4 & 1 & YES & YES & YES & NO & 808\\
$(48, 17)$ & 9 & $(5, 1)$ & 4 & 1 & YES & YES & YES & NO & 809\\
$(49, 22)$ & 9 & $(42, 19)$ & 9 & 7 & YES & YES & YES & NO & 810\\
$(50, 21)$ & 8 & $(11, 5)$ & 6 & 1 & YES & YES & YES & NO & 811\\
$(50, 21)$ & 8 & $(24, 7)$ & 7 & 2 & YES & YES & YES & NO & 812\\
$(50, 9)$ & 10 & $(36, 7)$ & 11 & 2 & YES & YES & YES & 1064 & 813\\
$(51, 16)$ & 10 & $(5, 1)$ & 4 & 1 & YES & YES & YES & NO & 814\\
$(51, 23)$ & 9 & $(7, 3)$ & 4 & 1 & YES & YES & YES & NO & 815\\
$(51, 16)$ & 10 & $(8, 3)$ & 4 & 1 & YES & YES & YES & NO & 816\\
$(51, 23)$ & 9 & $(42, 19)$ & 9 & 3 & YES & YES & YES & NO & 817\\
$(53, 11)$ & 10 & $(3, 1)$ & 2 & 1 & YES & YES & YES & NO & 818\\
$(53, 11)$ & 10 & $(4, 1)$ & 3 & 1 & YES & YES & YES & NO & 819\\
$(53, 11)$ & 10 & $(4, 1)$ & 3 & 1 & YES & YES & YES & NO & 820\\
$(53, 15)$ & 11 & $(5, 1)$ & 4 & 1 & YES & YES & YES & NO & 821\\
$(53, 20)$ & 10 & $(5, 1)$ & 4 & 1 & YES & YES & YES & NO & 822\\
$(53, 20)$ & 10 & $(5, 1)$ & 4 & 1 & YES & YES & YES & NO & 823\\
$(53, 20)$ & 10 & $(5, 1)$ & 4 & 1 & YES & YES & YES & NO & 824\\
$(53, 24)$ & 10 & $(7, 2)$ & 4 & 1 & YES & YES & YES & NO & 825\\
$(53, 14)$ & 9 & $(10, 3)$ & 5 & 1 & YES & YES & YES & NO & 826\\
$(53, 14)$ & 9 & $(11, 3)$ & 5 & 1 & YES & YES & YES & NO & 827\\
$(53, 11)$ & 10 & $(29, 6)$ & 9 & 1 & YES & YES & YES & NO & 828\\
$(53, 24)$ & 10 & $(29, 13)$ & 8 & 1 & YES & YES & YES & NO & 829\\
$(53, 24)$ & 10 & $(51, 23)$ & 9 & 1 & YES & YES & YES & 985 & 830\\
$(53, 11)$ & 10 & $(53, 11)$ & 10 & 53 & YES & YES & YES & NO & 831\\
$(54, 17)$ & 10 & $(5, 1)$ & 4 & 1 & YES & YES & YES & NO & 832\\
$(54, 17)$ & 10 & $(7, 2)$ & 4 & 1 & YES & YES & YES & NO & 833\\
$(54, 19)$ & 10 & $(7, 3)$ & 4 & 1 & YES & YES & YES & NO & 834\\
$(54, 17)$ & 10 & $(9, 4)$ & 5 & 9 & YES & YES & YES & NO & 835\\
$(55, 16)$ & 9 & $(14, 5)$ & 6 & 1 & YES & YES & YES & NO & 836\\
$(56, 15)$ & 9 & $(3, 1)$ & 2 & 1 & YES & YES & YES & NO & 837\\
$(56, 15)$ & 9 & $(7, 2)$ & 4 & 7 & YES & YES & YES & NO & 838\\
$(56, 15)$ & 9 & $(7, 2)$ & 4 & 7 & YES & YES & YES & NO & 839\\
$(56, 23)$ & 9 & $(28, 11)$ & 8 & 28 & YES & YES & YES & NO & 840\\
$(56, 15)$ & 9 & $(36, 11)$ & 8 & 4 & YES & YES & YES & NO & 841\\
$(57, 26)$ & 11 & $(13, 2)$ & 7 & 1 & YES & YES & YES & NO & 842\\
$(58, 15)$ & 11 & $(7, 1)$ & 6 & 1 & YES & YES & YES & NO & 843\\
$(58, 15)$ & 11 & $(7, 1)$ & 6 & 1 & YES & YES & YES & NO & 844\\
$(58, 15)$ & 11 & $(23, 6)$ & 8 & 1 & YES & YES & YES & NO & 845\\
$(59, 25)$ & 9 & $(7, 3)$ & 4 & 1 & YES & YES & YES & NO & 846\\
$(59, 18)$ & 9 & $(11, 5)$ & 6 & 1 & YES & YES & YES & NO & 847\\
$(59, 18)$ & 9 & $(11, 5)$ & 6 & 1 & YES & YES & YES & NO & 848\\
$(59, 23)$ & 9 & $(27, 11)$ & 8 & 1 & YES & YES & YES & NO & 849\\
$(59, 23)$ & 9 & $(33, 13)$ & 9 & 1 & YES & YES & YES & NO & 850\\
$(59, 25)$ & 9 & $(40, 17)$ & 9 & 1 & YES & YES & YES & NO & 851\\
$(60, 23)$ & 9 & $(4, 1)$ & 3 & 4 & YES & YES & YES & NO & 852\\
$(60, 19)$ & 11 & $(6, 1)$ & 5 & 6 & YES & YES & YES & NO & 853\\
$(60, 19)$ & 11 & $(6, 1)$ & 5 & 6 & YES & YES & YES & NO & 854\\
$(60, 11)$ & 11 & $(11, 5)$ & 6 & 1 & YES & YES & YES & NO & 855\\
$(60, 11)$ & 11 & $(14, 5)$ & 6 & 2 & YES & YES & YES & NO & 856\\
$(60, 11)$ & 11 & $(36, 7)$ & 11 & 12 & YES & YES & YES & NO & 857\\
$(61, 19)$ & 10 & $(3, 1)$ & 2 & 1 & YES & YES & YES & NO & 858\\
$(61, 16)$ & 10 & $(4, 1)$ & 3 & 1 & YES & YES & YES & NO & 859\\
$(61, 24)$ & 10 & $(4, 1)$ & 3 & 1 & YES & YES & YES & NO & 860\\
$(61, 25)$ & 9 & $(5, 2)$ & 3 & 1 & YES & YES & YES & NO & 861\\
$(61, 25)$ & 9 & $(5, 2)$ & 3 & 1 & YES & YES & YES & NO & 862\\
$(61, 28)$ & 10 & $(7, 3)$ & 4 & 1 & YES & YES & YES & 768 & 863\\
$(61, 24)$ & 10 & $(9, 4)$ & 5 & 1 & YES & YES & YES & NO & 864\\
$(61, 25)$ & 9 & $(32, 13)$ & 9 & 1 & YES & YES & YES & NO & 865\\
$(62, 29)$ & 12 & $(11, 3)$ & 5 & 1 & YES & YES & YES & NO & 866\\
$(63, 26)$ & 9 & $(5, 1)$ & 4 & 1 & YES & YES & YES & NO & 867\\
$(64, 17)$ & 10 & $(9, 2)$ & 5 & 1 & YES & YES & YES & NO & 868\\
$(64, 23)$ & 9 & $(17, 6)$ & 7 & 1 & YES & YES & YES & NO & 869\\
$(65, 17)$ & 10 & $(65, 17)$ & 10 & 65 & YES & YES & YES & NO & 870\\
$(67, 26)$ & 9 & $(5, 2)$ & 3 & 1 & YES & YES & YES & NO & 871\\
$(67, 28)$ & 10 & $(5, 1)$ & 4 & 1 & YES & YES & YES & NO & 872\\
$(67, 24)$ & 10 & $(7, 2)$ & 4 & 1 & YES & YES & YES & NO & 873\\
$(67, 28)$ & 10 & $(7, 3)$ & 4 & 1 & YES & YES & YES & NO & 874\\
$(67, 16)$ & 11 & $(9, 4)$ & 5 & 1 & YES & YES & YES & NO & 875\\
$(67, 28)$ & 10 & $(13, 5)$ & 5 & 1 & YES & YES & YES & NO & 876\\
$(67, 26)$ & 9 & $(21, 8)$ & 6 & 1 & YES & YES & YES & 1016 & 877\\
$(67, 28)$ & 10 & $(43, 18)$ & 8 & 1 & YES & YES & YES & 989 & 878\\
$(68, 21)$ & 11 & $(6, 1)$ & 5 & 2 & YES & YES & YES & NO & 879\\
$(69, 20)$ & 10 & $(10, 3)$ & 5 & 1 & YES & YES & YES & NO & 880\\
$(69, 13)$ & 11 & $(60, 11)$ & 11 & 3 & YES & YES & YES & NO & 881\\
$(70, 11)$ & 11 & $(11, 5)$ & 6 & 1 & YES & YES & YES & NO & 882\\
$(70, 11)$ & 11 & $(11, 5)$ & 6 & 1 & YES & YES & YES & NO & 883\\
$(71, 13)$ & 12 & $(7, 3)$ & 4 & 1 & YES & YES & YES & NO & 884\\
$(71, 21)$ & 9 & $(8, 3)$ & 4 & 1 & YES & YES & YES & NO & 885\\
$(71, 20)$ & 10 & $(11, 3)$ & 5 & 1 & YES & YES & YES & NO & 886\\
$(71, 25)$ & 11 & $(11, 2)$ & 6 & 1 & YES & YES & YES & NO & 887\\
$(71, 21)$ & 9 & $(16, 5)$ & 7 & 1 & YES & YES & YES & NO & 888\\
$(71, 13)$ & 12 & $(19, 3)$ & 8 & 1 & YES & YES & YES & NO & 889\\
$(72, 19)$ & 10 & $(5, 1)$ & 4 & 1 & YES & YES & YES & NO & 890\\
$(72, 19)$ & 10 & $(5, 1)$ & 4 & 1 & YES & YES & YES & NO & 891\\
$(72, 19)$ & 10 & $(7, 2)$ & 4 & 1 & YES & YES & YES & NO & 892\\
$(72, 25)$ & 12 & $(7, 3)$ & 4 & 1 & YES & YES & YES & NO & 893\\
$(72, 25)$ & 12 & $(8, 3)$ & 4 & 8 & YES & YES & YES & NO & 894\\
$(72, 25)$ & 12 & $(8, 3)$ & 4 & 8 & YES & YES & YES & NO & 895\\
$(72, 19)$ & 10 & $(16, 5)$ & 7 & 8 & YES & YES & YES & NO & 896\\
$(73, 20)$ & 11 & $(2, 1)$ & 1 & 1 & YES & YES & YES & NO & 897\\
$(73, 20)$ & 11 & $(2, 1)$ & 1 & 1 & YES & YES & YES & NO & 898\\
$(73, 20)$ & 11 & $(3, 1)$ & 2 & 1 & YES & YES & YES & NO & 899\\
$(73, 20)$ & 11 & $(3, 1)$ & 2 & 1 & YES & YES & YES & NO & 900\\
$(73, 23)$ & 11 & $(3, 1)$ & 2 & 1 & YES & YES & YES & NO & 901\\
$(73, 31)$ & 10 & $(3, 1)$ & 2 & 1 & YES & YES & YES & NO & 902\\
$(73, 33)$ & 10 & $(3, 1)$ & 2 & 1 & YES & YES & YES & NO & 903\\
$(73, 28)$ & 10 & $(4, 1)$ & 3 & 1 & YES & YES & YES & NO & 904\\
$(73, 11)$ & 11 & $(6, 1)$ & 5 & 1 & YES & YES & YES & NO & 905\\
$(73, 11)$ & 11 & $(6, 1)$ & 5 & 1 & YES & YES & YES & NO & 906\\
$(73, 11)$ & 11 & $(6, 1)$ & 5 & 1 & YES & YES & YES & NO & 907\\
$(73, 19)$ & 11 & $(6, 1)$ & 5 & 1 & YES & YES & YES & NO & 908\\
$(73, 19)$ & 11 & $(6, 1)$ & 5 & 1 & YES & YES & YES & NO & 909\\
$(73, 28)$ & 10 & $(6, 1)$ & 5 & 1 & YES & YES & YES & NO & 910\\
$(73, 20)$ & 11 & $(7, 2)$ & 4 & 1 & YES & YES & YES & NO & 911\\
$(73, 31)$ & 10 & $(12, 5)$ & 5 & 1 & YES & YES & YES & NO & 912\\
$(73, 19)$ & 11 & $(19, 5)$ & 7 & 1 & YES & YES & YES & NO & 913\\
$(73, 31)$ & 10 & $(19, 8)$ & 6 & 1 & YES & YES & YES & 802 & 914\\
$(73, 33)$ & 10 & $(20, 9)$ & 7 & 1 & YES & YES & YES & NO & 915\\
$(73, 28)$ & 10 & $(34, 13)$ & 7 & 1 & YES & YES & YES & NO & 916\\
$(73, 26)$ & 11 & $(59, 21)$ & 10 & 1 & YES & YES & YES & NO & 917\\
$(73, 28)$ & 10 & $(60, 23)$ & 9 & 1 & YES & YES & YES & NO & 918\\
$(74, 29)$ & 10 & $(4, 1)$ & 3 & 2 & YES & YES & YES & NO & 919\\
$(74, 29)$ & 10 & $(4, 1)$ & 3 & 2 & YES & YES & YES & NO & 920\\
$(74, 29)$ & 10 & $(7, 3)$ & 4 & 1 & YES & YES & YES & NO & 921\\
$(74, 29)$ & 10 & $(33, 13)$ & 9 & 1 & YES & YES & YES & 1034 & 922\\
$(76, 35)$ & 12 & $(37, 17)$ & 9 & 1 & YES & YES & YES & NO & 923\\
$(77, 32)$ & 11 & $(4, 1)$ & 3 & 1 & YES & YES & YES & NO & 924\\
$(77, 16)$ & 11 & $(9, 4)$ & 5 & 1 & YES & YES & YES & NO & 925\\
$(77, 34)$ & 10 & $(25, 11)$ & 7 & 1 & YES & YES & YES & NO & 926\\
$(79, 29)$ & 9 & $(2, 1)$ & 1 & 1 & YES & YES & NO(2) & NO & 927\\
$(79, 31)$ & 10 & $(3, 1)$ & 2 & 1 & YES & YES & YES & NO & 928\\
$(79, 33)$ & 11 & $(13, 2)$ & 7 & 1 & YES & YES & YES & NO & 929\\
$(79, 33)$ & 11 & $(74, 31)$ & 9 & 1 & YES & YES & YES & NO & 930\\
$(80, 19)$ & 11 & $(3, 1)$ & 2 & 1 & YES & YES & YES & NO & 931\\
$(80, 19)$ & 11 & $(3, 1)$ & 2 & 1 & YES & YES & YES & NO & 932\\
$(82, 31)$ & 10 & $(3, 1)$ & 2 & 1 & YES & YES & YES & NO & 933\\
$(82, 31)$ & 10 & $(3, 1)$ & 2 & 1 & YES & YES & YES & NO & 934\\
$(82, 31)$ & 10 & $(82, 31)$ & 10 & 82 & YES & YES & YES & NO & 935\\
$(83, 23)$ & 10 & $(2, 1)$ & 1 & 1 & YES & YES & YES & NO & 936\\
$(83, 24)$ & 11 & $(2, 1)$ & 1 & 1 & YES & YES & YES & NO & 937\\
$(83, 24)$ & 11 & $(3, 1)$ & 2 & 1 & YES & YES & YES & NO & 938\\
$(83, 24)$ & 11 & $(3, 1)$ & 2 & 1 & YES & YES & YES & NO & 939\\
$(83, 36)$ & 10 & $(4, 1)$ & 3 & 1 & YES & YES & YES & NO & 940\\
$(83, 36)$ & 10 & $(4, 1)$ & 3 & 1 & YES & YES & YES & NO & 941\\
$(83, 36)$ & 10 & $(5, 2)$ & 3 & 1 & YES & YES & YES & NO & 942\\
$(83, 29)$ & 12 & $(11, 3)$ & 5 & 1 & YES & YES & YES & NO & 943\\
$(83, 13)$ & 11 & $(17, 6)$ & 7 & 1 & YES & YES & YES & NO & 944\\
$(83, 24)$ & 11 & $(38, 11)$ & 9 & 1 & YES & YES & YES & NO & 945\\
$(83, 29)$ & 12 & $(49, 17)$ & 11 & 1 & YES & YES & YES & 1081 & 946\\
$(84, 13)$ & 13 & $(5, 1)$ & 4 & 1 & YES & YES & YES & NO & 947\\
$(84, 37)$ & 10 & $(5, 2)$ & 3 & 1 & YES & YES & YES & NO & 948\\
$(84, 13)$ & 13 & $(7, 3)$ & 4 & 7 & YES & YES & YES & NO & 949\\
$(85, 24)$ & 11 & $(4, 1)$ & 3 & 1 & YES & YES & YES & NO & 950\\
$(85, 24)$ & 11 & $(4, 1)$ & 3 & 1 & YES & YES & YES & NO & 951\\
$(85, 24)$ & 11 & $(4, 1)$ & 3 & 1 & YES & YES & YES & NO & 952\\
$(85, 24)$ & 11 & $(5, 1)$ & 4 & 5 & YES & YES & YES & NO & 953\\
$(85, 36)$ & 10 & $(33, 14)$ & 8 & 1 & YES & YES & YES & 976 & 954\\
$(86, 27)$ & 11 & $(2, 1)$ & 1 & 2 & YES & YES & YES & NO & 955\\
$(86, 35)$ & 11 & $(5, 2)$ & 3 & 1 & YES & YES & YES & NO & 956\\
$(86, 35)$ & 11 & $(9, 4)$ & 5 & 1 & YES & YES & YES & NO & 957\\
$(86, 23)$ & 11 & $(41, 11)$ & 8 & 1 & YES & YES & YES & NO & 958\\
$(86, 33)$ & 11 & $(86, 33)$ & 11 & 86 & YES & YES & YES & NO & 959\\
$(87, 20)$ & 12 & $(8, 1)$ & 7 & 1 & YES & YES & YES & NO & 960\\
$(87, 20)$ & 12 & $(48, 11)$ & 9 & 3 & YES & YES & YES & NO & 961\\
$(88, 31)$ & 12 & $(6, 1)$ & 5 & 2 & YES & YES & YES & NO & 962\\
$(88, 31)$ & 12 & $(54, 19)$ & 10 & 2 & YES & YES & YES & 1035 & 963\\
$(89, 24)$ & 10 & $(2, 1)$ & 1 & 1 & YES & YES & NO(2) & NO & 964\\
$(89, 26)$ & 10 & $(3, 1)$ & 2 & 1 & YES & YES & NO(2) & NO & 965\\
$(89, 26)$ & 10 & $(3, 1)$ & 2 & 1 & YES & YES & NO(2) & NO & 966\\
$(89, 40)$ & 11 & $(3, 1)$ & 2 & 1 & YES & YES & YES & NO & 967\\
$(89, 40)$ & 11 & $(3, 1)$ & 2 & 1 & YES & YES & YES & NO & 968\\
$(89, 26)$ & 10 & $(4, 1)$ & 3 & 1 & YES & YES & NO(2) & NO & 969\\
$(89, 35)$ & 11 & $(6, 1)$ & 5 & 1 & YES & YES & YES & NO & 970\\
$(89, 35)$ & 11 & $(6, 1)$ & 5 & 1 & YES & YES & YES & NO & 971\\
$(91, 27)$ & 10 & $(2, 1)$ & 1 & 1 & YES & YES & NO(2) & NO & 972\\
$(91, 29)$ & 13 & $(7, 1)$ & 6 & 7 & YES & YES & YES & NO & 973\\
$(91, 29)$ & 13 & $(13, 4)$ & 6 & 13 & YES & YES & YES & NO & 974\\
$(92, 39)$ & 10 & $(2, 1)$ & 1 & 2 & YES & YES & YES & NO & 975\\
$(92, 39)$ & 10 & $(26, 11)$ & 7 & 2 & YES & YES & YES & 954 & 976\\
$(93, 26)$ & 10 & $(7, 3)$ & 4 & 1 & YES & YES & YES & NO & 977\\
$(93, 34)$ & 10 & $(41, 15)$ & 8 & 1 & YES & YES & NO(2) & NO & 978\\
$(94, 35)$ & 11 & $(2, 1)$ & 1 & 2 & YES & YES & YES & NO & 979\\
$(94, 39)$ & 10 & $(3, 1)$ & 2 & 1 & YES & YES & NO(2) & NO & 980\\
$(94, 35)$ & 11 & $(5, 2)$ & 3 & 1 & YES & YES & YES & NO & 981\\
$(94, 35)$ & 11 & $(43, 16)$ & 9 & 1 & YES & YES & YES & NO & 982\\
$(94, 29)$ & 13 & $(68, 21)$ & 11 & 2 & YES & YES & YES & 1059 & 983\\
$(95, 44)$ & 12 & $(4, 1)$ & 3 & 1 & YES & YES & YES & NO & 984\\
$(95, 43)$ & 11 & $(20, 9)$ & 7 & 5 & YES & YES & YES & 830 & 985\\
$(95, 43)$ & 11 & $(42, 19)$ & 9 & 1 & YES & YES & YES & NO & 986\\
$(97, 20)$ & 12 & $(4, 1)$ & 3 & 1 & YES & YES & YES & NO & 987\\
$(98, 41)$ & 10 & $(5, 1)$ & 4 & 1 & YES & YES & YES & NO & 988\\
$(98, 41)$ & 10 & $(12, 5)$ & 5 & 2 & YES & YES & YES & 878 & 989\\
$(98, 29)$ & 10 & $(36, 11)$ & 8 & 2 & YES & YES & YES & NO & 990\\
$(99, 46)$ & 12 & $(5, 1)$ & 4 & 1 & YES & YES & YES & NO & 991\\
$(99, 31)$ & 13 & $(99, 31)$ & 13 & 99 & YES & YES & YES & NO & 992\\
$(100, 29)$ & 11 & $(4, 1)$ & 3 & 4 & YES & YES & YES & NO & 993\\
$(100, 29)$ & 11 & $(4, 1)$ & 3 & 4 & YES & YES & YES & NO & 994\\
$(101, 41)$ & 12 & $(7, 3)$ & 4 & 1 & YES & YES & YES & NO & 995\\
$(101, 41)$ & 12 & $(13, 5)$ & 5 & 1 & YES & YES & YES & NO & 996\\
$(103, 37)$ & 10 & $(8, 3)$ & 4 & 1 & YES & YES & YES & 1040 & 997\\
$(103, 47)$ & 12 & $(57, 26)$ & 11 & 1 & YES & YES & YES & NO & 998\\
$(104, 47)$ & 11 & $(2, 1)$ & 1 & 2 & YES & YES & YES & NO & 999\\
$(104, 47)$ & 11 & $(11, 5)$ & 6 & 1 & YES & YES & YES & NO & 1000\\
$(104, 41)$ & 12 & $(12, 5)$ & 5 & 4 & YES & YES & YES & NO & 1001\\
$(104, 47)$ & 11 & $(42, 19)$ & 9 & 2 & YES & YES & YES & 1020 & 1002\\
$(105, 44)$ & 10 & $(2, 1)$ & 1 & 1 & YES & YES & YES & NO & 1003\\
$(106, 39)$ & 11 & $(106, 39)$ & 11 & 106 & YES & YES & YES & NO & 1004\\
$(108, 41)$ & 10 & $(2, 1)$ & 1 & 2 & YES & YES & YES & NO & 1005\\
$(108, 41)$ & 10 & $(7, 3)$ & 4 & 1 & YES & YES & YES & NO & 1006\\
$(109, 23)$ & 12 & $(3, 1)$ & 2 & 1 & YES & YES & YES & NO & 1007\\
$(109, 40)$ & 10 & $(3, 1)$ & 2 & 1 & YES & YES & YES & NO & 1008\\
$(109, 40)$ & 10 & $(3, 1)$ & 2 & 1 & YES & YES & YES & NO & 1009\\
$(109, 23)$ & 12 & $(33, 7)$ & 8 & 1 & YES & YES & YES & NO & 1010\\
$(111, 32)$ & 13 & $(3, 1)$ & 2 & 3 & YES & YES & YES & NO & 1011\\
$(112, 33)$ & 12 & $(3, 1)$ & 2 & 1 & NO & YES & YES & NO & 1012\\
$(113, 32)$ & 13 & $(2, 1)$ & 1 & 1 & YES & YES & YES & NO & 1013\\
$(113, 20)$ & 13 & $(8, 1)$ & 7 & 1 & YES & YES & YES & NO & 1014\\
$(115, 44)$ & 10 & $(2, 1)$ & 1 & 1 & YES & YES & YES & NO & 1015\\
$(115, 44)$ & 10 & $(5, 2)$ & 3 & 5 & YES & YES & YES & 877 & 1016\\
$(115, 18)$ & 12 & $(6, 1)$ & 5 & 1 & NO & YES & YES & NO & 1017\\
$(115, 52)$ & 11 & $(9, 4)$ & 5 & 1 & YES & YES & YES & NO & 1018\\
$(115, 24)$ & 12 & $(14, 3)$ & 6 & 1 & YES & YES & YES & NO & 1019\\
$(115, 52)$ & 11 & $(31, 14)$ & 8 & 1 & YES & YES & YES & 1002 & 1020\\
$(116, 35)$ & 12 & $(3, 1)$ & 2 & 1 & NO & YES & YES & NO & 1021\\
$(118, 51)$ & 12 & $(5, 2)$ & 3 & 1 & YES & YES & YES & NO & 1022\\
$(119, 37)$ & 11 & $(3, 1)$ & 2 & 1 & YES & YES & NO(2) & NO & 1023\\
$(119, 45)$ & 11 & $(8, 3)$ & 4 & 1 & YES & YES & YES & NO & 1024\\
$(119, 37)$ & 11 & $(45, 14)$ & 9 & 1 & YES & YES & NO(2) & NO & 1025\\
$(120, 53)$ & 11 & $(2, 1)$ & 1 & 2 & NO & YES & YES & NO & 1026\\
$(120, 43)$ & 11 & $(3, 1)$ & 2 & 3 & YES & YES & YES & NO & 1027\\
$(121, 50)$ & 10 & $(2, 1)$ & 1 & 1 & NO & YES & NO(2) & NO & 1028\\
$(121, 35)$ & 12 & $(45, 13)$ & 10 & 1 & YES & YES & YES & 1041 & 1029\\
$(121, 32)$ & 11 & $(53, 14)$ & 9 & 1 & YES & YES & YES & 1052 & 1030\\
$(122, 37)$ & 11 & $(3, 1)$ & 2 & 1 & NO & YES & NO(2) & NO & 1031\\
$(124, 37)$ & 12 & $(7, 2)$ & 4 & 1 & YES & YES & YES & NO & 1032\\
$(125, 24)$ & 13 & $(3, 1)$ & 2 & 1 & YES & YES & YES & NO & 1033\\
$(125, 49)$ & 11 & $(5, 2)$ & 3 & 5 & YES & YES & YES & 922 & 1034\\
$(125, 44)$ & 12 & $(17, 6)$ & 7 & 1 & YES & YES & YES & 963 & 1035\\
$(125, 44)$ & 12 & $(71, 25)$ & 11 & 1 & YES & YES & YES & NO & 1036\\
$(126, 55)$ & 11 & $(2, 1)$ & 1 & 2 & YES & YES & YES & NO & 1037\\
$(127, 46)$ & 12 & $(5, 1)$ & 4 & 1 & YES & YES & YES & NO & 1038\\
$(127, 54)$ & 12 & $(7, 3)$ & 4 & 1 & YES & YES & YES & NO & 1039\\
$(128, 47)$ & 10 & $(3, 1)$ & 2 & 1 & YES & YES & YES & 997 & 1040\\
$(128, 37)$ & 12 & $(38, 11)$ & 9 & 2 & YES & YES & YES & 1029 & 1041\\
$(131, 39)$ & 11 & $(10, 3)$ & 5 & 1 & YES & YES & YES & NO & 1042\\
$(132, 23)$ & 13 & $(3, 1)$ & 2 & 3 & YES & YES & YES & NO & 1043\\
$(132, 47)$ & 12 & $(4, 1)$ & 3 & 4 & YES & YES & YES & NO & 1044\\
$(134, 35)$ & 13 & $(65, 17)$ & 10 & 1 & YES & YES & YES & NO & 1045\\
$(137, 53)$ & 11 & $(137, 53)$ & 11 & 137 & YES & YES & YES & NO & 1046\\
$(139, 53)$ & 12 & $(2, 1)$ & 1 & 1 & NO & YES & YES & NO & 1047\\
$(139, 61)$ & 11 & $(2, 1)$ & 1 & 1 & NO & YES & YES & NO & 1048\\
$(140, 37)$ & 11 & $(2, 1)$ & 1 & 2 & YES & YES & YES & NO & 1049\\
$(140, 37)$ & 11 & $(5, 1)$ & 4 & 5 & YES & YES & NO(2) & NO & 1050\\
$(140, 37)$ & 11 & $(19, 5)$ & 7 & 1 & YES & YES & YES & NO & 1051\\
$(140, 37)$ & 11 & $(34, 9)$ & 8 & 2 & YES & YES & YES & 1030 & 1052\\
$(140, 37)$ & 11 & $(53, 14)$ & 9 & 1 & YES & YES & YES & NO & 1053\\
$(141, 59)$ & 11 & $(50, 21)$ & 8 & 1 & YES & YES & YES & NO & 1054\\
$(142, 39)$ & 11 & $(11, 3)$ & 5 & 1 & YES & YES & YES & NO & 1055\\
$(143, 54)$ & 12 & $(2, 1)$ & 1 & 1 & YES & YES & YES & NO & 1056\\
$(146, 61)$ & 12 & $(6, 1)$ & 5 & 2 & YES & YES & YES & NO & 1057\\
$(149, 42)$ & 12 & $(3, 1)$ & 2 & 1 & NO & YES & YES & NO & 1058\\
$(149, 46)$ & 13 & $(13, 4)$ & 6 & 1 & YES & YES & YES & 983 & 1059\\
$(150, 47)$ & 13 & $(2, 1)$ & 1 & 2 & YES & YES & YES & NO & 1060\\
$(151, 53)$ & 12 & $(2, 1)$ & 1 & 1 & YES & YES & YES & NO & 1061\\
$(156, 73)$ & 14 & $(3, 1)$ & 2 & 3 & YES & YES & YES & NO & 1062\\
$(160, 31)$ & 15 & $(2, 1)$ & 1 & 2 & YES & YES & YES & NO & 1063\\
$(160, 31)$ & 15 & $(6, 1)$ & 5 & 2 & YES & YES & YES & 813 & 1064\\
$(160, 31)$ & 15 & $(36, 7)$ & 11 & 4 & YES & YES & YES & NO & 1065\\
$(160, 31)$ & 15 & $(160, 31)$ & 15 & 160 & YES & YES & YES & NO & 1066\\
$(161, 51)$ & 13 & $(19, 6)$ & 8 & 1 & YES & YES & YES & NO & 1067\\
$(165, 64)$ & 11 & $(2, 1)$ & 1 & 1 & NO & YES & NO(2) & NO & 1068\\
$(167, 58)$ & 14 & $(2, 1)$ & 1 & 1 & YES & YES & YES & NO & 1069\\
$(167, 58)$ & 14 & $(72, 25)$ & 12 & 1 & YES & YES & YES & NO & 1070\\
$(169, 71)$ & 11 & $(2, 1)$ & 1 & 1 & NO & YES & YES & NO & 1071\\
$(170, 29)$ & 15 & $(7, 1)$ & 6 & 1 & YES & YES & YES & NO & 1072\\
$(191, 31)$ & 16 & $(13, 2)$ & 7 & 1 & YES & YES & YES & NO & 1073\\
$(192, 31)$ & 16 & $(37, 6)$ & 11 & 1 & YES & YES & YES & NO & 1074\\
$(192, 31)$ & 16 & $(192, 31)$ & 16 & 192 & YES & YES & YES & NO & 1075\\
$(206, 73)$ & 12 & $(2, 1)$ & 1 & 2 & YES & YES & YES & NO & 1076\\
$(206, 73)$ & 12 & $(6, 1)$ & 5 & 2 & YES & YES & YES & NO & 1077\\
$(208, 87)$ & 12 & $(4, 1)$ & 3 & 4 & YES & YES & YES & NO & 1078\\
$(208, 87)$ & 12 & $(19, 8)$ & 6 & 1 & YES & YES & YES & NO & 1079\\
$(208, 37)$ & 13 & $(45, 8)$ & 9 & 1 & YES & YES & YES & NO & 1080\\
$(209, 73)$ & 14 & $(3, 1)$ & 2 & 1 & YES & YES & YES & 946 & 1081\\
$(210, 29)$ & 17 & $(2, 1)$ & 1 & 2 & YES & YES & YES & NO & 1082\\
$(210, 29)$ & 17 & $(36, 5)$ & 11 & 6 & YES & YES & YES & NO & 1083\\
$(223, 70)$ & 13 & $(2, 1)$ & 1 & 1 & YES & YES & YES & NO & 1084\\
$(239, 32)$ & 17 & $(7, 1)$ & 6 & 1 & YES & YES & YES & NO & 1085\\
$(267, 98)$ & 12 & $(5, 2)$ & 3 & 1 & YES & YES & YES & NO & 1086\\
$(274, 115)$ & 12 & $(2, 1)$ & 1 & 2 & YES & YES & YES & NO & 1087\\
$(286, 105)$ & 12 & $(6, 1)$ & 5 & 2 & YES & YES & YES & NO & 1088\\
$(293, 123)$ & 12 & $(3, 1)$ & 2 & 1 & YES & YES & YES & NO & 1089\\
$(295, 87)$ & 13 & $(3, 1)$ & 2 & 1 & YES & YES & YES & NO & 1090\\
$(a; 6, 1, 0; 49)$ & 11 & $(7, 3)$ & 4 & 7 & YES & YES & YES & NO & 1091\\
$(b; 0, 3, 0; 29)$ & 8 & $(7, 2)$ & 4 & 1 & YES & YES & YES & NO & 1092\\
$(b; 0, 3, 0; 29)$ & 8 & $(9, 2)$ & 5 & 1 & YES & YES & YES & NO & 1093\\
$(b; 0, 4, 0; 34)$ & 9 & $(4, 1)$ & 3 & 2 & YES & YES & YES & NO & 1094\\
$(b; 1, 0, 0; 5)$ & 6 & $(19, 6)$ & 8 & 1 & YES & YES & YES & NO & 1095\\
$(b; 1, 3, 0; 41)$ & 9 & $(4, 1)$ & 3 & 1 & YES & YES & YES & NO & 1096\\
$(b; 2, 2, 0; 44)$ & 9 & $(4, 1)$ & 3 & 4 & YES & YES & YES & NO & 1097\\
$(b; 3, 0, 0; 16)$ & 8 & $(7, 3)$ & 4 & 1 & YES & YES & YES & NO & 1098\\
$(b; 3, 1, 0; 43)$ & 9 & $(3, 1)$ & 2 & 1 & YES & YES & YES & NO & 1099\\
$(b; 4, 0, 0; 38)$ & 9 & $(4, 1)$ & 3 & 2 & YES & YES & YES & NO & 1100\\
$(c; 0, 0, 0; 4)$ & 4 & $(28, 13)$ & 9 & 4 & YES & YES & YES & NO & 1101\\
$(c; 0, 0, 0; 4)$ & 4 & $(34, 15)$ & 8 & 2 & YES & YES & YES & NO & 1102\\
$(c; 0, 0, 0; 4)$ & 4 & $(115, 24)$ & 12 & 1 & YES & YES & YES & NO & 1103\\
$(c; 0, 1, 0; 11)$ & 5 & $(17, 7)$ & 6 & 1 & YES & YES & NO(2) & NO & 1104\\
$(c; 0, 2, 0; 7)$ & 6 & $(16, 7)$ & 6 & 1 & YES & YES & YES & NO & 1105\\
$(c; 0, 2, 0; 7)$ & 6 & $(35, 13)$ & 8 & 7 & YES & YES & YES & NO & 1106\\
$(c; 0, 2, 0; 7)$ & 6 & $(56, 15)$ & 9 & 7 & YES & YES & YES & NO & 1107\\
$(c; 0, 2, 1; 19)$ & 7 & $(13, 5)$ & 5 & 1 & YES & YES & YES & NO & 1108\\
$(c; 0, 2, 2; 6)$ & 8 & $(21, 5)$ & 8 & 3 & YES & YES & YES & NO & 1109\\
$(c; 0, 4, 0; 10)$ & 8 & $(7, 2)$ & 4 & 1 & YES & YES & YES & NO & 1110\\
$(c; 0, 4, 0; 10)$ & 8 & $(9, 2)$ & 5 & 1 & YES & YES & YES & NO & 1111\\
$(c; 0, 5, 3; 47)$ & 12 & $(4, 1)$ & 3 & 1 & YES & YES & YES & NO & 1112\\
$(d; 0, 0, 0; 5)$ & 5 & $(29, 11)$ & 7 & 1 & YES & YES & YES & NO & 1113\\
$(d; 0, 0, 0; 5)$ & 5 & $(41, 11)$ & 8 & 1 & YES & YES & NO(2) & NO & 1114\\
$(d; 0, 0, 3; 22)$ & 8 & $(16, 5)$ & 7 & 2 & YES & YES & YES & NO & 1115\\
$(d; 0, 3, 0; 8)$ & 8 & $(7, 2)$ & 4 & 1 & YES & YES & YES & NO & 1116\\
$(d; 0, 4, 3; 42)$ & 12 & $(3, 1)$ & 2 & 3 & YES & YES & YES & NO & 1117\\
$(f; 0, 0, 0; 6)$ & 4 & $(52, 15)$ & 11 & 2 & YES & YES & YES & NO & 1118\\
$(f; 0, 0, 0; 6)$ & 4 & $(53, 15)$ & 11 & 1 & YES & YES & YES & NO & 1119\\
$(f; 0, 0, 0; 6)$ & 4 & $(72, 25)$ & 12 & 6 & YES & YES & YES & NO & 1120\\
$(f; 0, 0, 0; 6)$ & 4 & $(84, 13)$ & 13 & 6 & YES & YES & YES & NO & 1121\\
$(f; 0, 1, 0; 7)$ & 5 & $(18, 7)$ & 6 & 1 & YES & YES & NO(2) & NO & 1122\\
$(f; 0, 1, 0; 7)$ & 5 & $(22, 9)$ & 7 & 1 & YES & YES & NO(2) & NO & 1123\\
$(f; 0, 1, 0; 7)$ & 5 & $(23, 9)$ & 7 & 1 & YES & YES & NO(2) & NO & 1124\\
$(f; 0, 1, 0; 7)$ & 5 & $(27, 10)$ & 7 & 1 & YES & YES & YES & NO & 1125\\
$(g; 0, 0, 3; 40)$ & 9 & $(3, 1)$ & 2 & 1 & YES & YES & YES & NO & 1126\\
$(g; 0, 3, 0; 34)$ & 9 & $(3, 1)$ & 2 & 1 & YES & YES & YES & NO & 1127\\
$(g; 3, 0, 0; 23)$ & 9 & $(3, 1)$ & 2 & 1 & YES & YES & NO(2) & NO & 1128\\
$(j; 0, 0, 0; 8)$ & 5 & $(40, 17)$ & 9 & 8 & YES & YES & YES & NO & 1129\\
$(j; 0, 1, 0; 10)$ & 6 & $(27, 11)$ & 8 & 1 & YES & YES & YES & NO & 1130
\end{longtable}
\subsection{2 chains, $K^2 = 4$}
\begin{longtable}{|c|c|c|c|c|c|c|c|c|c|}
\hline
\multicolumn{10}{|c|}{2 chains, $K^2 = 4$}\\
\hline
$(n,a)$ & Length & $(n,a)$ & Length & GCD & Nef & $\mathbb Q$-ef & Obstruction 0 & WH & Index\\
\hline
\endfirsthead

\hline
$(n,a)$ & Length & $(n,a)$ & Length & GCD & Nef & $\mathbb Q$-ef & Obstruction 0 & WH & Index\\
\hline
\endhead
\hline
\endfoot

$(25, 11)$ & 7 & $(25, 11)$ & 7 & 25 & YES & YES & YES & NO & 1131\\
$(31, 14)$ & 8 & $(31, 13)$ & 7 & 31 & YES & YES & YES & NO & 1132\\
$(36, 11)$ & 8 & $(31, 14)$ & 8 & 1 & YES & YES & YES & NO & 1133\\
$(39, 11)$ & 9 & $(16, 3)$ & 7 & 1 & YES & YES & NO(2) & NO & 1134\\
$(39, 11)$ & 9 & $(16, 3)$ & 7 & 1 & YES & YES & NO(2) & NO & 1135\\
$(39, 11)$ & 9 & $(27, 5)$ & 8 & 3 & YES & YES & YES & NO & 1136\\
$(39, 11)$ & 9 & $(27, 5)$ & 8 & 3 & YES & YES & YES & NO & 1137\\
$(41, 15)$ & 8 & $(29, 13)$ & 8 & 1 & YES & YES & YES & NO & 1138\\
$(41, 15)$ & 8 & $(39, 11)$ & 9 & 1 & YES & YES & YES & NO & 1139\\
$(49, 22)$ & 9 & $(28, 11)$ & 8 & 7 & YES & YES & YES & NO & 1140\\
$(61, 16)$ & 10 & $(29, 11)$ & 7 & 1 & YES & YES & YES & NO & 1141\\
$(65, 18)$ & 9 & $(17, 8)$ & 9 & 1 & YES & YES & YES & NO & 1142\\
$(65, 18)$ & 9 & $(52, 15)$ & 11 & 13 & YES & YES & YES & NO & 1143\\
$(73, 21)$ & 14 & $(22, 3)$ & 9 & 1 & YES & YES & YES & NO & 1144\\
$(76, 31)$ & 10 & $(11, 4)$ & 5 & 1 & YES & YES & YES & NO & 1145\\
$(76, 29)$ & 9 & $(17, 8)$ & 9 & 1 & YES & YES & YES & NO & 1146\\
$(79, 24)$ & 10 & $(18, 7)$ & 6 & 1 & YES & YES & NO(2) & NO & 1147\\
$(89, 39)$ & 11 & $(9, 2)$ & 5 & 1 & YES & YES & YES & NO & 1148\\
$(89, 39)$ & 11 & $(11, 2)$ & 6 & 1 & YES & YES & YES & NO & 1149\\
$(94, 41)$ & 10 & $(37, 10)$ & 8 & 1 & YES & YES & YES & NO & 1150\\
$(96, 17)$ & 12 & $(26, 5)$ & 9 & 2 & YES & YES & YES & NO & 1151\\
$(98, 19)$ & 13 & $(19, 5)$ & 7 & 1 & YES & YES & YES & NO & 1152\\
$(103, 27)$ & 11 & $(14, 5)$ & 6 & 1 & YES & YES & YES & NO & 1153\\
$(107, 38)$ & 11 & $(18, 7)$ & 6 & 1 & YES & YES & NO(2) & NO & 1154\\
$(109, 16)$ & 13 & $(23, 6)$ & 8 & 1 & YES & YES & YES & NO & 1155\\
$(113, 17)$ & 13 & $(109, 16)$ & 13 & 1 & YES & YES & YES & NO & 1156\\
$(117, 41)$ & 13 & $(13, 2)$ & 7 & 13 & YES & YES & YES & NO & 1157\\
$(117, 31)$ & 11 & $(49, 15)$ & 9 & 1 & YES & YES & YES & NO & 1158\\
$(128, 37)$ & 12 & $(32, 9)$ & 8 & 32 & YES & YES & YES & NO & 1159\\
$(128, 37)$ & 12 & $(73, 21)$ & 14 & 1 & YES & YES & YES & NO & 1160\\
$(138, 61)$ & 12 & $(10, 3)$ & 5 & 2 & YES & YES & YES & NO & 1161\\
$(145, 42)$ & 12 & $(23, 7)$ & 7 & 1 & YES & YES & NO(2) & NO & 1162\\
$(151, 45)$ & 12 & $(11, 4)$ & 5 & 1 & YES & YES & YES & NO & 1163\\
$(153, 40)$ & 12 & $(5, 1)$ & 4 & 1 & YES & YES & YES & NO & 1164\\
$(157, 58)$ & 11 & $(11, 5)$ & 6 & 1 & YES & YES & YES & NO & 1165\\
$(157, 28)$ & 13 & $(41, 7)$ & 11 & 1 & YES & YES & YES & NO & 1166\\
$(163, 64)$ & 13 & $(74, 29)$ & 10 & 1 & YES & YES & YES & NO & 1167\\
$(164, 61)$ & 12 & $(4, 1)$ & 3 & 4 & YES & YES & NO(2) & NO & 1168\\
$(169, 70)$ & 11 & $(23, 7)$ & 7 & 1 & YES & YES & YES & NO & 1169\\
$(175, 67)$ & 11 & $(2, 1)$ & 1 & 1 & YES & YES & NO(2) & NO & 1170\\
$(183, 67)$ & 11 & $(7, 2)$ & 4 & 1 & YES & YES & NO(2) & NO & 1171\\
$(183, 38)$ & 13 & $(9, 4)$ & 5 & 3 & YES & YES & YES & NO & 1172\\
$(187, 79)$ & 11 & $(3, 1)$ & 2 & 1 & YES & YES & NO(2) & NO & 1173\\
$(187, 79)$ & 11 & $(3, 1)$ & 2 & 1 & YES & YES & NO(2) & NO & 1174\\
$(191, 75)$ & 14 & $(4, 1)$ & 3 & 1 & YES & YES & YES & NO & 1175\\
$(193, 53)$ & 12 & $(4, 1)$ & 3 & 1 & YES & YES & YES & NO & 1176\\
$(208, 37)$ & 13 & $(41, 7)$ & 11 & 1 & YES & YES & YES & NO & 1177\\
$(211, 41)$ & 16 & $(5, 2)$ & 3 & 1 & YES & YES & YES & NO & 1178\\
$(211, 41)$ & 16 & $(13, 2)$ & 7 & 1 & YES & YES & YES & NO & 1179\\
$(211, 41)$ & 16 & $(98, 19)$ & 13 & 1 & YES & YES & YES & NO & 1180\\
$(219, 65)$ & 12 & $(49, 15)$ & 9 & 1 & YES & YES & YES & NO & 1181\\
$(223, 54)$ & 15 & $(3, 1)$ & 2 & 1 & YES & YES & NO(3) & NO & 1182\\
$(227, 67)$ & 12 & $(17, 5)$ & 6 & 1 & YES & YES & NO(2) & NO & 1183\\
$(236, 65)$ & 12 & $(23, 7)$ & 7 & 1 & YES & YES & YES & NO & 1184\\
$(237, 85)$ & 12 & $(5, 1)$ & 4 & 1 & YES & YES & NO(2) & NO & 1185\\
$(238, 107)$ & 14 & $(2, 1)$ & 1 & 2 & YES & YES & YES & NO & 1186\\
$(238, 107)$ & 14 & $(109, 49)$ & 12 & 1 & YES & YES & YES & NO & 1187\\
$(243, 110)$ & 13 & $(3, 1)$ & 2 & 3 & YES & YES & YES & NO & 1188\\
$(243, 86)$ & 13 & $(6, 1)$ & 5 & 3 & YES & YES & NO(2) & NO & 1189\\
$(243, 110)$ & 13 & $(31, 14)$ & 8 & 1 & YES & YES & YES & NO & 1190\\
$(244, 33)$ & 16 & $(163, 22)$ & 14 & 1 & YES & YES & YES & NO & 1191\\
$(252, 41)$ & 17 & $(5, 2)$ & 3 & 1 & YES & YES & YES & NO & 1192\\
$(252, 107)$ & 13 & $(8, 3)$ & 4 & 4 & YES & YES & YES & NO & 1193\\
$(252, 107)$ & 13 & $(9, 4)$ & 5 & 9 & YES & YES & YES & NO & 1194\\
$(263, 93)$ & 14 & $(17, 6)$ & 7 & 1 & YES & YES & YES & NO & 1195\\
$(263, 82)$ & 14 & $(93, 29)$ & 12 & 1 & YES & YES & NO(3) & NO & 1196\\
$(265, 119)$ & 14 & $(265, 119)$ & 14 & 265 & YES & YES & YES & NO & 1197\\
$(277, 121)$ & 14 & $(2, 1)$ & 1 & 1 & YES & YES & YES & NO & 1198\\
$(286, 89)$ & 15 & $(5, 1)$ & 4 & 1 & YES & YES & YES & NO & 1199\\
$(287, 53)$ & 14 & $(26, 5)$ & 9 & 1 & YES & YES & YES & NO & 1200\\
$(289, 90)$ & 14 & $(4, 1)$ & 3 & 1 & YES & YES & NO(2) & NO & 1201\\
$(297, 79)$ & 15 & $(19, 5)$ & 7 & 1 & YES & YES & YES & NO & 1202\\
$(319, 95)$ & 15 & $(37, 11)$ & 8 & 1 & YES & YES & YES & NO & 1203\\
$(325, 137)$ & 14 & $(2, 1)$ & 1 & 1 & YES & YES & YES & NO & 1204\\
$(325, 137)$ & 14 & $(2, 1)$ & 1 & 1 & YES & YES & YES & NO & 1205\\
$(326, 71)$ & 14 & $(13, 4)$ & 6 & 1 & YES & YES & YES & NO & 1206\\
$(328, 121)$ & 14 & $(3, 1)$ & 2 & 1 & YES & YES & YES & NO & 1207\\
$(328, 63)$ & 16 & $(4, 1)$ & 3 & 4 & YES & YES & YES & NO & 1208\\
$(328, 121)$ & 14 & $(27, 10)$ & 7 & 1 & YES & YES & YES & NO & 1209\\
$(328, 63)$ & 16 & $(177, 34)$ & 15 & 1 & YES & YES & YES & NO & 1210\\
$(332, 89)$ & 13 & $(23, 7)$ & 7 & 1 & YES & YES & YES & NO & 1211\\
$(346, 95)$ & 15 & $(9, 1)$ & 8 & 1 & YES & YES & YES & NO & 1212\\
$(356, 93)$ & 15 & $(23, 6)$ & 8 & 1 & YES & YES & YES & NO & 1213\\
$(368, 141)$ & 13 & $(7, 3)$ & 4 & 1 & YES & YES & YES & NO & 1214\\
$(386, 75)$ & 17 & $(3, 1)$ & 2 & 1 & YES & YES & YES & NO & 1215\\
$(386, 75)$ & 17 & $(211, 41)$ & 16 & 1 & YES & YES & YES & NO & 1216\\
$(392, 53)$ & 20 & $(8, 1)$ & 7 & 8 & YES & YES & YES & NO & 1217\\
$(398, 147)$ & 13 & $(306, 113)$ & 13 & 2 & YES & YES & YES & NO & 1218\\
$(411, 109)$ & 14 & $(5, 1)$ & 4 & 1 & YES & YES & YES & NO & 1219\\
$(417, 161)$ & 13 & $(8, 3)$ & 4 & 1 & YES & YES & YES & NO & 1220\\
$(417, 161)$ & 13 & $(41, 16)$ & 8 & 1 & YES & YES & YES & NO & 1221\\
$(455, 192)$ & 13 & $(7, 3)$ & 4 & 7 & YES & YES & YES & NO & 1222\\
$(469, 76)$ & 18 & $(2, 1)$ & 1 & 1 & YES & YES & YES & NO & 1223\\
$(469, 76)$ & 18 & $(3, 1)$ & 2 & 1 & YES & YES & YES & NO & 1224\\
$(494, 111)$ & 16 & $(2, 1)$ & 1 & 2 & YES & YES & YES & NO & 1225\\
$(544, 223)$ & 14 & $(56, 23)$ & 9 & 8 & YES & YES & YES & NO & 1226\\
$(548, 227)$ & 14 & $(5, 2)$ & 3 & 1 & YES & YES & YES & NO & 1227\\
$(551, 240)$ & 14 & $(5, 2)$ & 3 & 1 & YES & YES & YES & NO & 1228\\
$(574, 155)$ & 14 & $(311, 84)$ & 13 & 1 & YES & YES & YES & NO & 1229\\
$(589, 80)$ & 19 & $(9, 1)$ & 8 & 1 & YES & YES & YES & NO & 1230\\
$(606, 115)$ & 16 & $(38, 7)$ & 9 & 2 & YES & YES & YES & NO & 1231\\
$(607, 256)$ & 14 & $(8, 3)$ & 4 & 1 & YES & YES & YES & NO & 1232\\
$(621, 140)$ & 15 & $(7, 3)$ & 4 & 1 & YES & YES & YES & NO & 1233\\
$(623, 241)$ & 14 & $(517, 200)$ & 14 & 1 & YES & YES & YES & NO & 1234\\
$(631, 191)$ & 15 & $(53, 16)$ & 10 & 1 & YES & YES & YES & NO & 1235\\
$(643, 196)$ & 15 & $(33, 10)$ & 8 & 1 & YES & YES & YES & NO & 1236\\
$(666, 101)$ & 18 & $(3, 1)$ & 2 & 3 & YES & YES & YES & NO & 1237\\
$(772, 279)$ & 15 & $(4, 1)$ & 3 & 4 & YES & YES & YES & NO & 1238\\
$(824, 227)$ & 15 & $(236, 65)$ & 12 & 4 & YES & YES & YES & NO & 1239\\
$(835, 323)$ & 15 & $(3, 1)$ & 2 & 1 & YES & YES & YES & NO & 1240\\
$(843, 326)$ & 15 & $(3, 1)$ & 2 & 3 & YES & YES & YES & NO & 1241\\
$(870, 269)$ & 16 & $(42, 13)$ & 9 & 6 & YES & YES & YES & NO & 1242\\
$(907, 269)$ & 16 & $(3, 1)$ & 2 & 1 & YES & YES & YES & NO & 1243\\
$(907, 335)$ & 15 & $(5, 1)$ & 4 & 1 & YES & YES & YES & NO & 1244\\
$(923, 259)$ & 16 & $(11, 3)$ & 5 & 1 & YES & YES & YES & NO & 1245\\
$(945, 388)$ & 15 & $(945, 388)$ & 15 & 945 & YES & YES & YES & NO & 1246\\
$(985, 289)$ & 16 & $(2, 1)$ & 1 & 1 & YES & YES & YES & NO & 1247\\
$(1057, 321)$ & 16 & $(3, 1)$ & 2 & 1 & YES & YES & YES & NO & 1248\\
$(1058, 409)$ & 15 & $(2, 1)$ & 1 & 2 & YES & YES & YES & NO & 1249\\
$(1212, 217)$ & 18 & $(1212, 217)$ & 18 & 1212 & YES & YES & YES & NO & 1250\\
$(1218, 463)$ & 15 & $(50, 19)$ & 8 & 2 & YES & YES & YES & NO & 1251\\
$(1237, 345)$ & 16 & $(7, 2)$ & 4 & 1 & YES & YES & YES & NO & 1252\\
$(1783, 331)$ & 18 & $(27, 5)$ & 8 & 1 & YES & YES & YES & NO & 1253\\
$(a; 0, 0, 0; 3)$ & 4 & $(265, 62)$ & 14 & 1 & YES & YES & YES & NO & 1254\\
$(a; 4, 1, 0; 37)$ & 9 & $(9, 4)$ & 5 & 1 & YES & YES & NO(2) & NO & 1255\\
$(a; 5, 1, 3; 106)$ & 13 & $(3, 1)$ & 2 & 1 & YES & YES & YES & NO & 1256\\
$(b; 0, 0, 4; 38)$ & 9 & $(11, 5)$ & 6 & 1 & YES & YES & YES & NO & 1257\\
$(b; 0, 2, 0; 8)$ & 7 & $(124, 29)$ & 11 & 4 & YES & YES & YES & NO & 1258\\
$(b; 0, 3, 5; 89)$ & 13 & $(3, 1)$ & 2 & 1 & YES & YES & YES & NO & 1259\\
$(b; 1, 3, 1; 59)$ & 10 & $(11, 4)$ & 5 & 1 & YES & YES & YES & NO & 1260\\
$(b; 2, 0, 3; 62)$ & 10 & $(9, 2)$ & 5 & 1 & YES & YES & YES & NO & 1261\\
$(b; 2, 1, 4; 33)$ & 12 & $(2, 1)$ & 1 & 1 & YES & YES & NO(2) & NO & 1262\\
$(c; 0, 3, 2; 29)$ & 9 & $(45, 8)$ & 9 & 1 & YES & YES & YES & NO & 1263\\
$(c; 0, 5, 2; 39)$ & 11 & $(37, 5)$ & 10 & 1 & YES & YES & YES & NO & 1264\\
$(d; 0, 0, 0; 5)$ & 5 & $(207, 80)$ & 12 & 1 & YES & YES & YES & NO & 1265\\
$(d; 0, 0, 2; 9)$ & 7 & $(31, 11)$ & 8 & 1 & YES & YES & NO(2) & NO & 1266\\
$(e; 1, 1, 0; 23)$ & 7 & $(99, 29)$ & 10 & 1 & YES & YES & YES & NO & 1267\\
$(f; 0, 0, 0; 6)$ & 4 & $(738, 137)$ & 16 & 6 & YES & YES & YES & NO & 1268\\
$(f; 0, 1, 0; 7)$ & 5 & $(37, 17)$ & 9 & 1 & YES & YES & YES & NO & 1269\\
$(g; 1, 0, 2; 24)$ & 9 & $(14, 5)$ & 6 & 2 & YES & YES & YES & NO & 1270\\
$(g; 3, 3, 0; 44)$ & 12 & $(4, 1)$ & 3 & 4 & YES & YES & YES & NO & 1271
\end{longtable}
\subsection{2 chains, $K^2 = 5$}
\begin{longtable}{|c|c|c|c|c|c|c|c|c|c|}
\hline
\multicolumn{10}{|c|}{2 chains, $K^2 = 5$}\\
\hline
$(n,a)$ & Length & $(n,a)$ & Length & GCD & Nef & $\mathbb Q$-ef & Obstruction 0 & WH & Index\\
\hline
\endfirsthead

\hline
$(n,a)$ & Length & $(n,a)$ & Length & GCD & Nef & $\mathbb Q$-ef & Obstruction 0 & WH & Index\\
\hline
\endhead
\hline
\endfoot

$(158, 33)$ & 12 & $(4, 1)$ & 3 & 2 & YES & YES & NO(3) & NO & 1272\\
$(555, 199)$ & 15 & $(6, 1)$ & 5 & 3 & YES & YES & NO(3) & NO & 1273\\
$(1099, 247)$ & 17 & $(89, 20)$ & 11 & 1 & YES & YES & NO(3) & NO & 1274
\end{longtable}



%%%%%%%%%%%%%%%%%%%%%%%%%%%%%%%%%%%%%%%%%%%
\section{$I_8 + I_2 + 2I_1$}

(2858 examples from 101122048 tests)

Base curves:
\begin{itemize}
  \item $L_1 = y - \sqrt{3}x$.
  \item $L_2 = 2y - 3z$.
  \item $L_3 = y + \sqrt{3}x$.
  \item $C = x^2 + (y-2z)^2 - z^2$.
  \item $L = x$.
\end{itemize}
Fibration given by pencil
\[F_{\lambda} = L_1L_2L_3 + \lambda CL.\]
Nine exceptionals are as follows:
\begin{itemize}
  \item $E_1$ - $E_3$ at $L_1 \cap L_2 \cap C = [\sqrt{3},3,2]$.
  \item $E_4$ - $E_5$ at $L_1 \cap L_3 \cap L = [0,0,1]$.
  \item $E_6$ at $L_2 \cap L = [0,3,2]$.
  \item $E_7$ - $E_9$ at $L_3 \cap L_2 \cap C = [-\sqrt{3},3,2]$.
\end{itemize}
Singular fibers are as follows:
\begin{itemize}
  \item $\lambda = \infty$: $I_2$ fiber given by $C$ and $L$. with nodes at $N_{I_2,1} = [0,3,1]$ and $N_{I_2,2} = [0,1,1]$.
  \item $\lambda = 0$: $I_8$ fiber given by $L_2$, $E_7$, $E_8$, $L_3$, $E_4$, $L_1$, $E_2$, $E_1$ in order.
  \item $\lambda = \frac{3\sqrt{3}}{2}$: $I_1$ fiber called $F_1$ with node at $N_{F_1} = [-\sqrt{3},0,1]$.
  \item $\lambda = -\frac{3\sqrt{3}}{2}$: $I_1$ fiber called $F_2$ with node at $N_{F_2} = [\sqrt{3},0,1]$.
\end{itemize}

\begin{center}
Classification of degree 1 double sections by intersections with $I_8$ and $I_2$
\end{center}

\begin{enumerate}
    \item $L_2 + E_4 + 2C$
    \[R_\alpha = y - \alpha x, \quad \alpha \in \C \setminus\{-\sqrt{3},\sqrt{3}\}\]
    Degenerations:
    \begin{itemize}
        \item $\alpha = 0$: $R_\alpha$ intersects $N_{F_1}$ and $N_{F_2}$
    \end{itemize}
    \item $E_1 + L_3 + C + L$
    \[M_\alpha^R = y - \alpha x + \dfrac{\sqrt{3}\alpha - 3}{2}z, \quad \alpha \in \widehat\C \setminus\{0,\sqrt{3}\}\]
    Degenerations:
    \begin{itemize}
        \item $\alpha = -\sqrt{3}$: $M_{\alpha}^R$ intersects $N_{F_2}$ and $N_{I_2,1}$
        \item $\alpha = \frac{1}{\sqrt{3}}$: $M_{\alpha}^R$ intersects $N_{F_1}$ and $N_{I_2,2}$
    \end{itemize}
    \item $E_7 + L_1 + C + L$
    \[M_\alpha^L = y + \alpha x + \dfrac{\sqrt{3}\alpha - 3}{2}z, \quad \alpha \in \widehat\C \setminus\{0,\sqrt{3}\}\]
    Degenerations:
    \begin{itemize}
        \item $\alpha = -\sqrt{3}$: $M_{\alpha}^L$ intersects $N_{F_1}$ and $N_{I_2,1}$
        \item $\alpha = \frac{1}{\sqrt{3}}$: $M_{\alpha}^L$ intersects $N_{F_2}$ and $N_{I_2,2}$
    \end{itemize}
    \item $L_1 + L_2 + 2C$ (also intersects $E_6$)
    \[S_\alpha = 2y - \alpha x - 3z, \quad \alpha \in \C \setminus\{0\}\]
    Degenerations:
    \begin{itemize}
        \item $\alpha = \sqrt{3}$: $S_\alpha$ intersects $N_{F_1}$
        \item $\alpha = -\sqrt{3}$: $S_\alpha$ intersects $N_{F_2}$
    \end{itemize}
    
\end{enumerate}

\begin{center}
Classification of degree 2 double sections by intersections with $I_8$ and $I_2$
\end{center}

\begin{enumerate}
  \item $E_1 + L_1 + 2L$ (also intersects $E_9$)
  $D_{\alpha}^L = L_3L_2 + \alpha C, \quad \alpha \in \C \setminus\{0\}$
  Degenerations:
  \begin{itemize}
      \item $\alpha = 3/2$: $D_{\alpha}^L$ intersects $N_{F_2}$
      \item $\alpha = -3/2$: $D_{\alpha}^L$ intersects $N_{F_1}$
  \end{itemize}
  \item $E_1 + E_7 + 2C$ (also intersects $E_5$)
  \[E_\alpha = LL_2 + \alpha L_1L_3, \quad \alpha \in \C \setminus\{0\}\]
  Degenerations:
  \begin{itemize}
      \item $\alpha = \frac{1}{\sqrt{3}}$: $E_{\alpha}$ intersects $N_{F_1}$
      \item $\alpha = -\frac{1}{\sqrt{3}}$: $E_{\alpha}$ intersects $N_{F_2}$
  \end{itemize}
  \item $E_2 + E_8 + 2L$
  \[A_\alpha = L_1L_3 + \alpha C, \quad \alpha \in \C \setminus\{0\}\]
  Degenerations:
  \begin{itemize}
      \item $\alpha = \frac{3}{2}$: $E_{\alpha}$ intersects $N_{F_1}$ and $N_{F_2}$
  \end{itemize}
  \item $2L_1 + C + L$ (also intersects $E_9$ and $E_6$)
  \[B_\alpha^L = C - \frac{1}{3}L_3M_{-\sqrt{3}}^L + \alpha L_2 L_3, \quad \alpha \in \C\]
  Degenerations:
  \begin{itemize}
      \item $\alpha = 0$: $B_\alpha^L$ intersects $N_{I_2,1}$
      \item $\alpha = 2/3$: $B_\alpha^L$ intersects $N_{I_2,2}$
      \item $\alpha = 4/3$: $B_\alpha^L$ intersects $N_{F_2}$
      \item $\alpha = -2/3$: $B_\alpha^L$ intersects $N_{F_1}$
  \end{itemize}
  \item $2L_3 + C + L$ (also intersects $E_3$ and $E_6$)
  \[B_\alpha^R = C - \frac{1}{3}L_1M_{-\sqrt{3}}^R + \alpha L_2 L_1, \quad \alpha \in \C\]
  Degenerations:
  \begin{itemize}
      \item $\alpha = 0$: $B_\alpha^R$ intersects $N_{I_2,1}$
      \item $\alpha = 2/3$: $B_\alpha^R$ intersects $N_{I_2,2}$
      \item $\alpha = 4/3$: $B_\alpha^R$ intersects $N_{F_1}$
      \item $\alpha = -2/3$: $B_\alpha^L$ intersects $N_{F_2}$
  \end{itemize}
  \item $E_7 + L_3 + 2L$ (also intersects $E_3$)
  \[D_{\alpha}^R = L_1L_2 + \alpha C, \quad \alpha \in \C \setminus\{0\}\]
  Degenerations:
  \begin{itemize}
      \item $\alpha = 3/2$: $D_{\alpha}^R$ intersects $N_{F_2}$
      \item $\alpha = -3/2$: $D_{\alpha}^R$ intersects $N_{F_1}$
  \end{itemize}
\end{enumerate}


Input:
\lstinputlisting[language=config]{../Tests/8211.txt}
Result:
%\usepackage{longtable}
\subsection{1 chain, $K^2 = 1$}
\begin{longtable}{|c|c|c|c|c|c|c|c|}
\hline
\multicolumn{8}{|c|}{1 chain, $K^2 = 1$}\\
\hline
$(n,a)$ & Len & Nef & $\mathbb Q$-ef & Obs 0 & $\overline c_1^2 / \overline c_2$ & $(P,K)$ & Index\\
\hline
\endfirsthead

\hline
$(n,a)$ & Len & Nef & $\mathbb Q$-ef & Obs 0 & $\overline c_1^2 / \overline c_2$ & $(P,K)$ & Index\\
\hline
\endhead
\hline
\endfoot

$(11,4)$ & 5 & YES & YES & YES & $0.75$ & $(1,1)$ & 1\\
$(13,5)$ & 5 & YES & YES & YES & $0.64$ & $(1,1)$ & 2\\
$(13,4)$ & 6 & YES & YES & YES & $0.75$ & $(1,1)$ & 3\\
$(14,5)$ & 6 & YES & YES & YES & $0.75$ & $(1,1)$ & 4\\
$(16,5)$ & 7 & YES & YES & YES & $0.55$ & $(1,1)$ & 5\\
$(16,7)$ & 6 & YES & YES & YES & $0.60$ & $(1,1)$ & 6\\
$(17,7)$ & 6 & YES & YES & YES & $0.64$ & $(1,1)$ & 7\\
$(19,5)$ & 7 & YES & YES & YES & $0.64$ & $(1,1)$ & 8\\
$(19,8)$ & 6 & YES & YES & YES & $0.64$ & $(1,1)$ & 9\\
$(21,5)$ & 8 & YES & YES & YES & $0.40$ & $(1,1)$ & 10\\
$(24,5)$ & 8 & YES & YES & YES & $0.50$ & $(1,1)$ & 11\\
$(26,7)$ & 7 & YES & YES & YES & $0.55$ & $(1,1)$ & 12\\
$(30,7)$ & 8 & YES & YES & YES & $0.67$ & $(1,1)$ & 13\\
$(a;1,0,0;13)$ & 5 & YES & YES & YES & $0.64$ & $(1,1)$ & 14\\
$(b;0,0,0;14)$ & 5 & YES & YES & YES & $0.64$ & $(1,1)$ & 15\\
$(j;0,0,0;8)$ & 5 & YES & YES & YES & $0.55$ & $(1,1)$ & 16\\
$(j;0,1,0;10)$ & 6 & YES & YES & YES & $0.67$ & $(1,1)$ & 17
\end{longtable}
\subsection{1 chain, $K^2 = 2$}
\begin{longtable}{|c|c|c|c|c|c|c|c|}
\hline
\multicolumn{8}{|c|}{1 chain, $K^2 = 2$}\\
\hline
$(n,a)$ & Len & Nef & $\mathbb Q$-ef & Obs 0 & $\overline c_1^2 / \overline c_2$ & $(P,K)$ & Index\\
\hline
\endfirsthead

\hline
$(n,a)$ & Len & Nef & $\mathbb Q$-ef & Obs 0 & $\overline c_1^2 / \overline c_2$ & $(P,K)$ & Index\\
\hline
\endhead
\hline
\endfoot

$(27,8)$ & 7 & YES & YES & YES & $0.90$ & $(1,2)$ & 18\\
$(29,8)$ & 7 & YES & YES & YES & $0.90$ & $(1,2)$ & 19\\
$(31,9)$ & 8 & YES & YES & YES & $1.00$ & $(1,2)$ & 20\\
$(31,7)$ & 8 & YES & YES & NO(2) & $1.17$ & $(1,2)$ & 21\\
$(32,9)$ & 8 & YES & YES & YES & $1.00$ & $(1,2)$ & 22\\
$(32,7)$ & 8 & YES & YES & YES & $0.67$ & $(5,0)$ & 23\\
$(33,13)$ & 9 & YES & YES & YES & $1.25$ & $(1,2)$ & 24\\
$(37,10)$ & 8 & YES & YES & YES & $1.00$ & $(1,2)$ & 25\\
$(37,8)$ & 8 & YES & YES & YES & $0.89$ & $(1,2)$ & 26\\
$(39,14)$ & 8 & YES & YES & YES & $0.78$ & $(3,1)$ & 27\\
$(40,17)$ & 9 & YES & YES & YES & $1.18$ & $(1,2)$ & 28\\
$(41,15)$ & 8 & YES & YES & YES & $1.00$ & $(1,2)$ & 29\\
$(41,11)$ & 8 & YES & YES & NO(2) & $0.90$ & $(3,1)$ & 30\\
$(42,13)$ & 9 & YES & YES & YES & $1.00$ & $(1,2)$ & 31\\
$(44,19)$ & 10 & YES & YES & YES & $1.18$ & $(1,2)$ & 32\\
$(45,13)$ & 10 & YES & YES & YES & $1.00$ & $(1,2)$ & 33\\
$(45,14)$ & 9 & YES & YES & YES & $1.00$ & $(1,2)$ & 34\\
$(46,21)$ & 10 & YES & YES & YES & $0.89$ & $(3,1)$ & 35\\
$(48,17)$ & 9 & YES & YES & YES & $1.00$ & $(1,2)$ & 36\\
$(49,13)$ & 9 & YES & YES & YES & $0.90$ & $(1,2)$ & 37\\
$(49,18)$ & 8 & YES & YES & YES & $0.90$ & $(1,2)$ & 38\\
$(49,22)$ & 9 & YES & YES & NO(2) & $1.09$ & $(1,2)$ & 39\\
$(49,15)$ & 9 & YES & YES & NO(2) & $0.78$ & $(7,-1)$ & 40\\
$(50,19)$ & 8 & YES & YES & NO(2) & $0.78$ & $(7,-1)$ & 41\\
$(51,20)$ & 9 & YES & YES & YES & $1.00$ & $(1,2)$ & 42\\
$(53,19)$ & 9 & YES & YES & YES & $0.78$ & $(3,1)$ & 43\\
$(55,24)$ & 9 & YES & YES & YES & $0.90$ & $(1,2)$ & 44\\
$(57,17)$ & 10 & YES & YES & NO(2) & $0.90$ & $(5,0)$ & 45\\
$(57,25)$ & 9 & YES & YES & YES & $0.90$ & $(1,2)$ & 46\\
$(59,13)$ & 11 & YES & YES & YES & $0.90$ & $(1,2)$ & 47\\
$(62,27)$ & 9 & YES & YES & YES & $1.00$ & $(1,2)$ & 48\\
$(64,17)$ & 10 & YES & YES & YES & $1.00$ & $(1,2)$ & 49\\
$(64,23)$ & 9 & YES & YES & YES & $1.00$ & $(1,2)$ & 50\\
$(65,24)$ & 9 & YES & YES & YES & $0.90$ & $(1,2)$ & 51\\
$(67,16)$ & 11 & YES & YES & YES & $0.90$ & $(1,2)$ & 52\\
$(71,13)$ & 12 & YES & YES & NO(2) & $1.09$ & $(1,2)$ & 53\\
$(71,17)$ & 11 & YES & YES & YES & $0.90$ & $(1,2)$ & 54\\
$(71,19)$ & 10 & YES & YES & YES & $1.00$ & $(1,2)$ & 55\\
$(71,22)$ & 10 & YES & YES & NO(2) & $1.09$ & $(1,2)$ & 56\\
$(72,19)$ & 10 & YES & YES & YES & $1.00$ & $(1,2)$ & 57\\
$(74,17)$ & 11 & YES & YES & YES & $0.90$ & $(1,2)$ & 58\\
$(77,16)$ & 11 & YES & YES & YES & $0.90$ & $(1,2)$ & 59\\
$(79,14)$ & 11 & YES & YES & YES & $0.90$ & $(1,2)$ & 60\\
$(80,19)$ & 11 & YES & YES & YES & $0.90$ & $(1,2)$ & 61\\
$(81,19)$ & 11 & YES & YES & NO(2) & $0.67$ & $(9,-2)$ & 62\\
$(89,27)$ & 10 & YES & YES & YES & $0.90$ & $(1,2)$ & 63\\
$(90,19)$ & 11 & YES & YES & NO(2) & $0.67$ & $(9,-2)$ & 64\\
$(91,19)$ & 11 & YES & YES & YES & $0.90$ & $(1,2)$ & 65\\
$(96,17)$ & 12 & YES & YES & YES & $1.00$ & $(1,2)$ & 66\\
$(a;3,1,0;31)$ & 8 & YES & YES & NO(2) & $1.09$ & $(1,2)$ & 67\\
$(b;0,0,3;32)$ & 8 & YES & YES & YES & $1.00$ & $(1,2)$ & 68\\
$(b;0,3,0;29)$ & 8 & YES & YES & YES & $1.00$ & $(1,2)$ & 69\\
$(c;0,3,1;23)$ & 8 & YES & YES & YES & $0.90$ & $(1,2)$ & 70\\
$(c;0,4,1;9)$ & 9 & YES & YES & YES & $0.90$ & $(1,2)$ & 71\\
$(d;0,0,3;22)$ & 8 & YES & YES & YES & $0.90$ & $(1,2)$ & 72\\
$(d;0,0,4;13)$ & 9 & YES & YES & YES & $0.90$ & $(1,2)$ & 73\\
$(d;0,1,3;27)$ & 9 & YES & YES & YES & $0.90$ & $(1,2)$ & 74\\
$(d;0,3,1;23)$ & 9 & YES & YES & YES & $0.90$ & $(1,2)$ & 75\\
$(e;3,0,0;10)$ & 8 & YES & YES & YES & $1.00$ & $(1,2)$ & 76
\end{longtable}
\subsection{1 chain, $K^2 = 3$}
\begin{longtable}{|c|c|c|c|c|c|c|c|}
\hline
\multicolumn{8}{|c|}{1 chain, $K^2 = 3$}\\
\hline
$(n,a)$ & Len & Nef & $\mathbb Q$-ef & Obs 0 & $\overline c_1^2 / \overline c_2$ & $(P,K)$ & Index\\
\hline
\endfirsthead

\hline
$(n,a)$ & Len & Nef & $\mathbb Q$-ef & Obs 0 & $\overline c_1^2 / \overline c_2$ & $(P,K)$ & Index\\
\hline
\endhead
\hline
\endfoot

$(64,25)$ & 9 & YES & YES & NO(2) & $1.36$ & $(1,3)$ & 77\\
$(71,26)$ & 9 & YES & YES & NO(2) & $1.30$ & $(1,3)$ & 78\\
$(76,31)$ & 10 & YES & YES & NO(2) & $1.36$ & $(1,3)$ & 79\\
$(92,39)$ & 10 & YES & YES & YES & $1.40$ & $(1,3)$ & 80\\
$(97,18)$ & 11 & YES & YES & YES & $1.12$ & $(5,1)$ & 81\\
$(98,41)$ & 10 & YES & YES & YES & $1.40$ & $(1,3)$ & 82\\
$(101,37)$ & 10 & YES & YES & NO(2) & $1.45$ & $(1,3)$ & 83\\
$(101,30)$ & 10 & YES & YES & NO(2) & $1.11$ & $(9,-1)$ & 84\\
$(101,22)$ & 11 & YES & YES & NO(2) & $1.40$ & $(3,2)$ & 85\\
$(104,45)$ & 11 & YES & YES & YES & $1.40$ & $(1,3)$ & 86\\
$(104,31)$ & 11 & YES & YES & NO(2) & $1.45$ & $(3,2)$ & 87\\
$(109,30)$ & 10 & YES & YES & NO(2) & $1.00$ & $(7,0)$ & 88\\
$(113,42)$ & 11 & YES & YES & YES & $1.38$ & $(1,3)$ & 89\\
$(113,35)$ & 11 & YES & YES & NO(2) & $1.33$ & $(3,2)$ & 90\\
$(115,52)$ & 11 & YES & YES & NO(2) & $1.45$ & $(3,2)$ & 91\\
$(119,45)$ & 11 & YES & YES & YES & $1.12$ & $(3,2)$ & 92\\
$(119,37)$ & 11 & YES & YES & NO(2) & $1.33$ & $(5,1)$ & 93\\
$(120,53)$ & 11 & YES & YES & NO(2) & $1.40$ & $(1,3)$ & 94\\
$(125,46)$ & 12 & YES & YES & YES & $1.55$ & $(1,3)$ & 95\\
$(125,49)$ & 11 & YES & YES & YES & $1.00$ & $(3,2)$ & 96\\
$(129,56)$ & 11 & YES & YES & NO(2) & $1.22$ & $(5,1)$ & 97\\
$(135,32)$ & 12 & YES & YES & YES & $1.33$ & $(3,2)$ & 98\\
$(137,63)$ & 12 & YES & YES & NO(2) & $1.33$ & $(3,2)$ & 99\\
$(144,43)$ & 13 & YES & YES & NO(2) & $1.33$ & $(3,2)$ & 100\\
$(145,51)$ & 12 & YES & YES & YES & $1.33$ & $(3,2)$ & 101\\
$(149,46)$ & 13 & YES & YES & YES & $1.33$ & $(1,3)$ & 102\\
$(151,53)$ & 12 & YES & YES & NO(2) & $1.33$ & $(3,2)$ & 103\\
$(151,62)$ & 11 & YES & YES & YES & $1.33$ & $(1,3)$ & 104\\
$(152,55)$ & 12 & YES & YES & YES & $1.25$ & $(3,2)$ & 105\\
$(152,67)$ & 11 & YES & YES & NO(2) & $0.88$ & $(7,0)$ & 106\\
$(153,64)$ & 11 & YES & YES & YES & $1.40$ & $(1,3)$ & 107\\
$(161,48)$ & 12 & YES & YES & NO(2) & $1.00$ & $(7,0)$ & 108\\
$(169,64)$ & 11 & YES & YES & YES & $1.33$ & $(1,3)$ & 109\\
$(171,71)$ & 12 & YES & YES & YES & $1.44$ & $(1,3)$ & 110\\
$(183,67)$ & 11 & YES & YES & YES & $1.22$ & $(1,3)$ & 111\\
$(188,39)$ & 13 & YES & YES & YES & $1.22$ & $(1,3)$ & 112\\
$(201,37)$ & 14 & YES & YES & NO(2) & $1.22$ & $(5,1)$ & 113\\
$(207,37)$ & 15 & YES & YES & YES & $1.33$ & $(3,2)$ & 114\\
$(211,50)$ & 14 & YES & YES & NO(2) & $1.45$ & $(3,2)$ & 115\\
$(213,38)$ & 15 & YES & YES & NO(2) & $1.22$ & $(3,2)$ & 116\\
$(213,62)$ & 12 & YES & YES & YES & $1.33$ & $(1,3)$ & 117\\
$(231,83)$ & 12 & YES & YES & YES & $1.38$ & $(1,3)$ & 118\\
$(241,63)$ & 13 & YES & YES & NO(2) & $1.33$ & $(3,2)$ & 119\\
$(243,38)$ & 16 & YES & YES & NO(2) & $1.33$ & $(3,2)$ & 120\\
$(246,91)$ & 12 & YES & YES & NO(2) & $1.40$ & $(3,2)$ & 121\\
$(272,59)$ & 13 & YES & YES & YES & $1.33$ & $(1,3)$ & 122\\
$(b;4,0,1;56)$ & 10 & YES & YES & YES & $1.33$ & $(1,3)$ & 123
\end{longtable}
\subsection{1 chain, $K^2 = 4$}
\begin{longtable}{|c|c|c|c|c|c|c|c|}
\hline
\multicolumn{8}{|c|}{1 chain, $K^2 = 4$}\\
\hline
$(n,a)$ & Len & Nef & $\mathbb Q$-ef & Obs 0 & $\overline c_1^2 / \overline c_2$ & $(P,K)$ & Index\\
\hline
\endfirsthead

\hline
$(n,a)$ & Len & Nef & $\mathbb Q$-ef & Obs 0 & $\overline c_1^2 / \overline c_2$ & $(P,K)$ & Index\\
\hline
\endhead
\hline
\endfoot

$(178,63)$ & 12 & YES & YES & YES & $1.67$ & $(1,4)$ & 124\\
$(252,107)$ & 13 & YES & YES & YES & $1.57$ & $(3,3)$ & 125\\
$(289,66)$ & 13 & YES & YES & NO(2) & $1.62$ & $(9,0)$ & 126\\
$(298,131)$ & 13 & YES & YES & NO(2) & $1.62$ & $(5,2)$ & 127\\
$(323,116)$ & 13 & YES & YES & NO(2) & $1.62$ & $(5,2)$ & 128\\
$(336,137)$ & 14 & YES & YES & YES & $1.57$ & $(3,3)$ & 129\\
$(375,143)$ & 14 & YES & YES & YES & $1.89$ & $(1,4)$ & 130\\
$(379,165)$ & 13 & YES & YES & YES & $1.75$ & $(1,4)$ & 131\\
$(412,107)$ & 16 & YES & YES & NO(2) & $1.75$ & $(3,3)$ & 132\\
$(497,107)$ & 15 & YES & YES & YES & $1.62$ & $(1,4)$ & 133\\
$(539,200)$ & 14 & YES & YES & NO(2) & $1.62$ & $(9,0)$ & 134\\
$(618,239)$ & 14 & YES & YES & NO(2) & $1.43$ & $(5,2)$ & 135\\
$(635,132)$ & 16 & YES & YES & YES & $1.75$ & $(3,3)$ & 136\\
$(636,179)$ & 16 & YES & YES & NO(2) & $1.57$ & $(9,0)$ & 137\\
$(727,282)$ & 14 & YES & YES & NO(2) & $1.78$ & $(3,3)$ & 138\\
$(832,191)$ & 17 & YES & YES & NO(2) & $1.43$ & $(9,0)$ & 139\\
$(1058,409)$ & 15 & YES & YES & YES & $2.00$ & $(7,1)$ & 140\\
$(1190,349)$ & 16 & YES & YES & YES & $1.86$ & $(3,3)$ & 141\\
$(g;2,3,1;19)$ & 12 & YES & YES & YES & $1.62$ & $(1,4)$ & 142
\end{longtable}
\subsection{1 chain, $K^2 = 5$}
\begin{longtable}{|c|c|c|c|c|c|c|c|}
\hline
\multicolumn{8}{|c|}{1 chain, $K^2 = 5$}\\
\hline
$(n,a)$ & Len & Nef & $\mathbb Q$-ef & Obs 0 & $\overline c_1^2 / \overline c_2$ & $(P,K)$ & Index\\
\hline
\endfirsthead

\hline
$(n,a)$ & Len & Nef & $\mathbb Q$-ef & Obs 0 & $\overline c_1^2 / \overline c_2$ & $(P,K)$ & Index\\
\hline
\endhead
\hline
\endfoot

$(1005,412)$ & 15 & YES & YES & NO(2) & $2.12$ & $(3,4)$ & 143
\end{longtable}
\subsection{2 chains, $K^2 = 1$}
\begin{longtable}{|c|c|c|c|c|c|c|c|c|c|c|c|}
\hline
\multicolumn{12}{|c|}{2 chains, $K^2 = 1$}\\
\hline
$(n,a)$ & Len & $(n,a)$ & Len & GCD & Nef & $\mathbb Q$-ef & Obs 0 & $\overline c_1^2 / \overline c_2$ & $(P,K)$ & WH & Index\\
\hline
\endfirsthead

\hline
$(n,a)$ & Len & $(n,a)$ & Len & GCD & Nef & $\mathbb Q$-ef & Obs 0 & $\overline c_1^2 / \overline c_2$ & $(P,K)$ & WH & Index\\
\hline
\endhead
\hline
\endfoot

$(5,2)$ & 3 & $(5,2)$ & 3 & 5 & YES & YES & YES & $0.60$ & $(2,1)$ & -- & 144\\
$(7,3)$ & 4 & $(5,1)$ & 4 & 1 & YES & YES & YES & $0.56$ & $(4,0)$ & NO & 145\\
$(7,3)$ & 4 & $(5,1)$ & 4 & 1 & YES & YES & YES & $0.56$ & $(4,0)$ & NO & 146\\
$(7,3)$ & 4 & $(7,2)$ & 4 & 7 & YES & YES & YES & $0.56$ & $(2,1)$ & NO & 147\\
$(7,3)$ & 4 & $(7,2)$ & 4 & 7 & YES & YES & YES & $0.56$ & $(2,1)$ & -- & 148\\
$(7,3)$ & 4 & $(7,2)$ & 4 & 7 & YES & YES & YES & $0.56$ & $(2,1)$ & NO & 149\\
$(8,3)$ & 4 & $(4,1)$ & 3 & 4 & YES & YES & YES & $0.56$ & $(2,1)$ & -- & 150\\
$(8,3)$ & 4 & $(4,1)$ & 3 & 4 & YES & YES & YES & $0.67$ & $(2,1)$ & NO & 151\\
$(8,3)$ & 4 & $(5,1)$ & 4 & 1 & YES & YES & YES & $0.56$ & $(2,1)$ & NO & 152\\
$(8,3)$ & 4 & $(5,1)$ & 4 & 1 & YES & YES & YES & $0.56$ & $(2,1)$ & -- & 153\\
$(8,3)$ & 4 & $(5,1)$ & 4 & 1 & YES & YES & YES & $0.67$ & $(2,1)$ & NO & 154\\
$(8,3)$ & 4 & $(5,2)$ & 3 & 1 & YES & YES & YES & $0.44$ & $(4,0)$ & -- & 155\\
$(8,3)$ & 4 & $(7,2)$ & 4 & 1 & YES & YES & YES & $0.44$ & $(4,0)$ & NO & 156\\
$(8,3)$ & 4 & $(7,2)$ & 4 & 1 & YES & YES & YES & $0.44$ & $(4,0)$ & -- & 157\\
$(8,3)$ & 4 & $(7,3)$ & 4 & 1 & YES & YES & YES & $0.56$ & $(2,1)$ & NO & 158\\
$(8,3)$ & 4 & $(7,3)$ & 4 & 1 & YES & YES & YES & $0.56$ & $(2,1)$ & -- & 159\\
$(8,3)$ & 4 & $(7,3)$ & 4 & 1 & YES & YES & YES & $0.56$ & $(2,1)$ & NO & 160\\
$(9,4)$ & 5 & $(4,1)$ & 3 & 1 & YES & YES & YES & $0.56$ & $(2,1)$ & NO & 161\\
$(9,4)$ & 5 & $(4,1)$ & 3 & 1 & YES & YES & YES & $0.80$ & $(2,1)$ & NO & 162\\
$(9,4)$ & 5 & $(5,2)$ & 3 & 1 & YES & YES & YES & $0.80$ & $(2,1)$ & NO & 163\\
$(9,2)$ & 5 & $(7,3)$ & 4 & 1 & YES & YES & YES & $0.56$ & $(2,1)$ & NO & 164\\
$(9,2)$ & 5 & $(7,3)$ & 4 & 1 & YES & YES & YES & $0.56$ & $(2,1)$ & -- & 165\\
$(9,4)$ & 5 & $(7,2)$ & 4 & 1 & YES & YES & NO(2) & $0.44$ & $(6,-1)$ & NO & 166\\
$(9,4)$ & 5 & $(7,2)$ & 4 & 1 & YES & YES & NO(2) & $0.44$ & $(6,-1)$ & -- & 167\\
$(9,4)$ & 5 & $(9,2)$ & 5 & 9 & YES & YES & NO(2) & $0.44$ & $(6,-1)$ & NO & 168\\
$(10,3)$ & 5 & $(4,1)$ & 3 & 2 & YES & YES & YES & $0.60$ & $(2,1)$ & 177 & 169\\
$(10,3)$ & 5 & $(4,1)$ & 3 & 2 & YES & YES & YES & $0.60$ & $(2,1)$ & -- & 170\\
$(10,3)$ & 5 & $(5,1)$ & 4 & 5 & YES & YES & YES & $0.60$ & $(2,1)$ & -- & 171\\
$(10,3)$ & 5 & $(5,1)$ & 4 & 5 & YES & YES & YES & $0.70$ & $(2,1)$ & NO & 172\\
$(10,3)$ & 5 & $(5,2)$ & 3 & 5 & YES & YES & YES & $0.60$ & $(2,1)$ & -- & 173\\
$(11,3)$ & 5 & $(2,1)$ & 1 & 1 & YES & YES & YES & $0.70$ & $(2,1)$ & -- & 174\\
$(11,3)$ & 5 & $(2,1)$ & 1 & 1 & YES & YES & YES & $0.70$ & $(2,1)$ & NO & 175\\
$(11,4)$ & 5 & $(2,1)$ & 1 & 1 & YES & YES & YES & $0.73$ & $(2,1)$ & -- & 176\\
$(11,3)$ & 5 & $(3,1)$ & 2 & 1 & YES & YES & YES & $0.60$ & $(2,1)$ & 169 & 177\\
$(11,3)$ & 5 & $(3,1)$ & 2 & 1 & YES & YES & YES & $0.60$ & $(2,1)$ & -- & 178\\
$(11,4)$ & 5 & $(3,1)$ & 2 & 1 & YES & YES & YES & $0.82$ & $(2,1)$ & NO & 179\\
$(11,4)$ & 5 & $(3,1)$ & 2 & 1 & YES & YES & YES & $0.82$ & $(2,1)$ & -- & 180\\
$(11,5)$ & 6 & $(3,1)$ & 2 & 1 & YES & YES & YES & $0.70$ & $(2,1)$ & -- & 181\\
$(11,5)$ & 6 & $(3,1)$ & 2 & 1 & YES & YES & YES & $0.70$ & $(2,1)$ & NO & 182\\
$(11,3)$ & 5 & $(4,1)$ & 3 & 1 & YES & YES & YES & $0.70$ & $(2,1)$ & NO & 183\\
$(11,3)$ & 5 & $(4,1)$ & 3 & 1 & YES & YES & YES & $0.70$ & $(2,1)$ & -- & 184\\
$(11,3)$ & 5 & $(4,1)$ & 3 & 1 & YES & YES & YES & $0.70$ & $(2,1)$ & NO & 185\\
$(11,4)$ & 5 & $(4,1)$ & 3 & 1 & YES & YES & YES & $0.56$ & $(4,0)$ & NO & 186\\
$(11,4)$ & 5 & $(4,1)$ & 3 & 1 & YES & YES & YES & $0.56$ & $(4,0)$ & -- & 187\\
$(11,5)$ & 6 & $(4,1)$ & 3 & 1 & YES & YES & YES & $0.80$ & $(2,1)$ & NO & 188\\
$(11,5)$ & 6 & $(4,1)$ & 3 & 1 & YES & YES & YES & $0.80$ & $(2,1)$ & -- & 189\\
$(11,5)$ & 6 & $(4,1)$ & 3 & 1 & YES & YES & YES & $0.80$ & $(2,1)$ & NO & 190\\
$(11,3)$ & 5 & $(5,1)$ & 4 & 1 & YES & YES & YES & $0.60$ & $(2,1)$ & -- & 191\\
$(11,3)$ & 5 & $(5,1)$ & 4 & 1 & YES & YES & YES & $0.70$ & $(2,1)$ & NO & 192\\
$(11,5)$ & 6 & $(5,2)$ & 3 & 1 & YES & YES & YES & $0.80$ & $(2,1)$ & NO & 193\\
$(11,5)$ & 6 & $(5,2)$ & 3 & 1 & YES & YES & YES & $0.80$ & $(2,1)$ & -- & 194\\
$(11,5)$ & 6 & $(6,1)$ & 5 & 1 & YES & YES & YES & $0.80$ & $(2,1)$ & NO & 195\\
$(11,5)$ & 6 & $(6,1)$ & 5 & 1 & YES & YES & YES & $0.80$ & $(2,1)$ & NO & 196\\
$(11,4)$ & 5 & $(8,3)$ & 4 & 1 & YES & YES & YES & $0.82$ & $(2,1)$ & NO & 197\\
$(11,5)$ & 6 & $(9,4)$ & 5 & 1 & YES & YES & YES & $0.80$ & $(2,1)$ & NO & 198\\
$(11,5)$ & 6 & $(11,5)$ & 6 & 11 & YES & YES & YES & $0.70$ & $(2,1)$ & NO & 199\\
$(12,5)$ & 5 & $(3,1)$ & 2 & 3 & YES & YES & YES & $0.60$ & $(2,1)$ & -- & 200\\
$(12,5)$ & 5 & $(3,1)$ & 2 & 3 & YES & YES & YES & $0.70$ & $(2,1)$ & NO & 201\\
$(12,5)$ & 5 & $(3,1)$ & 2 & 3 & YES & YES & YES & $0.70$ & $(2,1)$ & NO & 202\\
$(12,5)$ & 5 & $(4,1)$ & 3 & 4 & YES & YES & YES & $0.44$ & $(4,0)$ & NO & 203\\
$(12,5)$ & 5 & $(4,1)$ & 3 & 4 & YES & YES & YES & $0.80$ & $(2,1)$ & NO & 204\\
$(12,5)$ & 5 & $(4,1)$ & 3 & 4 & YES & YES & YES & $0.56$ & $(4,0)$ & -- & 205\\
$(12,5)$ & 5 & $(5,2)$ & 3 & 1 & YES & YES & NO(2) & $0.44$ & $(6,-1)$ & NO & 206\\
$(12,5)$ & 5 & $(5,2)$ & 3 & 1 & YES & YES & NO(2) & $0.44$ & $(6,-1)$ & -- & 207\\
$(12,5)$ & 5 & $(7,2)$ & 4 & 1 & YES & YES & YES & $0.80$ & $(2,1)$ & NO & 208\\
$(12,5)$ & 5 & $(7,2)$ & 4 & 1 & YES & YES & YES & $0.80$ & $(2,1)$ & -- & 209\\
$(12,5)$ & 5 & $(9,2)$ & 5 & 3 & YES & YES & NO(2) & $0.33$ & $(6,-1)$ & -- & 210\\
$(12,5)$ & 5 & $(9,4)$ & 5 & 3 & YES & YES & NO(2) & $0.44$ & $(6,-1)$ & NO & 211\\
$(13,3)$ & 6 & $(2,1)$ & 1 & 1 & YES & YES & YES & $0.70$ & $(2,1)$ & -- & 212\\
$(13,5)$ & 5 & $(2,1)$ & 1 & 1 & YES & YES & YES & $0.70$ & $(2,1)$ & NO & 213\\
$(13,4)$ & 6 & $(3,1)$ & 2 & 1 & YES & YES & YES & $0.82$ & $(2,1)$ & NO & 214\\
$(13,4)$ & 6 & $(3,1)$ & 2 & 1 & YES & YES & YES & $0.82$ & $(2,1)$ & -- & 215\\
$(13,5)$ & 5 & $(3,1)$ & 2 & 1 & YES & YES & YES & $0.60$ & $(2,1)$ & -- & 216\\
$(13,5)$ & 5 & $(3,1)$ & 2 & 1 & YES & YES & YES & $0.70$ & $(2,1)$ & NO & 217\\
$(13,3)$ & 6 & $(4,1)$ & 3 & 1 & YES & YES & YES & $0.70$ & $(2,1)$ & NO & 218\\
$(13,3)$ & 6 & $(4,1)$ & 3 & 1 & YES & YES & YES & $0.70$ & $(2,1)$ & -- & 219\\
$(13,4)$ & 6 & $(4,1)$ & 3 & 1 & YES & YES & YES & $0.80$ & $(2,1)$ & -- & 220\\
$(13,4)$ & 6 & $(4,1)$ & 3 & 1 & YES & YES & YES & $0.90$ & $(2,1)$ & NO & 221\\
$(13,4)$ & 6 & $(7,2)$ & 4 & 1 & YES & YES & YES & $0.80$ & $(2,1)$ & 246 & 222\\
$(13,4)$ & 6 & $(7,2)$ & 4 & 1 & YES & YES & YES & $0.80$ & $(2,1)$ & -- & 223\\
$(13,3)$ & 6 & $(11,3)$ & 5 & 1 & YES & YES & YES & $0.60$ & $(2,1)$ & NO & 224\\
$(13,5)$ & 5 & $(13,5)$ & 5 & 13 & YES & YES & YES & $0.70$ & $(2,1)$ & NO & 225\\
$(14,5)$ & 6 & $(3,1)$ & 2 & 1 & NO & YES & YES & $0.82$ & $(2,1)$ & -- & 226\\
$(15,4)$ & 6 & $(4,1)$ & 3 & 1 & NO & YES & YES & $0.70$ & $(2,1)$ & -- & 227\\
$(15,4)$ & 6 & $(9,2)$ & 5 & 3 & YES & YES & NO(2) & $0.33$ & $(6,-1)$ & NO & 228\\
$(16,5)$ & 7 & $(3,1)$ & 2 & 1 & YES & YES & YES & $0.60$ & $(2,1)$ & NO & 229\\
$(16,5)$ & 7 & $(3,1)$ & 2 & 1 & NO & YES & YES & $0.82$ & $(2,1)$ & -- & 230\\
$(16,7)$ & 6 & $(3,1)$ & 2 & 1 & YES & YES & NO(2) & $0.44$ & $(6,-1)$ & -- & 231\\
$(16,7)$ & 6 & $(3,1)$ & 2 & 1 & YES & YES & NO(2) & $0.44$ & $(6,-1)$ & NO & 232\\
$(16,7)$ & 6 & $(4,1)$ & 3 & 4 & YES & YES & NO(2) & $0.44$ & $(6,-1)$ & NO & 233\\
$(16,7)$ & 6 & $(4,1)$ & 3 & 4 & YES & YES & NO(2) & $0.44$ & $(6,-1)$ & -- & 234\\
$(16,5)$ & 7 & $(5,1)$ & 4 & 1 & YES & YES & NO(2) & $0.44$ & $(6,-1)$ & NO & 235\\
$(16,7)$ & 6 & $(5,1)$ & 4 & 1 & YES & YES & YES & $0.56$ & $(2,1)$ & NO & 236\\
$(16,7)$ & 6 & $(5,1)$ & 4 & 1 & YES & YES & YES & $0.56$ & $(2,1)$ & -- & 237\\
$(16,7)$ & 6 & $(5,2)$ & 3 & 1 & YES & YES & NO(2) & $0.44$ & $(6,-1)$ & NO & 238\\
$(16,5)$ & 7 & $(7,1)$ & 6 & 1 & YES & YES & YES & $0.60$ & $(2,1)$ & NO & 239\\
$(16,5)$ & 7 & $(7,2)$ & 4 & 1 & YES & YES & NO(2) & $0.44$ & $(6,-1)$ & NO & 240\\
$(16,7)$ & 6 & $(9,4)$ & 5 & 1 & YES & YES & NO(2) & $0.44$ & $(6,-1)$ & NO & 241\\
$(16,5)$ & 7 & $(13,4)$ & 6 & 1 & YES & YES & YES & $0.60$ & $(2,1)$ & NO & 242\\
$(16,7)$ & 6 & $(16,7)$ & 6 & 16 & YES & YES & NO(2) & $0.44$ & $(6,-1)$ & NO & 243\\
$(17,7)$ & 6 & $(2,1)$ & 1 & 1 & YES & YES & YES & $0.70$ & $(2,1)$ & NO & 244\\
$(17,5)$ & 6 & $(3,1)$ & 2 & 1 & NO & YES & YES & $0.60$ & $(2,1)$ & -- & 245\\
$(17,5)$ & 6 & $(3,1)$ & 2 & 1 & YES & YES & YES & $0.80$ & $(2,1)$ & 222 & 246\\
$(17,4)$ & 7 & $(4,1)$ & 3 & 1 & NO & YES & YES & $0.70$ & $(2,1)$ & NO & 247\\
$(17,4)$ & 7 & $(4,1)$ & 3 & 1 & NO & YES & YES & $0.70$ & $(2,1)$ & -- & 248\\
$(17,7)$ & 6 & $(12,5)$ & 5 & 1 & YES & YES & NO(2) & $0.33$ & $(6,-1)$ & NO & 249\\
$(17,7)$ & 6 & $(17,7)$ & 6 & 17 & YES & YES & NO(2) & $0.44$ & $(6,-1)$ & NO & 250\\
$(18,5)$ & 6 & $(3,1)$ & 2 & 3 & YES & YES & NO(3) & $0.33$ & $(2,1)$ & -- & 251\\
$(19,8)$ & 6 & $(2,1)$ & 1 & 1 & YES & YES & YES & $0.70$ & $(2,1)$ & NO & 252\\
$(19,8)$ & 6 & $(2,1)$ & 1 & 1 & NO & YES & YES & $0.70$ & $(2,1)$ & -- & 253\\
$(19,5)$ & 7 & $(4,1)$ & 3 & 1 & YES & YES & YES & $0.70$ & $(2,1)$ & NO & 254\\
$(19,8)$ & 6 & $(4,1)$ & 3 & 1 & YES & YES & YES & $0.70$ & $(2,1)$ & -- & 255\\
$(19,5)$ & 7 & $(5,1)$ & 4 & 1 & YES & YES & NO(2) & $0.44$ & $(6,-1)$ & NO & 256\\
$(19,5)$ & 7 & $(6,1)$ & 5 & 1 & YES & YES & NO(2) & $0.33$ & $(6,-1)$ & NO & 257\\
$(19,5)$ & 7 & $(7,1)$ & 6 & 1 & YES & YES & YES & $0.60$ & $(2,1)$ & NO & 258\\
$(19,5)$ & 7 & $(11,3)$ & 5 & 1 & YES & YES & YES & $0.60$ & $(2,1)$ & 267 & 259\\
$(19,5)$ & 7 & $(15,4)$ & 6 & 1 & YES & YES & NO(2) & $0.33$ & $(6,-1)$ & NO & 260\\
$(19,5)$ & 7 & $(19,5)$ & 7 & 19 & YES & YES & NO(2) & $0.44$ & $(6,-1)$ & NO & 261\\
$(19,8)$ & 6 & $(19,8)$ & 6 & 19 & YES & YES & NO(2) & $0.44$ & $(6,-1)$ & NO & 262\\
$(21,8)$ & 6 & $(2,1)$ & 1 & 1 & NO & YES & YES & $0.70$ & $(2,1)$ & -- & 263\\
$(23,10)$ & 7 & $(2,1)$ & 1 & 1 & NO & YES & YES & $0.70$ & $(2,1)$ & -- & 264\\
$(23,7)$ & 7 & $(3,1)$ & 2 & 1 & NO & YES & YES & $0.80$ & $(2,1)$ & -- & 265\\
$(25,11)$ & 7 & $(2,1)$ & 1 & 1 & NO & YES & NO(2) & $0.56$ & $(6,-1)$ & -- & 266\\
$(26,7)$ & 7 & $(4,1)$ & 3 & 2 & YES & YES & YES & $0.60$ & $(2,1)$ & 259 & 267\\
$(26,7)$ & 7 & $(26,7)$ & 7 & 26 & YES & YES & NO(2) & $0.33$ & $(6,-1)$ & NO & 268\\
$(30,7)$ & 8 & $(3,1)$ & 2 & 3 & YES & YES & NO(2) & $0.22$ & $(6,-1)$ & NO & 269\\
$(30,7)$ & 8 & $(9,2)$ & 5 & 3 & YES & YES & NO(2) & $0.22$ & $(6,-1)$ & NO & 270\\
$(a;1,0,0;13)$ & 5 & $(2,1)$ & 1 & 1 & YES & YES & YES & $0.70$ & $(2,1)$ & -- & 271\\
$(a;1,0,0;13)$ & 5 & $(5,2)$ & 3 & 1 & YES & YES & NO(2) & $0.44$ & $(6,-1)$ & -- & 272\\
$(b;0,0,0;14)$ & 5 & $(2,1)$ & 1 & 2 & YES & YES & NO(2) & $0.44$ & $(6,-1)$ & -- & 273\\
$(c;0,1,1;5)$ & 6 & $(2,1)$ & 1 & 1 & YES & YES & YES & $0.73$ & $(2,1)$ & -- & 274\\
$(c;0,2,0;7)$ & 6 & $(2,1)$ & 1 & 1 & YES & YES & YES & $0.60$ & $(2,1)$ & -- & 275\\
$(d;0,0,0;5)$ & 5 & $(2,1)$ & 1 & 1 & YES & YES & YES & $0.60$ & $(2,1)$ & -- & 276\\
$(d;0,0,0;5)$ & 5 & $(3,1)$ & 2 & 1 & YES & YES & YES & $0.60$ & $(2,1)$ & -- & 277\\
$(f;0,0,0;6)$ & 4 & $(4,1)$ & 3 & 2 & YES & YES & YES & $0.44$ & $(4,0)$ & -- & 278\\
$(f;0,0,0;6)$ & 4 & $(5,2)$ & 3 & 1 & YES & YES & YES & $0.56$ & $(2,1)$ & -- & 279\\
$(f;0,0,0;6)$ & 4 & $(7,2)$ & 4 & 1 & YES & YES & YES & $0.70$ & $(2,1)$ & -- & 280\\
$(f;0,0,0;6)$ & 4 & $(9,2)$ & 5 & 3 & YES & YES & YES & $0.82$ & $(2,1)$ & -- & 281\\
$(f;0,1,0;7)$ & 5 & $(2,1)$ & 1 & 1 & YES & YES & YES & $0.73$ & $(2,1)$ & -- & 282\\
$(f;0,1,0;7)$ & 5 & $(4,1)$ & 3 & 1 & YES & YES & YES & $0.82$ & $(2,1)$ & -- & 283\\
$(j;0,0,0;8)$ & 5 & $(3,1)$ & 2 & 1 & YES & YES & YES & $0.60$ & $(2,1)$ & -- & 284\\
$(j;0,1,0;10)$ & 6 & $(3,1)$ & 2 & 1 & YES & YES & NO(2) & $0.22$ & $(6,-1)$ & -- & 285\\
$(j;0,1,0;10)$ & 6 & $(4,1)$ & 3 & 2 & YES & YES & NO(2) & $0.22$ & $(6,-1)$ & -- & 286
\end{longtable}
\subsection{2 chains, $K^2 = 2$}
\begin{longtable}{|c|c|c|c|c|c|c|c|c|c|c|c|}
\hline
\multicolumn{12}{|c|}{2 chains, $K^2 = 2$}\\
\hline
$(n,a)$ & Len & $(n,a)$ & Len & GCD & Nef & $\mathbb Q$-ef & Obs 0 & $\overline c_1^2 / \overline c_2$ & $(P,K)$ & WH & Index\\
\hline
\endfirsthead

\hline
$(n,a)$ & Len & $(n,a)$ & Len & GCD & Nef & $\mathbb Q$-ef & Obs 0 & $\overline c_1^2 / \overline c_2$ & $(P,K)$ & WH & Index\\
\hline
\endhead
\hline
\endfoot

$(11,3)$ & 5 & $(5,2)$ & 3 & 1 & YES & YES & YES & $1.00$ & $(2,2)$ & -- & 287\\
$(11,4)$ & 5 & $(9,2)$ & 5 & 1 & YES & YES & YES & $1.00$ & $(2,2)$ & -- & 288\\
$(11,4)$ & 5 & $(9,2)$ & 5 & 1 & YES & YES & NO(2) & $1.27$ & $(2,2)$ & NO & 289\\
$(11,4)$ & 5 & $(10,3)$ & 5 & 1 & YES & YES & YES & $1.20$ & $(2,2)$ & NO & 290\\
$(12,5)$ & 5 & $(11,5)$ & 6 & 1 & YES & YES & YES & $0.88$ & $(2,2)$ & -- & 291\\
$(13,3)$ & 6 & $(9,4)$ & 5 & 1 & YES & YES & YES & $0.88$ & $(4,1)$ & NO & 292\\
$(13,3)$ & 6 & $(9,4)$ & 5 & 1 & YES & YES & YES & $0.88$ & $(4,1)$ & -- & 293\\
$(13,3)$ & 6 & $(9,4)$ & 5 & 1 & YES & YES & YES & $0.88$ & $(4,1)$ & NO & 294\\
$(13,5)$ & 5 & $(9,2)$ & 5 & 1 & YES & YES & YES & $0.88$ & $(4,1)$ & -- & 295\\
$(13,6)$ & 7 & $(10,3)$ & 5 & 1 & YES & YES & YES & $0.88$ & $(4,1)$ & NO & 296\\
$(13,6)$ & 7 & $(10,3)$ & 5 & 1 & YES & YES & YES & $0.88$ & $(4,1)$ & -- & 297\\
$(13,3)$ & 6 & $(11,4)$ & 5 & 1 & YES & YES & YES & $1.20$ & $(2,2)$ & NO & 298\\
$(13,3)$ & 6 & $(11,4)$ & 5 & 1 & YES & YES & YES & $1.20$ & $(2,2)$ & -- & 299\\
$(13,3)$ & 6 & $(11,4)$ & 5 & 1 & YES & YES & YES & $1.20$ & $(2,2)$ & 498 & 300\\
$(13,3)$ & 6 & $(11,5)$ & 6 & 1 & YES & YES & NO(2) & $1.10$ & $(2,2)$ & -- & 301\\
$(13,4)$ & 6 & $(11,2)$ & 6 & 1 & YES & YES & NO(2) & $1.11$ & $(4,1)$ & NO & 302\\
$(13,4)$ & 6 & $(11,2)$ & 6 & 1 & YES & YES & NO(2) & $1.11$ & $(4,1)$ & -- & 303\\
$(13,5)$ & 5 & $(11,4)$ & 5 & 1 & YES & YES & YES & $0.88$ & $(4,1)$ & -- & 304\\
$(13,5)$ & 5 & $(11,5)$ & 6 & 1 & YES & YES & YES & $0.88$ & $(2,2)$ & -- & 305\\
$(13,6)$ & 7 & $(13,3)$ & 6 & 13 & YES & YES & YES & $0.88$ & $(4,1)$ & NO & 306\\
$(14,5)$ & 6 & $(9,2)$ & 5 & 1 & YES & YES & YES & $0.75$ & $(4,1)$ & NO & 307\\
$(14,5)$ & 6 & $(9,2)$ & 5 & 1 & YES & YES & YES & $0.75$ & $(4,1)$ & -- & 308\\
$(14,3)$ & 6 & $(10,3)$ & 5 & 2 & YES & YES & YES & $0.88$ & $(4,1)$ & NO & 309\\
$(14,3)$ & 6 & $(10,3)$ & 5 & 2 & YES & YES & YES & $0.88$ & $(4,1)$ & -- & 310\\
$(14,5)$ & 6 & $(10,3)$ & 5 & 2 & YES & YES & YES & $0.88$ & $(4,1)$ & -- & 311\\
$(14,3)$ & 6 & $(11,3)$ & 5 & 1 & YES & YES & YES & $0.88$ & $(4,1)$ & NO & 312\\
$(14,3)$ & 6 & $(11,3)$ & 5 & 1 & YES & YES & YES & $0.88$ & $(4,1)$ & -- & 313\\
$(14,5)$ & 6 & $(11,3)$ & 5 & 1 & YES & YES & YES & $1.20$ & $(2,2)$ & NO & 314\\
$(14,5)$ & 6 & $(11,3)$ & 5 & 1 & YES & YES & YES & $1.20$ & $(2,2)$ & -- & 315\\
$(14,3)$ & 6 & $(13,4)$ & 6 & 1 & YES & YES & NO(2) & $0.89$ & $(10,-2)$ & -- & 316\\
$(14,3)$ & 6 & $(13,4)$ & 6 & 1 & YES & YES & NO(2) & $1.00$ & $(10,-2)$ & NO & 317\\
$(14,5)$ & 6 & $(13,3)$ & 6 & 1 & YES & YES & NO(2) & $0.75$ & $(8,-1)$ & -- & 318\\
$(15,4)$ & 6 & $(7,2)$ & 4 & 1 & YES & YES & NO(2) & $1.00$ & $(4,1)$ & -- & 319\\
$(15,4)$ & 6 & $(11,2)$ & 6 & 1 & YES & YES & NO(2) & $1.00$ & $(4,1)$ & NO & 320\\
$(15,4)$ & 6 & $(11,2)$ & 6 & 1 & YES & YES & NO(2) & $1.00$ & $(4,1)$ & -- & 321\\
$(15,4)$ & 6 & $(11,3)$ & 5 & 1 & YES & YES & NO(2) & $0.75$ & $(8,-1)$ & NO & 322\\
$(15,4)$ & 6 & $(11,3)$ & 5 & 1 & YES & YES & NO(2) & $0.75$ & $(8,-1)$ & -- & 323\\
$(15,4)$ & 6 & $(12,5)$ & 5 & 3 & YES & YES & YES & $1.00$ & $(2,2)$ & -- & 324\\
$(16,7)$ & 6 & $(8,3)$ & 4 & 8 & YES & YES & YES & $0.88$ & $(4,1)$ & -- & 325\\
$(16,5)$ & 7 & $(9,2)$ & 5 & 1 & YES & YES & YES & $1.11$ & $(2,2)$ & NO & 326\\
$(16,5)$ & 7 & $(9,2)$ & 5 & 1 & YES & YES & YES & $1.11$ & $(2,2)$ & -- & 327\\
$(16,5)$ & 7 & $(9,2)$ & 5 & 1 & YES & YES & YES & $1.11$ & $(2,2)$ & NO & 328\\
$(16,5)$ & 7 & $(9,4)$ & 5 & 1 & YES & YES & YES & $1.22$ & $(2,2)$ & NO & 329\\
$(16,5)$ & 7 & $(9,4)$ & 5 & 1 & YES & YES & YES & $1.22$ & $(2,2)$ & -- & 330\\
$(16,5)$ & 7 & $(10,3)$ & 5 & 2 & YES & YES & NO(2) & $1.00$ & $(6,0)$ & -- & 331\\
$(16,5)$ & 7 & $(11,2)$ & 6 & 1 & YES & YES & NO(2) & $1.20$ & $(2,2)$ & -- & 332\\
$(16,5)$ & 7 & $(11,3)$ & 5 & 1 & YES & YES & YES & $1.11$ & $(2,2)$ & NO & 333\\
$(16,5)$ & 7 & $(11,3)$ & 5 & 1 & YES & YES & YES & $1.11$ & $(2,2)$ & -- & 334\\
$(16,5)$ & 7 & $(12,5)$ & 5 & 4 & YES & YES & YES & $0.88$ & $(2,2)$ & NO & 335\\
$(16,5)$ & 7 & $(12,5)$ & 5 & 4 & YES & YES & YES & $1.11$ & $(2,2)$ & NO & 336\\
$(16,5)$ & 7 & $(12,5)$ & 5 & 4 & YES & YES & YES & $1.11$ & $(2,2)$ & -- & 337\\
$(16,7)$ & 6 & $(15,4)$ & 6 & 1 & YES & YES & YES & $1.00$ & $(2,2)$ & NO & 338\\
$(17,7)$ & 6 & $(5,1)$ & 4 & 1 & YES & YES & YES & $0.88$ & $(4,1)$ & -- & 339\\
$(17,7)$ & 6 & $(6,1)$ & 5 & 1 & YES & YES & YES & $0.88$ & $(4,1)$ & NO & 340\\
$(17,7)$ & 6 & $(6,1)$ & 5 & 1 & YES & YES & YES & $0.88$ & $(4,1)$ & -- & 341\\
$(17,6)$ & 7 & $(7,2)$ & 4 & 1 & YES & YES & YES & $1.11$ & $(2,2)$ & NO & 342\\
$(17,6)$ & 7 & $(7,2)$ & 4 & 1 & YES & YES & YES & $1.11$ & $(2,2)$ & -- & 343\\
$(17,7)$ & 6 & $(7,2)$ & 4 & 1 & YES & YES & NO(2) & $1.00$ & $(10,-2)$ & -- & 344\\
$(17,5)$ & 6 & $(9,4)$ & 5 & 1 & YES & YES & NO(2) & $1.20$ & $(4,1)$ & NO & 345\\
$(17,5)$ & 6 & $(9,4)$ & 5 & 1 & YES & YES & NO(2) & $1.20$ & $(4,1)$ & -- & 346\\
$(17,4)$ & 7 & $(11,5)$ & 6 & 1 & YES & YES & YES & $1.11$ & $(2,2)$ & NO & 347\\
$(17,6)$ & 7 & $(13,3)$ & 6 & 1 & YES & YES & YES & $1.20$ & $(2,2)$ & NO & 348\\
$(17,6)$ & 7 & $(13,5)$ & 5 & 1 & YES & YES & YES & $1.11$ & $(2,2)$ & NO & 349\\
$(17,7)$ & 6 & $(13,6)$ & 7 & 1 & YES & YES & YES & $0.88$ & $(4,1)$ & NO & 350\\
$(17,4)$ & 7 & $(14,5)$ & 6 & 1 & YES & YES & YES & $1.11$ & $(2,2)$ & NO & 351\\
$(17,6)$ & 7 & $(14,3)$ & 6 & 1 & YES & YES & NO(2) & $1.00$ & $(6,0)$ & NO & 352\\
$(17,4)$ & 7 & $(16,7)$ & 6 & 1 & YES & YES & YES & $1.00$ & $(2,2)$ & NO & 353\\
$(17,7)$ & 6 & $(16,7)$ & 6 & 1 & YES & YES & YES & $0.88$ & $(4,1)$ & NO & 354\\
$(17,7)$ & 6 & $(16,7)$ & 6 & 1 & YES & YES & NO(2) & $1.20$ & $(4,1)$ & -- & 355\\
$(18,7)$ & 6 & $(5,1)$ & 4 & 1 & YES & YES & YES & $0.88$ & $(4,1)$ & -- & 356\\
$(18,7)$ & 6 & $(5,1)$ & 4 & 1 & YES & YES & YES & $1.20$ & $(2,2)$ & NO & 357\\
$(18,7)$ & 6 & $(6,1)$ & 5 & 6 & YES & YES & YES & $0.88$ & $(4,1)$ & NO & 358\\
$(18,7)$ & 6 & $(6,1)$ & 5 & 6 & YES & YES & YES & $0.88$ & $(4,1)$ & -- & 359\\
$(18,7)$ & 6 & $(6,1)$ & 5 & 6 & YES & YES & YES & $1.20$ & $(2,2)$ & NO & 360\\
$(18,7)$ & 6 & $(9,2)$ & 5 & 9 & YES & YES & NO(2) & $0.88$ & $(8,-1)$ & NO & 361\\
$(18,7)$ & 6 & $(9,2)$ & 5 & 9 & YES & YES & NO(2) & $0.88$ & $(8,-1)$ & -- & 362\\
$(18,7)$ & 6 & $(9,4)$ & 5 & 9 & YES & YES & NO(2) & $1.11$ & $(6,0)$ & NO & 363\\
$(18,5)$ & 6 & $(11,5)$ & 6 & 1 & YES & YES & NO(2) & $1.00$ & $(6,0)$ & NO & 364\\
$(18,5)$ & 6 & $(14,5)$ & 6 & 2 & YES & YES & NO(2) & $1.00$ & $(6,0)$ & NO & 365\\
$(19,4)$ & 7 & $(5,2)$ & 3 & 1 & YES & YES & YES & $1.11$ & $(2,2)$ & NO & 366\\
$(19,4)$ & 7 & $(5,2)$ & 3 & 1 & YES & YES & YES & $1.11$ & $(2,2)$ & -- & 367\\
$(19,5)$ & 7 & $(5,2)$ & 3 & 1 & YES & YES & YES & $1.11$ & $(2,2)$ & NO & 368\\
$(19,6)$ & 8 & $(5,2)$ & 3 & 1 & YES & YES & YES & $1.30$ & $(2,2)$ & NO & 369\\
$(19,8)$ & 6 & $(5,1)$ & 4 & 1 & YES & YES & NO(2) & $1.11$ & $(4,1)$ & -- & 370\\
$(19,8)$ & 6 & $(6,1)$ & 5 & 1 & YES & YES & NO(2) & $1.11$ & $(4,1)$ & NO & 371\\
$(19,8)$ & 6 & $(6,1)$ & 5 & 1 & YES & YES & NO(2) & $1.11$ & $(4,1)$ & -- & 372\\
$(19,8)$ & 6 & $(6,1)$ & 5 & 1 & YES & YES & NO(2) & $1.11$ & $(4,1)$ & NO & 373\\
$(19,4)$ & 7 & $(7,2)$ & 4 & 1 & YES & YES & YES & $1.11$ & $(2,2)$ & NO & 374\\
$(19,4)$ & 7 & $(7,2)$ & 4 & 1 & YES & YES & YES & $1.11$ & $(2,2)$ & -- & 375\\
$(19,5)$ & 7 & $(7,2)$ & 4 & 1 & YES & YES & YES & $1.00$ & $(2,2)$ & -- & 376\\
$(19,5)$ & 7 & $(7,3)$ & 4 & 1 & YES & YES & YES & $1.20$ & $(2,2)$ & -- & 377\\
$(19,6)$ & 8 & $(7,3)$ & 4 & 1 & YES & YES & YES & $1.00$ & $(2,2)$ & NO & 378\\
$(19,6)$ & 8 & $(7,3)$ & 4 & 1 & YES & YES & YES & $1.00$ & $(2,2)$ & -- & 379\\
$(19,7)$ & 6 & $(8,3)$ & 4 & 1 & YES & YES & YES & $0.88$ & $(4,1)$ & -- & 380\\
$(19,8)$ & 6 & $(8,3)$ & 4 & 1 & YES & YES & YES & $1.11$ & $(2,2)$ & NO & 381\\
$(19,8)$ & 6 & $(8,3)$ & 4 & 1 & YES & YES & NO(2) & $1.00$ & $(6,0)$ & -- & 382\\
$(19,4)$ & 7 & $(9,4)$ & 5 & 1 & YES & YES & YES & $1.11$ & $(2,2)$ & NO & 383\\
$(19,4)$ & 7 & $(9,4)$ & 5 & 1 & YES & YES & YES & $1.11$ & $(2,2)$ & -- & 384\\
$(19,7)$ & 6 & $(9,4)$ & 5 & 1 & YES & YES & YES & $0.88$ & $(2,2)$ & NO & 385\\
$(19,7)$ & 6 & $(9,4)$ & 5 & 1 & YES & YES & YES & $0.88$ & $(2,2)$ & -- & 386\\
$(19,5)$ & 7 & $(10,3)$ & 5 & 1 & YES & YES & YES & $1.11$ & $(2,2)$ & NO & 387\\
$(19,5)$ & 7 & $(10,3)$ & 5 & 1 & YES & YES & YES & $1.11$ & $(2,2)$ & -- & 388\\
$(19,7)$ & 6 & $(10,3)$ & 5 & 1 & YES & YES & YES & $0.89$ & $(2,2)$ & -- & 389\\
$(19,4)$ & 7 & $(11,4)$ & 5 & 1 & YES & YES & YES & $1.11$ & $(2,2)$ & -- & 390\\
$(19,7)$ & 6 & $(11,5)$ & 6 & 1 & YES & YES & NO(2) & $1.20$ & $(4,1)$ & -- & 391\\
$(19,7)$ & 6 & $(11,5)$ & 6 & 1 & YES & YES & NO(2) & $1.30$ & $(4,1)$ & NO & 392\\
$(19,7)$ & 6 & $(14,5)$ & 6 & 1 & YES & YES & YES & $0.75$ & $(4,1)$ & NO & 393\\
$(19,3)$ & 8 & $(17,6)$ & 7 & 1 & YES & YES & YES & $0.88$ & $(2,2)$ & NO & 394\\
$(19,7)$ & 6 & $(17,4)$ & 7 & 1 & YES & YES & YES & $1.00$ & $(2,2)$ & NO & 395\\
$(19,7)$ & 6 & $(17,6)$ & 7 & 1 & YES & YES & NO(2) & $1.10$ & $(2,2)$ & 585 & 396\\
$(19,7)$ & 6 & $(18,7)$ & 6 & 1 & YES & YES & YES & $0.88$ & $(4,1)$ & NO & 397\\
$(20,9)$ & 7 & $(5,2)$ & 3 & 5 & YES & YES & NO(2) & $1.10$ & $(2,2)$ & -- & 398\\
$(20,9)$ & 7 & $(7,2)$ & 4 & 1 & YES & YES & YES & $0.89$ & $(2,2)$ & -- & 399\\
$(20,9)$ & 7 & $(8,3)$ & 4 & 4 & YES & YES & YES & $0.75$ & $(4,1)$ & -- & 400\\
$(20,9)$ & 7 & $(10,3)$ & 5 & 10 & YES & YES & YES & $0.88$ & $(2,2)$ & NO & 401\\
$(20,9)$ & 7 & $(11,3)$ & 5 & 1 & YES & YES & YES & $0.75$ & $(4,1)$ & NO & 402\\
$(20,9)$ & 7 & $(11,3)$ & 5 & 1 & YES & YES & NO(2) & $1.00$ & $(6,0)$ & -- & 403\\
$(20,9)$ & 7 & $(11,4)$ & 5 & 1 & YES & YES & YES & $0.88$ & $(2,2)$ & NO & 404\\
$(20,3)$ & 8 & $(13,6)$ & 7 & 1 & YES & YES & YES & $0.88$ & $(4,1)$ & NO & 405\\
$(20,7)$ & 8 & $(13,3)$ & 6 & 1 & YES & YES & NO(2) & $1.10$ & $(4,1)$ & -- & 406\\
$(20,9)$ & 7 & $(13,3)$ & 6 & 1 & YES & YES & YES & $0.75$ & $(4,1)$ & NO & 407\\
$(20,3)$ & 8 & $(17,6)$ & 7 & 1 & YES & YES & YES & $1.20$ & $(2,2)$ & NO & 408\\
$(20,9)$ & 7 & $(17,7)$ & 6 & 1 & YES & YES & YES & $1.11$ & $(2,2)$ & 542 & 409\\
$(20,9)$ & 7 & $(19,8)$ & 6 & 1 & YES & YES & YES & $1.11$ & $(2,2)$ & NO & 410\\
$(21,8)$ & 6 & $(5,1)$ & 4 & 1 & YES & YES & NO(2) & $1.00$ & $(6,0)$ & NO & 411\\
$(21,8)$ & 6 & $(6,1)$ & 5 & 3 & YES & YES & NO(2) & $1.11$ & $(4,1)$ & NO & 412\\
$(21,8)$ & 6 & $(6,1)$ & 5 & 3 & YES & YES & NO(2) & $1.11$ & $(4,1)$ & -- & 413\\
$(21,8)$ & 6 & $(6,1)$ & 5 & 3 & YES & YES & YES & $1.11$ & $(4,1)$ & NO & 414\\
$(21,5)$ & 8 & $(7,3)$ & 4 & 7 & YES & YES & NO(2) & $0.88$ & $(8,-1)$ & -- & 415\\
$(21,8)$ & 6 & $(9,4)$ & 5 & 3 & YES & YES & NO(2) & $1.00$ & $(6,0)$ & NO & 416\\
$(21,8)$ & 6 & $(9,4)$ & 5 & 3 & YES & YES & NO(2) & $1.00$ & $(6,0)$ & -- & 417\\
$(21,8)$ & 6 & $(9,4)$ & 5 & 3 & YES & YES & NO(2) & $1.20$ & $(2,2)$ & NO & 418\\
$(21,8)$ & 6 & $(11,5)$ & 6 & 1 & YES & YES & NO(2) & $1.00$ & $(6,0)$ & NO & 419\\
$(21,4)$ & 8 & $(13,6)$ & 7 & 1 & YES & YES & NO(2) & $1.20$ & $(4,1)$ & NO & 420\\
$(21,4)$ & 8 & $(13,6)$ & 7 & 1 & YES & YES & NO(2) & $1.20$ & $(4,1)$ & NO & 421\\
$(21,5)$ & 8 & $(13,4)$ & 6 & 1 & YES & YES & YES & $1.10$ & $(2,2)$ & NO & 422\\
$(21,8)$ & 6 & $(14,5)$ & 6 & 7 & YES & YES & NO(2) & $1.20$ & $(2,2)$ & NO & 423\\
$(21,5)$ & 8 & $(21,4)$ & 8 & 21 & YES & YES & YES & $1.00$ & $(2,2)$ & NO & 424\\
$(22,9)$ & 7 & $(9,4)$ & 5 & 1 & YES & YES & YES & $1.11$ & $(2,2)$ & NO & 425\\
$(22,5)$ & 7 & $(11,5)$ & 6 & 11 & YES & YES & YES & $1.11$ & $(2,2)$ & NO & 426\\
$(22,9)$ & 7 & $(11,5)$ & 6 & 11 & YES & YES & YES & $1.00$ & $(2,2)$ & NO & 427\\
$(22,5)$ & 7 & $(14,5)$ & 6 & 2 & YES & YES & NO(2) & $0.75$ & $(6,0)$ & NO & 428\\
$(23,5)$ & 7 & $(3,1)$ & 2 & 1 & YES & YES & YES & $1.00$ & $(2,2)$ & NO & 429\\
$(23,5)$ & 7 & $(4,1)$ & 3 & 1 & YES & YES & YES & $1.00$ & $(2,2)$ & -- & 430\\
$(23,5)$ & 7 & $(4,1)$ & 3 & 1 & YES & YES & YES & $1.10$ & $(2,2)$ & NO & 431\\
$(23,9)$ & 7 & $(5,1)$ & 4 & 1 & YES & YES & YES & $1.20$ & $(2,2)$ & NO & 432\\
$(23,9)$ & 7 & $(5,2)$ & 3 & 1 & YES & YES & YES & $1.11$ & $(2,2)$ & NO & 433\\
$(23,6)$ & 8 & $(7,3)$ & 4 & 1 & YES & YES & YES & $1.20$ & $(2,2)$ & -- & 434\\
$(23,9)$ & 7 & $(7,3)$ & 4 & 1 & YES & YES & YES & $1.11$ & $(2,2)$ & -- & 435\\
$(23,9)$ & 7 & $(7,3)$ & 4 & 1 & YES & YES & YES & $1.11$ & $(2,2)$ & NO & 436\\
$(23,9)$ & 7 & $(9,4)$ & 5 & 1 & YES & YES & YES & $1.11$ & $(2,2)$ & -- & 437\\
$(23,4)$ & 8 & $(11,5)$ & 6 & 1 & YES & YES & YES & $1.00$ & $(2,2)$ & -- & 438\\
$(23,4)$ & 8 & $(11,5)$ & 6 & 1 & YES & YES & YES & $1.11$ & $(2,2)$ & NO & 439\\
$(23,9)$ & 7 & $(11,4)$ & 5 & 1 & YES & YES & YES & $1.11$ & $(2,2)$ & NO & 440\\
$(23,10)$ & 7 & $(11,5)$ & 6 & 1 & YES & YES & NO(2) & $1.20$ & $(2,2)$ & 731 & 441\\
$(23,4)$ & 8 & $(13,6)$ & 7 & 1 & YES & YES & NO(2) & $1.20$ & $(4,1)$ & NO & 442\\
$(23,4)$ & 8 & $(13,6)$ & 7 & 1 & YES & YES & NO(2) & $1.20$ & $(4,1)$ & NO & 443\\
$(23,6)$ & 8 & $(13,4)$ & 6 & 1 & YES & YES & YES & $1.10$ & $(2,2)$ & NO & 444\\
$(23,4)$ & 8 & $(14,5)$ & 6 & 1 & YES & YES & YES & $1.11$ & $(2,2)$ & NO & 445\\
$(23,6)$ & 8 & $(14,3)$ & 6 & 1 & YES & YES & YES & $1.00$ & $(2,2)$ & -- & 446\\
$(23,10)$ & 7 & $(14,3)$ & 6 & 1 & YES & YES & NO(2) & $0.75$ & $(6,0)$ & NO & 447\\
$(23,6)$ & 8 & $(16,3)$ & 7 & 1 & YES & YES & YES & $1.11$ & $(2,2)$ & -- & 448\\
$(23,6)$ & 8 & $(16,3)$ & 7 & 1 & YES & YES & YES & $1.22$ & $(2,2)$ & NO & 449\\
$(23,6)$ & 8 & $(20,3)$ & 8 & 1 & YES & YES & YES & $1.11$ & $(2,2)$ & NO & 450\\
$(23,4)$ & 8 & $(21,5)$ & 8 & 1 & YES & YES & YES & $1.00$ & $(2,2)$ & NO & 451\\
$(24,7)$ & 7 & $(4,1)$ & 3 & 4 & YES & YES & YES & $0.88$ & $(4,1)$ & NO & 452\\
$(24,7)$ & 7 & $(4,1)$ & 3 & 4 & YES & YES & YES & $1.00$ & $(4,1)$ & -- & 453\\
$(24,7)$ & 7 & $(5,1)$ & 4 & 1 & YES & YES & YES & $1.11$ & $(4,1)$ & NO & 454\\
$(24,11)$ & 8 & $(5,2)$ & 3 & 1 & YES & YES & YES & $0.88$ & $(4,1)$ & -- & 455\\
$(24,7)$ & 7 & $(6,1)$ & 5 & 6 & YES & YES & YES & $1.00$ & $(4,1)$ & NO & 456\\
$(24,7)$ & 7 & $(6,1)$ & 5 & 6 & YES & YES & YES & $1.00$ & $(4,1)$ & -- & 457\\
$(24,7)$ & 7 & $(6,1)$ & 5 & 6 & YES & YES & YES & $1.11$ & $(4,1)$ & NO & 458\\
$(24,11)$ & 8 & $(7,3)$ & 4 & 1 & YES & YES & NO(2) & $1.30$ & $(4,1)$ & -- & 459\\
$(24,5)$ & 8 & $(9,4)$ & 5 & 3 & YES & YES & YES & $1.11$ & $(2,2)$ & -- & 460\\
$(24,5)$ & 8 & $(11,4)$ & 5 & 1 & YES & YES & YES & $1.11$ & $(2,2)$ & -- & 461\\
$(24,5)$ & 8 & $(11,4)$ & 5 & 1 & YES & YES & YES & $1.20$ & $(2,2)$ & NO & 462\\
$(24,5)$ & 8 & $(13,4)$ & 6 & 1 & YES & YES & YES & $1.10$ & $(2,2)$ & NO & 463\\
$(24,11)$ & 8 & $(20,9)$ & 7 & 4 & YES & YES & YES & $0.75$ & $(4,1)$ & NO & 464\\
$(24,5)$ & 8 & $(21,5)$ & 8 & 3 & YES & YES & YES & $1.00$ & $(2,2)$ & NO & 465\\
$(24,5)$ & 8 & $(23,4)$ & 8 & 1 & YES & YES & YES & $1.00$ & $(2,2)$ & NO & 466\\
$(25,9)$ & 7 & $(3,1)$ & 2 & 1 & YES & YES & YES & $0.88$ & $(4,1)$ & NO & 467\\
$(25,9)$ & 7 & $(3,1)$ & 2 & 1 & YES & YES & NO(2) & $1.20$ & $(2,2)$ & -- & 468\\
$(25,11)$ & 7 & $(3,1)$ & 2 & 1 & YES & YES & YES & $1.20$ & $(2,2)$ & NO & 469\\
$(25,9)$ & 7 & $(4,1)$ & 3 & 1 & YES & YES & YES & $0.88$ & $(4,1)$ & NO & 470\\
$(25,9)$ & 7 & $(4,1)$ & 3 & 1 & YES & YES & YES & $0.88$ & $(4,1)$ & -- & 471\\
$(25,9)$ & 7 & $(4,1)$ & 3 & 1 & YES & YES & YES & $0.88$ & $(4,1)$ & NO & 472\\
$(25,9)$ & 7 & $(5,2)$ & 3 & 5 & YES & YES & YES & $0.75$ & $(4,1)$ & -- & 473\\
$(25,6)$ & 9 & $(7,3)$ & 4 & 1 & YES & YES & YES & $0.88$ & $(2,2)$ & NO & 474\\
$(25,11)$ & 7 & $(7,2)$ & 4 & 1 & YES & YES & YES & $1.11$ & $(2,2)$ & NO & 475\\
$(25,11)$ & 7 & $(7,2)$ & 4 & 1 & YES & YES & YES & $1.11$ & $(2,2)$ & -- & 476\\
$(25,11)$ & 7 & $(8,3)$ & 4 & 1 & YES & YES & YES & $1.11$ & $(2,2)$ & 752 & 477\\
$(25,11)$ & 7 & $(8,3)$ & 4 & 1 & YES & YES & YES & $1.11$ & $(2,2)$ & -- & 478\\
$(25,7)$ & 7 & $(11,5)$ & 6 & 1 & YES & YES & NO(2) & $1.00$ & $(6,0)$ & -- & 479\\
$(25,9)$ & 7 & $(11,3)$ & 5 & 1 & YES & YES & NO(2) & $1.00$ & $(6,0)$ & NO & 480\\
$(25,9)$ & 7 & $(13,3)$ & 6 & 1 & YES & YES & YES & $1.10$ & $(2,2)$ & NO & 481\\
$(25,11)$ & 7 & $(13,3)$ & 6 & 1 & YES & YES & YES & $1.11$ & $(2,2)$ & NO & 482\\
$(25,11)$ & 7 & $(13,3)$ & 6 & 1 & YES & YES & YES & $1.11$ & $(2,2)$ & -- & 483\\
$(25,9)$ & 7 & $(19,7)$ & 6 & 1 & YES & YES & YES & $0.75$ & $(4,1)$ & NO & 484\\
$(25,6)$ & 9 & $(20,3)$ & 8 & 5 & YES & YES & YES & $1.00$ & $(2,2)$ & NO & 485\\
$(25,4)$ & 9 & $(24,5)$ & 8 & 1 & YES & YES & YES & $1.00$ & $(2,2)$ & NO & 486\\
$(26,7)$ & 7 & $(3,1)$ & 2 & 1 & YES & YES & NO(2) & $1.11$ & $(4,1)$ & -- & 487\\
$(26,7)$ & 7 & $(5,1)$ & 4 & 1 & YES & YES & NO(2) & $1.27$ & $(4,1)$ & NO & 488\\
$(26,7)$ & 7 & $(5,1)$ & 4 & 1 & YES & YES & NO(2) & $1.27$ & $(4,1)$ & -- & 489\\
$(26,11)$ & 7 & $(5,1)$ & 4 & 1 & YES & YES & NO(2) & $0.88$ & $(8,-1)$ & NO & 490\\
$(26,7)$ & 7 & $(7,2)$ & 4 & 1 & YES & YES & YES & $1.00$ & $(2,2)$ & -- & 491\\
$(26,11)$ & 7 & $(7,3)$ & 4 & 1 & YES & YES & YES & $1.22$ & $(2,2)$ & NO & 492\\
$(26,11)$ & 7 & $(7,3)$ & 4 & 1 & YES & YES & NO(2) & $1.20$ & $(4,1)$ & -- & 493\\
$(26,11)$ & 7 & $(8,3)$ & 4 & 2 & YES & YES & NO(2) & $0.75$ & $(6,0)$ & -- & 494\\
$(27,11)$ & 8 & $(3,1)$ & 2 & 3 & YES & YES & YES & $1.20$ & $(2,2)$ & -- & 495\\
$(27,11)$ & 8 & $(4,1)$ & 3 & 1 & YES & YES & YES & $1.11$ & $(2,2)$ & -- & 496\\
$(27,11)$ & 8 & $(4,1)$ & 3 & 1 & YES & YES & YES & $1.20$ & $(2,2)$ & NO & 497\\
$(27,10)$ & 7 & $(5,1)$ & 4 & 1 & YES & YES & YES & $1.20$ & $(2,2)$ & 300 & 498\\
$(27,10)$ & 7 & $(5,1)$ & 4 & 1 & YES & YES & YES & $1.20$ & $(2,2)$ & -- & 499\\
$(27,11)$ & 8 & $(6,1)$ & 5 & 3 & YES & YES & YES & $1.11$ & $(2,2)$ & -- & 500\\
$(27,8)$ & 7 & $(7,3)$ & 4 & 1 & YES & YES & YES & $1.00$ & $(2,2)$ & NO & 501\\
$(27,11)$ & 8 & $(7,2)$ & 4 & 1 & YES & YES & YES & $1.11$ & $(2,2)$ & NO & 502\\
$(27,11)$ & 8 & $(9,4)$ & 5 & 9 & YES & YES & YES & $1.11$ & $(2,2)$ & NO & 503\\
$(27,11)$ & 8 & $(12,5)$ & 5 & 3 & YES & YES & YES & $1.11$ & $(2,2)$ & NO & 504\\
$(27,10)$ & 7 & $(17,6)$ & 7 & 1 & YES & YES & YES & $1.20$ & $(2,2)$ & NO & 505\\
$(27,11)$ & 8 & $(22,9)$ & 7 & 1 & YES & YES & YES & $1.11$ & $(2,2)$ & NO & 506\\
$(27,11)$ & 8 & $(27,11)$ & 8 & 27 & YES & YES & YES & $1.20$ & $(2,2)$ & NO & 507\\
$(28,11)$ & 8 & $(2,1)$ & 1 & 2 & YES & YES & YES & $1.18$ & $(2,2)$ & -- & 508\\
$(28,11)$ & 8 & $(3,1)$ & 2 & 1 & YES & YES & YES & $1.20$ & $(2,2)$ & -- & 509\\
$(28,11)$ & 8 & $(4,1)$ & 3 & 4 & YES & YES & YES & $1.11$ & $(2,2)$ & -- & 510\\
$(28,11)$ & 8 & $(5,2)$ & 3 & 1 & YES & YES & YES & $1.11$ & $(2,2)$ & -- & 511\\
$(28,11)$ & 8 & $(6,1)$ & 5 & 2 & YES & YES & YES & $1.11$ & $(2,2)$ & -- & 512\\
$(28,11)$ & 8 & $(7,2)$ & 4 & 7 & YES & YES & NO(2) & $0.75$ & $(6,0)$ & -- & 513\\
$(28,5)$ & 8 & $(11,5)$ & 6 & 1 & YES & YES & YES & $1.11$ & $(2,2)$ & NO & 514\\
$(28,11)$ & 8 & $(11,2)$ & 6 & 1 & YES & YES & NO(2) & $0.75$ & $(6,0)$ & -- & 515\\
$(28,11)$ & 8 & $(13,5)$ & 5 & 1 & YES & YES & YES & $1.11$ & $(2,2)$ & NO & 516\\
$(28,5)$ & 8 & $(14,5)$ & 6 & 14 & YES & YES & YES & $1.10$ & $(2,2)$ & -- & 517\\
$(28,5)$ & 8 & $(14,5)$ & 6 & 14 & YES & YES & NO(2) & $0.75$ & $(6,0)$ & NO & 518\\
$(28,5)$ & 8 & $(21,5)$ & 8 & 7 & YES & YES & NO(2) & $0.75$ & $(6,0)$ & NO & 519\\
$(28,11)$ & 8 & $(23,9)$ & 7 & 1 & YES & YES & YES & $1.11$ & $(2,2)$ & NO & 520\\
$(28,11)$ & 8 & $(28,11)$ & 8 & 28 & YES & YES & YES & $1.20$ & $(2,2)$ & NO & 521\\
$(29,11)$ & 7 & $(3,1)$ & 2 & 1 & YES & YES & YES & $1.00$ & $(2,2)$ & -- & 522\\
$(29,9)$ & 8 & $(4,1)$ & 3 & 1 & YES & YES & YES & $1.20$ & $(2,2)$ & NO & 523\\
$(29,9)$ & 8 & $(4,1)$ & 3 & 1 & YES & YES & YES & $1.20$ & $(2,2)$ & -- & 524\\
$(29,9)$ & 8 & $(4,1)$ & 3 & 1 & YES & YES & YES & $1.20$ & $(2,2)$ & NO & 525\\
$(29,11)$ & 7 & $(4,1)$ & 3 & 1 & YES & YES & NO(2) & $0.75$ & $(8,-1)$ & -- & 526\\
$(29,13)$ & 8 & $(4,1)$ & 3 & 1 & YES & YES & YES & $1.00$ & $(2,2)$ & -- & 527\\
$(29,6)$ & 9 & $(5,2)$ & 3 & 1 & YES & YES & NO(2) & $1.00$ & $(2,2)$ & -- & 528\\
$(29,9)$ & 8 & $(5,1)$ & 4 & 1 & YES & YES & YES & $1.00$ & $(2,2)$ & NO & 529\\
$(29,9)$ & 8 & $(5,1)$ & 4 & 1 & YES & YES & YES & $1.00$ & $(2,2)$ & -- & 530\\
$(29,9)$ & 8 & $(5,2)$ & 3 & 1 & YES & YES & YES & $1.22$ & $(2,2)$ & NO & 531\\
$(29,9)$ & 8 & $(5,2)$ & 3 & 1 & YES & YES & YES & $1.22$ & $(2,2)$ & -- & 532\\
$(29,11)$ & 7 & $(5,2)$ & 3 & 1 & YES & YES & YES & $1.00$ & $(2,2)$ & -- & 533\\
$(29,13)$ & 8 & $(5,2)$ & 3 & 1 & YES & YES & YES & $0.88$ & $(2,2)$ & -- & 534\\
$(29,9)$ & 8 & $(7,3)$ & 4 & 1 & YES & YES & YES & $1.00$ & $(2,2)$ & NO & 535\\
$(29,11)$ & 7 & $(7,3)$ & 4 & 1 & YES & YES & YES & $1.00$ & $(2,2)$ & 750 & 536\\
$(29,9)$ & 8 & $(8,3)$ & 4 & 1 & YES & YES & YES & $0.75$ & $(4,1)$ & NO & 537\\
$(29,13)$ & 8 & $(9,2)$ & 5 & 1 & YES & YES & YES & $1.11$ & $(2,2)$ & NO & 538\\
$(29,13)$ & 8 & $(9,2)$ & 5 & 1 & YES & YES & YES & $1.11$ & $(2,2)$ & -- & 539\\
$(29,13)$ & 8 & $(9,2)$ & 5 & 1 & YES & YES & NO(2) & $1.20$ & $(4,1)$ & NO & 540\\
$(29,12)$ & 7 & $(10,3)$ & 5 & 1 & YES & YES & NO(2) & $0.62$ & $(6,0)$ & -- & 541\\
$(29,12)$ & 7 & $(11,5)$ & 6 & 1 & YES & YES & YES & $1.11$ & $(2,2)$ & 409 & 542\\
$(29,13)$ & 8 & $(12,5)$ & 5 & 1 & YES & YES & YES & $1.11$ & $(2,2)$ & NO & 543\\
$(29,6)$ & 9 & $(13,3)$ & 6 & 1 & YES & YES & NO(2) & $1.10$ & $(2,2)$ & NO & 544\\
$(29,7)$ & 10 & $(13,3)$ & 6 & 1 & YES & YES & NO(2) & $1.10$ & $(4,1)$ & -- & 545\\
$(29,6)$ & 9 & $(23,4)$ & 8 & 1 & YES & YES & NO(2) & $1.20$ & $(4,1)$ & NO & 546\\
$(29,9)$ & 8 & $(23,7)$ & 7 & 1 & YES & YES & YES & $1.11$ & $(2,2)$ & NO & 547\\
$(29,4)$ & 10 & $(25,6)$ & 9 & 1 & YES & YES & NO(2) & $1.10$ & $(4,1)$ & NO & 548\\
$(29,12)$ & 7 & $(26,11)$ & 7 & 1 & YES & YES & NO(2) & $0.62$ & $(6,0)$ & NO & 549\\
$(29,6)$ & 9 & $(29,4)$ & 10 & 29 & YES & YES & NO(2) & $1.10$ & $(4,1)$ & NO & 550\\
$(29,11)$ & 7 & $(29,11)$ & 7 & 29 & YES & YES & YES & $1.00$ & $(2,2)$ & NO & 551\\
$(30,11)$ & 7 & $(3,1)$ & 2 & 3 & YES & YES & YES & $0.89$ & $(2,2)$ & -- & 552\\
$(30,11)$ & 7 & $(5,1)$ & 4 & 5 & YES & YES & NO(2) & $0.88$ & $(8,-1)$ & NO & 553\\
$(30,11)$ & 7 & $(5,2)$ & 3 & 5 & YES & YES & YES & $1.00$ & $(2,2)$ & -- & 554\\
$(30,11)$ & 7 & $(5,2)$ & 3 & 5 & YES & YES & YES & $1.00$ & $(2,2)$ & 647 & 555\\
$(30,11)$ & 7 & $(7,2)$ & 4 & 1 & YES & YES & YES & $1.11$ & $(2,2)$ & NO & 556\\
$(30,11)$ & 7 & $(7,2)$ & 4 & 1 & YES & YES & YES & $1.11$ & $(2,2)$ & -- & 557\\
$(30,11)$ & 7 & $(7,3)$ & 4 & 1 & YES & YES & YES & $1.00$ & $(2,2)$ & 853 & 558\\
$(30,13)$ & 8 & $(7,3)$ & 4 & 1 & YES & YES & NO(2) & $1.20$ & $(4,1)$ & NO & 559\\
$(30,13)$ & 8 & $(7,3)$ & 4 & 1 & YES & YES & NO(2) & $1.20$ & $(4,1)$ & -- & 560\\
$(30,11)$ & 7 & $(10,3)$ & 5 & 10 & YES & YES & YES & $0.88$ & $(2,2)$ & NO & 561\\
$(30,11)$ & 7 & $(13,5)$ & 5 & 1 & YES & YES & YES & $1.11$ & $(2,2)$ & NO & 562\\
$(30,13)$ & 8 & $(13,6)$ & 7 & 1 & YES & YES & NO(2) & $1.20$ & $(4,1)$ & NO & 563\\
$(30,11)$ & 7 & $(17,6)$ & 7 & 1 & YES & YES & YES & $0.88$ & $(2,2)$ & NO & 564\\
$(30,11)$ & 7 & $(30,11)$ & 7 & 30 & YES & YES & NO(2) & $0.89$ & $(6,0)$ & NO & 565\\
$(31,7)$ & 8 & $(2,1)$ & 1 & 1 & YES & YES & YES & $1.00$ & $(2,2)$ & -- & 566\\
$(31,9)$ & 8 & $(2,1)$ & 1 & 1 & YES & YES & YES & $1.10$ & $(2,2)$ & -- & 567\\
$(31,7)$ & 8 & $(3,1)$ & 2 & 1 & YES & YES & YES & $1.11$ & $(2,2)$ & NO & 568\\
$(31,7)$ & 8 & $(3,1)$ & 2 & 1 & YES & YES & YES & $1.11$ & $(2,2)$ & -- & 569\\
$(31,11)$ & 8 & $(3,1)$ & 2 & 1 & YES & YES & YES & $1.11$ & $(2,2)$ & NO & 570\\
$(31,11)$ & 8 & $(3,1)$ & 2 & 1 & YES & YES & YES & $1.11$ & $(2,2)$ & -- & 571\\
$(31,11)$ & 8 & $(4,1)$ & 3 & 1 & YES & YES & NO(2) & $1.10$ & $(2,2)$ & -- & 572\\
$(31,14)$ & 8 & $(4,1)$ & 3 & 1 & YES & YES & YES & $1.00$ & $(2,2)$ & NO & 573\\
$(31,14)$ & 8 & $(4,1)$ & 3 & 1 & YES & YES & YES & $1.00$ & $(2,2)$ & -- & 574\\
$(31,7)$ & 8 & $(5,1)$ & 4 & 1 & YES & YES & YES & $1.00$ & $(2,2)$ & NO & 575\\
$(31,7)$ & 8 & $(5,1)$ & 4 & 1 & YES & YES & YES & $1.00$ & $(2,2)$ & -- & 576\\
$(31,7)$ & 8 & $(5,1)$ & 4 & 1 & YES & YES & YES & $1.00$ & $(2,2)$ & NO & 577\\
$(31,9)$ & 8 & $(5,2)$ & 3 & 1 & YES & YES & YES & $1.10$ & $(2,2)$ & -- & 578\\
$(31,11)$ & 8 & $(5,2)$ & 3 & 1 & YES & YES & YES & $1.00$ & $(2,2)$ & -- & 579\\
$(31,13)$ & 7 & $(5,2)$ & 3 & 1 & YES & YES & YES & $1.00$ & $(2,2)$ & NO & 580\\
$(31,13)$ & 7 & $(5,2)$ & 3 & 1 & YES & YES & YES & $1.00$ & $(2,2)$ & -- & 581\\
$(31,7)$ & 8 & $(7,3)$ & 4 & 1 & YES & YES & YES & $1.00$ & $(2,2)$ & -- & 582\\
$(31,11)$ & 8 & $(7,3)$ & 4 & 1 & YES & YES & YES & $1.00$ & $(2,2)$ & NO & 583\\
$(31,12)$ & 7 & $(7,3)$ & 4 & 1 & YES & YES & NO(2) & $0.75$ & $(6,0)$ & -- & 584\\
$(31,11)$ & 8 & $(8,3)$ & 4 & 1 & YES & YES & NO(2) & $1.10$ & $(2,2)$ & 396 & 585\\
$(31,6)$ & 10 & $(9,4)$ & 5 & 1 & YES & YES & YES & $1.20$ & $(2,2)$ & NO & 586\\
$(31,7)$ & 8 & $(9,2)$ & 5 & 1 & YES & YES & NO(2) & $1.27$ & $(2,2)$ & NO & 587\\
$(31,12)$ & 7 & $(9,4)$ & 5 & 1 & YES & YES & NO(2) & $1.20$ & $(4,1)$ & -- & 588\\
$(31,14)$ & 8 & $(13,6)$ & 7 & 1 & YES & YES & YES & $0.88$ & $(4,1)$ & NO & 589\\
$(31,6)$ & 10 & $(19,3)$ & 8 & 1 & YES & YES & YES & $0.88$ & $(2,2)$ & NO & 590\\
$(31,11)$ & 8 & $(19,7)$ & 6 & 1 & YES & YES & YES & $1.11$ & $(2,2)$ & NO & 591\\
$(31,14)$ & 8 & $(20,9)$ & 7 & 1 & YES & YES & YES & $1.00$ & $(2,2)$ & NO & 592\\
$(31,6)$ & 10 & $(23,4)$ & 8 & 1 & YES & YES & YES & $0.88$ & $(2,2)$ & NO & 593\\
$(31,7)$ & 8 & $(24,5)$ & 8 & 1 & YES & YES & YES & $1.00$ & $(2,2)$ & NO & 594\\
$(31,11)$ & 8 & $(25,9)$ & 7 & 1 & YES & YES & NO(2) & $1.00$ & $(6,0)$ & NO & 595\\
$(31,12)$ & 7 & $(28,11)$ & 8 & 1 & YES & YES & NO(2) & $0.75$ & $(6,0)$ & 896 & 596\\
$(31,11)$ & 8 & $(31,11)$ & 8 & 31 & YES & YES & YES & $1.00$ & $(2,2)$ & NO & 597\\
$(31,14)$ & 8 & $(31,14)$ & 8 & 31 & YES & YES & YES & $1.11$ & $(2,2)$ & NO & 598\\
$(32,7)$ & 8 & $(2,1)$ & 1 & 2 & YES & YES & YES & $0.89$ & $(4,1)$ & NO & 599\\
$(32,13)$ & 9 & $(2,1)$ & 1 & 2 & YES & YES & YES & $1.18$ & $(2,2)$ & -- & 600\\
$(32,7)$ & 8 & $(3,1)$ & 2 & 1 & YES & YES & YES & $0.88$ & $(4,1)$ & NO & 601\\
$(32,7)$ & 8 & $(3,1)$ & 2 & 1 & YES & YES & NO(2) & $1.00$ & $(6,0)$ & -- & 602\\
$(32,9)$ & 8 & $(3,1)$ & 2 & 1 & YES & YES & NO(2) & $0.75$ & $(8,-1)$ & NO & 603\\
$(32,13)$ & 9 & $(3,1)$ & 2 & 1 & YES & YES & YES & $1.20$ & $(2,2)$ & -- & 604\\
$(32,7)$ & 8 & $(4,1)$ & 3 & 4 & YES & YES & YES & $1.00$ & $(2,2)$ & NO & 605\\
$(32,7)$ & 8 & $(4,1)$ & 3 & 4 & YES & YES & YES & $1.00$ & $(2,2)$ & -- & 606\\
$(32,13)$ & 9 & $(4,1)$ & 3 & 4 & YES & YES & YES & $1.11$ & $(2,2)$ & -- & 607\\
$(32,7)$ & 8 & $(5,1)$ & 4 & 1 & YES & YES & YES & $0.89$ & $(4,1)$ & NO & 608\\
$(32,7)$ & 8 & $(5,1)$ & 4 & 1 & YES & YES & YES & $0.89$ & $(4,1)$ & -- & 609\\
$(32,9)$ & 8 & $(5,2)$ & 3 & 1 & YES & YES & YES & $1.10$ & $(2,2)$ & -- & 610\\
$(32,13)$ & 9 & $(5,1)$ & 4 & 1 & YES & YES & YES & $1.00$ & $(2,2)$ & -- & 611\\
$(32,13)$ & 9 & $(5,1)$ & 4 & 1 & YES & YES & YES & $1.11$ & $(2,2)$ & NO & 612\\
$(32,13)$ & 9 & $(6,1)$ & 5 & 2 & YES & YES & YES & $1.00$ & $(2,2)$ & NO & 613\\
$(32,9)$ & 8 & $(7,2)$ & 4 & 1 & YES & YES & YES & $1.10$ & $(2,2)$ & NO & 614\\
$(32,13)$ & 9 & $(7,3)$ & 4 & 1 & YES & YES & YES & $1.20$ & $(2,2)$ & NO & 615\\
$(32,7)$ & 8 & $(9,2)$ & 5 & 1 & YES & YES & YES & $1.00$ & $(2,2)$ & NO & 616\\
$(32,7)$ & 8 & $(11,4)$ & 5 & 1 & YES & YES & NO(2) & $1.10$ & $(4,1)$ & NO & 617\\
$(32,9)$ & 8 & $(13,4)$ & 6 & 1 & YES & YES & YES & $1.10$ & $(2,2)$ & NO & 618\\
$(32,7)$ & 8 & $(14,3)$ & 6 & 2 & YES & YES & YES & $0.88$ & $(4,1)$ & 713 & 619\\
$(32,13)$ & 9 & $(17,7)$ & 6 & 1 & YES & YES & YES & $1.00$ & $(2,2)$ & NO & 620\\
$(32,7)$ & 8 & $(21,5)$ & 8 & 1 & YES & YES & YES & $1.00$ & $(2,2)$ & NO & 621\\
$(32,13)$ & 9 & $(22,9)$ & 7 & 2 & YES & YES & YES & $1.00$ & $(2,2)$ & 871 & 622\\
$(32,13)$ & 9 & $(27,11)$ & 8 & 1 & YES & YES & YES & $1.11$ & $(2,2)$ & NO & 623\\
$(32,7)$ & 8 & $(32,7)$ & 8 & 32 & YES & YES & NO(2) & $1.00$ & $(6,0)$ & NO & 624\\
$(33,13)$ & 9 & $(2,1)$ & 1 & 1 & YES & YES & YES & $1.18$ & $(2,2)$ & -- & 625\\
$(33,13)$ & 9 & $(2,1)$ & 1 & 1 & YES & YES & YES & $1.36$ & $(2,2)$ & NO & 626\\
$(33,13)$ & 9 & $(3,1)$ & 2 & 3 & YES & YES & YES & $1.27$ & $(2,2)$ & -- & 627\\
$(33,14)$ & 8 & $(3,1)$ & 2 & 3 & YES & YES & YES & $1.00$ & $(2,2)$ & -- & 628\\
$(33,13)$ & 9 & $(4,1)$ & 3 & 1 & YES & YES & YES & $1.11$ & $(2,2)$ & -- & 629\\
$(33,13)$ & 9 & $(4,1)$ & 3 & 1 & YES & YES & YES & $1.20$ & $(2,2)$ & NO & 630\\
$(33,13)$ & 9 & $(5,1)$ & 4 & 1 & YES & YES & YES & $1.20$ & $(2,2)$ & -- & 631\\
$(33,13)$ & 9 & $(6,1)$ & 5 & 3 & YES & YES & YES & $1.00$ & $(2,2)$ & NO & 632\\
$(33,13)$ & 9 & $(6,1)$ & 5 & 3 & YES & YES & NO(2) & $1.11$ & $(6,0)$ & -- & 633\\
$(33,13)$ & 9 & $(8,3)$ & 4 & 1 & YES & YES & YES & $1.20$ & $(2,2)$ & NO & 634\\
$(33,14)$ & 8 & $(8,3)$ & 4 & 1 & YES & YES & YES & $1.00$ & $(2,2)$ & NO & 635\\
$(33,10)$ & 8 & $(11,4)$ & 5 & 11 & YES & YES & NO(2) & $1.20$ & $(4,1)$ & NO & 636\\
$(33,10)$ & 8 & $(13,4)$ & 6 & 1 & YES & YES & NO(2) & $1.00$ & $(6,0)$ & 689 & 637\\
$(33,13)$ & 9 & $(18,7)$ & 6 & 3 & YES & YES & YES & $1.00$ & $(2,2)$ & NO & 638\\
$(33,13)$ & 9 & $(23,9)$ & 7 & 1 & YES & YES & YES & $1.20$ & $(2,2)$ & 891 & 639\\
$(33,13)$ & 9 & $(28,11)$ & 8 & 1 & YES & YES & YES & $1.11$ & $(2,2)$ & NO & 640\\
$(33,14)$ & 8 & $(33,14)$ & 8 & 33 & YES & YES & YES & $1.00$ & $(2,2)$ & NO & 641\\
$(34,9)$ & 8 & $(2,1)$ & 1 & 2 & YES & YES & YES & $0.88$ & $(4,1)$ & NO & 642\\
$(34,13)$ & 7 & $(2,1)$ & 1 & 2 & YES & YES & YES & $1.00$ & $(2,2)$ & -- & 643\\
$(34,9)$ & 8 & $(3,1)$ & 2 & 1 & YES & YES & YES & $0.88$ & $(4,1)$ & NO & 644\\
$(34,9)$ & 8 & $(3,1)$ & 2 & 1 & YES & YES & YES & $0.88$ & $(4,1)$ & -- & 645\\
$(34,13)$ & 7 & $(3,1)$ & 2 & 1 & YES & YES & YES & $0.89$ & $(2,2)$ & -- & 646\\
$(34,13)$ & 7 & $(3,1)$ & 2 & 1 & YES & YES & YES & $1.00$ & $(2,2)$ & 555 & 647\\
$(34,13)$ & 7 & $(5,2)$ & 3 & 1 & YES & YES & YES & $1.00$ & $(2,2)$ & NO & 648\\
$(34,15)$ & 8 & $(5,2)$ & 3 & 1 & YES & YES & YES & $1.11$ & $(2,2)$ & NO & 649\\
$(34,15)$ & 8 & $(5,2)$ & 3 & 1 & YES & YES & YES & $1.11$ & $(2,2)$ & -- & 650\\
$(34,13)$ & 7 & $(7,3)$ & 4 & 1 & YES & YES & YES & $1.00$ & $(2,2)$ & NO & 651\\
$(34,9)$ & 8 & $(8,3)$ & 4 & 2 & YES & YES & NO(2) & $0.75$ & $(6,0)$ & -- & 652\\
$(34,15)$ & 8 & $(8,3)$ & 4 & 2 & YES & YES & YES & $1.11$ & $(2,2)$ & NO & 653\\
$(34,9)$ & 8 & $(11,3)$ & 5 & 1 & YES & YES & YES & $0.88$ & $(2,2)$ & NO & 654\\
$(34,15)$ & 8 & $(11,5)$ & 6 & 1 & YES & YES & YES & $1.11$ & $(2,2)$ & NO & 655\\
$(34,15)$ & 8 & $(12,5)$ & 5 & 2 & YES & YES & YES & $1.11$ & $(2,2)$ & 952 & 656\\
$(34,15)$ & 8 & $(13,6)$ & 7 & 1 & YES & YES & NO(2) & $1.20$ & $(4,1)$ & NO & 657\\
$(34,9)$ & 8 & $(19,5)$ & 7 & 1 & YES & YES & YES & $1.00$ & $(2,2)$ & NO & 658\\
$(34,15)$ & 8 & $(23,10)$ & 7 & 1 & YES & YES & NO(2) & $0.75$ & $(6,0)$ & 940 & 659\\
$(35,11)$ & 9 & $(2,1)$ & 1 & 1 & YES & YES & YES & $1.22$ & $(2,2)$ & NO & 660\\
$(35,11)$ & 9 & $(3,1)$ & 2 & 1 & YES & YES & YES & $1.00$ & $(2,2)$ & NO & 661\\
$(35,11)$ & 9 & $(3,1)$ & 2 & 1 & YES & YES & YES & $1.00$ & $(2,2)$ & -- & 662\\
$(35,13)$ & 8 & $(3,1)$ & 2 & 1 & YES & YES & YES & $1.10$ & $(2,2)$ & -- & 663\\
$(35,13)$ & 8 & $(4,1)$ & 3 & 1 & YES & YES & YES & $1.11$ & $(2,2)$ & -- & 664\\
$(35,6)$ & 10 & $(5,2)$ & 3 & 5 & YES & YES & NO(2) & $1.10$ & $(2,2)$ & NO & 665\\
$(35,6)$ & 10 & $(5,2)$ & 3 & 5 & YES & YES & NO(2) & $1.10$ & $(2,2)$ & -- & 666\\
$(35,6)$ & 10 & $(5,2)$ & 3 & 5 & YES & YES & NO(2) & $1.20$ & $(2,2)$ & NO & 667\\
$(35,13)$ & 8 & $(6,1)$ & 5 & 1 & YES & YES & YES & $1.00$ & $(2,2)$ & NO & 668\\
$(35,13)$ & 8 & $(6,1)$ & 5 & 1 & YES & YES & YES & $1.00$ & $(2,2)$ & -- & 669\\
$(35,8)$ & 8 & $(7,3)$ & 4 & 7 & YES & YES & YES & $1.00$ & $(2,2)$ & NO & 670\\
$(35,8)$ & 8 & $(7,3)$ & 4 & 7 & YES & YES & YES & $1.00$ & $(2,2)$ & -- & 671\\
$(35,11)$ & 9 & $(7,2)$ & 4 & 7 & YES & YES & YES & $1.00$ & $(2,2)$ & NO & 672\\
$(35,16)$ & 9 & $(9,4)$ & 5 & 1 & YES & YES & YES & $0.88$ & $(4,1)$ & NO & 673\\
$(35,16)$ & 9 & $(11,2)$ & 6 & 1 & YES & YES & NO(2) & $1.20$ & $(4,1)$ & NO & 674\\
$(35,11)$ & 9 & $(13,4)$ & 6 & 1 & YES & YES & YES & $1.00$ & $(2,2)$ & NO & 675\\
$(35,13)$ & 8 & $(14,5)$ & 6 & 7 & YES & YES & YES & $1.11$ & $(2,2)$ & NO & 676\\
$(35,16)$ & 9 & $(16,7)$ & 6 & 1 & YES & YES & NO(2) & $1.20$ & $(4,1)$ & NO & 677\\
$(35,6)$ & 10 & $(20,3)$ & 8 & 5 & YES & YES & YES & $0.75$ & $(4,1)$ & NO & 678\\
$(35,6)$ & 10 & $(22,3)$ & 9 & 1 & YES & YES & YES & $0.75$ & $(4,1)$ & NO & 679\\
$(35,8)$ & 8 & $(25,6)$ & 9 & 5 & YES & YES & YES & $1.00$ & $(2,2)$ & 965 & 680\\
$(36,11)$ & 8 & $(2,1)$ & 1 & 2 & YES & YES & NO(2) & $1.00$ & $(6,0)$ & -- & 681\\
$(36,11)$ & 8 & $(3,1)$ & 2 & 3 & YES & YES & YES & $1.20$ & $(2,2)$ & NO & 682\\
$(36,11)$ & 8 & $(3,1)$ & 2 & 3 & YES & YES & YES & $1.20$ & $(2,2)$ & -- & 683\\
$(36,13)$ & 8 & $(3,1)$ & 2 & 3 & YES & YES & YES & $1.20$ & $(2,2)$ & NO & 684\\
$(36,13)$ & 8 & $(3,1)$ & 2 & 3 & YES & YES & YES & $1.20$ & $(2,2)$ & -- & 685\\
$(36,11)$ & 8 & $(5,2)$ & 3 & 1 & YES & YES & YES & $1.22$ & $(2,2)$ & NO & 686\\
$(36,11)$ & 8 & $(5,2)$ & 3 & 1 & YES & YES & YES & $1.22$ & $(2,2)$ & -- & 687\\
$(36,11)$ & 8 & $(5,2)$ & 3 & 1 & YES & YES & YES & $1.22$ & $(2,2)$ & NO & 688\\
$(36,11)$ & 8 & $(10,3)$ & 5 & 2 & YES & YES & NO(2) & $1.00$ & $(6,0)$ & 637 & 689\\
$(36,11)$ & 8 & $(16,5)$ & 7 & 4 & YES & YES & YES & $1.22$ & $(2,2)$ & NO & 690\\
$(36,13)$ & 8 & $(36,13)$ & 8 & 36 & YES & YES & YES & $1.10$ & $(2,2)$ & NO & 691\\
$(37,8)$ & 8 & $(2,1)$ & 1 & 1 & YES & YES & NO(2) & $1.00$ & $(4,1)$ & -- & 692\\
$(37,14)$ & 8 & $(2,1)$ & 1 & 1 & YES & YES & YES & $1.20$ & $(2,2)$ & NO & 693\\
$(37,14)$ & 8 & $(2,1)$ & 1 & 1 & YES & YES & YES & $1.20$ & $(2,2)$ & -- & 694\\
$(37,8)$ & 8 & $(3,1)$ & 2 & 1 & YES & YES & NO(2) & $0.89$ & $(4,1)$ & -- & 695\\
$(37,8)$ & 8 & $(3,1)$ & 2 & 1 & YES & YES & NO(2) & $1.00$ & $(4,1)$ & NO & 696\\
$(37,10)$ & 8 & $(3,1)$ & 2 & 1 & YES & YES & NO(2) & $0.62$ & $(8,-1)$ & NO & 697\\
$(37,10)$ & 8 & $(3,1)$ & 2 & 1 & YES & YES & NO(2) & $0.62$ & $(8,-1)$ & -- & 698\\
$(37,14)$ & 8 & $(3,1)$ & 2 & 1 & YES & YES & YES & $1.10$ & $(2,2)$ & -- & 699\\
$(37,14)$ & 8 & $(3,1)$ & 2 & 1 & YES & YES & YES & $1.20$ & $(2,2)$ & NO & 700\\
$(37,14)$ & 8 & $(3,1)$ & 2 & 1 & YES & YES & NO(2) & $1.10$ & $(8,-1)$ & NO & 701\\
$(37,17)$ & 9 & $(3,1)$ & 2 & 1 & YES & YES & YES & $0.88$ & $(4,1)$ & NO & 702\\
$(37,8)$ & 8 & $(5,1)$ & 4 & 1 & YES & YES & NO(2) & $0.89$ & $(6,0)$ & -- & 703\\
$(37,8)$ & 8 & $(5,1)$ & 4 & 1 & YES & YES & NO(2) & $1.00$ & $(6,0)$ & NO & 704\\
$(37,8)$ & 8 & $(5,1)$ & 4 & 1 & YES & YES & NO(2) & $0.89$ & $(4,1)$ & NO & 705\\
$(37,13)$ & 9 & $(5,2)$ & 3 & 1 & YES & YES & NO(2) & $1.20$ & $(4,1)$ & NO & 706\\
$(37,13)$ & 9 & $(5,2)$ & 3 & 1 & YES & YES & NO(2) & $1.20$ & $(4,1)$ & -- & 707\\
$(37,10)$ & 8 & $(7,2)$ & 4 & 1 & YES & YES & NO(2) & $0.62$ & $(8,-1)$ & NO & 708\\
$(37,10)$ & 8 & $(7,3)$ & 4 & 1 & YES & YES & YES & $1.00$ & $(2,2)$ & NO & 709\\
$(37,16)$ & 9 & $(7,1)$ & 6 & 1 & YES & YES & YES & $1.11$ & $(2,2)$ & NO & 710\\
$(37,16)$ & 9 & $(7,1)$ & 6 & 1 & YES & YES & YES & $1.11$ & $(2,2)$ & NO & 711\\
$(37,13)$ & 9 & $(8,3)$ & 4 & 1 & YES & YES & YES & $0.88$ & $(4,1)$ & NO & 712\\
$(37,8)$ & 8 & $(9,2)$ & 5 & 1 & YES & YES & YES & $0.88$ & $(4,1)$ & 619 & 713\\
$(37,8)$ & 8 & $(11,4)$ & 5 & 1 & YES & YES & NO(2) & $0.75$ & $(6,0)$ & -- & 714\\
$(37,17)$ & 9 & $(13,6)$ & 7 & 1 & YES & YES & YES & $0.88$ & $(4,1)$ & NO & 715\\
$(37,8)$ & 8 & $(14,3)$ & 6 & 1 & YES & YES & YES & $0.88$ & $(4,1)$ & NO & 716\\
$(37,10)$ & 8 & $(14,3)$ & 6 & 1 & YES & YES & NO(2) & $0.62$ & $(6,0)$ & NO & 717\\
$(37,8)$ & 8 & $(21,5)$ & 8 & 1 & YES & YES & NO(2) & $0.75$ & $(6,0)$ & NO & 718\\
$(37,10)$ & 8 & $(34,9)$ & 8 & 1 & YES & YES & NO(2) & $0.62$ & $(6,0)$ & NO & 719\\
$(37,16)$ & 9 & $(37,16)$ & 9 & 37 & YES & YES & YES & $1.11$ & $(2,2)$ & NO & 720\\
$(38,17)$ & 9 & $(4,1)$ & 3 & 2 & YES & YES & YES & $1.00$ & $(2,2)$ & -- & 721\\
$(38,17)$ & 9 & $(5,1)$ & 4 & 1 & YES & YES & YES & $0.88$ & $(2,2)$ & -- & 722\\
$(38,17)$ & 9 & $(5,2)$ & 3 & 1 & YES & YES & NO(2) & $1.20$ & $(4,1)$ & NO & 723\\
$(38,17)$ & 9 & $(5,2)$ & 3 & 1 & YES & YES & NO(2) & $1.20$ & $(4,1)$ & -- & 724\\
$(38,17)$ & 9 & $(6,1)$ & 5 & 2 & YES & YES & YES & $1.00$ & $(2,2)$ & NO & 725\\
$(38,17)$ & 9 & $(6,1)$ & 5 & 2 & YES & YES & YES & $1.00$ & $(2,2)$ & NO & 726\\
$(38,17)$ & 9 & $(7,3)$ & 4 & 1 & YES & YES & YES & $1.11$ & $(2,2)$ & NO & 727\\
$(38,17)$ & 9 & $(8,3)$ & 4 & 2 & YES & YES & NO(2) & $1.20$ & $(4,1)$ & NO & 728\\
$(38,17)$ & 9 & $(29,13)$ & 8 & 1 & YES & YES & YES & $0.88$ & $(2,2)$ & NO & 729\\
$(38,17)$ & 9 & $(38,17)$ & 9 & 38 & YES & YES & YES & $1.00$ & $(2,2)$ & NO & 730\\
$(39,17)$ & 8 & $(2,1)$ & 1 & 1 & YES & YES & NO(2) & $1.20$ & $(2,2)$ & 441 & 731\\
$(39,14)$ & 8 & $(3,1)$ & 2 & 3 & YES & YES & YES & $0.88$ & $(4,1)$ & NO & 732\\
$(39,14)$ & 8 & $(3,1)$ & 2 & 3 & YES & YES & YES & $0.88$ & $(4,1)$ & -- & 733\\
$(39,17)$ & 8 & $(3,1)$ & 2 & 3 & YES & YES & YES & $1.20$ & $(2,2)$ & NO & 734\\
$(39,17)$ & 8 & $(3,1)$ & 2 & 3 & YES & YES & YES & $1.20$ & $(2,2)$ & -- & 735\\
$(39,14)$ & 8 & $(4,1)$ & 3 & 1 & YES & YES & YES & $1.11$ & $(2,2)$ & NO & 736\\
$(39,14)$ & 8 & $(4,1)$ & 3 & 1 & YES & YES & NO(2) & $1.20$ & $(4,1)$ & -- & 737\\
$(39,14)$ & 8 & $(4,1)$ & 3 & 1 & YES & YES & NO(2) & $1.27$ & $(4,1)$ & NO & 738\\
$(39,14)$ & 8 & $(7,2)$ & 4 & 1 & YES & YES & YES & $1.11$ & $(2,2)$ & NO & 739\\
$(39,17)$ & 8 & $(9,4)$ & 5 & 3 & YES & YES & YES & $1.20$ & $(2,2)$ & NO & 740\\
$(39,17)$ & 8 & $(39,17)$ & 8 & 39 & YES & YES & YES & $1.10$ & $(2,2)$ & NO & 741\\
$(40,11)$ & 8 & $(2,1)$ & 1 & 2 & YES & YES & NO(2) & $0.75$ & $(8,-1)$ & -- & 742\\
$(40,17)$ & 9 & $(2,1)$ & 1 & 2 & YES & YES & YES & $1.30$ & $(2,2)$ & -- & 743\\
$(40,17)$ & 9 & $(4,1)$ & 3 & 4 & YES & YES & YES & $1.20$ & $(2,2)$ & -- & 744\\
$(40,11)$ & 8 & $(11,3)$ & 5 & 1 & YES & YES & NO(2) & $0.75$ & $(8,-1)$ & NO & 745\\
$(41,11)$ & 8 & $(2,1)$ & 1 & 1 & YES & YES & NO(2) & $0.75$ & $(8,-1)$ & NO & 746\\
$(41,13)$ & 10 & $(2,1)$ & 1 & 1 & YES & YES & YES & $0.88$ & $(4,1)$ & NO & 747\\
$(41,15)$ & 8 & $(2,1)$ & 1 & 1 & YES & YES & YES & $1.11$ & $(2,2)$ & NO & 748\\
$(41,11)$ & 8 & $(3,1)$ & 2 & 1 & YES & YES & YES & $1.00$ & $(2,2)$ & -- & 749\\
$(41,17)$ & 8 & $(3,1)$ & 2 & 1 & YES & YES & YES & $1.00$ & $(2,2)$ & 536 & 750\\
$(41,17)$ & 8 & $(3,1)$ & 2 & 1 & YES & YES & YES & $1.00$ & $(2,2)$ & -- & 751\\
$(41,18)$ & 8 & $(3,1)$ & 2 & 1 & YES & YES & YES & $1.11$ & $(2,2)$ & 477 & 752\\
$(41,18)$ & 8 & $(3,1)$ & 2 & 1 & YES & YES & YES & $1.11$ & $(2,2)$ & -- & 753\\
$(41,13)$ & 10 & $(4,1)$ & 3 & 1 & YES & YES & YES & $0.88$ & $(4,1)$ & NO & 754\\
$(41,15)$ & 8 & $(4,1)$ & 3 & 1 & YES & YES & YES & $1.00$ & $(2,2)$ & -- & 755\\
$(41,18)$ & 8 & $(4,1)$ & 3 & 1 & YES & YES & YES & $1.00$ & $(2,2)$ & NO & 756\\
$(41,18)$ & 8 & $(4,1)$ & 3 & 1 & YES & YES & YES & $1.00$ & $(2,2)$ & -- & 757\\
$(41,15)$ & 8 & $(5,1)$ & 4 & 1 & YES & YES & YES & $1.11$ & $(2,2)$ & NO & 758\\
$(41,15)$ & 8 & $(5,1)$ & 4 & 1 & YES & YES & YES & $1.11$ & $(2,2)$ & -- & 759\\
$(41,15)$ & 8 & $(6,1)$ & 5 & 1 & YES & YES & YES & $1.00$ & $(2,2)$ & NO & 760\\
$(41,15)$ & 8 & $(6,1)$ & 5 & 1 & YES & YES & YES & $1.00$ & $(2,2)$ & -- & 761\\
$(41,11)$ & 8 & $(8,3)$ & 4 & 1 & YES & YES & YES & $1.11$ & $(2,2)$ & NO & 762\\
$(41,15)$ & 8 & $(11,4)$ & 5 & 1 & YES & YES & YES & $1.11$ & $(2,2)$ & NO & 763\\
$(41,18)$ & 8 & $(11,5)$ & 6 & 1 & YES & YES & YES & $1.11$ & $(2,2)$ & NO & 764\\
$(41,11)$ & 8 & $(23,6)$ & 8 & 1 & YES & YES & YES & $1.11$ & $(2,2)$ & NO & 765\\
$(41,15)$ & 8 & $(41,15)$ & 8 & 41 & YES & YES & YES & $1.11$ & $(2,2)$ & NO & 766\\
$(42,13)$ & 9 & $(2,1)$ & 1 & 2 & YES & YES & YES & $1.00$ & $(4,1)$ & NO & 767\\
$(42,19)$ & 9 & $(2,1)$ & 1 & 2 & YES & YES & YES & $1.00$ & $(4,1)$ & NO & 768\\
$(42,19)$ & 9 & $(2,1)$ & 1 & 2 & YES & YES & YES & $1.30$ & $(2,2)$ & -- & 769\\
$(42,13)$ & 9 & $(3,1)$ & 2 & 3 & YES & YES & NO(2) & $1.30$ & $(2,2)$ & NO & 770\\
$(42,13)$ & 9 & $(3,1)$ & 2 & 3 & YES & YES & NO(2) & $1.30$ & $(2,2)$ & -- & 771\\
$(42,19)$ & 9 & $(3,1)$ & 2 & 3 & YES & YES & YES & $0.88$ & $(4,1)$ & NO & 772\\
$(42,19)$ & 9 & $(3,1)$ & 2 & 3 & YES & YES & NO(2) & $1.11$ & $(6,0)$ & -- & 773\\
$(42,19)$ & 9 & $(4,1)$ & 3 & 2 & YES & YES & YES & $1.11$ & $(2,2)$ & -- & 774\\
$(42,19)$ & 9 & $(4,1)$ & 3 & 2 & YES & YES & NO(2) & $1.30$ & $(4,1)$ & NO & 775\\
$(42,13)$ & 9 & $(5,1)$ & 4 & 1 & YES & YES & YES & $0.88$ & $(4,1)$ & NO & 776\\
$(42,13)$ & 9 & $(5,1)$ & 4 & 1 & YES & YES & YES & $0.88$ & $(4,1)$ & -- & 777\\
$(42,19)$ & 9 & $(5,2)$ & 3 & 1 & YES & YES & YES & $1.22$ & $(2,2)$ & NO & 778\\
$(42,19)$ & 9 & $(5,2)$ & 3 & 1 & YES & YES & NO(2) & $1.20$ & $(4,1)$ & -- & 779\\
$(42,19)$ & 9 & $(6,1)$ & 5 & 6 & YES & YES & YES & $1.00$ & $(2,2)$ & NO & 780\\
$(42,19)$ & 9 & $(6,1)$ & 5 & 6 & YES & YES & YES & $1.00$ & $(2,2)$ & NO & 781\\
$(42,19)$ & 9 & $(7,3)$ & 4 & 7 & YES & YES & YES & $1.22$ & $(2,2)$ & NO & 782\\
$(42,19)$ & 9 & $(9,4)$ & 5 & 3 & YES & YES & YES & $0.88$ & $(4,1)$ & NO & 783\\
$(42,19)$ & 9 & $(42,19)$ & 9 & 42 & YES & YES & YES & $1.00$ & $(2,2)$ & NO & 784\\
$(43,19)$ & 9 & $(2,1)$ & 1 & 1 & YES & YES & YES & $1.22$ & $(2,2)$ & -- & 785\\
$(43,16)$ & 9 & $(3,1)$ & 2 & 1 & YES & YES & YES & $1.20$ & $(2,2)$ & -- & 786\\
$(43,19)$ & 9 & $(3,1)$ & 2 & 1 & YES & YES & YES & $1.11$ & $(2,2)$ & NO & 787\\
$(43,16)$ & 9 & $(4,1)$ & 3 & 1 & YES & YES & YES & $1.11$ & $(2,2)$ & -- & 788\\
$(43,19)$ & 9 & $(5,1)$ & 4 & 1 & YES & YES & YES & $1.11$ & $(2,2)$ & NO & 789\\
$(43,19)$ & 9 & $(5,1)$ & 4 & 1 & YES & YES & YES & $1.11$ & $(2,2)$ & -- & 790\\
$(43,19)$ & 9 & $(7,1)$ & 6 & 1 & YES & YES & YES & $1.11$ & $(2,2)$ & NO & 791\\
$(43,19)$ & 9 & $(7,1)$ & 6 & 1 & YES & YES & YES & $1.11$ & $(2,2)$ & NO & 792\\
$(43,19)$ & 9 & $(7,3)$ & 4 & 1 & YES & YES & YES & $1.11$ & $(2,2)$ & NO & 793\\
$(43,19)$ & 9 & $(9,4)$ & 5 & 1 & YES & YES & YES & $1.22$ & $(2,2)$ & NO & 794\\
$(43,16)$ & 9 & $(11,4)$ & 5 & 1 & YES & YES & YES & $1.20$ & $(2,2)$ & NO & 795\\
$(43,16)$ & 9 & $(35,13)$ & 8 & 1 & YES & YES & YES & $1.11$ & $(2,2)$ & NO & 796\\
$(43,19)$ & 9 & $(43,19)$ & 9 & 43 & YES & YES & YES & $1.11$ & $(2,2)$ & NO & 797\\
$(44,17)$ & 8 & $(2,1)$ & 1 & 2 & YES & YES & YES & $1.00$ & $(2,2)$ & -- & 798\\
$(44,19)$ & 10 & $(2,1)$ & 1 & 2 & YES & YES & YES & $1.30$ & $(2,2)$ & NO & 799\\
$(44,17)$ & 8 & $(3,1)$ & 2 & 1 & YES & YES & YES & $1.00$ & $(2,2)$ & NO & 800\\
$(44,17)$ & 8 & $(3,1)$ & 2 & 1 & YES & YES & YES & $1.00$ & $(2,2)$ & -- & 801\\
$(44,19)$ & 10 & $(3,1)$ & 2 & 1 & YES & YES & YES & $1.20$ & $(2,2)$ & NO & 802\\
$(44,17)$ & 8 & $(7,3)$ & 4 & 1 & YES & YES & YES & $1.00$ & $(2,2)$ & NO & 803\\
$(44,19)$ & 10 & $(9,4)$ & 5 & 1 & YES & YES & YES & $1.20$ & $(2,2)$ & NO & 804\\
$(45,13)$ & 10 & $(2,1)$ & 1 & 1 & YES & YES & YES & $1.10$ & $(2,2)$ & -- & 805\\
$(45,14)$ & 9 & $(2,1)$ & 1 & 1 & YES & YES & YES & $1.00$ & $(4,1)$ & NO & 806\\
$(45,14)$ & 9 & $(2,1)$ & 1 & 1 & YES & YES & YES & $1.00$ & $(4,1)$ & -- & 807\\
$(45,17)$ & 9 & $(2,1)$ & 1 & 1 & YES & YES & YES & $1.20$ & $(2,2)$ & NO & 808\\
$(45,14)$ & 9 & $(3,1)$ & 2 & 3 & YES & YES & YES & $1.11$ & $(2,2)$ & NO & 809\\
$(45,14)$ & 9 & $(3,1)$ & 2 & 3 & NO & YES & YES & $1.20$ & $(2,2)$ & -- & 810\\
$(45,17)$ & 9 & $(3,1)$ & 2 & 3 & YES & YES & YES & $1.10$ & $(2,2)$ & -- & 811\\
$(45,17)$ & 9 & $(4,1)$ & 3 & 1 & YES & YES & NO(2) & $0.88$ & $(6,0)$ & -- & 812\\
$(45,16)$ & 9 & $(5,1)$ & 4 & 5 & YES & YES & NO(2) & $0.89$ & $(6,0)$ & -- & 813\\
$(45,14)$ & 9 & $(6,1)$ & 5 & 3 & YES & YES & YES & $1.00$ & $(2,2)$ & NO & 814\\
$(45,14)$ & 9 & $(6,1)$ & 5 & 3 & YES & YES & YES & $1.00$ & $(2,2)$ & -- & 815\\
$(45,17)$ & 9 & $(6,1)$ & 5 & 3 & YES & YES & NO(2) & $0.75$ & $(6,0)$ & NO & 816\\
$(45,16)$ & 9 & $(17,6)$ & 7 & 1 & YES & YES & NO(2) & $1.00$ & $(6,0)$ & 848 & 817\\
$(45,17)$ & 9 & $(21,8)$ & 6 & 3 & YES & YES & NO(2) & $0.75$ & $(6,0)$ & NO & 818\\
$(45,17)$ & 9 & $(37,14)$ & 8 & 1 & YES & YES & NO(2) & $0.88$ & $(6,0)$ & NO & 819\\
$(46,19)$ & 8 & $(2,1)$ & 1 & 2 & YES & YES & YES & $1.00$ & $(2,2)$ & NO & 820\\
$(46,19)$ & 8 & $(3,1)$ & 2 & 1 & YES & YES & YES & $0.89$ & $(2,2)$ & -- & 821\\
$(46,19)$ & 8 & $(3,1)$ & 2 & 1 & YES & YES & YES & $1.11$ & $(2,2)$ & NO & 822\\
$(46,19)$ & 8 & $(3,1)$ & 2 & 1 & YES & YES & YES & $1.11$ & $(2,2)$ & NO & 823\\
$(46,13)$ & 10 & $(4,1)$ & 3 & 2 & YES & YES & YES & $1.11$ & $(2,2)$ & -- & 824\\
$(46,19)$ & 8 & $(4,1)$ & 3 & 2 & YES & YES & YES & $1.00$ & $(2,2)$ & -- & 825\\
$(46,21)$ & 10 & $(4,1)$ & 3 & 2 & YES & YES & NO(2) & $1.20$ & $(4,1)$ & -- & 826\\
$(46,21)$ & 10 & $(4,1)$ & 3 & 2 & YES & YES & NO(2) & $1.30$ & $(4,1)$ & NO & 827\\
$(46,19)$ & 8 & $(5,2)$ & 3 & 1 & YES & YES & YES & $1.00$ & $(2,2)$ & NO & 828\\
$(46,21)$ & 10 & $(5,1)$ & 4 & 1 & YES & YES & NO(2) & $1.20$ & $(4,1)$ & NO & 829\\
$(46,21)$ & 10 & $(5,1)$ & 4 & 1 & YES & YES & NO(2) & $1.30$ & $(4,1)$ & NO & 830\\
$(46,17)$ & 8 & $(7,2)$ & 4 & 1 & YES & YES & NO(2) & $0.62$ & $(6,0)$ & -- & 831\\
$(46,21)$ & 10 & $(7,3)$ & 4 & 1 & YES & YES & NO(2) & $1.20$ & $(4,1)$ & NO & 832\\
$(46,17)$ & 8 & $(14,5)$ & 6 & 2 & YES & YES & NO(2) & $0.75$ & $(6,0)$ & NO & 833\\
$(46,21)$ & 10 & $(24,11)$ & 8 & 2 & YES & YES & NO(2) & $1.20$ & $(4,1)$ & 941 & 834\\
$(46,21)$ & 10 & $(35,16)$ & 9 & 1 & YES & YES & NO(2) & $1.20$ & $(4,1)$ & NO & 835\\
$(46,13)$ & 10 & $(39,11)$ & 9 & 1 & YES & YES & YES & $1.11$ & $(2,2)$ & NO & 836\\
$(47,14)$ & 9 & $(2,1)$ & 1 & 1 & YES & YES & NO(2) & $1.00$ & $(6,0)$ & -- & 837\\
$(47,20)$ & 10 & $(2,1)$ & 1 & 1 & NO & YES & YES & $1.36$ & $(2,2)$ & -- & 838\\
$(47,14)$ & 9 & $(10,3)$ & 5 & 1 & YES & YES & NO(2) & $1.00$ & $(6,0)$ & NO & 839\\
$(48,17)$ & 9 & $(2,1)$ & 1 & 2 & YES & YES & YES & $1.00$ & $(2,2)$ & NO & 840\\
$(48,17)$ & 9 & $(2,1)$ & 1 & 2 & YES & YES & NO(2) & $1.00$ & $(6,0)$ & -- & 841\\
$(48,17)$ & 9 & $(3,1)$ & 2 & 3 & YES & YES & YES & $0.88$ & $(4,1)$ & NO & 842\\
$(48,17)$ & 9 & $(3,1)$ & 2 & 3 & YES & YES & YES & $1.11$ & $(2,2)$ & -- & 843\\
$(48,11)$ & 9 & $(4,1)$ & 3 & 4 & NO & YES & YES & $1.00$ & $(2,2)$ & -- & 844\\
$(48,17)$ & 9 & $(4,1)$ & 3 & 4 & YES & YES & YES & $1.11$ & $(2,2)$ & -- & 845\\
$(48,17)$ & 9 & $(6,1)$ & 5 & 6 & YES & YES & YES & $0.88$ & $(2,2)$ & NO & 846\\
$(48,17)$ & 9 & $(11,4)$ & 5 & 1 & YES & YES & YES & $0.75$ & $(4,1)$ & 954 & 847\\
$(48,17)$ & 9 & $(14,5)$ & 6 & 2 & YES & YES & NO(2) & $1.00$ & $(6,0)$ & 817 & 848\\
$(48,17)$ & 9 & $(17,6)$ & 7 & 1 & YES & YES & YES & $1.20$ & $(2,2)$ & NO & 849\\
$(48,17)$ & 9 & $(20,7)$ & 8 & 4 & YES & YES & NO(2) & $1.10$ & $(4,1)$ & NO & 850\\
$(48,17)$ & 9 & $(31,11)$ & 8 & 1 & YES & YES & YES & $1.11$ & $(2,2)$ & NO & 851\\
$(48,17)$ & 9 & $(48,17)$ & 9 & 48 & YES & YES & YES & $1.22$ & $(2,2)$ & NO & 852\\
$(49,18)$ & 8 & $(2,1)$ & 1 & 1 & YES & YES & YES & $1.00$ & $(2,2)$ & 558 & 853\\
$(49,18)$ & 8 & $(2,1)$ & 1 & 1 & YES & YES & YES & $1.11$ & $(2,2)$ & -- & 854\\
$(49,22)$ & 9 & $(2,1)$ & 1 & 1 & YES & YES & YES & $0.88$ & $(2,2)$ & -- & 855\\
$(49,15)$ & 9 & $(3,1)$ & 2 & 1 & YES & YES & YES & $1.00$ & $(4,1)$ & NO & 856\\
$(49,15)$ & 9 & $(3,1)$ & 2 & 1 & YES & YES & YES & $1.00$ & $(4,1)$ & -- & 857\\
$(49,19)$ & 8 & $(3,1)$ & 2 & 1 & YES & YES & YES & $1.11$ & $(2,2)$ & NO & 858\\
$(49,19)$ & 8 & $(3,1)$ & 2 & 1 & YES & YES & YES & $1.11$ & $(2,2)$ & -- & 859\\
$(49,19)$ & 8 & $(3,1)$ & 2 & 1 & YES & YES & YES & $1.11$ & $(2,2)$ & NO & 860\\
$(49,20)$ & 9 & $(3,1)$ & 2 & 1 & YES & YES & YES & $1.11$ & $(2,2)$ & NO & 861\\
$(49,22)$ & 9 & $(3,1)$ & 2 & 1 & YES & YES & YES & $0.75$ & $(4,1)$ & NO & 862\\
$(49,19)$ & 8 & $(4,1)$ & 3 & 1 & YES & YES & YES & $1.00$ & $(2,2)$ & -- & 863\\
$(49,22)$ & 9 & $(4,1)$ & 3 & 1 & YES & YES & YES & $1.00$ & $(2,2)$ & NO & 864\\
$(49,22)$ & 9 & $(4,1)$ & 3 & 1 & YES & YES & NO(2) & $1.10$ & $(4,1)$ & -- & 865\\
$(49,15)$ & 9 & $(5,1)$ & 4 & 1 & YES & YES & NO(2) & $0.75$ & $(8,-1)$ & -- & 866\\
$(49,15)$ & 9 & $(5,1)$ & 4 & 1 & YES & YES & NO(2) & $0.88$ & $(8,-1)$ & NO & 867\\
$(49,15)$ & 9 & $(5,2)$ & 3 & 1 & YES & YES & YES & $1.11$ & $(2,2)$ & -- & 868\\
$(49,20)$ & 9 & $(5,1)$ & 4 & 1 & YES & YES & YES & $0.88$ & $(2,2)$ & -- & 869\\
$(49,20)$ & 9 & $(5,1)$ & 4 & 1 & YES & YES & YES & $1.00$ & $(2,2)$ & NO & 870\\
$(49,20)$ & 9 & $(5,2)$ & 3 & 1 & YES & YES & YES & $1.00$ & $(2,2)$ & 622 & 871\\
$(49,18)$ & 8 & $(8,3)$ & 4 & 1 & YES & YES & YES & $1.00$ & $(2,2)$ & NO & 872\\
$(49,13)$ & 9 & $(9,2)$ & 5 & 1 & YES & YES & YES & $1.00$ & $(2,2)$ & NO & 873\\
$(49,13)$ & 9 & $(11,3)$ & 5 & 1 & YES & YES & YES & $1.00$ & $(2,2)$ & NO & 874\\
$(49,22)$ & 9 & $(20,9)$ & 7 & 1 & YES & YES & YES & $0.75$ & $(4,1)$ & NO & 875\\
$(49,11)$ & 10 & $(22,5)$ & 7 & 1 & YES & YES & YES & $1.10$ & $(2,2)$ & NO & 876\\
$(49,13)$ & 9 & $(23,6)$ & 8 & 1 & YES & YES & YES & $1.00$ & $(2,2)$ & 1007 & 877\\
$(49,9)$ & 10 & $(28,5)$ & 8 & 7 & YES & YES & YES & $1.10$ & $(2,2)$ & NO & 878\\
$(49,22)$ & 9 & $(29,13)$ & 8 & 1 & YES & YES & YES & $1.11$ & $(2,2)$ & NO & 879\\
$(49,20)$ & 9 & $(49,20)$ & 9 & 49 & YES & YES & YES & $1.00$ & $(2,2)$ & NO & 880\\
$(50,19)$ & 8 & $(2,1)$ & 1 & 2 & YES & YES & YES & $0.88$ & $(4,1)$ & NO & 881\\
$(50,23)$ & 10 & $(2,1)$ & 1 & 2 & NO & YES & YES & $1.00$ & $(4,1)$ & -- & 882\\
$(50,11)$ & 10 & $(7,2)$ & 4 & 1 & YES & YES & NO(2) & $0.62$ & $(6,0)$ & NO & 883\\
$(51,20)$ & 9 & $(2,1)$ & 1 & 1 & YES & YES & YES & $1.11$ & $(2,2)$ & NO & 884\\
$(51,23)$ & 9 & $(2,1)$ & 1 & 1 & YES & YES & YES & $1.22$ & $(2,2)$ & -- & 885\\
$(51,16)$ & 10 & $(3,1)$ & 2 & 3 & NO & YES & NO(2) & $1.30$ & $(2,2)$ & -- & 886\\
$(51,20)$ & 9 & $(3,1)$ & 2 & 3 & YES & YES & NO(2) & $0.75$ & $(6,0)$ & -- & 887\\
$(51,20)$ & 9 & $(4,1)$ & 3 & 1 & YES & YES & NO(2) & $0.75$ & $(6,0)$ & NO & 888\\
$(51,20)$ & 9 & $(5,1)$ & 4 & 1 & YES & YES & YES & $1.10$ & $(2,2)$ & -- & 889\\
$(51,20)$ & 9 & $(5,1)$ & 4 & 1 & YES & YES & NO(2) & $0.75$ & $(6,0)$ & NO & 890\\
$(51,20)$ & 9 & $(5,2)$ & 3 & 1 & YES & YES & YES & $1.20$ & $(2,2)$ & 639 & 891\\
$(51,23)$ & 9 & $(6,1)$ & 5 & 3 & YES & YES & NO(2) & $1.20$ & $(2,2)$ & NO & 892\\
$(51,23)$ & 9 & $(6,1)$ & 5 & 3 & YES & YES & NO(2) & $1.20$ & $(2,2)$ & -- & 893\\
$(51,20)$ & 9 & $(8,3)$ & 4 & 1 & YES & YES & NO(2) & $0.75$ & $(6,0)$ & NO & 894\\
$(51,23)$ & 9 & $(9,4)$ & 5 & 3 & YES & YES & YES & $0.75$ & $(4,1)$ & NO & 895\\
$(51,20)$ & 9 & $(13,5)$ & 5 & 1 & YES & YES & NO(2) & $0.75$ & $(6,0)$ & 596 & 896\\
$(51,23)$ & 9 & $(20,9)$ & 7 & 1 & YES & YES & YES & $0.75$ & $(4,1)$ & NO & 897\\
$(51,20)$ & 9 & $(51,20)$ & 9 & 51 & YES & YES & NO(2) & $0.75$ & $(6,0)$ & NO & 898\\
$(52,19)$ & 9 & $(3,1)$ & 2 & 1 & YES & YES & YES & $1.11$ & $(2,2)$ & NO & 899\\
$(52,19)$ & 9 & $(7,1)$ & 6 & 1 & YES & YES & YES & $1.00$ & $(2,2)$ & NO & 900\\
$(52,19)$ & 9 & $(11,4)$ & 5 & 1 & YES & YES & YES & $1.11$ & $(2,2)$ & NO & 901\\
$(52,19)$ & 9 & $(19,7)$ & 6 & 1 & YES & YES & YES & $1.00$ & $(2,2)$ & NO & 902\\
$(53,19)$ & 9 & $(3,1)$ & 2 & 1 & YES & YES & YES & $1.20$ & $(2,2)$ & NO & 903\\
$(53,19)$ & 9 & $(3,1)$ & 2 & 1 & YES & YES & NO(2) & $0.75$ & $(6,0)$ & -- & 904\\
$(53,19)$ & 9 & $(3,1)$ & 2 & 1 & YES & YES & NO(2) & $0.88$ & $(6,0)$ & NO & 905\\
$(53,14)$ & 9 & $(4,1)$ & 3 & 1 & YES & YES & YES & $0.89$ & $(2,2)$ & NO & 906\\
$(53,19)$ & 9 & $(4,1)$ & 3 & 1 & YES & YES & NO(2) & $0.75$ & $(6,0)$ & -- & 907\\
$(53,19)$ & 9 & $(5,1)$ & 4 & 1 & YES & YES & NO(2) & $0.75$ & $(6,0)$ & NO & 908\\
$(53,19)$ & 9 & $(5,2)$ & 3 & 1 & YES & YES & NO(2) & $0.75$ & $(6,0)$ & NO & 909\\
$(53,19)$ & 9 & $(8,3)$ & 4 & 1 & YES & YES & NO(2) & $0.75$ & $(6,0)$ & NO & 910\\
$(53,19)$ & 9 & $(14,5)$ & 6 & 1 & YES & YES & YES & $1.11$ & $(2,2)$ & NO & 911\\
$(53,19)$ & 9 & $(25,9)$ & 7 & 1 & YES & YES & YES & $1.10$ & $(2,2)$ & 958 & 912\\
$(53,19)$ & 9 & $(53,19)$ & 9 & 53 & YES & YES & NO(2) & $0.75$ & $(6,0)$ & NO & 913\\
$(55,16)$ & 9 & $(2,1)$ & 1 & 1 & YES & YES & YES & $0.89$ & $(2,2)$ & -- & 914\\
$(55,23)$ & 9 & $(2,1)$ & 1 & 1 & NO & YES & YES & $1.20$ & $(2,2)$ & -- & 915\\
$(55,16)$ & 9 & $(3,1)$ & 2 & 1 & NO & YES & YES & $0.88$ & $(4,1)$ & -- & 916\\
$(55,24)$ & 9 & $(3,1)$ & 2 & 1 & YES & YES & YES & $1.00$ & $(2,2)$ & -- & 917\\
$(55,24)$ & 9 & $(11,5)$ & 6 & 11 & YES & YES & NO(2) & $1.10$ & $(4,1)$ & NO & 918\\
$(55,24)$ & 9 & $(16,7)$ & 6 & 1 & YES & YES & YES & $1.00$ & $(2,2)$ & NO & 919\\
$(56,25)$ & 11 & $(2,1)$ & 1 & 2 & NO & YES & YES & $1.30$ & $(2,2)$ & -- & 920\\
$(56,13)$ & 10 & $(4,1)$ & 3 & 4 & YES & YES & NO(2) & $0.75$ & $(8,-1)$ & NO & 921\\
$(56,13)$ & 10 & $(4,1)$ & 3 & 4 & YES & YES & NO(2) & $0.75$ & $(8,-1)$ & -- & 922\\
$(56,15)$ & 9 & $(4,1)$ & 3 & 4 & YES & YES & YES & $0.88$ & $(4,1)$ & NO & 923\\
$(56,15)$ & 9 & $(4,1)$ & 3 & 4 & YES & YES & YES & $0.88$ & $(4,1)$ & -- & 924\\
$(56,17)$ & 9 & $(4,1)$ & 3 & 4 & YES & YES & YES & $1.11$ & $(2,2)$ & NO & 925\\
$(56,17)$ & 9 & $(4,1)$ & 3 & 4 & YES & YES & NO(2) & $1.10$ & $(4,1)$ & -- & 926\\
$(56,13)$ & 10 & $(13,3)$ & 6 & 1 & YES & YES & NO(2) & $0.75$ & $(8,-1)$ & NO & 927\\
$(56,13)$ & 10 & $(25,6)$ & 9 & 1 & YES & YES & NO(2) & $1.10$ & $(4,1)$ & NO & 928\\
$(57,25)$ & 9 & $(3,1)$ & 2 & 3 & YES & YES & YES & $1.00$ & $(2,2)$ & -- & 929\\
$(57,25)$ & 9 & $(5,2)$ & 3 & 1 & YES & YES & NO(2) & $1.10$ & $(4,1)$ & -- & 930\\
$(57,25)$ & 9 & $(16,7)$ & 6 & 1 & YES & YES & YES & $1.00$ & $(2,2)$ & NO & 931\\
$(59,13)$ & 11 & $(2,1)$ & 1 & 1 & YES & YES & YES & $1.00$ & $(2,2)$ & -- & 932\\
$(59,13)$ & 11 & $(3,1)$ & 2 & 1 & YES & YES & YES & $1.00$ & $(2,2)$ & -- & 933\\
$(59,13)$ & 11 & $(3,1)$ & 2 & 1 & YES & YES & NO(2) & $1.20$ & $(4,1)$ & NO & 934\\
$(59,26)$ & 9 & $(3,1)$ & 2 & 1 & YES & YES & NO(2) & $0.75$ & $(6,0)$ & NO & 935\\
$(59,14)$ & 10 & $(4,1)$ & 3 & 1 & YES & YES & NO(2) & $1.18$ & $(4,1)$ & -- & 936\\
$(59,27)$ & 10 & $(5,1)$ & 4 & 1 & YES & YES & NO(2) & $1.10$ & $(4,1)$ & NO & 937\\
$(59,27)$ & 10 & $(5,1)$ & 4 & 1 & YES & YES & NO(2) & $1.20$ & $(4,1)$ & NO & 938\\
$(59,14)$ & 10 & $(7,2)$ & 4 & 1 & YES & YES & YES & $1.11$ & $(2,2)$ & NO & 939\\
$(59,26)$ & 9 & $(7,3)$ & 4 & 1 & YES & YES & NO(2) & $0.75$ & $(6,0)$ & 659 & 940\\
$(59,27)$ & 10 & $(11,5)$ & 6 & 1 & YES & YES & NO(2) & $1.20$ & $(4,1)$ & 834 & 941\\
$(59,13)$ & 11 & $(13,3)$ & 6 & 1 & YES & YES & NO(2) & $1.10$ & $(4,1)$ & NO & 942\\
$(61,24)$ & 10 & $(2,1)$ & 1 & 1 & NO & YES & YES & $1.22$ & $(2,2)$ & -- & 943\\
$(61,19)$ & 10 & $(3,1)$ & 2 & 1 & NO & YES & YES & $1.00$ & $(4,1)$ & -- & 944\\
$(62,27)$ & 9 & $(2,1)$ & 1 & 2 & NO & YES & YES & $1.00$ & $(2,2)$ & -- & 945\\
$(63,26)$ & 9 & $(2,1)$ & 1 & 1 & NO & YES & YES & $1.11$ & $(2,2)$ & -- & 946\\
$(63,26)$ & 9 & $(2,1)$ & 1 & 1 & YES & YES & NO(2) & $0.88$ & $(6,0)$ & NO & 947\\
$(63,26)$ & 9 & $(4,1)$ & 3 & 1 & YES & YES & NO(2) & $0.75$ & $(6,0)$ & -- & 948\\
$(63,26)$ & 9 & $(5,2)$ & 3 & 1 & YES & YES & YES & $1.11$ & $(2,2)$ & NO & 949\\
$(64,23)$ & 9 & $(2,1)$ & 1 & 2 & YES & YES & YES & $0.88$ & $(2,2)$ & NO & 950\\
$(64,23)$ & 9 & $(2,1)$ & 1 & 2 & YES & YES & NO(2) & $0.89$ & $(6,0)$ & -- & 951\\
$(64,27)$ & 9 & $(2,1)$ & 1 & 2 & YES & YES & YES & $1.11$ & $(2,2)$ & 656 & 952\\
$(64,27)$ & 9 & $(2,1)$ & 1 & 2 & YES & YES & NO(2) & $0.75$ & $(6,0)$ & -- & 953\\
$(64,23)$ & 9 & $(3,1)$ & 2 & 1 & YES & YES & YES & $0.75$ & $(4,1)$ & 847 & 954\\
$(64,23)$ & 9 & $(3,1)$ & 2 & 1 & YES & YES & YES & $1.11$ & $(2,2)$ & -- & 955\\
$(64,23)$ & 9 & $(5,1)$ & 4 & 1 & YES & YES & NO(2) & $0.75$ & $(6,0)$ & NO & 956\\
$(64,27)$ & 9 & $(5,2)$ & 3 & 1 & YES & YES & NO(2) & $0.75$ & $(6,0)$ & NO & 957\\
$(64,23)$ & 9 & $(14,5)$ & 6 & 2 & YES & YES & YES & $1.10$ & $(2,2)$ & 912 & 958\\
$(64,23)$ & 9 & $(39,14)$ & 8 & 1 & YES & YES & YES & $1.11$ & $(2,2)$ & NO & 959\\
$(65,24)$ & 9 & $(3,1)$ & 2 & 1 & YES & YES & YES & $1.00$ & $(2,2)$ & -- & 960\\
$(65,24)$ & 9 & $(11,4)$ & 5 & 1 & YES & YES & NO(2) & $1.10$ & $(4,1)$ & NO & 961\\
$(65,17)$ & 10 & $(23,6)$ & 8 & 1 & YES & YES & YES & $1.22$ & $(2,2)$ & NO & 962\\
$(65,24)$ & 9 & $(65,24)$ & 9 & 65 & YES & YES & YES & $1.00$ & $(2,2)$ & NO & 963\\
$(66,25)$ & 9 & $(2,1)$ & 1 & 2 & NO & YES & YES & $0.88$ & $(4,1)$ & -- & 964\\
$(67,16)$ & 11 & $(9,2)$ & 5 & 1 & YES & YES & YES & $1.00$ & $(2,2)$ & 680 & 965\\
$(67,16)$ & 11 & $(21,5)$ & 8 & 1 & YES & YES & YES & $1.00$ & $(2,2)$ & NO & 966\\
$(68,25)$ & 9 & $(2,1)$ & 1 & 2 & NO & YES & YES & $0.89$ & $(4,1)$ & -- & 967\\
$(69,19)$ & 9 & $(3,1)$ & 2 & 3 & YES & YES & YES & $0.89$ & $(2,2)$ & NO & 968\\
$(71,13)$ & 12 & $(2,1)$ & 1 & 1 & YES & YES & YES & $0.88$ & $(2,2)$ & -- & 969\\
$(71,13)$ & 12 & $(2,1)$ & 1 & 1 & YES & YES & NO(2) & $1.00$ & $(6,0)$ & NO & 970\\
$(71,22)$ & 10 & $(2,1)$ & 1 & 1 & YES & YES & YES & $0.88$ & $(2,2)$ & NO & 971\\
$(71,31)$ & 10 & $(2,1)$ & 1 & 1 & YES & YES & NO(2) & $1.10$ & $(4,1)$ & NO & 972\\
$(71,17)$ & 11 & $(3,1)$ & 2 & 1 & YES & YES & YES & $1.00$ & $(2,2)$ & -- & 973\\
$(71,17)$ & 11 & $(3,1)$ & 2 & 1 & YES & YES & NO(2) & $1.20$ & $(4,1)$ & NO & 974\\
$(71,13)$ & 12 & $(6,1)$ & 5 & 1 & YES & YES & NO(2) & $1.00$ & $(6,0)$ & NO & 975\\
$(71,17)$ & 11 & $(25,6)$ & 9 & 1 & YES & YES & YES & $1.00$ & $(2,2)$ & NO & 976\\
$(71,17)$ & 11 & $(29,7)$ & 10 & 1 & YES & YES & NO(2) & $1.10$ & $(4,1)$ & NO & 977\\
$(72,13)$ & 12 & $(2,1)$ & 1 & 2 & YES & YES & NO(2) & $0.89$ & $(6,0)$ & -- & 978\\
$(72,13)$ & 12 & $(2,1)$ & 1 & 2 & YES & YES & NO(2) & $1.00$ & $(6,0)$ & NO & 979\\
$(72,19)$ & 10 & $(2,1)$ & 1 & 2 & YES & YES & NO(2) & $0.88$ & $(6,0)$ & -- & 980\\
$(72,13)$ & 12 & $(3,1)$ & 2 & 3 & YES & YES & YES & $1.11$ & $(2,2)$ & NO & 981\\
$(72,13)$ & 12 & $(4,1)$ & 3 & 4 & YES & YES & YES & $1.11$ & $(2,2)$ & NO & 982\\
$(72,17)$ & 11 & $(4,1)$ & 3 & 4 & NO & YES & NO(2) & $0.75$ & $(8,-1)$ & -- & 983\\
$(72,13)$ & 12 & $(5,1)$ & 4 & 1 & YES & YES & YES & $1.10$ & $(2,2)$ & NO & 984\\
$(72,13)$ & 12 & $(11,2)$ & 6 & 1 & YES & YES & YES & $0.75$ & $(4,1)$ & NO & 985\\
$(73,27)$ & 9 & $(3,1)$ & 2 & 1 & YES & YES & NO(2) & $0.88$ & $(6,0)$ & NO & 986\\
$(73,27)$ & 9 & $(4,1)$ & 3 & 1 & YES & YES & NO(2) & $0.75$ & $(6,0)$ & -- & 987\\
$(73,27)$ & 9 & $(8,3)$ & 4 & 1 & YES & YES & YES & $1.11$ & $(2,2)$ & NO & 988\\
$(74,17)$ & 11 & $(2,1)$ & 1 & 2 & YES & YES & NO(2) & $0.88$ & $(6,0)$ & -- & 989\\
$(74,29)$ & 10 & $(2,1)$ & 1 & 2 & NO & YES & YES & $1.00$ & $(4,1)$ & -- & 990\\
$(74,31)$ & 9 & $(2,1)$ & 1 & 2 & NO & YES & YES & $0.88$ & $(4,1)$ & -- & 991\\
$(76,13)$ & 12 & $(5,1)$ & 4 & 1 & YES & YES & YES & $0.75$ & $(4,1)$ & NO & 992\\
$(76,13)$ & 12 & $(35,6)$ & 10 & 1 & YES & YES & YES & $0.75$ & $(4,1)$ & NO & 993\\
$(77,16)$ & 11 & $(3,1)$ & 2 & 1 & YES & YES & YES & $1.00$ & $(2,2)$ & -- & 994\\
$(77,16)$ & 11 & $(3,1)$ & 2 & 1 & YES & YES & NO(2) & $1.20$ & $(4,1)$ & NO & 995\\
$(77,16)$ & 11 & $(11,2)$ & 6 & 11 & YES & YES & NO(2) & $1.10$ & $(4,1)$ & NO & 996\\
$(77,16)$ & 11 & $(24,5)$ & 8 & 1 & YES & YES & YES & $1.00$ & $(2,2)$ & NO & 997\\
$(79,17)$ & 11 & $(2,1)$ & 1 & 1 & YES & YES & NO(2) & $0.75$ & $(6,0)$ & NO & 998\\
$(79,30)$ & 9 & $(2,1)$ & 1 & 1 & NO & YES & NO(2) & $0.75$ & $(8,-1)$ & -- & 999\\
$(79,31)$ & 10 & $(2,1)$ & 1 & 1 & NO & YES & YES & $1.00$ & $(4,1)$ & -- & 1000\\
$(79,17)$ & 11 & $(4,1)$ & 3 & 1 & YES & YES & NO(2) & $0.75$ & $(6,0)$ & NO & 1001\\
$(79,17)$ & 11 & $(14,3)$ & 6 & 1 & YES & YES & YES & $1.11$ & $(2,2)$ & NO & 1002\\
$(80,19)$ & 11 & $(2,1)$ & 1 & 2 & YES & YES & NO(2) & $0.88$ & $(6,0)$ & -- & 1003\\
$(82,19)$ & 12 & $(4,1)$ & 3 & 2 & YES & YES & NO(2) & $1.10$ & $(4,1)$ & -- & 1004\\
$(82,19)$ & 12 & $(13,3)$ & 6 & 1 & YES & YES & NO(2) & $1.10$ & $(4,1)$ & NO & 1005\\
$(83,22)$ & 10 & $(3,1)$ & 2 & 1 & YES & YES & NO(2) & $0.75$ & $(6,0)$ & NO & 1006\\
$(83,22)$ & 10 & $(4,1)$ & 3 & 1 & YES & YES & YES & $1.00$ & $(2,2)$ & 877 & 1007\\
$(83,22)$ & 10 & $(4,1)$ & 3 & 1 & YES & YES & YES & $1.00$ & $(2,2)$ & -- & 1008\\
$(83,22)$ & 10 & $(34,9)$ & 8 & 1 & YES & YES & NO(2) & $0.75$ & $(6,0)$ & NO & 1009\\
$(84,37)$ & 10 & $(2,1)$ & 1 & 2 & NO & YES & NO(2) & $1.20$ & $(4,1)$ & -- & 1010\\
$(85,37)$ & 10 & $(2,1)$ & 1 & 1 & NO & YES & NO(2) & $0.88$ & $(6,0)$ & -- & 1011\\
$(88,21)$ & 12 & $(4,1)$ & 3 & 4 & YES & YES & NO(2) & $1.10$ & $(4,1)$ & NO & 1012\\
$(88,21)$ & 12 & $(67,16)$ & 11 & 1 & YES & YES & NO(2) & $1.10$ & $(4,1)$ & NO & 1013\\
$(89,27)$ & 10 & $(2,1)$ & 1 & 1 & YES & YES & NO(2) & $0.75$ & $(6,0)$ & NO & 1014\\
$(89,40)$ & 11 & $(2,1)$ & 1 & 1 & NO & YES & NO(2) & $1.20$ & $(4,1)$ & -- & 1015\\
$(89,17)$ & 12 & $(5,1)$ & 4 & 1 & YES & YES & YES & $0.88$ & $(2,2)$ & NO & 1016\\
$(91,19)$ & 11 & $(3,1)$ & 2 & 1 & YES & YES & YES & $1.00$ & $(2,2)$ & -- & 1017\\
$(91,19)$ & 11 & $(3,1)$ & 2 & 1 & YES & YES & NO(2) & $1.20$ & $(4,1)$ & NO & 1018\\
$(91,17)$ & 12 & $(6,1)$ & 5 & 1 & YES & YES & NO(2) & $0.62$ & $(6,0)$ & NO & 1019\\
$(91,17)$ & 12 & $(11,2)$ & 6 & 1 & YES & YES & NO(2) & $0.62$ & $(6,0)$ & NO & 1020\\
$(91,19)$ & 11 & $(24,5)$ & 8 & 1 & YES & YES & YES & $1.00$ & $(2,2)$ & NO & 1021\\
$(91,19)$ & 11 & $(29,6)$ & 9 & 1 & YES & YES & NO(2) & $1.10$ & $(4,1)$ & NO & 1022\\
$(92,19)$ & 12 & $(7,1)$ & 6 & 1 & YES & YES & NO(2) & $1.10$ & $(4,1)$ & NO & 1023\\
$(92,19)$ & 12 & $(29,6)$ & 9 & 1 & YES & YES & NO(2) & $1.20$ & $(4,1)$ & NO & 1024\\
$(96,17)$ & 12 & $(2,1)$ & 1 & 2 & YES & YES & NO(2) & $0.75$ & $(6,0)$ & NO & 1025\\
$(97,26)$ & 10 & $(2,1)$ & 1 & 1 & YES & YES & NO(2) & $0.62$ & $(6,0)$ & NO & 1026\\
$(97,26)$ & 10 & $(3,1)$ & 2 & 1 & YES & YES & NO(2) & $0.62$ & $(6,0)$ & NO & 1027\\
$(97,26)$ & 10 & $(4,1)$ & 3 & 1 & YES & YES & YES & $1.11$ & $(2,2)$ & NO & 1028\\
$(99,17)$ & 12 & $(2,1)$ & 1 & 1 & YES & YES & YES & $0.75$ & $(4,1)$ & NO & 1029\\
$(99,17)$ & 12 & $(35,6)$ & 10 & 1 & YES & YES & YES & $0.75$ & $(4,1)$ & NO & 1030\\
$(101,16)$ & 13 & $(7,1)$ & 6 & 1 & YES & YES & YES & $1.00$ & $(2,2)$ & NO & 1031\\
$(101,16)$ & 13 & $(19,3)$ & 8 & 1 & YES & YES & YES & $1.00$ & $(2,2)$ & NO & 1032\\
$(120,19)$ & 14 & $(19,3)$ & 8 & 1 & YES & YES & NO(2) & $1.20$ & $(4,1)$ & NO & 1033\\
$(a;2,0,0;17)$ & 6 & $(2,1)$ & 1 & 1 & YES & YES & YES & $0.78$ & $(2,2)$ & -- & 1034\\
$(a;2,0,0;17)$ & 6 & $(5,2)$ & 3 & 1 & YES & YES & NO(2) & $1.00$ & $(2,2)$ & -- & 1035\\
$(a;3,0,0;7)$ & 7 & $(3,1)$ & 2 & 1 & YES & YES & NO(2) & $1.42$ & $(2,2)$ & -- & 1036\\
$(a;3,0,0;7)$ & 7 & $(7,2)$ & 4 & 7 & YES & YES & YES & $1.11$ & $(2,2)$ & -- & 1037\\
$(a;3,0,1;31)$ & 8 & $(3,1)$ & 2 & 1 & YES & YES & YES & $1.00$ & $(2,2)$ & -- & 1038\\
$(a;3,1,0;31)$ & 8 & $(2,1)$ & 1 & 1 & YES & YES & YES & $0.88$ & $(2,2)$ & -- & 1039\\
$(a;3,1,0;31)$ & 8 & $(3,1)$ & 2 & 1 & YES & YES & NO(2) & $1.00$ & $(6,0)$ & -- & 1040\\
$(a;3,1,0;31)$ & 8 & $(4,1)$ & 3 & 1 & YES & YES & YES & $1.00$ & $(2,2)$ & -- & 1041\\
$(a;4,0,0;25)$ & 8 & $(3,1)$ & 2 & 1 & YES & YES & YES & $0.88$ & $(4,1)$ & -- & 1042\\
$(a;4,0,0;25)$ & 8 & $(7,3)$ & 4 & 1 & YES & YES & NO(2) & $1.20$ & $(4,1)$ & -- & 1043\\
$(a;4,2,0;7)$ & 10 & $(5,1)$ & 4 & 1 & YES & YES & NO(2) & $1.20$ & $(4,1)$ & -- & 1044\\
$(b;0,0,3;32)$ & 8 & $(2,1)$ & 1 & 2 & YES & YES & NO(2) & $0.75$ & $(6,0)$ & -- & 1045\\
$(b;0,1,0;19)$ & 6 & $(9,4)$ & 5 & 1 & YES & YES & NO(2) & $0.75$ & $(6,0)$ & -- & 1046\\
$(b;0,2,0;8)$ & 7 & $(4,1)$ & 3 & 4 & YES & YES & NO(2) & $0.75$ & $(10,-2)$ & -- & 1047\\
$(b;0,3,0;29)$ & 8 & $(2,1)$ & 1 & 1 & YES & YES & YES & $1.11$ & $(2,2)$ & -- & 1048\\
$(b;0,3,0;29)$ & 8 & $(3,1)$ & 2 & 1 & YES & YES & NO(2) & $0.88$ & $(6,0)$ & -- & 1049\\
$(b;0,3,0;29)$ & 8 & $(5,1)$ & 4 & 1 & YES & YES & NO(2) & $0.75$ & $(6,0)$ & -- & 1050\\
$(b;3,0,0;16)$ & 8 & $(2,1)$ & 1 & 2 & YES & YES & YES & $1.22$ & $(2,2)$ & -- & 1051\\
$(c;0,0,0;4)$ & 4 & $(15,4)$ & 6 & 1 & YES & YES & NO(2) & $0.62$ & $(8,-1)$ & -- & 1052\\
$(c;0,0,0;4)$ & 4 & $(16,7)$ & 6 & 4 & YES & YES & NO(2) & $1.00$ & $(6,0)$ & -- & 1053\\
$(c;0,0,0;4)$ & 4 & $(20,9)$ & 7 & 4 & YES & YES & YES & $1.11$ & $(2,2)$ & -- & 1054\\
$(c;0,0,0;4)$ & 4 & $(25,9)$ & 7 & 1 & YES & YES & YES & $1.11$ & $(2,2)$ & -- & 1055\\
$(c;0,1,0;11)$ & 5 & $(9,4)$ & 5 & 1 & YES & YES & YES & $1.00$ & $(2,2)$ & -- & 1056\\
$(c;0,1,0;11)$ & 5 & $(11,4)$ & 5 & 11 & YES & YES & YES & $1.00$ & $(2,2)$ & -- & 1057\\
$(c;0,1,0;11)$ & 5 & $(11,5)$ & 6 & 11 & YES & YES & YES & $1.00$ & $(2,2)$ & -- & 1058\\
$(c;0,1,1;5)$ & 6 & $(11,3)$ & 5 & 1 & YES & YES & YES & $1.00$ & $(2,2)$ & -- & 1059\\
$(c;0,1,1;5)$ & 6 & $(13,4)$ & 6 & 1 & YES & YES & YES & $1.00$ & $(2,2)$ & -- & 1060\\
$(c;0,2,0;7)$ & 6 & $(5,1)$ & 4 & 1 & YES & YES & NO(2) & $1.00$ & $(4,1)$ & -- & 1061\\
$(c;0,2,0;7)$ & 6 & $(5,2)$ & 3 & 1 & YES & YES & NO(2) & $1.00$ & $(6,0)$ & -- & 1062\\
$(c;0,2,0;7)$ & 6 & $(6,1)$ & 5 & 1 & YES & YES & NO(2) & $1.00$ & $(4,1)$ & -- & 1063\\
$(c;0,2,0;7)$ & 6 & $(8,3)$ & 4 & 1 & YES & YES & NO(2) & $1.33$ & $(2,2)$ & -- & 1064\\
$(c;0,2,1;19)$ & 7 & $(4,1)$ & 3 & 1 & YES & YES & NO(2) & $0.89$ & $(10,-2)$ & -- & 1065\\
$(c;0,2,1;19)$ & 7 & $(11,3)$ & 5 & 1 & YES & YES & YES & $1.10$ & $(2,2)$ & -- & 1066\\
$(c;0,3,0;17)$ & 7 & $(4,1)$ & 3 & 1 & YES & YES & YES & $1.10$ & $(2,2)$ & -- & 1067\\
$(c;0,3,0;17)$ & 7 & $(5,1)$ & 4 & 1 & YES & YES & YES & $1.00$ & $(2,2)$ & -- & 1068\\
$(c;0,3,0;17)$ & 7 & $(5,2)$ & 3 & 1 & YES & YES & YES & $1.10$ & $(2,2)$ & -- & 1069\\
$(c;0,3,0;17)$ & 7 & $(8,3)$ & 4 & 1 & YES & YES & YES & $0.75$ & $(4,1)$ & -- & 1070\\
$(c;0,3,1;23)$ & 8 & $(4,1)$ & 3 & 1 & YES & YES & YES & $1.00$ & $(2,2)$ & -- & 1071\\
$(c;0,3,1;23)$ & 8 & $(5,1)$ & 4 & 1 & YES & YES & YES & $1.00$ & $(2,2)$ & -- & 1072\\
$(c;0,3,1;23)$ & 8 & $(6,1)$ & 5 & 1 & YES & YES & YES & $1.00$ & $(2,2)$ & -- & 1073\\
$(c;0,3,2;29)$ & 9 & $(3,1)$ & 2 & 1 & YES & YES & YES & $1.00$ & $(2,2)$ & -- & 1074\\
$(c;0,3,2;29)$ & 9 & $(5,1)$ & 4 & 1 & YES & YES & YES & $0.88$ & $(2,2)$ & -- & 1075\\
$(c;0,4,0;10)$ & 8 & $(3,1)$ & 2 & 1 & YES & YES & YES & $1.00$ & $(2,2)$ & -- & 1076\\
$(c;0,4,0;10)$ & 8 & $(4,1)$ & 3 & 2 & YES & YES & YES & $1.00$ & $(2,2)$ & -- & 1077\\
$(c;0,4,0;10)$ & 8 & $(6,1)$ & 5 & 2 & YES & YES & YES & $1.00$ & $(2,2)$ & -- & 1078\\
$(c;0,4,1;9)$ & 9 & $(4,1)$ & 3 & 1 & YES & YES & YES & $1.00$ & $(2,2)$ & -- & 1079\\
$(c;0,4,1;9)$ & 9 & $(7,1)$ & 6 & 1 & YES & YES & YES & $1.00$ & $(2,2)$ & -- & 1080\\
$(d;0,0,0;5)$ & 5 & $(11,4)$ & 5 & 1 & YES & YES & YES & $1.00$ & $(2,2)$ & -- & 1081\\
$(d;0,0,1;14)$ & 6 & $(11,3)$ & 5 & 1 & YES & YES & YES & $1.00$ & $(2,2)$ & -- & 1082\\
$(d;0,0,2;9)$ & 7 & $(9,2)$ & 5 & 9 & YES & YES & NO(2) & $0.62$ & $(10,-2)$ & -- & 1083\\
$(d;0,0,3;22)$ & 8 & $(4,1)$ & 3 & 2 & YES & YES & YES & $1.00$ & $(2,2)$ & -- & 1084\\
$(d;0,0,3;22)$ & 8 & $(5,1)$ & 4 & 1 & YES & YES & YES & $1.00$ & $(2,2)$ & -- & 1085\\
$(d;0,0,3;22)$ & 8 & $(7,2)$ & 4 & 1 & YES & YES & YES & $1.11$ & $(2,2)$ & -- & 1086\\
$(d;0,0,4;13)$ & 9 & $(3,1)$ & 2 & 1 & YES & YES & YES & $1.00$ & $(2,2)$ & -- & 1087\\
$(d;0,1,0;6)$ & 6 & $(5,1)$ & 4 & 1 & YES & YES & NO(2) & $1.11$ & $(4,1)$ & -- & 1088\\
$(d;0,1,0;6)$ & 6 & $(5,2)$ & 3 & 1 & YES & YES & NO(2) & $0.89$ & $(6,0)$ & -- & 1089\\
$(d;0,1,0;6)$ & 6 & $(6,1)$ & 5 & 6 & YES & YES & NO(2) & $1.00$ & $(4,1)$ & -- & 1090\\
$(d;0,1,0;6)$ & 6 & $(8,3)$ & 4 & 2 & YES & YES & NO(2) & $1.10$ & $(2,2)$ & -- & 1091\\
$(d;0,1,2;11)$ & 8 & $(4,1)$ & 3 & 1 & YES & YES & NO(2) & $0.62$ & $(10,-2)$ & -- & 1092\\
$(d;0,2,0;7)$ & 7 & $(4,1)$ & 3 & 1 & YES & YES & YES & $1.10$ & $(2,2)$ & -- & 1093\\
$(d;0,2,1;20)$ & 8 & $(4,1)$ & 3 & 4 & YES & YES & YES & $1.11$ & $(2,2)$ & -- & 1094\\
$(d;0,3,1;23)$ & 9 & $(3,1)$ & 2 & 1 & YES & YES & YES & $1.00$ & $(2,2)$ & -- & 1095\\
$(d;0,3,1;23)$ & 9 & $(7,1)$ & 6 & 1 & YES & YES & YES & $1.00$ & $(2,2)$ & -- & 1096\\
$(e;0,1,0;5)$ & 6 & $(7,3)$ & 4 & 1 & YES & YES & NO(2) & $0.75$ & $(6,0)$ & -- & 1097\\
$(e;0,2,0;6)$ & 7 & $(7,2)$ & 4 & 1 & YES & YES & NO(2) & $0.62$ & $(6,0)$ & -- & 1098\\
$(e;0,3,0;7)$ & 8 & $(2,1)$ & 1 & 1 & YES & YES & YES & $1.11$ & $(2,2)$ & -- & 1099\\
$(e;0,3,0;7)$ & 8 & $(3,1)$ & 2 & 1 & YES & YES & NO(2) & $0.88$ & $(6,0)$ & -- & 1100\\
$(e;0,3,0;7)$ & 8 & $(5,1)$ & 4 & 1 & YES & YES & NO(2) & $0.75$ & $(6,0)$ & -- & 1101\\
$(e;3,0,0;10)$ & 8 & $(2,1)$ & 1 & 2 & YES & YES & NO(2) & $0.75$ & $(6,0)$ & -- & 1102\\
$(f;0,0,0;6)$ & 4 & $(16,5)$ & 7 & 2 & YES & YES & YES & $1.11$ & $(2,2)$ & -- & 1103\\
$(f;0,0,0;6)$ & 4 & $(18,7)$ & 6 & 6 & YES & YES & YES & $1.00$ & $(2,2)$ & -- & 1104\\
$(f;0,0,0;6)$ & 4 & $(19,5)$ & 7 & 1 & YES & YES & YES & $1.20$ & $(2,2)$ & -- & 1105\\
$(f;0,0,0;6)$ & 4 & $(19,6)$ & 8 & 1 & YES & YES & YES & $0.88$ & $(2,2)$ & -- & 1106\\
$(f;0,0,0;6)$ & 4 & $(23,7)$ & 7 & 1 & YES & YES & NO(2) & $0.75$ & $(10,-2)$ & -- & 1107\\
$(f;0,0,0;6)$ & 4 & $(23,9)$ & 7 & 1 & YES & YES & YES & $1.00$ & $(2,2)$ & -- & 1108\\
$(f;0,0,0;6)$ & 4 & $(24,11)$ & 8 & 6 & YES & YES & NO(2) & $1.20$ & $(4,1)$ & -- & 1109\\
$(f;0,0,0;6)$ & 4 & $(26,11)$ & 7 & 2 & YES & YES & YES & $1.11$ & $(2,2)$ & -- & 1110\\
$(f;0,0,0;6)$ & 4 & $(29,13)$ & 8 & 1 & YES & YES & NO(2) & $1.20$ & $(4,1)$ & -- & 1111\\
$(f;0,0,0;6)$ & 4 & $(30,13)$ & 8 & 6 & YES & YES & NO(2) & $1.20$ & $(4,1)$ & -- & 1112\\
$(f;0,0,0;6)$ & 4 & $(35,8)$ & 8 & 1 & YES & YES & YES & $1.00$ & $(2,2)$ & -- & 1113\\
$(f;0,1,0;7)$ & 5 & $(10,3)$ & 5 & 1 & YES & YES & YES & $0.88$ & $(4,1)$ & -- & 1114\\
$(f;0,1,0;7)$ & 5 & $(13,4)$ & 6 & 1 & YES & YES & YES & $1.11$ & $(2,2)$ & -- & 1115\\
$(f;0,1,0;7)$ & 5 & $(13,5)$ & 5 & 1 & YES & YES & YES & $0.88$ & $(4,1)$ & -- & 1116\\
$(f;0,1,0;7)$ & 5 & $(19,4)$ & 7 & 1 & YES & YES & YES & $1.11$ & $(2,2)$ & -- & 1117\\
$(f;0,2,0;8)$ & 6 & $(11,3)$ & 5 & 1 & YES & YES & NO(2) & $1.20$ & $(2,2)$ & -- & 1118\\
$(i;0,0,0;9)$ & 5 & $(6,1)$ & 5 & 3 & YES & YES & YES & $1.00$ & $(4,1)$ & -- & 1119\\
$(i;0,0,0;9)$ & 5 & $(9,4)$ & 5 & 9 & YES & YES & YES & $0.88$ & $(2,2)$ & -- & 1120\\
$(i;0,0,0;9)$ & 5 & $(10,3)$ & 5 & 1 & YES & YES & YES & $0.89$ & $(2,2)$ & -- & 1121\\
$(i;0,0,0;9)$ & 5 & $(12,5)$ & 5 & 3 & YES & YES & YES & $1.11$ & $(2,2)$ & -- & 1122\\
$(i;0,0,0;9)$ & 5 & $(19,4)$ & 7 & 1 & YES & YES & YES & $1.11$ & $(2,2)$ & -- & 1123\\
$(i;0,0,0;9)$ & 5 & $(22,5)$ & 7 & 1 & YES & YES & NO(2) & $0.75$ & $(6,0)$ & -- & 1124\\
$(i;0,1,0;12)$ & 6 & $(5,1)$ & 4 & 1 & YES & YES & NO(2) & $0.75$ & $(8,-1)$ & -- & 1125\\
$(i;0,1,0;12)$ & 6 & $(8,3)$ & 4 & 4 & YES & YES & YES & $0.88$ & $(2,2)$ & -- & 1126\\
$(i;0,1,0;12)$ & 6 & $(13,3)$ & 6 & 1 & YES & YES & YES & $1.00$ & $(2,2)$ & -- & 1127\\
$(i;0,1,0;12)$ & 6 & $(14,3)$ & 6 & 2 & YES & YES & YES & $1.11$ & $(2,2)$ & -- & 1128\\
$(i;0,2,0;15)$ & 7 & $(4,1)$ & 3 & 1 & YES & YES & YES & $1.11$ & $(2,2)$ & -- & 1129\\
$(i;0,3,0;18)$ & 8 & $(2,1)$ & 1 & 2 & YES & YES & YES & $0.88$ & $(2,2)$ & -- & 1130\\
$(i;0,3,0;18)$ & 8 & $(3,1)$ & 2 & 3 & YES & YES & YES & $1.11$ & $(2,2)$ & -- & 1131\\
$(i;0,3,0;18)$ & 8 & $(4,1)$ & 3 & 2 & YES & YES & YES & $1.11$ & $(2,2)$ & -- & 1132\\
$(i;0,3,0;18)$ & 8 & $(5,1)$ & 4 & 1 & YES & YES & YES & $0.75$ & $(4,1)$ & -- & 1133\\
$(j;0,0,0;8)$ & 5 & $(9,4)$ & 5 & 1 & YES & YES & YES & $1.11$ & $(2,2)$ & -- & 1134\\
$(j;0,0,0;8)$ & 5 & $(11,5)$ & 6 & 1 & YES & YES & YES & $0.88$ & $(2,2)$ & -- & 1135
\end{longtable}
\subsection{2 chains, $K^2 = 3$}
\begin{longtable}{|c|c|c|c|c|c|c|c|c|c|c|c|}
\hline
\multicolumn{12}{|c|}{2 chains, $K^2 = 3$}\\
\hline
$(n,a)$ & Len & $(n,a)$ & Len & GCD & Nef & $\mathbb Q$-ef & Obs 0 & $\overline c_1^2 / \overline c_2$ & $(P,K)$ & WH & Index\\
\hline
\endfirsthead

\hline
$(n,a)$ & Len & $(n,a)$ & Len & GCD & Nef & $\mathbb Q$-ef & Obs 0 & $\overline c_1^2 / \overline c_2$ & $(P,K)$ & WH & Index\\
\hline
\endhead
\hline
\endfoot

$(16,7)$ & 6 & $(16,7)$ & 6 & 16 & YES & YES & YES & $1.33$ & $(2,3)$ & -- & 1136\\
$(17,3)$ & 7 & $(14,5)$ & 6 & 1 & YES & YES & YES & $1.38$ & $(4,2)$ & -- & 1137\\
$(17,5)$ & 6 & $(14,3)$ & 6 & 1 & YES & YES & YES & $1.50$ & $(4,2)$ & -- & 1138\\
$(19,5)$ & 7 & $(10,3)$ & 5 & 1 & YES & YES & YES & $1.25$ & $(2,3)$ & -- & 1139\\
$(19,4)$ & 7 & $(16,7)$ & 6 & 1 & YES & YES & NO(2) & $1.38$ & $(4,2)$ & -- & 1140\\
$(19,6)$ & 8 & $(17,3)$ & 7 & 1 & YES & YES & YES & $1.29$ & $(4,2)$ & NO & 1141\\
$(19,6)$ & 8 & $(17,3)$ & 7 & 1 & YES & YES & YES & $1.29$ & $(4,2)$ & -- & 1142\\
$(19,6)$ & 8 & $(17,7)$ & 6 & 1 & YES & YES & YES & $1.29$ & $(4,2)$ & NO & 1143\\
$(20,9)$ & 7 & $(13,3)$ & 6 & 1 & YES & YES & NO(2) & $1.56$ & $(2,3)$ & NO & 1144\\
$(20,9)$ & 7 & $(13,3)$ & 6 & 1 & YES & YES & NO(2) & $1.56$ & $(2,3)$ & -- & 1145\\
$(20,9)$ & 7 & $(16,5)$ & 7 & 4 & YES & YES & YES & $1.50$ & $(2,3)$ & -- & 1146\\
$(20,7)$ & 8 & $(18,5)$ & 6 & 2 & YES & YES & NO(2) & $1.50$ & $(4,2)$ & NO & 1147\\
$(20,7)$ & 8 & $(18,5)$ & 6 & 2 & YES & YES & NO(2) & $1.50$ & $(4,2)$ & -- & 1148\\
$(20,7)$ & 8 & $(20,7)$ & 8 & 20 & YES & YES & YES & $1.57$ & $(4,2)$ & -- & 1149\\
$(21,8)$ & 6 & $(9,2)$ & 5 & 3 & YES & YES & YES & $1.25$ & $(4,2)$ & -- & 1150\\
$(21,4)$ & 8 & $(16,5)$ & 7 & 1 & YES & YES & YES & $1.67$ & $(2,3)$ & NO & 1151\\
$(21,4)$ & 8 & $(16,5)$ & 7 & 1 & YES & YES & YES & $1.67$ & $(2,3)$ & -- & 1152\\
$(21,4)$ & 8 & $(16,5)$ & 7 & 1 & YES & YES & YES & $1.67$ & $(2,3)$ & NO & 1153\\
$(22,7)$ & 9 & $(18,7)$ & 6 & 2 & YES & YES & YES & $1.29$ & $(4,2)$ & -- & 1154\\
$(23,6)$ & 8 & $(17,3)$ & 7 & 1 & YES & YES & YES & $1.29$ & $(4,2)$ & NO & 1155\\
$(23,6)$ & 8 & $(17,3)$ & 7 & 1 & YES & YES & YES & $1.29$ & $(4,2)$ & -- & 1156\\
$(23,8)$ & 9 & $(23,5)$ & 7 & 23 & YES & YES & YES & $1.62$ & $(2,3)$ & NO & 1157\\
$(24,7)$ & 7 & $(19,5)$ & 7 & 1 & YES & YES & NO(2) & $1.29$ & $(8,0)$ & -- & 1158\\
$(24,5)$ & 8 & $(24,5)$ & 8 & 24 & YES & YES & YES & $1.33$ & $(2,3)$ & -- & 1159\\
$(25,11)$ & 7 & $(16,5)$ & 7 & 1 & YES & YES & YES & $1.50$ & $(2,3)$ & NO & 1160\\
$(25,11)$ & 7 & $(16,5)$ & 7 & 1 & YES & YES & YES & $1.50$ & $(2,3)$ & -- & 1161\\
$(25,9)$ & 7 & $(21,5)$ & 8 & 1 & YES & YES & YES & $1.50$ & $(2,3)$ & NO & 1162\\
$(25,9)$ & 7 & $(21,5)$ & 8 & 1 & YES & YES & YES & $1.50$ & $(2,3)$ & -- & 1163\\
$(26,7)$ & 7 & $(13,3)$ & 6 & 13 & YES & YES & NO(2) & $1.56$ & $(2,3)$ & -- & 1164\\
$(26,7)$ & 7 & $(14,3)$ & 6 & 2 & YES & YES & NO(2) & $1.56$ & $(2,3)$ & -- & 1165\\
$(26,7)$ & 7 & $(18,5)$ & 6 & 2 & YES & YES & NO(2) & $1.38$ & $(6,1)$ & -- & 1166\\
$(26,7)$ & 7 & $(19,7)$ & 6 & 1 & YES & YES & NO(2) & $1.56$ & $(2,3)$ & -- & 1167\\
$(26,11)$ & 7 & $(23,10)$ & 7 & 1 & YES & YES & YES & $1.56$ & $(2,3)$ & -- & 1168\\
$(27,11)$ & 8 & $(9,2)$ & 5 & 9 & YES & YES & YES & $1.56$ & $(2,3)$ & -- & 1169\\
$(27,8)$ & 7 & $(19,7)$ & 6 & 1 & YES & YES & NO(2) & $1.50$ & $(2,3)$ & NO & 1170\\
$(27,8)$ & 7 & $(19,7)$ & 6 & 1 & YES & YES & NO(2) & $1.50$ & $(2,3)$ & -- & 1171\\
$(27,11)$ & 8 & $(19,8)$ & 6 & 1 & YES & YES & YES & $1.62$ & $(2,3)$ & -- & 1172\\
$(27,11)$ & 8 & $(19,8)$ & 6 & 1 & YES & YES & NO(2) & $1.60$ & $(6,1)$ & NO & 1173\\
$(28,11)$ & 8 & $(12,5)$ & 5 & 4 & YES & YES & YES & $1.50$ & $(2,3)$ & NO & 1174\\
$(28,11)$ & 8 & $(17,3)$ & 7 & 1 & YES & YES & YES & $1.29$ & $(4,2)$ & NO & 1175\\
$(28,11)$ & 8 & $(17,3)$ & 7 & 1 & YES & YES & YES & $1.29$ & $(4,2)$ & -- & 1176\\
$(29,11)$ & 7 & $(13,5)$ & 5 & 1 & YES & YES & NO(2) & $1.38$ & $(6,1)$ & -- & 1177\\
$(29,13)$ & 8 & $(13,4)$ & 6 & 1 & YES & YES & YES & $1.29$ & $(4,2)$ & NO & 1178\\
$(29,13)$ & 8 & $(13,4)$ & 6 & 1 & YES & YES & YES & $1.29$ & $(4,2)$ & -- & 1179\\
$(29,7)$ & 10 & $(18,7)$ & 6 & 1 & YES & YES & YES & $1.29$ & $(4,2)$ & NO & 1180\\
$(29,8)$ & 7 & $(19,6)$ & 8 & 1 & YES & YES & NO(2) & $1.56$ & $(6,1)$ & NO & 1181\\
$(29,8)$ & 7 & $(19,6)$ & 8 & 1 & YES & YES & NO(2) & $1.56$ & $(6,1)$ & -- & 1182\\
$(29,9)$ & 8 & $(19,6)$ & 8 & 1 & YES & YES & YES & $1.62$ & $(2,3)$ & -- & 1183\\
$(29,9)$ & 8 & $(19,8)$ & 6 & 1 & YES & YES & YES & $1.56$ & $(2,3)$ & NO & 1184\\
$(29,9)$ & 8 & $(19,8)$ & 6 & 1 & YES & YES & NO(2) & $1.25$ & $(6,1)$ & -- & 1185\\
$(29,12)$ & 7 & $(27,8)$ & 7 & 1 & YES & YES & NO(2) & $1.50$ & $(2,3)$ & NO & 1186\\
$(29,8)$ & 7 & $(28,11)$ & 8 & 1 & YES & YES & YES & $1.57$ & $(4,2)$ & -- & 1187\\
$(31,7)$ & 8 & $(5,2)$ & 3 & 1 & YES & YES & NO(2) & $1.38$ & $(4,2)$ & -- & 1188\\
$(31,11)$ & 8 & $(7,2)$ & 4 & 1 & YES & YES & NO(2) & $1.56$ & $(2,3)$ & NO & 1189\\
$(31,14)$ & 8 & $(7,2)$ & 4 & 1 & YES & YES & YES & $1.29$ & $(4,2)$ & NO & 1190\\
$(31,14)$ & 8 & $(7,2)$ & 4 & 1 & YES & YES & YES & $1.29$ & $(4,2)$ & -- & 1191\\
$(31,14)$ & 8 & $(10,3)$ & 5 & 1 & YES & YES & NO(2) & $1.38$ & $(6,1)$ & -- & 1192\\
$(31,11)$ & 8 & $(11,5)$ & 6 & 1 & YES & YES & NO(2) & $1.73$ & $(2,3)$ & -- & 1193\\
$(31,14)$ & 8 & $(13,5)$ & 5 & 1 & YES & YES & YES & $1.14$ & $(4,2)$ & -- & 1194\\
$(31,14)$ & 8 & $(16,5)$ & 7 & 1 & YES & YES & YES & $1.62$ & $(2,3)$ & -- & 1195\\
$(31,11)$ & 8 & $(17,4)$ & 7 & 1 & YES & YES & YES & $1.29$ & $(4,2)$ & NO & 1196\\
$(31,11)$ & 8 & $(17,4)$ & 7 & 1 & YES & YES & YES & $1.29$ & $(4,2)$ & -- & 1197\\
$(31,14)$ & 8 & $(27,11)$ & 8 & 1 & YES & YES & YES & $1.62$ & $(2,3)$ & NO & 1198\\
$(31,5)$ & 10 & $(29,6)$ & 9 & 1 & YES & YES & YES & $1.38$ & $(2,3)$ & -- & 1199\\
$(31,12)$ & 7 & $(31,12)$ & 7 & 31 & YES & YES & YES & $1.80$ & $(2,3)$ & -- & 1200\\
$(32,13)$ & 9 & $(13,2)$ & 7 & 1 & YES & YES & YES & $1.50$ & $(2,3)$ & -- & 1201\\
$(32,13)$ & 9 & $(13,2)$ & 7 & 1 & YES & YES & YES & $1.50$ & $(2,3)$ & NO & 1202\\
$(32,13)$ & 9 & $(13,6)$ & 7 & 1 & YES & YES & YES & $1.57$ & $(4,2)$ & -- & 1203\\
$(32,7)$ & 8 & $(20,7)$ & 8 & 4 & YES & YES & YES & $1.56$ & $(2,3)$ & NO & 1204\\
$(32,13)$ & 9 & $(20,7)$ & 8 & 4 & YES & YES & YES & $1.57$ & $(4,2)$ & NO & 1205\\
$(32,7)$ & 8 & $(22,7)$ & 9 & 2 & YES & YES & YES & $1.56$ & $(2,3)$ & NO & 1206\\
$(32,13)$ & 9 & $(22,5)$ & 7 & 2 & YES & YES & NO(2) & $1.70$ & $(2,3)$ & NO & 1207\\
$(33,14)$ & 8 & $(9,2)$ & 5 & 3 & YES & YES & YES & $1.44$ & $(2,3)$ & -- & 1208\\
$(33,14)$ & 8 & $(13,3)$ & 6 & 1 & YES & YES & YES & $1.62$ & $(2,3)$ & NO & 1209\\
$(33,10)$ & 8 & $(15,4)$ & 6 & 3 & YES & YES & NO(2) & $1.56$ & $(4,2)$ & NO & 1210\\
$(33,10)$ & 8 & $(15,4)$ & 6 & 3 & YES & YES & NO(2) & $1.64$ & $(4,2)$ & -- & 1211\\
$(33,10)$ & 8 & $(16,5)$ & 7 & 1 & YES & YES & NO(2) & $1.29$ & $(10,-1)$ & NO & 1212\\
$(33,10)$ & 8 & $(16,5)$ & 7 & 1 & YES & YES & NO(2) & $1.29$ & $(10,-1)$ & -- & 1213\\
$(33,10)$ & 8 & $(17,7)$ & 6 & 1 & YES & YES & NO(2) & $1.14$ & $(6,1)$ & NO & 1214\\
$(33,14)$ & 8 & $(17,7)$ & 6 & 1 & YES & YES & NO(2) & $1.70$ & $(2,3)$ & NO & 1215\\
$(33,14)$ & 8 & $(17,7)$ & 6 & 1 & YES & YES & NO(2) & $1.70$ & $(2,3)$ & -- & 1216\\
$(33,13)$ & 9 & $(19,3)$ & 8 & 1 & YES & YES & YES & $1.14$ & $(4,2)$ & NO & 1217\\
$(33,10)$ & 8 & $(23,10)$ & 7 & 1 & YES & YES & NO(2) & $1.14$ & $(6,1)$ & NO & 1218\\
$(34,9)$ & 8 & $(5,2)$ & 3 & 1 & YES & YES & YES & $1.56$ & $(2,3)$ & NO & 1219\\
$(34,9)$ & 8 & $(13,5)$ & 5 & 1 & YES & YES & YES & $1.38$ & $(2,3)$ & -- & 1220\\
$(34,13)$ & 7 & $(13,6)$ & 7 & 1 & YES & YES & NO(2) & $1.50$ & $(4,2)$ & -- & 1221\\
$(34,15)$ & 8 & $(18,5)$ & 6 & 2 & YES & YES & YES & $1.62$ & $(2,3)$ & -- & 1222\\
$(34,13)$ & 7 & $(20,7)$ & 8 & 2 & YES & YES & NO(2) & $1.50$ & $(4,2)$ & NO & 1223\\
$(34,13)$ & 7 & $(26,11)$ & 7 & 2 & YES & YES & YES & $1.78$ & $(2,3)$ & -- & 1224\\
$(35,11)$ & 9 & $(7,3)$ & 4 & 7 & YES & YES & YES & $1.50$ & $(2,3)$ & -- & 1225\\
$(35,16)$ & 9 & $(8,3)$ & 4 & 1 & YES & YES & YES & $1.56$ & $(2,3)$ & -- & 1226\\
$(35,11)$ & 9 & $(10,3)$ & 5 & 5 & YES & YES & YES & $1.50$ & $(2,3)$ & NO & 1227\\
$(35,11)$ & 9 & $(10,3)$ & 5 & 5 & YES & YES & YES & $1.50$ & $(2,3)$ & -- & 1228\\
$(35,16)$ & 9 & $(11,4)$ & 5 & 1 & YES & YES & YES & $1.29$ & $(4,2)$ & NO & 1229\\
$(35,11)$ & 9 & $(17,7)$ & 6 & 1 & YES & YES & NO(2) & $1.56$ & $(4,2)$ & -- & 1230\\
$(35,8)$ & 8 & $(23,10)$ & 7 & 1 & YES & YES & YES & $1.50$ & $(2,3)$ & -- & 1231\\
$(36,13)$ & 8 & $(11,4)$ & 5 & 1 & YES & YES & YES & $1.29$ & $(4,2)$ & -- & 1232\\
$(36,13)$ & 8 & $(17,7)$ & 6 & 1 & YES & YES & NO(2) & $1.14$ & $(6,1)$ & -- & 1233\\
$(36,11)$ & 8 & $(20,9)$ & 7 & 4 & YES & YES & YES & $1.50$ & $(2,3)$ & NO & 1234\\
$(36,13)$ & 8 & $(23,5)$ & 7 & 1 & YES & YES & NO(2) & $1.60$ & $(2,3)$ & -- & 1235\\
$(37,16)$ & 9 & $(9,2)$ & 5 & 1 & YES & YES & YES & $1.44$ & $(2,3)$ & -- & 1236\\
$(37,16)$ & 9 & $(11,2)$ & 6 & 1 & YES & YES & YES & $1.44$ & $(2,3)$ & NO & 1237\\
$(37,11)$ & 8 & $(13,6)$ & 7 & 1 & YES & YES & YES & $1.57$ & $(2,3)$ & -- & 1238\\
$(37,14)$ & 8 & $(13,4)$ & 6 & 1 & YES & YES & YES & $1.43$ & $(4,2)$ & NO & 1239\\
$(37,14)$ & 8 & $(13,4)$ & 6 & 1 & YES & YES & YES & $1.43$ & $(4,2)$ & -- & 1240\\
$(37,8)$ & 8 & $(20,7)$ & 8 & 1 & YES & YES & NO(2) & $1.29$ & $(8,0)$ & NO & 1241\\
$(37,8)$ & 8 & $(22,7)$ & 9 & 1 & YES & YES & NO(2) & $1.29$ & $(8,0)$ & NO & 1242\\
$(37,14)$ & 8 & $(23,4)$ & 8 & 1 & YES & YES & YES & $1.50$ & $(2,3)$ & NO & 1243\\
$(37,11)$ & 8 & $(31,12)$ & 7 & 1 & YES & YES & YES & $1.75$ & $(2,3)$ & -- & 1244\\
$(38,9)$ & 9 & $(14,3)$ & 6 & 2 & YES & YES & NO(2) & $1.50$ & $(4,2)$ & -- & 1245\\
$(38,17)$ & 9 & $(16,3)$ & 7 & 2 & YES & YES & YES & $1.14$ & $(4,2)$ & -- & 1246\\
$(38,7)$ & 9 & $(22,7)$ & 9 & 2 & YES & YES & NO(2) & $1.29$ & $(8,0)$ & NO & 1247\\
$(38,7)$ & 9 & $(27,7)$ & 9 & 1 & YES & YES & NO(2) & $1.29$ & $(8,0)$ & NO & 1248\\
$(38,9)$ & 9 & $(31,9)$ & 8 & 1 & YES & YES & NO(2) & $1.56$ & $(4,2)$ & NO & 1249\\
$(39,14)$ & 8 & $(7,2)$ & 4 & 1 & YES & YES & YES & $1.29$ & $(4,2)$ & NO & 1250\\
$(39,14)$ & 8 & $(7,2)$ & 4 & 1 & YES & YES & YES & $1.29$ & $(4,2)$ & -- & 1251\\
$(39,14)$ & 8 & $(24,7)$ & 7 & 3 & YES & YES & YES & $1.43$ & $(4,2)$ & -- & 1252\\
$(39,14)$ & 8 & $(25,7)$ & 7 & 1 & YES & YES & YES & $1.43$ & $(4,2)$ & -- & 1253\\
$(39,11)$ & 9 & $(38,7)$ & 9 & 1 & YES & YES & YES & $1.43$ & $(4,2)$ & NO & 1254\\
$(40,11)$ & 8 & $(7,2)$ & 4 & 1 & YES & YES & NO(2) & $1.29$ & $(8,0)$ & -- & 1255\\
$(40,11)$ & 8 & $(9,4)$ & 5 & 1 & YES & YES & YES & $1.29$ & $(4,2)$ & NO & 1256\\
$(40,11)$ & 8 & $(9,4)$ & 5 & 1 & YES & YES & YES & $1.29$ & $(4,2)$ & -- & 1257\\
$(40,11)$ & 8 & $(9,4)$ & 5 & 1 & YES & YES & YES & $1.29$ & $(4,2)$ & NO & 1258\\
$(40,11)$ & 8 & $(13,6)$ & 7 & 1 & YES & YES & YES & $1.56$ & $(2,3)$ & -- & 1259\\
$(40,17)$ & 9 & $(13,2)$ & 7 & 1 & YES & YES & YES & $1.38$ & $(2,3)$ & NO & 1260\\
$(40,17)$ & 9 & $(13,2)$ & 7 & 1 & YES & YES & YES & $1.38$ & $(2,3)$ & -- & 1261\\
$(40,17)$ & 9 & $(13,5)$ & 5 & 1 & YES & YES & YES & $1.50$ & $(2,3)$ & -- & 1262\\
$(40,11)$ & 8 & $(16,5)$ & 7 & 8 & YES & YES & YES & $1.29$ & $(4,2)$ & NO & 1263\\
$(40,11)$ & 8 & $(22,7)$ & 9 & 2 & YES & YES & YES & $1.56$ & $(2,3)$ & NO & 1264\\
$(40,9)$ & 9 & $(39,7)$ & 9 & 1 & YES & YES & NO(2) & $1.25$ & $(8,0)$ & -- & 1265\\
$(41,11)$ & 8 & $(5,2)$ & 3 & 1 & YES & YES & NO(2) & $1.38$ & $(6,1)$ & NO & 1266\\
$(41,13)$ & 10 & $(7,2)$ & 4 & 1 & YES & YES & YES & $1.43$ & $(6,1)$ & NO & 1267\\
$(41,13)$ & 10 & $(7,2)$ & 4 & 1 & YES & YES & YES & $1.43$ & $(6,1)$ & -- & 1268\\
$(41,19)$ & 10 & $(7,3)$ & 4 & 1 & YES & YES & NO(2) & $1.70$ & $(2,3)$ & -- & 1269\\
$(41,18)$ & 8 & $(8,3)$ & 4 & 1 & YES & YES & NO(2) & $1.44$ & $(6,1)$ & -- & 1270\\
$(41,9)$ & 9 & $(11,2)$ & 6 & 1 & YES & YES & NO(2) & $1.44$ & $(2,3)$ & -- & 1271\\
$(41,9)$ & 9 & $(11,2)$ & 6 & 1 & YES & YES & NO(2) & $1.56$ & $(2,3)$ & NO & 1272\\
$(41,18)$ & 8 & $(18,7)$ & 6 & 1 & YES & YES & YES & $1.43$ & $(4,2)$ & -- & 1273\\
$(41,17)$ & 8 & $(23,7)$ & 7 & 1 & YES & YES & YES & $1.62$ & $(2,3)$ & -- & 1274\\
$(41,15)$ & 8 & $(24,7)$ & 7 & 1 & YES & YES & YES & $1.88$ & $(2,3)$ & NO & 1275\\
$(41,15)$ & 8 & $(24,7)$ & 7 & 1 & YES & YES & YES & $1.88$ & $(2,3)$ & -- & 1276\\
$(41,18)$ & 8 & $(25,7)$ & 7 & 1 & YES & YES & YES & $1.43$ & $(4,2)$ & NO & 1277\\
$(41,12)$ & 8 & $(38,9)$ & 9 & 1 & YES & YES & NO(2) & $1.44$ & $(4,2)$ & NO & 1278\\
$(42,19)$ & 9 & $(5,2)$ & 3 & 1 & YES & YES & NO(2) & $1.64$ & $(2,3)$ & -- & 1279\\
$(42,19)$ & 9 & $(16,3)$ & 7 & 2 & YES & YES & YES & $1.14$ & $(4,2)$ & NO & 1280\\
$(42,13)$ & 9 & $(18,7)$ & 6 & 6 & YES & YES & NO(2) & $1.14$ & $(6,1)$ & -- & 1281\\
$(42,5)$ & 11 & $(23,8)$ & 9 & 1 & YES & YES & YES & $1.62$ & $(2,3)$ & NO & 1282\\
$(42,11)$ & 9 & $(23,7)$ & 7 & 1 & YES & YES & YES & $1.56$ & $(2,3)$ & NO & 1283\\
$(43,15)$ & 10 & $(7,3)$ & 4 & 1 & YES & YES & NO(2) & $1.70$ & $(2,3)$ & -- & 1284\\
$(43,19)$ & 9 & $(7,2)$ & 4 & 1 & YES & YES & NO(2) & $1.56$ & $(2,3)$ & -- & 1285\\
$(43,19)$ & 9 & $(13,4)$ & 6 & 1 & YES & YES & YES & $1.50$ & $(2,3)$ & -- & 1286\\
$(43,16)$ & 9 & $(25,9)$ & 7 & 1 & YES & YES & YES & $1.50$ & $(2,3)$ & NO & 1287\\
$(43,13)$ & 9 & $(28,5)$ & 8 & 1 & YES & YES & NO(2) & $1.50$ & $(2,3)$ & -- & 1288\\
$(44,13)$ & 8 & $(13,6)$ & 7 & 1 & YES & YES & NO(2) & $1.29$ & $(8,0)$ & -- & 1289\\
$(44,13)$ & 8 & $(19,5)$ & 7 & 1 & YES & YES & NO(2) & $1.29$ & $(8,0)$ & NO & 1290\\
$(44,13)$ & 8 & $(23,9)$ & 7 & 1 & YES & YES & YES & $1.89$ & $(2,3)$ & -- & 1291\\
$(44,17)$ & 8 & $(24,7)$ & 7 & 4 & YES & YES & YES & $1.88$ & $(2,3)$ & -- & 1292\\
$(45,14)$ & 9 & $(5,2)$ & 3 & 5 & YES & YES & NO(2) & $1.50$ & $(2,3)$ & -- & 1293\\
$(45,16)$ & 9 & $(8,3)$ & 4 & 1 & YES & YES & YES & $1.56$ & $(2,3)$ & -- & 1294\\
$(45,14)$ & 9 & $(10,3)$ & 5 & 5 & YES & YES & NO(2) & $1.14$ & $(8,0)$ & -- & 1295\\
$(45,17)$ & 9 & $(10,3)$ & 5 & 5 & YES & YES & YES & $1.62$ & $(2,3)$ & NO & 1296\\
$(45,17)$ & 9 & $(10,3)$ & 5 & 5 & YES & YES & YES & $1.62$ & $(2,3)$ & -- & 1297\\
$(45,19)$ & 8 & $(12,5)$ & 5 & 3 & YES & YES & NO(2) & $1.44$ & $(4,2)$ & -- & 1298\\
$(45,19)$ & 8 & $(24,7)$ & 7 & 3 & YES & YES & YES & $1.75$ & $(2,3)$ & -- & 1299\\
$(45,19)$ & 8 & $(33,14)$ & 8 & 3 & YES & YES & YES & $1.44$ & $(2,3)$ & NO & 1300\\
$(47,18)$ & 8 & $(9,4)$ & 5 & 1 & YES & YES & YES & $1.29$ & $(4,2)$ & NO & 1301\\
$(47,18)$ & 8 & $(9,4)$ & 5 & 1 & YES & YES & YES & $1.29$ & $(4,2)$ & -- & 1302\\
$(47,13)$ & 8 & $(13,6)$ & 7 & 1 & YES & YES & NO(2) & $1.29$ & $(8,0)$ & -- & 1303\\
$(47,20)$ & 10 & $(13,3)$ & 6 & 1 & YES & YES & NO(2) & $1.44$ & $(4,2)$ & -- & 1304\\
$(47,13)$ & 8 & $(17,6)$ & 7 & 1 & YES & YES & NO(2) & $1.29$ & $(8,0)$ & NO & 1305\\
$(47,17)$ & 9 & $(17,3)$ & 7 & 1 & YES & YES & NO(2) & $1.64$ & $(2,3)$ & -- & 1306\\
$(47,13)$ & 8 & $(22,7)$ & 9 & 1 & YES & YES & NO(2) & $1.29$ & $(8,0)$ & NO & 1307\\
$(47,13)$ & 8 & $(23,9)$ & 7 & 1 & YES & YES & YES & $2.00$ & $(2,3)$ & -- & 1308\\
$(47,13)$ & 8 & $(32,9)$ & 8 & 1 & YES & YES & NO(2) & $1.38$ & $(6,1)$ & NO & 1309\\
$(48,17)$ & 9 & $(7,2)$ & 4 & 1 & YES & YES & YES & $1.56$ & $(2,3)$ & -- & 1310\\
$(48,11)$ & 9 & $(11,3)$ & 5 & 1 & YES & YES & NO(2) & $1.60$ & $(4,2)$ & NO & 1311\\
$(48,11)$ & 9 & $(11,3)$ & 5 & 1 & YES & YES & NO(2) & $1.60$ & $(4,2)$ & -- & 1312\\
$(48,17)$ & 9 & $(19,7)$ & 6 & 1 & YES & YES & YES & $1.56$ & $(2,3)$ & 1512 & 1313\\
$(48,17)$ & 9 & $(20,7)$ & 8 & 4 & YES & YES & YES & $1.44$ & $(2,3)$ & NO & 1314\\
$(48,13)$ & 9 & $(38,7)$ & 9 & 2 & YES & YES & YES & $1.62$ & $(2,3)$ & -- & 1315\\
$(49,13)$ & 9 & $(5,2)$ & 3 & 1 & YES & YES & YES & $1.56$ & $(2,3)$ & NO & 1316\\
$(49,13)$ & 9 & $(5,2)$ & 3 & 1 & YES & YES & YES & $1.56$ & $(2,3)$ & -- & 1317\\
$(49,20)$ & 9 & $(5,1)$ & 4 & 1 & YES & YES & YES & $1.56$ & $(2,3)$ & -- & 1318\\
$(49,20)$ & 9 & $(7,2)$ & 4 & 7 & YES & YES & NO(2) & $1.56$ & $(2,3)$ & -- & 1319\\
$(49,9)$ & 10 & $(11,5)$ & 6 & 1 & YES & YES & YES & $1.60$ & $(2,3)$ & -- & 1320\\
$(49,13)$ & 9 & $(11,4)$ & 5 & 1 & YES & YES & YES & $1.62$ & $(2,3)$ & NO & 1321\\
$(49,13)$ & 9 & $(11,4)$ & 5 & 1 & YES & YES & YES & $1.62$ & $(2,3)$ & -- & 1322\\
$(49,18)$ & 8 & $(23,8)$ & 9 & 1 & YES & YES & YES & $1.62$ & $(2,3)$ & NO & 1323\\
$(49,19)$ & 8 & $(24,7)$ & 7 & 1 & YES & YES & YES & $1.88$ & $(2,3)$ & -- & 1324\\
$(49,11)$ & 10 & $(25,4)$ & 9 & 1 & YES & YES & YES & $1.38$ & $(2,3)$ & -- & 1325\\
$(49,18)$ & 8 & $(25,7)$ & 7 & 1 & YES & YES & YES & $1.75$ & $(2,3)$ & NO & 1326\\
$(49,18)$ & 8 & $(25,7)$ & 7 & 1 & YES & YES & YES & $1.75$ & $(2,3)$ & -- & 1327\\
$(49,20)$ & 9 & $(32,13)$ & 9 & 1 & YES & YES & YES & $1.50$ & $(2,3)$ & NO & 1328\\
$(50,13)$ & 10 & $(13,5)$ & 5 & 1 & YES & YES & NO(2) & $1.29$ & $(8,0)$ & -- & 1329\\
$(50,19)$ & 8 & $(18,7)$ & 6 & 2 & YES & YES & YES & $1.75$ & $(2,3)$ & -- & 1330\\
$(51,14)$ & 9 & $(7,2)$ & 4 & 1 & YES & YES & NO(2) & $1.38$ & $(6,1)$ & -- & 1331\\
$(51,23)$ & 9 & $(7,3)$ & 4 & 1 & YES & YES & YES & $1.14$ & $(4,2)$ & -- & 1332\\
$(51,16)$ & 10 & $(12,5)$ & 5 & 3 & YES & YES & YES & $1.56$ & $(2,3)$ & NO & 1333\\
$(51,11)$ & 9 & $(18,7)$ & 6 & 3 & YES & YES & NO(2) & $1.44$ & $(4,2)$ & NO & 1334\\
$(51,11)$ & 9 & $(27,10)$ & 7 & 3 & YES & YES & NO(2) & $1.14$ & $(6,1)$ & -- & 1335\\
$(52,19)$ & 9 & $(7,2)$ & 4 & 1 & YES & YES & NO(2) & $1.56$ & $(2,3)$ & -- & 1336\\
$(52,23)$ & 10 & $(7,2)$ & 4 & 1 & YES & YES & YES & $1.50$ & $(2,3)$ & -- & 1337\\
$(52,11)$ & 9 & $(17,7)$ & 6 & 1 & YES & YES & NO(2) & $1.44$ & $(4,2)$ & NO & 1338\\
$(52,11)$ & 9 & $(17,7)$ & 6 & 1 & YES & YES & NO(2) & $1.60$ & $(2,3)$ & -- & 1339\\
$(52,15)$ & 11 & $(17,3)$ & 7 & 1 & YES & YES & NO(2) & $1.14$ & $(8,0)$ & -- & 1340\\
$(52,11)$ & 9 & $(25,7)$ & 7 & 1 & YES & YES & NO(2) & $1.33$ & $(4,2)$ & NO & 1341\\
$(52,11)$ & 9 & $(25,7)$ & 7 & 1 & YES & YES & NO(2) & $1.33$ & $(4,2)$ & -- & 1342\\
$(52,11)$ & 9 & $(43,10)$ & 9 & 1 & YES & YES & NO(2) & $1.50$ & $(2,3)$ & NO & 1343\\
$(53,19)$ & 9 & $(4,1)$ & 3 & 1 & YES & YES & NO(2) & $1.64$ & $(2,3)$ & -- & 1344\\
$(53,14)$ & 9 & $(5,2)$ & 3 & 1 & YES & YES & YES & $1.50$ & $(4,2)$ & NO & 1345\\
$(53,14)$ & 9 & $(5,2)$ & 3 & 1 & YES & YES & YES & $1.50$ & $(4,2)$ & -- & 1346\\
$(53,15)$ & 11 & $(5,1)$ & 4 & 1 & YES & YES & YES & $1.43$ & $(6,1)$ & NO & 1347\\
$(53,15)$ & 11 & $(5,1)$ & 4 & 1 & YES & YES & YES & $1.43$ & $(6,1)$ & -- & 1348\\
$(53,19)$ & 9 & $(5,2)$ & 3 & 1 & YES & YES & YES & $1.29$ & $(4,2)$ & -- & 1349\\
$(53,22)$ & 9 & $(6,1)$ & 5 & 1 & YES & YES & YES & $1.43$ & $(4,2)$ & NO & 1350\\
$(53,12)$ & 9 & $(7,3)$ & 4 & 1 & YES & YES & NO(2) & $1.44$ & $(2,3)$ & -- & 1351\\
$(53,14)$ & 9 & $(7,2)$ & 4 & 1 & YES & YES & NO(2) & $1.38$ & $(6,1)$ & NO & 1352\\
$(53,14)$ & 9 & $(7,2)$ & 4 & 1 & YES & YES & NO(2) & $1.38$ & $(6,1)$ & -- & 1353\\
$(53,14)$ & 9 & $(7,3)$ & 4 & 1 & YES & YES & YES & $1.38$ & $(2,3)$ & -- & 1354\\
$(53,19)$ & 9 & $(7,3)$ & 4 & 1 & YES & YES & NO(2) & $1.64$ & $(2,3)$ & -- & 1355\\
$(53,24)$ & 10 & $(7,2)$ & 4 & 1 & YES & YES & YES & $1.29$ & $(4,2)$ & NO & 1356\\
$(53,11)$ & 10 & $(9,4)$ & 5 & 1 & YES & YES & YES & $1.38$ & $(2,3)$ & -- & 1357\\
$(53,14)$ & 9 & $(9,2)$ & 5 & 1 & YES & YES & YES & $1.25$ & $(4,2)$ & -- & 1358\\
$(53,14)$ & 9 & $(9,2)$ & 5 & 1 & YES & YES & YES & $1.38$ & $(4,2)$ & NO & 1359\\
$(53,14)$ & 9 & $(10,3)$ & 5 & 1 & YES & YES & YES & $1.38$ & $(4,2)$ & NO & 1360\\
$(53,20)$ & 10 & $(11,3)$ & 5 & 1 & YES & YES & YES & $1.50$ & $(2,3)$ & NO & 1361\\
$(53,24)$ & 10 & $(11,3)$ & 5 & 1 & YES & YES & NO(2) & $1.70$ & $(2,3)$ & -- & 1362\\
$(53,14)$ & 9 & $(12,5)$ & 5 & 1 & YES & YES & NO(2) & $1.38$ & $(6,1)$ & -- & 1363\\
$(53,24)$ & 10 & $(17,7)$ & 6 & 1 & YES & YES & NO(2) & $1.70$ & $(2,3)$ & NO & 1364\\
$(53,14)$ & 9 & $(18,5)$ & 6 & 1 & YES & YES & NO(2) & $1.38$ & $(6,1)$ & 1583 & 1365\\
$(53,24)$ & 10 & $(19,8)$ & 6 & 1 & YES & YES & NO(2) & $1.70$ & $(2,3)$ & NO & 1366\\
$(53,7)$ & 11 & $(20,7)$ & 8 & 1 & YES & YES & YES & $1.29$ & $(4,2)$ & NO & 1367\\
$(53,22)$ & 9 & $(22,5)$ & 7 & 1 & YES & YES & YES & $1.62$ & $(2,3)$ & -- & 1368\\
$(53,14)$ & 9 & $(23,5)$ & 7 & 1 & YES & YES & NO(2) & $1.25$ & $(6,1)$ & -- & 1369\\
$(53,14)$ & 9 & $(23,6)$ & 8 & 1 & YES & YES & YES & $1.29$ & $(4,2)$ & NO & 1370\\
$(53,14)$ & 9 & $(23,7)$ & 7 & 1 & YES & YES & NO(2) & $1.38$ & $(6,1)$ & NO & 1371\\
$(53,22)$ & 9 & $(23,5)$ & 7 & 1 & YES & YES & YES & $1.62$ & $(2,3)$ & -- & 1372\\
$(53,14)$ & 9 & $(26,7)$ & 7 & 1 & YES & YES & YES & $1.38$ & $(4,2)$ & NO & 1373\\
$(53,24)$ & 10 & $(29,13)$ & 8 & 1 & YES & YES & YES & $1.29$ & $(4,2)$ & NO & 1374\\
$(53,15)$ & 11 & $(39,11)$ & 9 & 1 & YES & YES & YES & $1.43$ & $(6,1)$ & 1571 & 1375\\
$(53,7)$ & 11 & $(43,7)$ & 12 & 1 & YES & YES & YES & $1.29$ & $(4,2)$ & NO & 1376\\
$(53,24)$ & 10 & $(51,23)$ & 9 & 1 & YES & YES & YES & $1.14$ & $(4,2)$ & 1658 & 1377\\
$(54,17)$ & 10 & $(9,4)$ & 5 & 9 & YES & YES & YES & $1.56$ & $(2,3)$ & NO & 1378\\
$(55,23)$ & 9 & $(6,1)$ & 5 & 1 & YES & YES & YES & $1.57$ & $(2,3)$ & -- & 1379\\
$(55,23)$ & 9 & $(8,3)$ & 4 & 1 & YES & YES & YES & $1.50$ & $(2,3)$ & NO & 1380\\
$(55,23)$ & 9 & $(8,3)$ & 4 & 1 & YES & YES & YES & $1.50$ & $(2,3)$ & -- & 1381\\
$(55,21)$ & 8 & $(11,5)$ & 6 & 11 & YES & YES & NO(2) & $1.50$ & $(4,2)$ & NO & 1382\\
$(55,21)$ & 8 & $(18,7)$ & 6 & 1 & YES & YES & YES & $1.75$ & $(2,3)$ & -- & 1383\\
$(55,16)$ & 9 & $(21,5)$ & 8 & 1 & YES & YES & YES & $1.50$ & $(2,3)$ & NO & 1384\\
$(56,15)$ & 9 & $(3,1)$ & 2 & 1 & YES & YES & YES & $1.60$ & $(2,3)$ & NO & 1385\\
$(56,15)$ & 9 & $(13,5)$ & 5 & 1 & YES & YES & YES & $1.50$ & $(2,3)$ & NO & 1386\\
$(56,15)$ & 9 & $(18,7)$ & 6 & 2 & YES & YES & YES & $1.62$ & $(2,3)$ & NO & 1387\\
$(56,15)$ & 9 & $(18,7)$ & 6 & 2 & YES & YES & YES & $1.62$ & $(2,3)$ & -- & 1388\\
$(57,17)$ & 10 & $(13,5)$ & 5 & 1 & YES & YES & YES & $1.50$ & $(2,3)$ & NO & 1389\\
$(57,22)$ & 9 & $(23,4)$ & 8 & 1 & YES & YES & NO(2) & $1.38$ & $(8,0)$ & NO & 1390\\
$(57,17)$ & 10 & $(29,9)$ & 8 & 1 & YES & YES & YES & $1.50$ & $(2,3)$ & NO & 1391\\
$(58,17)$ & 9 & $(16,5)$ & 7 & 2 & YES & YES & NO(2) & $1.38$ & $(10,-1)$ & NO & 1392\\
$(58,9)$ & 11 & $(17,6)$ & 7 & 1 & YES & YES & YES & $1.50$ & $(2,3)$ & NO & 1393\\
$(58,17)$ & 9 & $(22,7)$ & 9 & 2 & YES & YES & NO(2) & $1.29$ & $(8,0)$ & NO & 1394\\
$(58,9)$ & 11 & $(31,6)$ & 10 & 1 & YES & YES & YES & $1.50$ & $(2,3)$ & NO & 1395\\
$(58,13)$ & 11 & $(53,12)$ & 9 & 1 & YES & YES & NO(2) & $1.44$ & $(2,3)$ & NO & 1396\\
$(59,24)$ & 10 & $(4,1)$ & 3 & 1 & YES & YES & YES & $1.50$ & $(2,3)$ & -- & 1397\\
$(59,25)$ & 9 & $(4,1)$ & 3 & 1 & YES & YES & YES & $1.56$ & $(2,3)$ & NO & 1398\\
$(59,25)$ & 9 & $(4,1)$ & 3 & 1 & YES & YES & YES & $1.56$ & $(2,3)$ & -- & 1399\\
$(59,25)$ & 9 & $(5,2)$ & 3 & 1 & YES & YES & YES & $1.38$ & $(2,3)$ & -- & 1400\\
$(59,26)$ & 9 & $(5,2)$ & 3 & 1 & YES & YES & NO(2) & $1.12$ & $(6,1)$ & -- & 1401\\
$(59,25)$ & 9 & $(7,3)$ & 4 & 1 & YES & YES & YES & $1.29$ & $(4,2)$ & -- & 1402\\
$(59,24)$ & 10 & $(9,4)$ & 5 & 1 & YES & YES & YES & $1.50$ & $(2,3)$ & NO & 1403\\
$(59,24)$ & 10 & $(11,2)$ & 6 & 1 & YES & YES & YES & $1.44$ & $(2,3)$ & -- & 1404\\
$(59,25)$ & 9 & $(12,5)$ & 5 & 1 & YES & YES & NO(2) & $1.60$ & $(2,3)$ & -- & 1405\\
$(59,26)$ & 9 & $(12,5)$ & 5 & 1 & YES & YES & NO(2) & $1.25$ & $(6,1)$ & NO & 1406\\
$(59,23)$ & 9 & $(17,5)$ & 6 & 1 & YES & YES & YES & $1.75$ & $(2,3)$ & -- & 1407\\
$(59,23)$ & 9 & $(18,5)$ & 6 & 1 & YES & YES & YES & $1.75$ & $(2,3)$ & -- & 1408\\
$(59,24)$ & 10 & $(19,8)$ & 6 & 1 & YES & YES & YES & $1.56$ & $(2,3)$ & NO & 1409\\
$(59,26)$ & 9 & $(23,10)$ & 7 & 1 & YES & YES & NO(2) & $1.25$ & $(6,1)$ & NO & 1410\\
$(59,23)$ & 9 & $(33,13)$ & 9 & 1 & YES & YES & YES & $1.14$ & $(4,2)$ & NO & 1411\\
$(59,25)$ & 9 & $(40,17)$ & 9 & 1 & YES & YES & YES & $1.38$ & $(2,3)$ & NO & 1412\\
$(60,19)$ & 11 & $(7,3)$ & 4 & 1 & YES & YES & YES & $1.56$ & $(2,3)$ & -- & 1413\\
$(60,23)$ & 9 & $(7,3)$ & 4 & 1 & YES & YES & YES & $1.50$ & $(2,3)$ & NO & 1414\\
$(60,23)$ & 9 & $(7,3)$ & 4 & 1 & YES & YES & YES & $1.50$ & $(2,3)$ & -- & 1415\\
$(60,13)$ & 9 & $(11,4)$ & 5 & 1 & YES & YES & YES & $1.38$ & $(2,3)$ & NO & 1416\\
$(60,13)$ & 9 & $(11,4)$ & 5 & 1 & YES & YES & YES & $1.38$ & $(2,3)$ & -- & 1417\\
$(60,23)$ & 9 & $(12,5)$ & 5 & 12 & YES & YES & YES & $1.50$ & $(2,3)$ & NO & 1418\\
$(60,19)$ & 11 & $(54,17)$ & 10 & 6 & YES & YES & YES & $1.56$ & $(2,3)$ & NO & 1419\\
$(61,25)$ & 9 & $(3,1)$ & 2 & 1 & YES & YES & NO(2) & $1.50$ & $(2,3)$ & -- & 1420\\
$(61,25)$ & 9 & $(4,1)$ & 3 & 1 & YES & YES & YES & $1.56$ & $(2,3)$ & NO & 1421\\
$(61,25)$ & 9 & $(4,1)$ & 3 & 1 & YES & YES & YES & $1.56$ & $(2,3)$ & -- & 1422\\
$(61,18)$ & 9 & $(5,2)$ & 3 & 1 & YES & YES & NO(2) & $1.25$ & $(6,1)$ & -- & 1423\\
$(61,25)$ & 9 & $(5,2)$ & 3 & 1 & YES & YES & NO(2) & $1.50$ & $(2,3)$ & NO & 1424\\
$(61,25)$ & 9 & $(5,2)$ & 3 & 1 & YES & YES & YES & $1.56$ & $(2,3)$ & NO & 1425\\
$(61,25)$ & 9 & $(5,2)$ & 3 & 1 & YES & YES & YES & $1.56$ & $(2,3)$ & -- & 1426\\
$(61,22)$ & 9 & $(9,4)$ & 5 & 1 & YES & YES & NO(2) & $1.29$ & $(6,1)$ & -- & 1427\\
$(61,18)$ & 9 & $(11,3)$ & 5 & 1 & YES & YES & NO(2) & $1.25$ & $(6,1)$ & NO & 1428\\
$(61,16)$ & 10 & $(12,5)$ & 5 & 1 & YES & YES & YES & $1.67$ & $(2,3)$ & NO & 1429\\
$(61,16)$ & 10 & $(13,5)$ & 5 & 1 & YES & YES & YES & $1.67$ & $(2,3)$ & NO & 1430\\
$(61,23)$ & 11 & $(13,2)$ & 7 & 1 & YES & YES & NO(2) & $1.29$ & $(8,0)$ & -- & 1431\\
$(61,14)$ & 10 & $(16,5)$ & 7 & 1 & YES & YES & NO(2) & $1.56$ & $(4,2)$ & NO & 1432\\
$(61,17)$ & 9 & $(19,5)$ & 7 & 1 & YES & YES & NO(2) & $1.29$ & $(8,0)$ & NO & 1433\\
$(61,25)$ & 9 & $(32,13)$ & 9 & 1 & YES & YES & YES & $1.29$ & $(4,2)$ & NO & 1434\\
$(61,18)$ & 9 & $(42,13)$ & 9 & 1 & YES & YES & YES & $1.43$ & $(4,2)$ & NO & 1435\\
$(63,26)$ & 9 & $(13,6)$ & 7 & 1 & YES & YES & YES & $1.29$ & $(4,2)$ & NO & 1436\\
$(63,11)$ & 10 & $(19,5)$ & 7 & 1 & YES & YES & YES & $1.44$ & $(2,3)$ & -- & 1437\\
$(64,19)$ & 9 & $(12,5)$ & 5 & 4 & YES & YES & NO(2) & $1.44$ & $(4,2)$ & NO & 1438\\
$(64,15)$ & 10 & $(18,7)$ & 6 & 2 & YES & YES & YES & $1.62$ & $(2,3)$ & -- & 1439\\
$(64,15)$ & 10 & $(18,7)$ & 6 & 2 & YES & YES & YES & $1.62$ & $(2,3)$ & NO & 1440\\
$(65,23)$ & 10 & $(4,1)$ & 3 & 1 & YES & YES & NO(2) & $1.64$ & $(2,3)$ & -- & 1441\\
$(65,19)$ & 9 & $(7,3)$ & 4 & 1 & YES & YES & NO(2) & $1.33$ & $(4,2)$ & -- & 1442\\
$(65,23)$ & 10 & $(8,3)$ & 4 & 1 & YES & YES & NO(2) & $1.64$ & $(2,3)$ & NO & 1443\\
$(65,27)$ & 10 & $(11,3)$ & 5 & 1 & YES & YES & YES & $1.62$ & $(2,3)$ & -- & 1444\\
$(65,24)$ & 9 & $(20,7)$ & 8 & 5 & YES & YES & YES & $1.29$ & $(4,2)$ & NO & 1445\\
$(66,25)$ & 9 & $(61,23)$ & 11 & 1 & YES & YES & NO(2) & $1.29$ & $(8,0)$ & NO & 1446\\
$(67,21)$ & 11 & $(5,1)$ & 4 & 1 & YES & YES & YES & $1.50$ & $(2,3)$ & -- & 1447\\
$(67,18)$ & 9 & $(7,2)$ & 4 & 1 & YES & YES & NO(2) & $1.38$ & $(6,1)$ & NO & 1448\\
$(67,18)$ & 9 & $(7,2)$ & 4 & 1 & YES & YES & NO(2) & $1.38$ & $(6,1)$ & -- & 1449\\
$(67,20)$ & 11 & $(7,3)$ & 4 & 1 & YES & YES & YES & $1.50$ & $(2,3)$ & NO & 1450\\
$(67,24)$ & 10 & $(7,2)$ & 4 & 1 & YES & YES & YES & $1.29$ & $(4,2)$ & NO & 1451\\
$(67,26)$ & 9 & $(9,4)$ & 5 & 1 & YES & YES & NO(2) & $1.29$ & $(6,1)$ & -- & 1452\\
$(67,29)$ & 10 & $(44,19)$ & 10 & 1 & YES & YES & YES & $1.50$ & $(2,3)$ & NO & 1453\\
$(67,18)$ & 9 & $(53,14)$ & 9 & 1 & YES & YES & NO(2) & $1.25$ & $(6,1)$ & NO & 1454\\
$(68,19)$ & 9 & $(7,3)$ & 4 & 1 & YES & YES & YES & $1.50$ & $(2,3)$ & -- & 1455\\
$(68,25)$ & 9 & $(31,11)$ & 8 & 1 & YES & YES & NO(2) & $1.64$ & $(2,3)$ & NO & 1456\\
$(69,26)$ & 12 & $(8,1)$ & 7 & 1 & YES & YES & YES & $1.50$ & $(4,2)$ & NO & 1457\\
$(69,19)$ & 9 & $(9,4)$ & 5 & 3 & YES & YES & NO(2) & $1.33$ & $(4,2)$ & -- & 1458\\
$(69,19)$ & 9 & $(9,4)$ & 5 & 3 & YES & YES & NO(2) & $1.44$ & $(4,2)$ & NO & 1459\\
$(69,26)$ & 12 & $(29,11)$ & 7 & 1 & YES & YES & YES & $1.50$ & $(4,2)$ & NO & 1460\\
$(70,29)$ & 9 & $(17,5)$ & 6 & 1 & YES & YES & YES & $1.62$ & $(2,3)$ & -- & 1461\\
$(70,27)$ & 10 & $(20,3)$ & 8 & 10 & YES & YES & NO(2) & $1.29$ & $(6,1)$ & -- & 1462\\
$(71,26)$ & 9 & $(2,1)$ & 1 & 1 & YES & YES & NO(2) & $1.64$ & $(2,3)$ & -- & 1463\\
$(71,15)$ & 10 & $(3,1)$ & 2 & 1 & YES & YES & YES & $1.60$ & $(2,3)$ & NO & 1464\\
$(71,15)$ & 10 & $(3,1)$ & 2 & 1 & YES & YES & YES & $1.60$ & $(2,3)$ & -- & 1465\\
$(71,26)$ & 9 & $(4,1)$ & 3 & 1 & YES & YES & NO(2) & $1.12$ & $(6,1)$ & -- & 1466\\
$(71,13)$ & 12 & $(7,3)$ & 4 & 1 & YES & YES & YES & $1.14$ & $(4,2)$ & -- & 1467\\
$(71,17)$ & 11 & $(7,3)$ & 4 & 1 & YES & YES & YES & $1.50$ & $(2,3)$ & -- & 1468\\
$(71,15)$ & 10 & $(9,2)$ & 5 & 1 & YES & YES & YES & $1.50$ & $(2,3)$ & NO & 1469\\
$(71,27)$ & 9 & $(12,5)$ & 5 & 1 & YES & YES & YES & $1.43$ & $(4,2)$ & -- & 1470\\
$(71,16)$ & 10 & $(14,3)$ & 6 & 1 & YES & YES & NO(2) & $1.50$ & $(4,2)$ & NO & 1471\\
$(71,20)$ & 10 & $(15,4)$ & 6 & 1 & YES & YES & NO(2) & $1.29$ & $(8,0)$ & NO & 1472\\
$(71,19)$ & 10 & $(16,5)$ & 7 & 1 & YES & YES & NO(2) & $1.56$ & $(4,2)$ & NO & 1473\\
$(71,21)$ & 9 & $(17,7)$ & 6 & 1 & YES & YES & YES & $1.78$ & $(2,3)$ & -- & 1474\\
$(71,13)$ & 12 & $(19,3)$ & 8 & 1 & YES & YES & YES & $1.14$ & $(4,2)$ & NO & 1475\\
$(71,27)$ & 9 & $(45,17)$ & 9 & 1 & YES & YES & NO(2) & $1.44$ & $(4,2)$ & NO & 1476\\
$(72,19)$ & 10 & $(7,2)$ & 4 & 1 & YES & YES & NO(2) & $1.29$ & $(8,0)$ & -- & 1477\\
$(73,11)$ & 11 & $(2,1)$ & 1 & 1 & YES & YES & YES & $1.29$ & $(4,2)$ & NO & 1478\\
$(73,27)$ & 9 & $(5,2)$ & 3 & 1 & YES & YES & NO(2) & $1.44$ & $(4,2)$ & NO & 1479\\
$(73,27)$ & 9 & $(5,2)$ & 3 & 1 & YES & YES & NO(2) & $1.44$ & $(4,2)$ & -- & 1480\\
$(73,28)$ & 10 & $(5,1)$ & 4 & 1 & YES & YES & NO(2) & $1.29$ & $(8,0)$ & -- & 1481\\
$(73,11)$ & 11 & $(6,1)$ & 5 & 1 & YES & YES & YES & $1.14$ & $(6,1)$ & NO & 1482\\
$(73,11)$ & 11 & $(6,1)$ & 5 & 1 & YES & YES & YES & $1.14$ & $(6,1)$ & NO & 1483\\
$(73,11)$ & 11 & $(6,1)$ & 5 & 1 & YES & YES & YES & $1.14$ & $(6,1)$ & -- & 1484\\
$(73,19)$ & 11 & $(8,3)$ & 4 & 1 & YES & YES & YES & $1.57$ & $(2,3)$ & -- & 1485\\
$(73,14)$ & 11 & $(11,5)$ & 6 & 1 & YES & YES & YES & $1.62$ & $(2,3)$ & NO & 1486\\
$(73,11)$ & 11 & $(13,6)$ & 7 & 1 & YES & YES & YES & $1.29$ & $(4,2)$ & -- & 1487\\
$(73,31)$ & 10 & $(13,3)$ & 6 & 1 & YES & YES & NO(2) & $1.44$ & $(4,2)$ & -- & 1488\\
$(73,33)$ & 10 & $(13,3)$ & 6 & 1 & YES & YES & NO(2) & $1.60$ & $(2,3)$ & NO & 1489\\
$(73,11)$ & 11 & $(17,6)$ & 7 & 1 & YES & YES & YES & $1.29$ & $(4,2)$ & -- & 1490\\
$(73,19)$ & 11 & $(17,5)$ & 6 & 1 & YES & YES & YES & $1.57$ & $(2,3)$ & NO & 1491\\
$(73,11)$ & 11 & $(43,7)$ & 12 & 1 & YES & YES & YES & $1.29$ & $(4,2)$ & NO & 1492\\
$(73,11)$ & 11 & $(71,11)$ & 12 & 1 & YES & YES & YES & $1.29$ & $(4,2)$ & NO & 1493\\
$(74,13)$ & 11 & $(3,1)$ & 2 & 1 & YES & YES & NO(2) & $1.56$ & $(4,2)$ & NO & 1494\\
$(74,31)$ & 9 & $(17,5)$ & 6 & 1 & YES & YES & YES & $1.62$ & $(2,3)$ & -- & 1495\\
$(74,13)$ & 11 & $(31,6)$ & 10 & 1 & YES & YES & YES & $1.50$ & $(2,3)$ & NO & 1496\\
$(74,29)$ & 10 & $(33,13)$ & 9 & 1 & YES & YES & YES & $1.14$ & $(4,2)$ & 1831 & 1497\\
$(75,23)$ & 11 & $(6,1)$ & 5 & 3 & YES & YES & NO(2) & $1.56$ & $(2,3)$ & -- & 1498\\
$(75,29)$ & 9 & $(13,5)$ & 5 & 1 & YES & YES & YES & $1.80$ & $(2,3)$ & -- & 1499\\
$(75,17)$ & 10 & $(25,7)$ & 7 & 25 & YES & YES & YES & $1.43$ & $(4,2)$ & NO & 1500\\
$(75,17)$ & 10 & $(51,11)$ & 9 & 3 & YES & YES & NO(2) & $1.14$ & $(6,1)$ & NO & 1501\\
$(77,34)$ & 10 & $(3,1)$ & 2 & 1 & YES & YES & YES & $1.25$ & $(2,3)$ & -- & 1502\\
$(77,34)$ & 10 & $(5,2)$ & 3 & 1 & YES & YES & YES & $1.38$ & $(2,3)$ & NO & 1503\\
$(77,34)$ & 10 & $(7,2)$ & 4 & 7 & YES & YES & NO(2) & $1.56$ & $(4,2)$ & -- & 1504\\
$(77,34)$ & 10 & $(41,18)$ & 8 & 1 & YES & YES & NO(2) & $1.44$ & $(4,2)$ & NO & 1505\\
$(79,28)$ & 10 & $(4,1)$ & 3 & 1 & YES & YES & YES & $1.44$ & $(2,3)$ & -- & 1506\\
$(79,28)$ & 10 & $(4,1)$ & 3 & 1 & YES & YES & NO(2) & $1.44$ & $(6,1)$ & NO & 1507\\
$(79,17)$ & 11 & $(5,2)$ & 3 & 1 & YES & YES & NO(2) & $1.12$ & $(6,1)$ & -- & 1508\\
$(79,30)$ & 9 & $(5,2)$ & 3 & 1 & YES & YES & NO(2) & $1.60$ & $(2,3)$ & -- & 1509\\
$(79,33)$ & 11 & $(6,1)$ & 5 & 1 & YES & YES & YES & $1.56$ & $(2,3)$ & NO & 1510\\
$(79,31)$ & 10 & $(7,3)$ & 4 & 1 & YES & YES & YES & $1.50$ & $(2,3)$ & NO & 1511\\
$(79,28)$ & 10 & $(8,3)$ & 4 & 1 & YES & YES & YES & $1.56$ & $(2,3)$ & 1313 & 1512\\
$(79,30)$ & 9 & $(13,4)$ & 6 & 1 & YES & YES & YES & $1.75$ & $(2,3)$ & -- & 1513\\
$(79,30)$ & 9 & $(13,4)$ & 6 & 1 & YES & YES & YES & $1.75$ & $(2,3)$ & NO & 1514\\
$(79,23)$ & 10 & $(14,3)$ & 6 & 1 & YES & YES & NO(2) & $1.25$ & $(8,0)$ & -- & 1515\\
$(79,23)$ & 10 & $(17,5)$ & 6 & 1 & YES & YES & YES & $1.62$ & $(2,3)$ & -- & 1516\\
$(79,30)$ & 9 & $(34,13)$ & 7 & 1 & YES & YES & NO(2) & $1.44$ & $(4,2)$ & NO & 1517\\
$(79,30)$ & 9 & $(41,16)$ & 8 & 1 & YES & YES & YES & $1.78$ & $(2,3)$ & 1854 & 1518\\
$(79,33)$ & 11 & $(43,18)$ & 8 & 1 & YES & YES & YES & $1.56$ & $(2,3)$ & NO & 1519\\
$(79,18)$ & 10 & $(55,13)$ & 10 & 1 & YES & YES & YES & $1.43$ & $(4,2)$ & NO & 1520\\
$(79,14)$ & 11 & $(63,11)$ & 10 & 1 & YES & YES & YES & $1.44$ & $(2,3)$ & NO & 1521\\
$(79,33)$ & 11 & $(67,28)$ & 10 & 1 & YES & YES & YES & $1.56$ & $(2,3)$ & NO & 1522\\
$(79,33)$ & 11 & $(79,33)$ & 11 & 79 & YES & YES & YES & $1.56$ & $(2,3)$ & NO & 1523\\
$(80,19)$ & 11 & $(5,1)$ & 4 & 5 & YES & YES & NO(2) & $1.50$ & $(4,2)$ & NO & 1524\\
$(80,19)$ & 11 & $(5,1)$ & 4 & 5 & YES & YES & NO(2) & $1.50$ & $(4,2)$ & -- & 1525\\
$(80,31)$ & 9 & $(5,2)$ & 3 & 5 & YES & YES & YES & $1.38$ & $(2,3)$ & -- & 1526\\
$(80,33)$ & 10 & $(7,2)$ & 4 & 1 & YES & YES & NO(2) & $1.44$ & $(4,2)$ & -- & 1527\\
$(80,19)$ & 11 & $(13,3)$ & 6 & 1 & YES & YES & NO(2) & $1.50$ & $(4,2)$ & NO & 1528\\
$(80,19)$ & 11 & $(17,4)$ & 7 & 1 & YES & YES & NO(2) & $1.44$ & $(4,2)$ & -- & 1529\\
$(81,35)$ & 11 & $(4,1)$ & 3 & 1 & YES & YES & YES & $1.38$ & $(2,3)$ & -- & 1530\\
$(81,31)$ & 9 & $(9,4)$ & 5 & 9 & YES & YES & NO(2) & $1.44$ & $(4,2)$ & NO & 1531\\
$(81,32)$ & 12 & $(33,13)$ & 9 & 3 & YES & YES & YES & $1.50$ & $(2,3)$ & NO & 1532\\
$(81,35)$ & 11 & $(44,19)$ & 10 & 1 & YES & YES & YES & $1.50$ & $(2,3)$ & NO & 1533\\
$(82,31)$ & 10 & $(3,1)$ & 2 & 1 & YES & YES & YES & $1.38$ & $(2,3)$ & -- & 1534\\
$(82,31)$ & 10 & $(5,2)$ & 3 & 1 & YES & YES & YES & $1.38$ & $(2,3)$ & -- & 1535\\
$(82,31)$ & 10 & $(7,3)$ & 4 & 1 & YES & YES & YES & $1.38$ & $(2,3)$ & NO & 1536\\
$(82,23)$ & 10 & $(12,5)$ & 5 & 2 & YES & YES & YES & $1.62$ & $(2,3)$ & -- & 1537\\
$(82,31)$ & 10 & $(13,5)$ & 5 & 1 & YES & YES & YES & $1.38$ & $(2,3)$ & NO & 1538\\
$(82,31)$ & 10 & $(82,31)$ & 10 & 82 & YES & YES & YES & $1.38$ & $(2,3)$ & NO & 1539\\
$(83,18)$ & 10 & $(2,1)$ & 1 & 1 & YES & YES & YES & $1.50$ & $(2,3)$ & -- & 1540\\
$(83,24)$ & 11 & $(2,1)$ & 1 & 1 & YES & YES & YES & $1.50$ & $(4,2)$ & -- & 1541\\
$(83,18)$ & 10 & $(3,1)$ & 2 & 1 & YES & YES & NO(2) & $1.00$ & $(8,0)$ & -- & 1542\\
$(83,18)$ & 10 & $(3,1)$ & 2 & 1 & YES & YES & NO(2) & $1.14$ & $(8,0)$ & NO & 1543\\
$(83,24)$ & 11 & $(3,1)$ & 2 & 1 & YES & YES & NO(2) & $1.67$ & $(8,0)$ & -- & 1544\\
$(83,36)$ & 10 & $(4,1)$ & 3 & 1 & YES & YES & YES & $1.56$ & $(2,3)$ & NO & 1545\\
$(83,36)$ & 10 & $(4,1)$ & 3 & 1 & YES & YES & YES & $1.56$ & $(2,3)$ & -- & 1546\\
$(83,18)$ & 10 & $(5,2)$ & 3 & 1 & YES & YES & NO(2) & $1.00$ & $(8,0)$ & NO & 1547\\
$(83,18)$ & 10 & $(5,2)$ & 3 & 1 & YES & YES & NO(2) & $1.00$ & $(8,0)$ & -- & 1548\\
$(83,29)$ & 12 & $(5,1)$ & 4 & 1 & YES & YES & NO(2) & $1.60$ & $(2,3)$ & -- & 1549\\
$(83,24)$ & 11 & $(10,3)$ & 5 & 1 & YES & YES & NO(2) & $1.67$ & $(8,0)$ & NO & 1550\\
$(83,13)$ & 11 & $(11,5)$ & 6 & 1 & YES & YES & NO(2) & $1.50$ & $(4,2)$ & NO & 1551\\
$(83,13)$ & 11 & $(11,5)$ & 6 & 1 & YES & YES & NO(2) & $1.50$ & $(4,2)$ & -- & 1552\\
$(83,29)$ & 12 & $(11,4)$ & 5 & 1 & YES & YES & NO(2) & $1.70$ & $(2,3)$ & NO & 1553\\
$(83,18)$ & 10 & $(13,3)$ & 6 & 1 & YES & YES & NO(2) & $1.00$ & $(8,0)$ & NO & 1554\\
$(83,19)$ & 10 & $(17,7)$ & 6 & 1 & YES & YES & YES & $1.62$ & $(2,3)$ & -- & 1555\\
$(83,18)$ & 10 & $(52,11)$ & 9 & 1 & YES & YES & NO(2) & $1.44$ & $(4,2)$ & NO & 1556\\
$(83,18)$ & 10 & $(83,18)$ & 10 & 83 & YES & YES & NO(2) & $1.00$ & $(8,0)$ & NO & 1557\\
$(84,25)$ & 10 & $(3,1)$ & 2 & 3 & YES & YES & YES & $1.56$ & $(2,3)$ & NO & 1558\\
$(84,25)$ & 10 & $(3,1)$ & 2 & 3 & YES & YES & YES & $1.56$ & $(2,3)$ & -- & 1559\\
$(84,13)$ & 13 & $(7,2)$ & 4 & 7 & YES & YES & YES & $1.29$ & $(4,2)$ & -- & 1560\\
$(84,13)$ & 13 & $(7,2)$ & 4 & 7 & YES & YES & YES & $1.43$ & $(4,2)$ & NO & 1561\\
$(84,13)$ & 13 & $(7,3)$ & 4 & 7 & YES & YES & YES & $1.29$ & $(4,2)$ & NO & 1562\\
$(84,13)$ & 13 & $(7,3)$ & 4 & 7 & YES & YES & YES & $1.29$ & $(4,2)$ & -- & 1563\\
$(84,37)$ & 10 & $(7,2)$ & 4 & 7 & YES & YES & YES & $1.50$ & $(2,3)$ & -- & 1564\\
$(84,25)$ & 10 & $(23,7)$ & 7 & 1 & YES & YES & YES & $1.56$ & $(2,3)$ & NO & 1565\\
$(84,25)$ & 10 & $(37,11)$ & 8 & 1 & YES & YES & NO(2) & $1.38$ & $(6,1)$ & NO & 1566\\
$(85,24)$ & 11 & $(2,1)$ & 1 & 1 & YES & YES & YES & $1.62$ & $(2,3)$ & NO & 1567\\
$(85,24)$ & 11 & $(2,1)$ & 1 & 1 & YES & YES & YES & $1.70$ & $(2,3)$ & -- & 1568\\
$(85,24)$ & 11 & $(5,1)$ & 4 & 5 & YES & YES & YES & $1.29$ & $(6,1)$ & NO & 1569\\
$(85,24)$ & 11 & $(5,1)$ & 4 & 5 & YES & YES & YES & $1.29$ & $(6,1)$ & -- & 1570\\
$(85,24)$ & 11 & $(7,2)$ & 4 & 1 & YES & YES & YES & $1.43$ & $(6,1)$ & 1375 & 1571\\
$(85,26)$ & 10 & $(7,3)$ & 4 & 1 & YES & YES & NO(2) & $1.60$ & $(2,3)$ & -- & 1572\\
$(85,33)$ & 10 & $(7,3)$ & 4 & 1 & YES & YES & NO(2) & $1.50$ & $(8,0)$ & -- & 1573\\
$(85,38)$ & 11 & $(7,2)$ & 4 & 1 & YES & YES & YES & $1.50$ & $(2,3)$ & -- & 1574\\
$(85,24)$ & 11 & $(39,11)$ & 9 & 1 & YES & YES & NO(2) & $1.56$ & $(6,1)$ & NO & 1575\\
$(86,27)$ & 11 & $(2,1)$ & 1 & 2 & YES & YES & YES & $1.50$ & $(2,3)$ & -- & 1576\\
$(86,27)$ & 11 & $(3,1)$ & 2 & 1 & YES & YES & YES & $1.50$ & $(2,3)$ & NO & 1577\\
$(86,27)$ & 11 & $(3,1)$ & 2 & 1 & YES & YES & YES & $1.50$ & $(2,3)$ & -- & 1578\\
$(86,35)$ & 11 & $(5,2)$ & 3 & 1 & YES & YES & YES & $1.67$ & $(2,3)$ & -- & 1579\\
$(87,23)$ & 10 & $(4,1)$ & 3 & 1 & YES & YES & NO(2) & $1.38$ & $(6,1)$ & NO & 1580\\
$(87,23)$ & 10 & $(4,1)$ & 3 & 1 & YES & YES & NO(2) & $1.38$ & $(6,1)$ & -- & 1581\\
$(87,37)$ & 11 & $(5,2)$ & 3 & 1 & YES & YES & NO(2) & $1.56$ & $(4,2)$ & -- & 1582\\
$(87,23)$ & 10 & $(7,2)$ & 4 & 1 & YES & YES & NO(2) & $1.38$ & $(6,1)$ & 1365 & 1583\\
$(87,31)$ & 12 & $(7,1)$ & 6 & 1 & YES & YES & NO(2) & $1.73$ & $(2,3)$ & 2071 & 1584\\
$(87,37)$ & 11 & $(7,2)$ & 4 & 1 & YES & YES & NO(2) & $1.56$ & $(4,2)$ & NO & 1585\\
$(87,23)$ & 10 & $(9,4)$ & 5 & 3 & YES & YES & YES & $1.62$ & $(2,3)$ & -- & 1586\\
$(87,20)$ & 12 & $(10,3)$ & 5 & 1 & YES & YES & NO(2) & $1.60$ & $(2,3)$ & NO & 1587\\
$(87,19)$ & 10 & $(11,4)$ & 5 & 1 & YES & YES & NO(2) & $1.38$ & $(8,0)$ & NO & 1588\\
$(87,23)$ & 10 & $(11,3)$ & 5 & 1 & YES & YES & NO(2) & $1.25$ & $(6,1)$ & NO & 1589\\
$(87,19)$ & 10 & $(13,4)$ & 6 & 1 & YES & YES & NO(2) & $1.38$ & $(8,0)$ & NO & 1590\\
$(87,37)$ & 11 & $(13,2)$ & 7 & 1 & YES & YES & NO(2) & $1.60$ & $(2,3)$ & NO & 1591\\
$(87,32)$ & 10 & $(17,6)$ & 7 & 1 & YES & YES & YES & $1.50$ & $(2,3)$ & NO & 1592\\
$(87,37)$ & 11 & $(17,7)$ & 6 & 1 & YES & YES & NO(2) & $1.56$ & $(4,2)$ & NO & 1593\\
$(87,23)$ & 10 & $(53,14)$ & 9 & 1 & YES & YES & NO(2) & $1.38$ & $(6,1)$ & NO & 1594\\
$(87,31)$ & 12 & $(59,21)$ & 10 & 1 & YES & YES & NO(2) & $1.73$ & $(2,3)$ & 1861 & 1595\\
$(87,37)$ & 11 & $(59,25)$ & 9 & 1 & YES & YES & NO(2) & $1.60$ & $(2,3)$ & NO & 1596\\
$(87,37)$ & 11 & $(73,31)$ & 10 & 1 & YES & YES & NO(2) & $1.60$ & $(2,3)$ & NO & 1597\\
$(89,28)$ & 11 & $(3,1)$ & 2 & 1 & YES & YES & YES & $1.50$ & $(2,3)$ & NO & 1598\\
$(89,28)$ & 11 & $(3,1)$ & 2 & 1 & YES & YES & YES & $1.50$ & $(2,3)$ & -- & 1599\\
$(89,35)$ & 11 & $(3,1)$ & 2 & 1 & YES & YES & YES & $1.29$ & $(4,2)$ & -- & 1600\\
$(89,27)$ & 10 & $(5,2)$ & 3 & 1 & YES & YES & NO(2) & $1.44$ & $(4,2)$ & NO & 1601\\
$(89,27)$ & 10 & $(5,2)$ & 3 & 1 & YES & YES & NO(2) & $1.44$ & $(4,2)$ & -- & 1602\\
$(89,34)$ & 9 & $(5,2)$ & 3 & 1 & YES & YES & YES & $1.38$ & $(2,3)$ & -- & 1603\\
$(89,26)$ & 10 & $(7,3)$ & 4 & 1 & YES & YES & YES & $1.50$ & $(2,3)$ & NO & 1604\\
$(89,26)$ & 10 & $(7,3)$ & 4 & 1 & YES & YES & NO(2) & $1.25$ & $(8,0)$ & -- & 1605\\
$(89,20)$ & 11 & $(11,4)$ & 5 & 1 & YES & YES & NO(2) & $1.44$ & $(4,2)$ & -- & 1606\\
$(89,34)$ & 9 & $(11,4)$ & 5 & 1 & YES & YES & YES & $1.38$ & $(2,3)$ & NO & 1607\\
$(89,26)$ & 10 & $(12,5)$ & 5 & 1 & YES & YES & YES & $1.78$ & $(2,3)$ & -- & 1608\\
$(89,20)$ & 11 & $(15,4)$ & 6 & 1 & YES & YES & YES & $1.50$ & $(2,3)$ & NO & 1609\\
$(89,34)$ & 9 & $(28,11)$ & 8 & 1 & YES & YES & YES & $1.43$ & $(4,2)$ & NO & 1610\\
$(90,19)$ & 11 & $(3,1)$ & 2 & 3 & YES & YES & YES & $1.44$ & $(2,3)$ & NO & 1611\\
$(90,19)$ & 11 & $(3,1)$ & 2 & 3 & YES & YES & YES & $1.44$ & $(2,3)$ & -- & 1612\\
$(90,19)$ & 11 & $(24,5)$ & 8 & 6 & YES & YES & YES & $1.44$ & $(2,3)$ & NO & 1613\\
$(91,25)$ & 10 & $(2,1)$ & 1 & 1 & YES & YES & NO(2) & $1.38$ & $(6,1)$ & -- & 1614\\
$(91,25)$ & 10 & $(3,1)$ & 2 & 1 & YES & YES & NO(2) & $1.38$ & $(6,1)$ & -- & 1615\\
$(91,41)$ & 11 & $(3,1)$ & 2 & 1 & YES & YES & YES & $1.60$ & $(2,3)$ & NO & 1616\\
$(91,24)$ & 11 & $(4,1)$ & 3 & 1 & YES & YES & YES & $1.50$ & $(2,3)$ & -- & 1617\\
$(91,25)$ & 10 & $(4,1)$ & 3 & 1 & YES & YES & YES & $1.56$ & $(2,3)$ & NO & 1618\\
$(91,25)$ & 10 & $(4,1)$ & 3 & 1 & YES & YES & YES & $1.56$ & $(2,3)$ & -- & 1619\\
$(91,25)$ & 10 & $(4,1)$ & 3 & 1 & YES & YES & YES & $1.56$ & $(2,3)$ & NO & 1620\\
$(91,24)$ & 11 & $(5,1)$ & 4 & 1 & YES & YES & YES & $1.25$ & $(2,3)$ & -- & 1621\\
$(91,24)$ & 11 & $(5,1)$ & 4 & 1 & YES & YES & YES & $1.38$ & $(2,3)$ & NO & 1622\\
$(91,24)$ & 11 & $(7,2)$ & 4 & 7 & YES & YES & NO(2) & $1.38$ & $(6,1)$ & -- & 1623\\
$(91,24)$ & 11 & $(9,2)$ & 5 & 1 & YES & YES & YES & $1.50$ & $(2,3)$ & -- & 1624\\
$(91,27)$ & 10 & $(9,4)$ & 5 & 1 & YES & YES & NO(2) & $1.14$ & $(6,1)$ & -- & 1625\\
$(91,24)$ & 11 & $(13,4)$ & 6 & 13 & YES & YES & YES & $1.50$ & $(2,3)$ & NO & 1626\\
$(91,24)$ & 11 & $(14,3)$ & 6 & 7 & YES & YES & NO(2) & $1.60$ & $(2,3)$ & NO & 1627\\
$(91,25)$ & 10 & $(40,11)$ & 8 & 1 & YES & YES & NO(2) & $1.38$ & $(6,1)$ & NO & 1628\\
$(91,25)$ & 10 & $(51,14)$ & 9 & 1 & YES & YES & NO(2) & $1.38$ & $(6,1)$ & NO & 1629\\
$(91,24)$ & 11 & $(72,19)$ & 10 & 1 & YES & YES & YES & $1.25$ & $(2,3)$ & NO & 1630\\
$(91,24)$ & 11 & $(87,23)$ & 10 & 1 & YES & YES & NO(2) & $1.44$ & $(4,2)$ & 1988 & 1631\\
$(92,33)$ & 10 & $(3,1)$ & 2 & 1 & YES & YES & NO(2) & $1.38$ & $(6,1)$ & -- & 1632\\
$(92,39)$ & 10 & $(5,2)$ & 3 & 1 & YES & YES & NO(2) & $1.44$ & $(4,2)$ & -- & 1633\\
$(92,33)$ & 10 & $(36,13)$ & 8 & 4 & YES & YES & NO(2) & $1.60$ & $(2,3)$ & NO & 1634\\
$(92,33)$ & 10 & $(64,23)$ & 9 & 4 & YES & YES & NO(2) & $1.60$ & $(2,3)$ & NO & 1635\\
$(93,34)$ & 10 & $(3,1)$ & 2 & 3 & YES & YES & NO(2) & $1.56$ & $(2,3)$ & -- & 1636\\
$(93,26)$ & 10 & $(4,1)$ & 3 & 1 & YES & YES & YES & $1.38$ & $(2,3)$ & -- & 1637\\
$(93,26)$ & 10 & $(4,1)$ & 3 & 1 & YES & YES & YES & $1.50$ & $(2,3)$ & NO & 1638\\
$(93,26)$ & 10 & $(4,1)$ & 3 & 1 & YES & YES & YES & $1.50$ & $(2,3)$ & NO & 1639\\
$(93,29)$ & 12 & $(7,2)$ & 4 & 1 & YES & YES & YES & $1.50$ & $(2,3)$ & NO & 1640\\
$(93,22)$ & 11 & $(9,4)$ & 5 & 3 & YES & YES & YES & $1.50$ & $(2,3)$ & NO & 1641\\
$(93,22)$ & 11 & $(9,4)$ & 5 & 3 & YES & YES & YES & $1.50$ & $(2,3)$ & -- & 1642\\
$(93,26)$ & 10 & $(9,4)$ & 5 & 3 & YES & YES & NO(2) & $1.14$ & $(6,1)$ & NO & 1643\\
$(93,29)$ & 12 & $(10,3)$ & 5 & 1 & YES & YES & YES & $1.50$ & $(2,3)$ & NO & 1644\\
$(93,34)$ & 10 & $(10,3)$ & 5 & 1 & YES & YES & YES & $1.62$ & $(2,3)$ & -- & 1645\\
$(93,34)$ & 10 & $(52,19)$ & 9 & 1 & YES & YES & NO(2) & $1.56$ & $(2,3)$ & NO & 1646\\
$(93,34)$ & 10 & $(79,29)$ & 9 & 1 & YES & YES & YES & $1.62$ & $(2,3)$ & NO & 1647\\
$(94,43)$ & 11 & $(3,1)$ & 2 & 1 & YES & YES & YES & $1.50$ & $(4,2)$ & NO & 1648\\
$(94,43)$ & 11 & $(3,1)$ & 2 & 1 & YES & YES & YES & $1.50$ & $(4,2)$ & -- & 1649\\
$(95,44)$ & 12 & $(2,1)$ & 1 & 1 & YES & YES & NO(2) & $1.60$ & $(2,3)$ & -- & 1650\\
$(95,42)$ & 11 & $(3,1)$ & 2 & 1 & YES & YES & YES & $1.50$ & $(2,3)$ & -- & 1651\\
$(95,42)$ & 11 & $(4,1)$ & 3 & 1 & YES & YES & YES & $1.50$ & $(2,3)$ & NO & 1652\\
$(95,43)$ & 11 & $(4,1)$ & 3 & 1 & YES & YES & YES & $1.56$ & $(2,3)$ & NO & 1653\\
$(95,42)$ & 11 & $(5,2)$ & 3 & 5 & YES & YES & YES & $1.50$ & $(2,3)$ & NO & 1654\\
$(95,42)$ & 11 & $(5,2)$ & 3 & 5 & YES & YES & NO(2) & $1.56$ & $(4,2)$ & -- & 1655\\
$(95,36)$ & 10 & $(11,4)$ & 5 & 1 & YES & YES & YES & $1.62$ & $(2,3)$ & NO & 1656\\
$(95,36)$ & 10 & $(18,7)$ & 6 & 1 & YES & YES & NO(2) & $1.14$ & $(6,1)$ & NO & 1657\\
$(95,43)$ & 11 & $(20,9)$ & 7 & 5 & YES & YES & YES & $1.14$ & $(4,2)$ & 1377 & 1658\\
$(95,43)$ & 11 & $(42,19)$ & 9 & 1 & YES & YES & YES & $1.14$ & $(4,2)$ & NO & 1659\\
$(95,42)$ & 11 & $(52,23)$ & 10 & 1 & YES & YES & YES & $1.50$ & $(2,3)$ & NO & 1660\\
$(95,43)$ & 11 & $(95,43)$ & 11 & 95 & YES & YES & YES & $1.56$ & $(2,3)$ & NO & 1661\\
$(96,17)$ & 12 & $(5,2)$ & 3 & 1 & YES & YES & NO(2) & $1.25$ & $(6,1)$ & NO & 1662\\
$(96,17)$ & 12 & $(5,2)$ & 3 & 1 & YES & YES & NO(2) & $1.25$ & $(6,1)$ & -- & 1663\\
$(96,17)$ & 12 & $(5,2)$ & 3 & 1 & YES & YES & NO(2) & $1.38$ & $(6,1)$ & NO & 1664\\
$(96,17)$ & 12 & $(13,2)$ & 7 & 1 & YES & YES & YES & $1.38$ & $(2,3)$ & NO & 1665\\
$(97,18)$ & 11 & $(2,1)$ & 1 & 1 & YES & YES & NO(2) & $1.60$ & $(2,3)$ & -- & 1666\\
$(97,21)$ & 10 & $(3,1)$ & 2 & 1 & YES & YES & NO(2) & $1.00$ & $(8,0)$ & -- & 1667\\
$(97,21)$ & 10 & $(3,1)$ & 2 & 1 & YES & YES & NO(2) & $1.14$ & $(8,0)$ & NO & 1668\\
$(97,26)$ & 10 & $(5,2)$ & 3 & 1 & YES & YES & NO(2) & $1.50$ & $(4,2)$ & -- & 1669\\
$(97,28)$ & 12 & $(7,2)$ & 4 & 1 & YES & YES & NO(2) & $1.29$ & $(8,0)$ & -- & 1670\\
$(97,21)$ & 10 & $(14,3)$ & 6 & 1 & YES & YES & NO(2) & $1.00$ & $(8,0)$ & NO & 1671\\
$(97,30)$ & 11 & $(36,11)$ & 8 & 1 & YES & YES & YES & $1.50$ & $(2,3)$ & NO & 1672\\
$(97,28)$ & 12 & $(69,20)$ & 10 & 1 & YES & YES & NO(2) & $1.29$ & $(8,0)$ & NO & 1673\\
$(98,15)$ & 14 & $(2,1)$ & 1 & 2 & YES & YES & YES & $1.33$ & $(2,3)$ & -- & 1674\\
$(98,15)$ & 14 & $(2,1)$ & 1 & 2 & YES & YES & YES & $1.44$ & $(2,3)$ & NO & 1675\\
$(98,37)$ & 11 & $(7,2)$ & 4 & 7 & YES & YES & NO(2) & $1.56$ & $(4,2)$ & NO & 1676\\
$(98,43)$ & 10 & $(7,2)$ & 4 & 7 & YES & YES & YES & $1.62$ & $(2,3)$ & -- & 1677\\
$(98,43)$ & 10 & $(8,3)$ & 4 & 2 & YES & YES & YES & $1.62$ & $(2,3)$ & NO & 1678\\
$(98,27)$ & 10 & $(9,4)$ & 5 & 1 & YES & YES & YES & $1.43$ & $(4,2)$ & -- & 1679\\
$(98,31)$ & 13 & $(16,5)$ & 7 & 2 & YES & YES & YES & $1.67$ & $(2,3)$ & NO & 1680\\
$(98,27)$ & 10 & $(39,11)$ & 9 & 1 & YES & YES & YES & $1.43$ & $(4,2)$ & NO & 1681\\
$(98,37)$ & 11 & $(66,25)$ & 9 & 2 & YES & YES & NO(2) & $1.60$ & $(2,3)$ & NO & 1682\\
$(98,43)$ & 10 & $(66,29)$ & 9 & 2 & YES & YES & YES & $1.62$ & $(2,3)$ & NO & 1683\\
$(99,38)$ & 12 & $(5,1)$ & 4 & 1 & YES & YES & NO(2) & $1.29$ & $(8,0)$ & -- & 1684\\
$(99,38)$ & 12 & $(7,1)$ & 6 & 1 & YES & YES & NO(2) & $1.43$ & $(8,0)$ & NO & 1685\\
$(99,38)$ & 12 & $(47,18)$ & 8 & 1 & YES & YES & NO(2) & $1.43$ & $(8,0)$ & NO & 1686\\
$(99,38)$ & 12 & $(73,28)$ & 10 & 1 & YES & YES & NO(2) & $1.29$ & $(8,0)$ & 1968 & 1687\\
$(100,29)$ & 11 & $(4,1)$ & 3 & 4 & YES & YES & YES & $1.29$ & $(4,2)$ & NO & 1688\\
$(100,29)$ & 11 & $(4,1)$ & 3 & 4 & YES & YES & YES & $1.29$ & $(4,2)$ & -- & 1689\\
$(100,29)$ & 11 & $(4,1)$ & 3 & 4 & YES & YES & YES & $1.29$ & $(4,2)$ & NO & 1690\\
$(100,37)$ & 10 & $(7,3)$ & 4 & 1 & YES & YES & YES & $1.50$ & $(2,3)$ & NO & 1691\\
$(100,37)$ & 10 & $(13,5)$ & 5 & 1 & YES & YES & YES & $1.50$ & $(2,3)$ & NO & 1692\\
$(100,27)$ & 10 & $(25,7)$ & 7 & 25 & YES & YES & NO(2) & $1.33$ & $(4,2)$ & NO & 1693\\
$(100,29)$ & 11 & $(52,15)$ & 11 & 4 & YES & YES & NO(2) & $1.14$ & $(8,0)$ & NO & 1694\\
$(100,41)$ & 10 & $(83,34)$ & 10 & 1 & YES & YES & NO(2) & $1.60$ & $(2,3)$ & NO & 1695\\
$(101,24)$ & 12 & $(3,1)$ & 2 & 1 & YES & YES & YES & $1.25$ & $(2,3)$ & NO & 1696\\
$(101,16)$ & 13 & $(7,2)$ & 4 & 1 & YES & YES & NO(2) & $1.56$ & $(2,3)$ & -- & 1697\\
$(101,41)$ & 12 & $(12,5)$ & 5 & 1 & YES & YES & YES & $1.56$ & $(2,3)$ & NO & 1698\\
$(101,30)$ & 10 & $(23,7)$ & 7 & 1 & YES & YES & NO(2) & $1.38$ & $(6,1)$ & NO & 1699\\
$(103,32)$ & 11 & $(3,1)$ & 2 & 1 & YES & YES & NO(2) & $1.29$ & $(8,0)$ & NO & 1700\\
$(103,32)$ & 11 & $(3,1)$ & 2 & 1 & YES & YES & NO(2) & $1.29$ & $(8,0)$ & -- & 1701\\
$(103,47)$ & 12 & $(4,1)$ & 3 & 1 & YES & YES & YES & $1.56$ & $(2,3)$ & NO & 1702\\
$(103,40)$ & 11 & $(5,2)$ & 3 & 1 & YES & YES & NO(2) & $1.43$ & $(6,1)$ & -- & 1703\\
$(103,37)$ & 10 & $(7,3)$ & 4 & 1 & YES & YES & NO(2) & $1.44$ & $(4,2)$ & NO & 1704\\
$(103,39)$ & 10 & $(7,2)$ & 4 & 1 & YES & YES & NO(2) & $1.60$ & $(2,3)$ & NO & 1705\\
$(103,29)$ & 11 & $(17,5)$ & 6 & 1 & YES & YES & YES & $1.57$ & $(2,3)$ & NO & 1706\\
$(103,39)$ & 10 & $(34,13)$ & 7 & 1 & YES & YES & NO(2) & $1.44$ & $(4,2)$ & 2010 & 1707\\
$(103,40)$ & 11 & $(44,17)$ & 8 & 1 & YES & YES & NO(2) & $1.29$ & $(6,1)$ & NO & 1708\\
$(103,47)$ & 12 & $(103,47)$ & 12 & 103 & YES & YES & YES & $1.56$ & $(2,3)$ & NO & 1709\\
$(104,27)$ & 12 & $(3,1)$ & 2 & 1 & YES & YES & NO(2) & $1.62$ & $(4,2)$ & -- & 1710\\
$(104,41)$ & 12 & $(3,1)$ & 2 & 1 & YES & YES & YES & $1.43$ & $(2,3)$ & -- & 1711\\
$(104,47)$ & 11 & $(3,1)$ & 2 & 1 & YES & YES & NO(2) & $1.38$ & $(10,-1)$ & -- & 1712\\
$(104,45)$ & 11 & $(5,2)$ & 3 & 1 & YES & YES & YES & $1.62$ & $(2,3)$ & -- & 1713\\
$(104,31)$ & 11 & $(7,2)$ & 4 & 1 & YES & YES & NO(2) & $1.38$ & $(10,-1)$ & NO & 1714\\
$(104,45)$ & 11 & $(11,5)$ & 6 & 1 & YES & YES & YES & $1.62$ & $(2,3)$ & NO & 1715\\
$(104,47)$ & 11 & $(11,5)$ & 6 & 1 & YES & YES & YES & $1.29$ & $(4,2)$ & NO & 1716\\
$(104,47)$ & 11 & $(42,19)$ & 9 & 2 & YES & YES & YES & $1.14$ & $(4,2)$ & 1791 & 1717\\
$(105,41)$ & 10 & $(3,1)$ & 2 & 3 & YES & YES & YES & $1.50$ & $(2,3)$ & NO & 1718\\
$(105,38)$ & 11 & $(5,2)$ & 3 & 5 & YES & YES & NO(2) & $1.29$ & $(6,1)$ & -- & 1719\\
$(105,46)$ & 12 & $(5,1)$ & 4 & 5 & YES & YES & YES & $1.62$ & $(2,3)$ & -- & 1720\\
$(105,46)$ & 12 & $(5,1)$ & 4 & 5 & YES & YES & NO(2) & $1.56$ & $(4,2)$ & NO & 1721\\
$(105,31)$ & 10 & $(13,4)$ & 6 & 1 & YES & YES & YES & $1.62$ & $(2,3)$ & -- & 1722\\
$(105,41)$ & 10 & $(28,11)$ & 8 & 7 & YES & YES & NO(2) & $1.60$ & $(2,3)$ & NO & 1723\\
$(105,46)$ & 12 & $(73,32)$ & 10 & 1 & YES & YES & YES & $1.62$ & $(2,3)$ & 1982 & 1724\\
$(106,37)$ & 12 & $(2,1)$ & 1 & 2 & YES & YES & NO(2) & $1.70$ & $(2,3)$ & NO & 1725\\
$(106,45)$ & 11 & $(5,2)$ & 3 & 1 & YES & YES & YES & $1.50$ & $(2,3)$ & -- & 1726\\
$(106,41)$ & 10 & $(10,3)$ & 5 & 2 & YES & YES & YES & $1.78$ & $(2,3)$ & -- & 1727\\
$(106,41)$ & 10 & $(10,3)$ & 5 & 2 & YES & YES & YES & $1.78$ & $(2,3)$ & NO & 1728\\
$(106,41)$ & 10 & $(11,3)$ & 5 & 1 & YES & YES & YES & $1.78$ & $(2,3)$ & -- & 1729\\
$(106,37)$ & 12 & $(23,8)$ & 9 & 1 & YES & YES & YES & $1.62$ & $(2,3)$ & NO & 1730\\
$(107,25)$ & 11 & $(4,1)$ & 3 & 1 & YES & YES & YES & $1.50$ & $(4,2)$ & NO & 1731\\
$(107,25)$ & 11 & $(4,1)$ & 3 & 1 & YES & YES & YES & $1.50$ & $(4,2)$ & -- & 1732\\
$(107,47)$ & 10 & $(5,2)$ & 3 & 1 & YES & YES & NO(2) & $1.50$ & $(2,3)$ & -- & 1733\\
$(107,41)$ & 10 & $(8,3)$ & 4 & 1 & YES & YES & YES & $1.62$ & $(2,3)$ & -- & 1734\\
$(107,41)$ & 10 & $(11,3)$ & 5 & 1 & YES & YES & YES & $1.50$ & $(2,3)$ & -- & 1735\\
$(107,20)$ & 13 & $(13,3)$ & 6 & 1 & YES & YES & NO(2) & $1.44$ & $(4,2)$ & NO & 1736\\
$(107,47)$ & 10 & $(23,10)$ & 7 & 1 & YES & YES & YES & $1.50$ & $(2,3)$ & NO & 1737\\
$(107,41)$ & 10 & $(50,19)$ & 8 & 1 & YES & YES & YES & $1.62$ & $(2,3)$ & NO & 1738\\
$(107,47)$ & 10 & $(57,25)$ & 9 & 1 & YES & YES & YES & $1.62$ & $(2,3)$ & NO & 1739\\
$(107,41)$ & 10 & $(76,29)$ & 9 & 1 & YES & YES & YES & $1.50$ & $(2,3)$ & 1859 & 1740\\
$(108,41)$ & 10 & $(4,1)$ & 3 & 4 & YES & YES & YES & $1.38$ & $(2,3)$ & -- & 1741\\
$(108,41)$ & 10 & $(5,2)$ & 3 & 1 & YES & YES & YES & $1.50$ & $(2,3)$ & NO & 1742\\
$(109,30)$ & 10 & $(3,1)$ & 2 & 1 & YES & YES & NO(2) & $1.38$ & $(6,1)$ & -- & 1743\\
$(109,45)$ & 10 & $(3,1)$ & 2 & 1 & YES & YES & YES & $1.50$ & $(2,3)$ & -- & 1744\\
$(109,45)$ & 10 & $(7,3)$ & 4 & 1 & YES & YES & YES & $1.62$ & $(2,3)$ & NO & 1745\\
$(109,46)$ & 10 & $(10,3)$ & 5 & 1 & YES & YES & YES & $1.75$ & $(2,3)$ & -- & 1746\\
$(109,30)$ & 10 & $(13,4)$ & 6 & 1 & YES & YES & NO(2) & $1.38$ & $(8,0)$ & NO & 1747\\
$(109,50)$ & 12 & $(24,11)$ & 8 & 1 & YES & YES & YES & $1.67$ & $(2,3)$ & NO & 1748\\
$(109,45)$ & 10 & $(26,11)$ & 7 & 1 & YES & YES & YES & $1.62$ & $(2,3)$ & NO & 1749\\
$(109,46)$ & 10 & $(59,25)$ & 9 & 1 & YES & YES & YES & $1.62$ & $(2,3)$ & NO & 1750\\
$(109,50)$ & 12 & $(109,50)$ & 12 & 109 & YES & YES & YES & $1.56$ & $(2,3)$ & NO & 1751\\
$(110,29)$ & 12 & $(4,1)$ & 3 & 2 & YES & YES & YES & $1.50$ & $(2,3)$ & -- & 1752\\
$(110,43)$ & 11 & $(6,1)$ & 5 & 2 & YES & YES & YES & $1.29$ & $(2,3)$ & NO & 1753\\
$(110,29)$ & 12 & $(91,24)$ & 11 & 1 & YES & YES & YES & $1.50$ & $(2,3)$ & NO & 1754\\
$(110,43)$ & 11 & $(110,43)$ & 11 & 110 & YES & YES & YES & $1.43$ & $(2,3)$ & NO & 1755\\
$(111,34)$ & 11 & $(3,1)$ & 2 & 3 & NO & YES & YES & $1.56$ & $(2,3)$ & -- & 1756\\
$(111,46)$ & 10 & $(3,1)$ & 2 & 3 & YES & YES & YES & $1.50$ & $(2,3)$ & NO & 1757\\
$(111,46)$ & 10 & $(3,1)$ & 2 & 3 & YES & YES & YES & $1.50$ & $(2,3)$ & -- & 1758\\
$(111,32)$ & 13 & $(4,1)$ & 3 & 1 & YES & YES & YES & $1.50$ & $(4,2)$ & NO & 1759\\
$(111,46)$ & 10 & $(4,1)$ & 3 & 1 & YES & YES & NO(2) & $1.44$ & $(4,2)$ & -- & 1760\\
$(111,29)$ & 12 & $(5,2)$ & 3 & 1 & YES & YES & NO(2) & $1.56$ & $(4,2)$ & NO & 1761\\
$(111,29)$ & 12 & $(5,2)$ & 3 & 1 & YES & YES & NO(2) & $1.56$ & $(4,2)$ & -- & 1762\\
$(111,46)$ & 10 & $(5,2)$ & 3 & 1 & YES & YES & NO(2) & $1.44$ & $(4,2)$ & NO & 1763\\
$(111,29)$ & 12 & $(10,3)$ & 5 & 1 & YES & YES & NO(2) & $1.56$ & $(4,2)$ & NO & 1764\\
$(111,29)$ & 12 & $(11,2)$ & 6 & 1 & YES & YES & NO(2) & $1.60$ & $(2,3)$ & -- & 1765\\
$(111,29)$ & 12 & $(34,9)$ & 8 & 1 & YES & YES & NO(2) & $1.60$ & $(2,3)$ & NO & 1766\\
$(112,41)$ & 10 & $(3,1)$ & 2 & 1 & YES & YES & YES & $1.50$ & $(2,3)$ & -- & 1767\\
$(112,41)$ & 10 & $(19,7)$ & 6 & 1 & YES & YES & YES & $1.38$ & $(2,3)$ & 1849 & 1768\\
$(113,32)$ & 13 & $(2,1)$ & 1 & 1 & YES & YES & YES & $1.50$ & $(4,2)$ & -- & 1769\\
$(113,32)$ & 13 & $(2,1)$ & 1 & 1 & YES & YES & YES & $1.50$ & $(4,2)$ & NO & 1770\\
$(113,35)$ & 11 & $(2,1)$ & 1 & 1 & YES & YES & NO(2) & $1.56$ & $(2,3)$ & NO & 1771\\
$(113,42)$ & 11 & $(2,1)$ & 1 & 1 & YES & YES & YES & $1.43$ & $(4,2)$ & -- & 1772\\
$(113,35)$ & 11 & $(3,1)$ & 2 & 1 & YES & YES & NO(2) & $1.38$ & $(6,1)$ & NO & 1773\\
$(113,35)$ & 11 & $(3,1)$ & 2 & 1 & YES & YES & NO(2) & $1.38$ & $(6,1)$ & -- & 1774\\
$(113,48)$ & 11 & $(3,1)$ & 2 & 1 & YES & YES & NO(2) & $1.44$ & $(4,2)$ & -- & 1775\\
$(113,48)$ & 11 & $(4,1)$ & 3 & 1 & YES & YES & NO(2) & $1.56$ & $(4,2)$ & -- & 1776\\
$(113,24)$ & 11 & $(5,1)$ & 4 & 1 & YES & YES & NO(2) & $1.64$ & $(2,3)$ & NO & 1777\\
$(113,24)$ & 11 & $(5,1)$ & 4 & 1 & YES & YES & NO(2) & $1.64$ & $(2,3)$ & -- & 1778\\
$(113,42)$ & 11 & $(13,5)$ & 5 & 1 & YES & YES & YES & $1.67$ & $(2,3)$ & NO & 1779\\
$(113,35)$ & 11 & $(16,5)$ & 7 & 1 & YES & YES & NO(2) & $1.29$ & $(8,0)$ & 1806 & 1780\\
$(113,30)$ & 11 & $(53,14)$ & 9 & 1 & YES & YES & NO(2) & $1.56$ & $(4,2)$ & NO & 1781\\
$(113,32)$ & 13 & $(53,15)$ & 11 & 1 & YES & YES & YES & $1.62$ & $(2,3)$ & NO & 1782\\
$(113,48)$ & 11 & $(73,31)$ & 10 & 1 & YES & YES & NO(2) & $1.56$ & $(4,2)$ & NO & 1783\\
$(113,48)$ & 11 & $(113,48)$ & 11 & 113 & YES & YES & NO(2) & $1.44$ & $(4,2)$ & NO & 1784\\
$(114,53)$ & 12 & $(2,1)$ & 1 & 2 & YES & YES & YES & $1.62$ & $(2,3)$ & -- & 1785\\
$(115,18)$ & 12 & $(6,1)$ & 5 & 1 & NO & YES & YES & $1.14$ & $(6,1)$ & -- & 1786\\
$(115,44)$ & 10 & $(8,3)$ & 4 & 1 & YES & YES & YES & $1.75$ & $(2,3)$ & -- & 1787\\
$(115,52)$ & 11 & $(9,4)$ & 5 & 1 & YES & YES & YES & $1.14$ & $(4,2)$ & NO & 1788\\
$(115,26)$ & 11 & $(11,4)$ & 5 & 1 & YES & YES & YES & $1.75$ & $(2,3)$ & -- & 1789\\
$(115,44)$ & 10 & $(11,4)$ & 5 & 1 & YES & YES & NO(2) & $1.14$ & $(6,1)$ & NO & 1790\\
$(115,52)$ & 11 & $(31,14)$ & 8 & 1 & YES & YES & YES & $1.14$ & $(4,2)$ & 1717 & 1791\\
$(116,51)$ & 11 & $(3,1)$ & 2 & 1 & YES & YES & YES & $1.50$ & $(2,3)$ & NO & 1792\\
$(116,51)$ & 11 & $(3,1)$ & 2 & 1 & YES & YES & YES & $1.50$ & $(2,3)$ & -- & 1793\\
$(116,45)$ & 10 & $(8,3)$ & 4 & 4 & YES & YES & YES & $1.67$ & $(2,3)$ & -- & 1794\\
$(117,31)$ & 11 & $(10,3)$ & 5 & 1 & YES & YES & YES & $1.43$ & $(4,2)$ & -- & 1795\\
$(117,31)$ & 11 & $(25,7)$ & 7 & 1 & YES & YES & YES & $1.43$ & $(4,2)$ & NO & 1796\\
$(117,43)$ & 10 & $(109,40)$ & 10 & 1 & YES & YES & YES & $1.75$ & $(2,3)$ & NO & 1797\\
$(118,49)$ & 11 & $(5,2)$ & 3 & 1 & YES & YES & NO(2) & $1.56$ & $(4,2)$ & NO & 1798\\
$(118,49)$ & 11 & $(5,2)$ & 3 & 1 & YES & YES & NO(2) & $1.50$ & $(6,1)$ & -- & 1799\\
$(119,37)$ & 11 & $(2,1)$ & 1 & 1 & YES & YES & NO(2) & $1.29$ & $(8,0)$ & NO & 1800\\
$(119,37)$ & 11 & $(2,1)$ & 1 & 1 & YES & YES & NO(2) & $1.29$ & $(8,0)$ & -- & 1801\\
$(119,46)$ & 10 & $(4,1)$ & 3 & 1 & YES & YES & YES & $1.38$ & $(2,3)$ & -- & 1802\\
$(119,46)$ & 10 & $(5,2)$ & 3 & 1 & YES & YES & NO(2) & $1.44$ & $(4,2)$ & NO & 1803\\
$(119,45)$ & 11 & $(8,3)$ & 4 & 1 & YES & YES & YES & $1.29$ & $(4,2)$ & NO & 1804\\
$(119,46)$ & 10 & $(8,3)$ & 4 & 1 & YES & YES & YES & $1.78$ & $(2,3)$ & -- & 1805\\
$(119,37)$ & 11 & $(13,4)$ & 6 & 1 & YES & YES & NO(2) & $1.29$ & $(8,0)$ & 1780 & 1806\\
$(119,46)$ & 10 & $(13,3)$ & 6 & 1 & YES & YES & YES & $1.67$ & $(2,3)$ & -- & 1807\\
$(119,45)$ & 11 & $(82,31)$ & 10 & 1 & YES & YES & NO(2) & $1.56$ & $(4,2)$ & NO & 1808\\
$(120,43)$ & 11 & $(2,1)$ & 1 & 2 & YES & YES & NO(2) & $1.64$ & $(2,3)$ & -- & 1809\\
$(120,43)$ & 11 & $(3,1)$ & 2 & 3 & YES & YES & YES & $1.29$ & $(4,2)$ & NO & 1810\\
$(120,53)$ & 11 & $(5,2)$ & 3 & 5 & YES & YES & NO(2) & $1.38$ & $(6,1)$ & NO & 1811\\
$(120,49)$ & 11 & $(9,4)$ & 5 & 3 & YES & YES & YES & $1.50$ & $(2,3)$ & NO & 1812\\
$(121,35)$ & 12 & $(2,1)$ & 1 & 1 & YES & YES & YES & $1.44$ & $(2,3)$ & -- & 1813\\
$(121,35)$ & 12 & $(4,1)$ & 3 & 1 & YES & YES & NO(2) & $1.25$ & $(10,-1)$ & -- & 1814\\
$(121,46)$ & 10 & $(7,3)$ & 4 & 1 & YES & YES & YES & $1.62$ & $(2,3)$ & -- & 1815\\
$(121,46)$ & 10 & $(8,3)$ & 4 & 1 & YES & YES & YES & $1.62$ & $(2,3)$ & -- & 1816\\
$(121,36)$ & 11 & $(10,3)$ & 5 & 1 & YES & YES & YES & $1.78$ & $(2,3)$ & -- & 1817\\
$(121,32)$ & 11 & $(19,5)$ & 7 & 1 & YES & YES & YES & $1.38$ & $(2,3)$ & NO & 1818\\
$(121,36)$ & 11 & $(71,21)$ & 9 & 1 & YES & YES & YES & $1.78$ & $(2,3)$ & NO & 1819\\
$(122,51)$ & 11 & $(2,1)$ & 1 & 2 & YES & YES & YES & $1.50$ & $(2,3)$ & NO & 1820\\
$(122,37)$ & 11 & $(8,3)$ & 4 & 2 & YES & YES & YES & $1.62$ & $(2,3)$ & -- & 1821\\
$(122,37)$ & 11 & $(18,5)$ & 6 & 2 & YES & YES & YES & $1.62$ & $(2,3)$ & NO & 1822\\
$(123,47)$ & 10 & $(18,7)$ & 6 & 3 & YES & YES & NO(2) & $1.44$ & $(4,2)$ & NO & 1823\\
$(124,57)$ & 12 & $(2,1)$ & 1 & 2 & YES & YES & YES & $1.29$ & $(4,2)$ & -- & 1824\\
$(124,37)$ & 12 & $(3,1)$ & 2 & 1 & YES & YES & YES & $1.56$ & $(2,3)$ & NO & 1825\\
$(124,57)$ & 12 & $(3,1)$ & 2 & 1 & YES & YES & NO(2) & $1.29$ & $(8,0)$ & -- & 1826\\
$(124,37)$ & 12 & $(5,2)$ & 3 & 1 & YES & YES & NO(2) & $1.44$ & $(4,2)$ & NO & 1827\\
$(124,57)$ & 12 & $(24,11)$ & 8 & 4 & YES & YES & YES & $1.56$ & $(2,3)$ & NO & 1828\\
$(125,44)$ & 12 & $(3,1)$ & 2 & 1 & YES & YES & NO(2) & $1.60$ & $(2,3)$ & NO & 1829\\
$(125,49)$ & 11 & $(4,1)$ & 3 & 1 & YES & YES & NO(2) & $1.44$ & $(4,2)$ & NO & 1830\\
$(125,49)$ & 11 & $(5,2)$ & 3 & 5 & YES & YES & YES & $1.14$ & $(4,2)$ & 1497 & 1831\\
$(125,26)$ & 13 & $(6,1)$ & 5 & 1 & YES & YES & YES & $1.50$ & $(2,3)$ & NO & 1832\\
$(125,37)$ & 11 & $(8,3)$ & 4 & 1 & YES & YES & YES & $1.62$ & $(2,3)$ & -- & 1833\\
$(125,49)$ & 11 & $(8,3)$ & 4 & 1 & YES & YES & NO(2) & $1.44$ & $(4,2)$ & NO & 1834\\
$(125,27)$ & 11 & $(9,4)$ & 5 & 1 & YES & YES & YES & $1.43$ & $(4,2)$ & -- & 1835\\
$(125,33)$ & 11 & $(9,2)$ & 5 & 1 & YES & YES & YES & $1.38$ & $(2,3)$ & -- & 1836\\
$(125,37)$ & 11 & $(14,3)$ & 6 & 1 & YES & YES & YES & $1.62$ & $(2,3)$ & NO & 1837\\
$(125,26)$ & 13 & $(29,6)$ & 9 & 1 & YES & YES & YES & $1.50$ & $(2,3)$ & NO & 1838\\
$(125,33)$ & 11 & $(91,24)$ & 11 & 1 & YES & YES & YES & $1.50$ & $(2,3)$ & NO & 1839\\
$(125,37)$ & 11 & $(105,31)$ & 10 & 5 & YES & YES & YES & $1.62$ & $(2,3)$ & 2276 & 1840\\
$(126,55)$ & 11 & $(2,1)$ & 1 & 2 & YES & YES & YES & $1.29$ & $(4,2)$ & NO & 1841\\
$(127,54)$ & 12 & $(2,1)$ & 1 & 1 & YES & YES & NO(2) & $1.67$ & $(2,3)$ & -- & 1842\\
$(127,54)$ & 12 & $(3,1)$ & 2 & 1 & YES & YES & YES & $1.67$ & $(2,3)$ & -- & 1843\\
$(127,56)$ & 11 & $(5,2)$ & 3 & 1 & YES & YES & YES & $1.43$ & $(4,2)$ & -- & 1844\\
$(127,54)$ & 12 & $(33,14)$ & 8 & 1 & YES & YES & NO(2) & $1.70$ & $(2,3)$ & NO & 1845\\
$(127,56)$ & 11 & $(41,18)$ & 8 & 1 & YES & YES & YES & $1.43$ & $(4,2)$ & 2140 & 1846\\
$(128,37)$ & 12 & $(2,1)$ & 1 & 2 & YES & YES & YES & $1.44$ & $(2,3)$ & -- & 1847\\
$(128,47)$ & 10 & $(8,3)$ & 4 & 8 & YES & YES & YES & $1.78$ & $(2,3)$ & -- & 1848\\
$(128,47)$ & 10 & $(11,4)$ & 5 & 1 & YES & YES & YES & $1.38$ & $(2,3)$ & 1768 & 1849\\
$(128,47)$ & 10 & $(18,7)$ & 6 & 2 & YES & YES & YES & $1.78$ & $(2,3)$ & NO & 1850\\
$(128,45)$ & 12 & $(20,7)$ & 8 & 4 & YES & YES & YES & $1.56$ & $(2,3)$ & NO & 1851\\
$(129,49)$ & 10 & $(2,1)$ & 1 & 1 & YES & YES & NO(2) & $1.60$ & $(2,3)$ & -- & 1852\\
$(129,59)$ & 12 & $(13,6)$ & 7 & 1 & YES & YES & YES & $1.56$ & $(2,3)$ & NO & 1853\\
$(129,49)$ & 10 & $(23,9)$ & 7 & 1 & YES & YES & YES & $1.78$ & $(2,3)$ & 1518 & 1854\\
$(130,23)$ & 14 & $(3,1)$ & 2 & 1 & YES & YES & NO(2) & $1.50$ & $(4,2)$ & NO & 1855\\
$(130,51)$ & 11 & $(3,1)$ & 2 & 1 & YES & YES & NO(2) & $1.56$ & $(4,2)$ & -- & 1856\\
$(131,50)$ & 10 & $(7,3)$ & 4 & 1 & YES & YES & YES & $1.75$ & $(2,3)$ & -- & 1857\\
$(131,50)$ & 10 & $(13,3)$ & 6 & 1 & YES & YES & YES & $1.75$ & $(2,3)$ & NO & 1858\\
$(131,50)$ & 10 & $(60,23)$ & 9 & 1 & YES & YES & YES & $1.50$ & $(2,3)$ & 1740 & 1859\\
$(132,59)$ & 12 & $(5,2)$ & 3 & 1 & YES & YES & YES & $1.62$ & $(2,3)$ & NO & 1860\\
$(132,47)$ & 12 & $(14,5)$ & 6 & 2 & YES & YES & NO(2) & $1.73$ & $(2,3)$ & 1595 & 1861\\
$(132,59)$ & 12 & $(20,9)$ & 7 & 4 & YES & YES & YES & $1.50$ & $(2,3)$ & NO & 1862\\
$(132,59)$ & 12 & $(85,38)$ & 11 & 1 & YES & YES & YES & $1.50$ & $(2,3)$ & NO & 1863\\
$(132,59)$ & 12 & $(132,59)$ & 12 & 132 & YES & YES & YES & $1.62$ & $(2,3)$ & NO & 1864\\
$(134,39)$ & 11 & $(2,1)$ & 1 & 2 & YES & YES & NO(2) & $1.56$ & $(4,2)$ & -- & 1865\\
$(134,39)$ & 11 & $(4,1)$ & 3 & 2 & YES & YES & NO(2) & $1.60$ & $(2,3)$ & NO & 1866\\
$(134,39)$ & 11 & $(4,1)$ & 3 & 2 & YES & YES & NO(2) & $1.60$ & $(2,3)$ & -- & 1867\\
$(134,39)$ & 11 & $(8,3)$ & 4 & 2 & YES & YES & YES & $1.62$ & $(2,3)$ & -- & 1868\\
$(134,49)$ & 11 & $(52,19)$ & 9 & 2 & YES & YES & YES & $1.29$ & $(2,3)$ & 1912 & 1869\\
$(135,26)$ & 14 & $(4,1)$ & 3 & 1 & YES & YES & YES & $1.38$ & $(2,3)$ & NO & 1870\\
$(135,32)$ & 12 & $(4,1)$ & 3 & 1 & NO & YES & YES & $1.50$ & $(4,2)$ & -- & 1871\\
$(135,32)$ & 12 & $(38,9)$ & 9 & 1 & YES & YES & NO(2) & $1.29$ & $(8,0)$ & NO & 1872\\
$(137,43)$ & 12 & $(3,1)$ & 2 & 1 & NO & YES & NO(2) & $1.38$ & $(6,1)$ & -- & 1873\\
$(137,43)$ & 12 & $(3,1)$ & 2 & 1 & YES & YES & NO(2) & $1.50$ & $(4,2)$ & NO & 1874\\
$(137,51)$ & 12 & $(3,1)$ & 2 & 1 & YES & YES & NO(2) & $1.67$ & $(2,3)$ & -- & 1875\\
$(137,63)$ & 12 & $(24,11)$ & 8 & 1 & YES & YES & NO(2) & $1.29$ & $(8,0)$ & 1969 & 1876\\
$(138,49)$ & 12 & $(3,1)$ & 2 & 3 & YES & YES & YES & $1.67$ & $(2,3)$ & -- & 1877\\
$(138,61)$ & 12 & $(5,1)$ & 4 & 1 & YES & YES & NO(2) & $1.56$ & $(4,2)$ & NO & 1878\\
$(138,61)$ & 12 & $(5,1)$ & 4 & 1 & YES & YES & NO(2) & $1.56$ & $(4,2)$ & -- & 1879\\
$(138,61)$ & 12 & $(5,1)$ & 4 & 1 & YES & YES & NO(2) & $1.56$ & $(4,2)$ & NO & 1880\\
$(138,31)$ & 12 & $(7,3)$ & 4 & 1 & YES & YES & NO(2) & $1.44$ & $(4,2)$ & -- & 1881\\
$(138,61)$ & 12 & $(25,11)$ & 7 & 1 & YES & YES & NO(2) & $1.56$ & $(4,2)$ & NO & 1882\\
$(138,61)$ & 12 & $(95,42)$ & 11 & 1 & YES & YES & NO(2) & $1.44$ & $(4,2)$ & NO & 1883\\
$(138,61)$ & 12 & $(138,61)$ & 12 & 138 & YES & YES & NO(2) & $1.56$ & $(4,2)$ & NO & 1884\\
$(139,39)$ & 11 & $(2,1)$ & 1 & 1 & YES & YES & YES & $1.50$ & $(2,3)$ & -- & 1885\\
$(139,61)$ & 11 & $(2,1)$ & 1 & 1 & YES & YES & NO(2) & $1.38$ & $(10,-1)$ & NO & 1886\\
$(139,61)$ & 11 & $(5,2)$ & 3 & 1 & YES & YES & YES & $1.43$ & $(4,2)$ & -- & 1887\\
$(139,61)$ & 11 & $(12,5)$ & 5 & 1 & YES & YES & YES & $1.43$ & $(4,2)$ & NO & 1888\\
$(140,61)$ & 11 & $(16,7)$ & 6 & 4 & YES & YES & NO(2) & $1.44$ & $(4,2)$ & NO & 1889\\
$(142,59)$ & 12 & $(6,1)$ & 5 & 2 & YES & YES & NO(2) & $1.70$ & $(2,3)$ & -- & 1890\\
$(142,59)$ & 12 & $(7,1)$ & 6 & 1 & YES & YES & YES & $1.50$ & $(2,3)$ & -- & 1891\\
$(142,59)$ & 12 & $(29,12)$ & 7 & 1 & YES & YES & NO(2) & $1.56$ & $(4,2)$ & NO & 1892\\
$(143,54)$ & 12 & $(3,1)$ & 2 & 1 & YES & YES & YES & $1.50$ & $(2,3)$ & NO & 1893\\
$(143,59)$ & 11 & $(3,1)$ & 2 & 1 & YES & YES & NO(2) & $1.44$ & $(4,2)$ & -- & 1894\\
$(143,54)$ & 12 & $(8,3)$ & 4 & 1 & YES & YES & YES & $1.57$ & $(2,3)$ & NO & 1895\\
$(143,40)$ & 12 & $(29,8)$ & 7 & 1 & YES & YES & YES & $1.57$ & $(4,2)$ & NO & 1896\\
$(143,63)$ & 11 & $(84,37)$ & 10 & 1 & YES & YES & YES & $1.50$ & $(2,3)$ & NO & 1897\\
$(143,59)$ & 11 & $(143,59)$ & 11 & 143 & YES & YES & NO(2) & $1.60$ & $(2,3)$ & NO & 1898\\
$(144,61)$ & 11 & $(2,1)$ & 1 & 2 & YES & YES & YES & $1.44$ & $(2,3)$ & NO & 1899\\
$(144,43)$ & 13 & $(3,1)$ & 2 & 3 & YES & YES & NO(2) & $1.29$ & $(8,0)$ & -- & 1900\\
$(144,59)$ & 11 & $(3,1)$ & 2 & 3 & YES & YES & NO(2) & $1.33$ & $(4,2)$ & -- & 1901\\
$(144,61)$ & 11 & $(3,1)$ & 2 & 3 & YES & YES & YES & $1.29$ & $(2,3)$ & NO & 1902\\
$(144,55)$ & 10 & $(23,9)$ & 7 & 1 & YES & YES & YES & $1.89$ & $(2,3)$ & NO & 1903\\
$(144,59)$ & 11 & $(144,59)$ & 11 & 144 & YES & YES & NO(2) & $1.44$ & $(4,2)$ & NO & 1904\\
$(144,65)$ & 12 & $(144,65)$ & 12 & 144 & YES & YES & YES & $1.62$ & $(2,3)$ & NO & 1905\\
$(145,41)$ & 13 & $(2,1)$ & 1 & 1 & YES & YES & YES & $1.56$ & $(2,3)$ & NO & 1906\\
$(145,53)$ & 11 & $(2,1)$ & 1 & 1 & YES & YES & YES & $1.43$ & $(2,3)$ & -- & 1907\\
$(145,41)$ & 13 & $(3,1)$ & 2 & 1 & YES & YES & YES & $1.29$ & $(4,2)$ & NO & 1908\\
$(145,53)$ & 11 & $(3,1)$ & 2 & 1 & YES & YES & NO(2) & $1.33$ & $(4,2)$ & -- & 1909\\
$(145,53)$ & 11 & $(5,2)$ & 3 & 5 & YES & YES & NO(2) & $1.44$ & $(4,2)$ & NO & 1910\\
$(145,51)$ & 12 & $(20,7)$ & 8 & 5 & YES & YES & NO(2) & $1.29$ & $(8,0)$ & 1938 & 1911\\
$(145,53)$ & 11 & $(41,15)$ & 8 & 1 & YES & YES & YES & $1.29$ & $(2,3)$ & 1869 & 1912\\
$(145,41)$ & 13 & $(145,41)$ & 13 & 145 & YES & YES & NO(2) & $1.29$ & $(8,0)$ & NO & 1913\\
$(146,61)$ & 12 & $(67,28)$ & 10 & 1 & YES & YES & YES & $1.50$ & $(2,3)$ & NO & 1914\\
$(147,26)$ & 15 & $(2,1)$ & 1 & 1 & YES & YES & YES & $1.50$ & $(2,3)$ & -- & 1915\\
$(147,26)$ & 15 & $(2,1)$ & 1 & 1 & YES & YES & YES & $1.62$ & $(2,3)$ & NO & 1916\\
$(147,26)$ & 15 & $(11,2)$ & 6 & 1 & YES & YES & YES & $1.38$ & $(4,2)$ & NO & 1917\\
$(148,65)$ & 11 & $(4,1)$ & 3 & 4 & YES & YES & YES & $1.62$ & $(2,3)$ & -- & 1918\\
$(148,31)$ & 12 & $(5,2)$ & 3 & 1 & YES & YES & NO(2) & $1.60$ & $(2,3)$ & NO & 1919\\
$(148,31)$ & 12 & $(5,2)$ & 3 & 1 & YES & YES & NO(2) & $1.60$ & $(2,3)$ & NO & 1920\\
$(149,42)$ & 12 & $(2,1)$ & 1 & 1 & YES & YES & YES & $1.50$ & $(2,3)$ & NO & 1921\\
$(149,41)$ & 11 & $(3,1)$ & 2 & 1 & NO & YES & YES & $1.60$ & $(2,3)$ & -- & 1922\\
$(149,46)$ & 13 & $(4,1)$ & 3 & 1 & YES & YES & YES & $1.62$ & $(2,3)$ & -- & 1923\\
$(149,46)$ & 13 & $(5,1)$ & 4 & 1 & YES & YES & YES & $1.50$ & $(2,3)$ & -- & 1924\\
$(149,40)$ & 11 & $(7,3)$ & 4 & 1 & YES & YES & YES & $1.75$ & $(2,3)$ & -- & 1925\\
$(149,40)$ & 11 & $(7,3)$ & 4 & 1 & YES & YES & YES & $1.75$ & $(2,3)$ & NO & 1926\\
$(149,41)$ & 11 & $(7,3)$ & 4 & 1 & YES & YES & YES & $1.62$ & $(2,3)$ & -- & 1927\\
$(149,44)$ & 11 & $(7,3)$ & 4 & 1 & YES & YES & YES & $1.75$ & $(2,3)$ & NO & 1928\\
$(149,41)$ & 11 & $(13,4)$ & 6 & 1 & YES & YES & YES & $1.62$ & $(2,3)$ & NO & 1929\\
$(149,42)$ & 12 & $(18,5)$ & 6 & 1 & YES & YES & NO(2) & $1.14$ & $(10,-1)$ & NO & 1930\\
$(149,46)$ & 13 & $(55,17)$ & 10 & 1 & YES & YES & YES & $1.50$ & $(2,3)$ & NO & 1931\\
$(149,65)$ & 11 & $(55,24)$ & 9 & 1 & YES & YES & NO(2) & $1.60$ & $(2,3)$ & NO & 1932\\
$(149,44)$ & 11 & $(64,19)$ & 9 & 1 & YES & YES & YES & $1.43$ & $(4,2)$ & NO & 1933\\
$(151,53)$ & 12 & $(2,1)$ & 1 & 1 & YES & YES & YES & $1.38$ & $(4,2)$ & -- & 1934\\
$(151,47)$ & 12 & $(3,1)$ & 2 & 1 & YES & YES & NO(2) & $1.14$ & $(10,-1)$ & -- & 1935\\
$(151,47)$ & 12 & $(7,2)$ & 4 & 1 & YES & YES & YES & $1.50$ & $(2,3)$ & NO & 1936\\
$(151,47)$ & 12 & $(10,3)$ & 5 & 1 & YES & YES & YES & $1.62$ & $(2,3)$ & NO & 1937\\
$(151,53)$ & 12 & $(17,6)$ & 7 & 1 & YES & YES & NO(2) & $1.29$ & $(8,0)$ & 1911 & 1938\\
$(152,63)$ & 11 & $(2,1)$ & 1 & 2 & YES & YES & NO(2) & $1.44$ & $(4,2)$ & NO & 1939\\
$(152,63)$ & 11 & $(3,1)$ & 2 & 1 & YES & YES & NO(2) & $1.29$ & $(6,1)$ & NO & 1940\\
$(152,63)$ & 11 & $(3,1)$ & 2 & 1 & YES & YES & NO(2) & $1.29$ & $(6,1)$ & -- & 1941\\
$(152,67)$ & 11 & $(3,1)$ & 2 & 1 & YES & YES & NO(2) & $1.25$ & $(6,1)$ & -- & 1942\\
$(152,55)$ & 12 & $(5,2)$ & 3 & 1 & YES & YES & YES & $1.43$ & $(2,3)$ & NO & 1943\\
$(152,41)$ & 11 & $(7,3)$ & 4 & 1 & YES & YES & YES & $1.62$ & $(2,3)$ & -- & 1944\\
$(152,63)$ & 11 & $(7,3)$ & 4 & 1 & YES & YES & NO(2) & $1.38$ & $(8,0)$ & NO & 1945\\
$(152,67)$ & 11 & $(16,7)$ & 6 & 8 & YES & YES & NO(2) & $1.44$ & $(4,2)$ & NO & 1946\\
$(152,67)$ & 11 & $(152,67)$ & 11 & 152 & YES & YES & YES & $1.38$ & $(2,3)$ & NO & 1947\\
$(153,64)$ & 11 & $(2,1)$ & 1 & 1 & YES & YES & NO(2) & $1.38$ & $(6,1)$ & NO & 1948\\
$(153,64)$ & 11 & $(12,5)$ & 5 & 3 & YES & YES & NO(2) & $1.38$ & $(6,1)$ & NO & 1949\\
$(153,70)$ & 12 & $(13,6)$ & 7 & 1 & YES & YES & NO(2) & $1.29$ & $(8,0)$ & NO & 1950\\
$(153,56)$ & 11 & $(27,10)$ & 7 & 9 & YES & YES & NO(2) & $1.14$ & $(6,1)$ & NO & 1951\\
$(154,59)$ & 11 & $(5,2)$ & 3 & 1 & YES & YES & YES & $1.75$ & $(2,3)$ & -- & 1952\\
$(154,65)$ & 11 & $(5,2)$ & 3 & 1 & YES & YES & NO(2) & $1.14$ & $(6,1)$ & -- & 1953\\
$(154,59)$ & 11 & $(7,2)$ & 4 & 7 & YES & YES & YES & $1.75$ & $(2,3)$ & -- & 1954\\
$(154,59)$ & 11 & $(29,11)$ & 7 & 1 & YES & YES & YES & $1.75$ & $(2,3)$ & NO & 1955\\
$(155,41)$ & 12 & $(9,2)$ & 5 & 1 & YES & YES & NO(2) & $1.33$ & $(8,0)$ & NO & 1956\\
$(155,64)$ & 11 & $(9,4)$ & 5 & 1 & YES & YES & YES & $1.43$ & $(4,2)$ & NO & 1957\\
$(156,25)$ & 15 & $(4,1)$ & 3 & 4 & YES & YES & YES & $1.38$ & $(2,3)$ & -- & 1958\\
$(157,42)$ & 12 & $(4,1)$ & 3 & 1 & YES & YES & YES & $1.38$ & $(2,3)$ & -- & 1959\\
$(158,61)$ & 11 & $(9,2)$ & 5 & 1 & YES & YES & YES & $1.62$ & $(2,3)$ & NO & 1960\\
$(159,61)$ & 12 & $(2,1)$ & 1 & 1 & NO & YES & YES & $1.62$ & $(2,3)$ & -- & 1961\\
$(159,61)$ & 12 & $(2,1)$ & 1 & 1 & YES & YES & NO(2) & $1.56$ & $(8,0)$ & NO & 1962\\
$(159,59)$ & 11 & $(3,1)$ & 2 & 3 & YES & YES & NO(2) & $1.60$ & $(2,3)$ & -- & 1963\\
$(159,59)$ & 11 & $(4,1)$ & 3 & 1 & YES & YES & NO(2) & $1.50$ & $(2,3)$ & -- & 1964\\
$(159,61)$ & 12 & $(5,1)$ & 4 & 1 & YES & YES & NO(2) & $1.14$ & $(8,0)$ & -- & 1965\\
$(159,47)$ & 11 & $(7,3)$ & 4 & 1 & YES & YES & YES & $1.89$ & $(2,3)$ & NO & 1966\\
$(159,62)$ & 11 & $(9,2)$ & 5 & 3 & YES & YES & YES & $1.75$ & $(2,3)$ & NO & 1967\\
$(159,61)$ & 12 & $(13,5)$ & 5 & 1 & YES & YES & NO(2) & $1.29$ & $(8,0)$ & 1687 & 1968\\
$(159,73)$ & 12 & $(13,6)$ & 7 & 1 & YES & YES & NO(2) & $1.29$ & $(8,0)$ & 1876 & 1969\\
$(159,59)$ & 11 & $(19,7)$ & 6 & 1 & YES & YES & NO(2) & $1.50$ & $(2,3)$ & NO & 1970\\
$(159,37)$ & 12 & $(64,15)$ & 10 & 1 & YES & YES & YES & $1.62$ & $(2,3)$ & NO & 1971\\
$(159,59)$ & 11 & $(97,36)$ & 10 & 1 & YES & YES & NO(2) & $1.60$ & $(2,3)$ & NO & 1972\\
$(161,51)$ & 13 & $(2,1)$ & 1 & 1 & YES & YES & YES & $1.38$ & $(4,2)$ & NO & 1973\\
$(161,48)$ & 12 & $(3,1)$ & 2 & 1 & YES & YES & NO(2) & $1.38$ & $(6,1)$ & NO & 1974\\
$(161,66)$ & 11 & $(5,2)$ & 3 & 1 & YES & YES & NO(2) & $1.56$ & $(4,2)$ & NO & 1975\\
$(162,71)$ & 12 & $(5,1)$ & 4 & 1 & YES & YES & YES & $1.50$ & $(2,3)$ & -- & 1976\\
$(162,71)$ & 12 & $(5,1)$ & 4 & 1 & YES & YES & NO(2) & $1.56$ & $(4,2)$ & NO & 1977\\
$(162,73)$ & 12 & $(5,1)$ & 4 & 1 & YES & YES & YES & $1.50$ & $(2,3)$ & NO & 1978\\
$(162,73)$ & 12 & $(5,1)$ & 4 & 1 & YES & YES & YES & $1.50$ & $(2,3)$ & -- & 1979\\
$(162,73)$ & 12 & $(5,1)$ & 4 & 1 & YES & YES & NO(2) & $1.56$ & $(4,2)$ & NO & 1980\\
$(162,37)$ & 12 & $(8,3)$ & 4 & 2 & YES & YES & YES & $1.62$ & $(2,3)$ & NO & 1981\\
$(162,71)$ & 12 & $(16,7)$ & 6 & 2 & YES & YES & YES & $1.62$ & $(2,3)$ & 1724 & 1982\\
$(163,43)$ & 12 & $(3,1)$ & 2 & 1 & YES & YES & NO(2) & $1.44$ & $(4,2)$ & -- & 1983\\
$(163,43)$ & 12 & $(4,1)$ & 3 & 1 & YES & YES & NO(2) & $1.56$ & $(4,2)$ & -- & 1984\\
$(163,63)$ & 11 & $(5,2)$ & 3 & 1 & YES & YES & YES & $1.67$ & $(2,3)$ & -- & 1985\\
$(163,71)$ & 11 & $(7,3)$ & 4 & 1 & YES & YES & NO(2) & $1.44$ & $(4,2)$ & NO & 1986\\
$(163,43)$ & 12 & $(11,3)$ & 5 & 1 & YES & YES & NO(2) & $1.25$ & $(6,1)$ & NO & 1987\\
$(163,43)$ & 12 & $(34,9)$ & 8 & 1 & YES & YES & NO(2) & $1.44$ & $(4,2)$ & 1631 & 1988\\
$(163,43)$ & 12 & $(53,14)$ & 9 & 1 & YES & YES & NO(2) & $1.56$ & $(4,2)$ & NO & 1989\\
$(163,43)$ & 12 & $(91,24)$ & 11 & 1 & YES & YES & NO(2) & $1.38$ & $(6,1)$ & NO & 1990\\
$(163,63)$ & 11 & $(106,41)$ & 10 & 1 & YES & YES & YES & $1.67$ & $(2,3)$ & 2220 & 1991\\
$(163,44)$ & 11 & $(152,41)$ & 11 & 1 & YES & YES & YES & $1.75$ & $(2,3)$ & NO & 1992\\
$(165,64)$ & 11 & $(5,2)$ & 3 & 5 & YES & YES & YES & $1.80$ & $(2,3)$ & -- & 1993\\
$(166,63)$ & 12 & $(50,19)$ & 8 & 2 & YES & YES & YES & $1.57$ & $(4,2)$ & NO & 1994\\
$(167,64)$ & 11 & $(5,1)$ & 4 & 1 & YES & YES & NO(2) & $1.25$ & $(8,0)$ & NO & 1995\\
$(167,69)$ & 11 & $(5,2)$ & 3 & 1 & YES & YES & YES & $1.62$ & $(2,3)$ & -- & 1996\\
$(167,64)$ & 11 & $(60,23)$ & 9 & 1 & YES & YES & NO(2) & $1.38$ & $(8,0)$ & NO & 1997\\
$(168,65)$ & 12 & $(6,1)$ & 5 & 6 & YES & YES & NO(2) & $1.29$ & $(6,1)$ & -- & 1998\\
$(168,65)$ & 12 & $(75,29)$ & 9 & 3 & YES & YES & NO(2) & $1.29$ & $(6,1)$ & NO & 1999\\
$(169,62)$ & 12 & $(2,1)$ & 1 & 1 & YES & YES & YES & $1.56$ & $(2,3)$ & NO & 2000\\
$(169,66)$ & 11 & $(2,1)$ & 1 & 1 & YES & YES & NO(2) & $1.60$ & $(2,3)$ & -- & 2001\\
$(169,64)$ & 11 & $(3,1)$ & 2 & 1 & YES & YES & NO(2) & $1.44$ & $(4,2)$ & -- & 2002\\
$(169,64)$ & 11 & $(3,1)$ & 2 & 1 & YES & YES & NO(2) & $1.60$ & $(2,3)$ & NO & 2003\\
$(169,38)$ & 13 & $(5,1)$ & 4 & 1 & YES & YES & YES & $1.38$ & $(2,3)$ & NO & 2004\\
$(169,64)$ & 11 & $(5,1)$ & 4 & 1 & YES & YES & NO(2) & $1.25$ & $(8,0)$ & NO & 2005\\
$(169,66)$ & 11 & $(5,1)$ & 4 & 1 & YES & YES & NO(2) & $1.25$ & $(8,0)$ & NO & 2006\\
$(169,71)$ & 11 & $(5,2)$ & 3 & 1 & YES & YES & YES & $1.75$ & $(2,3)$ & -- & 2007\\
$(169,38)$ & 13 & $(7,2)$ & 4 & 1 & YES & YES & NO(2) & $1.56$ & $(4,2)$ & NO & 2008\\
$(169,70)$ & 11 & $(7,2)$ & 4 & 1 & YES & YES & YES & $1.75$ & $(2,3)$ & -- & 2009\\
$(169,64)$ & 11 & $(13,5)$ & 5 & 13 & YES & YES & NO(2) & $1.44$ & $(4,2)$ & 1707 & 2010\\
$(169,38)$ & 13 & $(49,11)$ & 10 & 1 & YES & YES & YES & $1.38$ & $(2,3)$ & NO & 2011\\
$(169,70)$ & 11 & $(53,22)$ & 9 & 1 & YES & YES & YES & $1.62$ & $(2,3)$ & NO & 2012\\
$(170,29)$ & 15 & $(2,1)$ & 1 & 2 & YES & YES & NO(2) & $1.60$ & $(2,3)$ & -- & 2013\\
$(170,29)$ & 15 & $(2,1)$ & 1 & 2 & YES & YES & NO(2) & $1.70$ & $(2,3)$ & NO & 2014\\
$(171,53)$ & 12 & $(2,1)$ & 1 & 1 & YES & YES & NO(2) & $1.60$ & $(2,3)$ & NO & 2015\\
$(171,71)$ & 12 & $(2,1)$ & 1 & 1 & YES & YES & NO(2) & $1.14$ & $(6,1)$ & -- & 2016\\
$(171,71)$ & 12 & $(3,1)$ & 2 & 3 & YES & YES & YES & $1.62$ & $(2,3)$ & -- & 2017\\
$(171,65)$ & 11 & $(5,2)$ & 3 & 1 & YES & YES & YES & $2.00$ & $(2,3)$ & -- & 2018\\
$(171,71)$ & 12 & $(5,2)$ & 3 & 1 & YES & YES & YES & $1.43$ & $(2,3)$ & NO & 2019\\
$(171,71)$ & 12 & $(12,5)$ & 5 & 3 & YES & YES & YES & $1.67$ & $(2,3)$ & NO & 2020\\
$(171,65)$ & 11 & $(18,7)$ & 6 & 9 & YES & YES & YES & $1.88$ & $(2,3)$ & NO & 2021\\
$(171,71)$ & 12 & $(118,49)$ & 11 & 1 & YES & YES & NO(2) & $1.56$ & $(4,2)$ & NO & 2022\\
$(172,71)$ & 11 & $(17,7)$ & 6 & 1 & YES & YES & NO(2) & $1.60$ & $(2,3)$ & NO & 2023\\
$(173,51)$ & 12 & $(2,1)$ & 1 & 1 & YES & YES & NO(2) & $1.33$ & $(8,0)$ & -- & 2024\\
$(173,73)$ & 11 & $(2,1)$ & 1 & 1 & YES & YES & YES & $1.50$ & $(2,3)$ & NO & 2025\\
$(173,78)$ & 12 & $(2,1)$ & 1 & 1 & YES & YES & NO(2) & $1.29$ & $(8,0)$ & NO & 2026\\
$(173,78)$ & 12 & $(2,1)$ & 1 & 1 & NO & YES & NO(2) & $1.38$ & $(10,-1)$ & -- & 2027\\
$(173,64)$ & 11 & $(5,2)$ & 3 & 1 & YES & YES & YES & $1.62$ & $(2,3)$ & -- & 2028\\
$(173,51)$ & 12 & $(78,23)$ & 10 & 1 & YES & YES & NO(2) & $1.33$ & $(8,0)$ & NO & 2029\\
$(175,62)$ & 12 & $(2,1)$ & 1 & 1 & YES & YES & YES & $1.56$ & $(2,3)$ & -- & 2030\\
$(175,62)$ & 12 & $(2,1)$ & 1 & 1 & YES & YES & NO(2) & $1.70$ & $(2,3)$ & NO & 2031\\
$(175,62)$ & 12 & $(5,2)$ & 3 & 5 & YES & YES & NO(2) & $1.29$ & $(8,0)$ & NO & 2032\\
$(175,67)$ & 11 & $(5,2)$ & 3 & 5 & YES & YES & YES & $1.62$ & $(2,3)$ & -- & 2033\\
$(175,62)$ & 12 & $(17,6)$ & 7 & 1 & YES & YES & YES & $1.29$ & $(4,2)$ & NO & 2034\\
$(175,67)$ & 11 & $(18,7)$ & 6 & 1 & YES & YES & YES & $1.62$ & $(2,3)$ & NO & 2035\\
$(175,67)$ & 11 & $(55,21)$ & 8 & 5 & YES & YES & YES & $1.75$ & $(2,3)$ & 2260 & 2036\\
$(176,65)$ & 11 & $(3,1)$ & 2 & 1 & YES & YES & NO(2) & $1.44$ & $(4,2)$ & NO & 2037\\
$(176,65)$ & 11 & $(11,4)$ & 5 & 11 & YES & YES & NO(2) & $1.38$ & $(8,0)$ & NO & 2038\\
$(177,47)$ & 12 & $(4,1)$ & 3 & 1 & YES & YES & YES & $1.56$ & $(2,3)$ & NO & 2039\\
$(177,80)$ & 12 & $(4,1)$ & 3 & 1 & YES & YES & NO(2) & $1.44$ & $(4,2)$ & -- & 2040\\
$(177,47)$ & 12 & $(5,1)$ & 4 & 1 & YES & YES & YES & $1.44$ & $(2,3)$ & -- & 2041\\
$(177,74)$ & 12 & $(12,5)$ & 5 & 3 & YES & YES & YES & $1.62$ & $(2,3)$ & NO & 2042\\
$(177,46)$ & 13 & $(27,7)$ & 9 & 3 & YES & YES & NO(2) & $1.29$ & $(8,0)$ & NO & 2043\\
$(178,47)$ & 12 & $(2,1)$ & 1 & 2 & YES & YES & NO(2) & $1.44$ & $(4,2)$ & -- & 2044\\
$(178,69)$ & 11 & $(5,2)$ & 3 & 1 & YES & YES & YES & $1.89$ & $(2,3)$ & NO & 2045\\
$(178,47)$ & 12 & $(15,4)$ & 6 & 1 & YES & YES & NO(2) & $1.50$ & $(2,3)$ & NO & 2046\\
$(179,48)$ & 12 & $(3,1)$ & 2 & 1 & NO & YES & NO(2) & $1.56$ & $(4,2)$ & -- & 2047\\
$(179,42)$ & 13 & $(4,1)$ & 3 & 1 & YES & YES & YES & $1.38$ & $(2,3)$ & -- & 2048\\
$(179,76)$ & 12 & $(5,2)$ & 3 & 1 & YES & YES & NO(2) & $1.56$ & $(4,2)$ & NO & 2049\\
$(181,65)$ & 12 & $(2,1)$ & 1 & 1 & YES & YES & YES & $1.50$ & $(2,3)$ & NO & 2050\\
$(181,48)$ & 12 & $(5,2)$ & 3 & 1 & YES & YES & NO(2) & $1.29$ & $(6,1)$ & NO & 2051\\
$(181,70)$ & 11 & $(5,2)$ & 3 & 1 & YES & YES & YES & $1.67$ & $(2,3)$ & -- & 2052\\
$(181,75)$ & 11 & $(5,2)$ & 3 & 1 & YES & YES & YES & $1.62$ & $(2,3)$ & -- & 2053\\
$(181,41)$ & 12 & $(48,11)$ & 9 & 1 & YES & YES & YES & $1.75$ & $(2,3)$ & NO & 2054\\
$(181,75)$ & 11 & $(53,22)$ & 9 & 1 & YES & YES & YES & $1.62$ & $(2,3)$ & NO & 2055\\
$(181,41)$ & 12 & $(115,26)$ & 11 & 1 & YES & YES & YES & $1.75$ & $(2,3)$ & NO & 2056\\
$(181,70)$ & 11 & $(119,46)$ & 10 & 1 & YES & YES & YES & $1.78$ & $(2,3)$ & NO & 2057\\
$(187,50)$ & 13 & $(4,1)$ & 3 & 1 & YES & YES & NO(2) & $1.44$ & $(4,2)$ & -- & 2058\\
$(187,79)$ & 11 & $(17,7)$ & 6 & 17 & YES & YES & YES & $1.78$ & $(2,3)$ & 2261 & 2059\\
$(188,57)$ & 13 & $(2,1)$ & 1 & 2 & YES & YES & NO(2) & $1.56$ & $(4,2)$ & NO & 2060\\
$(188,59)$ & 13 & $(2,1)$ & 1 & 2 & YES & YES & YES & $1.56$ & $(2,3)$ & NO & 2061\\
$(188,73)$ & 12 & $(2,1)$ & 1 & 2 & YES & YES & NO(2) & $1.43$ & $(6,1)$ & -- & 2062\\
$(188,57)$ & 13 & $(10,3)$ & 5 & 2 & YES & YES & YES & $1.56$ & $(2,3)$ & NO & 2063\\
$(188,73)$ & 12 & $(13,5)$ & 5 & 1 & YES & YES & NO(2) & $1.29$ & $(6,1)$ & NO & 2064\\
$(189,50)$ & 13 & $(34,9)$ & 8 & 1 & YES & YES & YES & $1.62$ & $(2,3)$ & NO & 2065\\
$(191,26)$ & 17 & $(2,1)$ & 1 & 1 & YES & YES & YES & $1.38$ & $(4,2)$ & NO & 2066\\
$(191,50)$ & 13 & $(6,1)$ & 5 & 1 & YES & YES & NO(2) & $1.33$ & $(8,0)$ & NO & 2067\\
$(191,59)$ & 13 & $(13,4)$ & 6 & 1 & YES & YES & YES & $1.62$ & $(2,3)$ & NO & 2068\\
$(191,50)$ & 13 & $(42,11)$ & 9 & 1 & YES & YES & YES & $1.50$ & $(2,3)$ & NO & 2069\\
$(192,31)$ & 16 & $(2,1)$ & 1 & 2 & YES & YES & NO(2) & $1.64$ & $(2,3)$ & -- & 2070\\
$(192,31)$ & 16 & $(2,1)$ & 1 & 2 & YES & YES & NO(2) & $1.73$ & $(2,3)$ & 1584 & 2071\\
$(192,71)$ & 11 & $(2,1)$ & 1 & 2 & YES & YES & NO(2) & $1.60$ & $(2,3)$ & NO & 2072\\
$(192,73)$ & 11 & $(2,1)$ & 1 & 2 & YES & YES & NO(2) & $1.38$ & $(8,0)$ & NO & 2073\\
$(194,75)$ & 11 & $(5,2)$ & 3 & 1 & YES & YES & YES & $1.90$ & $(2,3)$ & -- & 2074\\
$(194,75)$ & 11 & $(106,41)$ & 10 & 2 & YES & YES & YES & $1.78$ & $(2,3)$ & NO & 2075\\
$(196,45)$ & 13 & $(4,1)$ & 3 & 4 & YES & YES & YES & $1.50$ & $(2,3)$ & -- & 2076\\
$(196,45)$ & 13 & $(35,8)$ & 8 & 7 & YES & YES & NO(2) & $1.00$ & $(10,-1)$ & NO & 2077\\
$(197,52)$ & 12 & $(5,2)$ & 3 & 1 & YES & YES & NO(2) & $1.44$ & $(4,2)$ & -- & 2078\\
$(197,76)$ & 12 & $(5,1)$ & 4 & 1 & YES & YES & NO(2) & $1.14$ & $(6,1)$ & NO & 2079\\
$(197,43)$ & 12 & $(11,3)$ & 5 & 1 & YES & YES & YES & $1.78$ & $(2,3)$ & NO & 2080\\
$(197,52)$ & 12 & $(19,5)$ & 7 & 1 & YES & YES & YES & $1.44$ & $(2,3)$ & NO & 2081\\
$(197,76)$ & 12 & $(70,27)$ & 10 & 1 & YES & YES & NO(2) & $1.29$ & $(6,1)$ & NO & 2082\\
$(197,52)$ & 12 & $(91,24)$ & 11 & 1 & YES & YES & NO(2) & $1.44$ & $(4,2)$ & NO & 2083\\
$(198,71)$ & 12 & $(2,1)$ & 1 & 2 & YES & YES & NO(2) & $1.29$ & $(6,1)$ & -- & 2084\\
$(198,71)$ & 12 & $(39,14)$ & 8 & 3 & YES & YES & NO(2) & $1.14$ & $(6,1)$ & NO & 2085\\
$(199,78)$ & 12 & $(2,1)$ & 1 & 1 & YES & YES & NO(2) & $1.56$ & $(4,2)$ & NO & 2086\\
$(199,78)$ & 12 & $(2,1)$ & 1 & 1 & NO & YES & NO(2) & $1.56$ & $(2,3)$ & -- & 2087\\
$(199,78)$ & 12 & $(3,1)$ & 2 & 1 & YES & YES & YES & $1.43$ & $(4,2)$ & -- & 2088\\
$(199,78)$ & 12 & $(4,1)$ & 3 & 1 & YES & YES & YES & $1.43$ & $(4,2)$ & NO & 2089\\
$(199,78)$ & 12 & $(5,1)$ & 4 & 1 & YES & YES & NO(2) & $1.14$ & $(6,1)$ & NO & 2090\\
$(199,78)$ & 12 & $(74,29)$ & 10 & 1 & YES & YES & NO(2) & $1.14$ & $(6,1)$ & NO & 2091\\
$(199,78)$ & 12 & $(125,49)$ & 11 & 1 & YES & YES & YES & $1.43$ & $(4,2)$ & NO & 2092\\
$(199,78)$ & 12 & $(199,78)$ & 12 & 199 & YES & YES & YES & $1.43$ & $(4,2)$ & NO & 2093\\
$(201,59)$ & 13 & $(7,2)$ & 4 & 1 & YES & YES & NO(2) & $1.33$ & $(8,0)$ & NO & 2094\\
$(201,59)$ & 13 & $(92,27)$ & 11 & 1 & YES & YES & NO(2) & $1.25$ & $(6,1)$ & NO & 2095\\
$(202,59)$ & 12 & $(2,1)$ & 1 & 2 & YES & YES & NO(2) & $1.25$ & $(8,0)$ & -- & 2096\\
$(202,89)$ & 12 & $(3,1)$ & 2 & 1 & YES & YES & NO(2) & $1.14$ & $(6,1)$ & NO & 2097\\
$(202,89)$ & 12 & $(4,1)$ & 3 & 2 & YES & YES & NO(2) & $1.14$ & $(6,1)$ & -- & 2098\\
$(202,59)$ & 12 & $(5,2)$ & 3 & 1 & YES & YES & YES & $1.78$ & $(2,3)$ & NO & 2099\\
$(202,59)$ & 12 & $(5,2)$ & 3 & 1 & YES & YES & YES & $1.78$ & $(2,3)$ & -- & 2100\\
$(202,59)$ & 12 & $(17,5)$ & 6 & 1 & YES & YES & NO(2) & $1.12$ & $(8,0)$ & NO & 2101\\
$(202,53)$ & 13 & $(202,53)$ & 13 & 202 & YES & YES & NO(2) & $1.60$ & $(2,3)$ & NO & 2102\\
$(203,86)$ & 12 & $(4,1)$ & 3 & 1 & YES & YES & YES & $1.62$ & $(2,3)$ & NO & 2103\\
$(204,89)$ & 12 & $(2,1)$ & 1 & 2 & NO & YES & NO(2) & $1.44$ & $(4,2)$ & -- & 2104\\
$(204,89)$ & 12 & $(3,1)$ & 2 & 3 & YES & YES & NO(2) & $1.29$ & $(6,1)$ & NO & 2105\\
$(205,78)$ & 12 & $(205,78)$ & 12 & 205 & YES & YES & YES & $1.43$ & $(4,2)$ & NO & 2106\\
$(206,47)$ & 12 & $(83,19)$ & 10 & 1 & YES & YES & YES & $1.75$ & $(2,3)$ & NO & 2107\\
$(207,55)$ & 13 & $(2,1)$ & 1 & 1 & YES & YES & YES & $1.43$ & $(2,3)$ & NO & 2108\\
$(207,55)$ & 13 & $(3,1)$ & 2 & 3 & YES & YES & NO(2) & $1.14$ & $(6,1)$ & NO & 2109\\
$(207,37)$ & 15 & $(17,3)$ & 7 & 1 & YES & YES & NO(2) & $1.29$ & $(8,0)$ & NO & 2110\\
$(207,55)$ & 13 & $(207,55)$ & 13 & 207 & YES & YES & YES & $1.62$ & $(2,3)$ & NO & 2111\\
$(208,61)$ & 12 & $(9,2)$ & 5 & 1 & YES & YES & YES & $1.62$ & $(2,3)$ & NO & 2112\\
$(208,37)$ & 13 & $(39,7)$ & 9 & 13 & YES & YES & NO(2) & $1.25$ & $(8,0)$ & NO & 2113\\
$(209,82)$ & 12 & $(2,1)$ & 1 & 1 & NO & YES & NO(2) & $1.29$ & $(8,0)$ & -- & 2114\\
$(209,47)$ & 14 & $(4,1)$ & 3 & 1 & YES & YES & YES & $1.56$ & $(2,3)$ & NO & 2115\\
$(209,45)$ & 13 & $(5,2)$ & 3 & 1 & YES & YES & NO(2) & $1.14$ & $(6,1)$ & -- & 2116\\
$(209,56)$ & 12 & $(5,2)$ & 3 & 1 & YES & YES & NO(2) & $1.14$ & $(6,1)$ & NO & 2117\\
$(209,91)$ & 12 & $(5,2)$ & 3 & 1 & YES & YES & YES & $1.43$ & $(4,2)$ & NO & 2118\\
$(209,37)$ & 14 & $(6,1)$ & 5 & 1 & YES & YES & YES & $1.44$ & $(2,3)$ & NO & 2119\\
$(209,91)$ & 12 & $(9,4)$ & 5 & 1 & YES & YES & YES & $1.43$ & $(4,2)$ & NO & 2120\\
$(209,37)$ & 14 & $(13,2)$ & 7 & 1 & YES & YES & NO(2) & $1.44$ & $(4,2)$ & NO & 2121\\
$(209,37)$ & 14 & $(39,7)$ & 9 & 1 & YES & YES & NO(2) & $1.44$ & $(4,2)$ & NO & 2122\\
$(211,93)$ & 12 & $(9,4)$ & 5 & 1 & YES & YES & NO(2) & $1.56$ & $(4,2)$ & NO & 2123\\
$(211,50)$ & 14 & $(38,9)$ & 9 & 1 & YES & YES & YES & $1.62$ & $(2,3)$ & NO & 2124\\
$(211,50)$ & 14 & $(135,32)$ & 12 & 1 & YES & YES & YES & $1.50$ & $(2,3)$ & 2266 & 2125\\
$(213,38)$ & 15 & $(2,1)$ & 1 & 1 & YES & YES & NO(2) & $1.14$ & $(8,0)$ & -- & 2126\\
$(213,62)$ & 12 & $(9,2)$ & 5 & 3 & YES & YES & YES & $1.75$ & $(2,3)$ & NO & 2127\\
$(215,83)$ & 12 & $(3,1)$ & 2 & 1 & YES & YES & YES & $1.62$ & $(2,3)$ & NO & 2128\\
$(215,83)$ & 12 & $(3,1)$ & 2 & 1 & YES & YES & YES & $1.62$ & $(2,3)$ & -- & 2129\\
$(215,83)$ & 12 & $(4,1)$ & 3 & 1 & YES & YES & YES & $1.62$ & $(2,3)$ & NO & 2130\\
$(215,83)$ & 12 & $(18,7)$ & 6 & 1 & YES & YES & YES & $1.62$ & $(2,3)$ & NO & 2131\\
$(218,85)$ & 12 & $(4,1)$ & 3 & 2 & YES & YES & YES & $1.75$ & $(2,3)$ & NO & 2132\\
$(219,65)$ & 12 & $(5,2)$ & 3 & 1 & YES & YES & YES & $1.78$ & $(2,3)$ & -- & 2133\\
$(219,65)$ & 12 & $(11,3)$ & 5 & 1 & YES & YES & YES & $1.78$ & $(2,3)$ & 2292 & 2134\\
$(219,85)$ & 12 & $(18,7)$ & 6 & 3 & YES & YES & NO(2) & $1.14$ & $(6,1)$ & NO & 2135\\
$(221,58)$ & 13 & $(19,5)$ & 7 & 1 & YES & YES & YES & $1.50$ & $(2,3)$ & NO & 2136\\
$(222,59)$ & 13 & $(15,4)$ & 6 & 3 & YES & YES & YES & $1.62$ & $(2,3)$ & NO & 2137\\
$(222,85)$ & 12 & $(81,31)$ & 9 & 3 & YES & YES & YES & $1.89$ & $(2,3)$ & NO & 2138\\
$(223,98)$ & 12 & $(3,1)$ & 2 & 1 & YES & YES & YES & $1.43$ & $(4,2)$ & NO & 2139\\
$(223,98)$ & 12 & $(9,4)$ & 5 & 1 & YES & YES & YES & $1.43$ & $(4,2)$ & 1846 & 2140\\
$(225,98)$ & 12 & $(3,1)$ & 2 & 3 & YES & YES & YES & $1.43$ & $(4,2)$ & -- & 2141\\
$(229,95)$ & 12 & $(2,1)$ & 1 & 1 & YES & YES & NO(2) & $1.60$ & $(2,3)$ & NO & 2142\\
$(229,94)$ & 12 & $(3,1)$ & 2 & 1 & YES & YES & YES & $1.62$ & $(2,3)$ & -- & 2143\\
$(229,64)$ & 12 & $(5,2)$ & 3 & 1 & YES & YES & YES & $1.67$ & $(2,3)$ & NO & 2144\\
$(229,64)$ & 12 & $(5,2)$ & 3 & 1 & YES & YES & YES & $1.78$ & $(2,3)$ & -- & 2145\\
$(229,94)$ & 12 & $(229,94)$ & 12 & 229 & YES & YES & YES & $1.75$ & $(2,3)$ & NO & 2146\\
$(231,83)$ & 12 & $(2,1)$ & 1 & 1 & YES & YES & YES & $1.43$ & $(4,2)$ & -- & 2147\\
$(231,83)$ & 12 & $(3,1)$ & 2 & 3 & YES & YES & YES & $1.43$ & $(4,2)$ & -- & 2148\\
$(231,83)$ & 12 & $(39,14)$ & 8 & 3 & YES & YES & YES & $1.43$ & $(4,2)$ & NO & 2149\\
$(234,43)$ & 14 & $(6,1)$ & 5 & 6 & YES & YES & YES & $1.29$ & $(2,3)$ & NO & 2150\\
$(237,100)$ & 12 & $(3,1)$ & 2 & 3 & YES & YES & YES & $1.75$ & $(2,3)$ & -- & 2151\\
$(239,32)$ & 17 & $(2,1)$ & 1 & 1 & YES & YES & YES & $1.38$ & $(4,2)$ & NO & 2152\\
$(239,101)$ & 12 & $(2,1)$ & 1 & 1 & YES & YES & YES & $1.43$ & $(4,2)$ & NO & 2153\\
$(239,50)$ & 14 & $(5,1)$ & 4 & 1 & YES & YES & YES & $1.50$ & $(2,3)$ & NO & 2154\\
$(241,63)$ & 13 & $(3,1)$ & 2 & 1 & YES & YES & NO(2) & $1.29$ & $(8,0)$ & NO & 2155\\
$(241,89)$ & 12 & $(3,1)$ & 2 & 1 & YES & YES & NO(2) & $1.29$ & $(6,1)$ & NO & 2156\\
$(241,46)$ & 15 & $(4,1)$ & 3 & 1 & YES & YES & YES & $1.50$ & $(2,3)$ & -- & 2157\\
$(242,71)$ & 13 & $(3,1)$ & 2 & 1 & YES & YES & YES & $1.43$ & $(4,2)$ & -- & 2158\\
$(242,71)$ & 13 & $(5,1)$ & 4 & 1 & YES & YES & YES & $1.43$ & $(4,2)$ & NO & 2159\\
$(243,38)$ & 16 & $(2,1)$ & 1 & 1 & YES & YES & NO(2) & $1.29$ & $(8,0)$ & NO & 2160\\
$(243,46)$ & 15 & $(5,1)$ & 4 & 1 & YES & YES & YES & $1.50$ & $(2,3)$ & NO & 2161\\
$(243,38)$ & 16 & $(13,2)$ & 7 & 1 & YES & YES & NO(2) & $1.29$ & $(8,0)$ & NO & 2162\\
$(244,55)$ & 13 & $(5,2)$ & 3 & 1 & YES & YES & YES & $1.75$ & $(2,3)$ & NO & 2163\\
$(245,69)$ & 13 & $(5,1)$ & 4 & 5 & YES & YES & YES & $1.43$ & $(4,2)$ & NO & 2164\\
$(245,69)$ & 13 & $(103,29)$ & 11 & 1 & YES & YES & YES & $1.43$ & $(4,2)$ & 2213 & 2165\\
$(246,95)$ & 12 & $(8,3)$ & 4 & 2 & YES & YES & YES & $1.62$ & $(2,3)$ & NO & 2166\\
$(247,56)$ & 13 & $(5,2)$ & 3 & 1 & YES & YES & YES & $1.62$ & $(2,3)$ & NO & 2167\\
$(253,106)$ & 12 & $(2,1)$ & 1 & 1 & YES & YES & YES & $1.43$ & $(4,2)$ & -- & 2168\\
$(253,57)$ & 13 & $(5,1)$ & 4 & 1 & YES & YES & NO(2) & $1.25$ & $(8,0)$ & NO & 2169\\
$(253,57)$ & 13 & $(40,9)$ & 9 & 1 & YES & YES & NO(2) & $1.25$ & $(8,0)$ & NO & 2170\\
$(255,76)$ & 13 & $(2,1)$ & 1 & 1 & YES & YES & YES & $1.29$ & $(4,2)$ & NO & 2171\\
$(255,97)$ & 12 & $(163,62)$ & 11 & 1 & YES & YES & YES & $1.62$ & $(2,3)$ & NO & 2172\\
$(256,99)$ & 12 & $(3,1)$ & 2 & 1 & YES & YES & YES & $1.90$ & $(2,3)$ & NO & 2173\\
$(256,99)$ & 12 & $(3,1)$ & 2 & 1 & YES & YES & YES & $1.90$ & $(2,3)$ & -- & 2174\\
$(256,99)$ & 12 & $(4,1)$ & 3 & 4 & YES & YES & YES & $1.78$ & $(2,3)$ & -- & 2175\\
$(256,99)$ & 12 & $(4,1)$ & 3 & 4 & YES & YES & YES & $1.78$ & $(2,3)$ & NO & 2176\\
$(256,97)$ & 12 & $(5,2)$ & 3 & 1 & YES & YES & YES & $1.75$ & $(2,3)$ & NO & 2177\\
$(256,99)$ & 12 & $(106,41)$ & 10 & 2 & YES & YES & YES & $1.78$ & $(2,3)$ & 2238 & 2178\\
$(256,99)$ & 12 & $(181,70)$ & 11 & 1 & YES & YES & YES & $1.67$ & $(2,3)$ & NO & 2179\\
$(256,99)$ & 12 & $(256,99)$ & 12 & 256 & YES & YES & YES & $1.78$ & $(2,3)$ & NO & 2180\\
$(257,45)$ & 15 & $(3,1)$ & 2 & 1 & YES & YES & NO(2) & $1.60$ & $(2,3)$ & -- & 2181\\
$(258,109)$ & 12 & $(3,1)$ & 2 & 3 & YES & YES & YES & $1.89$ & $(2,3)$ & NO & 2182\\
$(258,109)$ & 12 & $(3,1)$ & 2 & 3 & YES & YES & YES & $1.89$ & $(2,3)$ & -- & 2183\\
$(258,109)$ & 12 & $(45,19)$ & 8 & 3 & YES & YES & YES & $1.89$ & $(2,3)$ & NO & 2184\\
$(259,76)$ & 13 & $(2,1)$ & 1 & 1 & YES & YES & YES & $1.29$ & $(4,2)$ & NO & 2185\\
$(259,59)$ & 13 & $(5,1)$ & 4 & 1 & YES & YES & NO(2) & $1.25$ & $(8,0)$ & NO & 2186\\
$(261,50)$ & 15 & $(5,1)$ & 4 & 1 & YES & YES & YES & $1.50$ & $(2,3)$ & NO & 2187\\
$(263,100)$ & 12 & $(4,1)$ & 3 & 1 & YES & YES & YES & $1.75$ & $(2,3)$ & -- & 2188\\
$(263,109)$ & 12 & $(4,1)$ & 3 & 1 & YES & YES & YES & $1.62$ & $(2,3)$ & NO & 2189\\
$(263,100)$ & 12 & $(6,1)$ & 5 & 1 & YES & YES & YES & $1.75$ & $(2,3)$ & NO & 2190\\
$(263,100)$ & 12 & $(6,1)$ & 5 & 1 & YES & YES & YES & $1.75$ & $(2,3)$ & -- & 2191\\
$(263,100)$ & 12 & $(6,1)$ & 5 & 1 & YES & YES & YES & $1.88$ & $(2,3)$ & NO & 2192\\
$(263,111)$ & 12 & $(7,3)$ & 4 & 1 & YES & YES & YES & $1.62$ & $(2,3)$ & NO & 2193\\
$(263,109)$ & 12 & $(17,7)$ & 6 & 1 & YES & YES & YES & $1.62$ & $(2,3)$ & NO & 2194\\
$(265,41)$ & 16 & $(2,1)$ & 1 & 1 & YES & YES & YES & $1.56$ & $(2,3)$ & NO & 2195\\
$(267,98)$ & 12 & $(3,1)$ & 2 & 3 & YES & YES & YES & $1.75$ & $(2,3)$ & NO & 2196\\
$(267,98)$ & 12 & $(3,1)$ & 2 & 3 & YES & YES & YES & $1.75$ & $(2,3)$ & -- & 2197\\
$(267,98)$ & 12 & $(8,3)$ & 4 & 1 & YES & YES & YES & $1.75$ & $(2,3)$ & NO & 2198\\
$(268,111)$ & 12 & $(99,41)$ & 10 & 1 & YES & YES & YES & $1.62$ & $(2,3)$ & NO & 2199\\
$(269,78)$ & 13 & $(2,1)$ & 1 & 1 & YES & YES & YES & $1.43$ & $(4,2)$ & NO & 2200\\
$(269,104)$ & 12 & $(44,17)$ & 8 & 1 & YES & YES & YES & $1.88$ & $(2,3)$ & NO & 2201\\
$(271,48)$ & 14 & $(3,1)$ & 2 & 1 & YES & YES & NO(2) & $1.25$ & $(6,1)$ & NO & 2202\\
$(273,76)$ & 13 & $(5,1)$ & 4 & 1 & YES & YES & YES & $1.43$ & $(4,2)$ & NO & 2203\\
$(274,115)$ & 12 & $(2,1)$ & 1 & 2 & YES & YES & YES & $1.89$ & $(2,3)$ & -- & 2204\\
$(274,81)$ & 12 & $(13,4)$ & 6 & 1 & YES & YES & YES & $1.62$ & $(2,3)$ & NO & 2205\\
$(274,43)$ & 15 & $(20,3)$ & 8 & 2 & YES & YES & NO(2) & $1.14$ & $(6,1)$ & NO & 2206\\
$(277,78)$ & 13 & $(2,1)$ & 1 & 1 & YES & YES & YES & $1.43$ & $(4,2)$ & -- & 2207\\
$(277,106)$ & 12 & $(2,1)$ & 1 & 1 & YES & YES & YES & $1.75$ & $(2,3)$ & NO & 2208\\
$(277,106)$ & 12 & $(2,1)$ & 1 & 1 & YES & YES & YES & $1.75$ & $(2,3)$ & -- & 2209\\
$(277,106)$ & 12 & $(8,3)$ & 4 & 1 & YES & YES & YES & $1.75$ & $(2,3)$ & NO & 2210\\
$(277,106)$ & 12 & $(13,5)$ & 5 & 1 & YES & YES & YES & $1.75$ & $(2,3)$ & NO & 2211\\
$(277,117)$ & 12 & $(19,8)$ & 6 & 1 & YES & YES & YES & $1.78$ & $(2,3)$ & NO & 2212\\
$(277,78)$ & 13 & $(71,20)$ & 10 & 1 & YES & YES & YES & $1.43$ & $(4,2)$ & 2165 & 2213\\
$(281,109)$ & 12 & $(2,1)$ & 1 & 1 & YES & YES & YES & $1.80$ & $(2,3)$ & -- & 2214\\
$(281,109)$ & 12 & $(13,5)$ & 5 & 1 & YES & YES & YES & $1.67$ & $(2,3)$ & NO & 2215\\
$(281,109)$ & 12 & $(116,45)$ & 10 & 1 & YES & YES & YES & $1.67$ & $(2,3)$ & NO & 2216\\
$(282,109)$ & 12 & $(2,1)$ & 1 & 2 & YES & YES & YES & $1.67$ & $(2,3)$ & -- & 2217\\
$(282,109)$ & 12 & $(4,1)$ & 3 & 2 & YES & YES & YES & $1.67$ & $(2,3)$ & -- & 2218\\
$(282,109)$ & 12 & $(13,5)$ & 5 & 1 & YES & YES & YES & $1.50$ & $(2,3)$ & NO & 2219\\
$(282,109)$ & 12 & $(31,12)$ & 7 & 1 & YES & YES & YES & $1.67$ & $(2,3)$ & 1991 & 2220\\
$(282,109)$ & 12 & $(119,46)$ & 10 & 1 & YES & YES & YES & $1.67$ & $(2,3)$ & NO & 2221\\
$(283,83)$ & 13 & $(2,1)$ & 1 & 1 & YES & YES & YES & $1.62$ & $(2,3)$ & -- & 2222\\
$(283,83)$ & 13 & $(2,1)$ & 1 & 1 & YES & YES & YES & $1.62$ & $(2,3)$ & NO & 2223\\
$(283,83)$ & 13 & $(4,1)$ & 3 & 1 & YES & YES & YES & $1.62$ & $(2,3)$ & NO & 2224\\
$(283,108)$ & 12 & $(6,1)$ & 5 & 1 & YES & YES & YES & $1.75$ & $(2,3)$ & NO & 2225\\
$(283,108)$ & 12 & $(6,1)$ & 5 & 1 & YES & YES & YES & $1.75$ & $(2,3)$ & -- & 2226\\
$(283,108)$ & 12 & $(6,1)$ & 5 & 1 & YES & YES & YES & $1.88$ & $(2,3)$ & NO & 2227\\
$(283,75)$ & 13 & $(49,13)$ & 9 & 1 & YES & YES & YES & $1.43$ & $(4,2)$ & NO & 2228\\
$(283,108)$ & 12 & $(131,50)$ & 10 & 1 & YES & YES & YES & $1.75$ & $(2,3)$ & 2286 & 2229\\
$(283,83)$ & 13 & $(283,83)$ & 13 & 283 & YES & YES & YES & $1.62$ & $(2,3)$ & NO & 2230\\
$(286,105)$ & 12 & $(11,4)$ & 5 & 11 & YES & YES & YES & $1.89$ & $(2,3)$ & NO & 2231\\
$(287,111)$ & 12 & $(2,1)$ & 1 & 1 & YES & YES & YES & $1.67$ & $(2,3)$ & -- & 2232\\
$(287,109)$ & 12 & $(3,1)$ & 2 & 1 & YES & YES & YES & $1.89$ & $(2,3)$ & -- & 2233\\
$(287,109)$ & 12 & $(3,1)$ & 2 & 1 & YES & YES & YES & $1.89$ & $(2,3)$ & NO & 2234\\
$(287,106)$ & 12 & $(5,1)$ & 4 & 1 & YES & YES & YES & $1.62$ & $(2,3)$ & -- & 2235\\
$(287,111)$ & 12 & $(5,1)$ & 4 & 1 & YES & YES & YES & $1.78$ & $(2,3)$ & NO & 2236\\
$(287,111)$ & 12 & $(5,1)$ & 4 & 1 & YES & YES & YES & $1.78$ & $(2,3)$ & -- & 2237\\
$(287,111)$ & 12 & $(75,29)$ & 9 & 1 & YES & YES & YES & $1.78$ & $(2,3)$ & 2178 & 2238\\
$(287,109)$ & 12 & $(79,30)$ & 9 & 1 & YES & YES & YES & $1.75$ & $(2,3)$ & NO & 2239\\
$(287,111)$ & 12 & $(106,41)$ & 10 & 1 & YES & YES & YES & $1.78$ & $(2,3)$ & NO & 2240\\
$(288,85)$ & 13 & $(4,1)$ & 3 & 4 & YES & YES & YES & $1.75$ & $(2,3)$ & -- & 2241\\
$(288,85)$ & 13 & $(44,13)$ & 8 & 4 & YES & YES & YES & $1.89$ & $(2,3)$ & NO & 2242\\
$(288,119)$ & 12 & $(121,50)$ & 10 & 1 & YES & YES & YES & $1.75$ & $(2,3)$ & NO & 2243\\
$(289,112)$ & 12 & $(31,12)$ & 7 & 1 & YES & YES & YES & $1.89$ & $(2,3)$ & NO & 2244\\
$(289,112)$ & 12 & $(49,19)$ & 8 & 1 & YES & YES & YES & $1.88$ & $(2,3)$ & NO & 2245\\
$(291,85)$ & 13 & $(10,3)$ & 5 & 1 & YES & YES & YES & $1.62$ & $(2,3)$ & NO & 2246\\
$(292,111)$ & 12 & $(2,1)$ & 1 & 2 & YES & YES & YES & $1.62$ & $(2,3)$ & -- & 2247\\
$(292,85)$ & 13 & $(3,1)$ & 2 & 1 & YES & YES & YES & $1.78$ & $(2,3)$ & -- & 2248\\
$(292,121)$ & 12 & $(3,1)$ & 2 & 1 & YES & YES & YES & $1.62$ & $(2,3)$ & -- & 2249\\
$(292,85)$ & 13 & $(4,1)$ & 3 & 4 & YES & YES & YES & $1.75$ & $(2,3)$ & NO & 2250\\
$(292,111)$ & 12 & $(5,2)$ & 3 & 1 & YES & YES & YES & $1.67$ & $(2,3)$ & NO & 2251\\
$(292,111)$ & 12 & $(121,46)$ & 10 & 1 & YES & YES & YES & $1.62$ & $(2,3)$ & NO & 2252\\
$(292,85)$ & 13 & $(134,39)$ & 11 & 2 & YES & YES & YES & $1.62$ & $(2,3)$ & 2288 & 2253\\
$(298,83)$ & 13 & $(3,1)$ & 2 & 1 & YES & YES & YES & $1.62$ & $(2,3)$ & NO & 2254\\
$(298,79)$ & 13 & $(49,13)$ & 9 & 1 & YES & YES & YES & $1.43$ & $(4,2)$ & NO & 2255\\
$(301,65)$ & 13 & $(5,2)$ & 3 & 1 & YES & YES & YES & $1.67$ & $(2,3)$ & -- & 2256\\
$(301,65)$ & 13 & $(5,2)$ & 3 & 1 & YES & YES & YES & $1.78$ & $(2,3)$ & NO & 2257\\
$(301,115)$ & 12 & $(5,2)$ & 3 & 1 & YES & YES & YES & $1.88$ & $(2,3)$ & NO & 2258\\
$(301,65)$ & 13 & $(13,3)$ & 6 & 1 & YES & YES & YES & $1.67$ & $(2,3)$ & 2301 & 2259\\
$(301,115)$ & 12 & $(13,5)$ & 5 & 1 & YES & YES & YES & $1.75$ & $(2,3)$ & 2036 & 2260\\
$(303,128)$ & 12 & $(5,2)$ & 3 & 1 & YES & YES & YES & $1.78$ & $(2,3)$ & 2059 & 2261\\
$(303,128)$ & 12 & $(19,8)$ & 6 & 1 & YES & YES & YES & $1.78$ & $(2,3)$ & NO & 2262\\
$(307,85)$ & 13 & $(4,1)$ & 3 & 1 & YES & YES & YES & $1.88$ & $(2,3)$ & -- & 2263\\
$(307,69)$ & 14 & $(5,1)$ & 4 & 1 & YES & YES & NO(2) & $1.14$ & $(6,1)$ & NO & 2264\\
$(307,69)$ & 14 & $(89,20)$ & 11 & 1 & YES & YES & NO(2) & $1.44$ & $(4,2)$ & NO & 2265\\
$(308,73)$ & 14 & $(38,9)$ & 9 & 2 & YES & YES & YES & $1.50$ & $(2,3)$ & 2125 & 2266\\
$(309,59)$ & 15 & $(4,1)$ & 3 & 1 & YES & YES & YES & $1.50$ & $(2,3)$ & NO & 2267\\
$(313,71)$ & 14 & $(2,1)$ & 1 & 1 & YES & YES & NO(2) & $1.44$ & $(4,2)$ & -- & 2268\\
$(313,71)$ & 14 & $(3,1)$ & 2 & 1 & YES & YES & YES & $1.43$ & $(4,2)$ & NO & 2269\\
$(313,121)$ & 12 & $(3,1)$ & 2 & 1 & YES & YES & YES & $1.62$ & $(2,3)$ & -- & 2270\\
$(313,86)$ & 13 & $(4,1)$ & 3 & 1 & YES & YES & YES & $1.62$ & $(2,3)$ & NO & 2271\\
$(313,71)$ & 14 & $(5,1)$ & 4 & 1 & YES & YES & NO(2) & $1.14$ & $(6,1)$ & NO & 2272\\
$(313,86)$ & 13 & $(5,1)$ & 4 & 1 & YES & YES & YES & $1.62$ & $(2,3)$ & NO & 2273\\
$(315,88)$ & 13 & $(2,1)$ & 1 & 1 & YES & YES & YES & $1.75$ & $(2,3)$ & NO & 2274\\
$(321,95)$ & 13 & $(2,1)$ & 1 & 1 & YES & YES & YES & $1.62$ & $(2,3)$ & -- & 2275\\
$(321,95)$ & 13 & $(17,5)$ & 6 & 1 & YES & YES & YES & $1.62$ & $(2,3)$ & 1840 & 2276\\
$(323,94)$ & 13 & $(2,1)$ & 1 & 1 & YES & YES & YES & $1.62$ & $(2,3)$ & -- & 2277\\
$(323,89)$ & 13 & $(5,1)$ & 4 & 1 & YES & YES & YES & $1.62$ & $(2,3)$ & NO & 2278\\
$(323,94)$ & 13 & $(134,39)$ & 11 & 1 & YES & YES & YES & $1.62$ & $(2,3)$ & NO & 2279\\
$(325,74)$ & 14 & $(2,1)$ & 1 & 1 & YES & YES & YES & $1.43$ & $(4,2)$ & -- & 2280\\
$(326,97)$ & 13 & $(121,36)$ & 11 & 1 & YES & YES & YES & $1.78$ & $(2,3)$ & NO & 2281\\
$(327,97)$ & 13 & $(5,1)$ & 4 & 1 & YES & YES & YES & $1.62$ & $(2,3)$ & NO & 2282\\
$(335,94)$ & 13 & $(2,1)$ & 1 & 1 & YES & YES & YES & $1.62$ & $(2,3)$ & NO & 2283\\
$(335,94)$ & 13 & $(3,1)$ & 2 & 1 & YES & YES & YES & $1.62$ & $(2,3)$ & -- & 2284\\
$(338,99)$ & 13 & $(2,1)$ & 1 & 2 & YES & YES & YES & $1.78$ & $(2,3)$ & NO & 2285\\
$(338,129)$ & 12 & $(76,29)$ & 9 & 2 & YES & YES & YES & $1.75$ & $(2,3)$ & 2229 & 2286\\
$(347,101)$ & 13 & $(4,1)$ & 3 & 1 & YES & YES & YES & $1.75$ & $(2,3)$ & NO & 2287\\
$(347,101)$ & 13 & $(79,23)$ & 10 & 1 & YES & YES & YES & $1.62$ & $(2,3)$ & 2253 & 2288\\
$(353,97)$ & 13 & $(2,1)$ & 1 & 1 & YES & YES & YES & $1.62$ & $(2,3)$ & NO & 2289\\
$(353,97)$ & 13 & $(2,1)$ & 1 & 1 & YES & YES & YES & $1.62$ & $(2,3)$ & -- & 2290\\
$(353,97)$ & 13 & $(7,2)$ & 4 & 1 & YES & YES & YES & $1.75$ & $(2,3)$ & NO & 2291\\
$(355,99)$ & 13 & $(3,1)$ & 2 & 1 & YES & YES & YES & $1.78$ & $(2,3)$ & 2134 & 2292\\
$(359,100)$ & 13 & $(5,1)$ & 4 & 1 & YES & YES & YES & $1.50$ & $(2,3)$ & -- & 2293\\
$(360,101)$ & 13 & $(2,1)$ & 1 & 2 & YES & YES & YES & $1.62$ & $(2,3)$ & NO & 2294\\
$(360,101)$ & 13 & $(2,1)$ & 1 & 2 & YES & YES & YES & $1.62$ & $(2,3)$ & -- & 2295\\
$(377,85)$ & 14 & $(3,1)$ & 2 & 1 & YES & YES & YES & $1.78$ & $(2,3)$ & NO & 2296\\
$(377,85)$ & 14 & $(71,16)$ & 10 & 1 & YES & YES & YES & $1.62$ & $(2,3)$ & NO & 2297\\
$(395,73)$ & 15 & $(3,1)$ & 2 & 1 & YES & YES & YES & $1.62$ & $(2,3)$ & -- & 2298\\
$(407,171)$ & 13 & $(2,1)$ & 1 & 1 & NO & YES & YES & $1.75$ & $(2,3)$ & -- & 2299\\
$(424,97)$ & 14 & $(3,1)$ & 2 & 1 & YES & YES & YES & $1.62$ & $(2,3)$ & NO & 2300\\
$(437,99)$ & 14 & $(5,1)$ & 4 & 1 & YES & YES & YES & $1.67$ & $(2,3)$ & 2259 & 2301\\
$(451,84)$ & 15 & $(11,2)$ & 6 & 11 & YES & YES & YES & $1.62$ & $(2,3)$ & NO & 2302\\
$(451,84)$ & 15 & $(27,5)$ & 8 & 1 & YES & YES & YES & $1.62$ & $(2,3)$ & NO & 2303\\
$(495,92)$ & 15 & $(2,1)$ & 1 & 1 & YES & YES & YES & $1.62$ & $(2,3)$ & -- & 2304\\
$(495,92)$ & 15 & $(2,1)$ & 1 & 1 & YES & YES & YES & $1.75$ & $(2,3)$ & NO & 2305\\
$(a;0,0,0;3)$ & 4 & $(81,31)$ & 9 & 3 & YES & YES & YES & $1.75$ & $(2,3)$ & -- & 2306\\
$(a;0,0,0;3)$ & 4 & $(89,34)$ & 9 & 1 & YES & YES & YES & $1.75$ & $(2,3)$ & -- & 2307\\
$(a;1,0,0;13)$ & 5 & $(33,14)$ & 8 & 1 & YES & YES & NO(2) & $1.70$ & $(2,3)$ & -- & 2308\\
$(a;1,0,0;13)$ & 5 & $(36,13)$ & 8 & 1 & YES & YES & NO(2) & $1.14$ & $(6,1)$ & -- & 2309\\
$(a;1,0,0;13)$ & 5 & $(55,21)$ & 8 & 1 & YES & YES & YES & $1.89$ & $(2,3)$ & -- & 2310\\
$(a;1,1,0;19)$ & 6 & $(19,6)$ & 8 & 19 & YES & YES & YES & $1.29$ & $(4,2)$ & -- & 2311\\
$(a;1,1,0;19)$ & 6 & $(35,11)$ & 9 & 1 & YES & YES & NO(2) & $1.56$ & $(4,2)$ & -- & 2312\\
$(a;2,1,1;37)$ & 8 & $(12,5)$ & 5 & 1 & YES & YES & YES & $1.62$ & $(2,3)$ & -- & 2313\\
$(a;3,1,0;31)$ & 8 & $(13,4)$ & 6 & 1 & YES & YES & NO(2) & $1.70$ & $(2,3)$ & -- & 2314\\
$(a;4,0,0;25)$ & 8 & $(7,3)$ & 4 & 1 & YES & YES & YES & $1.29$ & $(4,2)$ & -- & 2315\\
$(a;4,0,0;25)$ & 8 & $(8,3)$ & 4 & 1 & YES & YES & YES & $1.14$ & $(4,2)$ & -- & 2316\\
$(a;4,0,0;25)$ & 8 & $(16,3)$ & 7 & 1 & YES & YES & YES & $1.14$ & $(4,2)$ & -- & 2317\\
$(a;4,0,0;25)$ & 8 & $(17,3)$ & 7 & 1 & YES & YES & YES & $1.29$ & $(4,2)$ & -- & 2318\\
$(b;0,0,3;32)$ & 8 & $(5,2)$ & 3 & 1 & YES & YES & NO(2) & $1.25$ & $(6,1)$ & -- & 2319\\
$(b;0,1,0;19)$ & 6 & $(31,12)$ & 7 & 1 & YES & YES & YES & $1.70$ & $(2,3)$ & -- & 2320\\
$(b;0,2,1;34)$ & 8 & $(12,5)$ & 5 & 2 & YES & YES & YES & $1.62$ & $(2,3)$ & -- & 2321\\
$(b;0,3,0;29)$ & 8 & $(8,3)$ & 4 & 1 & YES & YES & NO(2) & $1.64$ & $(2,3)$ & -- & 2322\\
$(b;0,3,2;53)$ & 10 & $(4,1)$ & 3 & 1 & YES & YES & YES & $1.38$ & $(2,3)$ & -- & 2323\\
$(b;1,1,0;27)$ & 7 & $(12,5)$ & 5 & 3 & YES & YES & NO(2) & $1.33$ & $(4,2)$ & -- & 2324\\
$(b;1,2,0;17)$ & 8 & $(22,5)$ & 7 & 1 & YES & YES & YES & $1.62$ & $(2,3)$ & -- & 2325\\
$(b;1,2,1;7)$ & 9 & $(11,3)$ & 5 & 1 & YES & YES & YES & $1.78$ & $(2,3)$ & -- & 2326\\
$(b;2,0,0;26)$ & 7 & $(9,4)$ & 5 & 1 & YES & YES & YES & $1.38$ & $(2,3)$ & -- & 2327\\
$(b;2,0,0;26)$ & 7 & $(18,7)$ & 6 & 2 & YES & YES & YES & $1.43$ & $(4,2)$ & -- & 2328\\
$(b;2,1,0;7)$ & 8 & $(5,2)$ & 3 & 1 & YES & YES & NO(2) & $1.25$ & $(6,1)$ & -- & 2329\\
$(b;2,2,0;44)$ & 9 & $(3,1)$ & 2 & 1 & YES & YES & YES & $1.25$ & $(2,3)$ & -- & 2330\\
$(b;2,3,0;53)$ & 10 & $(3,1)$ & 2 & 1 & YES & YES & YES & $1.50$ & $(2,3)$ & -- & 2331\\
$(b;3,0,0;16)$ & 8 & $(5,2)$ & 3 & 1 & YES & YES & NO(2) & $1.44$ & $(2,3)$ & -- & 2332\\
$(b;3,0,0;16)$ & 8 & $(7,3)$ & 4 & 1 & YES & YES & YES & $1.14$ & $(4,2)$ & -- & 2333\\
$(b;3,0,0;16)$ & 8 & $(16,5)$ & 7 & 16 & YES & YES & NO(2) & $1.60$ & $(6,1)$ & -- & 2334\\
$(b;3,0,3;11)$ & 11 & $(5,1)$ & 4 & 1 & YES & YES & YES & $1.50$ & $(2,3)$ & -- & 2335\\
$(b;3,1,1;63)$ & 10 & $(3,1)$ & 2 & 3 & YES & YES & NO(2) & $1.00$ & $(10,-1)$ & -- & 2336\\
$(b;3,1,1;63)$ & 10 & $(4,1)$ & 3 & 1 & YES & YES & YES & $1.50$ & $(2,3)$ & -- & 2337\\
$(c;0,0,0;4)$ & 4 & $(34,15)$ & 8 & 2 & YES & YES & NO(2) & $1.00$ & $(8,0)$ & -- & 2338\\
$(c;0,0,0;4)$ & 4 & $(49,19)$ & 8 & 1 & YES & YES & NO(2) & $1.44$ & $(4,2)$ & -- & 2339\\
$(c;0,0,0;4)$ & 4 & $(61,25)$ & 9 & 1 & YES & YES & NO(2) & $1.33$ & $(4,2)$ & -- & 2340\\
$(c;0,0,0;4)$ & 4 & $(95,36)$ & 10 & 1 & YES & YES & YES & $1.89$ & $(2,3)$ & -- & 2341\\
$(c;0,1,0;11)$ & 5 & $(51,16)$ & 10 & 1 & YES & YES & YES & $1.57$ & $(2,3)$ & -- & 2342\\
$(c;0,1,0;11)$ & 5 & $(61,16)$ & 10 & 1 & YES & YES & YES & $1.57$ & $(2,3)$ & -- & 2343\\
$(c;0,1,0;11)$ & 5 & $(89,24)$ & 10 & 1 & YES & YES & YES & $1.43$ & $(4,2)$ & -- & 2344\\
$(c;0,1,1;5)$ & 6 & $(41,16)$ & 8 & 1 & YES & YES & YES & $1.62$ & $(2,3)$ & -- & 2345\\
$(c;0,1,1;5)$ & 6 & $(61,17)$ & 9 & 1 & YES & YES & YES & $1.67$ & $(2,3)$ & -- & 2346\\
$(c;0,2,0;7)$ & 6 & $(12,5)$ & 5 & 1 & YES & YES & NO(2) & $1.40$ & $(2,3)$ & -- & 2347\\
$(c;0,2,0;7)$ & 6 & $(29,9)$ & 8 & 1 & YES & YES & YES & $1.56$ & $(2,3)$ & -- & 2348\\
$(c;0,2,0;7)$ & 6 & $(36,11)$ & 8 & 1 & YES & YES & NO(2) & $1.38$ & $(6,1)$ & -- & 2349\\
$(c;0,2,0;7)$ & 6 & $(43,9)$ & 9 & 1 & YES & YES & YES & $1.56$ & $(2,3)$ & -- & 2350\\
$(c;0,2,0;7)$ & 6 & $(52,11)$ & 9 & 1 & YES & YES & NO(2) & $1.38$ & $(6,1)$ & -- & 2351\\
$(c;0,2,1;19)$ & 7 & $(27,8)$ & 7 & 1 & YES & YES & NO(2) & $1.50$ & $(2,3)$ & -- & 2352\\
$(c;0,3,0;17)$ & 7 & $(7,3)$ & 4 & 1 & YES & YES & NO(2) & $1.50$ & $(2,3)$ & -- & 2353\\
$(c;0,3,0;17)$ & 7 & $(19,8)$ & 6 & 1 & YES & YES & NO(2) & $1.25$ & $(6,1)$ & -- & 2354\\
$(c;0,3,1;23)$ & 8 & $(25,7)$ & 7 & 1 & YES & YES & YES & $1.29$ & $(4,2)$ & -- & 2355\\
$(c;0,3,1;23)$ & 8 & $(32,7)$ & 8 & 1 & YES & YES & YES & $1.43$ & $(4,2)$ & -- & 2356\\
$(c;0,3,2;29)$ & 9 & $(7,2)$ & 4 & 1 & YES & YES & NO(2) & $1.56$ & $(2,3)$ & -- & 2357\\
$(c;0,3,3;7)$ & 10 & $(9,2)$ & 5 & 1 & YES & YES & YES & $1.38$ & $(2,3)$ & -- & 2358\\
$(c;0,4,0;10)$ & 8 & $(10,3)$ & 5 & 10 & YES & YES & YES & $1.50$ & $(2,3)$ & -- & 2359\\
$(c;0,4,2;17)$ & 10 & $(11,2)$ & 6 & 1 & YES & YES & YES & $1.44$ & $(2,3)$ & -- & 2360\\
$(d;0,0,0;5)$ & 5 & $(64,27)$ & 9 & 1 & YES & YES & YES & $1.75$ & $(2,3)$ & -- & 2361\\
$(d;0,0,0;5)$ & 5 & $(65,24)$ & 9 & 5 & YES & YES & YES & $1.75$ & $(2,3)$ & -- & 2362\\
$(d;0,0,0;5)$ & 5 & $(79,24)$ & 10 & 1 & YES & YES & YES & $1.75$ & $(2,3)$ & -- & 2363\\
$(d;0,0,1;14)$ & 6 & $(44,17)$ & 8 & 2 & YES & YES & YES & $1.75$ & $(2,3)$ & -- & 2364\\
$(d;0,0,2;9)$ & 7 & $(37,11)$ & 8 & 1 & YES & YES & YES & $1.62$ & $(2,3)$ & -- & 2365\\
$(d;0,0,3;22)$ & 8 & $(9,2)$ & 5 & 1 & YES & YES & NO(2) & $1.50$ & $(4,2)$ & -- & 2366\\
$(d;0,1,1;17)$ & 7 & $(37,11)$ & 8 & 1 & YES & YES & YES & $1.75$ & $(2,3)$ & -- & 2367\\
$(d;0,2,0;7)$ & 7 & $(7,3)$ & 4 & 7 & YES & YES & NO(2) & $1.50$ & $(2,3)$ & -- & 2368\\
$(e;0,1,0;5)$ & 6 & $(31,12)$ & 7 & 1 & YES & YES & YES & $1.70$ & $(2,3)$ & -- & 2369\\
$(e;0,3,0;7)$ & 8 & $(8,3)$ & 4 & 1 & YES & YES & NO(2) & $1.64$ & $(2,3)$ & -- & 2370\\
$(e;1,1,0;23)$ & 7 & $(17,7)$ & 6 & 1 & YES & YES & NO(2) & $1.14$ & $(6,1)$ & -- & 2371\\
$(e;1,2,0;28)$ & 8 & $(18,5)$ & 6 & 2 & YES & YES & YES & $1.62$ & $(2,3)$ & -- & 2372\\
$(e;2,3,0;45)$ & 10 & $(4,1)$ & 3 & 1 & YES & YES & YES & $1.38$ & $(2,3)$ & -- & 2373\\
$(e;3,0,0;10)$ & 8 & $(9,4)$ & 5 & 1 & YES & YES & NO(2) & $1.44$ & $(4,2)$ & -- & 2374\\
$(f;0,0,0;6)$ & 4 & $(29,9)$ & 8 & 1 & YES & YES & NO(2) & $1.40$ & $(2,3)$ & -- & 2375\\
$(f;0,0,0;6)$ & 4 & $(43,16)$ & 9 & 1 & YES & YES & YES & $1.50$ & $(2,3)$ & -- & 2376\\
$(f;0,0,0;6)$ & 4 & $(47,20)$ & 10 & 1 & YES & YES & YES & $1.50$ & $(2,3)$ & -- & 2377\\
$(f;0,0,0;6)$ & 4 & $(55,16)$ & 9 & 1 & YES & YES & NO(2) & $1.56$ & $(4,2)$ & -- & 2378\\
$(f;0,0,0;6)$ & 4 & $(57,16)$ & 9 & 3 & YES & YES & YES & $1.50$ & $(2,3)$ & -- & 2379\\
$(f;0,0,0;6)$ & 4 & $(64,19)$ & 9 & 2 & YES & YES & YES & $1.62$ & $(2,3)$ & -- & 2380\\
$(f;0,0,0;6)$ & 4 & $(65,19)$ & 9 & 1 & YES & YES & YES & $1.50$ & $(2,3)$ & -- & 2381\\
$(f;0,0,0;6)$ & 4 & $(84,13)$ & 13 & 6 & YES & YES & YES & $1.14$ & $(4,2)$ & -- & 2382\\
$(f;0,0,0;6)$ & 4 & $(85,33)$ & 10 & 1 & YES & YES & NO(2) & $1.38$ & $(8,0)$ & -- & 2383\\
$(f;0,0,0;6)$ & 4 & $(131,24)$ & 13 & 1 & YES & YES & NO(2) & $1.44$ & $(4,2)$ & -- & 2384\\
$(f;0,0,0;6)$ & 4 & $(154,45)$ & 11 & 2 & YES & YES & YES & $1.62$ & $(2,3)$ & -- & 2385\\
$(f;0,1,0;7)$ & 5 & $(23,10)$ & 7 & 1 & YES & YES & YES & $1.29$ & $(4,2)$ & -- & 2386\\
$(f;0,1,0;7)$ & 5 & $(27,10)$ & 7 & 1 & YES & YES & YES & $1.29$ & $(4,2)$ & -- & 2387\\
$(g;0,0,1;26)$ & 7 & $(18,7)$ & 6 & 2 & YES & YES & YES & $1.57$ & $(4,2)$ & -- & 2388\\
$(g;0,1,0;24)$ & 7 & $(13,5)$ & 5 & 1 & YES & YES & NO(2) & $1.44$ & $(4,2)$ & -- & 2389\\
$(g;0,2,2;17)$ & 10 & $(2,1)$ & 1 & 1 & YES & YES & YES & $1.29$ & $(2,3)$ & -- & 2390\\
$(g;0,3,0;34)$ & 9 & $(5,2)$ & 3 & 1 & YES & YES & NO(2) & $1.44$ & $(4,2)$ & -- & 2391\\
$(h;0,3,0;12)$ & 8 & $(9,4)$ & 5 & 3 & YES & YES & YES & $1.50$ & $(2,3)$ & -- & 2392\\
$(h;0,3,0;12)$ & 8 & $(14,3)$ & 6 & 2 & YES & YES & NO(2) & $1.33$ & $(4,2)$ & -- & 2393\\
$(i;0,0,0;9)$ & 5 & $(57,13)$ & 9 & 3 & YES & YES & NO(2) & $1.25$ & $(8,0)$ & -- & 2394\\
$(i;0,0,0;9)$ & 5 & $(58,17)$ & 9 & 1 & YES & YES & YES & $1.43$ & $(4,2)$ & -- & 2395\\
$(i;0,0,0;9)$ & 5 & $(60,13)$ & 9 & 3 & YES & YES & NO(2) & $1.50$ & $(2,3)$ & -- & 2396\\
$(i;0,0,0;9)$ & 5 & $(75,17)$ & 10 & 3 & YES & YES & NO(2) & $1.14$ & $(6,1)$ & -- & 2397\\
$(i;0,2,0;15)$ & 7 & $(24,7)$ & 7 & 3 & YES & YES & YES & $1.43$ & $(4,2)$ & -- & 2398\\
$(i;0,2,0;15)$ & 7 & $(25,7)$ & 7 & 5 & YES & YES & YES & $1.43$ & $(4,2)$ & -- & 2399\\
$(j;0,0,0;8)$ & 5 & $(31,11)$ & 8 & 1 & YES & YES & NO(2) & $1.38$ & $(10,-1)$ & -- & 2400\\
$(j;0,0,0;8)$ & 5 & $(71,27)$ & 9 & 1 & YES & YES & YES & $1.75$ & $(2,3)$ & -- & 2401\\
$(j;0,0,0;8)$ & 5 & $(76,29)$ & 9 & 4 & YES & YES & YES & $1.62$ & $(2,3)$ & -- & 2402\\
$(j;0,1,0;10)$ & 6 & $(31,14)$ & 8 & 1 & YES & YES & NO(2) & $1.56$ & $(4,2)$ & -- & 2403\\
$(j;0,2,0;12)$ & 7 & $(16,5)$ & 7 & 4 & YES & YES & YES & $1.50$ & $(2,3)$ & -- & 2404\\
$(j;0,3,0;14)$ & 8 & $(11,4)$ & 5 & 1 & YES & YES & YES & $1.29$ & $(4,2)$ & -- & 2405
\end{longtable}
\subsection{2 chains, $K^2 = 4$}
\begin{longtable}{|c|c|c|c|c|c|c|c|c|c|c|c|}
\hline
\multicolumn{12}{|c|}{2 chains, $K^2 = 4$}\\
\hline
$(n,a)$ & Len & $(n,a)$ & Len & GCD & Nef & $\mathbb Q$-ef & Obs 0 & $\overline c_1^2 / \overline c_2$ & $(P,K)$ & WH & Index\\
\hline
\endfirsthead

\hline
$(n,a)$ & Len & $(n,a)$ & Len & GCD & Nef & $\mathbb Q$-ef & Obs 0 & $\overline c_1^2 / \overline c_2$ & $(P,K)$ & WH & Index\\
\hline
\endhead
\hline
\endfoot

$(24,11)$ & 8 & $(18,7)$ & 6 & 6 & YES & YES & NO(3) & $1.57$ & $(4,3)$ & -- & 2406\\
$(24,11)$ & 8 & $(24,7)$ & 7 & 24 & YES & YES & NO(3) & $1.57$ & $(4,3)$ & -- & 2407\\
$(34,13)$ & 7 & $(24,5)$ & 8 & 2 & YES & YES & NO(2) & $2.00$ & $(2,4)$ & -- & 2408\\
$(36,11)$ & 8 & $(31,14)$ & 8 & 1 & YES & YES & NO(2) & $1.75$ & $(8,1)$ & -- & 2409\\
$(37,10)$ & 8 & $(23,9)$ & 7 & 1 & YES & YES & YES & $1.83$ & $(4,3)$ & NO & 2410\\
$(39,11)$ & 9 & $(23,5)$ & 7 & 1 & YES & YES & NO(2) & $1.86$ & $(6,2)$ & -- & 2411\\
$(41,7)$ & 11 & $(18,7)$ & 6 & 1 & YES & YES & NO(3) & $1.57$ & $(4,3)$ & -- & 2412\\
$(41,7)$ & 11 & $(24,7)$ & 7 & 1 & YES & YES & NO(3) & $1.57$ & $(4,3)$ & -- & 2413\\
$(43,19)$ & 9 & $(33,10)$ & 8 & 1 & YES & YES & NO(2) & $2.00$ & $(4,3)$ & -- & 2414\\
$(44,17)$ & 8 & $(31,12)$ & 7 & 1 & YES & YES & NO(2) & $1.67$ & $(6,2)$ & -- & 2415\\
$(44,17)$ & 8 & $(33,14)$ & 8 & 11 & YES & YES & NO(2) & $1.83$ & $(6,2)$ & -- & 2416\\
$(47,20)$ & 10 & $(29,8)$ & 7 & 1 & YES & YES & NO(2) & $1.83$ & $(8,1)$ & -- & 2417\\
$(49,18)$ & 8 & $(33,14)$ & 8 & 1 & YES & YES & NO(2) & $2.00$ & $(2,4)$ & -- & 2418\\
$(51,19)$ & 10 & $(31,7)$ & 8 & 1 & YES & YES & NO(2) & $2.11$ & $(2,4)$ & -- & 2419\\
$(51,14)$ & 9 & $(40,7)$ & 9 & 1 & YES & YES & NO(3) & $1.71$ & $(2,4)$ & -- & 2420\\
$(52,23)$ & 10 & $(18,5)$ & 6 & 2 & YES & YES & YES & $1.67$ & $(4,3)$ & -- & 2421\\
$(52,23)$ & 10 & $(23,5)$ & 7 & 1 & YES & YES & YES & $1.67$ & $(4,3)$ & NO & 2422\\
$(56,23)$ & 9 & $(31,12)$ & 7 & 1 & YES & YES & NO(2) & $1.67$ & $(6,2)$ & -- & 2423\\
$(56,15)$ & 9 & $(44,13)$ & 8 & 4 & YES & YES & YES & $2.00$ & $(2,4)$ & -- & 2424\\
$(57,10)$ & 10 & $(55,23)$ & 9 & 1 & YES & YES & NO(2) & $2.00$ & $(4,3)$ & NO & 2425\\
$(59,26)$ & 9 & $(33,10)$ & 8 & 1 & YES & YES & NO(2) & $1.67$ & $(6,2)$ & -- & 2426\\
$(62,17)$ & 10 & $(26,7)$ & 7 & 2 & YES & YES & NO(2) & $2.00$ & $(2,4)$ & -- & 2427\\
$(63,26)$ & 9 & $(33,7)$ & 8 & 3 & YES & YES & NO(2) & $2.00$ & $(2,4)$ & -- & 2428\\
$(64,17)$ & 10 & $(35,8)$ & 8 & 1 & YES & YES & NO(2) & $1.89$ & $(2,4)$ & -- & 2429\\
$(65,19)$ & 9 & $(44,17)$ & 8 & 1 & YES & YES & YES & $2.00$ & $(8,1)$ & -- & 2430\\
$(67,21)$ & 11 & $(11,4)$ & 5 & 1 & YES & YES & YES & $1.83$ & $(4,3)$ & -- & 2431\\
$(67,20)$ & 11 & $(18,7)$ & 6 & 1 & YES & YES & NO(2) & $2.00$ & $(6,2)$ & -- & 2432\\
$(67,20)$ & 11 & $(32,7)$ & 8 & 1 & YES & YES & NO(2) & $2.00$ & $(6,2)$ & NO & 2433\\
$(68,19)$ & 9 & $(11,4)$ & 5 & 1 & YES & YES & YES & $1.83$ & $(4,3)$ & -- & 2434\\
$(68,19)$ & 9 & $(16,5)$ & 7 & 4 & YES & YES & YES & $2.00$ & $(2,4)$ & NO & 2435\\
$(68,19)$ & 9 & $(16,5)$ & 7 & 4 & YES & YES & YES & $2.00$ & $(2,4)$ & -- & 2436\\
$(68,19)$ & 9 & $(44,17)$ & 8 & 4 & YES & YES & YES & $2.12$ & $(2,4)$ & -- & 2437\\
$(71,27)$ & 9 & $(48,11)$ & 9 & 1 & YES & YES & YES & $2.11$ & $(2,4)$ & -- & 2438\\
$(76,31)$ & 10 & $(16,7)$ & 6 & 4 & YES & YES & NO(2) & $1.86$ & $(6,2)$ & -- & 2439\\
$(79,21)$ & 11 & $(23,5)$ & 7 & 1 & YES & YES & YES & $1.83$ & $(4,3)$ & -- & 2440\\
$(84,37)$ & 10 & $(23,7)$ & 7 & 1 & YES & YES & NO(2) & $1.67$ & $(6,2)$ & -- & 2441\\
$(84,37)$ & 10 & $(31,12)$ & 7 & 1 & YES & YES & NO(2) & $1.67$ & $(6,2)$ & NO & 2442\\
$(87,19)$ & 10 & $(11,4)$ & 5 & 1 & YES & YES & YES & $1.83$ & $(4,3)$ & NO & 2443\\
$(87,19)$ & 10 & $(11,4)$ & 5 & 1 & YES & YES & YES & $1.83$ & $(4,3)$ & -- & 2444\\
$(87,20)$ & 12 & $(19,8)$ & 6 & 1 & YES & YES & YES & $2.00$ & $(2,4)$ & NO & 2445\\
$(89,34)$ & 9 & $(37,11)$ & 8 & 1 & YES & YES & YES & $2.00$ & $(6,2)$ & -- & 2446\\
$(89,26)$ & 10 & $(67,20)$ & 11 & 1 & YES & YES & NO(2) & $2.00$ & $(6,2)$ & NO & 2447\\
$(92,21)$ & 10 & $(43,18)$ & 8 & 1 & YES & YES & YES & $2.00$ & $(6,2)$ & -- & 2448\\
$(95,42)$ & 11 & $(16,7)$ & 6 & 1 & YES & YES & NO(2) & $2.12$ & $(4,3)$ & -- & 2449\\
$(97,21)$ & 10 & $(21,5)$ & 8 & 1 & YES & YES & NO(2) & $1.83$ & $(8,1)$ & NO & 2450\\
$(98,41)$ & 10 & $(16,7)$ & 6 & 2 & YES & YES & YES & $2.00$ & $(2,4)$ & -- & 2451\\
$(98,41)$ & 10 & $(73,31)$ & 10 & 1 & YES & YES & YES & $2.00$ & $(2,4)$ & NO & 2452\\
$(99,29)$ & 10 & $(23,10)$ & 7 & 1 & YES & YES & YES & $2.11$ & $(2,4)$ & -- & 2453\\
$(99,29)$ & 10 & $(26,11)$ & 7 & 1 & YES & YES & YES & $2.12$ & $(2,4)$ & -- & 2454\\
$(103,37)$ & 10 & $(16,7)$ & 6 & 1 & YES & YES & NO(2) & $1.67$ & $(10,0)$ & -- & 2455\\
$(103,39)$ & 10 & $(53,20)$ & 10 & 1 & YES & YES & YES & $1.83$ & $(4,3)$ & NO & 2456\\
$(106,31)$ & 10 & $(19,8)$ & 6 & 1 & YES & YES & YES & $2.00$ & $(2,4)$ & -- & 2457\\
$(106,45)$ & 11 & $(49,20)$ & 9 & 1 & YES & YES & NO(2) & $2.00$ & $(4,3)$ & NO & 2458\\
$(107,41)$ & 10 & $(27,8)$ & 7 & 1 & YES & YES & YES & $1.83$ & $(6,2)$ & -- & 2459\\
$(107,47)$ & 10 & $(52,23)$ & 10 & 1 & YES & YES & YES & $1.67$ & $(4,3)$ & NO & 2460\\
$(109,40)$ & 10 & $(7,2)$ & 4 & 1 & YES & YES & NO(2) & $1.86$ & $(10,0)$ & -- & 2461\\
$(110,23)$ & 11 & $(7,2)$ & 4 & 1 & YES & YES & NO(3) & $1.83$ & $(2,4)$ & NO & 2462\\
$(110,23)$ & 11 & $(7,2)$ & 4 & 1 & YES & YES & NO(3) & $1.83$ & $(2,4)$ & -- & 2463\\
$(110,29)$ & 12 & $(16,5)$ & 7 & 2 & YES & YES & NO(2) & $2.12$ & $(4,3)$ & -- & 2464\\
$(113,31)$ & 11 & $(7,2)$ & 4 & 1 & YES & YES & YES & $1.86$ & $(2,4)$ & NO & 2465\\
$(113,31)$ & 11 & $(7,2)$ & 4 & 1 & YES & YES & YES & $1.86$ & $(2,4)$ & -- & 2466\\
$(115,42)$ & 11 & $(22,5)$ & 7 & 1 & YES & YES & YES & $2.00$ & $(2,4)$ & -- & 2467\\
$(117,49)$ & 10 & $(16,5)$ & 7 & 1 & YES & YES & YES & $2.00$ & $(2,4)$ & NO & 2468\\
$(117,49)$ & 10 & $(16,7)$ & 6 & 1 & YES & YES & YES & $2.00$ & $(2,4)$ & -- & 2469\\
$(117,31)$ & 11 & $(23,5)$ & 7 & 1 & YES & YES & YES & $1.86$ & $(2,4)$ & -- & 2470\\
$(120,43)$ & 11 & $(3,1)$ & 2 & 3 & YES & YES & YES & $1.83$ & $(4,3)$ & -- & 2471\\
$(121,32)$ & 11 & $(17,4)$ & 7 & 1 & YES & YES & NO(2) & $1.89$ & $(2,4)$ & -- & 2472\\
$(125,27)$ & 11 & $(11,3)$ & 5 & 1 & YES & YES & YES & $1.83$ & $(4,3)$ & NO & 2473\\
$(125,27)$ & 11 & $(11,3)$ & 5 & 1 & YES & YES & YES & $1.83$ & $(4,3)$ & -- & 2474\\
$(128,47)$ & 10 & $(115,42)$ & 11 & 1 & YES & YES & YES & $2.00$ & $(2,4)$ & NO & 2475\\
$(129,50)$ & 10 & $(18,7)$ & 6 & 3 & YES & YES & YES & $2.11$ & $(2,4)$ & -- & 2476\\
$(131,50)$ & 10 & $(18,7)$ & 6 & 1 & YES & YES & YES & $2.14$ & $(2,4)$ & -- & 2477\\
$(131,36)$ & 11 & $(31,7)$ & 8 & 1 & YES & YES & YES & $2.12$ & $(2,4)$ & -- & 2478\\
$(137,31)$ & 11 & $(56,13)$ & 10 & 1 & YES & YES & NO(2) & $1.89$ & $(2,4)$ & NO & 2479\\
$(138,37)$ & 11 & $(25,7)$ & 7 & 1 & YES & YES & YES & $2.12$ & $(2,4)$ & -- & 2480\\
$(140,53)$ & 11 & $(9,4)$ & 5 & 1 & YES & YES & YES & $2.00$ & $(2,4)$ & -- & 2481\\
$(140,53)$ & 11 & $(28,11)$ & 8 & 28 & YES & YES & YES & $2.00$ & $(2,4)$ & NO & 2482\\
$(144,61)$ & 11 & $(2,1)$ & 1 & 2 & YES & YES & NO(2) & $1.78$ & $(2,4)$ & -- & 2483\\
$(147,53)$ & 11 & $(103,37)$ & 10 & 1 & YES & YES & NO(2) & $1.67$ & $(10,0)$ & NO & 2484\\
$(148,53)$ & 12 & $(5,2)$ & 3 & 1 & YES & YES & YES & $1.86$ & $(2,4)$ & -- & 2485\\
$(148,53)$ & 12 & $(7,3)$ & 4 & 1 & YES & YES & YES & $1.86$ & $(2,4)$ & NO & 2486\\
$(148,53)$ & 12 & $(9,2)$ & 5 & 1 & YES & YES & YES & $1.86$ & $(2,4)$ & -- & 2487\\
$(148,53)$ & 12 & $(19,7)$ & 6 & 1 & YES & YES & YES & $1.86$ & $(2,4)$ & NO & 2488\\
$(149,45)$ & 12 & $(62,19)$ & 10 & 1 & YES & YES & NO(2) & $1.88$ & $(4,3)$ & NO & 2489\\
$(152,67)$ & 11 & $(52,23)$ & 10 & 4 & YES & YES & YES & $1.67$ & $(4,3)$ & NO & 2490\\
$(153,41)$ & 11 & $(12,5)$ & 5 & 3 & YES & YES & NO(2) & $2.00$ & $(2,4)$ & -- & 2491\\
$(155,47)$ & 12 & $(16,5)$ & 7 & 1 & YES & YES & NO(2) & $2.00$ & $(8,1)$ & -- & 2492\\
$(157,42)$ & 12 & $(28,5)$ & 8 & 1 & YES & YES & YES & $2.00$ & $(2,4)$ & -- & 2493\\
$(165,64)$ & 11 & $(8,3)$ & 4 & 1 & YES & YES & NO(2) & $1.67$ & $(6,2)$ & -- & 2494\\
$(166,61)$ & 11 & $(18,5)$ & 6 & 2 & YES & YES & YES & $2.12$ & $(2,4)$ & -- & 2495\\
$(173,75)$ & 13 & $(13,2)$ & 7 & 1 & YES & YES & NO(2) & $2.00$ & $(4,3)$ & -- & 2496\\
$(175,48)$ & 12 & $(113,31)$ & 11 & 1 & YES & YES & YES & $1.86$ & $(2,4)$ & NO & 2497\\
$(176,69)$ & 12 & $(8,3)$ & 4 & 8 & YES & YES & YES & $2.00$ & $(4,3)$ & -- & 2498\\
$(178,63)$ & 12 & $(4,1)$ & 3 & 2 & YES & YES & NO(2) & $1.90$ & $(2,4)$ & -- & 2499\\
$(178,47)$ & 12 & $(27,8)$ & 7 & 1 & YES & YES & YES & $1.83$ & $(4,3)$ & NO & 2500\\
$(178,49)$ & 11 & $(142,39)$ & 11 & 2 & YES & YES & NO(2) & $2.00$ & $(2,4)$ & NO & 2501\\
$(179,75)$ & 11 & $(17,5)$ & 6 & 1 & YES & YES & YES & $2.00$ & $(6,2)$ & -- & 2502\\
$(179,48)$ & 12 & $(85,23)$ & 10 & 1 & YES & YES & NO(2) & $1.89$ & $(6,2)$ & NO & 2503\\
$(183,67)$ & 11 & $(10,3)$ & 5 & 1 & YES & YES & NO(2) & $1.71$ & $(8,1)$ & -- & 2504\\
$(184,51)$ & 12 & $(4,1)$ & 3 & 4 & YES & YES & YES & $1.86$ & $(2,4)$ & -- & 2505\\
$(186,71)$ & 11 & $(97,37)$ & 10 & 1 & YES & YES & NO(2) & $1.88$ & $(4,3)$ & 2591 & 2506\\
$(187,71)$ & 11 & $(13,4)$ & 6 & 1 & YES & YES & NO(2) & $1.86$ & $(4,3)$ & -- & 2507\\
$(187,71)$ & 11 & $(60,23)$ & 9 & 1 & YES & YES & NO(2) & $1.86$ & $(4,3)$ & NO & 2508\\
$(189,40)$ & 12 & $(12,5)$ & 5 & 3 & YES & YES & NO(2) & $1.89$ & $(2,4)$ & -- & 2509\\
$(191,59)$ & 13 & $(9,4)$ & 5 & 1 & YES & YES & NO(2) & $2.00$ & $(4,3)$ & -- & 2510\\
$(191,59)$ & 13 & $(9,4)$ & 5 & 1 & YES & YES & NO(2) & $2.12$ & $(4,3)$ & NO & 2511\\
$(193,53)$ & 12 & $(22,5)$ & 7 & 1 & YES & YES & YES & $2.12$ & $(2,4)$ & -- & 2512\\
$(193,53)$ & 12 & $(167,46)$ & 11 & 1 & YES & YES & YES & $2.12$ & $(2,4)$ & NO & 2513\\
$(195,82)$ & 12 & $(23,10)$ & 7 & 1 & YES & YES & NO(2) & $2.00$ & $(4,3)$ & NO & 2514\\
$(205,38)$ & 15 & $(167,31)$ & 12 & 1 & YES & YES & NO(3) & $1.83$ & $(2,4)$ & NO & 2515\\
$(206,45)$ & 12 & $(14,5)$ & 6 & 2 & YES & YES & YES & $2.11$ & $(2,4)$ & -- & 2516\\
$(206,91)$ & 13 & $(197,87)$ & 12 & 1 & YES & YES & YES & $2.00$ & $(2,4)$ & 2615 & 2517\\
$(207,79)$ & 11 & $(17,5)$ & 6 & 1 & YES & YES & YES & $2.00$ & $(6,2)$ & -- & 2518\\
$(208,95)$ & 13 & $(4,1)$ & 3 & 4 & YES & YES & YES & $1.86$ & $(2,4)$ & -- & 2519\\
$(208,85)$ & 13 & $(9,2)$ & 5 & 1 & YES & YES & NO(2) & $2.00$ & $(4,3)$ & -- & 2520\\
$(208,37)$ & 13 & $(12,5)$ & 5 & 4 & YES & YES & NO(2) & $2.00$ & $(2,4)$ & NO & 2521\\
$(208,37)$ & 13 & $(12,5)$ & 5 & 4 & YES & YES & NO(2) & $2.00$ & $(4,3)$ & -- & 2522\\
$(208,61)$ & 12 & $(18,5)$ & 6 & 2 & YES & YES & YES & $2.00$ & $(6,2)$ & -- & 2523\\
$(208,37)$ & 13 & $(22,5)$ & 7 & 2 & YES & YES & NO(2) & $1.71$ & $(8,1)$ & NO & 2524\\
$(208,37)$ & 13 & $(97,17)$ & 11 & 1 & YES & YES & YES & $2.00$ & $(2,4)$ & NO & 2525\\
$(212,89)$ & 11 & $(5,2)$ & 3 & 1 & YES & YES & NO(2) & $1.89$ & $(2,4)$ & -- & 2526\\
$(212,81)$ & 11 & $(12,5)$ & 5 & 4 & YES & YES & YES & $2.00$ & $(6,2)$ & -- & 2527\\
$(212,89)$ & 11 & $(26,11)$ & 7 & 2 & YES & YES & NO(2) & $1.89$ & $(2,4)$ & NO & 2528\\
$(213,46)$ & 12 & $(9,4)$ & 5 & 3 & YES & YES & YES & $2.00$ & $(2,4)$ & NO & 2529\\
$(217,58)$ & 14 & $(4,1)$ & 3 & 1 & YES & YES & YES & $1.86$ & $(2,4)$ & -- & 2530\\
$(217,92)$ & 12 & $(191,81)$ & 13 & 1 & YES & YES & YES & $2.00$ & $(2,4)$ & 2607 & 2531\\
$(218,47)$ & 13 & $(27,5)$ & 8 & 1 & YES & YES & YES & $2.00$ & $(2,4)$ & NO & 2532\\
$(219,67)$ & 12 & $(3,1)$ & 2 & 3 & YES & YES & NO(2) & $1.86$ & $(6,2)$ & NO & 2533\\
$(219,67)$ & 12 & $(3,1)$ & 2 & 3 & YES & YES & NO(2) & $1.86$ & $(6,2)$ & -- & 2534\\
$(219,83)$ & 12 & $(15,4)$ & 6 & 3 & YES & YES & NO(2) & $2.00$ & $(4,3)$ & NO & 2535\\
$(220,97)$ & 12 & $(5,2)$ & 3 & 5 & YES & YES & YES & $2.00$ & $(2,4)$ & -- & 2536\\
$(222,61)$ & 12 & $(8,3)$ & 4 & 2 & YES & YES & NO(2) & $2.00$ & $(2,4)$ & -- & 2537\\
$(224,103)$ & 13 & $(224,103)$ & 13 & 224 & YES & YES & YES & $1.86$ & $(2,4)$ & NO & 2538\\
$(227,60)$ & 12 & $(7,3)$ & 4 & 1 & YES & YES & NO(2) & $1.86$ & $(6,2)$ & NO & 2539\\
$(227,60)$ & 12 & $(91,24)$ & 11 & 1 & YES & YES & NO(2) & $1.86$ & $(6,2)$ & NO & 2540\\
$(227,84)$ & 12 & $(119,44)$ & 10 & 1 & YES & YES & NO(2) & $2.00$ & $(2,4)$ & NO & 2541\\
$(229,64)$ & 12 & $(27,8)$ & 7 & 1 & YES & YES & YES & $1.83$ & $(4,3)$ & NO & 2542\\
$(231,61)$ & 13 & $(11,2)$ & 6 & 11 & YES & YES & YES & $1.86$ & $(2,4)$ & -- & 2543\\
$(232,89)$ & 13 & $(5,2)$ & 3 & 1 & YES & YES & YES & $2.00$ & $(2,4)$ & -- & 2544\\
$(232,89)$ & 13 & $(7,3)$ & 4 & 1 & YES & YES & NO(2) & $2.00$ & $(6,2)$ & NO & 2545\\
$(232,89)$ & 13 & $(7,3)$ & 4 & 1 & YES & YES & NO(2) & $2.00$ & $(6,2)$ & -- & 2546\\
$(233,89)$ & 11 & $(5,2)$ & 3 & 1 & YES & YES & NO(2) & $1.89$ & $(2,4)$ & -- & 2547\\
$(233,103)$ & 13 & $(6,1)$ & 5 & 1 & YES & YES & YES & $1.86$ & $(2,4)$ & -- & 2548\\
$(233,89)$ & 11 & $(29,11)$ & 7 & 1 & YES & YES & NO(2) & $1.89$ & $(2,4)$ & NO & 2549\\
$(233,89)$ & 11 & $(107,41)$ & 10 & 1 & YES & YES & YES & $1.83$ & $(6,2)$ & NO & 2550\\
$(236,53)$ & 14 & $(22,5)$ & 7 & 2 & YES & YES & YES & $1.88$ & $(2,4)$ & NO & 2551\\
$(236,65)$ & 12 & $(24,7)$ & 7 & 4 & YES & YES & YES & $2.22$ & $(2,4)$ & -- & 2552\\
$(239,107)$ & 13 & $(4,1)$ & 3 & 1 & YES & YES & NO(2) & $2.00$ & $(2,4)$ & -- & 2553\\
$(239,73)$ & 14 & $(7,1)$ & 6 & 1 & YES & YES & NO(2) & $1.88$ & $(8,1)$ & NO & 2554\\
$(239,104)$ & 13 & $(19,8)$ & 6 & 1 & YES & YES & NO(2) & $2.00$ & $(4,3)$ & NO & 2555\\
$(241,88)$ & 13 & $(5,1)$ & 4 & 1 & YES & YES & YES & $1.86$ & $(2,4)$ & NO & 2556\\
$(241,88)$ & 13 & $(11,2)$ & 6 & 1 & YES & YES & NO(2) & $2.00$ & $(4,3)$ & NO & 2557\\
$(243,106)$ & 12 & $(13,3)$ & 6 & 1 & YES & YES & YES & $2.11$ & $(2,4)$ & -- & 2558\\
$(244,67)$ & 13 & $(142,39)$ & 11 & 2 & YES & YES & NO(3) & $1.71$ & $(2,4)$ & 2596 & 2559\\
$(245,107)$ & 13 & $(2,1)$ & 1 & 1 & YES & YES & YES & $1.86$ & $(2,4)$ & -- & 2560\\
$(245,108)$ & 12 & $(5,2)$ & 3 & 5 & YES & YES & YES & $1.86$ & $(2,4)$ & -- & 2561\\
$(248,91)$ & 12 & $(128,47)$ & 10 & 8 & YES & YES & NO(2) & $2.00$ & $(2,4)$ & NO & 2562\\
$(257,69)$ & 12 & $(7,3)$ & 4 & 1 & YES & YES & NO(2) & $2.00$ & $(2,4)$ & -- & 2563\\
$(261,100)$ & 12 & $(3,1)$ & 2 & 3 & YES & YES & NO(2) & $1.88$ & $(4,3)$ & -- & 2564\\
$(261,100)$ & 12 & $(21,8)$ & 6 & 3 & YES & YES & NO(2) & $1.88$ & $(4,3)$ & NO & 2565\\
$(265,104)$ & 13 & $(5,1)$ & 4 & 5 & YES & YES & YES & $1.86$ & $(2,4)$ & NO & 2566\\
$(265,104)$ & 13 & $(135,53)$ & 12 & 5 & YES & YES & NO(2) & $2.00$ & $(4,3)$ & 2655 & 2567\\
$(268,111)$ & 12 & $(5,2)$ & 3 & 1 & YES & YES & NO(2) & $1.71$ & $(8,1)$ & -- & 2568\\
$(268,111)$ & 12 & $(10,3)$ & 5 & 2 & YES & YES & YES & $2.12$ & $(2,4)$ & -- & 2569\\
$(269,71)$ & 13 & $(49,13)$ & 9 & 1 & YES & YES & YES & $1.86$ & $(2,4)$ & NO & 2570\\
$(271,96)$ & 14 & $(25,9)$ & 7 & 1 & YES & YES & NO(2) & $2.00$ & $(6,2)$ & NO & 2571\\
$(273,85)$ & 13 & $(3,1)$ & 2 & 3 & NO & YES & NO(2) & $1.86$ & $(6,2)$ & -- & 2572\\
$(280,103)$ & 13 & $(79,29)$ & 9 & 1 & YES & YES & NO(2) & $2.00$ & $(4,3)$ & NO & 2573\\
$(283,52)$ & 15 & $(125,23)$ & 12 & 1 & YES & YES & NO(3) & $1.67$ & $(4,3)$ & NO & 2574\\
$(288,121)$ & 12 & $(3,1)$ & 2 & 3 & YES & YES & NO(2) & $1.88$ & $(4,3)$ & NO & 2575\\
$(288,121)$ & 12 & $(3,1)$ & 2 & 3 & YES & YES & NO(2) & $1.88$ & $(4,3)$ & -- & 2576\\
$(288,121)$ & 12 & $(9,4)$ & 5 & 9 & YES & YES & YES & $2.00$ & $(2,4)$ & NO & 2577\\
$(289,66)$ & 13 & $(43,10)$ & 9 & 1 & YES & YES & NO(2) & $2.00$ & $(4,3)$ & NO & 2578\\
$(292,111)$ & 12 & $(10,3)$ & 5 & 2 & YES & YES & YES & $2.11$ & $(2,4)$ & -- & 2579\\
$(293,123)$ & 12 & $(7,2)$ & 4 & 1 & YES & YES & YES & $2.00$ & $(2,4)$ & -- & 2580\\
$(298,131)$ & 13 & $(5,2)$ & 3 & 1 & YES & YES & NO(2) & $1.67$ & $(6,2)$ & -- & 2581\\
$(302,117)$ & 12 & $(13,3)$ & 6 & 1 & YES & YES & YES & $2.12$ & $(2,4)$ & NO & 2582\\
$(302,117)$ & 12 & $(13,3)$ & 6 & 1 & YES & YES & YES & $2.12$ & $(2,4)$ & -- & 2583\\
$(308,87)$ & 14 & $(3,1)$ & 2 & 1 & YES & YES & NO(3) & $1.83$ & $(2,4)$ & NO & 2584\\
$(310,83)$ & 13 & $(7,3)$ & 4 & 1 & YES & YES & YES & $2.00$ & $(2,4)$ & -- & 2585\\
$(313,121)$ & 12 & $(8,3)$ & 4 & 1 & YES & YES & YES & $2.14$ & $(2,4)$ & -- & 2586\\
$(313,121)$ & 12 & $(13,3)$ & 6 & 1 & YES & YES & YES & $2.14$ & $(2,4)$ & -- & 2587\\
$(314,83)$ & 13 & $(121,32)$ & 11 & 1 & YES & YES & NO(2) & $1.89$ & $(2,4)$ & NO & 2588\\
$(317,121)$ & 12 & $(3,1)$ & 2 & 1 & YES & YES & NO(2) & $2.00$ & $(2,4)$ & NO & 2589\\
$(317,121)$ & 12 & $(3,1)$ & 2 & 1 & YES & YES & NO(2) & $2.00$ & $(2,4)$ & -- & 2590\\
$(317,121)$ & 12 & $(21,8)$ & 6 & 1 & YES & YES & NO(2) & $1.88$ & $(4,3)$ & 2506 & 2591\\
$(325,87)$ & 13 & $(157,42)$ & 12 & 1 & YES & YES & YES & $2.00$ & $(2,4)$ & NO & 2592\\
$(332,97)$ & 13 & $(37,11)$ & 8 & 1 & YES & YES & YES & $2.11$ & $(2,4)$ & NO & 2593\\
$(335,92)$ & 13 & $(4,1)$ & 3 & 1 & YES & YES & NO(3) & $1.71$ & $(2,4)$ & NO & 2594\\
$(335,92)$ & 13 & $(13,3)$ & 6 & 1 & YES & YES & YES & $2.00$ & $(2,4)$ & NO & 2595\\
$(335,92)$ & 13 & $(51,14)$ & 9 & 1 & YES & YES & NO(3) & $1.71$ & $(2,4)$ & 2559 & 2596\\
$(335,92)$ & 13 & $(295,81)$ & 14 & 5 & YES & YES & YES & $2.00$ & $(2,4)$ & 2697 & 2597\\
$(336,137)$ & 14 & $(4,1)$ & 3 & 4 & YES & YES & NO(2) & $2.00$ & $(4,3)$ & -- & 2598\\
$(336,137)$ & 14 & $(233,95)$ & 13 & 1 & YES & YES & NO(2) & $2.00$ & $(4,3)$ & NO & 2599\\
$(340,143)$ & 14 & $(5,2)$ & 3 & 5 & YES & YES & YES & $2.00$ & $(2,4)$ & NO & 2600\\
$(340,143)$ & 14 & $(5,2)$ & 3 & 5 & YES & YES & NO(2) & $1.89$ & $(6,2)$ & -- & 2601\\
$(341,90)$ & 14 & $(5,2)$ & 3 & 1 & YES & YES & NO(2) & $2.00$ & $(2,4)$ & -- & 2602\\
$(341,90)$ & 14 & $(269,71)$ & 13 & 1 & YES & YES & YES & $1.86$ & $(2,4)$ & NO & 2603\\
$(347,153)$ & 13 & $(3,1)$ & 2 & 1 & YES & YES & YES & $1.86$ & $(2,4)$ & -- & 2604\\
$(348,103)$ & 13 & $(5,2)$ & 3 & 1 & YES & YES & YES & $2.00$ & $(2,4)$ & -- & 2605\\
$(348,125)$ & 13 & $(5,2)$ & 3 & 1 & YES & YES & YES & $2.00$ & $(4,3)$ & -- & 2606\\
$(349,148)$ & 14 & $(92,39)$ & 10 & 1 & YES & YES & YES & $2.00$ & $(2,4)$ & 2531 & 2607\\
$(353,154)$ & 13 & $(243,106)$ & 12 & 1 & YES & YES & YES & $2.11$ & $(2,4)$ & NO & 2608\\
$(355,63)$ & 15 & $(5,2)$ & 3 & 5 & YES & YES & YES & $1.86$ & $(2,4)$ & -- & 2609\\
$(356,105)$ & 13 & $(5,2)$ & 3 & 1 & YES & YES & YES & $2.00$ & $(2,4)$ & -- & 2610\\
$(363,58)$ & 17 & $(4,1)$ & 3 & 1 & YES & YES & YES & $1.86$ & $(2,4)$ & -- & 2611\\
$(363,152)$ & 13 & $(4,1)$ & 3 & 1 & YES & YES & NO(2) & $2.00$ & $(2,4)$ & -- & 2612\\
$(363,152)$ & 13 & $(117,49)$ & 10 & 3 & YES & YES & NO(2) & $2.00$ & $(2,4)$ & NO & 2613\\
$(367,154)$ & 13 & $(2,1)$ & 1 & 1 & YES & YES & NO(2) & $1.86$ & $(6,2)$ & NO & 2614\\
$(369,163)$ & 14 & $(77,34)$ & 10 & 1 & YES & YES & YES & $2.00$ & $(2,4)$ & 2517 & 2615\\
$(371,132)$ & 14 & $(3,1)$ & 2 & 1 & YES & YES & NO(2) & $2.00$ & $(2,4)$ & -- & 2616\\
$(371,144)$ & 13 & $(3,1)$ & 2 & 1 & YES & YES & NO(2) & $2.00$ & $(2,4)$ & -- & 2617\\
$(375,143)$ & 14 & $(2,1)$ & 1 & 1 & YES & YES & NO(2) & $2.00$ & $(6,2)$ & -- & 2618\\
$(375,88)$ & 15 & $(5,1)$ & 4 & 5 & YES & YES & YES & $1.86$ & $(2,4)$ & NO & 2619\\
$(375,88)$ & 15 & $(11,2)$ & 6 & 1 & YES & YES & NO(2) & $1.90$ & $(4,3)$ & NO & 2620\\
$(376,139)$ & 13 & $(4,1)$ & 3 & 4 & YES & YES & NO(2) & $2.00$ & $(2,4)$ & -- & 2621\\
$(376,139)$ & 13 & $(119,44)$ & 10 & 1 & YES & YES & NO(2) & $2.00$ & $(2,4)$ & NO & 2622\\
$(379,165)$ & 13 & $(4,1)$ & 3 & 1 & YES & YES & YES & $1.83$ & $(6,2)$ & NO & 2623\\
$(380,137)$ & 13 & $(9,2)$ & 5 & 1 & YES & YES & YES & $2.11$ & $(2,4)$ & -- & 2624\\
$(383,140)$ & 13 & $(7,2)$ & 4 & 1 & YES & YES & YES & $2.00$ & $(2,4)$ & -- & 2625\\
$(383,140)$ & 13 & $(7,3)$ & 4 & 1 & YES & YES & YES & $2.00$ & $(2,4)$ & NO & 2626\\
$(388,113)$ & 14 & $(24,7)$ & 7 & 4 & YES & YES & NO(2) & $2.00$ & $(4,3)$ & NO & 2627\\
$(391,73)$ & 16 & $(5,2)$ & 3 & 1 & YES & YES & NO(2) & $1.67$ & $(10,0)$ & NO & 2628\\
$(391,73)$ & 16 & $(5,2)$ & 3 & 1 & YES & YES & NO(2) & $1.67$ & $(10,0)$ & -- & 2629\\
$(393,116)$ & 13 & $(8,3)$ & 4 & 1 & YES & YES & NO(2) & $2.14$ & $(4,3)$ & -- & 2630\\
$(395,122)$ & 16 & $(3,1)$ & 2 & 1 & YES & YES & NO(2) & $2.12$ & $(4,3)$ & -- & 2631\\
$(395,123)$ & 14 & $(16,5)$ & 7 & 1 & YES & YES & NO(3) & $1.67$ & $(4,3)$ & NO & 2632\\
$(397,175)$ & 13 & $(397,175)$ & 13 & 397 & YES & YES & YES & $1.83$ & $(6,2)$ & NO & 2633\\
$(407,71)$ & 15 & $(3,1)$ & 2 & 1 & YES & YES & NO(3) & $1.71$ & $(2,4)$ & -- & 2634\\
$(413,157)$ & 13 & $(7,2)$ & 4 & 7 & YES & YES & YES & $2.12$ & $(2,4)$ & -- & 2635\\
$(415,127)$ & 14 & $(3,1)$ & 2 & 1 & YES & YES & NO(2) & $2.00$ & $(2,4)$ & -- & 2636\\
$(418,111)$ & 14 & $(4,1)$ & 3 & 2 & YES & YES & YES & $1.86$ & $(2,4)$ & -- & 2637\\
$(421,80)$ & 16 & $(5,2)$ & 3 & 1 & YES & YES & NO(2) & $2.00$ & $(4,3)$ & NO & 2638\\
$(421,80)$ & 16 & $(5,2)$ & 3 & 1 & YES & YES & NO(2) & $2.00$ & $(4,3)$ & NO & 2639\\
$(426,97)$ & 15 & $(3,1)$ & 2 & 3 & YES & YES & NO(2) & $1.89$ & $(2,4)$ & -- & 2640\\
$(428,101)$ & 16 & $(4,1)$ & 3 & 4 & YES & YES & NO(2) & $1.88$ & $(4,3)$ & -- & 2641\\
$(428,101)$ & 16 & $(13,3)$ & 6 & 1 & YES & YES & NO(2) & $2.00$ & $(4,3)$ & NO & 2642\\
$(432,181)$ & 13 & $(5,2)$ & 3 & 1 & YES & YES & YES & $2.12$ & $(2,4)$ & -- & 2643\\
$(432,181)$ & 13 & $(8,3)$ & 4 & 8 & YES & YES & NO(2) & $2.11$ & $(2,4)$ & NO & 2644\\
$(433,128)$ & 13 & $(169,50)$ & 11 & 1 & YES & YES & YES & $2.17$ & $(6,2)$ & NO & 2645\\
$(434,115)$ & 14 & $(200,53)$ & 12 & 2 & YES & YES & YES & $1.86$ & $(2,4)$ & 2692 & 2646\\
$(436,115)$ & 15 & $(53,14)$ & 9 & 1 & YES & YES & YES & $2.00$ & $(2,4)$ & NO & 2647\\
$(438,161)$ & 13 & $(14,5)$ & 6 & 2 & YES & YES & YES & $2.11$ & $(2,4)$ & NO & 2648\\
$(445,172)$ & 13 & $(9,2)$ & 5 & 1 & YES & YES & YES & $2.12$ & $(2,4)$ & -- & 2649\\
$(445,172)$ & 13 & $(9,2)$ & 5 & 1 & YES & YES & YES & $2.12$ & $(2,4)$ & NO & 2650\\
$(446,173)$ & 13 & $(44,17)$ & 8 & 2 & YES & YES & YES & $2.00$ & $(8,1)$ & NO & 2651\\
$(446,197)$ & 14 & $(77,34)$ & 10 & 1 & YES & YES & YES & $2.00$ & $(2,4)$ & 2690 & 2652\\
$(448,197)$ & 15 & $(4,1)$ & 3 & 4 & YES & YES & NO(2) & $1.89$ & $(6,2)$ & -- & 2653\\
$(448,197)$ & 15 & $(7,3)$ & 4 & 7 & YES & YES & NO(2) & $2.00$ & $(6,2)$ & NO & 2654\\
$(451,177)$ & 14 & $(28,11)$ & 8 & 1 & YES & YES & NO(2) & $2.00$ & $(4,3)$ & 2567 & 2655\\
$(461,74)$ & 17 & $(44,7)$ & 10 & 1 & YES & YES & NO(2) & $2.00$ & $(4,3)$ & NO & 2656\\
$(463,98)$ & 14 & $(2,1)$ & 1 & 1 & YES & YES & NO(2) & $1.75$ & $(8,1)$ & -- & 2657\\
$(463,98)$ & 14 & $(2,1)$ & 1 & 1 & YES & YES & NO(2) & $1.88$ & $(8,1)$ & NO & 2658\\
$(463,179)$ & 13 & $(313,121)$ & 12 & 1 & YES & YES & YES & $2.14$ & $(2,4)$ & NO & 2659\\
$(465,197)$ & 14 & $(3,1)$ & 2 & 3 & YES & YES & YES & $2.00$ & $(2,4)$ & -- & 2660\\
$(465,128)$ & 13 & $(65,18)$ & 9 & 5 & YES & YES & YES & $2.00$ & $(6,2)$ & NO & 2661\\
$(465,197)$ & 14 & $(465,197)$ & 14 & 465 & YES & YES & YES & $2.00$ & $(2,4)$ & NO & 2662\\
$(473,174)$ & 14 & $(3,1)$ & 2 & 1 & YES & YES & NO(2) & $2.00$ & $(4,3)$ & -- & 2663\\
$(473,125)$ & 14 & $(5,2)$ & 3 & 1 & YES & YES & YES & $2.00$ & $(4,3)$ & NO & 2664\\
$(473,140)$ & 14 & $(44,13)$ & 8 & 11 & YES & YES & YES & $2.00$ & $(2,4)$ & NO & 2665\\
$(473,174)$ & 14 & $(473,174)$ & 14 & 473 & YES & YES & NO(2) & $1.88$ & $(4,3)$ & NO & 2666\\
$(476,109)$ & 14 & $(40,9)$ & 9 & 4 & YES & YES & YES & $2.00$ & $(6,2)$ & NO & 2667\\
$(476,107)$ & 15 & $(49,11)$ & 10 & 7 & YES & YES & NO(3) & $1.86$ & $(2,4)$ & NO & 2668\\
$(476,109)$ & 14 & $(92,21)$ & 10 & 4 & YES & YES & YES & $2.00$ & $(6,2)$ & NO & 2669\\
$(477,187)$ & 14 & $(28,11)$ & 8 & 1 & YES & YES & YES & $1.83$ & $(4,3)$ & NO & 2670\\
$(478,201)$ & 14 & $(2,1)$ & 1 & 2 & YES & YES & NO(2) & $2.00$ & $(4,3)$ & -- & 2671\\
$(480,133)$ & 15 & $(8,1)$ & 7 & 8 & YES & YES & NO(2) & $1.83$ & $(8,1)$ & NO & 2672\\
$(482,177)$ & 13 & $(4,1)$ & 3 & 2 & YES & YES & NO(2) & $1.71$ & $(8,1)$ & -- & 2673\\
$(482,177)$ & 13 & $(30,11)$ & 7 & 2 & YES & YES & NO(2) & $1.86$ & $(8,1)$ & NO & 2674\\
$(485,188)$ & 13 & $(44,17)$ & 8 & 1 & YES & YES & YES & $2.00$ & $(2,4)$ & NO & 2675\\
$(487,101)$ & 15 & $(4,1)$ & 3 & 1 & YES & YES & NO(3) & $1.86$ & $(2,4)$ & NO & 2676\\
$(490,187)$ & 13 & $(5,2)$ & 3 & 5 & YES & YES & YES & $2.12$ & $(2,4)$ & -- & 2677\\
$(490,187)$ & 13 & $(18,7)$ & 6 & 2 & YES & YES & YES & $2.11$ & $(2,4)$ & NO & 2678\\
$(497,107)$ & 15 & $(23,5)$ & 7 & 1 & YES & YES & YES & $1.86$ & $(2,4)$ & NO & 2679\\
$(498,209)$ & 13 & $(5,2)$ & 3 & 1 & YES & YES & YES & $2.12$ & $(2,4)$ & -- & 2680\\
$(502,219)$ & 14 & $(353,154)$ & 13 & 1 & YES & YES & YES & $2.11$ & $(2,4)$ & NO & 2681\\
$(503,113)$ & 15 & $(2,1)$ & 1 & 1 & YES & YES & NO(3) & $1.86$ & $(2,4)$ & NO & 2682\\
$(503,132)$ & 15 & $(2,1)$ & 1 & 1 & YES & YES & NO(2) & $2.00$ & $(2,4)$ & -- & 2683\\
$(503,219)$ & 14 & $(4,1)$ & 3 & 1 & YES & YES & YES & $2.11$ & $(2,4)$ & NO & 2684\\
$(503,132)$ & 15 & $(7,2)$ & 4 & 1 & YES & YES & NO(2) & $1.83$ & $(8,1)$ & NO & 2685\\
$(507,140)$ & 14 & $(3,1)$ & 2 & 3 & YES & YES & YES & $1.88$ & $(2,4)$ & -- & 2686\\
$(507,140)$ & 14 & $(5,2)$ & 3 & 1 & YES & YES & YES & $2.00$ & $(2,4)$ & -- & 2687\\
$(507,140)$ & 14 & $(76,21)$ & 9 & 1 & YES & YES & YES & $1.88$ & $(2,4)$ & NO & 2688\\
$(514,181)$ & 18 & $(2,1)$ & 1 & 2 & YES & YES & NO(2) & $2.17$ & $(8,1)$ & NO & 2689\\
$(514,227)$ & 14 & $(43,19)$ & 9 & 1 & YES & YES & YES & $2.00$ & $(2,4)$ & 2652 & 2690\\
$(517,142)$ & 14 & $(18,5)$ & 6 & 1 & YES & YES & YES & $1.88$ & $(2,4)$ & NO & 2691\\
$(517,137)$ & 14 & $(117,31)$ & 11 & 1 & YES & YES & YES & $1.86$ & $(2,4)$ & 2646 & 2692\\
$(517,142)$ & 14 & $(131,36)$ & 11 & 1 & YES & YES & YES & $2.12$ & $(2,4)$ & 2760 & 2693\\
$(521,119)$ & 15 & $(35,8)$ & 8 & 1 & YES & YES & NO(3) & $1.86$ & $(2,4)$ & NO & 2694\\
$(537,164)$ & 15 & $(2,1)$ & 1 & 1 & YES & YES & NO(2) & $2.00$ & $(4,3)$ & -- & 2695\\
$(539,123)$ & 14 & $(53,12)$ & 9 & 1 & YES & YES & YES & $2.17$ & $(6,2)$ & NO & 2696\\
$(539,148)$ & 15 & $(142,39)$ & 11 & 1 & YES & YES & YES & $2.00$ & $(2,4)$ & 2597 & 2697\\
$(551,240)$ & 14 & $(3,1)$ & 2 & 1 & YES & YES & YES & $2.00$ & $(4,3)$ & NO & 2698\\
$(552,199)$ & 14 & $(319,115)$ & 13 & 1 & YES & YES & YES & $2.11$ & $(2,4)$ & NO & 2699\\
$(557,243)$ & 14 & $(3,1)$ & 2 & 1 & YES & YES & YES & $2.11$ & $(2,4)$ & NO & 2700\\
$(557,243)$ & 14 & $(353,154)$ & 13 & 1 & YES & YES & YES & $2.11$ & $(2,4)$ & NO & 2701\\
$(559,165)$ & 14 & $(5,2)$ & 3 & 1 & YES & YES & YES & $2.12$ & $(2,4)$ & -- & 2702\\
$(559,165)$ & 14 & $(11,3)$ & 5 & 1 & YES & YES & YES & $2.12$ & $(2,4)$ & NO & 2703\\
$(563,158)$ & 15 & $(7,2)$ & 4 & 1 & YES & YES & NO(2) & $1.67$ & $(10,0)$ & NO & 2704\\
$(583,226)$ & 14 & $(3,1)$ & 2 & 1 & YES & YES & NO(2) & $1.86$ & $(4,3)$ & -- & 2705\\
$(583,173)$ & 14 & $(5,2)$ & 3 & 1 & YES & YES & YES & $2.12$ & $(2,4)$ & NO & 2706\\
$(583,173)$ & 14 & $(5,2)$ & 3 & 1 & YES & YES & YES & $2.12$ & $(2,4)$ & -- & 2707\\
$(587,256)$ & 14 & $(5,1)$ & 4 & 1 & YES & YES & YES & $2.11$ & $(2,4)$ & NO & 2708\\
$(590,229)$ & 14 & $(13,5)$ & 5 & 1 & YES & YES & YES & $2.11$ & $(2,4)$ & NO & 2709\\
$(596,165)$ & 14 & $(25,7)$ & 7 & 1 & YES & YES & YES & $2.12$ & $(2,4)$ & NO & 2710\\
$(606,251)$ & 14 & $(4,1)$ & 3 & 2 & YES & YES & YES & $2.12$ & $(2,4)$ & -- & 2711\\
$(606,251)$ & 14 & $(268,111)$ & 12 & 2 & YES & YES & YES & $2.12$ & $(2,4)$ & 2742 & 2712\\
$(606,251)$ & 14 & $(437,181)$ & 13 & 1 & YES & YES & YES & $2.12$ & $(2,4)$ & NO & 2713\\
$(608,235)$ & 14 & $(445,172)$ & 13 & 1 & YES & YES & YES & $2.12$ & $(2,4)$ & NO & 2714\\
$(611,256)$ & 14 & $(105,44)$ & 10 & 1 & YES & YES & YES & $2.12$ & $(2,4)$ & NO & 2715\\
$(615,227)$ & 14 & $(19,7)$ & 6 & 1 & YES & YES & NO(2) & $1.50$ & $(10,0)$ & NO & 2716\\
$(623,241)$ & 14 & $(243,94)$ & 12 & 1 & YES & YES & YES & $2.11$ & $(2,4)$ & NO & 2717\\
$(628,265)$ & 14 & $(3,1)$ & 2 & 1 & YES & YES & YES & $2.12$ & $(2,4)$ & -- & 2718\\
$(628,265)$ & 14 & $(282,119)$ & 12 & 2 & YES & YES & YES & $2.12$ & $(2,4)$ & 2750 & 2719\\
$(634,241)$ & 14 & $(292,111)$ & 12 & 2 & YES & YES & YES & $2.11$ & $(2,4)$ & 2759 & 2720\\
$(635,132)$ & 16 & $(3,1)$ & 2 & 1 & YES & YES & NO(2) & $1.83$ & $(8,1)$ & NO & 2721\\
$(637,263)$ & 14 & $(3,1)$ & 2 & 1 & YES & YES & NO(2) & $2.25$ & $(4,3)$ & -- & 2722\\
$(649,251)$ & 14 & $(2,1)$ & 1 & 1 & YES & YES & NO(2) & $2.12$ & $(4,3)$ & -- & 2723\\
$(649,251)$ & 14 & $(5,1)$ & 4 & 1 & YES & YES & YES & $2.17$ & $(8,1)$ & NO & 2724\\
$(658,241)$ & 14 & $(3,1)$ & 2 & 1 & YES & YES & YES & $2.12$ & $(2,4)$ & -- & 2725\\
$(658,241)$ & 14 & $(5,2)$ & 3 & 1 & YES & YES & YES & $2.00$ & $(2,4)$ & NO & 2726\\
$(663,275)$ & 15 & $(6,1)$ & 5 & 3 & YES & YES & NO(2) & $2.00$ & $(4,3)$ & -- & 2727\\
$(675,154)$ & 15 & $(31,7)$ & 8 & 1 & YES & YES & YES & $2.00$ & $(2,4)$ & NO & 2728\\
$(680,263)$ & 14 & $(3,1)$ & 2 & 1 & YES & YES & YES & $2.11$ & $(2,4)$ & -- & 2729\\
$(680,263)$ & 14 & $(3,1)$ & 2 & 1 & YES & YES & NO(2) & $2.12$ & $(4,3)$ & NO & 2730\\
$(680,287)$ & 14 & $(263,111)$ & 12 & 1 & YES & YES & YES & $2.12$ & $(2,4)$ & NO & 2731\\
$(681,154)$ & 15 & $(75,17)$ & 10 & 3 & YES & YES & YES & $2.12$ & $(2,4)$ & NO & 2732\\
$(683,251)$ & 14 & $(166,61)$ & 11 & 1 & YES & YES & YES & $2.00$ & $(2,4)$ & NO & 2733\\
$(695,288)$ & 14 & $(3,1)$ & 2 & 1 & YES & YES & YES & $2.12$ & $(2,4)$ & -- & 2734\\
$(695,202)$ & 15 & $(7,2)$ & 4 & 1 & YES & YES & YES & $1.83$ & $(4,3)$ & NO & 2735\\
$(697,266)$ & 14 & $(2,1)$ & 1 & 1 & YES & YES & YES & $2.14$ & $(2,4)$ & -- & 2736\\
$(697,288)$ & 14 & $(3,1)$ & 2 & 1 & YES & YES & NO(2) & $2.25$ & $(4,3)$ & -- & 2737\\
$(697,266)$ & 14 & $(5,2)$ & 3 & 1 & YES & YES & YES & $2.11$ & $(2,4)$ & NO & 2738\\
$(697,266)$ & 14 & $(131,50)$ & 10 & 1 & YES & YES & NO(2) & $2.12$ & $(2,4)$ & NO & 2739\\
$(703,267)$ & 14 & $(129,49)$ & 10 & 1 & YES & YES & NO(2) & $2.14$ & $(4,3)$ & NO & 2740\\
$(705,268)$ & 14 & $(2,1)$ & 1 & 1 & YES & YES & NO(2) & $2.12$ & $(4,3)$ & NO & 2741\\
$(705,292)$ & 14 & $(169,70)$ & 11 & 1 & YES & YES & YES & $2.12$ & $(2,4)$ & 2712 & 2742\\
$(705,268)$ & 14 & $(413,157)$ & 13 & 1 & YES & YES & YES & $2.12$ & $(2,4)$ & NO & 2743\\
$(705,268)$ & 14 & $(705,268)$ & 14 & 705 & YES & YES & YES & $2.17$ & $(8,1)$ & NO & 2744\\
$(707,274)$ & 14 & $(13,5)$ & 5 & 1 & YES & YES & YES & $2.14$ & $(2,4)$ & NO & 2745\\
$(715,277)$ & 14 & $(13,5)$ & 5 & 13 & YES & YES & YES & $2.12$ & $(2,4)$ & NO & 2746\\
$(715,277)$ & 14 & $(302,117)$ & 12 & 1 & YES & YES & YES & $2.12$ & $(2,4)$ & NO & 2747\\
$(722,113)$ & 18 & $(2,1)$ & 1 & 2 & YES & YES & NO(2) & $2.00$ & $(2,4)$ & -- & 2748\\
$(722,113)$ & 18 & $(8,1)$ & 7 & 2 & YES & YES & NO(2) & $1.83$ & $(8,1)$ & NO & 2749\\
$(737,311)$ & 14 & $(173,73)$ & 11 & 1 & YES & YES & YES & $2.12$ & $(2,4)$ & 2719 & 2750\\
$(745,313)$ & 14 & $(5,1)$ & 4 & 5 & YES & YES & YES & $2.00$ & $(6,2)$ & -- & 2751\\
$(745,288)$ & 14 & $(313,121)$ & 12 & 1 & YES & YES & YES & $2.14$ & $(2,4)$ & NO & 2752\\
$(747,169)$ & 15 & $(75,17)$ & 10 & 3 & YES & YES & YES & $2.17$ & $(8,1)$ & NO & 2753\\
$(751,132)$ & 17 & $(3,1)$ & 2 & 1 & YES & YES & NO(2) & $2.00$ & $(4,3)$ & NO & 2754\\
$(751,132)$ & 17 & $(4,1)$ & 3 & 1 & YES & YES & NO(2) & $1.88$ & $(4,3)$ & NO & 2755\\
$(752,287)$ & 14 & $(3,1)$ & 2 & 1 & YES & YES & YES & $2.17$ & $(6,2)$ & -- & 2756\\
$(755,292)$ & 14 & $(2,1)$ & 1 & 1 & YES & YES & YES & $2.17$ & $(8,1)$ & -- & 2757\\
$(755,312)$ & 14 & $(12,5)$ & 5 & 1 & YES & YES & YES & $2.00$ & $(6,2)$ & NO & 2758\\
$(755,287)$ & 14 & $(171,65)$ & 11 & 1 & YES & YES & YES & $2.11$ & $(2,4)$ & 2720 & 2759\\
$(757,208)$ & 15 & $(51,14)$ & 9 & 1 & YES & YES & YES & $2.12$ & $(2,4)$ & 2693 & 2760\\
$(761,223)$ & 15 & $(273,80)$ & 13 & 1 & YES & YES & YES & $2.00$ & $(2,4)$ & NO & 2761\\
$(765,317)$ & 14 & $(7,3)$ & 4 & 1 & YES & YES & YES & $2.00$ & $(6,2)$ & NO & 2762\\
$(772,163)$ & 16 & $(3,1)$ & 2 & 1 & YES & YES & YES & $2.00$ & $(2,4)$ & NO & 2763\\
$(772,163)$ & 16 & $(9,2)$ & 5 & 1 & YES & YES & YES & $2.00$ & $(2,4)$ & NO & 2764\\
$(790,231)$ & 15 & $(2,1)$ & 1 & 2 & YES & YES & YES & $2.00$ & $(6,2)$ & NO & 2765\\
$(798,143)$ & 16 & $(5,2)$ & 3 & 1 & YES & YES & YES & $2.12$ & $(2,4)$ & -- & 2766\\
$(802,235)$ & 15 & $(2,1)$ & 1 & 2 & YES & YES & YES & $2.12$ & $(2,4)$ & NO & 2767\\
$(802,215)$ & 15 & $(138,37)$ & 11 & 2 & YES & YES & YES & $2.12$ & $(2,4)$ & NO & 2768\\
$(805,312)$ & 14 & $(4,1)$ & 3 & 1 & YES & YES & NO(2) & $2.00$ & $(4,3)$ & NO & 2769\\
$(809,226)$ & 15 & $(3,1)$ & 2 & 1 & YES & YES & YES & $2.17$ & $(6,2)$ & -- & 2770\\
$(811,219)$ & 15 & $(4,1)$ & 3 & 1 & YES & YES & YES & $2.11$ & $(2,4)$ & NO & 2771\\
$(835,148)$ & 17 & $(28,5)$ & 8 & 1 & YES & YES & YES & $2.00$ & $(2,4)$ & NO & 2772\\
$(843,322)$ & 14 & $(3,1)$ & 2 & 3 & YES & YES & YES & $2.17$ & $(6,2)$ & NO & 2773\\
$(843,322)$ & 14 & $(3,1)$ & 2 & 3 & YES & YES & YES & $2.17$ & $(6,2)$ & -- & 2774\\
$(880,199)$ & 16 & $(199,45)$ & 12 & 1 & YES & YES & YES & $2.11$ & $(2,4)$ & NO & 2775\\
$(883,243)$ & 15 & $(3,1)$ & 2 & 1 & YES & YES & YES & $2.12$ & $(2,4)$ & NO & 2776\\
$(893,246)$ & 15 & $(5,2)$ & 3 & 1 & YES & YES & YES & $2.22$ & $(2,4)$ & -- & 2777\\
$(893,246)$ & 15 & $(7,2)$ & 4 & 1 & YES & YES & YES & $2.11$ & $(2,4)$ & NO & 2778\\
$(893,246)$ & 15 & $(236,65)$ & 12 & 1 & YES & YES & YES & $2.22$ & $(2,4)$ & NO & 2779\\
$(901,264)$ & 15 & $(372,109)$ & 13 & 1 & YES & YES & YES & $2.11$ & $(2,4)$ & NO & 2780\\
$(907,265)$ & 15 & $(2,1)$ & 1 & 1 & YES & YES & YES & $2.11$ & $(2,4)$ & NO & 2781\\
$(908,207)$ & 16 & $(715,163)$ & 15 & 1 & YES & YES & YES & $2.12$ & $(2,4)$ & NO & 2782\\
$(911,199)$ & 16 & $(206,45)$ & 12 & 1 & YES & YES & YES & $2.11$ & $(2,4)$ & NO & 2783\\
$(923,255)$ & 15 & $(18,5)$ & 6 & 1 & YES & YES & YES & $2.00$ & $(6,2)$ & NO & 2784\\
$(927,256)$ & 15 & $(2,1)$ & 1 & 1 & YES & YES & YES & $2.00$ & $(6,2)$ & NO & 2785\\
$(937,261)$ & 15 & $(2,1)$ & 1 & 1 & YES & YES & YES & $1.83$ & $(6,2)$ & NO & 2786\\
$(957,284)$ & 15 & $(17,5)$ & 6 & 1 & YES & YES & NO(2) & $2.14$ & $(4,3)$ & NO & 2787\\
$(979,222)$ & 16 & $(3,1)$ & 2 & 1 & YES & YES & YES & $2.00$ & $(6,2)$ & NO & 2788\\
$(994,227)$ & 16 & $(3,1)$ & 2 & 1 & YES & YES & YES & $2.00$ & $(6,2)$ & NO & 2789\\
$(1013,299)$ & 15 & $(5,1)$ & 4 & 1 & YES & YES & NO(2) & $2.00$ & $(4,3)$ & -- & 2790\\
$(1027,305)$ & 15 & $(4,1)$ & 3 & 1 & YES & YES & NO(2) & $2.00$ & $(4,3)$ & -- & 2791\\
$(1027,305)$ & 15 & $(17,5)$ & 6 & 1 & YES & YES & YES & $2.00$ & $(6,2)$ & NO & 2792\\
$(1048,237)$ & 16 & $(199,45)$ & 12 & 1 & YES & YES & YES & $2.00$ & $(2,4)$ & NO & 2793\\
$(1085,237)$ & 16 & $(4,1)$ & 3 & 1 & YES & YES & YES & $2.12$ & $(2,4)$ & NO & 2794\\
$(1085,237)$ & 16 & $(23,5)$ & 7 & 1 & YES & YES & YES & $2.12$ & $(2,4)$ & NO & 2795\\
$(1117,432)$ & 15 & $(287,111)$ & 12 & 1 & YES & YES & YES & $2.22$ & $(2,4)$ & NO & 2796\\
$(1121,254)$ & 16 & $(2,1)$ & 1 & 1 & YES & YES & YES & $2.12$ & $(2,4)$ & -- & 2797\\
$(1420,393)$ & 16 & $(271,75)$ & 12 & 1 & YES & YES & YES & $2.33$ & $(2,4)$ & NO & 2798\\
$(a;1,0,0;13)$ & 5 & $(206,47)$ & 12 & 1 & YES & YES & YES & $2.17$ & $(6,2)$ & -- & 2799\\
$(a;1,1,0;19)$ & 6 & $(82,31)$ & 10 & 1 & YES & YES & NO(2) & $2.12$ & $(4,3)$ & -- & 2800\\
$(a;2,0,0;17)$ & 6 & $(73,31)$ & 10 & 1 & YES & YES & NO(2) & $2.00$ & $(4,3)$ & -- & 2801\\
$(a;3,0,0;7)$ & 7 & $(18,7)$ & 6 & 1 & YES & YES & NO(3) & $1.57$ & $(4,3)$ & -- & 2802\\
$(a;3,0,0;7)$ & 7 & $(24,7)$ & 7 & 1 & YES & YES & NO(3) & $1.57$ & $(4,3)$ & -- & 2803\\
$(a;4,0,1;37)$ & 9 & $(11,4)$ & 5 & 1 & YES & YES & YES & $1.83$ & $(4,3)$ & -- & 2804\\
$(b;0,0,0;14)$ & 5 & $(84,37)$ & 10 & 14 & YES & YES & NO(2) & $1.67$ & $(6,2)$ & -- & 2805\\
$(b;0,0,0;14)$ & 5 & $(101,37)$ & 10 & 1 & YES & YES & NO(2) & $1.67$ & $(6,2)$ & -- & 2806\\
$(b;0,0,1;4)$ & 6 & $(140,41)$ & 11 & 4 & YES & YES & YES & $2.22$ & $(2,4)$ & -- & 2807\\
$(b;0,1,0;19)$ & 6 & $(44,17)$ & 8 & 1 & YES & YES & NO(2) & $1.67$ & $(6,2)$ & -- & 2808\\
$(b;0,1,0;19)$ & 6 & $(56,23)$ & 9 & 1 & YES & YES & NO(2) & $1.67$ & $(6,2)$ & -- & 2809\\
$(b;0,1,0;19)$ & 6 & $(89,27)$ & 10 & 1 & YES & YES & YES & $2.11$ & $(2,4)$ & -- & 2810\\
$(b;0,2,0;8)$ & 7 & $(12,5)$ & 5 & 4 & YES & YES & NO(2) & $1.78$ & $(2,4)$ & -- & 2811\\
$(b;1,3,3;95)$ & 12 & $(5,2)$ & 3 & 5 & YES & YES & NO(2) & $1.67$ & $(10,0)$ & -- & 2812\\
$(b;2,1,0;7)$ & 8 & $(33,10)$ & 8 & 1 & YES & YES & NO(2) & $1.67$ & $(6,2)$ & -- & 2813\\
$(c;0,0,0;4)$ & 4 & $(50,23)$ & 10 & 2 & YES & YES & YES & $1.71$ & $(2,4)$ & -- & 2814\\
$(c;0,0,0;4)$ & 4 & $(61,22)$ & 9 & 1 & YES & YES & NO(2) & $1.90$ & $(2,4)$ & -- & 2815\\
$(c;0,0,0;4)$ & 4 & $(95,39)$ & 10 & 1 & YES & YES & NO(2) & $1.86$ & $(6,2)$ & -- & 2816\\
$(c;0,0,0;4)$ & 4 & $(97,41)$ & 10 & 1 & YES & YES & NO(2) & $2.00$ & $(4,3)$ & -- & 2817\\
$(c;0,0,0;4)$ & 4 & $(301,115)$ & 12 & 1 & YES & YES & YES & $2.17$ & $(6,2)$ & -- & 2818\\
$(c;0,1,0;11)$ & 5 & $(131,47)$ & 11 & 1 & YES & YES & YES & $1.83$ & $(4,3)$ & -- & 2819\\
$(c;0,1,0;11)$ & 5 & $(165,64)$ & 11 & 11 & YES & YES & YES & $2.00$ & $(2,4)$ & -- & 2820\\
$(c;0,1,0;11)$ & 5 & $(186,71)$ & 11 & 1 & YES & YES & NO(2) & $2.12$ & $(2,4)$ & -- & 2821\\
$(c;0,1,0;11)$ & 5 & $(194,75)$ & 11 & 1 & YES & YES & YES & $2.12$ & $(2,4)$ & -- & 2822\\
$(c;0,2,1;19)$ & 7 & $(53,14)$ & 9 & 1 & YES & YES & YES & $1.86$ & $(2,4)$ & -- & 2823\\
$(c;0,2,1;19)$ & 7 & $(116,25)$ & 11 & 1 & YES & YES & NO(2) & $2.00$ & $(4,3)$ & -- & 2824\\
$(d;0,0,0;5)$ & 5 & $(53,19)$ & 9 & 1 & YES & YES & YES & $1.86$ & $(2,4)$ & -- & 2825\\
$(d;0,0,0;5)$ & 5 & $(165,64)$ & 11 & 5 & YES & YES & YES & $2.14$ & $(2,4)$ & -- & 2826\\
$(d;0,0,0;5)$ & 5 & $(199,76)$ & 11 & 1 & YES & YES & YES & $2.00$ & $(6,2)$ & -- & 2827\\
$(d;0,0,0;5)$ & 5 & $(203,59)$ & 12 & 1 & YES & YES & YES & $2.12$ & $(2,4)$ & -- & 2828\\
$(d;0,0,0;5)$ & 5 & $(257,76)$ & 12 & 1 & YES & YES & YES & $2.17$ & $(6,2)$ & -- & 2829\\
$(d;0,0,1;14)$ & 6 & $(60,23)$ & 9 & 2 & YES & YES & NO(2) & $1.88$ & $(4,3)$ & -- & 2830\\
$(d;0,0,1;14)$ & 6 & $(79,23)$ & 10 & 1 & YES & YES & NO(2) & $1.78$ & $(6,2)$ & -- & 2831\\
$(d;0,0,1;14)$ & 6 & $(94,41)$ & 10 & 2 & YES & YES & YES & $2.11$ & $(2,4)$ & -- & 2832\\
$(d;0,0,1;14)$ & 6 & $(119,46)$ & 10 & 7 & YES & YES & YES & $2.12$ & $(2,4)$ & -- & 2833\\
$(d;0,2,1;20)$ & 8 & $(33,10)$ & 8 & 1 & YES & YES & NO(2) & $2.00$ & $(4,3)$ & -- & 2834\\
$(e;0,1,0;5)$ & 6 & $(105,31)$ & 10 & 5 & YES & YES & YES & $2.11$ & $(2,4)$ & -- & 2835\\
$(e;1,3,0;33)$ & 9 & $(23,5)$ & 7 & 1 & YES & YES & YES & $2.00$ & $(2,4)$ & -- & 2836\\
$(e;4,3,0;69)$ & 12 & $(5,2)$ & 3 & 1 & YES & YES & YES & $1.83$ & $(4,3)$ & -- & 2837\\
$(f;0,0,0;6)$ & 4 & $(320,57)$ & 14 & 2 & YES & YES & NO(2) & $2.00$ & $(2,4)$ & -- & 2838\\
$(g;0,2,0;29)$ & 8 & $(23,10)$ & 7 & 1 & YES & YES & NO(2) & $2.00$ & $(4,3)$ & -- & 2839\\
$(g;1,0,2;24)$ & 9 & $(12,5)$ & 5 & 12 & YES & YES & NO(2) & $2.00$ & $(2,4)$ & -- & 2840\\
$(g;1,0,2;24)$ & 9 & $(16,7)$ & 6 & 8 & YES & YES & YES & $2.00$ & $(2,4)$ & -- & 2841\\
$(g;1,0,2;24)$ & 9 & $(22,5)$ & 7 & 2 & YES & YES & NO(2) & $1.71$ & $(8,1)$ & -- & 2842\\
$(g;2,1,3;99)$ & 12 & $(4,1)$ & 3 & 1 & YES & YES & YES & $1.86$ & $(2,4)$ & -- & 2843\\
$(g;2,3,1;19)$ & 12 & $(3,1)$ & 2 & 1 & YES & YES & YES & $1.86$ & $(2,4)$ & -- & 2844\\
$(h;0,0,0;6)$ & 5 & $(24,11)$ & 8 & 6 & YES & YES & YES & $1.71$ & $(2,4)$ & -- & 2845\\
$(i;0,0,0;9)$ & 5 & $(108,29)$ & 10 & 9 & YES & YES & NO(2) & $2.00$ & $(2,4)$ & -- & 2846\\
$(i;0,1,0;12)$ & 6 & $(65,19)$ & 9 & 1 & YES & YES & NO(2) & $1.86$ & $(8,1)$ & -- & 2847\\
$(j;0,1,0;10)$ & 6 & $(106,45)$ & 11 & 2 & YES & YES & NO(2) & $2.00$ & $(4,3)$ & -- & 2848
\end{longtable}
\subsection{2 chains, $K^2 = 5$}
\begin{longtable}{|c|c|c|c|c|c|c|c|c|c|c|c|}
\hline
\multicolumn{12}{|c|}{2 chains, $K^2 = 5$}\\
\hline
$(n,a)$ & Len & $(n,a)$ & Len & GCD & Nef & $\mathbb Q$-ef & Obs 0 & $\overline c_1^2 / \overline c_2$ & $(P,K)$ & WH & Index\\
\hline
\endfirsthead

\hline
$(n,a)$ & Len & $(n,a)$ & Len & GCD & Nef & $\mathbb Q$-ef & Obs 0 & $\overline c_1^2 / \overline c_2$ & $(P,K)$ & WH & Index\\
\hline
\endhead
\hline
\endfoot

$(79,24)$ & 10 & $(64,27)$ & 9 & 1 & YES & YES & NO(3) & $2.40$ & $(6,3)$ & -- & 2849\\
$(251,78)$ & 13 & $(79,24)$ & 10 & 1 & YES & YES & NO(3) & $2.40$ & $(6,3)$ & NO & 2850\\
$(707,254)$ & 14 & $(5,2)$ & 3 & 1 & YES & YES & NO(3) & $2.38$ & $(2,5)$ & -- & 2851\\
$(707,254)$ & 14 & $(142,51)$ & 11 & 1 & YES & YES & NO(3) & $2.38$ & $(2,5)$ & NO & 2852\\
$(1192,503)$ & 15 & $(64,27)$ & 9 & 8 & YES & YES & NO(3) & $2.38$ & $(2,5)$ & NO & 2853\\
$(1233,277)$ & 17 & $(129,29)$ & 12 & 3 & YES & YES & NO(3) & $2.40$ & $(6,3)$ & NO & 2854\\
$(e;1,1,0;23)$ & 7 & $(101,37)$ & 10 & 1 & YES & YES & NO(3) & $2.40$ & $(6,3)$ & -- & 2855\\
$(g;0,0,0;19)$ & 6 & $(119,44)$ & 10 & 1 & YES & YES & NO(3) & $2.38$ & $(2,5)$ & -- & 2856\\
$(g;0,0,1;26)$ & 7 & $(106,41)$ & 10 & 2 & YES & YES & NO(3) & $2.57$ & $(2,5)$ & -- & 2857\\
$(i;0,0,0;9)$ & 5 & $(351,80)$ & 13 & 9 & YES & YES & NO(3) & $2.38$ & $(2,5)$ & -- & 2858
\end{longtable}



%%%%%%%%%%%%%%%%%%%%%%%%%%%%%%%%%%%%%%%%%%%
\section{$I_6 + I_3 + I_2 + I_1$}

Base curves:
\begin{itemize}
  \item $L_1 = x+z$.
  \item $L_2 = x+y$.
  \item $L_3 = y+z$.
  \item $x$.
  \item $y$.
  \item $z$.
  \item $C = xy + xz + yz$
  \item $L = x + y + z$
\end{itemize}
Fibration given by pencil
\[F_\lambda = L_1 L_2 L_3 + \lambda xyz\]

Nine exceptionals are as follows:
\begin{itemize}
  \item $E_1$ - $E_2$ at $z \cap x \cap L_1 = [0,1,0]$.
  \item $E_3$ - $E_4$ at $x \cap y \cap L_2 = [0,0,1]$.
  \item $E_5$ - $E_6$ at $y \cap z \cap L_3 = [1,0,0]$.
  \item $E_7$ at $y \cap L_1 = [-1,0,1]$.
  \item $E_8$ at $x \cap L_3 = [0,-1,1]$.
  \item $E_9$ at $z \cap L_2 = [-1,1,0]$.
\end{itemize}
Singular fibers are as follows:
\begin{itemize}
  \item $\lambda = \infty$: $I_6$ fiber given by $z$, $E_1$, $x$, $E_3$, $y$, $E_5$ in order.
  \item $\lambda = 0$: $I_3$ fiber given by $L_1$, $L_2$, $L_3$.
  \item $\lambda = 1$: $I_2$ fiber given by $C$ and $L$.
  \item $\lambda = -8$: $I_1$ fiber called $F_1$ with node at $[1,1,1]$.
\end{itemize}
Special curves:
\begin{itemize}
  \item $S = x + y - 2z$, double section through $[-1,1,0]$ and $[1,1,1]$
\end{itemize}
Input:
\lstinputlisting[language=config]{../Tests/6321.txt}
Result:
%\usepackage{longtable}
\subsection{1 chain, $K^2 = 1$}
\begin{longtable}{|c|c|c|c|c|c|c|c|}
\hline
\multicolumn{8}{|c|}{1 chain, $K^2 = 1$}\\
\hline
$(n,a)$ & Len & Nef & $\mathbb Q$-ef & Obs 0 & $\overline c_1^2 / \overline c_2$ & $(P,K)$ & Index\\
\hline
\endfirsthead

\hline
$(n,a)$ & Len & Nef & $\mathbb Q$-ef & Obs 0 & $\overline c_1^2 / \overline c_2$ & $(P,K)$ & Index\\
\hline
\endhead
\hline
\endfoot

$(19,8)$ & 6 & YES & YES & YES & $0.64$ & $(1,1)$ & 1\\
$(b;0,0,0;14)$ & 5 & YES & YES & YES & $0.64$ & $(1,1)$ & 2
\end{longtable}
\subsection{1 chain, $K^2 = 2$}
\begin{longtable}{|c|c|c|c|c|c|c|c|}
\hline
\multicolumn{8}{|c|}{1 chain, $K^2 = 2$}\\
\hline
$(n,a)$ & Len & Nef & $\mathbb Q$-ef & Obs 0 & $\overline c_1^2 / \overline c_2$ & $(P,K)$ & Index\\
\hline
\endfirsthead

\hline
$(n,a)$ & Len & Nef & $\mathbb Q$-ef & Obs 0 & $\overline c_1^2 / \overline c_2$ & $(P,K)$ & Index\\
\hline
\endhead
\hline
\endfoot

$(46,19)$ & 8 & YES & YES & YES & $0.90$ & $(1,2)$ & 3\\
$(49,19)$ & 8 & YES & YES & YES & $1.00$ & $(1,2)$ & 4\\
$(49,18)$ & 8 & YES & YES & YES & $1.00$ & $(1,2)$ & 5\\
$(49,15)$ & 9 & YES & YES & YES & $0.89$ & $(3,1)$ & 6\\
$(50,21)$ & 8 & YES & YES & YES & $1.00$ & $(1,2)$ & 7\\
$(50,19)$ & 8 & YES & YES & YES & $0.78$ & $(3,1)$ & 8\\
$(51,20)$ & 9 & YES & YES & YES & $0.78$ & $(3,1)$ & 9\\
$(53,19)$ & 9 & YES & YES & YES & $0.78$ & $(3,1)$ & 10\\
$(55,21)$ & 8 & YES & YES & YES & $0.89$ & $(3,1)$ & 11\\
$(59,18)$ & 9 & YES & YES & YES & $1.00$ & $(1,2)$ & 12\\
$(59,26)$ & 9 & YES & YES & YES & $1.10$ & $(1,2)$ & 13\\
$(61,18)$ & 9 & YES & YES & YES & $0.90$ & $(1,2)$ & 14\\
$(62,19)$ & 10 & YES & YES & YES & $1.10$ & $(1,2)$ & 15\\
$(63,26)$ & 9 & YES & YES & YES & $1.00$ & $(1,2)$ & 16\\
$(64,23)$ & 9 & YES & YES & YES & $1.00$ & $(1,2)$ & 17\\
$(64,27)$ & 9 & YES & YES & YES & $0.89$ & $(3,1)$ & 18\\
$(65,18)$ & 9 & YES & YES & YES & $0.90$ & $(1,2)$ & 19\\
$(65,24)$ & 9 & YES & YES & YES & $1.00$ & $(1,2)$ & 20\\
$(66,25)$ & 9 & YES & YES & YES & $0.89$ & $(1,2)$ & 21\\
$(67,18)$ & 9 & YES & YES & YES & $0.90$ & $(1,2)$ & 22\\
$(68,19)$ & 9 & YES & YES & YES & $0.90$ & $(1,2)$ & 23\\
$(69,19)$ & 9 & YES & YES & YES & $0.90$ & $(1,2)$ & 24\\
$(70,29)$ & 9 & YES & YES & YES & $0.78$ & $(3,1)$ & 25\\
$(71,30)$ & 9 & YES & YES & YES & $1.10$ & $(1,2)$ & 26\\
$(71,31)$ & 10 & YES & YES & YES & $1.00$ & $(3,1)$ & 27\\
$(71,21)$ & 9 & YES & YES & YES & $1.00$ & $(1,2)$ & 28\\
$(72,19)$ & 10 & YES & YES & YES & $1.00$ & $(1,2)$ & 29\\
$(73,27)$ & 9 & YES & YES & YES & $1.00$ & $(1,2)$ & 30\\
$(74,17)$ & 11 & YES & YES & YES & $0.90$ & $(1,2)$ & 31\\
$(74,31)$ & 9 & YES & YES & YES & $0.89$ & $(1,2)$ & 32\\
$(75,31)$ & 9 & YES & YES & YES & $1.00$ & $(1,2)$ & 33\\
$(76,21)$ & 9 & YES & YES & YES & $0.90$ & $(1,2)$ & 34\\
$(76,33)$ & 10 & YES & YES & YES & $0.89$ & $(3,1)$ & 35\\
$(78,29)$ & 10 & YES & YES & YES & $0.89$ & $(3,1)$ & 36\\
$(79,17)$ & 11 & YES & YES & YES & $1.00$ & $(1,2)$ & 37\\
$(79,29)$ & 9 & YES & YES & YES & $1.00$ & $(1,2)$ & 38\\
$(79,30)$ & 9 & YES & YES & YES & $1.00$ & $(1,2)$ & 39\\
$(80,19)$ & 11 & YES & YES & YES & $0.90$ & $(1,2)$ & 40\\
$(80,31)$ & 9 & YES & YES & YES & $1.00$ & $(1,2)$ & 41\\
$(81,19)$ & 11 & YES & YES & YES & $0.90$ & $(1,2)$ & 42\\
$(81,31)$ & 9 & YES & YES & YES & $0.89$ & $(1,2)$ & 43\\
$(82,25)$ & 10 & YES & YES & YES & $1.00$ & $(1,2)$ & 44\\
$(83,22)$ & 10 & YES & YES & YES & $0.90$ & $(1,2)$ & 45\\
$(89,27)$ & 10 & YES & YES & YES & $1.00$ & $(1,2)$ & 46\\
$(90,19)$ & 11 & YES & YES & YES & $1.00$ & $(1,2)$ & 47\\
$(91,17)$ & 12 & YES & YES & YES & $1.10$ & $(1,2)$ & 48\\
$(93,25)$ & 10 & YES & YES & YES & $0.89$ & $(1,2)$ & 49\\
$(96,17)$ & 12 & YES & YES & YES & $1.00$ & $(1,2)$ & 50\\
$(97,26)$ & 10 & YES & YES & YES & $1.00$ & $(1,2)$ & 51\\
$(100,27)$ & 10 & YES & YES & YES & $0.89$ & $(1,2)$ & 52\\
$(101,23)$ & 11 & YES & YES & YES & $0.90$ & $(1,2)$ & 53\\
$(105,31)$ & 10 & YES & YES & YES & $0.78$ & $(1,2)$ & 54\\
$(106,23)$ & 11 & YES & YES & YES & $1.00$ & $(1,2)$ & 55\\
$(123,22)$ & 12 & YES & YES & YES & $0.89$ & $(1,2)$ & 56\\
$(a;2,1,1;37)$ & 8 & YES & YES & YES & $0.78$ & $(1,2)$ & 57\\
$(b;0,0,3;32)$ & 8 & YES & YES & YES & $1.00$ & $(1,2)$ & 58\\
$(b;1,0,1;29)$ & 7 & YES & YES & YES & $0.90$ & $(1,2)$ & 59\\
$(b;1,0,2;19)$ & 8 & YES & YES & YES & $0.90$ & $(1,2)$ & 60\\
$(b;2,0,1;38)$ & 8 & YES & YES & YES & $0.89$ & $(1,2)$ & 61\\
$(e;1,2,0;28)$ & 8 & YES & YES & YES & $0.78$ & $(3,1)$ & 62\\
$(e;3,0,0;10)$ & 8 & YES & YES & YES & $1.00$ & $(1,2)$ & 63\\
$(g;0,0,1;26)$ & 7 & YES & YES & YES & $0.90$ & $(1,2)$ & 64\\
$(g;1,0,0;7)$ & 7 & YES & YES & YES & $0.80$ & $(1,2)$ & 65\\
$(h;0,2,0;10)$ & 7 & YES & YES & YES & $0.80$ & $(1,2)$ & 66
\end{longtable}
\subsection{1 chain, $K^2 = 3$}
\begin{longtable}{|c|c|c|c|c|c|c|c|}
\hline
\multicolumn{8}{|c|}{1 chain, $K^2 = 3$}\\
\hline
$(n,a)$ & Len & Nef & $\mathbb Q$-ef & Obs 0 & $\overline c_1^2 / \overline c_2$ & $(P,K)$ & Index\\
\hline
\endfirsthead

\hline
$(n,a)$ & Len & Nef & $\mathbb Q$-ef & Obs 0 & $\overline c_1^2 / \overline c_2$ & $(P,K)$ & Index\\
\hline
\endhead
\hline
\endfoot

$(111,31)$ & 10 & YES & YES & YES & $1.33$ & $(1,3)$ & 67\\
$(113,42)$ & 11 & YES & YES & YES & $1.25$ & $(3,2)$ & 68\\
$(119,46)$ & 10 & YES & YES & YES & $1.33$ & $(1,3)$ & 69\\
$(128,49)$ & 10 & YES & YES & YES & $1.33$ & $(1,3)$ & 70\\
$(136,59)$ & 11 & YES & YES & YES & $1.33$ & $(1,3)$ & 71\\
$(147,41)$ & 11 & YES & YES & YES & $1.25$ & $(1,3)$ & 72\\
$(151,62)$ & 11 & YES & YES & YES & $1.33$ & $(1,3)$ & 73\\
$(152,55)$ & 12 & YES & YES & YES & $1.25$ & $(3,2)$ & 74\\
$(159,61)$ & 12 & YES & YES & YES & $1.38$ & $(3,2)$ & 75\\
$(161,66)$ & 11 & YES & YES & YES & $1.33$ & $(1,3)$ & 76\\
$(167,60)$ & 11 & YES & YES & YES & $1.25$ & $(3,2)$ & 77\\
$(167,64)$ & 11 & YES & YES & YES & $1.33$ & $(1,3)$ & 78\\
$(169,66)$ & 11 & YES & YES & YES & $1.33$ & $(1,3)$ & 79\\
$(171,65)$ & 11 & YES & YES & YES & $1.25$ & $(3,2)$ & 80\\
$(173,73)$ & 11 & YES & YES & YES & $1.33$ & $(1,3)$ & 81\\
$(173,76)$ & 11 & YES & YES & YES & $1.25$ & $(1,3)$ & 82\\
$(173,78)$ & 12 & YES & YES & YES & $1.12$ & $(3,2)$ & 83\\
$(175,62)$ & 12 & YES & YES & YES & $1.44$ & $(1,3)$ & 84\\
$(176,79)$ & 12 & YES & YES & YES & $1.33$ & $(1,3)$ & 85\\
$(177,74)$ & 12 & YES & YES & YES & $1.44$ & $(1,3)$ & 86\\
$(181,65)$ & 12 & YES & YES & YES & $1.38$ & $(1,3)$ & 87\\
$(183,71)$ & 11 & YES & YES & YES & $1.44$ & $(1,3)$ & 88\\
$(189,50)$ & 13 & YES & YES & YES & $1.44$ & $(1,3)$ & 89\\
$(189,82)$ & 12 & YES & YES & YES & $1.38$ & $(1,3)$ & 90\\
$(191,59)$ & 13 & YES & YES & YES & $1.44$ & $(1,3)$ & 91\\
$(192,71)$ & 11 & YES & YES & YES & $1.38$ & $(1,3)$ & 92\\
$(193,74)$ & 12 & YES & YES & YES & $1.33$ & $(1,3)$ & 93\\
$(193,81)$ & 11 & YES & YES & YES & $1.44$ & $(1,3)$ & 94\\
$(194,71)$ & 12 & YES & YES & YES & $1.38$ & $(1,3)$ & 95\\
$(196,81)$ & 11 & YES & YES & YES & $1.22$ & $(1,3)$ & 96\\
$(198,71)$ & 12 & YES & YES & YES & $1.38$ & $(1,3)$ & 97\\
$(202,59)$ & 12 & YES & YES & YES & $1.33$ & $(1,3)$ & 98\\
$(207,79)$ & 11 & YES & YES & YES & $1.44$ & $(1,3)$ & 99\\
$(207,80)$ & 12 & YES & YES & YES & $1.38$ & $(1,3)$ & 100\\
$(207,91)$ & 12 & YES & YES & YES & $1.44$ & $(1,3)$ & 101\\
$(208,79)$ & 11 & YES & YES & YES & $1.33$ & $(1,3)$ & 102\\
$(209,81)$ & 11 & YES & YES & YES & $1.22$ & $(1,3)$ & 103\\
$(212,93)$ & 12 & YES & YES & YES & $1.38$ & $(1,3)$ & 104\\
$(213,83)$ & 12 & YES & YES & YES & $1.44$ & $(1,3)$ & 105\\
$(214,65)$ & 12 & YES & YES & YES & $1.33$ & $(1,3)$ & 106\\
$(219,65)$ & 12 & YES & YES & YES & $1.22$ & $(1,3)$ & 107\\
$(222,59)$ & 13 & YES & YES & YES & $1.44$ & $(1,3)$ & 108\\
$(222,91)$ & 12 & YES & YES & YES & $1.25$ & $(3,2)$ & 109\\
$(223,98)$ & 12 & YES & YES & YES & $1.25$ & $(1,3)$ & 110\\
$(226,95)$ & 12 & YES & YES & YES & $1.44$ & $(1,3)$ & 111\\
$(229,94)$ & 12 & YES & YES & YES & $1.33$ & $(1,3)$ & 112\\
$(229,97)$ & 12 & YES & YES & YES & $1.44$ & $(1,3)$ & 113\\
$(230,67)$ & 13 & YES & YES & YES & $1.33$ & $(1,3)$ & 114\\
$(230,97)$ & 12 & YES & YES & YES & $1.44$ & $(1,3)$ & 115\\
$(233,86)$ & 12 & YES & YES & YES & $1.44$ & $(1,3)$ & 116\\
$(236,69)$ & 12 & YES & YES & YES & $1.33$ & $(1,3)$ & 117\\
$(243,71)$ & 12 & YES & YES & YES & $1.33$ & $(1,3)$ & 118\\
$(243,106)$ & 12 & YES & YES & YES & $1.44$ & $(1,3)$ & 119\\
$(246,65)$ & 13 & YES & YES & YES & $1.44$ & $(1,3)$ & 120\\
$(248,91)$ & 12 & YES & YES & YES & $1.44$ & $(1,3)$ & 121\\
$(249,77)$ & 13 & YES & YES & YES & $1.44$ & $(1,3)$ & 122\\
$(251,74)$ & 13 & YES & YES & YES & $1.25$ & $(1,3)$ & 123\\
$(256,69)$ & 13 & YES & YES & YES & $1.44$ & $(1,3)$ & 124\\
$(259,76)$ & 13 & YES & YES & YES & $1.38$ & $(1,3)$ & 125\\
$(259,101)$ & 12 & YES & YES & YES & $1.33$ & $(1,3)$ & 126\\
$(263,78)$ & 13 & YES & YES & YES & $1.25$ & $(1,3)$ & 127\\
$(269,78)$ & 13 & YES & YES & YES & $1.25$ & $(1,3)$ & 128\\
$(274,65)$ & 14 & YES & YES & YES & $1.56$ & $(1,3)$ & 129\\
$(290,77)$ & 13 & YES & YES & YES & $1.44$ & $(1,3)$ & 130\\
$(293,62)$ & 14 & YES & YES & YES & $1.44$ & $(1,3)$ & 131\\
$(326,99)$ & 13 & YES & YES & YES & $1.33$ & $(1,3)$ & 132\\
$(327,62)$ & 15 & YES & YES & YES & $1.33$ & $(1,3)$ & 133\\
$(328,71)$ & 13 & YES & YES & YES & $1.33$ & $(1,3)$ & 134\\
$(337,76)$ & 14 & YES & YES & YES & $1.33$ & $(1,3)$ & 135\\
$(346,79)$ & 13 & YES & YES & YES & $1.33$ & $(1,3)$ & 136\\
$(371,88)$ & 14 & YES & YES & YES & $1.44$ & $(1,3)$ & 137\\
$(389,91)$ & 14 & YES & YES & YES & $1.44$ & $(1,3)$ & 138\\
$(424,97)$ & 14 & YES & YES & YES & $1.33$ & $(1,3)$ & 139
\end{longtable}
\subsection{1 chain, $K^2 = 4$}
\begin{longtable}{|c|c|c|c|c|c|c|c|}
\hline
\multicolumn{8}{|c|}{1 chain, $K^2 = 4$}\\
\hline
$(n,a)$ & Len & Nef & $\mathbb Q$-ef & Obs 0 & $\overline c_1^2 / \overline c_2$ & $(P,K)$ & Index\\
\hline
\endfirsthead

\hline
$(n,a)$ & Len & Nef & $\mathbb Q$-ef & Obs 0 & $\overline c_1^2 / \overline c_2$ & $(P,K)$ & Index\\
\hline
\endhead
\hline
\endfoot

$(284,119)$ & 12 & YES & YES & YES & $1.62$ & $(1,4)$ & 140\\
$(405,106)$ & 14 & YES & YES & YES & $1.75$ & $(1,4)$ & 141\\
$(418,147)$ & 15 & YES & YES & YES & $1.71$ & $(3,3)$ & 142\\
$(427,186)$ & 13 & YES & YES & NO(3) & $1.62$ & $(1,4)$ & 143\\
$(437,181)$ & 13 & YES & YES & NO(3) & $1.62$ & $(1,4)$ & 144\\
$(448,171)$ & 13 & YES & YES & YES & $1.57$ & $(3,3)$ & 145\\
$(457,192)$ & 13 & YES & YES & YES & $1.62$ & $(1,4)$ & 146\\
$(540,211)$ & 14 & YES & YES & YES & $1.86$ & $(1,4)$ & 147\\
$(547,167)$ & 14 & YES & YES & YES & $1.75$ & $(1,4)$ & 148\\
$(547,241)$ & 14 & YES & YES & YES & $1.86$ & $(1,4)$ & 149\\
$(571,223)$ & 14 & YES & YES & YES & $1.88$ & $(1,4)$ & 150\\
$(583,239)$ & 14 & YES & YES & YES & $1.86$ & $(1,4)$ & 151\\
$(618,239)$ & 14 & YES & YES & YES & $1.57$ & $(3,3)$ & 152\\
$(707,274)$ & 14 & YES & YES & YES & $1.75$ & $(1,4)$ & 153\\
$(725,277)$ & 14 & YES & YES & YES & $2.00$ & $(1,4)$ & 154\\
$(733,173)$ & 17 & YES & YES & YES & $1.86$ & $(1,4)$ & 155\\
$(782,297)$ & 14 & YES & YES & YES & $1.86$ & $(1,4)$ & 156\\
$(788,301)$ & 14 & YES & YES & YES & $1.88$ & $(1,4)$ & 157\\
$(823,251)$ & 15 & YES & YES & YES & $1.75$ & $(1,4)$ & 158\\
$(842,257)$ & 15 & YES & YES & YES & $1.71$ & $(1,4)$ & 159\\
$(860,263)$ & 15 & YES & YES & YES & $1.86$ & $(1,4)$ & 160
\end{longtable}
\subsection{1 chain, $K^2 = 5$}
\begin{longtable}{|c|c|c|c|c|c|c|c|}
\hline
\multicolumn{8}{|c|}{1 chain, $K^2 = 5$}\\
\hline
$(n,a)$ & Len & Nef & $\mathbb Q$-ef & Obs 0 & $\overline c_1^2 / \overline c_2$ & $(P,K)$ & Index\\
\hline
\endfirsthead

\hline
$(n,a)$ & Len & Nef & $\mathbb Q$-ef & Obs 0 & $\overline c_1^2 / \overline c_2$ & $(P,K)$ & Index\\
\hline
\endhead
\hline
\endfoot

$(1141,482)$ & 15 & YES & YES & NO(3) & $2.17$ & $(1,5)$ & 161
\end{longtable}
\subsection{2 chains, $K^2 = 1$}
\begin{longtable}{|c|c|c|c|c|c|c|c|c|c|c|c|}
\hline
\multicolumn{12}{|c|}{2 chains, $K^2 = 1$}\\
\hline
$(n,a)$ & Len & $(n,a)$ & Len & GCD & Nef & $\mathbb Q$-ef & Obs 0 & $\overline c_1^2 / \overline c_2$ & $(P,K)$ & WH & Index\\
\hline
\endfirsthead

\hline
$(n,a)$ & Len & $(n,a)$ & Len & GCD & Nef & $\mathbb Q$-ef & Obs 0 & $\overline c_1^2 / \overline c_2$ & $(P,K)$ & WH & Index\\
\hline
\endhead
\hline
\endfoot

$(7,3)$ & 4 & $(7,2)$ & 4 & 7 & YES & YES & YES & $0.56$ & $(4,0)$ & NO & 162\\
$(7,3)$ & 4 & $(7,2)$ & 4 & 7 & YES & YES & YES & $0.56$ & $(4,0)$ & -- & 163\\
$(7,3)$ & 4 & $(7,2)$ & 4 & 7 & YES & YES & YES & $0.90$ & $(2,1)$ & NO & 164\\
$(8,3)$ & 4 & $(5,2)$ & 3 & 1 & YES & YES & YES & $0.44$ & $(4,0)$ & -- & 165\\
$(8,3)$ & 4 & $(7,2)$ & 4 & 1 & YES & YES & YES & $0.44$ & $(4,0)$ & NO & 166\\
$(8,3)$ & 4 & $(7,2)$ & 4 & 1 & YES & YES & YES & $0.44$ & $(4,0)$ & -- & 167\\
$(8,3)$ & 4 & $(7,3)$ & 4 & 1 & YES & YES & YES & $0.44$ & $(4,0)$ & -- & 168\\
$(9,4)$ & 5 & $(5,2)$ & 3 & 1 & YES & YES & YES & $0.90$ & $(2,1)$ & -- & 169\\
$(9,4)$ & 5 & $(7,2)$ & 4 & 1 & YES & YES & YES & $0.67$ & $(2,1)$ & NO & 170\\
$(9,4)$ & 5 & $(7,2)$ & 4 & 1 & YES & YES & YES & $0.67$ & $(2,1)$ & -- & 171\\
$(9,4)$ & 5 & $(9,2)$ & 5 & 9 & YES & YES & YES & $0.56$ & $(2,1)$ & -- & 172\\
$(9,4)$ & 5 & $(9,2)$ & 5 & 9 & YES & YES & YES & $0.67$ & $(2,1)$ & NO & 173\\
$(10,3)$ & 5 & $(7,3)$ & 4 & 1 & YES & YES & YES & $0.67$ & $(4,0)$ & -- & 174\\
$(10,3)$ & 5 & $(8,3)$ & 4 & 2 & YES & YES & YES & $0.67$ & $(2,1)$ & -- & 175\\
$(10,3)$ & 5 & $(8,3)$ & 4 & 2 & YES & YES & YES & $0.78$ & $(2,1)$ & NO & 176\\
$(10,3)$ & 5 & $(8,3)$ & 4 & 2 & YES & YES & YES & $0.80$ & $(2,1)$ & NO & 177\\
$(10,3)$ & 5 & $(10,3)$ & 5 & 10 & YES & YES & YES & $0.80$ & $(2,1)$ & -- & 178\\
$(11,4)$ & 5 & $(4,1)$ & 3 & 1 & YES & YES & YES & $0.56$ & $(4,0)$ & NO & 179\\
$(11,4)$ & 5 & $(4,1)$ & 3 & 1 & YES & YES & YES & $0.56$ & $(4,0)$ & -- & 180\\
$(11,3)$ & 5 & $(7,3)$ & 4 & 1 & YES & YES & YES & $0.67$ & $(4,0)$ & -- & 181\\
$(11,4)$ & 5 & $(7,2)$ & 4 & 1 & YES & YES & YES & $0.56$ & $(4,0)$ & -- & 182\\
$(11,3)$ & 5 & $(8,3)$ & 4 & 1 & YES & YES & YES & $0.67$ & $(2,1)$ & -- & 183\\
$(11,3)$ & 5 & $(8,3)$ & 4 & 1 & YES & YES & YES & $0.78$ & $(2,1)$ & NO & 184\\
$(11,4)$ & 5 & $(8,3)$ & 4 & 1 & YES & YES & YES & $0.44$ & $(4,0)$ & NO & 185\\
$(12,5)$ & 5 & $(4,1)$ & 3 & 4 & YES & YES & YES & $0.80$ & $(2,1)$ & NO & 186\\
$(12,5)$ & 5 & $(4,1)$ & 3 & 4 & YES & YES & YES & $0.80$ & $(2,1)$ & NO & 187\\
$(12,5)$ & 5 & $(5,2)$ & 3 & 1 & YES & YES & YES & $0.56$ & $(4,0)$ & -- & 188\\
$(12,5)$ & 5 & $(5,2)$ & 3 & 1 & YES & YES & YES & $0.56$ & $(2,1)$ & NO & 189\\
$(12,5)$ & 5 & $(7,2)$ & 4 & 1 & YES & YES & YES & $0.56$ & $(4,0)$ & NO & 190\\
$(12,5)$ & 5 & $(7,2)$ & 4 & 1 & YES & YES & YES & $0.70$ & $(2,1)$ & -- & 191\\
$(12,5)$ & 5 & $(7,2)$ & 4 & 1 & YES & YES & YES & $0.80$ & $(2,1)$ & NO & 192\\
$(12,5)$ & 5 & $(9,2)$ & 5 & 3 & YES & YES & YES & $0.56$ & $(2,1)$ & NO & 193\\
$(12,5)$ & 5 & $(9,2)$ & 5 & 3 & YES & YES & YES & $0.70$ & $(2,1)$ & NO & 194\\
$(12,5)$ & 5 & $(9,4)$ & 5 & 3 & YES & YES & YES & $0.56$ & $(2,1)$ & NO & 195\\
$(13,5)$ & 5 & $(3,1)$ & 2 & 1 & YES & YES & YES & $0.70$ & $(2,1)$ & NO & 196\\
$(13,5)$ & 5 & $(3,1)$ & 2 & 1 & YES & YES & YES & $0.70$ & $(2,1)$ & -- & 197\\
$(13,4)$ & 6 & $(4,1)$ & 3 & 1 & YES & YES & YES & $0.80$ & $(2,1)$ & -- & 198\\
$(13,4)$ & 6 & $(4,1)$ & 3 & 1 & YES & YES & YES & $0.90$ & $(2,1)$ & NO & 199\\
$(13,5)$ & 5 & $(5,2)$ & 3 & 1 & YES & YES & YES & $0.78$ & $(2,1)$ & -- & 200\\
$(13,3)$ & 6 & $(7,3)$ & 4 & 1 & YES & YES & YES & $0.70$ & $(2,1)$ & NO & 201\\
$(13,4)$ & 6 & $(7,2)$ & 4 & 1 & YES & YES & YES & $0.80$ & $(2,1)$ & 239 & 202\\
$(13,4)$ & 6 & $(7,2)$ & 4 & 1 & YES & YES & YES & $0.80$ & $(2,1)$ & -- & 203\\
$(13,5)$ & 5 & $(7,2)$ & 4 & 1 & YES & YES & YES & $0.44$ & $(4,0)$ & NO & 204\\
$(13,5)$ & 5 & $(7,2)$ & 4 & 1 & YES & YES & YES & $0.70$ & $(2,1)$ & -- & 205\\
$(13,5)$ & 5 & $(7,3)$ & 4 & 1 & YES & YES & YES & $0.56$ & $(4,0)$ & -- & 206\\
$(13,4)$ & 6 & $(11,2)$ & 6 & 1 & YES & YES & YES & $0.56$ & $(2,1)$ & NO & 207\\
$(13,5)$ & 5 & $(11,4)$ & 5 & 1 & YES & YES & YES & $0.44$ & $(4,0)$ & NO & 208\\
$(13,5)$ & 5 & $(13,5)$ & 5 & 13 & YES & YES & YES & $0.70$ & $(2,1)$ & NO & 209\\
$(14,5)$ & 6 & $(3,1)$ & 2 & 1 & YES & YES & YES & $0.70$ & $(2,1)$ & -- & 210\\
$(14,5)$ & 6 & $(3,1)$ & 2 & 1 & YES & YES & YES & $0.80$ & $(2,1)$ & NO & 211\\
$(14,5)$ & 6 & $(4,1)$ & 3 & 2 & YES & YES & YES & $0.70$ & $(2,1)$ & NO & 212\\
$(14,5)$ & 6 & $(5,2)$ & 3 & 1 & YES & YES & YES & $0.80$ & $(2,1)$ & NO & 213\\
$(14,5)$ & 6 & $(6,1)$ & 5 & 2 & YES & YES & YES & $0.80$ & $(2,1)$ & NO & 214\\
$(14,5)$ & 6 & $(8,3)$ & 4 & 2 & YES & YES & YES & $0.80$ & $(2,1)$ & 276 & 215\\
$(14,5)$ & 6 & $(14,5)$ & 6 & 14 & YES & YES & YES & $0.70$ & $(2,1)$ & NO & 216\\
$(15,4)$ & 6 & $(7,2)$ & 4 & 1 & YES & YES & YES & $0.56$ & $(2,1)$ & -- & 217\\
$(15,4)$ & 6 & $(9,2)$ & 5 & 3 & YES & YES & YES & $0.56$ & $(2,1)$ & NO & 218\\
$(15,4)$ & 6 & $(11,2)$ & 6 & 1 & YES & YES & YES & $0.56$ & $(2,1)$ & NO & 219\\
$(16,5)$ & 7 & $(3,1)$ & 2 & 1 & YES & YES & YES & $0.56$ & $(2,1)$ & NO & 220\\
$(16,5)$ & 7 & $(3,1)$ & 2 & 1 & YES & YES & YES & $0.56$ & $(2,1)$ & -- & 221\\
$(16,7)$ & 6 & $(3,1)$ & 2 & 1 & YES & YES & YES & $0.67$ & $(2,1)$ & NO & 222\\
$(16,7)$ & 6 & $(3,1)$ & 2 & 1 & YES & YES & YES & $0.67$ & $(2,1)$ & -- & 223\\
$(16,5)$ & 7 & $(4,1)$ & 3 & 4 & YES & YES & YES & $0.60$ & $(2,1)$ & NO & 224\\
$(16,7)$ & 6 & $(4,1)$ & 3 & 4 & YES & YES & YES & $0.67$ & $(2,1)$ & NO & 225\\
$(16,7)$ & 6 & $(4,1)$ & 3 & 4 & YES & YES & YES & $0.67$ & $(2,1)$ & NO & 226\\
$(16,7)$ & 6 & $(4,1)$ & 3 & 4 & YES & YES & YES & $0.80$ & $(2,1)$ & -- & 227\\
$(16,5)$ & 7 & $(5,1)$ & 4 & 1 & YES & YES & YES & $0.67$ & $(2,1)$ & NO & 228\\
$(16,7)$ & 6 & $(5,2)$ & 3 & 1 & YES & YES & YES & $0.67$ & $(2,1)$ & NO & 229\\
$(16,7)$ & 6 & $(5,2)$ & 3 & 1 & YES & YES & YES & $0.80$ & $(2,1)$ & -- & 230\\
$(16,5)$ & 7 & $(6,1)$ & 5 & 2 & YES & YES & YES & $0.67$ & $(2,1)$ & NO & 231\\
$(16,5)$ & 7 & $(7,2)$ & 4 & 1 & YES & YES & YES & $0.67$ & $(2,1)$ & NO & 232\\
$(16,7)$ & 6 & $(9,4)$ & 5 & 1 & YES & YES & YES & $0.67$ & $(2,1)$ & NO & 233\\
$(16,5)$ & 7 & $(10,3)$ & 5 & 2 & YES & YES & YES & $0.67$ & $(2,1)$ & 315 & 234\\
$(16,5)$ & 7 & $(16,5)$ & 7 & 16 & YES & YES & YES & $0.56$ & $(2,1)$ & NO & 235\\
$(16,7)$ & 6 & $(16,7)$ & 6 & 16 & YES & YES & YES & $0.67$ & $(2,1)$ & NO & 236\\
$(17,7)$ & 6 & $(2,1)$ & 1 & 1 & YES & YES & YES & $0.70$ & $(2,1)$ & -- & 237\\
$(17,7)$ & 6 & $(2,1)$ & 1 & 1 & YES & YES & YES & $0.80$ & $(2,1)$ & NO & 238\\
$(17,5)$ & 6 & $(3,1)$ & 2 & 1 & YES & YES & YES & $0.80$ & $(2,1)$ & 202 & 239\\
$(17,5)$ & 6 & $(3,1)$ & 2 & 1 & YES & YES & YES & $0.80$ & $(2,1)$ & -- & 240\\
$(17,7)$ & 6 & $(3,1)$ & 2 & 1 & YES & YES & YES & $0.67$ & $(4,0)$ & NO & 241\\
$(17,7)$ & 6 & $(3,1)$ & 2 & 1 & YES & YES & YES & $0.67$ & $(2,1)$ & -- & 242\\
$(17,7)$ & 6 & $(3,1)$ & 2 & 1 & YES & YES & YES & $0.70$ & $(2,1)$ & NO & 243\\
$(17,7)$ & 6 & $(4,1)$ & 3 & 1 & YES & YES & YES & $0.44$ & $(4,0)$ & -- & 244\\
$(17,7)$ & 6 & $(4,1)$ & 3 & 1 & YES & YES & YES & $0.56$ & $(4,0)$ & NO & 245\\
$(17,5)$ & 6 & $(5,2)$ & 3 & 1 & YES & YES & YES & $0.56$ & $(2,1)$ & -- & 246\\
$(17,7)$ & 6 & $(5,1)$ & 4 & 1 & YES & YES & YES & $0.56$ & $(4,0)$ & NO & 247\\
$(17,7)$ & 6 & $(5,1)$ & 4 & 1 & YES & YES & YES & $0.56$ & $(4,0)$ & NO & 248\\
$(17,7)$ & 6 & $(7,3)$ & 4 & 1 & YES & YES & YES & $0.67$ & $(4,0)$ & 287 & 249\\
$(17,5)$ & 6 & $(11,3)$ & 5 & 1 & YES & YES & YES & $0.56$ & $(2,1)$ & NO & 250\\
$(17,7)$ & 6 & $(12,5)$ & 5 & 1 & YES & YES & YES & $0.44$ & $(4,0)$ & NO & 251\\
$(17,5)$ & 6 & $(13,4)$ & 6 & 1 & YES & YES & YES & $0.56$ & $(2,1)$ & NO & 252\\
$(17,7)$ & 6 & $(17,7)$ & 6 & 17 & YES & YES & YES & $0.56$ & $(4,0)$ & NO & 253\\
$(18,7)$ & 6 & $(2,1)$ & 1 & 2 & YES & YES & YES & $0.70$ & $(2,1)$ & NO & 254\\
$(18,7)$ & 6 & $(2,1)$ & 1 & 2 & YES & YES & YES & $0.70$ & $(2,1)$ & -- & 255\\
$(18,7)$ & 6 & $(3,1)$ & 2 & 3 & YES & YES & YES & $0.56$ & $(2,1)$ & NO & 256\\
$(18,7)$ & 6 & $(3,1)$ & 2 & 3 & YES & YES & YES & $0.56$ & $(2,1)$ & -- & 257\\
$(18,7)$ & 6 & $(4,1)$ & 3 & 2 & YES & YES & YES & $0.67$ & $(2,1)$ & NO & 258\\
$(18,7)$ & 6 & $(4,1)$ & 3 & 2 & YES & YES & YES & $0.67$ & $(2,1)$ & 320 & 259\\
$(18,7)$ & 6 & $(5,1)$ & 4 & 1 & YES & YES & YES & $0.67$ & $(2,1)$ & NO & 260\\
$(18,7)$ & 6 & $(5,1)$ & 4 & 1 & YES & YES & YES & $0.78$ & $(2,1)$ & NO & 261\\
$(18,7)$ & 6 & $(8,3)$ & 4 & 2 & YES & YES & YES & $0.67$ & $(2,1)$ & 305 & 262\\
$(18,5)$ & 6 & $(9,2)$ & 5 & 9 & YES & YES & YES & $0.44$ & $(2,1)$ & NO & 263\\
$(18,7)$ & 6 & $(13,5)$ & 5 & 1 & YES & YES & YES & $0.67$ & $(2,1)$ & NO & 264\\
$(18,5)$ & 6 & $(15,4)$ & 6 & 3 & YES & YES & YES & $0.44$ & $(2,1)$ & NO & 265\\
$(18,7)$ & 6 & $(18,7)$ & 6 & 18 & YES & YES & YES & $0.56$ & $(2,1)$ & NO & 266\\
$(19,5)$ & 7 & $(2,1)$ & 1 & 1 & YES & YES & YES & $0.70$ & $(2,1)$ & -- & 267\\
$(19,5)$ & 7 & $(2,1)$ & 1 & 1 & YES & YES & YES & $0.80$ & $(2,1)$ & NO & 268\\
$(19,8)$ & 6 & $(2,1)$ & 1 & 1 & YES & YES & YES & $0.67$ & $(4,0)$ & -- & 269\\
$(19,8)$ & 6 & $(2,1)$ & 1 & 1 & YES & YES & YES & $0.78$ & $(2,1)$ & NO & 270\\
$(19,5)$ & 7 & $(3,1)$ & 2 & 1 & YES & YES & YES & $0.67$ & $(2,1)$ & NO & 271\\
$(19,5)$ & 7 & $(3,1)$ & 2 & 1 & YES & YES & YES & $0.67$ & $(2,1)$ & -- & 272\\
$(19,5)$ & 7 & $(3,1)$ & 2 & 1 & YES & YES & YES & $0.70$ & $(2,1)$ & NO & 273\\
$(19,7)$ & 6 & $(3,1)$ & 2 & 1 & YES & YES & YES & $0.56$ & $(4,0)$ & NO & 274\\
$(19,7)$ & 6 & $(3,1)$ & 2 & 1 & YES & YES & YES & $0.56$ & $(4,0)$ & -- & 275\\
$(19,7)$ & 6 & $(3,1)$ & 2 & 1 & YES & YES & YES & $0.80$ & $(2,1)$ & 215 & 276\\
$(19,8)$ & 6 & $(3,1)$ & 2 & 1 & YES & YES & YES & $0.67$ & $(4,0)$ & 297 & 277\\
$(19,8)$ & 6 & $(3,1)$ & 2 & 1 & YES & YES & YES & $0.67$ & $(4,0)$ & -- & 278\\
$(19,8)$ & 6 & $(3,1)$ & 2 & 1 & YES & YES & YES & $0.80$ & $(2,1)$ & NO & 279\\
$(19,8)$ & 6 & $(4,1)$ & 3 & 1 & YES & YES & YES & $0.70$ & $(2,1)$ & NO & 280\\
$(19,8)$ & 6 & $(4,1)$ & 3 & 1 & YES & YES & YES & $0.80$ & $(2,1)$ & NO & 281\\
$(19,8)$ & 6 & $(4,1)$ & 3 & 1 & YES & YES & YES & $0.80$ & $(2,1)$ & -- & 282\\
$(19,5)$ & 7 & $(5,1)$ & 4 & 1 & YES & YES & YES & $0.67$ & $(2,1)$ & NO & 283\\
$(19,7)$ & 6 & $(5,2)$ & 3 & 1 & YES & YES & YES & $0.70$ & $(2,1)$ & NO & 284\\
$(19,8)$ & 6 & $(5,1)$ & 4 & 1 & YES & YES & YES & $0.56$ & $(4,0)$ & NO & 285\\
$(19,8)$ & 6 & $(5,1)$ & 4 & 1 & YES & YES & YES & $0.56$ & $(4,0)$ & NO & 286\\
$(19,8)$ & 6 & $(5,2)$ & 3 & 1 & YES & YES & YES & $0.67$ & $(4,0)$ & 249 & 287\\
$(19,5)$ & 7 & $(6,1)$ & 5 & 1 & YES & YES & YES & $0.56$ & $(2,1)$ & NO & 288\\
$(19,5)$ & 7 & $(7,2)$ & 4 & 1 & YES & YES & YES & $0.56$ & $(2,1)$ & NO & 289\\
$(19,8)$ & 6 & $(7,3)$ & 4 & 1 & YES & YES & YES & $0.67$ & $(4,0)$ & NO & 290\\
$(19,8)$ & 6 & $(12,5)$ & 5 & 1 & YES & YES & YES & $0.70$ & $(2,1)$ & NO & 291\\
$(19,5)$ & 7 & $(15,4)$ & 6 & 1 & YES & YES & YES & $0.56$ & $(2,1)$ & NO & 292\\
$(19,5)$ & 7 & $(19,5)$ & 7 & 19 & YES & YES & YES & $0.67$ & $(2,1)$ & NO & 293\\
$(19,7)$ & 6 & $(19,7)$ & 6 & 19 & YES & YES & YES & $0.44$ & $(4,0)$ & NO & 294\\
$(19,8)$ & 6 & $(19,8)$ & 6 & 19 & YES & YES & YES & $0.56$ & $(4,0)$ & NO & 295\\
$(21,8)$ & 6 & $(2,1)$ & 1 & 1 & YES & YES & YES & $0.67$ & $(2,1)$ & -- & 296\\
$(21,8)$ & 6 & $(2,1)$ & 1 & 1 & YES & YES & YES & $0.67$ & $(4,0)$ & 277 & 297\\
$(21,5)$ & 8 & $(3,1)$ & 2 & 3 & YES & YES & YES & $0.44$ & $(2,1)$ & NO & 298\\
$(21,8)$ & 6 & $(3,1)$ & 2 & 3 & YES & YES & YES & $0.56$ & $(4,0)$ & NO & 299\\
$(21,8)$ & 6 & $(3,1)$ & 2 & 3 & YES & YES & YES & $0.70$ & $(2,1)$ & NO & 300\\
$(21,8)$ & 6 & $(3,1)$ & 2 & 3 & YES & YES & YES & $0.70$ & $(2,1)$ & -- & 301\\
$(21,8)$ & 6 & $(4,1)$ & 3 & 1 & YES & YES & YES & $0.70$ & $(2,1)$ & NO & 302\\
$(21,8)$ & 6 & $(5,1)$ & 4 & 1 & YES & YES & YES & $0.56$ & $(2,1)$ & NO & 303\\
$(21,8)$ & 6 & $(5,1)$ & 4 & 1 & YES & YES & YES & $0.67$ & $(2,1)$ & NO & 304\\
$(21,8)$ & 6 & $(5,2)$ & 3 & 1 & YES & YES & YES & $0.67$ & $(2,1)$ & 262 & 305\\
$(21,8)$ & 6 & $(5,2)$ & 3 & 1 & YES & YES & YES & $0.78$ & $(2,1)$ & -- & 306\\
$(21,5)$ & 8 & $(6,1)$ & 5 & 3 & YES & YES & YES & $0.44$ & $(2,1)$ & NO & 307\\
$(21,8)$ & 6 & $(8,3)$ & 4 & 1 & YES & YES & YES & $0.78$ & $(2,1)$ & NO & 308\\
$(21,5)$ & 8 & $(9,2)$ & 5 & 3 & YES & YES & YES & $0.44$ & $(2,1)$ & NO & 309\\
$(21,8)$ & 6 & $(13,5)$ & 5 & 1 & YES & YES & YES & $0.70$ & $(2,1)$ & NO & 310\\
$(21,8)$ & 6 & $(21,8)$ & 6 & 21 & YES & YES & YES & $0.70$ & $(2,1)$ & NO & 311\\
$(23,10)$ & 7 & $(2,1)$ & 1 & 1 & NO & YES & YES & $0.80$ & $(2,1)$ & -- & 312\\
$(23,7)$ & 7 & $(3,1)$ & 2 & 1 & YES & YES & YES & $0.80$ & $(2,1)$ & -- & 313\\
$(23,7)$ & 7 & $(3,1)$ & 2 & 1 & YES & YES & YES & $0.90$ & $(2,1)$ & NO & 314\\
$(23,7)$ & 7 & $(3,1)$ & 2 & 1 & YES & YES & YES & $0.67$ & $(2,1)$ & 234 & 315\\
$(23,7)$ & 7 & $(5,1)$ & 4 & 1 & YES & YES & YES & $0.56$ & $(2,1)$ & NO & 316\\
$(23,7)$ & 7 & $(7,2)$ & 4 & 1 & YES & YES & YES & $0.60$ & $(2,1)$ & NO & 317\\
$(23,7)$ & 7 & $(23,7)$ & 7 & 23 & YES & YES & YES & $0.56$ & $(2,1)$ & NO & 318\\
$(24,7)$ & 7 & $(2,1)$ & 1 & 2 & YES & YES & YES & $0.67$ & $(2,1)$ & -- & 319\\
$(24,7)$ & 7 & $(2,1)$ & 1 & 2 & YES & YES & YES & $0.67$ & $(2,1)$ & 259 & 320\\
$(24,5)$ & 8 & $(3,1)$ & 2 & 3 & YES & YES & YES & $0.56$ & $(2,1)$ & NO & 321\\
$(24,7)$ & 7 & $(3,1)$ & 2 & 3 & YES & YES & YES & $0.56$ & $(4,0)$ & NO & 322\\
$(24,7)$ & 7 & $(4,1)$ & 3 & 4 & YES & YES & YES & $0.67$ & $(2,1)$ & NO & 323\\
$(24,5)$ & 8 & $(6,1)$ & 5 & 6 & YES & YES & YES & $0.56$ & $(2,1)$ & NO & 324\\
$(24,7)$ & 7 & $(7,2)$ & 4 & 1 & YES & YES & YES & $0.56$ & $(4,0)$ & NO & 325\\
$(24,7)$ & 7 & $(10,3)$ & 5 & 2 & YES & YES & YES & $0.60$ & $(2,1)$ & 342 & 326\\
$(24,5)$ & 8 & $(24,5)$ & 8 & 24 & YES & YES & YES & $0.56$ & $(2,1)$ & NO & 327\\
$(24,7)$ & 7 & $(24,7)$ & 7 & 24 & YES & YES & YES & $0.60$ & $(2,1)$ & NO & 328\\
$(25,7)$ & 7 & $(2,1)$ & 1 & 1 & YES & YES & YES & $0.44$ & $(2,1)$ & NO & 329\\
$(25,7)$ & 7 & $(2,1)$ & 1 & 1 & YES & YES & YES & $0.56$ & $(2,1)$ & -- & 330\\
$(25,11)$ & 7 & $(2,1)$ & 1 & 1 & NO & YES & YES & $0.78$ & $(2,1)$ & -- & 331\\
$(25,7)$ & 7 & $(3,1)$ & 2 & 1 & YES & YES & YES & $0.44$ & $(2,1)$ & NO & 332\\
$(26,7)$ & 7 & $(2,1)$ & 1 & 2 & YES & YES & YES & $0.60$ & $(2,1)$ & -- & 333\\
$(26,7)$ & 7 & $(2,1)$ & 1 & 2 & YES & YES & YES & $0.70$ & $(2,1)$ & NO & 334\\
$(26,7)$ & 7 & $(3,1)$ & 2 & 1 & YES & YES & YES & $0.56$ & $(2,1)$ & NO & 335\\
$(26,7)$ & 7 & $(3,1)$ & 2 & 1 & YES & YES & YES & $0.56$ & $(2,1)$ & -- & 336\\
$(26,7)$ & 7 & $(15,4)$ & 6 & 1 & YES & YES & YES & $0.56$ & $(2,1)$ & NO & 337\\
$(26,7)$ & 7 & $(26,7)$ & 7 & 26 & YES & YES & YES & $0.60$ & $(2,1)$ & NO & 338\\
$(27,8)$ & 7 & $(3,1)$ & 2 & 3 & YES & YES & YES & $0.80$ & $(2,1)$ & NO & 339\\
$(27,8)$ & 7 & $(3,1)$ & 2 & 3 & YES & YES & YES & $0.80$ & $(2,1)$ & -- & 340\\
$(27,8)$ & 7 & $(4,1)$ & 3 & 1 & YES & YES & YES & $0.60$ & $(2,1)$ & 344 & 341\\
$(27,8)$ & 7 & $(7,2)$ & 4 & 1 & YES & YES & YES & $0.60$ & $(2,1)$ & 326 & 342\\
$(29,8)$ & 7 & $(2,1)$ & 1 & 1 & YES & YES & YES & $0.70$ & $(2,1)$ & NO & 343\\
$(29,8)$ & 7 & $(3,1)$ & 2 & 1 & YES & YES & YES & $0.60$ & $(2,1)$ & 341 & 344\\
$(29,8)$ & 7 & $(4,1)$ & 3 & 1 & YES & YES & YES & $0.70$ & $(2,1)$ & NO & 345\\
$(29,8)$ & 7 & $(29,8)$ & 7 & 29 & YES & YES & YES & $0.70$ & $(2,1)$ & NO & 346\\
$(30,7)$ & 8 & $(3,1)$ & 2 & 3 & YES & YES & YES & $0.70$ & $(2,1)$ & NO & 347\\
$(30,7)$ & 8 & $(3,1)$ & 2 & 3 & YES & YES & YES & $0.70$ & $(2,1)$ & -- & 348\\
$(30,7)$ & 8 & $(9,2)$ & 5 & 3 & YES & YES & YES & $0.70$ & $(2,1)$ & NO & 349\\
$(31,7)$ & 8 & $(2,1)$ & 1 & 1 & YES & YES & YES & $0.44$ & $(2,1)$ & NO & 350\\
$(31,7)$ & 8 & $(2,1)$ & 1 & 1 & YES & YES & YES & $0.56$ & $(2,1)$ & -- & 351\\
$(31,13)$ & 7 & $(2,1)$ & 1 & 1 & NO & YES & YES & $0.67$ & $(4,0)$ & -- & 352\\
$(31,7)$ & 8 & $(4,1)$ & 3 & 1 & YES & YES & YES & $0.33$ & $(4,0)$ & NO & 353\\
$(31,7)$ & 8 & $(5,1)$ & 4 & 1 & YES & YES & YES & $0.56$ & $(2,1)$ & NO & 354\\
$(31,7)$ & 8 & $(31,7)$ & 8 & 31 & YES & YES & YES & $0.33$ & $(4,0)$ & NO & 355\\
$(a;0,0,0;3)$ & 4 & $(7,2)$ & 4 & 1 & YES & YES & YES & $0.78$ & $(2,1)$ & -- & 356\\
$(a;1,0,0;13)$ & 5 & $(2,1)$ & 1 & 1 & YES & YES & YES & $0.70$ & $(2,1)$ & -- & 357\\
$(a;1,0,0;13)$ & 5 & $(3,1)$ & 2 & 1 & YES & YES & YES & $0.67$ & $(4,0)$ & -- & 358\\
$(a;1,0,0;13)$ & 5 & $(4,1)$ & 3 & 1 & YES & YES & YES & $0.44$ & $(4,0)$ & -- & 359\\
$(a;1,0,0;13)$ & 5 & $(5,2)$ & 3 & 1 & YES & YES & YES & $0.56$ & $(4,0)$ & -- & 360\\
$(a;1,0,0;13)$ & 5 & $(7,2)$ & 4 & 1 & YES & YES & YES & $0.56$ & $(4,0)$ & -- & 361\\
$(a;1,1,0;19)$ & 6 & $(4,1)$ & 3 & 1 & YES & YES & YES & $0.56$ & $(4,0)$ & -- & 362\\
$(b;0,0,0;14)$ & 5 & $(2,1)$ & 1 & 2 & YES & YES & YES & $0.67$ & $(2,1)$ & -- & 363\\
$(b;0,0,0;14)$ & 5 & $(3,1)$ & 2 & 1 & YES & YES & YES & $0.56$ & $(2,1)$ & -- & 364\\
$(b;0,0,0;14)$ & 5 & $(5,2)$ & 3 & 1 & YES & YES & YES & $0.67$ & $(2,1)$ & -- & 365\\
$(b;0,0,0;14)$ & 5 & $(7,2)$ & 4 & 7 & YES & YES & YES & $0.80$ & $(2,1)$ & -- & 366\\
$(b;0,0,1;4)$ & 6 & $(4,1)$ & 3 & 4 & YES & YES & YES & $0.67$ & $(2,1)$ & -- & 367\\
$(b;0,1,0;19)$ & 6 & $(3,1)$ & 2 & 1 & YES & YES & YES & $0.67$ & $(2,1)$ & -- & 368\\
$(b;0,1,0;19)$ & 6 & $(5,1)$ & 4 & 1 & YES & YES & YES & $0.67$ & $(2,1)$ & -- & 369\\
$(b;1,0,0;5)$ & 6 & $(4,1)$ & 3 & 1 & YES & YES & YES & $0.70$ & $(2,1)$ & -- & 370\\
$(c;0,0,0;4)$ & 4 & $(7,3)$ & 4 & 1 & YES & YES & YES & $0.56$ & $(4,0)$ & -- & 371\\
$(c;0,1,0;11)$ & 5 & $(5,2)$ & 3 & 1 & YES & YES & YES & $0.60$ & $(2,1)$ & -- & 372\\
$(c;0,1,0;11)$ & 5 & $(7,2)$ & 4 & 1 & YES & YES & YES & $0.56$ & $(2,1)$ & -- & 373\\
$(c;0,1,1;5)$ & 6 & $(2,1)$ & 1 & 1 & YES & YES & YES & $0.44$ & $(2,1)$ & -- & 374\\
$(c;0,1,1;5)$ & 6 & $(3,1)$ & 2 & 1 & YES & YES & YES & $0.56$ & $(2,1)$ & -- & 375\\
$(c;0,1,1;5)$ & 6 & $(4,1)$ & 3 & 1 & YES & YES & YES & $0.33$ & $(4,0)$ & -- & 376\\
$(c;0,2,0;7)$ & 6 & $(3,1)$ & 2 & 1 & YES & YES & YES & $0.80$ & $(2,1)$ & -- & 377\\
$(c;0,2,0;7)$ & 6 & $(4,1)$ & 3 & 1 & YES & YES & YES & $0.60$ & $(2,1)$ & -- & 378\\
$(d;0,0,0;5)$ & 5 & $(2,1)$ & 1 & 1 & YES & YES & YES & $0.60$ & $(2,1)$ & -- & 379\\
$(d;0,0,0;5)$ & 5 & $(3,1)$ & 2 & 1 & YES & YES & YES & $0.60$ & $(2,1)$ & -- & 380\\
$(d;0,0,0;5)$ & 5 & $(5,2)$ & 3 & 5 & YES & YES & YES & $0.44$ & $(4,0)$ & -- & 381\\
$(d;0,0,0;5)$ & 5 & $(7,2)$ & 4 & 1 & YES & YES & YES & $0.44$ & $(4,0)$ & -- & 382\\
$(d;0,0,1;14)$ & 6 & $(2,1)$ & 1 & 2 & YES & YES & YES & $0.44$ & $(2,1)$ & -- & 383\\
$(d;0,0,1;14)$ & 6 & $(4,1)$ & 3 & 2 & YES & YES & YES & $0.33$ & $(4,0)$ & -- & 384\\
$(d;0,1,0;6)$ & 6 & $(3,1)$ & 2 & 3 & YES & YES & YES & $0.60$ & $(2,1)$ & -- & 385\\
$(d;0,1,0;6)$ & 6 & $(4,1)$ & 3 & 2 & YES & YES & YES & $0.60$ & $(2,1)$ & -- & 386\\
$(e;0,0,0;4)$ & 5 & $(2,1)$ & 1 & 2 & YES & YES & YES & $0.67$ & $(2,1)$ & -- & 387\\
$(e;0,0,0;4)$ & 5 & $(3,1)$ & 2 & 1 & YES & YES & YES & $0.67$ & $(2,1)$ & -- & 388\\
$(e;0,0,0;4)$ & 5 & $(4,1)$ & 3 & 4 & YES & YES & YES & $0.56$ & $(2,1)$ & -- & 389\\
$(e;0,1,0;5)$ & 6 & $(5,1)$ & 4 & 5 & YES & YES & YES & $0.67$ & $(2,1)$ & -- & 390\\
$(f;0,0,0;6)$ & 4 & $(5,2)$ & 3 & 1 & YES & YES & YES & $0.56$ & $(4,0)$ & -- & 391\\
$(f;0,0,0;6)$ & 4 & $(7,2)$ & 4 & 1 & YES & YES & YES & $0.70$ & $(2,1)$ & -- & 392\\
$(f;0,0,0;6)$ & 4 & $(10,3)$ & 5 & 2 & YES & YES & YES & $0.56$ & $(2,1)$ & -- & 393\\
$(h;0,0,0;6)$ & 5 & $(2,1)$ & 1 & 2 & YES & YES & YES & $0.70$ & $(2,1)$ & -- & 394\\
$(h;0,0,0;6)$ & 5 & $(3,1)$ & 2 & 3 & YES & YES & YES & $0.60$ & $(2,1)$ & -- & 395\\
$(i;0,0,0;9)$ & 5 & $(2,1)$ & 1 & 1 & YES & YES & YES & $0.80$ & $(2,1)$ & -- & 396\\
$(i;0,0,0;9)$ & 5 & $(3,1)$ & 2 & 3 & YES & YES & YES & $0.56$ & $(4,0)$ & -- & 397\\
$(i;0,0,0;9)$ & 5 & $(4,1)$ & 3 & 1 & YES & YES & YES & $0.44$ & $(4,0)$ & -- & 398\\
$(j;0,0,0;8)$ & 5 & $(3,1)$ & 2 & 1 & YES & YES & YES & $0.60$ & $(2,1)$ & -- & 399\\
$(j;0,0,0;8)$ & 5 & $(7,2)$ & 4 & 1 & YES & YES & YES & $0.60$ & $(2,1)$ & -- & 400\\
$(j;0,1,0;10)$ & 6 & $(3,1)$ & 2 & 1 & YES & YES & YES & $0.70$ & $(2,1)$ & -- & 401\\
$(j;0,1,0;10)$ & 6 & $(4,1)$ & 3 & 2 & YES & YES & YES & $0.70$ & $(2,1)$ & -- & 402
\end{longtable}
\subsection{2 chains, $K^2 = 2$}
\begin{longtable}{|c|c|c|c|c|c|c|c|c|c|c|c|}
\hline
\multicolumn{12}{|c|}{2 chains, $K^2 = 2$}\\
\hline
$(n,a)$ & Len & $(n,a)$ & Len & GCD & Nef & $\mathbb Q$-ef & Obs 0 & $\overline c_1^2 / \overline c_2$ & $(P,K)$ & WH & Index\\
\hline
\endfirsthead

\hline
$(n,a)$ & Len & $(n,a)$ & Len & GCD & Nef & $\mathbb Q$-ef & Obs 0 & $\overline c_1^2 / \overline c_2$ & $(P,K)$ & WH & Index\\
\hline
\endhead
\hline
\endfoot

$(11,4)$ & 5 & $(10,3)$ & 5 & 1 & YES & YES & YES & $0.88$ & $(4,1)$ & -- & 403\\
$(13,4)$ & 6 & $(10,3)$ & 5 & 1 & YES & YES & YES & $1.22$ & $(2,2)$ & -- & 404\\
$(13,5)$ & 5 & $(10,3)$ & 5 & 1 & YES & YES & YES & $1.22$ & $(2,2)$ & NO & 405\\
$(13,3)$ & 6 & $(11,4)$ & 5 & 1 & YES & YES & YES & $0.88$ & $(4,1)$ & NO & 406\\
$(13,3)$ & 6 & $(11,4)$ & 5 & 1 & YES & YES & YES & $0.88$ & $(4,1)$ & -- & 407\\
$(13,5)$ & 5 & $(11,3)$ & 5 & 1 & YES & YES & YES & $1.11$ & $(2,2)$ & -- & 408\\
$(13,5)$ & 5 & $(11,4)$ & 5 & 1 & YES & YES & YES & $0.88$ & $(4,1)$ & -- & 409\\
$(13,5)$ & 5 & $(13,3)$ & 6 & 13 & YES & YES & YES & $1.22$ & $(2,2)$ & NO & 410\\
$(13,5)$ & 5 & $(13,3)$ & 6 & 13 & YES & YES & YES & $1.22$ & $(2,2)$ & -- & 411\\
$(13,5)$ & 5 & $(13,3)$ & 6 & 13 & YES & YES & YES & $1.22$ & $(2,2)$ & NO & 412\\
$(13,5)$ & 5 & $(13,4)$ & 6 & 13 & YES & YES & YES & $1.00$ & $(2,2)$ & NO & 413\\
$(13,5)$ & 5 & $(13,4)$ & 6 & 13 & YES & YES & YES & $1.00$ & $(2,2)$ & -- & 414\\
$(14,5)$ & 6 & $(10,3)$ & 5 & 2 & YES & YES & YES & $0.88$ & $(4,1)$ & -- & 415\\
$(14,5)$ & 6 & $(11,3)$ & 5 & 1 & YES & YES & YES & $0.88$ & $(4,1)$ & NO & 416\\
$(14,5)$ & 6 & $(11,3)$ & 5 & 1 & YES & YES & YES & $0.88$ & $(4,1)$ & -- & 417\\
$(14,3)$ & 6 & $(13,4)$ & 6 & 1 & YES & YES & YES & $0.88$ & $(4,1)$ & NO & 418\\
$(14,3)$ & 6 & $(13,4)$ & 6 & 1 & YES & YES & YES & $0.88$ & $(4,1)$ & -- & 419\\
$(15,4)$ & 6 & $(13,6)$ & 7 & 1 & YES & YES & YES & $1.00$ & $(4,1)$ & NO & 420\\
$(15,4)$ & 6 & $(13,6)$ & 7 & 1 & YES & YES & YES & $1.00$ & $(4,1)$ & -- & 421\\
$(16,7)$ & 6 & $(8,3)$ & 4 & 8 & YES & YES & YES & $0.88$ & $(4,1)$ & -- & 422\\
$(16,5)$ & 7 & $(10,3)$ & 5 & 2 & YES & YES & YES & $1.22$ & $(2,2)$ & -- & 423\\
$(16,7)$ & 6 & $(11,5)$ & 6 & 1 & YES & YES & YES & $1.00$ & $(2,2)$ & -- & 424\\
$(16,5)$ & 7 & $(12,5)$ & 5 & 4 & YES & YES & YES & $0.88$ & $(2,2)$ & -- & 425\\
$(16,7)$ & 6 & $(12,5)$ & 5 & 4 & YES & YES & YES & $0.88$ & $(2,2)$ & -- & 426\\
$(16,7)$ & 6 & $(13,5)$ & 5 & 1 & YES & YES & YES & $0.88$ & $(4,1)$ & -- & 427\\
$(16,3)$ & 7 & $(14,5)$ & 6 & 2 & YES & YES & YES & $0.88$ & $(4,1)$ & NO & 428\\
$(16,3)$ & 7 & $(14,5)$ & 6 & 2 & YES & YES & YES & $0.88$ & $(4,1)$ & -- & 429\\
$(16,7)$ & 6 & $(14,5)$ & 6 & 2 & YES & YES & YES & $1.00$ & $(2,2)$ & -- & 430\\
$(16,7)$ & 6 & $(14,5)$ & 6 & 2 & YES & YES & YES & $1.12$ & $(2,2)$ & NO & 431\\
$(17,7)$ & 6 & $(7,2)$ & 4 & 1 & YES & YES & YES & $0.88$ & $(4,1)$ & -- & 432\\
$(17,7)$ & 6 & $(10,3)$ & 5 & 1 & YES & YES & YES & $0.88$ & $(4,1)$ & -- & 433\\
$(17,7)$ & 6 & $(10,3)$ & 5 & 1 & YES & YES & YES & $0.88$ & $(4,1)$ & NO & 434\\
$(17,5)$ & 6 & $(11,3)$ & 5 & 1 & YES & YES & YES & $0.88$ & $(4,1)$ & -- & 435\\
$(17,5)$ & 6 & $(11,3)$ & 5 & 1 & YES & YES & YES & $0.88$ & $(4,1)$ & NO & 436\\
$(17,5)$ & 6 & $(11,4)$ & 5 & 1 & YES & YES & YES & $0.88$ & $(4,1)$ & -- & 437\\
$(17,5)$ & 6 & $(11,4)$ & 5 & 1 & YES & YES & YES & $1.11$ & $(2,2)$ & NO & 438\\
$(17,5)$ & 6 & $(12,5)$ & 5 & 1 & YES & YES & YES & $1.22$ & $(2,2)$ & NO & 439\\
$(17,5)$ & 6 & $(12,5)$ & 5 & 1 & YES & YES & YES & $1.22$ & $(2,2)$ & -- & 440\\
$(17,5)$ & 6 & $(13,4)$ & 6 & 1 & YES & YES & YES & $1.22$ & $(2,2)$ & -- & 441\\
$(17,5)$ & 6 & $(13,5)$ & 5 & 1 & YES & YES & YES & $1.00$ & $(2,2)$ & NO & 442\\
$(17,5)$ & 6 & $(13,5)$ & 5 & 1 & YES & YES & YES & $1.00$ & $(2,2)$ & -- & 443\\
$(17,7)$ & 6 & $(13,5)$ & 5 & 1 & YES & YES & YES & $1.00$ & $(2,2)$ & NO & 444\\
$(17,7)$ & 6 & $(13,5)$ & 5 & 1 & YES & YES & YES & $1.00$ & $(2,2)$ & -- & 445\\
$(17,4)$ & 7 & $(14,5)$ & 6 & 1 & YES & YES & YES & $1.22$ & $(2,2)$ & NO & 446\\
$(17,6)$ & 7 & $(15,4)$ & 6 & 1 & YES & YES & YES & $0.88$ & $(4,1)$ & NO & 447\\
$(17,5)$ & 6 & $(16,7)$ & 6 & 1 & YES & YES & YES & $1.11$ & $(2,2)$ & -- & 448\\
$(17,7)$ & 6 & $(16,7)$ & 6 & 1 & YES & YES & YES & $0.88$ & $(4,1)$ & NO & 449\\
$(18,7)$ & 6 & $(7,2)$ & 4 & 1 & YES & YES & YES & $1.22$ & $(2,2)$ & NO & 450\\
$(18,7)$ & 6 & $(7,2)$ & 4 & 1 & YES & YES & YES & $1.22$ & $(2,2)$ & -- & 451\\
$(18,7)$ & 6 & $(8,3)$ & 4 & 2 & YES & YES & YES & $1.11$ & $(2,2)$ & NO & 452\\
$(18,7)$ & 6 & $(8,3)$ & 4 & 2 & YES & YES & YES & $1.11$ & $(2,2)$ & -- & 453\\
$(18,7)$ & 6 & $(9,2)$ & 5 & 9 & YES & YES & YES & $0.88$ & $(4,1)$ & NO & 454\\
$(18,7)$ & 6 & $(9,2)$ & 5 & 9 & YES & YES & YES & $0.88$ & $(4,1)$ & -- & 455\\
$(18,5)$ & 6 & $(10,3)$ & 5 & 2 & YES & YES & YES & $0.88$ & $(4,1)$ & NO & 456\\
$(18,5)$ & 6 & $(10,3)$ & 5 & 2 & YES & YES & YES & $0.88$ & $(4,1)$ & -- & 457\\
$(18,7)$ & 6 & $(10,3)$ & 5 & 2 & YES & YES & YES & $0.88$ & $(4,1)$ & -- & 458\\
$(18,5)$ & 6 & $(11,3)$ & 5 & 1 & YES & YES & YES & $0.88$ & $(4,1)$ & 460 & 459\\
$(18,5)$ & 6 & $(11,3)$ & 5 & 1 & YES & YES & YES & $0.88$ & $(4,1)$ & 459 & 460\\
$(18,5)$ & 6 & $(11,3)$ & 5 & 1 & YES & YES & YES & $0.88$ & $(4,1)$ & -- & 461\\
$(18,7)$ & 6 & $(12,5)$ & 5 & 6 & YES & YES & YES & $1.00$ & $(4,1)$ & NO & 462\\
$(18,7)$ & 6 & $(12,5)$ & 5 & 6 & YES & YES & YES & $1.00$ & $(4,1)$ & -- & 463\\
$(18,5)$ & 6 & $(13,4)$ & 6 & 1 & YES & YES & YES & $1.25$ & $(2,2)$ & NO & 464\\
$(18,5)$ & 6 & $(13,4)$ & 6 & 1 & YES & YES & YES & $1.25$ & $(2,2)$ & -- & 465\\
$(18,7)$ & 6 & $(13,6)$ & 7 & 1 & YES & YES & YES & $1.00$ & $(4,1)$ & 972 & 466\\
$(18,5)$ & 6 & $(16,5)$ & 7 & 2 & YES & YES & YES & $1.12$ & $(2,2)$ & -- & 467\\
$(18,5)$ & 6 & $(16,7)$ & 6 & 2 & YES & YES & YES & $1.11$ & $(2,2)$ & -- & 468\\
$(18,5)$ & 6 & $(16,7)$ & 6 & 2 & YES & YES & YES & $0.88$ & $(4,1)$ & NO & 469\\
$(18,7)$ & 6 & $(17,5)$ & 6 & 1 & YES & YES & YES & $1.11$ & $(2,2)$ & -- & 470\\
$(19,7)$ & 6 & $(7,2)$ & 4 & 1 & YES & YES & YES & $1.22$ & $(2,2)$ & NO & 471\\
$(19,8)$ & 6 & $(7,2)$ & 4 & 1 & YES & YES & YES & $1.11$ & $(2,2)$ & NO & 472\\
$(19,7)$ & 6 & $(8,3)$ & 4 & 1 & YES & YES & YES & $0.88$ & $(4,1)$ & -- & 473\\
$(19,8)$ & 6 & $(8,3)$ & 4 & 1 & YES & YES & YES & $1.11$ & $(2,2)$ & NO & 474\\
$(19,7)$ & 6 & $(10,3)$ & 5 & 1 & YES & YES & YES & $1.00$ & $(4,1)$ & -- & 475\\
$(19,8)$ & 6 & $(10,3)$ & 5 & 1 & YES & YES & YES & $1.00$ & $(2,2)$ & -- & 476\\
$(19,7)$ & 6 & $(11,3)$ & 5 & 1 & YES & YES & YES & $0.88$ & $(4,1)$ & NO & 477\\
$(19,7)$ & 6 & $(11,3)$ & 5 & 1 & YES & YES & YES & $0.88$ & $(4,1)$ & -- & 478\\
$(19,7)$ & 6 & $(12,5)$ & 5 & 1 & YES & YES & YES & $0.88$ & $(2,2)$ & -- & 479\\
$(19,8)$ & 6 & $(13,5)$ & 5 & 1 & YES & YES & YES & $1.11$ & $(2,2)$ & -- & 480\\
$(19,4)$ & 7 & $(17,7)$ & 6 & 1 & YES & YES & YES & $1.12$ & $(2,2)$ & NO & 481\\
$(19,4)$ & 7 & $(17,7)$ & 6 & 1 & YES & YES & YES & $1.12$ & $(2,2)$ & -- & 482\\
$(19,7)$ & 6 & $(17,4)$ & 7 & 1 & YES & YES & YES & $1.11$ & $(2,2)$ & NO & 483\\
$(19,5)$ & 7 & $(18,5)$ & 6 & 1 & YES & YES & YES & $0.88$ & $(4,1)$ & -- & 484\\
$(19,7)$ & 6 & $(18,7)$ & 6 & 1 & YES & YES & YES & $0.88$ & $(4,1)$ & NO & 485\\
$(20,9)$ & 7 & $(9,4)$ & 5 & 1 & YES & YES & YES & $1.12$ & $(2,2)$ & -- & 486\\
$(20,9)$ & 7 & $(14,5)$ & 6 & 2 & YES & YES & YES & $1.12$ & $(2,2)$ & NO & 487\\
$(21,8)$ & 6 & $(7,2)$ & 4 & 7 & YES & YES & YES & $1.00$ & $(2,2)$ & -- & 488\\
$(21,8)$ & 6 & $(8,3)$ & 4 & 1 & YES & YES & YES & $0.75$ & $(4,1)$ & -- & 489\\
$(21,8)$ & 6 & $(9,4)$ & 5 & 3 & YES & YES & YES & $0.88$ & $(4,1)$ & -- & 490\\
$(21,8)$ & 6 & $(10,3)$ & 5 & 1 & YES & YES & YES & $1.00$ & $(2,2)$ & -- & 491\\
$(21,8)$ & 6 & $(10,3)$ & 5 & 1 & YES & YES & YES & $1.00$ & $(2,2)$ & NO & 492\\
$(21,8)$ & 6 & $(12,5)$ & 5 & 3 & YES & YES & YES & $1.22$ & $(2,2)$ & NO & 493\\
$(21,8)$ & 6 & $(12,5)$ & 5 & 3 & YES & YES & YES & $1.22$ & $(2,2)$ & -- & 494\\
$(21,8)$ & 6 & $(14,5)$ & 6 & 7 & YES & YES & YES & $0.88$ & $(4,1)$ & NO & 495\\
$(21,5)$ & 8 & $(16,7)$ & 6 & 1 & YES & YES & YES & $1.12$ & $(2,2)$ & NO & 496\\
$(21,5)$ & 8 & $(16,7)$ & 6 & 1 & YES & YES & YES & $1.12$ & $(2,2)$ & -- & 497\\
$(21,8)$ & 6 & $(16,7)$ & 6 & 1 & YES & YES & YES & $0.88$ & $(4,1)$ & NO & 498\\
$(22,5)$ & 7 & $(11,4)$ & 5 & 11 & YES & YES & YES & $1.00$ & $(4,1)$ & -- & 499\\
$(22,9)$ & 7 & $(13,3)$ & 6 & 1 & YES & YES & YES & $1.11$ & $(2,2)$ & -- & 500\\
$(22,5)$ & 7 & $(14,5)$ & 6 & 2 & YES & YES & YES & $0.88$ & $(4,1)$ & NO & 501\\
$(22,9)$ & 7 & $(14,3)$ & 6 & 2 & YES & YES & YES & $1.11$ & $(2,2)$ & -- & 502\\
$(22,5)$ & 7 & $(16,5)$ & 7 & 2 & YES & YES & YES & $1.22$ & $(2,2)$ & -- & 503\\
$(22,5)$ & 7 & $(17,7)$ & 6 & 1 & YES & YES & YES & $1.00$ & $(2,2)$ & -- & 504\\
$(22,9)$ & 7 & $(18,7)$ & 6 & 2 & YES & YES & YES & $1.00$ & $(4,1)$ & NO & 505\\
$(22,5)$ & 7 & $(19,5)$ & 7 & 1 & YES & YES & YES & $1.22$ & $(2,2)$ & -- & 506\\
$(23,9)$ & 7 & $(5,1)$ & 4 & 1 & YES & YES & YES & $0.88$ & $(4,1)$ & NO & 507\\
$(23,9)$ & 7 & $(5,2)$ & 3 & 1 & YES & YES & YES & $1.22$ & $(2,2)$ & -- & 508\\
$(23,7)$ & 7 & $(7,2)$ & 4 & 1 & YES & YES & YES & $1.11$ & $(2,2)$ & -- & 509\\
$(23,9)$ & 7 & $(7,2)$ & 4 & 1 & YES & YES & YES & $1.00$ & $(2,2)$ & -- & 510\\
$(23,9)$ & 7 & $(7,2)$ & 4 & 1 & YES & YES & YES & $1.00$ & $(2,2)$ & NO & 511\\
$(23,9)$ & 7 & $(7,3)$ & 4 & 1 & YES & YES & YES & $1.22$ & $(2,2)$ & NO & 512\\
$(23,9)$ & 7 & $(7,3)$ & 4 & 1 & YES & YES & YES & $1.22$ & $(2,2)$ & -- & 513\\
$(23,5)$ & 7 & $(9,4)$ & 5 & 1 & YES & YES & YES & $1.11$ & $(2,2)$ & NO & 514\\
$(23,5)$ & 7 & $(9,4)$ & 5 & 1 & YES & YES & YES & $1.11$ & $(2,2)$ & -- & 515\\
$(23,7)$ & 7 & $(10,3)$ & 5 & 1 & YES & YES & YES & $1.00$ & $(2,2)$ & -- & 516\\
$(23,7)$ & 7 & $(10,3)$ & 5 & 1 & YES & YES & YES & $1.12$ & $(2,2)$ & NO & 517\\
$(23,5)$ & 7 & $(11,4)$ & 5 & 1 & YES & YES & YES & $1.00$ & $(2,2)$ & -- & 518\\
$(23,5)$ & 7 & $(11,4)$ & 5 & 1 & YES & YES & YES & $1.00$ & $(2,2)$ & NO & 519\\
$(23,7)$ & 7 & $(11,3)$ & 5 & 1 & YES & YES & YES & $1.11$ & $(2,2)$ & NO & 520\\
$(23,7)$ & 7 & $(11,3)$ & 5 & 1 & YES & YES & YES & $1.11$ & $(2,2)$ & -- & 521\\
$(23,9)$ & 7 & $(13,3)$ & 6 & 1 & YES & YES & YES & $1.00$ & $(2,2)$ & -- & 522\\
$(23,9)$ & 7 & $(13,3)$ & 6 & 1 & YES & YES & YES & $1.22$ & $(2,2)$ & NO & 523\\
$(23,5)$ & 7 & $(14,5)$ & 6 & 1 & YES & YES & YES & $1.12$ & $(2,2)$ & NO & 524\\
$(23,9)$ & 7 & $(14,3)$ & 6 & 1 & YES & YES & YES & $1.11$ & $(2,2)$ & -- & 525\\
$(23,5)$ & 7 & $(16,5)$ & 7 & 1 & YES & YES & YES & $1.12$ & $(2,2)$ & NO & 526\\
$(23,5)$ & 7 & $(16,5)$ & 7 & 1 & YES & YES & YES & $1.11$ & $(2,2)$ & -- & 527\\
$(23,9)$ & 7 & $(17,7)$ & 6 & 1 & YES & YES & YES & $0.88$ & $(4,1)$ & NO & 528\\
$(23,4)$ & 8 & $(19,5)$ & 7 & 1 & YES & YES & YES & $1.00$ & $(2,2)$ & -- & 529\\
$(23,5)$ & 7 & $(19,5)$ & 7 & 1 & YES & YES & YES & $1.11$ & $(2,2)$ & -- & 530\\
$(23,9)$ & 7 & $(21,8)$ & 6 & 1 & YES & YES & YES & $1.00$ & $(2,2)$ & 863 & 531\\
$(24,7)$ & 7 & $(4,1)$ & 3 & 4 & YES & YES & YES & $0.88$ & $(4,1)$ & NO & 532\\
$(24,7)$ & 7 & $(8,3)$ & 4 & 8 & YES & YES & YES & $0.89$ & $(2,2)$ & -- & 533\\
$(24,7)$ & 7 & $(9,4)$ & 5 & 3 & YES & YES & YES & $1.00$ & $(4,1)$ & NO & 534\\
$(24,7)$ & 7 & $(9,4)$ & 5 & 3 & YES & YES & YES & $1.00$ & $(4,1)$ & -- & 535\\
$(24,7)$ & 7 & $(10,3)$ & 5 & 2 & YES & YES & YES & $1.00$ & $(2,2)$ & -- & 536\\
$(24,5)$ & 8 & $(11,5)$ & 6 & 1 & YES & YES & YES & $1.25$ & $(2,2)$ & NO & 537\\
$(24,5)$ & 8 & $(11,5)$ & 6 & 1 & YES & YES & YES & $1.25$ & $(2,2)$ & -- & 538\\
$(24,7)$ & 7 & $(12,5)$ & 5 & 12 & YES & YES & YES & $1.12$ & $(2,2)$ & -- & 539\\
$(24,5)$ & 8 & $(13,4)$ & 6 & 1 & YES & YES & YES & $1.11$ & $(2,2)$ & -- & 540\\
$(24,5)$ & 8 & $(15,4)$ & 6 & 3 & YES & YES & YES & $1.00$ & $(2,2)$ & -- & 541\\
$(25,9)$ & 7 & $(5,2)$ & 3 & 5 & YES & YES & YES & $0.88$ & $(4,1)$ & -- & 542\\
$(25,11)$ & 7 & $(7,3)$ & 4 & 1 & YES & YES & YES & $1.00$ & $(2,2)$ & -- & 543\\
$(25,11)$ & 7 & $(8,3)$ & 4 & 1 & YES & YES & YES & $1.12$ & $(2,2)$ & -- & 544\\
$(25,11)$ & 7 & $(8,3)$ & 4 & 1 & YES & YES & YES & $1.11$ & $(2,2)$ & 827 & 545\\
$(25,7)$ & 7 & $(9,4)$ & 5 & 1 & YES & YES & YES & $1.00$ & $(4,1)$ & -- & 546\\
$(25,9)$ & 7 & $(9,4)$ & 5 & 1 & YES & YES & YES & $1.12$ & $(2,2)$ & NO & 547\\
$(25,9)$ & 7 & $(9,4)$ & 5 & 1 & YES & YES & YES & $1.12$ & $(2,2)$ & -- & 548\\
$(25,11)$ & 7 & $(9,4)$ & 5 & 1 & YES & YES & YES & $1.00$ & $(2,2)$ & -- & 549\\
$(25,9)$ & 7 & $(10,3)$ & 5 & 5 & YES & YES & YES & $0.88$ & $(2,2)$ & -- & 550\\
$(25,11)$ & 7 & $(10,3)$ & 5 & 5 & YES & YES & YES & $1.22$ & $(2,2)$ & -- & 551\\
$(25,9)$ & 7 & $(11,3)$ & 5 & 1 & YES & YES & YES & $0.88$ & $(4,1)$ & -- & 552\\
$(25,11)$ & 7 & $(11,3)$ & 5 & 1 & YES & YES & YES & $1.12$ & $(2,2)$ & NO & 553\\
$(25,11)$ & 7 & $(11,3)$ & 5 & 1 & YES & YES & YES & $1.12$ & $(2,2)$ & -- & 554\\
$(25,7)$ & 7 & $(12,5)$ & 5 & 1 & YES & YES & YES & $0.88$ & $(4,1)$ & -- & 555\\
$(25,7)$ & 7 & $(13,3)$ & 6 & 1 & YES & YES & NO(3) & $0.75$ & $(2,2)$ & -- & 556\\
$(25,7)$ & 7 & $(13,4)$ & 6 & 1 & YES & YES & YES & $1.11$ & $(2,2)$ & -- & 557\\
$(25,9)$ & 7 & $(13,5)$ & 5 & 1 & YES & YES & YES & $1.11$ & $(2,2)$ & NO & 558\\
$(25,11)$ & 7 & $(13,5)$ & 5 & 1 & YES & YES & YES & $1.12$ & $(2,2)$ & NO & 559\\
$(25,6)$ & 9 & $(16,7)$ & 6 & 1 & YES & YES & YES & $1.12$ & $(2,2)$ & -- & 560\\
$(25,11)$ & 7 & $(17,7)$ & 6 & 1 & YES & YES & YES & $1.00$ & $(2,2)$ & NO & 561\\
$(25,7)$ & 7 & $(19,4)$ & 7 & 1 & YES & YES & YES & $1.11$ & $(2,2)$ & -- & 562\\
$(25,9)$ & 7 & $(19,7)$ & 6 & 1 & YES & YES & YES & $0.75$ & $(4,1)$ & NO & 563\\
$(25,6)$ & 9 & $(21,4)$ & 8 & 1 & YES & YES & YES & $1.12$ & $(2,2)$ & NO & 564\\
$(25,6)$ & 9 & $(24,5)$ & 8 & 1 & YES & YES & YES & $1.12$ & $(2,2)$ & NO & 565\\
$(26,11)$ & 7 & $(5,1)$ & 4 & 1 & YES & YES & YES & $1.00$ & $(4,1)$ & NO & 566\\
$(26,11)$ & 7 & $(5,1)$ & 4 & 1 & YES & YES & YES & $1.00$ & $(4,1)$ & -- & 567\\
$(26,11)$ & 7 & $(5,2)$ & 3 & 1 & YES & YES & YES & $1.11$ & $(2,2)$ & -- & 568\\
$(26,11)$ & 7 & $(7,2)$ & 4 & 1 & YES & YES & YES & $1.00$ & $(4,1)$ & NO & 569\\
$(26,11)$ & 7 & $(7,2)$ & 4 & 1 & YES & YES & YES & $1.00$ & $(4,1)$ & -- & 570\\
$(26,11)$ & 7 & $(7,3)$ & 4 & 1 & YES & YES & YES & $1.12$ & $(2,2)$ & -- & 571\\
$(26,7)$ & 7 & $(8,3)$ & 4 & 2 & YES & YES & YES & $1.00$ & $(2,2)$ & -- & 572\\
$(26,7)$ & 7 & $(8,3)$ & 4 & 2 & YES & YES & YES & $1.00$ & $(2,2)$ & NO & 573\\
$(26,11)$ & 7 & $(8,3)$ & 4 & 2 & YES & YES & YES & $0.88$ & $(4,1)$ & -- & 574\\
$(26,11)$ & 7 & $(8,3)$ & 4 & 2 & YES & YES & YES & $1.00$ & $(4,1)$ & NO & 575\\
$(26,11)$ & 7 & $(8,3)$ & 4 & 2 & YES & YES & YES & $1.11$ & $(2,2)$ & 916 & 576\\
$(26,7)$ & 7 & $(10,3)$ & 5 & 2 & YES & YES & YES & $0.89$ & $(2,2)$ & -- & 577\\
$(26,11)$ & 7 & $(10,3)$ & 5 & 2 & YES & YES & YES & $0.88$ & $(4,1)$ & -- & 578\\
$(26,11)$ & 7 & $(11,3)$ & 5 & 1 & YES & YES & YES & $1.22$ & $(2,2)$ & -- & 579\\
$(26,11)$ & 7 & $(13,3)$ & 6 & 13 & YES & YES & YES & $1.00$ & $(2,2)$ & -- & 580\\
$(27,11)$ & 8 & $(5,2)$ & 3 & 1 & YES & YES & YES & $1.22$ & $(2,2)$ & -- & 581\\
$(27,8)$ & 7 & $(7,2)$ & 4 & 1 & YES & YES & YES & $1.12$ & $(2,2)$ & -- & 582\\
$(27,8)$ & 7 & $(7,2)$ & 4 & 1 & YES & YES & YES & $1.25$ & $(2,2)$ & NO & 583\\
$(27,8)$ & 7 & $(7,3)$ & 4 & 1 & YES & YES & YES & $1.22$ & $(2,2)$ & NO & 584\\
$(27,10)$ & 7 & $(7,2)$ & 4 & 1 & YES & YES & YES & $1.00$ & $(2,2)$ & -- & 585\\
$(27,10)$ & 7 & $(7,2)$ & 4 & 1 & YES & YES & YES & $1.00$ & $(2,2)$ & NO & 586\\
$(27,11)$ & 8 & $(7,2)$ & 4 & 1 & YES & YES & YES & $1.00$ & $(2,2)$ & -- & 587\\
$(27,8)$ & 7 & $(9,4)$ & 5 & 9 & YES & YES & YES & $1.22$ & $(2,2)$ & NO & 588\\
$(27,8)$ & 7 & $(9,4)$ & 5 & 9 & YES & YES & YES & $1.22$ & $(2,2)$ & -- & 589\\
$(27,8)$ & 7 & $(9,4)$ & 5 & 9 & YES & YES & YES & $1.00$ & $(2,2)$ & NO & 590\\
$(27,11)$ & 8 & $(9,2)$ & 5 & 9 & YES & YES & YES & $1.00$ & $(2,2)$ & -- & 591\\
$(27,11)$ & 8 & $(9,2)$ & 5 & 9 & YES & YES & YES & $1.00$ & $(2,2)$ & NO & 592\\
$(27,11)$ & 8 & $(11,2)$ & 6 & 1 & YES & YES & YES & $1.00$ & $(2,2)$ & -- & 593\\
$(27,8)$ & 7 & $(13,5)$ & 5 & 1 & YES & YES & YES & $1.12$ & $(2,2)$ & NO & 594\\
$(27,5)$ & 8 & $(14,5)$ & 6 & 1 & YES & YES & YES & $0.88$ & $(4,1)$ & -- & 595\\
$(27,5)$ & 8 & $(14,5)$ & 6 & 1 & YES & YES & YES & $1.00$ & $(4,1)$ & NO & 596\\
$(27,10)$ & 7 & $(14,3)$ & 6 & 1 & YES & YES & YES & $1.11$ & $(2,2)$ & -- & 597\\
$(27,5)$ & 8 & $(16,5)$ & 7 & 1 & YES & YES & YES & $1.00$ & $(2,2)$ & -- & 598\\
$(27,5)$ & 8 & $(16,5)$ & 7 & 1 & YES & YES & YES & $1.12$ & $(2,2)$ & NO & 599\\
$(27,5)$ & 8 & $(19,5)$ & 7 & 1 & YES & YES & YES & $0.75$ & $(4,1)$ & -- & 600\\
$(27,5)$ & 8 & $(19,5)$ & 7 & 1 & YES & YES & YES & $0.88$ & $(4,1)$ & NO & 601\\
$(27,5)$ & 8 & $(21,5)$ & 8 & 3 & YES & YES & YES & $1.00$ & $(2,2)$ & 1384 & 602\\
$(28,11)$ & 8 & $(5,2)$ & 3 & 1 & YES & YES & YES & $1.11$ & $(2,2)$ & NO & 603\\
$(28,11)$ & 8 & $(5,2)$ & 3 & 1 & YES & YES & YES & $1.11$ & $(2,2)$ & -- & 604\\
$(28,11)$ & 8 & $(9,2)$ & 5 & 1 & YES & YES & YES & $1.00$ & $(4,1)$ & -- & 605\\
$(28,11)$ & 8 & $(9,2)$ & 5 & 1 & YES & YES & YES & $1.00$ & $(4,1)$ & NO & 606\\
$(28,11)$ & 8 & $(11,2)$ & 6 & 1 & YES & YES & YES & $0.88$ & $(4,1)$ & -- & 607\\
$(28,5)$ & 8 & $(16,5)$ & 7 & 4 & YES & YES & YES & $0.88$ & $(4,1)$ & NO & 608\\
$(28,5)$ & 8 & $(19,5)$ & 7 & 1 & YES & YES & YES & $0.88$ & $(4,1)$ & NO & 609\\
$(28,11)$ & 8 & $(21,8)$ & 6 & 7 & YES & YES & YES & $1.00$ & $(4,1)$ & NO & 610\\
$(29,11)$ & 7 & $(4,1)$ & 3 & 1 & YES & YES & YES & $1.00$ & $(4,1)$ & -- & 611\\
$(29,11)$ & 7 & $(4,1)$ & 3 & 1 & YES & YES & YES & $1.11$ & $(2,2)$ & NO & 612\\
$(29,11)$ & 7 & $(5,1)$ & 4 & 1 & YES & YES & YES & $1.11$ & $(2,2)$ & NO & 613\\
$(29,11)$ & 7 & $(5,1)$ & 4 & 1 & YES & YES & YES & $1.11$ & $(2,2)$ & -- & 614\\
$(29,11)$ & 7 & $(5,1)$ & 4 & 1 & YES & YES & YES & $1.11$ & $(2,2)$ & NO & 615\\
$(29,11)$ & 7 & $(5,2)$ & 3 & 1 & YES & YES & YES & $1.00$ & $(4,1)$ & -- & 616\\
$(29,12)$ & 7 & $(5,2)$ & 3 & 1 & YES & YES & YES & $1.00$ & $(4,1)$ & -- & 617\\
$(29,8)$ & 7 & $(7,2)$ & 4 & 1 & YES & YES & YES & $1.00$ & $(2,2)$ & -- & 618\\
$(29,8)$ & 7 & $(7,2)$ & 4 & 1 & YES & YES & YES & $1.12$ & $(2,2)$ & 995 & 619\\
$(29,11)$ & 7 & $(7,2)$ & 4 & 1 & YES & YES & YES & $1.00$ & $(4,1)$ & NO & 620\\
$(29,11)$ & 7 & $(7,2)$ & 4 & 1 & YES & YES & YES & $1.00$ & $(4,1)$ & -- & 621\\
$(29,11)$ & 7 & $(7,3)$ & 4 & 1 & YES & YES & YES & $1.11$ & $(2,2)$ & 826 & 622\\
$(29,11)$ & 7 & $(7,3)$ & 4 & 1 & YES & YES & YES & $1.11$ & $(2,2)$ & -- & 623\\
$(29,12)$ & 7 & $(7,2)$ & 4 & 1 & YES & YES & YES & $1.00$ & $(2,2)$ & NO & 624\\
$(29,12)$ & 7 & $(7,2)$ & 4 & 1 & YES & YES & YES & $1.00$ & $(2,2)$ & -- & 625\\
$(29,13)$ & 8 & $(7,3)$ & 4 & 1 & YES & YES & YES & $1.00$ & $(2,2)$ & -- & 626\\
$(29,8)$ & 7 & $(8,3)$ & 4 & 1 & YES & YES & YES & $1.12$ & $(2,2)$ & NO & 627\\
$(29,8)$ & 7 & $(8,3)$ & 4 & 1 & YES & YES & YES & $1.12$ & $(2,2)$ & -- & 628\\
$(29,12)$ & 7 & $(8,3)$ & 4 & 1 & YES & YES & YES & $0.88$ & $(2,2)$ & -- & 629\\
$(29,12)$ & 7 & $(8,3)$ & 4 & 1 & YES & YES & YES & $1.00$ & $(2,2)$ & NO & 630\\
$(29,8)$ & 7 & $(9,4)$ & 5 & 1 & YES & YES & YES & $1.22$ & $(2,2)$ & -- & 631\\
$(29,8)$ & 7 & $(9,4)$ & 5 & 1 & YES & YES & YES & $1.33$ & $(2,2)$ & NO & 632\\
$(29,12)$ & 7 & $(9,4)$ & 5 & 1 & YES & YES & YES & $1.11$ & $(2,2)$ & NO & 633\\
$(29,12)$ & 7 & $(10,3)$ & 5 & 1 & YES & YES & YES & $0.75$ & $(4,1)$ & -- & 634\\
$(29,12)$ & 7 & $(10,3)$ & 5 & 1 & YES & YES & YES & $1.00$ & $(2,2)$ & NO & 635\\
$(29,8)$ & 7 & $(11,4)$ & 5 & 1 & YES & YES & YES & $1.12$ & $(2,2)$ & NO & 636\\
$(29,12)$ & 7 & $(13,3)$ & 6 & 1 & YES & YES & YES & $1.00$ & $(2,2)$ & -- & 637\\
$(29,8)$ & 7 & $(16,5)$ & 7 & 1 & YES & YES & YES & $1.12$ & $(2,2)$ & NO & 638\\
$(29,12)$ & 7 & $(16,7)$ & 6 & 1 & YES & YES & YES & $0.75$ & $(4,1)$ & NO & 639\\
$(29,9)$ & 8 & $(17,5)$ & 6 & 1 & YES & YES & YES & $1.00$ & $(2,2)$ & NO & 640\\
$(29,12)$ & 7 & $(18,7)$ & 6 & 1 & YES & YES & YES & $1.00$ & $(2,2)$ & NO & 641\\
$(29,9)$ & 8 & $(24,7)$ & 7 & 1 & YES & YES & YES & $1.12$ & $(2,2)$ & 873 & 642\\
$(29,13)$ & 8 & $(25,11)$ & 7 & 1 & YES & YES & YES & $1.00$ & $(2,2)$ & NO & 643\\
$(29,12)$ & 7 & $(26,11)$ & 7 & 1 & YES & YES & YES & $0.75$ & $(4,1)$ & NO & 644\\
$(29,9)$ & 8 & $(27,8)$ & 7 & 1 & YES & YES & YES & $1.12$ & $(2,2)$ & NO & 645\\
$(29,12)$ & 7 & $(27,11)$ & 8 & 1 & YES & YES & YES & $1.00$ & $(2,2)$ & 1052 & 646\\
$(30,11)$ & 7 & $(5,1)$ & 4 & 5 & YES & YES & YES & $0.88$ & $(4,1)$ & NO & 647\\
$(30,11)$ & 7 & $(5,1)$ & 4 & 5 & YES & YES & YES & $0.88$ & $(4,1)$ & -- & 648\\
$(30,11)$ & 7 & $(5,2)$ & 3 & 5 & YES & YES & YES & $1.11$ & $(2,2)$ & NO & 649\\
$(30,11)$ & 7 & $(5,2)$ & 3 & 5 & YES & YES & YES & $1.11$ & $(2,2)$ & -- & 650\\
$(30,11)$ & 7 & $(7,2)$ & 4 & 1 & YES & YES & YES & $1.12$ & $(2,2)$ & -- & 651\\
$(30,11)$ & 7 & $(7,3)$ & 4 & 1 & YES & YES & YES & $1.11$ & $(2,2)$ & -- & 652\\
$(30,13)$ & 8 & $(7,2)$ & 4 & 1 & YES & YES & YES & $1.00$ & $(4,1)$ & NO & 653\\
$(30,13)$ & 8 & $(7,2)$ & 4 & 1 & YES & YES & YES & $1.00$ & $(4,1)$ & -- & 654\\
$(30,11)$ & 7 & $(9,4)$ & 5 & 3 & YES & YES & YES & $1.00$ & $(2,2)$ & NO & 655\\
$(30,11)$ & 7 & $(11,3)$ & 5 & 1 & YES & YES & YES & $1.12$ & $(2,2)$ & NO & 656\\
$(30,13)$ & 8 & $(11,2)$ & 6 & 1 & YES & YES & YES & $1.00$ & $(2,2)$ & -- & 657\\
$(30,11)$ & 7 & $(13,3)$ & 6 & 1 & YES & YES & YES & $1.11$ & $(2,2)$ & -- & 658\\
$(31,13)$ & 7 & $(3,1)$ & 2 & 1 & YES & YES & YES & $1.00$ & $(2,2)$ & -- & 659\\
$(31,13)$ & 7 & $(3,1)$ & 2 & 1 & YES & YES & YES & $1.11$ & $(2,2)$ & NO & 660\\
$(31,13)$ & 7 & $(4,1)$ & 3 & 1 & YES & YES & YES & $1.00$ & $(2,2)$ & NO & 661\\
$(31,13)$ & 7 & $(4,1)$ & 3 & 1 & YES & YES & YES & $1.00$ & $(2,2)$ & -- & 662\\
$(31,9)$ & 8 & $(5,1)$ & 4 & 1 & YES & YES & YES & $1.11$ & $(2,2)$ & NO & 663\\
$(31,9)$ & 8 & $(5,1)$ & 4 & 1 & YES & YES & YES & $1.11$ & $(2,2)$ & -- & 664\\
$(31,12)$ & 7 & $(5,2)$ & 3 & 1 & YES & YES & YES & $1.00$ & $(2,2)$ & NO & 665\\
$(31,12)$ & 7 & $(5,2)$ & 3 & 1 & YES & YES & YES & $1.00$ & $(2,2)$ & -- & 666\\
$(31,13)$ & 7 & $(5,2)$ & 3 & 1 & YES & YES & YES & $1.12$ & $(2,2)$ & -- & 667\\
$(31,13)$ & 7 & $(5,2)$ & 3 & 1 & YES & YES & YES & $1.00$ & $(2,2)$ & NO & 668\\
$(31,7)$ & 8 & $(7,2)$ & 4 & 1 & YES & YES & YES & $1.00$ & $(2,2)$ & -- & 669\\
$(31,9)$ & 8 & $(7,2)$ & 4 & 1 & YES & YES & YES & $1.00$ & $(2,2)$ & -- & 670\\
$(31,9)$ & 8 & $(7,2)$ & 4 & 1 & YES & YES & YES & $1.12$ & $(2,2)$ & NO & 671\\
$(31,11)$ & 8 & $(7,3)$ & 4 & 1 & YES & YES & YES & $1.12$ & $(2,2)$ & -- & 672\\
$(31,12)$ & 7 & $(7,2)$ & 4 & 1 & YES & YES & YES & $1.11$ & $(2,2)$ & NO & 673\\
$(31,12)$ & 7 & $(7,2)$ & 4 & 1 & YES & YES & YES & $1.11$ & $(2,2)$ & -- & 674\\
$(31,12)$ & 7 & $(7,3)$ & 4 & 1 & YES & YES & YES & $0.88$ & $(4,1)$ & -- & 675\\
$(31,13)$ & 7 & $(7,2)$ & 4 & 1 & YES & YES & YES & $1.00$ & $(2,2)$ & NO & 676\\
$(31,13)$ & 7 & $(7,2)$ & 4 & 1 & YES & YES & YES & $1.00$ & $(2,2)$ & -- & 677\\
$(31,13)$ & 7 & $(7,3)$ & 4 & 1 & YES & YES & YES & $0.88$ & $(2,2)$ & -- & 678\\
$(31,7)$ & 8 & $(8,3)$ & 4 & 1 & YES & YES & NO(3) & $0.75$ & $(2,2)$ & -- & 679\\
$(31,12)$ & 7 & $(8,3)$ & 4 & 1 & YES & YES & YES & $1.11$ & $(2,2)$ & NO & 680\\
$(31,12)$ & 7 & $(8,3)$ & 4 & 1 & YES & YES & YES & $1.11$ & $(2,2)$ & -- & 681\\
$(31,13)$ & 7 & $(8,3)$ & 4 & 1 & YES & YES & YES & $1.00$ & $(2,2)$ & NO & 682\\
$(31,13)$ & 7 & $(8,3)$ & 4 & 1 & YES & YES & YES & $1.00$ & $(2,2)$ & -- & 683\\
$(31,7)$ & 8 & $(9,4)$ & 5 & 1 & YES & YES & YES & $1.12$ & $(2,2)$ & -- & 684\\
$(31,12)$ & 7 & $(10,3)$ & 5 & 1 & YES & YES & YES & $1.11$ & $(2,2)$ & -- & 685\\
$(31,12)$ & 7 & $(11,4)$ & 5 & 1 & YES & YES & YES & $1.11$ & $(2,2)$ & NO & 686\\
$(31,13)$ & 7 & $(11,3)$ & 5 & 1 & YES & YES & YES & $1.00$ & $(2,2)$ & NO & 687\\
$(31,7)$ & 8 & $(13,4)$ & 6 & 1 & YES & YES & YES & $1.00$ & $(2,2)$ & -- & 688\\
$(31,12)$ & 7 & $(13,3)$ & 6 & 1 & YES & YES & YES & $1.11$ & $(2,2)$ & NO & 689\\
$(31,12)$ & 7 & $(13,3)$ & 6 & 1 & YES & YES & YES & $1.11$ & $(2,2)$ & -- & 690\\
$(31,13)$ & 7 & $(13,3)$ & 6 & 1 & YES & YES & YES & $1.00$ & $(2,2)$ & -- & 691\\
$(31,9)$ & 8 & $(17,5)$ & 6 & 1 & YES & YES & YES & $1.11$ & $(2,2)$ & 854 & 692\\
$(31,7)$ & 8 & $(19,4)$ & 7 & 1 & YES & YES & YES & $1.00$ & $(2,2)$ & -- & 693\\
$(31,12)$ & 7 & $(21,8)$ & 6 & 1 & YES & YES & YES & $0.88$ & $(2,2)$ & NO & 694\\
$(31,9)$ & 8 & $(27,8)$ & 7 & 1 & YES & YES & YES & $0.89$ & $(2,2)$ & 1140 & 695\\
$(31,12)$ & 7 & $(28,11)$ & 8 & 1 & YES & YES & YES & $0.88$ & $(4,1)$ & 1097 & 696\\
$(31,13)$ & 7 & $(31,13)$ & 7 & 31 & YES & YES & YES & $1.00$ & $(2,2)$ & NO & 697\\
$(32,9)$ & 8 & $(4,1)$ & 3 & 4 & YES & YES & YES & $1.11$ & $(2,2)$ & -- & 698\\
$(32,9)$ & 8 & $(4,1)$ & 3 & 4 & YES & YES & YES & $1.22$ & $(2,2)$ & NO & 699\\
$(32,9)$ & 8 & $(5,2)$ & 3 & 1 & YES & YES & YES & $1.00$ & $(2,2)$ & -- & 700\\
$(32,9)$ & 8 & $(25,7)$ & 7 & 1 & YES & YES & YES & $1.11$ & $(2,2)$ & NO & 701\\
$(33,14)$ & 8 & $(4,1)$ & 3 & 1 & YES & YES & YES & $1.11$ & $(2,2)$ & NO & 702\\
$(33,10)$ & 8 & $(5,2)$ & 3 & 1 & YES & YES & YES & $1.00$ & $(2,2)$ & NO & 703\\
$(33,10)$ & 8 & $(5,2)$ & 3 & 1 & YES & YES & YES & $1.00$ & $(2,2)$ & -- & 704\\
$(33,14)$ & 8 & $(5,2)$ & 3 & 1 & YES & YES & YES & $1.25$ & $(2,2)$ & NO & 705\\
$(33,10)$ & 8 & $(7,2)$ & 4 & 1 & YES & YES & YES & $1.22$ & $(2,2)$ & NO & 706\\
$(33,10)$ & 8 & $(7,2)$ & 4 & 1 & YES & YES & YES & $1.22$ & $(2,2)$ & -- & 707\\
$(33,14)$ & 8 & $(8,3)$ & 4 & 1 & YES & YES & YES & $1.11$ & $(2,2)$ & NO & 708\\
$(33,7)$ & 8 & $(9,4)$ & 5 & 3 & YES & YES & YES & $1.00$ & $(2,2)$ & -- & 709\\
$(33,7)$ & 8 & $(9,4)$ & 5 & 3 & YES & YES & YES & $1.00$ & $(2,2)$ & NO & 710\\
$(33,10)$ & 8 & $(9,4)$ & 5 & 3 & YES & YES & YES & $1.00$ & $(2,2)$ & NO & 711\\
$(33,10)$ & 8 & $(14,3)$ & 6 & 1 & YES & YES & YES & $1.11$ & $(2,2)$ & -- & 712\\
$(33,7)$ & 8 & $(15,4)$ & 6 & 3 & YES & YES & YES & $0.88$ & $(2,2)$ & -- & 713\\
$(33,7)$ & 8 & $(15,4)$ & 6 & 3 & YES & YES & YES & $1.00$ & $(2,2)$ & NO & 714\\
$(33,7)$ & 8 & $(21,5)$ & 8 & 3 & YES & YES & YES & $1.00$ & $(2,2)$ & NO & 715\\
$(34,13)$ & 7 & $(2,1)$ & 1 & 2 & YES & YES & YES & $1.00$ & $(2,2)$ & -- & 716\\
$(34,13)$ & 7 & $(3,1)$ & 2 & 1 & YES & YES & YES & $1.00$ & $(2,2)$ & -- & 717\\
$(34,13)$ & 7 & $(4,1)$ & 3 & 2 & YES & YES & YES & $1.00$ & $(2,2)$ & NO & 718\\
$(34,13)$ & 7 & $(4,1)$ & 3 & 2 & YES & YES & YES & $1.00$ & $(2,2)$ & -- & 719\\
$(34,13)$ & 7 & $(5,2)$ & 3 & 1 & YES & YES & YES & $1.12$ & $(2,2)$ & -- & 720\\
$(34,13)$ & 7 & $(5,2)$ & 3 & 1 & YES & YES & YES & $1.11$ & $(2,2)$ & NO & 721\\
$(34,15)$ & 8 & $(5,2)$ & 3 & 1 & YES & YES & YES & $1.00$ & $(4,1)$ & -- & 722\\
$(34,9)$ & 8 & $(7,2)$ & 4 & 1 & YES & YES & YES & $0.89$ & $(2,2)$ & -- & 723\\
$(34,13)$ & 7 & $(7,2)$ & 4 & 1 & YES & YES & YES & $1.12$ & $(2,2)$ & NO & 724\\
$(34,13)$ & 7 & $(7,2)$ & 4 & 1 & YES & YES & YES & $1.12$ & $(2,2)$ & -- & 725\\
$(34,13)$ & 7 & $(7,3)$ & 4 & 1 & YES & YES & YES & $1.33$ & $(2,2)$ & NO & 726\\
$(34,13)$ & 7 & $(7,3)$ & 4 & 1 & YES & YES & YES & $1.33$ & $(2,2)$ & -- & 727\\
$(34,13)$ & 7 & $(7,3)$ & 4 & 1 & YES & YES & YES & $1.00$ & $(2,2)$ & NO & 728\\
$(34,15)$ & 8 & $(7,2)$ & 4 & 1 & YES & YES & YES & $1.22$ & $(2,2)$ & -- & 729\\
$(34,9)$ & 8 & $(8,3)$ & 4 & 2 & YES & YES & YES & $1.00$ & $(2,2)$ & -- & 730\\
$(34,9)$ & 8 & $(8,3)$ & 4 & 2 & YES & YES & YES & $1.12$ & $(2,2)$ & NO & 731\\
$(34,15)$ & 8 & $(8,3)$ & 4 & 2 & YES & YES & YES & $1.00$ & $(4,1)$ & NO & 732\\
$(34,13)$ & 7 & $(9,4)$ & 5 & 1 & YES & YES & YES & $1.00$ & $(2,2)$ & -- & 733\\
$(34,13)$ & 7 & $(9,4)$ & 5 & 1 & YES & YES & YES & $0.75$ & $(4,1)$ & NO & 734\\
$(34,15)$ & 8 & $(9,2)$ & 5 & 1 & YES & YES & YES & $1.11$ & $(2,2)$ & -- & 735\\
$(34,15)$ & 8 & $(9,2)$ & 5 & 1 & YES & YES & YES & $1.22$ & $(2,2)$ & NO & 736\\
$(34,15)$ & 8 & $(9,2)$ & 5 & 1 & YES & YES & YES & $1.22$ & $(2,2)$ & NO & 737\\
$(34,13)$ & 7 & $(10,3)$ & 5 & 2 & YES & YES & YES & $1.12$ & $(2,2)$ & NO & 738\\
$(34,13)$ & 7 & $(11,4)$ & 5 & 1 & YES & YES & YES & $1.00$ & $(4,1)$ & NO & 739\\
$(34,15)$ & 8 & $(12,5)$ & 5 & 2 & YES & YES & YES & $1.00$ & $(4,1)$ & 1257 & 740\\
$(34,13)$ & 7 & $(18,7)$ & 6 & 2 & YES & YES & YES & $1.00$ & $(2,2)$ & NO & 741\\
$(34,15)$ & 8 & $(20,9)$ & 7 & 2 & YES & YES & YES & $1.12$ & $(2,2)$ & NO & 742\\
$(34,13)$ & 7 & $(21,8)$ & 6 & 1 & YES & YES & YES & $1.11$ & $(2,2)$ & NO & 743\\
$(34,15)$ & 8 & $(23,10)$ & 7 & 1 & YES & YES & YES & $0.88$ & $(4,1)$ & 1193 & 744\\
$(34,13)$ & 7 & $(34,13)$ & 7 & 34 & YES & YES & YES & $1.00$ & $(2,2)$ & NO & 745\\
$(35,13)$ & 8 & $(5,2)$ & 3 & 5 & YES & YES & YES & $1.22$ & $(2,2)$ & -- & 746\\
$(35,8)$ & 8 & $(7,3)$ & 4 & 7 & YES & YES & YES & $1.00$ & $(2,2)$ & NO & 747\\
$(35,8)$ & 8 & $(7,3)$ & 4 & 7 & YES & YES & YES & $1.00$ & $(2,2)$ & -- & 748\\
$(35,13)$ & 8 & $(7,2)$ & 4 & 7 & YES & YES & YES & $1.00$ & $(4,1)$ & NO & 749\\
$(35,13)$ & 8 & $(7,2)$ & 4 & 7 & YES & YES & YES & $1.00$ & $(4,1)$ & -- & 750\\
$(35,8)$ & 8 & $(9,4)$ & 5 & 1 & YES & YES & YES & $1.22$ & $(2,2)$ & -- & 751\\
$(35,13)$ & 8 & $(11,2)$ & 6 & 1 & YES & YES & YES & $1.11$ & $(2,2)$ & NO & 752\\
$(36,13)$ & 8 & $(3,1)$ & 2 & 3 & YES & YES & YES & $0.88$ & $(4,1)$ & NO & 753\\
$(36,13)$ & 8 & $(3,1)$ & 2 & 3 & YES & YES & YES & $0.88$ & $(4,1)$ & -- & 754\\
$(36,11)$ & 8 & $(4,1)$ & 3 & 4 & YES & YES & YES & $1.11$ & $(2,2)$ & -- & 755\\
$(36,13)$ & 8 & $(5,1)$ & 4 & 1 & YES & YES & YES & $0.88$ & $(4,1)$ & NO & 756\\
$(36,13)$ & 8 & $(5,1)$ & 4 & 1 & YES & YES & YES & $0.88$ & $(4,1)$ & -- & 757\\
$(36,11)$ & 8 & $(7,2)$ & 4 & 1 & YES & YES & YES & $1.11$ & $(2,2)$ & 919 & 758\\
$(36,11)$ & 8 & $(7,2)$ & 4 & 1 & YES & YES & YES & $1.11$ & $(2,2)$ & -- & 759\\
$(36,11)$ & 8 & $(7,3)$ & 4 & 1 & YES & YES & YES & $0.88$ & $(2,2)$ & NO & 760\\
$(36,13)$ & 8 & $(7,2)$ & 4 & 1 & YES & YES & YES & $1.00$ & $(2,2)$ & -- & 761\\
$(36,13)$ & 8 & $(9,4)$ & 5 & 9 & YES & YES & YES & $1.00$ & $(2,2)$ & NO & 762\\
$(36,13)$ & 8 & $(14,5)$ & 6 & 2 & YES & YES & YES & $0.88$ & $(4,1)$ & 801 & 763\\
$(36,11)$ & 8 & $(23,7)$ & 7 & 1 & YES & YES & YES & $1.11$ & $(2,2)$ & NO & 764\\
$(36,13)$ & 8 & $(36,13)$ & 8 & 36 & YES & YES & YES & $0.88$ & $(4,1)$ & NO & 765\\
$(37,14)$ & 8 & $(4,1)$ & 3 & 1 & YES & YES & YES & $1.00$ & $(4,1)$ & NO & 766\\
$(37,14)$ & 8 & $(4,1)$ & 3 & 1 & YES & YES & YES & $1.12$ & $(2,2)$ & -- & 767\\
$(37,14)$ & 8 & $(5,1)$ & 4 & 1 & YES & YES & YES & $1.00$ & $(2,2)$ & NO & 768\\
$(37,14)$ & 8 & $(5,1)$ & 4 & 1 & YES & YES & YES & $1.00$ & $(2,2)$ & -- & 769\\
$(37,14)$ & 8 & $(5,2)$ & 3 & 1 & YES & YES & YES & $1.11$ & $(2,2)$ & -- & 770\\
$(37,16)$ & 9 & $(5,2)$ & 3 & 1 & YES & YES & YES & $1.25$ & $(2,2)$ & NO & 771\\
$(37,10)$ & 8 & $(7,3)$ & 4 & 1 & YES & YES & YES & $0.88$ & $(2,2)$ & -- & 772\\
$(37,14)$ & 8 & $(7,2)$ & 4 & 1 & YES & YES & YES & $1.22$ & $(2,2)$ & -- & 773\\
$(37,14)$ & 8 & $(7,2)$ & 4 & 1 & YES & YES & YES & $1.12$ & $(2,2)$ & NO & 774\\
$(37,14)$ & 8 & $(7,3)$ & 4 & 1 & YES & YES & YES & $1.11$ & $(2,2)$ & NO & 775\\
$(37,10)$ & 8 & $(8,3)$ & 4 & 1 & YES & YES & YES & $0.88$ & $(2,2)$ & -- & 776\\
$(37,10)$ & 8 & $(8,3)$ & 4 & 1 & YES & YES & YES & $1.11$ & $(2,2)$ & NO & 777\\
$(37,8)$ & 8 & $(9,4)$ & 5 & 1 & YES & YES & YES & $1.22$ & $(2,2)$ & -- & 778\\
$(37,14)$ & 8 & $(11,2)$ & 6 & 1 & YES & YES & YES & $1.11$ & $(2,2)$ & NO & 779\\
$(37,16)$ & 9 & $(11,5)$ & 6 & 1 & YES & YES & YES & $1.25$ & $(2,2)$ & NO & 780\\
$(37,14)$ & 8 & $(21,8)$ & 6 & 1 & YES & YES & YES & $1.00$ & $(2,2)$ & 1079 & 781\\
$(37,14)$ & 8 & $(29,11)$ & 7 & 1 & YES & YES & YES & $1.12$ & $(2,2)$ & NO & 782\\
$(37,14)$ & 8 & $(34,13)$ & 7 & 1 & YES & YES & YES & $1.11$ & $(2,2)$ & 1318 & 783\\
$(38,17)$ & 9 & $(9,2)$ & 5 & 1 & YES & YES & YES & $1.00$ & $(2,2)$ & -- & 784\\
$(38,11)$ & 9 & $(11,2)$ & 6 & 1 & YES & YES & YES & $1.22$ & $(2,2)$ & NO & 785\\
$(38,17)$ & 9 & $(16,7)$ & 6 & 2 & YES & YES & YES & $1.12$ & $(2,2)$ & NO & 786\\
$(39,14)$ & 8 & $(2,1)$ & 1 & 1 & YES & YES & YES & $0.88$ & $(4,1)$ & -- & 787\\
$(39,17)$ & 8 & $(3,1)$ & 2 & 3 & YES & YES & YES & $1.11$ & $(2,2)$ & -- & 788\\
$(39,11)$ & 9 & $(4,1)$ & 3 & 1 & YES & YES & YES & $1.11$ & $(2,2)$ & -- & 789\\
$(39,11)$ & 9 & $(4,1)$ & 3 & 1 & YES & YES & YES & $1.22$ & $(2,2)$ & NO & 790\\
$(39,11)$ & 9 & $(5,2)$ & 3 & 1 & YES & YES & YES & $0.88$ & $(4,1)$ & NO & 791\\
$(39,11)$ & 9 & $(5,2)$ & 3 & 1 & YES & YES & YES & $0.88$ & $(4,1)$ & -- & 792\\
$(39,16)$ & 8 & $(5,2)$ & 3 & 1 & YES & YES & YES & $1.22$ & $(2,2)$ & -- & 793\\
$(39,17)$ & 8 & $(5,2)$ & 3 & 1 & YES & YES & YES & $1.00$ & $(2,2)$ & -- & 794\\
$(39,14)$ & 8 & $(7,2)$ & 4 & 1 & YES & YES & YES & $1.00$ & $(2,2)$ & -- & 795\\
$(39,17)$ & 8 & $(7,2)$ & 4 & 1 & YES & YES & YES & $1.00$ & $(2,2)$ & -- & 796\\
$(39,17)$ & 8 & $(7,2)$ & 4 & 1 & YES & YES & YES & $1.11$ & $(2,2)$ & NO & 797\\
$(39,16)$ & 8 & $(9,2)$ & 5 & 3 & YES & YES & YES & $0.88$ & $(2,2)$ & NO & 798\\
$(39,16)$ & 8 & $(9,4)$ & 5 & 3 & YES & YES & YES & $1.00$ & $(2,2)$ & NO & 799\\
$(39,11)$ & 9 & $(11,2)$ & 6 & 1 & YES & YES & YES & $1.22$ & $(2,2)$ & NO & 800\\
$(39,14)$ & 8 & $(11,4)$ & 5 & 1 & YES & YES & YES & $0.88$ & $(4,1)$ & 763 & 801\\
$(39,14)$ & 8 & $(13,5)$ & 5 & 13 & YES & YES & YES & $1.12$ & $(2,2)$ & 1277 & 802\\
$(39,14)$ & 8 & $(14,5)$ & 6 & 1 & YES & YES & YES & $0.88$ & $(4,1)$ & NO & 803\\
$(39,17)$ & 8 & $(25,11)$ & 7 & 1 & YES & YES & YES & $1.12$ & $(2,2)$ & NO & 804\\
$(39,16)$ & 8 & $(27,11)$ & 8 & 3 & YES & YES & YES & $1.00$ & $(2,2)$ & NO & 805\\
$(39,11)$ & 9 & $(29,8)$ & 7 & 1 & YES & YES & YES & $1.22$ & $(2,2)$ & NO & 806\\
$(39,16)$ & 8 & $(29,12)$ & 7 & 1 & YES & YES & YES & $1.11$ & $(2,2)$ & NO & 807\\
$(40,11)$ & 8 & $(3,1)$ & 2 & 1 & YES & YES & YES & $1.00$ & $(4,1)$ & NO & 808\\
$(40,17)$ & 9 & $(4,1)$ & 3 & 4 & YES & YES & YES & $1.25$ & $(2,2)$ & NO & 809\\
$(40,17)$ & 9 & $(4,1)$ & 3 & 4 & YES & YES & YES & $1.25$ & $(2,2)$ & -- & 810\\
$(40,17)$ & 9 & $(5,1)$ & 4 & 5 & YES & YES & YES & $1.00$ & $(2,2)$ & -- & 811\\
$(40,17)$ & 9 & $(5,1)$ & 4 & 5 & YES & YES & YES & $1.12$ & $(2,2)$ & NO & 812\\
$(40,17)$ & 9 & $(6,1)$ & 5 & 2 & YES & YES & YES & $1.12$ & $(2,2)$ & NO & 813\\
$(40,17)$ & 9 & $(6,1)$ & 5 & 2 & YES & YES & YES & $1.12$ & $(2,2)$ & NO & 814\\
$(40,11)$ & 8 & $(18,5)$ & 6 & 2 & YES & YES & YES & $0.88$ & $(4,1)$ & 1005 & 815\\
$(40,17)$ & 9 & $(19,8)$ & 6 & 1 & YES & YES & YES & $1.12$ & $(2,2)$ & NO & 816\\
$(40,17)$ & 9 & $(26,11)$ & 7 & 2 & YES & YES & YES & $1.00$ & $(2,2)$ & 1192 & 817\\
$(40,17)$ & 9 & $(33,14)$ & 8 & 1 & YES & YES & YES & $1.25$ & $(2,2)$ & NO & 818\\
$(41,11)$ & 8 & $(2,1)$ & 1 & 1 & YES & YES & YES & $1.00$ & $(4,1)$ & NO & 819\\
$(41,15)$ & 8 & $(2,1)$ & 1 & 1 & YES & YES & YES & $1.11$ & $(2,2)$ & -- & 820\\
$(41,15)$ & 8 & $(3,1)$ & 2 & 1 & YES & YES & YES & $1.00$ & $(4,1)$ & NO & 821\\
$(41,15)$ & 8 & $(3,1)$ & 2 & 1 & YES & YES & YES & $1.12$ & $(2,2)$ & -- & 822\\
$(41,16)$ & 8 & $(3,1)$ & 2 & 1 & YES & YES & YES & $1.00$ & $(2,2)$ & -- & 823\\
$(41,16)$ & 8 & $(3,1)$ & 2 & 1 & YES & YES & YES & $1.00$ & $(2,2)$ & NO & 824\\
$(41,17)$ & 8 & $(3,1)$ & 2 & 1 & YES & YES & YES & $1.00$ & $(4,1)$ & -- & 825\\
$(41,17)$ & 8 & $(3,1)$ & 2 & 1 & YES & YES & YES & $1.11$ & $(2,2)$ & 622 & 826\\
$(41,18)$ & 8 & $(3,1)$ & 2 & 1 & YES & YES & YES & $1.11$ & $(2,2)$ & 545 & 827\\
$(41,18)$ & 8 & $(3,1)$ & 2 & 1 & YES & YES & YES & $1.11$ & $(2,2)$ & -- & 828\\
$(41,18)$ & 8 & $(3,1)$ & 2 & 1 & YES & YES & YES & $1.11$ & $(2,2)$ & NO & 829\\
$(41,11)$ & 8 & $(4,1)$ & 3 & 1 & YES & YES & YES & $1.11$ & $(2,2)$ & NO & 830\\
$(41,11)$ & 8 & $(4,1)$ & 3 & 1 & YES & YES & YES & $1.11$ & $(2,2)$ & -- & 831\\
$(41,12)$ & 8 & $(4,1)$ & 3 & 1 & YES & YES & YES & $1.11$ & $(2,2)$ & NO & 832\\
$(41,12)$ & 8 & $(4,1)$ & 3 & 1 & YES & YES & YES & $1.11$ & $(2,2)$ & -- & 833\\
$(41,16)$ & 8 & $(4,1)$ & 3 & 1 & YES & YES & YES & $1.00$ & $(2,2)$ & NO & 834\\
$(41,16)$ & 8 & $(4,1)$ & 3 & 1 & YES & YES & YES & $1.00$ & $(2,2)$ & -- & 835\\
$(41,17)$ & 8 & $(4,1)$ & 3 & 1 & YES & YES & YES & $1.00$ & $(4,1)$ & NO & 836\\
$(41,17)$ & 8 & $(4,1)$ & 3 & 1 & YES & YES & YES & $1.00$ & $(4,1)$ & -- & 837\\
$(41,18)$ & 8 & $(4,1)$ & 3 & 1 & YES & YES & YES & $1.00$ & $(2,2)$ & NO & 838\\
$(41,18)$ & 8 & $(4,1)$ & 3 & 1 & YES & YES & YES & $1.00$ & $(2,2)$ & -- & 839\\
$(41,18)$ & 8 & $(4,1)$ & 3 & 1 & YES & YES & YES & $1.00$ & $(2,2)$ & NO & 840\\
$(41,9)$ & 9 & $(5,2)$ & 3 & 1 & YES & YES & YES & $1.00$ & $(2,2)$ & NO & 841\\
$(41,12)$ & 8 & $(5,1)$ & 4 & 1 & YES & YES & YES & $1.00$ & $(2,2)$ & NO & 842\\
$(41,12)$ & 8 & $(5,1)$ & 4 & 1 & YES & YES & YES & $1.00$ & $(2,2)$ & -- & 843\\
$(41,12)$ & 8 & $(5,2)$ & 3 & 1 & YES & YES & YES & $0.88$ & $(4,1)$ & -- & 844\\
$(41,15)$ & 8 & $(5,1)$ & 4 & 1 & YES & YES & YES & $1.00$ & $(2,2)$ & NO & 845\\
$(41,15)$ & 8 & $(5,1)$ & 4 & 1 & YES & YES & YES & $1.00$ & $(2,2)$ & -- & 846\\
$(41,15)$ & 8 & $(5,2)$ & 3 & 1 & YES & YES & YES & $1.11$ & $(2,2)$ & -- & 847\\
$(41,16)$ & 8 & $(5,2)$ & 3 & 1 & YES & YES & YES & $1.11$ & $(2,2)$ & -- & 848\\
$(41,16)$ & 8 & $(5,2)$ & 3 & 1 & YES & YES & YES & $1.11$ & $(2,2)$ & NO & 849\\
$(41,17)$ & 8 & $(5,2)$ & 3 & 1 & YES & YES & YES & $1.12$ & $(2,2)$ & -- & 850\\
$(41,17)$ & 8 & $(5,2)$ & 3 & 1 & YES & YES & YES & $1.00$ & $(4,1)$ & NO & 851\\
$(41,18)$ & 8 & $(5,2)$ & 3 & 1 & YES & YES & YES & $1.22$ & $(2,2)$ & -- & 852\\
$(41,18)$ & 8 & $(5,2)$ & 3 & 1 & YES & YES & YES & $1.00$ & $(2,2)$ & NO & 853\\
$(41,12)$ & 8 & $(7,2)$ & 4 & 1 & YES & YES & YES & $1.11$ & $(2,2)$ & 692 & 854\\
$(41,12)$ & 8 & $(7,3)$ & 4 & 1 & YES & YES & YES & $1.11$ & $(2,2)$ & -- & 855\\
$(41,15)$ & 8 & $(7,2)$ & 4 & 1 & YES & YES & YES & $1.11$ & $(2,2)$ & -- & 856\\
$(41,15)$ & 8 & $(7,3)$ & 4 & 1 & YES & YES & YES & $1.00$ & $(2,2)$ & NO & 857\\
$(41,17)$ & 8 & $(7,2)$ & 4 & 1 & YES & YES & YES & $1.11$ & $(2,2)$ & NO & 858\\
$(41,18)$ & 8 & $(7,2)$ & 4 & 1 & YES & YES & YES & $1.11$ & $(2,2)$ & NO & 859\\
$(41,11)$ & 8 & $(8,3)$ & 4 & 1 & YES & YES & YES & $1.22$ & $(2,2)$ & -- & 860\\
$(41,12)$ & 8 & $(8,3)$ & 4 & 1 & YES & YES & YES & $1.00$ & $(4,1)$ & NO & 861\\
$(41,15)$ & 8 & $(8,3)$ & 4 & 1 & YES & YES & YES & $1.11$ & $(2,2)$ & NO & 862\\
$(41,16)$ & 8 & $(8,3)$ & 4 & 1 & YES & YES & YES & $1.00$ & $(2,2)$ & 531 & 863\\
$(41,15)$ & 8 & $(9,2)$ & 5 & 1 & YES & YES & YES & $1.11$ & $(2,2)$ & NO & 864\\
$(41,15)$ & 8 & $(9,2)$ & 5 & 1 & YES & YES & YES & $1.11$ & $(2,2)$ & -- & 865\\
$(41,16)$ & 8 & $(9,2)$ & 5 & 1 & YES & YES & YES & $0.88$ & $(4,1)$ & NO & 866\\
$(41,17)$ & 8 & $(9,4)$ & 5 & 1 & YES & YES & YES & $1.00$ & $(4,1)$ & NO & 867\\
$(41,18)$ & 8 & $(9,2)$ & 5 & 1 & YES & YES & YES & $1.22$ & $(2,2)$ & NO & 868\\
$(41,18)$ & 8 & $(9,2)$ & 5 & 1 & YES & YES & YES & $1.22$ & $(2,2)$ & -- & 869\\
$(41,15)$ & 8 & $(11,2)$ & 6 & 1 & YES & YES & YES & $1.11$ & $(2,2)$ & NO & 870\\
$(41,18)$ & 8 & $(12,5)$ & 5 & 1 & YES & YES & YES & $1.22$ & $(2,2)$ & NO & 871\\
$(41,16)$ & 8 & $(13,5)$ & 5 & 1 & YES & YES & YES & $1.00$ & $(2,2)$ & NO & 872\\
$(41,12)$ & 8 & $(16,5)$ & 7 & 1 & YES & YES & YES & $1.12$ & $(2,2)$ & 642 & 873\\
$(41,12)$ & 8 & $(18,5)$ & 6 & 1 & YES & YES & YES & $0.75$ & $(4,1)$ & NO & 874\\
$(41,15)$ & 8 & $(19,7)$ & 6 & 1 & YES & YES & YES & $0.88$ & $(4,1)$ & 1051 & 875\\
$(41,17)$ & 8 & $(22,9)$ & 7 & 1 & YES & YES & YES & $1.11$ & $(2,2)$ & 1348 & 876\\
$(41,18)$ & 8 & $(25,11)$ & 7 & 1 & YES & YES & YES & $1.11$ & $(2,2)$ & NO & 877\\
$(41,15)$ & 8 & $(27,10)$ & 7 & 1 & YES & YES & YES & $1.11$ & $(2,2)$ & 1367 & 878\\
$(41,16)$ & 8 & $(28,11)$ & 8 & 1 & YES & YES & YES & $1.00$ & $(4,1)$ & NO & 879\\
$(41,16)$ & 8 & $(31,12)$ & 7 & 1 & YES & YES & YES & $1.11$ & $(2,2)$ & NO & 880\\
$(41,18)$ & 8 & $(34,15)$ & 8 & 1 & YES & YES & YES & $1.22$ & $(2,2)$ & NO & 881\\
$(41,15)$ & 8 & $(41,15)$ & 8 & 41 & YES & YES & YES & $1.12$ & $(2,2)$ & NO & 882\\
$(41,16)$ & 8 & $(41,16)$ & 8 & 41 & YES & YES & YES & $1.00$ & $(2,2)$ & NO & 883\\
$(41,17)$ & 8 & $(41,17)$ & 8 & 41 & YES & YES & YES & $0.88$ & $(4,1)$ & NO & 884\\
$(42,13)$ & 9 & $(2,1)$ & 1 & 2 & YES & YES & YES & $1.00$ & $(4,1)$ & NO & 885\\
$(42,13)$ & 9 & $(7,2)$ & 4 & 7 & YES & YES & YES & $1.12$ & $(2,2)$ & -- & 886\\
$(42,13)$ & 9 & $(17,5)$ & 6 & 1 & YES & YES & YES & $1.22$ & $(2,2)$ & NO & 887\\
$(43,12)$ & 8 & $(3,1)$ & 2 & 1 & YES & YES & YES & $0.88$ & $(4,1)$ & NO & 888\\
$(43,12)$ & 8 & $(3,1)$ & 2 & 1 & YES & YES & YES & $0.88$ & $(4,1)$ & -- & 889\\
$(43,12)$ & 8 & $(3,1)$ & 2 & 1 & YES & YES & YES & $1.11$ & $(2,2)$ & NO & 890\\
$(43,19)$ & 9 & $(3,1)$ & 2 & 1 & YES & YES & YES & $1.22$ & $(2,2)$ & NO & 891\\
$(43,19)$ & 9 & $(3,1)$ & 2 & 1 & YES & YES & YES & $1.22$ & $(2,2)$ & -- & 892\\
$(43,18)$ & 8 & $(4,1)$ & 3 & 1 & YES & YES & YES & $1.12$ & $(2,2)$ & -- & 893\\
$(43,18)$ & 8 & $(4,1)$ & 3 & 1 & YES & YES & YES & $1.25$ & $(2,2)$ & NO & 894\\
$(43,19)$ & 9 & $(4,1)$ & 3 & 1 & YES & YES & YES & $1.12$ & $(2,2)$ & NO & 895\\
$(43,19)$ & 9 & $(4,1)$ & 3 & 1 & YES & YES & YES & $1.11$ & $(2,2)$ & -- & 896\\
$(43,13)$ & 9 & $(5,2)$ & 3 & 1 & YES & YES & YES & $1.00$ & $(4,1)$ & NO & 897\\
$(43,13)$ & 9 & $(5,2)$ & 3 & 1 & YES & YES & YES & $1.00$ & $(4,1)$ & -- & 898\\
$(43,18)$ & 8 & $(5,2)$ & 3 & 1 & YES & YES & YES & $1.12$ & $(2,2)$ & -- & 899\\
$(43,13)$ & 9 & $(6,1)$ & 5 & 1 & YES & YES & YES & $1.11$ & $(2,2)$ & NO & 900\\
$(43,13)$ & 9 & $(6,1)$ & 5 & 1 & YES & YES & YES & $1.11$ & $(2,2)$ & -- & 901\\
$(43,16)$ & 9 & $(7,1)$ & 6 & 1 & YES & YES & YES & $1.11$ & $(2,2)$ & NO & 902\\
$(43,19)$ & 9 & $(7,3)$ & 4 & 1 & YES & YES & YES & $1.22$ & $(2,2)$ & NO & 903\\
$(43,12)$ & 8 & $(8,3)$ & 4 & 1 & YES & YES & YES & $1.11$ & $(2,2)$ & -- & 904\\
$(43,18)$ & 8 & $(9,4)$ & 5 & 1 & YES & YES & YES & $1.12$ & $(2,2)$ & 1138 & 905\\
$(43,12)$ & 8 & $(11,3)$ & 5 & 1 & YES & YES & YES & $0.88$ & $(4,1)$ & NO & 906\\
$(43,19)$ & 9 & $(16,7)$ & 6 & 1 & YES & YES & YES & $1.12$ & $(2,2)$ & NO & 907\\
$(43,10)$ & 9 & $(19,4)$ & 7 & 1 & YES & YES & YES & $1.00$ & $(2,2)$ & NO & 908\\
$(43,19)$ & 9 & $(25,11)$ & 7 & 1 & YES & YES & YES & $1.00$ & $(2,2)$ & 1196 & 909\\
$(43,18)$ & 8 & $(26,11)$ & 7 & 1 & YES & YES & YES & $1.11$ & $(2,2)$ & 1407 & 910\\
$(43,18)$ & 8 & $(31,13)$ & 7 & 1 & YES & YES & YES & $1.00$ & $(2,2)$ & NO & 911\\
$(43,10)$ & 9 & $(35,8)$ & 8 & 1 & YES & YES & YES & $1.00$ & $(2,2)$ & NO & 912\\
$(43,16)$ & 9 & $(35,13)$ & 8 & 1 & YES & YES & YES & $1.11$ & $(2,2)$ & NO & 913\\
$(43,19)$ & 9 & $(43,19)$ & 9 & 43 & YES & YES & YES & $1.22$ & $(2,2)$ & NO & 914\\
$(44,17)$ & 8 & $(2,1)$ & 1 & 2 & YES & YES & YES & $1.00$ & $(2,2)$ & -- & 915\\
$(44,17)$ & 8 & $(2,1)$ & 1 & 2 & YES & YES & YES & $1.11$ & $(2,2)$ & 576 & 916\\
$(44,13)$ & 8 & $(3,1)$ & 2 & 1 & YES & YES & YES & $1.12$ & $(2,2)$ & NO & 917\\
$(44,13)$ & 8 & $(3,1)$ & 2 & 1 & YES & YES & YES & $1.12$ & $(2,2)$ & -- & 918\\
$(44,13)$ & 8 & $(3,1)$ & 2 & 1 & YES & YES & YES & $1.11$ & $(2,2)$ & 758 & 919\\
$(44,17)$ & 8 & $(3,1)$ & 2 & 1 & YES & YES & YES & $1.00$ & $(4,1)$ & -- & 920\\
$(44,13)$ & 8 & $(4,1)$ & 3 & 4 & YES & YES & YES & $1.00$ & $(2,2)$ & NO & 921\\
$(44,13)$ & 8 & $(4,1)$ & 3 & 4 & YES & YES & YES & $1.00$ & $(2,2)$ & -- & 922\\
$(44,19)$ & 10 & $(4,1)$ & 3 & 4 & YES & YES & YES & $1.25$ & $(2,2)$ & -- & 923\\
$(44,19)$ & 10 & $(4,1)$ & 3 & 4 & YES & YES & YES & $1.25$ & $(2,2)$ & NO & 924\\
$(44,13)$ & 8 & $(5,2)$ & 3 & 1 & YES & YES & YES & $1.22$ & $(2,2)$ & NO & 925\\
$(44,13)$ & 8 & $(5,2)$ & 3 & 1 & YES & YES & YES & $1.22$ & $(2,2)$ & -- & 926\\
$(44,13)$ & 8 & $(5,2)$ & 3 & 1 & YES & YES & YES & $1.22$ & $(2,2)$ & NO & 927\\
$(44,17)$ & 8 & $(5,2)$ & 3 & 1 & YES & YES & YES & $1.11$ & $(2,2)$ & -- & 928\\
$(44,17)$ & 8 & $(5,2)$ & 3 & 1 & YES & YES & YES & $1.11$ & $(2,2)$ & NO & 929\\
$(44,17)$ & 8 & $(5,2)$ & 3 & 1 & YES & YES & YES & $1.11$ & $(2,2)$ & NO & 930\\
$(44,19)$ & 10 & $(5,1)$ & 4 & 1 & YES & YES & YES & $1.12$ & $(2,2)$ & NO & 931\\
$(44,19)$ & 10 & $(6,1)$ & 5 & 2 & YES & YES & YES & $1.12$ & $(2,2)$ & -- & 932\\
$(44,13)$ & 8 & $(7,2)$ & 4 & 1 & YES & YES & YES & $1.00$ & $(2,2)$ & NO & 933\\
$(44,17)$ & 8 & $(7,2)$ & 4 & 1 & YES & YES & YES & $1.00$ & $(2,2)$ & -- & 934\\
$(44,19)$ & 10 & $(7,1)$ & 6 & 1 & YES & YES & YES & $1.25$ & $(2,2)$ & NO & 935\\
$(44,19)$ & 10 & $(7,1)$ & 6 & 1 & YES & YES & YES & $1.25$ & $(2,2)$ & NO & 936\\
$(44,13)$ & 8 & $(8,3)$ & 4 & 4 & YES & YES & YES & $0.88$ & $(2,2)$ & NO & 937\\
$(44,13)$ & 8 & $(9,2)$ & 5 & 1 & YES & YES & NO(3) & $0.75$ & $(2,2)$ & NO & 938\\
$(44,13)$ & 8 & $(16,5)$ & 7 & 4 & YES & YES & YES & $1.00$ & $(2,2)$ & NO & 939\\
$(44,19)$ & 10 & $(16,7)$ & 6 & 4 & YES & YES & YES & $1.25$ & $(2,2)$ & NO & 940\\
$(44,17)$ & 8 & $(18,7)$ & 6 & 2 & YES & YES & YES & $1.00$ & $(2,2)$ & 1054 & 941\\
$(44,17)$ & 8 & $(23,9)$ & 7 & 1 & YES & YES & YES & $1.11$ & $(2,2)$ & 1421 & 942\\
$(44,19)$ & 10 & $(23,10)$ & 7 & 1 & YES & YES & YES & $1.12$ & $(2,2)$ & NO & 943\\
$(44,13)$ & 8 & $(27,8)$ & 7 & 1 & YES & YES & YES & $1.12$ & $(2,2)$ & NO & 944\\
$(44,19)$ & 10 & $(30,13)$ & 8 & 2 & YES & YES & YES & $1.12$ & $(2,2)$ & 1329 & 945\\
$(44,19)$ & 10 & $(37,16)$ & 9 & 1 & YES & YES & YES & $1.25$ & $(2,2)$ & NO & 946\\
$(44,13)$ & 8 & $(44,13)$ & 8 & 44 & YES & YES & YES & $1.00$ & $(2,2)$ & NO & 947\\
$(44,17)$ & 8 & $(44,17)$ & 8 & 44 & YES & YES & YES & $0.88$ & $(4,1)$ & NO & 948\\
$(45,14)$ & 9 & $(3,1)$ & 2 & 3 & YES & YES & YES & $1.00$ & $(2,2)$ & -- & 949\\
$(45,19)$ & 8 & $(3,1)$ & 2 & 3 & YES & YES & YES & $1.00$ & $(2,2)$ & NO & 950\\
$(45,19)$ & 8 & $(3,1)$ & 2 & 3 & YES & YES & YES & $1.00$ & $(2,2)$ & -- & 951\\
$(45,17)$ & 9 & $(4,1)$ & 3 & 1 & YES & YES & YES & $1.00$ & $(4,1)$ & -- & 952\\
$(45,19)$ & 8 & $(4,1)$ & 3 & 1 & YES & YES & YES & $1.11$ & $(2,2)$ & NO & 953\\
$(45,19)$ & 8 & $(4,1)$ & 3 & 1 & YES & YES & YES & $1.11$ & $(2,2)$ & -- & 954\\
$(45,19)$ & 8 & $(5,2)$ & 3 & 5 & YES & YES & YES & $1.11$ & $(2,2)$ & NO & 955\\
$(45,19)$ & 8 & $(5,2)$ & 3 & 5 & YES & YES & YES & $1.11$ & $(2,2)$ & -- & 956\\
$(45,17)$ & 9 & $(6,1)$ & 5 & 3 & YES & YES & YES & $0.88$ & $(4,1)$ & NO & 957\\
$(45,14)$ & 9 & $(7,2)$ & 4 & 1 & YES & YES & YES & $1.11$ & $(2,2)$ & NO & 958\\
$(45,16)$ & 9 & $(7,3)$ & 4 & 1 & YES & YES & YES & $1.00$ & $(2,2)$ & NO & 959\\
$(45,19)$ & 8 & $(8,3)$ & 4 & 1 & YES & YES & YES & $1.11$ & $(2,2)$ & NO & 960\\
$(45,19)$ & 8 & $(9,4)$ & 5 & 9 & YES & YES & YES & $1.00$ & $(2,2)$ & NO & 961\\
$(45,14)$ & 9 & $(10,3)$ & 5 & 5 & YES & YES & YES & $1.11$ & $(2,2)$ & 1178 & 962\\
$(45,17)$ & 9 & $(21,8)$ & 6 & 3 & YES & YES & YES & $0.88$ & $(4,1)$ & NO & 963\\
$(45,19)$ & 8 & $(31,13)$ & 7 & 1 & YES & YES & YES & $1.11$ & $(2,2)$ & NO & 964\\
$(45,17)$ & 9 & $(37,14)$ & 8 & 1 & YES & YES & YES & $1.00$ & $(4,1)$ & NO & 965\\
$(46,19)$ & 8 & $(2,1)$ & 1 & 2 & YES & YES & YES & $1.11$ & $(2,2)$ & NO & 966\\
$(46,19)$ & 8 & $(2,1)$ & 1 & 2 & YES & YES & YES & $1.11$ & $(2,2)$ & -- & 967\\
$(46,21)$ & 10 & $(2,1)$ & 1 & 2 & YES & YES & YES & $1.00$ & $(4,1)$ & -- & 968\\
$(46,19)$ & 8 & $(3,1)$ & 2 & 1 & YES & YES & YES & $1.11$ & $(2,2)$ & -- & 969\\
$(46,19)$ & 8 & $(3,1)$ & 2 & 1 & YES & YES & YES & $1.12$ & $(2,2)$ & NO & 970\\
$(46,19)$ & 8 & $(3,1)$ & 2 & 1 & YES & YES & YES & $1.11$ & $(2,2)$ & NO & 971\\
$(46,21)$ & 10 & $(3,1)$ & 2 & 1 & YES & YES & YES & $1.00$ & $(4,1)$ & 466 & 972\\
$(46,21)$ & 10 & $(3,1)$ & 2 & 1 & YES & YES & YES & $1.00$ & $(4,1)$ & -- & 973\\
$(46,19)$ & 8 & $(4,1)$ & 3 & 2 & YES & YES & YES & $1.11$ & $(2,2)$ & -- & 974\\
$(46,19)$ & 8 & $(4,1)$ & 3 & 2 & YES & YES & YES & $1.22$ & $(2,2)$ & NO & 975\\
$(46,17)$ & 8 & $(5,2)$ & 3 & 1 & YES & YES & YES & $0.88$ & $(4,1)$ & NO & 976\\
$(46,17)$ & 8 & $(5,2)$ & 3 & 1 & YES & YES & YES & $1.11$ & $(2,2)$ & -- & 977\\
$(46,19)$ & 8 & $(5,2)$ & 3 & 1 & YES & YES & YES & $1.11$ & $(2,2)$ & -- & 978\\
$(46,19)$ & 8 & $(5,2)$ & 3 & 1 & YES & YES & YES & $1.11$ & $(2,2)$ & NO & 979\\
$(46,11)$ & 10 & $(7,3)$ & 4 & 1 & YES & YES & YES & $1.12$ & $(2,2)$ & -- & 980\\
$(46,17)$ & 8 & $(7,2)$ & 4 & 1 & YES & YES & YES & $1.11$ & $(2,2)$ & -- & 981\\
$(46,19)$ & 8 & $(7,2)$ & 4 & 1 & YES & YES & YES & $1.11$ & $(2,2)$ & -- & 982\\
$(46,19)$ & 8 & $(7,3)$ & 4 & 1 & YES & YES & YES & $1.00$ & $(2,2)$ & 1077 & 983\\
$(46,19)$ & 8 & $(8,3)$ & 4 & 2 & YES & YES & YES & $1.11$ & $(2,2)$ & NO & 984\\
$(46,19)$ & 8 & $(9,2)$ & 5 & 1 & YES & YES & YES & $1.00$ & $(2,2)$ & NO & 985\\
$(46,19)$ & 8 & $(9,4)$ & 5 & 1 & YES & YES & YES & $1.22$ & $(2,2)$ & 1415 & 986\\
$(46,19)$ & 8 & $(22,9)$ & 7 & 2 & YES & YES & YES & $1.11$ & $(2,2)$ & NO & 987\\
$(46,19)$ & 8 & $(29,12)$ & 7 & 1 & YES & YES & YES & $1.11$ & $(2,2)$ & NO & 988\\
$(46,17)$ & 8 & $(35,13)$ & 8 & 1 & YES & YES & YES & $1.11$ & $(2,2)$ & NO & 989\\
$(46,19)$ & 8 & $(46,19)$ & 8 & 46 & YES & YES & YES & $1.11$ & $(2,2)$ & NO & 990\\
$(47,13)$ & 8 & $(2,1)$ & 1 & 1 & YES & YES & YES & $1.00$ & $(4,1)$ & -- & 991\\
$(47,13)$ & 8 & $(2,1)$ & 1 & 1 & YES & YES & YES & $1.00$ & $(4,1)$ & NO & 992\\
$(47,13)$ & 8 & $(3,1)$ & 2 & 1 & YES & YES & YES & $1.00$ & $(2,2)$ & NO & 993\\
$(47,13)$ & 8 & $(3,1)$ & 2 & 1 & YES & YES & YES & $1.00$ & $(2,2)$ & -- & 994\\
$(47,13)$ & 8 & $(4,1)$ & 3 & 1 & YES & YES & YES & $1.12$ & $(2,2)$ & 619 & 995\\
$(47,13)$ & 8 & $(4,1)$ & 3 & 1 & YES & YES & YES & $1.12$ & $(2,2)$ & -- & 996\\
$(47,14)$ & 9 & $(4,1)$ & 3 & 1 & YES & YES & YES & $1.33$ & $(2,2)$ & NO & 997\\
$(47,17)$ & 9 & $(4,1)$ & 3 & 1 & YES & YES & YES & $1.00$ & $(4,1)$ & NO & 998\\
$(47,18)$ & 8 & $(4,1)$ & 3 & 1 & YES & YES & YES & $1.12$ & $(2,2)$ & -- & 999\\
$(47,18)$ & 8 & $(4,1)$ & 3 & 1 & YES & YES & YES & $1.25$ & $(2,2)$ & NO & 1000\\
$(47,13)$ & 8 & $(5,2)$ & 3 & 1 & YES & YES & NO(3) & $0.75$ & $(2,2)$ & -- & 1001\\
$(47,14)$ & 9 & $(5,1)$ & 4 & 1 & YES & YES & YES & $1.22$ & $(2,2)$ & -- & 1002\\
$(47,14)$ & 9 & $(5,1)$ & 4 & 1 & YES & YES & YES & $1.33$ & $(2,2)$ & NO & 1003\\
$(47,17)$ & 9 & $(6,1)$ & 5 & 1 & YES & YES & YES & $1.11$ & $(2,2)$ & -- & 1004\\
$(47,13)$ & 8 & $(11,3)$ & 5 & 1 & YES & YES & YES & $0.88$ & $(4,1)$ & 815 & 1005\\
$(47,13)$ & 8 & $(18,5)$ & 6 & 1 & YES & YES & YES & $0.88$ & $(4,1)$ & NO & 1006\\
$(47,13)$ & 8 & $(19,5)$ & 7 & 1 & YES & YES & YES & $0.75$ & $(4,1)$ & NO & 1007\\
$(47,13)$ & 8 & $(25,7)$ & 7 & 1 & YES & YES & NO(3) & $0.75$ & $(2,2)$ & NO & 1008\\
$(47,14)$ & 9 & $(27,8)$ & 7 & 1 & YES & YES & YES & $1.22$ & $(2,2)$ & 1283 & 1009\\
$(47,13)$ & 8 & $(29,8)$ & 7 & 1 & YES & YES & YES & $1.12$ & $(2,2)$ & NO & 1010\\
$(47,18)$ & 8 & $(34,13)$ & 7 & 1 & YES & YES & YES & $1.12$ & $(2,2)$ & NO & 1011\\
$(47,13)$ & 8 & $(47,13)$ & 8 & 47 & YES & YES & YES & $0.88$ & $(2,2)$ & NO & 1012\\
$(47,14)$ & 9 & $(47,14)$ & 9 & 47 & YES & YES & YES & $0.89$ & $(2,2)$ & NO & 1013\\
$(49,18)$ & 8 & $(2,1)$ & 1 & 1 & YES & YES & YES & $1.00$ & $(2,2)$ & -- & 1014\\
$(49,19)$ & 8 & $(2,1)$ & 1 & 1 & YES & YES & YES & $1.12$ & $(2,2)$ & -- & 1015\\
$(49,19)$ & 8 & $(2,1)$ & 1 & 1 & YES & YES & YES & $1.12$ & $(2,2)$ & NO & 1016\\
$(49,18)$ & 8 & $(3,1)$ & 2 & 1 & YES & YES & YES & $1.00$ & $(2,2)$ & NO & 1017\\
$(49,18)$ & 8 & $(3,1)$ & 2 & 1 & YES & YES & YES & $1.00$ & $(2,2)$ & -- & 1018\\
$(49,19)$ & 8 & $(3,1)$ & 2 & 1 & YES & YES & YES & $1.12$ & $(2,2)$ & NO & 1019\\
$(49,19)$ & 8 & $(3,1)$ & 2 & 1 & YES & YES & YES & $1.12$ & $(2,2)$ & -- & 1020\\
$(49,19)$ & 8 & $(3,1)$ & 2 & 1 & YES & YES & YES & $1.11$ & $(2,2)$ & NO & 1021\\
$(49,20)$ & 9 & $(3,1)$ & 2 & 1 & YES & YES & YES & $1.00$ & $(2,2)$ & -- & 1022\\
$(49,13)$ & 9 & $(4,1)$ & 3 & 1 & YES & YES & YES & $1.11$ & $(2,2)$ & -- & 1023\\
$(49,13)$ & 9 & $(4,1)$ & 3 & 1 & YES & YES & YES & $1.22$ & $(2,2)$ & NO & 1024\\
$(49,18)$ & 8 & $(4,1)$ & 3 & 1 & YES & YES & YES & $1.12$ & $(2,2)$ & NO & 1025\\
$(49,18)$ & 8 & $(4,1)$ & 3 & 1 & YES & YES & YES & $1.12$ & $(2,2)$ & -- & 1026\\
$(49,19)$ & 8 & $(4,1)$ & 3 & 1 & YES & YES & YES & $1.00$ & $(2,2)$ & -- & 1027\\
$(49,20)$ & 9 & $(4,1)$ & 3 & 1 & YES & YES & YES & $1.00$ & $(2,2)$ & NO & 1028\\
$(49,11)$ & 10 & $(5,2)$ & 3 & 1 & YES & YES & YES & $1.12$ & $(2,2)$ & NO & 1029\\
$(49,13)$ & 9 & $(5,2)$ & 3 & 1 & YES & YES & YES & $0.88$ & $(4,1)$ & NO & 1030\\
$(49,13)$ & 9 & $(5,2)$ & 3 & 1 & YES & YES & YES & $0.88$ & $(4,1)$ & -- & 1031\\
$(49,15)$ & 9 & $(5,1)$ & 4 & 1 & YES & YES & YES & $0.88$ & $(4,1)$ & NO & 1032\\
$(49,15)$ & 9 & $(5,1)$ & 4 & 1 & YES & YES & YES & $0.88$ & $(4,1)$ & -- & 1033\\
$(49,15)$ & 9 & $(5,2)$ & 3 & 1 & YES & YES & YES & $1.22$ & $(2,2)$ & -- & 1034\\
$(49,18)$ & 8 & $(5,1)$ & 4 & 1 & YES & YES & YES & $0.88$ & $(2,2)$ & NO & 1035\\
$(49,18)$ & 8 & $(5,1)$ & 4 & 1 & YES & YES & YES & $0.88$ & $(2,2)$ & -- & 1036\\
$(49,18)$ & 8 & $(5,2)$ & 3 & 1 & YES & YES & YES & $0.88$ & $(4,1)$ & -- & 1037\\
$(49,19)$ & 8 & $(5,2)$ & 3 & 1 & YES & YES & YES & $1.00$ & $(2,2)$ & -- & 1038\\
$(49,19)$ & 8 & $(5,2)$ & 3 & 1 & YES & YES & YES & $1.22$ & $(2,2)$ & NO & 1039\\
$(49,15)$ & 9 & $(6,1)$ & 5 & 1 & YES & YES & YES & $1.11$ & $(2,2)$ & NO & 1040\\
$(49,15)$ & 9 & $(6,1)$ & 5 & 1 & YES & YES & YES & $1.11$ & $(2,2)$ & -- & 1041\\
$(49,9)$ & 10 & $(7,2)$ & 4 & 7 & YES & YES & YES & $1.00$ & $(2,2)$ & NO & 1042\\
$(49,13)$ & 9 & $(7,2)$ & 4 & 7 & YES & YES & YES & $1.22$ & $(2,2)$ & -- & 1043\\
$(49,18)$ & 8 & $(7,2)$ & 4 & 7 & YES & YES & YES & $1.11$ & $(2,2)$ & NO & 1044\\
$(49,19)$ & 8 & $(7,3)$ & 4 & 7 & YES & YES & YES & $1.00$ & $(2,2)$ & NO & 1045\\
$(49,19)$ & 8 & $(8,3)$ & 4 & 1 & YES & YES & YES & $1.12$ & $(2,2)$ & 1137 & 1046\\
$(49,15)$ & 9 & $(9,2)$ & 5 & 1 & YES & YES & YES & $1.11$ & $(2,2)$ & NO & 1047\\
$(49,18)$ & 8 & $(9,2)$ & 5 & 1 & YES & YES & YES & $1.11$ & $(2,2)$ & NO & 1048\\
$(49,13)$ & 9 & $(11,3)$ & 5 & 1 & YES & YES & YES & $1.00$ & $(2,2)$ & NO & 1049\\
$(49,15)$ & 9 & $(11,3)$ & 5 & 1 & YES & YES & YES & $1.11$ & $(2,2)$ & NO & 1050\\
$(49,18)$ & 8 & $(11,4)$ & 5 & 1 & YES & YES & YES & $0.88$ & $(4,1)$ & 875 & 1051\\
$(49,20)$ & 9 & $(12,5)$ & 5 & 1 & YES & YES & YES & $1.00$ & $(2,2)$ & 646 & 1052\\
$(49,15)$ & 9 & $(13,4)$ & 6 & 1 & YES & YES & YES & $1.22$ & $(2,2)$ & NO & 1053\\
$(49,19)$ & 8 & $(13,5)$ & 5 & 1 & YES & YES & YES & $1.00$ & $(2,2)$ & 941 & 1054\\
$(49,18)$ & 8 & $(14,5)$ & 6 & 7 & YES & YES & YES & $1.00$ & $(4,1)$ & NO & 1055\\
$(49,15)$ & 9 & $(17,5)$ & 6 & 1 & YES & YES & YES & $1.11$ & $(2,2)$ & 1534 & 1056\\
$(49,19)$ & 8 & $(21,8)$ & 6 & 7 & YES & YES & YES & $1.11$ & $(2,2)$ & NO & 1057\\
$(49,19)$ & 8 & $(23,9)$ & 7 & 1 & YES & YES & YES & $1.00$ & $(2,2)$ & NO & 1058\\
$(49,18)$ & 8 & $(30,11)$ & 7 & 1 & YES & YES & YES & $1.12$ & $(2,2)$ & NO & 1059\\
$(49,15)$ & 9 & $(33,10)$ & 8 & 1 & YES & YES & YES & $1.11$ & $(2,2)$ & 1501 & 1060\\
$(49,18)$ & 8 & $(41,15)$ & 8 & 1 & YES & YES & YES & $1.11$ & $(2,2)$ & NO & 1061\\
$(49,19)$ & 8 & $(44,17)$ & 8 & 1 & YES & YES & YES & $1.00$ & $(2,2)$ & NO & 1062\\
$(49,18)$ & 8 & $(49,18)$ & 8 & 49 & YES & YES & YES & $0.88$ & $(2,2)$ & NO & 1063\\
$(49,20)$ & 9 & $(49,20)$ & 9 & 49 & YES & YES & YES & $1.00$ & $(2,2)$ & NO & 1064\\
$(50,19)$ & 8 & $(2,1)$ & 1 & 2 & YES & YES & YES & $0.88$ & $(4,1)$ & -- & 1065\\
$(50,21)$ & 8 & $(2,1)$ & 1 & 2 & YES & YES & YES & $1.11$ & $(2,2)$ & NO & 1066\\
$(50,19)$ & 8 & $(3,1)$ & 2 & 1 & YES & YES & YES & $1.00$ & $(2,2)$ & -- & 1067\\
$(50,19)$ & 8 & $(3,1)$ & 2 & 1 & YES & YES & YES & $0.88$ & $(4,1)$ & NO & 1068\\
$(50,21)$ & 8 & $(4,1)$ & 3 & 2 & YES & YES & YES & $1.22$ & $(2,2)$ & NO & 1069\\
$(50,21)$ & 8 & $(4,1)$ & 3 & 2 & YES & YES & YES & $1.22$ & $(2,2)$ & -- & 1070\\
$(50,11)$ & 10 & $(5,2)$ & 3 & 5 & YES & YES & YES & $0.88$ & $(2,2)$ & NO & 1071\\
$(50,19)$ & 8 & $(5,1)$ & 4 & 5 & YES & YES & YES & $0.88$ & $(2,2)$ & NO & 1072\\
$(50,19)$ & 8 & $(5,1)$ & 4 & 5 & YES & YES & YES & $0.88$ & $(2,2)$ & -- & 1073\\
$(50,19)$ & 8 & $(5,2)$ & 3 & 5 & YES & YES & YES & $1.11$ & $(2,2)$ & -- & 1074\\
$(50,21)$ & 8 & $(5,2)$ & 3 & 5 & YES & YES & YES & $1.11$ & $(2,2)$ & NO & 1075\\
$(50,21)$ & 8 & $(5,2)$ & 3 & 5 & YES & YES & YES & $1.11$ & $(2,2)$ & -- & 1076\\
$(50,21)$ & 8 & $(5,2)$ & 3 & 5 & YES & YES & YES & $1.00$ & $(2,2)$ & 983 & 1077\\
$(50,19)$ & 8 & $(7,3)$ & 4 & 1 & YES & YES & YES & $1.11$ & $(2,2)$ & NO & 1078\\
$(50,19)$ & 8 & $(8,3)$ & 4 & 2 & YES & YES & YES & $1.00$ & $(2,2)$ & 781 & 1079\\
$(50,21)$ & 8 & $(8,3)$ & 4 & 2 & YES & YES & YES & $1.22$ & $(2,2)$ & NO & 1080\\
$(50,21)$ & 8 & $(9,4)$ & 5 & 1 & YES & YES & YES & $1.22$ & $(2,2)$ & NO & 1081\\
$(50,19)$ & 8 & $(13,5)$ & 5 & 1 & YES & YES & YES & $1.00$ & $(2,2)$ & NO & 1082\\
$(50,21)$ & 8 & $(17,7)$ & 6 & 1 & YES & YES & YES & $1.22$ & $(2,2)$ & NO & 1083\\
$(50,19)$ & 8 & $(21,8)$ & 6 & 1 & YES & YES & YES & $0.75$ & $(4,1)$ & NO & 1084\\
$(50,21)$ & 8 & $(26,11)$ & 7 & 2 & YES & YES & YES & $1.22$ & $(2,2)$ & NO & 1085\\
$(51,20)$ & 9 & $(2,1)$ & 1 & 1 & YES & YES & YES & $1.11$ & $(2,2)$ & -- & 1086\\
$(51,20)$ & 9 & $(2,1)$ & 1 & 1 & YES & YES & YES & $1.22$ & $(2,2)$ & NO & 1087\\
$(51,20)$ & 9 & $(3,1)$ & 2 & 3 & YES & YES & YES & $1.12$ & $(4,1)$ & -- & 1088\\
$(51,14)$ & 9 & $(4,1)$ & 3 & 1 & YES & YES & YES & $1.22$ & $(2,2)$ & NO & 1089\\
$(51,14)$ & 9 & $(4,1)$ & 3 & 1 & YES & YES & YES & $1.11$ & $(2,2)$ & -- & 1090\\
$(51,20)$ & 9 & $(4,1)$ & 3 & 1 & YES & YES & YES & $0.88$ & $(4,1)$ & NO & 1091\\
$(51,14)$ & 9 & $(5,2)$ & 3 & 1 & YES & YES & YES & $1.22$ & $(2,2)$ & -- & 1092\\
$(51,20)$ & 9 & $(5,1)$ & 4 & 1 & YES & YES & YES & $0.88$ & $(4,1)$ & NO & 1093\\
$(51,20)$ & 9 & $(5,1)$ & 4 & 1 & YES & YES & YES & $0.88$ & $(4,1)$ & -- & 1094\\
$(51,20)$ & 9 & $(7,3)$ & 4 & 1 & YES & YES & YES & $1.12$ & $(2,2)$ & NO & 1095\\
$(51,20)$ & 9 & $(8,3)$ & 4 & 1 & YES & YES & YES & $0.88$ & $(4,1)$ & NO & 1096\\
$(51,20)$ & 9 & $(13,5)$ & 5 & 1 & YES & YES & YES & $0.88$ & $(4,1)$ & 696 & 1097\\
$(51,20)$ & 9 & $(18,7)$ & 6 & 3 & YES & YES & YES & $1.00$ & $(4,1)$ & NO & 1098\\
$(51,20)$ & 9 & $(23,9)$ & 7 & 1 & YES & YES & YES & $1.11$ & $(2,2)$ & NO & 1099\\
$(51,11)$ & 9 & $(24,5)$ & 8 & 3 & YES & YES & YES & $0.88$ & $(2,2)$ & NO & 1100\\
$(51,14)$ & 9 & $(25,7)$ & 7 & 1 & YES & YES & YES & $1.11$ & $(2,2)$ & NO & 1101\\
$(51,11)$ & 9 & $(33,7)$ & 8 & 3 & YES & YES & YES & $0.88$ & $(2,2)$ & NO & 1102\\
$(52,19)$ & 9 & $(2,1)$ & 1 & 2 & YES & YES & YES & $1.22$ & $(2,2)$ & -- & 1103\\
$(52,19)$ & 9 & $(3,1)$ & 2 & 1 & YES & YES & YES & $1.22$ & $(2,2)$ & NO & 1104\\
$(52,19)$ & 9 & $(4,1)$ & 3 & 4 & YES & YES & YES & $1.22$ & $(2,2)$ & -- & 1105\\
$(52,19)$ & 9 & $(5,1)$ & 4 & 1 & YES & YES & YES & $1.11$ & $(2,2)$ & NO & 1106\\
$(52,19)$ & 9 & $(7,1)$ & 6 & 1 & YES & YES & YES & $1.11$ & $(2,2)$ & NO & 1107\\
$(52,19)$ & 9 & $(11,4)$ & 5 & 1 & YES & YES & YES & $1.22$ & $(2,2)$ & NO & 1108\\
$(52,19)$ & 9 & $(19,7)$ & 6 & 1 & YES & YES & YES & $1.11$ & $(2,2)$ & NO & 1109\\
$(52,19)$ & 9 & $(30,11)$ & 7 & 2 & YES & YES & YES & $1.11$ & $(2,2)$ & 1368 & 1110\\
$(52,19)$ & 9 & $(41,15)$ & 8 & 1 & YES & YES & YES & $1.11$ & $(2,2)$ & NO & 1111\\
$(53,14)$ & 9 & $(3,1)$ & 2 & 1 & YES & YES & YES & $1.00$ & $(2,2)$ & -- & 1112\\
$(53,14)$ & 9 & $(3,1)$ & 2 & 1 & YES & YES & YES & $1.11$ & $(2,2)$ & NO & 1113\\
$(53,19)$ & 9 & $(3,1)$ & 2 & 1 & YES & YES & YES & $0.88$ & $(4,1)$ & -- & 1114\\
$(53,19)$ & 9 & $(3,1)$ & 2 & 1 & YES & YES & YES & $1.00$ & $(4,1)$ & NO & 1115\\
$(53,14)$ & 9 & $(4,1)$ & 3 & 1 & YES & YES & YES & $1.00$ & $(2,2)$ & -- & 1116\\
$(53,19)$ & 9 & $(4,1)$ & 3 & 1 & YES & YES & YES & $0.88$ & $(4,1)$ & -- & 1117\\
$(53,23)$ & 9 & $(4,1)$ & 3 & 1 & YES & YES & YES & $0.88$ & $(2,2)$ & NO & 1118\\
$(53,19)$ & 9 & $(5,2)$ & 3 & 1 & YES & YES & YES & $0.88$ & $(4,1)$ & NO & 1119\\
$(53,23)$ & 9 & $(5,2)$ & 3 & 1 & YES & YES & YES & $1.00$ & $(2,2)$ & NO & 1120\\
$(53,19)$ & 9 & $(6,1)$ & 5 & 1 & YES & YES & YES & $1.11$ & $(2,2)$ & -- & 1121\\
$(53,14)$ & 9 & $(7,2)$ & 4 & 1 & YES & YES & YES & $1.00$ & $(2,2)$ & NO & 1122\\
$(53,16)$ & 10 & $(7,1)$ & 6 & 1 & YES & YES & YES & $1.11$ & $(2,2)$ & NO & 1123\\
$(53,19)$ & 9 & $(8,3)$ & 4 & 1 & YES & YES & YES & $0.88$ & $(4,1)$ & NO & 1124\\
$(53,14)$ & 9 & $(11,3)$ & 5 & 1 & YES & YES & YES & $1.00$ & $(2,2)$ & 1326 & 1125\\
$(53,19)$ & 9 & $(14,5)$ & 6 & 1 & YES & YES & YES & $1.22$ & $(2,2)$ & NO & 1126\\
$(53,19)$ & 9 & $(25,9)$ & 7 & 1 & YES & YES & YES & $0.88$ & $(4,1)$ & 1288 & 1127\\
$(53,23)$ & 9 & $(30,13)$ & 8 & 1 & YES & YES & YES & $1.00$ & $(2,2)$ & NO & 1128\\
$(53,14)$ & 9 & $(34,9)$ & 8 & 1 & YES & YES & YES & $0.89$ & $(2,2)$ & NO & 1129\\
$(54,19)$ & 10 & $(2,1)$ & 1 & 2 & YES & YES & YES & $1.00$ & $(4,1)$ & NO & 1130\\
$(54,19)$ & 10 & $(17,6)$ & 7 & 1 & YES & YES & YES & $0.88$ & $(4,1)$ & NO & 1131\\
$(55,21)$ & 8 & $(3,1)$ & 2 & 1 & YES & YES & YES & $1.00$ & $(2,2)$ & -- & 1132\\
$(55,24)$ & 9 & $(3,1)$ & 2 & 1 & YES & YES & YES & $1.25$ & $(2,2)$ & NO & 1133\\
$(55,24)$ & 9 & $(3,1)$ & 2 & 1 & YES & YES & YES & $1.25$ & $(2,2)$ & -- & 1134\\
$(55,21)$ & 8 & $(5,1)$ & 4 & 5 & YES & YES & NO(3) & $0.75$ & $(2,2)$ & NO & 1135\\
$(55,21)$ & 8 & $(5,2)$ & 3 & 5 & YES & YES & YES & $1.00$ & $(2,2)$ & -- & 1136\\
$(55,21)$ & 8 & $(5,2)$ & 3 & 5 & YES & YES & YES & $1.12$ & $(2,2)$ & 1046 & 1137\\
$(55,24)$ & 9 & $(5,2)$ & 3 & 5 & YES & YES & YES & $1.12$ & $(2,2)$ & 905 & 1138\\
$(55,21)$ & 8 & $(8,3)$ & 4 & 1 & YES & YES & YES & $1.12$ & $(2,2)$ & NO & 1139\\
$(55,16)$ & 9 & $(10,3)$ & 5 & 5 & YES & YES & YES & $0.89$ & $(2,2)$ & 695 & 1140\\
$(55,21)$ & 8 & $(18,7)$ & 6 & 1 & YES & YES & YES & $1.11$ & $(2,2)$ & NO & 1141\\
$(55,13)$ & 10 & $(22,5)$ & 7 & 11 & YES & YES & YES & $1.00$ & $(2,2)$ & NO & 1142\\
$(55,24)$ & 9 & $(23,10)$ & 7 & 1 & YES & YES & YES & $0.88$ & $(4,1)$ & 1236 & 1143\\
$(56,17)$ & 9 & $(2,1)$ & 1 & 2 & YES & YES & YES & $0.89$ & $(2,2)$ & -- & 1144\\
$(56,17)$ & 9 & $(3,1)$ & 2 & 1 & YES & YES & YES & $1.12$ & $(2,2)$ & -- & 1145\\
$(56,17)$ & 9 & $(3,1)$ & 2 & 1 & YES & YES & YES & $1.25$ & $(2,2)$ & NO & 1146\\
$(56,15)$ & 9 & $(4,1)$ & 3 & 4 & YES & YES & YES & $1.11$ & $(2,2)$ & -- & 1147\\
$(56,15)$ & 9 & $(4,1)$ & 3 & 4 & YES & YES & YES & $1.22$ & $(2,2)$ & NO & 1148\\
$(56,23)$ & 9 & $(4,1)$ & 3 & 4 & YES & YES & YES & $1.11$ & $(2,2)$ & -- & 1149\\
$(56,13)$ & 10 & $(5,2)$ & 3 & 1 & YES & YES & YES & $1.00$ & $(2,2)$ & -- & 1150\\
$(56,15)$ & 9 & $(5,2)$ & 3 & 1 & YES & YES & YES & $1.11$ & $(2,2)$ & NO & 1151\\
$(56,15)$ & 9 & $(5,2)$ & 3 & 1 & YES & YES & YES & $1.11$ & $(2,2)$ & -- & 1152\\
$(56,17)$ & 9 & $(5,2)$ & 3 & 1 & YES & YES & YES & $1.11$ & $(2,2)$ & -- & 1153\\
$(56,15)$ & 9 & $(9,2)$ & 5 & 1 & YES & YES & YES & $1.11$ & $(2,2)$ & NO & 1154\\
$(56,13)$ & 10 & $(11,2)$ & 6 & 1 & YES & YES & YES & $1.00$ & $(2,2)$ & NO & 1155\\
$(56,17)$ & 9 & $(13,4)$ & 6 & 1 & YES & YES & YES & $1.12$ & $(2,2)$ & NO & 1156\\
$(56,17)$ & 9 & $(16,5)$ & 7 & 8 & YES & YES & YES & $1.11$ & $(2,2)$ & NO & 1157\\
$(56,23)$ & 9 & $(39,16)$ & 8 & 1 & YES & YES & YES & $1.11$ & $(2,2)$ & NO & 1158\\
$(57,25)$ & 9 & $(3,1)$ & 2 & 3 & YES & YES & YES & $1.12$ & $(2,2)$ & NO & 1159\\
$(57,25)$ & 9 & $(3,1)$ & 2 & 3 & YES & YES & YES & $1.12$ & $(2,2)$ & -- & 1160\\
$(57,25)$ & 9 & $(3,1)$ & 2 & 3 & YES & YES & YES & $1.12$ & $(2,2)$ & NO & 1161\\
$(57,25)$ & 9 & $(5,2)$ & 3 & 1 & YES & YES & YES & $1.12$ & $(2,2)$ & NO & 1162\\
$(57,17)$ & 10 & $(6,1)$ & 5 & 3 & YES & YES & YES & $1.22$ & $(2,2)$ & NO & 1163\\
$(57,25)$ & 9 & $(9,4)$ & 5 & 3 & YES & YES & YES & $1.00$ & $(4,1)$ & NO & 1164\\
$(57,25)$ & 9 & $(25,11)$ & 7 & 1 & YES & YES & YES & $1.22$ & $(2,2)$ & 1319 & 1165\\
$(57,17)$ & 10 & $(27,8)$ & 7 & 3 & YES & YES & YES & $1.22$ & $(2,2)$ & NO & 1166\\
$(57,25)$ & 9 & $(41,18)$ & 8 & 1 & YES & YES & YES & $1.22$ & $(2,2)$ & NO & 1167\\
$(57,16)$ & 9 & $(43,12)$ & 8 & 1 & YES & YES & YES & $1.11$ & $(2,2)$ & NO & 1168\\
$(58,17)$ & 9 & $(2,1)$ & 1 & 2 & YES & YES & YES & $1.00$ & $(2,2)$ & NO & 1169\\
$(58,17)$ & 9 & $(3,1)$ & 2 & 1 & YES & YES & YES & $1.00$ & $(2,2)$ & NO & 1170\\
$(58,17)$ & 9 & $(3,1)$ & 2 & 1 & YES & YES & YES & $1.00$ & $(2,2)$ & -- & 1171\\
$(58,17)$ & 9 & $(10,3)$ & 5 & 2 & YES & YES & YES & $1.00$ & $(2,2)$ & NO & 1172\\
$(59,18)$ & 9 & $(2,1)$ & 1 & 1 & YES & YES & YES & $1.12$ & $(2,2)$ & -- & 1173\\
$(59,25)$ & 9 & $(2,1)$ & 1 & 1 & YES & YES & YES & $1.00$ & $(2,2)$ & NO & 1174\\
$(59,25)$ & 9 & $(2,1)$ & 1 & 1 & YES & YES & YES & $1.00$ & $(2,2)$ & -- & 1175\\
$(59,18)$ & 9 & $(3,1)$ & 2 & 1 & YES & YES & YES & $1.12$ & $(2,2)$ & NO & 1176\\
$(59,18)$ & 9 & $(3,1)$ & 2 & 1 & YES & YES & YES & $1.12$ & $(2,2)$ & -- & 1177\\
$(59,18)$ & 9 & $(3,1)$ & 2 & 1 & YES & YES & YES & $1.11$ & $(2,2)$ & 962 & 1178\\
$(59,21)$ & 10 & $(3,1)$ & 2 & 1 & YES & YES & YES & $0.88$ & $(4,1)$ & NO & 1179\\
$(59,26)$ & 9 & $(3,1)$ & 2 & 1 & YES & YES & YES & $1.22$ & $(2,2)$ & -- & 1180\\
$(59,26)$ & 9 & $(3,1)$ & 2 & 1 & YES & YES & YES & $0.88$ & $(4,1)$ & NO & 1181\\
$(59,23)$ & 9 & $(4,1)$ & 3 & 1 & YES & YES & YES & $0.88$ & $(2,2)$ & NO & 1182\\
$(59,26)$ & 9 & $(4,1)$ & 3 & 1 & YES & YES & YES & $1.22$ & $(2,2)$ & NO & 1183\\
$(59,26)$ & 9 & $(4,1)$ & 3 & 1 & YES & YES & YES & $1.22$ & $(2,2)$ & -- & 1184\\
$(59,18)$ & 9 & $(5,1)$ & 4 & 1 & YES & YES & YES & $1.00$ & $(2,2)$ & NO & 1185\\
$(59,18)$ & 9 & $(5,2)$ & 3 & 1 & YES & YES & YES & $1.22$ & $(2,2)$ & NO & 1186\\
$(59,25)$ & 9 & $(5,1)$ & 4 & 1 & YES & YES & YES & $0.88$ & $(2,2)$ & -- & 1187\\
$(59,25)$ & 9 & $(5,1)$ & 4 & 1 & YES & YES & YES & $1.00$ & $(2,2)$ & NO & 1188\\
$(59,25)$ & 9 & $(5,2)$ & 3 & 1 & YES & YES & YES & $1.00$ & $(2,2)$ & NO & 1189\\
$(59,26)$ & 9 & $(5,2)$ & 3 & 1 & YES & YES & YES & $1.22$ & $(2,2)$ & NO & 1190\\
$(59,16)$ & 10 & $(7,1)$ & 6 & 1 & YES & YES & YES & $0.88$ & $(2,2)$ & NO & 1191\\
$(59,25)$ & 9 & $(7,3)$ & 4 & 1 & YES & YES & YES & $1.00$ & $(2,2)$ & 817 & 1192\\
$(59,26)$ & 9 & $(7,3)$ & 4 & 1 & YES & YES & YES & $0.88$ & $(4,1)$ & 744 & 1193\\
$(59,23)$ & 9 & $(8,3)$ & 4 & 1 & YES & YES & YES & $0.88$ & $(2,2)$ & NO & 1194\\
$(59,18)$ & 9 & $(9,2)$ & 5 & 1 & YES & YES & YES & $1.11$ & $(2,2)$ & NO & 1195\\
$(59,26)$ & 9 & $(9,4)$ & 5 & 1 & YES & YES & YES & $1.00$ & $(2,2)$ & 909 & 1196\\
$(59,18)$ & 9 & $(10,3)$ & 5 & 1 & YES & YES & YES & $1.12$ & $(2,2)$ & NO & 1197\\
$(59,23)$ & 9 & $(13,5)$ & 5 & 1 & YES & YES & YES & $0.88$ & $(2,2)$ & NO & 1198\\
$(59,18)$ & 9 & $(16,5)$ & 7 & 1 & YES & YES & YES & $1.22$ & $(2,2)$ & NO & 1199\\
$(59,26)$ & 9 & $(16,7)$ & 6 & 1 & YES & YES & YES & $1.22$ & $(2,2)$ & NO & 1200\\
$(59,18)$ & 9 & $(23,7)$ & 7 & 1 & YES & YES & YES & $1.12$ & $(2,2)$ & NO & 1201\\
$(59,25)$ & 9 & $(26,11)$ & 7 & 1 & YES & YES & YES & $0.88$ & $(2,2)$ & NO & 1202\\
$(59,18)$ & 9 & $(33,10)$ & 8 & 1 & YES & YES & YES & $1.11$ & $(2,2)$ & NO & 1203\\
$(59,26)$ & 9 & $(34,15)$ & 8 & 1 & YES & YES & YES & $1.22$ & $(2,2)$ & NO & 1204\\
$(59,16)$ & 10 & $(37,10)$ & 8 & 1 & YES & YES & YES & $0.88$ & $(2,2)$ & 1502 & 1205\\
$(59,23)$ & 9 & $(59,23)$ & 9 & 59 & YES & YES & YES & $0.88$ & $(2,2)$ & NO & 1206\\
$(59,26)$ & 9 & $(59,26)$ & 9 & 59 & YES & YES & YES & $1.22$ & $(2,2)$ & NO & 1207\\
$(61,18)$ & 9 & $(2,1)$ & 1 & 1 & YES & YES & YES & $1.12$ & $(2,2)$ & NO & 1208\\
$(61,22)$ & 9 & $(3,1)$ & 2 & 1 & YES & YES & YES & $1.00$ & $(2,2)$ & -- & 1209\\
$(61,22)$ & 9 & $(4,1)$ & 3 & 1 & YES & YES & YES & $1.00$ & $(2,2)$ & NO & 1210\\
$(61,22)$ & 9 & $(5,2)$ & 3 & 1 & YES & YES & YES & $0.88$ & $(4,1)$ & NO & 1211\\
$(61,22)$ & 9 & $(8,3)$ & 4 & 1 & YES & YES & YES & $1.00$ & $(4,1)$ & NO & 1212\\
$(61,18)$ & 9 & $(10,3)$ & 5 & 1 & YES & YES & YES & $1.00$ & $(2,2)$ & NO & 1213\\
$(61,14)$ & 10 & $(11,2)$ & 6 & 1 & YES & YES & YES & $1.00$ & $(2,2)$ & NO & 1214\\
$(61,18)$ & 9 & $(13,4)$ & 6 & 1 & YES & YES & YES & $1.00$ & $(2,2)$ & 1453 & 1215\\
$(61,14)$ & 10 & $(14,3)$ & 6 & 1 & YES & YES & YES & $1.00$ & $(2,2)$ & NO & 1216\\
$(61,22)$ & 9 & $(14,5)$ & 6 & 1 & YES & YES & YES & $1.00$ & $(2,2)$ & NO & 1217\\
$(61,14)$ & 10 & $(31,7)$ & 8 & 1 & YES & YES & YES & $1.00$ & $(2,2)$ & NO & 1218\\
$(61,22)$ & 9 & $(36,13)$ & 8 & 1 & YES & YES & YES & $1.00$ & $(2,2)$ & NO & 1219\\
$(61,18)$ & 9 & $(44,13)$ & 8 & 1 & YES & YES & NO(3) & $0.75$ & $(2,2)$ & NO & 1220\\
$(61,22)$ & 9 & $(61,22)$ & 9 & 61 & YES & YES & YES & $0.88$ & $(4,1)$ & NO & 1221\\
$(62,27)$ & 9 & $(2,1)$ & 1 & 2 & YES & YES & YES & $1.00$ & $(2,2)$ & -- & 1222\\
$(62,17)$ & 10 & $(3,1)$ & 2 & 1 & YES & YES & YES & $1.00$ & $(4,1)$ & NO & 1223\\
$(62,17)$ & 10 & $(3,1)$ & 2 & 1 & YES & YES & YES & $1.00$ & $(4,1)$ & -- & 1224\\
$(62,23)$ & 9 & $(3,1)$ & 2 & 1 & YES & YES & YES & $1.00$ & $(2,2)$ & -- & 1225\\
$(62,27)$ & 9 & $(3,1)$ & 2 & 1 & YES & YES & YES & $1.11$ & $(2,2)$ & -- & 1226\\
$(62,27)$ & 9 & $(3,1)$ & 2 & 1 & YES & YES & YES & $1.11$ & $(2,2)$ & NO & 1227\\
$(62,17)$ & 10 & $(4,1)$ & 3 & 2 & YES & YES & YES & $1.11$ & $(2,2)$ & -- & 1228\\
$(62,27)$ & 9 & $(4,1)$ & 3 & 2 & YES & YES & YES & $1.11$ & $(2,2)$ & NO & 1229\\
$(62,27)$ & 9 & $(4,1)$ & 3 & 2 & YES & YES & YES & $1.11$ & $(2,2)$ & -- & 1230\\
$(62,27)$ & 9 & $(5,2)$ & 3 & 1 & YES & YES & YES & $1.11$ & $(2,2)$ & NO & 1231\\
$(62,17)$ & 10 & $(6,1)$ & 5 & 2 & YES & YES & YES & $1.22$ & $(2,2)$ & NO & 1232\\
$(62,19)$ & 10 & $(7,1)$ & 6 & 1 & YES & YES & YES & $1.11$ & $(2,2)$ & NO & 1233\\
$(62,27)$ & 9 & $(9,4)$ & 5 & 1 & YES & YES & YES & $1.33$ & $(2,2)$ & 1316 & 1234\\
$(62,19)$ & 10 & $(13,4)$ & 6 & 1 & YES & YES & YES & $1.22$ & $(2,2)$ & NO & 1235\\
$(62,27)$ & 9 & $(16,7)$ & 6 & 2 & YES & YES & YES & $0.88$ & $(4,1)$ & 1143 & 1236\\
$(62,17)$ & 10 & $(18,5)$ & 6 & 2 & YES & YES & YES & $0.88$ & $(4,1)$ & NO & 1237\\
$(62,23)$ & 9 & $(19,7)$ & 6 & 1 & YES & YES & YES & $1.00$ & $(2,2)$ & NO & 1238\\
$(62,17)$ & 10 & $(29,8)$ & 7 & 1 & YES & YES & YES & $1.22$ & $(2,2)$ & NO & 1239\\
$(62,27)$ & 9 & $(39,17)$ & 8 & 1 & YES & YES & YES & $1.00$ & $(2,2)$ & NO & 1240\\
$(62,19)$ & 10 & $(49,15)$ & 9 & 1 & YES & YES & YES & $1.11$ & $(2,2)$ & NO & 1241\\
$(62,17)$ & 10 & $(51,14)$ & 9 & 1 & YES & YES & YES & $1.11$ & $(2,2)$ & NO & 1242\\
$(62,23)$ & 9 & $(62,23)$ & 9 & 62 & YES & YES & YES & $1.00$ & $(2,2)$ & NO & 1243\\
$(62,27)$ & 9 & $(62,27)$ & 9 & 62 & YES & YES & YES & $1.11$ & $(2,2)$ & NO & 1244\\
$(63,26)$ & 9 & $(3,1)$ & 2 & 3 & YES & YES & YES & $1.11$ & $(2,2)$ & -- & 1245\\
$(63,26)$ & 9 & $(4,1)$ & 3 & 1 & YES & YES & YES & $1.11$ & $(2,2)$ & NO & 1246\\
$(63,26)$ & 9 & $(4,1)$ & 3 & 1 & YES & YES & YES & $1.11$ & $(2,2)$ & -- & 1247\\
$(63,26)$ & 9 & $(6,1)$ & 5 & 3 & YES & YES & YES & $1.00$ & $(2,2)$ & NO & 1248\\
$(63,26)$ & 9 & $(6,1)$ & 5 & 3 & YES & YES & YES & $1.00$ & $(2,2)$ & -- & 1249\\
$(63,26)$ & 9 & $(17,7)$ & 6 & 1 & YES & YES & YES & $1.12$ & $(2,2)$ & NO & 1250\\
$(63,17)$ & 9 & $(19,5)$ & 7 & 1 & YES & YES & YES & $1.11$ & $(2,2)$ & NO & 1251\\
$(63,17)$ & 9 & $(41,11)$ & 8 & 1 & YES & YES & YES & $1.11$ & $(2,2)$ & NO & 1252\\
$(63,26)$ & 9 & $(46,19)$ & 8 & 1 & YES & YES & YES & $1.11$ & $(2,2)$ & NO & 1253\\
$(64,23)$ & 9 & $(2,1)$ & 1 & 2 & YES & YES & YES & $1.00$ & $(2,2)$ & -- & 1254\\
$(64,23)$ & 9 & $(2,1)$ & 1 & 2 & YES & YES & YES & $1.00$ & $(2,2)$ & NO & 1255\\
$(64,27)$ & 9 & $(2,1)$ & 1 & 2 & YES & YES & YES & $0.88$ & $(4,1)$ & -- & 1256\\
$(64,27)$ & 9 & $(2,1)$ & 1 & 2 & YES & YES & YES & $1.00$ & $(4,1)$ & 740 & 1257\\
$(64,19)$ & 9 & $(3,1)$ & 2 & 1 & NO & YES & YES & $1.11$ & $(2,2)$ & -- & 1258\\
$(64,23)$ & 9 & $(3,1)$ & 2 & 1 & YES & YES & YES & $1.00$ & $(2,2)$ & -- & 1259\\
$(64,23)$ & 9 & $(3,1)$ & 2 & 1 & YES & YES & YES & $1.22$ & $(2,2)$ & NO & 1260\\
$(64,25)$ & 9 & $(3,1)$ & 2 & 1 & YES & YES & YES & $1.22$ & $(2,2)$ & NO & 1261\\
$(64,25)$ & 9 & $(3,1)$ & 2 & 1 & YES & YES & YES & $1.22$ & $(2,2)$ & -- & 1262\\
$(64,27)$ & 9 & $(3,1)$ & 2 & 1 & YES & YES & YES & $1.11$ & $(2,2)$ & -- & 1263\\
$(64,27)$ & 9 & $(3,1)$ & 2 & 1 & YES & YES & YES & $0.88$ & $(4,1)$ & NO & 1264\\
$(64,17)$ & 10 & $(4,1)$ & 3 & 4 & YES & YES & YES & $1.00$ & $(2,2)$ & -- & 1265\\
$(64,23)$ & 9 & $(4,1)$ & 3 & 4 & YES & YES & YES & $1.00$ & $(2,2)$ & NO & 1266\\
$(64,23)$ & 9 & $(4,1)$ & 3 & 4 & YES & YES & YES & $1.11$ & $(2,2)$ & -- & 1267\\
$(64,25)$ & 9 & $(4,1)$ & 3 & 4 & YES & YES & YES & $1.11$ & $(2,2)$ & -- & 1268\\
$(64,27)$ & 9 & $(4,1)$ & 3 & 4 & YES & YES & YES & $1.11$ & $(2,2)$ & NO & 1269\\
$(64,15)$ & 10 & $(5,2)$ & 3 & 1 & YES & YES & YES & $1.22$ & $(2,2)$ & NO & 1270\\
$(64,17)$ & 10 & $(5,1)$ & 4 & 1 & YES & YES & YES & $1.00$ & $(2,2)$ & -- & 1271\\
$(64,19)$ & 9 & $(5,1)$ & 4 & 1 & YES & YES & YES & $1.11$ & $(2,2)$ & -- & 1272\\
$(64,19)$ & 9 & $(5,1)$ & 4 & 1 & YES & YES & YES & $1.22$ & $(2,2)$ & NO & 1273\\
$(64,19)$ & 9 & $(5,2)$ & 3 & 1 & YES & YES & YES & $1.00$ & $(2,2)$ & NO & 1274\\
$(64,19)$ & 9 & $(5,2)$ & 3 & 1 & YES & YES & YES & $1.11$ & $(2,2)$ & -- & 1275\\
$(64,23)$ & 9 & $(5,1)$ & 4 & 1 & YES & YES & YES & $1.00$ & $(2,2)$ & NO & 1276\\
$(64,23)$ & 9 & $(5,2)$ & 3 & 1 & YES & YES & YES & $1.12$ & $(2,2)$ & 802 & 1277\\
$(64,27)$ & 9 & $(5,2)$ & 3 & 1 & YES & YES & YES & $0.88$ & $(4,1)$ & NO & 1278\\
$(64,15)$ & 10 & $(7,2)$ & 4 & 1 & YES & YES & YES & $1.22$ & $(2,2)$ & NO & 1279\\
$(64,27)$ & 9 & $(7,3)$ & 4 & 1 & YES & YES & YES & $1.00$ & $(2,2)$ & NO & 1280\\
$(64,23)$ & 9 & $(8,3)$ & 4 & 8 & YES & YES & YES & $1.11$ & $(2,2)$ & NO & 1281\\
$(64,25)$ & 9 & $(8,3)$ & 4 & 8 & YES & YES & YES & $1.22$ & $(2,2)$ & NO & 1282\\
$(64,19)$ & 9 & $(10,3)$ & 5 & 2 & YES & YES & YES & $1.22$ & $(2,2)$ & 1009 & 1283\\
$(64,15)$ & 10 & $(11,2)$ & 6 & 1 & YES & YES & YES & $1.11$ & $(2,2)$ & NO & 1284\\
$(64,23)$ & 9 & $(11,4)$ & 5 & 1 & YES & YES & YES & $1.00$ & $(2,2)$ & NO & 1285\\
$(64,27)$ & 9 & $(12,5)$ & 5 & 4 & YES & YES & YES & $0.88$ & $(4,1)$ & NO & 1286\\
$(64,25)$ & 9 & $(13,5)$ & 5 & 1 & YES & YES & YES & $1.11$ & $(2,2)$ & 1466 & 1287\\
$(64,23)$ & 9 & $(14,5)$ & 6 & 2 & YES & YES & YES & $0.88$ & $(4,1)$ & 1127 & 1288\\
$(64,23)$ & 9 & $(25,9)$ & 7 & 1 & YES & YES & YES & $1.00$ & $(2,2)$ & NO & 1289\\
$(64,27)$ & 9 & $(26,11)$ & 7 & 2 & YES & YES & YES & $1.11$ & $(2,2)$ & 1373 & 1290\\
$(64,17)$ & 10 & $(34,9)$ & 8 & 2 & YES & YES & YES & $1.00$ & $(2,2)$ & 1490 & 1291\\
$(64,23)$ & 9 & $(39,14)$ & 8 & 1 & YES & YES & YES & $1.00$ & $(2,2)$ & NO & 1292\\
$(64,15)$ & 10 & $(43,10)$ & 9 & 1 & YES & YES & YES & $1.11$ & $(2,2)$ & 1572 & 1293\\
$(64,27)$ & 9 & $(45,19)$ & 8 & 1 & YES & YES & YES & $1.11$ & $(2,2)$ & NO & 1294\\
$(64,17)$ & 10 & $(49,13)$ & 9 & 1 & YES & YES & YES & $1.00$ & $(2,2)$ & NO & 1295\\
$(64,23)$ & 9 & $(64,23)$ & 9 & 64 & YES & YES & YES & $1.11$ & $(2,2)$ & NO & 1296\\
$(65,24)$ & 9 & $(2,1)$ & 1 & 1 & YES & YES & YES & $1.00$ & $(2,2)$ & -- & 1297\\
$(65,24)$ & 9 & $(2,1)$ & 1 & 1 & YES & YES & YES & $1.11$ & $(2,2)$ & NO & 1298\\
$(65,24)$ & 9 & $(3,1)$ & 2 & 1 & YES & YES & YES & $1.00$ & $(2,2)$ & -- & 1299\\
$(65,14)$ & 10 & $(4,1)$ & 3 & 1 & YES & YES & YES & $1.11$ & $(2,2)$ & -- & 1300\\
$(65,24)$ & 9 & $(4,1)$ & 3 & 1 & YES & YES & YES & $1.11$ & $(2,2)$ & -- & 1301\\
$(65,14)$ & 10 & $(5,2)$ & 3 & 5 & YES & YES & YES & $1.11$ & $(2,2)$ & -- & 1302\\
$(65,14)$ & 10 & $(5,2)$ & 3 & 5 & YES & YES & YES & $1.11$ & $(2,2)$ & NO & 1303\\
$(65,14)$ & 10 & $(5,2)$ & 3 & 5 & YES & YES & YES & $1.11$ & $(2,2)$ & NO & 1304\\
$(65,14)$ & 10 & $(7,2)$ & 4 & 1 & YES & YES & YES & $1.11$ & $(2,2)$ & NO & 1305\\
$(65,24)$ & 9 & $(8,3)$ & 4 & 1 & YES & YES & YES & $1.11$ & $(2,2)$ & NO & 1306\\
$(65,14)$ & 10 & $(13,3)$ & 6 & 13 & YES & YES & YES & $1.00$ & $(2,2)$ & NO & 1307\\
$(65,24)$ & 9 & $(27,10)$ & 7 & 1 & YES & YES & YES & $1.00$ & $(2,2)$ & 1393 & 1308\\
$(65,24)$ & 9 & $(46,17)$ & 8 & 1 & YES & YES & YES & $1.11$ & $(2,2)$ & NO & 1309\\
$(66,29)$ & 9 & $(2,1)$ & 1 & 2 & YES & YES & YES & $1.22$ & $(2,2)$ & -- & 1310\\
$(66,25)$ & 9 & $(3,1)$ & 2 & 3 & YES & YES & YES & $1.11$ & $(2,2)$ & -- & 1311\\
$(66,25)$ & 9 & $(3,1)$ & 2 & 3 & YES & YES & YES & $1.00$ & $(2,2)$ & NO & 1312\\
$(66,29)$ & 9 & $(3,1)$ & 2 & 3 & YES & YES & YES & $1.22$ & $(2,2)$ & -- & 1313\\
$(66,29)$ & 9 & $(3,1)$ & 2 & 3 & YES & YES & YES & $1.11$ & $(2,2)$ & NO & 1314\\
$(66,29)$ & 9 & $(4,1)$ & 3 & 2 & YES & YES & YES & $1.11$ & $(2,2)$ & -- & 1315\\
$(66,29)$ & 9 & $(7,3)$ & 4 & 1 & YES & YES & YES & $1.33$ & $(2,2)$ & 1234 & 1316\\
$(66,29)$ & 9 & $(9,4)$ & 5 & 3 & YES & YES & YES & $1.22$ & $(2,2)$ & NO & 1317\\
$(66,25)$ & 9 & $(13,5)$ & 5 & 1 & YES & YES & YES & $1.11$ & $(2,2)$ & 783 & 1318\\
$(66,29)$ & 9 & $(16,7)$ & 6 & 2 & YES & YES & YES & $1.22$ & $(2,2)$ & 1165 & 1319\\
$(66,25)$ & 9 & $(21,8)$ & 6 & 3 & YES & YES & YES & $0.88$ & $(2,2)$ & NO & 1320\\
$(66,25)$ & 9 & $(37,14)$ & 8 & 1 & YES & YES & YES & $1.22$ & $(2,2)$ & NO & 1321\\
$(66,29)$ & 9 & $(41,18)$ & 8 & 1 & YES & YES & YES & $1.11$ & $(2,2)$ & NO & 1322\\
$(66,25)$ & 9 & $(66,25)$ & 9 & 66 & YES & YES & YES & $1.00$ & $(2,2)$ & NO & 1323\\
$(66,29)$ & 9 & $(66,29)$ & 9 & 66 & YES & YES & YES & $1.22$ & $(2,2)$ & NO & 1324\\
$(67,29)$ & 10 & $(2,1)$ & 1 & 1 & YES & YES & YES & $1.12$ & $(2,2)$ & NO & 1325\\
$(67,18)$ & 9 & $(4,1)$ & 3 & 1 & YES & YES & YES & $1.00$ & $(2,2)$ & 1125 & 1326\\
$(67,29)$ & 10 & $(5,1)$ & 4 & 1 & YES & YES & YES & $1.00$ & $(2,2)$ & NO & 1327\\
$(67,29)$ & 10 & $(6,1)$ & 5 & 1 & YES & YES & YES & $1.00$ & $(2,2)$ & -- & 1328\\
$(67,29)$ & 10 & $(7,3)$ & 4 & 1 & YES & YES & YES & $1.12$ & $(2,2)$ & 945 & 1329\\
$(67,18)$ & 9 & $(9,2)$ & 5 & 1 & YES & YES & YES & $1.22$ & $(2,2)$ & NO & 1330\\
$(67,18)$ & 9 & $(19,5)$ & 7 & 1 & YES & YES & YES & $1.22$ & $(2,2)$ & NO & 1331\\
$(67,18)$ & 9 & $(26,7)$ & 7 & 1 & YES & YES & YES & $1.00$ & $(2,2)$ & NO & 1332\\
$(67,29)$ & 10 & $(30,13)$ & 8 & 1 & YES & YES & YES & $1.00$ & $(2,2)$ & NO & 1333\\
$(68,25)$ & 9 & $(3,1)$ & 2 & 1 & YES & YES & YES & $1.12$ & $(2,2)$ & NO & 1334\\
$(69,20)$ & 10 & $(2,1)$ & 1 & 1 & YES & YES & YES & $0.88$ & $(4,1)$ & -- & 1335\\
$(69,29)$ & 9 & $(2,1)$ & 1 & 1 & YES & YES & YES & $1.12$ & $(2,2)$ & NO & 1336\\
$(69,29)$ & 9 & $(3,1)$ & 2 & 3 & YES & YES & YES & $1.12$ & $(2,2)$ & NO & 1337\\
$(69,13)$ & 11 & $(5,2)$ & 3 & 1 & YES & YES & YES & $1.11$ & $(2,2)$ & NO & 1338\\
$(69,13)$ & 11 & $(5,2)$ & 3 & 1 & YES & YES & YES & $1.22$ & $(2,2)$ & NO & 1339\\
$(69,19)$ & 9 & $(5,2)$ & 3 & 1 & YES & YES & YES & $1.11$ & $(2,2)$ & NO & 1340\\
$(69,19)$ & 9 & $(5,2)$ & 3 & 1 & YES & YES & YES & $1.11$ & $(2,2)$ & -- & 1341\\
$(69,29)$ & 9 & $(12,5)$ & 5 & 3 & YES & YES & YES & $1.12$ & $(2,2)$ & NO & 1342\\
$(69,13)$ & 11 & $(27,5)$ & 8 & 3 & YES & YES & YES & $0.88$ & $(2,2)$ & NO & 1343\\
$(69,20)$ & 10 & $(31,9)$ & 8 & 1 & YES & YES & YES & $1.00$ & $(2,2)$ & NO & 1344\\
$(70,29)$ & 9 & $(2,1)$ & 1 & 2 & YES & YES & YES & $0.88$ & $(4,1)$ & NO & 1345\\
$(70,29)$ & 9 & $(3,1)$ & 2 & 1 & YES & YES & YES & $1.11$ & $(2,2)$ & NO & 1346\\
$(70,29)$ & 9 & $(4,1)$ & 3 & 2 & YES & YES & YES & $1.00$ & $(2,2)$ & -- & 1347\\
$(70,29)$ & 9 & $(5,2)$ & 3 & 5 & YES & YES & YES & $1.11$ & $(2,2)$ & 876 & 1348\\
$(70,29)$ & 9 & $(29,12)$ & 7 & 1 & YES & YES & YES & $0.75$ & $(4,1)$ & NO & 1349\\
$(70,29)$ & 9 & $(70,29)$ & 9 & 70 & YES & YES & YES & $1.11$ & $(2,2)$ & NO & 1350\\
$(71,19)$ & 10 & $(2,1)$ & 1 & 1 & YES & YES & YES & $1.00$ & $(2,2)$ & NO & 1351\\
$(71,22)$ & 10 & $(2,1)$ & 1 & 1 & YES & YES & YES & $1.12$ & $(2,2)$ & -- & 1352\\
$(71,22)$ & 10 & $(2,1)$ & 1 & 1 & YES & YES & YES & $1.00$ & $(4,1)$ & NO & 1353\\
$(71,26)$ & 9 & $(2,1)$ & 1 & 1 & YES & YES & YES & $1.11$ & $(2,2)$ & -- & 1354\\
$(71,30)$ & 9 & $(2,1)$ & 1 & 1 & YES & YES & YES & $1.22$ & $(2,2)$ & NO & 1355\\
$(71,30)$ & 9 & $(2,1)$ & 1 & 1 & NO & YES & YES & $1.12$ & $(2,2)$ & -- & 1356\\
$(71,22)$ & 10 & $(3,1)$ & 2 & 1 & YES & YES & YES & $1.12$ & $(2,2)$ & -- & 1357\\
$(71,22)$ & 10 & $(3,1)$ & 2 & 1 & YES & YES & YES & $1.22$ & $(2,2)$ & NO & 1358\\
$(71,26)$ & 9 & $(3,1)$ & 2 & 1 & YES & YES & YES & $1.11$ & $(2,2)$ & -- & 1359\\
$(71,19)$ & 10 & $(4,1)$ & 3 & 1 & YES & YES & YES & $1.00$ & $(2,2)$ & NO & 1360\\
$(71,26)$ & 9 & $(4,1)$ & 3 & 1 & YES & YES & YES & $1.11$ & $(2,2)$ & -- & 1361\\
$(71,26)$ & 9 & $(4,1)$ & 3 & 1 & YES & YES & YES & $1.11$ & $(2,2)$ & NO & 1362\\
$(71,26)$ & 9 & $(5,1)$ & 4 & 1 & YES & YES & YES & $1.00$ & $(2,2)$ & NO & 1363\\
$(71,26)$ & 9 & $(5,2)$ & 3 & 1 & YES & YES & YES & $1.11$ & $(2,2)$ & NO & 1364\\
$(71,19)$ & 10 & $(7,1)$ & 6 & 1 & YES & YES & YES & $0.88$ & $(2,2)$ & NO & 1365\\
$(71,22)$ & 10 & $(7,2)$ & 4 & 1 & YES & YES & YES & $1.11$ & $(2,2)$ & NO & 1366\\
$(71,26)$ & 9 & $(8,3)$ & 4 & 1 & YES & YES & YES & $1.11$ & $(2,2)$ & 878 & 1367\\
$(71,26)$ & 9 & $(11,4)$ & 5 & 1 & YES & YES & YES & $1.11$ & $(2,2)$ & 1110 & 1368\\
$(71,19)$ & 10 & $(15,4)$ & 6 & 1 & YES & YES & YES & $1.00$ & $(2,2)$ & NO & 1369\\
$(71,22)$ & 10 & $(16,5)$ & 7 & 1 & YES & YES & YES & $1.12$ & $(2,2)$ & NO & 1370\\
$(71,31)$ & 10 & $(16,7)$ & 6 & 1 & YES & YES & YES & $1.12$ & $(2,2)$ & NO & 1371\\
$(71,26)$ & 9 & $(19,7)$ & 6 & 1 & YES & YES & YES & $1.11$ & $(2,2)$ & NO & 1372\\
$(71,30)$ & 9 & $(19,8)$ & 6 & 1 & YES & YES & YES & $1.11$ & $(2,2)$ & 1290 & 1373\\
$(71,20)$ & 10 & $(25,7)$ & 7 & 1 & YES & YES & YES & $0.88$ & $(4,1)$ & NO & 1374\\
$(71,26)$ & 9 & $(30,11)$ & 7 & 1 & YES & YES & YES & $1.11$ & $(2,2)$ & NO & 1375\\
$(71,19)$ & 10 & $(41,11)$ & 8 & 1 & YES & YES & YES & $0.88$ & $(2,2)$ & 1558 & 1376\\
$(71,26)$ & 9 & $(41,15)$ & 8 & 1 & YES & YES & YES & $1.11$ & $(2,2)$ & NO & 1377\\
$(71,26)$ & 9 & $(71,26)$ & 9 & 71 & YES & YES & YES & $1.11$ & $(2,2)$ & NO & 1378\\
$(71,31)$ & 10 & $(71,31)$ & 10 & 71 & YES & YES & YES & $1.00$ & $(2,2)$ & NO & 1379\\
$(72,17)$ & 11 & $(3,1)$ & 2 & 3 & YES & YES & YES & $0.88$ & $(2,2)$ & -- & 1380\\
$(72,19)$ & 10 & $(3,1)$ & 2 & 3 & YES & YES & YES & $1.33$ & $(2,2)$ & NO & 1381\\
$(72,19)$ & 10 & $(3,1)$ & 2 & 3 & YES & YES & YES & $1.33$ & $(2,2)$ & -- & 1382\\
$(72,19)$ & 10 & $(4,1)$ & 3 & 4 & YES & YES & YES & $1.11$ & $(2,2)$ & -- & 1383\\
$(72,17)$ & 11 & $(6,1)$ & 5 & 6 & YES & YES & YES & $1.00$ & $(2,2)$ & 602 & 1384\\
$(72,19)$ & 10 & $(6,1)$ & 5 & 6 & YES & YES & YES & $1.00$ & $(2,2)$ & NO & 1385\\
$(72,19)$ & 10 & $(7,2)$ & 4 & 1 & YES & YES & YES & $1.22$ & $(2,2)$ & NO & 1386\\
$(72,19)$ & 10 & $(11,3)$ & 5 & 1 & YES & YES & YES & $1.22$ & $(2,2)$ & NO & 1387\\
$(72,19)$ & 10 & $(19,5)$ & 7 & 1 & YES & YES & YES & $1.12$ & $(2,2)$ & NO & 1388\\
$(72,19)$ & 10 & $(34,9)$ & 8 & 2 & YES & YES & YES & $0.88$ & $(4,1)$ & 1507 & 1389\\
$(73,27)$ & 9 & $(2,1)$ & 1 & 1 & YES & YES & YES & $1.12$ & $(2,2)$ & NO & 1390\\
$(73,31)$ & 10 & $(2,1)$ & 1 & 1 & NO & YES & YES & $1.22$ & $(2,2)$ & -- & 1391\\
$(73,27)$ & 9 & $(3,1)$ & 2 & 1 & YES & YES & YES & $1.11$ & $(2,2)$ & -- & 1392\\
$(73,27)$ & 9 & $(19,7)$ & 6 & 1 & YES & YES & YES & $1.00$ & $(2,2)$ & 1308 & 1393\\
$(73,27)$ & 9 & $(73,27)$ & 9 & 73 & YES & YES & YES & $1.11$ & $(2,2)$ & NO & 1394\\
$(74,23)$ & 10 & $(2,1)$ & 1 & 2 & YES & YES & YES & $1.12$ & $(2,2)$ & -- & 1395\\
$(74,23)$ & 10 & $(2,1)$ & 1 & 2 & YES & YES & YES & $1.00$ & $(4,1)$ & NO & 1396\\
$(74,29)$ & 10 & $(2,1)$ & 1 & 2 & NO & YES & YES & $1.00$ & $(4,1)$ & -- & 1397\\
$(74,31)$ & 9 & $(2,1)$ & 1 & 2 & YES & YES & YES & $1.00$ & $(2,2)$ & -- & 1398\\
$(74,31)$ & 9 & $(2,1)$ & 1 & 2 & YES & YES & YES & $1.12$ & $(2,2)$ & NO & 1399\\
$(74,17)$ & 11 & $(3,1)$ & 2 & 1 & YES & YES & YES & $1.33$ & $(2,2)$ & NO & 1400\\
$(74,17)$ & 11 & $(3,1)$ & 2 & 1 & YES & YES & YES & $1.33$ & $(2,2)$ & -- & 1401\\
$(74,17)$ & 11 & $(4,1)$ & 3 & 2 & YES & YES & YES & $1.12$ & $(2,2)$ & NO & 1402\\
$(74,23)$ & 10 & $(4,1)$ & 3 & 2 & YES & YES & YES & $1.12$ & $(2,2)$ & NO & 1403\\
$(74,31)$ & 9 & $(4,1)$ & 3 & 2 & YES & YES & YES & $1.00$ & $(2,2)$ & -- & 1404\\
$(74,31)$ & 9 & $(5,2)$ & 3 & 1 & YES & YES & YES & $1.22$ & $(2,2)$ & NO & 1405\\
$(74,17)$ & 11 & $(6,1)$ & 5 & 2 & YES & YES & YES & $1.22$ & $(2,2)$ & NO & 1406\\
$(74,31)$ & 9 & $(7,3)$ & 4 & 1 & YES & YES & YES & $1.11$ & $(2,2)$ & 910 & 1407\\
$(74,23)$ & 10 & $(13,4)$ & 6 & 1 & YES & YES & YES & $1.12$ & $(2,2)$ & NO & 1408\\
$(74,17)$ & 11 & $(22,5)$ & 7 & 2 & YES & YES & YES & $1.22$ & $(2,2)$ & NO & 1409\\
$(74,31)$ & 9 & $(31,13)$ & 7 & 1 & YES & YES & YES & $1.00$ & $(2,2)$ & NO & 1410\\
$(75,17)$ & 10 & $(2,1)$ & 1 & 1 & YES & YES & NO(3) & $0.75$ & $(2,2)$ & NO & 1411\\
$(75,17)$ & 10 & $(2,1)$ & 1 & 1 & YES & YES & NO(3) & $0.75$ & $(2,2)$ & -- & 1412\\
$(75,29)$ & 9 & $(2,1)$ & 1 & 1 & YES & YES & YES & $1.11$ & $(2,2)$ & -- & 1413\\
$(75,29)$ & 9 & $(2,1)$ & 1 & 1 & YES & YES & YES & $1.11$ & $(2,2)$ & NO & 1414\\
$(75,31)$ & 9 & $(2,1)$ & 1 & 1 & YES & YES & YES & $1.22$ & $(2,2)$ & 986 & 1415\\
$(75,17)$ & 10 & $(3,1)$ & 2 & 3 & YES & YES & NO(3) & $0.75$ & $(2,2)$ & NO & 1416\\
$(75,31)$ & 9 & $(3,1)$ & 2 & 3 & YES & YES & YES & $1.00$ & $(2,2)$ & -- & 1417\\
$(75,31)$ & 9 & $(3,1)$ & 2 & 3 & YES & YES & YES & $1.00$ & $(2,2)$ & NO & 1418\\
$(75,29)$ & 9 & $(4,1)$ & 3 & 1 & YES & YES & YES & $1.00$ & $(2,2)$ & -- & 1419\\
$(75,29)$ & 9 & $(4,1)$ & 3 & 1 & YES & YES & YES & $1.11$ & $(2,2)$ & NO & 1420\\
$(75,29)$ & 9 & $(5,2)$ & 3 & 5 & YES & YES & YES & $1.11$ & $(2,2)$ & 942 & 1421\\
$(75,17)$ & 10 & $(13,3)$ & 6 & 1 & YES & YES & NO(3) & $0.75$ & $(2,2)$ & NO & 1422\\
$(75,17)$ & 10 & $(14,3)$ & 6 & 1 & YES & YES & YES & $1.00$ & $(2,2)$ & NO & 1423\\
$(75,29)$ & 9 & $(31,12)$ & 7 & 1 & YES & YES & YES & $1.00$ & $(2,2)$ & NO & 1424\\
$(75,31)$ & 9 & $(46,19)$ & 8 & 1 & YES & YES & YES & $1.00$ & $(2,2)$ & NO & 1425\\
$(76,27)$ & 10 & $(2,1)$ & 1 & 2 & YES & YES & YES & $1.00$ & $(2,2)$ & NO & 1426\\
$(76,29)$ & 9 & $(2,1)$ & 1 & 2 & YES & YES & YES & $1.12$ & $(2,2)$ & NO & 1427\\
$(77,18)$ & 10 & $(7,2)$ & 4 & 7 & YES & YES & YES & $1.11$ & $(2,2)$ & NO & 1428\\
$(78,17)$ & 10 & $(5,2)$ & 3 & 1 & YES & YES & YES & $1.00$ & $(2,2)$ & -- & 1429\\
$(78,17)$ & 10 & $(13,3)$ & 6 & 13 & YES & YES & YES & $1.11$ & $(2,2)$ & NO & 1430\\
$(79,14)$ & 11 & $(2,1)$ & 1 & 1 & YES & YES & YES & $1.00$ & $(2,2)$ & NO & 1431\\
$(79,17)$ & 11 & $(2,1)$ & 1 & 1 & YES & YES & YES & $0.88$ & $(4,1)$ & NO & 1432\\
$(79,24)$ & 10 & $(2,1)$ & 1 & 1 & YES & YES & YES & $1.00$ & $(2,2)$ & NO & 1433\\
$(79,24)$ & 10 & $(2,1)$ & 1 & 1 & YES & YES & YES & $1.00$ & $(2,2)$ & -- & 1434\\
$(79,29)$ & 9 & $(2,1)$ & 1 & 1 & YES & YES & YES & $1.00$ & $(2,2)$ & -- & 1435\\
$(79,29)$ & 9 & $(2,1)$ & 1 & 1 & YES & YES & YES & $1.00$ & $(2,2)$ & NO & 1436\\
$(79,30)$ & 9 & $(2,1)$ & 1 & 1 & YES & YES & YES & $1.22$ & $(2,2)$ & NO & 1437\\
$(79,30)$ & 9 & $(2,1)$ & 1 & 1 & NO & YES & YES & $1.00$ & $(4,1)$ & -- & 1438\\
$(79,17)$ & 11 & $(3,1)$ & 2 & 1 & YES & YES & YES & $1.11$ & $(2,2)$ & -- & 1439\\
$(79,17)$ & 11 & $(3,1)$ & 2 & 1 & YES & YES & YES & $1.22$ & $(2,2)$ & NO & 1440\\
$(79,17)$ & 11 & $(3,1)$ & 2 & 1 & YES & YES & YES & $0.88$ & $(4,1)$ & NO & 1441\\
$(79,24)$ & 10 & $(3,1)$ & 2 & 1 & YES & YES & YES & $1.00$ & $(2,2)$ & -- & 1442\\
$(79,29)$ & 9 & $(3,1)$ & 2 & 1 & YES & YES & YES & $1.22$ & $(2,2)$ & NO & 1443\\
$(79,29)$ & 9 & $(3,1)$ & 2 & 1 & YES & YES & YES & $1.22$ & $(2,2)$ & NO & 1444\\
$(79,29)$ & 9 & $(3,1)$ & 2 & 1 & YES & YES & YES & $1.22$ & $(2,2)$ & -- & 1445\\
$(79,30)$ & 9 & $(3,1)$ & 2 & 1 & YES & YES & YES & $1.11$ & $(2,2)$ & -- & 1446\\
$(79,17)$ & 11 & $(4,1)$ & 3 & 1 & YES & YES & YES & $0.88$ & $(4,1)$ & NO & 1447\\
$(79,24)$ & 10 & $(4,1)$ & 3 & 1 & YES & YES & YES & $1.00$ & $(2,2)$ & NO & 1448\\
$(79,17)$ & 11 & $(5,1)$ & 4 & 1 & YES & YES & YES & $1.00$ & $(2,2)$ & NO & 1449\\
$(79,29)$ & 9 & $(5,2)$ & 3 & 1 & YES & YES & YES & $1.12$ & $(2,2)$ & 1522 & 1450\\
$(79,17)$ & 11 & $(6,1)$ & 5 & 1 & YES & YES & YES & $1.11$ & $(2,2)$ & NO & 1451\\
$(79,23)$ & 10 & $(7,2)$ & 4 & 1 & YES & YES & YES & $1.12$ & $(2,2)$ & NO & 1452\\
$(79,24)$ & 10 & $(7,2)$ & 4 & 1 & YES & YES & YES & $1.00$ & $(2,2)$ & 1215 & 1453\\
$(79,30)$ & 9 & $(8,3)$ & 4 & 1 & YES & YES & YES & $1.22$ & $(2,2)$ & NO & 1454\\
$(79,17)$ & 11 & $(9,2)$ & 5 & 1 & YES & YES & YES & $0.88$ & $(4,1)$ & NO & 1455\\
$(79,24)$ & 10 & $(13,4)$ & 6 & 1 & YES & YES & YES & $1.11$ & $(2,2)$ & NO & 1456\\
$(79,14)$ & 11 & $(23,4)$ & 8 & 1 & YES & YES & YES & $1.00$ & $(2,2)$ & NO & 1457\\
$(79,17)$ & 11 & $(23,5)$ & 7 & 1 & YES & YES & YES & $1.11$ & $(2,2)$ & NO & 1458\\
$(79,29)$ & 9 & $(30,11)$ & 7 & 1 & YES & YES & YES & $1.12$ & $(2,2)$ & NO & 1459\\
$(79,17)$ & 11 & $(65,14)$ & 10 & 1 & YES & YES & YES & $1.11$ & $(2,2)$ & NO & 1460\\
$(79,29)$ & 9 & $(79,29)$ & 9 & 79 & YES & YES & YES & $1.11$ & $(2,2)$ & NO & 1461\\
$(80,31)$ & 9 & $(2,1)$ & 1 & 2 & YES & YES & YES & $1.11$ & $(2,2)$ & NO & 1462\\
$(80,19)$ & 11 & $(3,1)$ & 2 & 1 & YES & YES & YES & $1.33$ & $(2,2)$ & NO & 1463\\
$(80,19)$ & 11 & $(3,1)$ & 2 & 1 & YES & YES & YES & $1.33$ & $(2,2)$ & -- & 1464\\
$(80,31)$ & 9 & $(3,1)$ & 2 & 1 & YES & YES & YES & $1.11$ & $(2,2)$ & -- & 1465\\
$(80,31)$ & 9 & $(5,2)$ & 3 & 5 & YES & YES & YES & $1.11$ & $(2,2)$ & 1287 & 1466\\
$(80,19)$ & 11 & $(9,2)$ & 5 & 1 & YES & YES & YES & $1.00$ & $(2,2)$ & NO & 1467\\
$(80,19)$ & 11 & $(13,3)$ & 6 & 1 & YES & YES & YES & $1.00$ & $(2,2)$ & NO & 1468\\
$(80,31)$ & 9 & $(13,5)$ & 5 & 1 & YES & YES & YES & $1.00$ & $(2,2)$ & NO & 1469\\
$(80,19)$ & 11 & $(21,5)$ & 8 & 1 & YES & YES & YES & $1.00$ & $(2,2)$ & NO & 1470\\
$(80,31)$ & 9 & $(31,12)$ & 7 & 1 & YES & YES & YES & $1.11$ & $(2,2)$ & NO & 1471\\
$(81,19)$ & 11 & $(5,1)$ & 4 & 1 & YES & YES & YES & $1.00$ & $(2,2)$ & NO & 1472\\
$(81,19)$ & 11 & $(13,3)$ & 6 & 1 & YES & YES & YES & $1.00$ & $(2,2)$ & NO & 1473\\
$(82,25)$ & 10 & $(2,1)$ & 1 & 2 & YES & YES & YES & $1.22$ & $(2,2)$ & NO & 1474\\
$(82,19)$ & 12 & $(4,1)$ & 3 & 2 & YES & YES & YES & $1.12$ & $(2,2)$ & NO & 1475\\
$(82,25)$ & 10 & $(4,1)$ & 3 & 2 & YES & YES & YES & $1.11$ & $(2,2)$ & NO & 1476\\
$(82,23)$ & 10 & $(7,2)$ & 4 & 1 & YES & YES & YES & $1.12$ & $(2,2)$ & NO & 1477\\
$(82,25)$ & 10 & $(13,4)$ & 6 & 1 & YES & YES & YES & $1.22$ & $(2,2)$ & NO & 1478\\
$(82,19)$ & 12 & $(43,10)$ & 9 & 1 & YES & YES & YES & $1.00$ & $(2,2)$ & NO & 1479\\
$(82,25)$ & 10 & $(59,18)$ & 9 & 1 & YES & YES & YES & $1.11$ & $(2,2)$ & NO & 1480\\
$(83,22)$ & 10 & $(2,1)$ & 1 & 1 & YES & YES & YES & $0.88$ & $(4,1)$ & NO & 1481\\
$(83,22)$ & 10 & $(3,1)$ & 2 & 1 & YES & YES & YES & $1.00$ & $(2,2)$ & -- & 1482\\
$(83,22)$ & 10 & $(3,1)$ & 2 & 1 & YES & YES & YES & $0.88$ & $(4,1)$ & NO & 1483\\
$(83,22)$ & 10 & $(3,1)$ & 2 & 1 & YES & YES & YES & $1.22$ & $(2,2)$ & NO & 1484\\
$(83,18)$ & 10 & $(4,1)$ & 3 & 1 & YES & YES & NO(3) & $0.75$ & $(2,2)$ & NO & 1485\\
$(83,22)$ & 10 & $(4,1)$ & 3 & 1 & YES & YES & YES & $1.00$ & $(2,2)$ & -- & 1486\\
$(83,22)$ & 10 & $(5,1)$ & 4 & 1 & YES & YES & YES & $0.88$ & $(2,2)$ & -- & 1487\\
$(83,22)$ & 10 & $(7,2)$ & 4 & 1 & YES & YES & YES & $1.00$ & $(2,2)$ & NO & 1488\\
$(83,22)$ & 10 & $(11,3)$ & 5 & 1 & YES & YES & YES & $1.00$ & $(2,2)$ & NO & 1489\\
$(83,22)$ & 10 & $(15,4)$ & 6 & 1 & YES & YES & YES & $1.00$ & $(2,2)$ & 1291 & 1490\\
$(83,23)$ & 10 & $(18,5)$ & 6 & 1 & YES & YES & YES & $1.12$ & $(2,2)$ & NO & 1491\\
$(83,18)$ & 10 & $(19,4)$ & 7 & 1 & YES & YES & YES & $1.00$ & $(2,2)$ & 1578 & 1492\\
$(83,22)$ & 10 & $(19,5)$ & 7 & 1 & YES & YES & YES & $0.88$ & $(4,1)$ & NO & 1493\\
$(83,22)$ & 10 & $(34,9)$ & 8 & 1 & YES & YES & YES & $0.88$ & $(4,1)$ & NO & 1494\\
$(83,22)$ & 10 & $(49,13)$ & 9 & 1 & YES & YES & YES & $1.11$ & $(2,2)$ & NO & 1495\\
$(83,22)$ & 10 & $(83,22)$ & 10 & 83 & YES & YES & YES & $1.11$ & $(2,2)$ & NO & 1496\\
$(84,37)$ & 10 & $(2,1)$ & 1 & 2 & NO & YES & YES & $1.00$ & $(4,1)$ & -- & 1497\\
$(85,26)$ & 10 & $(2,1)$ & 1 & 1 & YES & YES & YES & $1.22$ & $(2,2)$ & -- & 1498\\
$(85,23)$ & 10 & $(4,1)$ & 3 & 1 & YES & YES & YES & $0.88$ & $(2,2)$ & NO & 1499\\
$(85,16)$ & 12 & $(7,1)$ & 6 & 1 & YES & YES & YES & $0.88$ & $(2,2)$ & NO & 1500\\
$(85,26)$ & 10 & $(10,3)$ & 5 & 5 & YES & YES & YES & $1.11$ & $(2,2)$ & 1060 & 1501\\
$(85,23)$ & 10 & $(11,3)$ & 5 & 1 & YES & YES & YES & $0.88$ & $(2,2)$ & 1205 & 1502\\
$(85,23)$ & 10 & $(37,10)$ & 8 & 1 & YES & YES & YES & $0.88$ & $(2,2)$ & NO & 1503\\
$(86,25)$ & 10 & $(7,2)$ & 4 & 1 & YES & YES & YES & $1.12$ & $(2,2)$ & NO & 1504\\
$(87,23)$ & 10 & $(2,1)$ & 1 & 1 & YES & YES & YES & $1.00$ & $(2,2)$ & -- & 1505\\
$(87,23)$ & 10 & $(2,1)$ & 1 & 1 & YES & YES & YES & $1.12$ & $(2,2)$ & NO & 1506\\
$(87,23)$ & 10 & $(19,5)$ & 7 & 1 & YES & YES & YES & $0.88$ & $(4,1)$ & 1389 & 1507\\
$(87,23)$ & 10 & $(34,9)$ & 8 & 1 & YES & YES & YES & $1.00$ & $(2,2)$ & NO & 1508\\
$(88,21)$ & 12 & $(21,5)$ & 8 & 1 & YES & YES & YES & $1.12$ & $(2,2)$ & NO & 1509\\
$(88,21)$ & 12 & $(46,11)$ & 10 & 2 & YES & YES & YES & $1.00$ & $(2,2)$ & 1580 & 1510\\
$(89,20)$ & 11 & $(2,1)$ & 1 & 1 & YES & YES & YES & $1.00$ & $(2,2)$ & -- & 1511\\
$(89,20)$ & 11 & $(2,1)$ & 1 & 1 & YES & YES & YES & $1.12$ & $(2,2)$ & NO & 1512\\
$(89,26)$ & 10 & $(2,1)$ & 1 & 1 & YES & YES & YES & $1.11$ & $(2,2)$ & -- & 1513\\
$(89,27)$ & 10 & $(2,1)$ & 1 & 1 & YES & YES & YES & $1.00$ & $(2,2)$ & -- & 1514\\
$(89,27)$ & 10 & $(2,1)$ & 1 & 1 & YES & YES & YES & $0.88$ & $(4,1)$ & NO & 1515\\
$(89,34)$ & 9 & $(2,1)$ & 1 & 1 & NO & YES & YES & $1.00$ & $(4,1)$ & -- & 1516\\
$(89,17)$ & 12 & $(3,1)$ & 2 & 1 & YES & YES & YES & $0.88$ & $(2,2)$ & NO & 1517\\
$(89,26)$ & 10 & $(3,1)$ & 2 & 1 & YES & YES & YES & $1.11$ & $(2,2)$ & -- & 1518\\
$(89,27)$ & 10 & $(3,1)$ & 2 & 1 & YES & YES & YES & $1.11$ & $(2,2)$ & NO & 1519\\
$(89,27)$ & 10 & $(3,1)$ & 2 & 1 & YES & YES & YES & $1.22$ & $(2,2)$ & NO & 1520\\
$(89,27)$ & 10 & $(3,1)$ & 2 & 1 & YES & YES & YES & $1.22$ & $(2,2)$ & -- & 1521\\
$(89,34)$ & 9 & $(3,1)$ & 2 & 1 & YES & YES & YES & $1.12$ & $(2,2)$ & 1450 & 1522\\
$(89,19)$ & 12 & $(5,1)$ & 4 & 1 & YES & YES & YES & $1.00$ & $(2,2)$ & NO & 1523\\
$(89,25)$ & 10 & $(7,2)$ & 4 & 1 & YES & YES & YES & $1.12$ & $(2,2)$ & NO & 1524\\
$(89,17)$ & 12 & $(11,2)$ & 6 & 1 & YES & YES & YES & $0.88$ & $(2,2)$ & NO & 1525\\
$(89,27)$ & 10 & $(89,27)$ & 10 & 89 & YES & YES & YES & $1.11$ & $(2,2)$ & NO & 1526\\
$(90,19)$ & 11 & $(2,1)$ & 1 & 2 & YES & YES & YES & $1.11$ & $(2,2)$ & NO & 1527\\
$(90,19)$ & 11 & $(2,1)$ & 1 & 2 & YES & YES & YES & $1.11$ & $(2,2)$ & -- & 1528\\
$(90,19)$ & 11 & $(4,1)$ & 3 & 2 & YES & YES & YES & $1.11$ & $(2,2)$ & NO & 1529\\
$(90,19)$ & 11 & $(4,1)$ & 3 & 2 & YES & YES & YES & $1.11$ & $(2,2)$ & -- & 1530\\
$(90,19)$ & 11 & $(14,3)$ & 6 & 2 & YES & YES & YES & $1.11$ & $(2,2)$ & NO & 1531\\
$(91,40)$ & 10 & $(2,1)$ & 1 & 1 & NO & YES & YES & $1.12$ & $(2,2)$ & -- & 1532\\
$(91,17)$ & 12 & $(3,1)$ & 2 & 1 & YES & YES & YES & $1.22$ & $(2,2)$ & NO & 1533\\
$(91,27)$ & 10 & $(3,1)$ & 2 & 1 & YES & YES & YES & $1.11$ & $(2,2)$ & 1056 & 1534\\
$(91,17)$ & 12 & $(5,1)$ & 4 & 1 & YES & YES & YES & $0.88$ & $(2,2)$ & NO & 1535\\
$(91,17)$ & 12 & $(6,1)$ & 5 & 1 & YES & YES & YES & $0.75$ & $(4,1)$ & NO & 1536\\
$(91,17)$ & 12 & $(7,1)$ & 6 & 7 & YES & YES & YES & $0.75$ & $(4,1)$ & NO & 1537\\
$(91,17)$ & 12 & $(11,2)$ & 6 & 1 & YES & YES & YES & $0.75$ & $(4,1)$ & NO & 1538\\
$(91,17)$ & 12 & $(27,5)$ & 8 & 1 & YES & YES & YES & $0.75$ & $(4,1)$ & NO & 1539\\
$(91,27)$ & 10 & $(64,19)$ & 9 & 1 & YES & YES & YES & $1.00$ & $(2,2)$ & NO & 1540\\
$(93,41)$ & 10 & $(2,1)$ & 1 & 1 & NO & YES & YES & $1.33$ & $(2,2)$ & -- & 1541\\
$(93,26)$ & 10 & $(7,2)$ & 4 & 1 & YES & YES & YES & $1.11$ & $(2,2)$ & NO & 1542\\
$(93,26)$ & 10 & $(25,7)$ & 7 & 1 & YES & YES & YES & $1.11$ & $(2,2)$ & NO & 1543\\
$(96,17)$ & 12 & $(3,1)$ & 2 & 3 & YES & YES & YES & $1.22$ & $(2,2)$ & NO & 1544\\
$(96,17)$ & 12 & $(3,1)$ & 2 & 3 & YES & YES & YES & $1.22$ & $(2,2)$ & -- & 1545\\
$(96,17)$ & 12 & $(3,1)$ & 2 & 3 & YES & YES & YES & $1.33$ & $(2,2)$ & NO & 1546\\
$(96,17)$ & 12 & $(4,1)$ & 3 & 4 & YES & YES & YES & $1.11$ & $(2,2)$ & NO & 1547\\
$(96,17)$ & 12 & $(6,1)$ & 5 & 6 & YES & YES & YES & $1.00$ & $(2,2)$ & NO & 1548\\
$(96,17)$ & 12 & $(79,14)$ & 11 & 1 & YES & YES & YES & $1.00$ & $(2,2)$ & NO & 1549\\
$(97,26)$ & 10 & $(2,1)$ & 1 & 1 & YES & YES & YES & $0.75$ & $(4,1)$ & NO & 1550\\
$(97,23)$ & 11 & $(3,1)$ & 2 & 1 & YES & YES & YES & $1.00$ & $(2,2)$ & -- & 1551\\
$(97,23)$ & 11 & $(3,1)$ & 2 & 1 & YES & YES & YES & $1.22$ & $(2,2)$ & NO & 1552\\
$(97,26)$ & 10 & $(3,1)$ & 2 & 1 & YES & YES & YES & $1.11$ & $(2,2)$ & -- & 1553\\
$(97,26)$ & 10 & $(3,1)$ & 2 & 1 & YES & YES & YES & $0.75$ & $(4,1)$ & NO & 1554\\
$(97,26)$ & 10 & $(3,1)$ & 2 & 1 & YES & YES & YES & $1.11$ & $(2,2)$ & NO & 1555\\
$(97,26)$ & 10 & $(7,2)$ & 4 & 1 & YES & YES & YES & $0.75$ & $(4,1)$ & NO & 1556\\
$(97,23)$ & 11 & $(13,3)$ & 6 & 1 & YES & YES & YES & $1.00$ & $(2,2)$ & NO & 1557\\
$(97,26)$ & 10 & $(15,4)$ & 6 & 1 & YES & YES & YES & $0.88$ & $(2,2)$ & 1376 & 1558\\
$(97,26)$ & 10 & $(56,15)$ & 9 & 1 & YES & YES & YES & $1.11$ & $(2,2)$ & NO & 1559\\
$(98,43)$ & 10 & $(2,1)$ & 1 & 2 & NO & YES & YES & $1.22$ & $(2,2)$ & -- & 1560\\
$(99,29)$ & 10 & $(2,1)$ & 1 & 1 & YES & YES & YES & $1.00$ & $(2,2)$ & NO & 1561\\
$(101,44)$ & 10 & $(2,1)$ & 1 & 1 & NO & YES & YES & $1.22$ & $(2,2)$ & -- & 1562\\
$(101,23)$ & 11 & $(22,5)$ & 7 & 1 & YES & YES & YES & $1.22$ & $(2,2)$ & NO & 1563\\
$(105,23)$ & 11 & $(14,3)$ & 6 & 7 & YES & YES & YES & $1.11$ & $(2,2)$ & NO & 1564\\
$(105,23)$ & 11 & $(23,5)$ & 7 & 1 & YES & YES & YES & $1.11$ & $(2,2)$ & NO & 1565\\
$(106,41)$ & 10 & $(2,1)$ & 1 & 2 & NO & YES & YES & $1.00$ & $(2,2)$ & -- & 1566\\
$(106,23)$ & 11 & $(23,5)$ & 7 & 1 & YES & YES & YES & $1.11$ & $(2,2)$ & NO & 1567\\
$(109,46)$ & 10 & $(2,1)$ & 1 & 1 & NO & YES & YES & $1.22$ & $(2,2)$ & -- & 1568\\
$(111,26)$ & 11 & $(2,1)$ & 1 & 1 & YES & YES & YES & $1.11$ & $(2,2)$ & NO & 1569\\
$(111,26)$ & 11 & $(2,1)$ & 1 & 1 & YES & YES & YES & $1.00$ & $(2,2)$ & -- & 1570\\
$(111,43)$ & 10 & $(2,1)$ & 1 & 1 & NO & YES & YES & $1.12$ & $(2,2)$ & -- & 1571\\
$(111,26)$ & 11 & $(13,3)$ & 6 & 1 & YES & YES & YES & $1.11$ & $(2,2)$ & 1293 & 1572\\
$(113,24)$ & 11 & $(2,1)$ & 1 & 1 & YES & YES & YES & $1.00$ & $(2,2)$ & NO & 1573\\
$(113,24)$ & 11 & $(2,1)$ & 1 & 1 & YES & YES & YES & $1.11$ & $(2,2)$ & -- & 1574\\
$(113,24)$ & 11 & $(3,1)$ & 2 & 1 & YES & YES & YES & $1.00$ & $(2,2)$ & NO & 1575\\
$(113,24)$ & 11 & $(3,1)$ & 2 & 1 & YES & YES & YES & $1.00$ & $(2,2)$ & -- & 1576\\
$(113,24)$ & 11 & $(4,1)$ & 3 & 1 & YES & YES & YES & $1.00$ & $(2,2)$ & NO & 1577\\
$(113,24)$ & 11 & $(9,2)$ & 5 & 1 & YES & YES & YES & $1.00$ & $(2,2)$ & 1492 & 1578\\
$(113,24)$ & 11 & $(19,4)$ & 7 & 1 & YES & YES & YES & $1.11$ & $(2,2)$ & NO & 1579\\
$(113,27)$ & 12 & $(21,5)$ & 8 & 1 & YES & YES & YES & $1.00$ & $(2,2)$ & 1510 & 1580\\
$(115,26)$ & 11 & $(6,1)$ & 5 & 1 & YES & YES & YES & $1.00$ & $(2,2)$ & NO & 1581\\
$(115,26)$ & 11 & $(9,2)$ & 5 & 1 & YES & YES & YES & $1.00$ & $(2,2)$ & NO & 1582\\
$(115,26)$ & 11 & $(31,7)$ & 8 & 1 & YES & YES & YES & $1.00$ & $(2,2)$ & NO & 1583\\
$(120,19)$ & 14 & $(6,1)$ & 5 & 6 & YES & YES & YES & $1.00$ & $(2,2)$ & NO & 1584\\
$(123,22)$ & 12 & $(3,1)$ & 2 & 3 & YES & YES & YES & $1.00$ & $(2,2)$ & NO & 1585\\
$(123,22)$ & 12 & $(4,1)$ & 3 & 1 & YES & YES & YES & $1.00$ & $(2,2)$ & NO & 1586\\
$(123,22)$ & 12 & $(6,1)$ & 5 & 3 & YES & YES & YES & $1.00$ & $(2,2)$ & NO & 1587\\
$(124,23)$ & 12 & $(27,5)$ & 8 & 1 & YES & YES & YES & $1.00$ & $(2,2)$ & NO & 1588\\
$(a;0,0,0;3)$ & 4 & $(19,8)$ & 6 & 1 & YES & YES & YES & $1.12$ & $(2,2)$ & -- & 1589\\
$(a;1,0,0;13)$ & 5 & $(10,3)$ & 5 & 1 & YES & YES & YES & $0.88$ & $(4,1)$ & -- & 1590\\
$(a;1,0,0;13)$ & 5 & $(13,5)$ & 5 & 13 & YES & YES & YES & $1.00$ & $(2,2)$ & -- & 1591\\
$(a;1,0,0;13)$ & 5 & $(16,5)$ & 7 & 1 & YES & YES & YES & $1.22$ & $(2,2)$ & -- & 1592\\
$(a;1,0,0;13)$ & 5 & $(17,5)$ & 6 & 1 & YES & YES & YES & $0.88$ & $(4,1)$ & -- & 1593\\
$(a;1,1,0;19)$ & 6 & $(7,2)$ & 4 & 1 & YES & YES & YES & $1.00$ & $(2,2)$ & -- & 1594\\
$(a;2,0,0;17)$ & 6 & $(9,4)$ & 5 & 1 & YES & YES & YES & $1.12$ & $(2,2)$ & -- & 1595\\
$(a;2,0,0;17)$ & 6 & $(11,5)$ & 6 & 1 & YES & YES & YES & $1.12$ & $(2,2)$ & -- & 1596\\
$(a;2,0,1;25)$ & 7 & $(5,2)$ & 3 & 5 & YES & YES & YES & $1.22$ & $(2,2)$ & -- & 1597\\
$(a;2,0,1;25)$ & 7 & $(13,3)$ & 6 & 1 & YES & YES & YES & $1.11$ & $(2,2)$ & -- & 1598\\
$(a;2,1,0;5)$ & 7 & $(5,2)$ & 3 & 5 & YES & YES & YES & $1.12$ & $(2,2)$ & -- & 1599\\
$(a;2,1,0;5)$ & 7 & $(7,2)$ & 4 & 1 & YES & YES & YES & $1.11$ & $(2,2)$ & -- & 1600\\
$(a;2,1,1;37)$ & 8 & $(2,1)$ & 1 & 1 & YES & YES & YES & $0.88$ & $(2,2)$ & -- & 1601\\
$(a;2,1,1;37)$ & 8 & $(3,1)$ & 2 & 1 & YES & YES & YES & $1.12$ & $(2,2)$ & -- & 1602\\
$(a;2,2,0;33)$ & 8 & $(3,1)$ & 2 & 3 & YES & YES & YES & $1.00$ & $(2,2)$ & -- & 1603\\
$(a;3,0,1;31)$ & 8 & $(3,1)$ & 2 & 1 & YES & YES & YES & $1.00$ & $(2,2)$ & -- & 1604\\
$(a;3,0,1;31)$ & 8 & $(4,1)$ & 3 & 1 & YES & YES & YES & $1.00$ & $(2,2)$ & -- & 1605\\
$(a;4,0,2;49)$ & 10 & $(7,1)$ & 6 & 7 & YES & YES & YES & $1.12$ & $(2,2)$ & -- & 1606\\
$(b;0,0,0;14)$ & 5 & $(8,3)$ & 4 & 2 & YES & YES & YES & $0.75$ & $(4,1)$ & -- & 1607\\
$(b;0,0,0;14)$ & 5 & $(10,3)$ & 5 & 2 & YES & YES & YES & $1.00$ & $(2,2)$ & -- & 1608\\
$(b;0,0,0;14)$ & 5 & $(12,5)$ & 5 & 2 & YES & YES & YES & $1.12$ & $(2,2)$ & -- & 1609\\
$(b;0,0,0;14)$ & 5 & $(19,8)$ & 6 & 1 & YES & YES & YES & $1.00$ & $(2,2)$ & -- & 1610\\
$(b;0,0,1;4)$ & 6 & $(3,1)$ & 2 & 1 & YES & YES & YES & $1.00$ & $(2,2)$ & -- & 1611\\
$(b;0,0,1;4)$ & 6 & $(5,2)$ & 3 & 1 & YES & YES & YES & $1.00$ & $(2,2)$ & -- & 1612\\
$(b;0,0,2;26)$ & 7 & $(2,1)$ & 1 & 2 & YES & YES & YES & $0.89$ & $(2,2)$ & -- & 1613\\
$(b;0,0,2;26)$ & 7 & $(4,1)$ & 3 & 2 & YES & YES & YES & $0.89$ & $(2,2)$ & -- & 1614\\
$(b;0,0,2;26)$ & 7 & $(5,2)$ & 3 & 1 & YES & YES & YES & $1.11$ & $(2,2)$ & -- & 1615\\
$(b;0,0,2;26)$ & 7 & $(7,2)$ & 4 & 1 & YES & YES & YES & $1.11$ & $(2,2)$ & -- & 1616\\
$(b;0,0,2;26)$ & 7 & $(13,3)$ & 6 & 13 & YES & YES & YES & $1.11$ & $(2,2)$ & -- & 1617\\
$(b;0,0,3;32)$ & 8 & $(2,1)$ & 1 & 2 & YES & YES & YES & $1.00$ & $(2,2)$ & -- & 1618\\
$(b;0,0,3;32)$ & 8 & $(3,1)$ & 2 & 1 & YES & YES & YES & $1.12$ & $(2,2)$ & -- & 1619\\
$(b;0,0,3;32)$ & 8 & $(4,1)$ & 3 & 4 & YES & YES & YES & $1.11$ & $(2,2)$ & -- & 1620\\
$(b;0,0,3;32)$ & 8 & $(5,1)$ & 4 & 1 & YES & YES & YES & $1.12$ & $(2,2)$ & -- & 1621\\
$(b;0,1,0;19)$ & 6 & $(7,2)$ & 4 & 1 & YES & YES & YES & $1.22$ & $(2,2)$ & -- & 1622\\
$(b;0,1,1;27)$ & 7 & $(2,1)$ & 1 & 1 & YES & YES & NO(3) & $0.75$ & $(2,2)$ & -- & 1623\\
$(b;0,1,1;27)$ & 7 & $(3,1)$ & 2 & 3 & YES & YES & NO(3) & $0.75$ & $(2,2)$ & -- & 1624\\
$(b;0,2,0;8)$ & 7 & $(4,1)$ & 3 & 4 & YES & YES & YES & $1.11$ & $(2,2)$ & -- & 1625\\
$(b;0,2,0;8)$ & 7 & $(5,2)$ & 3 & 1 & YES & YES & YES & $1.11$ & $(2,2)$ & -- & 1626\\
$(b;0,2,0;8)$ & 7 & $(7,2)$ & 4 & 1 & YES & YES & YES & $1.11$ & $(2,2)$ & -- & 1627\\
$(b;0,2,0;8)$ & 7 & $(9,2)$ & 5 & 1 & YES & YES & YES & $1.11$ & $(2,2)$ & -- & 1628\\
$(b;0,3,0;29)$ & 8 & $(2,1)$ & 1 & 1 & YES & YES & YES & $1.11$ & $(2,2)$ & -- & 1629\\
$(b;0,3,0;29)$ & 8 & $(3,1)$ & 2 & 1 & YES & YES & YES & $1.00$ & $(4,1)$ & -- & 1630\\
$(b;0,3,0;29)$ & 8 & $(5,1)$ & 4 & 1 & YES & YES & YES & $0.88$ & $(4,1)$ & -- & 1631\\
$(b;0,3,0;29)$ & 8 & $(6,1)$ & 5 & 1 & YES & YES & YES & $1.11$ & $(2,2)$ & -- & 1632\\
$(b;1,0,0;5)$ & 6 & $(3,1)$ & 2 & 1 & YES & YES & YES & $1.00$ & $(2,2)$ & -- & 1633\\
$(b;1,0,0;5)$ & 6 & $(8,3)$ & 4 & 1 & YES & YES & YES & $1.00$ & $(2,2)$ & -- & 1634\\
$(b;1,0,2;19)$ & 8 & $(3,1)$ & 2 & 1 & YES & YES & YES & $1.22$ & $(2,2)$ & -- & 1635\\
$(b;1,1,0;27)$ & 7 & $(4,1)$ & 3 & 1 & YES & YES & YES & $1.11$ & $(2,2)$ & -- & 1636\\
$(b;1,1,0;27)$ & 7 & $(5,2)$ & 3 & 1 & YES & YES & YES & $1.11$ & $(2,2)$ & -- & 1637\\
$(b;1,1,0;27)$ & 7 & $(7,2)$ & 4 & 1 & YES & YES & YES & $1.11$ & $(2,2)$ & -- & 1638\\
$(b;1,1,0;27)$ & 7 & $(13,3)$ & 6 & 1 & YES & YES & YES & $1.11$ & $(2,2)$ & -- & 1639\\
$(b;1,1,1;39)$ & 8 & $(3,1)$ & 2 & 3 & YES & YES & YES & $1.00$ & $(2,2)$ & -- & 1640\\
$(b;1,2,0;17)$ & 8 & $(2,1)$ & 1 & 1 & YES & YES & YES & $1.00$ & $(2,2)$ & -- & 1641\\
$(b;1,2,0;17)$ & 8 & $(5,1)$ & 4 & 1 & YES & YES & YES & $0.88$ & $(2,2)$ & -- & 1642\\
$(b;2,0,0;26)$ & 7 & $(3,1)$ & 2 & 1 & YES & YES & YES & $1.00$ & $(2,2)$ & -- & 1643\\
$(b;2,0,0;26)$ & 7 & $(5,2)$ & 3 & 1 & YES & YES & YES & $1.11$ & $(2,2)$ & -- & 1644\\
$(b;2,0,0;26)$ & 7 & $(10,3)$ & 5 & 2 & YES & YES & YES & $1.11$ & $(2,2)$ & -- & 1645\\
$(b;2,0,1;38)$ & 8 & $(2,1)$ & 1 & 2 & YES & YES & YES & $1.11$ & $(2,2)$ & -- & 1646\\
$(b;2,0,1;38)$ & 8 & $(4,1)$ & 3 & 2 & YES & YES & YES & $1.00$ & $(2,2)$ & -- & 1647\\
$(b;2,1,0;7)$ & 8 & $(3,1)$ & 2 & 1 & YES & YES & YES & $1.11$ & $(2,2)$ & -- & 1648\\
$(c;0,0,0;4)$ & 4 & $(16,7)$ & 6 & 4 & YES & YES & YES & $0.89$ & $(2,2)$ & -- & 1649\\
$(c;0,0,0;4)$ & 4 & $(17,5)$ & 6 & 1 & YES & YES & YES & $0.75$ & $(4,1)$ & -- & 1650\\
$(c;0,0,0;4)$ & 4 & $(18,7)$ & 6 & 2 & YES & YES & YES & $1.00$ & $(2,2)$ & -- & 1651\\
$(c;0,0,0;4)$ & 4 & $(19,7)$ & 6 & 1 & YES & YES & YES & $1.25$ & $(2,2)$ & -- & 1652\\
$(c;0,0,0;4)$ & 4 & $(23,7)$ & 7 & 1 & YES & YES & YES & $0.89$ & $(2,2)$ & -- & 1653\\
$(c;0,0,0;4)$ & 4 & $(23,10)$ & 7 & 1 & YES & YES & YES & $1.12$ & $(2,2)$ & -- & 1654\\
$(c;0,0,0;4)$ & 4 & $(25,11)$ & 7 & 1 & YES & YES & YES & $1.22$ & $(2,2)$ & -- & 1655\\
$(c;0,0,0;4)$ & 4 & $(27,10)$ & 7 & 1 & YES & YES & YES & $1.12$ & $(2,2)$ & -- & 1656\\
$(c;0,0,0;4)$ & 4 & $(29,12)$ & 7 & 1 & YES & YES & YES & $1.11$ & $(2,2)$ & -- & 1657\\
$(c;0,0,0;4)$ & 4 & $(30,11)$ & 7 & 2 & YES & YES & YES & $1.12$ & $(2,2)$ & -- & 1658\\
$(c;0,1,0;11)$ & 5 & $(11,4)$ & 5 & 11 & YES & YES & YES & $1.11$ & $(2,2)$ & -- & 1659\\
$(c;0,1,0;11)$ & 5 & $(16,7)$ & 6 & 1 & YES & YES & YES & $1.11$ & $(2,2)$ & -- & 1660\\
$(c;0,1,0;11)$ & 5 & $(26,7)$ & 7 & 1 & YES & YES & YES & $1.22$ & $(2,2)$ & -- & 1661\\
$(c;0,1,1;5)$ & 6 & $(7,2)$ & 4 & 1 & YES & YES & YES & $1.00$ & $(2,2)$ & -- & 1662\\
$(c;0,1,1;5)$ & 6 & $(8,3)$ & 4 & 1 & YES & YES & NO(3) & $0.75$ & $(2,2)$ & -- & 1663\\
$(c;0,1,1;5)$ & 6 & $(10,3)$ & 5 & 5 & YES & YES & YES & $1.00$ & $(2,2)$ & -- & 1664\\
$(c;0,1,1;5)$ & 6 & $(11,3)$ & 5 & 1 & YES & YES & YES & $1.00$ & $(2,2)$ & -- & 1665\\
$(c;0,1,1;5)$ & 6 & $(18,5)$ & 6 & 1 & YES & YES & YES & $1.11$ & $(2,2)$ & -- & 1666\\
$(c;0,2,0;7)$ & 6 & $(7,2)$ & 4 & 7 & YES & YES & YES & $1.00$ & $(2,2)$ & -- & 1667\\
$(c;0,2,0;7)$ & 6 & $(10,3)$ & 5 & 1 & YES & YES & YES & $1.00$ & $(2,2)$ & -- & 1668\\
$(c;0,2,0;7)$ & 6 & $(14,3)$ & 6 & 7 & YES & YES & YES & $1.11$ & $(2,2)$ & -- & 1669\\
$(c;0,2,0;7)$ & 6 & $(19,4)$ & 7 & 1 & YES & YES & YES & $1.11$ & $(2,2)$ & -- & 1670\\
$(c;0,2,1;19)$ & 7 & $(4,1)$ & 3 & 1 & YES & YES & YES & $1.11$ & $(2,2)$ & -- & 1671\\
$(c;0,2,1;19)$ & 7 & $(7,2)$ & 4 & 1 & YES & YES & YES & $1.00$ & $(2,2)$ & -- & 1672\\
$(c;0,2,1;19)$ & 7 & $(10,3)$ & 5 & 1 & YES & YES & YES & $0.75$ & $(4,1)$ & -- & 1673\\
$(c;0,2,1;19)$ & 7 & $(13,3)$ & 6 & 1 & YES & YES & YES & $1.00$ & $(2,2)$ & -- & 1674\\
$(c;0,2,1;19)$ & 7 & $(14,3)$ & 6 & 1 & YES & YES & YES & $1.00$ & $(2,2)$ & -- & 1675\\
$(c;0,2,2;6)$ & 8 & $(5,2)$ & 3 & 1 & YES & YES & YES & $1.11$ & $(2,2)$ & -- & 1676\\
$(c;0,3,0;17)$ & 7 & $(9,2)$ & 5 & 1 & YES & YES & YES & $1.00$ & $(2,2)$ & -- & 1677\\
$(c;0,3,0;17)$ & 7 & $(13,3)$ & 6 & 1 & YES & YES & YES & $1.12$ & $(2,2)$ & -- & 1678\\
$(c;0,3,0;17)$ & 7 & $(14,3)$ & 6 & 1 & YES & YES & YES & $1.12$ & $(2,2)$ & -- & 1679\\
$(c;0,3,1;23)$ & 8 & $(3,1)$ & 2 & 1 & YES & YES & YES & $1.00$ & $(2,2)$ & -- & 1680\\
$(c;0,3,1;23)$ & 8 & $(5,1)$ & 4 & 1 & YES & YES & YES & $1.00$ & $(2,2)$ & -- & 1681\\
$(c;0,3,1;23)$ & 8 & $(5,2)$ & 3 & 1 & YES & YES & YES & $0.88$ & $(4,1)$ & -- & 1682\\
$(c;0,3,1;23)$ & 8 & $(7,2)$ & 4 & 1 & YES & YES & YES & $1.00$ & $(2,2)$ & -- & 1683\\
$(c;0,3,1;23)$ & 8 & $(9,2)$ & 5 & 1 & YES & YES & YES & $1.00$ & $(2,2)$ & -- & 1684\\
$(c;0,3,2;29)$ & 9 & $(3,1)$ & 2 & 1 & YES & YES & YES & $0.88$ & $(2,2)$ & -- & 1685\\
$(c;0,3,2;29)$ & 9 & $(4,1)$ & 3 & 1 & YES & YES & YES & $0.88$ & $(2,2)$ & -- & 1686\\
$(d;0,0,0;5)$ & 5 & $(7,3)$ & 4 & 1 & YES & YES & YES & $0.89$ & $(2,2)$ & -- & 1687\\
$(d;0,0,0;5)$ & 5 & $(12,5)$ & 5 & 1 & YES & YES & YES & $1.22$ & $(2,2)$ & -- & 1688\\
$(d;0,0,0;5)$ & 5 & $(13,5)$ & 5 & 1 & YES & YES & YES & $1.00$ & $(2,2)$ & -- & 1689\\
$(d;0,0,0;5)$ & 5 & $(16,7)$ & 6 & 1 & YES & YES & YES & $1.22$ & $(2,2)$ & -- & 1690\\
$(d;0,0,0;5)$ & 5 & $(17,5)$ & 6 & 1 & YES & YES & YES & $1.00$ & $(2,2)$ & -- & 1691\\
$(d;0,0,0;5)$ & 5 & $(26,7)$ & 7 & 1 & YES & YES & YES & $1.22$ & $(2,2)$ & -- & 1692\\
$(d;0,0,1;14)$ & 6 & $(7,3)$ & 4 & 7 & YES & YES & YES & $1.00$ & $(2,2)$ & -- & 1693\\
$(d;0,0,1;14)$ & 6 & $(8,3)$ & 4 & 2 & YES & YES & NO(3) & $0.75$ & $(2,2)$ & -- & 1694\\
$(d;0,0,1;14)$ & 6 & $(10,3)$ & 5 & 2 & YES & YES & YES & $0.88$ & $(2,2)$ & -- & 1695\\
$(d;0,0,1;14)$ & 6 & $(11,3)$ & 5 & 1 & YES & YES & YES & $1.00$ & $(2,2)$ & -- & 1696\\
$(d;0,0,1;14)$ & 6 & $(17,5)$ & 6 & 1 & YES & YES & YES & $1.00$ & $(2,2)$ & -- & 1697\\
$(d;0,0,1;14)$ & 6 & $(18,5)$ & 6 & 2 & YES & YES & YES & $1.00$ & $(2,2)$ & -- & 1698\\
$(d;0,0,2;9)$ & 7 & $(4,1)$ & 3 & 1 & YES & YES & YES & $1.11$ & $(2,2)$ & -- & 1699\\
$(d;0,0,2;9)$ & 7 & $(7,2)$ & 4 & 1 & YES & YES & YES & $1.00$ & $(2,2)$ & -- & 1700\\
$(d;0,0,2;9)$ & 7 & $(8,3)$ & 4 & 1 & YES & YES & YES & $1.00$ & $(2,2)$ & -- & 1701\\
$(d;0,0,2;9)$ & 7 & $(9,2)$ & 5 & 9 & YES & YES & YES & $1.00$ & $(2,2)$ & -- & 1702\\
$(d;0,0,3;22)$ & 8 & $(3,1)$ & 2 & 1 & YES & YES & YES & $1.00$ & $(2,2)$ & -- & 1703\\
$(d;0,0,3;22)$ & 8 & $(5,1)$ & 4 & 1 & YES & YES & YES & $1.00$ & $(2,2)$ & -- & 1704\\
$(d;0,0,3;22)$ & 8 & $(7,2)$ & 4 & 1 & YES & YES & YES & $1.11$ & $(2,2)$ & -- & 1705\\
$(d;0,1,0;6)$ & 6 & $(10,3)$ & 5 & 2 & YES & YES & YES & $1.00$ & $(2,2)$ & -- & 1706\\
$(d;0,1,0;6)$ & 6 & $(22,5)$ & 7 & 2 & YES & YES & YES & $1.11$ & $(2,2)$ & -- & 1707\\
$(d;0,1,1;17)$ & 7 & $(8,3)$ & 4 & 1 & YES & YES & YES & $1.00$ & $(2,2)$ & -- & 1708\\
$(d;0,1,1;17)$ & 7 & $(13,3)$ & 6 & 1 & YES & YES & YES & $1.11$ & $(2,2)$ & -- & 1709\\
$(d;0,1,1;17)$ & 7 & $(14,3)$ & 6 & 1 & YES & YES & YES & $1.00$ & $(2,2)$ & -- & 1710\\
$(d;0,1,2;11)$ & 8 & $(4,1)$ & 3 & 1 & YES & YES & YES & $1.11$ & $(2,2)$ & -- & 1711\\
$(d;0,1,2;11)$ & 8 & $(5,2)$ & 3 & 1 & YES & YES & YES & $1.00$ & $(2,2)$ & -- & 1712\\
$(d;0,1,3;27)$ & 9 & $(2,1)$ & 1 & 1 & YES & YES & YES & $1.00$ & $(2,2)$ & -- & 1713\\
$(d;0,1,3;27)$ & 9 & $(3,1)$ & 2 & 3 & YES & YES & YES & $1.00$ & $(2,2)$ & -- & 1714\\
$(d;0,1,3;27)$ & 9 & $(5,1)$ & 4 & 1 & YES & YES & YES & $1.00$ & $(2,2)$ & -- & 1715\\
$(d;0,2,2;13)$ & 9 & $(3,1)$ & 2 & 1 & YES & YES & YES & $0.88$ & $(2,2)$ & -- & 1716\\
$(e;0,0,0;4)$ & 5 & $(8,3)$ & 4 & 4 & YES & YES & YES & $0.88$ & $(4,1)$ & -- & 1717\\
$(e;0,0,0;4)$ & 5 & $(12,5)$ & 5 & 4 & YES & YES & YES & $0.88$ & $(4,1)$ & -- & 1718\\
$(e;0,2,0;6)$ & 7 & $(5,2)$ & 3 & 1 & YES & YES & YES & $1.11$ & $(2,2)$ & -- & 1719\\
$(e;0,2,0;6)$ & 7 & $(7,2)$ & 4 & 1 & YES & YES & YES & $1.11$ & $(2,2)$ & -- & 1720\\
$(e;0,2,0;6)$ & 7 & $(9,2)$ & 5 & 3 & YES & YES & YES & $1.11$ & $(2,2)$ & -- & 1721\\
$(e;1,0,0;18)$ & 6 & $(3,1)$ & 2 & 3 & YES & YES & YES & $1.00$ & $(2,2)$ & -- & 1722\\
$(e;1,0,0;18)$ & 6 & $(5,2)$ & 3 & 1 & YES & YES & YES & $1.00$ & $(2,2)$ & -- & 1723\\
$(e;1,0,0;18)$ & 6 & $(7,3)$ & 4 & 1 & YES & YES & YES & $0.75$ & $(4,1)$ & -- & 1724\\
$(e;1,1,0;23)$ & 7 & $(2,1)$ & 1 & 1 & YES & YES & NO(3) & $0.75$ & $(2,2)$ & -- & 1725\\
$(e;1,1,0;23)$ & 7 & $(3,1)$ & 2 & 1 & YES & YES & NO(3) & $0.75$ & $(2,2)$ & -- & 1726\\
$(e;1,1,0;23)$ & 7 & $(5,2)$ & 3 & 1 & YES & YES & YES & $1.00$ & $(2,2)$ & -- & 1727\\
$(e;1,2,0;28)$ & 8 & $(2,1)$ & 1 & 2 & YES & YES & YES & $0.88$ & $(2,2)$ & -- & 1728\\
$(e;1,2,0;28)$ & 8 & $(3,1)$ & 2 & 1 & YES & YES & YES & $0.88$ & $(2,2)$ & -- & 1729\\
$(e;2,0,0;24)$ & 7 & $(2,1)$ & 1 & 2 & YES & YES & YES & $0.89$ & $(2,2)$ & -- & 1730\\
$(e;2,0,0;24)$ & 7 & $(4,1)$ & 3 & 4 & YES & YES & YES & $1.11$ & $(2,2)$ & -- & 1731\\
$(e;2,0,0;24)$ & 7 & $(5,2)$ & 3 & 1 & YES & YES & YES & $1.11$ & $(2,2)$ & -- & 1732\\
$(e;2,0,0;24)$ & 7 & $(7,3)$ & 4 & 1 & YES & YES & YES & $1.11$ & $(2,2)$ & -- & 1733\\
$(e;3,0,0;10)$ & 8 & $(2,1)$ & 1 & 2 & YES & YES & YES & $1.00$ & $(2,2)$ & -- & 1734\\
$(e;3,0,0;10)$ & 8 & $(3,1)$ & 2 & 1 & YES & YES & YES & $1.12$ & $(2,2)$ & -- & 1735\\
$(e;3,0,0;10)$ & 8 & $(4,1)$ & 3 & 2 & YES & YES & YES & $1.11$ & $(2,2)$ & -- & 1736\\
$(f;0,0,0;6)$ & 4 & $(18,7)$ & 6 & 6 & YES & YES & YES & $1.11$ & $(2,2)$ & -- & 1737\\
$(f;0,0,0;6)$ & 4 & $(21,8)$ & 6 & 3 & YES & YES & YES & $0.75$ & $(4,1)$ & -- & 1738\\
$(f;0,0,0;6)$ & 4 & $(23,7)$ & 7 & 1 & YES & YES & YES & $1.11$ & $(2,2)$ & -- & 1739\\
$(f;0,0,0;6)$ & 4 & $(23,9)$ & 7 & 1 & YES & YES & YES & $1.11$ & $(2,2)$ & -- & 1740\\
$(f;0,0,0;6)$ & 4 & $(26,11)$ & 7 & 2 & YES & YES & YES & $1.00$ & $(2,2)$ & -- & 1741\\
$(f;0,0,0;6)$ & 4 & $(27,8)$ & 7 & 3 & YES & YES & YES & $1.11$ & $(2,2)$ & -- & 1742\\
$(f;0,0,0;6)$ & 4 & $(29,9)$ & 8 & 1 & YES & YES & YES & $1.11$ & $(2,2)$ & -- & 1743\\
$(f;0,0,0;6)$ & 4 & $(30,11)$ & 7 & 6 & YES & YES & YES & $1.00$ & $(2,2)$ & -- & 1744\\
$(f;0,0,0;6)$ & 4 & $(31,12)$ & 7 & 1 & YES & YES & YES & $1.11$ & $(2,2)$ & -- & 1745\\
$(f;0,0,0;6)$ & 4 & $(34,13)$ & 7 & 2 & YES & YES & YES & $1.00$ & $(2,2)$ & -- & 1746\\
$(f;0,1,0;7)$ & 5 & $(13,5)$ & 5 & 1 & YES & YES & YES & $0.88$ & $(4,1)$ & -- & 1747\\
$(g;0,0,0;19)$ & 6 & $(7,3)$ & 4 & 1 & YES & YES & YES & $0.75$ & $(4,1)$ & -- & 1748\\
$(g;0,0,1;26)$ & 7 & $(4,1)$ & 3 & 2 & YES & YES & NO(3) & $0.75$ & $(2,2)$ & -- & 1749\\
$(g;0,1,0;24)$ & 7 & $(4,1)$ & 3 & 4 & YES & YES & NO(3) & $0.75$ & $(2,2)$ & -- & 1750\\
$(h;0,0,0;6)$ & 5 & $(5,2)$ & 3 & 1 & YES & YES & YES & $1.00$ & $(2,2)$ & -- & 1751\\
$(h;0,2,0;10)$ & 7 & $(5,2)$ & 3 & 5 & YES & YES & YES & $1.00$ & $(2,2)$ & -- & 1752\\
$(h;0,3,0;12)$ & 8 & $(2,1)$ & 1 & 2 & YES & YES & YES & $0.88$ & $(2,2)$ & -- & 1753\\
$(h;0,3,0;12)$ & 8 & $(5,1)$ & 4 & 1 & YES & YES & YES & $0.75$ & $(4,1)$ & -- & 1754\\
$(i;0,0,0;9)$ & 5 & $(8,3)$ & 4 & 1 & YES & YES & YES & $0.88$ & $(4,1)$ & -- & 1755\\
$(i;0,0,0;9)$ & 5 & $(10,3)$ & 5 & 1 & YES & YES & YES & $1.00$ & $(4,1)$ & -- & 1756\\
$(i;0,0,0;9)$ & 5 & $(12,5)$ & 5 & 3 & YES & YES & YES & $0.88$ & $(2,2)$ & -- & 1757\\
$(i;0,0,0;9)$ & 5 & $(13,5)$ & 5 & 1 & YES & YES & YES & $0.88$ & $(4,1)$ & -- & 1758\\
$(i;0,0,0;9)$ & 5 & $(17,5)$ & 6 & 1 & YES & YES & YES & $0.88$ & $(4,1)$ & -- & 1759\\
$(i;0,0,0;9)$ & 5 & $(22,5)$ & 7 & 1 & YES & YES & YES & $1.00$ & $(2,2)$ & -- & 1760\\
$(i;0,0,0;9)$ & 5 & $(23,5)$ & 7 & 1 & YES & YES & YES & $1.00$ & $(2,2)$ & -- & 1761\\
$(i;0,1,0;12)$ & 6 & $(5,1)$ & 4 & 1 & YES & YES & YES & $0.88$ & $(4,1)$ & -- & 1762\\
$(i;0,2,0;15)$ & 7 & $(7,2)$ & 4 & 1 & YES & YES & YES & $1.00$ & $(2,2)$ & -- & 1763\\
$(j;0,0,0;8)$ & 5 & $(11,3)$ & 5 & 1 & YES & YES & YES & $1.11$ & $(2,2)$ & -- & 1764\\
$(j;0,0,0;8)$ & 5 & $(13,5)$ & 5 & 1 & YES & YES & YES & $0.88$ & $(2,2)$ & -- & 1765\\
$(j;0,0,0;8)$ & 5 & $(17,5)$ & 6 & 1 & YES & YES & YES & $1.11$ & $(2,2)$ & -- & 1766\\
$(j;0,0,0;8)$ & 5 & $(17,7)$ & 6 & 1 & YES & YES & YES & $0.88$ & $(2,2)$ & -- & 1767\\
$(j;0,1,0;10)$ & 6 & $(8,3)$ & 4 & 2 & YES & YES & YES & $0.88$ & $(4,1)$ & -- & 1768\\
$(j;0,1,0;10)$ & 6 & $(11,3)$ & 5 & 1 & YES & YES & YES & $1.00$ & $(2,2)$ & -- & 1769\\
$(j;0,1,0;10)$ & 6 & $(11,5)$ & 6 & 1 & YES & YES & YES & $1.12$ & $(2,2)$ & -- & 1770\\
$(j;0,1,0;10)$ & 6 & $(12,5)$ & 5 & 2 & YES & YES & YES & $0.88$ & $(2,2)$ & -- & 1771
\end{longtable}
\subsection{2 chains, $K^2 = 3$}
\begin{longtable}{|c|c|c|c|c|c|c|c|c|c|c|c|}
\hline
\multicolumn{12}{|c|}{2 chains, $K^2 = 3$}\\
\hline
$(n,a)$ & Len & $(n,a)$ & Len & GCD & Nef & $\mathbb Q$-ef & Obs 0 & $\overline c_1^2 / \overline c_2$ & $(P,K)$ & WH & Index\\
\hline
\endfirsthead

\hline
$(n,a)$ & Len & $(n,a)$ & Len & GCD & Nef & $\mathbb Q$-ef & Obs 0 & $\overline c_1^2 / \overline c_2$ & $(P,K)$ & WH & Index\\
\hline
\endhead
\hline
\endfoot

$(22,9)$ & 7 & $(22,5)$ & 7 & 22 & YES & YES & YES & $1.38$ & $(2,3)$ & -- & 1772\\
$(23,10)$ & 7 & $(11,4)$ & 5 & 1 & YES & YES & YES & $1.29$ & $(4,2)$ & -- & 1773\\
$(23,10)$ & 7 & $(19,7)$ & 6 & 1 & YES & YES & YES & $1.29$ & $(4,2)$ & -- & 1774\\
$(23,10)$ & 7 & $(21,8)$ & 6 & 1 & YES & YES & YES & $1.62$ & $(2,3)$ & NO & 1775\\
$(23,10)$ & 7 & $(21,8)$ & 6 & 1 & YES & YES & YES & $1.62$ & $(2,3)$ & -- & 1776\\
$(23,9)$ & 7 & $(23,7)$ & 7 & 23 & YES & YES & YES & $1.57$ & $(2,3)$ & NO & 1777\\
$(23,9)$ & 7 & $(23,7)$ & 7 & 23 & YES & YES & YES & $1.57$ & $(2,3)$ & -- & 1778\\
$(23,10)$ & 7 & $(23,10)$ & 7 & 23 & YES & YES & YES & $1.29$ & $(4,2)$ & -- & 1779\\
$(24,7)$ & 7 & $(14,3)$ & 6 & 2 & YES & YES & YES & $1.50$ & $(2,3)$ & -- & 1780\\
$(24,7)$ & 7 & $(19,5)$ & 7 & 1 & YES & YES & YES & $1.25$ & $(2,3)$ & -- & 1781\\
$(25,9)$ & 7 & $(23,10)$ & 7 & 1 & YES & YES & YES & $1.43$ & $(4,2)$ & -- & 1782\\
$(25,11)$ & 7 & $(25,11)$ & 7 & 25 & YES & YES & YES & $1.62$ & $(2,3)$ & -- & 1783\\
$(26,11)$ & 7 & $(23,9)$ & 7 & 1 & YES & YES & YES & $1.57$ & $(2,3)$ & -- & 1784\\
$(27,10)$ & 7 & $(11,4)$ & 5 & 1 & YES & YES & YES & $1.29$ & $(4,2)$ & -- & 1785\\
$(27,10)$ & 7 & $(13,5)$ & 5 & 1 & YES & YES & YES & $1.62$ & $(2,3)$ & -- & 1786\\
$(27,10)$ & 7 & $(17,7)$ & 6 & 1 & YES & YES & YES & $1.29$ & $(4,2)$ & -- & 1787\\
$(27,10)$ & 7 & $(17,7)$ & 6 & 1 & YES & YES & YES & $1.29$ & $(4,2)$ & NO & 1788\\
$(27,11)$ & 8 & $(17,5)$ & 6 & 1 & YES & YES & YES & $1.57$ & $(2,3)$ & NO & 1789\\
$(27,11)$ & 8 & $(17,5)$ & 6 & 1 & YES & YES & YES & $1.57$ & $(2,3)$ & -- & 1790\\
$(27,11)$ & 8 & $(27,8)$ & 7 & 27 & YES & YES & YES & $1.43$ & $(2,3)$ & -- & 1791\\
$(28,11)$ & 8 & $(23,7)$ & 7 & 1 & YES & YES & YES & $1.57$ & $(2,3)$ & -- & 1792\\
$(29,13)$ & 8 & $(17,5)$ & 6 & 1 & YES & YES & YES & $1.50$ & $(2,3)$ & -- & 1793\\
$(29,13)$ & 8 & $(24,7)$ & 7 & 1 & YES & YES & YES & $1.29$ & $(4,2)$ & -- & 1794\\
$(29,9)$ & 8 & $(27,8)$ & 7 & 1 & YES & YES & YES & $1.43$ & $(2,3)$ & -- & 1795\\
$(29,13)$ & 8 & $(27,8)$ & 7 & 1 & YES & YES & YES & $1.43$ & $(2,3)$ & -- & 1796\\
$(30,11)$ & 7 & $(13,5)$ & 5 & 1 & YES & YES & YES & $1.29$ & $(4,2)$ & -- & 1797\\
$(30,11)$ & 7 & $(17,7)$ & 6 & 1 & YES & YES & YES & $1.43$ & $(2,3)$ & NO & 1798\\
$(30,11)$ & 7 & $(17,7)$ & 6 & 1 & YES & YES & YES & $1.43$ & $(2,3)$ & -- & 1799\\
$(30,13)$ & 8 & $(17,5)$ & 6 & 1 & YES & YES & YES & $1.62$ & $(2,3)$ & -- & 1800\\
$(31,7)$ & 8 & $(11,4)$ & 5 & 1 & YES & YES & YES & $1.29$ & $(4,2)$ & NO & 1801\\
$(31,7)$ & 8 & $(11,4)$ & 5 & 1 & YES & YES & YES & $1.29$ & $(4,2)$ & -- & 1802\\
$(31,12)$ & 7 & $(16,5)$ & 7 & 1 & YES & YES & YES & $1.57$ & $(2,3)$ & NO & 1803\\
$(31,12)$ & 7 & $(16,5)$ & 7 & 1 & YES & YES & YES & $1.57$ & $(2,3)$ & -- & 1804\\
$(31,9)$ & 8 & $(20,9)$ & 7 & 1 & YES & YES & YES & $1.43$ & $(2,3)$ & -- & 1805\\
$(31,13)$ & 7 & $(22,9)$ & 7 & 1 & YES & YES & YES & $1.43$ & $(2,3)$ & -- & 1806\\
$(31,12)$ & 7 & $(23,9)$ & 7 & 1 & YES & YES & YES & $1.43$ & $(2,3)$ & -- & 1807\\
$(31,7)$ & 8 & $(29,13)$ & 8 & 1 & YES & YES & YES & $1.43$ & $(4,2)$ & NO & 1808\\
$(31,12)$ & 7 & $(29,12)$ & 7 & 1 & YES & YES & YES & $1.71$ & $(2,3)$ & -- & 1809\\
$(31,11)$ & 8 & $(31,7)$ & 8 & 31 & YES & YES & YES & $1.29$ & $(4,2)$ & NO & 1810\\
$(32,7)$ & 8 & $(11,4)$ & 5 & 1 & YES & YES & YES & $1.29$ & $(4,2)$ & NO & 1811\\
$(32,7)$ & 8 & $(11,4)$ & 5 & 1 & YES & YES & YES & $1.29$ & $(4,2)$ & -- & 1812\\
$(32,9)$ & 8 & $(20,9)$ & 7 & 4 & YES & YES & YES & $1.57$ & $(2,3)$ & NO & 1813\\
$(32,9)$ & 8 & $(20,9)$ & 7 & 4 & YES & YES & YES & $1.57$ & $(2,3)$ & -- & 1814\\
$(32,9)$ & 8 & $(22,5)$ & 7 & 2 & YES & YES & YES & $1.62$ & $(2,3)$ & -- & 1815\\
$(32,9)$ & 8 & $(22,5)$ & 7 & 2 & YES & YES & YES & $1.62$ & $(2,3)$ & NO & 1816\\
$(32,7)$ & 8 & $(26,11)$ & 7 & 2 & YES & YES & YES & $1.50$ & $(2,3)$ & -- & 1817\\
$(32,9)$ & 8 & $(29,11)$ & 7 & 1 & YES & YES & YES & $1.57$ & $(4,2)$ & -- & 1818\\
$(33,13)$ & 9 & $(18,5)$ & 6 & 3 & YES & YES & YES & $1.43$ & $(4,2)$ & -- & 1819\\
$(33,13)$ & 9 & $(19,3)$ & 8 & 1 & YES & YES & YES & $1.29$ & $(4,2)$ & NO & 1820\\
$(33,10)$ & 8 & $(23,10)$ & 7 & 1 & YES & YES & YES & $1.29$ & $(4,2)$ & NO & 1821\\
$(33,13)$ & 9 & $(23,5)$ & 7 & 1 & YES & YES & YES & $1.43$ & $(4,2)$ & NO & 1822\\
$(33,7)$ & 8 & $(26,11)$ & 7 & 1 & YES & YES & YES & $1.43$ & $(2,3)$ & -- & 1823\\
$(34,15)$ & 8 & $(18,7)$ & 6 & 2 & YES & YES & YES & $1.75$ & $(2,3)$ & -- & 1824\\
$(34,13)$ & 7 & $(23,9)$ & 7 & 1 & YES & YES & YES & $1.43$ & $(2,3)$ & -- & 1825\\
$(34,15)$ & 8 & $(24,5)$ & 8 & 2 & YES & YES & YES & $1.71$ & $(2,3)$ & -- & 1826\\
$(34,13)$ & 7 & $(32,9)$ & 8 & 2 & YES & YES & YES & $1.43$ & $(4,2)$ & -- & 1827\\
$(34,13)$ & 7 & $(33,10)$ & 8 & 1 & YES & YES & YES & $1.86$ & $(2,3)$ & -- & 1828\\
$(35,13)$ & 8 & $(9,2)$ & 5 & 1 & YES & YES & YES & $1.43$ & $(4,2)$ & -- & 1829\\
$(35,8)$ & 8 & $(11,3)$ & 5 & 1 & YES & YES & YES & $1.38$ & $(2,3)$ & -- & 1830\\
$(35,8)$ & 8 & $(11,4)$ & 5 & 1 & YES & YES & YES & $1.29$ & $(4,2)$ & NO & 1831\\
$(35,8)$ & 8 & $(11,4)$ & 5 & 1 & YES & YES & YES & $1.29$ & $(4,2)$ & -- & 1832\\
$(35,13)$ & 8 & $(16,3)$ & 7 & 1 & YES & YES & YES & $1.29$ & $(4,2)$ & NO & 1833\\
$(35,13)$ & 8 & $(16,3)$ & 7 & 1 & YES & YES & YES & $1.29$ & $(4,2)$ & -- & 1834\\
$(35,8)$ & 8 & $(17,7)$ & 6 & 1 & YES & YES & YES & $1.38$ & $(2,3)$ & NO & 1835\\
$(35,8)$ & 8 & $(17,7)$ & 6 & 1 & YES & YES & YES & $1.38$ & $(2,3)$ & -- & 1836\\
$(35,11)$ & 9 & $(23,7)$ & 7 & 1 & YES & YES & YES & $1.50$ & $(2,3)$ & -- & 1837\\
$(35,8)$ & 8 & $(30,11)$ & 7 & 5 & YES & YES & YES & $1.38$ & $(2,3)$ & NO & 1838\\
$(35,8)$ & 8 & $(31,11)$ & 8 & 1 & YES & YES & YES & $1.43$ & $(2,3)$ & NO & 1839\\
$(36,13)$ & 8 & $(11,4)$ & 5 & 1 & YES & YES & YES & $1.29$ & $(4,2)$ & -- & 1840\\
$(36,13)$ & 8 & $(13,5)$ & 5 & 1 & YES & YES & YES & $1.29$ & $(4,2)$ & -- & 1841\\
$(36,13)$ & 8 & $(20,7)$ & 8 & 4 & YES & YES & YES & $1.57$ & $(4,2)$ & -- & 1842\\
$(36,13)$ & 8 & $(31,7)$ & 8 & 1 & YES & YES & YES & $1.50$ & $(2,3)$ & -- & 1843\\
$(37,14)$ & 8 & $(11,4)$ & 5 & 1 & YES & YES & YES & $1.62$ & $(2,3)$ & -- & 1844\\
$(37,10)$ & 8 & $(12,5)$ & 5 & 1 & YES & YES & YES & $1.62$ & $(2,3)$ & NO & 1845\\
$(37,10)$ & 8 & $(12,5)$ & 5 & 1 & YES & YES & YES & $1.62$ & $(2,3)$ & -- & 1846\\
$(37,14)$ & 8 & $(12,5)$ & 5 & 1 & YES & YES & YES & $1.57$ & $(2,3)$ & NO & 1847\\
$(37,10)$ & 8 & $(13,5)$ & 5 & 1 & YES & YES & YES & $1.62$ & $(2,3)$ & NO & 1848\\
$(37,10)$ & 8 & $(13,5)$ & 5 & 1 & YES & YES & YES & $1.62$ & $(2,3)$ & -- & 1849\\
$(37,14)$ & 8 & $(14,3)$ & 6 & 1 & YES & YES & YES & $1.29$ & $(4,2)$ & NO & 1850\\
$(37,14)$ & 8 & $(14,3)$ & 6 & 1 & YES & YES & YES & $1.29$ & $(4,2)$ & -- & 1851\\
$(37,11)$ & 8 & $(16,5)$ & 7 & 1 & YES & YES & YES & $1.62$ & $(2,3)$ & -- & 1852\\
$(37,14)$ & 8 & $(16,7)$ & 6 & 1 & YES & YES & YES & $1.57$ & $(2,3)$ & NO & 1853\\
$(37,14)$ & 8 & $(16,7)$ & 6 & 1 & YES & YES & YES & $1.57$ & $(2,3)$ & -- & 1854\\
$(37,11)$ & 8 & $(23,9)$ & 7 & 1 & YES & YES & YES & $1.57$ & $(2,3)$ & NO & 1855\\
$(37,14)$ & 8 & $(23,5)$ & 7 & 1 & YES & YES & YES & $1.14$ & $(4,2)$ & -- & 1856\\
$(37,14)$ & 8 & $(23,5)$ & 7 & 1 & YES & YES & YES & $1.50$ & $(2,3)$ & NO & 1857\\
$(37,16)$ & 9 & $(23,5)$ & 7 & 1 & YES & YES & YES & $1.62$ & $(2,3)$ & -- & 1858\\
$(37,16)$ & 9 & $(28,5)$ & 8 & 1 & YES & YES & YES & $1.62$ & $(2,3)$ & NO & 1859\\
$(37,11)$ & 8 & $(31,12)$ & 7 & 1 & YES & YES & YES & $1.75$ & $(2,3)$ & -- & 1860\\
$(37,8)$ & 8 & $(33,10)$ & 8 & 1 & YES & YES & YES & $1.38$ & $(2,3)$ & NO & 1861\\
$(37,11)$ & 8 & $(34,13)$ & 7 & 1 & YES & YES & YES & $1.71$ & $(2,3)$ & -- & 1862\\
$(38,11)$ & 9 & $(23,7)$ & 7 & 1 & YES & YES & YES & $1.43$ & $(2,3)$ & -- & 1863\\
$(38,7)$ & 9 & $(28,11)$ & 8 & 2 & YES & YES & YES & $1.29$ & $(4,2)$ & NO & 1864\\
$(39,14)$ & 8 & $(13,4)$ & 6 & 13 & YES & YES & YES & $1.57$ & $(2,3)$ & NO & 1865\\
$(39,14)$ & 8 & $(18,7)$ & 6 & 3 & YES & YES & YES & $1.50$ & $(2,3)$ & -- & 1866\\
$(39,14)$ & 8 & $(26,11)$ & 7 & 13 & YES & YES & YES & $1.62$ & $(2,3)$ & NO & 1867\\
$(40,9)$ & 9 & $(19,6)$ & 8 & 1 & YES & YES & YES & $1.62$ & $(2,3)$ & -- & 1868\\
$(40,9)$ & 9 & $(29,9)$ & 8 & 1 & YES & YES & YES & $1.62$ & $(2,3)$ & -- & 1869\\
$(40,11)$ & 8 & $(34,13)$ & 7 & 2 & YES & YES & YES & $1.71$ & $(2,3)$ & -- & 1870\\
$(41,11)$ & 8 & $(8,3)$ & 4 & 1 & YES & YES & YES & $1.14$ & $(4,2)$ & NO & 1871\\
$(41,11)$ & 8 & $(8,3)$ & 4 & 1 & YES & YES & YES & $1.14$ & $(4,2)$ & -- & 1872\\
$(41,11)$ & 8 & $(9,4)$ & 5 & 1 & YES & YES & YES & $1.29$ & $(4,2)$ & NO & 1873\\
$(41,11)$ & 8 & $(9,4)$ & 5 & 1 & YES & YES & YES & $1.29$ & $(4,2)$ & -- & 1874\\
$(41,11)$ & 8 & $(11,4)$ & 5 & 1 & YES & YES & YES & $1.29$ & $(4,2)$ & NO & 1875\\
$(41,11)$ & 8 & $(11,4)$ & 5 & 1 & YES & YES & YES & $1.29$ & $(4,2)$ & -- & 1876\\
$(41,17)$ & 8 & $(11,4)$ & 5 & 1 & YES & YES & YES & $1.43$ & $(4,2)$ & NO & 1877\\
$(41,17)$ & 8 & $(11,4)$ & 5 & 1 & YES & YES & YES & $1.43$ & $(4,2)$ & -- & 1878\\
$(41,18)$ & 8 & $(12,5)$ & 5 & 1 & YES & YES & YES & $1.62$ & $(2,3)$ & -- & 1879\\
$(41,16)$ & 8 & $(13,4)$ & 6 & 1 & YES & YES & YES & $1.75$ & $(2,3)$ & -- & 1880\\
$(41,17)$ & 8 & $(17,7)$ & 6 & 1 & YES & YES & YES & $1.29$ & $(4,2)$ & -- & 1881\\
$(41,17)$ & 8 & $(29,13)$ & 8 & 1 & YES & YES & YES & $1.29$ & $(4,2)$ & NO & 1882\\
$(41,17)$ & 8 & $(31,7)$ & 8 & 1 & YES & YES & YES & $1.43$ & $(4,2)$ & NO & 1883\\
$(41,16)$ & 8 & $(41,15)$ & 8 & 41 & YES & YES & YES & $1.43$ & $(2,3)$ & NO & 1884\\
$(42,19)$ & 9 & $(8,3)$ & 4 & 2 & YES & YES & YES & $1.29$ & $(4,2)$ & -- & 1885\\
$(42,19)$ & 9 & $(11,3)$ & 5 & 1 & YES & YES & YES & $1.29$ & $(4,2)$ & NO & 1886\\
$(42,13)$ & 9 & $(18,7)$ & 6 & 6 & YES & YES & YES & $1.29$ & $(4,2)$ & -- & 1887\\
$(43,16)$ & 9 & $(11,3)$ & 5 & 1 & YES & YES & YES & $1.43$ & $(4,2)$ & -- & 1888\\
$(43,12)$ & 8 & $(12,5)$ & 5 & 1 & YES & YES & YES & $1.62$ & $(2,3)$ & -- & 1889\\
$(43,18)$ & 8 & $(20,9)$ & 7 & 1 & YES & YES & YES & $1.29$ & $(4,2)$ & NO & 1890\\
$(43,18)$ & 8 & $(32,7)$ & 8 & 1 & YES & YES & YES & $1.43$ & $(4,2)$ & NO & 1891\\
$(44,13)$ & 8 & $(9,2)$ & 5 & 1 & YES & YES & YES & $1.38$ & $(2,3)$ & -- & 1892\\
$(44,17)$ & 8 & $(25,7)$ & 7 & 1 & YES & YES & YES & $1.62$ & $(2,3)$ & -- & 1893\\
$(44,13)$ & 8 & $(31,12)$ & 7 & 1 & YES & YES & YES & $1.43$ & $(2,3)$ & -- & 1894\\
$(45,13)$ & 10 & $(10,3)$ & 5 & 5 & YES & YES & YES & $1.62$ & $(2,3)$ & -- & 1895\\
$(45,13)$ & 10 & $(12,5)$ & 5 & 3 & YES & YES & YES & $1.57$ & $(2,3)$ & NO & 1896\\
$(45,13)$ & 10 & $(12,5)$ & 5 & 3 & YES & YES & YES & $1.57$ & $(2,3)$ & -- & 1897\\
$(45,19)$ & 8 & $(13,4)$ & 6 & 1 & YES & YES & YES & $1.57$ & $(2,3)$ & NO & 1898\\
$(45,19)$ & 8 & $(13,4)$ & 6 & 1 & YES & YES & YES & $1.57$ & $(2,3)$ & -- & 1899\\
$(45,19)$ & 8 & $(16,5)$ & 7 & 1 & YES & YES & YES & $1.57$ & $(2,3)$ & -- & 1900\\
$(45,14)$ & 9 & $(17,5)$ & 6 & 1 & YES & YES & YES & $1.43$ & $(2,3)$ & -- & 1901\\
$(45,8)$ & 9 & $(23,9)$ & 7 & 1 & YES & YES & YES & $1.43$ & $(2,3)$ & NO & 1902\\
$(45,8)$ & 9 & $(40,9)$ & 9 & 5 & YES & YES & YES & $1.38$ & $(2,3)$ & NO & 1903\\
$(45,17)$ & 9 & $(41,16)$ & 8 & 1 & YES & YES & YES & $1.57$ & $(2,3)$ & NO & 1904\\
$(46,13)$ & 10 & $(10,3)$ & 5 & 2 & YES & YES & YES & $1.62$ & $(2,3)$ & -- & 1905\\
$(46,19)$ & 8 & $(12,5)$ & 5 & 2 & YES & YES & YES & $1.29$ & $(2,3)$ & -- & 1906\\
$(46,13)$ & 10 & $(13,5)$ & 5 & 1 & YES & YES & YES & $1.29$ & $(4,2)$ & -- & 1907\\
$(46,19)$ & 8 & $(13,5)$ & 5 & 1 & YES & YES & YES & $1.29$ & $(4,2)$ & -- & 1908\\
$(46,7)$ & 10 & $(20,7)$ & 8 & 2 & YES & YES & YES & $1.29$ & $(4,2)$ & NO & 1909\\
$(46,19)$ & 8 & $(24,7)$ & 7 & 2 & YES & YES & YES & $1.57$ & $(2,3)$ & -- & 1910\\
$(46,19)$ & 8 & $(25,7)$ & 7 & 1 & YES & YES & YES & $1.43$ & $(4,2)$ & -- & 1911\\
$(47,14)$ & 9 & $(13,3)$ & 6 & 1 & YES & YES & YES & $1.62$ & $(2,3)$ & NO & 1912\\
$(47,14)$ & 9 & $(13,3)$ & 6 & 1 & YES & YES & YES & $1.62$ & $(2,3)$ & -- & 1913\\
$(47,14)$ & 9 & $(13,4)$ & 6 & 1 & YES & YES & YES & $1.62$ & $(2,3)$ & -- & 1914\\
$(47,18)$ & 8 & $(22,5)$ & 7 & 1 & YES & YES & YES & $1.38$ & $(2,3)$ & -- & 1915\\
$(47,14)$ & 9 & $(24,7)$ & 7 & 1 & YES & YES & YES & $1.57$ & $(2,3)$ & -- & 1916\\
$(47,14)$ & 9 & $(25,7)$ & 7 & 1 & YES & YES & YES & $1.86$ & $(2,3)$ & -- & 1917\\
$(47,18)$ & 8 & $(31,7)$ & 8 & 1 & YES & YES & YES & $1.75$ & $(2,3)$ & -- & 1918\\
$(48,17)$ & 9 & $(8,3)$ & 4 & 8 & YES & YES & YES & $1.29$ & $(4,2)$ & -- & 1919\\
$(48,17)$ & 9 & $(17,5)$ & 6 & 1 & YES & YES & YES & $1.57$ & $(2,3)$ & -- & 1920\\
$(48,11)$ & 9 & $(19,6)$ & 8 & 1 & YES & YES & YES & $1.57$ & $(2,3)$ & NO & 1921\\
$(48,17)$ & 9 & $(36,13)$ & 8 & 12 & YES & YES & YES & $1.29$ & $(4,2)$ & NO & 1922\\
$(49,18)$ & 8 & $(8,3)$ & 4 & 1 & YES & YES & YES & $1.14$ & $(4,2)$ & -- & 1923\\
$(49,22)$ & 9 & $(8,3)$ & 4 & 1 & YES & YES & YES & $1.29$ & $(4,2)$ & -- & 1924\\
$(49,13)$ & 9 & $(11,4)$ & 5 & 1 & YES & YES & YES & $1.43$ & $(4,2)$ & -- & 1925\\
$(49,15)$ & 9 & $(13,4)$ & 6 & 1 & YES & YES & YES & $1.75$ & $(2,3)$ & -- & 1926\\
$(49,18)$ & 8 & $(13,4)$ & 6 & 1 & YES & YES & YES & $1.62$ & $(2,3)$ & NO & 1927\\
$(49,18)$ & 8 & $(13,4)$ & 6 & 1 & YES & YES & YES & $1.62$ & $(2,3)$ & -- & 1928\\
$(49,13)$ & 9 & $(17,7)$ & 6 & 1 & YES & YES & YES & $1.43$ & $(2,3)$ & -- & 1929\\
$(49,19)$ & 8 & $(26,11)$ & 7 & 1 & YES & YES & YES & $1.62$ & $(2,3)$ & NO & 1930\\
$(49,22)$ & 9 & $(42,19)$ & 9 & 7 & YES & YES & YES & $1.29$ & $(4,2)$ & NO & 1931\\
$(49,19)$ & 8 & $(45,17)$ & 9 & 1 & YES & YES & YES & $1.43$ & $(2,3)$ & NO & 1932\\
$(50,13)$ & 10 & $(11,4)$ & 5 & 1 & YES & YES & YES & $1.29$ & $(4,2)$ & -- & 1933\\
$(50,21)$ & 8 & $(11,4)$ & 5 & 1 & YES & YES & YES & $1.43$ & $(2,3)$ & -- & 1934\\
$(50,21)$ & 8 & $(11,4)$ & 5 & 1 & YES & YES & YES & $1.62$ & $(2,3)$ & NO & 1935\\
$(50,21)$ & 8 & $(12,5)$ & 5 & 2 & YES & YES & YES & $1.62$ & $(2,3)$ & -- & 1936\\
$(50,21)$ & 8 & $(24,7)$ & 7 & 2 & YES & YES & YES & $1.71$ & $(2,3)$ & -- & 1937\\
$(50,21)$ & 8 & $(39,17)$ & 8 & 1 & YES & YES & YES & $1.62$ & $(2,3)$ & NO & 1938\\
$(51,16)$ & 10 & $(8,3)$ & 4 & 1 & YES & YES & YES & $1.43$ & $(4,2)$ & -- & 1939\\
$(51,20)$ & 9 & $(8,3)$ & 4 & 1 & YES & YES & YES & $1.14$ & $(4,2)$ & -- & 1940\\
$(51,16)$ & 10 & $(13,5)$ & 5 & 1 & YES & YES & YES & $1.29$ & $(4,2)$ & -- & 1941\\
$(51,20)$ & 9 & $(17,4)$ & 7 & 17 & YES & YES & YES & $1.57$ & $(2,3)$ & NO & 1942\\
$(51,23)$ & 9 & $(17,4)$ & 7 & 17 & YES & YES & YES & $1.43$ & $(2,3)$ & NO & 1943\\
$(51,23)$ & 9 & $(22,9)$ & 7 & 1 & YES & YES & YES & $1.43$ & $(2,3)$ & NO & 1944\\
$(51,11)$ & 9 & $(23,10)$ & 7 & 1 & YES & YES & YES & $1.43$ & $(4,2)$ & -- & 1945\\
$(51,11)$ & 9 & $(27,10)$ & 7 & 3 & YES & YES & YES & $1.29$ & $(4,2)$ & -- & 1946\\
$(51,14)$ & 9 & $(29,9)$ & 8 & 1 & YES & YES & YES & $1.62$ & $(2,3)$ & NO & 1947\\
$(51,20)$ & 9 & $(37,14)$ & 8 & 1 & YES & YES & YES & $1.57$ & $(2,3)$ & NO & 1948\\
$(52,11)$ & 9 & $(19,6)$ & 8 & 1 & YES & YES & YES & $1.50$ & $(2,3)$ & NO & 1949\\
$(53,19)$ & 9 & $(8,3)$ & 4 & 1 & YES & YES & YES & $1.14$ & $(4,2)$ & -- & 1950\\
$(53,19)$ & 9 & $(13,4)$ & 6 & 1 & YES & YES & YES & $1.57$ & $(2,3)$ & -- & 1951\\
$(53,23)$ & 9 & $(13,4)$ & 6 & 1 & YES & YES & YES & $1.43$ & $(2,3)$ & -- & 1952\\
$(53,20)$ & 10 & $(16,3)$ & 7 & 1 & YES & YES & YES & $1.50$ & $(2,3)$ & -- & 1953\\
$(53,20)$ & 10 & $(16,3)$ & 7 & 1 & YES & YES & YES & $1.62$ & $(2,3)$ & NO & 1954\\
$(53,14)$ & 9 & $(18,5)$ & 6 & 1 & YES & YES & YES & $1.43$ & $(2,3)$ & -- & 1955\\
$(53,22)$ & 9 & $(20,9)$ & 7 & 1 & YES & YES & YES & $1.57$ & $(2,3)$ & NO & 1956\\
$(53,19)$ & 9 & $(23,9)$ & 7 & 1 & YES & YES & YES & $1.57$ & $(2,3)$ & NO & 1957\\
$(53,14)$ & 9 & $(27,8)$ & 7 & 1 & YES & YES & YES & $1.50$ & $(2,3)$ & NO & 1958\\
$(53,16)$ & 10 & $(29,8)$ & 7 & 1 & YES & YES & YES & $1.43$ & $(2,3)$ & NO & 1959\\
$(55,21)$ & 8 & $(7,2)$ & 4 & 1 & YES & YES & YES & $1.50$ & $(2,3)$ & NO & 1960\\
$(55,23)$ & 9 & $(8,3)$ & 4 & 1 & YES & YES & YES & $1.62$ & $(2,3)$ & -- & 1961\\
$(55,24)$ & 9 & $(11,4)$ & 5 & 11 & YES & YES & YES & $1.43$ & $(4,2)$ & NO & 1962\\
$(55,23)$ & 9 & $(12,5)$ & 5 & 1 & YES & YES & YES & $1.43$ & $(2,3)$ & -- & 1963\\
$(55,16)$ & 9 & $(18,7)$ & 6 & 1 & YES & YES & YES & $1.29$ & $(4,2)$ & -- & 1964\\
$(55,23)$ & 9 & $(20,9)$ & 7 & 5 & YES & YES & YES & $1.43$ & $(2,3)$ & NO & 1965\\
$(55,16)$ & 9 & $(23,7)$ & 7 & 1 & YES & YES & YES & $1.50$ & $(2,3)$ & NO & 1966\\
$(55,21)$ & 8 & $(24,7)$ & 7 & 1 & YES & YES & YES & $1.86$ & $(2,3)$ & -- & 1967\\
$(55,13)$ & 10 & $(26,7)$ & 7 & 1 & YES & YES & YES & $1.50$ & $(2,3)$ & -- & 1968\\
$(55,16)$ & 9 & $(27,8)$ & 7 & 1 & YES & YES & YES & $1.50$ & $(2,3)$ & NO & 1969\\
$(55,17)$ & 10 & $(27,8)$ & 7 & 1 & YES & YES & YES & $1.57$ & $(2,3)$ & NO & 1970\\
$(55,21)$ & 8 & $(33,13)$ & 9 & 11 & YES & YES & YES & $1.43$ & $(4,2)$ & NO & 1971\\
$(55,21)$ & 8 & $(41,16)$ & 8 & 1 & YES & YES & YES & $1.50$ & $(2,3)$ & NO & 1972\\
$(55,13)$ & 10 & $(51,11)$ & 9 & 1 & YES & YES & YES & $1.43$ & $(4,2)$ & 2380 & 1973\\
$(55,21)$ & 8 & $(53,20)$ & 10 & 1 & YES & YES & YES & $1.43$ & $(4,2)$ & NO & 1974\\
$(56,17)$ & 9 & $(8,3)$ & 4 & 8 & YES & YES & YES & $1.14$ & $(4,2)$ & -- & 1975\\
$(56,23)$ & 9 & $(10,3)$ & 5 & 2 & YES & YES & YES & $1.75$ & $(2,3)$ & NO & 1976\\
$(56,23)$ & 9 & $(10,3)$ & 5 & 2 & YES & YES & YES & $1.75$ & $(2,3)$ & -- & 1977\\
$(56,13)$ & 10 & $(11,4)$ & 5 & 1 & YES & YES & YES & $1.38$ & $(2,3)$ & -- & 1978\\
$(56,15)$ & 9 & $(19,6)$ & 8 & 1 & YES & YES & YES & $1.57$ & $(2,3)$ & NO & 1979\\
$(56,23)$ & 9 & $(20,7)$ & 8 & 4 & YES & YES & YES & $1.57$ & $(4,2)$ & 2184 & 1980\\
$(57,26)$ & 11 & $(12,5)$ & 5 & 3 & YES & YES & YES & $1.57$ & $(4,2)$ & -- & 1981\\
$(57,26)$ & 11 & $(13,5)$ & 5 & 1 & YES & YES & YES & $1.57$ & $(4,2)$ & -- & 1982\\
$(57,13)$ & 9 & $(14,3)$ & 6 & 1 & YES & YES & YES & $1.38$ & $(2,3)$ & NO & 1983\\
$(57,13)$ & 9 & $(17,6)$ & 7 & 1 & YES & YES & YES & $1.43$ & $(2,3)$ & NO & 1984\\
$(57,13)$ & 9 & $(18,7)$ & 6 & 3 & YES & YES & YES & $1.38$ & $(2,3)$ & -- & 1985\\
$(57,16)$ & 9 & $(26,7)$ & 7 & 1 & YES & YES & YES & $1.50$ & $(2,3)$ & NO & 1986\\
$(57,16)$ & 9 & $(29,8)$ & 7 & 1 & YES & YES & YES & $1.50$ & $(2,3)$ & NO & 1987\\
$(58,17)$ & 9 & $(16,7)$ & 6 & 2 & YES & YES & YES & $1.50$ & $(2,3)$ & -- & 1988\\
$(58,13)$ & 11 & $(23,4)$ & 8 & 1 & YES & YES & YES & $1.50$ & $(2,3)$ & -- & 1989\\
$(58,17)$ & 9 & $(31,7)$ & 8 & 1 & YES & YES & YES & $1.43$ & $(4,2)$ & NO & 1990\\
$(58,17)$ & 9 & $(43,13)$ & 9 & 1 & YES & YES & YES & $1.43$ & $(2,3)$ & NO & 1991\\
$(59,23)$ & 9 & $(10,3)$ & 5 & 1 & YES & YES & YES & $1.14$ & $(4,2)$ & -- & 1992\\
$(59,18)$ & 9 & $(11,4)$ & 5 & 1 & YES & YES & YES & $1.43$ & $(2,3)$ & -- & 1993\\
$(59,25)$ & 9 & $(13,4)$ & 6 & 1 & YES & YES & YES & $1.57$ & $(2,3)$ & -- & 1994\\
$(59,26)$ & 9 & $(17,4)$ & 7 & 1 & YES & YES & YES & $1.43$ & $(2,3)$ & NO & 1995\\
$(59,14)$ & 10 & $(23,7)$ & 7 & 1 & YES & YES & YES & $1.62$ & $(2,3)$ & -- & 1996\\
$(59,14)$ & 10 & $(26,7)$ & 7 & 1 & YES & YES & YES & $1.50$ & $(2,3)$ & -- & 1997\\
$(59,25)$ & 9 & $(31,13)$ & 7 & 1 & YES & YES & YES & $1.50$ & $(2,3)$ & NO & 1998\\
$(59,23)$ & 9 & $(33,13)$ & 9 & 1 & YES & YES & YES & $1.29$ & $(4,2)$ & NO & 1999\\
$(60,13)$ & 9 & $(13,4)$ & 6 & 1 & YES & YES & YES & $1.43$ & $(2,3)$ & NO & 2000\\
$(60,13)$ & 9 & $(13,4)$ & 6 & 1 & YES & YES & YES & $1.43$ & $(2,3)$ & -- & 2001\\
$(60,13)$ & 9 & $(18,7)$ & 6 & 6 & YES & YES & YES & $1.43$ & $(2,3)$ & NO & 2002\\
$(60,13)$ & 9 & $(19,6)$ & 8 & 1 & YES & YES & YES & $1.43$ & $(2,3)$ & NO & 2003\\
$(60,13)$ & 9 & $(31,9)$ & 8 & 1 & YES & YES & YES & $1.43$ & $(2,3)$ & NO & 2004\\
$(61,16)$ & 10 & $(12,5)$ & 5 & 1 & YES & YES & YES & $1.43$ & $(4,2)$ & NO & 2005\\
$(61,16)$ & 10 & $(13,5)$ & 5 & 1 & YES & YES & YES & $1.43$ & $(4,2)$ & -- & 2006\\
$(61,16)$ & 10 & $(13,5)$ & 5 & 1 & YES & YES & YES & $1.43$ & $(4,2)$ & NO & 2007\\
$(61,17)$ & 9 & $(18,7)$ & 6 & 1 & YES & YES & YES & $1.62$ & $(2,3)$ & -- & 2008\\
$(61,22)$ & 9 & $(53,19)$ & 9 & 1 & YES & YES & YES & $1.14$ & $(4,2)$ & NO & 2009\\
$(62,27)$ & 9 & $(11,4)$ & 5 & 1 & YES & YES & YES & $1.43$ & $(2,3)$ & -- & 2010\\
$(62,17)$ & 10 & $(12,5)$ & 5 & 2 & YES & YES & YES & $1.43$ & $(2,3)$ & -- & 2011\\
$(62,13)$ & 10 & $(14,5)$ & 6 & 2 & YES & YES & YES & $1.43$ & $(4,2)$ & -- & 2012\\
$(62,13)$ & 10 & $(18,7)$ & 6 & 2 & YES & YES & YES & $1.43$ & $(4,2)$ & -- & 2013\\
$(62,17)$ & 10 & $(23,7)$ & 7 & 1 & YES & YES & YES & $1.43$ & $(2,3)$ & NO & 2014\\
$(63,17)$ & 9 & $(44,13)$ & 8 & 1 & YES & YES & YES & $1.57$ & $(2,3)$ & NO & 2015\\
$(64,23)$ & 9 & $(5,2)$ & 3 & 1 & YES & YES & YES & $1.38$ & $(2,3)$ & NO & 2016\\
$(64,23)$ & 9 & $(5,2)$ & 3 & 1 & YES & YES & YES & $1.38$ & $(2,3)$ & -- & 2017\\
$(64,23)$ & 9 & $(7,3)$ & 4 & 1 & YES & YES & YES & $1.38$ & $(2,3)$ & NO & 2018\\
$(64,27)$ & 9 & $(12,5)$ & 5 & 4 & YES & YES & YES & $1.57$ & $(2,3)$ & NO & 2019\\
$(64,19)$ & 9 & $(13,3)$ & 6 & 1 & YES & YES & YES & $1.57$ & $(2,3)$ & -- & 2020\\
$(64,19)$ & 9 & $(18,7)$ & 6 & 2 & YES & YES & YES & $1.62$ & $(2,3)$ & -- & 2021\\
$(64,19)$ & 9 & $(24,7)$ & 7 & 8 & YES & YES & YES & $1.50$ & $(2,3)$ & -- & 2022\\
$(65,19)$ & 9 & $(5,1)$ & 4 & 5 & YES & YES & YES & $1.50$ & $(2,3)$ & NO & 2023\\
$(65,19)$ & 9 & $(5,1)$ & 4 & 5 & YES & YES & YES & $1.50$ & $(2,3)$ & -- & 2024\\
$(65,19)$ & 9 & $(5,1)$ & 4 & 5 & YES & YES & YES & $1.50$ & $(2,3)$ & NO & 2025\\
$(65,19)$ & 9 & $(7,2)$ & 4 & 1 & YES & YES & YES & $1.50$ & $(2,3)$ & -- & 2026\\
$(65,24)$ & 9 & $(9,4)$ & 5 & 1 & YES & YES & YES & $1.43$ & $(4,2)$ & NO & 2027\\
$(65,24)$ & 9 & $(10,3)$ & 5 & 5 & YES & YES & YES & $1.62$ & $(2,3)$ & -- & 2028\\
$(65,18)$ & 9 & $(11,5)$ & 6 & 1 & YES & YES & YES & $1.43$ & $(4,2)$ & NO & 2029\\
$(65,19)$ & 9 & $(11,5)$ & 6 & 1 & YES & YES & YES & $1.50$ & $(2,3)$ & -- & 2030\\
$(65,14)$ & 10 & $(13,4)$ & 6 & 13 & YES & YES & YES & $1.43$ & $(2,3)$ & -- & 2031\\
$(65,18)$ & 9 & $(14,5)$ & 6 & 1 & YES & YES & YES & $1.43$ & $(4,2)$ & NO & 2032\\
$(65,27)$ & 10 & $(14,3)$ & 6 & 1 & YES & YES & YES & $1.50$ & $(2,3)$ & -- & 2033\\
$(65,14)$ & 10 & $(21,4)$ & 8 & 1 & YES & YES & YES & $1.29$ & $(2,3)$ & -- & 2034\\
$(66,29)$ & 9 & $(7,2)$ & 4 & 1 & YES & YES & YES & $1.43$ & $(2,3)$ & -- & 2035\\
$(66,29)$ & 9 & $(11,3)$ & 5 & 11 & YES & YES & YES & $1.62$ & $(2,3)$ & -- & 2036\\
$(66,29)$ & 9 & $(11,4)$ & 5 & 11 & YES & YES & YES & $1.62$ & $(2,3)$ & NO & 2037\\
$(66,29)$ & 9 & $(11,4)$ & 5 & 11 & YES & YES & YES & $1.62$ & $(2,3)$ & -- & 2038\\
$(66,25)$ & 9 & $(28,11)$ & 8 & 2 & YES & YES & YES & $1.57$ & $(2,3)$ & NO & 2039\\
$(67,18)$ & 9 & $(7,2)$ & 4 & 1 & YES & YES & YES & $1.38$ & $(2,3)$ & NO & 2040\\
$(67,18)$ & 9 & $(7,2)$ & 4 & 1 & YES & YES & YES & $1.38$ & $(2,3)$ & -- & 2041\\
$(67,26)$ & 9 & $(9,4)$ & 5 & 1 & YES & YES & YES & $1.43$ & $(4,2)$ & -- & 2042\\
$(67,20)$ & 11 & $(11,3)$ & 5 & 1 & YES & YES & YES & $1.57$ & $(2,3)$ & -- & 2043\\
$(67,28)$ & 10 & $(14,3)$ & 6 & 1 & YES & YES & YES & $1.50$ & $(2,3)$ & -- & 2044\\
$(67,20)$ & 11 & $(16,3)$ & 7 & 1 & YES & YES & YES & $1.57$ & $(2,3)$ & -- & 2045\\
$(67,28)$ & 10 & $(16,3)$ & 7 & 1 & YES & YES & YES & $1.50$ & $(2,3)$ & -- & 2046\\
$(67,20)$ & 11 & $(17,3)$ & 7 & 1 & YES & YES & YES & $1.57$ & $(2,3)$ & -- & 2047\\
$(67,20)$ & 11 & $(17,3)$ & 7 & 1 & YES & YES & YES & $1.57$ & $(2,3)$ & NO & 2048\\
$(67,18)$ & 9 & $(18,5)$ & 6 & 1 & YES & YES & YES & $1.38$ & $(2,3)$ & NO & 2049\\
$(68,19)$ & 9 & $(3,1)$ & 2 & 1 & YES & YES & YES & $1.50$ & $(2,3)$ & -- & 2050\\
$(68,19)$ & 9 & $(3,1)$ & 2 & 1 & YES & YES & YES & $1.50$ & $(2,3)$ & NO & 2051\\
$(68,19)$ & 9 & $(7,2)$ & 4 & 1 & YES & YES & YES & $1.50$ & $(2,3)$ & -- & 2052\\
$(68,21)$ & 11 & $(11,5)$ & 6 & 1 & YES & YES & YES & $1.71$ & $(2,3)$ & NO & 2053\\
$(68,19)$ & 9 & $(15,4)$ & 6 & 1 & YES & YES & YES & $1.50$ & $(2,3)$ & 2304 & 2054\\
$(68,19)$ & 9 & $(19,6)$ & 8 & 1 & YES & YES & YES & $1.50$ & $(2,3)$ & NO & 2055\\
$(68,19)$ & 9 & $(27,8)$ & 7 & 1 & YES & YES & YES & $1.50$ & $(2,3)$ & NO & 2056\\
$(68,25)$ & 9 & $(31,11)$ & 8 & 1 & YES & YES & YES & $1.43$ & $(4,2)$ & NO & 2057\\
$(69,31)$ & 10 & $(5,2)$ & 3 & 1 & YES & YES & YES & $1.29$ & $(4,2)$ & -- & 2058\\
$(69,29)$ & 9 & $(8,3)$ & 4 & 1 & YES & YES & YES & $1.57$ & $(2,3)$ & NO & 2059\\
$(69,29)$ & 9 & $(8,3)$ & 4 & 1 & YES & YES & YES & $1.57$ & $(2,3)$ & -- & 2060\\
$(69,31)$ & 10 & $(10,3)$ & 5 & 1 & YES & YES & YES & $1.29$ & $(4,2)$ & -- & 2061\\
$(69,25)$ & 11 & $(12,5)$ & 5 & 3 & YES & YES & YES & $1.71$ & $(2,3)$ & -- & 2062\\
$(69,29)$ & 9 & $(23,5)$ & 7 & 23 & YES & YES & YES & $1.71$ & $(2,3)$ & NO & 2063\\
$(69,29)$ & 9 & $(23,5)$ & 7 & 23 & YES & YES & YES & $1.71$ & $(2,3)$ & -- & 2064\\
$(69,16)$ & 11 & $(25,4)$ & 9 & 1 & YES & YES & YES & $1.43$ & $(2,3)$ & -- & 2065\\
$(69,19)$ & 9 & $(44,13)$ & 8 & 1 & YES & YES & YES & $1.71$ & $(2,3)$ & NO & 2066\\
$(69,20)$ & 10 & $(47,14)$ & 9 & 1 & YES & YES & YES & $1.43$ & $(2,3)$ & NO & 2067\\
$(70,27)$ & 10 & $(20,3)$ & 8 & 10 & YES & YES & YES & $1.43$ & $(4,2)$ & -- & 2068\\
$(71,26)$ & 9 & $(5,2)$ & 3 & 1 & YES & YES & YES & $1.29$ & $(4,2)$ & -- & 2069\\
$(71,21)$ & 9 & $(8,3)$ & 4 & 1 & YES & YES & YES & $1.38$ & $(2,3)$ & NO & 2070\\
$(71,27)$ & 9 & $(8,3)$ & 4 & 1 & YES & YES & YES & $1.62$ & $(2,3)$ & NO & 2071\\
$(71,27)$ & 9 & $(8,3)$ & 4 & 1 & YES & YES & YES & $1.62$ & $(2,3)$ & -- & 2072\\
$(71,21)$ & 9 & $(9,4)$ & 5 & 1 & YES & YES & YES & $1.62$ & $(2,3)$ & NO & 2073\\
$(71,21)$ & 9 & $(9,4)$ & 5 & 1 & YES & YES & YES & $1.62$ & $(2,3)$ & -- & 2074\\
$(71,21)$ & 9 & $(11,5)$ & 6 & 1 & YES & YES & YES & $1.57$ & $(2,3)$ & -- & 2075\\
$(71,30)$ & 9 & $(13,4)$ & 6 & 1 & YES & YES & YES & $1.43$ & $(2,3)$ & NO & 2076\\
$(71,30)$ & 9 & $(22,5)$ & 7 & 1 & YES & YES & YES & $1.57$ & $(2,3)$ & NO & 2077\\
$(71,30)$ & 9 & $(25,11)$ & 7 & 1 & YES & YES & YES & $1.50$ & $(2,3)$ & NO & 2078\\
$(71,30)$ & 9 & $(27,11)$ & 8 & 1 & YES & YES & YES & $1.43$ & $(2,3)$ & NO & 2079\\
$(71,21)$ & 9 & $(33,10)$ & 8 & 1 & YES & YES & YES & $1.62$ & $(2,3)$ & NO & 2080\\
$(71,26)$ & 9 & $(49,18)$ & 8 & 1 & YES & YES & YES & $1.14$ & $(4,2)$ & NO & 2081\\
$(71,21)$ & 9 & $(67,20)$ & 11 & 1 & YES & YES & YES & $1.57$ & $(2,3)$ & NO & 2082\\
$(73,27)$ & 9 & $(5,2)$ & 3 & 1 & YES & YES & YES & $1.50$ & $(2,3)$ & NO & 2083\\
$(73,27)$ & 9 & $(5,2)$ & 3 & 1 & YES & YES & YES & $1.50$ & $(2,3)$ & -- & 2084\\
$(73,27)$ & 9 & $(7,3)$ & 4 & 1 & YES & YES & YES & $1.50$ & $(2,3)$ & NO & 2085\\
$(73,28)$ & 10 & $(8,3)$ & 4 & 1 & YES & YES & YES & $1.57$ & $(2,3)$ & -- & 2086\\
$(73,26)$ & 11 & $(9,4)$ & 5 & 1 & YES & YES & YES & $1.71$ & $(2,3)$ & -- & 2087\\
$(73,20)$ & 11 & $(11,3)$ & 5 & 1 & YES & YES & YES & $1.57$ & $(2,3)$ & -- & 2088\\
$(73,26)$ & 11 & $(11,5)$ & 6 & 1 & YES & YES & YES & $1.71$ & $(2,3)$ & NO & 2089\\
$(73,20)$ & 11 & $(17,3)$ & 7 & 1 & YES & YES & YES & $1.57$ & $(2,3)$ & NO & 2090\\
$(73,27)$ & 9 & $(30,11)$ & 7 & 1 & YES & YES & YES & $1.29$ & $(4,2)$ & NO & 2091\\
$(74,31)$ & 9 & $(8,3)$ & 4 & 2 & YES & YES & YES & $1.62$ & $(2,3)$ & -- & 2092\\
$(74,23)$ & 10 & $(10,3)$ & 5 & 2 & YES & YES & YES & $1.43$ & $(2,3)$ & -- & 2093\\
$(74,29)$ & 10 & $(10,3)$ & 5 & 2 & YES & YES & YES & $1.62$ & $(2,3)$ & -- & 2094\\
$(74,31)$ & 9 & $(10,3)$ & 5 & 2 & YES & YES & YES & $1.62$ & $(2,3)$ & NO & 2095\\
$(74,31)$ & 9 & $(10,3)$ & 5 & 2 & YES & YES & YES & $1.62$ & $(2,3)$ & -- & 2096\\
$(74,29)$ & 10 & $(29,11)$ & 7 & 1 & YES & YES & YES & $1.62$ & $(2,3)$ & NO & 2097\\
$(74,29)$ & 10 & $(33,13)$ & 9 & 1 & YES & YES & YES & $1.29$ & $(4,2)$ & 2379 & 2098\\
$(74,29)$ & 10 & $(49,19)$ & 8 & 1 & YES & YES & YES & $1.62$ & $(2,3)$ & NO & 2099\\
$(75,31)$ & 9 & $(4,1)$ & 3 & 1 & YES & YES & YES & $1.50$ & $(2,3)$ & NO & 2100\\
$(75,31)$ & 9 & $(4,1)$ & 3 & 1 & YES & YES & YES & $1.50$ & $(2,3)$ & -- & 2101\\
$(75,31)$ & 9 & $(4,1)$ & 3 & 1 & YES & YES & YES & $1.50$ & $(2,3)$ & NO & 2102\\
$(75,31)$ & 9 & $(8,3)$ & 4 & 1 & YES & YES & YES & $1.50$ & $(2,3)$ & NO & 2103\\
$(75,22)$ & 10 & $(11,4)$ & 5 & 1 & YES & YES & YES & $1.62$ & $(2,3)$ & NO & 2104\\
$(75,31)$ & 9 & $(18,5)$ & 6 & 3 & YES & YES & YES & $1.57$ & $(2,3)$ & NO & 2105\\
$(75,31)$ & 9 & $(18,5)$ & 6 & 3 & YES & YES & YES & $1.71$ & $(2,3)$ & -- & 2106\\
$(75,17)$ & 10 & $(51,11)$ & 9 & 3 & YES & YES & YES & $1.29$ & $(4,2)$ & NO & 2107\\
$(76,29)$ & 9 & $(7,2)$ & 4 & 1 & YES & YES & YES & $1.62$ & $(2,3)$ & -- & 2108\\
$(76,29)$ & 9 & $(9,2)$ & 5 & 1 & YES & YES & YES & $1.50$ & $(2,3)$ & NO & 2109\\
$(76,29)$ & 9 & $(9,2)$ & 5 & 1 & YES & YES & YES & $1.50$ & $(2,3)$ & -- & 2110\\
$(76,29)$ & 9 & $(9,4)$ & 5 & 1 & YES & YES & YES & $1.29$ & $(4,2)$ & -- & 2111\\
$(76,21)$ & 9 & $(11,5)$ & 6 & 1 & YES & YES & YES & $1.43$ & $(2,3)$ & -- & 2112\\
$(76,21)$ & 9 & $(11,5)$ & 6 & 1 & YES & YES & YES & $1.57$ & $(2,3)$ & NO & 2113\\
$(76,29)$ & 9 & $(11,5)$ & 6 & 1 & YES & YES & YES & $1.43$ & $(4,2)$ & NO & 2114\\
$(76,21)$ & 9 & $(14,5)$ & 6 & 2 & YES & YES & YES & $1.57$ & $(2,3)$ & NO & 2115\\
$(76,21)$ & 9 & $(19,6)$ & 8 & 19 & YES & YES & YES & $1.43$ & $(2,3)$ & NO & 2116\\
$(76,29)$ & 9 & $(19,7)$ & 6 & 19 & YES & YES & YES & $1.57$ & $(2,3)$ & NO & 2117\\
$(76,21)$ & 9 & $(73,20)$ & 11 & 1 & YES & YES & YES & $1.71$ & $(2,3)$ & NO & 2118\\
$(78,17)$ & 10 & $(43,10)$ & 9 & 1 & YES & YES & YES & $1.43$ & $(2,3)$ & NO & 2119\\
$(79,30)$ & 9 & $(3,1)$ & 2 & 1 & YES & YES & YES & $1.50$ & $(2,3)$ & NO & 2120\\
$(79,23)$ & 10 & $(4,1)$ & 3 & 1 & YES & YES & YES & $1.14$ & $(4,2)$ & NO & 2121\\
$(79,23)$ & 10 & $(4,1)$ & 3 & 1 & YES & YES & YES & $1.14$ & $(4,2)$ & -- & 2122\\
$(79,24)$ & 10 & $(5,2)$ & 3 & 1 & YES & YES & YES & $1.43$ & $(4,2)$ & NO & 2123\\
$(79,24)$ & 10 & $(5,2)$ & 3 & 1 & YES & YES & YES & $1.43$ & $(4,2)$ & -- & 2124\\
$(79,29)$ & 9 & $(5,2)$ & 3 & 1 & YES & YES & YES & $1.29$ & $(4,2)$ & -- & 2125\\
$(79,29)$ & 9 & $(7,2)$ & 4 & 1 & YES & YES & YES & $1.50$ & $(2,3)$ & -- & 2126\\
$(79,29)$ & 9 & $(7,3)$ & 4 & 1 & YES & YES & YES & $1.38$ & $(2,3)$ & 2385 & 2127\\
$(79,18)$ & 10 & $(9,4)$ & 5 & 1 & YES & YES & YES & $1.38$ & $(2,3)$ & -- & 2128\\
$(79,29)$ & 9 & $(9,4)$ & 5 & 1 & YES & YES & YES & $1.50$ & $(2,3)$ & -- & 2129\\
$(79,17)$ & 11 & $(13,4)$ & 6 & 1 & YES & YES & YES & $1.43$ & $(2,3)$ & NO & 2130\\
$(79,23)$ & 10 & $(13,5)$ & 5 & 1 & YES & YES & YES & $1.43$ & $(2,3)$ & NO & 2131\\
$(79,29)$ & 9 & $(13,4)$ & 6 & 1 & YES & YES & YES & $1.43$ & $(2,3)$ & -- & 2132\\
$(79,29)$ & 9 & $(23,9)$ & 7 & 1 & YES & YES & YES & $1.43$ & $(2,3)$ & NO & 2133\\
$(79,30)$ & 9 & $(28,11)$ & 8 & 1 & YES & YES & YES & $1.43$ & $(2,3)$ & NO & 2134\\
$(79,31)$ & 10 & $(33,13)$ & 9 & 1 & YES & YES & YES & $1.29$ & $(4,2)$ & NO & 2135\\
$(79,30)$ & 9 & $(37,14)$ & 8 & 1 & YES & YES & YES & $1.29$ & $(4,2)$ & NO & 2136\\
$(79,29)$ & 9 & $(52,19)$ & 9 & 1 & YES & YES & YES & $1.14$ & $(4,2)$ & NO & 2137\\
$(79,30)$ & 9 & $(71,27)$ & 9 & 1 & YES & YES & YES & $1.50$ & $(2,3)$ & NO & 2138\\
$(80,33)$ & 10 & $(7,3)$ & 4 & 1 & YES & YES & YES & $1.57$ & $(2,3)$ & NO & 2139\\
$(80,33)$ & 10 & $(7,3)$ & 4 & 1 & YES & YES & YES & $1.57$ & $(2,3)$ & -- & 2140\\
$(80,31)$ & 9 & $(9,4)$ & 5 & 1 & YES & YES & YES & $1.43$ & $(2,3)$ & -- & 2141\\
$(80,33)$ & 10 & $(10,3)$ & 5 & 10 & YES & YES & YES & $1.57$ & $(2,3)$ & NO & 2142\\
$(80,33)$ & 10 & $(10,3)$ & 5 & 10 & YES & YES & YES & $1.57$ & $(2,3)$ & -- & 2143\\
$(80,33)$ & 10 & $(11,5)$ & 6 & 1 & YES & YES & YES & $1.57$ & $(2,3)$ & NO & 2144\\
$(80,31)$ & 9 & $(12,5)$ & 5 & 4 & YES & YES & YES & $1.71$ & $(2,3)$ & -- & 2145\\
$(80,33)$ & 10 & $(26,11)$ & 7 & 2 & YES & YES & YES & $1.57$ & $(2,3)$ & NO & 2146\\
$(81,34)$ & 9 & $(5,2)$ & 3 & 1 & YES & YES & YES & $1.57$ & $(2,3)$ & NO & 2147\\
$(81,34)$ & 9 & $(5,2)$ & 3 & 1 & YES & YES & YES & $1.50$ & $(2,3)$ & -- & 2148\\
$(81,34)$ & 9 & $(8,3)$ & 4 & 1 & YES & YES & YES & $1.50$ & $(2,3)$ & NO & 2149\\
$(81,34)$ & 9 & $(8,3)$ & 4 & 1 & YES & YES & YES & $1.50$ & $(2,3)$ & -- & 2150\\
$(81,19)$ & 11 & $(13,4)$ & 6 & 1 & YES & YES & YES & $1.43$ & $(2,3)$ & NO & 2151\\
$(81,31)$ & 9 & $(13,4)$ & 6 & 1 & YES & YES & YES & $1.29$ & $(4,2)$ & NO & 2152\\
$(81,31)$ & 9 & $(13,5)$ & 5 & 1 & YES & YES & YES & $1.62$ & $(2,3)$ & -- & 2153\\
$(81,31)$ & 9 & $(17,5)$ & 6 & 1 & YES & YES & YES & $1.50$ & $(2,3)$ & -- & 2154\\
$(81,19)$ & 11 & $(25,4)$ & 9 & 1 & YES & YES & YES & $1.43$ & $(2,3)$ & NO & 2155\\
$(82,31)$ & 10 & $(16,3)$ & 7 & 2 & YES & YES & YES & $1.50$ & $(2,3)$ & NO & 2156\\
$(82,31)$ & 10 & $(16,3)$ & 7 & 2 & YES & YES & YES & $1.62$ & $(2,3)$ & -- & 2157\\
$(82,23)$ & 10 & $(17,5)$ & 6 & 1 & YES & YES & YES & $1.57$ & $(2,3)$ & -- & 2158\\
$(82,37)$ & 10 & $(19,8)$ & 6 & 1 & YES & YES & YES & $1.57$ & $(2,3)$ & NO & 2159\\
$(82,31)$ & 10 & $(31,12)$ & 7 & 1 & YES & YES & YES & $1.62$ & $(2,3)$ & NO & 2160\\
$(82,37)$ & 10 & $(42,19)$ & 9 & 2 & YES & YES & YES & $1.29$ & $(4,2)$ & NO & 2161\\
$(83,35)$ & 10 & $(11,2)$ & 6 & 1 & YES & YES & YES & $1.29$ & $(2,3)$ & -- & 2162\\
$(84,25)$ & 10 & $(17,4)$ & 7 & 1 & YES & YES & YES & $1.43$ & $(2,3)$ & NO & 2163\\
$(84,25)$ & 10 & $(38,11)$ & 9 & 2 & YES & YES & YES & $1.43$ & $(2,3)$ & NO & 2164\\
$(84,37)$ & 10 & $(39,17)$ & 8 & 3 & YES & YES & YES & $1.62$ & $(2,3)$ & NO & 2165\\
$(84,25)$ & 10 & $(64,19)$ & 9 & 4 & YES & YES & YES & $1.57$ & $(2,3)$ & NO & 2166\\
$(85,33)$ & 10 & $(5,1)$ & 4 & 5 & YES & YES & YES & $1.57$ & $(4,2)$ & NO & 2167\\
$(85,33)$ & 10 & $(5,1)$ & 4 & 5 & YES & YES & YES & $1.57$ & $(4,2)$ & -- & 2168\\
$(85,36)$ & 10 & $(5,2)$ & 3 & 5 & YES & YES & YES & $1.29$ & $(4,2)$ & -- & 2169\\
$(85,36)$ & 10 & $(7,2)$ & 4 & 1 & YES & YES & YES & $1.29$ & $(4,2)$ & NO & 2170\\
$(85,36)$ & 10 & $(17,7)$ & 6 & 17 & YES & YES & YES & $1.29$ & $(4,2)$ & NO & 2171\\
$(85,37)$ & 10 & $(34,15)$ & 8 & 17 & YES & YES & YES & $1.62$ & $(2,3)$ & NO & 2172\\
$(86,35)$ & 11 & $(5,2)$ & 3 & 1 & YES & YES & YES & $1.43$ & $(4,2)$ & -- & 2173\\
$(86,31)$ & 10 & $(20,7)$ & 8 & 2 & YES & YES & YES & $1.29$ & $(4,2)$ & NO & 2174\\
$(87,19)$ & 10 & $(3,1)$ & 2 & 3 & YES & YES & YES & $1.50$ & $(2,3)$ & NO & 2175\\
$(87,19)$ & 10 & $(3,1)$ & 2 & 3 & YES & YES & YES & $1.50$ & $(2,3)$ & -- & 2176\\
$(87,20)$ & 12 & $(7,3)$ & 4 & 1 & YES & YES & YES & $1.62$ & $(2,3)$ & -- & 2177\\
$(87,31)$ & 12 & $(7,3)$ & 4 & 1 & YES & YES & YES & $1.71$ & $(2,3)$ & -- & 2178\\
$(87,31)$ & 12 & $(9,4)$ & 5 & 3 & YES & YES & YES & $1.71$ & $(2,3)$ & NO & 2179\\
$(87,23)$ & 10 & $(10,3)$ & 5 & 1 & YES & YES & YES & $1.43$ & $(2,3)$ & -- & 2180\\
$(87,20)$ & 12 & $(16,3)$ & 7 & 1 & YES & YES & YES & $1.62$ & $(2,3)$ & NO & 2181\\
$(87,20)$ & 12 & $(40,9)$ & 9 & 1 & YES & YES & YES & $1.62$ & $(2,3)$ & NO & 2182\\
$(88,31)$ & 12 & $(7,3)$ & 4 & 1 & YES & YES & YES & $1.57$ & $(4,2)$ & -- & 2183\\
$(88,31)$ & 12 & $(12,5)$ & 5 & 4 & YES & YES & YES & $1.57$ & $(4,2)$ & 1980 & 2184\\
$(89,24)$ & 10 & $(5,2)$ & 3 & 1 & YES & YES & YES & $1.43$ & $(4,2)$ & NO & 2185\\
$(89,24)$ & 10 & $(5,2)$ & 3 & 1 & YES & YES & YES & $1.43$ & $(4,2)$ & -- & 2186\\
$(89,34)$ & 9 & $(5,2)$ & 3 & 1 & YES & YES & YES & $1.38$ & $(2,3)$ & -- & 2187\\
$(89,34)$ & 9 & $(9,2)$ & 5 & 1 & YES & YES & YES & $1.50$ & $(2,3)$ & -- & 2188\\
$(89,34)$ & 9 & $(9,4)$ & 5 & 1 & YES & YES & YES & $1.43$ & $(2,3)$ & -- & 2189\\
$(89,39)$ & 11 & $(9,2)$ & 5 & 1 & YES & YES & YES & $1.62$ & $(2,3)$ & -- & 2190\\
$(89,24)$ & 10 & $(10,3)$ & 5 & 1 & YES & YES & YES & $1.43$ & $(4,2)$ & NO & 2191\\
$(89,20)$ & 11 & $(11,3)$ & 5 & 1 & YES & YES & YES & $1.38$ & $(2,3)$ & -- & 2192\\
$(89,34)$ & 9 & $(11,5)$ & 6 & 1 & YES & YES & YES & $1.57$ & $(2,3)$ & NO & 2193\\
$(89,39)$ & 11 & $(11,2)$ & 6 & 1 & YES & YES & YES & $1.62$ & $(2,3)$ & -- & 2194\\
$(89,34)$ & 9 & $(12,5)$ & 5 & 1 & YES & YES & YES & $1.50$ & $(2,3)$ & NO & 2195\\
$(89,27)$ & 10 & $(25,7)$ & 7 & 1 & YES & YES & YES & $1.43$ & $(2,3)$ & NO & 2196\\
$(89,34)$ & 9 & $(29,11)$ & 7 & 1 & YES & YES & YES & $1.29$ & $(4,2)$ & NO & 2197\\
$(89,35)$ & 11 & $(41,16)$ & 8 & 1 & YES & YES & YES & $1.29$ & $(4,2)$ & NO & 2198\\
$(89,26)$ & 10 & $(64,19)$ & 9 & 1 & YES & YES & YES & $1.62$ & $(2,3)$ & NO & 2199\\
$(89,34)$ & 9 & $(76,29)$ & 9 & 1 & YES & YES & YES & $1.50$ & $(2,3)$ & NO & 2200\\
$(91,40)$ & 10 & $(7,3)$ & 4 & 7 & YES & YES & YES & $1.62$ & $(2,3)$ & -- & 2201\\
$(91,40)$ & 10 & $(8,3)$ & 4 & 1 & YES & YES & YES & $1.57$ & $(2,3)$ & NO & 2202\\
$(91,27)$ & 10 & $(9,4)$ & 5 & 1 & YES & YES & YES & $1.57$ & $(2,3)$ & -- & 2203\\
$(91,27)$ & 10 & $(10,3)$ & 5 & 1 & YES & YES & YES & $1.50$ & $(2,3)$ & -- & 2204\\
$(91,27)$ & 10 & $(12,5)$ & 5 & 1 & YES & YES & YES & $1.71$ & $(2,3)$ & NO & 2205\\
$(91,27)$ & 10 & $(13,5)$ & 5 & 13 & YES & YES & YES & $1.71$ & $(2,3)$ & -- & 2206\\
$(92,39)$ & 10 & $(4,1)$ & 3 & 4 & YES & YES & YES & $1.50$ & $(2,3)$ & NO & 2207\\
$(92,39)$ & 10 & $(4,1)$ & 3 & 4 & YES & YES & YES & $1.50$ & $(2,3)$ & -- & 2208\\
$(92,39)$ & 10 & $(7,2)$ & 4 & 1 & YES & YES & YES & $1.50$ & $(2,3)$ & -- & 2209\\
$(92,35)$ & 10 & $(13,3)$ & 6 & 1 & YES & YES & YES & $1.71$ & $(2,3)$ & -- & 2210\\
$(92,17)$ & 11 & $(14,5)$ & 6 & 2 & YES & YES & YES & $1.62$ & $(2,3)$ & -- & 2211\\
$(92,39)$ & 10 & $(19,8)$ & 6 & 1 & YES & YES & YES & $1.50$ & $(2,3)$ & 2337 & 2212\\
$(92,27)$ & 11 & $(23,7)$ & 7 & 23 & YES & YES & YES & $1.57$ & $(2,3)$ & NO & 2213\\
$(92,39)$ & 10 & $(31,13)$ & 7 & 1 & YES & YES & YES & $1.50$ & $(2,3)$ & 2503 & 2214\\
$(92,39)$ & 10 & $(59,25)$ & 9 & 1 & YES & YES & YES & $1.50$ & $(2,3)$ & NO & 2215\\
$(93,26)$ & 10 & $(9,4)$ & 5 & 3 & YES & YES & YES & $1.29$ & $(4,2)$ & NO & 2216\\
$(94,41)$ & 10 & $(8,3)$ & 4 & 2 & YES & YES & YES & $1.62$ & $(2,3)$ & -- & 2217\\
$(95,36)$ & 10 & $(4,1)$ & 3 & 1 & YES & YES & YES & $1.29$ & $(4,2)$ & NO & 2218\\
$(95,36)$ & 10 & $(4,1)$ & 3 & 1 & YES & YES & YES & $1.29$ & $(4,2)$ & -- & 2219\\
$(95,28)$ & 11 & $(7,3)$ & 4 & 1 & YES & YES & YES & $1.57$ & $(2,3)$ & -- & 2220\\
$(95,36)$ & 10 & $(7,2)$ & 4 & 1 & YES & YES & YES & $1.29$ & $(4,2)$ & NO & 2221\\
$(95,36)$ & 10 & $(7,3)$ & 4 & 1 & YES & YES & YES & $1.62$ & $(2,3)$ & -- & 2222\\
$(95,39)$ & 10 & $(10,3)$ & 5 & 5 & YES & YES & YES & $1.50$ & $(2,3)$ & -- & 2223\\
$(95,17)$ & 11 & $(14,5)$ & 6 & 1 & YES & YES & YES & $1.43$ & $(4,2)$ & NO & 2224\\
$(95,37)$ & 11 & $(17,7)$ & 6 & 1 & YES & YES & YES & $1.57$ & $(4,2)$ & NO & 2225\\
$(95,36)$ & 10 & $(18,7)$ & 6 & 1 & YES & YES & YES & $1.43$ & $(2,3)$ & NO & 2226\\
$(95,17)$ & 11 & $(21,5)$ & 8 & 1 & YES & YES & YES & $1.43$ & $(4,2)$ & NO & 2227\\
$(97,21)$ & 10 & $(3,1)$ & 2 & 1 & YES & YES & YES & $1.38$ & $(2,3)$ & -- & 2228\\
$(97,36)$ & 10 & $(4,1)$ & 3 & 1 & YES & YES & YES & $1.62$ & $(2,3)$ & NO & 2229\\
$(97,36)$ & 10 & $(4,1)$ & 3 & 1 & YES & YES & YES & $1.62$ & $(2,3)$ & -- & 2230\\
$(97,36)$ & 10 & $(4,1)$ & 3 & 1 & YES & YES & YES & $1.62$ & $(2,3)$ & NO & 2231\\
$(97,36)$ & 10 & $(5,2)$ & 3 & 1 & YES & YES & YES & $1.62$ & $(2,3)$ & -- & 2232\\
$(97,42)$ & 11 & $(5,2)$ & 3 & 1 & YES & YES & YES & $1.29$ & $(4,2)$ & -- & 2233\\
$(97,35)$ & 10 & $(7,3)$ & 4 & 1 & YES & YES & YES & $1.50$ & $(2,3)$ & -- & 2234\\
$(97,42)$ & 11 & $(7,2)$ & 4 & 1 & YES & YES & YES & $1.29$ & $(4,2)$ & NO & 2235\\
$(97,37)$ & 10 & $(13,3)$ & 6 & 1 & YES & YES & YES & $1.71$ & $(2,3)$ & NO & 2236\\
$(97,38)$ & 11 & $(31,12)$ & 7 & 1 & YES & YES & YES & $1.43$ & $(2,3)$ & NO & 2237\\
$(97,26)$ & 10 & $(41,11)$ & 8 & 1 & YES & YES & YES & $1.14$ & $(4,2)$ & NO & 2238\\
$(97,36)$ & 10 & $(89,33)$ & 10 & 1 & YES & YES & YES & $1.50$ & $(2,3)$ & NO & 2239\\
$(98,41)$ & 10 & $(4,1)$ & 3 & 2 & YES & YES & YES & $1.50$ & $(2,3)$ & -- & 2240\\
$(98,41)$ & 10 & $(5,2)$ & 3 & 1 & YES & YES & YES & $1.62$ & $(2,3)$ & -- & 2241\\
$(98,41)$ & 10 & $(7,2)$ & 4 & 7 & YES & YES & YES & $1.50$ & $(2,3)$ & -- & 2242\\
$(98,43)$ & 10 & $(8,3)$ & 4 & 2 & YES & YES & YES & $1.71$ & $(2,3)$ & -- & 2243\\
$(98,41)$ & 10 & $(13,3)$ & 6 & 1 & YES & YES & YES & $1.50$ & $(2,3)$ & -- & 2244\\
$(98,41)$ & 10 & $(31,13)$ & 7 & 1 & YES & YES & YES & $1.50$ & $(2,3)$ & NO & 2245\\
$(98,41)$ & 10 & $(50,21)$ & 8 & 2 & YES & YES & YES & $1.50$ & $(2,3)$ & NO & 2246\\
$(98,43)$ & 10 & $(89,39)$ & 11 & 1 & YES & YES & YES & $1.62$ & $(2,3)$ & 2575 & 2247\\
$(98,41)$ & 10 & $(98,41)$ & 10 & 98 & YES & YES & YES & $1.50$ & $(2,3)$ & NO & 2248\\
$(99,41)$ & 10 & $(8,3)$ & 4 & 1 & YES & YES & YES & $1.62$ & $(2,3)$ & -- & 2249\\
$(100,27)$ & 10 & $(3,1)$ & 2 & 1 & YES & YES & YES & $1.38$ & $(2,3)$ & NO & 2250\\
$(100,27)$ & 10 & $(3,1)$ & 2 & 1 & YES & YES & YES & $1.38$ & $(2,3)$ & -- & 2251\\
$(100,41)$ & 10 & $(4,1)$ & 3 & 4 & YES & YES & YES & $1.38$ & $(2,3)$ & -- & 2252\\
$(100,41)$ & 10 & $(5,2)$ & 3 & 5 & YES & YES & YES & $1.50$ & $(2,3)$ & -- & 2253\\
$(100,41)$ & 10 & $(7,2)$ & 4 & 1 & YES & YES & YES & $1.50$ & $(2,3)$ & -- & 2254\\
$(101,37)$ & 10 & $(3,1)$ & 2 & 1 & YES & YES & YES & $1.29$ & $(4,2)$ & NO & 2255\\
$(101,37)$ & 10 & $(7,3)$ & 4 & 1 & YES & YES & YES & $1.62$ & $(2,3)$ & -- & 2256\\
$(101,28)$ & 11 & $(8,3)$ & 4 & 1 & YES & YES & YES & $1.57$ & $(2,3)$ & NO & 2257\\
$(101,37)$ & 10 & $(9,4)$ & 5 & 1 & YES & YES & YES & $1.62$ & $(2,3)$ & NO & 2258\\
$(101,28)$ & 11 & $(13,4)$ & 6 & 1 & YES & YES & YES & $1.57$ & $(2,3)$ & NO & 2259\\
$(101,23)$ & 11 & $(23,5)$ & 7 & 1 & YES & YES & YES & $1.14$ & $(4,2)$ & NO & 2260\\
$(101,44)$ & 10 & $(37,16)$ & 9 & 1 & YES & YES & YES & $1.62$ & $(2,3)$ & NO & 2261\\
$(101,37)$ & 10 & $(101,37)$ & 10 & 101 & YES & YES & YES & $1.29$ & $(4,2)$ & NO & 2262\\
$(102,31)$ & 11 & $(5,2)$ & 3 & 1 & YES & YES & YES & $1.75$ & $(2,3)$ & -- & 2263\\
$(103,37)$ & 10 & $(4,1)$ & 3 & 1 & YES & YES & YES & $1.29$ & $(4,2)$ & NO & 2264\\
$(103,37)$ & 10 & $(4,1)$ & 3 & 1 & YES & YES & YES & $1.29$ & $(4,2)$ & -- & 2265\\
$(103,30)$ & 11 & $(5,1)$ & 4 & 1 & YES & YES & YES & $1.62$ & $(2,3)$ & -- & 2266\\
$(103,39)$ & 10 & $(53,20)$ & 10 & 1 & YES & YES & YES & $1.50$ & $(2,3)$ & NO & 2267\\
$(103,30)$ & 11 & $(69,20)$ & 10 & 1 & YES & YES & YES & $1.43$ & $(2,3)$ & 2931 & 2268\\
$(103,37)$ & 10 & $(103,37)$ & 10 & 103 & YES & YES & YES & $1.29$ & $(4,2)$ & NO & 2269\\
$(104,47)$ & 11 & $(3,1)$ & 2 & 1 & YES & YES & YES & $1.29$ & $(4,2)$ & -- & 2270\\
$(104,47)$ & 11 & $(5,2)$ & 3 & 1 & YES & YES & YES & $1.29$ & $(4,2)$ & -- & 2271\\
$(104,47)$ & 11 & $(29,13)$ & 8 & 1 & YES & YES & YES & $1.29$ & $(4,2)$ & NO & 2272\\
$(104,43)$ & 10 & $(75,31)$ & 9 & 1 & YES & YES & YES & $1.43$ & $(2,3)$ & NO & 2273\\
$(105,41)$ & 10 & $(3,1)$ & 2 & 3 & YES & YES & YES & $1.50$ & $(2,3)$ & -- & 2274\\
$(105,38)$ & 11 & $(7,2)$ & 4 & 7 & YES & YES & YES & $1.50$ & $(2,3)$ & -- & 2275\\
$(105,31)$ & 10 & $(12,5)$ & 5 & 3 & YES & YES & YES & $1.71$ & $(2,3)$ & -- & 2276\\
$(105,29)$ & 10 & $(13,5)$ & 5 & 1 & YES & YES & YES & $1.71$ & $(2,3)$ & -- & 2277\\
$(105,44)$ & 10 & $(13,5)$ & 5 & 1 & YES & YES & YES & $1.50$ & $(2,3)$ & NO & 2278\\
$(105,32)$ & 11 & $(49,15)$ & 9 & 7 & YES & YES & YES & $1.75$ & $(2,3)$ & NO & 2279\\
$(105,38)$ & 11 & $(61,22)$ & 9 & 1 & YES & YES & YES & $1.62$ & $(2,3)$ & NO & 2280\\
$(106,41)$ & 10 & $(7,3)$ & 4 & 1 & YES & YES & YES & $1.29$ & $(4,2)$ & -- & 2281\\
$(106,23)$ & 11 & $(23,4)$ & 8 & 1 & YES & YES & YES & $1.43$ & $(2,3)$ & NO & 2282\\
$(106,41)$ & 10 & $(28,11)$ & 8 & 2 & YES & YES & YES & $1.29$ & $(4,2)$ & NO & 2283\\
$(106,23)$ & 11 & $(65,14)$ & 10 & 1 & YES & YES & YES & $1.29$ & $(2,3)$ & NO & 2284\\
$(106,41)$ & 10 & $(106,41)$ & 10 & 106 & YES & YES & YES & $1.29$ & $(4,2)$ & NO & 2285\\
$(107,47)$ & 10 & $(2,1)$ & 1 & 1 & YES & YES & YES & $1.43$ & $(4,2)$ & -- & 2286\\
$(107,47)$ & 10 & $(3,1)$ & 2 & 1 & YES & YES & YES & $1.57$ & $(2,3)$ & -- & 2287\\
$(107,47)$ & 10 & $(4,1)$ & 3 & 1 & YES & YES & YES & $1.57$ & $(2,3)$ & NO & 2288\\
$(107,47)$ & 10 & $(4,1)$ & 3 & 1 & YES & YES & YES & $1.57$ & $(2,3)$ & -- & 2289\\
$(107,41)$ & 10 & $(7,2)$ & 4 & 1 & YES & YES & YES & $1.50$ & $(2,3)$ & -- & 2290\\
$(107,44)$ & 12 & $(7,3)$ & 4 & 1 & YES & YES & YES & $1.71$ & $(2,3)$ & -- & 2291\\
$(107,44)$ & 12 & $(13,5)$ & 5 & 1 & YES & YES & YES & $1.71$ & $(2,3)$ & 2325 & 2292\\
$(107,47)$ & 10 & $(107,47)$ & 10 & 107 & YES & YES & YES & $1.43$ & $(2,3)$ & NO & 2293\\
$(109,30)$ & 10 & $(3,1)$ & 2 & 1 & YES & YES & YES & $1.50$ & $(2,3)$ & NO & 2294\\
$(109,30)$ & 10 & $(3,1)$ & 2 & 1 & YES & YES & YES & $1.50$ & $(2,3)$ & -- & 2295\\
$(109,29)$ & 13 & $(7,3)$ & 4 & 1 & YES & YES & YES & $1.57$ & $(4,2)$ & NO & 2296\\
$(109,40)$ & 10 & $(7,3)$ & 4 & 1 & YES & YES & YES & $1.57$ & $(2,3)$ & -- & 2297\\
$(109,46)$ & 10 & $(7,2)$ & 4 & 1 & YES & YES & YES & $1.14$ & $(4,2)$ & NO & 2298\\
$(109,46)$ & 10 & $(7,2)$ & 4 & 1 & YES & YES & YES & $1.50$ & $(2,3)$ & -- & 2299\\
$(110,39)$ & 11 & $(3,1)$ & 2 & 1 & YES & YES & YES & $1.50$ & $(2,3)$ & NO & 2300\\
$(111,46)$ & 10 & $(3,1)$ & 2 & 3 & YES & YES & YES & $1.50$ & $(2,3)$ & -- & 2301\\
$(111,31)$ & 10 & $(4,1)$ & 3 & 1 & YES & YES & YES & $1.50$ & $(2,3)$ & NO & 2302\\
$(111,31)$ & 10 & $(4,1)$ & 3 & 1 & YES & YES & YES & $1.50$ & $(2,3)$ & -- & 2303\\
$(111,31)$ & 10 & $(4,1)$ & 3 & 1 & YES & YES & YES & $1.50$ & $(2,3)$ & 2054 & 2304\\
$(111,31)$ & 10 & $(7,3)$ & 4 & 1 & YES & YES & YES & $1.43$ & $(2,3)$ & NO & 2305\\
$(111,31)$ & 10 & $(8,3)$ & 4 & 1 & YES & YES & YES & $1.43$ & $(2,3)$ & NO & 2306\\
$(111,41)$ & 10 & $(9,2)$ & 5 & 3 & YES & YES & YES & $1.38$ & $(2,3)$ & NO & 2307\\
$(111,31)$ & 10 & $(12,5)$ & 5 & 3 & YES & YES & YES & $1.57$ & $(2,3)$ & NO & 2308\\
$(111,43)$ & 10 & $(14,3)$ & 6 & 1 & YES & YES & YES & $1.75$ & $(2,3)$ & NO & 2309\\
$(111,46)$ & 10 & $(17,7)$ & 6 & 1 & YES & YES & YES & $1.38$ & $(2,3)$ & 2369 & 2310\\
$(111,46)$ & 10 & $(19,8)$ & 6 & 1 & YES & YES & YES & $1.62$ & $(2,3)$ & NO & 2311\\
$(111,46)$ & 10 & $(111,46)$ & 10 & 111 & YES & YES & YES & $1.50$ & $(2,3)$ & NO & 2312\\
$(112,41)$ & 10 & $(3,1)$ & 2 & 1 & YES & YES & YES & $1.43$ & $(2,3)$ & NO & 2313\\
$(112,47)$ & 10 & $(5,1)$ & 4 & 1 & YES & YES & YES & $1.71$ & $(2,3)$ & NO & 2314\\
$(112,47)$ & 10 & $(5,1)$ & 4 & 1 & YES & YES & YES & $1.71$ & $(2,3)$ & -- & 2315\\
$(112,41)$ & 10 & $(7,2)$ & 4 & 7 & YES & YES & YES & $1.29$ & $(2,3)$ & -- & 2316\\
$(112,47)$ & 10 & $(7,3)$ & 4 & 7 & YES & YES & YES & $1.71$ & $(2,3)$ & NO & 2317\\
$(112,41)$ & 10 & $(27,10)$ & 7 & 1 & YES & YES & YES & $1.43$ & $(2,3)$ & 2686 & 2318\\
$(112,41)$ & 10 & $(71,26)$ & 9 & 1 & YES & YES & YES & $1.43$ & $(2,3)$ & NO & 2319\\
$(113,24)$ & 11 & $(2,1)$ & 1 & 1 & YES & YES & YES & $1.29$ & $(4,2)$ & -- & 2320\\
$(113,42)$ & 11 & $(2,1)$ & 1 & 1 & YES & YES & YES & $1.43$ & $(4,2)$ & -- & 2321\\
$(113,42)$ & 11 & $(3,1)$ & 2 & 1 & YES & YES & YES & $1.43$ & $(4,2)$ & -- & 2322\\
$(113,44)$ & 12 & $(9,4)$ & 5 & 1 & YES & YES & YES & $1.71$ & $(2,3)$ & NO & 2323\\
$(113,20)$ & 13 & $(11,3)$ & 5 & 1 & YES & YES & YES & $1.57$ & $(2,3)$ & NO & 2324\\
$(113,44)$ & 12 & $(12,5)$ & 5 & 1 & YES & YES & YES & $1.71$ & $(2,3)$ & 2292 & 2325\\
$(113,42)$ & 11 & $(13,5)$ & 5 & 1 & YES & YES & YES & $1.43$ & $(4,2)$ & NO & 2326\\
$(113,42)$ & 11 & $(27,10)$ & 7 & 1 & YES & YES & YES & $1.43$ & $(4,2)$ & NO & 2327\\
$(113,42)$ & 11 & $(113,42)$ & 11 & 113 & YES & YES & YES & $1.43$ & $(4,2)$ & NO & 2328\\
$(113,44)$ & 12 & $(113,44)$ & 12 & 113 & YES & YES & YES & $1.62$ & $(2,3)$ & NO & 2329\\
$(115,52)$ & 11 & $(5,2)$ & 3 & 5 & YES & YES & YES & $1.43$ & $(2,3)$ & -- & 2330\\
$(115,44)$ & 10 & $(7,3)$ & 4 & 1 & YES & YES & YES & $1.29$ & $(4,2)$ & -- & 2331\\
$(115,52)$ & 11 & $(7,2)$ & 4 & 1 & YES & YES & YES & $1.57$ & $(2,3)$ & -- & 2332\\
$(115,52)$ & 11 & $(9,4)$ & 5 & 1 & YES & YES & YES & $1.29$ & $(4,2)$ & NO & 2333\\
$(115,47)$ & 12 & $(13,5)$ & 5 & 1 & YES & YES & YES & $1.57$ & $(4,2)$ & NO & 2334\\
$(115,52)$ & 11 & $(29,13)$ & 8 & 1 & YES & YES & YES & $1.43$ & $(2,3)$ & 2703 & 2335\\
$(115,52)$ & 11 & $(51,23)$ & 9 & 1 & YES & YES & YES & $1.57$ & $(2,3)$ & NO & 2336\\
$(116,49)$ & 10 & $(7,3)$ & 4 & 1 & YES & YES & YES & $1.50$ & $(2,3)$ & 2212 & 2337\\
$(117,34)$ & 11 & $(5,2)$ & 3 & 1 & YES & YES & YES & $1.29$ & $(4,2)$ & -- & 2338\\
$(117,49)$ & 10 & $(5,2)$ & 3 & 1 & YES & YES & YES & $1.62$ & $(2,3)$ & -- & 2339\\
$(117,31)$ & 11 & $(8,3)$ & 4 & 1 & YES & YES & YES & $1.71$ & $(2,3)$ & -- & 2340\\
$(117,31)$ & 11 & $(17,5)$ & 6 & 1 & YES & YES & YES & $1.71$ & $(2,3)$ & 2984 & 2341\\
$(117,49)$ & 10 & $(17,7)$ & 6 & 1 & YES & YES & YES & $1.62$ & $(2,3)$ & NO & 2342\\
$(117,49)$ & 10 & $(67,28)$ & 10 & 1 & YES & YES & YES & $1.50$ & $(2,3)$ & NO & 2343\\
$(117,34)$ & 11 & $(79,23)$ & 10 & 1 & YES & YES & YES & $1.14$ & $(4,2)$ & 2765 & 2344\\
$(117,43)$ & 10 & $(117,43)$ & 10 & 117 & YES & YES & YES & $1.50$ & $(2,3)$ & NO & 2345\\
$(118,33)$ & 11 & $(5,2)$ & 3 & 1 & YES & YES & YES & $1.50$ & $(2,3)$ & -- & 2346\\
$(118,35)$ & 11 & $(16,3)$ & 7 & 2 & YES & YES & YES & $1.57$ & $(2,3)$ & -- & 2347\\
$(118,33)$ & 11 & $(29,8)$ & 7 & 1 & YES & YES & YES & $1.50$ & $(2,3)$ & NO & 2348\\
$(118,27)$ & 11 & $(79,18)$ & 10 & 1 & YES & YES & YES & $1.38$ & $(2,3)$ & NO & 2349\\
$(119,44)$ & 10 & $(2,1)$ & 1 & 1 & YES & YES & YES & $1.38$ & $(2,3)$ & NO & 2350\\
$(119,45)$ & 11 & $(2,1)$ & 1 & 1 & YES & YES & YES & $1.29$ & $(4,2)$ & -- & 2351\\
$(119,44)$ & 10 & $(3,1)$ & 2 & 1 & YES & YES & YES & $1.43$ & $(2,3)$ & -- & 2352\\
$(119,46)$ & 10 & $(4,1)$ & 3 & 1 & YES & YES & YES & $1.50$ & $(2,3)$ & -- & 2353\\
$(119,45)$ & 11 & $(5,2)$ & 3 & 1 & YES & YES & YES & $1.57$ & $(2,3)$ & -- & 2354\\
$(119,31)$ & 13 & $(7,3)$ & 4 & 7 & YES & YES & YES & $1.57$ & $(4,2)$ & NO & 2355\\
$(119,37)$ & 11 & $(7,2)$ & 4 & 7 & YES & YES & YES & $1.43$ & $(2,3)$ & -- & 2356\\
$(119,50)$ & 10 & $(7,2)$ & 4 & 7 & YES & YES & YES & $1.50$ & $(2,3)$ & -- & 2357\\
$(119,45)$ & 11 & $(8,3)$ & 4 & 1 & YES & YES & YES & $1.43$ & $(4,2)$ & NO & 2358\\
$(119,37)$ & 11 & $(9,2)$ & 5 & 1 & YES & YES & YES & $1.38$ & $(2,3)$ & NO & 2359\\
$(119,43)$ & 11 & $(9,2)$ & 5 & 1 & YES & YES & YES & $1.62$ & $(2,3)$ & -- & 2360\\
$(119,45)$ & 11 & $(18,7)$ & 6 & 1 & YES & YES & YES & $1.62$ & $(2,3)$ & NO & 2361\\
$(119,46)$ & 10 & $(23,9)$ & 7 & 1 & YES & YES & YES & $1.38$ & $(2,3)$ & 2732 & 2362\\
$(119,46)$ & 10 & $(75,29)$ & 9 & 1 & YES & YES & YES & $1.50$ & $(2,3)$ & NO & 2363\\
$(121,50)$ & 10 & $(3,1)$ & 2 & 1 & YES & YES & YES & $1.50$ & $(2,3)$ & -- & 2364\\
$(121,35)$ & 12 & $(4,1)$ & 3 & 1 & YES & YES & YES & $1.29$ & $(4,2)$ & -- & 2365\\
$(121,36)$ & 11 & $(4,1)$ & 3 & 1 & YES & YES & YES & $1.57$ & $(2,3)$ & -- & 2366\\
$(121,36)$ & 11 & $(5,2)$ & 3 & 1 & YES & YES & YES & $1.29$ & $(4,2)$ & NO & 2367\\
$(121,50)$ & 10 & $(11,3)$ & 5 & 11 & YES & YES & YES & $1.57$ & $(2,3)$ & -- & 2368\\
$(121,50)$ & 10 & $(12,5)$ & 5 & 1 & YES & YES & YES & $1.38$ & $(2,3)$ & 2310 & 2369\\
$(121,46)$ & 10 & $(13,3)$ & 6 & 1 & YES & YES & YES & $1.57$ & $(2,3)$ & -- & 2370\\
$(121,36)$ & 11 & $(61,18)$ & 9 & 1 & YES & YES & YES & $1.57$ & $(2,3)$ & NO & 2371\\
$(121,36)$ & 11 & $(67,20)$ & 11 & 1 & YES & YES & YES & $1.57$ & $(2,3)$ & NO & 2372\\
$(121,36)$ & 11 & $(84,25)$ & 10 & 1 & YES & YES & YES & $1.57$ & $(2,3)$ & NO & 2373\\
$(122,51)$ & 11 & $(98,41)$ & 10 & 2 & YES & YES & YES & $1.50$ & $(2,3)$ & NO & 2374\\
$(123,38)$ & 12 & $(7,2)$ & 4 & 1 & YES & YES & YES & $1.50$ & $(2,3)$ & -- & 2375\\
$(123,47)$ & 10 & $(8,3)$ & 4 & 1 & YES & YES & YES & $1.62$ & $(2,3)$ & -- & 2376\\
$(125,49)$ & 11 & $(2,1)$ & 1 & 1 & YES & YES & YES & $1.14$ & $(4,2)$ & -- & 2377\\
$(125,49)$ & 11 & $(5,2)$ & 3 & 5 & YES & YES & YES & $1.43$ & $(4,2)$ & -- & 2378\\
$(125,49)$ & 11 & $(5,2)$ & 3 & 5 & YES & YES & YES & $1.29$ & $(4,2)$ & 2098 & 2379\\
$(125,27)$ & 11 & $(21,5)$ & 8 & 1 & YES & YES & YES & $1.43$ & $(4,2)$ & 1973 & 2380\\
$(125,27)$ & 11 & $(31,7)$ & 8 & 1 & YES & YES & YES & $1.43$ & $(4,2)$ & NO & 2381\\
$(125,49)$ & 11 & $(51,20)$ & 9 & 1 & YES & YES & YES & $1.14$ & $(4,2)$ & NO & 2382\\
$(127,54)$ & 12 & $(2,1)$ & 1 & 1 & YES & YES & YES & $1.43$ & $(4,2)$ & -- & 2383\\
$(127,54)$ & 12 & $(3,1)$ & 2 & 1 & YES & YES & YES & $1.43$ & $(4,2)$ & -- & 2384\\
$(128,47)$ & 10 & $(2,1)$ & 1 & 2 & YES & YES & YES & $1.38$ & $(2,3)$ & 2127 & 2385\\
$(128,47)$ & 10 & $(3,1)$ & 2 & 1 & YES & YES & YES & $1.38$ & $(2,3)$ & -- & 2386\\
$(128,47)$ & 10 & $(4,1)$ & 3 & 4 & YES & YES & YES & $1.38$ & $(2,3)$ & NO & 2387\\
$(128,47)$ & 10 & $(4,1)$ & 3 & 4 & YES & YES & YES & $1.38$ & $(2,3)$ & -- & 2388\\
$(128,49)$ & 10 & $(4,1)$ & 3 & 4 & YES & YES & YES & $1.50$ & $(2,3)$ & NO & 2389\\
$(128,49)$ & 10 & $(4,1)$ & 3 & 4 & YES & YES & YES & $1.50$ & $(2,3)$ & -- & 2390\\
$(128,49)$ & 10 & $(13,3)$ & 6 & 1 & YES & YES & YES & $1.62$ & $(2,3)$ & NO & 2391\\
$(128,47)$ & 10 & $(27,10)$ & 7 & 1 & YES & YES & YES & $1.38$ & $(2,3)$ & NO & 2392\\
$(128,47)$ & 10 & $(79,29)$ & 9 & 1 & YES & YES & YES & $1.38$ & $(2,3)$ & NO & 2393\\
$(128,47)$ & 10 & $(128,47)$ & 10 & 128 & YES & YES & YES & $1.38$ & $(2,3)$ & NO & 2394\\
$(129,50)$ & 10 & $(2,1)$ & 1 & 1 & YES & YES & YES & $1.43$ & $(2,3)$ & NO & 2395\\
$(129,50)$ & 10 & $(3,1)$ & 2 & 3 & YES & YES & YES & $1.50$ & $(2,3)$ & NO & 2396\\
$(129,49)$ & 10 & $(5,2)$ & 3 & 1 & YES & YES & YES & $1.43$ & $(2,3)$ & -- & 2397\\
$(129,49)$ & 10 & $(7,3)$ & 4 & 1 & YES & YES & YES & $1.43$ & $(2,3)$ & NO & 2398\\
$(129,29)$ & 12 & $(8,3)$ & 4 & 1 & YES & YES & YES & $1.62$ & $(2,3)$ & -- & 2399\\
$(129,50)$ & 10 & $(23,9)$ & 7 & 1 & YES & YES & YES & $1.50$ & $(2,3)$ & NO & 2400\\
$(129,49)$ & 10 & $(37,14)$ & 8 & 1 & YES & YES & YES & $1.14$ & $(4,2)$ & NO & 2401\\
$(130,47)$ & 11 & $(5,2)$ & 3 & 5 & YES & YES & YES & $1.50$ & $(2,3)$ & -- & 2402\\
$(130,57)$ & 11 & $(9,2)$ & 5 & 1 & YES & YES & YES & $1.50$ & $(2,3)$ & NO & 2403\\
$(130,57)$ & 11 & $(89,39)$ & 11 & 1 & YES & YES & YES & $1.62$ & $(2,3)$ & NO & 2404\\
$(130,47)$ & 11 & $(119,43)$ & 11 & 1 & YES & YES & YES & $1.50$ & $(2,3)$ & NO & 2405\\
$(131,39)$ & 11 & $(5,1)$ & 4 & 1 & YES & YES & YES & $1.57$ & $(2,3)$ & NO & 2406\\
$(131,50)$ & 10 & $(8,3)$ & 4 & 1 & YES & YES & YES & $1.43$ & $(2,3)$ & NO & 2407\\
$(131,25)$ & 14 & $(9,4)$ & 5 & 1 & YES & YES & YES & $1.71$ & $(2,3)$ & -- & 2408\\
$(131,50)$ & 10 & $(9,2)$ & 5 & 1 & YES & YES & YES & $1.50$ & $(2,3)$ & NO & 2409\\
$(131,50)$ & 10 & $(10,3)$ & 5 & 1 & YES & YES & YES & $1.50$ & $(2,3)$ & -- & 2410\\
$(131,36)$ & 11 & $(11,3)$ & 5 & 1 & YES & YES & YES & $1.57$ & $(2,3)$ & -- & 2411\\
$(131,40)$ & 11 & $(11,3)$ & 5 & 1 & YES & YES & YES & $1.50$ & $(2,3)$ & NO & 2412\\
$(131,50)$ & 10 & $(11,3)$ & 5 & 1 & YES & YES & YES & $1.71$ & $(2,3)$ & -- & 2413\\
$(131,55)$ & 10 & $(11,3)$ & 5 & 1 & YES & YES & YES & $1.86$ & $(2,3)$ & NO & 2414\\
$(131,24)$ & 13 & $(13,3)$ & 6 & 1 & YES & YES & YES & $1.50$ & $(2,3)$ & NO & 2415\\
$(131,39)$ & 11 & $(67,20)$ & 11 & 1 & YES & YES & YES & $1.57$ & $(2,3)$ & NO & 2416\\
$(131,50)$ & 10 & $(76,29)$ & 9 & 1 & YES & YES & YES & $1.62$ & $(2,3)$ & NO & 2417\\
$(131,50)$ & 10 & $(81,31)$ & 9 & 1 & YES & YES & YES & $1.62$ & $(2,3)$ & NO & 2418\\
$(131,39)$ & 11 & $(121,36)$ & 11 & 1 & YES & YES & YES & $1.38$ & $(2,3)$ & NO & 2419\\
$(132,35)$ & 11 & $(10,3)$ & 5 & 2 & YES & YES & YES & $1.43$ & $(2,3)$ & -- & 2420\\
$(133,31)$ & 12 & $(5,2)$ & 3 & 1 & YES & YES & YES & $1.50$ & $(2,3)$ & NO & 2421\\
$(133,39)$ & 11 & $(7,2)$ & 4 & 7 & YES & YES & YES & $1.38$ & $(2,3)$ & -- & 2422\\
$(133,30)$ & 12 & $(58,13)$ & 11 & 1 & YES & YES & YES & $1.50$ & $(2,3)$ & NO & 2423\\
$(133,30)$ & 12 & $(89,20)$ & 11 & 1 & YES & YES & YES & $1.62$ & $(2,3)$ & 3072 & 2424\\
$(134,49)$ & 11 & $(3,1)$ & 2 & 1 & YES & YES & YES & $1.29$ & $(4,2)$ & NO & 2425\\
$(134,49)$ & 11 & $(3,1)$ & 2 & 1 & YES & YES & YES & $1.29$ & $(4,2)$ & -- & 2426\\
$(134,55)$ & 11 & $(4,1)$ & 3 & 2 & YES & YES & YES & $1.50$ & $(2,3)$ & -- & 2427\\
$(134,55)$ & 11 & $(5,2)$ & 3 & 1 & YES & YES & YES & $1.50$ & $(2,3)$ & -- & 2428\\
$(134,59)$ & 12 & $(6,1)$ & 5 & 2 & YES & YES & YES & $1.43$ & $(2,3)$ & -- & 2429\\
$(134,59)$ & 12 & $(7,1)$ & 6 & 1 & YES & YES & YES & $1.57$ & $(2,3)$ & NO & 2430\\
$(134,59)$ & 12 & $(7,1)$ & 6 & 1 & YES & YES & YES & $1.57$ & $(2,3)$ & NO & 2431\\
$(134,37)$ & 11 & $(16,3)$ & 7 & 2 & YES & YES & YES & $1.71$ & $(2,3)$ & -- & 2432\\
$(134,41)$ & 11 & $(29,9)$ & 8 & 1 & YES & YES & YES & $1.50$ & $(2,3)$ & NO & 2433\\
$(134,39)$ & 11 & $(44,13)$ & 8 & 2 & YES & YES & YES & $1.57$ & $(2,3)$ & NO & 2434\\
$(134,55)$ & 11 & $(95,39)$ & 10 & 1 & YES & YES & YES & $1.50$ & $(2,3)$ & NO & 2435\\
$(134,55)$ & 11 & $(134,55)$ & 11 & 134 & YES & YES & YES & $1.29$ & $(4,2)$ & NO & 2436\\
$(135,56)$ & 11 & $(4,1)$ & 3 & 1 & YES & YES & YES & $1.62$ & $(2,3)$ & NO & 2437\\
$(135,56)$ & 11 & $(4,1)$ & 3 & 1 & YES & YES & YES & $1.62$ & $(2,3)$ & -- & 2438\\
$(135,41)$ & 11 & $(5,2)$ & 3 & 5 & YES & YES & YES & $1.50$ & $(2,3)$ & -- & 2439\\
$(135,41)$ & 11 & $(9,2)$ & 5 & 9 & YES & YES & YES & $1.38$ & $(2,3)$ & NO & 2440\\
$(135,56)$ & 11 & $(22,9)$ & 7 & 1 & YES & YES & YES & $1.43$ & $(2,3)$ & NO & 2441\\
$(135,56)$ & 11 & $(65,27)$ & 10 & 5 & YES & YES & YES & $1.50$ & $(2,3)$ & 2855 & 2442\\
$(136,59)$ & 11 & $(37,16)$ & 9 & 1 & YES & YES & YES & $1.62$ & $(2,3)$ & NO & 2443\\
$(137,53)$ & 11 & $(4,1)$ & 3 & 1 & YES & YES & YES & $1.57$ & $(2,3)$ & -- & 2444\\
$(137,30)$ & 12 & $(5,2)$ & 3 & 1 & YES & YES & YES & $1.50$ & $(2,3)$ & NO & 2445\\
$(137,52)$ & 11 & $(5,1)$ & 4 & 1 & YES & YES & YES & $1.29$ & $(2,3)$ & -- & 2446\\
$(137,32)$ & 12 & $(81,19)$ & 11 & 1 & YES & YES & YES & $1.43$ & $(2,3)$ & NO & 2447\\
$(137,52)$ & 11 & $(108,41)$ & 10 & 1 & YES & YES & YES & $1.43$ & $(2,3)$ & NO & 2448\\
$(137,53)$ & 11 & $(137,53)$ & 11 & 137 & YES & YES & YES & $1.50$ & $(2,3)$ & NO & 2449\\
$(138,49)$ & 12 & $(3,1)$ & 2 & 3 & YES & YES & YES & $1.43$ & $(4,2)$ & -- & 2450\\
$(138,41)$ & 11 & $(16,3)$ & 7 & 2 & YES & YES & YES & $1.71$ & $(2,3)$ & NO & 2451\\
$(139,61)$ & 11 & $(2,1)$ & 1 & 1 & YES & YES & YES & $1.29$ & $(4,2)$ & NO & 2452\\
$(139,39)$ & 11 & $(3,1)$ & 2 & 1 & YES & YES & YES & $1.50$ & $(2,3)$ & NO & 2453\\
$(139,39)$ & 11 & $(3,1)$ & 2 & 1 & YES & YES & YES & $1.50$ & $(2,3)$ & -- & 2454\\
$(139,51)$ & 11 & $(4,1)$ & 3 & 1 & YES & YES & YES & $1.57$ & $(2,3)$ & -- & 2455\\
$(139,51)$ & 11 & $(7,3)$ & 4 & 1 & YES & YES & YES & $1.57$ & $(2,3)$ & NO & 2456\\
$(139,41)$ & 11 & $(8,3)$ & 4 & 1 & YES & YES & YES & $1.57$ & $(2,3)$ & -- & 2457\\
$(139,39)$ & 11 & $(139,39)$ & 11 & 139 & YES & YES & YES & $1.50$ & $(2,3)$ & NO & 2458\\
$(140,37)$ & 11 & $(5,2)$ & 3 & 5 & YES & YES & YES & $1.38$ & $(2,3)$ & -- & 2459\\
$(140,41)$ & 11 & $(8,3)$ & 4 & 4 & YES & YES & YES & $1.75$ & $(2,3)$ & -- & 2460\\
$(140,61)$ & 11 & $(9,4)$ & 5 & 1 & YES & YES & YES & $1.50$ & $(2,3)$ & NO & 2461\\
$(141,59)$ & 11 & $(4,1)$ & 3 & 1 & YES & YES & YES & $1.62$ & $(2,3)$ & NO & 2462\\
$(141,59)$ & 11 & $(4,1)$ & 3 & 1 & YES & YES & YES & $1.62$ & $(2,3)$ & -- & 2463\\
$(141,59)$ & 11 & $(67,28)$ & 10 & 1 & YES & YES & YES & $1.50$ & $(2,3)$ & 2872 & 2464\\
$(142,39)$ & 11 & $(5,2)$ & 3 & 1 & YES & YES & YES & $1.38$ & $(2,3)$ & NO & 2465\\
$(142,51)$ & 11 & $(5,2)$ & 3 & 1 & YES & YES & YES & $1.62$ & $(2,3)$ & -- & 2466\\
$(142,39)$ & 11 & $(10,3)$ & 5 & 2 & YES & YES & YES & $1.38$ & $(2,3)$ & NO & 2467\\
$(142,51)$ & 11 & $(36,13)$ & 8 & 2 & YES & YES & YES & $1.50$ & $(2,3)$ & 2792 & 2468\\
$(142,39)$ & 11 & $(73,20)$ & 11 & 1 & YES & YES & YES & $1.71$ & $(2,3)$ & NO & 2469\\
$(143,54)$ & 12 & $(3,1)$ & 2 & 1 & YES & YES & YES & $1.43$ & $(4,2)$ & -- & 2470\\
$(143,54)$ & 12 & $(3,1)$ & 2 & 1 & YES & YES & YES & $1.43$ & $(4,2)$ & NO & 2471\\
$(143,54)$ & 12 & $(6,1)$ & 5 & 1 & YES & YES & YES & $1.29$ & $(4,2)$ & NO & 2472\\
$(143,54)$ & 12 & $(6,1)$ & 5 & 1 & YES & YES & YES & $1.43$ & $(4,2)$ & -- & 2473\\
$(143,54)$ & 12 & $(13,5)$ & 5 & 13 & YES & YES & YES & $1.43$ & $(4,2)$ & NO & 2474\\
$(143,54)$ & 12 & $(29,11)$ & 7 & 1 & YES & YES & YES & $1.50$ & $(2,3)$ & NO & 2475\\
$(143,63)$ & 11 & $(34,15)$ & 8 & 1 & YES & YES & YES & $1.29$ & $(2,3)$ & NO & 2476\\
$(143,54)$ & 12 & $(37,14)$ & 8 & 1 & YES & YES & YES & $1.29$ & $(4,2)$ & NO & 2477\\
$(143,54)$ & 12 & $(143,54)$ & 12 & 143 & YES & YES & YES & $1.43$ & $(4,2)$ & NO & 2478\\
$(144,61)$ & 11 & $(2,1)$ & 1 & 2 & YES & YES & YES & $1.29$ & $(4,2)$ & -- & 2479\\
$(144,55)$ & 10 & $(18,7)$ & 6 & 18 & YES & YES & YES & $1.43$ & $(2,3)$ & NO & 2480\\
$(145,51)$ & 12 & $(2,1)$ & 1 & 1 & YES & YES & YES & $1.29$ & $(4,2)$ & -- & 2481\\
$(145,44)$ & 11 & $(5,2)$ & 3 & 5 & YES & YES & YES & $1.38$ & $(2,3)$ & -- & 2482\\
$(145,44)$ & 11 & $(18,5)$ & 6 & 1 & YES & YES & YES & $1.71$ & $(2,3)$ & NO & 2483\\
$(145,43)$ & 12 & $(145,43)$ & 12 & 145 & YES & YES & YES & $1.57$ & $(2,3)$ & NO & 2484\\
$(145,64)$ & 12 & $(145,64)$ & 12 & 145 & YES & YES & YES & $1.43$ & $(2,3)$ & NO & 2485\\
$(147,43)$ & 11 & $(14,3)$ & 6 & 7 & YES & YES & YES & $1.62$ & $(2,3)$ & NO & 2486\\
$(147,41)$ & 11 & $(17,5)$ & 6 & 1 & YES & YES & YES & $1.71$ & $(2,3)$ & NO & 2487\\
$(148,65)$ & 11 & $(3,1)$ & 2 & 1 & YES & YES & YES & $1.43$ & $(2,3)$ & -- & 2488\\
$(148,41)$ & 11 & $(4,1)$ & 3 & 4 & YES & YES & NO(3) & $1.14$ & $(2,3)$ & -- & 2489\\
$(148,35)$ & 12 & $(7,3)$ & 4 & 1 & YES & YES & YES & $1.50$ & $(2,3)$ & -- & 2490\\
$(149,44)$ & 11 & $(8,3)$ & 4 & 1 & YES & YES & YES & $1.50$ & $(2,3)$ & -- & 2491\\
$(149,42)$ & 12 & $(11,3)$ & 5 & 1 & YES & YES & YES & $1.50$ & $(2,3)$ & NO & 2492\\
$(149,45)$ & 12 & $(11,2)$ & 6 & 1 & YES & YES & YES & $1.50$ & $(2,3)$ & NO & 2493\\
$(149,45)$ & 12 & $(16,5)$ & 7 & 1 & YES & YES & YES & $1.62$ & $(2,3)$ & NO & 2494\\
$(149,42)$ & 12 & $(18,5)$ & 6 & 1 & YES & YES & YES & $1.50$ & $(2,3)$ & NO & 2495\\
$(149,40)$ & 11 & $(25,7)$ & 7 & 1 & YES & YES & YES & $1.86$ & $(2,3)$ & NO & 2496\\
$(149,44)$ & 11 & $(47,14)$ & 9 & 1 & YES & YES & YES & $1.57$ & $(2,3)$ & NO & 2497\\
$(149,44)$ & 11 & $(64,19)$ & 9 & 1 & YES & YES & YES & $1.50$ & $(2,3)$ & NO & 2498\\
$(151,47)$ & 12 & $(3,1)$ & 2 & 1 & YES & YES & YES & $1.50$ & $(2,3)$ & -- & 2499\\
$(151,64)$ & 11 & $(3,1)$ & 2 & 1 & YES & YES & YES & $1.62$ & $(2,3)$ & -- & 2500\\
$(151,62)$ & 11 & $(4,1)$ & 3 & 1 & YES & YES & YES & $1.50$ & $(2,3)$ & -- & 2501\\
$(151,34)$ & 12 & $(5,2)$ & 3 & 1 & YES & YES & YES & $1.29$ & $(4,2)$ & NO & 2502\\
$(151,64)$ & 11 & $(12,5)$ & 5 & 1 & YES & YES & YES & $1.50$ & $(2,3)$ & 2214 & 2503\\
$(151,64)$ & 11 & $(19,8)$ & 6 & 1 & YES & YES & YES & $1.50$ & $(2,3)$ & NO & 2504\\
$(151,62)$ & 11 & $(95,39)$ & 10 & 1 & YES & YES & YES & $1.50$ & $(2,3)$ & NO & 2505\\
$(152,47)$ & 12 & $(4,1)$ & 3 & 4 & YES & YES & YES & $1.62$ & $(2,3)$ & NO & 2506\\
$(152,47)$ & 12 & $(4,1)$ & 3 & 4 & YES & YES & YES & $1.62$ & $(2,3)$ & -- & 2507\\
$(152,55)$ & 12 & $(4,1)$ & 3 & 4 & YES & YES & YES & $1.43$ & $(4,2)$ & -- & 2508\\
$(152,45)$ & 12 & $(7,1)$ & 6 & 1 & YES & YES & YES & $1.43$ & $(2,3)$ & NO & 2509\\
$(152,59)$ & 11 & $(7,2)$ & 4 & 1 & YES & YES & YES & $1.71$ & $(2,3)$ & -- & 2510\\
$(152,63)$ & 11 & $(7,3)$ & 4 & 1 & YES & YES & YES & $1.50$ & $(2,3)$ & NO & 2511\\
$(152,47)$ & 12 & $(23,7)$ & 7 & 1 & YES & YES & YES & $1.62$ & $(2,3)$ & NO & 2512\\
$(152,55)$ & 12 & $(25,9)$ & 7 & 1 & YES & YES & YES & $1.43$ & $(4,2)$ & NO & 2513\\
$(152,59)$ & 11 & $(80,31)$ & 9 & 8 & YES & YES & YES & $1.71$ & $(2,3)$ & NO & 2514\\
$(152,45)$ & 12 & $(98,29)$ & 10 & 2 & YES & YES & YES & $1.43$ & $(2,3)$ & 2840 & 2515\\
$(152,55)$ & 12 & $(105,38)$ & 11 & 1 & YES & YES & YES & $1.43$ & $(4,2)$ & NO & 2516\\
$(152,45)$ & 12 & $(152,45)$ & 12 & 152 & YES & YES & YES & $1.57$ & $(2,3)$ & NO & 2517\\
$(152,47)$ & 12 & $(152,47)$ & 12 & 152 & YES & YES & YES & $1.62$ & $(2,3)$ & NO & 2518\\
$(153,56)$ & 11 & $(7,2)$ & 4 & 1 & YES & YES & YES & $1.43$ & $(4,2)$ & NO & 2519\\
$(153,56)$ & 11 & $(27,10)$ & 7 & 9 & YES & YES & YES & $1.29$ & $(4,2)$ & NO & 2520\\
$(153,41)$ & 11 & $(34,9)$ & 8 & 17 & YES & YES & YES & $1.50$ & $(2,3)$ & NO & 2521\\
$(154,47)$ & 11 & $(5,2)$ & 3 & 1 & YES & YES & YES & $1.38$ & $(2,3)$ & -- & 2522\\
$(154,45)$ & 11 & $(10,3)$ & 5 & 2 & YES & YES & YES & $1.62$ & $(2,3)$ & -- & 2523\\
$(154,43)$ & 11 & $(26,7)$ & 7 & 2 & YES & YES & YES & $1.71$ & $(2,3)$ & NO & 2524\\
$(154,47)$ & 11 & $(33,10)$ & 8 & 11 & YES & YES & YES & $1.50$ & $(2,3)$ & NO & 2525\\
$(154,45)$ & 11 & $(44,13)$ & 8 & 22 & YES & YES & YES & $1.57$ & $(2,3)$ & NO & 2526\\
$(154,45)$ & 11 & $(147,43)$ & 11 & 7 & YES & YES & YES & $1.62$ & $(2,3)$ & NO & 2527\\
$(154,65)$ & 11 & $(154,65)$ & 11 & 154 & YES & YES & YES & $1.29$ & $(2,3)$ & NO & 2528\\
$(155,64)$ & 11 & $(3,1)$ & 2 & 1 & YES & YES & YES & $1.50$ & $(2,3)$ & NO & 2529\\
$(155,46)$ & 11 & $(4,1)$ & 3 & 1 & YES & YES & YES & $1.57$ & $(2,3)$ & -- & 2530\\
$(155,47)$ & 12 & $(4,1)$ & 3 & 1 & YES & YES & YES & $1.57$ & $(2,3)$ & NO & 2531\\
$(155,64)$ & 11 & $(5,2)$ & 3 & 5 & YES & YES & YES & $1.29$ & $(4,2)$ & -- & 2532\\
$(155,46)$ & 11 & $(7,3)$ & 4 & 1 & YES & YES & YES & $1.71$ & $(2,3)$ & -- & 2533\\
$(155,64)$ & 11 & $(7,2)$ & 4 & 1 & YES & YES & YES & $1.57$ & $(2,3)$ & NO & 2534\\
$(155,46)$ & 11 & $(41,12)$ & 8 & 1 & YES & YES & YES & $1.75$ & $(2,3)$ & NO & 2535\\
$(155,64)$ & 11 & $(41,17)$ & 8 & 1 & YES & YES & YES & $1.29$ & $(4,2)$ & NO & 2536\\
$(157,69)$ & 11 & $(3,1)$ & 2 & 1 & YES & YES & YES & $1.43$ & $(2,3)$ & -- & 2537\\
$(157,58)$ & 11 & $(5,2)$ & 3 & 1 & YES & YES & YES & $1.43$ & $(2,3)$ & NO & 2538\\
$(157,60)$ & 11 & $(5,2)$ & 3 & 1 & YES & YES & YES & $1.43$ & $(4,2)$ & -- & 2539\\
$(157,58)$ & 11 & $(14,5)$ & 6 & 1 & YES & YES & YES & $1.43$ & $(4,2)$ & NO & 2540\\
$(157,60)$ & 11 & $(29,11)$ & 7 & 1 & YES & YES & YES & $1.57$ & $(4,2)$ & NO & 2541\\
$(159,59)$ & 11 & $(2,1)$ & 1 & 1 & YES & YES & YES & $1.57$ & $(2,3)$ & -- & 2542\\
$(159,59)$ & 11 & $(3,1)$ & 2 & 3 & YES & YES & YES & $1.57$ & $(2,3)$ & -- & 2543\\
$(159,62)$ & 11 & $(3,1)$ & 2 & 3 & YES & YES & YES & $1.50$ & $(2,3)$ & -- & 2544\\
$(159,47)$ & 11 & $(7,3)$ & 4 & 1 & YES & YES & YES & $1.86$ & $(2,3)$ & NO & 2545\\
$(159,47)$ & 11 & $(7,3)$ & 4 & 1 & YES & YES & YES & $1.86$ & $(2,3)$ & -- & 2546\\
$(159,62)$ & 11 & $(7,3)$ & 4 & 1 & YES & YES & YES & $1.50$ & $(2,3)$ & NO & 2547\\
$(159,62)$ & 11 & $(23,9)$ & 7 & 1 & YES & YES & YES & $1.14$ & $(4,2)$ & 2606 & 2548\\
$(159,59)$ & 11 & $(27,10)$ & 7 & 3 & YES & YES & YES & $1.57$ & $(2,3)$ & NO & 2549\\
$(159,47)$ & 11 & $(47,14)$ & 9 & 1 & YES & YES & YES & $1.86$ & $(2,3)$ & NO & 2550\\
$(159,59)$ & 11 & $(62,23)$ & 9 & 1 & YES & YES & YES & $1.57$ & $(2,3)$ & NO & 2551\\
$(159,61)$ & 12 & $(73,28)$ & 10 & 1 & YES & YES & YES & $1.57$ & $(4,2)$ & NO & 2552\\
$(159,62)$ & 11 & $(159,62)$ & 11 & 159 & YES & YES & YES & $1.50$ & $(2,3)$ & NO & 2553\\
$(160,67)$ & 11 & $(2,1)$ & 1 & 2 & YES & YES & YES & $1.57$ & $(2,3)$ & -- & 2554\\
$(161,48)$ & 12 & $(2,1)$ & 1 & 1 & YES & YES & YES & $1.43$ & $(2,3)$ & NO & 2555\\
$(161,57)$ & 12 & $(3,1)$ & 2 & 1 & YES & YES & YES & $1.29$ & $(4,2)$ & -- & 2556\\
$(161,61)$ & 11 & $(3,1)$ & 2 & 1 & YES & YES & YES & $1.29$ & $(4,2)$ & NO & 2557\\
$(161,66)$ & 11 & $(3,1)$ & 2 & 1 & YES & YES & YES & $1.62$ & $(2,3)$ & -- & 2558\\
$(161,66)$ & 11 & $(4,1)$ & 3 & 1 & YES & YES & YES & $1.50$ & $(2,3)$ & -- & 2559\\
$(161,57)$ & 12 & $(5,2)$ & 3 & 1 & YES & YES & YES & $1.50$ & $(2,3)$ & NO & 2560\\
$(161,71)$ & 12 & $(7,1)$ & 6 & 7 & YES & YES & YES & $1.57$ & $(2,3)$ & NO & 2561\\
$(161,71)$ & 12 & $(7,1)$ & 6 & 7 & YES & YES & YES & $1.57$ & $(2,3)$ & NO & 2562\\
$(161,57)$ & 12 & $(31,11)$ & 8 & 1 & YES & YES & YES & $1.29$ & $(4,2)$ & NO & 2563\\
$(161,71)$ & 12 & $(34,15)$ & 8 & 1 & YES & YES & YES & $1.71$ & $(2,3)$ & NO & 2564\\
$(161,66)$ & 11 & $(100,41)$ & 10 & 1 & YES & YES & YES & $1.50$ & $(2,3)$ & NO & 2565\\
$(161,66)$ & 11 & $(161,66)$ & 11 & 161 & YES & YES & YES & $1.50$ & $(2,3)$ & NO & 2566\\
$(161,71)$ & 12 & $(161,71)$ & 12 & 161 & YES & YES & YES & $1.57$ & $(2,3)$ & NO & 2567\\
$(162,71)$ & 12 & $(2,1)$ & 1 & 2 & YES & YES & YES & $1.57$ & $(2,3)$ & -- & 2568\\
$(162,71)$ & 12 & $(4,1)$ & 3 & 2 & YES & YES & YES & $1.62$ & $(2,3)$ & -- & 2569\\
$(162,71)$ & 12 & $(5,1)$ & 4 & 1 & YES & YES & YES & $1.62$ & $(2,3)$ & NO & 2570\\
$(162,71)$ & 12 & $(5,1)$ & 4 & 1 & YES & YES & YES & $1.62$ & $(2,3)$ & -- & 2571\\
$(162,73)$ & 12 & $(5,1)$ & 4 & 1 & YES & YES & YES & $1.50$ & $(2,3)$ & NO & 2572\\
$(162,73)$ & 12 & $(5,1)$ & 4 & 1 & YES & YES & YES & $1.50$ & $(2,3)$ & -- & 2573\\
$(162,43)$ & 12 & $(26,7)$ & 7 & 2 & YES & YES & YES & $1.50$ & $(2,3)$ & NO & 2574\\
$(162,71)$ & 12 & $(41,18)$ & 8 & 1 & YES & YES & YES & $1.62$ & $(2,3)$ & 2247 & 2575\\
$(162,71)$ & 12 & $(57,25)$ & 9 & 3 & YES & YES & YES & $1.62$ & $(2,3)$ & NO & 2576\\
$(163,71)$ & 11 & $(3,1)$ & 2 & 1 & YES & YES & YES & $1.57$ & $(2,3)$ & -- & 2577\\
$(163,45)$ & 12 & $(7,1)$ & 6 & 1 & YES & YES & YES & $1.29$ & $(2,3)$ & NO & 2578\\
$(163,62)$ & 11 & $(121,46)$ & 10 & 1 & YES & YES & YES & $1.57$ & $(2,3)$ & NO & 2579\\
$(163,45)$ & 12 & $(163,45)$ & 12 & 163 & YES & YES & YES & $1.43$ & $(2,3)$ & NO & 2580\\
$(163,62)$ & 11 & $(163,62)$ & 11 & 163 & YES & YES & YES & $1.50$ & $(2,3)$ & NO & 2581\\
$(164,71)$ & 12 & $(2,1)$ & 1 & 2 & YES & YES & YES & $1.57$ & $(2,3)$ & -- & 2582\\
$(164,51)$ & 12 & $(5,1)$ & 4 & 1 & YES & YES & YES & $1.57$ & $(2,3)$ & NO & 2583\\
$(164,51)$ & 12 & $(164,51)$ & 12 & 164 & YES & YES & YES & $1.57$ & $(2,3)$ & NO & 2584\\
$(165,64)$ & 11 & $(3,1)$ & 2 & 3 & YES & YES & YES & $1.50$ & $(2,3)$ & -- & 2585\\
$(165,64)$ & 11 & $(5,2)$ & 3 & 5 & YES & YES & YES & $1.57$ & $(2,3)$ & -- & 2586\\
$(165,49)$ & 11 & $(7,3)$ & 4 & 1 & YES & YES & YES & $1.71$ & $(2,3)$ & -- & 2587\\
$(165,49)$ & 11 & $(7,3)$ & 4 & 1 & YES & YES & YES & $1.86$ & $(2,3)$ & NO & 2588\\
$(165,64)$ & 11 & $(7,2)$ & 4 & 1 & YES & YES & YES & $1.57$ & $(2,3)$ & NO & 2589\\
$(166,49)$ & 11 & $(3,1)$ & 2 & 1 & NO & YES & YES & $1.50$ & $(2,3)$ & -- & 2590\\
$(167,51)$ & 12 & $(5,2)$ & 3 & 1 & YES & YES & YES & $1.57$ & $(2,3)$ & -- & 2591\\
$(167,64)$ & 11 & $(5,1)$ & 4 & 1 & YES & YES & YES & $1.38$ & $(2,3)$ & NO & 2592\\
$(167,64)$ & 11 & $(60,23)$ & 9 & 1 & YES & YES & YES & $1.50$ & $(2,3)$ & NO & 2593\\
$(167,52)$ & 12 & $(106,33)$ & 11 & 1 & YES & YES & YES & $1.43$ & $(2,3)$ & NO & 2594\\
$(167,60)$ & 11 & $(167,60)$ & 11 & 167 & YES & YES & YES & $1.43$ & $(2,3)$ & NO & 2595\\
$(169,64)$ & 11 & $(2,1)$ & 1 & 1 & YES & YES & YES & $1.43$ & $(2,3)$ & -- & 2596\\
$(169,66)$ & 11 & $(3,1)$ & 2 & 1 & YES & YES & YES & $1.50$ & $(2,3)$ & -- & 2597\\
$(169,38)$ & 13 & $(4,1)$ & 3 & 1 & YES & YES & YES & $1.57$ & $(2,3)$ & NO & 2598\\
$(169,50)$ & 11 & $(5,2)$ & 3 & 1 & YES & YES & YES & $1.38$ & $(2,3)$ & NO & 2599\\
$(169,64)$ & 11 & $(5,1)$ & 4 & 1 & YES & YES & YES & $1.38$ & $(2,3)$ & NO & 2600\\
$(169,71)$ & 11 & $(5,2)$ & 3 & 1 & YES & YES & YES & $1.43$ & $(4,2)$ & -- & 2601\\
$(169,50)$ & 11 & $(9,2)$ & 5 & 1 & YES & YES & YES & $1.38$ & $(2,3)$ & NO & 2602\\
$(169,50)$ & 11 & $(13,4)$ & 6 & 13 & YES & YES & YES & $1.38$ & $(2,3)$ & NO & 2603\\
$(169,53)$ & 13 & $(13,4)$ & 6 & 13 & YES & YES & YES & $1.50$ & $(2,3)$ & NO & 2604\\
$(169,66)$ & 11 & $(13,5)$ & 5 & 13 & YES & YES & YES & $1.38$ & $(2,3)$ & 2796 & 2605\\
$(169,66)$ & 11 & $(18,7)$ & 6 & 1 & YES & YES & YES & $1.14$ & $(4,2)$ & 2548 & 2606\\
$(169,66)$ & 11 & $(23,9)$ & 7 & 1 & YES & YES & YES & $1.29$ & $(4,2)$ & NO & 2607\\
$(169,40)$ & 13 & $(148,35)$ & 12 & 1 & YES & YES & YES & $1.50$ & $(2,3)$ & 3027 & 2608\\
$(170,47)$ & 11 & $(16,3)$ & 7 & 2 & YES & YES & YES & $1.71$ & $(2,3)$ & NO & 2609\\
$(171,65)$ & 11 & $(2,1)$ & 1 & 1 & YES & YES & YES & $1.43$ & $(2,3)$ & -- & 2610\\
$(171,74)$ & 12 & $(3,1)$ & 2 & 3 & YES & YES & YES & $1.62$ & $(2,3)$ & -- & 2611\\
$(171,65)$ & 11 & $(5,2)$ & 3 & 1 & YES & YES & YES & $1.57$ & $(4,2)$ & -- & 2612\\
$(171,65)$ & 11 & $(5,2)$ & 3 & 1 & YES & YES & YES & $1.43$ & $(2,3)$ & NO & 2613\\
$(171,50)$ & 11 & $(7,3)$ & 4 & 1 & YES & YES & YES & $1.71$ & $(2,3)$ & -- & 2614\\
$(171,71)$ & 12 & $(7,3)$ & 4 & 1 & YES & YES & YES & $1.43$ & $(2,3)$ & NO & 2615\\
$(171,74)$ & 12 & $(9,4)$ & 5 & 9 & YES & YES & YES & $1.62$ & $(2,3)$ & NO & 2616\\
$(171,71)$ & 12 & $(12,5)$ & 5 & 3 & YES & YES & YES & $1.43$ & $(4,2)$ & NO & 2617\\
$(171,65)$ & 11 & $(34,13)$ & 7 & 1 & YES & YES & YES & $1.57$ & $(4,2)$ & NO & 2618\\
$(171,71)$ & 12 & $(118,49)$ & 11 & 1 & YES & YES & YES & $1.62$ & $(2,3)$ & NO & 2619\\
$(172,63)$ & 11 & $(3,1)$ & 2 & 1 & YES & YES & YES & $1.29$ & $(2,3)$ & -- & 2620\\
$(172,71)$ & 11 & $(5,2)$ & 3 & 1 & YES & YES & YES & $1.43$ & $(2,3)$ & -- & 2621\\
$(172,71)$ & 11 & $(7,2)$ & 4 & 1 & YES & YES & YES & $1.57$ & $(2,3)$ & -- & 2622\\
$(172,63)$ & 11 & $(41,15)$ & 8 & 1 & YES & YES & YES & $1.43$ & $(2,3)$ & NO & 2623\\
$(172,71)$ & 11 & $(41,17)$ & 8 & 1 & YES & YES & YES & $1.43$ & $(2,3)$ & 2978 & 2624\\
$(172,71)$ & 11 & $(75,31)$ & 9 & 1 & YES & YES & YES & $1.57$ & $(2,3)$ & NO & 2625\\
$(173,78)$ & 12 & $(2,1)$ & 1 & 1 & YES & YES & YES & $1.43$ & $(4,2)$ & -- & 2626\\
$(173,76)$ & 11 & $(3,1)$ & 2 & 1 & YES & YES & YES & $1.29$ & $(2,3)$ & -- & 2627\\
$(173,73)$ & 11 & $(4,1)$ & 3 & 1 & YES & YES & YES & $1.38$ & $(2,3)$ & -- & 2628\\
$(173,76)$ & 11 & $(5,2)$ & 3 & 1 & YES & YES & YES & $1.29$ & $(2,3)$ & 2801 & 2629\\
$(173,73)$ & 11 & $(7,3)$ & 4 & 1 & YES & YES & YES & $1.50$ & $(2,3)$ & NO & 2630\\
$(173,41)$ & 13 & $(156,37)$ & 12 & 1 & YES & YES & YES & $1.50$ & $(2,3)$ & 3046 & 2631\\
$(173,73)$ & 11 & $(173,73)$ & 11 & 173 & YES & YES & YES & $1.50$ & $(2,3)$ & NO & 2632\\
$(175,62)$ & 12 & $(2,1)$ & 1 & 1 & YES & YES & YES & $1.43$ & $(4,2)$ & -- & 2633\\
$(175,62)$ & 12 & $(2,1)$ & 1 & 1 & YES & YES & YES & $1.57$ & $(2,3)$ & NO & 2634\\
$(175,48)$ & 12 & $(3,1)$ & 2 & 1 & YES & YES & YES & $1.50$ & $(2,3)$ & -- & 2635\\
$(175,62)$ & 12 & $(3,1)$ & 2 & 1 & YES & YES & YES & $1.43$ & $(4,2)$ & -- & 2636\\
$(175,67)$ & 11 & $(7,2)$ & 4 & 7 & YES & YES & YES & $1.62$ & $(2,3)$ & -- & 2637\\
$(175,62)$ & 12 & $(14,5)$ & 6 & 7 & YES & YES & YES & $1.43$ & $(4,2)$ & NO & 2638\\
$(175,62)$ & 12 & $(31,11)$ & 8 & 1 & YES & YES & YES & $1.43$ & $(4,2)$ & NO & 2639\\
$(175,62)$ & 12 & $(48,17)$ & 9 & 1 & YES & YES & YES & $1.57$ & $(2,3)$ & NO & 2640\\
$(175,67)$ & 11 & $(55,21)$ & 8 & 5 & YES & YES & YES & $1.75$ & $(2,3)$ & 3031 & 2641\\
$(176,65)$ & 11 & $(3,1)$ & 2 & 1 & YES & YES & YES & $1.50$ & $(2,3)$ & -- & 2642\\
$(176,69)$ & 12 & $(3,1)$ & 2 & 1 & YES & YES & YES & $1.57$ & $(2,3)$ & NO & 2643\\
$(176,69)$ & 12 & $(3,1)$ & 2 & 1 & YES & YES & YES & $1.57$ & $(2,3)$ & -- & 2644\\
$(176,79)$ & 12 & $(3,1)$ & 2 & 1 & YES & YES & YES & $1.43$ & $(4,2)$ & NO & 2645\\
$(176,69)$ & 12 & $(4,1)$ & 3 & 4 & YES & YES & YES & $1.29$ & $(4,2)$ & NO & 2646\\
$(176,51)$ & 12 & $(5,1)$ & 4 & 1 & YES & YES & YES & $1.43$ & $(2,3)$ & NO & 2647\\
$(176,51)$ & 12 & $(10,3)$ & 5 & 2 & YES & YES & YES & $1.43$ & $(2,3)$ & NO & 2648\\
$(176,69)$ & 12 & $(74,29)$ & 10 & 2 & YES & YES & YES & $1.29$ & $(4,2)$ & 2753 & 2649\\
$(177,73)$ & 12 & $(2,1)$ & 1 & 1 & YES & YES & YES & $1.71$ & $(2,3)$ & NO & 2650\\
$(177,49)$ & 11 & $(3,1)$ & 2 & 3 & NO & YES & YES & $1.50$ & $(2,3)$ & -- & 2651\\
$(177,64)$ & 12 & $(4,1)$ & 3 & 1 & YES & YES & YES & $1.62$ & $(2,3)$ & -- & 2652\\
$(177,73)$ & 12 & $(4,1)$ & 3 & 1 & YES & YES & YES & $1.57$ & $(2,3)$ & -- & 2653\\
$(177,80)$ & 12 & $(5,1)$ & 4 & 1 & YES & YES & YES & $1.50$ & $(2,3)$ & -- & 2654\\
$(177,64)$ & 12 & $(14,5)$ & 6 & 1 & YES & YES & YES & $1.62$ & $(2,3)$ & NO & 2655\\
$(177,47)$ & 12 & $(26,7)$ & 7 & 1 & YES & YES & YES & $1.50$ & $(2,3)$ & NO & 2656\\
$(177,74)$ & 12 & $(31,13)$ & 7 & 1 & YES & YES & YES & $1.62$ & $(2,3)$ & NO & 2657\\
$(177,74)$ & 12 & $(122,51)$ & 11 & 1 & YES & YES & YES & $1.62$ & $(2,3)$ & NO & 2658\\
$(178,53)$ & 12 & $(2,1)$ & 1 & 2 & YES & YES & YES & $1.57$ & $(2,3)$ & NO & 2659\\
$(178,63)$ & 12 & $(3,1)$ & 2 & 1 & YES & YES & YES & $1.43$ & $(2,3)$ & -- & 2660\\
$(178,69)$ & 11 & $(5,1)$ & 4 & 1 & YES & YES & YES & $1.50$ & $(2,3)$ & NO & 2661\\
$(178,69)$ & 11 & $(5,2)$ & 3 & 1 & YES & YES & YES & $1.75$ & $(2,3)$ & -- & 2662\\
$(178,69)$ & 11 & $(21,8)$ & 6 & 1 & YES & YES & YES & $1.75$ & $(2,3)$ & NO & 2663\\
$(178,63)$ & 12 & $(31,11)$ & 8 & 1 & YES & YES & YES & $1.43$ & $(2,3)$ & 2774 & 2664\\
$(178,69)$ & 11 & $(178,69)$ & 11 & 178 & YES & YES & YES & $1.50$ & $(2,3)$ & NO & 2665\\
$(179,74)$ & 11 & $(2,1)$ & 1 & 1 & YES & YES & YES & $1.50$ & $(2,3)$ & -- & 2666\\
$(179,74)$ & 11 & $(121,50)$ & 10 & 1 & YES & YES & YES & $1.86$ & $(2,3)$ & NO & 2667\\
$(181,65)$ & 12 & $(2,1)$ & 1 & 1 & YES & YES & YES & $1.57$ & $(2,3)$ & -- & 2668\\
$(181,75)$ & 11 & $(2,1)$ & 1 & 1 & NO & YES & YES & $1.38$ & $(2,3)$ & -- & 2669\\
$(181,54)$ & 13 & $(3,1)$ & 2 & 1 & YES & YES & YES & $1.57$ & $(2,3)$ & -- & 2670\\
$(181,70)$ & 11 & $(5,2)$ & 3 & 1 & YES & YES & YES & $1.71$ & $(2,3)$ & -- & 2671\\
$(181,75)$ & 11 & $(7,3)$ & 4 & 1 & YES & YES & YES & $1.50$ & $(2,3)$ & NO & 2672\\
$(181,41)$ & 12 & $(8,3)$ & 4 & 1 & YES & YES & YES & $1.62$ & $(2,3)$ & NO & 2673\\
$(181,70)$ & 11 & $(9,2)$ & 5 & 1 & YES & YES & YES & $1.86$ & $(2,3)$ & NO & 2674\\
$(181,65)$ & 12 & $(25,9)$ & 7 & 1 & YES & YES & YES & $1.43$ & $(2,3)$ & NO & 2675\\
$(181,70)$ & 11 & $(119,46)$ & 10 & 1 & YES & YES & YES & $1.62$ & $(2,3)$ & NO & 2676\\
$(181,65)$ & 12 & $(142,51)$ & 11 & 1 & YES & YES & YES & $1.50$ & $(2,3)$ & NO & 2677\\
$(181,54)$ & 13 & $(181,54)$ & 13 & 181 & YES & YES & YES & $1.57$ & $(2,3)$ & NO & 2678\\
$(182,79)$ & 12 & $(3,1)$ & 2 & 1 & YES & YES & YES & $1.62$ & $(2,3)$ & -- & 2679\\
$(182,79)$ & 12 & $(5,2)$ & 3 & 1 & YES & YES & YES & $1.50$ & $(2,3)$ & NO & 2680\\
$(182,79)$ & 12 & $(7,3)$ & 4 & 7 & YES & YES & YES & $1.43$ & $(4,2)$ & NO & 2681\\
$(183,67)$ & 11 & $(2,1)$ & 1 & 1 & YES & YES & YES & $1.43$ & $(2,3)$ & -- & 2682\\
$(183,67)$ & 11 & $(4,1)$ & 3 & 1 & YES & YES & YES & $1.38$ & $(2,3)$ & NO & 2683\\
$(183,67)$ & 11 & $(5,1)$ & 4 & 1 & YES & YES & YES & $1.29$ & $(2,3)$ & -- & 2684\\
$(183,82)$ & 14 & $(5,1)$ & 4 & 1 & YES & YES & YES & $1.71$ & $(2,3)$ & -- & 2685\\
$(183,67)$ & 11 & $(8,3)$ & 4 & 1 & YES & YES & YES & $1.43$ & $(2,3)$ & 2318 & 2686\\
$(184,83)$ & 12 & $(2,1)$ & 1 & 2 & YES & YES & YES & $1.29$ & $(4,2)$ & -- & 2687\\
$(184,57)$ & 12 & $(3,1)$ & 2 & 1 & YES & YES & YES & $1.43$ & $(2,3)$ & -- & 2688\\
$(184,83)$ & 12 & $(5,1)$ & 4 & 1 & YES & YES & YES & $1.71$ & $(2,3)$ & -- & 2689\\
$(184,83)$ & 12 & $(9,4)$ & 5 & 1 & YES & YES & YES & $1.29$ & $(4,2)$ & NO & 2690\\
$(184,83)$ & 12 & $(82,37)$ & 10 & 2 & YES & YES & YES & $1.57$ & $(2,3)$ & 2820 & 2691\\
$(185,58)$ & 13 & $(67,21)$ & 11 & 1 & YES & YES & YES & $1.57$ & $(2,3)$ & NO & 2692\\
$(185,58)$ & 13 & $(118,37)$ & 12 & 1 & YES & YES & YES & $1.43$ & $(2,3)$ & NO & 2693\\
$(186,71)$ & 11 & $(2,1)$ & 1 & 2 & YES & YES & YES & $1.29$ & $(4,2)$ & NO & 2694\\
$(186,71)$ & 11 & $(2,1)$ & 1 & 2 & YES & YES & YES & $1.43$ & $(2,3)$ & -- & 2695\\
$(187,79)$ & 11 & $(2,1)$ & 1 & 1 & YES & YES & YES & $1.29$ & $(2,3)$ & -- & 2696\\
$(187,79)$ & 11 & $(2,1)$ & 1 & 1 & YES & YES & YES & $1.29$ & $(2,3)$ & NO & 2697\\
$(187,71)$ & 11 & $(5,2)$ & 3 & 1 & YES & YES & YES & $1.71$ & $(2,3)$ & -- & 2698\\
$(187,79)$ & 11 & $(7,3)$ & 4 & 1 & YES & YES & YES & $1.62$ & $(2,3)$ & NO & 2699\\
$(187,71)$ & 11 & $(50,19)$ & 8 & 1 & YES & YES & YES & $1.62$ & $(2,3)$ & NO & 2700\\
$(188,85)$ & 12 & $(3,1)$ & 2 & 1 & YES & YES & YES & $1.43$ & $(2,3)$ & NO & 2701\\
$(188,85)$ & 12 & $(3,1)$ & 2 & 1 & YES & YES & YES & $1.57$ & $(2,3)$ & -- & 2702\\
$(188,85)$ & 12 & $(9,4)$ & 5 & 1 & YES & YES & YES & $1.43$ & $(2,3)$ & 2335 & 2703\\
$(188,85)$ & 12 & $(20,9)$ & 7 & 4 & YES & YES & YES & $1.57$ & $(2,3)$ & NO & 2704\\
$(189,67)$ & 12 & $(3,1)$ & 2 & 3 & YES & YES & YES & $1.57$ & $(2,3)$ & -- & 2705\\
$(189,50)$ & 13 & $(6,1)$ & 5 & 3 & YES & YES & YES & $1.62$ & $(2,3)$ & NO & 2706\\
$(189,67)$ & 12 & $(48,17)$ & 9 & 3 & YES & YES & YES & $1.57$ & $(2,3)$ & NO & 2707\\
$(191,58)$ & 12 & $(2,1)$ & 1 & 1 & YES & YES & YES & $1.50$ & $(2,3)$ & -- & 2708\\
$(191,56)$ & 12 & $(3,1)$ & 2 & 1 & YES & YES & YES & $1.50$ & $(2,3)$ & -- & 2709\\
$(191,59)$ & 13 & $(4,1)$ & 3 & 1 & YES & YES & YES & $1.62$ & $(2,3)$ & -- & 2710\\
$(191,50)$ & 13 & $(6,1)$ & 5 & 1 & YES & YES & YES & $1.43$ & $(2,3)$ & -- & 2711\\
$(191,50)$ & 13 & $(19,5)$ & 7 & 1 & YES & YES & YES & $1.57$ & $(2,3)$ & NO & 2712\\
$(191,59)$ & 13 & $(29,9)$ & 8 & 1 & YES & YES & YES & $1.62$ & $(2,3)$ & NO & 2713\\
$(191,74)$ & 11 & $(49,19)$ & 8 & 1 & YES & YES & YES & $1.50$ & $(2,3)$ & NO & 2714\\
$(191,58)$ & 12 & $(135,41)$ & 11 & 1 & YES & YES & YES & $1.38$ & $(2,3)$ & NO & 2715\\
$(192,73)$ & 11 & $(3,1)$ & 2 & 3 & YES & YES & YES & $1.50$ & $(2,3)$ & -- & 2716\\
$(192,73)$ & 11 & $(3,1)$ & 2 & 3 & YES & YES & YES & $1.50$ & $(2,3)$ & NO & 2717\\
$(192,73)$ & 11 & $(5,2)$ & 3 & 1 & YES & YES & YES & $1.57$ & $(2,3)$ & -- & 2718\\
$(192,73)$ & 11 & $(92,35)$ & 10 & 4 & YES & YES & YES & $1.71$ & $(2,3)$ & NO & 2719\\
$(193,81)$ & 11 & $(2,1)$ & 1 & 1 & YES & YES & YES & $1.29$ & $(2,3)$ & NO & 2720\\
$(193,60)$ & 12 & $(3,1)$ & 2 & 1 & YES & YES & YES & $1.43$ & $(2,3)$ & -- & 2721\\
$(193,81)$ & 11 & $(4,1)$ & 3 & 1 & YES & YES & YES & $1.38$ & $(2,3)$ & -- & 2722\\
$(193,81)$ & 11 & $(5,1)$ & 4 & 1 & YES & YES & YES & $1.50$ & $(2,3)$ & NO & 2723\\
$(193,81)$ & 11 & $(5,1)$ & 4 & 1 & YES & YES & YES & $1.50$ & $(2,3)$ & -- & 2724\\
$(193,81)$ & 11 & $(5,2)$ & 3 & 1 & YES & YES & YES & $1.71$ & $(2,3)$ & -- & 2725\\
$(193,71)$ & 12 & $(6,1)$ & 5 & 1 & YES & YES & YES & $1.50$ & $(2,3)$ & NO & 2726\\
$(193,74)$ & 12 & $(13,5)$ & 5 & 1 & YES & YES & YES & $1.29$ & $(4,2)$ & NO & 2727\\
$(193,81)$ & 11 & $(81,34)$ & 9 & 1 & YES & YES & YES & $1.50$ & $(2,3)$ & NO & 2728\\
$(193,81)$ & 11 & $(131,55)$ & 10 & 1 & YES & YES & YES & $1.86$ & $(2,3)$ & NO & 2729\\
$(193,60)$ & 12 & $(193,60)$ & 12 & 193 & YES & YES & YES & $1.43$ & $(2,3)$ & NO & 2730\\
$(193,81)$ & 11 & $(193,81)$ & 11 & 193 & YES & YES & YES & $1.38$ & $(2,3)$ & NO & 2731\\
$(194,75)$ & 11 & $(5,2)$ & 3 & 1 & YES & YES & YES & $1.38$ & $(2,3)$ & 2362 & 2732\\
$(194,75)$ & 11 & $(18,7)$ & 6 & 2 & YES & YES & YES & $1.38$ & $(2,3)$ & NO & 2733\\
$(194,75)$ & 11 & $(163,63)$ & 11 & 1 & YES & YES & YES & $1.57$ & $(2,3)$ & NO & 2734\\
$(194,75)$ & 11 & $(194,75)$ & 11 & 194 & YES & YES & YES & $1.38$ & $(2,3)$ & NO & 2735\\
$(196,81)$ & 11 & $(2,1)$ & 1 & 2 & NO & YES & YES & $1.38$ & $(2,3)$ & -- & 2736\\
$(196,75)$ & 11 & $(3,1)$ & 2 & 1 & YES & YES & YES & $1.50$ & $(2,3)$ & -- & 2737\\
$(196,45)$ & 13 & $(35,8)$ & 8 & 7 & YES & YES & YES & $1.38$ & $(2,3)$ & NO & 2738\\
$(196,75)$ & 11 & $(47,18)$ & 8 & 1 & YES & YES & YES & $1.50$ & $(2,3)$ & NO & 2739\\
$(197,54)$ & 13 & $(2,1)$ & 1 & 1 & YES & YES & YES & $1.43$ & $(4,2)$ & -- & 2740\\
$(197,54)$ & 13 & $(2,1)$ & 1 & 1 & YES & YES & YES & $1.43$ & $(4,2)$ & NO & 2741\\
$(197,52)$ & 12 & $(3,1)$ & 2 & 1 & YES & YES & YES & $1.43$ & $(2,3)$ & NO & 2742\\
$(197,54)$ & 13 & $(3,1)$ & 2 & 1 & YES & YES & YES & $1.57$ & $(2,3)$ & -- & 2743\\
$(197,76)$ & 12 & $(5,1)$ & 4 & 1 & YES & YES & YES & $1.29$ & $(4,2)$ & NO & 2744\\
$(197,43)$ & 12 & $(10,3)$ & 5 & 1 & YES & YES & YES & $1.62$ & $(2,3)$ & NO & 2745\\
$(197,76)$ & 12 & $(70,27)$ & 10 & 1 & YES & YES & YES & $1.43$ & $(4,2)$ & NO & 2746\\
$(198,71)$ & 12 & $(2,1)$ & 1 & 2 & YES & YES & YES & $1.43$ & $(4,2)$ & -- & 2747\\
$(198,71)$ & 12 & $(39,14)$ & 8 & 3 & YES & YES & YES & $1.29$ & $(4,2)$ & NO & 2748\\
$(199,78)$ & 12 & $(2,1)$ & 1 & 1 & YES & YES & YES & $1.29$ & $(4,2)$ & -- & 2749\\
$(199,47)$ & 13 & $(5,2)$ & 3 & 1 & YES & YES & YES & $1.50$ & $(2,3)$ & NO & 2750\\
$(199,89)$ & 14 & $(9,4)$ & 5 & 1 & YES & YES & YES & $1.71$ & $(2,3)$ & NO & 2751\\
$(199,89)$ & 14 & $(38,17)$ & 9 & 1 & YES & YES & YES & $1.71$ & $(2,3)$ & NO & 2752\\
$(199,78)$ & 12 & $(51,20)$ & 9 & 1 & YES & YES & YES & $1.29$ & $(4,2)$ & 2649 & 2753\\
$(200,61)$ & 12 & $(7,2)$ & 4 & 1 & YES & YES & YES & $1.50$ & $(2,3)$ & NO & 2754\\
$(201,76)$ & 12 & $(2,1)$ & 1 & 1 & YES & YES & YES & $1.57$ & $(2,3)$ & -- & 2755\\
$(201,76)$ & 12 & $(3,1)$ & 2 & 3 & YES & YES & YES & $1.43$ & $(2,3)$ & NO & 2756\\
$(201,76)$ & 12 & $(5,2)$ & 3 & 1 & YES & YES & YES & $1.43$ & $(4,2)$ & NO & 2757\\
$(201,76)$ & 12 & $(82,31)$ & 10 & 1 & YES & YES & YES & $1.57$ & $(2,3)$ & NO & 2758\\
$(202,53)$ & 13 & $(2,1)$ & 1 & 2 & YES & YES & YES & $1.50$ & $(2,3)$ & -- & 2759\\
$(202,89)$ & 12 & $(2,1)$ & 1 & 2 & YES & YES & YES & $1.29$ & $(4,2)$ & -- & 2760\\
$(202,73)$ & 12 & $(4,1)$ & 3 & 2 & YES & YES & YES & $1.62$ & $(2,3)$ & NO & 2761\\
$(202,61)$ & 13 & $(6,1)$ & 5 & 2 & YES & YES & YES & $1.43$ & $(2,3)$ & NO & 2762\\
$(202,61)$ & 13 & $(6,1)$ & 5 & 2 & YES & YES & YES & $1.43$ & $(2,3)$ & -- & 2763\\
$(202,89)$ & 12 & $(7,3)$ & 4 & 1 & YES & YES & YES & $1.29$ & $(4,2)$ & NO & 2764\\
$(203,59)$ & 12 & $(24,7)$ & 7 & 1 & YES & YES & YES & $1.14$ & $(4,2)$ & 2344 & 2765\\
$(205,61)$ & 12 & $(2,1)$ & 1 & 1 & YES & YES & YES & $1.50$ & $(2,3)$ & NO & 2766\\
$(205,92)$ & 12 & $(3,1)$ & 2 & 1 & YES & YES & YES & $1.57$ & $(2,3)$ & NO & 2767\\
$(205,89)$ & 12 & $(7,3)$ & 4 & 1 & YES & YES & YES & $1.50$ & $(2,3)$ & NO & 2768\\
$(205,92)$ & 12 & $(11,5)$ & 6 & 1 & YES & YES & YES & $1.43$ & $(2,3)$ & 2819 & 2769\\
$(206,73)$ & 12 & $(2,1)$ & 1 & 2 & YES & YES & YES & $1.57$ & $(2,3)$ & -- & 2770\\
$(206,73)$ & 12 & $(3,1)$ & 2 & 1 & YES & YES & YES & $1.57$ & $(2,3)$ & -- & 2771\\
$(206,85)$ & 12 & $(3,1)$ & 2 & 1 & YES & YES & YES & $1.43$ & $(2,3)$ & NO & 2772\\
$(206,73)$ & 12 & $(14,5)$ & 6 & 2 & YES & YES & YES & $1.57$ & $(2,3)$ & NO & 2773\\
$(206,73)$ & 12 & $(17,6)$ & 7 & 1 & YES & YES & YES & $1.43$ & $(2,3)$ & 2664 & 2774\\
$(207,37)$ & 15 & $(2,1)$ & 1 & 1 & YES & YES & YES & $1.29$ & $(4,2)$ & -- & 2775\\
$(207,49)$ & 14 & $(4,1)$ & 3 & 1 & YES & YES & YES & $1.62$ & $(2,3)$ & -- & 2776\\
$(207,80)$ & 12 & $(4,1)$ & 3 & 1 & YES & YES & YES & $1.43$ & $(2,3)$ & -- & 2777\\
$(207,91)$ & 12 & $(5,1)$ & 4 & 1 & YES & YES & YES & $1.50$ & $(2,3)$ & -- & 2778\\
$(207,91)$ & 12 & $(5,1)$ & 4 & 1 & YES & YES & YES & $1.62$ & $(2,3)$ & NO & 2779\\
$(207,76)$ & 11 & $(30,11)$ & 7 & 3 & YES & YES & YES & $1.50$ & $(2,3)$ & NO & 2780\\
$(207,79)$ & 11 & $(131,50)$ & 10 & 1 & YES & YES & YES & $1.50$ & $(2,3)$ & NO & 2781\\
$(207,49)$ & 14 & $(169,40)$ & 13 & 1 & YES & YES & YES & $1.62$ & $(2,3)$ & NO & 2782\\
$(208,61)$ & 12 & $(2,1)$ & 1 & 2 & YES & YES & YES & $1.38$ & $(2,3)$ & NO & 2783\\
$(208,79)$ & 11 & $(2,1)$ & 1 & 2 & YES & YES & YES & $1.50$ & $(2,3)$ & -- & 2784\\
$(208,55)$ & 12 & $(3,1)$ & 2 & 1 & YES & YES & YES & $1.43$ & $(2,3)$ & -- & 2785\\
$(208,75)$ & 12 & $(3,1)$ & 2 & 1 & YES & YES & YES & $1.62$ & $(2,3)$ & -- & 2786\\
$(208,75)$ & 12 & $(4,1)$ & 3 & 4 & YES & YES & YES & $1.43$ & $(4,2)$ & -- & 2787\\
$(208,55)$ & 12 & $(5,1)$ & 4 & 1 & YES & YES & YES & $1.29$ & $(2,3)$ & -- & 2788\\
$(208,79)$ & 11 & $(5,1)$ & 4 & 1 & YES & YES & YES & $1.38$ & $(2,3)$ & NO & 2789\\
$(208,79)$ & 11 & $(5,2)$ & 3 & 1 & YES & YES & YES & $1.57$ & $(2,3)$ & NO & 2790\\
$(208,75)$ & 12 & $(8,3)$ & 4 & 8 & YES & YES & YES & $1.71$ & $(2,3)$ & NO & 2791\\
$(208,75)$ & 12 & $(14,5)$ & 6 & 2 & YES & YES & YES & $1.50$ & $(2,3)$ & 2468 & 2792\\
$(208,37)$ & 13 & $(39,7)$ & 9 & 13 & YES & YES & YES & $1.38$ & $(2,3)$ & NO & 2793\\
$(209,81)$ & 11 & $(2,1)$ & 1 & 1 & YES & YES & YES & $1.62$ & $(2,3)$ & -- & 2794\\
$(209,56)$ & 12 & $(5,2)$ & 3 & 1 & YES & YES & YES & $1.29$ & $(4,2)$ & NO & 2795\\
$(209,81)$ & 11 & $(5,2)$ & 3 & 1 & YES & YES & YES & $1.38$ & $(2,3)$ & 2605 & 2796\\
$(211,50)$ & 14 & $(4,1)$ & 3 & 1 & YES & YES & YES & $1.62$ & $(2,3)$ & -- & 2797\\
$(211,50)$ & 14 & $(6,1)$ & 5 & 1 & YES & YES & YES & $1.62$ & $(2,3)$ & NO & 2798\\
$(211,93)$ & 12 & $(59,26)$ & 9 & 1 & YES & YES & YES & $1.43$ & $(2,3)$ & NO & 2799\\
$(211,50)$ & 14 & $(173,41)$ & 13 & 1 & YES & YES & YES & $1.62$ & $(2,3)$ & NO & 2800\\
$(212,89)$ & 11 & $(2,1)$ & 1 & 2 & YES & YES & YES & $1.29$ & $(2,3)$ & 2629 & 2801\\
$(212,89)$ & 11 & $(5,2)$ & 3 & 1 & YES & YES & YES & $1.71$ & $(2,3)$ & -- & 2802\\
$(212,89)$ & 11 & $(69,29)$ & 9 & 1 & YES & YES & YES & $1.71$ & $(2,3)$ & NO & 2803\\
$(212,89)$ & 11 & $(112,47)$ & 10 & 4 & YES & YES & YES & $1.71$ & $(2,3)$ & NO & 2804\\
$(213,88)$ & 12 & $(2,1)$ & 1 & 1 & YES & YES & YES & $1.71$ & $(2,3)$ & -- & 2805\\
$(213,88)$ & 12 & $(3,1)$ & 2 & 3 & YES & YES & YES & $1.57$ & $(2,3)$ & -- & 2806\\
$(213,62)$ & 12 & $(4,1)$ & 3 & 1 & YES & YES & YES & $1.50$ & $(2,3)$ & -- & 2807\\
$(213,65)$ & 12 & $(4,1)$ & 3 & 1 & YES & YES & YES & $1.50$ & $(2,3)$ & NO & 2808\\
$(213,83)$ & 12 & $(18,7)$ & 6 & 3 & YES & YES & YES & $1.62$ & $(2,3)$ & NO & 2809\\
$(213,88)$ & 12 & $(29,12)$ & 7 & 1 & YES & YES & YES & $1.57$ & $(2,3)$ & NO & 2810\\
$(213,62)$ & 12 & $(134,39)$ & 11 & 1 & YES & YES & YES & $1.50$ & $(2,3)$ & NO & 2811\\
$(214,65)$ & 12 & $(2,1)$ & 1 & 2 & YES & YES & YES & $1.50$ & $(2,3)$ & NO & 2812\\
$(214,83)$ & 12 & $(2,1)$ & 1 & 2 & YES & YES & YES & $1.43$ & $(4,2)$ & -- & 2813\\
$(214,83)$ & 12 & $(5,1)$ & 4 & 1 & YES & YES & YES & $1.57$ & $(2,3)$ & NO & 2814\\
$(214,83)$ & 12 & $(5,1)$ & 4 & 1 & YES & YES & YES & $1.57$ & $(2,3)$ & -- & 2815\\
$(214,83)$ & 12 & $(5,1)$ & 4 & 1 & YES & YES & YES & $1.71$ & $(2,3)$ & NO & 2816\\
$(214,83)$ & 12 & $(165,64)$ & 11 & 1 & YES & YES & YES & $1.57$ & $(2,3)$ & NO & 2817\\
$(215,97)$ & 12 & $(2,1)$ & 1 & 1 & YES & YES & YES & $1.43$ & $(2,3)$ & -- & 2818\\
$(215,97)$ & 12 & $(9,4)$ & 5 & 1 & YES & YES & YES & $1.43$ & $(2,3)$ & 2769 & 2819\\
$(215,97)$ & 12 & $(51,23)$ & 9 & 1 & YES & YES & YES & $1.57$ & $(2,3)$ & 2691 & 2820\\
$(217,60)$ & 12 & $(5,2)$ & 3 & 1 & YES & YES & YES & $1.71$ & $(2,3)$ & NO & 2821\\
$(217,60)$ & 12 & $(10,3)$ & 5 & 1 & YES & YES & YES & $1.57$ & $(2,3)$ & NO & 2822\\
$(217,64)$ & 12 & $(17,5)$ & 6 & 1 & YES & YES & YES & $1.43$ & $(2,3)$ & NO & 2823\\
$(218,79)$ & 14 & $(2,1)$ & 1 & 2 & YES & YES & YES & $1.71$ & $(4,2)$ & NO & 2824\\
$(218,47)$ & 13 & $(65,14)$ & 10 & 1 & YES & YES & YES & $1.43$ & $(2,3)$ & NO & 2825\\
$(218,79)$ & 14 & $(69,25)$ & 11 & 1 & YES & YES & YES & $1.57$ & $(4,2)$ & NO & 2826\\
$(219,95)$ & 12 & $(2,1)$ & 1 & 1 & YES & YES & YES & $1.62$ & $(2,3)$ & -- & 2827\\
$(219,85)$ & 12 & $(4,1)$ & 3 & 1 & YES & YES & YES & $1.43$ & $(4,2)$ & NO & 2828\\
$(219,95)$ & 12 & $(7,3)$ & 4 & 1 & YES & YES & YES & $1.62$ & $(2,3)$ & NO & 2829\\
$(219,85)$ & 12 & $(18,7)$ & 6 & 3 & YES & YES & YES & $1.29$ & $(4,2)$ & NO & 2830\\
$(221,58)$ & 13 & $(2,1)$ & 1 & 1 & YES & YES & YES & $1.57$ & $(2,3)$ & NO & 2831\\
$(221,58)$ & 13 & $(3,1)$ & 2 & 1 & YES & YES & YES & $1.43$ & $(2,3)$ & NO & 2832\\
$(221,84)$ & 12 & $(3,1)$ & 2 & 1 & YES & YES & YES & $1.57$ & $(4,2)$ & -- & 2833\\
$(221,84)$ & 12 & $(29,11)$ & 7 & 1 & YES & YES & YES & $1.57$ & $(4,2)$ & NO & 2834\\
$(221,84)$ & 12 & $(221,84)$ & 12 & 221 & YES & YES & YES & $1.43$ & $(4,2)$ & NO & 2835\\
$(222,59)$ & 13 & $(34,9)$ & 8 & 2 & YES & YES & YES & $1.62$ & $(2,3)$ & NO & 2836\\
$(222,91)$ & 12 & $(61,25)$ & 9 & 1 & YES & YES & YES & $1.57$ & $(2,3)$ & NO & 2837\\
$(223,82)$ & 12 & $(3,1)$ & 2 & 1 & YES & YES & YES & $1.43$ & $(2,3)$ & -- & 2838\\
$(223,48)$ & 13 & $(6,1)$ & 5 & 1 & YES & YES & YES & $1.29$ & $(2,3)$ & NO & 2839\\
$(223,66)$ & 12 & $(27,8)$ & 7 & 1 & YES & YES & YES & $1.43$ & $(2,3)$ & 2515 & 2840\\
$(223,82)$ & 12 & $(30,11)$ & 7 & 1 & YES & YES & YES & $1.43$ & $(2,3)$ & NO & 2841\\
$(225,98)$ & 12 & $(23,10)$ & 7 & 1 & YES & YES & YES & $1.43$ & $(4,2)$ & NO & 2842\\
$(226,95)$ & 12 & $(2,1)$ & 1 & 2 & YES & YES & YES & $1.43$ & $(2,3)$ & -- & 2843\\
$(227,86)$ & 12 & $(2,1)$ & 1 & 1 & YES & YES & YES & $1.62$ & $(2,3)$ & NO & 2844\\
$(227,60)$ & 12 & $(3,1)$ & 2 & 1 & YES & YES & YES & $1.43$ & $(2,3)$ & NO & 2845\\
$(227,60)$ & 12 & $(3,1)$ & 2 & 1 & YES & YES & YES & $1.43$ & $(2,3)$ & -- & 2846\\
$(227,86)$ & 12 & $(4,1)$ & 3 & 1 & YES & YES & YES & $1.29$ & $(4,2)$ & NO & 2847\\
$(227,86)$ & 12 & $(8,3)$ & 4 & 1 & YES & YES & YES & $1.71$ & $(2,3)$ & NO & 2848\\
$(227,60)$ & 12 & $(15,4)$ & 6 & 1 & YES & YES & YES & $1.43$ & $(2,3)$ & NO & 2849\\
$(227,88)$ & 12 & $(31,12)$ & 7 & 1 & YES & YES & YES & $1.43$ & $(4,2)$ & NO & 2850\\
$(227,82)$ & 14 & $(36,13)$ & 8 & 1 & YES & YES & YES & $1.57$ & $(4,2)$ & NO & 2851\\
$(227,60)$ & 12 & $(87,23)$ & 10 & 1 & YES & YES & YES & $1.43$ & $(2,3)$ & NO & 2852\\
$(228,61)$ & 13 & $(2,1)$ & 1 & 2 & YES & YES & YES & $1.43$ & $(2,3)$ & NO & 2853\\
$(229,64)$ & 12 & $(5,2)$ & 3 & 1 & YES & YES & YES & $1.57$ & $(2,3)$ & NO & 2854\\
$(229,95)$ & 12 & $(12,5)$ & 5 & 1 & YES & YES & YES & $1.50$ & $(2,3)$ & 2442 & 2855\\
$(229,87)$ & 12 & $(229,87)$ & 12 & 229 & YES & YES & YES & $1.43$ & $(4,2)$ & NO & 2856\\
$(231,53)$ & 13 & $(2,1)$ & 1 & 1 & YES & YES & YES & $1.43$ & $(2,3)$ & -- & 2857\\
$(234,49)$ & 14 & $(4,1)$ & 3 & 2 & YES & YES & YES & $1.50$ & $(2,3)$ & -- & 2858\\
$(234,53)$ & 13 & $(9,2)$ & 5 & 9 & YES & YES & YES & $1.38$ & $(2,3)$ & NO & 2859\\
$(234,49)$ & 14 & $(14,3)$ & 6 & 2 & YES & YES & YES & $1.50$ & $(2,3)$ & NO & 2860\\
$(234,71)$ & 12 & $(33,10)$ & 8 & 3 & YES & YES & YES & $1.50$ & $(2,3)$ & NO & 2861\\
$(234,89)$ & 12 & $(50,19)$ & 8 & 2 & YES & YES & YES & $1.43$ & $(4,2)$ & NO & 2862\\
$(235,66)$ & 12 & $(7,2)$ & 4 & 1 & YES & YES & YES & $1.43$ & $(2,3)$ & NO & 2863\\
$(235,97)$ & 12 & $(7,3)$ & 4 & 1 & YES & YES & YES & $1.57$ & $(2,3)$ & NO & 2864\\
$(236,69)$ & 12 & $(65,19)$ & 9 & 1 & YES & YES & YES & $1.50$ & $(2,3)$ & NO & 2865\\
$(237,100)$ & 12 & $(19,8)$ & 6 & 1 & YES & YES & YES & $1.57$ & $(2,3)$ & NO & 2866\\
$(239,50)$ & 14 & $(3,1)$ & 2 & 1 & YES & YES & YES & $1.50$ & $(2,3)$ & -- & 2867\\
$(239,85)$ & 14 & $(3,1)$ & 2 & 1 & YES & YES & YES & $1.57$ & $(4,2)$ & NO & 2868\\
$(239,50)$ & 14 & $(4,1)$ & 3 & 1 & YES & YES & YES & $1.50$ & $(2,3)$ & -- & 2869\\
$(239,101)$ & 12 & $(5,2)$ & 3 & 1 & YES & YES & YES & $1.29$ & $(4,2)$ & NO & 2870\\
$(239,50)$ & 14 & $(9,2)$ & 5 & 1 & YES & YES & YES & $1.50$ & $(2,3)$ & NO & 2871\\
$(239,100)$ & 12 & $(12,5)$ & 5 & 1 & YES & YES & YES & $1.50$ & $(2,3)$ & 2464 & 2872\\
$(239,50)$ & 14 & $(14,3)$ & 6 & 1 & YES & YES & YES & $1.50$ & $(2,3)$ & NO & 2873\\
$(239,99)$ & 12 & $(41,17)$ & 8 & 1 & YES & YES & YES & $1.29$ & $(4,2)$ & NO & 2874\\
$(239,71)$ & 12 & $(44,13)$ & 8 & 1 & YES & YES & YES & $1.57$ & $(2,3)$ & NO & 2875\\
$(240,71)$ & 12 & $(3,1)$ & 2 & 3 & YES & YES & YES & $1.50$ & $(2,3)$ & NO & 2876\\
$(241,89)$ & 12 & $(19,7)$ & 6 & 1 & YES & YES & YES & $1.57$ & $(2,3)$ & NO & 2877\\
$(241,89)$ & 12 & $(241,89)$ & 12 & 241 & YES & YES & YES & $1.57$ & $(2,3)$ & NO & 2878\\
$(242,45)$ & 14 & $(242,45)$ & 14 & 242 & YES & YES & YES & $1.29$ & $(2,3)$ & NO & 2879\\
$(243,71)$ & 12 & $(3,1)$ & 2 & 3 & YES & YES & YES & $1.38$ & $(2,3)$ & -- & 2880\\
$(243,94)$ & 12 & $(3,1)$ & 2 & 3 & YES & YES & YES & $1.71$ & $(2,3)$ & -- & 2881\\
$(243,94)$ & 12 & $(3,1)$ & 2 & 3 & YES & YES & YES & $1.71$ & $(2,3)$ & NO & 2882\\
$(243,71)$ & 12 & $(4,1)$ & 3 & 1 & YES & YES & YES & $1.38$ & $(2,3)$ & -- & 2883\\
$(243,71)$ & 12 & $(13,4)$ & 6 & 1 & YES & YES & YES & $1.57$ & $(2,3)$ & NO & 2884\\
$(243,46)$ & 15 & $(16,3)$ & 7 & 1 & YES & YES & YES & $1.50$ & $(2,3)$ & NO & 2885\\
$(245,69)$ & 13 & $(3,1)$ & 2 & 1 & YES & YES & YES & $1.57$ & $(2,3)$ & NO & 2886\\
$(245,74)$ & 13 & $(3,1)$ & 2 & 1 & YES & YES & YES & $1.62$ & $(2,3)$ & NO & 2887\\
$(245,93)$ & 12 & $(245,93)$ & 12 & 245 & YES & YES & YES & $1.57$ & $(2,3)$ & NO & 2888\\
$(246,53)$ & 13 & $(2,1)$ & 1 & 2 & YES & YES & YES & $1.43$ & $(2,3)$ & NO & 2889\\
$(246,73)$ & 12 & $(3,1)$ & 2 & 3 & YES & YES & YES & $1.50$ & $(2,3)$ & NO & 2890\\
$(246,91)$ & 12 & $(3,1)$ & 2 & 3 & YES & YES & YES & $1.43$ & $(4,2)$ & NO & 2891\\
$(246,95)$ & 12 & $(4,1)$ & 3 & 2 & YES & YES & YES & $1.86$ & $(2,3)$ & NO & 2892\\
$(246,95)$ & 12 & $(4,1)$ & 3 & 2 & YES & YES & YES & $1.86$ & $(2,3)$ & -- & 2893\\
$(246,53)$ & 13 & $(5,1)$ & 4 & 1 & YES & YES & YES & $1.29$ & $(2,3)$ & -- & 2894\\
$(246,91)$ & 12 & $(27,10)$ & 7 & 3 & YES & YES & YES & $1.43$ & $(2,3)$ & NO & 2895\\
$(246,95)$ & 12 & $(31,12)$ & 7 & 1 & YES & YES & YES & $1.86$ & $(2,3)$ & NO & 2896\\
$(247,68)$ & 12 & $(2,1)$ & 1 & 1 & YES & YES & YES & $1.50$ & $(2,3)$ & -- & 2897\\
$(249,95)$ & 12 & $(2,1)$ & 1 & 1 & YES & YES & YES & $1.43$ & $(4,2)$ & -- & 2898\\
$(249,95)$ & 12 & $(55,21)$ & 8 & 1 & YES & YES & YES & $1.29$ & $(4,2)$ & NO & 2899\\
$(249,95)$ & 12 & $(249,95)$ & 12 & 249 & YES & YES & YES & $1.57$ & $(2,3)$ & NO & 2900\\
$(250,59)$ & 14 & $(250,59)$ & 14 & 250 & YES & YES & YES & $1.50$ & $(2,3)$ & NO & 2901\\
$(251,104)$ & 12 & $(5,2)$ & 3 & 1 & YES & YES & YES & $1.57$ & $(2,3)$ & NO & 2902\\
$(251,104)$ & 12 & $(41,17)$ & 8 & 1 & YES & YES & YES & $1.29$ & $(4,2)$ & NO & 2903\\
$(253,74)$ & 12 & $(2,1)$ & 1 & 1 & YES & YES & YES & $1.50$ & $(2,3)$ & NO & 2904\\
$(253,60)$ & 13 & $(3,1)$ & 2 & 1 & YES & YES & YES & $1.43$ & $(2,3)$ & NO & 2905\\
$(253,106)$ & 12 & $(5,2)$ & 3 & 1 & YES & YES & YES & $1.43$ & $(4,2)$ & NO & 2906\\
$(253,57)$ & 13 & $(9,2)$ & 5 & 1 & YES & YES & YES & $1.38$ & $(2,3)$ & NO & 2907\\
$(253,48)$ & 14 & $(37,7)$ & 9 & 1 & YES & YES & YES & $1.29$ & $(2,3)$ & NO & 2908\\
$(254,105)$ & 12 & $(2,1)$ & 1 & 2 & YES & YES & YES & $1.29$ & $(4,2)$ & -- & 2909\\
$(254,105)$ & 12 & $(17,7)$ & 6 & 1 & YES & YES & YES & $1.43$ & $(4,2)$ & NO & 2910\\
$(254,105)$ & 12 & $(46,19)$ & 8 & 2 & YES & YES & YES & $1.71$ & $(2,3)$ & NO & 2911\\
$(254,105)$ & 12 & $(254,105)$ & 12 & 254 & YES & YES & YES & $1.57$ & $(2,3)$ & NO & 2912\\
$(255,47)$ & 15 & $(16,3)$ & 7 & 1 & YES & YES & YES & $1.43$ & $(2,3)$ & NO & 2913\\
$(255,97)$ & 12 & $(255,97)$ & 12 & 255 & YES & YES & YES & $1.71$ & $(2,3)$ & NO & 2914\\
$(256,97)$ & 12 & $(4,1)$ & 3 & 4 & YES & YES & YES & $1.57$ & $(2,3)$ & NO & 2915\\
$(256,99)$ & 12 & $(8,3)$ & 4 & 8 & YES & YES & YES & $1.62$ & $(2,3)$ & NO & 2916\\
$(256,99)$ & 12 & $(13,5)$ & 5 & 1 & YES & YES & YES & $1.29$ & $(4,2)$ & NO & 2917\\
$(256,99)$ & 12 & $(44,17)$ & 8 & 4 & YES & YES & YES & $1.62$ & $(2,3)$ & NO & 2918\\
$(257,108)$ & 12 & $(2,1)$ & 1 & 1 & YES & YES & YES & $1.71$ & $(2,3)$ & -- & 2919\\
$(257,108)$ & 12 & $(3,1)$ & 2 & 1 & YES & YES & YES & $1.57$ & $(2,3)$ & -- & 2920\\
$(257,45)$ & 15 & $(23,4)$ & 8 & 1 & YES & YES & YES & $1.50$ & $(2,3)$ & NO & 2921\\
$(258,71)$ & 12 & $(2,1)$ & 1 & 2 & YES & YES & YES & $1.50$ & $(2,3)$ & -- & 2922\\
$(258,55)$ & 14 & $(6,1)$ & 5 & 6 & YES & YES & YES & $1.43$ & $(2,3)$ & NO & 2923\\
$(258,55)$ & 14 & $(258,55)$ & 14 & 258 & YES & YES & YES & $1.43$ & $(2,3)$ & NO & 2924\\
$(259,100)$ & 12 & $(3,1)$ & 2 & 1 & YES & YES & YES & $1.57$ & $(2,3)$ & -- & 2925\\
$(259,100)$ & 12 & $(3,1)$ & 2 & 1 & YES & YES & YES & $1.57$ & $(2,3)$ & NO & 2926\\
$(259,59)$ & 13 & $(5,1)$ & 4 & 1 & YES & YES & YES & $1.38$ & $(2,3)$ & NO & 2927\\
$(259,59)$ & 13 & $(22,5)$ & 7 & 1 & YES & YES & YES & $1.38$ & $(2,3)$ & NO & 2928\\
$(261,50)$ & 15 & $(3,1)$ & 2 & 3 & YES & YES & YES & $1.50$ & $(2,3)$ & -- & 2929\\
$(261,50)$ & 15 & $(4,1)$ & 3 & 1 & YES & YES & YES & $1.50$ & $(2,3)$ & -- & 2930\\
$(261,76)$ & 13 & $(7,2)$ & 4 & 1 & YES & YES & YES & $1.43$ & $(2,3)$ & 2268 & 2931\\
$(261,50)$ & 15 & $(11,2)$ & 6 & 1 & YES & YES & YES & $1.50$ & $(2,3)$ & NO & 2932\\
$(261,50)$ & 15 & $(16,3)$ & 7 & 1 & YES & YES & YES & $1.50$ & $(2,3)$ & NO & 2933\\
$(261,76)$ & 13 & $(55,16)$ & 9 & 1 & YES & YES & YES & $1.43$ & $(2,3)$ & NO & 2934\\
$(261,100)$ & 12 & $(261,100)$ & 12 & 261 & YES & YES & YES & $1.43$ & $(2,3)$ & NO & 2935\\
$(262,47)$ & 15 & $(3,1)$ & 2 & 1 & YES & YES & YES & $1.43$ & $(2,3)$ & -- & 2936\\
$(262,47)$ & 15 & $(17,3)$ & 7 & 1 & YES & YES & YES & $1.43$ & $(2,3)$ & NO & 2937\\
$(263,60)$ & 13 & $(9,2)$ & 5 & 1 & YES & YES & YES & $1.38$ & $(2,3)$ & NO & 2938\\
$(263,61)$ & 14 & $(69,16)$ & 11 & 1 & YES & YES & YES & $1.43$ & $(2,3)$ & NO & 2939\\
$(263,57)$ & 13 & $(263,57)$ & 13 & 263 & YES & YES & YES & $1.50$ & $(2,3)$ & NO & 2940\\
$(264,109)$ & 12 & $(2,1)$ & 1 & 2 & YES & YES & YES & $1.71$ & $(2,3)$ & -- & 2941\\
$(264,109)$ & 12 & $(3,1)$ & 2 & 3 & YES & YES & YES & $1.57$ & $(2,3)$ & NO & 2942\\
$(264,109)$ & 12 & $(3,1)$ & 2 & 3 & YES & YES & YES & $1.71$ & $(2,3)$ & -- & 2943\\
$(265,73)$ & 12 & $(4,1)$ & 3 & 1 & YES & YES & YES & $1.50$ & $(2,3)$ & NO & 2944\\
$(265,112)$ & 12 & $(4,1)$ & 3 & 1 & YES & YES & YES & $1.57$ & $(2,3)$ & NO & 2945\\
$(265,112)$ & 12 & $(5,2)$ & 3 & 5 & YES & YES & YES & $1.43$ & $(2,3)$ & NO & 2946\\
$(266,101)$ & 12 & $(2,1)$ & 1 & 2 & YES & YES & YES & $1.29$ & $(4,2)$ & NO & 2947\\
$(268,111)$ & 12 & $(41,17)$ & 8 & 1 & YES & YES & YES & $1.43$ & $(2,3)$ & 3009 & 2948\\
$(269,104)$ & 12 & $(2,1)$ & 1 & 1 & YES & YES & YES & $1.43$ & $(4,2)$ & NO & 2949\\
$(269,104)$ & 12 & $(13,5)$ & 5 & 1 & YES & YES & YES & $1.29$ & $(4,2)$ & NO & 2950\\
$(269,104)$ & 12 & $(18,7)$ & 6 & 1 & YES & YES & YES & $1.71$ & $(2,3)$ & NO & 2951\\
$(269,104)$ & 12 & $(31,12)$ & 7 & 1 & YES & YES & YES & $1.62$ & $(2,3)$ & NO & 2952\\
$(270,103)$ & 12 & $(173,66)$ & 11 & 1 & YES & YES & YES & $1.57$ & $(2,3)$ & NO & 2953\\
$(271,112)$ & 12 & $(2,1)$ & 1 & 1 & YES & YES & YES & $1.29$ & $(4,2)$ & -- & 2954\\
$(271,112)$ & 12 & $(3,1)$ & 2 & 1 & YES & YES & YES & $1.57$ & $(2,3)$ & -- & 2955\\
$(271,112)$ & 12 & $(12,5)$ & 5 & 1 & YES & YES & YES & $1.43$ & $(2,3)$ & NO & 2956\\
$(272,59)$ & 13 & $(3,1)$ & 2 & 1 & YES & YES & YES & $1.38$ & $(2,3)$ & -- & 2957\\
$(273,101)$ & 12 & $(2,1)$ & 1 & 1 & YES & YES & YES & $1.43$ & $(2,3)$ & -- & 2958\\
$(273,101)$ & 12 & $(100,37)$ & 10 & 1 & YES & YES & YES & $1.43$ & $(2,3)$ & NO & 2959\\
$(274,115)$ & 12 & $(2,1)$ & 1 & 2 & NO & YES & YES & $1.29$ & $(4,2)$ & -- & 2960\\
$(274,81)$ & 12 & $(11,3)$ & 5 & 1 & YES & YES & YES & $1.71$ & $(2,3)$ & NO & 2961\\
$(274,43)$ & 15 & $(20,3)$ & 8 & 2 & YES & YES & YES & $1.29$ & $(4,2)$ & NO & 2962\\
$(274,115)$ & 12 & $(50,21)$ & 8 & 2 & YES & YES & YES & $1.71$ & $(2,3)$ & NO & 2963\\
$(274,65)$ & 14 & $(156,37)$ & 12 & 2 & YES & YES & YES & $1.62$ & $(2,3)$ & 3101 & 2964\\
$(277,106)$ & 12 & $(3,1)$ & 2 & 1 & YES & YES & YES & $1.62$ & $(2,3)$ & NO & 2965\\
$(277,106)$ & 12 & $(3,1)$ & 2 & 1 & YES & YES & YES & $1.62$ & $(2,3)$ & -- & 2966\\
$(277,116)$ & 12 & $(3,1)$ & 2 & 1 & YES & YES & YES & $1.71$ & $(2,3)$ & -- & 2967\\
$(277,116)$ & 12 & $(3,1)$ & 2 & 1 & YES & YES & YES & $1.86$ & $(2,3)$ & NO & 2968\\
$(277,116)$ & 12 & $(5,2)$ & 3 & 1 & YES & YES & YES & $1.71$ & $(2,3)$ & NO & 2969\\
$(277,106)$ & 12 & $(8,3)$ & 4 & 1 & YES & YES & YES & $1.75$ & $(2,3)$ & NO & 2970\\
$(277,60)$ & 13 & $(9,2)$ & 5 & 1 & YES & YES & YES & $1.50$ & $(2,3)$ & NO & 2971\\
$(277,116)$ & 12 & $(19,8)$ & 6 & 1 & YES & YES & YES & $1.71$ & $(2,3)$ & NO & 2972\\
$(281,109)$ & 12 & $(2,1)$ & 1 & 1 & YES & YES & YES & $1.57$ & $(2,3)$ & -- & 2973\\
$(281,85)$ & 13 & $(3,1)$ & 2 & 1 & YES & YES & YES & $1.62$ & $(2,3)$ & NO & 2974\\
$(281,116)$ & 12 & $(3,1)$ & 2 & 1 & YES & YES & YES & $1.57$ & $(2,3)$ & -- & 2975\\
$(281,109)$ & 12 & $(4,1)$ & 3 & 1 & YES & YES & YES & $1.57$ & $(2,3)$ & -- & 2976\\
$(281,109)$ & 12 & $(5,1)$ & 4 & 1 & YES & YES & YES & $1.57$ & $(2,3)$ & NO & 2977\\
$(281,116)$ & 12 & $(12,5)$ & 5 & 1 & YES & YES & YES & $1.43$ & $(2,3)$ & 2624 & 2978\\
$(281,109)$ & 12 & $(281,109)$ & 12 & 281 & YES & YES & YES & $1.57$ & $(2,3)$ & NO & 2979\\
$(282,109)$ & 12 & $(2,1)$ & 1 & 2 & YES & YES & YES & $1.57$ & $(2,3)$ & -- & 2980\\
$(282,109)$ & 12 & $(282,109)$ & 12 & 282 & YES & YES & YES & $1.57$ & $(2,3)$ & NO & 2981\\
$(283,108)$ & 12 & $(2,1)$ & 1 & 1 & YES & YES & YES & $1.71$ & $(2,3)$ & -- & 2982\\
$(283,117)$ & 12 & $(2,1)$ & 1 & 1 & YES & YES & YES & $1.43$ & $(2,3)$ & -- & 2983\\
$(283,75)$ & 13 & $(3,1)$ & 2 & 1 & YES & YES & YES & $1.71$ & $(2,3)$ & 2341 & 2984\\
$(283,84)$ & 13 & $(3,1)$ & 2 & 1 & YES & YES & YES & $1.71$ & $(2,3)$ & -- & 2985\\
$(283,108)$ & 12 & $(4,1)$ & 3 & 1 & YES & YES & YES & $1.75$ & $(2,3)$ & -- & 2986\\
$(283,108)$ & 12 & $(34,13)$ & 7 & 1 & YES & YES & YES & $1.71$ & $(2,3)$ & NO & 2987\\
$(283,108)$ & 12 & $(55,21)$ & 8 & 1 & YES & YES & YES & $1.62$ & $(2,3)$ & NO & 2988\\
$(283,83)$ & 13 & $(283,83)$ & 13 & 283 & YES & YES & YES & $1.57$ & $(2,3)$ & NO & 2989\\
$(284,83)$ & 13 & $(3,1)$ & 2 & 1 & YES & YES & YES & $1.57$ & $(2,3)$ & -- & 2990\\
$(286,61)$ & 14 & $(5,1)$ & 4 & 1 & YES & YES & YES & $1.43$ & $(2,3)$ & NO & 2991\\
$(287,111)$ & 12 & $(2,1)$ & 1 & 1 & YES & YES & YES & $1.71$ & $(2,3)$ & -- & 2992\\
$(287,109)$ & 12 & $(3,1)$ & 2 & 1 & YES & YES & YES & $1.75$ & $(2,3)$ & -- & 2993\\
$(287,111)$ & 12 & $(4,1)$ & 3 & 1 & YES & YES & YES & $1.86$ & $(2,3)$ & -- & 2994\\
$(287,109)$ & 12 & $(13,5)$ & 5 & 1 & YES & YES & YES & $1.71$ & $(2,3)$ & NO & 2995\\
$(287,109)$ & 12 & $(21,8)$ & 6 & 7 & YES & YES & YES & $1.75$ & $(2,3)$ & NO & 2996\\
$(288,121)$ & 12 & $(7,3)$ & 4 & 1 & YES & YES & YES & $1.71$ & $(2,3)$ & NO & 2997\\
$(289,63)$ & 13 & $(3,1)$ & 2 & 1 & YES & YES & YES & $1.50$ & $(2,3)$ & -- & 2998\\
$(289,63)$ & 13 & $(5,2)$ & 3 & 1 & YES & YES & YES & $1.71$ & $(2,3)$ & -- & 2999\\
$(289,112)$ & 12 & $(13,5)$ & 5 & 1 & YES & YES & YES & $1.86$ & $(2,3)$ & NO & 3000\\
$(289,112)$ & 12 & $(18,7)$ & 6 & 1 & YES & YES & YES & $1.71$ & $(2,3)$ & NO & 3001\\
$(289,112)$ & 12 & $(31,12)$ & 7 & 1 & YES & YES & YES & $1.88$ & $(2,3)$ & NO & 3002\\
$(289,112)$ & 12 & $(49,19)$ & 8 & 1 & YES & YES & YES & $1.75$ & $(2,3)$ & NO & 3003\\
$(289,86)$ & 13 & $(289,86)$ & 13 & 289 & YES & YES & YES & $1.71$ & $(2,3)$ & NO & 3004\\
$(290,111)$ & 12 & $(2,1)$ & 1 & 2 & YES & YES & YES & $1.71$ & $(2,3)$ & NO & 3005\\
$(291,85)$ & 13 & $(2,1)$ & 1 & 1 & YES & YES & YES & $1.43$ & $(4,2)$ & -- & 3006\\
$(291,85)$ & 13 & $(3,1)$ & 2 & 3 & YES & YES & YES & $1.43$ & $(2,3)$ & NO & 3007\\
$(292,121)$ & 12 & $(3,1)$ & 2 & 1 & YES & YES & YES & $1.71$ & $(2,3)$ & -- & 3008\\
$(292,121)$ & 12 & $(29,12)$ & 7 & 1 & YES & YES & YES & $1.43$ & $(2,3)$ & 2948 & 3009\\
$(292,121)$ & 12 & $(41,17)$ & 8 & 1 & YES & YES & YES & $1.43$ & $(2,3)$ & NO & 3010\\
$(292,85)$ & 13 & $(292,85)$ & 13 & 292 & YES & YES & YES & $1.86$ & $(2,3)$ & NO & 3011\\
$(293,52)$ & 14 & $(2,1)$ & 1 & 1 & YES & YES & YES & $1.43$ & $(2,3)$ & NO & 3012\\
$(293,123)$ & 12 & $(2,1)$ & 1 & 1 & YES & YES & YES & $1.43$ & $(4,2)$ & -- & 3013\\
$(293,123)$ & 12 & $(3,1)$ & 2 & 1 & YES & YES & YES & $1.71$ & $(2,3)$ & -- & 3014\\
$(293,123)$ & 12 & $(3,1)$ & 2 & 1 & YES & YES & YES & $1.86$ & $(2,3)$ & NO & 3015\\
$(293,52)$ & 14 & $(5,1)$ & 4 & 1 & YES & YES & YES & $1.29$ & $(2,3)$ & NO & 3016\\
$(293,62)$ & 14 & $(5,1)$ & 4 & 1 & YES & YES & YES & $1.62$ & $(2,3)$ & NO & 3017\\
$(293,52)$ & 14 & $(45,8)$ & 9 & 1 & YES & YES & YES & $1.43$ & $(2,3)$ & NO & 3018\\
$(294,67)$ & 13 & $(2,1)$ & 1 & 2 & YES & YES & YES & $1.38$ & $(2,3)$ & NO & 3019\\
$(294,67)$ & 13 & $(57,13)$ & 9 & 3 & YES & YES & YES & $1.38$ & $(2,3)$ & NO & 3020\\
$(295,112)$ & 12 & $(2,1)$ & 1 & 1 & YES & YES & YES & $1.43$ & $(2,3)$ & NO & 3021\\
$(297,68)$ & 13 & $(2,1)$ & 1 & 1 & YES & YES & YES & $1.38$ & $(2,3)$ & -- & 3022\\
$(298,83)$ & 13 & $(140,39)$ & 11 & 2 & YES & YES & YES & $1.71$ & $(2,3)$ & 3093 & 3023\\
$(299,116)$ & 12 & $(31,12)$ & 7 & 1 & YES & YES & YES & $1.43$ & $(2,3)$ & NO & 3024\\
$(300,71)$ & 14 & $(13,3)$ & 6 & 1 & YES & YES & YES & $1.50$ & $(2,3)$ & NO & 3025\\
$(300,89)$ & 13 & $(17,5)$ & 6 & 1 & YES & YES & YES & $1.57$ & $(2,3)$ & NO & 3026\\
$(300,71)$ & 14 & $(55,13)$ & 10 & 5 & YES & YES & YES & $1.50$ & $(2,3)$ & 2608 & 3027\\
$(300,89)$ & 13 & $(118,35)$ & 11 & 2 & YES & YES & YES & $1.57$ & $(2,3)$ & 3068 & 3028\\
$(301,115)$ & 12 & $(2,1)$ & 1 & 1 & YES & YES & YES & $1.50$ & $(2,3)$ & -- & 3029\\
$(301,115)$ & 12 & $(5,2)$ & 3 & 1 & YES & YES & YES & $1.75$ & $(2,3)$ & NO & 3030\\
$(301,115)$ & 12 & $(13,5)$ & 5 & 1 & YES & YES & YES & $1.75$ & $(2,3)$ & 2641 & 3031\\
$(301,88)$ & 13 & $(24,7)$ & 7 & 1 & YES & YES & YES & $1.57$ & $(2,3)$ & NO & 3032\\
$(301,115)$ & 12 & $(34,13)$ & 7 & 1 & YES & YES & YES & $1.86$ & $(2,3)$ & NO & 3033\\
$(302,117)$ & 12 & $(111,43)$ & 10 & 1 & YES & YES & YES & $1.75$ & $(2,3)$ & NO & 3034\\
$(303,116)$ & 12 & $(4,1)$ & 3 & 1 & YES & YES & YES & $1.62$ & $(2,3)$ & NO & 3035\\
$(304,85)$ & 13 & $(3,1)$ & 2 & 1 & YES & YES & YES & $1.71$ & $(2,3)$ & NO & 3036\\
$(304,85)$ & 13 & $(3,1)$ & 2 & 1 & YES & YES & YES & $1.71$ & $(2,3)$ & -- & 3037\\
$(305,69)$ & 13 & $(2,1)$ & 1 & 1 & YES & YES & YES & $1.38$ & $(2,3)$ & NO & 3038\\
$(305,128)$ & 12 & $(112,47)$ & 10 & 1 & YES & YES & YES & $1.71$ & $(2,3)$ & NO & 3039\\
$(307,57)$ & 14 & $(2,1)$ & 1 & 1 & YES & YES & YES & $1.50$ & $(2,3)$ & NO & 3040\\
$(307,119)$ & 12 & $(2,1)$ & 1 & 1 & YES & YES & YES & $1.57$ & $(2,3)$ & -- & 3041\\
$(307,119)$ & 12 & $(4,1)$ & 3 & 1 & YES & YES & YES & $1.62$ & $(2,3)$ & NO & 3042\\
$(307,129)$ & 12 & $(7,3)$ & 4 & 1 & YES & YES & YES & $1.57$ & $(2,3)$ & NO & 3043\\
$(308,73)$ & 14 & $(3,1)$ & 2 & 1 & YES & YES & YES & $1.50$ & $(2,3)$ & -- & 3044\\
$(308,73)$ & 14 & $(13,3)$ & 6 & 1 & YES & YES & YES & $1.50$ & $(2,3)$ & NO & 3045\\
$(308,73)$ & 14 & $(59,14)$ & 10 & 1 & YES & YES & YES & $1.50$ & $(2,3)$ & 2631 & 3046\\
$(311,71)$ & 13 & $(35,8)$ & 8 & 1 & YES & YES & YES & $1.38$ & $(2,3)$ & NO & 3047\\
$(312,131)$ & 12 & $(19,8)$ & 6 & 1 & YES & YES & YES & $1.71$ & $(2,3)$ & NO & 3048\\
$(313,86)$ & 13 & $(2,1)$ & 1 & 1 & YES & YES & YES & $1.43$ & $(4,2)$ & -- & 3049\\
$(313,121)$ & 12 & $(2,1)$ & 1 & 1 & YES & YES & YES & $1.57$ & $(2,3)$ & NO & 3050\\
$(313,91)$ & 13 & $(3,1)$ & 2 & 1 & YES & YES & YES & $1.71$ & $(2,3)$ & -- & 3051\\
$(313,91)$ & 13 & $(3,1)$ & 2 & 1 & YES & YES & YES & $1.86$ & $(2,3)$ & NO & 3052\\
$(313,71)$ & 14 & $(5,1)$ & 4 & 1 & YES & YES & YES & $1.29$ & $(4,2)$ & NO & 3053\\
$(313,93)$ & 13 & $(5,1)$ & 4 & 1 & YES & YES & YES & $1.71$ & $(2,3)$ & NO & 3054\\
$(313,93)$ & 13 & $(7,2)$ & 4 & 1 & YES & YES & YES & $1.57$ & $(2,3)$ & NO & 3055\\
$(313,86)$ & 13 & $(131,36)$ & 11 & 1 & YES & YES & YES & $1.57$ & $(2,3)$ & 3087 & 3056\\
$(315,88)$ & 13 & $(3,1)$ & 2 & 3 & YES & YES & YES & $1.71$ & $(2,3)$ & -- & 3057\\
$(315,88)$ & 13 & $(25,7)$ & 7 & 5 & YES & YES & YES & $1.86$ & $(2,3)$ & NO & 3058\\
$(316,69)$ & 13 & $(55,12)$ & 9 & 1 & YES & YES & YES & $1.50$ & $(2,3)$ & NO & 3059\\
$(317,131)$ & 12 & $(2,1)$ & 1 & 1 & YES & YES & YES & $1.43$ & $(2,3)$ & -- & 3060\\
$(319,74)$ & 14 & $(4,1)$ & 3 & 1 & YES & YES & YES & $1.62$ & $(2,3)$ & NO & 3061\\
$(322,123)$ & 12 & $(2,1)$ & 1 & 2 & YES & YES & YES & $1.57$ & $(2,3)$ & -- & 3062\\
$(322,123)$ & 12 & $(2,1)$ & 1 & 2 & YES & YES & YES & $1.71$ & $(2,3)$ & NO & 3063\\
$(322,123)$ & 12 & $(8,3)$ & 4 & 2 & YES & YES & YES & $1.71$ & $(2,3)$ & NO & 3064\\
$(322,123)$ & 12 & $(34,13)$ & 7 & 2 & YES & YES & YES & $1.71$ & $(2,3)$ & NO & 3065\\
$(325,76)$ & 13 & $(47,11)$ & 9 & 1 & YES & YES & YES & $1.50$ & $(2,3)$ & NO & 3066\\
$(327,62)$ & 15 & $(5,1)$ & 4 & 1 & YES & YES & YES & $1.50$ & $(2,3)$ & NO & 3067\\
$(327,97)$ & 13 & $(91,27)$ & 10 & 1 & YES & YES & YES & $1.57$ & $(2,3)$ & 3028 & 3068\\
$(327,97)$ & 13 & $(327,97)$ & 13 & 327 & YES & YES & YES & $1.71$ & $(2,3)$ & NO & 3069\\
$(329,61)$ & 15 & $(151,28)$ & 13 & 1 & YES & YES & YES & $1.71$ & $(2,3)$ & NO & 3070\\
$(337,91)$ & 13 & $(2,1)$ & 1 & 1 & YES & YES & YES & $1.71$ & $(2,3)$ & NO & 3071\\
$(337,76)$ & 14 & $(9,2)$ & 5 & 1 & YES & YES & YES & $1.62$ & $(2,3)$ & 2424 & 3072\\
$(338,131)$ & 12 & $(2,1)$ & 1 & 2 & YES & YES & YES & $1.57$ & $(2,3)$ & -- & 3073\\
$(338,129)$ & 12 & $(3,1)$ & 2 & 1 & YES & YES & YES & $1.57$ & $(2,3)$ & -- & 3074\\
$(338,131)$ & 12 & $(3,1)$ & 2 & 1 & YES & YES & YES & $1.43$ & $(2,3)$ & -- & 3075\\
$(338,129)$ & 12 & $(8,3)$ & 4 & 2 & YES & YES & YES & $1.57$ & $(2,3)$ & NO & 3076\\
$(338,131)$ & 12 & $(8,3)$ & 4 & 2 & YES & YES & YES & $1.57$ & $(2,3)$ & 3103 & 3077\\
$(339,100)$ & 13 & $(4,1)$ & 3 & 1 & YES & YES & YES & $1.57$ & $(2,3)$ & NO & 3078\\
$(341,100)$ & 13 & $(4,1)$ & 3 & 1 & YES & YES & YES & $1.71$ & $(2,3)$ & NO & 3079\\
$(343,144)$ & 12 & $(19,8)$ & 6 & 1 & YES & YES & YES & $1.57$ & $(2,3)$ & NO & 3080\\
$(344,95)$ & 13 & $(11,3)$ & 5 & 1 & YES & YES & YES & $1.71$ & $(2,3)$ & NO & 3081\\
$(344,95)$ & 13 & $(134,37)$ & 11 & 2 & YES & YES & YES & $1.71$ & $(2,3)$ & 3102 & 3082\\
$(346,79)$ & 13 & $(4,1)$ & 3 & 2 & YES & YES & YES & $1.50$ & $(2,3)$ & NO & 3083\\
$(347,101)$ & 13 & $(5,1)$ & 4 & 1 & YES & YES & YES & $1.57$ & $(2,3)$ & -- & 3084\\
$(347,101)$ & 13 & $(134,39)$ & 11 & 1 & YES & YES & YES & $1.57$ & $(2,3)$ & NO & 3085\\
$(348,103)$ & 13 & $(2,1)$ & 1 & 2 & YES & YES & YES & $1.57$ & $(2,3)$ & NO & 3086\\
$(353,97)$ & 13 & $(91,25)$ & 10 & 1 & YES & YES & YES & $1.57$ & $(2,3)$ & 3056 & 3087\\
$(356,105)$ & 13 & $(10,3)$ & 5 & 2 & YES & YES & YES & $1.71$ & $(2,3)$ & NO & 3088\\
$(359,100)$ & 13 & $(2,1)$ & 1 & 1 & YES & YES & YES & $1.57$ & $(2,3)$ & -- & 3089\\
$(359,100)$ & 13 & $(3,1)$ & 2 & 1 & YES & YES & YES & $1.43$ & $(2,3)$ & NO & 3090\\
$(359,100)$ & 13 & $(3,1)$ & 2 & 1 & YES & YES & YES & $1.86$ & $(2,3)$ & -- & 3091\\
$(359,100)$ & 13 & $(11,3)$ & 5 & 1 & YES & YES & YES & $1.71$ & $(2,3)$ & NO & 3092\\
$(359,100)$ & 13 & $(79,22)$ & 10 & 1 & YES & YES & YES & $1.71$ & $(2,3)$ & 3023 & 3093\\
$(367,83)$ & 14 & $(2,1)$ & 1 & 1 & YES & YES & YES & $1.57$ & $(2,3)$ & NO & 3094\\
$(367,83)$ & 14 & $(2,1)$ & 1 & 1 & YES & YES & YES & $1.71$ & $(2,3)$ & -- & 3095\\
$(367,112)$ & 13 & $(3,1)$ & 2 & 1 & YES & YES & YES & $1.86$ & $(2,3)$ & -- & 3096\\
$(368,107)$ & 13 & $(2,1)$ & 1 & 2 & YES & YES & YES & $1.71$ & $(2,3)$ & NO & 3097\\
$(368,107)$ & 13 & $(24,7)$ & 7 & 8 & YES & YES & YES & $1.86$ & $(2,3)$ & NO & 3098\\
$(368,107)$ & 13 & $(141,41)$ & 11 & 1 & YES & YES & YES & $1.57$ & $(2,3)$ & NO & 3099\\
$(370,59)$ & 16 & $(6,1)$ & 5 & 2 & YES & YES & YES & $1.43$ & $(2,3)$ & NO & 3100\\
$(371,88)$ & 14 & $(59,14)$ & 10 & 1 & YES & YES & YES & $1.62$ & $(2,3)$ & 2964 & 3101\\
$(373,103)$ & 13 & $(105,29)$ & 10 & 1 & YES & YES & YES & $1.71$ & $(2,3)$ & 3082 & 3102\\
$(377,144)$ & 12 & $(5,2)$ & 3 & 1 & YES & YES & YES & $1.57$ & $(2,3)$ & 3077 & 3103\\
$(379,105)$ & 13 & $(11,3)$ & 5 & 1 & YES & YES & YES & $1.71$ & $(2,3)$ & NO & 3104\\
$(380,83)$ & 14 & $(2,1)$ & 1 & 2 & YES & YES & YES & $1.71$ & $(2,3)$ & NO & 3105\\
$(380,83)$ & 14 & $(4,1)$ & 3 & 4 & YES & YES & YES & $1.57$ & $(2,3)$ & NO & 3106\\
$(382,89)$ & 14 & $(382,89)$ & 14 & 382 & YES & YES & YES & $1.71$ & $(2,3)$ & NO & 3107\\
$(389,91)$ & 14 & $(4,1)$ & 3 & 1 & YES & YES & YES & $1.62$ & $(2,3)$ & NO & 3108\\
$(391,91)$ & 14 & $(2,1)$ & 1 & 1 & YES & YES & YES & $1.71$ & $(2,3)$ & NO & 3109\\
$(391,91)$ & 14 & $(30,7)$ & 8 & 1 & YES & YES & YES & $1.86$ & $(2,3)$ & NO & 3110\\
$(393,116)$ & 13 & $(2,1)$ & 1 & 1 & YES & YES & YES & $1.62$ & $(2,3)$ & NO & 3111\\
$(393,116)$ & 13 & $(10,3)$ & 5 & 1 & YES & YES & YES & $1.62$ & $(2,3)$ & NO & 3112\\
$(404,91)$ & 14 & $(3,1)$ & 2 & 1 & YES & YES & YES & $1.71$ & $(2,3)$ & NO & 3113\\
$(404,91)$ & 14 & $(3,1)$ & 2 & 1 & YES & YES & YES & $1.71$ & $(2,3)$ & -- & 3114\\
$(407,112)$ & 13 & $(2,1)$ & 1 & 1 & YES & YES & YES & $1.57$ & $(2,3)$ & NO & 3115\\
$(407,119)$ & 13 & $(2,1)$ & 1 & 1 & YES & YES & YES & $1.57$ & $(2,3)$ & -- & 3116\\
$(407,119)$ & 13 & $(3,1)$ & 2 & 1 & YES & YES & YES & $1.86$ & $(2,3)$ & NO & 3117\\
$(407,112)$ & 13 & $(7,2)$ & 4 & 1 & YES & YES & YES & $1.57$ & $(2,3)$ & NO & 3118\\
$(426,179)$ & 13 & $(2,1)$ & 1 & 2 & NO & YES & YES & $1.71$ & $(2,3)$ & -- & 3119\\
$(434,101)$ & 14 & $(2,1)$ & 1 & 2 & YES & YES & YES & $1.57$ & $(2,3)$ & NO & 3120\\
$(474,199)$ & 13 & $(2,1)$ & 1 & 2 & NO & YES & YES & $1.71$ & $(2,3)$ & -- & 3121\\
$(477,88)$ & 15 & $(3,1)$ & 2 & 3 & YES & YES & YES & $1.71$ & $(2,3)$ & -- & 3122\\
$(477,88)$ & 15 & $(3,1)$ & 2 & 3 & YES & YES & YES & $1.71$ & $(2,3)$ & NO & 3123\\
$(522,97)$ & 15 & $(16,3)$ & 7 & 2 & YES & YES & YES & $1.71$ & $(2,3)$ & NO & 3124\\
$(a;0,0,0;3)$ & 4 & $(37,14)$ & 8 & 1 & YES & YES & YES & $1.62$ & $(2,3)$ & -- & 3125\\
$(a;0,0,0;3)$ & 4 & $(50,19)$ & 8 & 1 & YES & YES & YES & $1.62$ & $(2,3)$ & -- & 3126\\
$(a;0,0,0;3)$ & 4 & $(70,29)$ & 9 & 1 & YES & YES & YES & $1.62$ & $(2,3)$ & -- & 3127\\
$(a;1,0,0;13)$ & 5 & $(24,7)$ & 7 & 1 & YES & YES & YES & $1.25$ & $(2,3)$ & -- & 3128\\
$(a;1,0,0;13)$ & 5 & $(27,10)$ & 7 & 1 & YES & YES & YES & $1.29$ & $(4,2)$ & -- & 3129\\
$(a;1,0,0;13)$ & 5 & $(30,11)$ & 7 & 1 & YES & YES & YES & $1.43$ & $(2,3)$ & -- & 3130\\
$(a;1,0,0;13)$ & 5 & $(34,13)$ & 7 & 1 & YES & YES & YES & $1.62$ & $(2,3)$ & -- & 3131\\
$(a;1,0,0;13)$ & 5 & $(41,17)$ & 8 & 1 & YES & YES & YES & $1.29$ & $(4,2)$ & -- & 3132\\
$(a;1,0,0;13)$ & 5 & $(47,21)$ & 10 & 1 & YES & YES & YES & $1.57$ & $(4,2)$ & -- & 3133\\
$(a;1,0,0;13)$ & 5 & $(50,19)$ & 8 & 1 & YES & YES & YES & $1.62$ & $(2,3)$ & -- & 3134\\
$(a;1,0,0;13)$ & 5 & $(50,21)$ & 8 & 1 & YES & YES & YES & $1.57$ & $(2,3)$ & -- & 3135\\
$(a;1,1,0;19)$ & 6 & $(19,7)$ & 6 & 19 & YES & YES & YES & $1.57$ & $(2,3)$ & -- & 3136\\
$(a;1,1,0;19)$ & 6 & $(25,11)$ & 7 & 1 & YES & YES & YES & $1.62$ & $(2,3)$ & -- & 3137\\
$(a;1,1,0;19)$ & 6 & $(30,11)$ & 7 & 1 & YES & YES & YES & $1.50$ & $(2,3)$ & -- & 3138\\
$(a;2,0,0;17)$ & 6 & $(23,10)$ & 7 & 1 & YES & YES & YES & $1.29$ & $(4,2)$ & -- & 3139\\
$(a;2,0,0;17)$ & 6 & $(41,12)$ & 8 & 1 & YES & YES & YES & $1.62$ & $(2,3)$ & -- & 3140\\
$(a;2,0,1;25)$ & 7 & $(18,7)$ & 6 & 1 & YES & YES & YES & $1.43$ & $(2,3)$ & -- & 3141\\
$(a;2,1,0;5)$ & 7 & $(17,7)$ & 6 & 1 & YES & YES & YES & $1.43$ & $(2,3)$ & -- & 3142\\
$(a;2,1,0;5)$ & 7 & $(24,7)$ & 7 & 1 & YES & YES & YES & $1.43$ & $(2,3)$ & -- & 3143\\
$(a;3,0,0;7)$ & 7 & $(24,7)$ & 7 & 1 & YES & YES & YES & $1.29$ & $(4,2)$ & -- & 3144\\
$(a;3,0,0;7)$ & 7 & $(25,7)$ & 7 & 1 & YES & YES & YES & $1.43$ & $(4,2)$ & -- & 3145\\
$(a;3,0,0;7)$ & 7 & $(27,8)$ & 7 & 1 & YES & YES & YES & $1.43$ & $(2,3)$ & -- & 3146\\
$(a;3,1,0;31)$ & 8 & $(11,4)$ & 5 & 1 & YES & YES & YES & $1.62$ & $(2,3)$ & -- & 3147\\
$(a;3,1,0;31)$ & 8 & $(17,5)$ & 6 & 1 & YES & YES & YES & $1.57$ & $(2,3)$ & -- & 3148\\
$(a;3,1,1;46)$ & 9 & $(8,3)$ & 4 & 2 & YES & YES & YES & $1.50$ & $(2,3)$ & -- & 3149\\
$(a;3,1,2;61)$ & 10 & $(5,2)$ & 3 & 1 & YES & YES & YES & $1.50$ & $(2,3)$ & -- & 3150\\
$(a;4,2,1;73)$ & 11 & $(5,1)$ & 4 & 1 & YES & YES & YES & $1.50$ & $(2,3)$ & -- & 3151\\
$(b;0,0,0;14)$ & 5 & $(14,3)$ & 6 & 14 & YES & YES & YES & $1.29$ & $(4,2)$ & -- & 3152\\
$(b;0,0,0;14)$ & 5 & $(29,12)$ & 7 & 1 & YES & YES & YES & $1.29$ & $(4,2)$ & -- & 3153\\
$(b;0,0,0;14)$ & 5 & $(29,13)$ & 8 & 1 & YES & YES & YES & $1.29$ & $(4,2)$ & -- & 3154\\
$(b;0,0,0;14)$ & 5 & $(37,11)$ & 8 & 1 & YES & YES & YES & $1.62$ & $(2,3)$ & -- & 3155\\
$(b;0,0,0;14)$ & 5 & $(41,17)$ & 8 & 1 & YES & YES & YES & $1.57$ & $(4,2)$ & -- & 3156\\
$(b;0,0,0;14)$ & 5 & $(44,17)$ & 8 & 2 & YES & YES & YES & $1.43$ & $(4,2)$ & -- & 3157\\
$(b;0,0,1;4)$ & 6 & $(23,9)$ & 7 & 1 & YES & YES & YES & $1.43$ & $(4,2)$ & -- & 3158\\
$(b;0,0,1;4)$ & 6 & $(26,11)$ & 7 & 2 & YES & YES & YES & $1.43$ & $(2,3)$ & -- & 3159\\
$(b;0,0,2;26)$ & 7 & $(12,5)$ & 5 & 2 & YES & YES & YES & $1.50$ & $(2,3)$ & -- & 3160\\
$(b;0,0,2;26)$ & 7 & $(18,7)$ & 6 & 2 & YES & YES & YES & $1.43$ & $(2,3)$ & -- & 3161\\
$(b;0,0,2;26)$ & 7 & $(24,7)$ & 7 & 2 & YES & YES & YES & $1.43$ & $(4,2)$ & -- & 3162\\
$(b;0,1,0;19)$ & 6 & $(23,10)$ & 7 & 1 & YES & YES & YES & $1.71$ & $(2,3)$ & -- & 3163\\
$(b;0,2,0;8)$ & 7 & $(13,4)$ & 6 & 1 & YES & YES & YES & $1.57$ & $(2,3)$ & -- & 3164\\
$(b;0,2,0;8)$ & 7 & $(13,5)$ & 5 & 1 & YES & YES & YES & $1.29$ & $(4,2)$ & -- & 3165\\
$(b;0,2,0;8)$ & 7 & $(18,5)$ & 6 & 2 & YES & YES & YES & $1.50$ & $(2,3)$ & -- & 3166\\
$(b;0,2,0;8)$ & 7 & $(18,7)$ & 6 & 2 & YES & YES & YES & $1.43$ & $(2,3)$ & -- & 3167\\
$(b;0,2,2;44)$ & 9 & $(5,2)$ & 3 & 1 & YES & YES & YES & $1.50$ & $(2,3)$ & -- & 3168\\
$(b;1,0,1;29)$ & 7 & $(17,7)$ & 6 & 1 & YES & YES & YES & $1.43$ & $(2,3)$ & -- & 3169\\
$(b;1,0,2;19)$ & 8 & $(7,3)$ & 4 & 1 & YES & YES & YES & $1.14$ & $(4,2)$ & -- & 3170\\
$(b;1,0,2;19)$ & 8 & $(11,4)$ & 5 & 1 & YES & YES & YES & $1.43$ & $(2,3)$ & -- & 3171\\
$(b;1,1,0;27)$ & 7 & $(11,3)$ & 5 & 1 & YES & YES & YES & $1.62$ & $(2,3)$ & -- & 3172\\
$(b;1,1,1;39)$ & 8 & $(9,4)$ & 5 & 3 & YES & YES & YES & $1.57$ & $(2,3)$ & -- & 3173\\
$(b;1,1,1;39)$ & 8 & $(12,5)$ & 5 & 3 & YES & YES & YES & $1.71$ & $(2,3)$ & -- & 3174\\
$(b;1,1,1;39)$ & 8 & $(13,5)$ & 5 & 13 & YES & YES & YES & $1.62$ & $(2,3)$ & -- & 3175\\
$(b;1,2,0;17)$ & 8 & $(13,4)$ & 6 & 1 & YES & YES & YES & $1.57$ & $(2,3)$ & -- & 3176\\
$(b;1,2,1;7)$ & 9 & $(5,2)$ & 3 & 1 & YES & YES & YES & $1.14$ & $(4,2)$ & -- & 3177\\
$(b;1,2,1;7)$ & 9 & $(10,3)$ & 5 & 1 & YES & YES & YES & $1.57$ & $(2,3)$ & -- & 3178\\
$(b;2,0,0;26)$ & 7 & $(12,5)$ & 5 & 2 & YES & YES & YES & $1.50$ & $(2,3)$ & -- & 3179\\
$(b;3,0,3;11)$ & 11 & $(3,1)$ & 2 & 1 & YES & YES & YES & $1.50$ & $(2,3)$ & -- & 3180\\
$(b;3,0,3;11)$ & 11 & $(4,1)$ & 3 & 1 & YES & YES & YES & $1.50$ & $(2,3)$ & -- & 3181\\
$(b;3,1,1;63)$ & 10 & $(3,1)$ & 2 & 3 & YES & YES & YES & $1.38$ & $(2,3)$ & -- & 3182\\
$(c;0,0,0;4)$ & 4 & $(35,8)$ & 8 & 1 & YES & YES & YES & $1.38$ & $(2,3)$ & -- & 3183\\
$(c;0,0,0;4)$ & 4 & $(36,13)$ & 8 & 4 & YES & YES & YES & $1.14$ & $(4,2)$ & -- & 3184\\
$(c;0,0,0;4)$ & 4 & $(39,11)$ & 9 & 1 & YES & YES & YES & $1.25$ & $(2,3)$ & -- & 3185\\
$(c;0,0,0;4)$ & 4 & $(39,17)$ & 8 & 1 & YES & YES & YES & $1.14$ & $(4,2)$ & -- & 3186\\
$(c;0,0,0;4)$ & 4 & $(44,17)$ & 8 & 4 & YES & YES & YES & $1.29$ & $(4,2)$ & -- & 3187\\
$(c;0,0,0;4)$ & 4 & $(45,19)$ & 8 & 1 & YES & YES & YES & $1.62$ & $(2,3)$ & -- & 3188\\
$(c;0,0,0;4)$ & 4 & $(49,19)$ & 8 & 1 & YES & YES & YES & $1.50$ & $(2,3)$ & -- & 3189\\
$(c;0,0,0;4)$ & 4 & $(50,11)$ & 10 & 2 & YES & YES & YES & $1.38$ & $(2,3)$ & -- & 3190\\
$(c;0,0,0;4)$ & 4 & $(50,19)$ & 8 & 2 & YES & YES & YES & $1.50$ & $(2,3)$ & -- & 3191\\
$(c;0,0,0;4)$ & 4 & $(55,16)$ & 9 & 1 & YES & YES & YES & $1.57$ & $(2,3)$ & -- & 3192\\
$(c;0,0,0;4)$ & 4 & $(57,25)$ & 9 & 1 & YES & YES & YES & $1.29$ & $(2,3)$ & -- & 3193\\
$(c;0,0,0;4)$ & 4 & $(68,25)$ & 9 & 4 & YES & YES & YES & $1.43$ & $(4,2)$ & -- & 3194\\
$(c;0,0,0;4)$ & 4 & $(73,33)$ & 10 & 1 & YES & YES & YES & $1.71$ & $(2,3)$ & -- & 3195\\
$(c;0,0,0;4)$ & 4 & $(75,31)$ & 9 & 1 & YES & YES & YES & $1.50$ & $(2,3)$ & -- & 3196\\
$(c;0,0,0;4)$ & 4 & $(106,41)$ & 10 & 2 & YES & YES & YES & $1.29$ & $(4,2)$ & -- & 3197\\
$(c;0,0,0;4)$ & 4 & $(113,20)$ & 13 & 1 & YES & YES & YES & $1.71$ & $(2,3)$ & -- & 3198\\
$(c;0,0,0;4)$ & 4 & $(115,44)$ & 10 & 1 & YES & YES & YES & $1.86$ & $(2,3)$ & -- & 3199\\
$(c;0,0,0;4)$ & 4 & $(119,50)$ & 10 & 1 & YES & YES & YES & $1.71$ & $(2,3)$ & -- & 3200\\
$(c;0,0,0;4)$ & 4 & $(125,33)$ & 11 & 1 & YES & YES & YES & $1.71$ & $(2,3)$ & -- & 3201\\
$(c;0,0,0;4)$ & 4 & $(144,55)$ & 10 & 4 & YES & YES & YES & $1.71$ & $(2,3)$ & -- & 3202\\
$(c;0,0,0;4)$ & 4 & $(146,41)$ & 11 & 2 & YES & YES & YES & $1.86$ & $(2,3)$ & -- & 3203\\
$(c;0,1,0;11)$ & 5 & $(28,11)$ & 8 & 1 & YES & YES & YES & $1.38$ & $(2,3)$ & -- & 3204\\
$(c;0,1,0;11)$ & 5 & $(37,10)$ & 8 & 1 & YES & YES & YES & $1.71$ & $(2,3)$ & -- & 3205\\
$(c;0,1,0;11)$ & 5 & $(39,17)$ & 8 & 1 & YES & YES & YES & $1.29$ & $(4,2)$ & -- & 3206\\
$(c;0,1,0;11)$ & 5 & $(41,18)$ & 8 & 1 & YES & YES & YES & $1.29$ & $(2,3)$ & -- & 3207\\
$(c;0,1,0;11)$ & 5 & $(64,27)$ & 9 & 1 & YES & YES & YES & $1.71$ & $(2,3)$ & -- & 3208\\
$(c;0,1,0;11)$ & 5 & $(68,25)$ & 9 & 1 & YES & YES & YES & $1.43$ & $(4,2)$ & -- & 3209\\
$(c;0,1,0;11)$ & 5 & $(89,24)$ & 10 & 1 & YES & YES & YES & $1.71$ & $(2,3)$ & -- & 3210\\
$(c;0,1,1;5)$ & 6 & $(61,17)$ & 9 & 1 & YES & YES & YES & $1.43$ & $(4,2)$ & -- & 3211\\
$(c;0,2,0;7)$ & 6 & $(23,9)$ & 7 & 1 & YES & YES & YES & $1.50$ & $(2,3)$ & -- & 3212\\
$(c;0,2,0;7)$ & 6 & $(40,9)$ & 9 & 1 & YES & YES & YES & $1.50$ & $(2,3)$ & -- & 3213\\
$(c;0,2,0;7)$ & 6 & $(47,11)$ & 9 & 1 & YES & YES & YES & $1.62$ & $(2,3)$ & -- & 3214\\
$(c;0,2,0;7)$ & 6 & $(64,19)$ & 9 & 1 & YES & YES & YES & $1.71$ & $(2,3)$ & -- & 3215\\
$(c;0,2,1;19)$ & 7 & $(37,10)$ & 8 & 1 & YES & YES & YES & $1.43$ & $(2,3)$ & -- & 3216\\
$(c;0,3,0;17)$ & 7 & $(37,8)$ & 8 & 1 & YES & YES & YES & $1.38$ & $(2,3)$ & -- & 3217\\
$(c;0,3,1;23)$ & 8 & $(24,5)$ & 8 & 1 & YES & YES & YES & $1.57$ & $(2,3)$ & -- & 3218\\
$(c;0,3,1;23)$ & 8 & $(25,7)$ & 7 & 1 & YES & YES & YES & $1.43$ & $(4,2)$ & -- & 3219\\
$(d;0,0,0;5)$ & 5 & $(17,5)$ & 6 & 1 & YES & YES & NO(3) & $1.25$ & $(2,3)$ & -- & 3220\\
$(d;0,0,0;5)$ & 5 & $(21,8)$ & 6 & 1 & YES & YES & YES & $1.38$ & $(2,3)$ & -- & 3221\\
$(d;0,0,0;5)$ & 5 & $(25,7)$ & 7 & 5 & YES & YES & NO(3) & $1.29$ & $(2,3)$ & -- & 3222\\
$(d;0,0,0;5)$ & 5 & $(39,14)$ & 8 & 1 & YES & YES & YES & $1.50$ & $(2,3)$ & -- & 3223\\
$(d;0,0,0;5)$ & 5 & $(69,29)$ & 9 & 1 & YES & YES & YES & $1.71$ & $(2,3)$ & -- & 3224\\
$(d;0,0,1;14)$ & 6 & $(14,5)$ & 6 & 14 & YES & YES & YES & $1.38$ & $(2,3)$ & -- & 3225\\
$(d;0,0,1;14)$ & 6 & $(33,10)$ & 8 & 1 & YES & YES & YES & $1.43$ & $(2,3)$ & -- & 3226\\
$(d;0,0,2;9)$ & 7 & $(18,7)$ & 6 & 9 & YES & YES & YES & $1.43$ & $(2,3)$ & -- & 3227\\
$(d;0,0,2;9)$ & 7 & $(23,9)$ & 7 & 1 & YES & YES & YES & $1.43$ & $(2,3)$ & -- & 3228\\
$(d;0,1,0;6)$ & 6 & $(18,7)$ & 6 & 6 & YES & YES & YES & $1.50$ & $(2,3)$ & -- & 3229\\
$(d;0,1,0;6)$ & 6 & $(31,7)$ & 8 & 1 & YES & YES & YES & $1.50$ & $(2,3)$ & -- & 3230\\
$(d;0,1,1;17)$ & 7 & $(23,9)$ & 7 & 1 & YES & YES & YES & $1.43$ & $(2,3)$ & -- & 3231\\
$(d;0,1,1;17)$ & 7 & $(25,7)$ & 7 & 1 & YES & YES & YES & $1.38$ & $(2,3)$ & -- & 3232\\
$(d;0,1,1;17)$ & 7 & $(41,9)$ & 9 & 1 & YES & YES & YES & $1.43$ & $(2,3)$ & -- & 3233\\
$(d;0,2,1;20)$ & 8 & $(24,5)$ & 8 & 4 & YES & YES & YES & $1.57$ & $(2,3)$ & -- & 3234\\
$(e;0,0,0;4)$ & 5 & $(29,12)$ & 7 & 1 & YES & YES & YES & $1.29$ & $(2,3)$ & -- & 3235\\
$(e;0,0,0;4)$ & 5 & $(34,15)$ & 8 & 2 & YES & YES & YES & $1.43$ & $(4,2)$ & -- & 3236\\
$(e;0,0,0;4)$ & 5 & $(41,15)$ & 8 & 1 & YES & YES & YES & $1.43$ & $(4,2)$ & -- & 3237\\
$(e;1,1,0;23)$ & 7 & $(17,7)$ & 6 & 1 & YES & YES & YES & $1.29$ & $(4,2)$ & -- & 3238\\
$(e;2,0,0;24)$ & 7 & $(8,3)$ & 4 & 8 & YES & YES & YES & $1.14$ & $(4,2)$ & -- & 3239\\
$(e;2,0,0;24)$ & 7 & $(18,7)$ & 6 & 6 & YES & YES & YES & $1.43$ & $(4,2)$ & -- & 3240\\
$(e;3,0,0;10)$ & 8 & $(19,4)$ & 7 & 1 & YES & YES & YES & $1.57$ & $(2,3)$ & -- & 3241\\
$(f;0,0,0;6)$ & 4 & $(49,18)$ & 8 & 1 & YES & YES & YES & $1.50$ & $(2,3)$ & -- & 3242\\
$(f;0,0,0;6)$ & 4 & $(65,19)$ & 9 & 1 & YES & YES & YES & $1.50$ & $(2,3)$ & -- & 3243\\
$(f;0,0,0;6)$ & 4 & $(66,29)$ & 9 & 6 & YES & YES & YES & $1.62$ & $(2,3)$ & -- & 3244\\
$(f;0,0,0;6)$ & 4 & $(68,25)$ & 9 & 2 & YES & YES & YES & $1.57$ & $(2,3)$ & -- & 3245\\
$(f;0,0,0;6)$ & 4 & $(69,19)$ & 9 & 3 & YES & YES & YES & $1.43$ & $(2,3)$ & -- & 3246\\
$(f;0,0,0;6)$ & 4 & $(75,31)$ & 9 & 3 & YES & YES & YES & $1.29$ & $(2,3)$ & -- & 3247\\
$(f;0,0,0;6)$ & 4 & $(81,34)$ & 9 & 3 & YES & YES & YES & $1.50$ & $(2,3)$ & -- & 3248\\
$(f;0,0,0;6)$ & 4 & $(87,31)$ & 12 & 3 & YES & YES & YES & $1.71$ & $(2,3)$ & -- & 3249\\
$(f;0,0,0;6)$ & 4 & $(88,31)$ & 12 & 2 & YES & YES & YES & $1.57$ & $(4,2)$ & -- & 3250\\
$(f;0,0,0;6)$ & 4 & $(88,37)$ & 10 & 2 & YES & YES & YES & $1.57$ & $(2,3)$ & -- & 3251\\
$(f;0,0,0;6)$ & 4 & $(89,34)$ & 9 & 1 & YES & YES & YES & $1.38$ & $(2,3)$ & -- & 3252\\
$(f;0,0,0;6)$ & 4 & $(91,27)$ & 10 & 1 & YES & YES & YES & $1.43$ & $(2,3)$ & -- & 3253\\
$(f;0,0,0;6)$ & 4 & $(97,35)$ & 10 & 1 & YES & YES & YES & $1.50$ & $(2,3)$ & -- & 3254\\
$(f;0,0,0;6)$ & 4 & $(107,20)$ & 13 & 1 & YES & YES & YES & $1.50$ & $(2,3)$ & -- & 3255\\
$(f;0,0,0;6)$ & 4 & $(138,41)$ & 11 & 6 & YES & YES & YES & $1.57$ & $(2,3)$ & -- & 3256\\
$(f;0,1,0;7)$ & 5 & $(41,17)$ & 8 & 1 & YES & YES & YES & $1.43$ & $(4,2)$ & -- & 3257\\
$(f;0,1,0;7)$ & 5 & $(51,14)$ & 9 & 1 & YES & YES & YES & $1.43$ & $(2,3)$ & -- & 3258\\
$(g;0,0,0;19)$ & 6 & $(29,12)$ & 7 & 1 & YES & YES & YES & $1.71$ & $(2,3)$ & -- & 3259\\
$(g;0,1,0;24)$ & 7 & $(17,7)$ & 6 & 1 & YES & YES & YES & $1.29$ & $(4,2)$ & -- & 3260\\
$(g;0,1,2;14)$ & 9 & $(7,2)$ & 4 & 7 & YES & YES & YES & $1.29$ & $(2,3)$ & -- & 3261\\
$(g;0,2,1;40)$ & 9 & $(7,2)$ & 4 & 1 & YES & YES & YES & $1.29$ & $(2,3)$ & -- & 3262\\
$(g;1,0,0;7)$ & 7 & $(8,3)$ & 4 & 1 & YES & YES & YES & $1.38$ & $(2,3)$ & -- & 3263\\
$(g;1,0,2;24)$ & 9 & $(5,1)$ & 4 & 1 & YES & YES & YES & $1.57$ & $(2,3)$ & -- & 3264\\
$(g;1,1,0;9)$ & 8 & $(12,5)$ & 5 & 3 & YES & YES & YES & $1.57$ & $(2,3)$ & -- & 3265\\
$(g;1,1,1;49)$ & 9 & $(7,3)$ & 4 & 7 & YES & YES & YES & $1.43$ & $(2,3)$ & -- & 3266\\
$(g;1,1,2;31)$ & 10 & $(2,1)$ & 1 & 1 & YES & YES & YES & $1.29$ & $(2,3)$ & -- & 3267\\
$(h;0,0,0;6)$ & 5 & $(29,12)$ & 7 & 1 & YES & YES & YES & $1.29$ & $(2,3)$ & -- & 3268\\
$(h;0,0,0;6)$ & 5 & $(44,17)$ & 8 & 2 & YES & YES & YES & $1.86$ & $(2,3)$ & -- & 3269\\
$(h;0,1,0;8)$ & 6 & $(29,12)$ & 7 & 1 & YES & YES & YES & $1.71$ & $(2,3)$ & -- & 3270\\
$(h;0,2,0;10)$ & 7 & $(18,7)$ & 6 & 2 & YES & YES & YES & $1.43$ & $(4,2)$ & -- & 3271\\
$(h;0,3,0;12)$ & 8 & $(10,3)$ & 5 & 2 & YES & YES & YES & $1.43$ & $(2,3)$ & -- & 3272\\
$(i;0,0,0;9)$ & 5 & $(37,11)$ & 8 & 1 & YES & YES & YES & $1.29$ & $(2,3)$ & -- & 3273\\
$(i;0,0,0;9)$ & 5 & $(49,13)$ & 9 & 1 & YES & YES & YES & $1.43$ & $(2,3)$ & -- & 3274\\
$(i;0,0,0;9)$ & 5 & $(64,15)$ & 10 & 1 & YES & YES & YES & $1.57$ & $(2,3)$ & -- & 3275\\
$(i;0,1,0;12)$ & 6 & $(35,8)$ & 8 & 1 & YES & YES & YES & $1.38$ & $(2,3)$ & -- & 3276\\
$(i;0,1,0;12)$ & 6 & $(37,10)$ & 8 & 1 & YES & YES & YES & $1.43$ & $(2,3)$ & -- & 3277\\
$(i;0,2,0;15)$ & 7 & $(31,7)$ & 8 & 1 & YES & YES & YES & $1.43$ & $(4,2)$ & -- & 3278\\
$(j;0,0,0;8)$ & 5 & $(33,14)$ & 8 & 1 & YES & YES & YES & $1.71$ & $(2,3)$ & -- & 3279\\
$(j;0,0,0;8)$ & 5 & $(42,19)$ & 9 & 2 & YES & YES & YES & $1.50$ & $(2,3)$ & -- & 3280\\
$(j;0,0,0;8)$ & 5 & $(43,19)$ & 9 & 1 & YES & YES & YES & $1.57$ & $(2,3)$ & -- & 3281\\
$(j;0,0,0;8)$ & 5 & $(45,13)$ & 10 & 1 & YES & YES & YES & $1.43$ & $(2,3)$ & -- & 3282\\
$(j;0,0,0;8)$ & 5 & $(45,17)$ & 9 & 1 & YES & YES & YES & $1.57$ & $(2,3)$ & -- & 3283\\
$(j;0,0,0;8)$ & 5 & $(50,21)$ & 8 & 2 & YES & YES & YES & $1.38$ & $(2,3)$ & -- & 3284\\
$(j;0,0,0;8)$ & 5 & $(56,25)$ & 11 & 8 & YES & YES & YES & $1.71$ & $(2,3)$ & -- & 3285\\
$(j;0,0,0;8)$ & 5 & $(62,27)$ & 9 & 2 & YES & YES & YES & $1.50$ & $(2,3)$ & -- & 3286\\
$(j;0,0,0;8)$ & 5 & $(64,27)$ & 9 & 8 & YES & YES & YES & $1.50$ & $(2,3)$ & -- & 3287\\
$(j;0,0,0;8)$ & 5 & $(76,29)$ & 9 & 4 & YES & YES & YES & $1.71$ & $(2,3)$ & -- & 3288\\
$(j;0,1,0;10)$ & 6 & $(19,8)$ & 6 & 1 & YES & YES & YES & $1.38$ & $(2,3)$ & -- & 3289\\
$(j;0,1,0;10)$ & 6 & $(37,11)$ & 8 & 1 & YES & YES & YES & $1.57$ & $(2,3)$ & -- & 3290\\
$(j;0,1,0;10)$ & 6 & $(41,17)$ & 8 & 1 & YES & YES & YES & $1.50$ & $(2,3)$ & -- & 3291\\
$(j;0,1,0;10)$ & 6 & $(41,18)$ & 8 & 1 & YES & YES & YES & $1.50$ & $(2,3)$ & -- & 3292\\
$(j;0,1,0;10)$ & 6 & $(44,17)$ & 8 & 2 & YES & YES & YES & $1.50$ & $(2,3)$ & -- & 3293\\
$(j;0,1,0;10)$ & 6 & $(47,14)$ & 9 & 1 & YES & YES & YES & $1.43$ & $(2,3)$ & -- & 3294\\
$(j;0,2,0;12)$ & 7 & $(31,9)$ & 8 & 1 & YES & YES & YES & $1.43$ & $(2,3)$ & -- & 3295
\end{longtable}
\subsection{2 chains, $K^2 = 4$}
\begin{longtable}{|c|c|c|c|c|c|c|c|c|c|c|c|}
\hline
\multicolumn{12}{|c|}{2 chains, $K^2 = 4$}\\
\hline
$(n,a)$ & Len & $(n,a)$ & Len & GCD & Nef & $\mathbb Q$-ef & Obs 0 & $\overline c_1^2 / \overline c_2$ & $(P,K)$ & WH & Index\\
\hline
\endfirsthead

\hline
$(n,a)$ & Len & $(n,a)$ & Len & GCD & Nef & $\mathbb Q$-ef & Obs 0 & $\overline c_1^2 / \overline c_2$ & $(P,K)$ & WH & Index\\
\hline
\endhead
\hline
\endfoot

$(25,9)$ & 7 & $(24,7)$ & 7 & 1 & YES & YES & NO(3) & $1.71$ & $(2,4)$ & -- & 3296\\
$(40,17)$ & 9 & $(37,11)$ & 8 & 1 & YES & YES & YES & $2.00$ & $(2,4)$ & -- & 3297\\
$(47,13)$ & 8 & $(40,17)$ & 9 & 1 & YES & YES & YES & $2.00$ & $(2,4)$ & NO & 3298\\
$(49,19)$ & 8 & $(34,13)$ & 7 & 1 & YES & YES & YES & $2.00$ & $(2,4)$ & -- & 3299\\
$(55,21)$ & 8 & $(19,8)$ & 6 & 1 & YES & YES & YES & $1.83$ & $(2,4)$ & -- & 3300\\
$(59,25)$ & 9 & $(12,5)$ & 5 & 1 & YES & YES & YES & $1.86$ & $(2,4)$ & -- & 3301\\
$(59,23)$ & 9 & $(35,8)$ & 8 & 1 & YES & YES & NO(3) & $1.67$ & $(4,3)$ & -- & 3302\\
$(61,19)$ & 10 & $(37,8)$ & 8 & 1 & YES & YES & YES & $1.83$ & $(2,4)$ & -- & 3303\\
$(63,23)$ & 10 & $(27,8)$ & 7 & 9 & YES & YES & YES & $2.00$ & $(2,4)$ & -- & 3304\\
$(63,23)$ & 10 & $(29,8)$ & 7 & 1 & YES & YES & YES & $2.00$ & $(2,4)$ & -- & 3305\\
$(65,27)$ & 10 & $(34,13)$ & 7 & 1 & YES & YES & YES & $2.17$ & $(4,3)$ & -- & 3306\\
$(67,28)$ & 10 & $(29,8)$ & 7 & 1 & YES & YES & YES & $2.00$ & $(2,4)$ & NO & 3307\\
$(67,14)$ & 10 & $(37,11)$ & 8 & 1 & YES & YES & YES & $2.00$ & $(2,4)$ & -- & 3308\\
$(71,19)$ & 10 & $(15,4)$ & 6 & 1 & YES & YES & YES & $1.71$ & $(2,4)$ & -- & 3309\\
$(71,27)$ & 9 & $(31,12)$ & 7 & 1 & YES & YES & NO(3) & $2.00$ & $(2,4)$ & -- & 3310\\
$(73,31)$ & 10 & $(23,7)$ & 7 & 1 & YES & YES & YES & $2.00$ & $(2,4)$ & -- & 3311\\
$(73,28)$ & 10 & $(29,8)$ & 7 & 1 & YES & YES & YES & $2.00$ & $(2,4)$ & NO & 3312\\
$(73,32)$ & 10 & $(49,9)$ & 10 & 1 & YES & YES & YES & $2.17$ & $(2,4)$ & NO & 3313\\
$(74,29)$ & 10 & $(14,5)$ & 6 & 2 & YES & YES & YES & $1.83$ & $(4,3)$ & -- & 3314\\
$(74,31)$ & 9 & $(29,8)$ & 7 & 1 & YES & YES & YES & $2.00$ & $(2,4)$ & NO & 3315\\
$(79,18)$ & 10 & $(31,12)$ & 7 & 1 & YES & YES & NO(3) & $1.86$ & $(2,4)$ & -- & 3316\\
$(79,21)$ & 11 & $(32,7)$ & 8 & 1 & YES & YES & YES & $1.83$ & $(4,3)$ & NO & 3317\\
$(84,19)$ & 10 & $(19,6)$ & 8 & 1 & YES & YES & YES & $1.86$ & $(2,4)$ & -- & 3318\\
$(87,32)$ & 10 & $(41,9)$ & 9 & 1 & YES & YES & YES & $2.29$ & $(2,4)$ & -- & 3319\\
$(88,23)$ & 11 & $(79,21)$ & 11 & 1 & YES & YES & YES & $1.83$ & $(4,3)$ & NO & 3320\\
$(91,27)$ & 10 & $(21,8)$ & 6 & 7 & YES & YES & YES & $2.00$ & $(2,4)$ & -- & 3321\\
$(92,35)$ & 10 & $(29,8)$ & 7 & 1 & YES & YES & YES & $2.14$ & $(2,4)$ & -- & 3322\\
$(103,39)$ & 10 & $(11,4)$ & 5 & 1 & YES & YES & YES & $1.83$ & $(4,3)$ & -- & 3323\\
$(111,41)$ & 10 & $(17,5)$ & 6 & 1 & YES & YES & NO(3) & $1.86$ & $(2,4)$ & -- & 3324\\
$(116,49)$ & 10 & $(59,25)$ & 9 & 1 & YES & YES & YES & $1.86$ & $(2,4)$ & NO & 3325\\
$(119,33)$ & 12 & $(7,2)$ & 4 & 7 & YES & YES & YES & $1.71$ & $(2,4)$ & -- & 3326\\
$(121,50)$ & 10 & $(40,17)$ & 9 & 1 & YES & YES & YES & $2.00$ & $(2,4)$ & NO & 3327\\
$(121,43)$ & 11 & $(110,39)$ & 11 & 11 & YES & YES & YES & $1.83$ & $(4,3)$ & NO & 3328\\
$(142,39)$ & 11 & $(8,3)$ & 4 & 2 & YES & YES & YES & $1.83$ & $(4,3)$ & -- & 3329\\
$(149,65)$ & 11 & $(8,3)$ & 4 & 1 & YES & YES & NO(3) & $1.67$ & $(4,3)$ & -- & 3330\\
$(149,40)$ & 11 & $(71,19)$ & 10 & 1 & YES & YES & YES & $1.86$ & $(2,4)$ & NO & 3331\\
$(155,46)$ & 11 & $(131,39)$ & 11 & 1 & YES & YES & NO(3) & $1.83$ & $(2,4)$ & NO & 3332\\
$(157,42)$ & 12 & $(13,4)$ & 6 & 1 & YES & YES & YES & $2.00$ & $(2,4)$ & -- & 3333\\
$(159,44)$ & 11 & $(44,13)$ & 8 & 1 & YES & YES & NO(3) & $2.00$ & $(2,4)$ & NO & 3334\\
$(161,66)$ & 11 & $(14,5)$ & 6 & 7 & YES & YES & YES & $2.17$ & $(4,3)$ & -- & 3335\\
$(161,66)$ & 11 & $(65,27)$ & 10 & 1 & YES & YES & YES & $2.17$ & $(4,3)$ & NO & 3336\\
$(167,64)$ & 11 & $(7,2)$ & 4 & 1 & YES & YES & YES & $1.83$ & $(2,4)$ & -- & 3337\\
$(167,64)$ & 11 & $(8,3)$ & 4 & 1 & YES & YES & NO(3) & $1.67$ & $(4,3)$ & -- & 3338\\
$(169,71)$ & 11 & $(7,2)$ & 4 & 1 & YES & YES & YES & $1.86$ & $(2,4)$ & -- & 3339\\
$(169,30)$ & 12 & $(19,5)$ & 7 & 1 & YES & YES & YES & $2.00$ & $(2,4)$ & -- & 3340\\
$(169,50)$ & 11 & $(29,8)$ & 7 & 1 & YES & YES & YES & $2.00$ & $(2,4)$ & NO & 3341\\
$(173,76)$ & 11 & $(13,5)$ & 5 & 1 & YES & YES & NO(3) & $1.67$ & $(4,3)$ & NO & 3342\\
$(176,49)$ & 12 & $(7,2)$ & 4 & 1 & YES & YES & NO(3) & $1.71$ & $(2,4)$ & -- & 3343\\
$(177,49)$ & 11 & $(149,41)$ & 11 & 1 & YES & YES & YES & $2.00$ & $(2,4)$ & NO & 3344\\
$(185,33)$ & 14 & $(7,2)$ & 4 & 1 & YES & YES & YES & $1.86$ & $(2,4)$ & NO & 3345\\
$(187,57)$ & 12 & $(14,5)$ & 6 & 1 & YES & YES & YES & $2.17$ & $(4,3)$ & -- & 3346\\
$(196,57)$ & 12 & $(13,5)$ & 5 & 1 & YES & YES & YES & $2.00$ & $(2,4)$ & NO & 3347\\
$(202,59)$ & 12 & $(13,3)$ & 6 & 1 & YES & YES & NO(3) & $1.67$ & $(4,3)$ & -- & 3348\\
$(203,59)$ & 12 & $(8,3)$ & 4 & 1 & YES & YES & NO(3) & $1.83$ & $(2,4)$ & -- & 3349\\
$(203,59)$ & 12 & $(58,17)$ & 9 & 29 & YES & YES & NO(3) & $1.67$ & $(4,3)$ & NO & 3350\\
$(219,64)$ & 12 & $(202,59)$ & 12 & 1 & YES & YES & NO(3) & $1.67$ & $(4,3)$ & NO & 3351\\
$(221,50)$ & 12 & $(137,31)$ & 11 & 1 & YES & YES & YES & $1.86$ & $(2,4)$ & NO & 3352\\
$(232,91)$ & 13 & $(31,12)$ & 7 & 1 & YES & YES & YES & $2.00$ & $(2,4)$ & NO & 3353\\
$(247,56)$ & 13 & $(190,43)$ & 12 & 19 & YES & YES & YES & $2.00$ & $(2,4)$ & NO & 3354\\
$(249,95)$ & 12 & $(7,2)$ & 4 & 1 & YES & YES & YES & $2.00$ & $(2,4)$ & NO & 3355\\
$(249,95)$ & 12 & $(7,2)$ & 4 & 1 & YES & YES & YES & $2.00$ & $(2,4)$ & -- & 3356\\
$(261,100)$ & 12 & $(167,64)$ & 11 & 1 & YES & YES & NO(3) & $1.67$ & $(4,3)$ & NO & 3357\\
$(264,115)$ & 12 & $(7,2)$ & 4 & 1 & YES & YES & NO(3) & $1.86$ & $(2,4)$ & NO & 3358\\
$(271,80)$ & 12 & $(8,3)$ & 4 & 1 & YES & YES & YES & $2.00$ & $(2,4)$ & NO & 3359\\
$(273,107)$ & 13 & $(2,1)$ & 1 & 1 & YES & YES & YES & $1.67$ & $(4,3)$ & -- & 3360\\
$(274,107)$ & 12 & $(2,1)$ & 1 & 2 & YES & YES & YES & $1.83$ & $(4,3)$ & -- & 3361\\
$(274,107)$ & 12 & $(59,23)$ & 9 & 1 & YES & YES & NO(3) & $1.67$ & $(4,3)$ & NO & 3362\\
$(281,109)$ & 12 & $(7,2)$ & 4 & 1 & YES & YES & YES & $2.00$ & $(2,4)$ & -- & 3363\\
$(283,75)$ & 13 & $(5,2)$ & 3 & 1 & YES & YES & NO(3) & $1.86$ & $(2,4)$ & -- & 3364\\
$(288,121)$ & 12 & $(3,1)$ & 2 & 3 & YES & YES & YES & $2.00$ & $(2,4)$ & -- & 3365\\
$(288,121)$ & 12 & $(288,121)$ & 12 & 288 & YES & YES & YES & $2.00$ & $(2,4)$ & NO & 3366\\
$(289,84)$ & 13 & $(44,13)$ & 8 & 1 & YES & YES & NO(3) & $2.00$ & $(2,4)$ & NO & 3367\\
$(291,89)$ & 13 & $(5,2)$ & 3 & 1 & YES & YES & NO(3) & $1.86$ & $(2,4)$ & -- & 3368\\
$(295,89)$ & 14 & $(4,1)$ & 3 & 1 & YES & YES & YES & $1.86$ & $(2,4)$ & -- & 3369\\
$(296,107)$ & 13 & $(13,5)$ & 5 & 1 & YES & YES & YES & $2.00$ & $(2,4)$ & NO & 3370\\
$(299,125)$ & 13 & $(19,8)$ & 6 & 1 & YES & YES & YES & $1.83$ & $(4,3)$ & NO & 3371\\
$(307,119)$ & 12 & $(209,81)$ & 11 & 1 & YES & YES & YES & $2.14$ & $(2,4)$ & NO & 3372\\
$(311,119)$ & 12 & $(7,3)$ & 4 & 1 & YES & YES & YES & $2.00$ & $(2,4)$ & -- & 3373\\
$(317,121)$ & 12 & $(97,37)$ & 10 & 1 & YES & YES & NO(3) & $1.83$ & $(2,4)$ & NO & 3374\\
$(318,97)$ & 14 & $(7,3)$ & 4 & 1 & YES & YES & YES & $2.17$ & $(4,3)$ & -- & 3375\\
$(331,129)$ & 13 & $(331,129)$ & 13 & 331 & YES & YES & YES & $2.00$ & $(2,4)$ & NO & 3376\\
$(337,128)$ & 12 & $(92,35)$ & 10 & 1 & YES & YES & YES & $2.00$ & $(2,4)$ & NO & 3377\\
$(340,101)$ & 13 & $(4,1)$ & 3 & 4 & YES & YES & YES & $2.00$ & $(2,4)$ & NO & 3378\\
$(340,101)$ & 13 & $(4,1)$ & 3 & 4 & YES & YES & YES & $2.00$ & $(2,4)$ & -- & 3379\\
$(340,101)$ & 13 & $(4,1)$ & 3 & 4 & YES & YES & YES & $2.00$ & $(2,4)$ & NO & 3380\\
$(346,131)$ & 13 & $(2,1)$ & 1 & 2 & YES & YES & YES & $1.83$ & $(4,3)$ & -- & 3381\\
$(355,128)$ & 13 & $(208,75)$ & 12 & 1 & YES & YES & NO(3) & $1.67$ & $(4,3)$ & NO & 3382\\
$(356,139)$ & 13 & $(2,1)$ & 1 & 2 & YES & YES & YES & $2.00$ & $(4,3)$ & -- & 3383\\
$(363,113)$ & 14 & $(2,1)$ & 1 & 1 & YES & YES & YES & $1.86$ & $(2,4)$ & -- & 3384\\
$(365,159)$ & 13 & $(3,1)$ & 2 & 1 & YES & YES & YES & $1.86$ & $(2,4)$ & -- & 3385\\
$(367,144)$ & 14 & $(5,1)$ & 4 & 1 & YES & YES & YES & $2.00$ & $(2,4)$ & NO & 3386\\
$(367,99)$ & 13 & $(13,4)$ & 6 & 1 & YES & YES & YES & $2.14$ & $(2,4)$ & -- & 3387\\
$(367,144)$ & 14 & $(13,5)$ & 5 & 1 & YES & YES & YES & $2.00$ & $(2,4)$ & NO & 3388\\
$(371,142)$ & 13 & $(4,1)$ & 3 & 1 & YES & YES & YES & $2.00$ & $(2,4)$ & NO & 3389\\
$(372,113)$ & 14 & $(247,75)$ & 13 & 1 & YES & YES & YES & $2.29$ & $(2,4)$ & NO & 3390\\
$(376,85)$ & 14 & $(35,8)$ & 8 & 1 & YES & YES & NO(3) & $1.67$ & $(4,3)$ & NO & 3391\\
$(376,111)$ & 13 & $(64,19)$ & 9 & 8 & YES & YES & YES & $2.00$ & $(2,4)$ & NO & 3392\\
$(388,89)$ & 14 & $(57,13)$ & 9 & 1 & YES & YES & NO(3) & $1.86$ & $(2,4)$ & NO & 3393\\
$(394,167)$ & 13 & $(394,167)$ & 13 & 394 & YES & YES & YES & $2.00$ & $(2,4)$ & NO & 3394\\
$(395,142)$ & 13 & $(8,3)$ & 4 & 1 & YES & YES & YES & $2.00$ & $(2,4)$ & NO & 3395\\
$(401,155)$ & 13 & $(5,2)$ & 3 & 1 & YES & YES & YES & $2.17$ & $(2,4)$ & -- & 3396\\
$(402,157)$ & 13 & $(8,3)$ & 4 & 2 & YES & YES & NO(3) & $1.83$ & $(2,4)$ & NO & 3397\\
$(403,177)$ & 13 & $(2,1)$ & 1 & 1 & YES & YES & YES & $2.00$ & $(2,4)$ & -- & 3398\\
$(403,123)$ & 13 & $(9,2)$ & 5 & 1 & YES & YES & NO(3) & $1.86$ & $(2,4)$ & NO & 3399\\
$(403,123)$ & 13 & $(33,10)$ & 8 & 1 & YES & YES & NO(3) & $1.86$ & $(2,4)$ & NO & 3400\\
$(405,118)$ & 15 & $(6,1)$ & 5 & 3 & YES & YES & YES & $2.00$ & $(2,4)$ & -- & 3401\\
$(413,94)$ & 15 & $(7,1)$ & 6 & 7 & YES & YES & NO(3) & $1.83$ & $(2,4)$ & NO & 3402\\
$(419,173)$ & 13 & $(3,1)$ & 2 & 1 & YES & YES & YES & $1.86$ & $(2,4)$ & -- & 3403\\
$(433,159)$ & 13 & $(5,2)$ & 3 & 1 & YES & YES & NO(3) & $2.00$ & $(2,4)$ & -- & 3404\\
$(437,183)$ & 13 & $(74,31)$ & 9 & 1 & YES & YES & YES & $1.86$ & $(2,4)$ & 3423 & 3405\\
$(439,180)$ & 14 & $(5,2)$ & 3 & 1 & YES & YES & YES & $2.00$ & $(4,3)$ & -- & 3406\\
$(453,187)$ & 13 & $(2,1)$ & 1 & 1 & YES & YES & YES & $2.00$ & $(2,4)$ & NO & 3407\\
$(453,137)$ & 15 & $(23,7)$ & 7 & 1 & YES & YES & YES & $2.00$ & $(2,4)$ & NO & 3408\\
$(457,142)$ & 14 & $(2,1)$ & 1 & 1 & YES & YES & YES & $1.83$ & $(2,4)$ & NO & 3409\\
$(458,123)$ & 13 & $(5,2)$ & 3 & 1 & YES & YES & NO(3) & $1.86$ & $(2,4)$ & -- & 3410\\
$(458,123)$ & 13 & $(5,2)$ & 3 & 1 & YES & YES & NO(3) & $1.86$ & $(2,4)$ & NO & 3411\\
$(458,123)$ & 13 & $(149,40)$ & 11 & 1 & YES & YES & NO(3) & $1.86$ & $(2,4)$ & NO & 3412\\
$(458,123)$ & 13 & $(242,65)$ & 12 & 2 & YES & YES & NO(3) & $1.86$ & $(2,4)$ & NO & 3413\\
$(466,177)$ & 13 & $(34,13)$ & 7 & 2 & YES & YES & YES & $2.00$ & $(2,4)$ & NO & 3414\\
$(467,181)$ & 13 & $(3,1)$ & 2 & 1 & YES & YES & YES & $2.17$ & $(2,4)$ & NO & 3415\\
$(467,181)$ & 13 & $(3,1)$ & 2 & 1 & YES & YES & YES & $2.17$ & $(2,4)$ & -- & 3416\\
$(469,179)$ & 13 & $(2,1)$ & 1 & 1 & YES & YES & YES & $2.00$ & $(2,4)$ & -- & 3417\\
$(481,142)$ & 14 & $(2,1)$ & 1 & 1 & YES & YES & YES & $2.00$ & $(2,4)$ & -- & 3418\\
$(487,185)$ & 13 & $(4,1)$ & 3 & 1 & YES & YES & YES & $2.00$ & $(2,4)$ & -- & 3419\\
$(487,136)$ & 14 & $(29,8)$ & 7 & 1 & YES & YES & YES & $2.14$ & $(2,4)$ & NO & 3420\\
$(487,186)$ & 13 & $(123,47)$ & 10 & 1 & YES & YES & YES & $2.17$ & $(2,4)$ & NO & 3421\\
$(495,137)$ & 14 & $(5,2)$ & 3 & 5 & YES & YES & YES & $2.14$ & $(2,4)$ & -- & 3422\\
$(499,209)$ & 13 & $(43,18)$ & 8 & 1 & YES & YES & YES & $1.86$ & $(2,4)$ & 3405 & 3423\\
$(503,113)$ & 15 & $(49,11)$ & 10 & 1 & YES & YES & YES & $2.00$ & $(2,4)$ & NO & 3424\\
$(526,139)$ & 14 & $(19,5)$ & 7 & 1 & YES & YES & YES & $2.00$ & $(2,4)$ & NO & 3425\\
$(529,121)$ & 15 & $(341,78)$ & 14 & 1 & YES & YES & YES & $1.86$ & $(2,4)$ & NO & 3426\\
$(542,199)$ & 13 & $(2,1)$ & 1 & 2 & YES & YES & YES & $2.00$ & $(2,4)$ & NO & 3427\\
$(547,160)$ & 14 & $(106,31)$ & 10 & 1 & YES & YES & YES & $2.00$ & $(2,4)$ & NO & 3428\\
$(574,155)$ & 14 & $(311,84)$ & 13 & 1 & YES & YES & YES & $2.33$ & $(2,4)$ & NO & 3429\\
$(580,177)$ & 14 & $(2,1)$ & 1 & 2 & YES & YES & NO(3) & $1.83$ & $(2,4)$ & NO & 3430\\
$(580,177)$ & 14 & $(13,4)$ & 6 & 1 & YES & YES & NO(3) & $1.83$ & $(2,4)$ & NO & 3431\\
$(582,223)$ & 15 & $(34,13)$ & 7 & 2 & YES & YES & YES & $2.17$ & $(4,3)$ & NO & 3432\\
$(597,106)$ & 16 & $(169,30)$ & 12 & 1 & YES & YES & YES & $2.00$ & $(2,4)$ & NO & 3433\\
$(611,256)$ & 14 & $(105,44)$ & 10 & 1 & YES & YES & NO(3) & $2.00$ & $(2,4)$ & NO & 3434\\
$(621,140)$ & 15 & $(7,3)$ & 4 & 1 & YES & YES & YES & $2.17$ & $(2,4)$ & -- & 3435\\
$(628,265)$ & 14 & $(3,1)$ & 2 & 1 & YES & YES & NO(3) & $2.00$ & $(2,4)$ & -- & 3436\\
$(628,265)$ & 14 & $(12,5)$ & 5 & 4 & YES & YES & NO(3) & $2.00$ & $(2,4)$ & NO & 3437\\
$(631,259)$ & 15 & $(229,94)$ & 12 & 1 & YES & YES & YES & $2.29$ & $(2,4)$ & NO & 3438\\
$(640,243)$ & 14 & $(108,41)$ & 10 & 4 & YES & YES & NO(3) & $2.00$ & $(2,4)$ & NO & 3439\\
$(647,140)$ & 15 & $(37,8)$ & 8 & 1 & YES & YES & YES & $1.83$ & $(2,4)$ & NO & 3440\\
$(653,250)$ & 14 & $(21,8)$ & 6 & 1 & YES & YES & YES & $2.00$ & $(2,4)$ & NO & 3441\\
$(657,148)$ & 15 & $(3,1)$ & 2 & 3 & YES & YES & NO(3) & $1.86$ & $(2,4)$ & -- & 3442\\
$(657,148)$ & 15 & $(22,5)$ & 7 & 1 & YES & YES & NO(3) & $1.86$ & $(2,4)$ & NO & 3443\\
$(661,279)$ & 15 & $(661,279)$ & 15 & 661 & YES & YES & YES & $2.14$ & $(2,4)$ & NO & 3444\\
$(675,154)$ & 15 & $(3,1)$ & 2 & 3 & YES & YES & NO(3) & $1.86$ & $(2,4)$ & -- & 3445\\
$(675,154)$ & 15 & $(92,21)$ & 10 & 1 & YES & YES & NO(3) & $1.86$ & $(2,4)$ & NO & 3446\\
$(683,282)$ & 14 & $(5,1)$ & 4 & 1 & YES & YES & YES & $2.00$ & $(2,4)$ & -- & 3447\\
$(691,264)$ & 14 & $(5,2)$ & 3 & 1 & YES & YES & YES & $2.00$ & $(2,4)$ & NO & 3448\\
$(703,269)$ & 14 & $(21,8)$ & 6 & 1 & YES & YES & YES & $2.17$ & $(2,4)$ & NO & 3449\\
$(729,215)$ & 15 & $(4,1)$ & 3 & 1 & YES & YES & NO(3) & $2.00$ & $(2,4)$ & NO & 3450\\
$(747,169)$ & 15 & $(2,1)$ & 1 & 1 & YES & YES & NO(3) & $1.86$ & $(2,4)$ & -- & 3451\\
$(752,287)$ & 14 & $(2,1)$ & 1 & 2 & YES & YES & YES & $2.00$ & $(2,4)$ & -- & 3452\\
$(752,287)$ & 14 & $(8,3)$ & 4 & 8 & YES & YES & YES & $2.00$ & $(2,4)$ & NO & 3453\\
$(781,215)$ & 15 & $(29,8)$ & 7 & 1 & YES & YES & YES & $2.14$ & $(2,4)$ & NO & 3454\\
$(788,301)$ & 14 & $(5,2)$ & 3 & 1 & YES & YES & YES & $2.00$ & $(2,4)$ & NO & 3455\\
$(788,301)$ & 14 & $(8,3)$ & 4 & 4 & YES & YES & YES & $2.00$ & $(2,4)$ & NO & 3456\\
$(830,253)$ & 16 & $(187,57)$ & 12 & 1 & YES & YES & YES & $2.17$ & $(4,3)$ & NO & 3457\\
$(882,157)$ & 16 & $(6,1)$ & 5 & 6 & YES & YES & YES & $2.00$ & $(2,4)$ & NO & 3458\\
$(923,259)$ & 16 & $(11,3)$ & 5 & 1 & YES & YES & YES & $2.17$ & $(2,4)$ & NO & 3459\\
$(943,398)$ & 15 & $(263,111)$ & 12 & 1 & YES & YES & YES & $2.14$ & $(2,4)$ & NO & 3460\\
$(945,388)$ & 15 & $(945,388)$ & 15 & 945 & YES & YES & YES & $2.17$ & $(2,4)$ & NO & 3461\\
$(986,157)$ & 19 & $(7,1)$ & 6 & 1 & YES & YES & YES & $2.14$ & $(2,4)$ & NO & 3462\\
$(992,277)$ & 15 & $(25,7)$ & 7 & 1 & YES & YES & YES & $2.00$ & $(2,4)$ & NO & 3463\\
$(1058,409)$ & 15 & $(2,1)$ & 1 & 2 & YES & YES & YES & $2.33$ & $(2,4)$ & -- & 3464\\
$(1783,331)$ & 18 & $(27,5)$ & 8 & 1 & YES & YES & YES & $2.33$ & $(2,4)$ & NO & 3465\\
$(a;1,0,0;13)$ & 5 & $(116,51)$ & 11 & 1 & YES & YES & YES & $2.17$ & $(2,4)$ & -- & 3466\\
$(a;4,1,1;55)$ & 10 & $(13,5)$ & 5 & 1 & YES & YES & YES & $1.83$ & $(4,3)$ & -- & 3467\\
$(b;0,0,0;14)$ & 5 & $(76,29)$ & 9 & 2 & YES & YES & YES & $2.17$ & $(2,4)$ & -- & 3468\\
$(b;0,0,0;14)$ & 5 & $(91,27)$ & 10 & 7 & YES & YES & YES & $2.00$ & $(2,4)$ & -- & 3469\\
$(b;0,0,0;14)$ & 5 & $(105,44)$ & 10 & 7 & YES & YES & NO(3) & $2.00$ & $(2,4)$ & -- & 3470\\
$(b;0,0,0;14)$ & 5 & $(106,41)$ & 10 & 2 & YES & YES & YES & $1.83$ & $(4,3)$ & -- & 3471\\
$(b;0,0,0;14)$ & 5 & $(115,44)$ & 10 & 1 & YES & YES & YES & $2.14$ & $(2,4)$ & -- & 3472\\
$(b;0,0,2;26)$ & 7 & $(56,23)$ & 9 & 2 & YES & YES & YES & $2.14$ & $(2,4)$ & -- & 3473\\
$(b;0,1,0;19)$ & 6 & $(71,27)$ & 9 & 1 & YES & YES & NO(3) & $2.00$ & $(2,4)$ & -- & 3474\\
$(b;0,1,0;19)$ & 6 & $(79,18)$ & 10 & 1 & YES & YES & NO(3) & $1.86$ & $(2,4)$ & -- & 3475\\
$(b;1,0,1;29)$ & 7 & $(41,17)$ & 8 & 1 & YES & YES & NO(3) & $2.00$ & $(2,4)$ & -- & 3476\\
$(b;1,1,1;39)$ & 8 & $(21,8)$ & 6 & 3 & YES & YES & YES & $2.00$ & $(2,4)$ & -- & 3477\\
$(c;0,0,0;4)$ & 4 & $(66,25)$ & 9 & 2 & YES & YES & YES & $1.71$ & $(2,4)$ & -- & 3478\\
$(c;0,0,0;4)$ & 4 & $(85,36)$ & 10 & 1 & YES & YES & YES & $1.67$ & $(4,3)$ & -- & 3479\\
$(c;0,0,0;4)$ & 4 & $(94,41)$ & 10 & 2 & YES & YES & YES & $1.71$ & $(2,4)$ & -- & 3480\\
$(c;0,0,0;4)$ & 4 & $(151,34)$ & 12 & 1 & YES & YES & YES & $1.71$ & $(2,4)$ & -- & 3481\\
$(c;0,0,0;4)$ & 4 & $(153,64)$ & 11 & 1 & YES & YES & YES & $1.83$ & $(2,4)$ & -- & 3482\\
$(c;0,1,0;11)$ & 5 & $(108,41)$ & 10 & 1 & YES & YES & YES & $1.86$ & $(2,4)$ & -- & 3483\\
$(c;0,1,0;11)$ & 5 & $(115,44)$ & 10 & 1 & YES & YES & YES & $1.83$ & $(4,3)$ & -- & 3484\\
$(c;0,1,0;11)$ & 5 & $(236,69)$ & 12 & 1 & YES & YES & NO(3) & $2.00$ & $(2,4)$ & -- & 3485\\
$(d;0,0,0;5)$ & 5 & $(207,80)$ & 12 & 1 & YES & YES & YES & $2.17$ & $(2,4)$ & -- & 3486\\
$(d;0,1,1;17)$ & 7 & $(106,23)$ & 11 & 1 & YES & YES & YES & $2.00$ & $(2,4)$ & -- & 3487\\
$(e;0,1,0;5)$ & 6 & $(71,27)$ & 9 & 1 & YES & YES & YES & $2.17$ & $(2,4)$ & -- & 3488\\
$(e;1,0,0;18)$ & 6 & $(39,17)$ & 8 & 3 & YES & YES & NO(3) & $1.83$ & $(2,4)$ & -- & 3489\\
$(e;1,1,0;23)$ & 7 & $(31,12)$ & 7 & 1 & YES & YES & NO(3) & $1.86$ & $(2,4)$ & -- & 3490\\
$(e;4,0,0;36)$ & 9 & $(22,5)$ & 7 & 2 & YES & YES & YES & $1.86$ & $(2,4)$ & -- & 3491\\
$(f;0,0,0;6)$ & 4 & $(303,116)$ & 12 & 3 & YES & YES & YES & $2.00$ & $(2,4)$ & -- & 3492\\
$(g;0,0,0;19)$ & 6 & $(23,9)$ & 7 & 1 & YES & YES & YES & $1.71$ & $(2,4)$ & -- & 3493\\
$(g;0,0,0;19)$ & 6 & $(49,18)$ & 8 & 1 & YES & YES & YES & $1.83$ & $(2,4)$ & -- & 3494\\
$(g;0,0,0;19)$ & 6 & $(71,27)$ & 9 & 1 & YES & YES & YES & $2.17$ & $(2,4)$ & -- & 3495\\
$(g;0,0,1;26)$ & 7 & $(47,18)$ & 8 & 1 & YES & YES & YES & $2.00$ & $(2,4)$ & -- & 3496\\
$(g;0,0,1;26)$ & 7 & $(61,18)$ & 9 & 1 & YES & YES & YES & $2.00$ & $(2,4)$ & -- & 3497\\
$(g;0,1,0;24)$ & 7 & $(47,18)$ & 8 & 1 & YES & YES & YES & $2.17$ & $(2,4)$ & -- & 3498\\
$(g;0,1,0;24)$ & 7 & $(61,18)$ & 9 & 1 & YES & YES & YES & $2.17$ & $(2,4)$ & -- & 3499\\
$(g;1,0,2;24)$ & 9 & $(8,3)$ & 4 & 8 & YES & YES & YES & $1.67$ & $(4,3)$ & -- & 3500\\
$(g;1,0,2;24)$ & 9 & $(10,3)$ & 5 & 2 & YES & YES & YES & $1.71$ & $(2,4)$ & -- & 3501\\
$(g;1,0,2;24)$ & 9 & $(11,3)$ & 5 & 1 & YES & YES & YES & $1.71$ & $(2,4)$ & -- & 3502\\
$(g;1,0,2;24)$ & 9 & $(22,5)$ & 7 & 2 & YES & YES & NO(3) & $1.83$ & $(2,4)$ & -- & 3503\\
$(h;0,1,0;8)$ & 6 & $(71,27)$ & 9 & 1 & YES & YES & YES & $2.17$ & $(2,4)$ & -- & 3504\\
$(j;0,0,0;8)$ & 5 & $(116,51)$ & 11 & 4 & YES & YES & YES & $2.00$ & $(2,4)$ & -- & 3505
\end{longtable}




%%%%%%%%%%%%%%%%%%%%%%%%%%%%%%%%%%%%%%%%%%%
\section{$2I_5 + 2I_1$}

(3886 examples from 37715968 tests)

Base curves:
\begin{itemize}
  \item $x$.
  \item $y$.
  \item $z$.
  \item $A = x + z$.
  \item $B = x + y + z$.
  \item $C = x+y$.
\end{itemize}
Fibration given by pencil
\[F_\lambda = ABC + \lambda xyz\]

Nine exceptionals are as follows:
\begin{itemize}
  \item $E_1$ - $E_2$ at $y \cap A \cap B = [-1,0,1]$.
  \item $E_3$ - $E_4$ at $x \cap y \cap C = [0,0,1]$.
  \item $E_5$ - $E_6$ at $z \cap B \cap C = [-1,1,0]$.
  \item $E_7$ - $E_8$ at $x \cap z \cap A = [0,1,0]$.
  \item $E_9$ at $x \cap B = [0,-1,1]$.
\end{itemize}
Singular fibers are as follows:
\begin{itemize}
  \item $\lambda = \infty$: $I_5$ fiber given by $x$, $E_3$, $y$, $z$, $E_7$ in order.
  \item $\lambda = 0$: $I_5$ fiber given by $A$, $C$, $E_5$, $B$, $E_1$ in order.
  \item $\lambda = \dfrac{-11 + 5\sqrt{5}}{2}$: $I_1$ fiber called $F_1$ with node at $[-1-\sqrt{5},2,2]$.
  \item $\lambda = \dfrac{-11 - 5\sqrt{5}}{2}$: $I_1$ fiber called $F_2$ with node at $[-1+\sqrt{5},2,2]$.
\end{itemize}
Special curves:
\begin{itemize}
  \item $S = 2x - (-1 -\sqrt{5})y$, double section through $[0,0,1]$ and $[-1-\sqrt{5},2,2]$.
  \item $R = y - z$, triple section through $y \cap z$, $A \cap C$ and both nodes of $I_1$'s.
  \item $Q = x^2 + x - y$, triple section through $y \cap A \cap B$ (tangent with $B$), $x \cap y \cap C$, $x \cap z \cap A$ (tangent with $z$), and both nodes of $I_1$'s.
  \item $T = y + z$, double section through $[1,0,0]$ and $[0,-1,1]$.
\end{itemize}

Input:
\lstinputlisting[language=config]{../Tests/5511.txt}
Result:
%\usepackage{longtable}
\subsection{1 chain, $K^2 = 1$}
\begin{longtable}{|c|c|c|c|c|c|}
\hline
\multicolumn{6}{|c|}{1 chain, $K^2 = 1$}\\
\hline
$(n,a)$ & Length & Nef & $\mathbb Q$-ef & Obstruction 0 & Index\\
\hline
\endfirsthead

\hline
$(n,a)$ & Length & Nef & $\mathbb Q$-ef & Obstruction 0 & Index\\
\hline
\endhead
\hline
\endfoot

$(13, 3)$ & 6 & YES & YES & YES & 1\\
$(13, 4)$ & 6 & YES & YES & YES & 2\\
$(16, 5)$ & 7 & YES & YES & YES & 3\\
$(16, 7)$ & 6 & YES & YES & YES & 4\\
$(17, 7)$ & 6 & YES & YES & YES & 5\\
$(19, 5)$ & 7 & YES & YES & YES & 6\\
$(19, 8)$ & 6 & YES & YES & YES & 7\\
$(21, 5)$ & 8 & YES & YES & YES & 8\\
$(24, 5)$ & 8 & YES & YES & YES & 9\\
$(26, 7)$ & 7 & YES & YES & YES & 10\\
$(a; 1, 0, 0; 13)$ & 5 & YES & YES & YES & 11
\end{longtable}
\subsection{1 chain, $K^2 = 2$}
\begin{longtable}{|c|c|c|c|c|c|}
\hline
\multicolumn{6}{|c|}{1 chain, $K^2 = 2$}\\
\hline
$(n,a)$ & Length & Nef & $\mathbb Q$-ef & Obstruction 0 & Index\\
\hline
\endfirsthead

\hline
$(n,a)$ & Length & Nef & $\mathbb Q$-ef & Obstruction 0 & Index\\
\hline
\endhead
\hline
\endfoot

$(34, 13)$ & 7 & YES & YES & YES & 12\\
$(37, 13)$ & 9 & YES & YES & YES & 13\\
$(37, 17)$ & 9 & YES & YES & YES & 14\\
$(39, 14)$ & 8 & YES & YES & YES & 15\\
$(39, 16)$ & 8 & YES & YES & YES & 16\\
$(41, 13)$ & 10 & YES & YES & YES & 17\\
$(41, 15)$ & 8 & YES & YES & YES & 18\\
$(41, 16)$ & 8 & YES & YES & YES & 19\\
$(41, 17)$ & 8 & YES & YES & YES & 20\\
$(42, 19)$ & 9 & YES & YES & YES & 21\\
$(43, 19)$ & 9 & YES & YES & YES & 22\\
$(44, 17)$ & 8 & YES & YES & YES & 23\\
$(45, 19)$ & 8 & YES & YES & YES & 24\\
$(46, 19)$ & 8 & YES & YES & YES & 25\\
$(48, 17)$ & 9 & YES & YES & YES & 26\\
$(49, 13)$ & 9 & YES & YES & YES & 27\\
$(49, 15)$ & 9 & YES & YES & YES & 28\\
$(49, 18)$ & 8 & YES & YES & YES & 29\\
$(49, 19)$ & 8 & YES & YES & YES & 30\\
$(49, 20)$ & 9 & YES & YES & YES & 31\\
$(49, 22)$ & 9 & YES & YES & YES & 32\\
$(51, 20)$ & 9 & YES & YES & YES & 33\\
$(51, 23)$ & 9 & YES & YES & YES & 34\\
$(52, 19)$ & 9 & YES & YES & YES & 35\\
$(53, 19)$ & 9 & YES & YES & YES & 36\\
$(55, 16)$ & 9 & YES & YES & YES & 37\\
$(59, 23)$ & 9 & YES & YES & YES & 38\\
$(62, 23)$ & 9 & YES & YES & YES & 39\\
$(64, 23)$ & 9 & YES & YES & YES & 40\\
$(65, 24)$ & 9 & YES & YES & YES & 41\\
$(67, 26)$ & 9 & YES & YES & YES & 42\\
$(71, 15)$ & 10 & YES & YES & YES & 43\\
$(71, 27)$ & 9 & YES & YES & YES & 44\\
$(72, 13)$ & 12 & YES & YES & YES & 45\\
$(75, 22)$ & 10 & YES & YES & YES & 46\\
$(76, 13)$ & 12 & YES & YES & YES & 47\\
$(76, 29)$ & 9 & YES & YES & YES & 48\\
$(79, 14)$ & 11 & YES & YES & YES & 49\\
$(79, 22)$ & 10 & YES & YES & YES & 50\\
$(79, 30)$ & 9 & YES & YES & YES & 51\\
$(81, 31)$ & 9 & YES & YES & YES & 52\\
$(85, 33)$ & 10 & YES & YES & YES & 53\\
$(89, 17)$ & 12 & YES & YES & YES & 54\\
$(92, 35)$ & 10 & YES & YES & YES & 55\\
$(95, 36)$ & 10 & YES & YES & YES & 56\\
$(99, 17)$ & 12 & YES & YES & YES & 57\\
$(101, 16)$ & 13 & YES & YES & YES & 58\\
$(105, 31)$ & 10 & YES & YES & YES & 59\\
$(a; 3, 0, 1; 31)$ & 8 & YES & YES & YES & 60\\
$(b; 0, 3, 0; 29)$ & 8 & YES & YES & YES & 61\\
$(b; 1, 1, 0; 27)$ & 7 & YES & YES & YES & 62\\
$(c; 0, 2, 2; 6)$ & 8 & YES & YES & YES & 63\\
$(c; 0, 3, 1; 23)$ & 8 & YES & YES & YES & 64\\
$(c; 0, 3, 2; 29)$ & 9 & YES & YES & YES & 65\\
$(d; 0, 1, 2; 11)$ & 8 & YES & YES & YES & 66\\
$(d; 0, 1, 3; 27)$ & 9 & YES & YES & YES & 67\\
$(d; 0, 2, 2; 13)$ & 9 & YES & YES & YES & 68\\
$(e; 0, 3, 0; 7)$ & 8 & YES & YES & YES & 69\\
$(i; 0, 3, 0; 18)$ & 8 & YES & YES & YES & 70
\end{longtable}
\subsection{1 chain, $K^2 = 3$}
\begin{longtable}{|c|c|c|c|c|c|}
\hline
\multicolumn{6}{|c|}{1 chain, $K^2 = 3$}\\
\hline
$(n,a)$ & Length & Nef & $\mathbb Q$-ef & Obstruction 0 & Index\\
\hline
\endfirsthead

\hline
$(n,a)$ & Length & Nef & $\mathbb Q$-ef & Obstruction 0 & Index\\
\hline
\endhead
\hline
\endfoot

$(67, 26)$ & 9 & YES & YES & NO(2) & 71\\
$(71, 21)$ & 9 & YES & YES & YES & 72\\
$(73, 33)$ & 10 & YES & YES & NO(3) & 73\\
$(79, 29)$ & 9 & YES & YES & NO(2) & 74\\
$(82, 37)$ & 10 & YES & YES & NO(3) & 75\\
$(83, 34)$ & 10 & YES & YES & NO(2) & 76\\
$(85, 36)$ & 10 & YES & YES & NO(2) & 77\\
$(89, 26)$ & 10 & YES & YES & YES & 78\\
$(91, 27)$ & 10 & YES & YES & NO(2) & 79\\
$(93, 26)$ & 10 & YES & YES & YES & 80\\
$(94, 39)$ & 10 & YES & YES & NO(2) & 81\\
$(97, 36)$ & 10 & YES & YES & YES & 82\\
$(97, 41)$ & 10 & YES & YES & NO(2) & 83\\
$(98, 41)$ & 10 & YES & YES & YES & 84\\
$(100, 31)$ & 11 & YES & YES & YES & 85\\
$(100, 37)$ & 10 & YES & YES & YES & 86\\
$(100, 41)$ & 10 & YES & YES & NO(2) & 87\\
$(101, 37)$ & 10 & YES & YES & NO(2) & 88\\
$(103, 47)$ & 12 & YES & YES & YES & 89\\
$(107, 41)$ & 10 & YES & YES & YES & 90\\
$(108, 41)$ & 10 & YES & YES & YES & 91\\
$(111, 46)$ & 10 & YES & YES & YES & 92\\
$(113, 42)$ & 11 & YES & YES & YES & 93\\
$(113, 49)$ & 11 & YES & YES & YES & 94\\
$(116, 51)$ & 11 & YES & YES & YES & 95\\
$(128, 49)$ & 10 & YES & YES & YES & 96\\
$(130, 47)$ & 11 & YES & YES & YES & 97\\
$(132, 47)$ & 12 & YES & YES & YES & 98\\
$(133, 48)$ & 11 & YES & YES & YES & 99\\
$(147, 43)$ & 11 & YES & YES & YES & 100\\
$(151, 32)$ & 12 & YES & YES & YES & 101\\
$(151, 62)$ & 11 & YES & YES & YES & 102\\
$(152, 55)$ & 12 & YES & YES & YES & 103\\
$(160, 67)$ & 11 & YES & YES & NO(2) & 104\\
$(175, 41)$ & 12 & YES & YES & NO(2) & 105\\
$(192, 73)$ & 11 & YES & YES & YES & 106\\
$(199, 74)$ & 12 & YES & YES & YES & 107\\
$(201, 37)$ & 14 & YES & YES & YES & 108\\
$(203, 59)$ & 12 & YES & YES & NO(3) & 109\\
$(205, 78)$ & 12 & YES & YES & YES & 110\\
$(207, 79)$ & 11 & YES & YES & YES & 111\\
$(212, 93)$ & 12 & YES & YES & YES & 112\\
$(215, 63)$ & 12 & YES & YES & YES & 113\\
$(223, 98)$ & 12 & YES & YES & YES & 114\\
$(227, 88)$ & 12 & YES & YES & YES & 115\\
$(229, 87)$ & 12 & YES & YES & YES & 116\\
$(231, 83)$ & 12 & YES & YES & YES & 117\\
$(239, 105)$ & 12 & YES & YES & YES & 118\\
$(246, 73)$ & 12 & YES & YES & YES & 119\\
$(246, 91)$ & 12 & YES & YES & YES & 120\\
$(246, 95)$ & 12 & YES & YES & YES & 121\\
$(251, 74)$ & 13 & YES & YES & YES & 122\\
$(253, 106)$ & 12 & YES & YES & YES & 123\\
$(254, 75)$ & 12 & YES & YES & YES & 124\\
$(256, 75)$ & 12 & YES & YES & YES & 125\\
$(259, 76)$ & 13 & YES & YES & YES & 126\\
$(263, 78)$ & 13 & YES & YES & YES & 127\\
$(269, 78)$ & 13 & YES & YES & YES & 128\\
$(269, 104)$ & 12 & YES & YES & YES & 129\\
$(271, 84)$ & 13 & YES & YES & YES & 130\\
$(271, 112)$ & 12 & YES & YES & YES & 131\\
$(273, 76)$ & 13 & YES & YES & YES & 132\\
$(274, 115)$ & 12 & YES & YES & YES & 133\\
$(280, 107)$ & 12 & YES & YES & YES & 134\\
$(286, 105)$ & 12 & YES & YES & YES & 135\\
$(288, 119)$ & 12 & YES & YES & YES & 136\\
$(292, 85)$ & 13 & YES & YES & YES & 137\\
$(293, 123)$ & 12 & YES & YES & YES & 138\\
$(295, 87)$ & 13 & YES & YES & YES & 139\\
$(305, 112)$ & 12 & YES & YES & YES & 140\\
$(307, 119)$ & 12 & YES & YES & YES & 141\\
$(309, 92)$ & 13 & YES & YES & YES & 142\\
$(313, 86)$ & 13 & YES & YES & YES & 143\\
$(313, 121)$ & 12 & YES & YES & YES & 144\\
$(317, 121)$ & 12 & YES & YES & YES & 145\\
$(320, 93)$ & 13 & YES & YES & YES & 146\\
$(321, 94)$ & 13 & YES & YES & YES & 147\\
$(323, 94)$ & 13 & YES & YES & YES & 148\\
$(325, 74)$ & 14 & YES & YES & YES & 149\\
$(326, 71)$ & 14 & YES & YES & YES & 150\\
$(326, 99)$ & 13 & YES & YES & YES & 151\\
$(338, 129)$ & 12 & YES & YES & YES & 152\\
$(339, 100)$ & 13 & YES & YES & YES & 153\\
$(341, 100)$ & 13 & YES & YES & YES & 154\\
$(343, 131)$ & 12 & YES & YES & YES & 155\\
$(344, 95)$ & 13 & YES & YES & YES & 156\\
$(353, 97)$ & 13 & YES & YES & YES & 157\\
$(359, 100)$ & 13 & YES & YES & YES & 158\\
$(365, 108)$ & 13 & YES & YES & YES & 159\\
$(373, 104)$ & 13 & YES & YES & YES & 160\\
$(376, 105)$ & 13 & YES & YES & YES & 161\\
$(382, 87)$ & 14 & YES & YES & YES & 162\\
$(393, 116)$ & 13 & YES & YES & YES & 163\\
$(397, 116)$ & 13 & YES & YES & YES & 164\\
$(398, 111)$ & 13 & YES & YES & YES & 165\\
$(401, 111)$ & 13 & YES & YES & YES & 166\\
$(409, 121)$ & 13 & YES & YES & YES & 167\\
$(413, 121)$ & 13 & YES & YES & YES & 168\\
$(464, 105)$ & 14 & YES & YES & YES & 169\\
$(487, 111)$ & 14 & YES & YES & YES & 170\\
$(495, 92)$ & 15 & YES & YES & YES & 171\\
$(b; 0, 2, 3; 6)$ & 10 & YES & YES & YES & 172\\
$(e; 3, 2, 0; 16)$ & 10 & YES & YES & YES & 173
\end{longtable}
\subsection{1 chain, $K^2 = 4$}
\begin{longtable}{|c|c|c|c|c|c|}
\hline
\multicolumn{6}{|c|}{1 chain, $K^2 = 4$}\\
\hline
$(n,a)$ & Length & Nef & $\mathbb Q$-ef & Obstruction 0 & Index\\
\hline
\endfirsthead

\hline
$(n,a)$ & Length & Nef & $\mathbb Q$-ef & Obstruction 0 & Index\\
\hline
\endhead
\hline
\endfoot

$(158, 61)$ & 11 & YES & YES & NO(2) & 174\\
$(202, 83)$ & 12 & YES & YES & NO(2) & 175\\
$(331, 119)$ & 13 & YES & YES & YES & 176\\
$(404, 169)$ & 13 & YES & YES & NO(3) & 177\\
$(445, 72)$ & 18 & YES & YES & YES & 178\\
$(448, 171)$ & 13 & YES & YES & YES & 179\\
$(459, 194)$ & 14 & YES & YES & YES & 180\\
$(487, 186)$ & 13 & YES & YES & YES & 181\\
$(535, 158)$ & 14 & YES & YES & YES & 182\\
$(539, 159)$ & 14 & YES & YES & YES & 183\\
$(573, 217)$ & 14 & YES & YES & YES & 184\\
$(577, 239)$ & 14 & YES & YES & YES & 185\\
$(597, 176)$ & 15 & YES & YES & YES & 186\\
$(605, 183)$ & 15 & YES & YES & YES & 187\\
$(611, 237)$ & 14 & YES & YES & YES & 188\\
$(622, 257)$ & 14 & YES & YES & YES & 189\\
$(631, 231)$ & 15 & YES & YES & YES & 190\\
$(647, 246)$ & 14 & YES & YES & YES & 191\\
$(647, 271)$ & 14 & YES & YES & YES & 192\\
$(649, 240)$ & 14 & YES & YES & YES & 193\\
$(673, 196)$ & 15 & YES & YES & YES & 194\\
$(685, 253)$ & 14 & YES & YES & YES & 195\\
$(694, 305)$ & 15 & YES & YES & YES & 196\\
$(697, 266)$ & 14 & YES & YES & YES & 197\\
$(708, 209)$ & 14 & YES & YES & YES & 198\\
$(745, 288)$ & 14 & YES & YES & YES & 199\\
$(755, 312)$ & 14 & YES & YES & YES & 200\\
$(780, 323)$ & 15 & YES & YES & YES & 201\\
$(818, 239)$ & 15 & YES & YES & NO(2) & 202\\
$(853, 313)$ & 15 & YES & YES & YES & 203\\
$(875, 363)$ & 15 & YES & YES & YES & 204\\
$(881, 326)$ & 15 & YES & YES & YES & 205\\
$(882, 337)$ & 14 & YES & YES & YES & 206\\
$(907, 264)$ & 15 & YES & YES & YES & 207\\
$(941, 264)$ & 15 & YES & YES & YES & 208\\
$(997, 295)$ & 15 & YES & YES & YES & 209\\
$(1027, 305)$ & 15 & YES & YES & YES & 210\\
$(1037, 278)$ & 16 & YES & YES & YES & 211\\
$(1047, 307)$ & 16 & YES & YES & YES & 212\\
$(1173, 266)$ & 17 & YES & YES & YES & 213\\
$(1193, 273)$ & 16 & YES & YES & NO(2) & 214\\
$(1415, 593)$ & 16 & YES & YES & YES & 215\\
$(1515, 443)$ & 16 & YES & YES & YES & 216\\
$(1565, 436)$ & 17 & YES & YES & YES & 217\\
$(1663, 487)$ & 17 & YES & YES & YES & 218\\
$(1696, 473)$ & 16 & YES & YES & YES & 219\\
$(1933, 438)$ & 17 & YES & YES & YES & 220\\
$(2204, 503)$ & 17 & YES & YES & YES & 221\\
$(b; 4, 0, 4; 110)$ & 13 & YES & YES & YES & 222
\end{longtable}
\subsection{1 chain, $K^2 = 5$}
\begin{longtable}{|c|c|c|c|c|c|}
\hline
\multicolumn{6}{|c|}{1 chain, $K^2 = 5$}\\
\hline
$(n,a)$ & Length & Nef & $\mathbb Q$-ef & Obstruction 0 & Index\\
\hline
\endfirsthead

\hline
$(n,a)$ & Length & Nef & $\mathbb Q$-ef & Obstruction 0 & Index\\
\hline
\endhead
\hline
\endfoot

$(1435, 403)$ & 16 & YES & YES & NO(3) & 223\\
$(1953, 544)$ & 17 & YES & YES & NO(3) & 224
\end{longtable}
\subsection{2 chains, $K^2 = 1$}
\begin{longtable}{|c|c|c|c|c|c|c|c|c|c|}
\hline
\multicolumn{10}{|c|}{2 chains, $K^2 = 1$}\\
\hline
$(n,a)$ & Length & $(n,a)$ & Length & GCD & Nef & $\mathbb Q$-ef & Obstruction 0 & WH & Index\\
\hline
\endfirsthead

\hline
$(n,a)$ & Length & $(n,a)$ & Length & GCD & Nef & $\mathbb Q$-ef & Obstruction 0 & WH & Index\\
\hline
\endhead
\hline
\endfoot

$(6, 1)$ & 5 & $(5, 2)$ & 3 & 1 & YES & YES & YES & NO & 225\\
$(6, 1)$ & 5 & $(5, 2)$ & 3 & 1 & YES & YES & YES & NO & 226\\
$(7, 3)$ & 4 & $(5, 1)$ & 4 & 1 & YES & YES & YES & NO & 227\\
$(7, 3)$ & 4 & $(5, 1)$ & 4 & 1 & YES & YES & YES & NO & 228\\
$(7, 3)$ & 4 & $(7, 2)$ & 4 & 7 & YES & YES & YES & -- & 229\\
$(7, 3)$ & 4 & $(7, 2)$ & 4 & 7 & YES & YES & YES & NO & 230\\
$(7, 3)$ & 4 & $(7, 2)$ & 4 & 7 & YES & YES & YES & NO & 231\\
$(7, 3)$ & 4 & $(7, 3)$ & 4 & 7 & YES & YES & YES & NO & 232\\
$(8, 3)$ & 4 & $(7, 3)$ & 4 & 1 & YES & YES & YES & -- & 233\\
$(8, 3)$ & 4 & $(7, 3)$ & 4 & 1 & YES & YES & YES & NO & 234\\
$(8, 3)$ & 4 & $(7, 3)$ & 4 & 1 & YES & YES & YES & NO & 235\\
$(9, 2)$ & 5 & $(4, 1)$ & 3 & 1 & YES & YES & YES & -- & 236\\
$(9, 2)$ & 5 & $(4, 1)$ & 3 & 1 & YES & YES & YES & NO & 237\\
$(9, 4)$ & 5 & $(4, 1)$ & 3 & 1 & YES & YES & YES & NO & 238\\
$(9, 4)$ & 5 & $(4, 1)$ & 3 & 1 & YES & YES & YES & NO & 239\\
$(9, 2)$ & 5 & $(5, 1)$ & 4 & 1 & YES & YES & YES & -- & 240\\
$(9, 2)$ & 5 & $(5, 1)$ & 4 & 1 & YES & YES & YES & NO & 241\\
$(9, 2)$ & 5 & $(5, 1)$ & 4 & 1 & YES & YES & YES & NO & 242\\
$(9, 4)$ & 5 & $(5, 2)$ & 3 & 1 & YES & YES & YES & NO & 243\\
$(9, 2)$ & 5 & $(7, 3)$ & 4 & 1 & YES & YES & YES & -- & 244\\
$(9, 2)$ & 5 & $(7, 3)$ & 4 & 1 & YES & YES & YES & NO & 245\\
$(9, 4)$ & 5 & $(7, 2)$ & 4 & 1 & YES & YES & YES & NO & 246\\
$(9, 4)$ & 5 & $(8, 3)$ & 4 & 1 & YES & YES & YES & 294 & 247\\
$(10, 3)$ & 5 & $(5, 2)$ & 3 & 5 & YES & YES & YES & -- & 248\\
$(11, 2)$ & 6 & $(2, 1)$ & 1 & 1 & YES & YES & YES & NO & 249\\
$(11, 3)$ & 5 & $(2, 1)$ & 1 & 1 & YES & YES & YES & -- & 250\\
$(11, 4)$ & 5 & $(3, 1)$ & 2 & 1 & YES & YES & YES & -- & 251\\
$(11, 4)$ & 5 & $(3, 1)$ & 2 & 1 & YES & YES & YES & NO & 252\\
$(11, 5)$ & 6 & $(3, 1)$ & 2 & 1 & YES & YES & YES & -- & 253\\
$(11, 5)$ & 6 & $(3, 1)$ & 2 & 1 & YES & YES & YES & NO & 254\\
$(11, 3)$ & 5 & $(4, 1)$ & 3 & 1 & YES & YES & YES & -- & 255\\
$(11, 3)$ & 5 & $(4, 1)$ & 3 & 1 & YES & YES & YES & NO & 256\\
$(11, 3)$ & 5 & $(4, 1)$ & 3 & 1 & YES & YES & YES & NO & 257\\
$(11, 4)$ & 5 & $(4, 1)$ & 3 & 1 & YES & YES & YES & -- & 258\\
$(11, 4)$ & 5 & $(4, 1)$ & 3 & 1 & YES & YES & YES & NO & 259\\
$(11, 5)$ & 6 & $(4, 1)$ & 3 & 1 & YES & YES & YES & -- & 260\\
$(11, 5)$ & 6 & $(4, 1)$ & 3 & 1 & YES & YES & YES & NO & 261\\
$(11, 5)$ & 6 & $(4, 1)$ & 3 & 1 & YES & YES & YES & NO & 262\\
$(11, 2)$ & 6 & $(5, 1)$ & 4 & 1 & YES & YES & YES & -- & 263\\
$(11, 2)$ & 6 & $(5, 1)$ & 4 & 1 & YES & YES & YES & NO & 264\\
$(11, 2)$ & 6 & $(5, 1)$ & 4 & 1 & YES & YES & YES & NO & 265\\
$(11, 3)$ & 5 & $(5, 2)$ & 3 & 1 & YES & YES & YES & -- & 266\\
$(11, 3)$ & 5 & $(5, 2)$ & 3 & 1 & YES & YES & YES & NO & 267\\
$(11, 4)$ & 5 & $(5, 2)$ & 3 & 1 & YES & YES & YES & -- & 268\\
$(11, 4)$ & 5 & $(5, 2)$ & 3 & 1 & YES & YES & YES & 288 & 269\\
$(11, 5)$ & 6 & $(5, 1)$ & 4 & 1 & YES & YES & YES & NO & 270\\
$(11, 5)$ & 6 & $(5, 1)$ & 4 & 1 & YES & YES & YES & NO & 271\\
$(11, 5)$ & 6 & $(5, 2)$ & 3 & 1 & YES & YES & YES & -- & 272\\
$(11, 5)$ & 6 & $(5, 2)$ & 3 & 1 & YES & YES & YES & NO & 273\\
$(11, 5)$ & 6 & $(6, 1)$ & 5 & 1 & YES & YES & YES & NO & 274\\
$(11, 5)$ & 6 & $(6, 1)$ & 5 & 1 & YES & YES & YES & NO & 275\\
$(11, 5)$ & 6 & $(7, 3)$ & 4 & 1 & YES & YES & YES & 292 & 276\\
$(11, 4)$ & 5 & $(8, 3)$ & 4 & 1 & YES & YES & YES & NO & 277\\
$(11, 2)$ & 6 & $(9, 4)$ & 5 & 1 & YES & YES & YES & NO & 278\\
$(11, 5)$ & 6 & $(9, 4)$ & 5 & 1 & YES & YES & YES & NO & 279\\
$(11, 4)$ & 5 & $(11, 4)$ & 5 & 11 & YES & YES & YES & NO & 280\\
$(11, 5)$ & 6 & $(11, 5)$ & 6 & 11 & YES & YES & YES & NO & 281\\
$(12, 5)$ & 5 & $(3, 1)$ & 2 & 3 & YES & YES & YES & -- & 282\\
$(12, 5)$ & 5 & $(3, 1)$ & 2 & 3 & YES & YES & YES & NO & 283\\
$(12, 5)$ & 5 & $(3, 1)$ & 2 & 3 & YES & YES & YES & NO & 284\\
$(13, 5)$ & 5 & $(2, 1)$ & 1 & 1 & YES & YES & YES & NO & 285\\
$(13, 5)$ & 5 & $(3, 1)$ & 2 & 1 & YES & YES & YES & -- & 286\\
$(13, 5)$ & 5 & $(3, 1)$ & 2 & 1 & YES & YES & YES & NO & 287\\
$(13, 5)$ & 5 & $(3, 1)$ & 2 & 1 & YES & YES & YES & 269 & 288\\
$(13, 3)$ & 6 & $(11, 3)$ & 5 & 1 & YES & YES & YES & NO & 289\\
$(14, 3)$ & 6 & $(5, 1)$ & 4 & 1 & NO & YES & YES & -- & 290\\
$(15, 4)$ & 6 & $(4, 1)$ & 3 & 1 & NO & YES & YES & -- & 291\\
$(16, 7)$ & 6 & $(2, 1)$ & 1 & 2 & YES & YES & YES & 276 & 292\\
$(16, 5)$ & 7 & $(3, 1)$ & 2 & 1 & YES & YES & YES & NO & 293\\
$(16, 7)$ & 6 & $(3, 1)$ & 2 & 1 & YES & YES & YES & 247 & 294\\
$(16, 3)$ & 7 & $(5, 1)$ & 4 & 1 & NO & YES & YES & -- & 295\\
$(16, 3)$ & 7 & $(5, 1)$ & 4 & 1 & NO & YES & YES & NO & 296\\
$(16, 7)$ & 6 & $(5, 1)$ & 4 & 1 & YES & YES & YES & -- & 297\\
$(16, 7)$ & 6 & $(5, 1)$ & 4 & 1 & YES & YES & YES & NO & 298\\
$(16, 7)$ & 6 & $(5, 1)$ & 4 & 1 & YES & YES & YES & NO & 299\\
$(16, 5)$ & 7 & $(7, 1)$ & 6 & 1 & YES & YES & YES & NO & 300\\
$(16, 7)$ & 6 & $(7, 3)$ & 4 & 1 & YES & YES & YES & NO & 301\\
$(16, 7)$ & 6 & $(9, 4)$ & 5 & 1 & YES & YES & YES & NO & 302\\
$(16, 5)$ & 7 & $(13, 4)$ & 6 & 1 & YES & YES & YES & NO & 303\\
$(17, 7)$ & 6 & $(2, 1)$ & 1 & 1 & YES & YES & YES & NO & 304\\
$(19, 8)$ & 6 & $(2, 1)$ & 1 & 1 & NO & YES & YES & -- & 305\\
$(19, 8)$ & 6 & $(2, 1)$ & 1 & 1 & YES & YES & YES & NO & 306\\
$(19, 5)$ & 7 & $(4, 1)$ & 3 & 1 & YES & YES & YES & NO & 307\\
$(19, 5)$ & 7 & $(7, 1)$ & 6 & 1 & YES & YES & YES & NO & 308\\
$(19, 4)$ & 7 & $(11, 2)$ & 6 & 1 & YES & YES & YES & NO & 309\\
$(19, 5)$ & 7 & $(11, 3)$ & 5 & 1 & YES & YES & YES & 318 & 310\\
$(20, 9)$ & 7 & $(2, 1)$ & 1 & 2 & NO & YES & YES & -- & 311\\
$(21, 5)$ & 8 & $(4, 1)$ & 3 & 1 & YES & YES & YES & NO & 312\\
$(23, 10)$ & 7 & $(2, 1)$ & 1 & 1 & NO & YES & YES & -- & 313\\
$(24, 5)$ & 8 & $(5, 1)$ & 4 & 1 & YES & YES & YES & NO & 314\\
$(24, 5)$ & 8 & $(7, 1)$ & 6 & 1 & YES & YES & YES & NO & 315\\
$(24, 5)$ & 8 & $(19, 4)$ & 7 & 1 & YES & YES & YES & NO & 316\\
$(25, 9)$ & 7 & $(2, 1)$ & 1 & 1 & NO & YES & YES & -- & 317\\
$(26, 7)$ & 7 & $(4, 1)$ & 3 & 2 & YES & YES & YES & 310 & 318\\
$(a; 1, 0, 0; 13)$ & 5 & $(2, 1)$ & 1 & 1 & YES & YES & YES & -- & 319\\
$(a; 2, 0, 0; 17)$ & 6 & $(5, 1)$ & 4 & 1 & YES & YES & YES & -- & 320\\
$(c; 0, 1, 1; 5)$ & 6 & $(2, 1)$ & 1 & 1 & YES & YES & YES & -- & 321\\
$(c; 0, 2, 0; 7)$ & 6 & $(2, 1)$ & 1 & 1 & YES & YES & YES & -- & 322\\
$(f; 0, 0, 0; 6)$ & 4 & $(4, 1)$ & 3 & 2 & YES & YES & YES & -- & 323\\
$(f; 0, 0, 0; 6)$ & 4 & $(5, 2)$ & 3 & 1 & YES & YES & YES & -- & 324\\
$(f; 0, 0, 0; 6)$ & 4 & $(7, 3)$ & 4 & 1 & YES & YES & YES & -- & 325\\
$(f; 0, 0, 0; 6)$ & 4 & $(9, 2)$ & 5 & 3 & YES & YES & YES & -- & 326\\
$(f; 0, 1, 0; 7)$ & 5 & $(3, 1)$ & 2 & 1 & YES & YES & YES & -- & 327\\
$(f; 0, 1, 0; 7)$ & 5 & $(4, 1)$ & 3 & 1 & YES & YES & YES & -- & 328\\
$(f; 0, 1, 0; 7)$ & 5 & $(5, 1)$ & 4 & 1 & YES & YES & YES & -- & 329\\
$(j; 0, 0, 0; 8)$ & 5 & $(3, 1)$ & 2 & 1 & YES & YES & YES & -- & 330\\
$(j; 0, 0, 0; 8)$ & 5 & $(5, 1)$ & 4 & 1 & YES & YES & YES & -- & 331
\end{longtable}
\subsection{2 chains, $K^2 = 2$}
\begin{longtable}{|c|c|c|c|c|c|c|c|c|c|}
\hline
\multicolumn{10}{|c|}{2 chains, $K^2 = 2$}\\
\hline
$(n,a)$ & Length & $(n,a)$ & Length & GCD & Nef & $\mathbb Q$-ef & Obstruction 0 & WH & Index\\
\hline
\endfirsthead

\hline
$(n,a)$ & Length & $(n,a)$ & Length & GCD & Nef & $\mathbb Q$-ef & Obstruction 0 & WH & Index\\
\hline
\endhead
\hline
\endfoot

$(11, 4)$ & 5 & $(7, 3)$ & 4 & 1 & YES & YES & NO(2) & -- & 332\\
$(11, 3)$ & 5 & $(9, 4)$ & 5 & 1 & YES & YES & YES & -- & 333\\
$(11, 5)$ & 6 & $(9, 2)$ & 5 & 1 & YES & YES & YES & -- & 334\\
$(11, 5)$ & 6 & $(9, 2)$ & 5 & 1 & YES & YES & YES & NO & 335\\
$(11, 5)$ & 6 & $(9, 2)$ & 5 & 1 & YES & YES & YES & NO & 336\\
$(11, 5)$ & 6 & $(10, 3)$ & 5 & 1 & YES & YES & YES & NO & 337\\
$(11, 5)$ & 6 & $(11, 3)$ & 5 & 11 & YES & YES & YES & -- & 338\\
$(11, 5)$ & 6 & $(11, 3)$ & 5 & 11 & YES & YES & YES & NO & 339\\
$(11, 5)$ & 6 & $(11, 3)$ & 5 & 11 & YES & YES & YES & NO & 340\\
$(11, 5)$ & 6 & $(11, 5)$ & 6 & 11 & YES & YES & YES & -- & 341\\
$(12, 5)$ & 5 & $(7, 3)$ & 4 & 1 & YES & YES & NO(2) & -- & 342\\
$(12, 5)$ & 5 & $(9, 4)$ & 5 & 3 & YES & YES & YES & -- & 343\\
$(12, 5)$ & 5 & $(10, 3)$ & 5 & 2 & YES & YES & YES & -- & 344\\
$(12, 5)$ & 5 & $(11, 3)$ & 5 & 1 & YES & YES & YES & -- & 345\\
$(12, 5)$ & 5 & $(11, 4)$ & 5 & 1 & YES & YES & NO(2) & NO & 346\\
$(13, 5)$ & 5 & $(7, 3)$ & 4 & 1 & YES & YES & NO(2) & -- & 347\\
$(13, 4)$ & 6 & $(8, 3)$ & 4 & 1 & YES & YES & YES & NO & 348\\
$(13, 3)$ & 6 & $(9, 4)$ & 5 & 1 & YES & YES & YES & -- & 349\\
$(13, 3)$ & 6 & $(9, 4)$ & 5 & 1 & YES & YES & YES & NO & 350\\
$(13, 3)$ & 6 & $(9, 4)$ & 5 & 1 & YES & YES & YES & NO & 351\\
$(13, 4)$ & 6 & $(9, 4)$ & 5 & 1 & YES & YES & YES & NO & 352\\
$(13, 5)$ & 5 & $(9, 4)$ & 5 & 1 & YES & YES & NO(2) & -- & 353\\
$(13, 5)$ & 5 & $(10, 3)$ & 5 & 1 & YES & YES & YES & -- & 354\\
$(13, 5)$ & 5 & $(12, 5)$ & 5 & 1 & YES & YES & YES & -- & 355\\
$(14, 5)$ & 6 & $(7, 2)$ & 4 & 7 & YES & YES & YES & -- & 356\\
$(14, 5)$ & 6 & $(7, 2)$ & 4 & 7 & YES & YES & YES & NO & 357\\
$(14, 5)$ & 6 & $(9, 2)$ & 5 & 1 & YES & YES & YES & -- & 358\\
$(14, 5)$ & 6 & $(9, 2)$ & 5 & 1 & YES & YES & YES & NO & 359\\
$(14, 5)$ & 6 & $(10, 3)$ & 5 & 2 & YES & YES & YES & -- & 360\\
$(15, 4)$ & 6 & $(5, 1)$ & 4 & 5 & YES & YES & YES & NO & 361\\
$(15, 4)$ & 6 & $(9, 4)$ & 5 & 3 & YES & YES & YES & -- & 362\\
$(15, 4)$ & 6 & $(12, 5)$ & 5 & 3 & YES & YES & YES & -- & 363\\
$(16, 5)$ & 7 & $(7, 3)$ & 4 & 1 & YES & YES & YES & NO & 364\\
$(16, 7)$ & 6 & $(7, 3)$ & 4 & 1 & YES & YES & YES & -- & 365\\
$(16, 5)$ & 7 & $(9, 4)$ & 5 & 1 & YES & YES & YES & -- & 366\\
$(16, 5)$ & 7 & $(9, 4)$ & 5 & 1 & YES & YES & YES & NO & 367\\
$(16, 3)$ & 7 & $(11, 5)$ & 6 & 1 & YES & YES & YES & -- & 368\\
$(16, 7)$ & 6 & $(11, 4)$ & 5 & 1 & YES & YES & YES & 417 & 369\\
$(16, 5)$ & 7 & $(12, 5)$ & 5 & 4 & YES & YES & YES & -- & 370\\
$(16, 5)$ & 7 & $(12, 5)$ & 5 & 4 & YES & YES & YES & NO & 371\\
$(17, 5)$ & 6 & $(7, 3)$ & 4 & 1 & YES & YES & NO(2) & -- & 372\\
$(17, 5)$ & 6 & $(7, 3)$ & 4 & 1 & YES & YES & YES & NO & 373\\
$(17, 5)$ & 6 & $(8, 3)$ & 4 & 1 & YES & YES & YES & -- & 374\\
$(17, 5)$ & 6 & $(8, 3)$ & 4 & 1 & YES & YES & YES & NO & 375\\
$(17, 6)$ & 7 & $(9, 2)$ & 5 & 1 & YES & YES & YES & -- & 376\\
$(17, 6)$ & 7 & $(9, 4)$ & 5 & 1 & YES & YES & YES & -- & 377\\
$(17, 6)$ & 7 & $(9, 4)$ & 5 & 1 & YES & YES & YES & NO & 378\\
$(17, 7)$ & 6 & $(9, 2)$ & 5 & 1 & YES & YES & YES & -- & 379\\
$(17, 7)$ & 6 & $(10, 3)$ & 5 & 1 & YES & YES & YES & -- & 380\\
$(17, 7)$ & 6 & $(10, 3)$ & 5 & 1 & YES & YES & YES & NO & 381\\
$(17, 3)$ & 7 & $(11, 5)$ & 6 & 1 & YES & YES & YES & -- & 382\\
$(17, 3)$ & 7 & $(11, 5)$ & 6 & 1 & YES & YES & YES & NO & 383\\
$(17, 6)$ & 7 & $(11, 5)$ & 6 & 1 & YES & YES & YES & NO & 384\\
$(17, 7)$ & 6 & $(11, 3)$ & 5 & 1 & YES & YES & YES & -- & 385\\
$(17, 7)$ & 6 & $(11, 5)$ & 6 & 1 & YES & YES & YES & NO & 386\\
$(17, 5)$ & 6 & $(13, 5)$ & 5 & 1 & YES & YES & YES & -- & 387\\
$(17, 5)$ & 6 & $(13, 5)$ & 5 & 1 & YES & YES & YES & NO & 388\\
$(17, 7)$ & 6 & $(17, 5)$ & 6 & 17 & YES & YES & YES & -- & 389\\
$(18, 5)$ & 6 & $(7, 3)$ & 4 & 1 & YES & YES & YES & -- & 390\\
$(18, 5)$ & 6 & $(7, 3)$ & 4 & 1 & YES & YES & YES & NO & 391\\
$(18, 7)$ & 6 & $(7, 3)$ & 4 & 1 & YES & YES & YES & -- & 392\\
$(18, 5)$ & 6 & $(8, 3)$ & 4 & 2 & YES & YES & YES & -- & 393\\
$(18, 5)$ & 6 & $(8, 3)$ & 4 & 2 & YES & YES & YES & NO & 394\\
$(18, 7)$ & 6 & $(9, 2)$ & 5 & 9 & YES & YES & YES & -- & 395\\
$(18, 7)$ & 6 & $(9, 2)$ & 5 & 9 & YES & YES & YES & NO & 396\\
$(18, 7)$ & 6 & $(9, 4)$ & 5 & 9 & YES & YES & YES & -- & 397\\
$(18, 7)$ & 6 & $(9, 4)$ & 5 & 9 & YES & YES & YES & NO & 398\\
$(18, 7)$ & 6 & $(11, 3)$ & 5 & 1 & YES & YES & YES & -- & 399\\
$(18, 7)$ & 6 & $(11, 3)$ & 5 & 1 & YES & YES & YES & NO & 400\\
$(18, 5)$ & 6 & $(13, 4)$ & 6 & 1 & YES & YES & YES & NO & 401\\
$(18, 5)$ & 6 & $(13, 5)$ & 5 & 1 & YES & YES & YES & -- & 402\\
$(18, 7)$ & 6 & $(15, 4)$ & 6 & 3 & YES & YES & YES & -- & 403\\
$(18, 7)$ & 6 & $(15, 4)$ & 6 & 3 & YES & YES & YES & NO & 404\\
$(18, 7)$ & 6 & $(16, 7)$ & 6 & 2 & YES & YES & YES & -- & 405\\
$(18, 5)$ & 6 & $(17, 7)$ & 6 & 1 & YES & YES & YES & NO & 406\\
$(18, 7)$ & 6 & $(17, 4)$ & 7 & 1 & YES & YES & YES & -- & 407\\
$(18, 7)$ & 6 & $(17, 4)$ & 7 & 1 & YES & YES & YES & NO & 408\\
$(18, 7)$ & 6 & $(18, 5)$ & 6 & 18 & YES & YES & YES & -- & 409\\
$(18, 7)$ & 6 & $(18, 5)$ & 6 & 18 & YES & YES & YES & NO & 410\\
$(19, 8)$ & 6 & $(5, 2)$ & 3 & 1 & YES & YES & YES & -- & 411\\
$(19, 7)$ & 6 & $(7, 3)$ & 4 & 1 & YES & YES & YES & -- & 412\\
$(19, 7)$ & 6 & $(7, 3)$ & 4 & 1 & YES & YES & YES & NO & 413\\
$(19, 8)$ & 6 & $(7, 3)$ & 4 & 1 & YES & YES & YES & -- & 414\\
$(19, 4)$ & 7 & $(9, 4)$ & 5 & 1 & YES & YES & YES & -- & 415\\
$(19, 4)$ & 7 & $(9, 4)$ & 5 & 1 & YES & YES & YES & NO & 416\\
$(19, 7)$ & 6 & $(9, 4)$ & 5 & 1 & YES & YES & YES & 369 & 417\\
$(19, 8)$ & 6 & $(9, 4)$ & 5 & 1 & YES & YES & YES & -- & 418\\
$(19, 7)$ & 6 & $(10, 3)$ & 5 & 1 & YES & YES & YES & -- & 419\\
$(19, 7)$ & 6 & $(10, 3)$ & 5 & 1 & YES & YES & YES & NO & 420\\
$(19, 8)$ & 6 & $(10, 3)$ & 5 & 1 & YES & YES & YES & -- & 421\\
$(19, 8)$ & 6 & $(10, 3)$ & 5 & 1 & YES & YES & YES & NO & 422\\
$(19, 4)$ & 7 & $(11, 4)$ & 5 & 1 & YES & YES & YES & -- & 423\\
$(19, 8)$ & 6 & $(11, 4)$ & 5 & 1 & YES & YES & YES & NO & 424\\
$(19, 8)$ & 6 & $(13, 4)$ & 6 & 1 & YES & YES & YES & -- & 425\\
$(19, 8)$ & 6 & $(13, 4)$ & 6 & 1 & YES & YES & YES & NO & 426\\
$(19, 7)$ & 6 & $(14, 5)$ & 6 & 1 & YES & YES & YES & NO & 427\\
$(19, 8)$ & 6 & $(15, 4)$ & 6 & 1 & YES & YES & YES & -- & 428\\
$(19, 8)$ & 6 & $(15, 4)$ & 6 & 1 & YES & YES & YES & NO & 429\\
$(19, 8)$ & 6 & $(15, 4)$ & 6 & 1 & YES & YES & YES & NO & 430\\
$(19, 4)$ & 7 & $(17, 4)$ & 7 & 1 & YES & YES & YES & -- & 431\\
$(19, 5)$ & 7 & $(17, 3)$ & 7 & 1 & YES & YES & YES & -- & 432\\
$(19, 7)$ & 6 & $(17, 6)$ & 7 & 1 & YES & YES & YES & 639 & 433\\
$(19, 8)$ & 6 & $(17, 5)$ & 6 & 1 & YES & YES & YES & NO & 434\\
$(19, 8)$ & 6 & $(17, 7)$ & 6 & 1 & YES & YES & NO(2) & NO & 435\\
$(19, 7)$ & 6 & $(18, 7)$ & 6 & 1 & YES & YES & YES & NO & 436\\
$(19, 8)$ & 6 & $(18, 5)$ & 6 & 1 & YES & YES & YES & -- & 437\\
$(19, 8)$ & 6 & $(18, 5)$ & 6 & 1 & YES & YES & YES & NO & 438\\
$(19, 4)$ & 7 & $(19, 4)$ & 7 & 19 & YES & YES & YES & -- & 439\\
$(20, 9)$ & 7 & $(5, 2)$ & 3 & 5 & YES & YES & YES & -- & 440\\
$(20, 9)$ & 7 & $(5, 2)$ & 3 & 5 & YES & YES & YES & NO & 441\\
$(20, 9)$ & 7 & $(8, 3)$ & 4 & 4 & YES & YES & YES & NO & 442\\
$(20, 9)$ & 7 & $(16, 7)$ & 6 & 4 & YES & YES & YES & NO & 443\\
$(21, 8)$ & 6 & $(5, 2)$ & 3 & 1 & YES & YES & YES & -- & 444\\
$(21, 8)$ & 6 & $(5, 2)$ & 3 & 1 & YES & YES & YES & NO & 445\\
$(21, 8)$ & 6 & $(7, 3)$ & 4 & 7 & YES & YES & YES & -- & 446\\
$(21, 8)$ & 6 & $(7, 3)$ & 4 & 7 & YES & YES & YES & NO & 447\\
$(21, 8)$ & 6 & $(9, 4)$ & 5 & 3 & YES & YES & YES & NO & 448\\
$(21, 8)$ & 6 & $(10, 3)$ & 5 & 1 & YES & YES & YES & -- & 449\\
$(21, 8)$ & 6 & $(10, 3)$ & 5 & 1 & YES & YES & YES & NO & 450\\
$(21, 8)$ & 6 & $(11, 3)$ & 5 & 1 & YES & YES & YES & -- & 451\\
$(21, 8)$ & 6 & $(11, 3)$ & 5 & 1 & YES & YES & YES & NO & 452\\
$(21, 8)$ & 6 & $(12, 5)$ & 5 & 3 & YES & YES & YES & -- & 453\\
$(21, 8)$ & 6 & $(13, 4)$ & 6 & 1 & YES & YES & YES & -- & 454\\
$(21, 8)$ & 6 & $(15, 4)$ & 6 & 3 & YES & YES & YES & -- & 455\\
$(21, 8)$ & 6 & $(17, 4)$ & 7 & 1 & YES & YES & YES & -- & 456\\
$(21, 8)$ & 6 & $(17, 4)$ & 7 & 1 & YES & YES & YES & NO & 457\\
$(21, 8)$ & 6 & $(17, 5)$ & 6 & 1 & YES & YES & YES & -- & 458\\
$(21, 8)$ & 6 & $(17, 5)$ & 6 & 1 & YES & YES & YES & NO & 459\\
$(21, 8)$ & 6 & $(18, 5)$ & 6 & 3 & YES & YES & YES & -- & 460\\
$(21, 8)$ & 6 & $(18, 7)$ & 6 & 3 & YES & YES & YES & NO & 461\\
$(21, 5)$ & 8 & $(21, 4)$ & 8 & 21 & YES & YES & YES & NO & 462\\
$(22, 9)$ & 7 & $(4, 1)$ & 3 & 2 & YES & YES & YES & -- & 463\\
$(22, 9)$ & 7 & $(7, 2)$ & 4 & 1 & YES & YES & YES & -- & 464\\
$(22, 9)$ & 7 & $(7, 2)$ & 4 & 1 & YES & YES & YES & NO & 465\\
$(22, 9)$ & 7 & $(9, 4)$ & 5 & 1 & YES & YES & YES & NO & 466\\
$(22, 9)$ & 7 & $(11, 2)$ & 6 & 11 & YES & YES & YES & NO & 467\\
$(22, 9)$ & 7 & $(17, 4)$ & 7 & 1 & YES & YES & YES & NO & 468\\
$(22, 5)$ & 7 & $(18, 7)$ & 6 & 2 & YES & YES & YES & -- & 469\\
$(22, 5)$ & 7 & $(18, 7)$ & 6 & 2 & YES & YES & YES & NO & 470\\
$(22, 9)$ & 7 & $(19, 4)$ & 7 & 1 & YES & YES & YES & NO & 471\\
$(22, 5)$ & 7 & $(21, 8)$ & 6 & 1 & YES & YES & YES & NO & 472\\
$(23, 9)$ & 7 & $(4, 1)$ & 3 & 1 & YES & YES & YES & -- & 473\\
$(23, 9)$ & 7 & $(4, 1)$ & 3 & 1 & YES & YES & YES & NO & 474\\
$(23, 9)$ & 7 & $(5, 2)$ & 3 & 1 & YES & YES & YES & NO & 475\\
$(23, 7)$ & 7 & $(7, 3)$ & 4 & 1 & YES & YES & NO(2) & -- & 476\\
$(23, 9)$ & 7 & $(7, 3)$ & 4 & 1 & YES & YES & YES & -- & 477\\
$(23, 9)$ & 7 & $(7, 3)$ & 4 & 1 & YES & YES & YES & NO & 478\\
$(23, 6)$ & 8 & $(9, 4)$ & 5 & 1 & YES & YES & YES & NO & 479\\
$(23, 9)$ & 7 & $(10, 3)$ & 5 & 1 & YES & YES & YES & -- & 480\\
$(23, 9)$ & 7 & $(10, 3)$ & 5 & 1 & YES & YES & YES & NO & 481\\
$(23, 9)$ & 7 & $(11, 4)$ & 5 & 1 & YES & YES & YES & NO & 482\\
$(23, 7)$ & 7 & $(12, 5)$ & 5 & 1 & YES & YES & YES & -- & 483\\
$(23, 4)$ & 8 & $(14, 5)$ & 6 & 1 & YES & YES & YES & -- & 484\\
$(23, 4)$ & 8 & $(14, 5)$ & 6 & 1 & YES & YES & YES & NO & 485\\
$(23, 5)$ & 7 & $(17, 7)$ & 6 & 1 & YES & YES & YES & -- & 486\\
$(23, 10)$ & 7 & $(18, 5)$ & 6 & 1 & YES & YES & YES & -- & 487\\
$(23, 5)$ & 7 & $(19, 8)$ & 6 & 1 & YES & YES & YES & 647 & 488\\
$(23, 4)$ & 8 & $(21, 5)$ & 8 & 1 & YES & YES & YES & NO & 489\\
$(23, 10)$ & 7 & $(23, 5)$ & 7 & 23 & YES & YES & YES & -- & 490\\
$(24, 11)$ & 8 & $(3, 1)$ & 2 & 3 & YES & YES & YES & NO & 491\\
$(24, 11)$ & 8 & $(5, 1)$ & 4 & 1 & YES & YES & YES & -- & 492\\
$(24, 11)$ & 8 & $(5, 2)$ & 3 & 1 & YES & YES & YES & NO & 493\\
$(24, 5)$ & 8 & $(9, 4)$ & 5 & 3 & YES & YES & YES & -- & 494\\
$(24, 7)$ & 7 & $(10, 3)$ & 5 & 2 & YES & YES & YES & -- & 495\\
$(24, 7)$ & 7 & $(10, 3)$ & 5 & 2 & YES & YES & YES & NO & 496\\
$(24, 5)$ & 8 & $(11, 3)$ & 5 & 1 & YES & YES & YES & NO & 497\\
$(24, 5)$ & 8 & $(11, 4)$ & 5 & 1 & YES & YES & YES & -- & 498\\
$(24, 7)$ & 7 & $(11, 3)$ & 5 & 1 & YES & YES & YES & -- & 499\\
$(24, 7)$ & 7 & $(11, 3)$ & 5 & 1 & YES & YES & YES & NO & 500\\
$(24, 7)$ & 7 & $(11, 4)$ & 5 & 1 & YES & YES & YES & NO & 501\\
$(24, 7)$ & 7 & $(12, 5)$ & 5 & 12 & YES & YES & YES & -- & 502\\
$(24, 7)$ & 7 & $(12, 5)$ & 5 & 12 & YES & YES & YES & NO & 503\\
$(24, 7)$ & 7 & $(13, 5)$ & 5 & 1 & YES & YES & YES & -- & 504\\
$(24, 5)$ & 8 & $(21, 5)$ & 8 & 3 & YES & YES & YES & NO & 505\\
$(24, 7)$ & 7 & $(23, 5)$ & 7 & 1 & YES & YES & YES & NO & 506\\
$(25, 9)$ & 7 & $(3, 1)$ & 2 & 1 & YES & YES & YES & -- & 507\\
$(25, 9)$ & 7 & $(3, 1)$ & 2 & 1 & YES & YES & YES & NO & 508\\
$(25, 9)$ & 7 & $(4, 1)$ & 3 & 1 & YES & YES & YES & -- & 509\\
$(25, 9)$ & 7 & $(4, 1)$ & 3 & 1 & YES & YES & YES & NO & 510\\
$(25, 9)$ & 7 & $(4, 1)$ & 3 & 1 & YES & YES & YES & NO & 511\\
$(25, 9)$ & 7 & $(5, 2)$ & 3 & 5 & YES & YES & YES & -- & 512\\
$(25, 9)$ & 7 & $(5, 2)$ & 3 & 5 & YES & YES & YES & NO & 513\\
$(25, 9)$ & 7 & $(7, 3)$ & 4 & 1 & YES & YES & YES & -- & 514\\
$(25, 9)$ & 7 & $(7, 3)$ & 4 & 1 & YES & YES & YES & NO & 515\\
$(25, 9)$ & 7 & $(9, 4)$ & 5 & 1 & YES & YES & YES & NO & 516\\
$(25, 7)$ & 7 & $(12, 5)$ & 5 & 1 & YES & YES & YES & -- & 517\\
$(25, 7)$ & 7 & $(12, 5)$ & 5 & 1 & YES & YES & YES & NO & 518\\
$(25, 7)$ & 7 & $(13, 5)$ & 5 & 1 & YES & YES & YES & -- & 519\\
$(25, 9)$ & 7 & $(13, 3)$ & 6 & 1 & YES & YES & YES & NO & 520\\
$(25, 7)$ & 7 & $(23, 7)$ & 7 & 1 & YES & YES & YES & NO & 521\\
$(25, 9)$ & 7 & $(25, 9)$ & 7 & 25 & YES & YES & YES & NO & 522\\
$(26, 11)$ & 7 & $(3, 1)$ & 2 & 1 & YES & YES & NO(2) & -- & 523\\
$(26, 11)$ & 7 & $(3, 1)$ & 2 & 1 & YES & YES & YES & NO & 524\\
$(26, 11)$ & 7 & $(4, 1)$ & 3 & 2 & YES & YES & NO(2) & -- & 525\\
$(26, 11)$ & 7 & $(4, 1)$ & 3 & 2 & YES & YES & YES & NO & 526\\
$(26, 11)$ & 7 & $(5, 2)$ & 3 & 1 & YES & YES & YES & -- & 527\\
$(26, 11)$ & 7 & $(5, 2)$ & 3 & 1 & YES & YES & YES & NO & 528\\
$(26, 11)$ & 7 & $(7, 2)$ & 4 & 1 & YES & YES & YES & -- & 529\\
$(26, 11)$ & 7 & $(7, 2)$ & 4 & 1 & YES & YES & YES & NO & 530\\
$(26, 11)$ & 7 & $(8, 3)$ & 4 & 2 & YES & YES & YES & -- & 531\\
$(26, 11)$ & 7 & $(8, 3)$ & 4 & 2 & YES & YES & YES & 836 & 532\\
$(26, 7)$ & 7 & $(12, 5)$ & 5 & 2 & YES & YES & YES & -- & 533\\
$(26, 11)$ & 7 & $(12, 5)$ & 5 & 2 & YES & YES & NO(2) & 638 & 534\\
$(26, 11)$ & 7 & $(13, 3)$ & 6 & 13 & YES & YES & YES & -- & 535\\
$(26, 11)$ & 7 & $(13, 3)$ & 6 & 13 & YES & YES & YES & NO & 536\\
$(26, 11)$ & 7 & $(14, 3)$ & 6 & 2 & YES & YES & YES & NO & 537\\
$(26, 11)$ & 7 & $(19, 8)$ & 6 & 1 & YES & YES & NO(2) & NO & 538\\
$(26, 7)$ & 7 & $(23, 7)$ & 7 & 1 & YES & YES & YES & NO & 539\\
$(26, 5)$ & 9 & $(26, 5)$ & 9 & 26 & YES & YES & YES & NO & 540\\
$(26, 11)$ & 7 & $(26, 11)$ & 7 & 26 & YES & YES & YES & NO & 541\\
$(27, 10)$ & 7 & $(2, 1)$ & 1 & 1 & YES & YES & NO(2) & -- & 542\\
$(27, 10)$ & 7 & $(2, 1)$ & 1 & 1 & YES & YES & NO(2) & NO & 543\\
$(27, 11)$ & 8 & $(4, 1)$ & 3 & 1 & YES & YES & YES & -- & 544\\
$(27, 8)$ & 7 & $(5, 2)$ & 3 & 1 & YES & YES & YES & -- & 545\\
$(27, 8)$ & 7 & $(5, 2)$ & 3 & 1 & YES & YES & YES & NO & 546\\
$(27, 11)$ & 8 & $(5, 1)$ & 4 & 1 & YES & YES & YES & -- & 547\\
$(27, 11)$ & 8 & $(6, 1)$ & 5 & 3 & YES & YES & YES & -- & 548\\
$(27, 8)$ & 7 & $(7, 3)$ & 4 & 1 & YES & YES & YES & NO & 549\\
$(27, 10)$ & 7 & $(7, 3)$ & 4 & 1 & YES & YES & YES & -- & 550\\
$(27, 10)$ & 7 & $(7, 3)$ & 4 & 1 & YES & YES & YES & NO & 551\\
$(27, 8)$ & 7 & $(8, 3)$ & 4 & 1 & YES & YES & YES & NO & 552\\
$(27, 11)$ & 8 & $(9, 4)$ & 5 & 9 & YES & YES & YES & NO & 553\\
$(27, 8)$ & 7 & $(12, 5)$ & 5 & 3 & YES & YES & YES & -- & 554\\
$(27, 8)$ & 7 & $(12, 5)$ & 5 & 3 & YES & YES & YES & NO & 555\\
$(27, 11)$ & 8 & $(12, 5)$ & 5 & 3 & YES & YES & YES & NO & 556\\
$(27, 8)$ & 7 & $(13, 5)$ & 5 & 1 & YES & YES & YES & -- & 557\\
$(27, 8)$ & 7 & $(13, 5)$ & 5 & 1 & YES & YES & YES & NO & 558\\
$(27, 11)$ & 8 & $(17, 7)$ & 6 & 1 & YES & YES & YES & 753 & 559\\
$(27, 11)$ & 8 & $(22, 9)$ & 7 & 1 & YES & YES & YES & NO & 560\\
$(27, 10)$ & 7 & $(23, 5)$ & 7 & 1 & YES & YES & YES & -- & 561\\
$(27, 10)$ & 7 & $(25, 9)$ & 7 & 1 & YES & YES & YES & NO & 562\\
$(28, 11)$ & 8 & $(4, 1)$ & 3 & 4 & YES & YES & YES & -- & 563\\
$(28, 11)$ & 8 & $(4, 1)$ & 3 & 4 & YES & YES & YES & NO & 564\\
$(28, 11)$ & 8 & $(5, 1)$ & 4 & 1 & YES & YES & YES & -- & 565\\
$(28, 11)$ & 8 & $(5, 2)$ & 3 & 1 & YES & YES & YES & NO & 566\\
$(28, 11)$ & 8 & $(6, 1)$ & 5 & 2 & YES & YES & YES & -- & 567\\
$(28, 11)$ & 8 & $(6, 1)$ & 5 & 2 & YES & YES & YES & NO & 568\\
$(28, 11)$ & 8 & $(13, 3)$ & 6 & 1 & YES & YES & YES & -- & 569\\
$(28, 11)$ & 8 & $(13, 5)$ & 5 & 1 & YES & YES & YES & NO & 570\\
$(28, 11)$ & 8 & $(14, 3)$ & 6 & 14 & YES & YES & YES & -- & 571\\
$(28, 11)$ & 8 & $(16, 3)$ & 7 & 4 & YES & YES & YES & -- & 572\\
$(28, 11)$ & 8 & $(17, 3)$ & 7 & 1 & YES & YES & YES & NO & 573\\
$(28, 11)$ & 8 & $(18, 7)$ & 6 & 2 & YES & YES & YES & 781 & 574\\
$(28, 11)$ & 8 & $(23, 9)$ & 7 & 1 & YES & YES & YES & NO & 575\\
$(29, 11)$ & 7 & $(3, 1)$ & 2 & 1 & YES & YES & YES & -- & 576\\
$(29, 12)$ & 7 & $(3, 1)$ & 2 & 1 & YES & YES & YES & -- & 577\\
$(29, 11)$ & 7 & $(4, 1)$ & 3 & 1 & YES & YES & YES & -- & 578\\
$(29, 12)$ & 7 & $(4, 1)$ & 3 & 1 & YES & YES & YES & -- & 579\\
$(29, 9)$ & 8 & $(5, 2)$ & 3 & 1 & YES & YES & YES & -- & 580\\
$(29, 9)$ & 8 & $(5, 2)$ & 3 & 1 & YES & YES & YES & NO & 581\\
$(29, 11)$ & 7 & $(5, 2)$ & 3 & 1 & YES & YES & YES & -- & 582\\
$(29, 12)$ & 7 & $(5, 2)$ & 3 & 1 & YES & YES & YES & -- & 583\\
$(29, 8)$ & 7 & $(7, 3)$ & 4 & 1 & YES & YES & YES & -- & 584\\
$(29, 8)$ & 7 & $(7, 3)$ & 4 & 1 & YES & YES & YES & NO & 585\\
$(29, 11)$ & 7 & $(7, 2)$ & 4 & 1 & YES & YES & YES & -- & 586\\
$(29, 11)$ & 7 & $(7, 3)$ & 4 & 1 & YES & YES & YES & -- & 587\\
$(29, 11)$ & 7 & $(7, 3)$ & 4 & 1 & YES & YES & YES & 774 & 588\\
$(29, 12)$ & 7 & $(7, 3)$ & 4 & 1 & YES & YES & YES & NO & 589\\
$(29, 8)$ & 7 & $(8, 3)$ & 4 & 1 & YES & YES & YES & -- & 590\\
$(29, 8)$ & 7 & $(8, 3)$ & 4 & 1 & YES & YES & YES & NO & 591\\
$(29, 8)$ & 7 & $(8, 3)$ & 4 & 1 & YES & YES & YES & NO & 592\\
$(29, 11)$ & 7 & $(10, 3)$ & 5 & 1 & YES & YES & YES & -- & 593\\
$(29, 12)$ & 7 & $(10, 3)$ & 5 & 1 & YES & YES & YES & -- & 594\\
$(29, 12)$ & 7 & $(10, 3)$ & 5 & 1 & YES & YES & YES & NO & 595\\
$(29, 8)$ & 7 & $(13, 5)$ & 5 & 1 & YES & YES & YES & -- & 596\\
$(29, 8)$ & 7 & $(13, 5)$ & 5 & 1 & YES & YES & YES & NO & 597\\
$(29, 11)$ & 7 & $(13, 3)$ & 6 & 1 & YES & YES & YES & -- & 598\\
$(29, 11)$ & 7 & $(13, 3)$ & 6 & 1 & YES & YES & YES & NO & 599\\
$(29, 11)$ & 7 & $(13, 3)$ & 6 & 1 & YES & YES & YES & NO & 600\\
$(29, 11)$ & 7 & $(13, 5)$ & 5 & 1 & YES & YES & YES & 693 & 601\\
$(29, 11)$ & 7 & $(14, 3)$ & 6 & 1 & YES & YES & YES & -- & 602\\
$(29, 11)$ & 7 & $(14, 3)$ & 6 & 1 & YES & YES & YES & NO & 603\\
$(29, 11)$ & 7 & $(14, 3)$ & 6 & 1 & YES & YES & YES & NO & 604\\
$(29, 12)$ & 7 & $(17, 4)$ & 7 & 1 & YES & YES & YES & NO & 605\\
$(29, 11)$ & 7 & $(21, 8)$ & 6 & 1 & YES & YES & YES & NO & 606\\
$(29, 12)$ & 7 & $(22, 9)$ & 7 & 1 & YES & YES & YES & NO & 607\\
$(29, 8)$ & 7 & $(23, 5)$ & 7 & 1 & YES & YES & YES & NO & 608\\
$(29, 8)$ & 7 & $(23, 7)$ & 7 & 1 & YES & YES & YES & NO & 609\\
$(29, 11)$ & 7 & $(29, 11)$ & 7 & 29 & YES & YES & YES & NO & 610\\
$(30, 11)$ & 7 & $(3, 1)$ & 2 & 3 & YES & YES & YES & -- & 611\\
$(30, 11)$ & 7 & $(4, 1)$ & 3 & 2 & YES & YES & YES & -- & 612\\
$(30, 11)$ & 7 & $(4, 1)$ & 3 & 2 & YES & YES & YES & NO & 613\\
$(30, 13)$ & 8 & $(4, 1)$ & 3 & 2 & YES & YES & YES & -- & 614\\
$(30, 11)$ & 7 & $(5, 2)$ & 3 & 5 & YES & YES & YES & -- & 615\\
$(30, 11)$ & 7 & $(5, 2)$ & 3 & 5 & YES & YES & YES & 682 & 616\\
$(30, 13)$ & 8 & $(5, 1)$ & 4 & 5 & YES & YES & YES & -- & 617\\
$(30, 13)$ & 8 & $(5, 1)$ & 4 & 5 & YES & YES & YES & NO & 618\\
$(30, 11)$ & 7 & $(7, 3)$ & 4 & 1 & YES & YES & YES & -- & 619\\
$(30, 11)$ & 7 & $(7, 3)$ & 4 & 1 & YES & YES & YES & 889 & 620\\
$(30, 11)$ & 7 & $(9, 4)$ & 5 & 3 & YES & YES & YES & NO & 621\\
$(30, 11)$ & 7 & $(10, 3)$ & 5 & 10 & YES & YES & YES & -- & 622\\
$(30, 11)$ & 7 & $(11, 3)$ & 5 & 1 & YES & YES & YES & -- & 623\\
$(30, 11)$ & 7 & $(30, 11)$ & 7 & 30 & YES & YES & YES & NO & 624\\
$(31, 13)$ & 7 & $(2, 1)$ & 1 & 1 & YES & YES & YES & -- & 625\\
$(31, 12)$ & 7 & $(3, 1)$ & 2 & 1 & YES & YES & YES & -- & 626\\
$(31, 13)$ & 7 & $(3, 1)$ & 2 & 1 & YES & YES & YES & -- & 627\\
$(31, 13)$ & 7 & $(3, 1)$ & 2 & 1 & YES & YES & YES & NO & 628\\
$(31, 13)$ & 7 & $(3, 1)$ & 2 & 1 & YES & YES & YES & NO & 629\\
$(31, 14)$ & 8 & $(3, 1)$ & 2 & 1 & YES & YES & YES & -- & 630\\
$(31, 14)$ & 8 & $(3, 1)$ & 2 & 1 & YES & YES & YES & NO & 631\\
$(31, 12)$ & 7 & $(4, 1)$ & 3 & 1 & YES & YES & YES & -- & 632\\
$(31, 11)$ & 8 & $(5, 2)$ & 3 & 1 & YES & YES & YES & NO & 633\\
$(31, 13)$ & 7 & $(5, 2)$ & 3 & 1 & YES & YES & YES & NO & 634\\
$(31, 9)$ & 8 & $(7, 3)$ & 4 & 1 & YES & YES & YES & -- & 635\\
$(31, 9)$ & 8 & $(7, 3)$ & 4 & 1 & YES & YES & YES & NO & 636\\
$(31, 13)$ & 7 & $(7, 3)$ & 4 & 1 & YES & YES & YES & -- & 637\\
$(31, 13)$ & 7 & $(7, 3)$ & 4 & 1 & YES & YES & NO(2) & 534 & 638\\
$(31, 11)$ & 8 & $(8, 3)$ & 4 & 1 & YES & YES & YES & 433 & 639\\
$(31, 12)$ & 7 & $(8, 3)$ & 4 & 1 & YES & YES & YES & NO & 640\\
$(31, 12)$ & 7 & $(9, 4)$ & 5 & 1 & YES & YES & YES & -- & 641\\
$(31, 9)$ & 8 & $(10, 3)$ & 5 & 1 & YES & YES & YES & -- & 642\\
$(31, 13)$ & 7 & $(10, 3)$ & 5 & 1 & YES & YES & YES & NO & 643\\
$(31, 9)$ & 8 & $(11, 3)$ & 5 & 1 & YES & YES & YES & -- & 644\\
$(31, 13)$ & 7 & $(11, 3)$ & 5 & 1 & YES & YES & YES & NO & 645\\
$(31, 13)$ & 7 & $(13, 3)$ & 6 & 1 & YES & YES & YES & NO & 646\\
$(31, 13)$ & 7 & $(14, 3)$ & 6 & 1 & YES & YES & YES & 488 & 647\\
$(31, 9)$ & 8 & $(16, 3)$ & 7 & 1 & YES & YES & YES & -- & 648\\
$(31, 7)$ & 8 & $(17, 5)$ & 6 & 1 & YES & YES & YES & NO & 649\\
$(31, 12)$ & 7 & $(18, 7)$ & 6 & 1 & YES & YES & YES & NO & 650\\
$(31, 7)$ & 8 & $(19, 4)$ & 7 & 1 & YES & YES & YES & -- & 651\\
$(31, 7)$ & 8 & $(19, 7)$ & 6 & 1 & YES & YES & YES & -- & 652\\
$(31, 9)$ & 8 & $(23, 7)$ & 7 & 1 & YES & YES & YES & 1234 & 653\\
$(32, 9)$ & 8 & $(7, 3)$ & 4 & 1 & YES & YES & YES & -- & 654\\
$(32, 9)$ & 8 & $(7, 3)$ & 4 & 1 & YES & YES & YES & NO & 655\\
$(32, 9)$ & 8 & $(10, 3)$ & 5 & 2 & YES & YES & YES & -- & 656\\
$(32, 9)$ & 8 & $(19, 4)$ & 7 & 1 & YES & YES & YES & NO & 657\\
$(32, 7)$ & 8 & $(21, 5)$ & 8 & 1 & YES & YES & YES & NO & 658\\
$(33, 14)$ & 8 & $(3, 1)$ & 2 & 3 & YES & YES & YES & -- & 659\\
$(33, 10)$ & 8 & $(4, 1)$ & 3 & 1 & YES & YES & YES & -- & 660\\
$(33, 10)$ & 8 & $(4, 1)$ & 3 & 1 & YES & YES & YES & NO & 661\\
$(33, 10)$ & 8 & $(4, 1)$ & 3 & 1 & YES & YES & YES & NO & 662\\
$(33, 14)$ & 8 & $(5, 1)$ & 4 & 1 & YES & YES & NO(2) & NO & 663\\
$(33, 14)$ & 8 & $(6, 1)$ & 5 & 3 & YES & YES & YES & -- & 664\\
$(33, 14)$ & 8 & $(6, 1)$ & 5 & 3 & YES & YES & YES & NO & 665\\
$(33, 10)$ & 8 & $(7, 3)$ & 4 & 1 & YES & YES & YES & NO & 666\\
$(33, 14)$ & 8 & $(7, 2)$ & 4 & 1 & YES & YES & YES & -- & 667\\
$(33, 10)$ & 8 & $(10, 3)$ & 5 & 1 & YES & YES & YES & -- & 668\\
$(33, 14)$ & 8 & $(11, 2)$ & 6 & 11 & YES & YES & YES & -- & 669\\
$(33, 10)$ & 8 & $(13, 4)$ & 6 & 1 & YES & YES & YES & 715 & 670\\
$(33, 10)$ & 8 & $(14, 3)$ & 6 & 1 & YES & YES & YES & -- & 671\\
$(33, 14)$ & 8 & $(19, 8)$ & 6 & 1 & YES & YES & NO(2) & 858 & 672\\
$(33, 14)$ & 8 & $(26, 11)$ & 7 & 1 & YES & YES & YES & NO & 673\\
$(33, 14)$ & 8 & $(31, 13)$ & 7 & 1 & YES & YES & YES & 1014 & 674\\
$(33, 14)$ & 8 & $(33, 14)$ & 8 & 33 & YES & YES & YES & NO & 675\\
$(34, 9)$ & 8 & $(2, 1)$ & 1 & 2 & YES & YES & YES & NO & 676\\
$(34, 13)$ & 7 & $(2, 1)$ & 1 & 2 & YES & YES & YES & -- & 677\\
$(34, 13)$ & 7 & $(2, 1)$ & 1 & 2 & YES & YES & YES & NO & 678\\
$(34, 9)$ & 8 & $(3, 1)$ & 2 & 1 & YES & YES & YES & -- & 679\\
$(34, 9)$ & 8 & $(3, 1)$ & 2 & 1 & YES & YES & YES & NO & 680\\
$(34, 13)$ & 7 & $(3, 1)$ & 2 & 1 & YES & YES & YES & -- & 681\\
$(34, 13)$ & 7 & $(3, 1)$ & 2 & 1 & YES & YES & YES & 616 & 682\\
$(34, 15)$ & 8 & $(3, 1)$ & 2 & 1 & YES & YES & YES & -- & 683\\
$(34, 15)$ & 8 & $(3, 1)$ & 2 & 1 & YES & YES & YES & NO & 684\\
$(34, 15)$ & 8 & $(4, 1)$ & 3 & 2 & YES & YES & YES & -- & 685\\
$(34, 15)$ & 8 & $(4, 1)$ & 3 & 2 & YES & YES & YES & NO & 686\\
$(34, 13)$ & 7 & $(5, 2)$ & 3 & 1 & YES & YES & YES & NO & 687\\
$(34, 9)$ & 8 & $(7, 3)$ & 4 & 1 & YES & YES & YES & NO & 688\\
$(34, 13)$ & 7 & $(7, 3)$ & 4 & 1 & YES & YES & YES & -- & 689\\
$(34, 13)$ & 7 & $(7, 3)$ & 4 & 1 & YES & YES & YES & NO & 690\\
$(34, 9)$ & 8 & $(8, 3)$ & 4 & 2 & YES & YES & YES & NO & 691\\
$(34, 13)$ & 7 & $(8, 3)$ & 4 & 2 & YES & YES & YES & -- & 692\\
$(34, 13)$ & 7 & $(8, 3)$ & 4 & 2 & YES & YES & YES & 601 & 693\\
$(34, 15)$ & 8 & $(8, 3)$ & 4 & 2 & YES & YES & YES & -- & 694\\
$(34, 13)$ & 7 & $(11, 3)$ & 5 & 1 & YES & YES & YES & -- & 695\\
$(34, 13)$ & 7 & $(11, 3)$ & 5 & 1 & YES & YES & YES & NO & 696\\
$(34, 15)$ & 8 & $(11, 3)$ & 5 & 1 & YES & YES & YES & NO & 697\\
$(34, 13)$ & 7 & $(13, 3)$ & 6 & 1 & YES & YES & YES & -- & 698\\
$(34, 13)$ & 7 & $(13, 3)$ & 6 & 1 & YES & YES & YES & NO & 699\\
$(34, 13)$ & 7 & $(13, 3)$ & 6 & 1 & YES & YES & YES & NO & 700\\
$(34, 13)$ & 7 & $(31, 12)$ & 7 & 1 & YES & YES & YES & NO & 701\\
$(35, 13)$ & 8 & $(4, 1)$ & 3 & 1 & YES & YES & YES & -- & 702\\
$(35, 13)$ & 8 & $(4, 1)$ & 3 & 1 & YES & YES & YES & NO & 703\\
$(35, 6)$ & 10 & $(5, 2)$ & 3 & 5 & YES & YES & YES & -- & 704\\
$(35, 6)$ & 10 & $(5, 2)$ & 3 & 5 & YES & YES & YES & NO & 705\\
$(35, 6)$ & 10 & $(9, 2)$ & 5 & 1 & YES & YES & YES & NO & 706\\
$(35, 8)$ & 8 & $(13, 4)$ & 6 & 1 & YES & YES & YES & NO & 707\\
$(35, 13)$ & 8 & $(14, 5)$ & 6 & 7 & YES & YES & YES & NO & 708\\
$(35, 8)$ & 8 & $(17, 5)$ & 6 & 1 & YES & YES & YES & NO & 709\\
$(36, 11)$ & 8 & $(2, 1)$ & 1 & 2 & YES & YES & YES & NO & 710\\
$(36, 11)$ & 8 & $(5, 1)$ & 4 & 1 & YES & YES & YES & -- & 711\\
$(36, 11)$ & 8 & $(5, 1)$ & 4 & 1 & YES & YES & YES & NO & 712\\
$(36, 11)$ & 8 & $(5, 2)$ & 3 & 1 & YES & YES & YES & -- & 713\\
$(36, 13)$ & 8 & $(7, 3)$ & 4 & 1 & YES & YES & YES & NO & 714\\
$(36, 11)$ & 8 & $(10, 3)$ & 5 & 2 & YES & YES & YES & 670 & 715\\
$(36, 13)$ & 8 & $(10, 3)$ & 5 & 2 & YES & YES & YES & -- & 716\\
$(36, 11)$ & 8 & $(13, 4)$ & 6 & 1 & YES & YES & YES & NO & 717\\
$(36, 13)$ & 8 & $(14, 3)$ & 6 & 2 & YES & YES & YES & -- & 718\\
$(36, 11)$ & 8 & $(15, 4)$ & 6 & 3 & YES & YES & YES & NO & 719\\
$(36, 11)$ & 8 & $(27, 8)$ & 7 & 9 & YES & YES & YES & NO & 720\\
$(37, 11)$ & 8 & $(2, 1)$ & 1 & 1 & YES & YES & YES & NO & 721\\
$(37, 11)$ & 8 & $(3, 1)$ & 2 & 1 & YES & YES & YES & -- & 722\\
$(37, 11)$ & 8 & $(3, 1)$ & 2 & 1 & YES & YES & YES & NO & 723\\
$(37, 11)$ & 8 & $(5, 2)$ & 3 & 1 & YES & YES & YES & -- & 724\\
$(37, 11)$ & 8 & $(5, 2)$ & 3 & 1 & YES & YES & YES & NO & 725\\
$(37, 16)$ & 9 & $(6, 1)$ & 5 & 1 & YES & YES & YES & -- & 726\\
$(37, 11)$ & 8 & $(7, 3)$ & 4 & 1 & YES & YES & YES & -- & 727\\
$(37, 16)$ & 9 & $(7, 1)$ & 6 & 1 & YES & YES & YES & NO & 728\\
$(37, 16)$ & 9 & $(7, 1)$ & 6 & 1 & YES & YES & YES & NO & 729\\
$(37, 11)$ & 8 & $(8, 3)$ & 4 & 1 & YES & YES & YES & -- & 730\\
$(37, 11)$ & 8 & $(11, 3)$ & 5 & 1 & YES & YES & YES & -- & 731\\
$(37, 11)$ & 8 & $(11, 3)$ & 5 & 1 & YES & YES & YES & 1029 & 732\\
$(37, 14)$ & 8 & $(11, 2)$ & 6 & 1 & YES & YES & YES & -- & 733\\
$(37, 14)$ & 8 & $(11, 2)$ & 6 & 1 & YES & YES & YES & NO & 734\\
$(37, 14)$ & 8 & $(11, 2)$ & 6 & 1 & YES & YES & YES & NO & 735\\
$(37, 10)$ & 8 & $(13, 4)$ & 6 & 1 & YES & YES & YES & -- & 736\\
$(37, 11)$ & 8 & $(13, 3)$ & 6 & 1 & YES & YES & YES & NO & 737\\
$(37, 11)$ & 8 & $(14, 3)$ & 6 & 1 & YES & YES & YES & NO & 738\\
$(37, 11)$ & 8 & $(15, 4)$ & 6 & 1 & YES & YES & YES & NO & 739\\
$(37, 11)$ & 8 & $(18, 5)$ & 6 & 1 & YES & YES & YES & NO & 740\\
$(37, 14)$ & 8 & $(29, 11)$ & 7 & 1 & YES & YES & YES & NO & 741\\
$(37, 16)$ & 9 & $(30, 13)$ & 8 & 1 & YES & YES & YES & NO & 742\\
$(37, 11)$ & 8 & $(31, 9)$ & 8 & 1 & YES & YES & YES & NO & 743\\
$(37, 10)$ & 8 & $(32, 9)$ & 8 & 1 & YES & YES & YES & NO & 744\\
$(37, 16)$ & 9 & $(37, 16)$ & 9 & 37 & YES & YES & YES & NO & 745\\
$(38, 11)$ & 9 & $(24, 7)$ & 7 & 2 & YES & YES & YES & 975 & 746\\
$(39, 16)$ & 8 & $(2, 1)$ & 1 & 1 & YES & YES & YES & NO & 747\\
$(39, 17)$ & 8 & $(2, 1)$ & 1 & 1 & YES & YES & YES & -- & 748\\
$(39, 14)$ & 8 & $(3, 1)$ & 2 & 3 & YES & YES & YES & -- & 749\\
$(39, 16)$ & 8 & $(3, 1)$ & 2 & 3 & YES & YES & YES & -- & 750\\
$(39, 16)$ & 8 & $(3, 1)$ & 2 & 3 & YES & YES & YES & NO & 751\\
$(39, 16)$ & 8 & $(5, 1)$ & 4 & 1 & YES & YES & YES & -- & 752\\
$(39, 16)$ & 8 & $(5, 2)$ & 3 & 1 & YES & YES & YES & 559 & 753\\
$(39, 16)$ & 8 & $(7, 2)$ & 4 & 1 & YES & YES & YES & -- & 754\\
$(39, 14)$ & 8 & $(8, 3)$ & 4 & 1 & YES & YES & YES & 894 & 755\\
$(39, 14)$ & 8 & $(13, 3)$ & 6 & 13 & YES & YES & YES & -- & 756\\
$(39, 17)$ & 8 & $(13, 5)$ & 5 & 13 & YES & YES & YES & NO & 757\\
$(39, 16)$ & 8 & $(19, 8)$ & 6 & 1 & YES & YES & YES & NO & 758\\
$(39, 17)$ & 8 & $(34, 15)$ & 8 & 1 & YES & YES & YES & NO & 759\\
$(40, 11)$ & 8 & $(5, 2)$ & 3 & 5 & YES & YES & YES & -- & 760\\
$(40, 11)$ & 8 & $(7, 3)$ & 4 & 1 & YES & YES & YES & -- & 761\\
$(40, 11)$ & 8 & $(10, 3)$ & 5 & 10 & YES & YES & YES & 999 & 762\\
$(40, 9)$ & 9 & $(11, 4)$ & 5 & 1 & YES & YES & YES & -- & 763\\
$(40, 11)$ & 8 & $(17, 5)$ & 6 & 1 & YES & YES & YES & NO & 764\\
$(41, 15)$ & 8 & $(2, 1)$ & 1 & 1 & YES & YES & YES & NO & 765\\
$(41, 16)$ & 8 & $(2, 1)$ & 1 & 1 & YES & YES & YES & -- & 766\\
$(41, 16)$ & 8 & $(2, 1)$ & 1 & 1 & YES & YES & YES & NO & 767\\
$(41, 17)$ & 8 & $(2, 1)$ & 1 & 1 & NO & YES & NO(2) & -- & 768\\
$(41, 17)$ & 8 & $(2, 1)$ & 1 & 1 & YES & YES & YES & NO & 769\\
$(41, 15)$ & 8 & $(3, 1)$ & 2 & 1 & YES & YES & YES & -- & 770\\
$(41, 16)$ & 8 & $(3, 1)$ & 2 & 1 & YES & YES & YES & -- & 771\\
$(41, 16)$ & 8 & $(3, 1)$ & 2 & 1 & YES & YES & YES & NO & 772\\
$(41, 17)$ & 8 & $(3, 1)$ & 2 & 1 & YES & YES & YES & -- & 773\\
$(41, 17)$ & 8 & $(3, 1)$ & 2 & 1 & YES & YES & YES & 588 & 774\\
$(41, 17)$ & 8 & $(3, 1)$ & 2 & 1 & YES & YES & YES & NO & 775\\
$(41, 15)$ & 8 & $(4, 1)$ & 3 & 1 & YES & YES & YES & -- & 776\\
$(41, 17)$ & 8 & $(4, 1)$ & 3 & 1 & YES & YES & YES & -- & 777\\
$(41, 17)$ & 8 & $(4, 1)$ & 3 & 1 & YES & YES & YES & NO & 778\\
$(41, 16)$ & 8 & $(5, 1)$ & 4 & 1 & YES & YES & YES & -- & 779\\
$(41, 16)$ & 8 & $(5, 1)$ & 4 & 1 & YES & YES & YES & NO & 780\\
$(41, 16)$ & 8 & $(5, 2)$ & 3 & 1 & YES & YES & YES & 574 & 781\\
$(41, 11)$ & 8 & $(7, 3)$ & 4 & 1 & YES & YES & YES & -- & 782\\
$(41, 12)$ & 8 & $(7, 3)$ & 4 & 1 & YES & YES & YES & -- & 783\\
$(41, 15)$ & 8 & $(7, 2)$ & 4 & 1 & YES & YES & YES & NO & 784\\
$(41, 15)$ & 8 & $(7, 3)$ & 4 & 1 & YES & YES & YES & NO & 785\\
$(41, 16)$ & 8 & $(7, 2)$ & 4 & 1 & YES & YES & YES & -- & 786\\
$(41, 16)$ & 8 & $(7, 3)$ & 4 & 1 & YES & YES & YES & NO & 787\\
$(41, 11)$ & 8 & $(8, 3)$ & 4 & 1 & YES & YES & YES & -- & 788\\
$(41, 11)$ & 8 & $(8, 3)$ & 4 & 1 & YES & YES & YES & NO & 789\\
$(41, 12)$ & 8 & $(8, 3)$ & 4 & 1 & YES & YES & YES & -- & 790\\
$(41, 12)$ & 8 & $(8, 3)$ & 4 & 1 & YES & YES & YES & NO & 791\\
$(41, 15)$ & 8 & $(9, 2)$ & 5 & 1 & YES & YES & YES & -- & 792\\
$(41, 15)$ & 8 & $(9, 2)$ & 5 & 1 & YES & YES & YES & NO & 793\\
$(41, 12)$ & 8 & $(10, 3)$ & 5 & 1 & YES & YES & YES & -- & 794\\
$(41, 11)$ & 8 & $(11, 3)$ & 5 & 1 & YES & YES & YES & -- & 795\\
$(41, 12)$ & 8 & $(11, 3)$ & 5 & 1 & YES & YES & YES & -- & 796\\
$(41, 15)$ & 8 & $(11, 3)$ & 5 & 1 & YES & YES & YES & NO & 797\\
$(41, 17)$ & 8 & $(11, 3)$ & 5 & 1 & YES & YES & YES & NO & 798\\
$(41, 11)$ & 8 & $(13, 4)$ & 6 & 1 & YES & YES & YES & NO & 799\\
$(41, 16)$ & 8 & $(13, 3)$ & 6 & 1 & YES & YES & YES & -- & 800\\
$(41, 16)$ & 8 & $(18, 7)$ & 6 & 1 & YES & YES & YES & NO & 801\\
$(41, 17)$ & 8 & $(19, 8)$ & 6 & 1 & YES & YES & YES & NO & 802\\
$(41, 12)$ & 8 & $(23, 7)$ & 7 & 1 & YES & YES & YES & NO & 803\\
$(41, 11)$ & 8 & $(29, 8)$ & 7 & 1 & YES & YES & YES & NO & 804\\
$(41, 12)$ & 8 & $(37, 11)$ & 8 & 1 & YES & YES & YES & NO & 805\\
$(41, 15)$ & 8 & $(41, 15)$ & 8 & 41 & YES & YES & YES & NO & 806\\
$(41, 17)$ & 8 & $(41, 17)$ & 8 & 41 & YES & YES & YES & NO & 807\\
$(42, 13)$ & 9 & $(2, 1)$ & 1 & 2 & YES & YES & YES & NO & 808\\
$(42, 13)$ & 9 & $(5, 2)$ & 3 & 1 & YES & YES & YES & -- & 809\\
$(42, 19)$ & 9 & $(6, 1)$ & 5 & 6 & YES & YES & YES & -- & 810\\
$(42, 19)$ & 9 & $(6, 1)$ & 5 & 6 & YES & YES & YES & NO & 811\\
$(42, 13)$ & 9 & $(8, 3)$ & 4 & 2 & YES & YES & YES & -- & 812\\
$(42, 13)$ & 9 & $(18, 5)$ & 6 & 6 & YES & YES & YES & NO & 813\\
$(43, 16)$ & 9 & $(4, 1)$ & 3 & 1 & YES & YES & YES & -- & 814\\
$(43, 16)$ & 9 & $(5, 1)$ & 4 & 1 & YES & YES & YES & -- & 815\\
$(43, 18)$ & 8 & $(5, 2)$ & 3 & 1 & YES & YES & YES & -- & 816\\
$(43, 19)$ & 9 & $(5, 1)$ & 4 & 1 & YES & YES & YES & -- & 817\\
$(43, 19)$ & 9 & $(5, 1)$ & 4 & 1 & YES & YES & YES & NO & 818\\
$(43, 16)$ & 9 & $(6, 1)$ & 5 & 1 & YES & YES & YES & -- & 819\\
$(43, 12)$ & 8 & $(7, 3)$ & 4 & 1 & YES & YES & YES & -- & 820\\
$(43, 19)$ & 9 & $(7, 1)$ & 6 & 1 & YES & YES & YES & NO & 821\\
$(43, 19)$ & 9 & $(7, 1)$ & 6 & 1 & YES & YES & YES & NO & 822\\
$(43, 12)$ & 8 & $(8, 3)$ & 4 & 1 & YES & YES & YES & -- & 823\\
$(43, 12)$ & 8 & $(9, 4)$ & 5 & 1 & YES & YES & YES & NO & 824\\
$(43, 19)$ & 9 & $(9, 4)$ & 5 & 1 & YES & YES & YES & NO & 825\\
$(43, 12)$ & 8 & $(11, 4)$ & 5 & 1 & YES & YES & YES & NO & 826\\
$(43, 18)$ & 8 & $(11, 2)$ & 6 & 1 & YES & YES & YES & -- & 827\\
$(43, 16)$ & 9 & $(19, 7)$ & 6 & 1 & YES & YES & YES & NO & 828\\
$(43, 18)$ & 8 & $(26, 11)$ & 7 & 1 & YES & YES & YES & 1176 & 829\\
$(43, 16)$ & 9 & $(27, 10)$ & 7 & 1 & YES & YES & YES & 1051 & 830\\
$(43, 16)$ & 9 & $(35, 13)$ & 8 & 1 & YES & YES & YES & NO & 831\\
$(43, 10)$ & 9 & $(40, 9)$ & 9 & 1 & YES & YES & YES & NO & 832\\
$(43, 12)$ & 8 & $(40, 11)$ & 8 & 1 & YES & YES & YES & NO & 833\\
$(43, 19)$ & 9 & $(43, 19)$ & 9 & 43 & YES & YES & YES & NO & 834\\
$(44, 17)$ & 8 & $(2, 1)$ & 1 & 2 & YES & YES & YES & -- & 835\\
$(44, 17)$ & 8 & $(2, 1)$ & 1 & 2 & YES & YES & YES & 532 & 836\\
$(44, 17)$ & 8 & $(3, 1)$ & 2 & 1 & YES & YES & YES & -- & 837\\
$(44, 17)$ & 8 & $(3, 1)$ & 2 & 1 & YES & YES & YES & NO & 838\\
$(44, 17)$ & 8 & $(5, 2)$ & 3 & 1 & YES & YES & YES & NO & 839\\
$(44, 13)$ & 8 & $(7, 3)$ & 4 & 1 & YES & YES & YES & -- & 840\\
$(44, 13)$ & 8 & $(7, 3)$ & 4 & 1 & YES & YES & YES & NO & 841\\
$(44, 17)$ & 8 & $(7, 2)$ & 4 & 1 & YES & YES & YES & -- & 842\\
$(44, 17)$ & 8 & $(7, 3)$ & 4 & 1 & YES & YES & YES & -- & 843\\
$(44, 17)$ & 8 & $(9, 2)$ & 5 & 1 & YES & YES & YES & -- & 844\\
$(44, 17)$ & 8 & $(9, 2)$ & 5 & 1 & YES & YES & YES & NO & 845\\
$(44, 13)$ & 8 & $(13, 3)$ & 6 & 1 & YES & YES & YES & NO & 846\\
$(44, 13)$ & 8 & $(15, 4)$ & 6 & 1 & YES & YES & YES & NO & 847\\
$(44, 13)$ & 8 & $(18, 5)$ & 6 & 2 & YES & YES & YES & NO & 848\\
$(44, 17)$ & 8 & $(21, 8)$ & 6 & 1 & YES & YES & YES & NO & 849\\
$(44, 13)$ & 8 & $(23, 7)$ & 7 & 1 & YES & YES & YES & NO & 850\\
$(44, 13)$ & 8 & $(31, 9)$ & 8 & 1 & YES & YES & YES & NO & 851\\
$(44, 13)$ & 8 & $(41, 12)$ & 8 & 1 & YES & YES & YES & NO & 852\\
$(45, 14)$ & 9 & $(2, 1)$ & 1 & 1 & YES & YES & YES & NO & 853\\
$(45, 19)$ & 8 & $(2, 1)$ & 1 & 1 & YES & YES & YES & -- & 854\\
$(45, 14)$ & 9 & $(5, 1)$ & 4 & 5 & YES & YES & YES & NO & 855\\
$(45, 19)$ & 8 & $(5, 1)$ & 4 & 5 & YES & YES & NO(2) & NO & 856\\
$(45, 19)$ & 8 & $(5, 2)$ & 3 & 5 & YES & YES & YES & -- & 857\\
$(45, 19)$ & 8 & $(7, 3)$ & 4 & 1 & YES & YES & NO(2) & 672 & 858\\
$(45, 14)$ & 9 & $(29, 9)$ & 8 & 1 & YES & YES & YES & NO & 859\\
$(46, 19)$ & 8 & $(2, 1)$ & 1 & 2 & YES & YES & YES & NO & 860\\
$(46, 19)$ & 8 & $(3, 1)$ & 2 & 1 & YES & YES & YES & -- & 861\\
$(46, 19)$ & 8 & $(3, 1)$ & 2 & 1 & YES & YES & YES & NO & 862\\
$(46, 19)$ & 8 & $(5, 2)$ & 3 & 1 & YES & YES & YES & -- & 863\\
$(46, 19)$ & 8 & $(5, 2)$ & 3 & 1 & YES & YES & YES & NO & 864\\
$(46, 19)$ & 8 & $(7, 2)$ & 4 & 1 & YES & YES & YES & NO & 865\\
$(46, 19)$ & 8 & $(9, 2)$ & 5 & 1 & YES & YES & YES & NO & 866\\
$(46, 19)$ & 8 & $(19, 8)$ & 6 & 1 & YES & YES & YES & NO & 867\\
$(47, 18)$ & 8 & $(3, 1)$ & 2 & 1 & YES & YES & YES & -- & 868\\
$(47, 18)$ & 8 & $(3, 1)$ & 2 & 1 & YES & YES & YES & NO & 869\\
$(47, 18)$ & 8 & $(4, 1)$ & 3 & 1 & YES & YES & YES & -- & 870\\
$(47, 18)$ & 8 & $(4, 1)$ & 3 & 1 & YES & YES & YES & NO & 871\\
$(47, 18)$ & 8 & $(5, 1)$ & 4 & 1 & YES & YES & YES & -- & 872\\
$(47, 18)$ & 8 & $(5, 1)$ & 4 & 1 & YES & YES & YES & NO & 873\\
$(47, 18)$ & 8 & $(5, 2)$ & 3 & 1 & YES & YES & YES & -- & 874\\
$(47, 13)$ & 8 & $(7, 3)$ & 4 & 1 & YES & YES & YES & NO & 875\\
$(47, 18)$ & 8 & $(7, 2)$ & 4 & 1 & YES & YES & YES & -- & 876\\
$(47, 18)$ & 8 & $(7, 3)$ & 4 & 1 & YES & YES & YES & NO & 877\\
$(47, 13)$ & 8 & $(8, 3)$ & 4 & 1 & YES & YES & YES & NO & 878\\
$(47, 18)$ & 8 & $(9, 2)$ & 5 & 1 & YES & YES & YES & -- & 879\\
$(47, 18)$ & 8 & $(9, 2)$ & 5 & 1 & YES & YES & YES & NO & 880\\
$(47, 18)$ & 8 & $(11, 2)$ & 6 & 1 & YES & YES & YES & -- & 881\\
$(47, 18)$ & 8 & $(11, 2)$ & 6 & 1 & YES & YES & YES & NO & 882\\
$(47, 13)$ & 8 & $(13, 4)$ & 6 & 1 & YES & YES & YES & NO & 883\\
$(47, 13)$ & 8 & $(17, 5)$ & 6 & 1 & YES & YES & YES & NO & 884\\
$(47, 18)$ & 8 & $(18, 7)$ & 6 & 1 & YES & YES & YES & NO & 885\\
$(47, 18)$ & 8 & $(21, 8)$ & 6 & 1 & YES & YES & YES & 980 & 886\\
$(47, 18)$ & 8 & $(29, 11)$ & 7 & 1 & YES & YES & YES & 1290 & 887\\
$(47, 18)$ & 8 & $(47, 18)$ & 8 & 47 & YES & YES & YES & NO & 888\\
$(49, 18)$ & 8 & $(2, 1)$ & 1 & 1 & YES & YES & YES & 620 & 889\\
$(49, 19)$ & 8 & $(2, 1)$ & 1 & 1 & YES & YES & NO(2) & -- & 890\\
$(49, 20)$ & 9 & $(2, 1)$ & 1 & 1 & YES & YES & YES & NO & 891\\
$(49, 15)$ & 9 & $(3, 1)$ & 2 & 1 & NO & YES & YES & -- & 892\\
$(49, 18)$ & 8 & $(3, 1)$ & 2 & 1 & YES & YES & YES & -- & 893\\
$(49, 18)$ & 8 & $(3, 1)$ & 2 & 1 & YES & YES & YES & 755 & 894\\
$(49, 19)$ & 8 & $(3, 1)$ & 2 & 1 & YES & YES & YES & -- & 895\\
$(49, 19)$ & 8 & $(3, 1)$ & 2 & 1 & YES & YES & YES & NO & 896\\
$(49, 20)$ & 9 & $(3, 1)$ & 2 & 1 & YES & YES & YES & NO & 897\\
$(49, 9)$ & 10 & $(4, 1)$ & 3 & 1 & YES & YES & YES & -- & 898\\
$(49, 9)$ & 10 & $(4, 1)$ & 3 & 1 & YES & YES & YES & NO & 899\\
$(49, 13)$ & 9 & $(5, 1)$ & 4 & 1 & YES & YES & YES & -- & 900\\
$(49, 13)$ & 9 & $(5, 1)$ & 4 & 1 & YES & YES & YES & NO & 901\\
$(49, 19)$ & 8 & $(5, 2)$ & 3 & 1 & YES & YES & YES & NO & 902\\
$(49, 15)$ & 9 & $(6, 1)$ & 5 & 1 & YES & YES & YES & NO & 903\\
$(49, 20)$ & 9 & $(6, 1)$ & 5 & 1 & YES & YES & YES & NO & 904\\
$(49, 18)$ & 8 & $(7, 3)$ & 4 & 7 & YES & YES & YES & NO & 905\\
$(49, 19)$ & 8 & $(7, 2)$ & 4 & 7 & YES & YES & YES & -- & 906\\
$(49, 18)$ & 8 & $(8, 3)$ & 4 & 1 & YES & YES & YES & NO & 907\\
$(49, 19)$ & 8 & $(8, 3)$ & 4 & 1 & YES & YES & YES & 970 & 908\\
$(49, 9)$ & 10 & $(9, 2)$ & 5 & 1 & YES & YES & YES & 1038 & 909\\
$(49, 15)$ & 9 & $(9, 2)$ & 5 & 1 & YES & YES & YES & NO & 910\\
$(49, 19)$ & 8 & $(9, 2)$ & 5 & 1 & YES & YES & YES & NO & 911\\
$(49, 18)$ & 8 & $(13, 5)$ & 5 & 1 & YES & YES & YES & NO & 912\\
$(49, 13)$ & 9 & $(15, 4)$ & 6 & 1 & YES & YES & YES & NO & 913\\
$(49, 11)$ & 10 & $(17, 3)$ & 7 & 1 & YES & YES & YES & NO & 914\\
$(49, 20)$ & 9 & $(17, 7)$ & 6 & 1 & YES & YES & YES & NO & 915\\
$(49, 13)$ & 9 & $(19, 5)$ & 7 & 1 & YES & YES & YES & 952 & 916\\
$(49, 9)$ & 10 & $(23, 4)$ & 8 & 1 & YES & YES & YES & NO & 917\\
$(49, 19)$ & 8 & $(28, 11)$ & 8 & 7 & YES & YES & YES & NO & 918\\
$(49, 15)$ & 9 & $(33, 10)$ & 8 & 1 & YES & YES & YES & 1317 & 919\\
$(49, 15)$ & 9 & $(36, 11)$ & 8 & 1 & YES & YES & YES & NO & 920\\
$(49, 19)$ & 8 & $(44, 17)$ & 8 & 1 & YES & YES & YES & NO & 921\\
$(50, 21)$ & 8 & $(2, 1)$ & 1 & 2 & NO & YES & YES & -- & 922\\
$(50, 19)$ & 8 & $(3, 1)$ & 2 & 1 & YES & YES & YES & -- & 923\\
$(50, 19)$ & 8 & $(5, 2)$ & 3 & 5 & YES & YES & YES & -- & 924\\
$(50, 19)$ & 8 & $(5, 2)$ & 3 & 5 & YES & YES & YES & NO & 925\\
$(50, 21)$ & 8 & $(5, 2)$ & 3 & 5 & YES & YES & YES & -- & 926\\
$(50, 19)$ & 8 & $(7, 2)$ & 4 & 1 & YES & YES & YES & -- & 927\\
$(50, 19)$ & 8 & $(7, 3)$ & 4 & 1 & YES & YES & YES & NO & 928\\
$(50, 19)$ & 8 & $(9, 4)$ & 5 & 1 & YES & YES & YES & NO & 929\\
$(50, 19)$ & 8 & $(13, 5)$ & 5 & 1 & YES & YES & YES & NO & 930\\
$(50, 21)$ & 8 & $(26, 11)$ & 7 & 2 & YES & YES & YES & NO & 931\\
$(50, 19)$ & 8 & $(34, 13)$ & 7 & 2 & YES & YES & YES & NO & 932\\
$(51, 14)$ & 9 & $(2, 1)$ & 1 & 1 & YES & YES & YES & -- & 933\\
$(51, 20)$ & 9 & $(2, 1)$ & 1 & 1 & YES & YES & YES & -- & 934\\
$(51, 20)$ & 9 & $(2, 1)$ & 1 & 1 & YES & YES & YES & NO & 935\\
$(51, 20)$ & 9 & $(41, 16)$ & 8 & 1 & YES & YES & YES & NO & 936\\
$(53, 14)$ & 9 & $(2, 1)$ & 1 & 1 & YES & YES & YES & -- & 937\\
$(53, 14)$ & 9 & $(2, 1)$ & 1 & 1 & YES & YES & YES & NO & 938\\
$(53, 19)$ & 9 & $(2, 1)$ & 1 & 1 & YES & YES & YES & NO & 939\\
$(53, 23)$ & 9 & $(2, 1)$ & 1 & 1 & NO & YES & YES & -- & 940\\
$(53, 14)$ & 9 & $(3, 1)$ & 2 & 1 & YES & YES & YES & NO & 941\\
$(53, 22)$ & 9 & $(4, 1)$ & 3 & 1 & YES & YES & YES & -- & 942\\
$(53, 22)$ & 9 & $(4, 1)$ & 3 & 1 & YES & YES & YES & NO & 943\\
$(53, 14)$ & 9 & $(5, 1)$ & 4 & 1 & YES & YES & YES & -- & 944\\
$(53, 14)$ & 9 & $(5, 1)$ & 4 & 1 & YES & YES & YES & NO & 945\\
$(53, 19)$ & 9 & $(5, 1)$ & 4 & 1 & YES & YES & YES & -- & 946\\
$(53, 19)$ & 9 & $(5, 1)$ & 4 & 1 & YES & YES & YES & NO & 947\\
$(53, 19)$ & 9 & $(5, 2)$ & 3 & 1 & YES & YES & YES & -- & 948\\
$(53, 19)$ & 9 & $(6, 1)$ & 5 & 1 & YES & YES & YES & NO & 949\\
$(53, 22)$ & 9 & $(7, 3)$ & 4 & 1 & YES & YES & YES & NO & 950\\
$(53, 19)$ & 9 & $(14, 5)$ & 6 & 1 & YES & YES & YES & NO & 951\\
$(53, 14)$ & 9 & $(15, 4)$ & 6 & 1 & YES & YES & YES & 916 & 952\\
$(53, 14)$ & 9 & $(19, 5)$ & 7 & 1 & YES & YES & YES & NO & 953\\
$(53, 19)$ & 9 & $(19, 7)$ & 6 & 1 & YES & YES & YES & 1410 & 954\\
$(53, 19)$ & 9 & $(36, 13)$ & 8 & 1 & YES & YES & YES & 1372 & 955\\
$(55, 16)$ & 9 & $(2, 1)$ & 1 & 1 & YES & YES & YES & -- & 956\\
$(55, 21)$ & 8 & $(2, 1)$ & 1 & 1 & YES & YES & YES & -- & 957\\
$(55, 23)$ & 9 & $(2, 1)$ & 1 & 1 & YES & YES & YES & -- & 958\\
$(55, 24)$ & 9 & $(2, 1)$ & 1 & 1 & NO & YES & YES & -- & 959\\
$(55, 16)$ & 9 & $(3, 1)$ & 2 & 1 & NO & YES & YES & -- & 960\\
$(55, 21)$ & 8 & $(3, 1)$ & 2 & 1 & YES & YES & YES & -- & 961\\
$(55, 23)$ & 9 & $(3, 1)$ & 2 & 1 & YES & YES & YES & -- & 962\\
$(55, 23)$ & 9 & $(3, 1)$ & 2 & 1 & YES & YES & YES & NO & 963\\
$(55, 16)$ & 9 & $(4, 1)$ & 3 & 1 & YES & YES & YES & NO & 964\\
$(55, 23)$ & 9 & $(4, 1)$ & 3 & 1 & YES & YES & YES & NO & 965\\
$(55, 16)$ & 9 & $(5, 2)$ & 3 & 5 & YES & YES & YES & NO & 966\\
$(55, 21)$ & 8 & $(5, 1)$ & 4 & 5 & YES & YES & YES & -- & 967\\
$(55, 21)$ & 8 & $(5, 1)$ & 4 & 5 & YES & YES & YES & NO & 968\\
$(55, 21)$ & 8 & $(5, 2)$ & 3 & 5 & YES & YES & YES & -- & 969\\
$(55, 21)$ & 8 & $(5, 2)$ & 3 & 5 & YES & YES & YES & 908 & 970\\
$(55, 23)$ & 9 & $(5, 1)$ & 4 & 5 & YES & YES & YES & -- & 971\\
$(55, 23)$ & 9 & $(6, 1)$ & 5 & 1 & YES & YES & YES & -- & 972\\
$(55, 23)$ & 9 & $(6, 1)$ & 5 & 1 & YES & YES & YES & NO & 973\\
$(55, 23)$ & 9 & $(6, 1)$ & 5 & 1 & YES & YES & YES & NO & 974\\
$(55, 16)$ & 9 & $(7, 2)$ & 4 & 1 & YES & YES & YES & 746 & 975\\
$(55, 23)$ & 9 & $(7, 3)$ & 4 & 1 & YES & YES & YES & NO & 976\\
$(55, 21)$ & 8 & $(8, 3)$ & 4 & 1 & YES & YES & YES & NO & 977\\
$(55, 21)$ & 8 & $(9, 2)$ & 5 & 1 & YES & YES & YES & -- & 978\\
$(55, 21)$ & 8 & $(9, 2)$ & 5 & 1 & YES & YES & YES & NO & 979\\
$(55, 21)$ & 8 & $(13, 5)$ & 5 & 1 & YES & YES & YES & 886 & 980\\
$(55, 21)$ & 8 & $(18, 7)$ & 6 & 1 & YES & YES & YES & NO & 981\\
$(55, 23)$ & 9 & $(19, 8)$ & 6 & 1 & YES & YES & YES & NO & 982\\
$(55, 21)$ & 8 & $(21, 8)$ & 6 & 1 & YES & YES & YES & NO & 983\\
$(55, 13)$ & 10 & $(23, 5)$ & 7 & 1 & YES & YES & YES & NO & 984\\
$(55, 21)$ & 8 & $(29, 11)$ & 7 & 1 & YES & YES & YES & NO & 985\\
$(55, 23)$ & 9 & $(31, 13)$ & 7 & 1 & YES & YES & YES & 1177 & 986\\
$(55, 23)$ & 9 & $(43, 18)$ & 8 & 1 & YES & YES & YES & NO & 987\\
$(55, 21)$ & 8 & $(47, 18)$ & 8 & 1 & YES & YES & YES & NO & 988\\
$(55, 23)$ & 9 & $(55, 23)$ & 9 & 55 & YES & YES & YES & NO & 989\\
$(57, 22)$ & 9 & $(3, 1)$ & 2 & 3 & YES & YES & YES & -- & 990\\
$(57, 22)$ & 9 & $(3, 1)$ & 2 & 3 & YES & YES & YES & NO & 991\\
$(57, 22)$ & 9 & $(4, 1)$ & 3 & 1 & YES & YES & YES & NO & 992\\
$(57, 22)$ & 9 & $(18, 7)$ & 6 & 3 & YES & YES & YES & NO & 993\\
$(57, 22)$ & 9 & $(31, 12)$ & 7 & 1 & YES & YES & YES & 1194 & 994\\
$(58, 17)$ & 9 & $(2, 1)$ & 1 & 2 & YES & YES & YES & -- & 995\\
$(58, 17)$ & 9 & $(3, 1)$ & 2 & 1 & YES & YES & YES & -- & 996\\
$(58, 17)$ & 9 & $(3, 1)$ & 2 & 1 & YES & YES & YES & NO & 997\\
$(58, 17)$ & 9 & $(4, 1)$ & 3 & 2 & YES & YES & YES & -- & 998\\
$(58, 17)$ & 9 & $(4, 1)$ & 3 & 2 & YES & YES & YES & 762 & 999\\
$(58, 17)$ & 9 & $(5, 2)$ & 3 & 1 & YES & YES & YES & -- & 1000\\
$(58, 17)$ & 9 & $(7, 2)$ & 4 & 1 & YES & YES & YES & -- & 1001\\
$(58, 17)$ & 9 & $(9, 2)$ & 5 & 1 & YES & YES & YES & NO & 1002\\
$(58, 17)$ & 9 & $(10, 3)$ & 5 & 2 & YES & YES & YES & NO & 1003\\
$(58, 17)$ & 9 & $(17, 5)$ & 6 & 1 & YES & YES & YES & NO & 1004\\
$(58, 17)$ & 9 & $(18, 5)$ & 6 & 2 & YES & YES & YES & NO & 1005\\
$(58, 17)$ & 9 & $(58, 17)$ & 9 & 58 & YES & YES & YES & NO & 1006\\
$(59, 23)$ & 9 & $(2, 1)$ & 1 & 1 & YES & YES & YES & NO & 1007\\
$(59, 25)$ & 9 & $(2, 1)$ & 1 & 1 & NO & YES & YES & -- & 1008\\
$(59, 25)$ & 9 & $(3, 1)$ & 2 & 1 & YES & YES & YES & -- & 1009\\
$(59, 23)$ & 9 & $(4, 1)$ & 3 & 1 & YES & YES & YES & -- & 1010\\
$(59, 23)$ & 9 & $(4, 1)$ & 3 & 1 & YES & YES & YES & NO & 1011\\
$(59, 25)$ & 9 & $(4, 1)$ & 3 & 1 & YES & YES & YES & NO & 1012\\
$(59, 23)$ & 9 & $(5, 2)$ & 3 & 1 & YES & YES & YES & NO & 1013\\
$(59, 25)$ & 9 & $(12, 5)$ & 5 & 1 & YES & YES & YES & 674 & 1014\\
$(59, 23)$ & 9 & $(13, 5)$ & 5 & 1 & YES & YES & YES & NO & 1015\\
$(59, 23)$ & 9 & $(28, 11)$ & 8 & 1 & YES & YES & YES & 1407 & 1016\\
$(59, 25)$ & 9 & $(33, 14)$ & 8 & 1 & YES & YES & YES & NO & 1017\\
$(59, 25)$ & 9 & $(59, 25)$ & 9 & 59 & YES & YES & YES & NO & 1018\\
$(60, 23)$ & 9 & $(2, 1)$ & 1 & 2 & YES & YES & YES & NO & 1019\\
$(60, 23)$ & 9 & $(6, 1)$ & 5 & 6 & YES & YES & YES & -- & 1020\\
$(60, 23)$ & 9 & $(6, 1)$ & 5 & 6 & YES & YES & YES & NO & 1021\\
$(60, 23)$ & 9 & $(6, 1)$ & 5 & 6 & YES & YES & YES & NO & 1022\\
$(60, 23)$ & 9 & $(11, 4)$ & 5 & 1 & YES & YES & YES & NO & 1023\\
$(60, 23)$ & 9 & $(21, 8)$ & 6 & 3 & YES & YES & YES & NO & 1024\\
$(60, 23)$ & 9 & $(34, 13)$ & 7 & 2 & YES & YES & YES & 1291 & 1025\\
$(60, 23)$ & 9 & $(60, 23)$ & 9 & 60 & YES & YES & YES & NO & 1026\\
$(61, 18)$ & 9 & $(2, 1)$ & 1 & 1 & YES & YES & YES & NO & 1027\\
$(61, 17)$ & 9 & $(3, 1)$ & 2 & 1 & YES & YES & YES & -- & 1028\\
$(61, 17)$ & 9 & $(3, 1)$ & 2 & 1 & YES & YES & YES & 732 & 1029\\
$(61, 25)$ & 9 & $(3, 1)$ & 2 & 1 & YES & YES & YES & -- & 1030\\
$(61, 18)$ & 9 & $(4, 1)$ & 3 & 1 & YES & YES & YES & -- & 1031\\
$(61, 18)$ & 9 & $(4, 1)$ & 3 & 1 & YES & YES & YES & NO & 1032\\
$(61, 25)$ & 9 & $(4, 1)$ & 3 & 1 & YES & YES & YES & NO & 1033\\
$(61, 13)$ & 10 & $(5, 1)$ & 4 & 1 & YES & YES & YES & NO & 1034\\
$(61, 17)$ & 9 & $(5, 2)$ & 3 & 1 & YES & YES & YES & -- & 1035\\
$(61, 17)$ & 9 & $(5, 2)$ & 3 & 1 & YES & YES & YES & NO & 1036\\
$(61, 18)$ & 9 & $(5, 2)$ & 3 & 1 & YES & YES & YES & -- & 1037\\
$(61, 13)$ & 10 & $(6, 1)$ & 5 & 1 & YES & YES & YES & 909 & 1038\\
$(61, 18)$ & 9 & $(7, 2)$ & 4 & 1 & YES & YES & YES & -- & 1039\\
$(61, 25)$ & 9 & $(7, 3)$ & 4 & 1 & YES & YES & YES & NO & 1040\\
$(61, 18)$ & 9 & $(9, 2)$ & 5 & 1 & YES & YES & YES & NO & 1041\\
$(61, 18)$ & 9 & $(10, 3)$ & 5 & 1 & YES & YES & YES & NO & 1042\\
$(61, 17)$ & 9 & $(11, 3)$ & 5 & 1 & YES & YES & YES & NO & 1043\\
$(61, 25)$ & 9 & $(12, 5)$ & 5 & 1 & YES & YES & YES & 1191 & 1044\\
$(61, 13)$ & 10 & $(19, 4)$ & 7 & 1 & YES & YES & YES & NO & 1045\\
$(61, 25)$ & 9 & $(39, 16)$ & 8 & 1 & YES & YES & YES & NO & 1046\\
$(61, 25)$ & 9 & $(61, 25)$ & 9 & 61 & YES & YES & YES & NO & 1047\\
$(62, 23)$ & 9 & $(3, 1)$ & 2 & 1 & YES & YES & YES & NO & 1048\\
$(62, 23)$ & 9 & $(5, 1)$ & 4 & 1 & YES & YES & YES & -- & 1049\\
$(62, 19)$ & 10 & $(7, 2)$ & 4 & 1 & YES & YES & YES & NO & 1050\\
$(62, 23)$ & 9 & $(8, 3)$ & 4 & 2 & YES & YES & YES & 830 & 1051\\
$(63, 26)$ & 9 & $(2, 1)$ & 1 & 1 & YES & YES & YES & -- & 1052\\
$(63, 26)$ & 9 & $(3, 1)$ & 2 & 3 & YES & YES & YES & -- & 1053\\
$(63, 26)$ & 9 & $(3, 1)$ & 2 & 3 & YES & YES & YES & NO & 1054\\
$(63, 26)$ & 9 & $(3, 1)$ & 2 & 3 & YES & YES & YES & NO & 1055\\
$(63, 26)$ & 9 & $(4, 1)$ & 3 & 1 & YES & YES & YES & -- & 1056\\
$(63, 26)$ & 9 & $(4, 1)$ & 3 & 1 & YES & YES & YES & NO & 1057\\
$(63, 26)$ & 9 & $(5, 2)$ & 3 & 1 & YES & YES & YES & NO & 1058\\
$(63, 26)$ & 9 & $(12, 5)$ & 5 & 3 & YES & YES & YES & NO & 1059\\
$(63, 26)$ & 9 & $(22, 9)$ & 7 & 1 & YES & YES & YES & NO & 1060\\
$(63, 26)$ & 9 & $(29, 12)$ & 7 & 1 & YES & YES & YES & 1195 & 1061\\
$(63, 26)$ & 9 & $(46, 19)$ & 8 & 1 & YES & YES & YES & NO & 1062\\
$(64, 25)$ & 9 & $(2, 1)$ & 1 & 2 & NO & YES & YES & -- & 1063\\
$(64, 27)$ & 9 & $(2, 1)$ & 1 & 2 & NO & YES & YES & -- & 1064\\
$(64, 25)$ & 9 & $(3, 1)$ & 2 & 1 & YES & YES & YES & -- & 1065\\
$(64, 25)$ & 9 & $(3, 1)$ & 2 & 1 & YES & YES & YES & NO & 1066\\
$(64, 25)$ & 9 & $(4, 1)$ & 3 & 4 & YES & YES & YES & NO & 1067\\
$(64, 27)$ & 9 & $(5, 2)$ & 3 & 1 & YES & YES & YES & -- & 1068\\
$(64, 27)$ & 9 & $(8, 3)$ & 4 & 8 & YES & YES & YES & NO & 1069\\
$(64, 27)$ & 9 & $(12, 5)$ & 5 & 4 & YES & YES & YES & NO & 1070\\
$(64, 25)$ & 9 & $(13, 5)$ & 5 & 1 & YES & YES & YES & 1273 & 1071\\
$(64, 19)$ & 9 & $(24, 7)$ & 7 & 8 & YES & YES & YES & NO & 1072\\
$(64, 27)$ & 9 & $(26, 11)$ & 7 & 2 & YES & YES & YES & 1158 & 1073\\
$(64, 25)$ & 9 & $(28, 11)$ & 8 & 4 & YES & YES & YES & NO & 1074\\
$(64, 25)$ & 9 & $(41, 16)$ & 8 & 1 & YES & YES & YES & NO & 1075\\
$(64, 27)$ & 9 & $(45, 19)$ & 8 & 1 & YES & YES & YES & NO & 1076\\
$(64, 25)$ & 9 & $(64, 25)$ & 9 & 64 & YES & YES & YES & NO & 1077\\
$(65, 24)$ & 9 & $(2, 1)$ & 1 & 1 & YES & YES & YES & NO & 1078\\
$(65, 27)$ & 10 & $(2, 1)$ & 1 & 1 & YES & YES & YES & -- & 1079\\
$(65, 18)$ & 9 & $(3, 1)$ & 2 & 1 & YES & YES & YES & -- & 1080\\
$(65, 18)$ & 9 & $(3, 1)$ & 2 & 1 & YES & YES & YES & NO & 1081\\
$(65, 24)$ & 9 & $(3, 1)$ & 2 & 1 & YES & YES & YES & -- & 1082\\
$(65, 24)$ & 9 & $(3, 1)$ & 2 & 1 & YES & YES & YES & NO & 1083\\
$(65, 18)$ & 9 & $(5, 2)$ & 3 & 5 & YES & YES & YES & -- & 1084\\
$(65, 18)$ & 9 & $(5, 2)$ & 3 & 5 & YES & YES & YES & NO & 1085\\
$(65, 19)$ & 9 & $(5, 2)$ & 3 & 5 & YES & YES & YES & -- & 1086\\
$(65, 18)$ & 9 & $(7, 2)$ & 4 & 1 & YES & YES & YES & NO & 1087\\
$(65, 19)$ & 9 & $(7, 2)$ & 4 & 1 & YES & YES & YES & -- & 1088\\
$(65, 19)$ & 9 & $(9, 2)$ & 5 & 1 & YES & YES & YES & NO & 1089\\
$(65, 18)$ & 9 & $(10, 3)$ & 5 & 5 & YES & YES & YES & NO & 1090\\
$(65, 19)$ & 9 & $(27, 8)$ & 7 & 1 & YES & YES & YES & NO & 1091\\
$(66, 25)$ & 9 & $(2, 1)$ & 1 & 2 & NO & YES & YES & -- & 1092\\
$(66, 25)$ & 9 & $(4, 1)$ & 3 & 2 & YES & YES & YES & -- & 1093\\
$(66, 25)$ & 9 & $(4, 1)$ & 3 & 2 & YES & YES & YES & NO & 1094\\
$(66, 29)$ & 9 & $(5, 2)$ & 3 & 1 & YES & YES & YES & -- & 1095\\
$(66, 29)$ & 9 & $(23, 10)$ & 7 & 1 & YES & YES & YES & NO & 1096\\
$(66, 25)$ & 9 & $(37, 14)$ & 8 & 1 & YES & YES & YES & NO & 1097\\
$(67, 26)$ & 9 & $(2, 1)$ & 1 & 1 & YES & YES & YES & -- & 1098\\
$(67, 26)$ & 9 & $(3, 1)$ & 2 & 1 & YES & YES & YES & -- & 1099\\
$(67, 26)$ & 9 & $(3, 1)$ & 2 & 1 & YES & YES & YES & NO & 1100\\
$(67, 26)$ & 9 & $(3, 1)$ & 2 & 1 & YES & YES & YES & NO & 1101\\
$(67, 26)$ & 9 & $(4, 1)$ & 3 & 1 & YES & YES & YES & -- & 1102\\
$(67, 26)$ & 9 & $(4, 1)$ & 3 & 1 & YES & YES & YES & NO & 1103\\
$(67, 26)$ & 9 & $(4, 1)$ & 3 & 1 & YES & YES & YES & NO & 1104\\
$(67, 26)$ & 9 & $(8, 3)$ & 4 & 1 & YES & YES & YES & NO & 1105\\
$(68, 25)$ & 9 & $(3, 1)$ & 2 & 1 & YES & YES & YES & -- & 1106\\
$(68, 25)$ & 9 & $(3, 1)$ & 2 & 1 & YES & YES & YES & NO & 1107\\
$(68, 19)$ & 9 & $(5, 2)$ & 3 & 1 & YES & YES & YES & -- & 1108\\
$(68, 19)$ & 9 & $(5, 2)$ & 3 & 1 & YES & YES & YES & NO & 1109\\
$(68, 25)$ & 9 & $(5, 1)$ & 4 & 1 & YES & YES & YES & -- & 1110\\
$(68, 25)$ & 9 & $(5, 2)$ & 3 & 1 & YES & YES & YES & NO & 1111\\
$(68, 25)$ & 9 & $(19, 7)$ & 6 & 1 & YES & YES & YES & NO & 1112\\
$(68, 25)$ & 9 & $(30, 11)$ & 7 & 2 & YES & YES & YES & 1260 & 1113\\
$(68, 25)$ & 9 & $(68, 25)$ & 9 & 68 & YES & YES & YES & NO & 1114\\
$(69, 29)$ & 9 & $(2, 1)$ & 1 & 1 & YES & YES & YES & -- & 1115\\
$(69, 29)$ & 9 & $(3, 1)$ & 2 & 3 & YES & YES & YES & -- & 1116\\
$(69, 29)$ & 9 & $(3, 1)$ & 2 & 3 & YES & YES & YES & NO & 1117\\
$(69, 29)$ & 9 & $(3, 1)$ & 2 & 3 & YES & YES & YES & NO & 1118\\
$(69, 29)$ & 9 & $(5, 1)$ & 4 & 1 & YES & YES & YES & -- & 1119\\
$(69, 29)$ & 9 & $(6, 1)$ & 5 & 3 & YES & YES & YES & -- & 1120\\
$(69, 29)$ & 9 & $(6, 1)$ & 5 & 3 & YES & YES & YES & NO & 1121\\
$(69, 29)$ & 9 & $(7, 3)$ & 4 & 1 & YES & YES & YES & NO & 1122\\
$(69, 29)$ & 9 & $(12, 5)$ & 5 & 3 & YES & YES & YES & NO & 1123\\
$(69, 29)$ & 9 & $(19, 8)$ & 6 & 1 & YES & YES & YES & NO & 1124\\
$(69, 29)$ & 9 & $(31, 13)$ & 7 & 1 & YES & YES & YES & 1292 & 1125\\
$(69, 29)$ & 9 & $(69, 29)$ & 9 & 69 & YES & YES & YES & NO & 1126\\
$(70, 27)$ & 10 & $(2, 1)$ & 1 & 2 & YES & YES & YES & -- & 1127\\
$(70, 27)$ & 10 & $(2, 1)$ & 1 & 2 & YES & YES & YES & NO & 1128\\
$(70, 29)$ & 9 & $(3, 1)$ & 2 & 1 & YES & YES & YES & -- & 1129\\
$(70, 27)$ & 10 & $(5, 1)$ & 4 & 5 & YES & YES & YES & -- & 1130\\
$(70, 27)$ & 10 & $(6, 1)$ & 5 & 2 & YES & YES & YES & NO & 1131\\
$(70, 29)$ & 9 & $(7, 3)$ & 4 & 7 & YES & YES & YES & NO & 1132\\
$(70, 29)$ & 9 & $(17, 7)$ & 6 & 1 & YES & YES & YES & NO & 1133\\
$(70, 27)$ & 10 & $(44, 17)$ & 8 & 2 & YES & YES & YES & 1414 & 1134\\
$(70, 27)$ & 10 & $(57, 22)$ & 9 & 1 & YES & YES & YES & NO & 1135\\
$(71, 27)$ & 9 & $(2, 1)$ & 1 & 1 & YES & YES & YES & -- & 1136\\
$(71, 30)$ & 9 & $(2, 1)$ & 1 & 1 & YES & YES & YES & -- & 1137\\
$(71, 11)$ & 12 & $(3, 1)$ & 2 & 1 & YES & YES & YES & NO & 1138\\
$(71, 26)$ & 9 & $(3, 1)$ & 2 & 1 & YES & YES & YES & -- & 1139\\
$(71, 27)$ & 9 & $(3, 1)$ & 2 & 1 & YES & YES & YES & -- & 1140\\
$(71, 27)$ & 9 & $(3, 1)$ & 2 & 1 & YES & YES & YES & NO & 1141\\
$(71, 30)$ & 9 & $(3, 1)$ & 2 & 1 & YES & YES & YES & -- & 1142\\
$(71, 26)$ & 9 & $(4, 1)$ & 3 & 1 & YES & YES & YES & -- & 1143\\
$(71, 26)$ & 9 & $(4, 1)$ & 3 & 1 & YES & YES & YES & NO & 1144\\
$(71, 27)$ & 9 & $(4, 1)$ & 3 & 1 & YES & YES & YES & -- & 1145\\
$(71, 27)$ & 9 & $(4, 1)$ & 3 & 1 & YES & YES & YES & NO & 1146\\
$(71, 27)$ & 9 & $(4, 1)$ & 3 & 1 & YES & YES & YES & NO & 1147\\
$(71, 11)$ & 12 & $(5, 1)$ & 4 & 1 & YES & YES & YES & NO & 1148\\
$(71, 26)$ & 9 & $(5, 2)$ & 3 & 1 & YES & YES & YES & NO & 1149\\
$(71, 27)$ & 9 & $(5, 1)$ & 4 & 1 & YES & YES & YES & -- & 1150\\
$(71, 26)$ & 9 & $(7, 2)$ & 4 & 1 & YES & YES & YES & NO & 1151\\
$(71, 27)$ & 9 & $(7, 3)$ & 4 & 1 & YES & YES & YES & NO & 1152\\
$(71, 21)$ & 9 & $(9, 2)$ & 5 & 1 & YES & YES & YES & NO & 1153\\
$(71, 21)$ & 9 & $(11, 3)$ & 5 & 1 & YES & YES & YES & NO & 1154\\
$(71, 30)$ & 9 & $(12, 5)$ & 5 & 1 & YES & YES & YES & 1289 & 1155\\
$(71, 27)$ & 9 & $(13, 5)$ & 5 & 1 & YES & YES & YES & NO & 1156\\
$(71, 15)$ & 10 & $(19, 4)$ & 7 & 1 & YES & YES & YES & NO & 1157\\
$(71, 30)$ & 9 & $(19, 8)$ & 6 & 1 & YES & YES & YES & 1073 & 1158\\
$(71, 21)$ & 9 & $(24, 7)$ & 7 & 1 & YES & YES & YES & NO & 1159\\
$(71, 26)$ & 9 & $(27, 10)$ & 7 & 1 & YES & YES & YES & NO & 1160\\
$(71, 27)$ & 9 & $(29, 11)$ & 7 & 1 & YES & YES & YES & 1261 & 1161\\
$(71, 26)$ & 9 & $(41, 15)$ & 8 & 1 & YES & YES & YES & NO & 1162\\
$(71, 27)$ & 9 & $(50, 19)$ & 8 & 1 & YES & YES & YES & NO & 1163\\
$(71, 21)$ & 9 & $(61, 18)$ & 9 & 1 & YES & YES & YES & NO & 1164\\
$(71, 27)$ & 9 & $(71, 27)$ & 9 & 71 & YES & YES & YES & NO & 1165\\
$(74, 29)$ & 10 & $(2, 1)$ & 1 & 2 & NO & YES & YES & -- & 1166\\
$(74, 31)$ & 9 & $(2, 1)$ & 1 & 2 & YES & YES & YES & -- & 1167\\
$(74, 31)$ & 9 & $(3, 1)$ & 2 & 1 & YES & YES & YES & -- & 1168\\
$(74, 31)$ & 9 & $(3, 1)$ & 2 & 1 & YES & YES & YES & NO & 1169\\
$(74, 29)$ & 10 & $(4, 1)$ & 3 & 2 & YES & YES & YES & -- & 1170\\
$(74, 29)$ & 10 & $(4, 1)$ & 3 & 2 & YES & YES & YES & NO & 1171\\
$(74, 31)$ & 9 & $(4, 1)$ & 3 & 2 & YES & YES & YES & NO & 1172\\
$(74, 17)$ & 11 & $(5, 2)$ & 3 & 1 & YES & YES & YES & -- & 1173\\
$(74, 31)$ & 9 & $(5, 1)$ & 4 & 1 & YES & YES & YES & -- & 1174\\
$(74, 31)$ & 9 & $(5, 2)$ & 3 & 1 & YES & YES & YES & NO & 1175\\
$(74, 31)$ & 9 & $(7, 3)$ & 4 & 1 & YES & YES & YES & 829 & 1176\\
$(74, 31)$ & 9 & $(12, 5)$ & 5 & 2 & YES & YES & YES & 986 & 1177\\
$(74, 29)$ & 10 & $(13, 5)$ & 5 & 1 & YES & YES & YES & NO & 1178\\
$(74, 17)$ & 11 & $(14, 3)$ & 6 & 2 & YES & YES & YES & NO & 1179\\
$(74, 31)$ & 9 & $(19, 8)$ & 6 & 1 & YES & YES & YES & NO & 1180\\
$(74, 31)$ & 9 & $(31, 13)$ & 7 & 1 & YES & YES & YES & NO & 1181\\
$(74, 29)$ & 10 & $(51, 20)$ & 9 & 1 & YES & YES & YES & NO & 1182\\
$(75, 22)$ & 10 & $(2, 1)$ & 1 & 1 & YES & YES & YES & NO & 1183\\
$(75, 29)$ & 9 & $(2, 1)$ & 1 & 1 & YES & YES & YES & -- & 1184\\
$(75, 31)$ & 9 & $(2, 1)$ & 1 & 1 & YES & YES & YES & -- & 1185\\
$(75, 29)$ & 9 & $(3, 1)$ & 2 & 3 & YES & YES & YES & -- & 1186\\
$(75, 31)$ & 9 & $(3, 1)$ & 2 & 3 & YES & YES & YES & -- & 1187\\
$(75, 31)$ & 9 & $(3, 1)$ & 2 & 3 & YES & YES & YES & NO & 1188\\
$(75, 29)$ & 9 & $(4, 1)$ & 3 & 1 & YES & YES & YES & -- & 1189\\
$(75, 29)$ & 9 & $(4, 1)$ & 3 & 1 & YES & YES & YES & NO & 1190\\
$(75, 31)$ & 9 & $(5, 2)$ & 3 & 5 & YES & YES & YES & 1044 & 1191\\
$(75, 29)$ & 9 & $(8, 3)$ & 4 & 1 & YES & YES & YES & NO & 1192\\
$(75, 22)$ & 10 & $(10, 3)$ & 5 & 5 & YES & YES & YES & NO & 1193\\
$(75, 29)$ & 9 & $(13, 5)$ & 5 & 1 & YES & YES & YES & 994 & 1194\\
$(75, 31)$ & 9 & $(17, 7)$ & 6 & 1 & YES & YES & YES & 1061 & 1195\\
$(75, 29)$ & 9 & $(18, 7)$ & 6 & 3 & YES & YES & YES & NO & 1196\\
$(75, 17)$ & 10 & $(23, 5)$ & 7 & 1 & YES & YES & YES & NO & 1197\\
$(75, 29)$ & 9 & $(44, 17)$ & 8 & 1 & YES & YES & YES & NO & 1198\\
$(75, 31)$ & 9 & $(46, 19)$ & 8 & 1 & YES & YES & YES & NO & 1199\\
$(75, 29)$ & 9 & $(75, 29)$ & 9 & 75 & YES & YES & YES & NO & 1200\\
$(76, 29)$ & 9 & $(2, 1)$ & 1 & 2 & YES & YES & YES & -- & 1201\\
$(76, 29)$ & 9 & $(2, 1)$ & 1 & 2 & YES & YES & YES & NO & 1202\\
$(76, 23)$ & 10 & $(3, 1)$ & 2 & 1 & YES & YES & YES & -- & 1203\\
$(76, 29)$ & 9 & $(3, 1)$ & 2 & 1 & YES & YES & YES & -- & 1204\\
$(76, 29)$ & 9 & $(3, 1)$ & 2 & 1 & YES & YES & YES & NO & 1205\\
$(76, 29)$ & 9 & $(3, 1)$ & 2 & 1 & YES & YES & YES & NO & 1206\\
$(76, 29)$ & 9 & $(4, 1)$ & 3 & 4 & YES & YES & YES & NO & 1207\\
$(76, 21)$ & 9 & $(5, 2)$ & 3 & 1 & YES & YES & YES & -- & 1208\\
$(76, 29)$ & 9 & $(5, 2)$ & 3 & 1 & YES & YES & YES & NO & 1209\\
$(76, 29)$ & 9 & $(6, 1)$ & 5 & 2 & YES & YES & YES & -- & 1210\\
$(76, 29)$ & 9 & $(6, 1)$ & 5 & 2 & YES & YES & YES & NO & 1211\\
$(76, 29)$ & 9 & $(6, 1)$ & 5 & 2 & YES & YES & YES & NO & 1212\\
$(76, 23)$ & 10 & $(7, 2)$ & 4 & 1 & YES & YES & YES & NO & 1213\\
$(76, 29)$ & 9 & $(8, 3)$ & 4 & 4 & YES & YES & YES & NO & 1214\\
$(76, 29)$ & 9 & $(13, 5)$ & 5 & 1 & YES & YES & YES & NO & 1215\\
$(76, 29)$ & 9 & $(21, 8)$ & 6 & 1 & YES & YES & YES & NO & 1216\\
$(76, 29)$ & 9 & $(34, 13)$ & 7 & 2 & YES & YES & YES & 1353 & 1217\\
$(76, 29)$ & 9 & $(55, 21)$ & 8 & 1 & YES & YES & YES & NO & 1218\\
$(76, 29)$ & 9 & $(76, 29)$ & 9 & 76 & YES & YES & YES & NO & 1219\\
$(78, 23)$ & 10 & $(2, 1)$ & 1 & 2 & YES & YES & YES & -- & 1220\\
$(78, 23)$ & 10 & $(3, 1)$ & 2 & 3 & YES & YES & YES & -- & 1221\\
$(78, 23)$ & 10 & $(3, 1)$ & 2 & 3 & YES & YES & YES & NO & 1222\\
$(78, 23)$ & 10 & $(4, 1)$ & 3 & 2 & YES & YES & YES & NO & 1223\\
$(78, 23)$ & 10 & $(10, 3)$ & 5 & 2 & YES & YES & YES & NO & 1224\\
$(78, 23)$ & 10 & $(44, 13)$ & 8 & 2 & YES & YES & YES & 1433 & 1225\\
$(79, 29)$ & 9 & $(2, 1)$ & 1 & 1 & YES & YES & YES & -- & 1226\\
$(79, 29)$ & 9 & $(2, 1)$ & 1 & 1 & YES & YES & YES & NO & 1227\\
$(79, 30)$ & 9 & $(2, 1)$ & 1 & 1 & YES & YES & YES & -- & 1228\\
$(79, 30)$ & 9 & $(2, 1)$ & 1 & 1 & YES & YES & YES & NO & 1229\\
$(79, 14)$ & 11 & $(3, 1)$ & 2 & 1 & YES & YES & YES & -- & 1230\\
$(79, 22)$ & 10 & $(3, 1)$ & 2 & 1 & YES & YES & YES & NO & 1231\\
$(79, 23)$ & 10 & $(3, 1)$ & 2 & 1 & YES & YES & YES & -- & 1232\\
$(79, 23)$ & 10 & $(3, 1)$ & 2 & 1 & YES & YES & YES & NO & 1233\\
$(79, 23)$ & 10 & $(3, 1)$ & 2 & 1 & YES & YES & YES & 653 & 1234\\
$(79, 30)$ & 9 & $(3, 1)$ & 2 & 1 & YES & YES & YES & -- & 1235\\
$(79, 30)$ & 9 & $(3, 1)$ & 2 & 1 & YES & YES & YES & NO & 1236\\
$(79, 29)$ & 9 & $(4, 1)$ & 3 & 1 & YES & YES & YES & NO & 1237\\
$(79, 30)$ & 9 & $(4, 1)$ & 3 & 1 & YES & YES & YES & -- & 1238\\
$(79, 30)$ & 9 & $(4, 1)$ & 3 & 1 & YES & YES & YES & NO & 1239\\
$(79, 17)$ & 11 & $(5, 2)$ & 3 & 1 & YES & YES & YES & -- & 1240\\
$(79, 18)$ & 10 & $(5, 2)$ & 3 & 1 & YES & YES & YES & -- & 1241\\
$(79, 18)$ & 10 & $(5, 2)$ & 3 & 1 & YES & YES & YES & NO & 1242\\
$(79, 23)$ & 10 & $(5, 1)$ & 4 & 1 & YES & YES & YES & NO & 1243\\
$(79, 29)$ & 9 & $(5, 1)$ & 4 & 1 & YES & YES & YES & -- & 1244\\
$(79, 29)$ & 9 & $(5, 2)$ & 3 & 1 & YES & YES & YES & -- & 1245\\
$(79, 30)$ & 9 & $(5, 1)$ & 4 & 1 & YES & YES & YES & -- & 1246\\
$(79, 30)$ & 9 & $(5, 1)$ & 4 & 1 & YES & YES & YES & NO & 1247\\
$(79, 30)$ & 9 & $(5, 1)$ & 4 & 1 & YES & YES & YES & NO & 1248\\
$(79, 30)$ & 9 & $(5, 2)$ & 3 & 1 & YES & YES & YES & NO & 1249\\
$(79, 14)$ & 11 & $(6, 1)$ & 5 & 1 & YES & YES & YES & NO & 1250\\
$(79, 30)$ & 9 & $(8, 3)$ & 4 & 1 & YES & YES & YES & NO & 1251\\
$(79, 23)$ & 10 & $(10, 3)$ & 5 & 1 & YES & YES & YES & NO & 1252\\
$(79, 14)$ & 11 & $(11, 2)$ & 6 & 1 & YES & YES & YES & NO & 1253\\
$(79, 22)$ & 10 & $(11, 3)$ & 5 & 1 & YES & YES & YES & NO & 1254\\
$(79, 23)$ & 10 & $(11, 3)$ & 5 & 1 & YES & YES & YES & NO & 1255\\
$(79, 17)$ & 11 & $(13, 3)$ & 6 & 1 & YES & YES & YES & NO & 1256\\
$(79, 30)$ & 9 & $(13, 5)$ & 5 & 1 & YES & YES & YES & 1349 & 1257\\
$(79, 18)$ & 10 & $(14, 3)$ & 6 & 1 & YES & YES & YES & NO & 1258\\
$(79, 14)$ & 11 & $(17, 3)$ & 7 & 1 & YES & YES & YES & NO & 1259\\
$(79, 29)$ & 9 & $(19, 7)$ & 6 & 1 & YES & YES & YES & 1113 & 1260\\
$(79, 30)$ & 9 & $(21, 8)$ & 6 & 1 & YES & YES & YES & 1161 & 1261\\
$(79, 30)$ & 9 & $(29, 11)$ & 7 & 1 & YES & YES & YES & NO & 1262\\
$(79, 29)$ & 9 & $(30, 11)$ & 7 & 1 & YES & YES & YES & NO & 1263\\
$(79, 23)$ & 10 & $(31, 9)$ & 8 & 1 & YES & YES & YES & 1328 & 1264\\
$(79, 29)$ & 9 & $(41, 15)$ & 8 & 1 & YES & YES & YES & NO & 1265\\
$(79, 30)$ & 9 & $(50, 19)$ & 8 & 1 & YES & YES & YES & NO & 1266\\
$(79, 23)$ & 10 & $(79, 23)$ & 10 & 79 & YES & YES & YES & NO & 1267\\
$(79, 30)$ & 9 & $(79, 30)$ & 9 & 79 & YES & YES & YES & NO & 1268\\
$(80, 31)$ & 9 & $(2, 1)$ & 1 & 2 & YES & YES & YES & NO & 1269\\
$(80, 31)$ & 9 & $(3, 1)$ & 2 & 1 & YES & YES & YES & -- & 1270\\
$(80, 31)$ & 9 & $(3, 1)$ & 2 & 1 & YES & YES & YES & NO & 1271\\
$(80, 31)$ & 9 & $(4, 1)$ & 3 & 4 & YES & YES & YES & NO & 1272\\
$(80, 31)$ & 9 & $(5, 2)$ & 3 & 5 & YES & YES & YES & 1071 & 1273\\
$(80, 31)$ & 9 & $(8, 3)$ & 4 & 8 & YES & YES & YES & NO & 1274\\
$(80, 31)$ & 9 & $(13, 5)$ & 5 & 1 & YES & YES & YES & NO & 1275\\
$(80, 31)$ & 9 & $(49, 19)$ & 8 & 1 & YES & YES & YES & NO & 1276\\
$(80, 31)$ & 9 & $(80, 31)$ & 9 & 80 & YES & YES & YES & NO & 1277\\
$(81, 31)$ & 9 & $(2, 1)$ & 1 & 1 & YES & YES & YES & -- & 1278\\
$(81, 31)$ & 9 & $(2, 1)$ & 1 & 1 & YES & YES & YES & NO & 1279\\
$(81, 34)$ & 9 & $(2, 1)$ & 1 & 1 & YES & YES & YES & -- & 1280\\
$(81, 34)$ & 9 & $(3, 1)$ & 2 & 3 & YES & YES & YES & NO & 1281\\
$(81, 31)$ & 9 & $(4, 1)$ & 3 & 1 & YES & YES & YES & -- & 1282\\
$(81, 31)$ & 9 & $(4, 1)$ & 3 & 1 & YES & YES & YES & NO & 1283\\
$(81, 31)$ & 9 & $(5, 1)$ & 4 & 1 & YES & YES & YES & -- & 1284\\
$(81, 31)$ & 9 & $(5, 2)$ & 3 & 1 & YES & YES & YES & NO & 1285\\
$(81, 34)$ & 9 & $(5, 1)$ & 4 & 1 & YES & YES & YES & -- & 1286\\
$(81, 34)$ & 9 & $(5, 1)$ & 4 & 1 & YES & YES & YES & NO & 1287\\
$(81, 34)$ & 9 & $(5, 2)$ & 3 & 1 & YES & YES & YES & NO & 1288\\
$(81, 34)$ & 9 & $(7, 3)$ & 4 & 1 & YES & YES & YES & 1155 & 1289\\
$(81, 31)$ & 9 & $(8, 3)$ & 4 & 1 & YES & YES & YES & 887 & 1290\\
$(81, 31)$ & 9 & $(13, 5)$ & 5 & 1 & YES & YES & YES & 1025 & 1291\\
$(81, 34)$ & 9 & $(19, 8)$ & 6 & 1 & YES & YES & YES & 1125 & 1292\\
$(81, 34)$ & 9 & $(31, 13)$ & 7 & 1 & YES & YES & YES & NO & 1293\\
$(81, 31)$ & 9 & $(34, 13)$ & 7 & 1 & YES & YES & YES & NO & 1294\\
$(81, 31)$ & 9 & $(47, 18)$ & 8 & 1 & YES & YES & YES & NO & 1295\\
$(82, 23)$ & 10 & $(5, 1)$ & 4 & 1 & YES & YES & YES & NO & 1296\\
$(82, 23)$ & 10 & $(5, 2)$ & 3 & 1 & YES & YES & YES & -- & 1297\\
$(82, 23)$ & 10 & $(10, 3)$ & 5 & 2 & YES & YES & YES & NO & 1298\\
$(82, 23)$ & 10 & $(32, 9)$ & 8 & 2 & YES & YES & YES & 1354 & 1299\\
$(82, 23)$ & 10 & $(82, 23)$ & 10 & 82 & YES & YES & YES & NO & 1300\\
$(83, 23)$ & 10 & $(2, 1)$ & 1 & 1 & YES & YES & YES & -- & 1301\\
$(83, 23)$ & 10 & $(2, 1)$ & 1 & 1 & YES & YES & YES & NO & 1302\\
$(83, 23)$ & 10 & $(3, 1)$ & 2 & 1 & YES & YES & YES & NO & 1303\\
$(83, 23)$ & 10 & $(4, 1)$ & 3 & 1 & YES & YES & YES & NO & 1304\\
$(83, 23)$ & 10 & $(11, 3)$ & 5 & 1 & YES & YES & YES & NO & 1305\\
$(84, 25)$ & 10 & $(2, 1)$ & 1 & 2 & YES & YES & YES & -- & 1306\\
$(84, 25)$ & 10 & $(2, 1)$ & 1 & 2 & YES & YES & YES & NO & 1307\\
$(84, 37)$ & 10 & $(2, 1)$ & 1 & 2 & YES & YES & YES & -- & 1308\\
$(84, 25)$ & 10 & $(3, 1)$ & 2 & 3 & YES & YES & YES & NO & 1309\\
$(84, 37)$ & 10 & $(3, 1)$ & 2 & 3 & YES & YES & YES & -- & 1310\\
$(84, 37)$ & 10 & $(3, 1)$ & 2 & 3 & YES & YES & YES & NO & 1311\\
$(84, 37)$ & 10 & $(4, 1)$ & 3 & 4 & YES & YES & YES & -- & 1312\\
$(84, 25)$ & 10 & $(7, 2)$ & 4 & 7 & YES & YES & YES & NO & 1313\\
$(84, 37)$ & 10 & $(7, 3)$ & 4 & 7 & YES & YES & YES & NO & 1314\\
$(85, 26)$ & 10 & $(2, 1)$ & 1 & 1 & YES & YES & YES & -- & 1315\\
$(85, 37)$ & 10 & $(3, 1)$ & 2 & 1 & YES & YES & YES & NO & 1316\\
$(85, 26)$ & 10 & $(10, 3)$ & 5 & 5 & YES & YES & YES & 919 & 1317\\
$(85, 37)$ & 10 & $(16, 7)$ & 6 & 1 & YES & YES & YES & NO & 1318\\
$(85, 37)$ & 10 & $(39, 17)$ & 8 & 1 & YES & YES & YES & 1415 & 1319\\
$(86, 25)$ & 10 & $(2, 1)$ & 1 & 2 & YES & YES & YES & -- & 1320\\
$(86, 25)$ & 10 & $(2, 1)$ & 1 & 2 & YES & YES & YES & NO & 1321\\
$(86, 25)$ & 10 & $(3, 1)$ & 2 & 1 & YES & YES & YES & NO & 1322\\
$(86, 25)$ & 10 & $(4, 1)$ & 3 & 2 & YES & YES & YES & NO & 1323\\
$(86, 25)$ & 10 & $(5, 1)$ & 4 & 1 & YES & YES & YES & -- & 1324\\
$(86, 25)$ & 10 & $(5, 1)$ & 4 & 1 & YES & YES & YES & NO & 1325\\
$(86, 25)$ & 10 & $(10, 3)$ & 5 & 2 & YES & YES & YES & NO & 1326\\
$(86, 25)$ & 10 & $(17, 5)$ & 6 & 1 & YES & YES & YES & 1442 & 1327\\
$(86, 25)$ & 10 & $(24, 7)$ & 7 & 2 & YES & YES & YES & 1264 & 1328\\
$(86, 25)$ & 10 & $(31, 9)$ & 8 & 1 & YES & YES & YES & NO & 1329\\
$(86, 25)$ & 10 & $(86, 25)$ & 10 & 86 & YES & YES & YES & NO & 1330\\
$(87, 32)$ & 10 & $(4, 1)$ & 3 & 1 & YES & YES & YES & -- & 1331\\
$(88, 37)$ & 10 & $(2, 1)$ & 1 & 2 & NO & YES & YES & -- & 1332\\
$(89, 25)$ & 10 & $(2, 1)$ & 1 & 1 & YES & YES & YES & -- & 1333\\
$(89, 25)$ & 10 & $(2, 1)$ & 1 & 1 & YES & YES & YES & NO & 1334\\
$(89, 26)$ & 10 & $(2, 1)$ & 1 & 1 & YES & YES & YES & -- & 1335\\
$(89, 34)$ & 9 & $(2, 1)$ & 1 & 1 & YES & YES & YES & -- & 1336\\
$(89, 34)$ & 9 & $(2, 1)$ & 1 & 1 & YES & YES & YES & NO & 1337\\
$(89, 25)$ & 10 & $(3, 1)$ & 2 & 1 & YES & YES & YES & NO & 1338\\
$(89, 26)$ & 10 & $(3, 1)$ & 2 & 1 & YES & YES & YES & -- & 1339\\
$(89, 26)$ & 10 & $(3, 1)$ & 2 & 1 & YES & YES & YES & NO & 1340\\
$(89, 26)$ & 10 & $(3, 1)$ & 2 & 1 & YES & YES & YES & NO & 1341\\
$(89, 34)$ & 9 & $(3, 1)$ & 2 & 1 & YES & YES & YES & -- & 1342\\
$(89, 34)$ & 9 & $(3, 1)$ & 2 & 1 & YES & YES & YES & NO & 1343\\
$(89, 25)$ & 10 & $(4, 1)$ & 3 & 1 & YES & YES & YES & NO & 1344\\
$(89, 26)$ & 10 & $(4, 1)$ & 3 & 1 & YES & YES & YES & NO & 1345\\
$(89, 25)$ & 10 & $(5, 1)$ & 4 & 1 & YES & YES & YES & NO & 1346\\
$(89, 34)$ & 9 & $(5, 2)$ & 3 & 1 & YES & YES & YES & NO & 1347\\
$(89, 17)$ & 12 & $(6, 1)$ & 5 & 1 & YES & YES & YES & NO & 1348\\
$(89, 34)$ & 9 & $(8, 3)$ & 4 & 1 & YES & YES & YES & 1257 & 1349\\
$(89, 34)$ & 9 & $(13, 5)$ & 5 & 1 & YES & YES & YES & NO & 1350\\
$(89, 26)$ & 10 & $(17, 5)$ & 6 & 1 & YES & YES & YES & NO & 1351\\
$(89, 17)$ & 12 & $(21, 4)$ & 8 & 1 & YES & YES & YES & NO & 1352\\
$(89, 34)$ & 9 & $(21, 8)$ & 6 & 1 & YES & YES & YES & 1217 & 1353\\
$(89, 25)$ & 10 & $(25, 7)$ & 7 & 1 & YES & YES & YES & 1299 & 1354\\
$(89, 25)$ & 10 & $(32, 9)$ & 8 & 1 & YES & YES & YES & NO & 1355\\
$(89, 27)$ & 10 & $(33, 10)$ & 8 & 1 & YES & YES & YES & NO & 1356\\
$(89, 26)$ & 10 & $(41, 12)$ & 8 & 1 & YES & YES & YES & 1444 & 1357\\
$(89, 25)$ & 10 & $(57, 16)$ & 9 & 1 & YES & YES & YES & NO & 1358\\
$(89, 26)$ & 10 & $(65, 19)$ & 9 & 1 & YES & YES & YES & NO & 1359\\
$(89, 26)$ & 10 & $(89, 26)$ & 10 & 89 & YES & YES & YES & NO & 1360\\
$(91, 27)$ & 10 & $(2, 1)$ & 1 & 1 & YES & YES & YES & -- & 1361\\
$(91, 27)$ & 10 & $(3, 1)$ & 2 & 1 & YES & YES & YES & -- & 1362\\
$(91, 25)$ & 10 & $(4, 1)$ & 3 & 1 & YES & YES & YES & NO & 1363\\
$(91, 27)$ & 10 & $(4, 1)$ & 3 & 1 & YES & YES & YES & NO & 1364\\
$(91, 27)$ & 10 & $(7, 2)$ & 4 & 7 & YES & YES & YES & NO & 1365\\
$(91, 27)$ & 10 & $(17, 5)$ & 6 & 1 & YES & YES & YES & NO & 1366\\
$(91, 27)$ & 10 & $(37, 11)$ & 8 & 1 & YES & YES & YES & 1416 & 1367\\
$(91, 27)$ & 10 & $(91, 27)$ & 10 & 91 & YES & YES & YES & NO & 1368\\
$(92, 33)$ & 10 & $(2, 1)$ & 1 & 2 & YES & YES & YES & -- & 1369\\
$(92, 33)$ & 10 & $(4, 1)$ & 3 & 4 & YES & YES & YES & -- & 1370\\
$(92, 35)$ & 10 & $(8, 3)$ & 4 & 4 & YES & YES & YES & NO & 1371\\
$(92, 33)$ & 10 & $(11, 4)$ & 5 & 1 & YES & YES & YES & 955 & 1372\\
$(92, 33)$ & 10 & $(39, 14)$ & 8 & 1 & YES & YES & YES & NO & 1373\\
$(93, 26)$ & 10 & $(2, 1)$ & 1 & 1 & YES & YES & YES & -- & 1374\\
$(93, 26)$ & 10 & $(2, 1)$ & 1 & 1 & YES & YES & YES & NO & 1375\\
$(93, 26)$ & 10 & $(3, 1)$ & 2 & 3 & YES & YES & YES & -- & 1376\\
$(93, 26)$ & 10 & $(5, 1)$ & 4 & 1 & YES & YES & YES & NO & 1377\\
$(93, 26)$ & 10 & $(11, 3)$ & 5 & 1 & YES & YES & YES & NO & 1378\\
$(93, 26)$ & 10 & $(18, 5)$ & 6 & 3 & YES & YES & YES & NO & 1379\\
$(93, 26)$ & 10 & $(93, 26)$ & 10 & 93 & YES & YES & YES & NO & 1380\\
$(94, 41)$ & 10 & $(3, 1)$ & 2 & 1 & YES & YES & YES & NO & 1381\\
$(96, 17)$ & 12 & $(5, 2)$ & 3 & 1 & YES & YES & YES & -- & 1382\\
$(96, 17)$ & 12 & $(16, 3)$ & 7 & 16 & YES & YES & YES & NO & 1383\\
$(97, 26)$ & 10 & $(5, 2)$ & 3 & 1 & YES & YES & YES & NO & 1384\\
$(97, 35)$ & 10 & $(5, 1)$ & 4 & 1 & YES & YES & YES & -- & 1385\\
$(97, 35)$ & 10 & $(36, 13)$ & 8 & 1 & YES & YES & YES & NO & 1386\\
$(98, 27)$ & 10 & $(2, 1)$ & 1 & 2 & YES & YES & YES & -- & 1387\\
$(98, 27)$ & 10 & $(2, 1)$ & 1 & 2 & YES & YES & YES & NO & 1388\\
$(98, 29)$ & 10 & $(2, 1)$ & 1 & 2 & YES & YES & YES & -- & 1389\\
$(98, 29)$ & 10 & $(2, 1)$ & 1 & 2 & YES & YES & YES & NO & 1390\\
$(98, 29)$ & 10 & $(3, 1)$ & 2 & 1 & YES & YES & YES & -- & 1391\\
$(98, 29)$ & 10 & $(4, 1)$ & 3 & 2 & YES & YES & YES & NO & 1392\\
$(98, 27)$ & 10 & $(7, 2)$ & 4 & 7 & YES & YES & YES & NO & 1393\\
$(98, 29)$ & 10 & $(7, 2)$ & 4 & 7 & YES & YES & YES & NO & 1394\\
$(98, 29)$ & 10 & $(17, 5)$ & 6 & 1 & YES & YES & YES & NO & 1395\\
$(99, 29)$ & 10 & $(2, 1)$ & 1 & 1 & YES & YES & YES & -- & 1396\\
$(99, 29)$ & 10 & $(2, 1)$ & 1 & 1 & YES & YES & YES & NO & 1397\\
$(99, 41)$ & 10 & $(3, 1)$ & 2 & 3 & YES & YES & YES & -- & 1398\\
$(99, 41)$ & 10 & $(3, 1)$ & 2 & 3 & YES & YES & YES & NO & 1399\\
$(99, 29)$ & 10 & $(4, 1)$ & 3 & 1 & YES & YES & YES & NO & 1400\\
$(99, 29)$ & 10 & $(10, 3)$ & 5 & 1 & YES & YES & YES & NO & 1401\\
$(99, 29)$ & 10 & $(24, 7)$ & 7 & 3 & YES & YES & YES & NO & 1402\\
$(99, 29)$ & 10 & $(41, 12)$ & 8 & 1 & YES & YES & YES & NO & 1403\\
$(99, 29)$ & 10 & $(58, 17)$ & 9 & 1 & YES & YES & YES & NO & 1404\\
$(100, 31)$ & 11 & $(2, 1)$ & 1 & 2 & YES & YES & YES & NO & 1405\\
$(100, 37)$ & 10 & $(3, 1)$ & 2 & 1 & YES & YES & YES & NO & 1406\\
$(100, 39)$ & 10 & $(5, 2)$ & 3 & 5 & YES & YES & YES & 1016 & 1407\\
$(100, 19)$ & 12 & $(11, 2)$ & 6 & 1 & YES & YES & YES & NO & 1408\\
$(101, 30)$ & 10 & $(2, 1)$ & 1 & 1 & YES & YES & YES & -- & 1409\\
$(101, 37)$ & 10 & $(3, 1)$ & 2 & 1 & YES & YES & YES & 954 & 1410\\
$(101, 39)$ & 10 & $(5, 1)$ & 4 & 1 & YES & YES & YES & -- & 1411\\
$(101, 39)$ & 10 & $(5, 1)$ & 4 & 1 & YES & YES & YES & NO & 1412\\
$(101, 16)$ & 13 & $(6, 1)$ & 5 & 1 & YES & YES & YES & NO & 1413\\
$(101, 39)$ & 10 & $(13, 5)$ & 5 & 1 & YES & YES & YES & 1134 & 1414\\
$(101, 44)$ & 10 & $(23, 10)$ & 7 & 1 & YES & YES & YES & 1319 & 1415\\
$(101, 30)$ & 10 & $(27, 8)$ & 7 & 1 & YES & YES & YES & 1367 & 1416\\
$(102, 43)$ & 11 & $(2, 1)$ & 1 & 2 & NO & YES & YES & -- & 1417\\
$(102, 23)$ & 11 & $(14, 3)$ & 6 & 2 & YES & YES & YES & NO & 1418\\
$(103, 30)$ & 11 & $(2, 1)$ & 1 & 1 & YES & YES & YES & -- & 1419\\
$(104, 29)$ & 10 & $(2, 1)$ & 1 & 2 & YES & YES & YES & NO & 1420\\
$(104, 29)$ & 10 & $(11, 3)$ & 5 & 1 & YES & YES & YES & NO & 1421\\
$(105, 29)$ & 10 & $(2, 1)$ & 1 & 1 & YES & YES & YES & NO & 1422\\
$(105, 31)$ & 10 & $(2, 1)$ & 1 & 1 & YES & YES & YES & -- & 1423\\
$(105, 31)$ & 10 & $(2, 1)$ & 1 & 1 & YES & YES & YES & NO & 1424\\
$(105, 29)$ & 10 & $(3, 1)$ & 2 & 3 & YES & YES & YES & -- & 1425\\
$(105, 29)$ & 10 & $(3, 1)$ & 2 & 3 & YES & YES & YES & NO & 1426\\
$(105, 31)$ & 10 & $(3, 1)$ & 2 & 3 & YES & YES & YES & -- & 1427\\
$(105, 31)$ & 10 & $(4, 1)$ & 3 & 1 & YES & YES & YES & NO & 1428\\
$(105, 29)$ & 10 & $(7, 2)$ & 4 & 7 & YES & YES & YES & NO & 1429\\
$(105, 31)$ & 10 & $(7, 2)$ & 4 & 7 & YES & YES & YES & NO & 1430\\
$(105, 23)$ & 11 & $(13, 3)$ & 6 & 1 & YES & YES & YES & NO & 1431\\
$(105, 23)$ & 11 & $(14, 3)$ & 6 & 7 & YES & YES & YES & NO & 1432\\
$(105, 31)$ & 10 & $(17, 5)$ & 6 & 1 & YES & YES & YES & 1225 & 1433\\
$(105, 23)$ & 11 & $(23, 5)$ & 7 & 1 & YES & YES & YES & NO & 1434\\
$(105, 31)$ & 10 & $(105, 31)$ & 10 & 105 & YES & YES & YES & NO & 1435\\
$(106, 31)$ & 10 & $(2, 1)$ & 1 & 2 & YES & YES & YES & -- & 1436\\
$(106, 41)$ & 10 & $(2, 1)$ & 1 & 2 & YES & YES & YES & -- & 1437\\
$(106, 41)$ & 10 & $(2, 1)$ & 1 & 2 & YES & YES & YES & NO & 1438\\
$(106, 31)$ & 10 & $(3, 1)$ & 2 & 1 & YES & YES & YES & -- & 1439\\
$(106, 31)$ & 10 & $(4, 1)$ & 3 & 2 & YES & YES & YES & NO & 1440\\
$(106, 31)$ & 10 & $(5, 1)$ & 4 & 1 & YES & YES & YES & NO & 1441\\
$(106, 31)$ & 10 & $(7, 2)$ & 4 & 1 & YES & YES & YES & 1327 & 1442\\
$(106, 31)$ & 10 & $(17, 5)$ & 6 & 1 & YES & YES & YES & NO & 1443\\
$(106, 31)$ & 10 & $(24, 7)$ & 7 & 2 & YES & YES & YES & 1357 & 1444\\
$(106, 31)$ & 10 & $(41, 12)$ & 8 & 1 & YES & YES & YES & NO & 1445\\
$(106, 31)$ & 10 & $(65, 19)$ & 9 & 1 & YES & YES & YES & NO & 1446\\
$(107, 25)$ & 11 & $(13, 3)$ & 6 & 1 & YES & YES & YES & NO & 1447\\
$(109, 30)$ & 10 & $(2, 1)$ & 1 & 1 & YES & YES & YES & -- & 1448\\
$(109, 30)$ & 10 & $(2, 1)$ & 1 & 1 & YES & YES & YES & NO & 1449\\
$(109, 30)$ & 10 & $(7, 2)$ & 4 & 1 & YES & YES & YES & NO & 1450\\
$(109, 30)$ & 10 & $(11, 3)$ & 5 & 1 & YES & YES & YES & NO & 1451\\
$(111, 31)$ & 10 & $(2, 1)$ & 1 & 1 & YES & YES & YES & NO & 1452\\
$(111, 43)$ & 10 & $(2, 1)$ & 1 & 1 & NO & YES & YES & -- & 1453\\
$(111, 25)$ & 11 & $(3, 1)$ & 2 & 3 & YES & YES & YES & NO & 1454\\
$(111, 31)$ & 10 & $(3, 1)$ & 2 & 3 & YES & YES & YES & NO & 1455\\
$(111, 41)$ & 10 & $(3, 1)$ & 2 & 3 & YES & YES & YES & NO & 1456\\
$(111, 31)$ & 10 & $(18, 5)$ & 6 & 3 & YES & YES & YES & NO & 1457\\
$(112, 31)$ & 10 & $(2, 1)$ & 1 & 2 & YES & YES & YES & -- & 1458\\
$(112, 47)$ & 10 & $(2, 1)$ & 1 & 2 & NO & YES & YES & -- & 1459\\
$(113, 24)$ & 11 & $(19, 4)$ & 7 & 1 & YES & YES & YES & NO & 1460\\
$(115, 26)$ & 11 & $(3, 1)$ & 2 & 1 & YES & YES & YES & -- & 1461\\
$(115, 26)$ & 11 & $(3, 1)$ & 2 & 1 & YES & YES & YES & NO & 1462\\
$(115, 26)$ & 11 & $(3, 1)$ & 2 & 1 & YES & YES & YES & NO & 1463\\
$(115, 26)$ & 11 & $(9, 2)$ & 5 & 1 & YES & YES & YES & NO & 1464\\
$(115, 26)$ & 11 & $(31, 7)$ & 8 & 1 & YES & YES & YES & NO & 1465\\
$(116, 49)$ & 10 & $(2, 1)$ & 1 & 2 & NO & YES & YES & -- & 1466\\
$(117, 49)$ & 10 & $(2, 1)$ & 1 & 1 & NO & YES & YES & -- & 1467\\
$(117, 31)$ & 11 & $(3, 1)$ & 2 & 3 & YES & YES & YES & NO & 1468\\
$(118, 27)$ & 11 & $(2, 1)$ & 1 & 2 & YES & YES & YES & NO & 1469\\
$(118, 27)$ & 11 & $(9, 2)$ & 5 & 1 & YES & YES & YES & NO & 1470\\
$(119, 50)$ & 10 & $(2, 1)$ & 1 & 1 & NO & YES & YES & -- & 1471\\
$(119, 27)$ & 12 & $(5, 1)$ & 4 & 1 & YES & YES & YES & NO & 1472\\
$(119, 22)$ & 12 & $(16, 3)$ & 7 & 1 & YES & YES & YES & NO & 1473\\
$(124, 23)$ & 12 & $(2, 1)$ & 1 & 2 & YES & YES & YES & -- & 1474\\
$(124, 23)$ & 12 & $(2, 1)$ & 1 & 2 & YES & YES & YES & NO & 1475\\
$(124, 27)$ & 12 & $(4, 1)$ & 3 & 4 & YES & YES & YES & NO & 1476\\
$(124, 23)$ & 12 & $(5, 1)$ & 4 & 1 & YES & YES & YES & NO & 1477\\
$(124, 23)$ & 12 & $(6, 1)$ & 5 & 2 & YES & YES & YES & NO & 1478\\
$(124, 23)$ & 12 & $(11, 2)$ & 6 & 1 & YES & YES & YES & NO & 1479\\
$(124, 23)$ & 12 & $(16, 3)$ & 7 & 4 & YES & YES & YES & NO & 1480\\
$(127, 29)$ & 11 & $(9, 2)$ & 5 & 1 & YES & YES & YES & NO & 1481\\
$(129, 23)$ & 12 & $(2, 1)$ & 1 & 1 & YES & YES & YES & -- & 1482\\
$(129, 23)$ & 12 & $(2, 1)$ & 1 & 1 & YES & YES & YES & NO & 1483\\
$(129, 23)$ & 12 & $(5, 1)$ & 4 & 1 & YES & YES & YES & NO & 1484\\
$(134, 29)$ & 11 & $(2, 1)$ & 1 & 2 & YES & YES & YES & -- & 1485\\
$(134, 29)$ & 11 & $(2, 1)$ & 1 & 2 & YES & YES & YES & NO & 1486\\
$(148, 31)$ & 12 & $(2, 1)$ & 1 & 2 & YES & YES & YES & NO & 1487\\
$(148, 35)$ & 12 & $(2, 1)$ & 1 & 2 & YES & YES & YES & NO & 1488\\
$(148, 31)$ & 12 & $(4, 1)$ & 3 & 4 & YES & YES & YES & NO & 1489\\
$(149, 34)$ & 11 & $(2, 1)$ & 1 & 1 & YES & YES & YES & -- & 1490\\
$(149, 34)$ & 11 & $(2, 1)$ & 1 & 1 & YES & YES & YES & NO & 1491\\
$(149, 34)$ & 11 & $(9, 2)$ & 5 & 1 & YES & YES & YES & NO & 1492\\
$(149, 34)$ & 11 & $(22, 5)$ & 7 & 1 & YES & YES & YES & NO & 1493\\
$(151, 27)$ & 13 & $(5, 1)$ & 4 & 1 & YES & YES & YES & NO & 1494\\
$(154, 65)$ & 11 & $(2, 1)$ & 1 & 2 & NO & YES & YES & -- & 1495\\
$(156, 29)$ & 12 & $(5, 1)$ & 4 & 1 & YES & YES & YES & NO & 1496\\
$(a; 1, 0, 0; 13)$ & 5 & $(11, 3)$ & 5 & 1 & YES & YES & YES & -- & 1497\\
$(a; 1, 1, 1; 4)$ & 7 & $(5, 2)$ & 3 & 1 & YES & YES & YES & -- & 1498\\
$(a; 2, 0, 1; 25)$ & 7 & $(4, 1)$ & 3 & 1 & YES & YES & YES & -- & 1499\\
$(a; 2, 1, 1; 37)$ & 8 & $(3, 1)$ & 2 & 1 & YES & YES & YES & -- & 1500\\
$(a; 2, 1, 1; 37)$ & 8 & $(5, 2)$ & 3 & 1 & YES & YES & YES & -- & 1501\\
$(a; 3, 0, 0; 7)$ & 7 & $(3, 1)$ & 2 & 1 & YES & YES & YES & -- & 1502\\
$(a; 3, 0, 1; 31)$ & 8 & $(2, 1)$ & 1 & 1 & YES & YES & YES & -- & 1503\\
$(a; 3, 0, 1; 31)$ & 8 & $(5, 1)$ & 4 & 1 & YES & YES & YES & -- & 1504\\
$(b; 0, 0, 0; 14)$ & 5 & $(10, 3)$ & 5 & 2 & YES & YES & YES & -- & 1505\\
$(b; 0, 0, 1; 4)$ & 6 & $(7, 3)$ & 4 & 1 & YES & YES & YES & -- & 1506\\
$(b; 0, 1, 0; 19)$ & 6 & $(11, 3)$ & 5 & 1 & YES & YES & YES & -- & 1507\\
$(b; 0, 1, 1; 27)$ & 7 & $(5, 2)$ & 3 & 1 & YES & YES & YES & -- & 1508\\
$(b; 0, 1, 1; 27)$ & 7 & $(7, 3)$ & 4 & 1 & YES & YES & YES & -- & 1509\\
$(b; 0, 1, 3; 43)$ & 9 & $(5, 1)$ & 4 & 1 & YES & YES & YES & -- & 1510\\
$(b; 0, 2, 0; 8)$ & 7 & $(3, 1)$ & 2 & 1 & YES & YES & YES & -- & 1511\\
$(b; 0, 2, 1; 34)$ & 8 & $(5, 2)$ & 3 & 1 & YES & YES & YES & -- & 1512\\
$(b; 0, 3, 0; 29)$ & 8 & $(2, 1)$ & 1 & 1 & YES & YES & YES & -- & 1513\\
$(b; 0, 3, 0; 29)$ & 8 & $(11, 2)$ & 6 & 1 & YES & YES & YES & -- & 1514\\
$(b; 1, 0, 0; 5)$ & 6 & $(7, 3)$ & 4 & 1 & YES & YES & YES & -- & 1515\\
$(b; 1, 0, 0; 5)$ & 6 & $(13, 4)$ & 6 & 1 & YES & YES & YES & -- & 1516\\
$(b; 1, 0, 1; 29)$ & 7 & $(5, 2)$ & 3 & 1 & YES & YES & YES & -- & 1517\\
$(b; 1, 0, 1; 29)$ & 7 & $(10, 3)$ & 5 & 1 & YES & YES & YES & -- & 1518\\
$(b; 1, 1, 0; 27)$ & 7 & $(5, 2)$ & 3 & 1 & YES & YES & YES & -- & 1519\\
$(b; 1, 1, 0; 27)$ & 7 & $(13, 3)$ & 6 & 1 & YES & YES & YES & -- & 1520\\
$(b; 1, 1, 1; 39)$ & 8 & $(2, 1)$ & 1 & 1 & YES & YES & YES & -- & 1521\\
$(b; 1, 1, 1; 39)$ & 8 & $(3, 1)$ & 2 & 3 & YES & YES & YES & -- & 1522\\
$(b; 1, 2, 0; 17)$ & 8 & $(3, 1)$ & 2 & 1 & YES & YES & YES & -- & 1523\\
$(c; 0, 0, 0; 4)$ & 4 & $(18, 7)$ & 6 & 2 & YES & YES & YES & -- & 1524\\
$(c; 0, 0, 0; 4)$ & 4 & $(22, 9)$ & 7 & 2 & YES & YES & YES & -- & 1525\\
$(c; 0, 0, 0; 4)$ & 4 & $(26, 11)$ & 7 & 2 & YES & YES & YES & -- & 1526\\
$(c; 0, 0, 0; 4)$ & 4 & $(29, 11)$ & 7 & 1 & YES & YES & YES & -- & 1527\\
$(c; 0, 0, 0; 4)$ & 4 & $(29, 12)$ & 7 & 1 & YES & YES & YES & -- & 1528\\
$(c; 0, 0, 0; 4)$ & 4 & $(31, 9)$ & 8 & 1 & YES & YES & YES & -- & 1529\\
$(c; 0, 0, 0; 4)$ & 4 & $(31, 12)$ & 7 & 1 & YES & YES & YES & -- & 1530\\
$(c; 0, 0, 0; 4)$ & 4 & $(34, 13)$ & 7 & 2 & YES & YES & YES & -- & 1531\\
$(c; 0, 1, 0; 11)$ & 5 & $(9, 4)$ & 5 & 1 & YES & YES & YES & -- & 1532\\
$(c; 0, 1, 0; 11)$ & 5 & $(13, 5)$ & 5 & 1 & YES & YES & YES & -- & 1533\\
$(c; 0, 1, 0; 11)$ & 5 & $(17, 7)$ & 6 & 1 & YES & YES & YES & -- & 1534\\
$(c; 0, 1, 0; 11)$ & 5 & $(18, 7)$ & 6 & 1 & YES & YES & YES & -- & 1535\\
$(c; 0, 1, 0; 11)$ & 5 & $(19, 7)$ & 6 & 1 & YES & YES & YES & -- & 1536\\
$(c; 0, 1, 0; 11)$ & 5 & $(21, 8)$ & 6 & 1 & YES & YES & YES & -- & 1537\\
$(c; 0, 1, 0; 11)$ & 5 & $(24, 7)$ & 7 & 1 & YES & YES & YES & -- & 1538\\
$(c; 0, 1, 0; 11)$ & 5 & $(47, 11)$ & 9 & 1 & YES & YES & YES & -- & 1539\\
$(c; 0, 1, 1; 5)$ & 6 & $(13, 5)$ & 5 & 1 & YES & YES & YES & -- & 1540\\
$(c; 0, 1, 1; 5)$ & 6 & $(17, 5)$ & 6 & 1 & YES & YES & YES & -- & 1541\\
$(c; 0, 2, 0; 7)$ & 6 & $(5, 2)$ & 3 & 1 & YES & YES & YES & -- & 1542\\
$(c; 0, 2, 0; 7)$ & 6 & $(7, 3)$ & 4 & 7 & YES & YES & YES & -- & 1543\\
$(c; 0, 2, 0; 7)$ & 6 & $(8, 3)$ & 4 & 1 & YES & YES & YES & -- & 1544\\
$(c; 0, 2, 0; 7)$ & 6 & $(9, 2)$ & 5 & 1 & YES & YES & YES & -- & 1545\\
$(c; 0, 2, 0; 7)$ & 6 & $(9, 4)$ & 5 & 1 & YES & YES & YES & -- & 1546\\
$(c; 0, 2, 0; 7)$ & 6 & $(13, 4)$ & 6 & 1 & YES & YES & YES & -- & 1547\\
$(c; 0, 2, 0; 7)$ & 6 & $(13, 5)$ & 5 & 1 & YES & YES & YES & -- & 1548\\
$(c; 0, 2, 0; 7)$ & 6 & $(15, 4)$ & 6 & 1 & YES & YES & YES & -- & 1549\\
$(c; 0, 2, 0; 7)$ & 6 & $(17, 4)$ & 7 & 1 & YES & YES & YES & -- & 1550\\
$(c; 0, 2, 0; 7)$ & 6 & $(17, 5)$ & 6 & 1 & YES & YES & YES & -- & 1551\\
$(c; 0, 2, 0; 7)$ & 6 & $(18, 5)$ & 6 & 1 & YES & YES & YES & -- & 1552\\
$(c; 0, 2, 0; 7)$ & 6 & $(22, 5)$ & 7 & 1 & YES & YES & YES & -- & 1553\\
$(c; 0, 2, 1; 19)$ & 7 & $(3, 1)$ & 2 & 1 & YES & YES & YES & -- & 1554\\
$(c; 0, 2, 1; 19)$ & 7 & $(4, 1)$ & 3 & 1 & YES & YES & YES & -- & 1555\\
$(c; 0, 2, 1; 19)$ & 7 & $(9, 2)$ & 5 & 1 & YES & YES & YES & -- & 1556\\
$(c; 0, 2, 1; 19)$ & 7 & $(10, 3)$ & 5 & 1 & YES & YES & YES & -- & 1557\\
$(c; 0, 2, 1; 19)$ & 7 & $(17, 4)$ & 7 & 1 & YES & YES & YES & -- & 1558\\
$(c; 0, 2, 2; 6)$ & 8 & $(3, 1)$ & 2 & 3 & YES & YES & YES & -- & 1559\\
$(c; 0, 2, 2; 6)$ & 8 & $(5, 1)$ & 4 & 1 & YES & YES & YES & -- & 1560\\
$(c; 0, 2, 2; 6)$ & 8 & $(7, 2)$ & 4 & 1 & YES & YES & YES & -- & 1561\\
$(c; 0, 3, 1; 23)$ & 8 & $(2, 1)$ & 1 & 1 & YES & YES & YES & -- & 1562\\
$(c; 0, 3, 1; 23)$ & 8 & $(3, 1)$ & 2 & 1 & YES & YES & YES & -- & 1563\\
$(c; 0, 3, 1; 23)$ & 8 & $(5, 1)$ & 4 & 1 & YES & YES & YES & -- & 1564\\
$(c; 0, 3, 1; 23)$ & 8 & $(6, 1)$ & 5 & 1 & YES & YES & YES & -- & 1565\\
$(c; 0, 3, 2; 29)$ & 9 & $(3, 1)$ & 2 & 1 & YES & YES & YES & -- & 1566\\
$(c; 0, 3, 2; 29)$ & 9 & $(6, 1)$ & 5 & 1 & YES & YES & YES & -- & 1567\\
$(d; 0, 0, 0; 5)$ & 5 & $(9, 4)$ & 5 & 1 & YES & YES & YES & -- & 1568\\
$(d; 0, 0, 0; 5)$ & 5 & $(13, 5)$ & 5 & 1 & YES & YES & YES & -- & 1569\\
$(d; 0, 0, 0; 5)$ & 5 & $(17, 5)$ & 6 & 1 & YES & YES & YES & -- & 1570\\
$(d; 0, 0, 0; 5)$ & 5 & $(17, 7)$ & 6 & 1 & YES & YES & YES & -- & 1571\\
$(d; 0, 0, 0; 5)$ & 5 & $(18, 7)$ & 6 & 1 & YES & YES & YES & -- & 1572\\
$(d; 0, 0, 0; 5)$ & 5 & $(19, 8)$ & 6 & 1 & YES & YES & YES & -- & 1573\\
$(d; 0, 0, 0; 5)$ & 5 & $(21, 8)$ & 6 & 1 & YES & YES & YES & -- & 1574\\
$(d; 0, 0, 0; 5)$ & 5 & $(24, 7)$ & 7 & 1 & YES & YES & YES & -- & 1575\\
$(d; 0, 0, 0; 5)$ & 5 & $(29, 8)$ & 7 & 1 & YES & YES & YES & -- & 1576\\
$(d; 0, 0, 1; 14)$ & 6 & $(12, 5)$ & 5 & 2 & YES & YES & YES & -- & 1577\\
$(d; 0, 0, 1; 14)$ & 6 & $(13, 4)$ & 6 & 1 & YES & YES & YES & -- & 1578\\
$(d; 0, 0, 1; 14)$ & 6 & $(13, 5)$ & 5 & 1 & YES & YES & YES & -- & 1579\\
$(d; 0, 0, 2; 9)$ & 7 & $(3, 1)$ & 2 & 3 & YES & YES & YES & -- & 1580\\
$(d; 0, 0, 3; 22)$ & 8 & $(2, 1)$ & 1 & 2 & YES & YES & YES & -- & 1581\\
$(d; 0, 0, 3; 22)$ & 8 & $(6, 1)$ & 5 & 2 & YES & YES & YES & -- & 1582\\
$(d; 0, 1, 0; 6)$ & 6 & $(9, 2)$ & 5 & 3 & YES & YES & YES & -- & 1583\\
$(d; 0, 1, 0; 6)$ & 6 & $(12, 5)$ & 5 & 6 & YES & YES & YES & -- & 1584\\
$(d; 0, 1, 0; 6)$ & 6 & $(13, 4)$ & 6 & 1 & YES & YES & YES & -- & 1585\\
$(d; 0, 1, 0; 6)$ & 6 & $(13, 5)$ & 5 & 1 & YES & YES & YES & -- & 1586\\
$(d; 0, 1, 0; 6)$ & 6 & $(15, 4)$ & 6 & 3 & YES & YES & YES & -- & 1587\\
$(d; 0, 1, 2; 11)$ & 8 & $(2, 1)$ & 1 & 1 & YES & YES & YES & -- & 1588\\
$(d; 0, 1, 2; 11)$ & 8 & $(5, 1)$ & 4 & 1 & YES & YES & YES & -- & 1589\\
$(d; 0, 1, 2; 11)$ & 8 & $(7, 2)$ & 4 & 1 & YES & YES & YES & -- & 1590\\
$(d; 0, 1, 3; 27)$ & 9 & $(2, 1)$ & 1 & 1 & YES & YES & YES & -- & 1591\\
$(d; 0, 2, 2; 13)$ & 9 & $(2, 1)$ & 1 & 1 & YES & YES & YES & -- & 1592\\
$(d; 0, 2, 2; 13)$ & 9 & $(6, 1)$ & 5 & 1 & YES & YES & YES & -- & 1593\\
$(e; 0, 0, 0; 4)$ & 5 & $(7, 3)$ & 4 & 1 & YES & YES & YES & -- & 1594\\
$(e; 0, 0, 0; 4)$ & 5 & $(10, 3)$ & 5 & 2 & YES & YES & YES & -- & 1595\\
$(e; 0, 0, 0; 4)$ & 5 & $(17, 5)$ & 6 & 1 & YES & YES & YES & -- & 1596\\
$(e; 0, 1, 0; 5)$ & 6 & $(3, 1)$ & 2 & 1 & YES & YES & YES & -- & 1597\\
$(e; 0, 3, 0; 7)$ & 8 & $(2, 1)$ & 1 & 1 & YES & YES & YES & -- & 1598\\
$(e; 0, 3, 0; 7)$ & 8 & $(6, 1)$ & 5 & 1 & YES & YES & YES & -- & 1599\\
$(e; 0, 3, 0; 7)$ & 8 & $(11, 2)$ & 6 & 1 & YES & YES & YES & -- & 1600\\
$(e; 1, 0, 0; 18)$ & 6 & $(7, 3)$ & 4 & 1 & YES & YES & YES & -- & 1601\\
$(e; 1, 0, 0; 18)$ & 6 & $(8, 3)$ & 4 & 2 & YES & YES & YES & -- & 1602\\
$(e; 1, 0, 0; 18)$ & 6 & $(10, 3)$ & 5 & 2 & YES & YES & YES & -- & 1603\\
$(e; 1, 1, 0; 23)$ & 7 & $(5, 2)$ & 3 & 1 & YES & YES & YES & -- & 1604\\
$(e; 1, 1, 0; 23)$ & 7 & $(7, 3)$ & 4 & 1 & YES & YES & YES & -- & 1605\\
$(e; 2, 0, 0; 24)$ & 7 & $(2, 1)$ & 1 & 2 & YES & YES & NO(2) & -- & 1606\\
$(f; 0, 0, 0; 6)$ & 4 & $(11, 4)$ & 5 & 1 & YES & YES & NO(2) & -- & 1607\\
$(f; 0, 0, 0; 6)$ & 4 & $(12, 5)$ & 5 & 6 & YES & YES & YES & -- & 1608\\
$(f; 0, 0, 0; 6)$ & 4 & $(13, 4)$ & 6 & 1 & YES & YES & YES & -- & 1609\\
$(f; 0, 0, 0; 6)$ & 4 & $(16, 5)$ & 7 & 2 & YES & YES & YES & -- & 1610\\
$(f; 0, 0, 0; 6)$ & 4 & $(18, 7)$ & 6 & 6 & YES & YES & YES & -- & 1611\\
$(f; 0, 0, 0; 6)$ & 4 & $(27, 10)$ & 7 & 3 & YES & YES & YES & -- & 1612\\
$(f; 0, 0, 0; 6)$ & 4 & $(29, 11)$ & 7 & 1 & YES & YES & YES & -- & 1613\\
$(f; 0, 0, 0; 6)$ & 4 & $(40, 11)$ & 8 & 2 & YES & YES & YES & -- & 1614\\
$(f; 0, 0, 0; 6)$ & 4 & $(44, 17)$ & 8 & 2 & YES & YES & YES & -- & 1615\\
$(f; 0, 1, 0; 7)$ & 5 & $(10, 3)$ & 5 & 1 & YES & YES & YES & -- & 1616\\
$(g; 0, 0, 0; 19)$ & 6 & $(7, 3)$ & 4 & 1 & YES & YES & YES & -- & 1617\\
$(g; 0, 0, 0; 19)$ & 6 & $(8, 3)$ & 4 & 1 & YES & YES & YES & -- & 1618\\
$(g; 0, 0, 0; 19)$ & 6 & $(13, 4)$ & 6 & 1 & YES & YES & YES & -- & 1619\\
$(g; 0, 0, 1; 26)$ & 7 & $(5, 2)$ & 3 & 1 & YES & YES & YES & -- & 1620\\
$(g; 0, 0, 2; 11)$ & 8 & $(2, 1)$ & 1 & 1 & YES & YES & YES & -- & 1621\\
$(g; 0, 0, 2; 11)$ & 8 & $(3, 1)$ & 2 & 1 & YES & YES & YES & -- & 1622\\
$(g; 0, 0, 2; 11)$ & 8 & $(5, 1)$ & 4 & 1 & YES & YES & YES & -- & 1623\\
$(g; 0, 0, 2; 11)$ & 8 & $(11, 2)$ & 6 & 11 & YES & YES & YES & -- & 1624\\
$(g; 0, 1, 0; 24)$ & 7 & $(5, 2)$ & 3 & 1 & YES & YES & YES & -- & 1625\\
$(g; 0, 1, 0; 24)$ & 7 & $(13, 3)$ & 6 & 1 & YES & YES & YES & -- & 1626\\
$(g; 0, 2, 0; 29)$ & 8 & $(2, 1)$ & 1 & 1 & YES & YES & YES & -- & 1627\\
$(g; 0, 2, 0; 29)$ & 8 & $(5, 1)$ & 4 & 1 & YES & YES & YES & -- & 1628\\
$(g; 1, 0, 0; 7)$ & 7 & $(5, 2)$ & 3 & 1 & YES & YES & YES & -- & 1629\\
$(g; 1, 0, 1; 38)$ & 8 & $(2, 1)$ & 1 & 2 & YES & YES & YES & -- & 1630\\
$(g; 1, 0, 1; 38)$ & 8 & $(4, 1)$ & 3 & 2 & YES & YES & YES & -- & 1631\\
$(g; 1, 1, 0; 9)$ & 8 & $(2, 1)$ & 1 & 1 & YES & YES & YES & -- & 1632\\
$(h; 0, 0, 0; 6)$ & 5 & $(8, 3)$ & 4 & 2 & YES & YES & YES & -- & 1633\\
$(h; 0, 0, 0; 6)$ & 5 & $(10, 3)$ & 5 & 2 & YES & YES & YES & -- & 1634\\
$(h; 0, 0, 0; 6)$ & 5 & $(12, 5)$ & 5 & 6 & YES & YES & YES & -- & 1635\\
$(h; 0, 0, 0; 6)$ & 5 & $(13, 5)$ & 5 & 1 & YES & YES & YES & -- & 1636\\
$(h; 0, 1, 0; 8)$ & 6 & $(7, 3)$ & 4 & 1 & YES & YES & YES & -- & 1637\\
$(h; 0, 1, 0; 8)$ & 6 & $(13, 4)$ & 6 & 1 & YES & YES & YES & -- & 1638\\
$(i; 0, 0, 0; 9)$ & 5 & $(5, 2)$ & 3 & 1 & YES & YES & NO(2) & -- & 1639\\
$(i; 0, 0, 0; 9)$ & 5 & $(7, 2)$ & 4 & 1 & YES & YES & YES & -- & 1640\\
$(i; 0, 0, 0; 9)$ & 5 & $(8, 3)$ & 4 & 1 & YES & YES & YES & -- & 1641\\
$(i; 0, 0, 0; 9)$ & 5 & $(10, 3)$ & 5 & 1 & YES & YES & YES & -- & 1642\\
$(i; 0, 0, 0; 9)$ & 5 & $(17, 5)$ & 6 & 1 & YES & YES & YES & -- & 1643\\
$(i; 0, 0, 0; 9)$ & 5 & $(18, 5)$ & 6 & 9 & YES & YES & YES & -- & 1644\\
$(i; 0, 0, 0; 9)$ & 5 & $(19, 4)$ & 7 & 1 & YES & YES & YES & -- & 1645\\
$(i; 0, 1, 0; 12)$ & 6 & $(4, 1)$ & 3 & 4 & YES & YES & YES & -- & 1646\\
$(i; 0, 1, 0; 12)$ & 6 & $(7, 3)$ & 4 & 1 & YES & YES & YES & -- & 1647\\
$(i; 0, 1, 0; 12)$ & 6 & $(10, 3)$ & 5 & 2 & YES & YES & YES & -- & 1648\\
$(i; 0, 1, 0; 12)$ & 6 & $(11, 3)$ & 5 & 1 & YES & YES & YES & -- & 1649\\
$(i; 0, 2, 0; 15)$ & 7 & $(3, 1)$ & 2 & 3 & YES & YES & YES & -- & 1650\\
$(i; 0, 2, 0; 15)$ & 7 & $(13, 3)$ & 6 & 1 & YES & YES & YES & -- & 1651\\
$(j; 0, 0, 0; 8)$ & 5 & $(8, 3)$ & 4 & 8 & YES & YES & YES & -- & 1652\\
$(j; 0, 0, 0; 8)$ & 5 & $(9, 4)$ & 5 & 1 & YES & YES & YES & -- & 1653\\
$(j; 0, 0, 0; 8)$ & 5 & $(10, 3)$ & 5 & 2 & YES & YES & YES & -- & 1654\\
$(j; 0, 0, 0; 8)$ & 5 & $(11, 4)$ & 5 & 1 & YES & YES & YES & -- & 1655\\
$(j; 0, 0, 0; 8)$ & 5 & $(17, 7)$ & 6 & 1 & YES & YES & YES & -- & 1656\\
$(j; 0, 0, 0; 8)$ & 5 & $(23, 7)$ & 7 & 1 & YES & YES & YES & -- & 1657\\
$(j; 0, 0, 0; 8)$ & 5 & $(24, 7)$ & 7 & 8 & YES & YES & YES & -- & 1658\\
$(j; 0, 1, 0; 10)$ & 6 & $(9, 4)$ & 5 & 1 & YES & YES & YES & -- & 1659\\
$(j; 0, 1, 0; 10)$ & 6 & $(11, 4)$ & 5 & 1 & YES & YES & YES & -- & 1660\\
$(j; 0, 1, 0; 10)$ & 6 & $(18, 7)$ & 6 & 2 & YES & YES & YES & -- & 1661
\end{longtable}
\subsection{2 chains, $K^2 = 3$}
\begin{longtable}{|c|c|c|c|c|c|c|c|c|c|}
\hline
\multicolumn{10}{|c|}{2 chains, $K^2 = 3$}\\
\hline
$(n,a)$ & Length & $(n,a)$ & Length & GCD & Nef & $\mathbb Q$-ef & Obstruction 0 & WH & Index\\
\hline
\endfirsthead

\hline
$(n,a)$ & Length & $(n,a)$ & Length & GCD & Nef & $\mathbb Q$-ef & Obstruction 0 & WH & Index\\
\hline
\endhead
\hline
\endfoot

$(16, 7)$ & 6 & $(14, 5)$ & 6 & 2 & YES & YES & YES & -- & 1662\\
$(18, 7)$ & 6 & $(11, 4)$ & 5 & 1 & YES & YES & NO(2) & NO & 1663\\
$(19, 8)$ & 6 & $(12, 5)$ & 5 & 1 & YES & YES & YES & -- & 1664\\
$(22, 9)$ & 7 & $(11, 3)$ & 5 & 11 & YES & YES & NO(2) & -- & 1665\\
$(22, 9)$ & 7 & $(11, 3)$ & 5 & 11 & YES & YES & NO(2) & NO & 1666\\
$(23, 9)$ & 7 & $(16, 5)$ & 7 & 1 & YES & YES & YES & -- & 1667\\
$(23, 9)$ & 7 & $(16, 5)$ & 7 & 1 & YES & YES & YES & NO & 1668\\
$(23, 10)$ & 7 & $(18, 7)$ & 6 & 1 & YES & YES & NO(2) & NO & 1669\\
$(25, 9)$ & 7 & $(21, 5)$ & 8 & 1 & YES & YES & YES & -- & 1670\\
$(25, 9)$ & 7 & $(21, 5)$ & 8 & 1 & YES & YES & YES & NO & 1671\\
$(25, 7)$ & 7 & $(23, 10)$ & 7 & 1 & YES & YES & NO(2) & -- & 1672\\
$(26, 11)$ & 7 & $(7, 3)$ & 4 & 1 & YES & YES & NO(2) & -- & 1673\\
$(26, 11)$ & 7 & $(9, 4)$ & 5 & 1 & YES & YES & NO(2) & -- & 1674\\
$(26, 11)$ & 7 & $(24, 5)$ & 8 & 2 & YES & YES & YES & -- & 1675\\
$(27, 8)$ & 7 & $(10, 3)$ & 5 & 1 & YES & YES & NO(2) & -- & 1676\\
$(27, 10)$ & 7 & $(11, 5)$ & 6 & 1 & YES & YES & YES & -- & 1677\\
$(27, 11)$ & 8 & $(13, 4)$ & 6 & 1 & YES & YES & YES & -- & 1678\\
$(27, 11)$ & 8 & $(13, 4)$ & 6 & 1 & YES & YES & YES & NO & 1679\\
$(27, 8)$ & 7 & $(21, 8)$ & 6 & 3 & YES & YES & YES & -- & 1680\\
$(27, 8)$ & 7 & $(21, 8)$ & 6 & 3 & YES & YES & YES & NO & 1681\\
$(27, 10)$ & 7 & $(21, 8)$ & 6 & 3 & YES & YES & YES & -- & 1682\\
$(28, 11)$ & 8 & $(27, 8)$ & 7 & 1 & YES & YES & YES & -- & 1683\\
$(29, 13)$ & 8 & $(14, 5)$ & 6 & 1 & YES & YES & YES & NO & 1684\\
$(29, 12)$ & 7 & $(16, 5)$ & 7 & 1 & YES & YES & YES & -- & 1685\\
$(29, 12)$ & 7 & $(17, 5)$ & 6 & 1 & YES & YES & YES & -- & 1686\\
$(29, 8)$ & 7 & $(21, 8)$ & 6 & 1 & YES & YES & YES & -- & 1687\\
$(29, 8)$ & 7 & $(24, 7)$ & 7 & 1 & YES & YES & YES & -- & 1688\\
$(29, 8)$ & 7 & $(24, 7)$ & 7 & 1 & YES & YES & YES & NO & 1689\\
$(29, 12)$ & 7 & $(27, 10)$ & 7 & 1 & YES & YES & YES & -- & 1690\\
$(29, 8)$ & 7 & $(28, 11)$ & 8 & 1 & YES & YES & YES & -- & 1691\\
$(29, 12)$ & 7 & $(29, 8)$ & 7 & 29 & YES & YES & YES & -- & 1692\\
$(29, 12)$ & 7 & $(29, 11)$ & 7 & 29 & YES & YES & YES & -- & 1693\\
$(30, 13)$ & 8 & $(9, 4)$ & 5 & 3 & YES & YES & NO(2) & -- & 1694\\
$(30, 11)$ & 7 & $(25, 7)$ & 7 & 5 & YES & YES & YES & -- & 1695\\
$(30, 11)$ & 7 & $(25, 7)$ & 7 & 5 & YES & YES & YES & NO & 1696\\
$(31, 7)$ & 8 & $(13, 4)$ & 6 & 1 & YES & YES & YES & -- & 1697\\
$(31, 9)$ & 8 & $(17, 4)$ & 7 & 1 & YES & YES & YES & -- & 1698\\
$(31, 9)$ & 8 & $(17, 4)$ & 7 & 1 & YES & YES & YES & NO & 1699\\
$(31, 9)$ & 8 & $(24, 7)$ & 7 & 1 & YES & YES & YES & -- & 1700\\
$(31, 13)$ & 7 & $(24, 7)$ & 7 & 1 & YES & YES & YES & -- & 1701\\
$(31, 13)$ & 7 & $(25, 7)$ & 7 & 1 & YES & YES & YES & -- & 1702\\
$(31, 7)$ & 8 & $(26, 11)$ & 7 & 1 & YES & YES & YES & -- & 1703\\
$(31, 12)$ & 7 & $(26, 11)$ & 7 & 1 & YES & YES & YES & -- & 1704\\
$(31, 12)$ & 7 & $(27, 8)$ & 7 & 1 & YES & YES & YES & NO & 1705\\
$(31, 12)$ & 7 & $(28, 11)$ & 8 & 1 & YES & YES & YES & -- & 1706\\
$(31, 9)$ & 8 & $(29, 11)$ & 7 & 1 & YES & YES & YES & -- & 1707\\
$(32, 9)$ & 8 & $(24, 7)$ & 7 & 8 & YES & YES & YES & -- & 1708\\
$(33, 10)$ & 8 & $(9, 4)$ & 5 & 3 & YES & YES & YES & -- & 1709\\
$(33, 10)$ & 8 & $(9, 4)$ & 5 & 3 & YES & YES & YES & NO & 1710\\
$(33, 14)$ & 8 & $(23, 4)$ & 8 & 1 & YES & YES & YES & -- & 1711\\
$(33, 10)$ & 8 & $(24, 7)$ & 7 & 3 & YES & YES & YES & -- & 1712\\
$(33, 10)$ & 8 & $(25, 7)$ & 7 & 1 & YES & YES & YES & NO & 1713\\
$(33, 10)$ & 8 & $(31, 12)$ & 7 & 1 & YES & YES & YES & -- & 1714\\
$(34, 13)$ & 7 & $(12, 5)$ & 5 & 2 & YES & YES & YES & -- & 1715\\
$(34, 13)$ & 7 & $(17, 5)$ & 6 & 17 & YES & YES & YES & -- & 1716\\
$(34, 13)$ & 7 & $(17, 5)$ & 6 & 17 & YES & YES & YES & NO & 1717\\
$(34, 13)$ & 7 & $(23, 9)$ & 7 & 1 & YES & YES & YES & -- & 1718\\
$(34, 13)$ & 7 & $(25, 7)$ & 7 & 1 & YES & YES & YES & -- & 1719\\
$(34, 13)$ & 7 & $(25, 7)$ & 7 & 1 & YES & YES & YES & NO & 1720\\
$(34, 13)$ & 7 & $(31, 12)$ & 7 & 1 & YES & YES & YES & -- & 1721\\
$(34, 15)$ & 8 & $(32, 7)$ & 8 & 2 & YES & YES & YES & NO & 1722\\
$(35, 13)$ & 8 & $(9, 4)$ & 5 & 1 & YES & YES & NO(2) & NO & 1723\\
$(35, 13)$ & 8 & $(31, 7)$ & 8 & 1 & YES & YES & YES & -- & 1724\\
$(35, 13)$ & 8 & $(35, 8)$ & 8 & 35 & YES & YES & YES & -- & 1725\\
$(36, 13)$ & 8 & $(9, 4)$ & 5 & 9 & YES & YES & YES & -- & 1726\\
$(36, 13)$ & 8 & $(9, 4)$ & 5 & 9 & YES & YES & YES & NO & 1727\\
$(36, 13)$ & 8 & $(11, 5)$ & 6 & 1 & YES & YES & YES & NO & 1728\\
$(36, 11)$ & 8 & $(24, 7)$ & 7 & 12 & YES & YES & YES & -- & 1729\\
$(36, 11)$ & 8 & $(24, 7)$ & 7 & 12 & YES & YES & YES & NO & 1730\\
$(36, 11)$ & 8 & $(25, 7)$ & 7 & 1 & YES & YES & YES & -- & 1731\\
$(36, 11)$ & 8 & $(25, 7)$ & 7 & 1 & YES & YES & YES & NO & 1732\\
$(37, 11)$ & 8 & $(7, 3)$ & 4 & 1 & YES & YES & NO(2) & NO & 1733\\
$(37, 14)$ & 8 & $(9, 4)$ & 5 & 1 & YES & YES & NO(2) & NO & 1734\\
$(37, 11)$ & 8 & $(17, 7)$ & 6 & 1 & YES & YES & YES & -- & 1735\\
$(37, 14)$ & 8 & $(17, 5)$ & 6 & 1 & YES & YES & YES & -- & 1736\\
$(37, 14)$ & 8 & $(17, 5)$ & 6 & 1 & YES & YES & YES & NO & 1737\\
$(37, 14)$ & 8 & $(31, 7)$ & 8 & 1 & YES & YES & YES & -- & 1738\\
$(37, 14)$ & 8 & $(32, 7)$ & 8 & 1 & YES & YES & YES & -- & 1739\\
$(37, 14)$ & 8 & $(32, 7)$ & 8 & 1 & YES & YES & YES & NO & 1740\\
$(37, 8)$ & 8 & $(35, 13)$ & 8 & 1 & YES & YES & YES & NO & 1741\\
$(38, 9)$ & 9 & $(11, 4)$ & 5 & 1 & YES & YES & YES & -- & 1742\\
$(38, 9)$ & 9 & $(11, 4)$ & 5 & 1 & YES & YES & YES & NO & 1743\\
$(39, 14)$ & 8 & $(5, 2)$ & 3 & 1 & YES & YES & YES & -- & 1744\\
$(39, 16)$ & 8 & $(17, 5)$ & 6 & 1 & YES & YES & YES & -- & 1745\\
$(39, 16)$ & 8 & $(21, 5)$ & 8 & 3 & YES & YES & YES & NO & 1746\\
$(39, 16)$ & 8 & $(21, 8)$ & 6 & 3 & YES & YES & YES & -- & 1747\\
$(39, 14)$ & 8 & $(24, 7)$ & 7 & 3 & YES & YES & YES & -- & 1748\\
$(39, 14)$ & 8 & $(31, 12)$ & 7 & 1 & YES & YES & YES & 1917 & 1749\\
$(39, 7)$ & 9 & $(38, 11)$ & 9 & 1 & YES & YES & YES & NO & 1750\\
$(39, 11)$ & 9 & $(38, 7)$ & 9 & 1 & YES & YES & YES & NO & 1751\\
$(40, 11)$ & 8 & $(17, 4)$ & 7 & 1 & YES & YES & YES & -- & 1752\\
$(40, 11)$ & 8 & $(17, 5)$ & 6 & 1 & YES & YES & YES & -- & 1753\\
$(40, 9)$ & 9 & $(18, 7)$ & 6 & 2 & YES & YES & YES & -- & 1754\\
$(40, 9)$ & 9 & $(18, 7)$ & 6 & 2 & YES & YES & YES & NO & 1755\\
$(40, 9)$ & 9 & $(21, 8)$ & 6 & 1 & YES & YES & YES & -- & 1756\\
$(40, 9)$ & 9 & $(21, 8)$ & 6 & 1 & YES & YES & YES & NO & 1757\\
$(40, 11)$ & 8 & $(23, 10)$ & 7 & 1 & YES & YES & YES & -- & 1758\\
$(40, 11)$ & 8 & $(23, 10)$ & 7 & 1 & YES & YES & YES & NO & 1759\\
$(40, 9)$ & 9 & $(24, 7)$ & 7 & 8 & YES & YES & YES & -- & 1760\\
$(40, 11)$ & 8 & $(27, 10)$ & 7 & 1 & YES & YES & YES & -- & 1761\\
$(40, 11)$ & 8 & $(31, 9)$ & 8 & 1 & YES & YES & YES & -- & 1762\\
$(40, 11)$ & 8 & $(32, 9)$ & 8 & 8 & YES & YES & YES & -- & 1763\\
$(41, 16)$ & 8 & $(9, 4)$ & 5 & 1 & YES & YES & YES & -- & 1764\\
$(41, 17)$ & 8 & $(17, 5)$ & 6 & 1 & YES & YES & YES & -- & 1765\\
$(41, 16)$ & 8 & $(21, 8)$ & 6 & 1 & YES & YES & YES & -- & 1766\\
$(41, 17)$ & 8 & $(22, 5)$ & 7 & 1 & YES & YES & YES & NO & 1767\\
$(41, 12)$ & 8 & $(23, 9)$ & 7 & 1 & YES & YES & YES & -- & 1768\\
$(41, 12)$ & 8 & $(29, 12)$ & 7 & 1 & YES & YES & YES & -- & 1769\\
$(41, 17)$ & 8 & $(29, 8)$ & 7 & 1 & YES & YES & YES & -- & 1770\\
$(41, 17)$ & 8 & $(29, 8)$ & 7 & 1 & YES & YES & YES & NO & 1771\\
$(41, 12)$ & 8 & $(31, 12)$ & 7 & 1 & YES & YES & YES & -- & 1772\\
$(41, 17)$ & 8 & $(31, 7)$ & 8 & 1 & YES & YES & YES & NO & 1773\\
$(43, 18)$ & 8 & $(15, 4)$ & 6 & 1 & YES & YES & YES & -- & 1774\\
$(43, 18)$ & 8 & $(17, 4)$ & 7 & 1 & YES & YES & YES & -- & 1775\\
$(43, 12)$ & 8 & $(18, 5)$ & 6 & 1 & YES & YES & YES & -- & 1776\\
$(43, 12)$ & 8 & $(18, 5)$ & 6 & 1 & YES & YES & YES & NO & 1777\\
$(43, 12)$ & 8 & $(21, 8)$ & 6 & 1 & YES & YES & YES & -- & 1778\\
$(43, 12)$ & 8 & $(21, 8)$ & 6 & 1 & YES & YES & YES & NO & 1779\\
$(43, 13)$ & 9 & $(21, 8)$ & 6 & 1 & YES & YES & YES & -- & 1780\\
$(43, 16)$ & 9 & $(25, 9)$ & 7 & 1 & YES & YES & YES & NO & 1781\\
$(43, 10)$ & 9 & $(26, 11)$ & 7 & 1 & YES & YES & YES & NO & 1782\\
$(43, 13)$ & 9 & $(37, 8)$ & 8 & 1 & YES & YES & YES & NO & 1783\\
$(44, 13)$ & 8 & $(13, 5)$ & 5 & 1 & YES & YES & YES & -- & 1784\\
$(44, 13)$ & 8 & $(13, 5)$ & 5 & 1 & YES & YES & YES & NO & 1785\\
$(44, 17)$ & 8 & $(19, 5)$ & 7 & 1 & YES & YES & YES & -- & 1786\\
$(44, 17)$ & 8 & $(19, 5)$ & 7 & 1 & YES & YES & YES & NO & 1787\\
$(44, 17)$ & 8 & $(21, 5)$ & 8 & 1 & YES & YES & YES & -- & 1788\\
$(44, 17)$ & 8 & $(21, 5)$ & 8 & 1 & YES & YES & YES & NO & 1789\\
$(44, 13)$ & 8 & $(24, 7)$ & 7 & 4 & YES & YES & YES & -- & 1790\\
$(44, 13)$ & 8 & $(43, 10)$ & 9 & 1 & YES & YES & YES & NO & 1791\\
$(45, 17)$ & 9 & $(6, 1)$ & 5 & 3 & YES & YES & YES & -- & 1792\\
$(45, 17)$ & 9 & $(7, 3)$ & 4 & 1 & YES & YES & YES & NO & 1793\\
$(45, 17)$ & 9 & $(12, 5)$ & 5 & 3 & YES & YES & YES & -- & 1794\\
$(45, 19)$ & 8 & $(29, 11)$ & 7 & 1 & YES & YES & YES & NO & 1795\\
$(46, 17)$ & 8 & $(17, 7)$ & 6 & 1 & YES & YES & YES & -- & 1796\\
$(46, 19)$ & 8 & $(24, 7)$ & 7 & 2 & YES & YES & YES & -- & 1797\\
$(46, 17)$ & 8 & $(26, 11)$ & 7 & 2 & YES & YES & YES & NO & 1798\\
$(46, 17)$ & 8 & $(31, 13)$ & 7 & 1 & YES & YES & YES & NO & 1799\\
$(46, 17)$ & 8 & $(44, 17)$ & 8 & 2 & YES & YES & YES & NO & 1800\\
$(47, 13)$ & 8 & $(17, 4)$ & 7 & 1 & YES & YES & YES & -- & 1801\\
$(47, 13)$ & 8 & $(17, 7)$ & 6 & 1 & YES & YES & YES & NO & 1802\\
$(47, 18)$ & 8 & $(17, 5)$ & 6 & 1 & YES & YES & YES & -- & 1803\\
$(47, 18)$ & 8 & $(18, 5)$ & 6 & 1 & YES & YES & YES & -- & 1804\\
$(47, 18)$ & 8 & $(18, 7)$ & 6 & 1 & YES & YES & YES & -- & 1805\\
$(47, 13)$ & 8 & $(21, 8)$ & 6 & 1 & YES & YES & YES & -- & 1806\\
$(47, 13)$ & 8 & $(23, 9)$ & 7 & 1 & YES & YES & YES & -- & 1807\\
$(47, 13)$ & 8 & $(23, 9)$ & 7 & 1 & YES & YES & YES & NO & 1808\\
$(47, 10)$ & 9 & $(31, 9)$ & 8 & 1 & YES & YES & YES & NO & 1809\\
$(47, 14)$ & 9 & $(38, 7)$ & 9 & 1 & YES & YES & YES & NO & 1810\\
$(48, 13)$ & 9 & $(11, 3)$ & 5 & 1 & YES & YES & YES & -- & 1811\\
$(48, 11)$ & 9 & $(27, 10)$ & 7 & 3 & YES & YES & YES & -- & 1812\\
$(48, 13)$ & 9 & $(32, 7)$ & 8 & 16 & YES & YES & YES & -- & 1813\\
$(48, 11)$ & 9 & $(41, 11)$ & 8 & 1 & YES & YES & YES & -- & 1814\\
$(49, 15)$ & 9 & $(5, 2)$ & 3 & 1 & YES & YES & NO(2) & -- & 1815\\
$(49, 15)$ & 9 & $(7, 2)$ & 4 & 7 & YES & YES & YES & -- & 1816\\
$(49, 15)$ & 9 & $(7, 2)$ & 4 & 7 & YES & YES & YES & NO & 1817\\
$(49, 15)$ & 9 & $(13, 5)$ & 5 & 1 & YES & YES & YES & -- & 1818\\
$(49, 18)$ & 8 & $(13, 5)$ & 5 & 1 & YES & YES & YES & -- & 1819\\
$(49, 19)$ & 8 & $(16, 5)$ & 7 & 1 & YES & YES & YES & -- & 1820\\
$(49, 19)$ & 8 & $(16, 5)$ & 7 & 1 & YES & YES & YES & NO & 1821\\
$(49, 19)$ & 8 & $(17, 6)$ & 7 & 1 & YES & YES & YES & -- & 1822\\
$(49, 19)$ & 8 & $(18, 7)$ & 6 & 1 & YES & YES & YES & -- & 1823\\
$(49, 19)$ & 8 & $(23, 7)$ & 7 & 1 & YES & YES & YES & -- & 1824\\
$(49, 19)$ & 8 & $(25, 7)$ & 7 & 1 & YES & YES & YES & -- & 1825\\
$(49, 18)$ & 8 & $(31, 12)$ & 7 & 1 & YES & YES & YES & NO & 1826\\
$(49, 13)$ & 9 & $(37, 11)$ & 8 & 1 & YES & YES & YES & 2074 & 1827\\
$(49, 15)$ & 9 & $(41, 12)$ & 8 & 1 & YES & YES & YES & NO & 1828\\
$(49, 13)$ & 9 & $(44, 13)$ & 8 & 1 & YES & YES & YES & NO & 1829\\
$(50, 21)$ & 8 & $(2, 1)$ & 1 & 2 & YES & YES & NO(2) & -- & 1830\\
$(50, 19)$ & 8 & $(12, 5)$ & 5 & 2 & YES & YES & YES & -- & 1831\\
$(50, 19)$ & 8 & $(17, 6)$ & 7 & 1 & YES & YES & YES & -- & 1832\\
$(50, 19)$ & 8 & $(18, 7)$ & 6 & 2 & YES & YES & YES & -- & 1833\\
$(50, 21)$ & 8 & $(18, 7)$ & 6 & 2 & YES & YES & YES & -- & 1834\\
$(50, 19)$ & 8 & $(24, 7)$ & 7 & 2 & YES & YES & YES & -- & 1835\\
$(51, 16)$ & 10 & $(5, 1)$ & 4 & 1 & YES & YES & YES & -- & 1836\\
$(51, 16)$ & 10 & $(5, 1)$ & 4 & 1 & YES & YES & YES & NO & 1837\\
$(51, 16)$ & 10 & $(5, 1)$ & 4 & 1 & YES & YES & YES & NO & 1838\\
$(51, 11)$ & 9 & $(22, 9)$ & 7 & 1 & YES & YES & YES & -- & 1839\\
$(51, 11)$ & 9 & $(23, 9)$ & 7 & 1 & YES & YES & YES & -- & 1840\\
$(51, 11)$ & 9 & $(23, 9)$ & 7 & 1 & YES & YES & YES & NO & 1841\\
$(52, 19)$ & 9 & $(16, 3)$ & 7 & 4 & YES & YES & YES & -- & 1842\\
$(53, 16)$ & 10 & $(9, 4)$ & 5 & 1 & YES & YES & YES & -- & 1843\\
$(53, 19)$ & 9 & $(18, 7)$ & 6 & 1 & YES & YES & YES & -- & 1844\\
$(53, 12)$ & 9 & $(21, 8)$ & 6 & 1 & YES & YES & YES & NO & 1845\\
$(53, 20)$ & 10 & $(49, 19)$ & 8 & 1 & YES & YES & YES & NO & 1846\\
$(55, 23)$ & 9 & $(9, 4)$ & 5 & 1 & YES & YES & YES & -- & 1847\\
$(55, 21)$ & 8 & $(10, 3)$ & 5 & 5 & YES & YES & YES & -- & 1848\\
$(55, 21)$ & 8 & $(11, 3)$ & 5 & 11 & YES & YES & YES & -- & 1849\\
$(55, 21)$ & 8 & $(13, 5)$ & 5 & 1 & YES & YES & YES & -- & 1850\\
$(55, 21)$ & 8 & $(17, 7)$ & 6 & 1 & YES & YES & YES & -- & 1851\\
$(55, 16)$ & 9 & $(18, 7)$ & 6 & 1 & YES & YES & YES & -- & 1852\\
$(55, 21)$ & 8 & $(18, 5)$ & 6 & 1 & YES & YES & YES & -- & 1853\\
$(55, 21)$ & 8 & $(18, 7)$ & 6 & 1 & YES & YES & YES & -- & 1854\\
$(55, 23)$ & 9 & $(18, 7)$ & 6 & 1 & YES & YES & YES & NO & 1855\\
$(55, 24)$ & 9 & $(18, 7)$ & 6 & 1 & YES & YES & YES & -- & 1856\\
$(55, 13)$ & 10 & $(21, 8)$ & 6 & 1 & YES & YES & YES & -- & 1857\\
$(55, 21)$ & 8 & $(25, 7)$ & 7 & 5 & YES & YES & YES & NO & 1858\\
$(56, 13)$ & 10 & $(18, 7)$ & 6 & 2 & YES & YES & YES & NO & 1859\\
$(56, 17)$ & 9 & $(29, 8)$ & 7 & 1 & YES & YES & YES & NO & 1860\\
$(56, 13)$ & 10 & $(51, 11)$ & 9 & 1 & YES & YES & YES & NO & 1861\\
$(57, 22)$ & 9 & $(18, 5)$ & 6 & 3 & YES & YES & YES & -- & 1862\\
$(57, 16)$ & 9 & $(19, 7)$ & 6 & 19 & YES & YES & YES & -- & 1863\\
$(57, 22)$ & 9 & $(23, 5)$ & 7 & 1 & YES & YES & YES & -- & 1864\\
$(57, 13)$ & 9 & $(30, 11)$ & 7 & 3 & YES & YES & YES & -- & 1865\\
$(58, 17)$ & 9 & $(11, 3)$ & 5 & 1 & YES & YES & YES & -- & 1866\\
$(58, 17)$ & 9 & $(13, 3)$ & 6 & 1 & YES & YES & YES & -- & 1867\\
$(58, 17)$ & 9 & $(17, 7)$ & 6 & 1 & YES & YES & YES & -- & 1868\\
$(58, 17)$ & 9 & $(19, 7)$ & 6 & 1 & YES & YES & YES & -- & 1869\\
$(58, 21)$ & 10 & $(39, 14)$ & 8 & 1 & YES & YES & YES & NO & 1870\\
$(58, 17)$ & 9 & $(40, 11)$ & 8 & 2 & YES & YES & YES & NO & 1871\\
$(59, 23)$ & 9 & $(12, 5)$ & 5 & 1 & YES & YES & YES & -- & 1872\\
$(59, 11)$ & 10 & $(32, 9)$ & 8 & 1 & YES & YES & YES & NO & 1873\\
$(59, 18)$ & 9 & $(40, 11)$ & 8 & 1 & YES & YES & YES & NO & 1874\\
$(59, 25)$ & 9 & $(55, 23)$ & 9 & 1 & YES & YES & YES & NO & 1875\\
$(60, 23)$ & 9 & $(4, 1)$ & 3 & 4 & YES & YES & YES & -- & 1876\\
$(60, 23)$ & 9 & $(10, 3)$ & 5 & 10 & YES & YES & YES & -- & 1877\\
$(60, 23)$ & 9 & $(13, 5)$ & 5 & 1 & YES & YES & YES & -- & 1878\\
$(60, 11)$ & 11 & $(14, 5)$ & 6 & 2 & YES & YES & YES & -- & 1879\\
$(60, 23)$ & 9 & $(18, 5)$ & 6 & 6 & YES & YES & YES & -- & 1880\\
$(60, 13)$ & 9 & $(23, 9)$ & 7 & 1 & YES & YES & YES & NO & 1881\\
$(60, 23)$ & 9 & $(27, 5)$ & 8 & 3 & YES & YES & YES & NO & 1882\\
$(60, 13)$ & 9 & $(31, 9)$ & 8 & 1 & YES & YES & YES & NO & 1883\\
$(61, 25)$ & 9 & $(2, 1)$ & 1 & 1 & YES & YES & NO(2) & -- & 1884\\
$(61, 25)$ & 9 & $(3, 1)$ & 2 & 1 & YES & YES & NO(2) & -- & 1885\\
$(61, 25)$ & 9 & $(3, 1)$ & 2 & 1 & YES & YES & NO(2) & NO & 1886\\
$(61, 25)$ & 9 & $(5, 1)$ & 4 & 1 & YES & YES & NO(2) & -- & 1887\\
$(61, 25)$ & 9 & $(7, 3)$ & 4 & 1 & YES & YES & YES & -- & 1888\\
$(61, 17)$ & 9 & $(9, 4)$ & 5 & 1 & YES & YES & NO(2) & -- & 1889\\
$(61, 18)$ & 9 & $(10, 3)$ & 5 & 1 & YES & YES & YES & -- & 1890\\
$(61, 18)$ & 9 & $(10, 3)$ & 5 & 1 & YES & YES & YES & NO & 1891\\
$(61, 22)$ & 9 & $(10, 3)$ & 5 & 1 & YES & YES & YES & -- & 1892\\
$(61, 25)$ & 9 & $(10, 3)$ & 5 & 1 & YES & YES & YES & -- & 1893\\
$(61, 25)$ & 9 & $(10, 3)$ & 5 & 1 & YES & YES & YES & NO & 1894\\
$(61, 17)$ & 9 & $(12, 5)$ & 5 & 1 & YES & YES & YES & -- & 1895\\
$(61, 17)$ & 9 & $(13, 4)$ & 6 & 1 & YES & YES & YES & -- & 1896\\
$(61, 18)$ & 9 & $(13, 5)$ & 5 & 1 & YES & YES & YES & -- & 1897\\
$(61, 25)$ & 9 & $(13, 4)$ & 6 & 1 & YES & YES & YES & -- & 1898\\
$(61, 17)$ & 9 & $(17, 7)$ & 6 & 1 & YES & YES & YES & -- & 1899\\
$(61, 18)$ & 9 & $(17, 5)$ & 6 & 1 & YES & YES & YES & -- & 1900\\
$(61, 17)$ & 9 & $(19, 8)$ & 6 & 1 & YES & YES & YES & -- & 1901\\
$(61, 17)$ & 9 & $(21, 8)$ & 6 & 1 & YES & YES & YES & -- & 1902\\
$(61, 25)$ & 9 & $(22, 9)$ & 7 & 1 & YES & YES & NO(2) & NO & 1903\\
$(61, 18)$ & 9 & $(33, 7)$ & 8 & 1 & YES & YES & YES & -- & 1904\\
$(61, 17)$ & 9 & $(37, 11)$ & 8 & 1 & YES & YES & YES & NO & 1905\\
$(61, 14)$ & 10 & $(47, 10)$ & 9 & 1 & YES & YES & YES & NO & 1906\\
$(61, 14)$ & 10 & $(51, 11)$ & 9 & 1 & YES & YES & YES & NO & 1907\\
$(62, 27)$ & 9 & $(15, 4)$ & 6 & 1 & YES & YES & YES & -- & 1908\\
$(63, 26)$ & 9 & $(10, 3)$ & 5 & 1 & YES & YES & YES & -- & 1909\\
$(64, 25)$ & 9 & $(2, 1)$ & 1 & 2 & YES & YES & NO(2) & -- & 1910\\
$(64, 27)$ & 9 & $(2, 1)$ & 1 & 2 & YES & YES & NO(2) & -- & 1911\\
$(64, 25)$ & 9 & $(3, 1)$ & 2 & 1 & YES & YES & NO(2) & -- & 1912\\
$(64, 25)$ & 9 & $(3, 1)$ & 2 & 1 & YES & YES & NO(2) & NO & 1913\\
$(64, 25)$ & 9 & $(5, 1)$ & 4 & 1 & YES & YES & NO(2) & -- & 1914\\
$(64, 23)$ & 9 & $(10, 3)$ & 5 & 2 & YES & YES & YES & -- & 1915\\
$(64, 19)$ & 9 & $(18, 7)$ & 6 & 2 & YES & YES & YES & -- & 1916\\
$(64, 23)$ & 9 & $(18, 7)$ & 6 & 2 & YES & YES & YES & 1749 & 1917\\
$(64, 27)$ & 9 & $(18, 5)$ & 6 & 2 & YES & YES & YES & -- & 1918\\
$(64, 19)$ & 9 & $(23, 7)$ & 7 & 1 & YES & YES & YES & -- & 1919\\
$(64, 19)$ & 9 & $(24, 7)$ & 7 & 8 & YES & YES & YES & -- & 1920\\
$(64, 25)$ & 9 & $(34, 13)$ & 7 & 2 & YES & YES & YES & NO & 1921\\
$(65, 19)$ & 9 & $(10, 3)$ & 5 & 5 & YES & YES & YES & -- & 1922\\
$(65, 19)$ & 9 & $(11, 4)$ & 5 & 1 & YES & YES & YES & -- & 1923\\
$(65, 19)$ & 9 & $(13, 3)$ & 6 & 13 & YES & YES & YES & -- & 1924\\
$(65, 24)$ & 9 & $(13, 5)$ & 5 & 13 & YES & YES & YES & -- & 1925\\
$(65, 18)$ & 9 & $(17, 7)$ & 6 & 1 & YES & YES & YES & NO & 1926\\
$(65, 18)$ & 9 & $(18, 7)$ & 6 & 1 & YES & YES & YES & NO & 1927\\
$(65, 19)$ & 9 & $(18, 7)$ & 6 & 1 & YES & YES & YES & -- & 1928\\
$(65, 18)$ & 9 & $(21, 8)$ & 6 & 1 & YES & YES & YES & NO & 1929\\
$(65, 14)$ & 10 & $(31, 7)$ & 8 & 1 & YES & YES & YES & NO & 1930\\
$(65, 24)$ & 9 & $(53, 19)$ & 9 & 1 & YES & YES & YES & NO & 1931\\
$(66, 25)$ & 9 & $(10, 3)$ & 5 & 2 & YES & YES & YES & -- & 1932\\
$(66, 25)$ & 9 & $(10, 3)$ & 5 & 2 & YES & YES & YES & NO & 1933\\
$(66, 25)$ & 9 & $(13, 5)$ & 5 & 1 & YES & YES & YES & -- & 1934\\
$(66, 25)$ & 9 & $(22, 5)$ & 7 & 22 & YES & YES & YES & -- & 1935\\
$(67, 28)$ & 10 & $(6, 1)$ & 5 & 1 & YES & YES & YES & -- & 1936\\
$(67, 28)$ & 10 & $(6, 1)$ & 5 & 1 & YES & YES & YES & NO & 1937\\
$(67, 28)$ & 10 & $(7, 3)$ & 4 & 1 & YES & YES & YES & -- & 1938\\
$(67, 28)$ & 10 & $(13, 5)$ & 5 & 1 & YES & YES & YES & NO & 1939\\
$(67, 26)$ & 9 & $(30, 11)$ & 7 & 1 & YES & YES & YES & NO & 1940\\
$(67, 26)$ & 9 & $(50, 19)$ & 8 & 1 & YES & YES & YES & NO & 1941\\
$(68, 19)$ & 9 & $(10, 3)$ & 5 & 2 & YES & YES & YES & -- & 1942\\
$(68, 25)$ & 9 & $(11, 3)$ & 5 & 1 & YES & YES & YES & -- & 1943\\
$(68, 19)$ & 9 & $(17, 7)$ & 6 & 17 & YES & YES & YES & -- & 1944\\
$(69, 29)$ & 9 & $(23, 5)$ & 7 & 23 & YES & YES & YES & -- & 1945\\
$(69, 19)$ & 9 & $(24, 7)$ & 7 & 3 & YES & YES & YES & -- & 1946\\
$(69, 13)$ & 11 & $(60, 11)$ & 11 & 3 & YES & YES & YES & NO & 1947\\
$(70, 29)$ & 9 & $(13, 4)$ & 6 & 1 & YES & YES & YES & -- & 1948\\
$(70, 29)$ & 9 & $(13, 5)$ & 5 & 1 & YES & YES & YES & -- & 1949\\
$(70, 29)$ & 9 & $(15, 4)$ & 6 & 5 & YES & YES & YES & -- & 1950\\
$(70, 29)$ & 9 & $(17, 5)$ & 6 & 1 & YES & YES & YES & -- & 1951\\
$(71, 21)$ & 9 & $(2, 1)$ & 1 & 1 & YES & YES & NO(2) & NO & 1952\\
$(71, 26)$ & 9 & $(4, 1)$ & 3 & 1 & YES & YES & NO(2) & -- & 1953\\
$(71, 30)$ & 9 & $(5, 1)$ & 4 & 1 & YES & YES & NO(3) & NO & 1954\\
$(71, 21)$ & 9 & $(10, 3)$ & 5 & 1 & YES & YES & NO(2) & NO & 1955\\
$(71, 22)$ & 10 & $(10, 3)$ & 5 & 1 & YES & YES & YES & -- & 1956\\
$(71, 27)$ & 9 & $(10, 3)$ & 5 & 1 & YES & YES & YES & -- & 1957\\
$(71, 27)$ & 9 & $(10, 3)$ & 5 & 1 & YES & YES & YES & NO & 1958\\
$(71, 21)$ & 9 & $(13, 5)$ & 5 & 1 & YES & YES & YES & -- & 1959\\
$(71, 27)$ & 9 & $(13, 5)$ & 5 & 1 & YES & YES & YES & -- & 1960\\
$(71, 30)$ & 9 & $(14, 5)$ & 6 & 1 & YES & YES & YES & NO & 1961\\
$(71, 30)$ & 9 & $(17, 5)$ & 6 & 1 & YES & YES & YES & -- & 1962\\
$(71, 27)$ & 9 & $(18, 5)$ & 6 & 1 & YES & YES & YES & -- & 1963\\
$(71, 27)$ & 9 & $(23, 10)$ & 7 & 1 & YES & YES & YES & NO & 1964\\
$(71, 19)$ & 10 & $(31, 9)$ & 8 & 1 & YES & YES & YES & NO & 1965\\
$(71, 26)$ & 9 & $(41, 15)$ & 8 & 1 & YES & YES & NO(2) & NO & 1966\\
$(73, 27)$ & 9 & $(19, 8)$ & 6 & 1 & YES & YES & YES & NO & 1967\\
$(73, 27)$ & 9 & $(22, 5)$ & 7 & 1 & YES & YES & YES & NO & 1968\\
$(73, 26)$ & 11 & $(59, 21)$ & 10 & 1 & YES & YES & YES & NO & 1969\\
$(74, 29)$ & 10 & $(4, 1)$ & 3 & 2 & YES & YES & YES & -- & 1970\\
$(74, 29)$ & 10 & $(4, 1)$ & 3 & 2 & YES & YES & YES & NO & 1971\\
$(74, 31)$ & 9 & $(13, 5)$ & 5 & 1 & YES & YES & YES & -- & 1972\\
$(74, 31)$ & 9 & $(17, 4)$ & 7 & 1 & YES & YES & YES & NO & 1973\\
$(75, 22)$ & 10 & $(7, 3)$ & 4 & 1 & YES & YES & YES & -- & 1974\\
$(75, 22)$ & 10 & $(11, 3)$ & 5 & 1 & YES & YES & YES & -- & 1975\\
$(75, 29)$ & 9 & $(13, 5)$ & 5 & 1 & YES & YES & YES & -- & 1976\\
$(75, 29)$ & 9 & $(14, 5)$ & 6 & 1 & YES & YES & YES & -- & 1977\\
$(75, 17)$ & 10 & $(17, 7)$ & 6 & 1 & YES & YES & YES & NO & 1978\\
$(75, 22)$ & 10 & $(18, 5)$ & 6 & 3 & YES & YES & YES & -- & 1979\\
$(75, 29)$ & 9 & $(18, 5)$ & 6 & 3 & YES & YES & YES & -- & 1980\\
$(75, 22)$ & 10 & $(19, 4)$ & 7 & 1 & YES & YES & YES & NO & 1981\\
$(75, 22)$ & 10 & $(27, 5)$ & 8 & 3 & YES & YES & YES & -- & 1982\\
$(75, 22)$ & 10 & $(27, 5)$ & 8 & 3 & YES & YES & YES & NO & 1983\\
$(76, 29)$ & 9 & $(7, 2)$ & 4 & 1 & YES & YES & YES & -- & 1984\\
$(76, 29)$ & 9 & $(7, 2)$ & 4 & 1 & YES & YES & YES & NO & 1985\\
$(76, 21)$ & 9 & $(8, 3)$ & 4 & 4 & YES & YES & YES & -- & 1986\\
$(76, 21)$ & 9 & $(11, 4)$ & 5 & 1 & YES & YES & YES & -- & 1987\\
$(76, 21)$ & 9 & $(11, 4)$ & 5 & 1 & YES & YES & YES & NO & 1988\\
$(76, 21)$ & 9 & $(13, 3)$ & 6 & 1 & YES & YES & YES & -- & 1989\\
$(76, 21)$ & 9 & $(13, 3)$ & 6 & 1 & YES & YES & YES & NO & 1990\\
$(76, 29)$ & 9 & $(41, 16)$ & 8 & 1 & YES & YES & YES & NO & 1991\\
$(76, 29)$ & 9 & $(60, 23)$ & 9 & 4 & YES & YES & YES & NO & 1992\\
$(78, 23)$ & 10 & $(4, 1)$ & 3 & 2 & YES & YES & YES & -- & 1993\\
$(78, 29)$ & 10 & $(5, 1)$ & 4 & 1 & YES & YES & YES & -- & 1994\\
$(78, 23)$ & 10 & $(10, 3)$ & 5 & 2 & YES & YES & YES & NO & 1995\\
$(78, 29)$ & 10 & $(10, 3)$ & 5 & 2 & YES & YES & YES & -- & 1996\\
$(78, 29)$ & 10 & $(11, 4)$ & 5 & 1 & YES & YES & YES & NO & 1997\\
$(79, 29)$ & 9 & $(2, 1)$ & 1 & 1 & YES & YES & NO(2) & NO & 1998\\
$(79, 30)$ & 9 & $(9, 4)$ & 5 & 1 & YES & YES & YES & -- & 1999\\
$(79, 18)$ & 10 & $(10, 3)$ & 5 & 1 & YES & YES & YES & -- & 2000\\
$(79, 29)$ & 9 & $(10, 3)$ & 5 & 1 & YES & YES & YES & -- & 2001\\
$(79, 18)$ & 10 & $(11, 4)$ & 5 & 1 & YES & YES & YES & NO & 2002\\
$(79, 29)$ & 9 & $(11, 4)$ & 5 & 1 & YES & YES & YES & -- & 2003\\
$(79, 22)$ & 10 & $(13, 5)$ & 5 & 1 & YES & YES & YES & NO & 2004\\
$(79, 23)$ & 10 & $(13, 5)$ & 5 & 1 & YES & YES & YES & -- & 2005\\
$(79, 29)$ & 9 & $(13, 4)$ & 6 & 1 & YES & YES & YES & -- & 2006\\
$(79, 30)$ & 9 & $(13, 3)$ & 6 & 1 & YES & YES & YES & NO & 2007\\
$(79, 29)$ & 9 & $(17, 7)$ & 6 & 1 & YES & YES & YES & NO & 2008\\
$(79, 30)$ & 9 & $(17, 5)$ & 6 & 1 & YES & YES & YES & -- & 2009\\
$(79, 30)$ & 9 & $(17, 7)$ & 6 & 1 & YES & YES & YES & NO & 2010\\
$(79, 24)$ & 10 & $(18, 5)$ & 6 & 1 & YES & YES & YES & -- & 2011\\
$(79, 30)$ & 9 & $(19, 8)$ & 6 & 1 & YES & YES & YES & NO & 2012\\
$(79, 18)$ & 10 & $(21, 8)$ & 6 & 1 & YES & YES & YES & NO & 2013\\
$(79, 29)$ & 9 & $(23, 9)$ & 7 & 1 & YES & YES & YES & NO & 2014\\
$(79, 30)$ & 9 & $(28, 11)$ & 8 & 1 & YES & YES & YES & NO & 2015\\
$(79, 18)$ & 10 & $(55, 13)$ & 10 & 1 & YES & YES & YES & NO & 2016\\
$(79, 30)$ & 9 & $(60, 23)$ & 9 & 1 & YES & YES & YES & NO & 2017\\
$(80, 31)$ & 9 & $(7, 2)$ & 4 & 1 & YES & YES & NO(2) & -- & 2018\\
$(80, 31)$ & 9 & $(7, 2)$ & 4 & 1 & YES & YES & NO(2) & NO & 2019\\
$(80, 31)$ & 9 & $(8, 3)$ & 4 & 8 & YES & YES & YES & -- & 2020\\
$(80, 31)$ & 9 & $(19, 7)$ & 6 & 1 & YES & YES & YES & NO & 2021\\
$(80, 33)$ & 10 & $(70, 29)$ & 9 & 10 & YES & YES & YES & 2678 & 2022\\
$(81, 31)$ & 9 & $(7, 3)$ & 4 & 1 & YES & YES & YES & -- & 2023\\
$(81, 34)$ & 9 & $(7, 3)$ & 4 & 1 & YES & YES & YES & -- & 2024\\
$(81, 31)$ & 9 & $(8, 3)$ & 4 & 1 & YES & YES & YES & -- & 2025\\
$(81, 31)$ & 9 & $(10, 3)$ & 5 & 1 & YES & YES & YES & -- & 2026\\
$(81, 31)$ & 9 & $(12, 5)$ & 5 & 3 & YES & YES & YES & -- & 2027\\
$(81, 31)$ & 9 & $(13, 3)$ & 6 & 1 & YES & YES & YES & -- & 2028\\
$(82, 31)$ & 10 & $(5, 2)$ & 3 & 1 & YES & YES & YES & -- & 2029\\
$(82, 23)$ & 10 & $(13, 5)$ & 5 & 1 & YES & YES & YES & -- & 2030\\
$(82, 25)$ & 10 & $(23, 5)$ & 7 & 1 & YES & YES & YES & NO & 2031\\
$(83, 36)$ & 10 & $(2, 1)$ & 1 & 1 & YES & YES & YES & -- & 2032\\
$(83, 36)$ & 10 & $(5, 1)$ & 4 & 1 & YES & YES & YES & -- & 2033\\
$(83, 18)$ & 10 & $(14, 5)$ & 6 & 1 & YES & YES & YES & NO & 2034\\
$(83, 18)$ & 10 & $(16, 5)$ & 7 & 1 & YES & YES & YES & NO & 2035\\
$(83, 19)$ & 10 & $(17, 7)$ & 6 & 1 & YES & YES & YES & -- & 2036\\
$(84, 25)$ & 10 & $(2, 1)$ & 1 & 2 & YES & YES & NO(2) & NO & 2037\\
$(84, 25)$ & 10 & $(13, 4)$ & 6 & 1 & YES & YES & YES & NO & 2038\\
$(84, 19)$ & 10 & $(17, 7)$ & 6 & 1 & YES & YES & YES & -- & 2039\\
$(84, 25)$ & 10 & $(37, 11)$ & 8 & 1 & YES & YES & YES & NO & 2040\\
$(85, 33)$ & 10 & $(13, 3)$ & 6 & 1 & YES & YES & YES & -- & 2041\\
$(86, 25)$ & 10 & $(7, 3)$ & 4 & 1 & YES & YES & YES & NO & 2042\\
$(86, 31)$ & 10 & $(7, 2)$ & 4 & 1 & YES & YES & YES & -- & 2043\\
$(86, 25)$ & 10 & $(13, 5)$ & 5 & 1 & YES & YES & YES & -- & 2044\\
$(89, 26)$ & 10 & $(2, 1)$ & 1 & 1 & YES & YES & YES & NO & 2045\\
$(89, 26)$ & 10 & $(3, 1)$ & 2 & 1 & YES & YES & NO(2) & -- & 2046\\
$(89, 26)$ & 10 & $(3, 1)$ & 2 & 1 & YES & YES & NO(2) & NO & 2047\\
$(89, 26)$ & 10 & $(4, 1)$ & 3 & 1 & YES & YES & YES & -- & 2048\\
$(89, 25)$ & 10 & $(7, 3)$ & 4 & 1 & YES & YES & YES & NO & 2049\\
$(89, 34)$ & 9 & $(7, 3)$ & 4 & 1 & YES & YES & YES & -- & 2050\\
$(89, 39)$ & 11 & $(7, 1)$ & 6 & 1 & YES & YES & YES & NO & 2051\\
$(89, 39)$ & 11 & $(7, 1)$ & 6 & 1 & YES & YES & YES & NO & 2052\\
$(89, 25)$ & 10 & $(8, 3)$ & 4 & 1 & YES & YES & YES & -- & 2053\\
$(89, 25)$ & 10 & $(8, 3)$ & 4 & 1 & YES & YES & YES & NO & 2054\\
$(89, 26)$ & 10 & $(9, 4)$ & 5 & 1 & YES & YES & YES & NO & 2055\\
$(89, 32)$ & 10 & $(10, 3)$ & 5 & 1 & YES & YES & YES & -- & 2056\\
$(89, 34)$ & 9 & $(10, 3)$ & 5 & 1 & YES & YES & YES & -- & 2057\\
$(89, 34)$ & 9 & $(10, 3)$ & 5 & 1 & YES & YES & YES & NO & 2058\\
$(89, 34)$ & 9 & $(11, 3)$ & 5 & 1 & YES & YES & YES & -- & 2059\\
$(89, 34)$ & 9 & $(12, 5)$ & 5 & 1 & YES & YES & YES & -- & 2060\\
$(89, 24)$ & 10 & $(13, 5)$ & 5 & 1 & YES & YES & YES & -- & 2061\\
$(89, 24)$ & 10 & $(18, 5)$ & 6 & 1 & YES & YES & YES & -- & 2062\\
$(89, 24)$ & 10 & $(24, 7)$ & 7 & 1 & YES & YES & YES & NO & 2063\\
$(89, 34)$ & 9 & $(28, 11)$ & 8 & 1 & YES & YES & YES & NO & 2064\\
$(89, 34)$ & 9 & $(37, 14)$ & 8 & 1 & YES & YES & YES & NO & 2065\\
$(89, 25)$ & 10 & $(61, 17)$ & 9 & 1 & YES & YES & YES & NO & 2066\\
$(89, 26)$ & 10 & $(64, 19)$ & 9 & 1 & YES & YES & YES & NO & 2067\\
$(89, 34)$ & 9 & $(81, 31)$ & 9 & 1 & YES & YES & YES & NO & 2068\\
$(90, 37)$ & 11 & $(5, 1)$ & 4 & 5 & YES & YES & YES & -- & 2069\\
$(91, 27)$ & 10 & $(2, 1)$ & 1 & 1 & YES & YES & NO(2) & NO & 2070\\
$(91, 40)$ & 10 & $(5, 2)$ & 3 & 1 & YES & YES & YES & -- & 2071\\
$(91, 27)$ & 10 & $(12, 5)$ & 5 & 1 & YES & YES & YES & -- & 2072\\
$(91, 25)$ & 10 & $(13, 5)$ & 5 & 13 & YES & YES & YES & -- & 2073\\
$(91, 27)$ & 10 & $(19, 5)$ & 7 & 1 & YES & YES & YES & 1827 & 2074\\
$(93, 26)$ & 10 & $(2, 1)$ & 1 & 1 & YES & YES & YES & NO & 2075\\
$(93, 26)$ & 10 & $(6, 1)$ & 5 & 3 & YES & YES & YES & -- & 2076\\
$(93, 26)$ & 10 & $(6, 1)$ & 5 & 3 & YES & YES & YES & NO & 2077\\
$(93, 26)$ & 10 & $(7, 2)$ & 4 & 1 & YES & YES & YES & NO & 2078\\
$(93, 26)$ & 10 & $(25, 7)$ & 7 & 1 & YES & YES & YES & NO & 2079\\
$(93, 34)$ & 10 & $(41, 15)$ & 8 & 1 & YES & YES & NO(2) & NO & 2080\\
$(93, 26)$ & 10 & $(47, 13)$ & 8 & 1 & YES & YES & YES & NO & 2081\\
$(94, 39)$ & 10 & $(5, 1)$ & 4 & 1 & YES & YES & YES & -- & 2082\\
$(94, 39)$ & 10 & $(8, 3)$ & 4 & 2 & YES & YES & YES & -- & 2083\\
$(94, 39)$ & 10 & $(11, 3)$ & 5 & 1 & YES & YES & YES & -- & 2084\\
$(94, 39)$ & 10 & $(11, 3)$ & 5 & 1 & YES & YES & YES & NO & 2085\\
$(95, 39)$ & 10 & $(2, 1)$ & 1 & 1 & YES & YES & YES & NO & 2086\\
$(95, 37)$ & 11 & $(6, 1)$ & 5 & 1 & YES & YES & YES & NO & 2087\\
$(95, 36)$ & 10 & $(10, 3)$ & 5 & 5 & YES & YES & YES & -- & 2088\\
$(97, 41)$ & 10 & $(2, 1)$ & 1 & 1 & YES & YES & YES & -- & 2089\\
$(97, 22)$ & 11 & $(7, 3)$ & 4 & 1 & YES & YES & YES & NO & 2090\\
$(97, 36)$ & 10 & $(7, 3)$ & 4 & 1 & YES & YES & YES & NO & 2091\\
$(97, 22)$ & 11 & $(11, 4)$ & 5 & 1 & YES & YES & YES & -- & 2092\\
$(97, 22)$ & 11 & $(11, 4)$ & 5 & 1 & YES & YES & YES & NO & 2093\\
$(97, 37)$ & 10 & $(17, 7)$ & 6 & 1 & YES & YES & YES & NO & 2094\\
$(97, 37)$ & 10 & $(18, 7)$ & 6 & 1 & YES & YES & YES & NO & 2095\\
$(97, 41)$ & 10 & $(43, 18)$ & 8 & 1 & YES & YES & YES & NO & 2096\\
$(98, 29)$ & 10 & $(8, 3)$ & 4 & 2 & YES & YES & YES & -- & 2097\\
$(98, 27)$ & 10 & $(9, 4)$ & 5 & 1 & YES & YES & YES & -- & 2098\\
$(98, 27)$ & 10 & $(9, 4)$ & 5 & 1 & YES & YES & YES & NO & 2099\\
$(98, 27)$ & 10 & $(11, 4)$ & 5 & 1 & YES & YES & YES & -- & 2100\\
$(98, 27)$ & 10 & $(24, 7)$ & 7 & 2 & YES & YES & YES & NO & 2101\\
$(98, 27)$ & 10 & $(39, 11)$ & 9 & 1 & YES & YES & YES & NO & 2102\\
$(98, 27)$ & 10 & $(47, 13)$ & 8 & 1 & YES & YES & YES & NO & 2103\\
$(99, 41)$ & 10 & $(7, 3)$ & 4 & 1 & YES & YES & YES & -- & 2104\\
$(99, 29)$ & 10 & $(8, 3)$ & 4 & 1 & YES & YES & YES & -- & 2105\\
$(99, 41)$ & 10 & $(8, 3)$ & 4 & 1 & YES & YES & YES & -- & 2106\\
$(99, 29)$ & 10 & $(10, 3)$ & 5 & 1 & YES & YES & YES & -- & 2107\\
$(99, 41)$ & 10 & $(11, 3)$ & 5 & 11 & YES & YES & YES & -- & 2108\\
$(99, 41)$ & 10 & $(11, 3)$ & 5 & 11 & YES & YES & YES & NO & 2109\\
$(99, 29)$ & 10 & $(13, 4)$ & 6 & 1 & YES & YES & YES & -- & 2110\\
$(99, 29)$ & 10 & $(89, 26)$ & 10 & 1 & YES & YES & YES & NO & 2111\\
$(100, 29)$ & 11 & $(7, 3)$ & 4 & 1 & YES & YES & YES & NO & 2112\\
$(100, 29)$ & 11 & $(8, 3)$ & 4 & 4 & YES & YES & YES & 2387 & 2113\\
$(100, 39)$ & 10 & $(10, 3)$ & 5 & 10 & YES & YES & YES & -- & 2114\\
$(100, 39)$ & 10 & $(11, 3)$ & 5 & 1 & YES & YES & YES & -- & 2115\\
$(100, 29)$ & 11 & $(13, 3)$ & 6 & 1 & YES & YES & YES & -- & 2116\\
$(100, 27)$ & 10 & $(22, 5)$ & 7 & 2 & YES & YES & YES & -- & 2117\\
$(100, 29)$ & 11 & $(58, 17)$ & 9 & 2 & YES & YES & YES & NO & 2118\\
$(101, 39)$ & 10 & $(5, 1)$ & 4 & 1 & YES & YES & YES & -- & 2119\\
$(101, 37)$ & 10 & $(7, 3)$ & 4 & 1 & YES & YES & YES & -- & 2120\\
$(101, 39)$ & 10 & $(7, 3)$ & 4 & 1 & YES & YES & YES & -- & 2121\\
$(101, 37)$ & 10 & $(21, 8)$ & 6 & 1 & YES & YES & YES & NO & 2122\\
$(102, 31)$ & 11 & $(10, 3)$ & 5 & 2 & YES & YES & YES & -- & 2123\\
$(103, 39)$ & 10 & $(5, 1)$ & 4 & 1 & YES & YES & YES & -- & 2124\\
$(103, 39)$ & 10 & $(5, 2)$ & 3 & 1 & YES & YES & YES & NO & 2125\\
$(103, 39)$ & 10 & $(7, 3)$ & 4 & 1 & YES & YES & YES & NO & 2126\\
$(103, 29)$ & 11 & $(10, 3)$ & 5 & 1 & YES & YES & YES & -- & 2127\\
$(103, 29)$ & 11 & $(11, 3)$ & 5 & 1 & YES & YES & YES & -- & 2128\\
$(103, 30)$ & 11 & $(14, 3)$ & 6 & 1 & YES & YES & YES & -- & 2129\\
$(103, 40)$ & 11 & $(75, 29)$ & 9 & 1 & YES & YES & YES & NO & 2130\\
$(104, 43)$ & 10 & $(5, 2)$ & 3 & 1 & YES & YES & YES & -- & 2131\\
$(104, 29)$ & 10 & $(13, 4)$ & 6 & 13 & YES & YES & YES & -- & 2132\\
$(104, 43)$ & 10 & $(63, 26)$ & 9 & 1 & YES & YES & YES & 2562 & 2133\\
$(105, 38)$ & 11 & $(4, 1)$ & 3 & 1 & YES & YES & YES & -- & 2134\\
$(105, 43)$ & 11 & $(5, 1)$ & 4 & 5 & YES & YES & YES & -- & 2135\\
$(105, 44)$ & 10 & $(5, 2)$ & 3 & 5 & YES & YES & YES & -- & 2136\\
$(105, 44)$ & 10 & $(7, 2)$ & 4 & 7 & YES & YES & YES & NO & 2137\\
$(105, 31)$ & 10 & $(8, 3)$ & 4 & 1 & YES & YES & YES & -- & 2138\\
$(105, 29)$ & 10 & $(9, 4)$ & 5 & 3 & YES & YES & YES & NO & 2139\\
$(105, 44)$ & 10 & $(9, 4)$ & 5 & 3 & YES & YES & NO(2) & NO & 2140\\
$(105, 29)$ & 10 & $(11, 4)$ & 5 & 1 & YES & YES & YES & -- & 2141\\
$(105, 29)$ & 10 & $(11, 4)$ & 5 & 1 & YES & YES & YES & NO & 2142\\
$(105, 44)$ & 10 & $(11, 3)$ & 5 & 1 & YES & YES & YES & -- & 2143\\
$(105, 29)$ & 10 & $(12, 5)$ & 5 & 3 & YES & YES & YES & -- & 2144\\
$(105, 29)$ & 10 & $(16, 5)$ & 7 & 1 & YES & YES & YES & NO & 2145\\
$(105, 29)$ & 10 & $(24, 7)$ & 7 & 3 & YES & YES & YES & NO & 2146\\
$(105, 38)$ & 11 & $(58, 21)$ & 10 & 1 & YES & YES & YES & NO & 2147\\
$(105, 29)$ & 10 & $(68, 19)$ & 9 & 1 & YES & YES & YES & NO & 2148\\
$(105, 43)$ & 11 & $(83, 34)$ & 10 & 1 & YES & YES & YES & NO & 2149\\
$(105, 31)$ & 10 & $(98, 29)$ & 10 & 7 & YES & YES & YES & NO & 2150\\
$(106, 31)$ & 10 & $(5, 2)$ & 3 & 1 & YES & YES & YES & -- & 2151\\
$(106, 41)$ & 10 & $(5, 2)$ & 3 & 1 & YES & YES & YES & -- & 2152\\
$(106, 41)$ & 10 & $(7, 3)$ & 4 & 1 & YES & YES & YES & -- & 2153\\
$(106, 23)$ & 11 & $(8, 3)$ & 4 & 2 & YES & YES & YES & NO & 2154\\
$(106, 31)$ & 10 & $(8, 3)$ & 4 & 2 & YES & YES & YES & -- & 2155\\
$(106, 41)$ & 10 & $(8, 3)$ & 4 & 2 & YES & YES & YES & -- & 2156\\
$(106, 23)$ & 11 & $(9, 4)$ & 5 & 1 & YES & YES & YES & -- & 2157\\
$(106, 23)$ & 11 & $(10, 3)$ & 5 & 2 & YES & YES & YES & NO & 2158\\
$(106, 41)$ & 10 & $(10, 3)$ & 5 & 2 & YES & YES & YES & -- & 2159\\
$(106, 41)$ & 10 & $(11, 3)$ & 5 & 1 & YES & YES & YES & -- & 2160\\
$(106, 41)$ & 10 & $(11, 3)$ & 5 & 1 & YES & YES & YES & NO & 2161\\
$(106, 41)$ & 10 & $(11, 3)$ & 5 & 1 & YES & YES & YES & NO & 2162\\
$(106, 31)$ & 10 & $(58, 17)$ & 9 & 2 & YES & YES & YES & NO & 2163\\
$(106, 41)$ & 10 & $(101, 39)$ & 10 & 1 & YES & YES & YES & NO & 2164\\
$(107, 41)$ & 10 & $(7, 3)$ & 4 & 1 & YES & YES & YES & -- & 2165\\
$(107, 41)$ & 10 & $(11, 3)$ & 5 & 1 & YES & YES & YES & -- & 2166\\
$(107, 41)$ & 10 & $(29, 11)$ & 7 & 1 & YES & YES & YES & NO & 2167\\
$(107, 41)$ & 10 & $(81, 31)$ & 9 & 1 & YES & YES & YES & NO & 2168\\
$(107, 44)$ & 12 & $(90, 37)$ & 11 & 1 & YES & YES & YES & NO & 2169\\
$(108, 41)$ & 10 & $(5, 2)$ & 3 & 1 & YES & YES & YES & -- & 2170\\
$(108, 41)$ & 10 & $(7, 3)$ & 4 & 1 & YES & YES & YES & -- & 2171\\
$(108, 41)$ & 10 & $(10, 3)$ & 5 & 2 & YES & YES & YES & -- & 2172\\
$(108, 41)$ & 10 & $(34, 13)$ & 7 & 2 & YES & YES & YES & 2597 & 2173\\
$(109, 40)$ & 10 & $(5, 2)$ & 3 & 1 & YES & YES & NO(2) & NO & 2174\\
$(109, 40)$ & 10 & $(8, 3)$ & 4 & 1 & YES & YES & YES & -- & 2175\\
$(109, 45)$ & 10 & $(10, 3)$ & 5 & 1 & YES & YES & YES & -- & 2176\\
$(109, 46)$ & 10 & $(10, 3)$ & 5 & 1 & YES & YES & YES & -- & 2177\\
$(109, 40)$ & 10 & $(18, 7)$ & 6 & 1 & YES & YES & YES & NO & 2178\\
$(109, 45)$ & 10 & $(26, 11)$ & 7 & 1 & YES & YES & YES & NO & 2179\\
$(109, 45)$ & 10 & $(31, 13)$ & 7 & 1 & YES & YES & YES & NO & 2180\\
$(110, 43)$ & 11 & $(6, 1)$ & 5 & 2 & YES & YES & YES & NO & 2181\\
$(110, 43)$ & 11 & $(110, 43)$ & 11 & 110 & YES & YES & YES & NO & 2182\\
$(111, 41)$ & 10 & $(3, 1)$ & 2 & 3 & YES & YES & NO(2) & -- & 2183\\
$(111, 46)$ & 10 & $(3, 1)$ & 2 & 3 & YES & YES & YES & -- & 2184\\
$(111, 41)$ & 10 & $(10, 3)$ & 5 & 1 & YES & YES & YES & -- & 2185\\
$(111, 43)$ & 10 & $(14, 3)$ & 6 & 1 & YES & YES & YES & NO & 2186\\
$(111, 46)$ & 10 & $(17, 7)$ & 6 & 1 & YES & YES & YES & 2248 & 2187\\
$(111, 41)$ & 10 & $(27, 10)$ & 7 & 3 & YES & YES & NO(2) & NO & 2188\\
$(112, 47)$ & 10 & $(5, 2)$ & 3 & 1 & YES & YES & YES & -- & 2189\\
$(112, 47)$ & 10 & $(7, 2)$ & 4 & 7 & YES & YES & YES & -- & 2190\\
$(112, 41)$ & 10 & $(8, 3)$ & 4 & 8 & YES & YES & YES & -- & 2191\\
$(112, 47)$ & 10 & $(11, 3)$ & 5 & 1 & YES & YES & YES & NO & 2192\\
$(112, 41)$ & 10 & $(13, 3)$ & 6 & 1 & YES & YES & YES & -- & 2193\\
$(112, 47)$ & 10 & $(17, 7)$ & 6 & 1 & YES & YES & YES & NO & 2194\\
$(112, 47)$ & 10 & $(26, 11)$ & 7 & 2 & YES & YES & YES & 2405 & 2195\\
$(112, 47)$ & 10 & $(43, 18)$ & 8 & 1 & YES & YES & YES & NO & 2196\\
$(112, 47)$ & 10 & $(69, 29)$ & 9 & 1 & YES & YES & YES & 2629 & 2197\\
$(113, 42)$ & 11 & $(5, 2)$ & 3 & 1 & YES & YES & YES & NO & 2198\\
$(113, 49)$ & 11 & $(6, 1)$ & 5 & 1 & YES & YES & YES & NO & 2199\\
$(113, 42)$ & 11 & $(7, 3)$ & 4 & 1 & YES & YES & YES & NO & 2200\\
$(113, 44)$ & 12 & $(113, 44)$ & 12 & 113 & YES & YES & YES & NO & 2201\\
$(115, 34)$ & 10 & $(5, 2)$ & 3 & 5 & YES & YES & YES & -- & 2202\\
$(115, 44)$ & 10 & $(5, 2)$ & 3 & 5 & YES & YES & YES & -- & 2203\\
$(115, 31)$ & 11 & $(8, 3)$ & 4 & 1 & YES & YES & YES & -- & 2204\\
$(115, 44)$ & 10 & $(8, 3)$ & 4 & 1 & YES & YES & YES & -- & 2205\\
$(115, 26)$ & 11 & $(9, 4)$ & 5 & 1 & YES & YES & YES & -- & 2206\\
$(115, 44)$ & 10 & $(9, 4)$ & 5 & 1 & YES & YES & YES & NO & 2207\\
$(115, 44)$ & 10 & $(10, 3)$ & 5 & 5 & YES & YES & YES & -- & 2208\\
$(115, 34)$ & 10 & $(24, 7)$ & 7 & 1 & YES & YES & YES & NO & 2209\\
$(115, 44)$ & 10 & $(55, 21)$ & 8 & 5 & YES & YES & YES & NO & 2210\\
$(115, 26)$ & 11 & $(79, 18)$ & 10 & 1 & YES & YES & YES & NO & 2211\\
$(115, 47)$ & 12 & $(93, 38)$ & 11 & 1 & YES & YES & YES & NO & 2212\\
$(115, 44)$ & 10 & $(107, 41)$ & 10 & 1 & YES & YES & YES & NO & 2213\\
$(116, 49)$ & 10 & $(10, 3)$ & 5 & 2 & YES & YES & YES & -- & 2214\\
$(116, 49)$ & 10 & $(11, 3)$ & 5 & 1 & YES & YES & YES & NO & 2215\\
$(116, 51)$ & 11 & $(25, 11)$ & 7 & 1 & YES & YES & YES & NO & 2216\\
$(116, 49)$ & 10 & $(29, 12)$ & 7 & 29 & YES & YES & YES & NO & 2217\\
$(116, 49)$ & 10 & $(43, 18)$ & 8 & 1 & YES & YES & YES & NO & 2218\\
$(116, 51)$ & 11 & $(116, 51)$ & 11 & 116 & YES & YES & YES & NO & 2219\\
$(117, 49)$ & 10 & $(5, 2)$ & 3 & 1 & YES & YES & YES & -- & 2220\\
$(117, 31)$ & 11 & $(29, 8)$ & 7 & 1 & YES & YES & YES & NO & 2221\\
$(118, 45)$ & 11 & $(6, 1)$ & 5 & 2 & YES & YES & YES & -- & 2222\\
$(118, 45)$ & 11 & $(6, 1)$ & 5 & 2 & YES & YES & YES & NO & 2223\\
$(118, 27)$ & 11 & $(11, 4)$ & 5 & 1 & YES & YES & YES & -- & 2224\\
$(118, 27)$ & 11 & $(32, 7)$ & 8 & 2 & YES & YES & YES & NO & 2225\\
$(119, 44)$ & 10 & $(2, 1)$ & 1 & 1 & YES & YES & NO(2) & -- & 2226\\
$(119, 45)$ & 11 & $(5, 2)$ & 3 & 1 & YES & YES & YES & NO & 2227\\
$(119, 46)$ & 10 & $(5, 2)$ & 3 & 1 & YES & YES & YES & -- & 2228\\
$(119, 26)$ & 11 & $(8, 3)$ & 4 & 1 & YES & YES & YES & NO & 2229\\
$(119, 44)$ & 10 & $(8, 3)$ & 4 & 1 & YES & YES & YES & -- & 2230\\
$(119, 26)$ & 11 & $(10, 3)$ & 5 & 1 & YES & YES & YES & NO & 2231\\
$(119, 46)$ & 10 & $(10, 3)$ & 5 & 1 & YES & YES & YES & -- & 2232\\
$(119, 50)$ & 10 & $(10, 3)$ & 5 & 1 & YES & YES & YES & -- & 2233\\
$(119, 44)$ & 10 & $(13, 3)$ & 6 & 1 & YES & YES & YES & NO & 2234\\
$(119, 46)$ & 10 & $(13, 3)$ & 6 & 1 & YES & YES & YES & -- & 2235\\
$(119, 46)$ & 10 & $(21, 8)$ & 6 & 7 & YES & YES & YES & NO & 2236\\
$(119, 45)$ & 11 & $(31, 12)$ & 7 & 1 & YES & YES & YES & NO & 2237\\
$(119, 45)$ & 11 & $(34, 13)$ & 7 & 17 & YES & YES & YES & NO & 2238\\
$(119, 44)$ & 10 & $(41, 15)$ & 8 & 1 & YES & YES & YES & NO & 2239\\
$(119, 46)$ & 10 & $(41, 16)$ & 8 & 1 & YES & YES & YES & NO & 2240\\
$(119, 50)$ & 10 & $(74, 31)$ & 9 & 1 & YES & YES & YES & NO & 2241\\
$(119, 44)$ & 10 & $(111, 41)$ & 10 & 1 & YES & YES & YES & NO & 2242\\
$(120, 47)$ & 12 & $(120, 47)$ & 12 & 120 & YES & YES & YES & NO & 2243\\
$(121, 50)$ & 10 & $(2, 1)$ & 1 & 1 & NO & YES & NO(2) & -- & 2244\\
$(121, 50)$ & 10 & $(3, 1)$ & 2 & 1 & YES & YES & YES & -- & 2245\\
$(121, 46)$ & 10 & $(5, 2)$ & 3 & 1 & YES & YES & YES & -- & 2246\\
$(121, 46)$ & 10 & $(8, 3)$ & 4 & 1 & YES & YES & YES & -- & 2247\\
$(121, 50)$ & 10 & $(12, 5)$ & 5 & 1 & YES & YES & YES & 2187 & 2248\\
$(121, 50)$ & 10 & $(13, 3)$ & 6 & 1 & YES & YES & YES & NO & 2249\\
$(121, 32)$ & 11 & $(34, 9)$ & 8 & 1 & YES & YES & YES & NO & 2250\\
$(121, 46)$ & 10 & $(66, 25)$ & 9 & 11 & YES & YES & YES & NO & 2251\\
$(121, 46)$ & 10 & $(79, 30)$ & 9 & 1 & YES & YES & YES & NO & 2252\\
$(121, 46)$ & 10 & $(92, 35)$ & 10 & 1 & YES & YES & YES & NO & 2253\\
$(122, 51)$ & 11 & $(2, 1)$ & 1 & 2 & YES & YES & YES & NO & 2254\\
$(122, 37)$ & 11 & $(3, 1)$ & 2 & 1 & NO & YES & NO(2) & -- & 2255\\
$(122, 51)$ & 11 & $(5, 2)$ & 3 & 1 & YES & YES & YES & NO & 2256\\
$(122, 37)$ & 11 & $(7, 2)$ & 4 & 1 & YES & YES & YES & -- & 2257\\
$(122, 37)$ & 11 & $(7, 3)$ & 4 & 1 & YES & YES & YES & -- & 2258\\
$(122, 33)$ & 11 & $(8, 3)$ & 4 & 2 & YES & YES & YES & -- & 2259\\
$(122, 37)$ & 11 & $(102, 31)$ & 11 & 2 & YES & YES & YES & NO & 2260\\
$(123, 47)$ & 10 & $(2, 1)$ & 1 & 1 & YES & YES & YES & NO & 2261\\
$(123, 47)$ & 10 & $(4, 1)$ & 3 & 1 & YES & YES & YES & -- & 2262\\
$(123, 47)$ & 10 & $(4, 1)$ & 3 & 1 & YES & YES & YES & NO & 2263\\
$(123, 47)$ & 10 & $(5, 2)$ & 3 & 1 & YES & YES & YES & -- & 2264\\
$(123, 52)$ & 11 & $(5, 1)$ & 4 & 1 & YES & YES & YES & -- & 2265\\
$(123, 52)$ & 11 & $(6, 1)$ & 5 & 3 & YES & YES & YES & -- & 2266\\
$(123, 52)$ & 11 & $(6, 1)$ & 5 & 3 & YES & YES & YES & NO & 2267\\
$(123, 47)$ & 10 & $(7, 2)$ & 4 & 1 & YES & YES & YES & -- & 2268\\
$(123, 47)$ & 10 & $(8, 3)$ & 4 & 1 & YES & YES & YES & -- & 2269\\
$(123, 47)$ & 10 & $(9, 4)$ & 5 & 3 & YES & YES & YES & NO & 2270\\
$(123, 47)$ & 10 & $(11, 4)$ & 5 & 1 & YES & YES & YES & NO & 2271\\
$(123, 47)$ & 10 & $(37, 14)$ & 8 & 1 & YES & YES & YES & NO & 2272\\
$(123, 47)$ & 10 & $(47, 18)$ & 8 & 1 & YES & YES & YES & NO & 2273\\
$(123, 47)$ & 10 & $(76, 29)$ & 9 & 1 & YES & YES & YES & 2716 & 2274\\
$(123, 52)$ & 11 & $(97, 41)$ & 10 & 1 & YES & YES & YES & NO & 2275\\
$(123, 47)$ & 10 & $(123, 47)$ & 10 & 123 & YES & YES & YES & NO & 2276\\
$(123, 52)$ & 11 & $(123, 52)$ & 11 & 123 & YES & YES & YES & NO & 2277\\
$(124, 23)$ & 12 & $(7, 3)$ & 4 & 1 & YES & YES & YES & NO & 2278\\
$(125, 53)$ & 11 & $(2, 1)$ & 1 & 1 & YES & YES & YES & -- & 2279\\
$(125, 53)$ & 11 & $(6, 1)$ & 5 & 1 & YES & YES & YES & NO & 2280\\
$(125, 37)$ & 11 & $(11, 3)$ & 5 & 1 & YES & YES & YES & NO & 2281\\
$(125, 53)$ & 11 & $(33, 14)$ & 8 & 1 & YES & YES & YES & NO & 2282\\
$(127, 35)$ & 11 & $(3, 1)$ & 2 & 1 & YES & YES & YES & -- & 2283\\
$(127, 29)$ & 11 & $(33, 7)$ & 8 & 1 & YES & YES & YES & NO & 2284\\
$(127, 35)$ & 11 & $(40, 11)$ & 8 & 1 & YES & YES & YES & NO & 2285\\
$(127, 29)$ & 11 & $(84, 19)$ & 10 & 1 & YES & YES & YES & NO & 2286\\
$(128, 49)$ & 10 & $(3, 1)$ & 2 & 1 & YES & YES & YES & -- & 2287\\
$(128, 49)$ & 10 & $(5, 2)$ & 3 & 1 & YES & YES & YES & -- & 2288\\
$(128, 53)$ & 11 & $(5, 2)$ & 3 & 1 & YES & YES & YES & -- & 2289\\
$(128, 49)$ & 10 & $(8, 3)$ & 4 & 8 & YES & YES & YES & -- & 2290\\
$(128, 49)$ & 10 & $(8, 3)$ & 4 & 8 & YES & YES & YES & NO & 2291\\
$(128, 47)$ & 10 & $(13, 5)$ & 5 & 1 & YES & YES & YES & NO & 2292\\
$(128, 49)$ & 10 & $(21, 8)$ & 6 & 1 & YES & YES & YES & 2345 & 2293\\
$(128, 47)$ & 10 & $(35, 13)$ & 8 & 1 & YES & YES & YES & NO & 2294\\
$(128, 49)$ & 10 & $(55, 21)$ & 8 & 1 & YES & YES & YES & NO & 2295\\
$(128, 49)$ & 10 & $(76, 29)$ & 9 & 4 & YES & YES & YES & NO & 2296\\
$(128, 49)$ & 10 & $(128, 49)$ & 10 & 128 & YES & YES & YES & NO & 2297\\
$(129, 50)$ & 10 & $(2, 1)$ & 1 & 1 & NO & YES & NO(2) & -- & 2298\\
$(129, 56)$ & 11 & $(2, 1)$ & 1 & 1 & NO & YES & YES & -- & 2299\\
$(129, 53)$ & 11 & $(5, 1)$ & 4 & 1 & YES & YES & YES & -- & 2300\\
$(129, 50)$ & 10 & $(8, 3)$ & 4 & 1 & YES & YES & YES & 2342 & 2301\\
$(129, 49)$ & 10 & $(11, 3)$ & 5 & 1 & YES & YES & YES & NO & 2302\\
$(129, 49)$ & 10 & $(37, 14)$ & 8 & 1 & YES & YES & YES & NO & 2303\\
$(131, 50)$ & 10 & $(3, 1)$ & 2 & 1 & YES & YES & YES & -- & 2304\\
$(131, 48)$ & 11 & $(5, 2)$ & 3 & 1 & YES & YES & YES & -- & 2305\\
$(131, 55)$ & 10 & $(5, 2)$ & 3 & 1 & YES & YES & YES & -- & 2306\\
$(131, 50)$ & 10 & $(7, 2)$ & 4 & 1 & YES & YES & YES & -- & 2307\\
$(131, 50)$ & 10 & $(8, 3)$ & 4 & 1 & YES & YES & YES & -- & 2308\\
$(131, 50)$ & 10 & $(10, 3)$ & 5 & 1 & YES & YES & YES & -- & 2309\\
$(131, 48)$ & 11 & $(13, 5)$ & 5 & 1 & YES & YES & YES & NO & 2310\\
$(131, 50)$ & 10 & $(34, 13)$ & 7 & 1 & YES & YES & YES & NO & 2311\\
$(131, 50)$ & 10 & $(123, 47)$ & 10 & 1 & YES & YES & YES & NO & 2312\\
$(133, 39)$ & 11 & $(8, 3)$ & 4 & 1 & YES & YES & YES & -- & 2313\\
$(133, 58)$ & 11 & $(13, 5)$ & 5 & 1 & YES & YES & YES & NO & 2314\\
$(133, 31)$ & 12 & $(23, 5)$ & 7 & 1 & YES & YES & YES & NO & 2315\\
$(134, 39)$ & 11 & $(8, 3)$ & 4 & 2 & YES & YES & YES & -- & 2316\\
$(134, 37)$ & 11 & $(112, 31)$ & 10 & 2 & YES & YES & YES & 3198 & 2317\\
$(135, 56)$ & 11 & $(5, 2)$ & 3 & 5 & YES & YES & YES & -- & 2318\\
$(135, 56)$ & 11 & $(7, 2)$ & 4 & 1 & YES & YES & YES & NO & 2319\\
$(136, 57)$ & 11 & $(43, 18)$ & 8 & 1 & YES & YES & YES & NO & 2320\\
$(137, 37)$ & 11 & $(3, 1)$ & 2 & 1 & YES & YES & YES & -- & 2321\\
$(137, 37)$ & 11 & $(7, 3)$ & 4 & 1 & YES & YES & YES & -- & 2322\\
$(137, 37)$ & 11 & $(11, 3)$ & 5 & 1 & YES & YES & YES & -- & 2323\\
$(137, 37)$ & 11 & $(56, 15)$ & 9 & 1 & YES & YES & YES & NO & 2324\\
$(139, 57)$ & 11 & $(2, 1)$ & 1 & 1 & YES & YES & YES & -- & 2325\\
$(139, 51)$ & 11 & $(68, 25)$ & 9 & 1 & YES & YES & YES & NO & 2326\\
$(140, 41)$ & 11 & $(3, 1)$ & 2 & 1 & YES & YES & YES & -- & 2327\\
$(140, 53)$ & 11 & $(3, 1)$ & 2 & 1 & YES & YES & YES & -- & 2328\\
$(140, 61)$ & 11 & $(5, 2)$ & 3 & 5 & YES & YES & YES & NO & 2329\\
$(140, 41)$ & 11 & $(7, 3)$ & 4 & 7 & YES & YES & YES & -- & 2330\\
$(140, 41)$ & 11 & $(8, 3)$ & 4 & 4 & YES & YES & YES & -- & 2331\\
$(140, 41)$ & 11 & $(44, 13)$ & 8 & 4 & YES & YES & YES & NO & 2332\\
$(140, 41)$ & 11 & $(58, 17)$ & 9 & 2 & YES & YES & YES & 2424 & 2333\\
$(140, 41)$ & 11 & $(140, 41)$ & 11 & 140 & YES & YES & YES & NO & 2334\\
$(141, 59)$ & 11 & $(26, 11)$ & 7 & 1 & YES & YES & YES & NO & 2335\\
$(142, 51)$ & 11 & $(3, 1)$ & 2 & 1 & YES & YES & YES & -- & 2336\\
$(142, 55)$ & 11 & $(44, 17)$ & 8 & 2 & YES & YES & YES & NO & 2337\\
$(144, 55)$ & 10 & $(2, 1)$ & 1 & 2 & YES & YES & YES & -- & 2338\\
$(144, 55)$ & 10 & $(3, 1)$ & 2 & 3 & YES & YES & YES & -- & 2339\\
$(144, 55)$ & 10 & $(3, 1)$ & 2 & 3 & YES & YES & YES & NO & 2340\\
$(144, 55)$ & 10 & $(5, 2)$ & 3 & 1 & YES & YES & YES & -- & 2341\\
$(144, 55)$ & 10 & $(5, 2)$ & 3 & 1 & YES & YES & YES & 2301 & 2342\\
$(144, 55)$ & 10 & $(8, 3)$ & 4 & 8 & YES & YES & YES & -- & 2343\\
$(144, 55)$ & 10 & $(11, 4)$ & 5 & 1 & YES & YES & YES & NO & 2344\\
$(144, 55)$ & 10 & $(13, 5)$ & 5 & 1 & YES & YES & YES & 2293 & 2345\\
$(144, 55)$ & 10 & $(21, 8)$ & 6 & 3 & YES & YES & YES & NO & 2346\\
$(144, 55)$ & 10 & $(23, 9)$ & 7 & 1 & YES & YES & YES & NO & 2347\\
$(144, 55)$ & 10 & $(55, 21)$ & 8 & 1 & YES & YES & YES & NO & 2348\\
$(144, 55)$ & 10 & $(60, 23)$ & 9 & 12 & YES & YES & YES & NO & 2349\\
$(144, 55)$ & 10 & $(97, 37)$ & 10 & 1 & YES & YES & YES & NO & 2350\\
$(145, 53)$ & 11 & $(5, 1)$ & 4 & 5 & YES & YES & YES & -- & 2351\\
$(145, 56)$ & 11 & $(7, 2)$ & 4 & 1 & YES & YES & YES & NO & 2352\\
$(145, 53)$ & 11 & $(8, 3)$ & 4 & 1 & YES & YES & YES & NO & 2353\\
$(145, 43)$ & 12 & $(11, 3)$ & 5 & 1 & YES & YES & YES & NO & 2354\\
$(145, 53)$ & 11 & $(52, 19)$ & 9 & 1 & YES & YES & YES & NO & 2355\\
$(145, 44)$ & 11 & $(122, 37)$ & 11 & 1 & YES & YES & YES & NO & 2356\\
$(146, 57)$ & 11 & $(4, 1)$ & 3 & 2 & YES & YES & YES & NO & 2357\\
$(146, 57)$ & 11 & $(8, 3)$ & 4 & 2 & YES & YES & YES & NO & 2358\\
$(147, 43)$ & 11 & $(3, 1)$ & 2 & 3 & YES & YES & YES & -- & 2359\\
$(147, 43)$ & 11 & $(3, 1)$ & 2 & 3 & YES & YES & YES & NO & 2360\\
$(147, 41)$ & 11 & $(7, 3)$ & 4 & 7 & YES & YES & YES & -- & 2361\\
$(147, 41)$ & 11 & $(7, 3)$ & 4 & 7 & YES & YES & YES & NO & 2362\\
$(147, 43)$ & 11 & $(11, 3)$ & 5 & 1 & YES & YES & YES & NO & 2363\\
$(147, 43)$ & 11 & $(13, 4)$ & 6 & 1 & YES & YES & YES & NO & 2364\\
$(147, 43)$ & 11 & $(14, 3)$ & 6 & 7 & YES & YES & YES & NO & 2365\\
$(147, 43)$ & 11 & $(23, 7)$ & 7 & 1 & YES & YES & YES & NO & 2366\\
$(147, 43)$ & 11 & $(31, 9)$ & 8 & 1 & YES & YES & YES & 2611 & 2367\\
$(147, 41)$ & 11 & $(93, 26)$ & 10 & 3 & YES & YES & YES & NO & 2368\\
$(148, 65)$ & 11 & $(5, 2)$ & 3 & 1 & YES & YES & YES & -- & 2369\\
$(148, 65)$ & 11 & $(34, 15)$ & 8 & 2 & YES & YES & YES & 2670 & 2370\\
$(149, 40)$ & 11 & $(3, 1)$ & 2 & 1 & YES & YES & YES & -- & 2371\\
$(149, 40)$ & 11 & $(3, 1)$ & 2 & 1 & YES & YES & YES & NO & 2372\\
$(149, 44)$ & 11 & $(3, 1)$ & 2 & 1 & YES & YES & YES & -- & 2373\\
$(149, 44)$ & 11 & $(8, 3)$ & 4 & 1 & YES & YES & YES & -- & 2374\\
$(149, 41)$ & 11 & $(11, 3)$ & 5 & 1 & YES & YES & YES & -- & 2375\\
$(149, 41)$ & 11 & $(13, 4)$ & 6 & 1 & YES & YES & YES & NO & 2376\\
$(149, 41)$ & 11 & $(32, 9)$ & 8 & 1 & YES & YES & YES & NO & 2377\\
$(149, 44)$ & 11 & $(47, 14)$ & 9 & 1 & YES & YES & YES & NO & 2378\\
$(149, 44)$ & 11 & $(64, 19)$ & 9 & 1 & YES & YES & YES & NO & 2379\\
$(151, 62)$ & 11 & $(3, 1)$ & 2 & 1 & YES & YES & YES & -- & 2380\\
$(151, 34)$ & 12 & $(5, 2)$ & 3 & 1 & YES & YES & YES & NO & 2381\\
$(151, 62)$ & 11 & $(5, 2)$ & 3 & 1 & YES & YES & YES & -- & 2382\\
$(151, 62)$ & 11 & $(9, 2)$ & 5 & 1 & YES & YES & YES & -- & 2383\\
$(151, 62)$ & 11 & $(22, 9)$ & 7 & 1 & YES & YES & YES & 2455 & 2384\\
$(152, 59)$ & 11 & $(2, 1)$ & 1 & 2 & NO & YES & NO(2) & -- & 2385\\
$(152, 59)$ & 11 & $(3, 1)$ & 2 & 1 & YES & YES & YES & -- & 2386\\
$(152, 55)$ & 12 & $(4, 1)$ & 3 & 4 & YES & YES & YES & 2113 & 2387\\
$(152, 63)$ & 11 & $(5, 2)$ & 3 & 1 & YES & YES & YES & -- & 2388\\
$(152, 41)$ & 11 & $(7, 3)$ & 4 & 1 & YES & YES & YES & -- & 2389\\
$(152, 55)$ & 12 & $(8, 3)$ & 4 & 8 & YES & YES & YES & NO & 2390\\
$(152, 63)$ & 11 & $(8, 3)$ & 4 & 8 & YES & YES & YES & NO & 2391\\
$(152, 45)$ & 12 & $(11, 3)$ & 5 & 1 & YES & YES & YES & NO & 2392\\
$(152, 41)$ & 11 & $(13, 4)$ & 6 & 1 & YES & YES & YES & NO & 2393\\
$(152, 45)$ & 12 & $(24, 7)$ & 7 & 8 & YES & YES & YES & NO & 2394\\
$(152, 63)$ & 11 & $(111, 46)$ & 10 & 1 & YES & YES & YES & NO & 2395\\
$(153, 64)$ & 11 & $(7, 3)$ & 4 & 1 & YES & YES & YES & NO & 2396\\
$(153, 35)$ & 12 & $(31, 7)$ & 8 & 1 & YES & YES & YES & NO & 2397\\
$(154, 59)$ & 11 & $(2, 1)$ & 1 & 2 & YES & YES & YES & -- & 2398\\
$(154, 65)$ & 11 & $(3, 1)$ & 2 & 1 & YES & YES & YES & -- & 2399\\
$(154, 45)$ & 11 & $(4, 1)$ & 3 & 2 & YES & YES & YES & -- & 2400\\
$(154, 59)$ & 11 & $(5, 2)$ & 3 & 1 & YES & YES & YES & -- & 2401\\
$(154, 65)$ & 11 & $(5, 2)$ & 3 & 1 & YES & YES & YES & -- & 2402\\
$(154, 59)$ & 11 & $(7, 2)$ & 4 & 7 & YES & YES & YES & -- & 2403\\
$(154, 45)$ & 11 & $(10, 3)$ & 5 & 2 & YES & YES & YES & -- & 2404\\
$(154, 65)$ & 11 & $(12, 5)$ & 5 & 2 & YES & YES & YES & 2195 & 2405\\
$(154, 65)$ & 11 & $(17, 7)$ & 6 & 1 & YES & YES & YES & NO & 2406\\
$(154, 59)$ & 11 & $(107, 41)$ & 10 & 1 & YES & YES & YES & NO & 2407\\
$(154, 45)$ & 11 & $(147, 43)$ & 11 & 7 & YES & YES & YES & NO & 2408\\
$(155, 48)$ & 12 & $(3, 1)$ & 2 & 1 & YES & YES & YES & -- & 2409\\
$(155, 64)$ & 11 & $(5, 2)$ & 3 & 5 & YES & YES & YES & -- & 2410\\
$(155, 64)$ & 11 & $(9, 4)$ & 5 & 1 & YES & YES & YES & NO & 2411\\
$(155, 48)$ & 12 & $(71, 22)$ & 10 & 1 & YES & YES & YES & 2591 & 2412\\
$(156, 43)$ & 12 & $(5, 2)$ & 3 & 1 & YES & YES & YES & -- & 2413\\
$(156, 43)$ & 12 & $(10, 3)$ & 5 & 2 & YES & YES & YES & NO & 2414\\
$(157, 69)$ & 11 & $(2, 1)$ & 1 & 1 & YES & YES & YES & -- & 2415\\
$(157, 46)$ & 11 & $(3, 1)$ & 2 & 1 & YES & YES & YES & -- & 2416\\
$(157, 46)$ & 11 & $(3, 1)$ & 2 & 1 & YES & YES & YES & NO & 2417\\
$(157, 46)$ & 11 & $(5, 2)$ & 3 & 1 & YES & YES & YES & -- & 2418\\
$(157, 58)$ & 11 & $(5, 2)$ & 3 & 1 & YES & YES & YES & -- & 2419\\
$(157, 28)$ & 13 & $(6, 1)$ & 5 & 1 & YES & YES & YES & NO & 2420\\
$(157, 58)$ & 11 & $(7, 3)$ & 4 & 1 & YES & YES & YES & NO & 2421\\
$(157, 60)$ & 11 & $(7, 3)$ & 4 & 1 & YES & YES & YES & NO & 2422\\
$(157, 58)$ & 11 & $(25, 9)$ & 7 & 1 & YES & YES & YES & NO & 2423\\
$(157, 46)$ & 11 & $(41, 12)$ & 8 & 1 & YES & YES & YES & 2333 & 2424\\
$(157, 65)$ & 12 & $(128, 53)$ & 11 & 1 & YES & YES & YES & NO & 2425\\
$(157, 65)$ & 12 & $(157, 65)$ & 12 & 157 & YES & YES & YES & NO & 2426\\
$(158, 57)$ & 11 & $(4, 1)$ & 3 & 2 & YES & YES & YES & -- & 2427\\
$(158, 57)$ & 11 & $(4, 1)$ & 3 & 2 & YES & YES & YES & NO & 2428\\
$(158, 61)$ & 11 & $(7, 2)$ & 4 & 1 & YES & YES & YES & NO & 2429\\
$(158, 61)$ & 11 & $(8, 3)$ & 4 & 2 & YES & YES & YES & NO & 2430\\
$(158, 61)$ & 11 & $(75, 29)$ & 9 & 1 & YES & YES & YES & NO & 2431\\
$(158, 57)$ & 11 & $(158, 57)$ & 11 & 158 & YES & YES & YES & NO & 2432\\
$(158, 61)$ & 11 & $(158, 61)$ & 11 & 158 & YES & YES & YES & NO & 2433\\
$(159, 44)$ & 11 & $(3, 1)$ & 2 & 3 & YES & YES & YES & -- & 2434\\
$(159, 44)$ & 11 & $(3, 1)$ & 2 & 3 & YES & YES & YES & NO & 2435\\
$(159, 47)$ & 11 & $(5, 2)$ & 3 & 1 & YES & YES & YES & -- & 2436\\
$(159, 47)$ & 11 & $(5, 2)$ & 3 & 1 & YES & YES & YES & NO & 2437\\
$(159, 44)$ & 11 & $(7, 3)$ & 4 & 1 & YES & YES & YES & -- & 2438\\
$(159, 44)$ & 11 & $(7, 3)$ & 4 & 1 & YES & YES & YES & NO & 2439\\
$(159, 47)$ & 11 & $(7, 2)$ & 4 & 1 & YES & YES & YES & -- & 2440\\
$(159, 47)$ & 11 & $(7, 2)$ & 4 & 1 & YES & YES & YES & NO & 2441\\
$(159, 62)$ & 11 & $(7, 2)$ & 4 & 1 & YES & YES & YES & -- & 2442\\
$(159, 37)$ & 12 & $(8, 3)$ & 4 & 1 & YES & YES & YES & -- & 2443\\
$(159, 37)$ & 12 & $(8, 3)$ & 4 & 1 & YES & YES & YES & NO & 2444\\
$(159, 47)$ & 11 & $(13, 4)$ & 6 & 1 & YES & YES & YES & NO & 2445\\
$(159, 44)$ & 11 & $(17, 5)$ & 6 & 1 & YES & YES & YES & NO & 2446\\
$(159, 44)$ & 11 & $(105, 29)$ & 10 & 3 & YES & YES & YES & NO & 2447\\
$(160, 67)$ & 11 & $(3, 1)$ & 2 & 1 & YES & YES & YES & -- & 2448\\
$(160, 67)$ & 11 & $(3, 1)$ & 2 & 1 & YES & YES & YES & NO & 2449\\
$(160, 67)$ & 11 & $(5, 2)$ & 3 & 5 & YES & YES & YES & -- & 2450\\
$(160, 67)$ & 11 & $(5, 2)$ & 3 & 5 & YES & YES & YES & NO & 2451\\
$(160, 67)$ & 11 & $(9, 4)$ & 5 & 1 & YES & YES & YES & NO & 2452\\
$(161, 68)$ & 11 & $(2, 1)$ & 1 & 1 & YES & YES & YES & -- & 2453\\
$(161, 66)$ & 11 & $(3, 1)$ & 2 & 1 & YES & YES & YES & -- & 2454\\
$(161, 66)$ & 11 & $(17, 7)$ & 6 & 1 & YES & YES & YES & 2384 & 2455\\
$(162, 49)$ & 12 & $(2, 1)$ & 1 & 2 & YES & YES & YES & -- & 2456\\
$(162, 49)$ & 12 & $(2, 1)$ & 1 & 2 & YES & YES & YES & NO & 2457\\
$(163, 63)$ & 11 & $(2, 1)$ & 1 & 1 & NO & YES & YES & -- & 2458\\
$(163, 62)$ & 11 & $(4, 1)$ & 3 & 1 & YES & YES & YES & -- & 2459\\
$(163, 62)$ & 11 & $(4, 1)$ & 3 & 1 & YES & YES & YES & NO & 2460\\
$(163, 45)$ & 12 & $(5, 2)$ & 3 & 1 & YES & YES & YES & -- & 2461\\
$(163, 71)$ & 11 & $(5, 2)$ & 3 & 1 & YES & YES & YES & NO & 2462\\
$(163, 62)$ & 11 & $(7, 3)$ & 4 & 1 & YES & YES & YES & NO & 2463\\
$(163, 63)$ & 11 & $(7, 2)$ & 4 & 1 & YES & YES & YES & -- & 2464\\
$(164, 45)$ & 12 & $(25, 7)$ & 7 & 1 & YES & YES & YES & NO & 2465\\
$(165, 64)$ & 11 & $(2, 1)$ & 1 & 1 & YES & YES & YES & -- & 2466\\
$(165, 61)$ & 11 & $(3, 1)$ & 2 & 3 & YES & YES & YES & -- & 2467\\
$(165, 61)$ & 11 & $(3, 1)$ & 2 & 3 & YES & YES & YES & NO & 2468\\
$(165, 64)$ & 11 & $(3, 1)$ & 2 & 3 & YES & YES & YES & -- & 2469\\
$(165, 64)$ & 11 & $(3, 1)$ & 2 & 3 & YES & YES & YES & NO & 2470\\
$(165, 61)$ & 11 & $(4, 1)$ & 3 & 1 & YES & YES & YES & NO & 2471\\
$(165, 61)$ & 11 & $(5, 2)$ & 3 & 5 & YES & YES & YES & -- & 2472\\
$(165, 46)$ & 11 & $(7, 3)$ & 4 & 1 & YES & YES & YES & -- & 2473\\
$(166, 61)$ & 11 & $(3, 1)$ & 2 & 1 & YES & YES & YES & -- & 2474\\
$(166, 49)$ & 11 & $(71, 21)$ & 9 & 1 & YES & YES & YES & NO & 2475\\
$(166, 61)$ & 11 & $(166, 61)$ & 11 & 166 & YES & YES & YES & NO & 2476\\
$(167, 64)$ & 11 & $(2, 1)$ & 1 & 1 & YES & YES & YES & -- & 2477\\
$(167, 69)$ & 11 & $(2, 1)$ & 1 & 1 & YES & YES & YES & -- & 2478\\
$(167, 69)$ & 11 & $(3, 1)$ & 2 & 1 & YES & YES & YES & -- & 2479\\
$(167, 69)$ & 11 & $(3, 1)$ & 2 & 1 & YES & YES & YES & NO & 2480\\
$(167, 51)$ & 12 & $(5, 2)$ & 3 & 1 & YES & YES & YES & -- & 2481\\
$(167, 69)$ & 11 & $(5, 2)$ & 3 & 1 & YES & YES & YES & -- & 2482\\
$(167, 69)$ & 11 & $(7, 2)$ & 4 & 1 & YES & YES & YES & -- & 2483\\
$(167, 64)$ & 11 & $(8, 3)$ & 4 & 1 & YES & YES & YES & NO & 2484\\
$(167, 46)$ & 11 & $(18, 5)$ & 6 & 1 & YES & YES & YES & NO & 2485\\
$(167, 69)$ & 11 & $(22, 9)$ & 7 & 1 & YES & YES & YES & NO & 2486\\
$(167, 69)$ & 11 & $(41, 17)$ & 8 & 1 & YES & YES & YES & 2834 & 2487\\
$(168, 71)$ & 11 & $(2, 1)$ & 1 & 2 & YES & YES & YES & -- & 2488\\
$(168, 71)$ & 11 & $(3, 1)$ & 2 & 3 & YES & YES & YES & -- & 2489\\
$(168, 65)$ & 12 & $(4, 1)$ & 3 & 4 & YES & YES & YES & NO & 2490\\
$(168, 65)$ & 12 & $(44, 17)$ & 8 & 4 & YES & YES & YES & NO & 2491\\
$(168, 65)$ & 12 & $(75, 29)$ & 9 & 3 & YES & YES & YES & NO & 2492\\
$(168, 71)$ & 11 & $(168, 71)$ & 11 & 168 & YES & YES & YES & NO & 2493\\
$(169, 64)$ & 11 & $(2, 1)$ & 1 & 1 & YES & YES & YES & NO & 2494\\
$(169, 71)$ & 11 & $(2, 1)$ & 1 & 1 & YES & YES & YES & -- & 2495\\
$(169, 70)$ & 11 & $(3, 1)$ & 2 & 1 & YES & YES & YES & -- & 2496\\
$(169, 70)$ & 11 & $(3, 1)$ & 2 & 1 & YES & YES & YES & NO & 2497\\
$(169, 50)$ & 11 & $(5, 2)$ & 3 & 1 & YES & YES & YES & -- & 2498\\
$(169, 70)$ & 11 & $(5, 2)$ & 3 & 1 & YES & YES & YES & -- & 2499\\
$(169, 50)$ & 11 & $(7, 3)$ & 4 & 1 & YES & YES & YES & -- & 2500\\
$(169, 70)$ & 11 & $(7, 2)$ & 4 & 1 & YES & YES & YES & -- & 2501\\
$(169, 71)$ & 11 & $(8, 3)$ & 4 & 1 & YES & YES & YES & NO & 2502\\
$(169, 71)$ & 11 & $(17, 7)$ & 6 & 1 & YES & YES & YES & NO & 2503\\
$(169, 71)$ & 11 & $(31, 13)$ & 7 & 1 & YES & YES & YES & NO & 2504\\
$(169, 38)$ & 13 & $(40, 9)$ & 9 & 1 & YES & YES & YES & NO & 2505\\
$(169, 50)$ & 11 & $(61, 18)$ & 9 & 1 & YES & YES & YES & NO & 2506\\
$(170, 47)$ & 11 & $(5, 2)$ & 3 & 5 & YES & YES & YES & NO & 2507\\
$(170, 47)$ & 11 & $(7, 3)$ & 4 & 1 & YES & YES & YES & -- & 2508\\
$(170, 47)$ & 11 & $(7, 3)$ & 4 & 1 & YES & YES & YES & NO & 2509\\
$(170, 47)$ & 11 & $(8, 3)$ & 4 & 2 & YES & YES & YES & NO & 2510\\
$(171, 50)$ & 11 & $(2, 1)$ & 1 & 1 & YES & YES & YES & NO & 2511\\
$(171, 65)$ & 11 & $(2, 1)$ & 1 & 1 & YES & YES & YES & NO & 2512\\
$(171, 65)$ & 11 & $(3, 1)$ & 2 & 3 & YES & YES & YES & -- & 2513\\
$(171, 65)$ & 11 & $(3, 1)$ & 2 & 3 & YES & YES & YES & NO & 2514\\
$(171, 65)$ & 11 & $(5, 2)$ & 3 & 1 & YES & YES & YES & -- & 2515\\
$(171, 50)$ & 11 & $(7, 3)$ & 4 & 1 & YES & YES & YES & -- & 2516\\
$(171, 65)$ & 11 & $(7, 3)$ & 4 & 1 & YES & YES & YES & 2829 & 2517\\
$(171, 65)$ & 11 & $(9, 4)$ & 5 & 9 & YES & YES & YES & NO & 2518\\
$(171, 50)$ & 11 & $(13, 3)$ & 6 & 1 & YES & YES & YES & NO & 2519\\
$(171, 65)$ & 11 & $(37, 14)$ & 8 & 1 & YES & YES & YES & NO & 2520\\
$(172, 71)$ & 11 & $(3, 1)$ & 2 & 1 & YES & YES & YES & NO & 2521\\
$(172, 75)$ & 12 & $(3, 1)$ & 2 & 1 & YES & YES & YES & -- & 2522\\
$(172, 75)$ & 12 & $(5, 2)$ & 3 & 1 & YES & YES & YES & NO & 2523\\
$(172, 71)$ & 11 & $(29, 12)$ & 7 & 1 & YES & YES & YES & 2646 & 2524\\
$(172, 63)$ & 11 & $(112, 41)$ & 10 & 4 & YES & YES & YES & NO & 2525\\
$(173, 64)$ & 11 & $(5, 2)$ & 3 & 1 & YES & YES & YES & -- & 2526\\
$(173, 73)$ & 11 & $(5, 2)$ & 3 & 1 & YES & YES & YES & -- & 2527\\
$(173, 66)$ & 11 & $(7, 3)$ & 4 & 1 & YES & YES & YES & NO & 2528\\
$(173, 66)$ & 11 & $(9, 2)$ & 5 & 1 & YES & YES & YES & NO & 2529\\
$(173, 64)$ & 11 & $(11, 4)$ & 5 & 1 & YES & YES & YES & NO & 2530\\
$(173, 64)$ & 11 & $(119, 44)$ & 10 & 1 & YES & YES & YES & NO & 2531\\
$(173, 73)$ & 11 & $(154, 65)$ & 11 & 1 & YES & YES & YES & NO & 2532\\
$(173, 64)$ & 11 & $(173, 64)$ & 11 & 173 & YES & YES & YES & NO & 2533\\
$(175, 67)$ & 11 & $(2, 1)$ & 1 & 1 & NO & YES & YES & -- & 2534\\
$(175, 67)$ & 11 & $(5, 2)$ & 3 & 5 & YES & YES & YES & -- & 2535\\
$(175, 67)$ & 11 & $(5, 2)$ & 3 & 5 & YES & YES & YES & NO & 2536\\
$(175, 67)$ & 11 & $(13, 5)$ & 5 & 1 & YES & YES & YES & NO & 2537\\
$(175, 67)$ & 11 & $(115, 44)$ & 10 & 5 & YES & YES & YES & 3121 & 2538\\
$(176, 65)$ & 11 & $(5, 2)$ & 3 & 1 & YES & YES & YES & -- & 2539\\
$(176, 65)$ & 11 & $(7, 3)$ & 4 & 1 & YES & YES & YES & NO & 2540\\
$(177, 65)$ & 11 & $(2, 1)$ & 1 & 1 & YES & YES & YES & -- & 2541\\
$(177, 65)$ & 11 & $(3, 1)$ & 2 & 3 & YES & YES & YES & -- & 2542\\
$(177, 65)$ & 11 & $(5, 2)$ & 3 & 1 & YES & YES & YES & -- & 2543\\
$(177, 49)$ & 11 & $(7, 3)$ & 4 & 1 & YES & YES & YES & NO & 2544\\
$(177, 49)$ & 11 & $(17, 5)$ & 6 & 1 & YES & YES & YES & NO & 2545\\
$(177, 65)$ & 11 & $(27, 10)$ & 7 & 3 & YES & YES & YES & NO & 2546\\
$(177, 65)$ & 11 & $(30, 11)$ & 7 & 3 & YES & YES & YES & NO & 2547\\
$(178, 69)$ & 11 & $(2, 1)$ & 1 & 2 & YES & YES & YES & -- & 2548\\
$(178, 69)$ & 11 & $(3, 1)$ & 2 & 1 & YES & YES & YES & -- & 2549\\
$(178, 69)$ & 11 & $(3, 1)$ & 2 & 1 & YES & YES & YES & NO & 2550\\
$(178, 69)$ & 11 & $(3, 1)$ & 2 & 1 & YES & YES & YES & NO & 2551\\
$(178, 69)$ & 11 & $(5, 2)$ & 3 & 1 & YES & YES & YES & -- & 2552\\
$(178, 69)$ & 11 & $(13, 5)$ & 5 & 1 & YES & YES & YES & NO & 2553\\
$(178, 69)$ & 11 & $(21, 8)$ & 6 & 1 & YES & YES & YES & NO & 2554\\
$(178, 69)$ & 11 & $(23, 9)$ & 7 & 1 & YES & YES & YES & NO & 2555\\
$(179, 75)$ & 11 & $(2, 1)$ & 1 & 1 & YES & YES & YES & -- & 2556\\
$(179, 50)$ & 11 & $(3, 1)$ & 2 & 1 & YES & YES & YES & -- & 2557\\
$(179, 74)$ & 11 & $(3, 1)$ & 2 & 1 & YES & YES & YES & NO & 2558\\
$(179, 75)$ & 11 & $(3, 1)$ & 2 & 1 & YES & YES & YES & -- & 2559\\
$(179, 75)$ & 11 & $(3, 1)$ & 2 & 1 & YES & YES & YES & NO & 2560\\
$(179, 78)$ & 12 & $(3, 1)$ & 2 & 1 & YES & YES & YES & -- & 2561\\
$(179, 74)$ & 11 & $(17, 7)$ & 6 & 1 & YES & YES & YES & 2133 & 2562\\
$(179, 74)$ & 11 & $(121, 50)$ & 10 & 1 & YES & YES & YES & NO & 2563\\
$(179, 75)$ & 11 & $(179, 75)$ & 11 & 179 & YES & YES & YES & NO & 2564\\
$(180, 41)$ & 12 & $(7, 3)$ & 4 & 1 & YES & YES & YES & -- & 2565\\
$(180, 41)$ & 12 & $(8, 3)$ & 4 & 4 & YES & YES & YES & -- & 2566\\
$(181, 50)$ & 11 & $(2, 1)$ & 1 & 1 & YES & YES & YES & -- & 2567\\
$(181, 65)$ & 12 & $(2, 1)$ & 1 & 1 & YES & YES & YES & NO & 2568\\
$(181, 75)$ & 11 & $(2, 1)$ & 1 & 1 & YES & YES & YES & -- & 2569\\
$(181, 76)$ & 11 & $(2, 1)$ & 1 & 1 & YES & YES & YES & -- & 2570\\
$(181, 53)$ & 12 & $(3, 1)$ & 2 & 1 & YES & YES & YES & -- & 2571\\
$(181, 53)$ & 12 & $(3, 1)$ & 2 & 1 & YES & YES & YES & NO & 2572\\
$(181, 70)$ & 11 & $(5, 2)$ & 3 & 1 & YES & YES & YES & -- & 2573\\
$(181, 75)$ & 11 & $(5, 2)$ & 3 & 1 & YES & YES & YES & -- & 2574\\
$(181, 76)$ & 11 & $(5, 2)$ & 3 & 1 & YES & YES & YES & -- & 2575\\
$(181, 53)$ & 12 & $(11, 3)$ & 5 & 1 & YES & YES & YES & NO & 2576\\
$(181, 55)$ & 12 & $(11, 3)$ & 5 & 1 & YES & YES & YES & NO & 2577\\
$(181, 76)$ & 11 & $(19, 8)$ & 6 & 1 & YES & YES & YES & NO & 2578\\
$(181, 53)$ & 12 & $(24, 7)$ & 7 & 1 & YES & YES & YES & NO & 2579\\
$(181, 76)$ & 11 & $(31, 13)$ & 7 & 1 & YES & YES & YES & NO & 2580\\
$(181, 50)$ & 11 & $(76, 21)$ & 9 & 1 & YES & YES & YES & NO & 2581\\
$(181, 41)$ & 12 & $(115, 26)$ & 11 & 1 & YES & YES & YES & NO & 2582\\
$(181, 70)$ & 11 & $(119, 46)$ & 10 & 1 & YES & YES & YES & NO & 2583\\
$(182, 71)$ & 12 & $(3, 1)$ & 2 & 1 & YES & YES & YES & -- & 2584\\
$(182, 71)$ & 12 & $(3, 1)$ & 2 & 1 & YES & YES & YES & NO & 2585\\
$(183, 71)$ & 11 & $(2, 1)$ & 1 & 1 & YES & YES & YES & -- & 2586\\
$(183, 71)$ & 11 & $(4, 1)$ & 3 & 1 & YES & YES & YES & NO & 2587\\
$(183, 71)$ & 11 & $(8, 3)$ & 4 & 1 & YES & YES & YES & NO & 2588\\
$(183, 71)$ & 11 & $(85, 33)$ & 10 & 1 & YES & YES & YES & NO & 2589\\
$(184, 71)$ & 12 & $(4, 1)$ & 3 & 4 & YES & YES & YES & NO & 2590\\
$(184, 57)$ & 12 & $(42, 13)$ & 9 & 2 & YES & YES & YES & 2412 & 2591\\
$(184, 77)$ & 12 & $(184, 77)$ & 12 & 184 & YES & YES & YES & NO & 2592\\
$(186, 71)$ & 11 & $(2, 1)$ & 1 & 2 & YES & YES & YES & -- & 2593\\
$(186, 71)$ & 11 & $(4, 1)$ & 3 & 2 & YES & YES & YES & -- & 2594\\
$(186, 71)$ & 11 & $(4, 1)$ & 3 & 2 & YES & YES & YES & NO & 2595\\
$(186, 71)$ & 11 & $(7, 3)$ & 4 & 1 & YES & YES & YES & NO & 2596\\
$(186, 71)$ & 11 & $(8, 3)$ & 4 & 2 & YES & YES & YES & 2173 & 2597\\
$(186, 71)$ & 11 & $(34, 13)$ & 7 & 2 & YES & YES & YES & NO & 2598\\
$(187, 71)$ & 11 & $(2, 1)$ & 1 & 1 & YES & YES & YES & NO & 2599\\
$(187, 71)$ & 11 & $(3, 1)$ & 2 & 1 & YES & YES & YES & -- & 2600\\
$(187, 71)$ & 11 & $(4, 1)$ & 3 & 1 & YES & YES & YES & NO & 2601\\
$(187, 71)$ & 11 & $(18, 7)$ & 6 & 1 & YES & YES & YES & NO & 2602\\
$(187, 71)$ & 11 & $(50, 19)$ & 8 & 1 & YES & YES & YES & NO & 2603\\
$(188, 69)$ & 11 & $(3, 1)$ & 2 & 1 & YES & YES & YES & -- & 2604\\
$(188, 79)$ & 11 & $(7, 2)$ & 4 & 1 & YES & YES & YES & NO & 2605\\
$(188, 79)$ & 11 & $(8, 3)$ & 4 & 4 & YES & YES & YES & NO & 2606\\
$(188, 79)$ & 11 & $(17, 7)$ & 6 & 1 & YES & YES & YES & NO & 2607\\
$(188, 79)$ & 11 & $(43, 18)$ & 8 & 1 & YES & YES & YES & 3132 & 2608\\
$(189, 73)$ & 12 & $(5, 1)$ & 4 & 1 & YES & YES & YES & -- & 2609\\
$(189, 73)$ & 12 & $(5, 1)$ & 4 & 1 & YES & YES & YES & NO & 2610\\
$(189, 55)$ & 12 & $(17, 5)$ & 6 & 1 & YES & YES & YES & 2367 & 2611\\
$(189, 55)$ & 12 & $(38, 11)$ & 9 & 1 & YES & YES & YES & NO & 2612\\
$(189, 83)$ & 12 & $(66, 29)$ & 9 & 3 & YES & YES & YES & NO & 2613\\
$(191, 80)$ & 11 & $(2, 1)$ & 1 & 1 & YES & YES & YES & -- & 2614\\
$(191, 71)$ & 12 & $(3, 1)$ & 2 & 1 & YES & YES & YES & -- & 2615\\
$(191, 56)$ & 12 & $(5, 2)$ & 3 & 1 & YES & YES & YES & NO & 2616\\
$(191, 74)$ & 11 & $(5, 2)$ & 3 & 1 & YES & YES & YES & -- & 2617\\
$(191, 58)$ & 12 & $(33, 10)$ & 8 & 1 & YES & YES & YES & NO & 2618\\
$(191, 56)$ & 12 & $(75, 22)$ & 10 & 1 & YES & YES & YES & 2697 & 2619\\
$(192, 73)$ & 11 & $(2, 1)$ & 1 & 2 & YES & YES & YES & NO & 2620\\
$(192, 71)$ & 11 & $(3, 1)$ & 2 & 3 & YES & YES & YES & -- & 2621\\
$(192, 71)$ & 11 & $(3, 1)$ & 2 & 3 & YES & YES & YES & NO & 2622\\
$(192, 73)$ & 11 & $(4, 1)$ & 3 & 4 & YES & YES & YES & NO & 2623\\
$(192, 73)$ & 11 & $(8, 3)$ & 4 & 8 & YES & YES & YES & NO & 2624\\
$(192, 73)$ & 11 & $(21, 8)$ & 6 & 3 & YES & YES & YES & NO & 2625\\
$(192, 73)$ & 11 & $(192, 73)$ & 11 & 192 & YES & YES & YES & NO & 2626\\
$(193, 81)$ & 11 & $(2, 1)$ & 1 & 1 & YES & YES & YES & -- & 2627\\
$(193, 81)$ & 11 & $(8, 3)$ & 4 & 1 & YES & YES & YES & NO & 2628\\
$(193, 81)$ & 11 & $(19, 8)$ & 6 & 1 & YES & YES & YES & 2197 & 2629\\
$(193, 80)$ & 12 & $(70, 29)$ & 9 & 1 & YES & YES & YES & NO & 2630\\
$(193, 81)$ & 11 & $(81, 34)$ & 9 & 1 & YES & YES & YES & NO & 2631\\
$(193, 81)$ & 11 & $(131, 55)$ & 10 & 1 & YES & YES & YES & NO & 2632\\
$(194, 75)$ & 11 & $(2, 1)$ & 1 & 2 & YES & YES & YES & -- & 2633\\
$(194, 75)$ & 11 & $(4, 1)$ & 3 & 2 & YES & YES & YES & NO & 2634\\
$(194, 75)$ & 11 & $(5, 2)$ & 3 & 1 & YES & YES & YES & -- & 2635\\
$(194, 75)$ & 11 & $(8, 3)$ & 4 & 2 & YES & YES & YES & NO & 2636\\
$(194, 75)$ & 11 & $(57, 22)$ & 9 & 1 & YES & YES & YES & NO & 2637\\
$(194, 75)$ & 11 & $(106, 41)$ & 10 & 2 & YES & YES & YES & NO & 2638\\
$(194, 75)$ & 11 & $(119, 46)$ & 10 & 1 & YES & YES & YES & NO & 2639\\
$(196, 75)$ & 11 & $(2, 1)$ & 1 & 2 & YES & YES & YES & -- & 2640\\
$(196, 75)$ & 11 & $(3, 1)$ & 2 & 1 & YES & YES & YES & -- & 2641\\
$(196, 75)$ & 11 & $(3, 1)$ & 2 & 1 & YES & YES & YES & NO & 2642\\
$(196, 75)$ & 11 & $(4, 1)$ & 3 & 4 & YES & YES & YES & -- & 2643\\
$(196, 75)$ & 11 & $(4, 1)$ & 3 & 4 & YES & YES & YES & NO & 2644\\
$(196, 75)$ & 11 & $(8, 3)$ & 4 & 4 & YES & YES & YES & NO & 2645\\
$(196, 81)$ & 11 & $(17, 7)$ & 6 & 1 & YES & YES & YES & 2524 & 2646\\
$(196, 75)$ & 11 & $(21, 8)$ & 6 & 7 & YES & YES & YES & NO & 2647\\
$(196, 81)$ & 11 & $(22, 9)$ & 7 & 2 & YES & YES & YES & NO & 2648\\
$(196, 55)$ & 12 & $(29, 8)$ & 7 & 1 & YES & YES & YES & NO & 2649\\
$(196, 81)$ & 11 & $(41, 17)$ & 8 & 1 & YES & YES & YES & NO & 2650\\
$(196, 75)$ & 11 & $(81, 31)$ & 9 & 1 & YES & YES & YES & NO & 2651\\
$(197, 76)$ & 12 & $(4, 1)$ & 3 & 1 & YES & YES & YES & NO & 2652\\
$(197, 61)$ & 13 & $(13, 4)$ & 6 & 1 & YES & YES & YES & NO & 2653\\
$(197, 43)$ & 12 & $(33, 7)$ & 8 & 1 & YES & YES & YES & NO & 2654\\
$(198, 71)$ & 12 & $(3, 1)$ & 2 & 3 & YES & YES & YES & -- & 2655\\
$(199, 76)$ & 11 & $(2, 1)$ & 1 & 1 & YES & YES & YES & -- & 2656\\
$(199, 76)$ & 11 & $(3, 1)$ & 2 & 1 & YES & YES & YES & -- & 2657\\
$(199, 55)$ & 11 & $(5, 2)$ & 3 & 1 & YES & YES & YES & -- & 2658\\
$(199, 76)$ & 11 & $(5, 2)$ & 3 & 1 & YES & YES & YES & -- & 2659\\
$(199, 76)$ & 11 & $(13, 5)$ & 5 & 1 & YES & YES & YES & NO & 2660\\
$(199, 76)$ & 11 & $(34, 13)$ & 7 & 1 & YES & YES & YES & NO & 2661\\
$(199, 74)$ & 12 & $(78, 29)$ & 10 & 1 & YES & YES & YES & NO & 2662\\
$(199, 76)$ & 11 & $(89, 34)$ & 9 & 1 & YES & YES & YES & 2803 & 2663\\
$(200, 59)$ & 12 & $(4, 1)$ & 3 & 4 & YES & YES & YES & -- & 2664\\
$(200, 61)$ & 12 & $(36, 11)$ & 8 & 4 & YES & YES & YES & NO & 2665\\
$(201, 37)$ & 14 & $(2, 1)$ & 1 & 1 & YES & YES & YES & -- & 2666\\
$(201, 77)$ & 12 & $(5, 1)$ & 4 & 1 & YES & YES & YES & NO & 2667\\
$(201, 61)$ & 12 & $(33, 10)$ & 8 & 3 & YES & YES & YES & NO & 2668\\
$(201, 83)$ & 12 & $(155, 64)$ & 11 & 1 & YES & YES & YES & NO & 2669\\
$(202, 89)$ & 12 & $(16, 7)$ & 6 & 2 & YES & YES & YES & 2370 & 2670\\
$(202, 59)$ & 12 & $(89, 26)$ & 10 & 1 & YES & YES & YES & NO & 2671\\
$(203, 57)$ & 12 & $(2, 1)$ & 1 & 1 & YES & YES & YES & -- & 2672\\
$(203, 75)$ & 12 & $(3, 1)$ & 2 & 1 & YES & YES & YES & -- & 2673\\
$(203, 75)$ & 12 & $(11, 4)$ & 5 & 1 & YES & YES & YES & NO & 2674\\
$(204, 89)$ & 12 & $(3, 1)$ & 2 & 3 & YES & YES & YES & -- & 2675\\
$(204, 89)$ & 12 & $(3, 1)$ & 2 & 3 & YES & YES & YES & NO & 2676\\
$(205, 78)$ & 12 & $(5, 2)$ & 3 & 5 & YES & YES & YES & NO & 2677\\
$(206, 85)$ & 12 & $(12, 5)$ & 5 & 2 & YES & YES & YES & 2022 & 2678\\
$(206, 47)$ & 12 & $(19, 4)$ & 7 & 1 & YES & YES & YES & NO & 2679\\
$(207, 76)$ & 11 & $(2, 1)$ & 1 & 1 & YES & YES & YES & -- & 2680\\
$(207, 76)$ & 11 & $(3, 1)$ & 2 & 3 & YES & YES & YES & -- & 2681\\
$(207, 85)$ & 12 & $(3, 1)$ & 2 & 3 & YES & YES & YES & -- & 2682\\
$(207, 85)$ & 12 & $(3, 1)$ & 2 & 3 & YES & YES & YES & NO & 2683\\
$(207, 79)$ & 11 & $(4, 1)$ & 3 & 1 & YES & YES & YES & -- & 2684\\
$(207, 79)$ & 11 & $(4, 1)$ & 3 & 1 & YES & YES & YES & NO & 2685\\
$(207, 79)$ & 11 & $(7, 2)$ & 4 & 1 & YES & YES & YES & NO & 2686\\
$(207, 79)$ & 11 & $(34, 13)$ & 7 & 1 & YES & YES & YES & 2802 & 2687\\
$(207, 79)$ & 11 & $(47, 18)$ & 8 & 1 & YES & YES & YES & 3192 & 2688\\
$(207, 79)$ & 11 & $(97, 37)$ & 10 & 1 & YES & YES & YES & NO & 2689\\
$(207, 79)$ & 11 & $(131, 50)$ & 10 & 1 & YES & YES & YES & NO & 2690\\
$(207, 85)$ & 12 & $(151, 62)$ & 11 & 1 & YES & YES & YES & NO & 2691\\
$(207, 79)$ & 11 & $(207, 79)$ & 11 & 207 & YES & YES & YES & NO & 2692\\
$(207, 85)$ & 12 & $(207, 85)$ & 12 & 207 & YES & YES & YES & NO & 2693\\
$(208, 79)$ & 11 & $(2, 1)$ & 1 & 2 & YES & YES & YES & -- & 2694\\
$(208, 79)$ & 11 & $(3, 1)$ & 2 & 1 & YES & YES & YES & -- & 2695\\
$(208, 79)$ & 11 & $(37, 14)$ & 8 & 1 & YES & YES & YES & NO & 2696\\
$(208, 61)$ & 12 & $(58, 17)$ & 9 & 2 & YES & YES & YES & 2619 & 2697\\
$(209, 80)$ & 11 & $(2, 1)$ & 1 & 1 & YES & YES & YES & -- & 2698\\
$(209, 80)$ & 11 & $(3, 1)$ & 2 & 1 & YES & YES & YES & -- & 2699\\
$(209, 81)$ & 11 & $(5, 2)$ & 3 & 1 & YES & YES & YES & -- & 2700\\
$(209, 81)$ & 11 & $(13, 5)$ & 5 & 1 & YES & YES & YES & NO & 2701\\
$(209, 80)$ & 11 & $(21, 8)$ & 6 & 1 & YES & YES & YES & NO & 2702\\
$(209, 80)$ & 11 & $(34, 13)$ & 7 & 1 & YES & YES & YES & NO & 2703\\
$(211, 89)$ & 12 & $(2, 1)$ & 1 & 1 & YES & YES & YES & -- & 2704\\
$(211, 78)$ & 12 & $(46, 17)$ & 8 & 1 & YES & YES & YES & NO & 2705\\
$(212, 81)$ & 11 & $(2, 1)$ & 1 & 2 & YES & YES & YES & -- & 2706\\
$(212, 93)$ & 12 & $(2, 1)$ & 1 & 2 & YES & YES & YES & -- & 2707\\
$(212, 81)$ & 11 & $(3, 1)$ & 2 & 1 & YES & YES & YES & -- & 2708\\
$(212, 81)$ & 11 & $(3, 1)$ & 2 & 1 & YES & YES & YES & NO & 2709\\
$(212, 81)$ & 11 & $(4, 1)$ & 3 & 4 & YES & YES & YES & -- & 2710\\
$(212, 81)$ & 11 & $(4, 1)$ & 3 & 4 & YES & YES & YES & NO & 2711\\
$(212, 93)$ & 12 & $(4, 1)$ & 3 & 4 & YES & YES & YES & -- & 2712\\
$(212, 89)$ & 11 & $(5, 2)$ & 3 & 1 & YES & YES & YES & -- & 2713\\
$(212, 81)$ & 11 & $(7, 3)$ & 4 & 1 & YES & YES & YES & NO & 2714\\
$(212, 93)$ & 12 & $(7, 3)$ & 4 & 1 & YES & YES & YES & NO & 2715\\
$(212, 81)$ & 11 & $(21, 8)$ & 6 & 1 & YES & YES & YES & 2274 & 2716\\
$(212, 63)$ & 13 & $(27, 8)$ & 7 & 1 & YES & YES & YES & NO & 2717\\
$(212, 89)$ & 11 & $(112, 47)$ & 10 & 4 & YES & YES & YES & NO & 2718\\
$(212, 81)$ & 11 & $(123, 47)$ & 10 & 1 & YES & YES & YES & NO & 2719\\
$(212, 81)$ & 11 & $(212, 81)$ & 11 & 212 & YES & YES & YES & NO & 2720\\
$(213, 59)$ & 12 & $(3, 1)$ & 2 & 3 & YES & YES & YES & -- & 2721\\
$(213, 65)$ & 12 & $(3, 1)$ & 2 & 3 & YES & YES & YES & -- & 2722\\
$(213, 65)$ & 12 & $(3, 1)$ & 2 & 3 & YES & YES & YES & NO & 2723\\
$(213, 59)$ & 12 & $(5, 2)$ & 3 & 1 & YES & YES & YES & NO & 2724\\
$(213, 62)$ & 12 & $(7, 2)$ & 4 & 1 & YES & YES & YES & NO & 2725\\
$(213, 59)$ & 12 & $(10, 3)$ & 5 & 1 & YES & YES & YES & NO & 2726\\
$(213, 88)$ & 12 & $(29, 12)$ & 7 & 1 & YES & YES & YES & NO & 2727\\
$(213, 65)$ & 12 & $(36, 11)$ & 8 & 3 & YES & YES & YES & NO & 2728\\
$(213, 88)$ & 12 & $(167, 69)$ & 11 & 1 & YES & YES & YES & NO & 2729\\
$(214, 79)$ & 12 & $(3, 1)$ & 2 & 1 & YES & YES & YES & -- & 2730\\
$(214, 79)$ & 12 & $(4, 1)$ & 3 & 2 & YES & YES & YES & -- & 2731\\
$(214, 79)$ & 12 & $(27, 10)$ & 7 & 1 & YES & YES & YES & NO & 2732\\
$(214, 79)$ & 12 & $(46, 17)$ & 8 & 2 & YES & YES & YES & NO & 2733\\
$(215, 83)$ & 12 & $(4, 1)$ & 3 & 1 & YES & YES & YES & -- & 2734\\
$(215, 83)$ & 12 & $(4, 1)$ & 3 & 1 & YES & YES & YES & NO & 2735\\
$(215, 82)$ & 12 & $(6, 1)$ & 5 & 1 & YES & YES & YES & -- & 2736\\
$(215, 63)$ & 12 & $(7, 2)$ & 4 & 1 & YES & YES & YES & NO & 2737\\
$(215, 63)$ & 12 & $(11, 2)$ & 6 & 1 & YES & YES & YES & NO & 2738\\
$(215, 79)$ & 12 & $(11, 4)$ & 5 & 1 & YES & YES & YES & NO & 2739\\
$(215, 51)$ & 13 & $(17, 4)$ & 7 & 1 & YES & YES & YES & NO & 2740\\
$(215, 63)$ & 12 & $(24, 7)$ & 7 & 1 & YES & YES & YES & NO & 2741\\
$(215, 83)$ & 12 & $(31, 12)$ & 7 & 1 & YES & YES & YES & NO & 2742\\
$(215, 82)$ & 12 & $(97, 37)$ & 10 & 1 & YES & YES & YES & NO & 2743\\
$(215, 58)$ & 12 & $(100, 27)$ & 10 & 5 & YES & YES & YES & NO & 2744\\
$(215, 83)$ & 12 & $(101, 39)$ & 10 & 1 & YES & YES & YES & 2946 & 2745\\
$(217, 60)$ & 12 & $(2, 1)$ & 1 & 1 & YES & YES & YES & -- & 2746\\
$(217, 60)$ & 12 & $(5, 2)$ & 3 & 1 & YES & YES & YES & -- & 2747\\
$(217, 60)$ & 12 & $(5, 2)$ & 3 & 1 & YES & YES & YES & NO & 2748\\
$(217, 60)$ & 12 & $(10, 3)$ & 5 & 1 & YES & YES & YES & NO & 2749\\
$(217, 78)$ & 12 & $(39, 14)$ & 8 & 1 & YES & YES & YES & NO & 2750\\
$(217, 90)$ & 13 & $(217, 90)$ & 13 & 217 & YES & YES & YES & NO & 2751\\
$(218, 49)$ & 13 & $(3, 1)$ & 2 & 1 & YES & YES & YES & NO & 2752\\
$(218, 85)$ & 12 & $(3, 1)$ & 2 & 1 & YES & YES & YES & -- & 2753\\
$(218, 85)$ & 12 & $(100, 39)$ & 10 & 2 & YES & YES & YES & 2947 & 2754\\
$(218, 85)$ & 12 & $(218, 85)$ & 12 & 218 & YES & YES & YES & NO & 2755\\
$(219, 79)$ & 12 & $(2, 1)$ & 1 & 1 & YES & YES & YES & -- & 2756\\
$(219, 64)$ & 12 & $(3, 1)$ & 2 & 3 & YES & YES & YES & -- & 2757\\
$(219, 79)$ & 12 & $(3, 1)$ & 2 & 3 & YES & YES & YES & -- & 2758\\
$(219, 79)$ & 12 & $(4, 1)$ & 3 & 1 & YES & YES & YES & NO & 2759\\
$(219, 85)$ & 12 & $(4, 1)$ & 3 & 1 & YES & YES & YES & NO & 2760\\
$(219, 61)$ & 12 & $(5, 2)$ & 3 & 1 & YES & YES & YES & NO & 2761\\
$(219, 65)$ & 12 & $(5, 2)$ & 3 & 1 & YES & YES & YES & -- & 2762\\
$(219, 65)$ & 12 & $(5, 2)$ & 3 & 1 & YES & YES & YES & NO & 2763\\
$(219, 65)$ & 12 & $(11, 3)$ & 5 & 1 & YES & YES & YES & 3210 & 2764\\
$(219, 79)$ & 12 & $(14, 5)$ & 6 & 1 & YES & YES & YES & NO & 2765\\
$(219, 79)$ & 12 & $(25, 9)$ & 7 & 1 & YES & YES & YES & NO & 2766\\
$(219, 64)$ & 12 & $(41, 12)$ & 8 & 1 & YES & YES & YES & NO & 2767\\
$(221, 84)$ & 12 & $(3, 1)$ & 2 & 1 & YES & YES & YES & NO & 2768\\
$(221, 84)$ & 12 & $(8, 3)$ & 4 & 1 & YES & YES & YES & NO & 2769\\
$(222, 65)$ & 13 & $(2, 1)$ & 1 & 2 & YES & YES & YES & NO & 2770\\
$(222, 65)$ & 13 & $(24, 7)$ & 7 & 6 & YES & YES & YES & NO & 2771\\
$(222, 85)$ & 12 & $(34, 13)$ & 7 & 2 & YES & YES & YES & NO & 2772\\
$(225, 98)$ & 12 & $(3, 1)$ & 2 & 3 & YES & YES & YES & -- & 2773\\
$(226, 83)$ & 12 & $(2, 1)$ & 1 & 2 & YES & YES & YES & NO & 2774\\
$(226, 63)$ & 12 & $(3, 1)$ & 2 & 1 & YES & YES & YES & -- & 2775\\
$(226, 61)$ & 12 & $(5, 2)$ & 3 & 1 & YES & YES & YES & -- & 2776\\
$(226, 69)$ & 12 & $(17, 5)$ & 6 & 1 & YES & YES & YES & NO & 2777\\
$(227, 66)$ & 12 & $(2, 1)$ & 1 & 1 & YES & YES & YES & -- & 2778\\
$(227, 86)$ & 12 & $(2, 1)$ & 1 & 1 & YES & YES & YES & -- & 2779\\
$(227, 94)$ & 12 & $(2, 1)$ & 1 & 1 & YES & YES & YES & -- & 2780\\
$(227, 86)$ & 12 & $(3, 1)$ & 2 & 1 & YES & YES & YES & -- & 2781\\
$(227, 86)$ & 12 & $(3, 1)$ & 2 & 1 & YES & YES & YES & NO & 2782\\
$(227, 86)$ & 12 & $(4, 1)$ & 3 & 1 & YES & YES & YES & NO & 2783\\
$(227, 52)$ & 13 & $(5, 1)$ & 4 & 1 & YES & YES & YES & NO & 2784\\
$(227, 52)$ & 13 & $(5, 2)$ & 3 & 1 & YES & YES & YES & -- & 2785\\
$(227, 88)$ & 12 & $(5, 2)$ & 3 & 1 & YES & YES & YES & NO & 2786\\
$(227, 86)$ & 12 & $(13, 5)$ & 5 & 1 & YES & YES & YES & NO & 2787\\
$(227, 86)$ & 12 & $(66, 25)$ & 9 & 1 & YES & YES & YES & NO & 2788\\
$(227, 86)$ & 12 & $(95, 36)$ & 10 & 1 & YES & YES & YES & 2923 & 2789\\
$(227, 94)$ & 12 & $(99, 41)$ & 10 & 1 & YES & YES & YES & NO & 2790\\
$(227, 86)$ & 12 & $(227, 86)$ & 12 & 227 & YES & YES & YES & NO & 2791\\
$(229, 95)$ & 12 & $(2, 1)$ & 1 & 1 & YES & YES & YES & -- & 2792\\
$(229, 95)$ & 12 & $(3, 1)$ & 2 & 1 & YES & YES & YES & NO & 2793\\
$(229, 63)$ & 13 & $(4, 1)$ & 3 & 1 & YES & YES & YES & -- & 2794\\
$(229, 64)$ & 12 & $(5, 2)$ & 3 & 1 & YES & YES & YES & NO & 2795\\
$(229, 63)$ & 13 & $(29, 8)$ & 7 & 1 & YES & YES & YES & NO & 2796\\
$(231, 83)$ & 12 & $(3, 1)$ & 2 & 3 & YES & YES & YES & -- & 2797\\
$(231, 83)$ & 12 & $(11, 4)$ & 5 & 11 & YES & YES & YES & NO & 2798\\
$(233, 89)$ & 11 & $(2, 1)$ & 1 & 1 & YES & YES & YES & -- & 2799\\
$(233, 89)$ & 11 & $(3, 1)$ & 2 & 1 & YES & YES & YES & -- & 2800\\
$(233, 89)$ & 11 & $(13, 5)$ & 5 & 1 & YES & YES & YES & NO & 2801\\
$(233, 89)$ & 11 & $(21, 8)$ & 6 & 1 & YES & YES & YES & 2687 & 2802\\
$(233, 89)$ & 11 & $(55, 21)$ & 8 & 1 & YES & YES & YES & 2663 & 2803\\
$(234, 43)$ & 14 & $(2, 1)$ & 1 & 2 & YES & YES & YES & -- & 2804\\
$(234, 71)$ & 12 & $(2, 1)$ & 1 & 2 & YES & YES & YES & -- & 2805\\
$(234, 53)$ & 13 & $(5, 2)$ & 3 & 1 & YES & YES & YES & -- & 2806\\
$(234, 43)$ & 14 & $(6, 1)$ & 5 & 6 & YES & YES & YES & NO & 2807\\
$(234, 53)$ & 13 & $(35, 8)$ & 8 & 1 & YES & YES & YES & 2888 & 2808\\
$(234, 71)$ & 12 & $(79, 24)$ & 10 & 1 & YES & YES & YES & NO & 2809\\
$(235, 66)$ & 12 & $(2, 1)$ & 1 & 1 & YES & YES & YES & -- & 2810\\
$(235, 97)$ & 12 & $(2, 1)$ & 1 & 1 & YES & YES & YES & NO & 2811\\
$(236, 69)$ & 12 & $(2, 1)$ & 1 & 2 & YES & YES & YES & -- & 2812\\
$(236, 69)$ & 12 & $(3, 1)$ & 2 & 1 & YES & YES & YES & -- & 2813\\
$(236, 69)$ & 12 & $(3, 1)$ & 2 & 1 & YES & YES & YES & NO & 2814\\
$(236, 69)$ & 12 & $(5, 1)$ & 4 & 1 & YES & YES & YES & NO & 2815\\
$(236, 69)$ & 12 & $(17, 5)$ & 6 & 1 & YES & YES & YES & NO & 2816\\
$(236, 69)$ & 12 & $(41, 12)$ & 8 & 1 & YES & YES & YES & NO & 2817\\
$(237, 100)$ & 12 & $(3, 1)$ & 2 & 3 & YES & YES & YES & -- & 2818\\
$(237, 64)$ & 12 & $(5, 2)$ & 3 & 1 & YES & YES & YES & NO & 2819\\
$(237, 100)$ & 12 & $(109, 46)$ & 10 & 1 & YES & YES & YES & 3044 & 2820\\
$(238, 69)$ & 13 & $(2, 1)$ & 1 & 2 & YES & YES & YES & -- & 2821\\
$(238, 69)$ & 13 & $(5, 1)$ & 4 & 1 & YES & YES & YES & NO & 2822\\
$(238, 69)$ & 13 & $(10, 3)$ & 5 & 2 & YES & YES & YES & NO & 2823\\
$(238, 69)$ & 13 & $(31, 9)$ & 8 & 1 & YES & YES & YES & NO & 2824\\
$(239, 99)$ & 12 & $(2, 1)$ & 1 & 1 & YES & YES & YES & -- & 2825\\
$(239, 70)$ & 12 & $(3, 1)$ & 2 & 1 & YES & YES & YES & -- & 2826\\
$(239, 70)$ & 12 & $(3, 1)$ & 2 & 1 & YES & YES & YES & NO & 2827\\
$(239, 99)$ & 12 & $(3, 1)$ & 2 & 1 & YES & YES & YES & -- & 2828\\
$(239, 99)$ & 12 & $(3, 1)$ & 2 & 1 & YES & YES & YES & 2517 & 2829\\
$(239, 101)$ & 12 & $(5, 2)$ & 3 & 1 & YES & YES & YES & NO & 2830\\
$(239, 99)$ & 12 & $(7, 3)$ & 4 & 1 & YES & YES & YES & NO & 2831\\
$(239, 101)$ & 12 & $(12, 5)$ & 5 & 1 & YES & YES & YES & NO & 2832\\
$(239, 70)$ & 12 & $(13, 4)$ & 6 & 1 & YES & YES & YES & NO & 2833\\
$(239, 99)$ & 12 & $(17, 7)$ & 6 & 1 & YES & YES & YES & 2487 & 2834\\
$(239, 67)$ & 13 & $(18, 5)$ & 6 & 1 & YES & YES & YES & NO & 2835\\
$(239, 71)$ & 12 & $(24, 7)$ & 7 & 1 & YES & YES & YES & NO & 2836\\
$(239, 99)$ & 12 & $(41, 17)$ & 8 & 1 & YES & YES & YES & NO & 2837\\
$(239, 99)$ & 12 & $(239, 99)$ & 12 & 239 & YES & YES & YES & NO & 2838\\
$(240, 71)$ & 12 & $(3, 1)$ & 2 & 3 & NO & YES & YES & -- & 2839\\
$(240, 71)$ & 12 & $(44, 13)$ & 8 & 4 & YES & YES & YES & NO & 2840\\
$(241, 89)$ & 12 & $(2, 1)$ & 1 & 1 & YES & YES & YES & -- & 2841\\
$(241, 101)$ & 12 & $(3, 1)$ & 2 & 1 & YES & YES & YES & -- & 2842\\
$(241, 94)$ & 12 & $(8, 3)$ & 4 & 1 & YES & YES & YES & NO & 2843\\
$(241, 94)$ & 12 & $(13, 5)$ & 5 & 1 & YES & YES & YES & NO & 2844\\
$(241, 89)$ & 12 & $(46, 17)$ & 8 & 1 & YES & YES & YES & NO & 2845\\
$(241, 89)$ & 12 & $(111, 41)$ & 10 & 1 & YES & YES & YES & 3067 & 2846\\
$(241, 89)$ & 12 & $(176, 65)$ & 11 & 1 & YES & YES & YES & NO & 2847\\
$(242, 45)$ & 14 & $(2, 1)$ & 1 & 2 & YES & YES & YES & -- & 2848\\
$(242, 71)$ & 13 & $(2, 1)$ & 1 & 2 & YES & YES & YES & -- & 2849\\
$(242, 71)$ & 13 & $(2, 1)$ & 1 & 2 & YES & YES & YES & NO & 2850\\
$(242, 71)$ & 13 & $(4, 1)$ & 3 & 2 & YES & YES & YES & NO & 2851\\
$(242, 65)$ & 12 & $(5, 2)$ & 3 & 1 & YES & YES & YES & NO & 2852\\
$(242, 71)$ & 13 & $(10, 3)$ & 5 & 2 & YES & YES & YES & NO & 2853\\
$(242, 71)$ & 13 & $(92, 27)$ & 11 & 2 & YES & YES & YES & 2948 & 2854\\
$(243, 71)$ & 12 & $(2, 1)$ & 1 & 1 & YES & YES & YES & -- & 2855\\
$(243, 94)$ & 12 & $(2, 1)$ & 1 & 1 & YES & YES & YES & -- & 2856\\
$(243, 94)$ & 12 & $(3, 1)$ & 2 & 3 & YES & YES & YES & NO & 2857\\
$(243, 53)$ & 13 & $(19, 4)$ & 7 & 1 & YES & YES & YES & NO & 2858\\
$(243, 53)$ & 13 & $(37, 8)$ & 8 & 1 & YES & YES & YES & 2965 & 2859\\
$(243, 71)$ & 12 & $(89, 26)$ & 10 & 1 & YES & YES & YES & NO & 2860\\
$(243, 94)$ & 12 & $(106, 41)$ & 10 & 1 & YES & YES & YES & NO & 2861\\
$(243, 94)$ & 12 & $(243, 94)$ & 12 & 243 & YES & YES & YES & NO & 2862\\
$(245, 69)$ & 13 & $(2, 1)$ & 1 & 1 & YES & YES & YES & -- & 2863\\
$(245, 69)$ & 13 & $(4, 1)$ & 3 & 1 & YES & YES & YES & NO & 2864\\
$(245, 69)$ & 13 & $(5, 1)$ & 4 & 5 & YES & YES & YES & NO & 2865\\
$(245, 69)$ & 13 & $(32, 9)$ & 8 & 1 & YES & YES & YES & NO & 2866\\
$(245, 69)$ & 13 & $(103, 29)$ & 11 & 1 & YES & YES & YES & 3029 & 2867\\
$(246, 73)$ & 12 & $(2, 1)$ & 1 & 2 & YES & YES & YES & NO & 2868\\
$(246, 91)$ & 12 & $(2, 1)$ & 1 & 2 & YES & YES & YES & -- & 2869\\
$(246, 91)$ & 12 & $(2, 1)$ & 1 & 2 & YES & YES & YES & NO & 2870\\
$(246, 95)$ & 12 & $(2, 1)$ & 1 & 2 & YES & YES & YES & -- & 2871\\
$(246, 91)$ & 12 & $(3, 1)$ & 2 & 3 & YES & YES & YES & -- & 2872\\
$(246, 95)$ & 12 & $(3, 1)$ & 2 & 3 & YES & YES & YES & -- & 2873\\
$(246, 95)$ & 12 & $(3, 1)$ & 2 & 3 & YES & YES & YES & NO & 2874\\
$(246, 101)$ & 12 & $(3, 1)$ & 2 & 3 & YES & YES & YES & NO & 2875\\
$(246, 91)$ & 12 & $(11, 4)$ & 5 & 1 & YES & YES & YES & NO & 2876\\
$(246, 91)$ & 12 & $(19, 7)$ & 6 & 1 & YES & YES & YES & NO & 2877\\
$(246, 101)$ & 12 & $(39, 16)$ & 8 & 3 & YES & YES & YES & NO & 2878\\
$(246, 95)$ & 12 & $(57, 22)$ & 9 & 3 & YES & YES & YES & NO & 2879\\
$(246, 95)$ & 12 & $(101, 39)$ & 10 & 1 & YES & YES & YES & NO & 2880\\
$(246, 91)$ & 12 & $(173, 64)$ & 11 & 1 & YES & YES & YES & NO & 2881\\
$(247, 69)$ & 12 & $(2, 1)$ & 1 & 1 & YES & YES & YES & -- & 2882\\
$(247, 69)$ & 12 & $(3, 1)$ & 2 & 1 & YES & YES & YES & -- & 2883\\
$(247, 69)$ & 12 & $(5, 1)$ & 4 & 1 & YES & YES & YES & NO & 2884\\
$(247, 69)$ & 12 & $(18, 5)$ & 6 & 1 & YES & YES & YES & NO & 2885\\
$(249, 95)$ & 12 & $(3, 1)$ & 2 & 3 & YES & YES & YES & -- & 2886\\
$(249, 58)$ & 13 & $(5, 2)$ & 3 & 1 & YES & YES & YES & -- & 2887\\
$(250, 57)$ & 13 & $(31, 7)$ & 8 & 1 & YES & YES & YES & 2808 & 2888\\
$(251, 74)$ & 13 & $(2, 1)$ & 1 & 1 & YES & YES & YES & -- & 2889\\
$(251, 104)$ & 12 & $(2, 1)$ & 1 & 1 & YES & YES & YES & -- & 2890\\
$(251, 74)$ & 13 & $(4, 1)$ & 3 & 1 & YES & YES & YES & NO & 2891\\
$(251, 46)$ & 15 & $(5, 1)$ & 4 & 1 & YES & YES & YES & NO & 2892\\
$(251, 104)$ & 12 & $(7, 3)$ & 4 & 1 & YES & YES & YES & NO & 2893\\
$(251, 74)$ & 13 & $(10, 3)$ & 5 & 1 & YES & YES & YES & NO & 2894\\
$(251, 104)$ & 12 & $(29, 12)$ & 7 & 1 & YES & YES & YES & NO & 2895\\
$(251, 74)$ & 13 & $(44, 13)$ & 8 & 1 & YES & YES & YES & NO & 2896\\
$(253, 60)$ & 13 & $(2, 1)$ & 1 & 1 & YES & YES & YES & -- & 2897\\
$(253, 106)$ & 12 & $(2, 1)$ & 1 & 1 & YES & YES & YES & -- & 2898\\
$(253, 98)$ & 12 & $(3, 1)$ & 2 & 1 & YES & YES & YES & NO & 2899\\
$(253, 106)$ & 12 & $(5, 2)$ & 3 & 1 & YES & YES & YES & NO & 2900\\
$(253, 106)$ & 12 & $(105, 44)$ & 10 & 1 & YES & YES & YES & 3055 & 2901\\
$(254, 105)$ & 12 & $(2, 1)$ & 1 & 2 & YES & YES & YES & -- & 2902\\
$(254, 75)$ & 12 & $(3, 1)$ & 2 & 1 & YES & YES & YES & -- & 2903\\
$(254, 71)$ & 12 & $(10, 3)$ & 5 & 2 & YES & YES & YES & NO & 2904\\
$(254, 75)$ & 12 & $(13, 4)$ & 6 & 1 & YES & YES & YES & NO & 2905\\
$(254, 75)$ & 12 & $(27, 8)$ & 7 & 1 & YES & YES & YES & NO & 2906\\
$(255, 71)$ & 13 & $(3, 1)$ & 2 & 3 & YES & YES & YES & NO & 2907\\
$(255, 71)$ & 13 & $(11, 3)$ & 5 & 1 & YES & YES & YES & NO & 2908\\
$(255, 71)$ & 13 & $(97, 27)$ & 11 & 1 & YES & YES & YES & 3006 & 2909\\
$(255, 76)$ & 13 & $(104, 31)$ & 11 & 1 & YES & YES & YES & NO & 2910\\
$(256, 75)$ & 12 & $(2, 1)$ & 1 & 2 & YES & YES & YES & -- & 2911\\
$(256, 75)$ & 12 & $(2, 1)$ & 1 & 2 & YES & YES & YES & NO & 2912\\
$(256, 99)$ & 12 & $(2, 1)$ & 1 & 2 & YES & YES & YES & -- & 2913\\
$(256, 75)$ & 12 & $(3, 1)$ & 2 & 1 & YES & YES & YES & -- & 2914\\
$(256, 97)$ & 12 & $(3, 1)$ & 2 & 1 & YES & YES & YES & -- & 2915\\
$(256, 99)$ & 12 & $(3, 1)$ & 2 & 1 & YES & YES & YES & -- & 2916\\
$(256, 99)$ & 12 & $(3, 1)$ & 2 & 1 & YES & YES & YES & NO & 2917\\
$(256, 99)$ & 12 & $(3, 1)$ & 2 & 1 & YES & YES & YES & NO & 2918\\
$(256, 99)$ & 12 & $(4, 1)$ & 3 & 4 & YES & YES & YES & -- & 2919\\
$(256, 99)$ & 12 & $(4, 1)$ & 3 & 4 & YES & YES & YES & NO & 2920\\
$(256, 75)$ & 12 & $(24, 7)$ & 7 & 8 & YES & YES & YES & NO & 2921\\
$(256, 99)$ & 12 & $(31, 12)$ & 7 & 1 & YES & YES & YES & NO & 2922\\
$(256, 97)$ & 12 & $(66, 25)$ & 9 & 2 & YES & YES & YES & 2789 & 2923\\
$(256, 99)$ & 12 & $(75, 29)$ & 9 & 1 & YES & YES & YES & NO & 2924\\
$(256, 75)$ & 12 & $(99, 29)$ & 10 & 1 & YES & YES & YES & NO & 2925\\
$(256, 99)$ & 12 & $(106, 41)$ & 10 & 2 & YES & YES & YES & 3068 & 2926\\
$(256, 99)$ & 12 & $(181, 70)$ & 11 & 1 & YES & YES & YES & NO & 2927\\
$(256, 99)$ & 12 & $(256, 99)$ & 12 & 256 & YES & YES & YES & NO & 2928\\
$(257, 108)$ & 12 & $(2, 1)$ & 1 & 1 & YES & YES & YES & -- & 2929\\
$(257, 76)$ & 12 & $(3, 1)$ & 2 & 1 & YES & YES & YES & -- & 2930\\
$(257, 108)$ & 12 & $(3, 1)$ & 2 & 1 & YES & YES & YES & -- & 2931\\
$(257, 108)$ & 12 & $(3, 1)$ & 2 & 1 & YES & YES & YES & NO & 2932\\
$(257, 76)$ & 12 & $(7, 2)$ & 4 & 1 & YES & YES & YES & NO & 2933\\
$(257, 76)$ & 12 & $(17, 5)$ & 6 & 1 & YES & YES & YES & NO & 2934\\
$(257, 108)$ & 12 & $(50, 21)$ & 8 & 1 & YES & YES & YES & NO & 2935\\
$(257, 59)$ & 14 & $(74, 17)$ & 11 & 1 & YES & YES & YES & NO & 2936\\
$(257, 108)$ & 12 & $(119, 50)$ & 10 & 1 & YES & YES & YES & 3134 & 2937\\
$(258, 109)$ & 12 & $(4, 1)$ & 3 & 2 & YES & YES & YES & -- & 2938\\
$(258, 109)$ & 12 & $(4, 1)$ & 3 & 2 & YES & YES & YES & NO & 2939\\
$(258, 109)$ & 12 & $(116, 49)$ & 10 & 2 & YES & YES & YES & 3123 & 2940\\
$(259, 76)$ & 13 & $(2, 1)$ & 1 & 1 & YES & YES & YES & -- & 2941\\
$(259, 100)$ & 12 & $(3, 1)$ & 2 & 1 & YES & YES & YES & -- & 2942\\
$(259, 100)$ & 12 & $(3, 1)$ & 2 & 1 & YES & YES & YES & NO & 2943\\
$(259, 100)$ & 12 & $(3, 1)$ & 2 & 1 & YES & YES & YES & NO & 2944\\
$(259, 76)$ & 13 & $(4, 1)$ & 3 & 1 & YES & YES & YES & NO & 2945\\
$(259, 100)$ & 12 & $(57, 22)$ & 9 & 1 & YES & YES & YES & 2745 & 2946\\
$(259, 101)$ & 12 & $(59, 23)$ & 9 & 1 & YES & YES & YES & 2754 & 2947\\
$(259, 76)$ & 13 & $(75, 22)$ & 10 & 1 & YES & YES & YES & 2854 & 2948\\
$(259, 101)$ & 12 & $(100, 39)$ & 10 & 1 & YES & YES & YES & NO & 2949\\
$(259, 100)$ & 12 & $(158, 61)$ & 11 & 1 & YES & YES & YES & NO & 2950\\
$(259, 101)$ & 12 & $(159, 62)$ & 11 & 1 & YES & YES & YES & NO & 2951\\
$(259, 101)$ & 12 & $(259, 101)$ & 12 & 259 & YES & YES & YES & NO & 2952\\
$(261, 100)$ & 12 & $(2, 1)$ & 1 & 1 & YES & YES & YES & -- & 2953\\
$(261, 100)$ & 12 & $(3, 1)$ & 2 & 3 & YES & YES & YES & -- & 2954\\
$(261, 100)$ & 12 & $(4, 1)$ & 3 & 1 & YES & YES & YES & -- & 2955\\
$(261, 100)$ & 12 & $(60, 23)$ & 9 & 3 & YES & YES & YES & NO & 2956\\
$(261, 100)$ & 12 & $(107, 41)$ & 10 & 1 & YES & YES & YES & NO & 2957\\
$(263, 78)$ & 13 & $(2, 1)$ & 1 & 1 & YES & YES & YES & -- & 2958\\
$(263, 109)$ & 12 & $(2, 1)$ & 1 & 1 & YES & YES & YES & -- & 2959\\
$(263, 109)$ & 12 & $(3, 1)$ & 2 & 1 & YES & YES & YES & -- & 2960\\
$(263, 111)$ & 12 & $(3, 1)$ & 2 & 1 & YES & YES & YES & -- & 2961\\
$(263, 60)$ & 13 & $(5, 2)$ & 3 & 1 & YES & YES & YES & -- & 2962\\
$(263, 78)$ & 13 & $(7, 2)$ & 4 & 1 & YES & YES & YES & NO & 2963\\
$(263, 109)$ & 12 & $(17, 7)$ & 6 & 1 & YES & YES & YES & NO & 2964\\
$(263, 57)$ & 13 & $(32, 7)$ & 8 & 1 & YES & YES & YES & 2859 & 2965\\
$(263, 71)$ & 12 & $(89, 24)$ & 10 & 1 & YES & YES & YES & NO & 2966\\
$(263, 78)$ & 13 & $(118, 35)$ & 11 & 1 & YES & YES & YES & NO & 2967\\
$(263, 71)$ & 12 & $(137, 37)$ & 11 & 1 & YES & YES & YES & NO & 2968\\
$(263, 111)$ & 12 & $(263, 111)$ & 12 & 263 & YES & YES & YES & NO & 2969\\
$(264, 109)$ & 12 & $(109, 45)$ & 10 & 1 & YES & YES & YES & NO & 2970\\
$(265, 98)$ & 12 & $(11, 4)$ & 5 & 1 & YES & YES & YES & NO & 2971\\
$(265, 97)$ & 12 & $(112, 41)$ & 10 & 1 & YES & YES & YES & NO & 2972\\
$(266, 101)$ & 12 & $(2, 1)$ & 1 & 2 & YES & YES & YES & -- & 2973\\
$(266, 101)$ & 12 & $(2, 1)$ & 1 & 2 & YES & YES & YES & NO & 2974\\
$(267, 74)$ & 13 & $(2, 1)$ & 1 & 1 & YES & YES & YES & NO & 2975\\
$(267, 74)$ & 13 & $(3, 1)$ & 2 & 3 & YES & YES & YES & NO & 2976\\
$(267, 98)$ & 12 & $(3, 1)$ & 2 & 3 & YES & YES & YES & -- & 2977\\
$(267, 98)$ & 12 & $(8, 3)$ & 4 & 1 & YES & YES & YES & NO & 2978\\
$(267, 98)$ & 12 & $(19, 7)$ & 6 & 1 & YES & YES & YES & NO & 2979\\
$(267, 98)$ & 12 & $(30, 11)$ & 7 & 3 & YES & YES & YES & NO & 2980\\
$(268, 111)$ & 12 & $(2, 1)$ & 1 & 2 & YES & YES & YES & -- & 2981\\
$(268, 111)$ & 12 & $(3, 1)$ & 2 & 1 & YES & YES & YES & -- & 2982\\
$(268, 111)$ & 12 & $(3, 1)$ & 2 & 1 & YES & YES & YES & NO & 2983\\
$(268, 111)$ & 12 & $(4, 1)$ & 3 & 4 & YES & YES & YES & -- & 2984\\
$(268, 99)$ & 12 & $(5, 2)$ & 3 & 1 & YES & YES & YES & NO & 2985\\
$(268, 111)$ & 12 & $(7, 3)$ & 4 & 1 & YES & YES & YES & NO & 2986\\
$(268, 111)$ & 12 & $(17, 7)$ & 6 & 1 & YES & YES & YES & NO & 2987\\
$(268, 111)$ & 12 & $(29, 12)$ & 7 & 1 & YES & YES & YES & NO & 2988\\
$(268, 111)$ & 12 & $(41, 17)$ & 8 & 1 & YES & YES & YES & 3096 & 2989\\
$(268, 111)$ & 12 & $(268, 111)$ & 12 & 268 & YES & YES & YES & NO & 2990\\
$(269, 78)$ & 13 & $(2, 1)$ & 1 & 1 & YES & YES & YES & -- & 2991\\
$(269, 78)$ & 13 & $(2, 1)$ & 1 & 1 & YES & YES & YES & NO & 2992\\
$(269, 104)$ & 12 & $(3, 1)$ & 2 & 1 & YES & YES & YES & -- & 2993\\
$(269, 104)$ & 12 & $(3, 1)$ & 2 & 1 & YES & YES & YES & NO & 2994\\
$(269, 78)$ & 13 & $(5, 1)$ & 4 & 1 & YES & YES & YES & NO & 2995\\
$(269, 104)$ & 12 & $(8, 3)$ & 4 & 1 & YES & YES & YES & NO & 2996\\
$(271, 105)$ & 12 & $(3, 1)$ & 2 & 1 & YES & YES & YES & -- & 2997\\
$(271, 112)$ & 12 & $(3, 1)$ & 2 & 1 & YES & YES & YES & -- & 2998\\
$(271, 112)$ & 12 & $(46, 19)$ & 8 & 1 & YES & YES & YES & NO & 2999\\
$(273, 76)$ & 13 & $(2, 1)$ & 1 & 1 & YES & YES & YES & NO & 3000\\
$(273, 106)$ & 13 & $(2, 1)$ & 1 & 1 & YES & YES & YES & -- & 3001\\
$(273, 76)$ & 13 & $(3, 1)$ & 2 & 3 & YES & YES & YES & NO & 3002\\
$(273, 100)$ & 12 & $(5, 2)$ & 3 & 1 & YES & YES & YES & NO & 3003\\
$(273, 106)$ & 13 & $(13, 5)$ & 5 & 13 & YES & YES & YES & NO & 3004\\
$(273, 80)$ & 13 & $(41, 12)$ & 8 & 1 & YES & YES & YES & NO & 3005\\
$(273, 76)$ & 13 & $(79, 22)$ & 10 & 1 & YES & YES & YES & 2909 & 3006\\
$(273, 80)$ & 13 & $(99, 29)$ & 10 & 3 & YES & YES & YES & NO & 3007\\
$(273, 80)$ & 13 & $(215, 63)$ & 12 & 1 & YES & YES & YES & NO & 3008\\
$(274, 81)$ & 12 & $(2, 1)$ & 1 & 2 & YES & YES & YES & -- & 3009\\
$(274, 115)$ & 12 & $(2, 1)$ & 1 & 2 & YES & YES & YES & -- & 3010\\
$(274, 81)$ & 12 & $(3, 1)$ & 2 & 1 & YES & YES & YES & -- & 3011\\
$(274, 81)$ & 12 & $(3, 1)$ & 2 & 1 & YES & YES & YES & NO & 3012\\
$(274, 105)$ & 12 & $(3, 1)$ & 2 & 1 & YES & YES & YES & -- & 3013\\
$(274, 115)$ & 12 & $(3, 1)$ & 2 & 1 & YES & YES & YES & -- & 3014\\
$(274, 115)$ & 12 & $(3, 1)$ & 2 & 1 & YES & YES & YES & NO & 3015\\
$(274, 81)$ & 12 & $(11, 3)$ & 5 & 1 & YES & YES & YES & NO & 3016\\
$(274, 115)$ & 12 & $(19, 8)$ & 6 & 1 & YES & YES & YES & NO & 3017\\
$(274, 81)$ & 12 & $(24, 7)$ & 7 & 2 & YES & YES & YES & NO & 3018\\
$(275, 76)$ & 12 & $(2, 1)$ & 1 & 1 & YES & YES & YES & -- & 3019\\
$(275, 76)$ & 12 & $(2, 1)$ & 1 & 1 & YES & YES & YES & NO & 3020\\
$(275, 76)$ & 12 & $(7, 2)$ & 4 & 1 & YES & YES & YES & NO & 3021\\
$(277, 76)$ & 13 & $(2, 1)$ & 1 & 1 & YES & YES & YES & NO & 3022\\
$(277, 78)$ & 13 & $(2, 1)$ & 1 & 1 & YES & YES & YES & NO & 3023\\
$(277, 81)$ & 12 & $(2, 1)$ & 1 & 1 & YES & YES & YES & -- & 3024\\
$(277, 106)$ & 12 & $(3, 1)$ & 2 & 1 & YES & YES & YES & -- & 3025\\
$(277, 117)$ & 12 & $(4, 1)$ & 3 & 1 & YES & YES & YES & -- & 3026\\
$(277, 60)$ & 13 & $(5, 2)$ & 3 & 1 & YES & YES & YES & -- & 3027\\
$(277, 76)$ & 13 & $(7, 2)$ & 4 & 1 & YES & YES & YES & NO & 3028\\
$(277, 78)$ & 13 & $(71, 20)$ & 10 & 1 & YES & YES & YES & 2867 & 3029\\
$(277, 78)$ & 13 & $(103, 29)$ & 11 & 1 & YES & YES & YES & NO & 3030\\
$(277, 81)$ & 12 & $(106, 31)$ & 10 & 1 & YES & YES & YES & NO & 3031\\
$(277, 117)$ & 12 & $(116, 49)$ & 10 & 1 & YES & YES & YES & NO & 3032\\
$(277, 117)$ & 12 & $(277, 117)$ & 12 & 277 & YES & YES & YES & NO & 3033\\
$(280, 107)$ & 12 & $(5, 1)$ & 4 & 5 & YES & YES & YES & -- & 3034\\
$(281, 64)$ & 13 & $(2, 1)$ & 1 & 1 & YES & YES & YES & NO & 3035\\
$(281, 109)$ & 12 & $(13, 5)$ & 5 & 1 & YES & YES & YES & NO & 3036\\
$(281, 109)$ & 12 & $(116, 45)$ & 10 & 1 & YES & YES & YES & NO & 3037\\
$(282, 109)$ & 12 & $(2, 1)$ & 1 & 2 & YES & YES & YES & -- & 3038\\
$(282, 119)$ & 12 & $(3, 1)$ & 2 & 3 & YES & YES & YES & -- & 3039\\
$(282, 109)$ & 12 & $(4, 1)$ & 3 & 2 & YES & YES & YES & -- & 3040\\
$(282, 119)$ & 12 & $(5, 1)$ & 4 & 1 & YES & YES & YES & NO & 3041\\
$(282, 119)$ & 12 & $(5, 2)$ & 3 & 1 & YES & YES & YES & NO & 3042\\
$(282, 119)$ & 12 & $(45, 19)$ & 8 & 3 & YES & YES & YES & NO & 3043\\
$(282, 119)$ & 12 & $(64, 27)$ & 9 & 2 & YES & YES & YES & 2820 & 3044\\
$(282, 109)$ & 12 & $(119, 46)$ & 10 & 1 & YES & YES & YES & NO & 3045\\
$(282, 119)$ & 12 & $(173, 73)$ & 11 & 1 & YES & YES & YES & NO & 3046\\
$(283, 108)$ & 12 & $(2, 1)$ & 1 & 1 & YES & YES & YES & -- & 3047\\
$(283, 83)$ & 13 & $(3, 1)$ & 2 & 1 & YES & YES & YES & -- & 3048\\
$(283, 108)$ & 12 & $(4, 1)$ & 3 & 1 & YES & YES & YES & -- & 3049\\
$(283, 108)$ & 12 & $(6, 1)$ & 5 & 1 & YES & YES & YES & NO & 3050\\
$(283, 108)$ & 12 & $(13, 5)$ & 5 & 1 & YES & YES & YES & NO & 3051\\
$(283, 104)$ & 12 & $(30, 11)$ & 7 & 1 & YES & YES & YES & NO & 3052\\
$(283, 108)$ & 12 & $(55, 21)$ & 8 & 1 & YES & YES & YES & NO & 3053\\
$(283, 83)$ & 13 & $(133, 39)$ & 11 & 1 & YES & YES & YES & 3193 & 3054\\
$(284, 119)$ & 12 & $(74, 31)$ & 9 & 2 & YES & YES & YES & 2901 & 3055\\
$(284, 105)$ & 12 & $(284, 105)$ & 12 & 284 & YES & YES & YES & NO & 3056\\
$(286, 105)$ & 12 & $(2, 1)$ & 1 & 2 & YES & YES & YES & -- & 3057\\
$(287, 106)$ & 12 & $(2, 1)$ & 1 & 1 & YES & YES & YES & -- & 3058\\
$(287, 109)$ & 12 & $(2, 1)$ & 1 & 1 & YES & YES & YES & -- & 3059\\
$(287, 111)$ & 12 & $(2, 1)$ & 1 & 1 & YES & YES & YES & -- & 3060\\
$(287, 109)$ & 12 & $(3, 1)$ & 2 & 1 & YES & YES & YES & -- & 3061\\
$(287, 106)$ & 12 & $(5, 2)$ & 3 & 1 & YES & YES & YES & NO & 3062\\
$(287, 111)$ & 12 & $(5, 1)$ & 4 & 1 & YES & YES & YES & -- & 3063\\
$(287, 111)$ & 12 & $(5, 1)$ & 4 & 1 & YES & YES & YES & NO & 3064\\
$(287, 53)$ & 14 & $(7, 2)$ & 4 & 7 & YES & YES & YES & NO & 3065\\
$(287, 109)$ & 12 & $(21, 8)$ & 6 & 7 & YES & YES & YES & NO & 3066\\
$(287, 106)$ & 12 & $(65, 24)$ & 9 & 1 & YES & YES & YES & 2846 & 3067\\
$(287, 111)$ & 12 & $(75, 29)$ & 9 & 1 & YES & YES & YES & 2926 & 3068\\
$(287, 80)$ & 13 & $(104, 29)$ & 10 & 1 & YES & YES & YES & NO & 3069\\
$(287, 111)$ & 12 & $(106, 41)$ & 10 & 1 & YES & YES & YES & NO & 3070\\
$(287, 111)$ & 12 & $(181, 70)$ & 11 & 1 & YES & YES & YES & NO & 3071\\
$(288, 85)$ & 13 & $(2, 1)$ & 1 & 2 & YES & YES & YES & -- & 3072\\
$(288, 119)$ & 12 & $(3, 1)$ & 2 & 3 & YES & YES & YES & -- & 3073\\
$(288, 119)$ & 12 & $(3, 1)$ & 2 & 3 & YES & YES & YES & NO & 3074\\
$(288, 119)$ & 12 & $(4, 1)$ & 3 & 4 & YES & YES & YES & NO & 3075\\
$(288, 121)$ & 12 & $(12, 5)$ & 5 & 12 & YES & YES & YES & NO & 3076\\
$(288, 85)$ & 13 & $(166, 49)$ & 11 & 2 & YES & YES & YES & 3228 & 3077\\
$(288, 119)$ & 12 & $(167, 69)$ & 11 & 1 & YES & YES & YES & NO & 3078\\
$(288, 119)$ & 12 & $(288, 119)$ & 12 & 288 & YES & YES & YES & NO & 3079\\
$(289, 80)$ & 12 & $(2, 1)$ & 1 & 1 & YES & YES & YES & NO & 3080\\
$(289, 84)$ & 13 & $(3, 1)$ & 2 & 1 & YES & YES & YES & -- & 3081\\
$(289, 112)$ & 12 & $(13, 5)$ & 5 & 1 & YES & YES & YES & NO & 3082\\
$(289, 112)$ & 12 & $(49, 19)$ & 8 & 1 & YES & YES & YES & NO & 3083\\
$(290, 81)$ & 12 & $(2, 1)$ & 1 & 2 & YES & YES & YES & -- & 3084\\
$(290, 81)$ & 12 & $(2, 1)$ & 1 & 2 & YES & YES & YES & NO & 3085\\
$(290, 111)$ & 12 & $(8, 3)$ & 4 & 2 & YES & YES & YES & NO & 3086\\
$(290, 81)$ & 12 & $(18, 5)$ & 6 & 2 & YES & YES & YES & NO & 3087\\
$(290, 111)$ & 12 & $(34, 13)$ & 7 & 2 & YES & YES & YES & NO & 3088\\
$(291, 85)$ & 13 & $(3, 1)$ & 2 & 3 & YES & YES & YES & -- & 3089\\
$(291, 85)$ & 13 & $(4, 1)$ & 3 & 1 & YES & YES & YES & -- & 3090\\
$(291, 85)$ & 13 & $(10, 3)$ & 5 & 1 & YES & YES & YES & NO & 3091\\
$(291, 85)$ & 13 & $(65, 19)$ & 9 & 1 & YES & YES & YES & NO & 3092\\
$(292, 111)$ & 12 & $(2, 1)$ & 1 & 2 & YES & YES & YES & -- & 3093\\
$(292, 111)$ & 12 & $(3, 1)$ & 2 & 1 & YES & YES & YES & -- & 3094\\
$(292, 121)$ & 12 & $(3, 1)$ & 2 & 1 & YES & YES & YES & -- & 3095\\
$(292, 121)$ & 12 & $(29, 12)$ & 7 & 1 & YES & YES & YES & 2989 & 3096\\
$(292, 85)$ & 13 & $(31, 9)$ & 8 & 1 & YES & YES & YES & NO & 3097\\
$(292, 85)$ & 13 & $(55, 16)$ & 9 & 1 & YES & YES & YES & NO & 3098\\
$(292, 111)$ & 12 & $(121, 46)$ & 10 & 1 & YES & YES & YES & NO & 3099\\
$(293, 123)$ & 12 & $(2, 1)$ & 1 & 1 & YES & YES & YES & -- & 3100\\
$(293, 79)$ & 13 & $(3, 1)$ & 2 & 1 & YES & YES & YES & NO & 3101\\
$(293, 123)$ & 12 & $(7, 3)$ & 4 & 1 & YES & YES & YES & NO & 3102\\
$(295, 108)$ & 12 & $(4, 1)$ & 3 & 1 & YES & YES & YES & -- & 3103\\
$(295, 87)$ & 13 & $(61, 18)$ & 9 & 1 & YES & YES & YES & NO & 3104\\
$(297, 83)$ & 13 & $(2, 1)$ & 1 & 1 & YES & YES & YES & -- & 3105\\
$(297, 83)$ & 13 & $(3, 1)$ & 2 & 3 & YES & YES & YES & -- & 3106\\
$(297, 83)$ & 13 & $(18, 5)$ & 6 & 9 & YES & YES & YES & NO & 3107\\
$(298, 83)$ & 13 & $(3, 1)$ & 2 & 1 & YES & YES & YES & -- & 3108\\
$(298, 83)$ & 13 & $(298, 83)$ & 13 & 298 & YES & YES & YES & NO & 3109\\
$(301, 115)$ & 12 & $(2, 1)$ & 1 & 1 & YES & YES & YES & -- & 3110\\
$(301, 65)$ & 13 & $(5, 2)$ & 3 & 1 & YES & YES & YES & -- & 3111\\
$(301, 65)$ & 13 & $(5, 2)$ & 3 & 1 & YES & YES & YES & NO & 3112\\
$(301, 115)$ & 12 & $(5, 2)$ & 3 & 1 & YES & YES & YES & NO & 3113\\
$(301, 65)$ & 13 & $(13, 3)$ & 6 & 1 & YES & YES & YES & 3243 & 3114\\
$(301, 115)$ & 12 & $(21, 8)$ & 6 & 7 & YES & YES & YES & NO & 3115\\
$(301, 88)$ & 13 & $(41, 12)$ & 8 & 1 & YES & YES & YES & NO & 3116\\
$(303, 85)$ & 13 & $(2, 1)$ & 1 & 1 & YES & YES & YES & -- & 3117\\
$(303, 85)$ & 13 & $(3, 1)$ & 2 & 3 & YES & YES & YES & -- & 3118\\
$(303, 85)$ & 13 & $(11, 3)$ & 5 & 1 & YES & YES & YES & NO & 3119\\
$(303, 85)$ & 13 & $(32, 9)$ & 8 & 1 & YES & YES & YES & NO & 3120\\
$(303, 116)$ & 12 & $(34, 13)$ & 7 & 1 & YES & YES & YES & 2538 & 3121\\
$(303, 85)$ & 13 & $(57, 16)$ & 9 & 3 & YES & YES & YES & NO & 3122\\
$(303, 128)$ & 12 & $(71, 30)$ & 9 & 1 & YES & YES & YES & 2940 & 3123\\
$(304, 85)$ & 13 & $(3, 1)$ & 2 & 1 & YES & YES & YES & -- & 3124\\
$(304, 85)$ & 13 & $(11, 3)$ & 5 & 1 & YES & YES & YES & NO & 3125\\
$(305, 84)$ & 13 & $(2, 1)$ & 1 & 1 & YES & YES & YES & -- & 3126\\
$(305, 118)$ & 13 & $(137, 53)$ & 11 & 1 & YES & YES & YES & NO & 3127\\
$(307, 119)$ & 12 & $(2, 1)$ & 1 & 1 & YES & YES & YES & -- & 3128\\
$(307, 129)$ & 12 & $(2, 1)$ & 1 & 1 & YES & YES & YES & -- & 3129\\
$(307, 69)$ & 14 & $(3, 1)$ & 2 & 1 & YES & YES & YES & NO & 3130\\
$(307, 119)$ & 12 & $(3, 1)$ & 2 & 1 & YES & YES & YES & -- & 3131\\
$(307, 129)$ & 12 & $(12, 5)$ & 5 & 1 & YES & YES & YES & 2608 & 3132\\
$(307, 85)$ & 13 & $(47, 13)$ & 8 & 1 & YES & YES & YES & NO & 3133\\
$(307, 129)$ & 12 & $(69, 29)$ & 9 & 1 & YES & YES & YES & 2937 & 3134\\
$(313, 86)$ & 13 & $(2, 1)$ & 1 & 1 & YES & YES & YES & NO & 3135\\
$(313, 121)$ & 12 & $(2, 1)$ & 1 & 1 & NO & YES & YES & -- & 3136\\
$(313, 86)$ & 13 & $(3, 1)$ & 2 & 1 & YES & YES & YES & -- & 3137\\
$(313, 86)$ & 13 & $(3, 1)$ & 2 & 1 & YES & YES & YES & NO & 3138\\
$(313, 86)$ & 13 & $(3, 1)$ & 2 & 1 & YES & YES & YES & NO & 3139\\
$(313, 121)$ & 12 & $(3, 1)$ & 2 & 1 & YES & YES & YES & -- & 3140\\
$(313, 121)$ & 12 & $(13, 5)$ & 5 & 1 & YES & YES & YES & NO & 3141\\
$(313, 86)$ & 13 & $(18, 5)$ & 6 & 1 & YES & YES & YES & NO & 3142\\
$(313, 91)$ & 13 & $(55, 16)$ & 9 & 1 & YES & YES & YES & NO & 3143\\
$(313, 119)$ & 12 & $(121, 46)$ & 10 & 1 & YES & YES & YES & NO & 3144\\
$(315, 88)$ & 13 & $(3, 1)$ & 2 & 3 & YES & YES & YES & -- & 3145\\
$(315, 88)$ & 13 & $(18, 5)$ & 6 & 9 & YES & YES & YES & NO & 3146\\
$(317, 121)$ & 12 & $(5, 1)$ & 4 & 1 & YES & YES & YES & -- & 3147\\
$(317, 121)$ & 12 & $(5, 2)$ & 3 & 1 & YES & YES & YES & NO & 3148\\
$(317, 121)$ & 12 & $(13, 5)$ & 5 & 1 & YES & YES & YES & NO & 3149\\
$(321, 94)$ & 13 & $(2, 1)$ & 1 & 1 & YES & YES & YES & -- & 3150\\
$(321, 95)$ & 13 & $(2, 1)$ & 1 & 1 & YES & YES & YES & -- & 3151\\
$(321, 94)$ & 13 & $(3, 1)$ & 2 & 3 & YES & YES & YES & -- & 3152\\
$(321, 95)$ & 13 & $(4, 1)$ & 3 & 1 & YES & YES & YES & NO & 3153\\
$(321, 94)$ & 13 & $(140, 41)$ & 11 & 1 & YES & YES & YES & NO & 3154\\
$(322, 73)$ & 14 & $(5, 1)$ & 4 & 1 & YES & YES & YES & NO & 3155\\
$(323, 60)$ & 14 & $(2, 1)$ & 1 & 1 & YES & YES & YES & -- & 3156\\
$(323, 60)$ & 14 & $(2, 1)$ & 1 & 1 & YES & YES & YES & NO & 3157\\
$(323, 94)$ & 13 & $(2, 1)$ & 1 & 1 & YES & YES & YES & -- & 3158\\
$(323, 98)$ & 13 & $(2, 1)$ & 1 & 1 & YES & YES & YES & NO & 3159\\
$(323, 98)$ & 13 & $(3, 1)$ & 2 & 1 & NO & YES & YES & -- & 3160\\
$(323, 94)$ & 13 & $(17, 5)$ & 6 & 17 & YES & YES & YES & NO & 3161\\
$(323, 98)$ & 13 & $(23, 7)$ & 7 & 1 & YES & YES & YES & NO & 3162\\
$(323, 94)$ & 13 & $(31, 9)$ & 8 & 1 & YES & YES & YES & NO & 3163\\
$(323, 98)$ & 13 & $(56, 17)$ & 9 & 1 & YES & YES & YES & NO & 3164\\
$(323, 89)$ & 13 & $(98, 27)$ & 10 & 1 & YES & YES & YES & NO & 3165\\
$(323, 94)$ & 13 & $(134, 39)$ & 11 & 1 & YES & YES & YES & NO & 3166\\
$(324, 95)$ & 13 & $(10, 3)$ & 5 & 2 & YES & YES & YES & NO & 3167\\
$(324, 95)$ & 13 & $(75, 22)$ & 10 & 3 & YES & YES & YES & NO & 3168\\
$(325, 74)$ & 14 & $(5, 1)$ & 4 & 5 & YES & YES & YES & NO & 3169\\
$(326, 99)$ & 13 & $(2, 1)$ & 1 & 2 & YES & YES & YES & NO & 3170\\
$(326, 97)$ & 13 & $(3, 1)$ & 2 & 1 & YES & YES & YES & -- & 3171\\
$(326, 99)$ & 13 & $(3, 1)$ & 2 & 1 & YES & YES & YES & -- & 3172\\
$(326, 71)$ & 14 & $(4, 1)$ & 3 & 2 & YES & YES & YES & NO & 3173\\
$(326, 99)$ & 13 & $(79, 24)$ & 10 & 1 & YES & YES & YES & NO & 3174\\
$(326, 97)$ & 13 & $(326, 97)$ & 13 & 326 & YES & YES & YES & NO & 3175\\
$(333, 101)$ & 13 & $(2, 1)$ & 1 & 1 & YES & YES & YES & -- & 3176\\
$(333, 101)$ & 13 & $(2, 1)$ & 1 & 1 & YES & YES & YES & NO & 3177\\
$(333, 92)$ & 13 & $(3, 1)$ & 2 & 3 & YES & YES & YES & -- & 3178\\
$(333, 76)$ & 13 & $(9, 2)$ & 5 & 9 & YES & YES & YES & NO & 3179\\
$(333, 76)$ & 13 & $(22, 5)$ & 7 & 1 & YES & YES & YES & NO & 3180\\
$(333, 101)$ & 13 & $(23, 7)$ & 7 & 1 & YES & YES & YES & NO & 3181\\
$(335, 73)$ & 14 & $(2, 1)$ & 1 & 1 & YES & YES & YES & -- & 3182\\
$(335, 73)$ & 14 & $(2, 1)$ & 1 & 1 & YES & YES & YES & NO & 3183\\
$(335, 73)$ & 14 & $(3, 1)$ & 2 & 1 & YES & YES & YES & -- & 3184\\
$(335, 73)$ & 14 & $(3, 1)$ & 2 & 1 & YES & YES & YES & NO & 3185\\
$(335, 76)$ & 14 & $(3, 1)$ & 2 & 1 & YES & YES & YES & NO & 3186\\
$(337, 98)$ & 13 & $(24, 7)$ & 7 & 1 & YES & YES & YES & NO & 3187\\
$(337, 91)$ & 13 & $(137, 37)$ & 11 & 1 & YES & YES & YES & 3223 & 3188\\
$(338, 99)$ & 13 & $(2, 1)$ & 1 & 2 & YES & YES & YES & NO & 3189\\
$(338, 129)$ & 12 & $(2, 1)$ & 1 & 2 & YES & YES & YES & -- & 3190\\
$(338, 77)$ & 14 & $(5, 1)$ & 4 & 1 & YES & YES & YES & NO & 3191\\
$(338, 129)$ & 12 & $(13, 5)$ & 5 & 13 & YES & YES & YES & 2688 & 3192\\
$(341, 100)$ & 13 & $(75, 22)$ & 10 & 1 & YES & YES & YES & 3054 & 3193\\
$(341, 100)$ & 13 & $(133, 39)$ & 11 & 1 & YES & YES & YES & NO & 3194\\
$(342, 101)$ & 13 & $(193, 57)$ & 12 & 1 & YES & YES & YES & NO & 3195\\
$(344, 95)$ & 13 & $(2, 1)$ & 1 & 2 & YES & YES & YES & -- & 3196\\
$(344, 95)$ & 13 & $(2, 1)$ & 1 & 2 & YES & YES & YES & NO & 3197\\
$(344, 95)$ & 13 & $(18, 5)$ & 6 & 2 & YES & YES & YES & 2317 & 3198\\
$(347, 93)$ & 13 & $(3, 1)$ & 2 & 1 & YES & YES & YES & NO & 3199\\
$(347, 101)$ & 13 & $(3, 1)$ & 2 & 1 & YES & YES & YES & -- & 3200\\
$(347, 93)$ & 13 & $(4, 1)$ & 3 & 1 & YES & YES & YES & -- & 3201\\
$(347, 101)$ & 13 & $(4, 1)$ & 3 & 1 & YES & YES & YES & NO & 3202\\
$(347, 93)$ & 13 & $(41, 11)$ & 8 & 1 & YES & YES & YES & NO & 3203\\
$(347, 101)$ & 13 & $(134, 39)$ & 11 & 1 & YES & YES & YES & NO & 3204\\
$(349, 135)$ & 13 & $(31, 12)$ & 7 & 1 & YES & YES & YES & NO & 3205\\
$(353, 97)$ & 13 & $(3, 1)$ & 2 & 1 & YES & YES & YES & -- & 3206\\
$(353, 97)$ & 13 & $(3, 1)$ & 2 & 1 & YES & YES & YES & NO & 3207\\
$(353, 97)$ & 13 & $(7, 2)$ & 4 & 1 & YES & YES & YES & NO & 3208\\
$(355, 77)$ & 14 & $(2, 1)$ & 1 & 1 & YES & YES & YES & NO & 3209\\
$(355, 99)$ & 13 & $(3, 1)$ & 2 & 1 & YES & YES & YES & 2764 & 3210\\
$(355, 77)$ & 14 & $(14, 3)$ & 6 & 1 & YES & YES & YES & NO & 3211\\
$(359, 100)$ & 13 & $(2, 1)$ & 1 & 1 & YES & YES & YES & -- & 3212\\
$(359, 57)$ & 16 & $(3, 1)$ & 2 & 1 & YES & YES & YES & -- & 3213\\
$(359, 100)$ & 13 & $(61, 17)$ & 9 & 1 & YES & YES & YES & NO & 3214\\
$(360, 101)$ & 13 & $(2, 1)$ & 1 & 2 & YES & YES & YES & NO & 3215\\
$(360, 101)$ & 13 & $(57, 16)$ & 9 & 3 & YES & YES & YES & NO & 3216\\
$(366, 83)$ & 14 & $(3, 1)$ & 2 & 3 & YES & YES & YES & -- & 3217\\
$(367, 99)$ & 13 & $(3, 1)$ & 2 & 1 & YES & YES & YES & -- & 3218\\
$(367, 112)$ & 13 & $(23, 7)$ & 7 & 1 & YES & YES & YES & NO & 3219\\
$(367, 99)$ & 13 & $(89, 24)$ & 10 & 1 & YES & YES & YES & NO & 3220\\
$(372, 109)$ & 13 & $(4, 1)$ & 3 & 4 & YES & YES & YES & -- & 3221\\
$(372, 109)$ & 13 & $(17, 5)$ & 6 & 1 & YES & YES & YES & NO & 3222\\
$(374, 101)$ & 13 & $(100, 27)$ & 10 & 2 & YES & YES & YES & 3188 & 3223\\
$(383, 106)$ & 13 & $(18, 5)$ & 6 & 1 & YES & YES & YES & NO & 3224\\
$(389, 89)$ & 14 & $(2, 1)$ & 1 & 1 & YES & YES & YES & NO & 3225\\
$(389, 89)$ & 14 & $(83, 19)$ & 10 & 1 & YES & YES & YES & NO & 3226\\
$(393, 116)$ & 13 & $(2, 1)$ & 1 & 1 & YES & YES & YES & NO & 3227\\
$(393, 116)$ & 13 & $(61, 18)$ & 9 & 1 & YES & YES & YES & 3077 & 3228\\
$(393, 116)$ & 13 & $(166, 49)$ & 11 & 1 & YES & YES & YES & NO & 3229\\
$(394, 165)$ & 13 & $(2, 1)$ & 1 & 2 & NO & YES & YES & -- & 3230\\
$(397, 75)$ & 15 & $(3, 1)$ & 2 & 1 & YES & YES & YES & -- & 3231\\
$(398, 111)$ & 13 & $(2, 1)$ & 1 & 2 & YES & YES & YES & -- & 3232\\
$(398, 111)$ & 13 & $(7, 2)$ & 4 & 1 & YES & YES & YES & NO & 3233\\
$(403, 87)$ & 14 & $(4, 1)$ & 3 & 1 & YES & YES & YES & -- & 3234\\
$(407, 119)$ & 13 & $(17, 5)$ & 6 & 1 & YES & YES & YES & NO & 3235\\
$(419, 89)$ & 14 & $(2, 1)$ & 1 & 1 & YES & YES & YES & -- & 3236\\
$(419, 89)$ & 14 & $(2, 1)$ & 1 & 1 & YES & YES & YES & NO & 3237\\
$(423, 97)$ & 14 & $(2, 1)$ & 1 & 1 & YES & YES & YES & -- & 3238\\
$(423, 97)$ & 14 & $(2, 1)$ & 1 & 1 & YES & YES & YES & NO & 3239\\
$(423, 97)$ & 14 & $(3, 1)$ & 2 & 3 & YES & YES & YES & -- & 3240\\
$(424, 97)$ & 14 & $(3, 1)$ & 2 & 1 & YES & YES & YES & -- & 3241\\
$(424, 97)$ & 14 & $(48, 11)$ & 9 & 8 & YES & YES & YES & NO & 3242\\
$(437, 99)$ & 14 & $(5, 1)$ & 4 & 1 & YES & YES & YES & 3114 & 3243\\
$(451, 84)$ & 15 & $(2, 1)$ & 1 & 1 & YES & YES & YES & -- & 3244\\
$(451, 84)$ & 15 & $(2, 1)$ & 1 & 1 & YES & YES & YES & NO & 3245\\
$(451, 84)$ & 15 & $(3, 1)$ & 2 & 1 & YES & YES & YES & NO & 3246\\
$(461, 98)$ & 14 & $(4, 1)$ & 3 & 1 & YES & YES & YES & NO & 3247\\
$(461, 98)$ & 14 & $(33, 7)$ & 8 & 1 & YES & YES & YES & NO & 3248\\
$(466, 109)$ & 14 & $(13, 3)$ & 6 & 1 & YES & YES & YES & NO & 3249\\
$(466, 109)$ & 14 & $(30, 7)$ & 8 & 2 & YES & YES & YES & NO & 3250\\
$(469, 107)$ & 14 & $(22, 5)$ & 7 & 1 & YES & YES & YES & NO & 3251\\
$(477, 88)$ & 15 & $(27, 5)$ & 8 & 9 & YES & YES & YES & NO & 3252\\
$(495, 92)$ & 15 & $(3, 1)$ & 2 & 3 & YES & YES & YES & -- & 3253\\
$(495, 92)$ & 15 & $(11, 2)$ & 6 & 11 & YES & YES & YES & NO & 3254\\
$(495, 92)$ & 15 & $(27, 5)$ & 8 & 9 & YES & YES & YES & NO & 3255\\
$(522, 119)$ & 14 & $(22, 5)$ & 7 & 2 & YES & YES & YES & NO & 3256\\
$(522, 119)$ & 14 & $(57, 13)$ & 9 & 3 & YES & YES & YES & NO & 3257\\
$(a; 0, 0, 0; 3)$ & 4 & $(65, 19)$ & 9 & 1 & YES & YES & YES & -- & 3258\\
$(a; 0, 0, 0; 3)$ & 4 & $(76, 29)$ & 9 & 1 & YES & YES & YES & -- & 3259\\
$(a; 0, 0, 0; 3)$ & 4 & $(79, 18)$ & 10 & 1 & YES & YES & YES & -- & 3260\\
$(a; 0, 0, 0; 3)$ & 4 & $(89, 24)$ & 10 & 1 & YES & YES & YES & -- & 3261\\
$(a; 0, 0, 0; 3)$ & 4 & $(101, 22)$ & 11 & 1 & YES & YES & YES & -- & 3262\\
$(a; 0, 0, 0; 3)$ & 4 & $(101, 23)$ & 11 & 1 & YES & YES & YES & -- & 3263\\
$(a; 1, 0, 0; 13)$ & 5 & $(46, 19)$ & 8 & 1 & YES & YES & YES & -- & 3264\\
$(a; 1, 0, 0; 13)$ & 5 & $(55, 23)$ & 9 & 1 & YES & YES & YES & -- & 3265\\
$(a; 1, 0, 0; 13)$ & 5 & $(61, 17)$ & 9 & 1 & YES & YES & YES & -- & 3266\\
$(a; 1, 0, 0; 13)$ & 5 & $(89, 27)$ & 10 & 1 & YES & YES & YES & -- & 3267\\
$(a; 1, 1, 0; 19)$ & 6 & $(29, 11)$ & 7 & 1 & YES & YES & YES & -- & 3268\\
$(a; 1, 1, 0; 19)$ & 6 & $(31, 12)$ & 7 & 1 & YES & YES & YES & -- & 3269\\
$(a; 1, 1, 0; 19)$ & 6 & $(37, 14)$ & 8 & 1 & YES & YES & YES & -- & 3270\\
$(a; 1, 1, 1; 4)$ & 7 & $(12, 5)$ & 5 & 4 & YES & YES & YES & -- & 3271\\
$(a; 2, 0, 0; 17)$ & 6 & $(58, 17)$ & 9 & 1 & YES & YES & YES & -- & 3272\\
$(a; 2, 0, 0; 17)$ & 6 & $(79, 18)$ & 10 & 1 & YES & YES & YES & -- & 3273\\
$(a; 2, 1, 1; 37)$ & 8 & $(13, 5)$ & 5 & 1 & YES & YES & YES & -- & 3274\\
$(a; 2, 1, 1; 37)$ & 8 & $(16, 5)$ & 7 & 1 & YES & YES & YES & -- & 3275\\
$(a; 3, 3, 0; 17)$ & 10 & $(2, 1)$ & 1 & 1 & YES & YES & YES & -- & 3276\\
$(a; 3, 3, 0; 17)$ & 10 & $(5, 1)$ & 4 & 1 & YES & YES & YES & -- & 3277\\
$(b; 0, 0, 0; 14)$ & 5 & $(25, 7)$ & 7 & 1 & YES & YES & YES & -- & 3278\\
$(b; 0, 0, 0; 14)$ & 5 & $(29, 12)$ & 7 & 1 & YES & YES & YES & -- & 3279\\
$(b; 0, 0, 0; 14)$ & 5 & $(31, 12)$ & 7 & 1 & YES & YES & YES & -- & 3280\\
$(b; 0, 0, 0; 14)$ & 5 & $(40, 9)$ & 9 & 2 & YES & YES & YES & -- & 3281\\
$(b; 0, 0, 0; 14)$ & 5 & $(44, 17)$ & 8 & 2 & YES & YES & YES & -- & 3282\\
$(b; 0, 0, 1; 4)$ & 6 & $(17, 7)$ & 6 & 1 & YES & YES & YES & -- & 3283\\
$(b; 0, 0, 1; 4)$ & 6 & $(23, 9)$ & 7 & 1 & YES & YES & YES & -- & 3284\\
$(b; 0, 0, 1; 4)$ & 6 & $(26, 11)$ & 7 & 2 & YES & YES & YES & -- & 3285\\
$(b; 0, 0, 1; 4)$ & 6 & $(31, 12)$ & 7 & 1 & YES & YES & YES & -- & 3286\\
$(b; 0, 1, 0; 19)$ & 6 & $(24, 7)$ & 7 & 1 & YES & YES & YES & -- & 3287\\
$(b; 0, 1, 0; 19)$ & 6 & $(26, 11)$ & 7 & 1 & YES & YES & YES & -- & 3288\\
$(b; 0, 1, 0; 19)$ & 6 & $(29, 12)$ & 7 & 1 & YES & YES & YES & -- & 3289\\
$(b; 0, 1, 0; 19)$ & 6 & $(31, 9)$ & 8 & 1 & YES & YES & YES & -- & 3290\\
$(b; 0, 1, 0; 19)$ & 6 & $(42, 13)$ & 9 & 1 & YES & YES & YES & -- & 3291\\
$(b; 0, 1, 1; 27)$ & 7 & $(12, 5)$ & 5 & 3 & YES & YES & YES & -- & 3292\\
$(b; 0, 1, 1; 27)$ & 7 & $(17, 5)$ & 6 & 1 & YES & YES & YES & -- & 3293\\
$(b; 0, 1, 1; 27)$ & 7 & $(17, 7)$ & 6 & 1 & YES & YES & YES & -- & 3294\\
$(b; 0, 1, 1; 27)$ & 7 & $(24, 7)$ & 7 & 3 & YES & YES & YES & -- & 3295\\
$(b; 0, 1, 2; 5)$ & 8 & $(13, 5)$ & 5 & 1 & YES & YES & YES & -- & 3296\\
$(b; 0, 2, 0; 8)$ & 7 & $(7, 2)$ & 4 & 1 & YES & YES & YES & -- & 3297\\
$(b; 0, 2, 0; 8)$ & 7 & $(18, 7)$ & 6 & 2 & YES & YES & YES & -- & 3298\\
$(b; 0, 2, 0; 8)$ & 7 & $(21, 8)$ & 6 & 1 & YES & YES & YES & -- & 3299\\
$(b; 0, 2, 0; 8)$ & 7 & $(25, 7)$ & 7 & 1 & YES & YES & YES & -- & 3300\\
$(b; 0, 2, 0; 8)$ & 7 & $(27, 8)$ & 7 & 1 & YES & YES & YES & -- & 3301\\
$(b; 0, 2, 0; 8)$ & 7 & $(31, 7)$ & 8 & 1 & YES & YES & YES & -- & 3302\\
$(b; 0, 2, 0; 8)$ & 7 & $(35, 8)$ & 8 & 1 & YES & YES & YES & -- & 3303\\
$(b; 0, 2, 1; 34)$ & 8 & $(13, 5)$ & 5 & 1 & YES & YES & YES & -- & 3304\\
$(b; 0, 2, 1; 34)$ & 8 & $(17, 5)$ & 6 & 17 & YES & YES & YES & -- & 3305\\
$(b; 0, 3, 2; 53)$ & 10 & $(6, 1)$ & 5 & 1 & YES & YES & YES & -- & 3306\\
$(b; 1, 0, 1; 29)$ & 7 & $(13, 4)$ & 6 & 1 & YES & YES & YES & -- & 3307\\
$(b; 1, 1, 0; 27)$ & 7 & $(17, 7)$ & 6 & 1 & YES & YES & YES & -- & 3308\\
$(b; 1, 1, 1; 39)$ & 8 & $(7, 3)$ & 4 & 1 & YES & YES & YES & -- & 3309\\
$(b; 1, 1, 1; 39)$ & 8 & $(10, 3)$ & 5 & 1 & YES & YES & YES & -- & 3310\\
$(b; 1, 1, 1; 39)$ & 8 & $(11, 4)$ & 5 & 1 & YES & YES & YES & -- & 3311\\
$(b; 1, 1, 1; 39)$ & 8 & $(13, 5)$ & 5 & 13 & YES & YES & YES & -- & 3312\\
$(b; 1, 1, 2; 51)$ & 9 & $(5, 2)$ & 3 & 1 & YES & YES & YES & -- & 3313\\
$(b; 1, 1, 2; 51)$ & 9 & $(7, 3)$ & 4 & 1 & YES & YES & YES & -- & 3314\\
$(b; 1, 2, 0; 17)$ & 8 & $(13, 4)$ & 6 & 1 & YES & YES & YES & -- & 3315\\
$(b; 2, 0, 1; 38)$ & 8 & $(13, 5)$ & 5 & 1 & YES & YES & YES & -- & 3316\\
$(b; 2, 0, 1; 38)$ & 8 & $(17, 5)$ & 6 & 1 & YES & YES & YES & -- & 3317\\
$(c; 0, 0, 0; 4)$ & 4 & $(47, 18)$ & 8 & 1 & YES & YES & YES & -- & 3318\\
$(c; 0, 0, 0; 4)$ & 4 & $(49, 19)$ & 8 & 1 & YES & YES & YES & -- & 3319\\
$(c; 0, 0, 0; 4)$ & 4 & $(57, 22)$ & 9 & 1 & YES & YES & YES & -- & 3320\\
$(c; 0, 0, 0; 4)$ & 4 & $(58, 17)$ & 9 & 2 & YES & YES & YES & -- & 3321\\
$(c; 0, 0, 0; 4)$ & 4 & $(61, 17)$ & 9 & 1 & YES & YES & YES & -- & 3322\\
$(c; 0, 0, 0; 4)$ & 4 & $(69, 29)$ & 9 & 1 & YES & YES & YES & -- & 3323\\
$(c; 0, 0, 0; 4)$ & 4 & $(75, 31)$ & 9 & 1 & YES & YES & YES & -- & 3324\\
$(c; 0, 0, 0; 4)$ & 4 & $(76, 29)$ & 9 & 4 & YES & YES & YES & -- & 3325\\
$(c; 0, 0, 0; 4)$ & 4 & $(79, 22)$ & 10 & 1 & YES & YES & YES & -- & 3326\\
$(c; 0, 0, 0; 4)$ & 4 & $(82, 23)$ & 10 & 2 & YES & YES & YES & -- & 3327\\
$(c; 0, 0, 0; 4)$ & 4 & $(92, 35)$ & 10 & 4 & YES & YES & YES & -- & 3328\\
$(c; 0, 0, 0; 4)$ & 4 & $(95, 36)$ & 10 & 1 & YES & YES & YES & -- & 3329\\
$(c; 0, 0, 0; 4)$ & 4 & $(99, 41)$ & 10 & 1 & YES & YES & YES & -- & 3330\\
$(c; 0, 0, 0; 4)$ & 4 & $(106, 31)$ & 10 & 2 & YES & YES & YES & -- & 3331\\
$(c; 0, 0, 0; 4)$ & 4 & $(108, 41)$ & 10 & 4 & YES & YES & YES & -- & 3332\\
$(c; 0, 1, 0; 11)$ & 5 & $(17, 7)$ & 6 & 1 & YES & YES & NO(2) & -- & 3333\\
$(c; 0, 1, 0; 11)$ & 5 & $(45, 19)$ & 8 & 1 & YES & YES & YES & -- & 3334\\
$(c; 0, 1, 0; 11)$ & 5 & $(56, 23)$ & 9 & 1 & YES & YES & YES & -- & 3335\\
$(c; 0, 1, 0; 11)$ & 5 & $(58, 17)$ & 9 & 1 & YES & YES & YES & -- & 3336\\
$(c; 0, 1, 0; 11)$ & 5 & $(61, 17)$ & 9 & 1 & YES & YES & YES & -- & 3337\\
$(c; 0, 1, 0; 11)$ & 5 & $(64, 27)$ & 9 & 1 & YES & YES & YES & -- & 3338\\
$(c; 0, 1, 0; 11)$ & 5 & $(65, 24)$ & 9 & 1 & YES & YES & YES & -- & 3339\\
$(c; 0, 1, 0; 11)$ & 5 & $(70, 29)$ & 9 & 1 & YES & YES & YES & -- & 3340\\
$(c; 0, 1, 0; 11)$ & 5 & $(79, 22)$ & 10 & 1 & YES & YES & YES & -- & 3341\\
$(c; 0, 1, 0; 11)$ & 5 & $(79, 24)$ & 10 & 1 & YES & YES & YES & -- & 3342\\
$(c; 0, 1, 0; 11)$ & 5 & $(99, 29)$ & 10 & 11 & YES & YES & YES & -- & 3343\\
$(c; 0, 1, 1; 5)$ & 6 & $(30, 11)$ & 7 & 5 & YES & YES & YES & -- & 3344\\
$(c; 0, 1, 1; 5)$ & 6 & $(41, 17)$ & 8 & 1 & YES & YES & YES & -- & 3345\\
$(c; 0, 2, 0; 7)$ & 6 & $(26, 11)$ & 7 & 1 & YES & YES & YES & -- & 3346\\
$(c; 0, 2, 0; 7)$ & 6 & $(37, 11)$ & 8 & 1 & YES & YES & YES & -- & 3347\\
$(c; 0, 2, 0; 7)$ & 6 & $(48, 11)$ & 9 & 1 & YES & YES & YES & -- & 3348\\
$(c; 0, 2, 1; 19)$ & 7 & $(16, 5)$ & 7 & 1 & YES & YES & YES & -- & 3349\\
$(c; 0, 2, 1; 19)$ & 7 & $(41, 12)$ & 8 & 1 & YES & YES & YES & -- & 3350\\
$(c; 0, 2, 2; 6)$ & 8 & $(21, 5)$ & 8 & 3 & YES & YES & YES & -- & 3351\\
$(c; 0, 3, 0; 17)$ & 7 & $(16, 5)$ & 7 & 1 & YES & YES & YES & -- & 3352\\
$(c; 0, 3, 0; 17)$ & 7 & $(24, 5)$ & 8 & 1 & YES & YES & YES & -- & 3353\\
$(d; 0, 0, 0; 5)$ & 5 & $(63, 26)$ & 9 & 1 & YES & YES & YES & -- & 3354\\
$(d; 0, 0, 0; 5)$ & 5 & $(64, 27)$ & 9 & 1 & YES & YES & YES & -- & 3355\\
$(d; 0, 0, 0; 5)$ & 5 & $(65, 24)$ & 9 & 5 & YES & YES & YES & -- & 3356\\
$(d; 0, 0, 0; 5)$ & 5 & $(70, 29)$ & 9 & 5 & YES & YES & YES & -- & 3357\\
$(d; 0, 0, 0; 5)$ & 5 & $(75, 31)$ & 9 & 5 & YES & YES & YES & -- & 3358\\
$(d; 0, 0, 0; 5)$ & 5 & $(79, 24)$ & 10 & 1 & YES & YES & YES & -- & 3359\\
$(d; 0, 0, 0; 5)$ & 5 & $(104, 29)$ & 10 & 1 & YES & YES & YES & -- & 3360\\
$(d; 0, 0, 1; 14)$ & 6 & $(23, 9)$ & 7 & 1 & YES & YES & YES & -- & 3361\\
$(d; 0, 0, 1; 14)$ & 6 & $(39, 16)$ & 8 & 1 & YES & YES & YES & -- & 3362\\
$(d; 0, 0, 1; 14)$ & 6 & $(41, 17)$ & 8 & 1 & YES & YES & YES & -- & 3363\\
$(d; 0, 0, 1; 14)$ & 6 & $(46, 17)$ & 8 & 2 & YES & YES & YES & -- & 3364\\
$(d; 0, 0, 2; 9)$ & 7 & $(7, 3)$ & 4 & 1 & YES & YES & NO(2) & -- & 3365\\
$(d; 0, 0, 2; 9)$ & 7 & $(16, 5)$ & 7 & 1 & YES & YES & YES & -- & 3366\\
$(d; 0, 1, 0; 6)$ & 6 & $(41, 12)$ & 8 & 1 & YES & YES & YES & -- & 3367\\
$(d; 0, 1, 0; 6)$ & 6 & $(43, 12)$ & 8 & 1 & YES & YES & YES & -- & 3368\\
$(d; 0, 1, 1; 17)$ & 7 & $(34, 13)$ & 7 & 17 & YES & YES & YES & -- & 3369\\
$(d; 0, 1, 1; 17)$ & 7 & $(41, 12)$ & 8 & 1 & YES & YES & YES & -- & 3370\\
$(d; 0, 1, 2; 11)$ & 8 & $(9, 4)$ & 5 & 1 & YES & YES & YES & -- & 3371\\
$(e; 1, 0, 0; 18)$ & 6 & $(12, 5)$ & 5 & 6 & YES & YES & YES & -- & 3372\\
$(e; 1, 0, 0; 18)$ & 6 & $(17, 7)$ & 6 & 1 & YES & YES & YES & -- & 3373\\
$(e; 1, 0, 0; 18)$ & 6 & $(21, 8)$ & 6 & 3 & YES & YES & YES & -- & 3374\\
$(e; 1, 0, 0; 18)$ & 6 & $(23, 9)$ & 7 & 1 & YES & YES & YES & -- & 3375\\
$(e; 1, 0, 0; 18)$ & 6 & $(24, 7)$ & 7 & 6 & YES & YES & YES & -- & 3376\\
$(e; 1, 0, 0; 18)$ & 6 & $(33, 10)$ & 8 & 3 & YES & YES & YES & -- & 3377\\
$(e; 1, 1, 0; 23)$ & 7 & $(12, 5)$ & 5 & 1 & YES & YES & YES & -- & 3378\\
$(e; 1, 1, 0; 23)$ & 7 & $(13, 5)$ & 5 & 1 & YES & YES & YES & -- & 3379\\
$(e; 1, 2, 0; 28)$ & 8 & $(13, 4)$ & 6 & 1 & YES & YES & YES & -- & 3380\\
$(e; 1, 2, 0; 28)$ & 8 & $(13, 5)$ & 5 & 1 & YES & YES & YES & -- & 3381\\
$(e; 2, 0, 0; 24)$ & 7 & $(13, 5)$ & 5 & 1 & YES & YES & YES & -- & 3382\\
$(e; 2, 0, 0; 24)$ & 7 & $(17, 5)$ & 6 & 1 & YES & YES & YES & -- & 3383\\
$(e; 2, 0, 0; 24)$ & 7 & $(18, 7)$ & 6 & 6 & YES & YES & YES & -- & 3384\\
$(e; 2, 0, 0; 24)$ & 7 & $(21, 8)$ & 6 & 3 & YES & YES & YES & -- & 3385\\
$(e; 2, 3, 0; 45)$ & 10 & $(6, 1)$ & 5 & 3 & YES & YES & YES & -- & 3386\\
$(f; 0, 0, 0; 6)$ & 4 & $(22, 9)$ & 7 & 2 & YES & YES & NO(2) & -- & 3387\\
$(f; 0, 0, 0; 6)$ & 4 & $(23, 9)$ & 7 & 1 & YES & YES & NO(2) & -- & 3388\\
$(f; 0, 0, 0; 6)$ & 4 & $(26, 11)$ & 7 & 2 & YES & YES & NO(2) & -- & 3389\\
$(f; 0, 0, 0; 6)$ & 4 & $(30, 11)$ & 7 & 6 & YES & YES & NO(2) & -- & 3390\\
$(f; 0, 0, 0; 6)$ & 4 & $(37, 11)$ & 8 & 1 & YES & YES & YES & -- & 3391\\
$(f; 0, 0, 0; 6)$ & 4 & $(37, 16)$ & 9 & 1 & YES & YES & YES & -- & 3392\\
$(f; 0, 0, 0; 6)$ & 4 & $(41, 15)$ & 8 & 1 & YES & YES & YES & -- & 3393\\
$(f; 0, 0, 0; 6)$ & 4 & $(45, 16)$ & 9 & 3 & YES & YES & YES & -- & 3394\\
$(f; 0, 0, 0; 6)$ & 4 & $(45, 17)$ & 9 & 3 & YES & YES & YES & -- & 3395\\
$(f; 0, 0, 0; 6)$ & 4 & $(69, 29)$ & 9 & 3 & YES & YES & YES & -- & 3396\\
$(f; 0, 0, 0; 6)$ & 4 & $(80, 33)$ & 10 & 2 & YES & YES & YES & -- & 3397\\
$(f; 0, 0, 0; 6)$ & 4 & $(89, 25)$ & 10 & 1 & YES & YES & YES & -- & 3398\\
$(f; 0, 0, 0; 6)$ & 4 & $(91, 27)$ & 10 & 1 & YES & YES & YES & -- & 3399\\
$(f; 0, 0, 0; 6)$ & 4 & $(97, 37)$ & 10 & 1 & YES & YES & YES & -- & 3400\\
$(f; 0, 0, 0; 6)$ & 4 & $(98, 27)$ & 10 & 2 & YES & YES & YES & -- & 3401\\
$(f; 0, 0, 0; 6)$ & 4 & $(106, 41)$ & 10 & 2 & YES & YES & YES & -- & 3402\\
$(f; 0, 0, 0; 6)$ & 4 & $(111, 46)$ & 10 & 3 & YES & YES & YES & -- & 3403\\
$(f; 0, 0, 0; 6)$ & 4 & $(123, 47)$ & 10 & 3 & YES & YES & YES & -- & 3404\\
$(f; 0, 0, 0; 6)$ & 4 & $(124, 23)$ & 12 & 2 & YES & YES & YES & -- & 3405\\
$(f; 0, 0, 0; 6)$ & 4 & $(140, 39)$ & 11 & 2 & YES & YES & YES & -- & 3406\\
$(f; 0, 0, 0; 6)$ & 4 & $(140, 41)$ & 11 & 2 & YES & YES & YES & -- & 3407\\
$(f; 0, 1, 0; 7)$ & 5 & $(19, 4)$ & 7 & 1 & YES & YES & YES & -- & 3408\\
$(f; 0, 1, 0; 7)$ & 5 & $(24, 11)$ & 8 & 1 & YES & YES & YES & -- & 3409\\
$(f; 0, 1, 0; 7)$ & 5 & $(29, 7)$ & 10 & 1 & YES & YES & YES & -- & 3410\\
$(g; 0, 0, 0; 19)$ & 6 & $(12, 5)$ & 5 & 1 & YES & YES & YES & -- & 3411\\
$(g; 0, 0, 0; 19)$ & 6 & $(17, 7)$ & 6 & 1 & YES & YES & YES & -- & 3412\\
$(g; 0, 0, 0; 19)$ & 6 & $(21, 8)$ & 6 & 1 & YES & YES & YES & -- & 3413\\
$(g; 0, 0, 0; 19)$ & 6 & $(23, 9)$ & 7 & 1 & YES & YES & YES & -- & 3414\\
$(g; 0, 0, 0; 19)$ & 6 & $(23, 10)$ & 7 & 1 & YES & YES & YES & -- & 3415\\
$(g; 0, 0, 0; 19)$ & 6 & $(24, 7)$ & 7 & 1 & YES & YES & YES & -- & 3416\\
$(g; 0, 0, 1; 26)$ & 7 & $(13, 5)$ & 5 & 13 & YES & YES & YES & -- & 3417\\
$(g; 0, 0, 1; 26)$ & 7 & $(17, 5)$ & 6 & 1 & YES & YES & YES & -- & 3418\\
$(g; 0, 0, 1; 26)$ & 7 & $(17, 7)$ & 6 & 1 & YES & YES & YES & -- & 3419\\
$(g; 0, 0, 2; 11)$ & 8 & $(10, 3)$ & 5 & 1 & YES & YES & YES & -- & 3420\\
$(g; 0, 0, 2; 11)$ & 8 & $(11, 3)$ & 5 & 11 & YES & YES & YES & -- & 3421\\
$(g; 0, 0, 2; 11)$ & 8 & $(13, 5)$ & 5 & 1 & YES & YES & YES & -- & 3422\\
$(g; 0, 1, 0; 24)$ & 7 & $(9, 4)$ & 5 & 3 & YES & YES & NO(2) & -- & 3423\\
$(g; 0, 1, 0; 24)$ & 7 & $(11, 4)$ & 5 & 1 & YES & YES & YES & -- & 3424\\
$(g; 0, 1, 0; 24)$ & 7 & $(13, 5)$ & 5 & 1 & YES & YES & YES & -- & 3425\\
$(g; 0, 1, 0; 24)$ & 7 & $(17, 5)$ & 6 & 1 & YES & YES & YES & -- & 3426\\
$(g; 0, 1, 1; 33)$ & 8 & $(8, 3)$ & 4 & 1 & YES & YES & YES & -- & 3427\\
$(g; 0, 1, 1; 33)$ & 8 & $(10, 3)$ & 5 & 1 & YES & YES & YES & -- & 3428\\
$(g; 0, 2, 0; 29)$ & 8 & $(10, 3)$ & 5 & 1 & YES & YES & YES & -- & 3429\\
$(g; 0, 2, 2; 17)$ & 10 & $(5, 1)$ & 4 & 1 & YES & YES & YES & -- & 3430\\
$(g; 1, 0, 1; 38)$ & 8 & $(16, 5)$ & 7 & 2 & YES & YES & YES & -- & 3431\\
$(g; 1, 1, 0; 9)$ & 8 & $(7, 3)$ & 4 & 1 & YES & YES & YES & -- & 3432\\
$(g; 1, 1, 0; 9)$ & 8 & $(13, 5)$ & 5 & 1 & YES & YES & YES & -- & 3433\\
$(g; 3, 1, 0; 30)$ & 10 & $(2, 1)$ & 1 & 2 & YES & YES & YES & -- & 3434\\
$(h; 0, 0, 0; 6)$ & 5 & $(21, 8)$ & 6 & 3 & YES & YES & YES & -- & 3435\\
$(h; 0, 0, 0; 6)$ & 5 & $(27, 10)$ & 7 & 3 & YES & YES & YES & -- & 3436\\
$(h; 0, 0, 0; 6)$ & 5 & $(31, 12)$ & 7 & 1 & YES & YES & YES & -- & 3437\\
$(h; 0, 0, 0; 6)$ & 5 & $(37, 14)$ & 8 & 1 & YES & YES & YES & -- & 3438\\
$(h; 0, 1, 0; 8)$ & 6 & $(12, 5)$ & 5 & 4 & YES & YES & YES & -- & 3439\\
$(h; 0, 1, 0; 8)$ & 6 & $(17, 7)$ & 6 & 1 & YES & YES & YES & -- & 3440\\
$(h; 0, 1, 0; 8)$ & 6 & $(21, 8)$ & 6 & 1 & YES & YES & YES & -- & 3441\\
$(h; 0, 1, 0; 8)$ & 6 & $(23, 9)$ & 7 & 1 & YES & YES & YES & -- & 3442\\
$(h; 0, 1, 0; 8)$ & 6 & $(24, 7)$ & 7 & 8 & YES & YES & YES & -- & 3443\\
$(h; 0, 2, 0; 10)$ & 7 & $(13, 5)$ & 5 & 1 & YES & YES & YES & -- & 3444\\
$(h; 0, 2, 0; 10)$ & 7 & $(18, 7)$ & 6 & 2 & YES & YES & YES & -- & 3445\\
$(h; 0, 2, 0; 10)$ & 7 & $(24, 7)$ & 7 & 2 & YES & YES & YES & -- & 3446\\
$(i; 0, 0, 0; 9)$ & 5 & $(12, 5)$ & 5 & 3 & YES & YES & NO(2) & -- & 3447\\
$(i; 0, 0, 0; 9)$ & 5 & $(16, 7)$ & 6 & 1 & YES & YES & YES & -- & 3448\\
$(i; 0, 0, 0; 9)$ & 5 & $(26, 11)$ & 7 & 1 & YES & YES & YES & -- & 3449\\
$(i; 0, 0, 0; 9)$ & 5 & $(35, 13)$ & 8 & 1 & YES & YES & YES & -- & 3450\\
$(i; 0, 0, 0; 9)$ & 5 & $(43, 12)$ & 8 & 1 & YES & YES & YES & -- & 3451\\
$(i; 0, 1, 0; 12)$ & 6 & $(13, 4)$ & 6 & 1 & YES & YES & YES & -- & 3452\\
$(i; 0, 1, 0; 12)$ & 6 & $(33, 10)$ & 8 & 3 & YES & YES & YES & -- & 3453\\
$(i; 0, 2, 0; 15)$ & 7 & $(9, 4)$ & 5 & 3 & YES & YES & YES & -- & 3454\\
$(i; 0, 2, 0; 15)$ & 7 & $(24, 7)$ & 7 & 3 & YES & YES & YES & -- & 3455\\
$(j; 0, 0, 0; 8)$ & 5 & $(32, 13)$ & 9 & 8 & YES & YES & YES & -- & 3456\\
$(j; 0, 0, 0; 8)$ & 5 & $(40, 17)$ & 9 & 8 & YES & YES & YES & -- & 3457\\
$(j; 0, 0, 0; 8)$ & 5 & $(75, 29)$ & 9 & 1 & YES & YES & YES & -- & 3458\\
$(j; 0, 0, 0; 8)$ & 5 & $(76, 29)$ & 9 & 4 & YES & YES & YES & -- & 3459\\
$(j; 0, 0, 0; 8)$ & 5 & $(89, 26)$ & 10 & 1 & YES & YES & YES & -- & 3460\\
$(j; 0, 1, 0; 10)$ & 6 & $(27, 11)$ & 8 & 1 & YES & YES & YES & -- & 3461\\
$(j; 0, 1, 0; 10)$ & 6 & $(37, 11)$ & 8 & 1 & YES & YES & YES & -- & 3462\\
$(j; 0, 1, 0; 10)$ & 6 & $(43, 13)$ & 9 & 1 & YES & YES & YES & -- & 3463
\end{longtable}
\subsection{2 chains, $K^2 = 4$}
\begin{longtable}{|c|c|c|c|c|c|c|c|c|c|}
\hline
\multicolumn{10}{|c|}{2 chains, $K^2 = 4$}\\
\hline
$(n,a)$ & Length & $(n,a)$ & Length & GCD & Nef & $\mathbb Q$-ef & Obstruction 0 & WH & Index\\
\hline
\endfirsthead

\hline
$(n,a)$ & Length & $(n,a)$ & Length & GCD & Nef & $\mathbb Q$-ef & Obstruction 0 & WH & Index\\
\hline
\endhead
\hline
\endfoot

$(29, 9)$ & 8 & $(25, 9)$ & 7 & 1 & YES & YES & YES & -- & 3464\\
$(39, 14)$ & 8 & $(12, 5)$ & 5 & 3 & YES & YES & YES & -- & 3465\\
$(45, 19)$ & 8 & $(44, 13)$ & 8 & 1 & YES & YES & NO(2) & -- & 3466\\
$(49, 19)$ & 8 & $(40, 11)$ & 8 & 1 & YES & YES & NO(2) & -- & 3467\\
$(56, 15)$ & 9 & $(43, 18)$ & 8 & 1 & YES & YES & NO(2) & -- & 3468\\
$(57, 16)$ & 9 & $(41, 12)$ & 8 & 1 & YES & YES & YES & -- & 3469\\
$(58, 17)$ & 9 & $(50, 19)$ & 8 & 2 & YES & YES & YES & -- & 3470\\
$(61, 17)$ & 9 & $(55, 21)$ & 8 & 1 & YES & YES & YES & -- & 3471\\
$(63, 26)$ & 9 & $(35, 8)$ & 8 & 7 & YES & YES & YES & -- & 3472\\
$(64, 27)$ & 9 & $(40, 11)$ & 8 & 8 & YES & YES & YES & NO & 3473\\
$(64, 19)$ & 9 & $(45, 19)$ & 8 & 1 & YES & YES & YES & -- & 3474\\
$(65, 27)$ & 10 & $(34, 13)$ & 7 & 1 & YES & YES & YES & -- & 3475\\
$(65, 18)$ & 9 & $(46, 17)$ & 8 & 1 & YES & YES & YES & -- & 3476\\
$(69, 29)$ & 9 & $(40, 11)$ & 8 & 1 & YES & YES & YES & -- & 3477\\
$(71, 30)$ & 9 & $(27, 8)$ & 7 & 1 & YES & YES & NO(2) & -- & 3478\\
$(71, 21)$ & 9 & $(44, 17)$ & 8 & 1 & YES & YES & NO(2) & -- & 3479\\
$(76, 21)$ & 9 & $(44, 17)$ & 8 & 4 & YES & YES & YES & -- & 3480\\
$(79, 24)$ & 10 & $(19, 8)$ & 6 & 1 & YES & YES & YES & -- & 3481\\
$(79, 30)$ & 9 & $(23, 9)$ & 7 & 1 & YES & YES & YES & -- & 3482\\
$(80, 31)$ & 9 & $(37, 11)$ & 8 & 1 & YES & YES & YES & -- & 3483\\
$(83, 23)$ & 10 & $(32, 7)$ & 8 & 1 & YES & YES & YES & NO & 3484\\
$(89, 25)$ & 10 & $(19, 8)$ & 6 & 1 & YES & YES & NO(3) & -- & 3485\\
$(91, 27)$ & 10 & $(27, 10)$ & 7 & 1 & YES & YES & NO(2) & -- & 3486\\
$(92, 35)$ & 10 & $(29, 8)$ & 7 & 1 & YES & YES & YES & -- & 3487\\
$(95, 36)$ & 10 & $(24, 7)$ & 7 & 1 & YES & YES & YES & -- & 3488\\
$(97, 37)$ & 10 & $(32, 7)$ & 8 & 1 & YES & YES & YES & NO & 3489\\
$(98, 41)$ & 10 & $(18, 7)$ & 6 & 2 & YES & YES & YES & -- & 3490\\
$(98, 27)$ & 10 & $(22, 9)$ & 7 & 2 & YES & YES & YES & -- & 3491\\
$(98, 27)$ & 10 & $(26, 11)$ & 7 & 2 & YES & YES & YES & NO & 3492\\
$(98, 27)$ & 10 & $(44, 17)$ & 8 & 2 & YES & YES & YES & NO & 3493\\
$(98, 27)$ & 10 & $(61, 18)$ & 9 & 1 & YES & YES & YES & NO & 3494\\
$(100, 37)$ & 10 & $(31, 7)$ & 8 & 1 & YES & YES & NO(2) & NO & 3495\\
$(101, 30)$ & 10 & $(18, 7)$ & 6 & 1 & YES & YES & NO(2) & -- & 3496\\
$(101, 39)$ & 10 & $(18, 7)$ & 6 & 1 & YES & YES & YES & -- & 3497\\
$(106, 41)$ & 10 & $(13, 5)$ & 5 & 1 & YES & YES & YES & -- & 3498\\
$(108, 41)$ & 10 & $(17, 4)$ & 7 & 1 & YES & YES & YES & NO & 3499\\
$(109, 45)$ & 10 & $(25, 7)$ & 7 & 1 & YES & YES & NO(2) & NO & 3500\\
$(109, 30)$ & 10 & $(32, 9)$ & 8 & 1 & YES & YES & YES & -- & 3501\\
$(111, 43)$ & 10 & $(25, 7)$ & 7 & 1 & YES & YES & YES & -- & 3502\\
$(112, 31)$ & 10 & $(21, 8)$ & 6 & 7 & YES & YES & YES & NO & 3503\\
$(112, 31)$ & 10 & $(32, 9)$ & 8 & 16 & YES & YES & YES & -- & 3504\\
$(112, 47)$ & 10 & $(56, 23)$ & 9 & 56 & YES & YES & NO(2) & NO & 3505\\
$(113, 49)$ & 11 & $(13, 4)$ & 6 & 1 & YES & YES & YES & -- & 3506\\
$(119, 46)$ & 10 & $(18, 5)$ & 6 & 1 & YES & YES & YES & -- & 3507\\
$(121, 37)$ & 11 & $(12, 5)$ & 5 & 1 & YES & YES & YES & -- & 3508\\
$(121, 37)$ & 11 & $(29, 8)$ & 7 & 1 & YES & YES & YES & -- & 3509\\
$(121, 37)$ & 11 & $(44, 13)$ & 8 & 11 & YES & YES & YES & NO & 3510\\
$(124, 23)$ & 12 & $(21, 8)$ & 6 & 1 & YES & YES & YES & -- & 3511\\
$(127, 29)$ & 11 & $(37, 11)$ & 8 & 1 & YES & YES & YES & NO & 3512\\
$(129, 50)$ & 10 & $(25, 7)$ & 7 & 1 & YES & YES & YES & -- & 3513\\
$(131, 50)$ & 10 & $(10, 3)$ & 5 & 1 & YES & YES & NO(2) & -- & 3514\\
$(131, 55)$ & 10 & $(63, 26)$ & 9 & 1 & YES & YES & NO(2) & NO & 3515\\
$(134, 39)$ & 11 & $(29, 8)$ & 7 & 1 & YES & YES & YES & -- & 3516\\
$(137, 37)$ & 11 & $(37, 11)$ & 8 & 1 & YES & YES & NO(2) & NO & 3517\\
$(149, 41)$ & 11 & $(10, 3)$ & 5 & 1 & YES & YES & YES & -- & 3518\\
$(149, 44)$ & 11 & $(13, 5)$ & 5 & 1 & YES & YES & YES & -- & 3519\\
$(153, 56)$ & 11 & $(13, 5)$ & 5 & 1 & YES & YES & YES & -- & 3520\\
$(154, 45)$ & 11 & $(10, 3)$ & 5 & 2 & YES & YES & YES & -- & 3521\\
$(157, 46)$ & 11 & $(17, 7)$ & 6 & 1 & YES & YES & NO(2) & NO & 3522\\
$(163, 44)$ & 11 & $(17, 7)$ & 6 & 1 & YES & YES & YES & -- & 3523\\
$(163, 44)$ & 11 & $(33, 10)$ & 8 & 1 & YES & YES & YES & NO & 3524\\
$(166, 61)$ & 11 & $(18, 7)$ & 6 & 2 & YES & YES & YES & -- & 3525\\
$(166, 61)$ & 11 & $(44, 17)$ & 8 & 2 & YES & YES & YES & NO & 3526\\
$(169, 50)$ & 11 & $(23, 7)$ & 7 & 1 & YES & YES & YES & -- & 3527\\
$(170, 47)$ & 11 & $(44, 13)$ & 8 & 2 & YES & YES & YES & NO & 3528\\
$(170, 47)$ & 11 & $(89, 25)$ & 10 & 1 & YES & YES & YES & NO & 3529\\
$(171, 50)$ & 11 & $(17, 7)$ & 6 & 1 & YES & YES & NO(2) & NO & 3530\\
$(189, 55)$ & 12 & $(64, 19)$ & 9 & 1 & YES & YES & NO(2) & NO & 3531\\
$(194, 75)$ & 11 & $(13, 4)$ & 6 & 1 & YES & YES & NO(2) & -- & 3532\\
$(203, 60)$ & 12 & $(12, 5)$ & 5 & 1 & YES & YES & YES & -- & 3533\\
$(214, 79)$ & 12 & $(10, 3)$ & 5 & 2 & YES & YES & YES & -- & 3534\\
$(227, 87)$ & 12 & $(5, 1)$ & 4 & 1 & YES & YES & YES & -- & 3535\\
$(234, 89)$ & 12 & $(7, 2)$ & 4 & 1 & YES & YES & NO(2) & -- & 3536\\
$(235, 97)$ & 12 & $(10, 3)$ & 5 & 5 & YES & YES & YES & -- & 3537\\
$(236, 65)$ & 12 & $(24, 7)$ & 7 & 4 & YES & YES & YES & -- & 3538\\
$(237, 100)$ & 12 & $(10, 3)$ & 5 & 1 & YES & YES & NO(2) & -- & 3539\\
$(242, 65)$ & 12 & $(13, 4)$ & 6 & 1 & YES & YES & YES & -- & 3540\\
$(242, 65)$ & 12 & $(24, 7)$ & 7 & 2 & YES & YES & YES & NO & 3541\\
$(246, 73)$ & 12 & $(10, 3)$ & 5 & 2 & YES & YES & YES & -- & 3542\\
$(253, 106)$ & 12 & $(7, 3)$ & 4 & 1 & YES & YES & YES & -- & 3543\\
$(253, 68)$ & 12 & $(22, 5)$ & 7 & 11 & YES & YES & YES & -- & 3544\\
$(254, 105)$ & 12 & $(26, 11)$ & 7 & 2 & YES & YES & YES & NO & 3545\\
$(257, 108)$ & 12 & $(11, 3)$ & 5 & 1 & YES & YES & YES & -- & 3546\\
$(265, 112)$ & 12 & $(11, 3)$ & 5 & 1 & YES & YES & NO(2) & NO & 3547\\
$(266, 101)$ & 12 & $(44, 17)$ & 8 & 2 & YES & YES & YES & NO & 3548\\
$(271, 112)$ & 12 & $(7, 2)$ & 4 & 1 & YES & YES & NO(2) & -- & 3549\\
$(274, 115)$ & 12 & $(22, 9)$ & 7 & 2 & YES & YES & YES & NO & 3550\\
$(277, 116)$ & 12 & $(10, 3)$ & 5 & 1 & YES & YES & YES & NO & 3551\\
$(277, 116)$ & 12 & $(179, 75)$ & 11 & 1 & YES & YES & YES & NO & 3552\\
$(292, 85)$ & 13 & $(8, 3)$ & 4 & 4 & YES & YES & YES & -- & 3553\\
$(292, 111)$ & 12 & $(8, 3)$ & 4 & 4 & YES & YES & YES & -- & 3554\\
$(292, 111)$ & 12 & $(263, 100)$ & 12 & 1 & YES & YES & YES & NO & 3555\\
$(295, 112)$ & 12 & $(11, 3)$ & 5 & 1 & YES & YES & NO(2) & NO & 3556\\
$(298, 123)$ & 13 & $(5, 2)$ & 3 & 1 & YES & YES & YES & -- & 3557\\
$(301, 115)$ & 12 & $(8, 3)$ & 4 & 1 & YES & YES & YES & -- & 3558\\
$(303, 116)$ & 12 & $(10, 3)$ & 5 & 1 & YES & YES & YES & -- & 3559\\
$(304, 85)$ & 13 & $(11, 4)$ & 5 & 1 & YES & YES & YES & -- & 3560\\
$(312, 131)$ & 12 & $(17, 7)$ & 6 & 1 & YES & YES & NO(2) & NO & 3561\\
$(313, 121)$ & 12 & $(5, 2)$ & 3 & 1 & YES & YES & YES & -- & 3562\\
$(313, 91)$ & 13 & $(10, 3)$ & 5 & 1 & YES & YES & YES & -- & 3563\\
$(313, 91)$ & 13 & $(37, 11)$ & 8 & 1 & YES & YES & YES & NO & 3564\\
$(313, 91)$ & 13 & $(44, 13)$ & 8 & 1 & YES & YES & YES & NO & 3565\\
$(317, 131)$ & 12 & $(5, 2)$ & 3 & 1 & YES & YES & NO(2) & -- & 3566\\
$(317, 89)$ & 14 & $(7, 1)$ & 6 & 1 & YES & YES & NO(3) & NO & 3567\\
$(317, 131)$ & 12 & $(9, 2)$ & 5 & 1 & YES & YES & NO(2) & NO & 3568\\
$(317, 131)$ & 12 & $(167, 69)$ & 11 & 1 & YES & YES & NO(2) & NO & 3569\\
$(321, 95)$ & 13 & $(5, 2)$ & 3 & 1 & YES & YES & YES & NO & 3570\\
$(323, 134)$ & 13 & $(7, 2)$ & 4 & 1 & YES & YES & YES & NO & 3571\\
$(324, 91)$ & 13 & $(203, 57)$ & 12 & 1 & YES & YES & YES & 3669 & 3572\\
$(326, 99)$ & 13 & $(7, 3)$ & 4 & 1 & YES & YES & NO(2) & -- & 3573\\
$(326, 99)$ & 13 & $(25, 7)$ & 7 & 1 & YES & YES & YES & NO & 3574\\
$(332, 97)$ & 13 & $(3, 1)$ & 2 & 1 & YES & YES & YES & -- & 3575\\
$(332, 97)$ & 13 & $(16, 3)$ & 7 & 4 & YES & YES & NO(2) & NO & 3576\\
$(332, 97)$ & 13 & $(41, 12)$ & 8 & 1 & YES & YES & YES & NO & 3577\\
$(333, 101)$ & 13 & $(201, 61)$ & 12 & 3 & YES & YES & NO(2) & NO & 3578\\
$(337, 100)$ & 13 & $(5, 2)$ & 3 & 1 & YES & YES & NO(2) & -- & 3579\\
$(337, 100)$ & 13 & $(101, 30)$ & 10 & 1 & YES & YES & NO(2) & 3668 & 3580\\
$(338, 129)$ & 12 & $(7, 3)$ & 4 & 1 & YES & YES & YES & -- & 3581\\
$(338, 129)$ & 12 & $(131, 50)$ & 10 & 1 & YES & YES & NO(2) & NO & 3582\\
$(346, 131)$ & 13 & $(34, 13)$ & 7 & 2 & YES & YES & YES & NO & 3583\\
$(347, 134)$ & 13 & $(7, 2)$ & 4 & 1 & YES & YES & YES & NO & 3584\\
$(356, 139)$ & 13 & $(4, 1)$ & 3 & 4 & YES & YES & YES & -- & 3585\\
$(356, 139)$ & 13 & $(4, 1)$ & 3 & 4 & YES & YES & YES & NO & 3586\\
$(356, 139)$ & 13 & $(8, 3)$ & 4 & 4 & YES & YES & YES & NO & 3587\\
$(361, 151)$ & 13 & $(2, 1)$ & 1 & 1 & YES & YES & NO(3) & NO & 3588\\
$(363, 100)$ & 13 & $(13, 4)$ & 6 & 1 & YES & YES & YES & NO & 3589\\
$(365, 108)$ & 13 & $(2, 1)$ & 1 & 1 & YES & YES & YES & -- & 3590\\
$(365, 108)$ & 13 & $(7, 2)$ & 4 & 1 & YES & YES & YES & -- & 3591\\
$(365, 108)$ & 13 & $(61, 18)$ & 9 & 1 & YES & YES & YES & NO & 3592\\
$(383, 112)$ & 13 & $(2, 1)$ & 1 & 1 & YES & YES & YES & -- & 3593\\
$(383, 161)$ & 13 & $(157, 66)$ & 11 & 1 & YES & YES & YES & NO & 3594\\
$(385, 167)$ & 14 & $(30, 13)$ & 8 & 5 & YES & YES & YES & NO & 3595\\
$(391, 108)$ & 13 & $(13, 4)$ & 6 & 1 & YES & YES & NO(2) & NO & 3596\\
$(397, 116)$ & 13 & $(37, 11)$ & 8 & 1 & YES & YES & YES & NO & 3597\\
$(397, 116)$ & 13 & $(154, 45)$ & 11 & 1 & YES & YES & YES & NO & 3598\\
$(398, 111)$ & 13 & $(40, 11)$ & 8 & 2 & YES & YES & YES & NO & 3599\\
$(400, 117)$ & 13 & $(7, 3)$ & 4 & 1 & YES & YES & YES & -- & 3600\\
$(401, 155)$ & 13 & $(3, 1)$ & 2 & 1 & YES & YES & YES & -- & 3601\\
$(401, 155)$ & 13 & $(5, 2)$ & 3 & 1 & YES & YES & YES & -- & 3602\\
$(401, 155)$ & 13 & $(19, 7)$ & 6 & 1 & YES & YES & YES & NO & 3603\\
$(402, 175)$ & 14 & $(4, 1)$ & 3 & 2 & YES & YES & YES & -- & 3604\\
$(402, 175)$ & 14 & $(7, 3)$ & 4 & 1 & YES & YES & YES & NO & 3605\\
$(403, 153)$ & 13 & $(108, 41)$ & 10 & 1 & YES & YES & YES & NO & 3606\\
$(407, 112)$ & 13 & $(10, 3)$ & 5 & 1 & YES & YES & YES & -- & 3607\\
$(407, 171)$ & 13 & $(19, 8)$ & 6 & 1 & YES & YES & YES & NO & 3608\\
$(407, 112)$ & 13 & $(167, 46)$ & 11 & 1 & YES & YES & YES & NO & 3609\\
$(407, 119)$ & 13 & $(383, 112)$ & 13 & 1 & YES & YES & YES & NO & 3610\\
$(409, 121)$ & 13 & $(365, 108)$ & 13 & 1 & YES & YES & YES & NO & 3611\\
$(422, 183)$ & 14 & $(113, 49)$ & 11 & 1 & YES & YES & YES & NO & 3612\\
$(424, 155)$ & 14 & $(13, 5)$ & 5 & 1 & YES & YES & YES & NO & 3613\\
$(431, 128)$ & 13 & $(394, 117)$ & 13 & 1 & YES & YES & YES & NO & 3614\\
$(433, 128)$ & 13 & $(3, 1)$ & 2 & 1 & YES & YES & YES & -- & 3615\\
$(433, 128)$ & 13 & $(3, 1)$ & 2 & 1 & YES & YES & YES & NO & 3616\\
$(433, 131)$ & 14 & $(4, 1)$ & 3 & 1 & YES & YES & YES & -- & 3617\\
$(435, 182)$ & 14 & $(5, 2)$ & 3 & 5 & YES & YES & YES & -- & 3618\\
$(437, 100)$ & 14 & $(10, 3)$ & 5 & 1 & YES & YES & NO(2) & NO & 3619\\
$(437, 183)$ & 13 & $(26, 11)$ & 7 & 1 & YES & YES & YES & NO & 3620\\
$(437, 181)$ & 13 & $(128, 53)$ & 11 & 1 & YES & YES & YES & NO & 3621\\
$(438, 181)$ & 13 & $(196, 81)$ & 11 & 2 & YES & YES & NO(2) & 3657 & 3622\\
$(438, 181)$ & 13 & $(317, 131)$ & 12 & 1 & YES & YES & NO(2) & NO & 3623\\
$(438, 185)$ & 13 & $(438, 185)$ & 13 & 438 & YES & YES & NO(2) & NO & 3624\\
$(441, 169)$ & 13 & $(5, 1)$ & 4 & 1 & YES & YES & YES & -- & 3625\\
$(448, 173)$ & 14 & $(347, 134)$ & 13 & 1 & YES & YES & YES & NO & 3626\\
$(455, 188)$ & 13 & $(5, 2)$ & 3 & 5 & YES & YES & YES & -- & 3627\\
$(459, 179)$ & 14 & $(218, 85)$ & 12 & 1 & YES & YES & YES & NO & 3628\\
$(463, 176)$ & 13 & $(3, 1)$ & 2 & 1 & YES & YES & NO(2) & -- & 3629\\
$(463, 171)$ & 13 & $(4, 1)$ & 3 & 1 & YES & YES & NO(2) & -- & 3630\\
$(463, 171)$ & 13 & $(4, 1)$ & 3 & 1 & YES & YES & NO(2) & NO & 3631\\
$(463, 170)$ & 13 & $(5, 2)$ & 3 & 1 & YES & YES & NO(2) & -- & 3632\\
$(467, 181)$ & 13 & $(5, 2)$ & 3 & 1 & YES & YES & NO(2) & -- & 3633\\
$(467, 181)$ & 13 & $(49, 19)$ & 8 & 1 & YES & YES & NO(2) & NO & 3634\\
$(467, 196)$ & 13 & $(193, 81)$ & 11 & 1 & YES & YES & YES & NO & 3635\\
$(467, 193)$ & 13 & $(271, 112)$ & 12 & 1 & YES & YES & NO(2) & NO & 3636\\
$(474, 131)$ & 13 & $(7, 3)$ & 4 & 1 & YES & YES & YES & -- & 3637\\
$(474, 131)$ & 13 & $(32, 9)$ & 8 & 2 & YES & YES & YES & NO & 3638\\
$(477, 131)$ & 14 & $(5, 2)$ & 3 & 1 & YES & YES & YES & -- & 3639\\
$(481, 140)$ & 14 & $(7, 2)$ & 4 & 1 & YES & YES & NO(2) & -- & 3640\\
$(484, 89)$ & 16 & $(484, 89)$ & 16 & 484 & YES & YES & NO(3) & NO & 3641\\
$(485, 188)$ & 13 & $(4, 1)$ & 3 & 1 & YES & YES & YES & NO & 3642\\
$(485, 188)$ & 13 & $(485, 188)$ & 13 & 485 & YES & YES & NO(2) & NO & 3643\\
$(487, 186)$ & 13 & $(13, 5)$ & 5 & 1 & YES & YES & YES & NO & 3644\\
$(487, 136)$ & 14 & $(29, 8)$ & 7 & 1 & YES & YES & YES & NO & 3645\\
$(490, 207)$ & 13 & $(3, 1)$ & 2 & 1 & YES & YES & NO(2) & -- & 3646\\
$(490, 207)$ & 13 & $(4, 1)$ & 3 & 2 & YES & YES & NO(2) & -- & 3647\\
$(493, 207)$ & 13 & $(5, 2)$ & 3 & 1 & YES & YES & YES & -- & 3648\\
$(495, 137)$ & 14 & $(5, 2)$ & 3 & 5 & YES & YES & YES & -- & 3649\\
$(499, 139)$ & 14 & $(5, 2)$ & 3 & 1 & YES & YES & YES & NO & 3650\\
$(505, 212)$ & 13 & $(26, 11)$ & 7 & 1 & YES & YES & YES & NO & 3651\\
$(507, 196)$ & 13 & $(5, 1)$ & 4 & 1 & YES & YES & YES & -- & 3652\\
$(507, 196)$ & 13 & $(5, 1)$ & 4 & 1 & YES & YES & YES & NO & 3653\\
$(513, 215)$ & 14 & $(4, 1)$ & 3 & 1 & YES & YES & NO(2) & NO & 3654\\
$(513, 155)$ & 15 & $(43, 13)$ & 9 & 1 & YES & YES & YES & NO & 3655\\
$(513, 215)$ & 14 & $(43, 18)$ & 8 & 1 & YES & YES & NO(2) & NO & 3656\\
$(513, 212)$ & 13 & $(121, 50)$ & 10 & 1 & YES & YES & NO(2) & 3622 & 3657\\
$(517, 144)$ & 14 & $(140, 39)$ & 11 & 1 & YES & YES & YES & NO & 3658\\
$(519, 140)$ & 14 & $(241, 65)$ & 12 & 1 & YES & YES & YES & NO & 3659\\
$(522, 119)$ & 14 & $(5, 2)$ & 3 & 1 & YES & YES & NO(2) & -- & 3660\\
$(522, 119)$ & 14 & $(5, 2)$ & 3 & 1 & YES & YES & NO(2) & NO & 3661\\
$(536, 207)$ & 14 & $(158, 61)$ & 11 & 2 & YES & YES & YES & 3794 & 3662\\
$(548, 225)$ & 14 & $(4, 1)$ & 3 & 4 & YES & YES & YES & NO & 3663\\
$(551, 161)$ & 14 & $(2, 1)$ & 1 & 1 & YES & YES & NO(2) & -- & 3664\\
$(559, 157)$ & 14 & $(2, 1)$ & 1 & 1 & YES & YES & YES & -- & 3665\\
$(559, 165)$ & 14 & $(2, 1)$ & 1 & 1 & YES & YES & NO(2) & -- & 3666\\
$(559, 214)$ & 14 & $(5, 2)$ & 3 & 1 & YES & YES & NO(2) & NO & 3667\\
$(559, 166)$ & 14 & $(27, 8)$ & 7 & 1 & YES & YES & NO(2) & 3580 & 3668\\
$(559, 157)$ & 14 & $(57, 16)$ & 9 & 1 & YES & YES & YES & 3572 & 3669\\
$(565, 219)$ & 14 & $(4, 1)$ & 3 & 1 & YES & YES & YES & NO & 3670\\
$(565, 128)$ & 15 & $(35, 8)$ & 8 & 5 & YES & YES & YES & NO & 3671\\
$(577, 239)$ & 14 & $(2, 1)$ & 1 & 1 & YES & YES & YES & -- & 3672\\
$(577, 169)$ & 14 & $(5, 2)$ & 3 & 1 & YES & YES & YES & NO & 3673\\
$(577, 213)$ & 14 & $(5, 1)$ & 4 & 1 & YES & YES & YES & NO & 3674\\
$(577, 239)$ & 14 & $(12, 5)$ & 5 & 1 & YES & YES & YES & NO & 3675\\
$(577, 213)$ & 14 & $(214, 79)$ & 12 & 1 & YES & YES & YES & NO & 3676\\
$(579, 239)$ & 14 & $(3, 1)$ & 2 & 3 & YES & YES & NO(2) & NO & 3677\\
$(579, 221)$ & 14 & $(186, 71)$ & 11 & 3 & YES & YES & YES & NO & 3678\\
$(582, 215)$ & 14 & $(11, 4)$ & 5 & 1 & YES & YES & YES & NO & 3679\\
$(582, 215)$ & 14 & $(19, 7)$ & 6 & 1 & YES & YES & YES & NO & 3680\\
$(582, 223)$ & 15 & $(34, 13)$ & 7 & 2 & YES & YES & YES & NO & 3681\\
$(583, 246)$ & 14 & $(2, 1)$ & 1 & 1 & YES & YES & NO(2) & -- & 3682\\
$(592, 173)$ & 14 & $(10, 3)$ & 5 & 2 & YES & YES & YES & NO & 3683\\
$(592, 173)$ & 14 & $(41, 12)$ & 8 & 1 & YES & YES & YES & NO & 3684\\
$(592, 175)$ & 14 & $(433, 128)$ & 13 & 1 & YES & YES & YES & NO & 3685\\
$(595, 227)$ & 14 & $(4, 1)$ & 3 & 1 & YES & YES & YES & NO & 3686\\
$(597, 250)$ & 14 & $(437, 183)$ & 13 & 1 & YES & YES & YES & NO & 3687\\
$(599, 165)$ & 14 & $(18, 5)$ & 6 & 1 & YES & YES & NO(2) & NO & 3688\\
$(601, 137)$ & 15 & $(31, 7)$ & 8 & 1 & YES & YES & YES & NO & 3689\\
$(613, 237)$ & 14 & $(5, 1)$ & 4 & 1 & YES & YES & YES & NO & 3690\\
$(613, 234)$ & 14 & $(131, 50)$ & 10 & 1 & YES & YES & YES & NO & 3691\\
$(613, 234)$ & 14 & $(613, 234)$ & 14 & 613 & YES & YES & YES & NO & 3692\\
$(617, 182)$ & 15 & $(617, 182)$ & 15 & 617 & YES & YES & YES & NO & 3693\\
$(625, 258)$ & 14 & $(2, 1)$ & 1 & 1 & YES & YES & YES & -- & 3694\\
$(626, 263)$ & 14 & $(69, 29)$ & 9 & 1 & YES & YES & YES & NO & 3695\\
$(631, 231)$ & 15 & $(4, 1)$ & 3 & 1 & YES & YES & YES & -- & 3696\\
$(631, 234)$ & 14 & $(89, 33)$ & 10 & 1 & YES & YES & NO(2) & NO & 3697\\
$(632, 137)$ & 15 & $(19, 4)$ & 7 & 1 & YES & YES & YES & NO & 3698\\
$(633, 266)$ & 14 & $(257, 108)$ & 12 & 1 & YES & YES & YES & 3732 & 3699\\
$(633, 266)$ & 14 & $(445, 187)$ & 13 & 1 & YES & YES & YES & NO & 3700\\
$(640, 243)$ & 14 & $(5, 2)$ & 3 & 5 & YES & YES & YES & NO & 3701\\
$(641, 146)$ & 15 & $(9, 2)$ & 5 & 1 & YES & YES & YES & -- & 3702\\
$(642, 265)$ & 14 & $(4, 1)$ & 3 & 2 & YES & YES & YES & -- & 3703\\
$(642, 265)$ & 14 & $(642, 265)$ & 14 & 642 & YES & YES & YES & NO & 3704\\
$(647, 246)$ & 14 & $(2, 1)$ & 1 & 1 & YES & YES & YES & -- & 3705\\
$(647, 271)$ & 14 & $(2, 1)$ & 1 & 1 & YES & YES & YES & -- & 3706\\
$(649, 240)$ & 14 & $(2, 1)$ & 1 & 1 & YES & YES & YES & -- & 3707\\
$(650, 283)$ & 15 & $(3, 1)$ & 2 & 1 & YES & YES & YES & -- & 3708\\
$(653, 253)$ & 14 & $(3, 1)$ & 2 & 1 & YES & YES & YES & -- & 3709\\
$(653, 250)$ & 14 & $(6, 1)$ & 5 & 1 & YES & YES & NO(2) & NO & 3710\\
$(659, 184)$ & 15 & $(25, 7)$ & 7 & 1 & YES & YES & YES & NO & 3711\\
$(663, 196)$ & 14 & $(389, 115)$ & 13 & 1 & YES & YES & YES & NO & 3712\\
$(664, 185)$ & 15 & $(5, 2)$ & 3 & 1 & YES & YES & YES & -- & 3713\\
$(665, 258)$ & 14 & $(3, 1)$ & 2 & 1 & YES & YES & NO(2) & -- & 3714\\
$(665, 258)$ & 14 & $(67, 26)$ & 9 & 1 & YES & YES & NO(2) & NO & 3715\\
$(665, 258)$ & 14 & $(116, 45)$ & 10 & 1 & YES & YES & NO(2) & NO & 3716\\
$(674, 283)$ & 14 & $(2, 1)$ & 1 & 2 & YES & YES & YES & -- & 3717\\
$(674, 283)$ & 14 & $(131, 55)$ & 10 & 1 & YES & YES & YES & NO & 3718\\
$(683, 287)$ & 14 & $(3, 1)$ & 2 & 1 & YES & YES & YES & NO & 3719\\
$(691, 254)$ & 14 & $(3, 1)$ & 2 & 1 & YES & YES & YES & -- & 3720\\
$(691, 264)$ & 14 & $(301, 115)$ & 12 & 1 & YES & YES & YES & NO & 3721\\
$(691, 254)$ & 14 & $(691, 254)$ & 14 & 691 & YES & YES & YES & NO & 3722\\
$(694, 305)$ & 15 & $(3, 1)$ & 2 & 1 & YES & YES & YES & -- & 3723\\
$(697, 266)$ & 14 & $(34, 13)$ & 7 & 17 & YES & YES & YES & NO & 3724\\
$(698, 265)$ & 14 & $(3, 1)$ & 2 & 1 & YES & YES & YES & -- & 3725\\
$(698, 265)$ & 14 & $(13, 5)$ & 5 & 1 & YES & YES & YES & NO & 3726\\
$(698, 295)$ & 14 & $(265, 112)$ & 12 & 1 & YES & YES & NO(2) & NO & 3727\\
$(701, 204)$ & 15 & $(2, 1)$ & 1 & 1 & YES & YES & YES & NO & 3728\\
$(701, 207)$ & 15 & $(4, 1)$ & 3 & 1 & YES & YES & YES & -- & 3729\\
$(701, 207)$ & 15 & $(403, 119)$ & 13 & 1 & YES & YES & YES & 3812 & 3730\\
$(702, 295)$ & 14 & $(2, 1)$ & 1 & 2 & YES & YES & YES & -- & 3731\\
$(702, 295)$ & 14 & $(188, 79)$ & 11 & 2 & YES & YES & YES & 3699 & 3732\\
$(702, 295)$ & 14 & $(702, 295)$ & 14 & 702 & YES & YES & YES & NO & 3733\\
$(703, 267)$ & 14 & $(13, 5)$ & 5 & 1 & YES & YES & YES & NO & 3734\\
$(707, 274)$ & 14 & $(129, 50)$ & 10 & 1 & YES & YES & YES & NO & 3735\\
$(709, 293)$ & 14 & $(2, 1)$ & 1 & 1 & YES & YES & YES & -- & 3736\\
$(714, 299)$ & 14 & $(3, 1)$ & 2 & 3 & YES & YES & YES & -- & 3737\\
$(714, 299)$ & 14 & $(437, 183)$ & 13 & 1 & YES & YES & YES & NO & 3738\\
$(717, 212)$ & 14 & $(3, 1)$ & 2 & 3 & YES & YES & YES & -- & 3739\\
$(717, 212)$ & 14 & $(3, 1)$ & 2 & 3 & YES & YES & YES & NO & 3740\\
$(718, 213)$ & 15 & $(91, 27)$ & 10 & 1 & YES & YES & NO(2) & NO & 3741\\
$(729, 212)$ & 15 & $(4, 1)$ & 3 & 1 & YES & YES & YES & -- & 3742\\
$(729, 212)$ & 15 & $(533, 155)$ & 14 & 1 & YES & YES & YES & NO & 3743\\
$(734, 281)$ & 14 & $(5, 1)$ & 4 & 1 & YES & YES & YES & -- & 3744\\
$(734, 303)$ & 14 & $(5, 1)$ & 4 & 1 & YES & YES & YES & -- & 3745\\
$(741, 283)$ & 14 & $(4, 1)$ & 3 & 1 & YES & YES & YES & -- & 3746\\
$(752, 287)$ & 14 & $(3, 1)$ & 2 & 1 & YES & YES & YES & -- & 3747\\
$(752, 219)$ & 15 & $(4, 1)$ & 3 & 4 & YES & YES & YES & NO & 3748\\
$(752, 287)$ & 14 & $(131, 50)$ & 10 & 1 & YES & YES & YES & NO & 3749\\
$(753, 286)$ & 14 & $(2, 1)$ & 1 & 1 & YES & YES & YES & -- & 3750\\
$(753, 328)$ & 15 & $(62, 27)$ & 9 & 1 & YES & YES & YES & NO & 3751\\
$(753, 220)$ & 15 & $(332, 97)$ & 13 & 1 & YES & YES & NO(2) & NO & 3752\\
$(755, 229)$ & 15 & $(5, 1)$ & 4 & 5 & YES & YES & NO(2) & -- & 3753\\
$(755, 292)$ & 14 & $(44, 17)$ & 8 & 1 & YES & YES & YES & 3781 & 3754\\
$(755, 229)$ & 15 & $(755, 229)$ & 15 & 755 & YES & YES & NO(2) & NO & 3755\\
$(761, 223)$ & 15 & $(3, 1)$ & 2 & 1 & YES & YES & YES & -- & 3756\\
$(761, 223)$ & 15 & $(157, 46)$ & 11 & 1 & YES & YES & YES & 3814 & 3757\\
$(761, 226)$ & 15 & $(431, 128)$ & 13 & 1 & YES & YES & YES & 3823 & 3758\\
$(767, 322)$ & 14 & $(5, 2)$ & 3 & 1 & YES & YES & YES & NO & 3759\\
$(767, 223)$ & 15 & $(141, 41)$ & 11 & 1 & YES & YES & YES & NO & 3760\\
$(775, 143)$ & 16 & $(2, 1)$ & 1 & 1 & YES & YES & YES & -- & 3761\\
$(775, 143)$ & 16 & $(2, 1)$ & 1 & 1 & YES & YES & YES & NO & 3762\\
$(777, 214)$ & 15 & $(2, 1)$ & 1 & 1 & YES & YES & NO(2) & -- & 3763\\
$(777, 295)$ & 14 & $(4, 1)$ & 3 & 1 & YES & YES & NO(2) & -- & 3764\\
$(777, 295)$ & 14 & $(295, 112)$ & 12 & 1 & YES & YES & NO(2) & NO & 3765\\
$(780, 227)$ & 15 & $(2, 1)$ & 1 & 2 & YES & YES & YES & -- & 3766\\
$(781, 215)$ & 15 & $(29, 8)$ & 7 & 1 & YES & YES & YES & NO & 3767\\
$(784, 229)$ & 15 & $(4, 1)$ & 3 & 4 & YES & YES & YES & NO & 3768\\
$(788, 301)$ & 14 & $(2, 1)$ & 1 & 2 & YES & YES & YES & -- & 3769\\
$(788, 291)$ & 15 & $(5, 2)$ & 3 & 1 & YES & YES & YES & NO & 3770\\
$(788, 301)$ & 14 & $(8, 3)$ & 4 & 4 & YES & YES & YES & NO & 3771\\
$(790, 217)$ & 15 & $(2, 1)$ & 1 & 2 & YES & YES & YES & -- & 3772\\
$(790, 217)$ & 15 & $(3, 1)$ & 2 & 1 & YES & YES & YES & NO & 3773\\
$(793, 242)$ & 15 & $(3, 1)$ & 2 & 1 & YES & YES & YES & -- & 3774\\
$(797, 219)$ & 15 & $(3, 1)$ & 2 & 1 & YES & YES & NO(2) & -- & 3775\\
$(797, 219)$ & 15 & $(131, 36)$ & 11 & 1 & YES & YES & NO(2) & NO & 3776\\
$(802, 225)$ & 15 & $(2, 1)$ & 1 & 2 & YES & YES & YES & -- & 3777\\
$(802, 337)$ & 14 & $(2, 1)$ & 1 & 2 & YES & YES & YES & -- & 3778\\
$(803, 305)$ & 14 & $(5, 1)$ & 4 & 1 & YES & YES & YES & -- & 3779\\
$(808, 185)$ & 15 & $(2, 1)$ & 1 & 2 & YES & YES & NO(2) & -- & 3780\\
$(820, 317)$ & 14 & $(31, 12)$ & 7 & 1 & YES & YES & YES & 3754 & 3781\\
$(820, 317)$ & 14 & $(44, 17)$ & 8 & 4 & YES & YES & NO(2) & NO & 3782\\
$(822, 239)$ & 15 & $(2, 1)$ & 1 & 2 & YES & YES & YES & -- & 3783\\
$(822, 239)$ & 15 & $(86, 25)$ & 10 & 2 & YES & YES & YES & NO & 3784\\
$(830, 253)$ & 16 & $(10, 3)$ & 5 & 10 & YES & YES & YES & NO & 3785\\
$(833, 246)$ & 15 & $(2, 1)$ & 1 & 1 & YES & YES & YES & -- & 3786\\
$(833, 253)$ & 15 & $(56, 17)$ & 9 & 7 & YES & YES & YES & NO & 3787\\
$(833, 246)$ & 15 & $(342, 101)$ & 13 & 1 & YES & YES & YES & NO & 3788\\
$(852, 229)$ & 15 & $(346, 93)$ & 13 & 2 & YES & YES & YES & 3811 & 3789\\
$(860, 263)$ & 15 & $(3, 1)$ & 2 & 1 & YES & YES & YES & -- & 3790\\
$(863, 256)$ & 15 & $(5, 1)$ & 4 & 1 & YES & YES & YES & -- & 3791\\
$(877, 266)$ & 15 & $(2, 1)$ & 1 & 1 & YES & YES & NO(2) & -- & 3792\\
$(878, 339)$ & 15 & $(5, 2)$ & 3 & 1 & YES & YES & YES & NO & 3793\\
$(878, 339)$ & 15 & $(44, 17)$ & 8 & 2 & YES & YES & YES & 3662 & 3794\\
$(882, 337)$ & 14 & $(5, 2)$ & 3 & 1 & YES & YES & YES & NO & 3795\\
$(889, 246)$ & 15 & $(2, 1)$ & 1 & 1 & YES & YES & YES & -- & 3796\\
$(893, 246)$ & 15 & $(5, 2)$ & 3 & 1 & YES & YES & YES & -- & 3797\\
$(893, 246)$ & 15 & $(236, 65)$ & 12 & 1 & YES & YES & YES & NO & 3798\\
$(903, 274)$ & 15 & $(56, 17)$ & 9 & 7 & YES & YES & YES & NO & 3799\\
$(907, 264)$ & 15 & $(2, 1)$ & 1 & 1 & YES & YES & YES & NO & 3800\\
$(913, 207)$ & 16 & $(13, 3)$ & 6 & 1 & YES & YES & YES & NO & 3801\\
$(915, 338)$ & 15 & $(3, 1)$ & 2 & 3 & YES & YES & YES & -- & 3802\\
$(920, 273)$ & 15 & $(64, 19)$ & 9 & 8 & YES & YES & NO(2) & NO & 3803\\
$(928, 353)$ & 15 & $(5, 2)$ & 3 & 1 & YES & YES & YES & NO & 3804\\
$(932, 283)$ & 16 & $(79, 24)$ & 10 & 1 & YES & YES & YES & NO & 3805\\
$(935, 259)$ & 15 & $(11, 3)$ & 5 & 11 & YES & YES & YES & NO & 3806\\
$(943, 215)$ & 16 & $(2, 1)$ & 1 & 1 & YES & YES & NO(2) & NO & 3807\\
$(943, 215)$ & 16 & $(943, 215)$ & 16 & 943 & YES & YES & NO(2) & NO & 3808\\
$(944, 261)$ & 15 & $(29, 8)$ & 7 & 1 & YES & YES & YES & NO & 3809\\
$(945, 254)$ & 15 & $(4, 1)$ & 3 & 1 & YES & YES & YES & -- & 3810\\
$(945, 254)$ & 15 & $(253, 68)$ & 12 & 1 & YES & YES & YES & 3789 & 3811\\
$(955, 282)$ & 15 & $(149, 44)$ & 11 & 1 & YES & YES & YES & 3730 & 3812\\
$(957, 284)$ & 15 & $(10, 3)$ & 5 & 1 & YES & YES & YES & NO & 3813\\
$(959, 281)$ & 15 & $(58, 17)$ & 9 & 1 & YES & YES & YES & 3757 & 3814\\
$(965, 282)$ & 15 & $(7, 2)$ & 4 & 1 & YES & YES & YES & NO & 3815\\
$(985, 407)$ & 15 & $(2, 1)$ & 1 & 1 & YES & YES & YES & -- & 3816\\
$(987, 292)$ & 15 & $(17, 5)$ & 6 & 1 & YES & YES & YES & NO & 3817\\
$(992, 277)$ & 15 & $(11, 3)$ & 5 & 1 & YES & YES & YES & NO & 3818\\
$(997, 295)$ & 15 & $(4, 1)$ & 3 & 1 & YES & YES & YES & NO & 3819\\
$(997, 295)$ & 15 & $(365, 108)$ & 13 & 1 & YES & YES & YES & NO & 3820\\
$(1024, 283)$ & 15 & $(7, 2)$ & 4 & 1 & YES & YES & YES & NO & 3821\\
$(1025, 303)$ & 15 & $(2, 1)$ & 1 & 1 & YES & YES & YES & -- & 3822\\
$(1027, 305)$ & 15 & $(165, 49)$ & 11 & 1 & YES & YES & YES & 3758 & 3823\\
$(1042, 403)$ & 15 & $(5, 2)$ & 3 & 1 & YES & YES & YES & NO & 3824\\
$(1055, 242)$ & 16 & $(4, 1)$ & 3 & 1 & YES & YES & YES & -- & 3825\\
$(1055, 242)$ & 16 & $(22, 5)$ & 7 & 1 & YES & YES & YES & NO & 3826\\
$(1096, 303)$ & 15 & $(7, 2)$ & 4 & 1 & YES & YES & YES & NO & 3827\\
$(1117, 432)$ & 15 & $(287, 111)$ & 12 & 1 & YES & YES & YES & NO & 3828\\
$(1149, 206)$ & 17 & $(3, 1)$ & 2 & 3 & YES & YES & YES & NO & 3829\\
$(1149, 206)$ & 17 & $(4, 1)$ & 3 & 1 & YES & YES & YES & NO & 3830\\
$(1420, 393)$ & 16 & $(271, 75)$ & 12 & 1 & YES & YES & YES & NO & 3831\\
$(a; 0, 0, 0; 3)$ & 4 & $(290, 81)$ & 12 & 1 & YES & YES & YES & -- & 3832\\
$(a; 1, 0, 0; 13)$ & 5 & $(140, 41)$ & 11 & 1 & YES & YES & NO(2) & -- & 3833\\
$(b; 0, 0, 0; 14)$ & 5 & $(112, 47)$ & 10 & 14 & YES & YES & YES & -- & 3834\\
$(b; 0, 0, 0; 14)$ & 5 & $(123, 47)$ & 10 & 1 & YES & YES & YES & -- & 3835\\
$(b; 0, 0, 0; 14)$ & 5 & $(124, 23)$ & 12 & 2 & YES & YES & YES & -- & 3836\\
$(b; 0, 0, 0; 14)$ & 5 & $(145, 56)$ & 11 & 1 & YES & YES & YES & -- & 3837\\
$(b; 0, 0, 1; 4)$ & 6 & $(65, 19)$ & 9 & 1 & YES & YES & YES & -- & 3838\\
$(b; 0, 0, 1; 4)$ & 6 & $(105, 31)$ & 10 & 1 & YES & YES & YES & -- & 3839\\
$(b; 0, 0, 1; 4)$ & 6 & $(140, 41)$ & 11 & 4 & YES & YES & YES & -- & 3840\\
$(b; 0, 0, 2; 26)$ & 7 & $(40, 11)$ & 8 & 2 & YES & YES & YES & -- & 3841\\
$(b; 0, 0, 2; 26)$ & 7 & $(79, 24)$ & 10 & 1 & YES & YES & YES & -- & 3842\\
$(b; 0, 1, 0; 19)$ & 6 & $(95, 29)$ & 10 & 19 & YES & YES & YES & -- & 3843\\
$(b; 0, 1, 0; 19)$ & 6 & $(98, 29)$ & 10 & 1 & YES & YES & NO(2) & -- & 3844\\
$(b; 0, 1, 1; 27)$ & 7 & $(41, 17)$ & 8 & 1 & YES & YES & YES & -- & 3845\\
$(b; 0, 1, 1; 27)$ & 7 & $(56, 13)$ & 10 & 1 & YES & YES & YES & -- & 3846\\
$(b; 0, 1, 1; 27)$ & 7 & $(59, 18)$ & 9 & 1 & YES & YES & YES & -- & 3847\\
$(b; 1, 0, 1; 29)$ & 7 & $(41, 17)$ & 8 & 1 & YES & YES & YES & -- & 3848\\
$(b; 1, 1, 0; 27)$ & 7 & $(64, 19)$ & 9 & 1 & YES & YES & YES & -- & 3849\\
$(b; 2, 0, 1; 38)$ & 8 & $(17, 7)$ & 6 & 1 & YES & YES & YES & -- & 3850\\
$(c; 0, 0, 0; 4)$ & 4 & $(167, 69)$ & 11 & 1 & YES & YES & YES & -- & 3851\\
$(c; 0, 0, 0; 4)$ & 4 & $(256, 99)$ & 12 & 4 & YES & YES & NO(2) & -- & 3852\\
$(c; 0, 1, 0; 11)$ & 5 & $(116, 49)$ & 10 & 1 & YES & YES & NO(2) & -- & 3853\\
$(c; 0, 1, 0; 11)$ & 5 & $(140, 41)$ & 11 & 1 & YES & YES & YES & -- & 3854\\
$(c; 0, 1, 0; 11)$ & 5 & $(149, 44)$ & 11 & 1 & YES & YES & YES & -- & 3855\\
$(c; 0, 1, 0; 11)$ & 5 & $(169, 50)$ & 11 & 1 & YES & YES & NO(2) & -- & 3856\\
$(c; 0, 1, 0; 11)$ & 5 & $(169, 70)$ & 11 & 1 & YES & YES & YES & -- & 3857\\
$(c; 0, 1, 0; 11)$ & 5 & $(186, 71)$ & 11 & 1 & YES & YES & YES & -- & 3858\\
$(c; 0, 2, 0; 7)$ & 6 & $(89, 25)$ & 10 & 1 & YES & YES & NO(3) & -- & 3859\\
$(c; 0, 2, 0; 7)$ & 6 & $(124, 47)$ & 11 & 1 & YES & YES & YES & -- & 3860\\
$(c; 0, 2, 0; 7)$ & 6 & $(154, 45)$ & 11 & 7 & YES & YES & YES & -- & 3861\\
$(c; 0, 2, 1; 19)$ & 7 & $(41, 18)$ & 8 & 1 & YES & YES & NO(3) & -- & 3862\\
$(d; 0, 0, 0; 5)$ & 5 & $(49, 20)$ & 9 & 1 & YES & YES & YES & -- & 3863\\
$(e; 0, 0, 0; 4)$ & 5 & $(89, 26)$ & 10 & 1 & YES & YES & NO(2) & -- & 3864\\
$(e; 0, 0, 0; 4)$ & 5 & $(134, 37)$ & 11 & 2 & YES & YES & YES & -- & 3865\\
$(e; 0, 1, 0; 5)$ & 6 & $(71, 27)$ & 9 & 1 & YES & YES & YES & -- & 3866\\
$(e; 1, 0, 0; 18)$ & 6 & $(50, 21)$ & 8 & 2 & YES & YES & NO(2) & -- & 3867\\
$(e; 1, 0, 0; 18)$ & 6 & $(56, 23)$ & 9 & 2 & YES & YES & YES & -- & 3868\\
$(e; 1, 1, 0; 23)$ & 7 & $(61, 18)$ & 9 & 1 & YES & YES & NO(2) & -- & 3869\\
$(e; 2, 1, 0; 31)$ & 8 & $(58, 17)$ & 9 & 1 & YES & YES & YES & -- & 3870\\
$(f; 0, 0, 0; 6)$ & 4 & $(215, 64)$ & 12 & 1 & YES & YES & YES & -- & 3871\\
$(f; 0, 0, 0; 6)$ & 4 & $(246, 95)$ & 12 & 6 & YES & YES & YES & -- & 3872\\
$(g; 0, 0, 0; 19)$ & 6 & $(26, 11)$ & 7 & 1 & YES & YES & NO(2) & -- & 3873\\
$(g; 0, 0, 1; 26)$ & 7 & $(41, 17)$ & 8 & 1 & YES & YES & YES & -- & 3874\\
$(g; 0, 0, 2; 11)$ & 8 & $(13, 5)$ & 5 & 1 & YES & YES & NO(3) & -- & 3875\\
$(g; 0, 0, 2; 11)$ & 8 & $(40, 11)$ & 8 & 1 & YES & YES & YES & -- & 3876\\
$(h; 0, 0, 0; 6)$ & 5 & $(108, 41)$ & 10 & 6 & YES & YES & YES & -- & 3877\\
$(h; 0, 0, 0; 6)$ & 5 & $(119, 46)$ & 10 & 1 & YES & YES & YES & -- & 3878\\
$(h; 0, 1, 0; 8)$ & 6 & $(26, 11)$ & 7 & 2 & YES & YES & NO(2) & -- & 3879\\
$(h; 0, 1, 0; 8)$ & 6 & $(44, 17)$ & 8 & 4 & YES & YES & YES & -- & 3880\\
$(h; 0, 1, 0; 8)$ & 6 & $(69, 29)$ & 9 & 1 & YES & YES & YES & -- & 3881\\
$(h; 0, 1, 0; 8)$ & 6 & $(71, 27)$ & 9 & 1 & YES & YES & YES & -- & 3882\\
$(i; 0, 0, 0; 9)$ & 5 & $(166, 61)$ & 11 & 1 & YES & YES & YES & -- & 3883\\
$(j; 0, 0, 0; 8)$ & 5 & $(208, 79)$ & 11 & 8 & YES & YES & YES & -- & 3884
\end{longtable}
\subsection{2 chains, $K^2 = 5$}
\begin{longtable}{|c|c|c|c|c|c|c|c|c|c|}
\hline
\multicolumn{10}{|c|}{2 chains, $K^2 = 5$}\\
\hline
$(n,a)$ & Length & $(n,a)$ & Length & GCD & Nef & $\mathbb Q$-ef & Obstruction 0 & WH & Index\\
\hline
\endfirsthead

\hline
$(n,a)$ & Length & $(n,a)$ & Length & GCD & Nef & $\mathbb Q$-ef & Obstruction 0 & WH & Index\\
\hline
\endhead
\hline
\endfoot

$(b; 0, 0, 0; 14)$ & 5 & $(167, 69)$ & 11 & 1 & YES & YES & NO(3) & -- & 3885
\end{longtable}





%%%%%%%%%%%%%%%%%%%%%%%%%%%%%%%%%%%%%%%%%%%
\section{$2I_4 + 2I_2$}


Base curves:
\begin{itemize}
  \item $L_x = x$.
  \item $L_y = y$.
  \item $L_z = z$.
  \item $A = x - z$.
  \item $B = x + y + z$.
  \item $C = x - y + z$.
  \item $Q_1 = (x+z)^2 - y(x-z)$.
  \item $L_1 = x + y - z$.
  \item $Q_2 = (x+z)^2 + y(x-z)$.
  \item $L_2 = x - y - z$.
\end{itemize}
Fibration given by pencil
\[F_\lambda = ABC + \lambda L_xL_yL_z\]

Nine exceptionals are as follows:
\begin{itemize}
  \item $E_1$ - $E_2$ at $L_x \cap L_z \cap A = [0,1,0]$.
  \item $E_3$ - $E_4$ at $L_y \cap B \cap C = [-1,0,1]$.
  \item $E_5$ at $L_y \cap A = [1,0,1]$.
  \item $E_6$ at $L_x \cap C = [0,1,1]$.
  \item $E_7$ at $L_x \cap B = [0,-1,1]$.
  \item $E_8$ at $L_z \cap C = [1,1,0]$.
  \item $E_9$ at $L_z \cap B = [-1,1,0]$.
\end{itemize}
Singular fibers are as follows:
\begin{itemize}
  \item $\lambda = \infty$: $I_4$ fiber given by $L_z$, $L_x$, $L_y$, $E_1$ in order.
  \item $\lambda = 0$: $I_4$ fiber given by $B$, $A$, $C$, $E_3$ in order.
  \item $\lambda = 4$: $I_2$ fiber given by $Q_1$, $L_1$ with nodes at $B_1 = [-i,1+i,1]$ and $T_1 = [i,1-i,1]$.
  \item $\lambda = -4$: $I_2$ fiber given by $Q_2$, $L_2$ with nodes at $B_2 = [-i,-1-i,1]$ and $T_2 = [i,-1+i,1]$.
\end{itemize}
Special curves:
\begin{itemize}
  \item $H = x+z$, a section through $[0,1,0]$ and $[-1,0,1]$.
  \item $N = (2+i)x + iz + iy$, a double section through $[0,-1,1]$ and $T_1$.
  \item $BT = x + iy + z$, a double section through $B_1, T_2$ and $[-1,0,1]$.
  \item $TB = x - iy + z$, a double section through $T_1, B_2$ and $[-1,0,1]$.
  \item $BB = x + iz$, a double section through $B_1, B_2$ and $[0,1,0]$.
  \item $BT = x - iz$, a double section through $T_1, T_2$ and $[0,1,0]$.
\end{itemize}

Input:
\lstinputlisting[language=config]{../Tests/4422.txt}
Result:
%\usepackage{longtable}
\subsection{1 chain, $K^2 = 2$}
\begin{longtable}{|c|c|c|c|c|c|c|c|}
\hline
\multicolumn{8}{|c|}{1 chain, $K^2 = 2$}\\
\hline
$(n,a)$ & Len & Nef & $\mathbb Q$-ef & Obs 0 & $\overline c_1^2 / \overline c_2$ & $(P,K)$ & Index\\
\hline
\endfirsthead

\hline
$(n,a)$ & Len & Nef & $\mathbb Q$-ef & Obs 0 & $\overline c_1^2 / \overline c_2$ & $(P,K)$ & Index\\
\hline
\endhead
\hline
\endfoot

$(64,19)$ & 9 & YES & YES & NO(2) & $1.00$ & $(5,0)$ & 1\\
$(65,19)$ & 9 & YES & YES & NO(2) & $0.90$ & $(5,0)$ & 2\\
$(70,29)$ & 9 & YES & YES & YES & $1.00$ & $(1,2)$ & 3\\
$(71,26)$ & 9 & YES & YES & YES & $1.00$ & $(1,2)$ & 4\\
$(74,31)$ & 9 & YES & YES & YES & $1.10$ & $(1,2)$ & 5\\
$(75,31)$ & 9 & YES & YES & YES & $1.00$ & $(1,2)$ & 6\\
$(79,24)$ & 10 & YES & YES & YES & $1.10$ & $(1,2)$ & 7\\
$(79,29)$ & 9 & YES & YES & YES & $1.00$ & $(1,2)$ & 8\\
$(79,30)$ & 9 & YES & YES & YES & $1.20$ & $(1,2)$ & 9\\
$(80,31)$ & 9 & YES & YES & YES & $1.10$ & $(1,2)$ & 10\\
$(81,31)$ & 9 & YES & YES & YES & $1.10$ & $(1,2)$ & 11\\
$(84,25)$ & 10 & YES & YES & YES & $1.10$ & $(1,2)$ & 12\\
$(85,26)$ & 10 & YES & YES & YES & $1.10$ & $(1,2)$ & 13\\
$(86,25)$ & 10 & YES & YES & YES & $1.00$ & $(1,2)$ & 14\\
$(89,26)$ & 10 & YES & YES & YES & $1.10$ & $(1,2)$ & 15\\
$(89,34)$ & 9 & YES & YES & YES & $1.00$ & $(1,2)$ & 16\\
$(91,25)$ & 10 & YES & YES & YES & $1.00$ & $(1,2)$ & 17\\
$(91,27)$ & 10 & YES & YES & YES & $1.10$ & $(1,2)$ & 18\\
$(93,26)$ & 10 & YES & YES & YES & $1.10$ & $(1,2)$ & 19\\
$(98,27)$ & 10 & YES & YES & YES & $1.00$ & $(1,2)$ & 20\\
$(98,29)$ & 10 & YES & YES & YES & $1.11$ & $(1,2)$ & 21\\
$(99,29)$ & 10 & YES & YES & YES & $1.00$ & $(1,2)$ & 22\\
$(101,30)$ & 10 & YES & YES & YES & $0.89$ & $(1,2)$ & 23\\
$(103,39)$ & 10 & YES & YES & YES & $0.89$ & $(3,1)$ & 24\\
$(104,29)$ & 10 & YES & YES & YES & $1.10$ & $(1,2)$ & 25\\
$(105,31)$ & 10 & YES & YES & YES & $0.89$ & $(1,2)$ & 26\\
$(105,41)$ & 10 & YES & YES & YES & $1.00$ & $(3,1)$ & 27\\
$(105,44)$ & 10 & YES & YES & YES & $1.11$ & $(3,1)$ & 28\\
$(106,31)$ & 10 & YES & YES & YES & $1.00$ & $(1,2)$ & 29\\
$(107,41)$ & 10 & YES & YES & YES & $1.11$ & $(3,1)$ & 30\\
$(109,30)$ & 10 & YES & YES & YES & $0.90$ & $(1,2)$ & 31\\
$(109,40)$ & 10 & YES & YES & YES & $0.89$ & $(3,1)$ & 32\\
$(111,31)$ & 10 & YES & YES & YES & $1.00$ & $(1,2)$ & 33\\
$(112,31)$ & 10 & YES & YES & YES & $0.89$ & $(1,2)$ & 34\\
$(115,26)$ & 11 & YES & YES & YES & $1.18$ & $(1,2)$ & 35\\
$(115,44)$ & 10 & YES & YES & YES & $1.00$ & $(3,1)$ & 36\\
$(117,34)$ & 11 & YES & YES & YES & $1.00$ & $(3,1)$ & 37\\
$(117,43)$ & 10 & YES & YES & YES & $1.00$ & $(3,1)$ & 38\\
$(118,27)$ & 11 & YES & YES & YES & $1.18$ & $(1,2)$ & 39\\
$(119,26)$ & 11 & YES & YES & YES & $1.27$ & $(1,2)$ & 40\\
$(121,34)$ & 11 & YES & YES & YES & $1.00$ & $(3,1)$ & 41\\
$(131,30)$ & 11 & YES & YES & YES & $1.18$ & $(1,2)$ & 42\\
$(131,50)$ & 10 & YES & YES & YES & $1.00$ & $(3,1)$ & 43\\
$(133,39)$ & 11 & YES & YES & YES & $0.89$ & $(3,1)$ & 44\\
$(134,29)$ & 11 & YES & YES & YES & $0.89$ & $(1,2)$ & 45\\
$(134,39)$ & 11 & YES & YES & YES & $1.00$ & $(3,1)$ & 46\\
$(141,43)$ & 11 & YES & YES & YES & $0.89$ & $(3,1)$ & 47\\
$(149,34)$ & 11 & YES & YES & YES & $1.00$ & $(1,2)$ & 48\\
$(b;1,1,1;39)$ & 8 & YES & YES & YES & $1.00$ & $(1,2)$ & 49\\
$(g;0,1,1;33)$ & 8 & YES & YES & YES & $1.18$ & $(1,2)$ & 50\\
$(g;0,2,0;29)$ & 8 & YES & YES & YES & $1.00$ & $(1,2)$ & 51\\
$(g;1,0,1;38)$ & 8 & YES & YES & YES & $0.78$ & $(1,2)$ & 52
\end{longtable}
\subsection{1 chain, $K^2 = 3$}
\begin{longtable}{|c|c|c|c|c|c|c|c|}
\hline
\multicolumn{8}{|c|}{1 chain, $K^2 = 3$}\\
\hline
$(n,a)$ & Len & Nef & $\mathbb Q$-ef & Obs 0 & $\overline c_1^2 / \overline c_2$ & $(P,K)$ & Index\\
\hline
\endfirsthead

\hline
$(n,a)$ & Len & Nef & $\mathbb Q$-ef & Obs 0 & $\overline c_1^2 / \overline c_2$ & $(P,K)$ & Index\\
\hline
\endhead
\hline
\endfoot

$(128,49)$ & 10 & YES & YES & NO(2) & $1.36$ & $(1,3)$ & 53\\
$(144,55)$ & 10 & YES & YES & YES & $1.33$ & $(1,3)$ & 54\\
$(157,46)$ & 11 & YES & YES & NO(2) & $1.40$ & $(3,2)$ & 55\\
$(159,59)$ & 11 & YES & YES & YES & $1.25$ & $(3,2)$ & 56\\
$(163,62)$ & 11 & YES & YES & NO(2) & $1.30$ & $(3,2)$ & 57\\
$(163,63)$ & 11 & YES & YES & NO(2) & $1.30$ & $(3,2)$ & 58\\
$(163,44)$ & 11 & YES & YES & NO(2) & $1.40$ & $(3,2)$ & 59\\
$(165,61)$ & 11 & YES & YES & YES & $1.33$ & $(1,3)$ & 60\\
$(167,69)$ & 11 & YES & YES & NO(2) & $1.36$ & $(1,3)$ & 61\\
$(167,51)$ & 12 & YES & YES & YES & $1.56$ & $(1,3)$ & 62\\
$(173,76)$ & 11 & YES & YES & NO(2) & $1.45$ & $(1,3)$ & 63\\
$(173,51)$ & 12 & YES & YES & NO(2) & $1.50$ & $(7,0)$ & 64\\
$(175,67)$ & 11 & YES & YES & NO(2) & $1.33$ & $(5,1)$ & 65\\
$(176,65)$ & 11 & YES & YES & NO(2) & $1.40$ & $(3,2)$ & 66\\
$(179,68)$ & 11 & YES & YES & NO(2) & $1.22$ & $(5,1)$ & 67\\
$(179,74)$ & 11 & YES & YES & NO(2) & $1.22$ & $(5,1)$ & 68\\
$(181,53)$ & 12 & YES & YES & YES & $1.33$ & $(1,3)$ & 69\\
$(181,76)$ & 11 & YES & YES & YES & $1.44$ & $(1,3)$ & 70\\
$(181,79)$ & 12 & YES & YES & YES & $1.25$ & $(3,2)$ & 71\\
$(181,75)$ & 11 & YES & YES & NO(2) & $1.40$ & $(3,2)$ & 72\\
$(183,67)$ & 11 & YES & YES & YES & $1.22$ & $(1,3)$ & 73\\
$(183,71)$ & 11 & YES & YES & NO(2) & $1.50$ & $(7,0)$ & 74\\
$(185,68)$ & 11 & YES & YES & NO(2) & $1.33$ & $(5,1)$ & 75\\
$(186,71)$ & 11 & YES & YES & NO(2) & $1.64$ & $(1,3)$ & 76\\
$(187,71)$ & 11 & YES & YES & NO(2) & $1.50$ & $(7,0)$ & 77\\
$(187,84)$ & 12 & YES & YES & NO(2) & $1.50$ & $(3,2)$ & 78\\
$(188,79)$ & 11 & YES & YES & YES & $1.22$ & $(1,3)$ & 79\\
$(188,69)$ & 11 & YES & YES & NO(2) & $1.33$ & $(5,1)$ & 80\\
$(191,58)$ & 12 & YES & YES & NO(2) & $1.45$ & $(1,3)$ & 81\\
$(191,80)$ & 11 & YES & YES & NO(2) & $1.64$ & $(1,3)$ & 82\\
$(192,73)$ & 11 & YES & YES & NO(2) & $1.22$ & $(5,1)$ & 83\\
$(192,71)$ & 11 & YES & YES & NO(2) & $1.33$ & $(5,1)$ & 84\\
$(193,81)$ & 11 & YES & YES & NO(2) & $1.36$ & $(1,3)$ & 85\\
$(193,87)$ & 12 & YES & YES & NO(2) & $1.33$ & $(5,1)$ & 86\\
$(194,75)$ & 11 & YES & YES & NO(2) & $1.20$ & $(3,2)$ & 87\\
$(195,88)$ & 12 & YES & YES & NO(2) & $1.60$ & $(3,2)$ & 88\\
$(196,81)$ & 11 & YES & YES & YES & $1.38$ & $(1,3)$ & 89\\
$(197,71)$ & 12 & YES & YES & NO(2) & $1.33$ & $(5,1)$ & 90\\
$(197,61)$ & 13 & YES & YES & NO(2) & $1.33$ & $(5,1)$ & 91\\
$(205,92)$ & 12 & YES & YES & NO(2) & $1.40$ & $(3,2)$ & 92\\
$(206,73)$ & 12 & YES & YES & NO(2) & $1.60$ & $(3,2)$ & 93\\
$(207,61)$ & 13 & YES & YES & NO(2) & $1.40$ & $(3,2)$ & 94\\
$(207,76)$ & 11 & YES & YES & NO(2) & $1.33$ & $(9,-1)$ & 95\\
$(207,79)$ & 11 & YES & YES & YES & $1.22$ & $(1,3)$ & 96\\
$(207,80)$ & 12 & YES & YES & NO(2) & $1.60$ & $(3,2)$ & 97\\
$(207,91)$ & 12 & YES & YES & NO(2) & $1.60$ & $(3,2)$ & 98\\
$(208,79)$ & 11 & YES & YES & NO(2) & $1.22$ & $(5,1)$ & 99\\
$(209,80)$ & 11 & YES & YES & YES & $1.25$ & $(3,2)$ & 100\\
$(210,89)$ & 12 & YES & YES & NO(2) & $1.50$ & $(3,2)$ & 101\\
$(212,59)$ & 13 & YES & YES & NO(2) & $1.50$ & $(3,2)$ & 102\\
$(212,63)$ & 13 & YES & YES & NO(2) & $1.33$ & $(5,1)$ & 103\\
$(212,81)$ & 11 & YES & YES & NO(2) & $1.45$ & $(1,3)$ & 104\\
$(215,64)$ & 12 & YES & YES & YES & $1.33$ & $(1,3)$ & 105\\
$(215,82)$ & 12 & YES & YES & NO(2) & $1.64$ & $(1,3)$ & 106\\
$(215,83)$ & 12 & YES & YES & NO(2) & $1.33$ & $(5,1)$ & 107\\
$(219,67)$ & 12 & YES & YES & NO(2) & $1.30$ & $(3,2)$ & 108\\
$(219,79)$ & 12 & YES & YES & YES & $1.33$ & $(1,3)$ & 109\\
$(219,85)$ & 12 & YES & YES & NO(2) & $1.33$ & $(5,1)$ & 110\\
$(219,95)$ & 12 & YES & YES & YES & $1.56$ & $(1,3)$ & 111\\
$(222,91)$ & 12 & YES & YES & NO(2) & $1.50$ & $(3,2)$ & 112\\
$(223,66)$ & 12 & YES & YES & YES & $1.33$ & $(1,3)$ & 113\\
$(223,98)$ & 12 & YES & YES & NO(2) & $1.60$ & $(3,2)$ & 114\\
$(225,98)$ & 12 & YES & YES & NO(2) & $1.50$ & $(3,2)$ & 115\\
$(226,61)$ & 12 & YES & YES & YES & $1.44$ & $(1,3)$ & 116\\
$(227,66)$ & 12 & YES & YES & YES & $1.33$ & $(1,3)$ & 117\\
$(227,94)$ & 12 & YES & YES & NO(2) & $1.33$ & $(3,2)$ & 118\\
$(227,99)$ & 12 & YES & YES & NO(2) & $1.40$ & $(3,2)$ & 119\\
$(227,100)$ & 12 & YES & YES & YES & $1.25$ & $(3,2)$ & 120\\
$(227,84)$ & 12 & YES & YES & NO(2) & $1.12$ & $(7,0)$ & 121\\
$(229,63)$ & 13 & YES & YES & NO(2) & $1.44$ & $(9,-1)$ & 122\\
$(229,68)$ & 12 & YES & YES & NO(2) & $1.40$ & $(3,2)$ & 123\\
$(229,71)$ & 13 & YES & YES & NO(2) & $1.40$ & $(3,2)$ & 124\\
$(229,94)$ & 12 & YES & YES & NO(2) & $1.40$ & $(3,2)$ & 125\\
$(229,97)$ & 12 & YES & YES & NO(2) & $1.55$ & $(5,1)$ & 126\\
$(229,85)$ & 12 & YES & YES & NO(2) & $1.44$ & $(5,1)$ & 127\\
$(230,97)$ & 12 & YES & YES & YES & $1.62$ & $(1,3)$ & 128\\
$(231,64)$ & 12 & YES & YES & YES & $1.33$ & $(1,3)$ & 129\\
$(231,83)$ & 12 & YES & YES & YES & $1.56$ & $(1,3)$ & 130\\
$(232,95)$ & 13 & YES & YES & NO(2) & $1.44$ & $(9,-1)$ & 131\\
$(233,64)$ & 12 & YES & YES & YES & $1.33$ & $(1,3)$ & 132\\
$(233,84)$ & 12 & YES & YES & NO(2) & $1.40$ & $(3,2)$ & 133\\
$(233,89)$ & 11 & YES & YES & YES & $1.12$ & $(5,1)$ & 134\\
$(233,86)$ & 12 & YES & YES & NO(2) & $1.12$ & $(7,0)$ & 135\\
$(234,89)$ & 12 & YES & YES & NO(2) & $1.12$ & $(7,0)$ & 136\\
$(235,66)$ & 12 & YES & YES & YES & $1.33$ & $(1,3)$ & 137\\
$(235,72)$ & 13 & YES & YES & NO(2) & $1.40$ & $(7,0)$ & 138\\
$(236,65)$ & 12 & YES & YES & YES & $1.33$ & $(1,3)$ & 139\\
$(236,69)$ & 12 & YES & YES & YES & $1.33$ & $(1,3)$ & 140\\
$(237,85)$ & 12 & YES & YES & NO(2) & $1.22$ & $(9,-1)$ & 141\\
$(237,92)$ & 12 & YES & YES & NO(2) & $1.40$ & $(7,0)$ & 142\\
$(237,100)$ & 12 & YES & YES & NO(2) & $1.33$ & $(3,2)$ & 143\\
$(238,93)$ & 12 & YES & YES & NO(2) & $1.44$ & $(5,1)$ & 144\\
$(239,99)$ & 12 & YES & YES & NO(2) & $1.40$ & $(3,2)$ & 145\\
$(239,100)$ & 12 & YES & YES & YES & $1.56$ & $(1,3)$ & 146\\
$(242,67)$ & 12 & YES & YES & YES & $1.33$ & $(1,3)$ & 147\\
$(243,92)$ & 12 & YES & YES & NO(2) & $1.44$ & $(5,1)$ & 148\\
$(243,94)$ & 12 & YES & YES & NO(2) & $1.33$ & $(5,1)$ & 149\\
$(245,69)$ & 13 & YES & YES & NO(2) & $1.40$ & $(3,2)$ & 150\\
$(245,88)$ & 12 & YES & YES & NO(2) & $1.60$ & $(3,2)$ & 151\\
$(245,93)$ & 12 & YES & YES & NO(2) & $1.50$ & $(3,2)$ & 152\\
$(245,104)$ & 13 & YES & YES & NO(2) & $1.60$ & $(7,0)$ & 153\\
$(246,91)$ & 12 & YES & YES & NO(2) & $1.60$ & $(3,2)$ & 154\\
$(247,68)$ & 12 & YES & YES & YES & $1.44$ & $(1,3)$ & 155\\
$(248,91)$ & 12 & YES & YES & YES & $1.62$ & $(1,3)$ & 156\\
$(248,109)$ & 12 & YES & YES & NO(2) & $1.60$ & $(3,2)$ & 157\\
$(250,73)$ & 13 & YES & YES & NO(2) & $1.33$ & $(5,1)$ & 158\\
$(252,71)$ & 13 & YES & YES & NO(2) & $1.33$ & $(5,1)$ & 159\\
$(252,73)$ & 13 & YES & YES & NO(2) & $1.50$ & $(3,2)$ & 160\\
$(253,105)$ & 13 & YES & YES & NO(2) & $1.44$ & $(5,1)$ & 161\\
$(254,71)$ & 12 & YES & YES & NO(2) & $1.45$ & $(1,3)$ & 162\\
$(254,93)$ & 12 & YES & YES & YES & $1.56$ & $(1,3)$ & 163\\
$(254,105)$ & 12 & YES & YES & YES & $1.38$ & $(1,3)$ & 164\\
$(255,71)$ & 13 & YES & YES & NO(2) & $1.55$ & $(1,3)$ & 165\\
$(255,76)$ & 13 & YES & YES & NO(2) & $1.45$ & $(1,3)$ & 166\\
$(257,76)$ & 12 & YES & YES & YES & $1.25$ & $(5,1)$ & 167\\
$(257,106)$ & 13 & YES & YES & NO(2) & $1.40$ & $(7,0)$ & 168\\
$(259,100)$ & 12 & YES & YES & NO(2) & $1.40$ & $(7,0)$ & 169\\
$(261,80)$ & 14 & YES & YES & NO(2) & $1.40$ & $(7,0)$ & 170\\
$(261,100)$ & 12 & YES & YES & YES & $1.25$ & $(1,3)$ & 171\\
$(263,78)$ & 13 & YES & YES & YES & $1.33$ & $(1,3)$ & 172\\
$(263,109)$ & 12 & YES & YES & YES & $1.38$ & $(3,2)$ & 173\\
$(264,109)$ & 12 & YES & YES & YES & $1.25$ & $(3,2)$ & 174\\
$(265,77)$ & 13 & YES & YES & NO(2) & $1.50$ & $(3,2)$ & 175\\
$(265,97)$ & 12 & YES & YES & YES & $1.38$ & $(1,3)$ & 176\\
$(265,98)$ & 12 & YES & YES & YES & $1.38$ & $(3,2)$ & 177\\
$(266,101)$ & 12 & YES & YES & YES & $1.56$ & $(1,3)$ & 178\\
$(267,98)$ & 12 & YES & YES & NO(2) & $1.33$ & $(3,2)$ & 179\\
$(268,61)$ & 14 & YES & YES & NO(2) & $1.22$ & $(5,1)$ & 180\\
$(269,72)$ & 13 & YES & YES & NO(2) & $1.40$ & $(7,0)$ & 181\\
$(270,103)$ & 12 & YES & YES & YES & $1.38$ & $(3,2)$ & 182\\
$(271,59)$ & 14 & YES & YES & NO(2) & $1.33$ & $(5,1)$ & 183\\
$(271,83)$ & 13 & YES & YES & NO(2) & $1.36$ & $(5,1)$ & 184\\
$(271,105)$ & 12 & YES & YES & YES & $1.56$ & $(1,3)$ & 185\\
$(271,112)$ & 12 & YES & YES & YES & $1.25$ & $(3,2)$ & 186\\
$(271,111)$ & 13 & YES & YES & NO(2) & $1.55$ & $(5,1)$ & 187\\
$(272,103)$ & 12 & YES & YES & NO(2) & $1.40$ & $(3,2)$ & 188\\
$(273,76)$ & 13 & YES & YES & NO(2) & $1.55$ & $(1,3)$ & 189\\
$(273,100)$ & 12 & YES & YES & NO(2) & $1.22$ & $(5,1)$ & 190\\
$(274,115)$ & 12 & YES & YES & YES & $1.38$ & $(3,2)$ & 191\\
$(274,121)$ & 13 & YES & YES & NO(2) & $1.50$ & $(7,0)$ & 192\\
$(277,75)$ & 14 & YES & YES & NO(2) & $1.50$ & $(7,0)$ & 193\\
$(277,76)$ & 13 & YES & YES & NO(2) & $1.22$ & $(9,-1)$ & 194\\
$(277,106)$ & 12 & YES & YES & NO(2) & $1.22$ & $(3,2)$ & 195\\
$(277,117)$ & 12 & YES & YES & NO(2) & $1.45$ & $(5,1)$ & 196\\
$(281,109)$ & 12 & YES & YES & YES & $1.25$ & $(1,3)$ & 197\\
$(282,109)$ & 12 & YES & YES & NO(2) & $1.50$ & $(1,3)$ & 198\\
$(283,78)$ & 13 & YES & YES & NO(2) & $1.33$ & $(9,-1)$ & 199\\
$(283,83)$ & 13 & YES & YES & NO(2) & $1.33$ & $(3,2)$ & 200\\
$(283,86)$ & 13 & YES & YES & YES & $1.56$ & $(1,3)$ & 201\\
$(283,104)$ & 12 & YES & YES & NO(2) & $1.45$ & $(5,1)$ & 202\\
$(284,119)$ & 12 & YES & YES & YES & $1.44$ & $(3,2)$ & 203\\
$(285,53)$ & 15 & YES & YES & NO(2) & $1.50$ & $(3,2)$ & 204\\
$(286,105)$ & 12 & YES & YES & YES & $1.38$ & $(5,1)$ & 205\\
$(287,106)$ & 12 & YES & YES & YES & $1.25$ & $(3,2)$ & 206\\
$(287,111)$ & 12 & YES & YES & NO(2) & $1.33$ & $(5,1)$ & 207\\
$(288,121)$ & 12 & YES & YES & YES & $1.38$ & $(3,2)$ & 208\\
$(290,81)$ & 12 & YES & YES & YES & $1.25$ & $(5,1)$ & 209\\
$(292,121)$ & 12 & YES & YES & YES & $1.44$ & $(3,2)$ & 210\\
$(294,109)$ & 13 & YES & YES & NO(2) & $1.33$ & $(5,1)$ & 211\\
$(297,68)$ & 13 & YES & YES & YES & $1.33$ & $(1,3)$ & 212\\
$(297,109)$ & 12 & YES & YES & NO(2) & $1.40$ & $(3,2)$ & 213\\
$(298,53)$ & 15 & YES & YES & NO(2) & $1.60$ & $(3,2)$ & 214\\
$(298,79)$ & 13 & YES & YES & YES & $1.38$ & $(1,3)$ & 215\\
$(299,80)$ & 14 & YES & YES & NO(2) & $1.40$ & $(7,0)$ & 216\\
$(299,116)$ & 12 & YES & YES & YES & $1.56$ & $(3,2)$ & 217\\
$(301,88)$ & 13 & YES & YES & YES & $1.56$ & $(5,1)$ & 218\\
$(301,89)$ & 12 & YES & YES & YES & $1.25$ & $(5,1)$ & 219\\
$(301,115)$ & 12 & YES & YES & YES & $1.12$ & $(5,1)$ & 220\\
$(302,111)$ & 12 & YES & YES & YES & $1.44$ & $(3,2)$ & 221\\
$(302,117)$ & 12 & YES & YES & YES & $1.80$ & $(1,3)$ & 222\\
$(303,116)$ & 12 & YES & YES & YES & $1.25$ & $(3,2)$ & 223\\
$(305,112)$ & 12 & YES & YES & YES & $1.56$ & $(1,3)$ & 224\\
$(305,128)$ & 12 & YES & YES & YES & $1.56$ & $(3,2)$ & 225\\
$(307,57)$ & 14 & YES & YES & NO(2) & $1.22$ & $(5,1)$ & 226\\
$(307,65)$ & 13 & YES & YES & NO(2) & $1.45$ & $(1,3)$ & 227\\
$(307,129)$ & 12 & YES & YES & YES & $1.50$ & $(5,1)$ & 228\\
$(307,119)$ & 12 & YES & YES & YES & $1.56$ & $(3,2)$ & 229\\
$(308,129)$ & 12 & YES & YES & YES & $1.56$ & $(5,1)$ & 230\\
$(310,83)$ & 13 & YES & YES & NO(2) & $1.30$ & $(7,0)$ & 231\\
$(311,67)$ & 14 & YES & YES & NO(2) & $1.50$ & $(7,0)$ & 232\\
$(311,84)$ & 13 & YES & YES & NO(2) & $1.40$ & $(7,0)$ & 233\\
$(311,119)$ & 12 & YES & YES & YES & $1.73$ & $(1,3)$ & 234\\
$(312,119)$ & 13 & YES & YES & YES & $1.38$ & $(3,2)$ & 235\\
$(312,131)$ & 12 & YES & YES & YES & $1.44$ & $(3,2)$ & 236\\
$(313,119)$ & 12 & YES & YES & YES & $1.44$ & $(3,2)$ & 237\\
$(313,121)$ & 12 & YES & YES & YES & $1.44$ & $(3,2)$ & 238\\
$(315,88)$ & 13 & YES & YES & YES & $1.38$ & $(5,1)$ & 239\\
$(316,69)$ & 13 & YES & YES & YES & $1.22$ & $(1,3)$ & 240\\
$(317,84)$ & 13 & YES & YES & NO(2) & $1.40$ & $(3,2)$ & 241\\
$(317,85)$ & 13 & YES & YES & NO(2) & $1.40$ & $(7,0)$ & 242\\
$(317,96)$ & 14 & YES & YES & NO(2) & $1.22$ & $(9,-1)$ & 243\\
$(317,121)$ & 12 & YES & YES & YES & $1.56$ & $(3,2)$ & 244\\
$(317,131)$ & 12 & YES & YES & YES & $1.50$ & $(5,1)$ & 245\\
$(320,93)$ & 13 & YES & YES & YES & $1.50$ & $(3,2)$ & 246\\
$(321,95)$ & 13 & YES & YES & YES & $1.38$ & $(5,1)$ & 247\\
$(322,73)$ & 14 & YES & YES & YES & $1.44$ & $(1,3)$ & 248\\
$(322,89)$ & 12 & YES & YES & YES & $1.12$ & $(5,1)$ & 249\\
$(322,123)$ & 12 & YES & YES & YES & $1.38$ & $(3,2)$ & 250\\
$(323,89)$ & 13 & YES & YES & YES & $1.38$ & $(3,2)$ & 251\\
$(323,98)$ & 13 & YES & YES & YES & $1.56$ & $(5,1)$ & 252\\
$(327,97)$ & 13 & YES & YES & YES & $1.25$ & $(5,1)$ & 253\\
$(327,100)$ & 13 & YES & YES & NO(2) & $1.36$ & $(5,1)$ & 254\\
$(330,89)$ & 13 & YES & YES & YES & $1.56$ & $(3,2)$ & 255\\
$(333,92)$ & 13 & YES & YES & YES & $1.25$ & $(3,2)$ & 256\\
$(335,73)$ & 14 & YES & YES & NO(2) & $1.22$ & $(3,2)$ & 257\\
$(335,92)$ & 13 & YES & YES & YES & $1.56$ & $(1,3)$ & 258\\
$(335,98)$ & 13 & YES & YES & YES & $1.56$ & $(5,1)$ & 259\\
$(335,123)$ & 12 & YES & YES & YES & $1.38$ & $(5,1)$ & 260\\
$(335,128)$ & 12 & YES & YES & YES & $1.56$ & $(3,2)$ & 261\\
$(337,129)$ & 12 & YES & YES & YES & $1.62$ & $(3,2)$ & 262\\
$(337,141)$ & 13 & YES & YES & YES & $1.67$ & $(3,2)$ & 263\\
$(338,77)$ & 14 & YES & YES & YES & $1.38$ & $(1,3)$ & 264\\
$(338,129)$ & 12 & YES & YES & YES & $1.62$ & $(3,2)$ & 265\\
$(340,101)$ & 13 & YES & YES & YES & $1.44$ & $(3,2)$ & 266\\
$(341,95)$ & 13 & YES & YES & YES & $1.56$ & $(3,2)$ & 267\\
$(341,100)$ & 13 & YES & YES & YES & $1.38$ & $(3,2)$ & 268\\
$(341,140)$ & 13 & YES & YES & YES & $1.25$ & $(5,1)$ & 269\\
$(343,131)$ & 12 & YES & YES & YES & $1.44$ & $(3,2)$ & 270\\
$(344,95)$ & 13 & YES & YES & YES & $1.25$ & $(3,2)$ & 271\\
$(344,105)$ & 13 & YES & YES & YES & $1.56$ & $(3,2)$ & 272\\
$(347,92)$ & 13 & YES & YES & YES & $1.38$ & $(1,3)$ & 273\\
$(347,97)$ & 13 & YES & YES & YES & $1.83$ & $(1,3)$ & 274\\
$(347,101)$ & 13 & YES & YES & YES & $1.25$ & $(3,2)$ & 275\\
$(347,144)$ & 13 & YES & YES & YES & $1.62$ & $(5,1)$ & 276\\
$(347,134)$ & 13 & YES & YES & YES & $1.82$ & $(1,3)$ & 277\\
$(349,106)$ & 13 & YES & YES & YES & $1.56$ & $(3,2)$ & 278\\
$(349,135)$ & 13 & YES & YES & YES & $1.56$ & $(1,3)$ & 279\\
$(349,136)$ & 13 & YES & YES & YES & $1.25$ & $(5,1)$ & 280\\
$(349,144)$ & 13 & YES & YES & YES & $1.56$ & $(3,2)$ & 281\\
$(351,98)$ & 13 & YES & YES & YES & $1.56$ & $(5,1)$ & 282\\
$(353,75)$ & 14 & YES & YES & YES & $1.25$ & $(1,3)$ & 283\\
$(355,96)$ & 14 & YES & YES & NO(2) & $1.40$ & $(7,0)$ & 284\\
$(355,104)$ & 13 & YES & YES & YES & $1.73$ & $(1,3)$ & 285\\
$(358,151)$ & 13 & YES & YES & YES & $1.67$ & $(3,2)$ & 286\\
$(359,100)$ & 13 & YES & YES & YES & $1.38$ & $(3,2)$ & 287\\
$(359,105)$ & 13 & YES & YES & YES & $1.70$ & $(1,3)$ & 288\\
$(359,106)$ & 13 & YES & YES & YES & $1.33$ & $(1,3)$ & 289\\
$(360,101)$ & 13 & YES & YES & YES & $1.25$ & $(3,2)$ & 290\\
$(363,98)$ & 13 & YES & YES & YES & $1.33$ & $(1,3)$ & 291\\
$(364,111)$ & 13 & YES & YES & YES & $1.60$ & $(3,2)$ & 292\\
$(365,98)$ & 13 & YES & YES & YES & $1.25$ & $(5,1)$ & 293\\
$(365,108)$ & 13 & YES & YES & YES & $1.56$ & $(1,3)$ & 294\\
$(367,83)$ & 14 & YES & YES & YES & $1.25$ & $(3,2)$ & 295\\
$(367,84)$ & 14 & YES & YES & YES & $1.44$ & $(3,2)$ & 296\\
$(367,101)$ & 13 & YES & YES & YES & $1.38$ & $(3,2)$ & 297\\
$(367,112)$ & 13 & YES & YES & YES & $1.38$ & $(3,2)$ & 298\\
$(371,108)$ & 14 & YES & YES & YES & $1.38$ & $(5,1)$ & 299\\
$(373,104)$ & 13 & YES & YES & YES & $1.60$ & $(1,3)$ & 300\\
$(374,111)$ & 13 & YES & YES & YES & $1.33$ & $(1,3)$ & 301\\
$(376,105)$ & 13 & YES & YES & YES & $1.60$ & $(1,3)$ & 302\\
$(377,79)$ & 14 & YES & YES & NO(2) & $1.33$ & $(5,1)$ & 303\\
$(377,112)$ & 13 & YES & YES & YES & $1.44$ & $(3,2)$ & 304\\
$(377,136)$ & 13 & YES & YES & YES & $1.56$ & $(3,2)$ & 305\\
$(377,144)$ & 12 & YES & YES & YES & $1.25$ & $(1,3)$ & 306\\
$(377,159)$ & 13 & YES & YES & YES & $1.38$ & $(5,1)$ & 307\\
$(379,111)$ & 13 & YES & YES & YES & $1.56$ & $(1,3)$ & 308\\
$(379,115)$ & 13 & YES & YES & YES & $1.38$ & $(5,1)$ & 309\\
$(380,83)$ & 14 & YES & YES & YES & $1.12$ & $(3,2)$ & 310\\
$(380,137)$ & 13 & YES & YES & YES & $1.67$ & $(3,2)$ & 311\\
$(380,157)$ & 13 & YES & YES & YES & $1.67$ & $(3,2)$ & 312\\
$(382,89)$ & 14 & YES & YES & YES & $1.38$ & $(3,2)$ & 313\\
$(382,103)$ & 14 & YES & YES & NO(2) & $1.33$ & $(5,1)$ & 314\\
$(382,141)$ & 13 & YES & YES & YES & $1.44$ & $(3,2)$ & 315\\
$(383,84)$ & 14 & YES & YES & NO(2) & $1.40$ & $(7,0)$ & 316\\
$(383,106)$ & 13 & YES & YES & YES & $1.33$ & $(1,3)$ & 317\\
$(383,112)$ & 13 & YES & YES & YES & $1.70$ & $(1,3)$ & 318\\
$(385,108)$ & 14 & YES & YES & YES & $1.38$ & $(5,1)$ & 319\\
$(388,161)$ & 13 & YES & YES & YES & $1.56$ & $(3,2)$ & 320\\
$(389,84)$ & 14 & YES & YES & YES & $1.56$ & $(3,2)$ & 321\\
$(389,88)$ & 14 & YES & YES & YES & $1.44$ & $(3,2)$ & 322\\
$(391,105)$ & 13 & YES & YES & YES & $1.25$ & $(5,1)$ & 323\\
$(391,108)$ & 13 & YES & YES & YES & $1.56$ & $(1,3)$ & 324\\
$(391,109)$ & 13 & YES & YES & YES & $1.56$ & $(3,2)$ & 325\\
$(393,106)$ & 13 & YES & YES & YES & $1.56$ & $(3,2)$ & 326\\
$(393,116)$ & 13 & YES & YES & YES & $1.56$ & $(1,3)$ & 327\\
$(393,142)$ & 14 & YES & YES & YES & $1.38$ & $(5,1)$ & 328\\
$(395,73)$ & 15 & YES & YES & NO(2) & $1.22$ & $(5,1)$ & 329\\
$(397,111)$ & 14 & YES & YES & YES & $1.70$ & $(3,2)$ & 330\\
$(397,116)$ & 13 & YES & YES & YES & $1.56$ & $(1,3)$ & 331\\
$(398,111)$ & 13 & YES & YES & YES & $1.60$ & $(1,3)$ & 332\\
$(399,86)$ & 14 & YES & YES & NO(2) & $1.40$ & $(7,0)$ & 333\\
$(401,92)$ & 14 & YES & YES & NO(2) & $1.40$ & $(7,0)$ & 334\\
$(401,111)$ & 13 & YES & YES & YES & $1.60$ & $(1,3)$ & 335\\
$(401,112)$ & 13 & YES & YES & YES & $1.38$ & $(3,2)$ & 336\\
$(401,119)$ & 13 & YES & YES & YES & $1.44$ & $(3,2)$ & 337\\
$(403,88)$ & 14 & YES & YES & YES & $1.56$ & $(3,2)$ & 338\\
$(403,92)$ & 14 & YES & YES & YES & $1.56$ & $(1,3)$ & 339\\
$(403,109)$ & 14 & YES & YES & NO(2) & $1.45$ & $(5,1)$ & 340\\
$(403,111)$ & 13 & YES & YES & YES & $1.73$ & $(1,3)$ & 341\\
$(403,113)$ & 14 & YES & YES & YES & $1.70$ & $(3,2)$ & 342\\
$(403,119)$ & 13 & YES & YES & YES & $1.33$ & $(3,2)$ & 343\\
$(407,112)$ & 13 & YES & YES & YES & $1.56$ & $(3,2)$ & 344\\
$(407,119)$ & 13 & YES & YES & YES & $1.56$ & $(1,3)$ & 345\\
$(407,149)$ & 13 & YES & YES & YES & $1.70$ & $(3,2)$ & 346\\
$(407,171)$ & 13 & YES & YES & YES & $1.62$ & $(1,3)$ & 347\\
$(409,121)$ & 13 & YES & YES & YES & $1.44$ & $(1,3)$ & 348\\
$(413,121)$ & 13 & YES & YES & YES & $1.44$ & $(1,3)$ & 349\\
$(413,157)$ & 13 & YES & YES & YES & $1.56$ & $(3,2)$ & 350\\
$(414,97)$ & 14 & YES & YES & YES & $1.38$ & $(1,3)$ & 351\\
$(415,116)$ & 13 & YES & YES & YES & $1.56$ & $(3,2)$ & 352\\
$(416,123)$ & 13 & YES & YES & YES & $1.38$ & $(1,3)$ & 353\\
$(417,65)$ & 16 & YES & YES & NO(2) & $1.40$ & $(3,2)$ & 354\\
$(418,159)$ & 13 & YES & YES & YES & $1.25$ & $(5,1)$ & 355\\
$(419,89)$ & 14 & YES & YES & YES & $1.25$ & $(1,3)$ & 356\\
$(419,116)$ & 13 & YES & YES & YES & $1.44$ & $(3,2)$ & 357\\
$(421,98)$ & 14 & YES & YES & YES & $1.44$ & $(1,3)$ & 358\\
$(422,161)$ & 13 & YES & YES & YES & $1.38$ & $(1,3)$ & 359\\
$(425,92)$ & 14 & YES & YES & YES & $1.60$ & $(3,2)$ & 360\\
$(425,97)$ & 14 & YES & YES & YES & $1.38$ & $(5,1)$ & 361\\
$(426,115)$ & 13 & YES & YES & YES & $1.12$ & $(5,1)$ & 362\\
$(427,92)$ & 14 & YES & YES & NO(2) & $1.40$ & $(7,0)$ & 363\\
$(429,115)$ & 14 & YES & YES & YES & $1.56$ & $(3,2)$ & 364\\
$(430,91)$ & 14 & YES & YES & NO(2) & $1.40$ & $(3,2)$ & 365\\
$(433,119)$ & 14 & YES & YES & YES & $1.56$ & $(3,2)$ & 366\\
$(434,121)$ & 13 & YES & YES & YES & $1.50$ & $(3,2)$ & 367\\
$(436,129)$ & 13 & YES & YES & YES & $1.38$ & $(1,3)$ & 368\\
$(441,67)$ & 16 & YES & YES & NO(2) & $1.33$ & $(5,1)$ & 369\\
$(441,134)$ & 14 & YES & YES & YES & $1.56$ & $(3,2)$ & 370\\
$(445,78)$ & 15 & YES & YES & NO(2) & $1.45$ & $(5,1)$ & 371\\
$(445,123)$ & 13 & YES & YES & YES & $1.38$ & $(1,3)$ & 372\\
$(447,121)$ & 14 & YES & YES & YES & $1.70$ & $(3,2)$ & 373\\
$(448,97)$ & 14 & YES & YES & YES & $1.25$ & $(5,1)$ & 374\\
$(448,171)$ & 13 & YES & YES & YES & $1.50$ & $(1,3)$ & 375\\
$(449,105)$ & 14 & YES & YES & YES & $1.38$ & $(5,1)$ & 376\\
$(457,133)$ & 14 & YES & YES & YES & $1.33$ & $(3,2)$ & 377\\
$(457,134)$ & 14 & YES & YES & YES & $1.56$ & $(3,2)$ & 378\\
$(457,169)$ & 13 & YES & YES & YES & $1.44$ & $(3,2)$ & 379\\
$(459,104)$ & 14 & YES & YES & YES & $1.33$ & $(3,2)$ & 380\\
$(463,129)$ & 14 & YES & YES & YES & $1.25$ & $(5,1)$ & 381\\
$(464,105)$ & 14 & YES & YES & YES & $1.44$ & $(3,2)$ & 382\\
$(465,106)$ & 14 & YES & YES & YES & $1.33$ & $(3,2)$ & 383\\
$(467,102)$ & 15 & YES & YES & YES & $1.70$ & $(3,2)$ & 384\\
$(467,129)$ & 13 & YES & YES & YES & $1.38$ & $(1,3)$ & 385\\
$(469,131)$ & 13 & YES & YES & YES & $1.38$ & $(1,3)$ & 386\\
$(470,111)$ & 15 & YES & YES & NO(2) & $1.33$ & $(5,1)$ & 387\\
$(473,108)$ & 14 & YES & YES & YES & $1.60$ & $(1,3)$ & 388\\
$(474,133)$ & 14 & YES & YES & YES & $1.44$ & $(3,2)$ & 389\\
$(477,103)$ & 15 & YES & YES & YES & $1.67$ & $(3,2)$ & 390\\
$(477,104)$ & 14 & YES & YES & YES & $1.73$ & $(1,3)$ & 391\\
$(484,111)$ & 15 & YES & YES & YES & $1.33$ & $(3,2)$ & 392\\
$(487,111)$ & 14 & YES & YES & YES & $1.60$ & $(1,3)$ & 393\\
$(491,111)$ & 15 & YES & YES & YES & $1.56$ & $(3,2)$ & 394\\
$(497,134)$ & 14 & YES & YES & YES & $1.56$ & $(3,2)$ & 395\\
$(499,108)$ & 14 & YES & YES & YES & $1.44$ & $(3,2)$ & 396\\
$(499,135)$ & 14 & YES & YES & YES & $1.25$ & $(5,1)$ & 397\\
$(501,113)$ & 15 & YES & YES & YES & $1.70$ & $(3,2)$ & 398\\
$(502,117)$ & 15 & YES & YES & YES & $1.56$ & $(3,2)$ & 399\\
$(504,115)$ & 14 & YES & YES & YES & $1.33$ & $(3,2)$ & 400\\
$(519,119)$ & 15 & YES & YES & YES & $1.25$ & $(5,1)$ & 401\\
$(526,123)$ & 14 & YES & YES & YES & $1.25$ & $(5,1)$ & 402\\
$(542,151)$ & 14 & YES & YES & YES & $1.50$ & $(3,2)$ & 403\\
$(553,102)$ & 16 & YES & YES & YES & $1.56$ & $(3,2)$ & 404\\
$(553,103)$ & 16 & YES & YES & YES & $1.70$ & $(3,2)$ & 405\\
$(557,169)$ & 14 & YES & YES & YES & $1.50$ & $(1,3)$ & 406\\
$(561,128)$ & 14 & YES & YES & YES & $1.25$ & $(3,2)$ & 407\\
$(571,154)$ & 14 & YES & YES & YES & $1.60$ & $(3,2)$ & 408\\
$(574,131)$ & 14 & YES & YES & YES & $1.33$ & $(3,2)$ & 409\\
$(579,137)$ & 15 & YES & YES & YES & $1.56$ & $(3,2)$ & 410\\
$(615,131)$ & 15 & YES & YES & YES & $1.70$ & $(3,2)$ & 411\\
$(615,134)$ & 15 & YES & YES & YES & $1.44$ & $(3,2)$ & 412\\
$(a;3,3,3;6)$ & 13 & YES & YES & YES & $1.50$ & $(1,3)$ & 413
\end{longtable}
\subsection{1 chain, $K^2 = 4$}
\begin{longtable}{|c|c|c|c|c|c|c|c|}
\hline
\multicolumn{8}{|c|}{1 chain, $K^2 = 4$}\\
\hline
$(n,a)$ & Len & Nef & $\mathbb Q$-ef & Obs 0 & $\overline c_1^2 / \overline c_2$ & $(P,K)$ & Index\\
\hline
\endfirsthead

\hline
$(n,a)$ & Len & Nef & $\mathbb Q$-ef & Obs 0 & $\overline c_1^2 / \overline c_2$ & $(P,K)$ & Index\\
\hline
\endhead
\hline
\endfoot

$(287,106)$ & 12 & YES & YES & NO(2) & $1.70$ & $(1,4)$ & 414\\
$(317,121)$ & 12 & YES & YES & NO(2) & $1.62$ & $(5,2)$ & 415\\
$(335,123)$ & 12 & YES & YES & NO(2) & $1.80$ & $(5,2)$ & 416\\
$(343,144)$ & 12 & YES & YES & NO(2) & $1.92$ & $(1,4)$ & 417\\
$(348,125)$ & 13 & YES & YES & YES & $1.57$ & $(3,3)$ & 418\\
$(359,140)$ & 13 & YES & YES & YES & $1.75$ & $(1,4)$ & 419\\
$(377,144)$ & 12 & YES & YES & NO(2) & $1.80$ & $(9,0)$ & 420\\
$(417,158)$ & 13 & YES & YES & NO(2) & $1.80$ & $(1,4)$ & 421\\
$(418,159)$ & 13 & YES & YES & NO(2) & $1.75$ & $(5,2)$ & 422\\
$(419,154)$ & 13 & YES & YES & NO(2) & $1.92$ & $(5,2)$ & 423\\
$(426,163)$ & 13 & YES & YES & NO(2) & $1.75$ & $(5,2)$ & 424\\
$(427,186)$ & 13 & YES & YES & NO(2) & $1.75$ & $(9,0)$ & 425\\
$(432,131)$ & 14 & YES & YES & YES & $1.88$ & $(1,4)$ & 426\\
$(433,159)$ & 13 & YES & YES & NO(2) & $2.00$ & $(1,4)$ & 427\\
$(434,165)$ & 13 & YES & YES & NO(2) & $1.75$ & $(5,2)$ & 428\\
$(438,185)$ & 13 & YES & YES & NO(2) & $1.90$ & $(1,4)$ & 429\\
$(448,171)$ & 13 & YES & YES & NO(2) & $1.78$ & $(3,3)$ & 430\\
$(457,192)$ & 13 & YES & YES & NO(2) & $1.67$ & $(3,3)$ & 431\\
$(463,179)$ & 13 & YES & YES & NO(2) & $1.62$ & $(9,0)$ & 432\\
$(463,194)$ & 13 & YES & YES & NO(2) & $1.82$ & $(7,1)$ & 433\\
$(463,201)$ & 14 & YES & YES & NO(2) & $1.80$ & $(1,4)$ & 434\\
$(463,170)$ & 13 & YES & YES & NO(2) & $1.75$ & $(9,0)$ & 435\\
$(466,141)$ & 14 & YES & YES & NO(2) & $1.73$ & $(3,3)$ & 436\\
$(469,179)$ & 13 & YES & YES & YES & $1.75$ & $(1,4)$ & 437\\
$(479,132)$ & 14 & YES & YES & NO(2) & $1.67$ & $(3,3)$ & 438\\
$(482,183)$ & 13 & YES & YES & NO(2) & $1.78$ & $(7,1)$ & 439\\
$(485,178)$ & 13 & YES & YES & NO(2) & $1.70$ & $(1,4)$ & 440\\
$(487,186)$ & 13 & YES & YES & NO(2) & $1.78$ & $(7,1)$ & 441\\
$(502,181)$ & 14 & YES & YES & NO(2) & $1.78$ & $(3,3)$ & 442\\
$(503,208)$ & 14 & YES & YES & NO(2) & $1.75$ & $(9,0)$ & 443\\
$(507,196)$ & 13 & YES & YES & NO(2) & $1.62$ & $(3,3)$ & 444\\
$(509,141)$ & 14 & YES & YES & NO(2) & $1.80$ & $(1,4)$ & 445\\
$(512,189)$ & 14 & YES & YES & NO(2) & $1.82$ & $(3,3)$ & 446\\
$(513,196)$ & 13 & YES & YES & NO(2) & $1.62$ & $(3,3)$ & 447\\
$(519,199)$ & 14 & YES & YES & NO(2) & $1.91$ & $(3,3)$ & 448\\
$(520,187)$ & 14 & YES & YES & NO(2) & $1.91$ & $(3,3)$ & 449\\
$(523,188)$ & 14 & YES & YES & NO(2) & $1.78$ & $(3,3)$ & 450\\
$(523,204)$ & 14 & YES & YES & NO(2) & $1.90$ & $(9,0)$ & 451\\
$(527,190)$ & 14 & YES & YES & NO(2) & $1.80$ & $(5,2)$ & 452\\
$(533,219)$ & 14 & YES & YES & NO(2) & $1.90$ & $(5,2)$ & 453\\
$(537,157)$ & 14 & YES & YES & NO(2) & $1.80$ & $(1,4)$ & 454\\
$(539,200)$ & 14 & YES & YES & NO(2) & $1.83$ & $(5,2)$ & 455\\
$(542,199)$ & 13 & YES & YES & YES & $1.57$ & $(5,2)$ & 456\\
$(542,201)$ & 14 & YES & YES & NO(2) & $1.78$ & $(3,3)$ & 457\\
$(545,207)$ & 13 & YES & YES & YES & $1.78$ & $(1,4)$ & 458\\
$(545,208)$ & 13 & YES & YES & NO(2) & $1.80$ & $(1,4)$ & 459\\
$(548,197)$ & 14 & YES & YES & NO(2) & $1.80$ & $(5,2)$ & 460\\
$(552,229)$ & 14 & YES & YES & NO(2) & $1.78$ & $(3,3)$ & 461\\
$(555,212)$ & 13 & YES & YES & YES & $1.75$ & $(1,4)$ & 462\\
$(556,165)$ & 14 & YES & YES & NO(2) & $1.78$ & $(1,4)$ & 463\\
$(562,157)$ & 14 & YES & YES & NO(2) & $1.80$ & $(1,4)$ & 464\\
$(562,213)$ & 14 & YES & YES & NO(2) & $1.75$ & $(5,2)$ & 465\\
$(563,158)$ & 15 & YES & YES & YES & $1.88$ & $(1,4)$ & 466\\
$(569,240)$ & 14 & YES & YES & NO(2) & $1.88$ & $(3,3)$ & 467\\
$(573,242)$ & 14 & YES & YES & NO(2) & $1.67$ & $(11,-1)$ & 468\\
$(578,169)$ & 14 & YES & YES & YES & $1.88$ & $(1,4)$ & 469\\
$(579,221)$ & 14 & YES & YES & NO(2) & $1.75$ & $(3,3)$ & 470\\
$(587,172)$ & 14 & YES & YES & NO(2) & $1.78$ & $(3,3)$ & 471\\
$(587,215)$ & 14 & YES & YES & NO(2) & $1.78$ & $(3,3)$ & 472\\
$(594,227)$ & 14 & YES & YES & NO(2) & $2.00$ & $(5,2)$ & 473\\
$(597,250)$ & 14 & YES & YES & NO(2) & $1.78$ & $(7,1)$ & 474\\
$(604,175)$ & 16 & YES & YES & NO(2) & $2.00$ & $(3,3)$ & 475\\
$(605,179)$ & 14 & YES & YES & NO(2) & $2.00$ & $(1,4)$ & 476\\
$(610,233)$ & 13 & YES & YES & YES & $1.78$ & $(1,4)$ & 477\\
$(613,233)$ & 14 & YES & YES & NO(2) & $1.62$ & $(9,0)$ & 478\\
$(615,227)$ & 14 & YES & YES & NO(2) & $1.89$ & $(1,4)$ & 479\\
$(616,269)$ & 15 & YES & YES & NO(2) & $2.00$ & $(3,3)$ & 480\\
$(617,226)$ & 14 & YES & YES & NO(2) & $1.80$ & $(5,2)$ & 481\\
$(619,181)$ & 14 & YES & YES & YES & $1.78$ & $(1,4)$ & 482\\
$(622,255)$ & 14 & YES & YES & NO(2) & $1.78$ & $(3,3)$ & 483\\
$(622,261)$ & 14 & YES & YES & NO(2) & $1.82$ & $(3,3)$ & 484\\
$(625,258)$ & 14 & YES & YES & YES & $2.00$ & $(1,4)$ & 485\\
$(630,257)$ & 15 & YES & YES & NO(2) & $2.00$ & $(9,0)$ & 486\\
$(631,234)$ & 14 & YES & YES & YES & $1.88$ & $(1,4)$ & 487\\
$(633,266)$ & 14 & YES & YES & NO(2) & $1.78$ & $(7,1)$ & 488\\
$(643,178)$ & 14 & YES & YES & NO(2) & $1.78$ & $(1,4)$ & 489\\
$(643,236)$ & 14 & YES & YES & NO(2) & $1.78$ & $(3,3)$ & 490\\
$(643,264)$ & 14 & YES & YES & NO(2) & $1.89$ & $(3,3)$ & 491\\
$(645,181)$ & 15 & YES & YES & NO(2) & $1.82$ & $(3,3)$ & 492\\
$(648,179)$ & 14 & YES & YES & YES & $1.75$ & $(1,4)$ & 493\\
$(648,181)$ & 14 & YES & YES & YES & $1.78$ & $(1,4)$ & 494\\
$(649,192)$ & 14 & YES & YES & YES & $1.75$ & $(1,4)$ & 495\\
$(653,191)$ & 14 & YES & YES & YES & $1.88$ & $(1,4)$ & 496\\
$(654,271)$ & 14 & YES & YES & YES & $1.75$ & $(3,3)$ & 497\\
$(665,258)$ & 14 & YES & YES & YES & $1.86$ & $(3,3)$ & 498\\
$(666,241)$ & 15 & YES & YES & NO(2) & $2.00$ & $(3,3)$ & 499\\
$(670,259)$ & 15 & YES & YES & NO(2) & $1.78$ & $(7,1)$ & 500\\
$(673,199)$ & 14 & YES & YES & YES & $1.89$ & $(1,4)$ & 501\\
$(673,247)$ & 14 & YES & YES & NO(2) & $1.80$ & $(5,2)$ & 502\\
$(680,287)$ & 14 & YES & YES & YES & $1.71$ & $(3,3)$ & 503\\
$(683,198)$ & 16 & YES & YES & NO(2) & $1.91$ & $(3,3)$ & 504\\
$(683,287)$ & 14 & YES & YES & YES & $2.00$ & $(1,4)$ & 505\\
$(697,266)$ & 14 & YES & YES & YES & $2.00$ & $(1,4)$ & 506\\
$(699,287)$ & 15 & YES & YES & NO(2) & $1.89$ & $(3,3)$ & 507\\
$(701,189)$ & 15 & YES & YES & NO(2) & $1.75$ & $(3,3)$ & 508\\
$(704,205)$ & 15 & YES & YES & NO(2) & $1.67$ & $(3,3)$ & 509\\
$(707,274)$ & 14 & YES & YES & YES & $2.00$ & $(1,4)$ & 510\\
$(708,191)$ & 15 & YES & YES & NO(2) & $1.75$ & $(3,3)$ & 511\\
$(713,272)$ & 14 & YES & YES & NO(2) & $1.78$ & $(3,3)$ & 512\\
$(715,152)$ & 16 & YES & YES & YES & $1.86$ & $(1,4)$ & 513\\
$(715,277)$ & 14 & YES & YES & YES & $1.90$ & $(1,4)$ & 514\\
$(718,263)$ & 14 & YES & YES & YES & $1.71$ & $(3,3)$ & 515\\
$(721,214)$ & 15 & YES & YES & YES & $1.88$ & $(1,4)$ & 516\\
$(721,265)$ & 14 & YES & YES & NO(2) & $1.67$ & $(11,-1)$ & 517\\
$(722,261)$ & 15 & YES & YES & NO(2) & $1.78$ & $(7,1)$ & 518\\
$(733,307)$ & 14 & YES & YES & YES & $1.89$ & $(1,4)$ & 519\\
$(738,271)$ & 14 & YES & YES & YES & $1.71$ & $(3,3)$ & 520\\
$(740,321)$ & 15 & YES & YES & YES & $1.86$ & $(3,3)$ & 521\\
$(745,313)$ & 14 & YES & YES & YES & $1.86$ & $(3,3)$ & 522\\
$(747,200)$ & 15 & YES & YES & NO(2) & $1.91$ & $(3,3)$ & 523\\
$(752,287)$ & 14 & YES & YES & YES & $1.89$ & $(1,4)$ & 524\\
$(755,292)$ & 14 & YES & YES & YES & $1.88$ & $(3,3)$ & 525\\
$(755,312)$ & 14 & YES & YES & YES & $1.57$ & $(5,2)$ & 526\\
$(757,201)$ & 15 & YES & YES & NO(2) & $2.00$ & $(5,2)$ & 527\\
$(759,212)$ & 14 & YES & YES & YES & $1.88$ & $(1,4)$ & 528\\
$(763,173)$ & 16 & YES & YES & YES & $1.71$ & $(3,3)$ & 529\\
$(765,292)$ & 14 & YES & YES & YES & $1.57$ & $(3,3)$ & 530\\
$(767,227)$ & 15 & YES & YES & NO(2) & $1.75$ & $(3,3)$ & 531\\
$(771,208)$ & 15 & YES & YES & NO(2) & $1.57$ & $(5,2)$ & 532\\
$(777,218)$ & 16 & YES & YES & YES & $1.86$ & $(1,4)$ & 533\\
$(778,297)$ & 15 & YES & YES & NO(2) & $1.89$ & $(3,3)$ & 534\\
$(780,323)$ & 15 & YES & YES & YES & $1.88$ & $(3,3)$ & 535\\
$(784,179)$ & 15 & YES & YES & NO(2) & $1.78$ & $(1,4)$ & 536\\
$(785,233)$ & 15 & YES & YES & NO(2) & $1.75$ & $(3,3)$ & 537\\
$(788,241)$ & 15 & YES & YES & YES & $1.71$ & $(3,3)$ & 538\\
$(788,301)$ & 14 & YES & YES & YES & $1.88$ & $(1,4)$ & 539\\
$(789,302)$ & 14 & YES & YES & YES & $1.71$ & $(1,4)$ & 540\\
$(790,231)$ & 15 & YES & YES & YES & $2.00$ & $(1,4)$ & 541\\
$(791,327)$ & 15 & YES & YES & YES & $1.86$ & $(3,3)$ & 542\\
$(793,335)$ & 14 & YES & YES & YES & $1.86$ & $(5,2)$ & 543\\
$(795,308)$ & 14 & YES & YES & YES & $2.00$ & $(1,4)$ & 544\\
$(797,232)$ & 15 & YES & YES & NO(2) & $1.75$ & $(3,3)$ & 545\\
$(803,305)$ & 14 & YES & YES & YES & $1.90$ & $(1,4)$ & 546\\
$(817,178)$ & 16 & YES & YES & NO(2) & $1.57$ & $(5,2)$ & 547\\
$(818,311)$ & 14 & YES & YES & YES & $1.75$ & $(1,4)$ & 548\\
$(818,313)$ & 14 & YES & YES & YES & $1.88$ & $(1,4)$ & 549\\
$(820,313)$ & 14 & YES & YES & YES & $1.89$ & $(1,4)$ & 550\\
$(823,357)$ & 15 & YES & YES & YES & $1.71$ & $(3,3)$ & 551\\
$(842,349)$ & 15 & YES & YES & YES & $1.86$ & $(3,3)$ & 552\\
$(843,322)$ & 14 & YES & YES & YES & $1.86$ & $(1,4)$ & 553\\
$(858,239)$ & 15 & YES & YES & YES & $1.75$ & $(1,4)$ & 554\\
$(862,255)$ & 15 & YES & YES & YES & $1.88$ & $(1,4)$ & 555\\
$(863,358)$ & 15 & YES & YES & YES & $2.00$ & $(1,4)$ & 556\\
$(875,363)$ & 15 & YES & YES & YES & $1.86$ & $(3,3)$ & 557\\
$(877,324)$ & 15 & YES & YES & NO(2) & $1.80$ & $(5,2)$ & 558\\
$(877,335)$ & 14 & YES & YES & YES & $1.88$ & $(1,4)$ & 559\\
$(878,339)$ & 15 & YES & YES & YES & $2.00$ & $(1,4)$ & 560\\
$(882,337)$ & 14 & YES & YES & YES & $1.88$ & $(1,4)$ & 561\\
$(883,261)$ & 15 & YES & YES & YES & $1.88$ & $(1,4)$ & 562\\
$(885,343)$ & 14 & YES & YES & NO(2) & $2.00$ & $(3,3)$ & 563\\
$(886,367)$ & 15 & YES & YES & YES & $1.88$ & $(3,3)$ & 564\\
$(900,251)$ & 15 & YES & YES & YES & $1.75$ & $(1,4)$ & 565\\
$(927,358)$ & 15 & YES & YES & YES & $2.11$ & $(1,4)$ & 566\\
$(928,255)$ & 15 & YES & YES & YES & $1.86$ & $(5,2)$ & 567\\
$(935,259)$ & 15 & YES & YES & YES & $1.88$ & $(1,4)$ & 568\\
$(937,277)$ & 15 & YES & YES & YES & $1.89$ & $(1,4)$ & 569\\
$(938,253)$ & 15 & YES & YES & YES & $1.88$ & $(3,3)$ & 570\\
$(940,363)$ & 15 & YES & YES & YES & $1.88$ & $(3,3)$ & 571\\
$(941,264)$ & 15 & YES & YES & YES & $1.88$ & $(1,4)$ & 572\\
$(941,364)$ & 15 & YES & YES & YES & $2.00$ & $(3,3)$ & 573\\
$(945,388)$ & 15 & YES & YES & YES & $2.00$ & $(1,4)$ & 574\\
$(949,265)$ & 15 & YES & YES & YES & $2.00$ & $(1,4)$ & 575\\
$(955,219)$ & 16 & YES & YES & YES & $1.71$ & $(3,3)$ & 576\\
$(955,266)$ & 16 & YES & YES & YES & $1.86$ & $(3,3)$ & 577\\
$(955,282)$ & 15 & YES & YES & YES & $1.75$ & $(1,4)$ & 578\\
$(957,218)$ & 16 & YES & YES & NO(2) & $1.78$ & $(1,4)$ & 579\\
$(957,397)$ & 15 & YES & YES & YES & $2.00$ & $(1,4)$ & 580\\
$(965,373)$ & 15 & YES & YES & YES & $2.12$ & $(1,4)$ & 581\\
$(974,227)$ & 16 & YES & YES & NO(2) & $1.78$ & $(1,4)$ & 582\\
$(977,353)$ & 16 & YES & YES & YES & $1.71$ & $(5,2)$ & 583\\
$(997,292)$ & 15 & YES & YES & YES & $1.88$ & $(1,4)$ & 584\\
$(997,295)$ & 15 & YES & YES & YES & $1.90$ & $(1,4)$ & 585\\
$(998,275)$ & 16 & YES & YES & YES & $2.00$ & $(1,4)$ & 586\\
$(1025,303)$ & 15 & YES & YES & YES & $2.00$ & $(1,4)$ & 587\\
$(1027,305)$ & 15 & YES & YES & YES & $1.90$ & $(1,4)$ & 588\\
$(1028,189)$ & 17 & YES & YES & NO(2) & $1.82$ & $(3,3)$ & 589\\
$(1043,432)$ & 15 & YES & YES & YES & $2.14$ & $(1,4)$ & 590\\
$(1047,292)$ & 15 & YES & YES & YES & $1.88$ & $(1,4)$ & 591\\
$(1047,307)$ & 16 & YES & YES & YES & $1.86$ & $(3,3)$ & 592\\
$(1048,237)$ & 16 & YES & YES & YES & $1.90$ & $(1,4)$ & 593\\
$(1052,389)$ & 15 & YES & YES & YES & $1.88$ & $(3,3)$ & 594\\
$(1058,321)$ & 16 & YES & YES & YES & $2.00$ & $(3,3)$ & 595\\
$(1058,409)$ & 15 & YES & YES & YES & $2.14$ & $(1,4)$ & 596\\
$(1067,323)$ & 16 & YES & YES & YES & $2.00$ & $(1,4)$ & 597\\
$(1103,323)$ & 16 & YES & YES & YES & $2.00$ & $(1,4)$ & 598\\
$(1115,308)$ & 15 & YES & YES & YES & $1.89$ & $(3,3)$ & 599\\
$(1147,348)$ & 16 & YES & YES & YES & $1.89$ & $(3,3)$ & 600\\
$(1171,265)$ & 16 & YES & YES & YES & $1.88$ & $(1,4)$ & 601\\
$(1180,433)$ & 16 & YES & YES & YES & $2.12$ & $(3,3)$ & 602\\
$(1181,256)$ & 16 & YES & YES & YES & $1.88$ & $(3,3)$ & 603\\
$(1190,349)$ & 16 & YES & YES & YES & $1.71$ & $(3,3)$ & 604\\
$(1196,271)$ & 16 & YES & YES & YES & $1.71$ & $(1,4)$ & 605\\
$(1210,367)$ & 16 & YES & YES & YES & $1.75$ & $(3,3)$ & 606\\
$(1214,277)$ & 16 & YES & YES & YES & $1.88$ & $(1,4)$ & 607\\
$(1216,329)$ & 17 & YES & YES & YES & $1.86$ & $(5,2)$ & 608\\
$(1218,463)$ & 15 & YES & YES & YES & $2.00$ & $(1,4)$ & 609\\
$(1234,269)$ & 17 & YES & YES & YES & $1.71$ & $(5,2)$ & 610\\
$(1257,466)$ & 16 & YES & YES & YES & $1.88$ & $(3,3)$ & 611\\
$(1283,298)$ & 17 & YES & YES & YES & $1.71$ & $(3,3)$ & 612\\
$(1297,241)$ & 17 & YES & YES & YES & $1.57$ & $(5,2)$ & 613\\
$(1318,399)$ & 17 & YES & YES & YES & $2.00$ & $(3,3)$ & 614\\
$(1355,393)$ & 17 & YES & YES & YES & $2.00$ & $(3,3)$ & 615\\
$(1407,247)$ & 18 & YES & YES & NO(2) & $1.80$ & $(5,2)$ & 616\\
$(1463,335)$ & 16 & YES & YES & YES & $1.89$ & $(3,3)$ & 617\\
$(1504,341)$ & 17 & YES & YES & YES & $2.14$ & $(1,4)$ & 618\\
$(1783,331)$ & 18 & YES & YES & YES & $2.14$ & $(1,4)$ & 619
\end{longtable}
\subsection{1 chain, $K^2 = 5$}
\begin{longtable}{|c|c|c|c|c|c|c|c|}
\hline
\multicolumn{8}{|c|}{1 chain, $K^2 = 5$}\\
\hline
$(n,a)$ & Len & Nef & $\mathbb Q$-ef & Obs 0 & $\overline c_1^2 / \overline c_2$ & $(P,K)$ & Index\\
\hline
\endfirsthead

\hline
$(n,a)$ & Len & Nef & $\mathbb Q$-ef & Obs 0 & $\overline c_1^2 / \overline c_2$ & $(P,K)$ & Index\\
\hline
\endhead
\hline
\endfoot

$(1165,431)$ & 15 & YES & YES & NO(2) & $2.18$ & $(5,3)$ & 620\\
$(1306,547)$ & 15 & YES & YES & NO(2) & $2.09$ & $(5,3)$ & 621\\
$(1412,417)$ & 16 & YES & YES & NO(2) & $2.10$ & $(7,2)$ & 622\\
$(1605,452)$ & 17 & YES & YES & NO(2) & $2.27$ & $(1,5)$ & 623\\
$(1962,547)$ & 17 & YES & YES & NO(2) & $2.18$ & $(5,3)$ & 624
\end{longtable}
\subsection{2 chains, $K^2 = 1$}
\begin{longtable}{|c|c|c|c|c|c|c|c|c|c|c|c|}
\hline
\multicolumn{12}{|c|}{2 chains, $K^2 = 1$}\\
\hline
$(n,a)$ & Len & $(n,a)$ & Len & GCD & Nef & $\mathbb Q$-ef & Obs 0 & $\overline c_1^2 / \overline c_2$ & $(P,K)$ & WH & Index\\
\hline
\endfirsthead

\hline
$(n,a)$ & Len & $(n,a)$ & Len & GCD & Nef & $\mathbb Q$-ef & Obs 0 & $\overline c_1^2 / \overline c_2$ & $(P,K)$ & WH & Index\\
\hline
\endhead
\hline
\endfoot

$(10,3)$ & 5 & $(7,3)$ & 4 & 1 & YES & YES & YES & $0.70$ & $(2,1)$ & -- & 625\\
$(11,3)$ & 5 & $(7,3)$ & 4 & 1 & YES & YES & YES & $0.80$ & $(2,1)$ & -- & 626\\
$(11,4)$ & 5 & $(7,2)$ & 4 & 1 & YES & YES & YES & $0.80$ & $(2,1)$ & -- & 627\\
$(12,5)$ & 5 & $(5,2)$ & 3 & 1 & YES & YES & YES & $0.80$ & $(2,1)$ & -- & 628\\
$(12,5)$ & 5 & $(7,2)$ & 4 & 1 & YES & YES & YES & $0.80$ & $(2,1)$ & NO & 629\\
$(12,5)$ & 5 & $(7,2)$ & 4 & 1 & YES & YES & YES & $1.00$ & $(2,1)$ & -- & 630\\
$(13,4)$ & 6 & $(7,2)$ & 4 & 1 & YES & YES & YES & $0.80$ & $(2,1)$ & -- & 631\\
$(13,5)$ & 5 & $(7,2)$ & 4 & 1 & YES & YES & YES & $0.70$ & $(2,1)$ & NO & 632\\
$(13,5)$ & 5 & $(7,3)$ & 4 & 1 & YES & YES & YES & $0.67$ & $(4,0)$ & -- & 633\\
$(13,5)$ & 5 & $(11,4)$ & 5 & 1 & YES & YES & YES & $0.70$ & $(2,1)$ & NO & 634\\
$(17,7)$ & 6 & $(3,1)$ & 2 & 1 & YES & YES & YES & $0.80$ & $(2,1)$ & NO & 635\\
$(17,7)$ & 6 & $(3,1)$ & 2 & 1 & YES & YES & YES & $0.80$ & $(2,1)$ & -- & 636\\
$(17,7)$ & 6 & $(4,1)$ & 3 & 1 & YES & YES & YES & $0.80$ & $(2,1)$ & NO & 637\\
$(17,7)$ & 6 & $(4,1)$ & 3 & 1 & YES & YES & YES & $0.80$ & $(2,1)$ & -- & 638\\
$(17,7)$ & 6 & $(4,1)$ & 3 & 1 & YES & YES & YES & $0.80$ & $(2,1)$ & NO & 639\\
$(17,7)$ & 6 & $(5,1)$ & 4 & 1 & YES & YES & YES & $0.80$ & $(2,1)$ & NO & 640\\
$(17,7)$ & 6 & $(5,1)$ & 4 & 1 & YES & YES & YES & $0.80$ & $(2,1)$ & NO & 641\\
$(17,7)$ & 6 & $(7,3)$ & 4 & 1 & YES & YES & YES & $0.80$ & $(2,1)$ & 655 & 642\\
$(17,7)$ & 6 & $(12,5)$ & 5 & 1 & YES & YES & YES & $0.80$ & $(2,1)$ & NO & 643\\
$(17,7)$ & 6 & $(17,7)$ & 6 & 17 & YES & YES & YES & $0.80$ & $(2,1)$ & NO & 644\\
$(18,7)$ & 6 & $(3,1)$ & 2 & 3 & YES & YES & YES & $0.70$ & $(2,1)$ & -- & 645\\
$(18,7)$ & 6 & $(3,1)$ & 2 & 3 & YES & YES & YES & $0.90$ & $(2,1)$ & NO & 646\\
$(18,7)$ & 6 & $(18,7)$ & 6 & 18 & YES & YES & YES & $0.60$ & $(2,1)$ & NO & 647\\
$(19,8)$ & 6 & $(2,1)$ & 1 & 1 & YES & YES & YES & $0.70$ & $(2,1)$ & -- & 648\\
$(19,7)$ & 6 & $(3,1)$ & 2 & 1 & YES & YES & YES & $0.70$ & $(2,1)$ & -- & 649\\
$(19,7)$ & 6 & $(3,1)$ & 2 & 1 & YES & YES & YES & $0.80$ & $(2,1)$ & NO & 650\\
$(19,8)$ & 6 & $(3,1)$ & 2 & 1 & YES & YES & YES & $0.80$ & $(2,1)$ & 660 & 651\\
$(19,8)$ & 6 & $(3,1)$ & 2 & 1 & YES & YES & YES & $0.80$ & $(2,1)$ & -- & 652\\
$(19,8)$ & 6 & $(5,1)$ & 4 & 1 & YES & YES & YES & $0.80$ & $(2,1)$ & NO & 653\\
$(19,8)$ & 6 & $(5,1)$ & 4 & 1 & YES & YES & YES & $0.90$ & $(2,1)$ & NO & 654\\
$(19,8)$ & 6 & $(5,2)$ & 3 & 1 & YES & YES & YES & $0.80$ & $(2,1)$ & 642 & 655\\
$(19,8)$ & 6 & $(5,2)$ & 3 & 1 & YES & YES & YES & $0.67$ & $(4,0)$ & -- & 656\\
$(19,8)$ & 6 & $(7,3)$ & 4 & 1 & YES & YES & YES & $0.70$ & $(2,1)$ & NO & 657\\
$(19,7)$ & 6 & $(19,7)$ & 6 & 19 & YES & YES & YES & $0.70$ & $(2,1)$ & NO & 658\\
$(19,8)$ & 6 & $(19,8)$ & 6 & 19 & YES & YES & YES & $0.80$ & $(2,1)$ & NO & 659\\
$(21,8)$ & 6 & $(2,1)$ & 1 & 1 & YES & YES & YES & $0.80$ & $(2,1)$ & 651 & 660\\
$(21,8)$ & 6 & $(3,1)$ & 2 & 3 & YES & YES & YES & $0.70$ & $(2,1)$ & NO & 661\\
$(24,7)$ & 7 & $(3,1)$ & 2 & 3 & YES & YES & YES & $0.80$ & $(2,1)$ & NO & 662\\
$(24,7)$ & 7 & $(7,2)$ & 4 & 1 & YES & YES & YES & $0.70$ & $(2,1)$ & NO & 663\\
$(25,7)$ & 7 & $(2,1)$ & 1 & 1 & YES & YES & YES & $0.80$ & $(2,1)$ & NO & 664\\
$(25,7)$ & 7 & $(3,1)$ & 2 & 1 & YES & YES & YES & $0.70$ & $(2,1)$ & NO & 665\\
$(25,7)$ & 7 & $(3,1)$ & 2 & 1 & YES & YES & YES & $0.80$ & $(2,1)$ & -- & 666\\
$(25,7)$ & 7 & $(7,2)$ & 4 & 1 & YES & YES & YES & $0.90$ & $(2,1)$ & NO & 667\\
$(31,7)$ & 8 & $(2,1)$ & 1 & 1 & YES & YES & YES & $0.80$ & $(2,1)$ & NO & 668\\
$(31,13)$ & 7 & $(2,1)$ & 1 & 1 & NO & YES & YES & $0.80$ & $(2,1)$ & -- & 669\\
$(31,7)$ & 8 & $(4,1)$ & 3 & 1 & YES & YES & YES & $0.80$ & $(2,1)$ & NO & 670\\
$(31,7)$ & 8 & $(31,7)$ & 8 & 31 & YES & YES & YES & $0.70$ & $(2,1)$ & NO & 671\\
$(a;1,0,0;13)$ & 5 & $(3,1)$ & 2 & 1 & YES & YES & YES & $0.80$ & $(2,1)$ & -- & 672\\
$(a;1,0,0;13)$ & 5 & $(4,1)$ & 3 & 1 & YES & YES & YES & $0.80$ & $(2,1)$ & -- & 673\\
$(a;1,0,0;13)$ & 5 & $(5,2)$ & 3 & 1 & YES & YES & YES & $0.70$ & $(2,1)$ & -- & 674\\
$(a;1,0,0;13)$ & 5 & $(7,2)$ & 4 & 1 & YES & YES & YES & $0.90$ & $(2,1)$ & -- & 675\\
$(a;1,1,0;19)$ & 6 & $(4,1)$ & 3 & 1 & YES & YES & YES & $0.90$ & $(2,1)$ & -- & 676\\
$(b;0,0,0;14)$ & 5 & $(2,1)$ & 1 & 2 & YES & YES & YES & $0.70$ & $(2,1)$ & -- & 677\\
$(b;0,0,0;14)$ & 5 & $(3,1)$ & 2 & 1 & YES & YES & YES & $0.70$ & $(2,1)$ & -- & 678\\
$(b;0,0,0;14)$ & 5 & $(5,2)$ & 3 & 1 & YES & YES & YES & $0.90$ & $(2,1)$ & -- & 679\\
$(c;0,0,0;4)$ & 4 & $(7,3)$ & 4 & 1 & YES & YES & YES & $0.90$ & $(2,1)$ & -- & 680\\
$(c;0,0,0;4)$ & 4 & $(8,3)$ & 4 & 4 & YES & YES & YES & $0.90$ & $(2,1)$ & -- & 681\\
$(c;0,0,0;4)$ & 4 & $(13,4)$ & 6 & 1 & YES & YES & YES & $0.80$ & $(2,1)$ & -- & 682\\
$(c;0,1,1;5)$ & 6 & $(2,1)$ & 1 & 1 & YES & YES & YES & $0.80$ & $(2,1)$ & -- & 683\\
$(c;0,1,1;5)$ & 6 & $(4,1)$ & 3 & 1 & YES & YES & YES & $0.80$ & $(2,1)$ & -- & 684\\
$(d;0,0,0;5)$ & 5 & $(5,2)$ & 3 & 5 & YES & YES & YES & $0.80$ & $(2,1)$ & -- & 685\\
$(d;0,0,0;5)$ & 5 & $(7,2)$ & 4 & 1 & YES & YES & YES & $0.70$ & $(2,1)$ & -- & 686\\
$(d;0,0,1;14)$ & 6 & $(2,1)$ & 1 & 2 & YES & YES & YES & $0.80$ & $(2,1)$ & -- & 687\\
$(d;0,0,1;14)$ & 6 & $(4,1)$ & 3 & 2 & YES & YES & YES & $0.70$ & $(2,1)$ & -- & 688\\
$(e;0,0,0;4)$ & 5 & $(2,1)$ & 1 & 2 & YES & YES & YES & $0.70$ & $(2,1)$ & -- & 689\\
$(i;0,0,0;9)$ & 5 & $(3,1)$ & 2 & 3 & YES & YES & YES & $0.70$ & $(2,1)$ & -- & 690\\
$(i;0,0,0;9)$ & 5 & $(4,1)$ & 3 & 1 & YES & YES & YES & $0.70$ & $(2,1)$ & -- & 691
\end{longtable}
\subsection{2 chains, $K^2 = 2$}
\begin{longtable}{|c|c|c|c|c|c|c|c|c|c|c|c|}
\hline
\multicolumn{12}{|c|}{2 chains, $K^2 = 2$}\\
\hline
$(n,a)$ & Len & $(n,a)$ & Len & GCD & Nef & $\mathbb Q$-ef & Obs 0 & $\overline c_1^2 / \overline c_2$ & $(P,K)$ & WH & Index\\
\hline
\endfirsthead

\hline
$(n,a)$ & Len & $(n,a)$ & Len & GCD & Nef & $\mathbb Q$-ef & Obs 0 & $\overline c_1^2 / \overline c_2$ & $(P,K)$ & WH & Index\\
\hline
\endhead
\hline
\endfoot

$(13,5)$ & 5 & $(13,4)$ & 6 & 13 & YES & YES & YES & $1.33$ & $(2,2)$ & NO & 692\\
$(13,5)$ & 5 & $(13,4)$ & 6 & 13 & YES & YES & YES & $1.00$ & $(4,1)$ & -- & 693\\
$(13,5)$ & 5 & $(13,5)$ & 5 & 13 & YES & YES & YES & $1.22$ & $(2,2)$ & -- & 694\\
$(16,7)$ & 6 & $(13,5)$ & 5 & 1 & YES & YES & YES & $1.22$ & $(2,2)$ & -- & 695\\
$(16,7)$ & 6 & $(16,7)$ & 6 & 16 & YES & YES & YES & $1.00$ & $(2,2)$ & -- & 696\\
$(17,7)$ & 6 & $(10,3)$ & 5 & 1 & YES & YES & YES & $1.00$ & $(2,2)$ & -- & 697\\
$(17,7)$ & 6 & $(11,3)$ & 5 & 1 & YES & YES & YES & $1.11$ & $(2,2)$ & -- & 698\\
$(17,7)$ & 6 & $(11,3)$ & 5 & 1 & YES & YES & NO(2) & $1.30$ & $(4,1)$ & NO & 699\\
$(17,5)$ & 6 & $(12,5)$ & 5 & 1 & YES & YES & YES & $1.00$ & $(2,2)$ & -- & 700\\
$(17,5)$ & 6 & $(13,5)$ & 5 & 1 & YES & YES & YES & $1.11$ & $(2,2)$ & NO & 701\\
$(17,5)$ & 6 & $(13,5)$ & 5 & 1 & YES & YES & YES & $1.11$ & $(2,2)$ & -- & 702\\
$(17,4)$ & 7 & $(16,5)$ & 7 & 1 & YES & YES & NO(2) & $0.88$ & $(8,-1)$ & -- & 703\\
$(17,4)$ & 7 & $(16,5)$ & 7 & 1 & YES & YES & NO(2) & $1.00$ & $(8,-1)$ & NO & 704\\
$(17,5)$ & 6 & $(16,5)$ & 7 & 1 & YES & YES & NO(2) & $1.00$ & $(6,0)$ & -- & 705\\
$(17,7)$ & 6 & $(16,7)$ & 6 & 1 & YES & YES & NO(2) & $1.30$ & $(4,1)$ & -- & 706\\
$(17,5)$ & 6 & $(17,5)$ & 6 & 17 & YES & YES & YES & $1.25$ & $(2,2)$ & -- & 707\\
$(17,6)$ & 7 & $(17,4)$ & 7 & 17 & YES & YES & NO(2) & $1.11$ & $(6,0)$ & -- & 708\\
$(17,6)$ & 7 & $(17,5)$ & 6 & 17 & YES & YES & NO(2) & $1.00$ & $(6,0)$ & -- & 709\\
$(17,7)$ & 6 & $(17,5)$ & 6 & 17 & YES & YES & YES & $1.00$ & $(2,2)$ & -- & 710\\
$(17,7)$ & 6 & $(17,5)$ & 6 & 17 & YES & YES & NO(2) & $1.22$ & $(4,1)$ & NO & 711\\
$(18,7)$ & 6 & $(11,3)$ & 5 & 1 & YES & YES & YES & $1.11$ & $(2,2)$ & -- & 712\\
$(18,5)$ & 6 & $(12,5)$ & 5 & 6 & YES & YES & NO(2) & $1.30$ & $(4,1)$ & NO & 713\\
$(18,5)$ & 6 & $(12,5)$ & 5 & 6 & YES & YES & NO(2) & $1.30$ & $(4,1)$ & -- & 714\\
$(18,7)$ & 6 & $(12,5)$ & 5 & 6 & YES & YES & YES & $1.22$ & $(2,2)$ & -- & 715\\
$(18,7)$ & 6 & $(12,5)$ & 5 & 6 & YES & YES & NO(2) & $1.40$ & $(4,1)$ & NO & 716\\
$(18,5)$ & 6 & $(13,5)$ & 5 & 1 & YES & YES & NO(2) & $1.30$ & $(4,1)$ & NO & 717\\
$(18,5)$ & 6 & $(13,5)$ & 5 & 1 & YES & YES & NO(2) & $1.30$ & $(4,1)$ & -- & 718\\
$(18,5)$ & 6 & $(16,7)$ & 6 & 2 & YES & YES & YES & $1.11$ & $(2,2)$ & NO & 719\\
$(18,5)$ & 6 & $(16,7)$ & 6 & 2 & YES & YES & YES & $1.22$ & $(2,2)$ & -- & 720\\
$(18,5)$ & 6 & $(16,7)$ & 6 & 2 & YES & YES & NO(2) & $1.20$ & $(4,1)$ & NO & 721\\
$(18,5)$ & 6 & $(17,5)$ & 6 & 1 & YES & YES & YES & $1.12$ & $(2,2)$ & -- & 722\\
$(18,5)$ & 6 & $(17,5)$ & 6 & 1 & YES & YES & YES & $1.25$ & $(2,2)$ & NO & 723\\
$(18,5)$ & 6 & $(17,7)$ & 6 & 1 & YES & YES & YES & $1.22$ & $(2,2)$ & -- & 724\\
$(18,5)$ & 6 & $(17,7)$ & 6 & 1 & YES & YES & YES & $1.33$ & $(2,2)$ & NO & 725\\
$(18,7)$ & 6 & $(17,5)$ & 6 & 1 & YES & YES & YES & $1.00$ & $(2,2)$ & -- & 726\\
$(19,8)$ & 6 & $(8,3)$ & 4 & 1 & YES & YES & YES & $1.11$ & $(2,2)$ & -- & 727\\
$(19,8)$ & 6 & $(10,3)$ & 5 & 1 & YES & YES & YES & $1.00$ & $(4,1)$ & NO & 728\\
$(19,8)$ & 6 & $(10,3)$ & 5 & 1 & YES & YES & YES & $1.00$ & $(4,1)$ & -- & 729\\
$(19,7)$ & 6 & $(11,5)$ & 6 & 1 & YES & YES & YES & $1.12$ & $(4,1)$ & -- & 730\\
$(19,8)$ & 6 & $(11,3)$ & 5 & 1 & YES & YES & YES & $1.11$ & $(2,2)$ & -- & 731\\
$(19,8)$ & 6 & $(11,3)$ & 5 & 1 & YES & YES & NO(2) & $1.00$ & $(6,0)$ & NO & 732\\
$(19,7)$ & 6 & $(12,5)$ & 5 & 1 & YES & YES & NO(2) & $1.22$ & $(6,0)$ & -- & 733\\
$(19,8)$ & 6 & $(12,5)$ & 5 & 1 & YES & YES & YES & $1.00$ & $(2,2)$ & -- & 734\\
$(19,7)$ & 6 & $(13,4)$ & 6 & 1 & YES & YES & YES & $1.12$ & $(2,2)$ & -- & 735\\
$(19,8)$ & 6 & $(13,5)$ & 5 & 1 & YES & YES & NO(2) & $1.30$ & $(4,1)$ & -- & 736\\
$(19,8)$ & 6 & $(15,4)$ & 6 & 1 & YES & YES & YES & $1.11$ & $(2,2)$ & -- & 737\\
$(19,4)$ & 7 & $(16,5)$ & 7 & 1 & YES & YES & NO(2) & $0.88$ & $(8,-1)$ & -- & 738\\
$(19,4)$ & 7 & $(16,5)$ & 7 & 1 & YES & YES & NO(2) & $1.00$ & $(8,-1)$ & NO & 739\\
$(19,7)$ & 6 & $(16,7)$ & 6 & 1 & YES & YES & NO(2) & $1.40$ & $(4,1)$ & NO & 740\\
$(19,7)$ & 6 & $(16,7)$ & 6 & 1 & YES & YES & NO(2) & $1.40$ & $(4,1)$ & -- & 741\\
$(19,4)$ & 7 & $(17,6)$ & 7 & 1 & YES & YES & NO(2) & $1.11$ & $(6,0)$ & -- & 742\\
$(19,7)$ & 6 & $(17,5)$ & 6 & 1 & YES & YES & YES & $1.00$ & $(2,2)$ & -- & 743\\
$(19,8)$ & 6 & $(17,5)$ & 6 & 1 & YES & YES & YES & $1.33$ & $(2,2)$ & -- & 744\\
$(19,4)$ & 7 & $(18,7)$ & 6 & 1 & YES & YES & YES & $1.22$ & $(2,2)$ & -- & 745\\
$(19,8)$ & 6 & $(18,5)$ & 6 & 1 & YES & YES & YES & $1.12$ & $(2,2)$ & NO & 746\\
$(19,8)$ & 6 & $(18,5)$ & 6 & 1 & YES & YES & YES & $1.12$ & $(2,2)$ & -- & 747\\
$(19,7)$ & 6 & $(19,7)$ & 6 & 19 & YES & YES & YES & $1.12$ & $(6,0)$ & -- & 748\\
$(19,8)$ & 6 & $(19,7)$ & 6 & 19 & YES & YES & YES & $1.33$ & $(4,1)$ & -- & 749\\
$(20,9)$ & 7 & $(12,5)$ & 5 & 4 & YES & YES & NO(2) & $1.50$ & $(4,1)$ & -- & 750\\
$(20,9)$ & 7 & $(13,5)$ & 5 & 1 & YES & YES & NO(2) & $1.22$ & $(6,0)$ & -- & 751\\
$(20,9)$ & 7 & $(15,4)$ & 6 & 5 & YES & YES & NO(2) & $1.30$ & $(4,1)$ & -- & 752\\
$(20,9)$ & 7 & $(17,5)$ & 6 & 1 & YES & YES & NO(2) & $1.40$ & $(4,1)$ & 2057 & 753\\
$(20,9)$ & 7 & $(18,7)$ & 6 & 2 & YES & YES & YES & $1.00$ & $(4,1)$ & NO & 754\\
$(20,9)$ & 7 & $(19,7)$ & 6 & 1 & YES & YES & NO(2) & $1.40$ & $(4,1)$ & NO & 755\\
$(21,8)$ & 6 & $(7,2)$ & 4 & 7 & YES & YES & YES & $0.88$ & $(4,1)$ & NO & 756\\
$(21,8)$ & 6 & $(7,2)$ & 4 & 7 & YES & YES & YES & $1.12$ & $(4,1)$ & -- & 757\\
$(21,8)$ & 6 & $(7,3)$ & 4 & 7 & YES & YES & YES & $1.00$ & $(2,2)$ & -- & 758\\
$(21,8)$ & 6 & $(8,3)$ & 4 & 1 & YES & YES & YES & $1.00$ & $(2,2)$ & -- & 759\\
$(21,8)$ & 6 & $(9,2)$ & 5 & 3 & YES & YES & NO(2) & $1.30$ & $(4,1)$ & NO & 760\\
$(21,8)$ & 6 & $(9,2)$ & 5 & 3 & YES & YES & NO(2) & $1.30$ & $(4,1)$ & -- & 761\\
$(21,8)$ & 6 & $(9,2)$ & 5 & 3 & YES & YES & NO(2) & $1.11$ & $(6,0)$ & NO & 762\\
$(21,8)$ & 6 & $(9,4)$ & 5 & 3 & YES & YES & YES & $1.33$ & $(2,2)$ & -- & 763\\
$(21,8)$ & 6 & $(10,3)$ & 5 & 1 & YES & YES & YES & $0.88$ & $(4,1)$ & -- & 764\\
$(21,8)$ & 6 & $(10,3)$ & 5 & 1 & YES & YES & YES & $1.44$ & $(2,2)$ & NO & 765\\
$(21,8)$ & 6 & $(12,5)$ & 5 & 3 & YES & YES & YES & $1.00$ & $(2,2)$ & -- & 766\\
$(21,8)$ & 6 & $(13,4)$ & 6 & 1 & YES & YES & YES & $1.33$ & $(2,2)$ & -- & 767\\
$(21,8)$ & 6 & $(13,5)$ & 5 & 1 & YES & YES & YES & $1.00$ & $(2,2)$ & -- & 768\\
$(21,8)$ & 6 & $(13,5)$ & 5 & 1 & YES & YES & YES & $1.33$ & $(2,2)$ & NO & 769\\
$(21,8)$ & 6 & $(16,7)$ & 6 & 1 & YES & YES & YES & $1.11$ & $(2,2)$ & NO & 770\\
$(21,4)$ & 8 & $(17,6)$ & 7 & 1 & YES & YES & YES & $1.12$ & $(4,1)$ & NO & 771\\
$(21,4)$ & 8 & $(17,6)$ & 7 & 1 & YES & YES & NO(2) & $1.11$ & $(6,0)$ & -- & 772\\
$(21,8)$ & 6 & $(17,5)$ & 6 & 1 & YES & YES & YES & $0.88$ & $(4,1)$ & -- & 773\\
$(21,8)$ & 6 & $(17,5)$ & 6 & 1 & YES & YES & YES & $1.00$ & $(4,1)$ & NO & 774\\
$(21,8)$ & 6 & $(18,5)$ & 6 & 3 & YES & YES & YES & $1.12$ & $(4,1)$ & -- & 775\\
$(21,8)$ & 6 & $(18,5)$ & 6 & 3 & YES & YES & YES & $1.25$ & $(4,1)$ & NO & 776\\
$(21,8)$ & 6 & $(18,5)$ & 6 & 3 & YES & YES & YES & $1.00$ & $(2,2)$ & NO & 777\\
$(21,5)$ & 8 & $(19,5)$ & 7 & 1 & YES & YES & NO(2) & $1.00$ & $(6,0)$ & -- & 778\\
$(21,8)$ & 6 & $(19,7)$ & 6 & 1 & YES & YES & YES & $1.12$ & $(6,0)$ & -- & 779\\
$(21,5)$ & 8 & $(21,5)$ & 8 & 21 & YES & YES & YES & $1.25$ & $(2,2)$ & -- & 780\\
$(21,8)$ & 6 & $(21,8)$ & 6 & 21 & YES & YES & YES & $1.33$ & $(4,1)$ & -- & 781\\
$(22,9)$ & 7 & $(11,3)$ & 5 & 11 & YES & YES & YES & $1.00$ & $(2,2)$ & -- & 782\\
$(22,5)$ & 7 & $(13,4)$ & 6 & 1 & YES & YES & YES & $1.11$ & $(2,2)$ & -- & 783\\
$(22,5)$ & 7 & $(14,5)$ & 6 & 2 & YES & YES & YES & $1.11$ & $(2,2)$ & NO & 784\\
$(22,9)$ & 7 & $(14,3)$ & 6 & 2 & YES & YES & YES & $1.00$ & $(2,2)$ & NO & 785\\
$(22,5)$ & 7 & $(19,6)$ & 8 & 1 & YES & YES & NO(2) & $1.20$ & $(4,1)$ & NO & 786\\
$(22,5)$ & 7 & $(19,8)$ & 6 & 1 & YES & YES & NO(2) & $1.00$ & $(6,0)$ & NO & 787\\
$(22,5)$ & 7 & $(19,8)$ & 6 & 1 & YES & YES & NO(2) & $1.00$ & $(6,0)$ & -- & 788\\
$(22,9)$ & 7 & $(22,5)$ & 7 & 22 & YES & YES & YES & $1.22$ & $(4,1)$ & -- & 789\\
$(22,9)$ & 7 & $(22,5)$ & 7 & 22 & YES & YES & YES & $1.22$ & $(4,1)$ & NO & 790\\
$(23,9)$ & 7 & $(7,2)$ & 4 & 1 & YES & YES & YES & $1.44$ & $(2,2)$ & NO & 791\\
$(23,7)$ & 7 & $(8,3)$ & 4 & 1 & YES & YES & YES & $1.11$ & $(2,2)$ & -- & 792\\
$(23,7)$ & 7 & $(8,3)$ & 4 & 1 & YES & YES & YES & $1.22$ & $(2,2)$ & NO & 793\\
$(23,9)$ & 7 & $(8,3)$ & 4 & 1 & YES & YES & YES & $1.11$ & $(2,2)$ & -- & 794\\
$(23,9)$ & 7 & $(8,3)$ & 4 & 1 & YES & YES & YES & $1.22$ & $(2,2)$ & NO & 795\\
$(23,9)$ & 7 & $(10,3)$ & 5 & 1 & YES & YES & YES & $1.22$ & $(2,2)$ & -- & 796\\
$(23,7)$ & 7 & $(11,3)$ & 5 & 1 & YES & YES & YES & $0.88$ & $(4,1)$ & -- & 797\\
$(23,7)$ & 7 & $(11,3)$ & 5 & 1 & YES & YES & YES & $1.00$ & $(4,1)$ & NO & 798\\
$(23,7)$ & 7 & $(11,4)$ & 5 & 1 & YES & YES & YES & $1.12$ & $(2,2)$ & -- & 799\\
$(23,10)$ & 7 & $(11,3)$ & 5 & 1 & YES & YES & NO(2) & $1.00$ & $(6,0)$ & NO & 800\\
$(23,10)$ & 7 & $(11,3)$ & 5 & 1 & YES & YES & NO(2) & $1.00$ & $(6,0)$ & -- & 801\\
$(23,10)$ & 7 & $(11,4)$ & 5 & 1 & YES & YES & YES & $1.12$ & $(4,1)$ & -- & 802\\
$(23,10)$ & 7 & $(11,4)$ & 5 & 1 & YES & YES & NO(2) & $1.22$ & $(6,0)$ & NO & 803\\
$(23,10)$ & 7 & $(12,5)$ & 5 & 1 & YES & YES & NO(2) & $1.11$ & $(6,0)$ & -- & 804\\
$(23,7)$ & 7 & $(13,4)$ & 6 & 1 & YES & YES & YES & $1.00$ & $(2,2)$ & -- & 805\\
$(23,7)$ & 7 & $(13,5)$ & 5 & 1 & YES & YES & YES & $1.12$ & $(2,2)$ & -- & 806\\
$(23,9)$ & 7 & $(13,3)$ & 6 & 1 & YES & YES & YES & $1.33$ & $(2,2)$ & NO & 807\\
$(23,10)$ & 7 & $(13,3)$ & 6 & 1 & YES & YES & YES & $0.88$ & $(4,1)$ & NO & 808\\
$(23,10)$ & 7 & $(13,3)$ & 6 & 1 & YES & YES & NO(2) & $1.00$ & $(6,0)$ & NO & 809\\
$(23,10)$ & 7 & $(13,3)$ & 6 & 1 & YES & YES & NO(2) & $1.00$ & $(6,0)$ & -- & 810\\
$(23,10)$ & 7 & $(13,5)$ & 5 & 1 & YES & YES & YES & $1.22$ & $(2,2)$ & -- & 811\\
$(23,9)$ & 7 & $(14,3)$ & 6 & 1 & YES & YES & YES & $1.11$ & $(2,2)$ & NO & 812\\
$(23,6)$ & 8 & $(15,4)$ & 6 & 1 & YES & YES & NO(2) & $1.11$ & $(6,0)$ & -- & 813\\
$(23,7)$ & 7 & $(15,4)$ & 6 & 1 & YES & YES & YES & $1.11$ & $(2,2)$ & -- & 814\\
$(23,10)$ & 7 & $(15,4)$ & 6 & 1 & YES & YES & YES & $1.12$ & $(4,1)$ & 1181 & 815\\
$(23,5)$ & 7 & $(16,5)$ & 7 & 1 & YES & YES & YES & $1.11$ & $(2,2)$ & NO & 816\\
$(23,4)$ & 8 & $(17,6)$ & 7 & 1 & YES & YES & YES & $1.12$ & $(4,1)$ & NO & 817\\
$(23,4)$ & 8 & $(17,6)$ & 7 & 1 & YES & YES & NO(2) & $1.11$ & $(6,0)$ & -- & 818\\
$(23,5)$ & 7 & $(17,7)$ & 6 & 1 & YES & YES & YES & $1.00$ & $(2,2)$ & -- & 819\\
$(23,5)$ & 7 & $(17,7)$ & 6 & 1 & YES & YES & NO(2) & $1.22$ & $(4,1)$ & NO & 820\\
$(23,7)$ & 7 & $(17,4)$ & 7 & 1 & YES & YES & NO(2) & $1.00$ & $(8,-1)$ & -- & 821\\
$(23,7)$ & 7 & $(17,7)$ & 6 & 1 & YES & YES & YES & $1.25$ & $(6,0)$ & -- & 822\\
$(23,10)$ & 7 & $(17,5)$ & 6 & 1 & YES & YES & YES & $1.44$ & $(2,2)$ & -- & 823\\
$(23,10)$ & 7 & $(17,5)$ & 6 & 1 & YES & YES & YES & $1.44$ & $(2,2)$ & NO & 824\\
$(23,5)$ & 7 & $(18,7)$ & 6 & 1 & YES & YES & YES & $1.11$ & $(2,2)$ & -- & 825\\
$(23,7)$ & 7 & $(18,7)$ & 6 & 1 & YES & YES & YES & $1.12$ & $(6,0)$ & -- & 826\\
$(23,10)$ & 7 & $(18,5)$ & 6 & 1 & YES & YES & YES & $1.33$ & $(2,2)$ & NO & 827\\
$(23,10)$ & 7 & $(18,5)$ & 6 & 1 & YES & YES & YES & $1.33$ & $(2,2)$ & -- & 828\\
$(23,7)$ & 7 & $(19,7)$ & 6 & 1 & YES & YES & YES & $1.12$ & $(6,0)$ & -- & 829\\
$(23,7)$ & 7 & $(19,8)$ & 6 & 1 & YES & YES & YES & $1.22$ & $(4,1)$ & -- & 830\\
$(23,10)$ & 7 & $(19,7)$ & 6 & 1 & YES & YES & NO(2) & $1.20$ & $(4,1)$ & NO & 831\\
$(23,5)$ & 7 & $(20,9)$ & 7 & 1 & YES & YES & NO(2) & $1.30$ & $(4,1)$ & NO & 832\\
$(23,7)$ & 7 & $(21,5)$ & 8 & 1 & YES & YES & NO(2) & $0.89$ & $(6,0)$ & -- & 833\\
$(23,7)$ & 7 & $(21,8)$ & 6 & 1 & YES & YES & YES & $1.22$ & $(4,1)$ & -- & 834\\
$(23,5)$ & 7 & $(22,9)$ & 7 & 1 & YES & YES & YES & $1.22$ & $(4,1)$ & -- & 835\\
$(23,5)$ & 7 & $(22,9)$ & 7 & 1 & YES & YES & YES & $1.22$ & $(4,1)$ & NO & 836\\
$(23,7)$ & 7 & $(23,7)$ & 7 & 23 & YES & YES & YES & $1.40$ & $(4,1)$ & -- & 837\\
$(24,7)$ & 7 & $(7,2)$ & 4 & 1 & YES & YES & YES & $0.88$ & $(4,1)$ & -- & 838\\
$(24,7)$ & 7 & $(7,2)$ & 4 & 1 & YES & YES & YES & $1.00$ & $(4,1)$ & NO & 839\\
$(24,7)$ & 7 & $(9,4)$ & 5 & 3 & YES & YES & YES & $1.33$ & $(2,2)$ & -- & 840\\
$(24,7)$ & 7 & $(9,4)$ & 5 & 3 & YES & YES & YES & $1.44$ & $(2,2)$ & NO & 841\\
$(24,7)$ & 7 & $(10,3)$ & 5 & 2 & YES & YES & YES & $1.11$ & $(2,2)$ & -- & 842\\
$(24,7)$ & 7 & $(12,5)$ & 5 & 12 & YES & YES & YES & $1.25$ & $(2,2)$ & NO & 843\\
$(24,7)$ & 7 & $(12,5)$ & 5 & 12 & YES & YES & YES & $1.25$ & $(2,2)$ & -- & 844\\
$(24,7)$ & 7 & $(13,4)$ & 6 & 1 & YES & YES & YES & $1.12$ & $(4,1)$ & -- & 845\\
$(24,7)$ & 7 & $(13,5)$ & 5 & 1 & YES & YES & YES & $1.12$ & $(2,2)$ & -- & 846\\
$(24,7)$ & 7 & $(14,3)$ & 6 & 2 & YES & YES & NO(2) & $1.20$ & $(4,1)$ & -- & 847\\
$(24,5)$ & 8 & $(16,5)$ & 7 & 8 & YES & YES & NO(2) & $1.00$ & $(6,0)$ & -- & 848\\
$(24,7)$ & 7 & $(17,5)$ & 6 & 1 & YES & YES & YES & $0.88$ & $(4,1)$ & -- & 849\\
$(24,7)$ & 7 & $(18,5)$ & 6 & 6 & YES & YES & YES & $0.88$ & $(4,1)$ & -- & 850\\
$(24,7)$ & 7 & $(19,4)$ & 7 & 1 & YES & YES & YES & $1.00$ & $(2,2)$ & -- & 851\\
$(24,5)$ & 8 & $(21,5)$ & 8 & 3 & YES & YES & YES & $1.12$ & $(2,2)$ & -- & 852\\
$(24,7)$ & 7 & $(22,5)$ & 7 & 2 & YES & YES & YES & $1.00$ & $(4,1)$ & -- & 853\\
$(24,7)$ & 7 & $(23,7)$ & 7 & 1 & YES & YES & NO(2) & $1.00$ & $(6,0)$ & NO & 854\\
$(24,5)$ & 8 & $(24,5)$ & 8 & 24 & YES & YES & YES & $1.00$ & $(2,2)$ & -- & 855\\
$(24,7)$ & 7 & $(24,7)$ & 7 & 24 & YES & YES & YES & $1.11$ & $(4,1)$ & -- & 856\\
$(25,7)$ & 7 & $(7,2)$ & 4 & 1 & YES & YES & YES & $1.00$ & $(4,1)$ & -- & 857\\
$(25,7)$ & 7 & $(9,2)$ & 5 & 1 & YES & YES & YES & $1.00$ & $(4,1)$ & -- & 858\\
$(25,7)$ & 7 & $(9,2)$ & 5 & 1 & YES & YES & YES & $1.12$ & $(4,1)$ & NO & 859\\
$(25,7)$ & 7 & $(9,4)$ & 5 & 1 & YES & YES & YES & $1.22$ & $(2,2)$ & -- & 860\\
$(25,11)$ & 7 & $(10,3)$ & 5 & 5 & YES & YES & NO(2) & $1.30$ & $(4,1)$ & NO & 861\\
$(25,11)$ & 7 & $(10,3)$ & 5 & 5 & YES & YES & NO(2) & $1.30$ & $(4,1)$ & -- & 862\\
$(25,11)$ & 7 & $(10,3)$ & 5 & 5 & YES & YES & NO(2) & $1.20$ & $(8,-1)$ & NO & 863\\
$(25,9)$ & 7 & $(11,3)$ & 5 & 1 & YES & YES & YES & $1.00$ & $(2,2)$ & -- & 864\\
$(25,7)$ & 7 & $(12,5)$ & 5 & 1 & YES & YES & YES & $1.44$ & $(2,2)$ & NO & 865\\
$(25,7)$ & 7 & $(12,5)$ & 5 & 1 & YES & YES & YES & $1.44$ & $(2,2)$ & -- & 866\\
$(25,9)$ & 7 & $(12,5)$ & 5 & 1 & YES & YES & NO(2) & $1.11$ & $(6,0)$ & -- & 867\\
$(25,7)$ & 7 & $(13,4)$ & 6 & 1 & YES & YES & YES & $1.12$ & $(4,1)$ & -- & 868\\
$(25,7)$ & 7 & $(13,4)$ & 6 & 1 & YES & YES & YES & $1.25$ & $(4,1)$ & NO & 869\\
$(25,7)$ & 7 & $(13,5)$ & 5 & 1 & YES & YES & YES & $1.12$ & $(2,2)$ & -- & 870\\
$(25,9)$ & 7 & $(13,5)$ & 5 & 1 & YES & YES & YES & $1.22$ & $(2,2)$ & -- & 871\\
$(25,6)$ & 9 & $(15,4)$ & 6 & 5 & YES & YES & NO(2) & $1.11$ & $(6,0)$ & -- & 872\\
$(25,7)$ & 7 & $(17,5)$ & 6 & 1 & YES & YES & YES & $1.12$ & $(4,1)$ & -- & 873\\
$(25,11)$ & 7 & $(17,4)$ & 7 & 1 & YES & YES & NO(2) & $0.88$ & $(8,-1)$ & NO & 874\\
$(25,11)$ & 7 & $(17,4)$ & 7 & 1 & YES & YES & NO(2) & $0.88$ & $(8,-1)$ & -- & 875\\
$(25,7)$ & 7 & $(18,5)$ & 6 & 1 & YES & YES & YES & $1.00$ & $(4,1)$ & -- & 876\\
$(25,7)$ & 7 & $(19,4)$ & 7 & 1 & YES & YES & YES & $1.12$ & $(2,2)$ & -- & 877\\
$(25,7)$ & 7 & $(22,5)$ & 7 & 1 & YES & YES & YES & $0.88$ & $(4,1)$ & -- & 878\\
$(25,4)$ & 9 & $(23,6)$ & 8 & 1 & YES & YES & NO(2) & $1.00$ & $(6,0)$ & NO & 879\\
$(25,4)$ & 9 & $(23,6)$ & 8 & 1 & YES & YES & NO(2) & $1.11$ & $(6,0)$ & -- & 880\\
$(25,6)$ & 9 & $(23,4)$ & 8 & 1 & YES & YES & YES & $1.00$ & $(4,1)$ & NO & 881\\
$(25,7)$ & 7 & $(24,7)$ & 7 & 1 & YES & YES & YES & $1.12$ & $(2,2)$ & -- & 882\\
$(25,6)$ & 9 & $(25,4)$ & 9 & 25 & YES & YES & NO(2) & $1.00$ & $(6,0)$ & NO & 883\\
$(25,7)$ & 7 & $(25,7)$ & 7 & 25 & YES & YES & YES & $1.00$ & $(2,2)$ & -- & 884\\
$(26,11)$ & 7 & $(7,2)$ & 4 & 1 & YES & YES & YES & $1.22$ & $(2,2)$ & NO & 885\\
$(26,11)$ & 7 & $(7,2)$ & 4 & 1 & YES & YES & YES & $1.33$ & $(2,2)$ & -- & 886\\
$(26,11)$ & 7 & $(8,3)$ & 4 & 2 & YES & YES & YES & $1.11$ & $(2,2)$ & -- & 887\\
$(26,11)$ & 7 & $(8,3)$ & 4 & 2 & YES & YES & YES & $1.22$ & $(2,2)$ & NO & 888\\
$(26,11)$ & 7 & $(10,3)$ & 5 & 2 & YES & YES & YES & $1.12$ & $(2,2)$ & -- & 889\\
$(26,7)$ & 7 & $(11,5)$ & 6 & 1 & YES & YES & NO(2) & $1.30$ & $(4,1)$ & NO & 890\\
$(26,7)$ & 7 & $(11,5)$ & 6 & 1 & YES & YES & NO(2) & $1.30$ & $(4,1)$ & -- & 891\\
$(26,7)$ & 7 & $(11,5)$ & 6 & 1 & YES & YES & NO(2) & $1.22$ & $(6,0)$ & NO & 892\\
$(26,11)$ & 7 & $(11,3)$ & 5 & 1 & YES & YES & YES & $0.88$ & $(4,1)$ & NO & 893\\
$(26,11)$ & 7 & $(11,3)$ & 5 & 1 & YES & YES & YES & $0.88$ & $(4,1)$ & -- & 894\\
$(26,11)$ & 7 & $(11,3)$ & 5 & 1 & YES & YES & YES & $1.22$ & $(2,2)$ & NO & 895\\
$(26,7)$ & 7 & $(13,4)$ & 6 & 13 & YES & YES & YES & $0.88$ & $(2,2)$ & -- & 896\\
$(26,11)$ & 7 & $(13,3)$ & 6 & 13 & YES & YES & YES & $1.12$ & $(2,2)$ & NO & 897\\
$(26,7)$ & 7 & $(14,5)$ & 6 & 2 & YES & YES & NO(2) & $1.11$ & $(6,0)$ & NO & 898\\
$(26,11)$ & 7 & $(14,3)$ & 6 & 2 & YES & YES & YES & $1.00$ & $(2,2)$ & NO & 899\\
$(26,7)$ & 7 & $(19,5)$ & 7 & 1 & YES & YES & NO(2) & $1.10$ & $(4,1)$ & -- & 900\\
$(26,7)$ & 7 & $(19,6)$ & 8 & 1 & YES & YES & NO(2) & $1.20$ & $(4,1)$ & NO & 901\\
$(26,7)$ & 7 & $(19,8)$ & 6 & 1 & YES & YES & YES & $1.33$ & $(4,1)$ & -- & 902\\
$(26,11)$ & 7 & $(20,9)$ & 7 & 2 & YES & YES & NO(2) & $1.40$ & $(4,1)$ & NO & 903\\
$(26,7)$ & 7 & $(21,8)$ & 6 & 1 & YES & YES & YES & $1.22$ & $(4,1)$ & -- & 904\\
$(27,10)$ & 7 & $(7,2)$ & 4 & 1 & YES & YES & YES & $1.00$ & $(2,2)$ & -- & 905\\
$(27,10)$ & 7 & $(7,2)$ & 4 & 1 & YES & YES & YES & $1.11$ & $(2,2)$ & NO & 906\\
$(27,8)$ & 7 & $(9,2)$ & 5 & 9 & YES & YES & NO(2) & $1.10$ & $(4,1)$ & NO & 907\\
$(27,8)$ & 7 & $(9,2)$ & 5 & 9 & YES & YES & NO(2) & $1.10$ & $(4,1)$ & -- & 908\\
$(27,8)$ & 7 & $(9,2)$ & 5 & 9 & YES & YES & NO(2) & $1.20$ & $(4,1)$ & NO & 909\\
$(27,10)$ & 7 & $(9,4)$ & 5 & 9 & YES & YES & NO(2) & $1.11$ & $(6,0)$ & NO & 910\\
$(27,10)$ & 7 & $(9,4)$ & 5 & 9 & YES & YES & NO(2) & $1.11$ & $(6,0)$ & -- & 911\\
$(27,8)$ & 7 & $(10,3)$ & 5 & 1 & YES & YES & NO(2) & $1.10$ & $(4,1)$ & -- & 912\\
$(27,8)$ & 7 & $(10,3)$ & 5 & 1 & YES & YES & NO(2) & $1.20$ & $(4,1)$ & NO & 913\\
$(27,10)$ & 7 & $(10,3)$ & 5 & 1 & YES & YES & YES & $1.33$ & $(2,2)$ & -- & 914\\
$(27,8)$ & 7 & $(11,3)$ & 5 & 1 & YES & YES & YES & $1.12$ & $(2,2)$ & -- & 915\\
$(27,10)$ & 7 & $(11,3)$ & 5 & 1 & YES & YES & YES & $1.11$ & $(2,2)$ & -- & 916\\
$(27,8)$ & 7 & $(12,5)$ & 5 & 3 & YES & YES & NO(2) & $1.20$ & $(4,1)$ & NO & 917\\
$(27,8)$ & 7 & $(12,5)$ & 5 & 3 & YES & YES & NO(2) & $1.20$ & $(4,1)$ & -- & 918\\
$(27,8)$ & 7 & $(13,5)$ & 5 & 1 & YES & YES & YES & $1.44$ & $(6,0)$ & -- & 919\\
$(27,11)$ & 8 & $(13,3)$ & 6 & 1 & YES & YES & NO(2) & $1.00$ & $(6,0)$ & -- & 920\\
$(27,5)$ & 8 & $(14,5)$ & 6 & 1 & YES & YES & YES & $1.11$ & $(2,2)$ & NO & 921\\
$(27,5)$ & 8 & $(14,5)$ & 6 & 1 & YES & YES & YES & $1.22$ & $(2,2)$ & -- & 922\\
$(27,10)$ & 7 & $(14,3)$ & 6 & 1 & YES & YES & NO(2) & $1.20$ & $(4,1)$ & NO & 923\\
$(27,8)$ & 7 & $(17,5)$ & 6 & 1 & YES & YES & YES & $1.00$ & $(6,0)$ & -- & 924\\
$(27,10)$ & 7 & $(17,5)$ & 6 & 1 & YES & YES & YES & $1.11$ & $(4,1)$ & -- & 925\\
$(27,8)$ & 7 & $(18,5)$ & 6 & 9 & YES & YES & YES & $0.88$ & $(6,0)$ & -- & 926\\
$(27,10)$ & 7 & $(18,5)$ & 6 & 9 & YES & YES & YES & $1.22$ & $(4,1)$ & -- & 927\\
$(27,10)$ & 7 & $(18,5)$ & 6 & 9 & YES & YES & YES & $1.33$ & $(2,2)$ & NO & 928\\
$(27,8)$ & 7 & $(19,7)$ & 6 & 1 & YES & YES & NO(2) & $1.20$ & $(4,1)$ & NO & 929\\
$(27,8)$ & 7 & $(22,5)$ & 7 & 1 & YES & YES & YES & $1.22$ & $(4,1)$ & -- & 930\\
$(27,10)$ & 7 & $(22,5)$ & 7 & 1 & YES & YES & YES & $0.88$ & $(6,0)$ & -- & 931\\
$(28,11)$ & 8 & $(7,2)$ & 4 & 7 & YES & YES & YES & $0.88$ & $(4,1)$ & -- & 932\\
$(28,11)$ & 8 & $(7,2)$ & 4 & 7 & YES & YES & NO(2) & $1.30$ & $(4,1)$ & NO & 933\\
$(28,11)$ & 8 & $(9,2)$ & 5 & 1 & YES & YES & YES & $1.33$ & $(2,2)$ & -- & 934\\
$(28,11)$ & 8 & $(9,2)$ & 5 & 1 & YES & YES & YES & $1.44$ & $(2,2)$ & NO & 935\\
$(28,11)$ & 8 & $(9,2)$ & 5 & 1 & YES & YES & YES & $1.00$ & $(4,1)$ & NO & 936\\
$(28,11)$ & 8 & $(11,2)$ & 6 & 1 & YES & YES & YES & $1.22$ & $(2,2)$ & NO & 937\\
$(28,11)$ & 8 & $(11,2)$ & 6 & 1 & YES & YES & YES & $1.22$ & $(2,2)$ & -- & 938\\
$(28,11)$ & 8 & $(13,3)$ & 6 & 1 & YES & YES & YES & $1.33$ & $(2,2)$ & -- & 939\\
$(28,11)$ & 8 & $(14,3)$ & 6 & 14 & YES & YES & YES & $1.22$ & $(2,2)$ & -- & 940\\
$(28,11)$ & 8 & $(14,3)$ & 6 & 14 & YES & YES & YES & $1.22$ & $(2,2)$ & NO & 941\\
$(28,11)$ & 8 & $(16,3)$ & 7 & 4 & YES & YES & YES & $1.22$ & $(2,2)$ & -- & 942\\
$(28,11)$ & 8 & $(16,3)$ & 7 & 4 & YES & YES & YES & $1.44$ & $(2,2)$ & NO & 943\\
$(28,11)$ & 8 & $(17,3)$ & 7 & 1 & YES & YES & YES & $1.33$ & $(2,2)$ & NO & 944\\
$(28,11)$ & 8 & $(17,3)$ & 7 & 1 & YES & YES & YES & $1.44$ & $(2,2)$ & NO & 945\\
$(28,11)$ & 8 & $(17,3)$ & 7 & 1 & YES & YES & YES & $1.44$ & $(2,2)$ & -- & 946\\
$(28,11)$ & 8 & $(21,8)$ & 6 & 7 & YES & YES & YES & $1.44$ & $(2,2)$ & NO & 947\\
$(29,8)$ & 7 & $(5,2)$ & 3 & 1 & YES & YES & YES & $1.00$ & $(4,1)$ & NO & 948\\
$(29,8)$ & 7 & $(5,2)$ & 3 & 1 & YES & YES & YES & $1.12$ & $(4,1)$ & -- & 949\\
$(29,8)$ & 7 & $(7,3)$ & 4 & 1 & YES & YES & YES & $1.11$ & $(2,2)$ & NO & 950\\
$(29,8)$ & 7 & $(7,3)$ & 4 & 1 & YES & YES & YES & $1.11$ & $(2,2)$ & -- & 951\\
$(29,9)$ & 8 & $(7,3)$ & 4 & 1 & YES & YES & NO(2) & $1.11$ & $(6,0)$ & -- & 952\\
$(29,11)$ & 7 & $(7,2)$ & 4 & 1 & YES & YES & YES & $1.11$ & $(2,2)$ & NO & 953\\
$(29,11)$ & 7 & $(7,2)$ & 4 & 1 & YES & YES & YES & $1.11$ & $(2,2)$ & -- & 954\\
$(29,11)$ & 7 & $(7,3)$ & 4 & 1 & YES & YES & YES & $1.33$ & $(2,2)$ & NO & 955\\
$(29,11)$ & 7 & $(7,3)$ & 4 & 1 & YES & YES & YES & $1.33$ & $(2,2)$ & -- & 956\\
$(29,12)$ & 7 & $(7,2)$ & 4 & 1 & YES & YES & YES & $0.88$ & $(4,1)$ & -- & 957\\
$(29,8)$ & 7 & $(8,3)$ & 4 & 1 & YES & YES & YES & $1.11$ & $(2,2)$ & -- & 958\\
$(29,8)$ & 7 & $(8,3)$ & 4 & 1 & YES & YES & YES & $1.11$ & $(2,2)$ & NO & 959\\
$(29,9)$ & 8 & $(8,3)$ & 4 & 1 & YES & YES & YES & $1.22$ & $(2,2)$ & -- & 960\\
$(29,9)$ & 8 & $(8,3)$ & 4 & 1 & YES & YES & NO(2) & $1.20$ & $(4,1)$ & NO & 961\\
$(29,12)$ & 7 & $(8,3)$ & 4 & 1 & YES & YES & YES & $1.11$ & $(2,2)$ & -- & 962\\
$(29,12)$ & 7 & $(8,3)$ & 4 & 1 & YES & YES & YES & $1.22$ & $(2,2)$ & NO & 963\\
$(29,8)$ & 7 & $(9,2)$ & 5 & 1 & YES & YES & YES & $1.00$ & $(4,1)$ & NO & 964\\
$(29,8)$ & 7 & $(9,2)$ & 5 & 1 & YES & YES & YES & $1.00$ & $(4,1)$ & -- & 965\\
$(29,8)$ & 7 & $(9,4)$ & 5 & 1 & YES & YES & YES & $1.11$ & $(2,2)$ & NO & 966\\
$(29,8)$ & 7 & $(9,4)$ & 5 & 1 & YES & YES & YES & $1.22$ & $(2,2)$ & -- & 967\\
$(29,11)$ & 7 & $(9,2)$ & 5 & 1 & YES & YES & YES & $1.00$ & $(4,1)$ & -- & 968\\
$(29,11)$ & 7 & $(9,2)$ & 5 & 1 & YES & YES & YES & $1.12$ & $(4,1)$ & NO & 969\\
$(29,12)$ & 7 & $(9,4)$ & 5 & 1 & YES & YES & NO(2) & $1.30$ & $(4,1)$ & -- & 970\\
$(29,9)$ & 8 & $(10,3)$ & 5 & 1 & YES & YES & NO(2) & $1.00$ & $(6,0)$ & -- & 971\\
$(29,11)$ & 7 & $(10,3)$ & 5 & 1 & YES & YES & YES & $1.00$ & $(4,1)$ & NO & 972\\
$(29,11)$ & 7 & $(10,3)$ & 5 & 1 & YES & YES & YES & $1.00$ & $(4,1)$ & -- & 973\\
$(29,12)$ & 7 & $(10,3)$ & 5 & 1 & YES & YES & YES & $1.11$ & $(2,2)$ & -- & 974\\
$(29,12)$ & 7 & $(10,3)$ & 5 & 1 & YES & YES & YES & $1.22$ & $(2,2)$ & NO & 975\\
$(29,13)$ & 8 & $(10,3)$ & 5 & 1 & YES & YES & NO(2) & $1.22$ & $(6,0)$ & NO & 976\\
$(29,8)$ & 7 & $(11,4)$ & 5 & 1 & YES & YES & YES & $1.00$ & $(2,2)$ & NO & 977\\
$(29,11)$ & 7 & $(11,3)$ & 5 & 1 & YES & YES & YES & $1.22$ & $(2,2)$ & -- & 978\\
$(29,13)$ & 8 & $(11,4)$ & 5 & 1 & YES & YES & NO(2) & $1.22$ & $(6,0)$ & NO & 979\\
$(29,6)$ & 9 & $(12,5)$ & 5 & 1 & YES & YES & NO(2) & $1.40$ & $(4,1)$ & NO & 980\\
$(29,6)$ & 9 & $(12,5)$ & 5 & 1 & YES & YES & NO(2) & $1.40$ & $(4,1)$ & NO & 981\\
$(29,9)$ & 8 & $(12,5)$ & 5 & 1 & YES & YES & NO(2) & $1.30$ & $(4,1)$ & NO & 982\\
$(29,12)$ & 7 & $(12,5)$ & 5 & 1 & YES & YES & YES & $1.12$ & $(6,0)$ & -- & 983\\
$(29,8)$ & 7 & $(13,5)$ & 5 & 1 & YES & YES & YES & $0.88$ & $(4,1)$ & -- & 984\\
$(29,8)$ & 7 & $(13,5)$ & 5 & 1 & YES & YES & YES & $1.12$ & $(2,2)$ & NO & 985\\
$(29,12)$ & 7 & $(13,4)$ & 6 & 1 & YES & YES & NO(2) & $1.20$ & $(4,1)$ & -- & 986\\
$(29,12)$ & 7 & $(13,5)$ & 5 & 1 & YES & YES & YES & $1.25$ & $(6,0)$ & -- & 987\\
$(29,13)$ & 8 & $(13,3)$ & 6 & 1 & YES & YES & YES & $1.12$ & $(4,1)$ & -- & 988\\
$(29,8)$ & 7 & $(16,5)$ & 7 & 1 & YES & YES & YES & $1.11$ & $(2,2)$ & NO & 989\\
$(29,12)$ & 7 & $(16,7)$ & 6 & 1 & YES & YES & YES & $1.11$ & $(2,2)$ & NO & 990\\
$(29,8)$ & 7 & $(17,5)$ & 6 & 1 & YES & YES & YES & $1.00$ & $(6,0)$ & -- & 991\\
$(29,11)$ & 7 & $(17,5)$ & 6 & 1 & YES & YES & YES & $1.12$ & $(2,2)$ & -- & 992\\
$(29,6)$ & 9 & $(18,5)$ & 6 & 1 & YES & YES & NO(2) & $1.20$ & $(4,1)$ & NO & 993\\
$(29,8)$ & 7 & $(18,5)$ & 6 & 1 & YES & YES & YES & $0.75$ & $(6,0)$ & -- & 994\\
$(29,8)$ & 7 & $(18,7)$ & 6 & 1 & YES & YES & YES & $1.25$ & $(2,2)$ & -- & 995\\
$(29,8)$ & 7 & $(18,7)$ & 6 & 1 & YES & YES & YES & $1.38$ & $(2,2)$ & NO & 996\\
$(29,11)$ & 7 & $(18,5)$ & 6 & 1 & YES & YES & YES & $1.12$ & $(2,2)$ & -- & 997\\
$(29,12)$ & 7 & $(18,7)$ & 6 & 1 & YES & YES & YES & $1.00$ & $(4,1)$ & NO & 998\\
$(29,8)$ & 7 & $(22,5)$ & 7 & 1 & YES & YES & NO(2) & $1.00$ & $(6,0)$ & NO & 999\\
$(29,8)$ & 7 & $(22,5)$ & 7 & 1 & YES & YES & YES & $1.11$ & $(4,1)$ & -- & 1000\\
$(29,11)$ & 7 & $(22,5)$ & 7 & 1 & YES & YES & YES & $1.25$ & $(2,2)$ & -- & 1001\\
$(29,11)$ & 7 & $(22,5)$ & 7 & 1 & YES & YES & YES & $1.38$ & $(2,2)$ & NO & 1002\\
$(29,8)$ & 7 & $(23,7)$ & 7 & 1 & YES & YES & YES & $1.12$ & $(2,2)$ & NO & 1003\\
$(29,11)$ & 7 & $(23,5)$ & 7 & 1 & YES & YES & YES & $1.25$ & $(2,2)$ & NO & 1004\\
$(29,11)$ & 7 & $(23,5)$ & 7 & 1 & YES & YES & YES & $1.25$ & $(2,2)$ & -- & 1005\\
$(29,11)$ & 7 & $(23,9)$ & 7 & 1 & YES & YES & YES & $1.22$ & $(2,2)$ & NO & 1006\\
$(29,12)$ & 7 & $(23,10)$ & 7 & 1 & YES & YES & NO(2) & $1.30$ & $(4,1)$ & NO & 1007\\
$(29,13)$ & 8 & $(23,10)$ & 7 & 1 & YES & YES & YES & $1.12$ & $(4,1)$ & NO & 1008\\
$(29,8)$ & 7 & $(25,7)$ & 7 & 1 & YES & YES & YES & $1.00$ & $(4,1)$ & NO & 1009\\
$(29,12)$ & 7 & $(26,11)$ & 7 & 1 & YES & YES & YES & $1.22$ & $(2,2)$ & NO & 1010\\
$(30,11)$ & 7 & $(5,2)$ & 3 & 5 & YES & YES & YES & $1.11$ & $(2,2)$ & -- & 1011\\
$(30,11)$ & 7 & $(7,2)$ & 4 & 1 & YES & YES & YES & $0.88$ & $(4,1)$ & NO & 1012\\
$(30,11)$ & 7 & $(7,2)$ & 4 & 1 & YES & YES & YES & $0.88$ & $(4,1)$ & -- & 1013\\
$(30,13)$ & 8 & $(7,2)$ & 4 & 1 & YES & YES & NO(2) & $1.00$ & $(8,-1)$ & -- & 1014\\
$(30,13)$ & 8 & $(7,2)$ & 4 & 1 & YES & YES & NO(2) & $1.20$ & $(8,-1)$ & NO & 1015\\
$(30,11)$ & 7 & $(8,3)$ & 4 & 2 & YES & YES & NO(2) & $1.11$ & $(6,0)$ & NO & 1016\\
$(30,11)$ & 7 & $(8,3)$ & 4 & 2 & YES & YES & NO(2) & $1.11$ & $(6,0)$ & -- & 1017\\
$(30,13)$ & 8 & $(8,3)$ & 4 & 2 & YES & YES & YES & $1.00$ & $(4,1)$ & -- & 1018\\
$(30,13)$ & 8 & $(8,3)$ & 4 & 2 & YES & YES & NO(2) & $1.11$ & $(6,0)$ & NO & 1019\\
$(30,11)$ & 7 & $(10,3)$ & 5 & 10 & YES & YES & YES & $1.11$ & $(2,2)$ & -- & 1020\\
$(30,11)$ & 7 & $(11,3)$ & 5 & 1 & YES & YES & YES & $1.12$ & $(2,2)$ & -- & 1021\\
$(30,11)$ & 7 & $(11,3)$ & 5 & 1 & YES & YES & YES & $1.12$ & $(4,1)$ & NO & 1022\\
$(30,13)$ & 8 & $(11,3)$ & 5 & 1 & YES & YES & NO(2) & $1.00$ & $(6,0)$ & NO & 1023\\
$(30,11)$ & 7 & $(12,5)$ & 5 & 6 & YES & YES & YES & $1.33$ & $(4,1)$ & -- & 1024\\
$(30,7)$ & 8 & $(13,4)$ & 6 & 1 & YES & YES & YES & $1.00$ & $(2,2)$ & -- & 1025\\
$(30,11)$ & 7 & $(13,5)$ & 5 & 1 & YES & YES & YES & $1.22$ & $(2,2)$ & NO & 1026\\
$(30,11)$ & 7 & $(13,5)$ & 5 & 1 & YES & YES & YES & $1.22$ & $(4,1)$ & -- & 1027\\
$(30,11)$ & 7 & $(13,5)$ & 5 & 1 & YES & YES & YES & $1.44$ & $(4,1)$ & NO & 1028\\
$(30,7)$ & 8 & $(15,4)$ & 6 & 15 & YES & YES & YES & $1.11$ & $(2,2)$ & -- & 1029\\
$(30,7)$ & 8 & $(17,5)$ & 6 & 1 & YES & YES & YES & $1.12$ & $(2,2)$ & -- & 1030\\
$(30,11)$ & 7 & $(18,5)$ & 6 & 6 & YES & YES & YES & $1.22$ & $(4,1)$ & -- & 1031\\
$(30,13)$ & 8 & $(20,9)$ & 7 & 10 & YES & YES & YES & $1.00$ & $(4,1)$ & 2290 & 1032\\
$(30,7)$ & 8 & $(21,5)$ & 8 & 3 & YES & YES & YES & $1.12$ & $(2,2)$ & -- & 1033\\
$(30,7)$ & 8 & $(30,7)$ & 8 & 30 & YES & YES & YES & $1.00$ & $(2,2)$ & -- & 1034\\
$(31,13)$ & 7 & $(5,2)$ & 3 & 1 & YES & YES & YES & $1.00$ & $(2,2)$ & -- & 1035\\
$(31,11)$ & 8 & $(7,3)$ & 4 & 1 & YES & YES & YES & $1.00$ & $(4,1)$ & -- & 1036\\
$(31,12)$ & 7 & $(7,3)$ & 4 & 1 & YES & YES & YES & $1.22$ & $(2,2)$ & NO & 1037\\
$(31,12)$ & 7 & $(7,3)$ & 4 & 1 & YES & YES & YES & $1.22$ & $(2,2)$ & -- & 1038\\
$(31,13)$ & 7 & $(7,2)$ & 4 & 1 & YES & YES & YES & $1.00$ & $(4,1)$ & NO & 1039\\
$(31,13)$ & 7 & $(7,2)$ & 4 & 1 & YES & YES & YES & $1.00$ & $(4,1)$ & -- & 1040\\
$(31,13)$ & 7 & $(7,3)$ & 4 & 1 & YES & YES & YES & $1.00$ & $(2,2)$ & -- & 1041\\
$(31,7)$ & 8 & $(8,3)$ & 4 & 1 & YES & YES & YES & $1.00$ & $(2,2)$ & NO & 1042\\
$(31,11)$ & 8 & $(8,3)$ & 4 & 1 & YES & YES & YES & $1.12$ & $(4,1)$ & -- & 1043\\
$(31,11)$ & 8 & $(8,3)$ & 4 & 1 & YES & YES & YES & $1.25$ & $(4,1)$ & NO & 1044\\
$(31,12)$ & 7 & $(8,3)$ & 4 & 1 & YES & YES & YES & $1.11$ & $(2,2)$ & NO & 1045\\
$(31,12)$ & 7 & $(8,3)$ & 4 & 1 & YES & YES & YES & $1.11$ & $(2,2)$ & -- & 1046\\
$(31,13)$ & 7 & $(8,3)$ & 4 & 1 & YES & YES & YES & $1.12$ & $(2,2)$ & -- & 1047\\
$(31,9)$ & 8 & $(10,3)$ & 5 & 1 & YES & YES & YES & $1.12$ & $(2,2)$ & -- & 1048\\
$(31,9)$ & 8 & $(10,3)$ & 5 & 1 & YES & YES & YES & $1.12$ & $(4,1)$ & NO & 1049\\
$(31,11)$ & 8 & $(10,3)$ & 5 & 1 & YES & YES & NO(2) & $1.00$ & $(6,0)$ & -- & 1050\\
$(31,12)$ & 7 & $(10,3)$ & 5 & 1 & YES & YES & YES & $1.00$ & $(2,2)$ & -- & 1051\\
$(31,13)$ & 7 & $(10,3)$ & 5 & 1 & YES & YES & NO(2) & $1.20$ & $(4,1)$ & -- & 1052\\
$(31,14)$ & 8 & $(10,3)$ & 5 & 1 & YES & YES & NO(2) & $1.30$ & $(4,1)$ & NO & 1053\\
$(31,9)$ & 8 & $(11,3)$ & 5 & 1 & YES & YES & YES & $1.12$ & $(2,2)$ & -- & 1054\\
$(31,12)$ & 7 & $(11,5)$ & 6 & 1 & YES & YES & YES & $1.12$ & $(4,1)$ & NO & 1055\\
$(31,13)$ & 7 & $(11,3)$ & 5 & 1 & YES & YES & YES & $1.12$ & $(2,2)$ & NO & 1056\\
$(31,13)$ & 7 & $(11,3)$ & 5 & 1 & YES & YES & YES & $1.33$ & $(2,2)$ & -- & 1057\\
$(31,11)$ & 8 & $(12,5)$ & 5 & 1 & YES & YES & NO(2) & $1.11$ & $(6,0)$ & NO & 1058\\
$(31,12)$ & 7 & $(12,5)$ & 5 & 1 & YES & YES & YES & $1.00$ & $(6,0)$ & -- & 1059\\
$(31,13)$ & 7 & $(12,5)$ & 5 & 1 & YES & YES & YES & $1.33$ & $(4,1)$ & -- & 1060\\
$(31,7)$ & 8 & $(13,4)$ & 6 & 1 & YES & YES & YES & $1.22$ & $(2,2)$ & -- & 1061\\
$(31,9)$ & 8 & $(13,3)$ & 6 & 1 & YES & YES & YES & $1.11$ & $(2,2)$ & -- & 1062\\
$(31,12)$ & 7 & $(13,3)$ & 6 & 1 & YES & YES & YES & $1.22$ & $(2,2)$ & NO & 1063\\
$(31,13)$ & 7 & $(13,3)$ & 6 & 1 & YES & YES & YES & $1.12$ & $(2,2)$ & NO & 1064\\
$(31,13)$ & 7 & $(13,4)$ & 6 & 1 & YES & YES & YES & $1.12$ & $(2,2)$ & -- & 1065\\
$(31,13)$ & 7 & $(13,5)$ & 5 & 1 & YES & YES & YES & $1.00$ & $(2,2)$ & NO & 1066\\
$(31,9)$ & 8 & $(14,3)$ & 6 & 1 & YES & YES & YES & $1.11$ & $(2,2)$ & NO & 1067\\
$(31,9)$ & 8 & $(16,3)$ & 7 & 1 & YES & YES & YES & $1.00$ & $(2,2)$ & -- & 1068\\
$(31,11)$ & 8 & $(16,3)$ & 7 & 1 & YES & YES & NO(2) & $1.00$ & $(6,0)$ & -- & 1069\\
$(31,7)$ & 8 & $(17,5)$ & 6 & 1 & YES & YES & YES & $1.22$ & $(2,2)$ & NO & 1070\\
$(31,9)$ & 8 & $(17,4)$ & 7 & 1 & YES & YES & YES & $1.00$ & $(6,0)$ & -- & 1071\\
$(31,12)$ & 7 & $(17,7)$ & 6 & 1 & YES & YES & YES & $1.12$ & $(4,1)$ & NO & 1072\\
$(31,7)$ & 8 & $(19,7)$ & 6 & 1 & YES & YES & YES & $1.33$ & $(2,2)$ & -- & 1073\\
$(31,9)$ & 8 & $(19,4)$ & 7 & 1 & YES & YES & YES & $1.00$ & $(6,0)$ & -- & 1074\\
$(31,13)$ & 7 & $(20,9)$ & 7 & 1 & YES & YES & NO(2) & $1.30$ & $(4,1)$ & NO & 1075\\
$(31,9)$ & 8 & $(22,5)$ & 7 & 1 & YES & YES & YES & $1.22$ & $(4,1)$ & -- & 1076\\
$(31,13)$ & 7 & $(22,9)$ & 7 & 1 & YES & YES & YES & $1.00$ & $(2,2)$ & NO & 1077\\
$(31,9)$ & 8 & $(23,7)$ & 7 & 1 & YES & YES & YES & $1.00$ & $(2,2)$ & 2354 & 1078\\
$(31,7)$ & 8 & $(25,7)$ & 7 & 1 & YES & YES & YES & $1.12$ & $(2,2)$ & -- & 1079\\
$(31,13)$ & 7 & $(25,11)$ & 7 & 1 & YES & YES & NO(2) & $1.20$ & $(4,1)$ & NO & 1080\\
$(31,12)$ & 7 & $(28,11)$ & 8 & 1 & YES & YES & YES & $1.22$ & $(2,2)$ & 1629 & 1081\\
$(31,12)$ & 7 & $(29,11)$ & 7 & 1 & YES & YES & YES & $1.11$ & $(2,2)$ & NO & 1082\\
$(32,7)$ & 8 & $(11,4)$ & 5 & 1 & YES & YES & YES & $1.11$ & $(2,2)$ & NO & 1083\\
$(32,7)$ & 8 & $(11,5)$ & 6 & 1 & YES & YES & YES & $1.00$ & $(4,1)$ & -- & 1084\\
$(32,13)$ & 9 & $(11,2)$ & 6 & 1 & YES & YES & NO(2) & $1.30$ & $(4,1)$ & NO & 1085\\
$(32,13)$ & 9 & $(11,2)$ & 6 & 1 & YES & YES & NO(2) & $1.30$ & $(4,1)$ & -- & 1086\\
$(32,7)$ & 8 & $(13,4)$ & 6 & 1 & YES & YES & YES & $1.11$ & $(2,2)$ & NO & 1087\\
$(32,7)$ & 8 & $(14,5)$ & 6 & 2 & YES & YES & YES & $1.33$ & $(2,2)$ & -- & 1088\\
$(32,9)$ & 8 & $(17,4)$ & 7 & 1 & YES & YES & YES & $1.00$ & $(6,0)$ & -- & 1089\\
$(32,9)$ & 8 & $(19,4)$ & 7 & 1 & YES & YES & YES & $1.22$ & $(4,1)$ & -- & 1090\\
$(32,7)$ & 8 & $(25,6)$ & 9 & 1 & YES & YES & YES & $1.00$ & $(4,1)$ & 2579 & 1091\\
$(33,10)$ & 8 & $(5,2)$ & 3 & 1 & YES & YES & NO(2) & $1.11$ & $(6,0)$ & NO & 1092\\
$(33,14)$ & 8 & $(5,2)$ & 3 & 1 & YES & YES & YES & $1.11$ & $(2,2)$ & -- & 1093\\
$(33,10)$ & 8 & $(7,2)$ & 4 & 1 & YES & YES & YES & $1.33$ & $(2,2)$ & -- & 1094\\
$(33,10)$ & 8 & $(7,3)$ & 4 & 1 & YES & YES & YES & $1.12$ & $(2,2)$ & -- & 1095\\
$(33,14)$ & 8 & $(8,3)$ & 4 & 1 & YES & YES & YES & $1.22$ & $(2,2)$ & NO & 1096\\
$(33,14)$ & 8 & $(8,3)$ & 4 & 1 & YES & YES & YES & $1.22$ & $(2,2)$ & -- & 1097\\
$(33,10)$ & 8 & $(9,2)$ & 5 & 3 & YES & YES & NO(2) & $1.00$ & $(6,0)$ & -- & 1098\\
$(33,10)$ & 8 & $(9,2)$ & 5 & 3 & YES & YES & NO(2) & $1.11$ & $(6,0)$ & NO & 1099\\
$(33,14)$ & 8 & $(9,2)$ & 5 & 3 & YES & YES & YES & $1.12$ & $(2,2)$ & NO & 1100\\
$(33,14)$ & 8 & $(9,2)$ & 5 & 3 & YES & YES & YES & $1.12$ & $(2,2)$ & -- & 1101\\
$(33,10)$ & 8 & $(11,3)$ & 5 & 11 & YES & YES & YES & $1.22$ & $(2,2)$ & -- & 1102\\
$(33,10)$ & 8 & $(13,3)$ & 6 & 1 & YES & YES & YES & $1.11$ & $(2,2)$ & -- & 1103\\
$(33,10)$ & 8 & $(13,4)$ & 6 & 1 & YES & YES & YES & $1.40$ & $(4,1)$ & -- & 1104\\
$(33,10)$ & 8 & $(14,3)$ & 6 & 1 & YES & YES & NO(2) & $1.10$ & $(4,1)$ & NO & 1105\\
$(33,10)$ & 8 & $(15,4)$ & 6 & 3 & YES & YES & YES & $1.22$ & $(4,1)$ & -- & 1106\\
$(33,10)$ & 8 & $(29,9)$ & 8 & 1 & YES & YES & YES & $0.88$ & $(4,1)$ & NO & 1107\\
$(34,13)$ & 7 & $(4,1)$ & 3 & 2 & YES & YES & YES & $1.00$ & $(4,1)$ & NO & 1108\\
$(34,13)$ & 7 & $(5,1)$ & 4 & 1 & YES & YES & NO(2) & $1.20$ & $(4,1)$ & NO & 1109\\
$(34,13)$ & 7 & $(5,1)$ & 4 & 1 & YES & YES & NO(2) & $1.20$ & $(4,1)$ & -- & 1110\\
$(34,13)$ & 7 & $(5,1)$ & 4 & 1 & YES & YES & NO(2) & $1.20$ & $(4,1)$ & NO & 1111\\
$(34,13)$ & 7 & $(5,2)$ & 3 & 1 & YES & YES & YES & $1.00$ & $(2,2)$ & -- & 1112\\
$(34,13)$ & 7 & $(5,2)$ & 3 & 1 & YES & YES & YES & $1.11$ & $(2,2)$ & NO & 1113\\
$(34,15)$ & 8 & $(5,2)$ & 3 & 1 & YES & YES & YES & $1.11$ & $(2,2)$ & -- & 1114\\
$(34,13)$ & 7 & $(7,2)$ & 4 & 1 & YES & YES & YES & $1.00$ & $(4,1)$ & NO & 1115\\
$(34,13)$ & 7 & $(7,2)$ & 4 & 1 & YES & YES & YES & $1.00$ & $(4,1)$ & -- & 1116\\
$(34,13)$ & 7 & $(7,2)$ & 4 & 1 & YES & YES & YES & $1.25$ & $(2,2)$ & NO & 1117\\
$(34,13)$ & 7 & $(7,3)$ & 4 & 1 & YES & YES & YES & $1.11$ & $(2,2)$ & NO & 1118\\
$(34,13)$ & 7 & $(7,3)$ & 4 & 1 & YES & YES & YES & $1.12$ & $(2,2)$ & -- & 1119\\
$(34,15)$ & 8 & $(7,2)$ & 4 & 1 & YES & YES & NO(2) & $1.30$ & $(4,1)$ & NO & 1120\\
$(34,9)$ & 8 & $(8,3)$ & 4 & 2 & YES & YES & YES & $1.11$ & $(2,2)$ & -- & 1121\\
$(34,13)$ & 7 & $(8,3)$ & 4 & 2 & YES & YES & YES & $1.11$ & $(2,2)$ & -- & 1122\\
$(34,15)$ & 8 & $(8,3)$ & 4 & 2 & YES & YES & YES & $1.22$ & $(2,2)$ & NO & 1123\\
$(34,15)$ & 8 & $(8,3)$ & 4 & 2 & YES & YES & YES & $1.22$ & $(2,2)$ & -- & 1124\\
$(34,13)$ & 7 & $(9,2)$ & 5 & 1 & YES & YES & YES & $1.00$ & $(4,1)$ & -- & 1125\\
$(34,13)$ & 7 & $(9,4)$ & 5 & 1 & YES & YES & YES & $1.11$ & $(2,2)$ & NO & 1126\\
$(34,13)$ & 7 & $(9,4)$ & 5 & 1 & YES & YES & NO(2) & $1.20$ & $(4,1)$ & -- & 1127\\
$(34,13)$ & 7 & $(10,3)$ & 5 & 2 & YES & YES & YES & $0.88$ & $(4,1)$ & -- & 1128\\
$(34,13)$ & 7 & $(10,3)$ & 5 & 2 & YES & YES & YES & $1.12$ & $(4,1)$ & NO & 1129\\
$(34,13)$ & 7 & $(11,3)$ & 5 & 1 & YES & YES & YES & $0.88$ & $(4,1)$ & -- & 1130\\
$(34,13)$ & 7 & $(11,3)$ & 5 & 1 & YES & YES & YES & $1.22$ & $(2,2)$ & NO & 1131\\
$(34,15)$ & 8 & $(11,3)$ & 5 & 1 & YES & YES & YES & $1.22$ & $(2,2)$ & NO & 1132\\
$(34,15)$ & 8 & $(11,3)$ & 5 & 1 & YES & YES & YES & $1.22$ & $(2,2)$ & -- & 1133\\
$(34,13)$ & 7 & $(12,5)$ & 5 & 2 & YES & YES & YES & $1.22$ & $(4,1)$ & -- & 1134\\
$(34,15)$ & 8 & $(12,5)$ & 5 & 2 & YES & YES & YES & $1.22$ & $(2,2)$ & 1942 & 1135\\
$(34,9)$ & 8 & $(23,5)$ & 7 & 1 & YES & YES & NO(2) & $1.20$ & $(4,1)$ & NO & 1136\\
$(34,9)$ & 8 & $(23,7)$ & 7 & 1 & YES & YES & NO(2) & $1.20$ & $(4,1)$ & NO & 1137\\
$(34,13)$ & 7 & $(23,9)$ & 7 & 1 & YES & YES & YES & $1.11$ & $(2,2)$ & NO & 1138\\
$(34,15)$ & 8 & $(23,10)$ & 7 & 1 & YES & YES & YES & $1.22$ & $(2,2)$ & 1807 & 1139\\
$(34,13)$ & 7 & $(29,11)$ & 7 & 1 & YES & YES & YES & $1.12$ & $(4,1)$ & NO & 1140\\
$(35,13)$ & 8 & $(7,2)$ & 4 & 7 & YES & YES & NO(2) & $1.00$ & $(6,0)$ & NO & 1141\\
$(35,13)$ & 8 & $(7,2)$ & 4 & 7 & YES & YES & NO(2) & $1.00$ & $(6,0)$ & -- & 1142\\
$(35,8)$ & 8 & $(9,4)$ & 5 & 1 & YES & YES & NO(2) & $1.20$ & $(4,1)$ & -- & 1143\\
$(35,13)$ & 8 & $(10,3)$ & 5 & 5 & YES & YES & NO(2) & $1.11$ & $(6,0)$ & NO & 1144\\
$(35,11)$ & 9 & $(13,3)$ & 6 & 1 & YES & YES & NO(2) & $1.20$ & $(4,1)$ & NO & 1145\\
$(35,13)$ & 8 & $(13,3)$ & 6 & 1 & YES & YES & YES & $1.00$ & $(6,0)$ & -- & 1146\\
$(35,13)$ & 8 & $(14,3)$ & 6 & 7 & YES & YES & YES & $1.00$ & $(6,0)$ & -- & 1147\\
$(35,11)$ & 9 & $(15,4)$ & 6 & 5 & YES & YES & NO(2) & $1.20$ & $(4,1)$ & NO & 1148\\
$(35,11)$ & 9 & $(17,3)$ & 7 & 1 & YES & YES & NO(2) & $1.10$ & $(4,1)$ & NO & 1149\\
$(35,13)$ & 8 & $(17,6)$ & 7 & 1 & YES & YES & NO(2) & $1.11$ & $(6,0)$ & NO & 1150\\
$(35,8)$ & 8 & $(18,7)$ & 6 & 1 & YES & YES & YES & $1.12$ & $(2,2)$ & -- & 1151\\
$(35,8)$ & 8 & $(18,7)$ & 6 & 1 & YES & YES & YES & $1.38$ & $(2,2)$ & NO & 1152\\
$(35,11)$ & 9 & $(27,8)$ & 7 & 1 & YES & YES & NO(2) & $1.10$ & $(4,1)$ & NO & 1153\\
$(35,13)$ & 8 & $(30,11)$ & 7 & 5 & YES & YES & NO(2) & $0.88$ & $(8,-1)$ & 1911 & 1154\\
$(36,11)$ & 8 & $(5,2)$ & 3 & 1 & YES & YES & NO(2) & $1.11$ & $(6,0)$ & NO & 1155\\
$(36,11)$ & 8 & $(5,2)$ & 3 & 1 & YES & YES & NO(2) & $1.11$ & $(6,0)$ & -- & 1156\\
$(36,11)$ & 8 & $(7,3)$ & 4 & 1 & YES & YES & YES & $1.12$ & $(2,2)$ & -- & 1157\\
$(36,13)$ & 8 & $(7,2)$ & 4 & 1 & YES & YES & YES & $1.11$ & $(2,2)$ & -- & 1158\\
$(36,11)$ & 8 & $(9,2)$ & 5 & 9 & YES & YES & NO(2) & $1.00$ & $(6,0)$ & NO & 1159\\
$(36,11)$ & 8 & $(9,2)$ & 5 & 9 & YES & YES & NO(2) & $1.00$ & $(6,0)$ & -- & 1160\\
$(36,13)$ & 8 & $(9,2)$ & 5 & 9 & YES & YES & NO(2) & $1.00$ & $(6,0)$ & -- & 1161\\
$(36,13)$ & 8 & $(10,3)$ & 5 & 2 & YES & YES & YES & $1.11$ & $(2,2)$ & -- & 1162\\
$(36,11)$ & 8 & $(11,3)$ & 5 & 1 & YES & YES & YES & $1.11$ & $(2,2)$ & 1786 & 1163\\
$(36,13)$ & 8 & $(11,3)$ & 5 & 1 & YES & YES & NO(2) & $1.00$ & $(6,0)$ & NO & 1164\\
$(36,13)$ & 8 & $(11,3)$ & 5 & 1 & YES & YES & YES & $1.33$ & $(2,2)$ & -- & 1165\\
$(36,11)$ & 8 & $(13,4)$ & 6 & 1 & YES & YES & YES & $1.22$ & $(4,1)$ & -- & 1166\\
$(36,13)$ & 8 & $(13,3)$ & 6 & 1 & YES & YES & YES & $1.22$ & $(2,2)$ & -- & 1167\\
$(36,11)$ & 8 & $(17,5)$ & 6 & 1 & YES & YES & YES & $1.11$ & $(2,2)$ & NO & 1168\\
$(36,11)$ & 8 & $(18,5)$ & 6 & 18 & YES & YES & YES & $1.12$ & $(2,2)$ & NO & 1169\\
$(36,11)$ & 8 & $(24,7)$ & 7 & 12 & YES & YES & YES & $1.12$ & $(2,2)$ & NO & 1170\\
$(36,13)$ & 8 & $(31,11)$ & 8 & 1 & YES & YES & NO(2) & $1.00$ & $(6,0)$ & NO & 1171\\
$(36,11)$ & 8 & $(33,10)$ & 8 & 3 & YES & YES & NO(2) & $1.00$ & $(6,0)$ & NO & 1172\\
$(37,10)$ & 8 & $(5,2)$ & 3 & 1 & YES & YES & YES & $0.75$ & $(4,1)$ & -- & 1173\\
$(37,11)$ & 8 & $(5,2)$ & 3 & 1 & YES & YES & NO(2) & $1.30$ & $(4,1)$ & -- & 1174\\
$(37,14)$ & 8 & $(5,2)$ & 3 & 1 & YES & YES & YES & $1.11$ & $(2,2)$ & -- & 1175\\
$(37,11)$ & 8 & $(7,3)$ & 4 & 1 & YES & YES & YES & $1.12$ & $(2,2)$ & -- & 1176\\
$(37,14)$ & 8 & $(7,2)$ & 4 & 1 & YES & YES & YES & $1.00$ & $(2,2)$ & -- & 1177\\
$(37,16)$ & 9 & $(7,2)$ & 4 & 1 & YES & YES & NO(2) & $1.11$ & $(6,0)$ & NO & 1178\\
$(37,16)$ & 9 & $(7,2)$ & 4 & 1 & YES & YES & NO(2) & $1.11$ & $(6,0)$ & -- & 1179\\
$(37,10)$ & 8 & $(8,3)$ & 4 & 1 & YES & YES & YES & $1.11$ & $(2,2)$ & -- & 1180\\
$(37,10)$ & 8 & $(9,4)$ & 5 & 1 & YES & YES & YES & $1.12$ & $(4,1)$ & 815 & 1181\\
$(37,10)$ & 8 & $(9,4)$ & 5 & 1 & YES & YES & YES & $1.12$ & $(4,1)$ & -- & 1182\\
$(37,14)$ & 8 & $(9,2)$ & 5 & 1 & YES & YES & YES & $1.00$ & $(2,2)$ & -- & 1183\\
$(37,14)$ & 8 & $(9,2)$ & 5 & 1 & YES & YES & YES & $1.00$ & $(2,2)$ & NO & 1184\\
$(37,16)$ & 9 & $(9,2)$ & 5 & 1 & YES & YES & YES & $1.12$ & $(2,2)$ & -- & 1185\\
$(37,16)$ & 9 & $(9,2)$ & 5 & 1 & YES & YES & YES & $1.25$ & $(2,2)$ & NO & 1186\\
$(37,10)$ & 8 & $(10,3)$ & 5 & 1 & YES & YES & YES & $1.00$ & $(2,2)$ & NO & 1187\\
$(37,11)$ & 8 & $(10,3)$ & 5 & 1 & YES & YES & YES & $1.00$ & $(4,1)$ & -- & 1188\\
$(37,8)$ & 8 & $(11,4)$ & 5 & 1 & YES & YES & YES & $1.00$ & $(2,2)$ & -- & 1189\\
$(37,11)$ & 8 & $(11,2)$ & 6 & 1 & YES & YES & NO(2) & $1.00$ & $(6,0)$ & -- & 1190\\
$(37,11)$ & 8 & $(11,2)$ & 6 & 1 & YES & YES & NO(2) & $1.11$ & $(6,0)$ & NO & 1191\\
$(37,11)$ & 8 & $(11,3)$ & 5 & 1 & YES & YES & YES & $1.00$ & $(4,1)$ & -- & 1192\\
$(37,14)$ & 8 & $(11,2)$ & 6 & 1 & YES & YES & YES & $1.12$ & $(2,2)$ & -- & 1193\\
$(37,14)$ & 8 & $(11,2)$ & 6 & 1 & YES & YES & YES & $1.22$ & $(2,2)$ & NO & 1194\\
$(37,16)$ & 9 & $(11,2)$ & 6 & 1 & YES & YES & YES & $1.12$ & $(2,2)$ & NO & 1195\\
$(37,16)$ & 9 & $(11,2)$ & 6 & 1 & YES & YES & YES & $1.25$ & $(2,2)$ & -- & 1196\\
$(37,16)$ & 9 & $(11,2)$ & 6 & 1 & YES & YES & YES & $1.25$ & $(2,2)$ & NO & 1197\\
$(37,10)$ & 8 & $(13,4)$ & 6 & 1 & YES & YES & YES & $1.22$ & $(4,1)$ & -- & 1198\\
$(37,11)$ & 8 & $(13,3)$ & 6 & 1 & YES & YES & YES & $1.00$ & $(4,1)$ & -- & 1199\\
$(37,11)$ & 8 & $(14,3)$ & 6 & 1 & YES & YES & YES & $0.88$ & $(4,1)$ & -- & 1200\\
$(37,14)$ & 8 & $(14,3)$ & 6 & 1 & YES & YES & YES & $1.12$ & $(6,0)$ & -- & 1201\\
$(37,14)$ & 8 & $(14,3)$ & 6 & 1 & YES & YES & YES & $1.12$ & $(6,0)$ & NO & 1202\\
$(37,10)$ & 8 & $(15,4)$ & 6 & 1 & YES & YES & YES & $1.22$ & $(4,1)$ & -- & 1203\\
$(37,10)$ & 8 & $(17,4)$ & 7 & 1 & YES & YES & YES & $1.12$ & $(6,0)$ & -- & 1204\\
$(37,14)$ & 8 & $(18,7)$ & 6 & 1 & YES & YES & YES & $1.22$ & $(2,2)$ & NO & 1205\\
$(37,16)$ & 9 & $(25,11)$ & 7 & 1 & YES & YES & NO(2) & $1.11$ & $(6,0)$ & NO & 1206\\
$(37,14)$ & 8 & $(34,13)$ & 7 & 1 & YES & YES & YES & $1.12$ & $(2,2)$ & 2036 & 1207\\
$(38,11)$ & 9 & $(5,2)$ & 3 & 1 & YES & YES & YES & $1.33$ & $(2,2)$ & NO & 1208\\
$(38,11)$ & 9 & $(5,2)$ & 3 & 1 & YES & YES & YES & $1.33$ & $(2,2)$ & -- & 1209\\
$(38,17)$ & 9 & $(5,2)$ & 3 & 1 & YES & YES & YES & $1.12$ & $(4,1)$ & -- & 1210\\
$(38,17)$ & 9 & $(7,2)$ & 4 & 1 & YES & YES & NO(2) & $1.30$ & $(4,1)$ & -- & 1211\\
$(38,17)$ & 9 & $(7,2)$ & 4 & 1 & YES & YES & NO(2) & $1.30$ & $(4,1)$ & NO & 1212\\
$(38,17)$ & 9 & $(7,2)$ & 4 & 1 & YES & YES & NO(2) & $1.22$ & $(6,0)$ & NO & 1213\\
$(38,17)$ & 9 & $(8,3)$ & 4 & 2 & YES & YES & YES & $1.00$ & $(4,1)$ & NO & 1214\\
$(38,17)$ & 9 & $(8,3)$ & 4 & 2 & YES & YES & NO(2) & $1.40$ & $(4,1)$ & -- & 1215\\
$(38,11)$ & 9 & $(9,2)$ & 5 & 1 & YES & YES & YES & $1.33$ & $(2,2)$ & NO & 1216\\
$(38,17)$ & 9 & $(31,14)$ & 8 & 1 & YES & YES & NO(2) & $1.30$ & $(4,1)$ & 2064 & 1217\\
$(39,11)$ & 9 & $(5,2)$ & 3 & 1 & YES & YES & YES & $1.11$ & $(2,2)$ & -- & 1218\\
$(39,11)$ & 9 & $(5,2)$ & 3 & 1 & YES & YES & YES & $1.22$ & $(2,2)$ & NO & 1219\\
$(39,14)$ & 8 & $(5,2)$ & 3 & 1 & YES & YES & NO(2) & $1.11$ & $(6,0)$ & -- & 1220\\
$(39,14)$ & 8 & $(5,2)$ & 3 & 1 & YES & YES & NO(2) & $1.20$ & $(8,-1)$ & NO & 1221\\
$(39,16)$ & 8 & $(5,2)$ & 3 & 1 & YES & YES & YES & $1.12$ & $(2,2)$ & -- & 1222\\
$(39,14)$ & 8 & $(7,3)$ & 4 & 1 & YES & YES & NO(2) & $1.11$ & $(6,0)$ & -- & 1223\\
$(39,14)$ & 8 & $(7,3)$ & 4 & 1 & YES & YES & NO(2) & $1.20$ & $(8,-1)$ & NO & 1224\\
$(39,16)$ & 8 & $(7,2)$ & 4 & 1 & YES & YES & YES & $0.88$ & $(2,2)$ & -- & 1225\\
$(39,17)$ & 8 & $(7,2)$ & 4 & 1 & YES & YES & NO(2) & $1.10$ & $(4,1)$ & -- & 1226\\
$(39,17)$ & 8 & $(7,2)$ & 4 & 1 & YES & YES & NO(2) & $1.00$ & $(10,-2)$ & NO & 1227\\
$(39,17)$ & 8 & $(7,3)$ & 4 & 1 & YES & YES & YES & $1.12$ & $(2,2)$ & -- & 1228\\
$(39,14)$ & 8 & $(8,3)$ & 4 & 1 & YES & YES & YES & $1.11$ & $(2,2)$ & -- & 1229\\
$(39,17)$ & 8 & $(8,3)$ & 4 & 1 & YES & YES & YES & $0.88$ & $(4,1)$ & NO & 1230\\
$(39,17)$ & 8 & $(8,3)$ & 4 & 1 & YES & YES & YES & $1.33$ & $(2,2)$ & -- & 1231\\
$(39,14)$ & 8 & $(9,2)$ & 5 & 3 & YES & YES & NO(2) & $1.00$ & $(6,0)$ & -- & 1232\\
$(39,17)$ & 8 & $(10,3)$ & 5 & 1 & YES & YES & NO(2) & $1.11$ & $(6,0)$ & NO & 1233\\
$(39,14)$ & 8 & $(11,3)$ & 5 & 1 & YES & YES & NO(2) & $1.20$ & $(4,1)$ & NO & 1234\\
$(39,11)$ & 9 & $(13,2)$ & 7 & 13 & YES & YES & NO(2) & $1.20$ & $(4,1)$ & NO & 1235\\
$(39,16)$ & 8 & $(13,3)$ & 6 & 13 & YES & YES & NO(2) & $1.20$ & $(4,1)$ & NO & 1236\\
$(39,16)$ & 8 & $(13,3)$ & 6 & 13 & YES & YES & NO(2) & $1.20$ & $(4,1)$ & -- & 1237\\
$(39,16)$ & 8 & $(13,5)$ & 5 & 13 & YES & YES & NO(2) & $1.22$ & $(6,0)$ & NO & 1238\\
$(39,16)$ & 8 & $(16,7)$ & 6 & 1 & YES & YES & NO(2) & $1.30$ & $(4,1)$ & NO & 1239\\
$(39,16)$ & 8 & $(19,8)$ & 6 & 1 & YES & YES & YES & $0.88$ & $(2,2)$ & NO & 1240\\
$(39,17)$ & 8 & $(20,9)$ & 7 & 1 & YES & YES & NO(2) & $1.11$ & $(6,0)$ & NO & 1241\\
$(39,14)$ & 8 & $(27,10)$ & 7 & 3 & YES & YES & YES & $1.22$ & $(2,2)$ & NO & 1242\\
$(39,16)$ & 8 & $(29,12)$ & 7 & 1 & YES & YES & YES & $1.00$ & $(2,2)$ & NO & 1243\\
$(39,14)$ & 8 & $(31,11)$ & 8 & 1 & YES & YES & NO(2) & $1.11$ & $(6,0)$ & NO & 1244\\
$(39,16)$ & 8 & $(32,13)$ & 9 & 1 & YES & YES & NO(2) & $1.30$ & $(4,1)$ & 1816 & 1245\\
$(39,17)$ & 8 & $(34,15)$ & 8 & 1 & YES & YES & YES & $1.22$ & $(2,2)$ & NO & 1246\\
$(39,14)$ & 8 & $(36,13)$ & 8 & 3 & YES & YES & NO(2) & $1.00$ & $(6,0)$ & NO & 1247\\
$(39,17)$ & 8 & $(37,16)$ & 9 & 1 & YES & YES & YES & $1.12$ & $(2,2)$ & 2066 & 1248\\
$(40,11)$ & 8 & $(4,1)$ & 3 & 4 & YES & YES & YES & $1.00$ & $(4,1)$ & NO & 1249\\
$(40,11)$ & 8 & $(4,1)$ & 3 & 4 & YES & YES & YES & $1.00$ & $(4,1)$ & -- & 1250\\
$(40,11)$ & 8 & $(8,3)$ & 4 & 8 & YES & YES & YES & $1.11$ & $(2,2)$ & -- & 1251\\
$(40,11)$ & 8 & $(8,3)$ & 4 & 8 & YES & YES & YES & $1.22$ & $(2,2)$ & NO & 1252\\
$(40,11)$ & 8 & $(8,3)$ & 4 & 8 & YES & YES & YES & $1.00$ & $(4,1)$ & NO & 1253\\
$(40,9)$ & 9 & $(9,4)$ & 5 & 1 & YES & YES & NO(2) & $1.00$ & $(6,0)$ & -- & 1254\\
$(40,11)$ & 8 & $(10,3)$ & 5 & 10 & YES & YES & YES & $1.22$ & $(6,0)$ & -- & 1255\\
$(40,11)$ & 8 & $(11,3)$ & 5 & 1 & YES & YES & YES & $1.00$ & $(4,1)$ & -- & 1256\\
$(40,11)$ & 8 & $(13,3)$ & 6 & 1 & YES & YES & YES & $0.88$ & $(4,1)$ & -- & 1257\\
$(40,11)$ & 8 & $(13,4)$ & 6 & 1 & YES & YES & YES & $1.11$ & $(2,2)$ & NO & 1258\\
$(40,9)$ & 9 & $(17,4)$ & 7 & 1 & YES & YES & YES & $1.11$ & $(4,1)$ & -- & 1259\\
$(40,9)$ & 9 & $(19,4)$ & 7 & 1 & YES & YES & YES & $1.22$ & $(4,1)$ & -- & 1260\\
$(40,11)$ & 8 & $(23,6)$ & 8 & 1 & YES & YES & NO(2) & $1.11$ & $(6,0)$ & NO & 1261\\
$(40,9)$ & 9 & $(25,6)$ & 9 & 5 & YES & YES & NO(2) & $1.00$ & $(6,0)$ & NO & 1262\\
$(40,11)$ & 8 & $(29,8)$ & 7 & 1 & YES & YES & YES & $1.00$ & $(4,1)$ & NO & 1263\\
$(40,11)$ & 8 & $(37,10)$ & 8 & 1 & YES & YES & NO(2) & $1.11$ & $(6,0)$ & NO & 1264\\
$(41,12)$ & 8 & $(4,1)$ & 3 & 1 & YES & YES & YES & $0.88$ & $(4,1)$ & NO & 1265\\
$(41,12)$ & 8 & $(4,1)$ & 3 & 1 & YES & YES & YES & $0.88$ & $(4,1)$ & -- & 1266\\
$(41,16)$ & 8 & $(4,1)$ & 3 & 1 & YES & YES & YES & $1.33$ & $(2,2)$ & NO & 1267\\
$(41,16)$ & 8 & $(4,1)$ & 3 & 1 & YES & YES & NO(2) & $1.30$ & $(4,1)$ & -- & 1268\\
$(41,11)$ & 8 & $(5,2)$ & 3 & 1 & YES & YES & YES & $1.11$ & $(2,2)$ & -- & 1269\\
$(41,11)$ & 8 & $(5,2)$ & 3 & 1 & YES & YES & NO(2) & $1.00$ & $(6,0)$ & NO & 1270\\
$(41,15)$ & 8 & $(5,1)$ & 4 & 1 & YES & YES & YES & $1.11$ & $(2,2)$ & -- & 1271\\
$(41,15)$ & 8 & $(5,1)$ & 4 & 1 & YES & YES & YES & $1.22$ & $(2,2)$ & NO & 1272\\
$(41,17)$ & 8 & $(5,2)$ & 3 & 1 & YES & YES & YES & $1.22$ & $(2,2)$ & -- & 1273\\
$(41,17)$ & 8 & $(5,2)$ & 3 & 1 & YES & YES & YES & $1.33$ & $(2,2)$ & NO & 1274\\
$(41,18)$ & 8 & $(5,2)$ & 3 & 1 & YES & YES & NO(2) & $1.20$ & $(4,1)$ & -- & 1275\\
$(41,12)$ & 8 & $(7,3)$ & 4 & 1 & YES & YES & YES & $1.12$ & $(2,2)$ & -- & 1276\\
$(41,15)$ & 8 & $(7,2)$ & 4 & 1 & YES & YES & NO(2) & $1.30$ & $(4,1)$ & -- & 1277\\
$(41,16)$ & 8 & $(7,2)$ & 4 & 1 & YES & YES & YES & $1.12$ & $(2,2)$ & -- & 1278\\
$(41,16)$ & 8 & $(7,2)$ & 4 & 1 & YES & YES & YES & $1.00$ & $(2,2)$ & NO & 1279\\
$(41,17)$ & 8 & $(7,2)$ & 4 & 1 & YES & YES & YES & $1.25$ & $(4,1)$ & NO & 1280\\
$(41,17)$ & 8 & $(7,2)$ & 4 & 1 & YES & YES & YES & $1.33$ & $(2,2)$ & -- & 1281\\
$(41,17)$ & 8 & $(7,3)$ & 4 & 1 & YES & YES & YES & $1.00$ & $(4,1)$ & -- & 1282\\
$(41,18)$ & 8 & $(7,3)$ & 4 & 1 & YES & YES & NO(2) & $1.40$ & $(4,1)$ & -- & 1283\\
$(41,12)$ & 8 & $(8,3)$ & 4 & 1 & YES & YES & YES & $1.33$ & $(2,2)$ & NO & 1284\\
$(41,17)$ & 8 & $(8,3)$ & 4 & 1 & YES & YES & YES & $1.33$ & $(2,2)$ & -- & 1285\\
$(41,18)$ & 8 & $(8,3)$ & 4 & 1 & YES & YES & NO(2) & $0.89$ & $(6,0)$ & -- & 1286\\
$(41,11)$ & 8 & $(9,4)$ & 5 & 1 & YES & YES & NO(2) & $1.11$ & $(6,0)$ & -- & 1287\\
$(41,16)$ & 8 & $(9,2)$ & 5 & 1 & YES & YES & YES & $1.33$ & $(2,2)$ & NO & 1288\\
$(41,16)$ & 8 & $(9,2)$ & 5 & 1 & YES & YES & YES & $1.33$ & $(2,2)$ & -- & 1289\\
$(41,16)$ & 8 & $(9,2)$ & 5 & 1 & YES & YES & YES & $0.88$ & $(4,1)$ & NO & 1290\\
$(41,17)$ & 8 & $(9,4)$ & 5 & 1 & YES & YES & YES & $1.22$ & $(2,2)$ & NO & 1291\\
$(41,11)$ & 8 & $(10,3)$ & 5 & 1 & YES & YES & YES & $1.00$ & $(2,2)$ & 2048 & 1292\\
$(41,12)$ & 8 & $(10,3)$ & 5 & 1 & YES & YES & YES & $1.11$ & $(4,1)$ & -- & 1293\\
$(41,12)$ & 8 & $(10,3)$ & 5 & 1 & YES & YES & YES & $1.12$ & $(4,1)$ & NO & 1294\\
$(41,11)$ & 8 & $(11,4)$ & 5 & 1 & YES & YES & YES & $1.33$ & $(4,1)$ & -- & 1295\\
$(41,12)$ & 8 & $(11,3)$ & 5 & 1 & YES & YES & YES & $1.11$ & $(6,0)$ & -- & 1296\\
$(41,12)$ & 8 & $(11,4)$ & 5 & 1 & YES & YES & YES & $1.12$ & $(6,0)$ & -- & 1297\\
$(41,15)$ & 8 & $(11,3)$ & 5 & 1 & YES & YES & YES & $1.12$ & $(6,0)$ & -- & 1298\\
$(41,15)$ & 8 & $(11,3)$ & 5 & 1 & YES & YES & YES & $1.22$ & $(2,2)$ & NO & 1299\\
$(41,17)$ & 8 & $(11,3)$ & 5 & 1 & YES & YES & YES & $1.33$ & $(2,2)$ & NO & 1300\\
$(41,16)$ & 8 & $(12,5)$ & 5 & 1 & YES & YES & NO(2) & $1.11$ & $(6,0)$ & NO & 1301\\
$(41,18)$ & 8 & $(12,5)$ & 5 & 1 & YES & YES & NO(2) & $1.20$ & $(4,1)$ & NO & 1302\\
$(41,11)$ & 8 & $(13,4)$ & 6 & 1 & YES & YES & YES & $1.33$ & $(4,1)$ & -- & 1303\\
$(41,12)$ & 8 & $(13,5)$ & 5 & 1 & YES & YES & YES & $1.11$ & $(4,1)$ & -- & 1304\\
$(41,16)$ & 8 & $(13,3)$ & 6 & 1 & YES & YES & YES & $1.22$ & $(2,2)$ & NO & 1305\\
$(41,16)$ & 8 & $(13,3)$ & 6 & 1 & YES & YES & YES & $1.33$ & $(2,2)$ & -- & 1306\\
$(41,12)$ & 8 & $(17,4)$ & 7 & 1 & YES & YES & YES & $1.00$ & $(6,0)$ & -- & 1307\\
$(41,15)$ & 8 & $(17,6)$ & 7 & 1 & YES & YES & NO(2) & $1.11$ & $(6,0)$ & NO & 1308\\
$(41,17)$ & 8 & $(17,7)$ & 6 & 1 & YES & YES & YES & $1.11$ & $(2,2)$ & 1443 & 1309\\
$(41,12)$ & 8 & $(18,5)$ & 6 & 1 & YES & YES & YES & $1.11$ & $(2,2)$ & NO & 1310\\
$(41,9)$ & 9 & $(19,4)$ & 7 & 1 & YES & YES & YES & $1.00$ & $(6,0)$ & -- & 1311\\
$(41,18)$ & 8 & $(19,8)$ & 6 & 1 & YES & YES & NO(2) & $1.30$ & $(4,1)$ & NO & 1312\\
$(41,16)$ & 8 & $(21,8)$ & 6 & 1 & YES & YES & YES & $1.00$ & $(2,2)$ & NO & 1313\\
$(41,12)$ & 8 & $(22,5)$ & 7 & 1 & YES & YES & YES & $1.00$ & $(4,1)$ & -- & 1314\\
$(41,12)$ & 8 & $(23,7)$ & 7 & 1 & YES & YES & YES & $1.12$ & $(2,2)$ & NO & 1315\\
$(41,18)$ & 8 & $(23,10)$ & 7 & 1 & YES & YES & YES & $0.88$ & $(4,1)$ & NO & 1316\\
$(41,12)$ & 8 & $(24,7)$ & 7 & 1 & YES & YES & YES & $0.88$ & $(4,1)$ & NO & 1317\\
$(41,16)$ & 8 & $(28,11)$ & 8 & 1 & YES & YES & YES & $1.33$ & $(2,2)$ & NO & 1318\\
$(41,17)$ & 8 & $(29,12)$ & 7 & 1 & YES & YES & YES & $1.11$ & $(2,2)$ & NO & 1319\\
$(41,18)$ & 8 & $(30,13)$ & 8 & 1 & YES & YES & NO(2) & $1.00$ & $(6,0)$ & NO & 1320\\
$(41,18)$ & 8 & $(39,17)$ & 8 & 1 & YES & YES & NO(2) & $1.00$ & $(6,0)$ & NO & 1321\\
$(41,15)$ & 8 & $(41,15)$ & 8 & 41 & YES & YES & YES & $1.11$ & $(2,2)$ & NO & 1322\\
$(41,17)$ & 8 & $(41,17)$ & 8 & 41 & YES & YES & YES & $1.11$ & $(2,2)$ & NO & 1323\\
$(42,13)$ & 9 & $(5,2)$ & 3 & 1 & YES & YES & NO(2) & $1.11$ & $(6,0)$ & NO & 1324\\
$(42,13)$ & 9 & $(5,2)$ & 3 & 1 & YES & YES & NO(2) & $1.11$ & $(6,0)$ & -- & 1325\\
$(42,11)$ & 9 & $(7,3)$ & 4 & 7 & YES & YES & NO(2) & $1.22$ & $(6,0)$ & NO & 1326\\
$(42,11)$ & 9 & $(7,3)$ & 4 & 7 & YES & YES & NO(2) & $1.22$ & $(6,0)$ & -- & 1327\\
$(42,13)$ & 9 & $(7,2)$ & 4 & 7 & YES & YES & NO(2) & $1.20$ & $(4,1)$ & -- & 1328\\
$(42,13)$ & 9 & $(7,3)$ & 4 & 7 & YES & YES & NO(2) & $1.11$ & $(6,0)$ & NO & 1329\\
$(42,13)$ & 9 & $(8,3)$ & 4 & 2 & YES & YES & YES & $1.22$ & $(2,2)$ & -- & 1330\\
$(42,13)$ & 9 & $(8,3)$ & 4 & 2 & YES & YES & YES & $1.33$ & $(2,2)$ & NO & 1331\\
$(42,13)$ & 9 & $(11,3)$ & 5 & 1 & YES & YES & NO(2) & $1.00$ & $(6,0)$ & NO & 1332\\
$(42,11)$ & 9 & $(17,3)$ & 7 & 1 & YES & YES & NO(2) & $1.20$ & $(4,1)$ & NO & 1333\\
$(42,13)$ & 9 & $(18,5)$ & 6 & 6 & YES & YES & YES & $1.33$ & $(2,2)$ & NO & 1334\\
$(42,13)$ & 9 & $(23,7)$ & 7 & 1 & YES & YES & NO(2) & $1.11$ & $(6,0)$ & NO & 1335\\
$(43,18)$ & 8 & $(4,1)$ & 3 & 1 & YES & YES & YES & $1.00$ & $(2,2)$ & NO & 1336\\
$(43,18)$ & 8 & $(4,1)$ & 3 & 1 & YES & YES & NO(2) & $0.75$ & $(8,-1)$ & -- & 1337\\
$(43,13)$ & 9 & $(5,2)$ & 3 & 1 & YES & YES & YES & $1.12$ & $(2,2)$ & -- & 1338\\
$(43,18)$ & 8 & $(5,2)$ & 3 & 1 & YES & YES & YES & $1.22$ & $(2,2)$ & -- & 1339\\
$(43,12)$ & 8 & $(7,2)$ & 4 & 1 & YES & YES & YES & $1.22$ & $(2,2)$ & -- & 1340\\
$(43,12)$ & 8 & $(7,2)$ & 4 & 1 & YES & YES & YES & $1.33$ & $(2,2)$ & NO & 1341\\
$(43,12)$ & 8 & $(7,3)$ & 4 & 1 & YES & YES & NO(2) & $1.20$ & $(4,1)$ & -- & 1342\\
$(43,12)$ & 8 & $(7,3)$ & 4 & 1 & YES & YES & YES & $1.33$ & $(2,2)$ & NO & 1343\\
$(43,18)$ & 8 & $(7,2)$ & 4 & 1 & YES & YES & NO(2) & $1.20$ & $(4,1)$ & -- & 1344\\
$(43,18)$ & 8 & $(7,2)$ & 4 & 1 & YES & YES & YES & $1.00$ & $(4,1)$ & NO & 1345\\
$(43,18)$ & 8 & $(7,3)$ & 4 & 1 & YES & YES & NO(2) & $1.20$ & $(4,1)$ & -- & 1346\\
$(43,12)$ & 8 & $(8,3)$ & 4 & 1 & YES & YES & NO(2) & $1.20$ & $(4,1)$ & -- & 1347\\
$(43,13)$ & 9 & $(8,3)$ & 4 & 1 & YES & YES & NO(2) & $1.11$ & $(6,0)$ & NO & 1348\\
$(43,18)$ & 8 & $(8,3)$ & 4 & 1 & YES & YES & YES & $1.22$ & $(2,2)$ & NO & 1349\\
$(43,18)$ & 8 & $(8,3)$ & 4 & 1 & YES & YES & NO(2) & $1.20$ & $(4,1)$ & -- & 1350\\
$(43,10)$ & 9 & $(9,4)$ & 5 & 1 & YES & YES & NO(2) & $1.11$ & $(6,0)$ & NO & 1351\\
$(43,10)$ & 9 & $(9,4)$ & 5 & 1 & YES & YES & NO(2) & $1.11$ & $(6,0)$ & -- & 1352\\
$(43,18)$ & 8 & $(9,4)$ & 5 & 1 & YES & YES & YES & $1.11$ & $(2,2)$ & 1690 & 1353\\
$(43,12)$ & 8 & $(10,3)$ & 5 & 1 & YES & YES & YES & $0.88$ & $(4,1)$ & -- & 1354\\
$(43,18)$ & 8 & $(10,3)$ & 5 & 1 & YES & YES & YES & $1.33$ & $(4,1)$ & -- & 1355\\
$(43,12)$ & 8 & $(11,3)$ & 5 & 1 & YES & YES & YES & $1.22$ & $(6,0)$ & -- & 1356\\
$(43,12)$ & 8 & $(11,4)$ & 5 & 1 & YES & YES & YES & $1.00$ & $(6,0)$ & -- & 1357\\
$(43,12)$ & 8 & $(11,4)$ & 5 & 1 & YES & YES & YES & $1.44$ & $(2,2)$ & NO & 1358\\
$(43,13)$ & 9 & $(11,3)$ & 5 & 1 & YES & YES & YES & $1.12$ & $(2,2)$ & NO & 1359\\
$(43,9)$ & 9 & $(12,5)$ & 5 & 1 & YES & YES & NO(2) & $1.20$ & $(4,1)$ & NO & 1360\\
$(43,10)$ & 9 & $(15,4)$ & 6 & 1 & YES & YES & YES & $1.11$ & $(4,1)$ & -- & 1361\\
$(43,13)$ & 9 & $(16,5)$ & 7 & 1 & YES & YES & NO(2) & $1.00$ & $(6,0)$ & NO & 1362\\
$(43,18)$ & 8 & $(16,7)$ & 6 & 1 & YES & YES & NO(2) & $1.30$ & $(4,1)$ & NO & 1363\\
$(43,10)$ & 9 & $(17,4)$ & 7 & 1 & YES & YES & YES & $0.75$ & $(6,0)$ & -- & 1364\\
$(43,12)$ & 8 & $(17,4)$ & 7 & 1 & YES & YES & YES & $1.11$ & $(4,1)$ & -- & 1365\\
$(43,10)$ & 9 & $(18,5)$ & 6 & 1 & YES & YES & YES & $1.11$ & $(4,1)$ & -- & 1366\\
$(43,18)$ & 8 & $(19,8)$ & 6 & 1 & YES & YES & YES & $1.11$ & $(2,2)$ & 1598 & 1367\\
$(43,10)$ & 9 & $(22,5)$ & 7 & 1 & YES & YES & YES & $0.75$ & $(6,0)$ & -- & 1368\\
$(43,10)$ & 9 & $(23,5)$ & 7 & 1 & YES & YES & YES & $1.22$ & $(2,2)$ & NO & 1369\\
$(43,12)$ & 8 & $(25,7)$ & 7 & 1 & YES & YES & YES & $1.00$ & $(4,1)$ & NO & 1370\\
$(43,18)$ & 8 & $(26,11)$ & 7 & 1 & YES & YES & YES & $1.12$ & $(2,2)$ & 2247 & 1371\\
$(43,13)$ & 9 & $(36,11)$ & 8 & 1 & YES & YES & YES & $1.00$ & $(2,2)$ & 2321 & 1372\\
$(43,12)$ & 8 & $(39,11)$ & 9 & 1 & YES & YES & NO(2) & $1.20$ & $(4,1)$ & 2182 & 1373\\
$(44,17)$ & 8 & $(4,1)$ & 3 & 4 & YES & YES & YES & $1.22$ & $(2,2)$ & -- & 1374\\
$(44,13)$ & 8 & $(5,2)$ & 3 & 1 & YES & YES & YES & $1.22$ & $(2,2)$ & NO & 1375\\
$(44,13)$ & 8 & $(5,2)$ & 3 & 1 & YES & YES & YES & $1.22$ & $(2,2)$ & -- & 1376\\
$(44,17)$ & 8 & $(5,2)$ & 3 & 1 & YES & YES & YES & $1.11$ & $(2,2)$ & -- & 1377\\
$(44,13)$ & 8 & $(7,3)$ & 4 & 1 & YES & YES & NO(2) & $0.89$ & $(6,0)$ & -- & 1378\\
$(44,17)$ & 8 & $(7,2)$ & 4 & 1 & YES & YES & YES & $1.00$ & $(2,2)$ & -- & 1379\\
$(44,17)$ & 8 & $(7,3)$ & 4 & 1 & YES & YES & YES & $1.44$ & $(2,2)$ & NO & 1380\\
$(44,17)$ & 8 & $(7,3)$ & 4 & 1 & YES & YES & YES & $1.44$ & $(2,2)$ & -- & 1381\\
$(44,13)$ & 8 & $(8,3)$ & 4 & 4 & YES & YES & YES & $0.88$ & $(4,1)$ & -- & 1382\\
$(44,13)$ & 8 & $(8,3)$ & 4 & 4 & YES & YES & YES & $1.00$ & $(4,1)$ & NO & 1383\\
$(44,17)$ & 8 & $(9,4)$ & 5 & 1 & YES & YES & YES & $1.12$ & $(4,1)$ & NO & 1384\\
$(44,13)$ & 8 & $(10,3)$ & 5 & 2 & YES & YES & YES & $1.00$ & $(6,0)$ & -- & 1385\\
$(44,13)$ & 8 & $(11,3)$ & 5 & 11 & YES & YES & YES & $1.00$ & $(6,0)$ & -- & 1386\\
$(44,13)$ & 8 & $(15,4)$ & 6 & 1 & YES & YES & YES & $1.00$ & $(2,2)$ & NO & 1387\\
$(44,13)$ & 8 & $(18,5)$ & 6 & 2 & YES & YES & YES & $1.00$ & $(2,2)$ & NO & 1388\\
$(44,17)$ & 8 & $(18,7)$ & 6 & 2 & YES & YES & YES & $1.11$ & $(2,2)$ & 1557 & 1389\\
$(44,17)$ & 8 & $(28,11)$ & 8 & 4 & YES & YES & YES & $1.44$ & $(2,2)$ & NO & 1390\\
$(44,13)$ & 8 & $(31,9)$ & 8 & 1 & YES & YES & YES & $1.00$ & $(2,2)$ & NO & 1391\\
$(44,13)$ & 8 & $(41,12)$ & 8 & 1 & YES & YES & YES & $1.00$ & $(2,2)$ & NO & 1392\\
$(45,17)$ & 9 & $(4,1)$ & 3 & 1 & YES & YES & YES & $1.22$ & $(2,2)$ & -- & 1393\\
$(45,19)$ & 8 & $(4,1)$ & 3 & 1 & YES & YES & NO(2) & $1.30$ & $(4,1)$ & NO & 1394\\
$(45,19)$ & 8 & $(4,1)$ & 3 & 1 & YES & YES & NO(2) & $1.40$ & $(4,1)$ & NO & 1395\\
$(45,16)$ & 9 & $(5,2)$ & 3 & 5 & YES & YES & NO(2) & $1.11$ & $(6,0)$ & NO & 1396\\
$(45,16)$ & 9 & $(5,2)$ & 3 & 5 & YES & YES & NO(2) & $1.11$ & $(6,0)$ & -- & 1397\\
$(45,17)$ & 9 & $(5,1)$ & 4 & 5 & YES & YES & YES & $1.22$ & $(2,2)$ & NO & 1398\\
$(45,17)$ & 9 & $(5,1)$ & 4 & 5 & YES & YES & YES & $1.00$ & $(4,1)$ & -- & 1399\\
$(45,19)$ & 8 & $(5,2)$ & 3 & 5 & YES & YES & YES & $1.12$ & $(2,2)$ & -- & 1400\\
$(45,19)$ & 8 & $(5,2)$ & 3 & 5 & YES & YES & NO(2) & $1.30$ & $(4,1)$ & NO & 1401\\
$(45,17)$ & 9 & $(6,1)$ & 5 & 3 & YES & YES & YES & $1.33$ & $(2,2)$ & NO & 1402\\
$(45,16)$ & 9 & $(7,3)$ & 4 & 1 & YES & YES & YES & $1.00$ & $(4,1)$ & NO & 1403\\
$(45,19)$ & 8 & $(7,2)$ & 4 & 1 & YES & YES & YES & $1.00$ & $(2,2)$ & NO & 1404\\
$(45,19)$ & 8 & $(7,2)$ & 4 & 1 & YES & YES & NO(2) & $1.30$ & $(4,1)$ & -- & 1405\\
$(45,19)$ & 8 & $(7,2)$ & 4 & 1 & YES & YES & NO(2) & $1.00$ & $(10,-2)$ & NO & 1406\\
$(45,19)$ & 8 & $(7,3)$ & 4 & 1 & YES & YES & YES & $1.25$ & $(6,0)$ & -- & 1407\\
$(45,19)$ & 8 & $(8,3)$ & 4 & 1 & YES & YES & YES & $1.11$ & $(2,2)$ & NO & 1408\\
$(45,19)$ & 8 & $(9,2)$ & 5 & 9 & YES & YES & YES & $1.00$ & $(2,2)$ & NO & 1409\\
$(45,19)$ & 8 & $(9,2)$ & 5 & 9 & YES & YES & YES & $1.00$ & $(2,2)$ & -- & 1410\\
$(45,14)$ & 9 & $(11,3)$ & 5 & 1 & YES & YES & NO(2) & $1.20$ & $(4,1)$ & NO & 1411\\
$(45,19)$ & 8 & $(17,7)$ & 6 & 1 & YES & YES & YES & $1.00$ & $(2,2)$ & NO & 1412\\
$(45,17)$ & 9 & $(21,8)$ & 6 & 3 & YES & YES & YES & $1.33$ & $(2,2)$ & NO & 1413\\
$(45,14)$ & 9 & $(23,7)$ & 7 & 1 & YES & YES & NO(2) & $1.20$ & $(4,1)$ & NO & 1414\\
$(45,16)$ & 9 & $(25,9)$ & 7 & 5 & YES & YES & YES & $1.00$ & $(4,1)$ & NO & 1415\\
$(45,17)$ & 9 & $(29,11)$ & 7 & 1 & YES & YES & YES & $1.22$ & $(2,2)$ & 2035 & 1416\\
$(45,19)$ & 8 & $(31,13)$ & 7 & 1 & YES & YES & YES & $1.12$ & $(2,2)$ & NO & 1417\\
$(45,19)$ & 8 & $(33,14)$ & 8 & 3 & YES & YES & YES & $1.12$ & $(2,2)$ & NO & 1418\\
$(45,17)$ & 9 & $(37,14)$ & 8 & 1 & YES & YES & YES & $1.22$ & $(2,2)$ & NO & 1419\\
$(46,19)$ & 8 & $(2,1)$ & 1 & 2 & YES & YES & YES & $1.00$ & $(2,2)$ & -- & 1420\\
$(46,17)$ & 8 & $(3,1)$ & 2 & 1 & YES & YES & YES & $1.11$ & $(2,2)$ & NO & 1421\\
$(46,17)$ & 8 & $(3,1)$ & 2 & 1 & YES & YES & YES & $1.11$ & $(2,2)$ & -- & 1422\\
$(46,19)$ & 8 & $(3,1)$ & 2 & 1 & YES & YES & YES & $1.00$ & $(2,2)$ & NO & 1423\\
$(46,19)$ & 8 & $(3,1)$ & 2 & 1 & YES & YES & YES & $1.00$ & $(2,2)$ & -- & 1424\\
$(46,19)$ & 8 & $(3,1)$ & 2 & 1 & YES & YES & YES & $1.11$ & $(2,2)$ & NO & 1425\\
$(46,19)$ & 8 & $(4,1)$ & 3 & 2 & YES & YES & YES & $1.00$ & $(4,1)$ & -- & 1426\\
$(46,19)$ & 8 & $(4,1)$ & 3 & 2 & YES & YES & NO(2) & $1.11$ & $(6,0)$ & NO & 1427\\
$(46,13)$ & 10 & $(5,2)$ & 3 & 1 & YES & YES & NO(2) & $1.11$ & $(6,0)$ & NO & 1428\\
$(46,13)$ & 10 & $(5,2)$ & 3 & 1 & YES & YES & NO(2) & $1.11$ & $(6,0)$ & -- & 1429\\
$(46,19)$ & 8 & $(5,2)$ & 3 & 1 & YES & YES & YES & $1.12$ & $(2,2)$ & -- & 1430\\
$(46,17)$ & 8 & $(7,2)$ & 4 & 1 & YES & YES & YES & $1.00$ & $(2,2)$ & -- & 1431\\
$(46,17)$ & 8 & $(7,3)$ & 4 & 1 & YES & YES & NO(2) & $0.88$ & $(8,-1)$ & NO & 1432\\
$(46,17)$ & 8 & $(7,3)$ & 4 & 1 & YES & YES & NO(2) & $0.88$ & $(8,-1)$ & -- & 1433\\
$(46,19)$ & 8 & $(7,2)$ & 4 & 1 & YES & YES & YES & $1.12$ & $(2,2)$ & NO & 1434\\
$(46,19)$ & 8 & $(7,2)$ & 4 & 1 & YES & YES & YES & $1.11$ & $(2,2)$ & -- & 1435\\
$(46,19)$ & 8 & $(7,3)$ & 4 & 1 & YES & YES & YES & $1.11$ & $(2,2)$ & 1588 & 1436\\
$(46,17)$ & 8 & $(8,3)$ & 4 & 2 & YES & YES & YES & $1.00$ & $(6,0)$ & -- & 1437\\
$(46,17)$ & 8 & $(8,3)$ & 4 & 2 & YES & YES & YES & $1.25$ & $(6,0)$ & NO & 1438\\
$(46,13)$ & 10 & $(9,2)$ & 5 & 1 & YES & YES & NO(2) & $1.11$ & $(6,0)$ & NO & 1439\\
$(46,19)$ & 8 & $(9,2)$ & 5 & 1 & YES & YES & YES & $1.00$ & $(2,2)$ & -- & 1440\\
$(46,19)$ & 8 & $(9,4)$ & 5 & 1 & YES & YES & YES & $1.22$ & $(2,2)$ & 2263 & 1441\\
$(46,17)$ & 8 & $(11,3)$ & 5 & 1 & YES & YES & YES & $1.00$ & $(6,0)$ & -- & 1442\\
$(46,19)$ & 8 & $(12,5)$ & 5 & 2 & YES & YES & YES & $1.11$ & $(2,2)$ & 1309 & 1443\\
$(46,17)$ & 8 & $(13,5)$ & 5 & 1 & YES & YES & NO(2) & $0.88$ & $(8,-1)$ & NO & 1444\\
$(46,13)$ & 10 & $(15,4)$ & 6 & 1 & YES & YES & NO(2) & $1.11$ & $(6,0)$ & NO & 1445\\
$(46,19)$ & 8 & $(17,7)$ & 6 & 1 & YES & YES & YES & $1.00$ & $(2,2)$ & NO & 1446\\
$(46,19)$ & 8 & $(19,8)$ & 6 & 1 & YES & YES & YES & $1.12$ & $(2,2)$ & NO & 1447\\
$(46,19)$ & 8 & $(22,9)$ & 7 & 2 & YES & YES & YES & $1.00$ & $(2,2)$ & NO & 1448\\
$(46,17)$ & 8 & $(27,10)$ & 7 & 1 & YES & YES & YES & $1.00$ & $(2,2)$ & NO & 1449\\
$(46,19)$ & 8 & $(46,19)$ & 8 & 46 & YES & YES & YES & $1.11$ & $(2,2)$ & NO & 1450\\
$(47,18)$ & 8 & $(2,1)$ & 1 & 1 & YES & YES & YES & $1.11$ & $(2,2)$ & -- & 1451\\
$(47,14)$ & 9 & $(3,1)$ & 2 & 1 & YES & YES & YES & $1.33$ & $(2,2)$ & -- & 1452\\
$(47,14)$ & 9 & $(3,1)$ & 2 & 1 & YES & YES & YES & $1.44$ & $(2,2)$ & NO & 1453\\
$(47,18)$ & 8 & $(3,1)$ & 2 & 1 & YES & YES & YES & $1.00$ & $(2,2)$ & NO & 1454\\
$(47,18)$ & 8 & $(3,1)$ & 2 & 1 & YES & YES & YES & $1.00$ & $(2,2)$ & -- & 1455\\
$(47,14)$ & 9 & $(4,1)$ & 3 & 1 & YES & YES & NO(2) & $1.30$ & $(4,1)$ & -- & 1456\\
$(47,14)$ & 9 & $(4,1)$ & 3 & 1 & YES & YES & NO(2) & $1.40$ & $(4,1)$ & NO & 1457\\
$(47,18)$ & 8 & $(4,1)$ & 3 & 1 & YES & YES & YES & $1.22$ & $(2,2)$ & NO & 1458\\
$(47,18)$ & 8 & $(4,1)$ & 3 & 1 & YES & YES & YES & $1.22$ & $(2,2)$ & -- & 1459\\
$(47,14)$ & 9 & $(5,2)$ & 3 & 1 & YES & YES & YES & $1.00$ & $(2,2)$ & -- & 1460\\
$(47,14)$ & 9 & $(5,2)$ & 3 & 1 & YES & YES & YES & $1.00$ & $(2,2)$ & NO & 1461\\
$(47,18)$ & 8 & $(5,2)$ & 3 & 1 & YES & YES & YES & $1.25$ & $(4,1)$ & -- & 1462\\
$(47,14)$ & 9 & $(6,1)$ & 5 & 1 & YES & YES & YES & $1.22$ & $(2,2)$ & NO & 1463\\
$(47,14)$ & 9 & $(6,1)$ & 5 & 1 & YES & YES & YES & $1.22$ & $(2,2)$ & -- & 1464\\
$(47,13)$ & 8 & $(7,3)$ & 4 & 1 & YES & YES & NO(2) & $0.89$ & $(6,0)$ & -- & 1465\\
$(47,13)$ & 8 & $(7,3)$ & 4 & 1 & YES & YES & NO(2) & $1.11$ & $(6,0)$ & NO & 1466\\
$(47,17)$ & 9 & $(7,2)$ & 4 & 1 & YES & YES & NO(2) & $1.11$ & $(6,0)$ & NO & 1467\\
$(47,18)$ & 8 & $(7,2)$ & 4 & 1 & YES & YES & YES & $1.33$ & $(4,1)$ & -- & 1468\\
$(47,18)$ & 8 & $(7,2)$ & 4 & 1 & YES & YES & YES & $1.33$ & $(4,1)$ & NO & 1469\\
$(47,18)$ & 8 & $(7,2)$ & 4 & 1 & YES & YES & YES & $1.00$ & $(4,1)$ & NO & 1470\\
$(47,18)$ & 8 & $(7,3)$ & 4 & 1 & YES & YES & YES & $1.22$ & $(2,2)$ & NO & 1471\\
$(47,13)$ & 8 & $(8,3)$ & 4 & 1 & YES & YES & YES & $1.11$ & $(2,2)$ & NO & 1472\\
$(47,14)$ & 9 & $(9,2)$ & 5 & 1 & YES & YES & YES & $1.00$ & $(2,2)$ & NO & 1473\\
$(47,18)$ & 8 & $(9,2)$ & 5 & 1 & YES & YES & YES & $1.22$ & $(4,1)$ & NO & 1474\\
$(47,18)$ & 8 & $(9,2)$ & 5 & 1 & YES & YES & YES & $1.22$ & $(4,1)$ & -- & 1475\\
$(47,13)$ & 8 & $(10,3)$ & 5 & 1 & YES & YES & YES & $1.00$ & $(6,0)$ & -- & 1476\\
$(47,18)$ & 8 & $(10,3)$ & 5 & 1 & YES & YES & YES & $1.11$ & $(4,1)$ & -- & 1477\\
$(47,11)$ & 9 & $(11,3)$ & 5 & 1 & YES & YES & YES & $1.00$ & $(2,2)$ & -- & 1478\\
$(47,13)$ & 8 & $(11,3)$ & 5 & 1 & YES & YES & YES & $0.88$ & $(6,0)$ & -- & 1479\\
$(47,11)$ & 9 & $(13,5)$ & 5 & 1 & YES & YES & YES & $1.00$ & $(6,0)$ & -- & 1480\\
$(47,13)$ & 8 & $(13,4)$ & 6 & 1 & YES & YES & YES & $1.22$ & $(2,2)$ & NO & 1481\\
$(47,14)$ & 9 & $(13,2)$ & 7 & 1 & YES & YES & NO(2) & $1.20$ & $(4,1)$ & NO & 1482\\
$(47,18)$ & 8 & $(13,3)$ & 6 & 1 & YES & YES & YES & $1.00$ & $(2,2)$ & -- & 1483\\
$(47,18)$ & 8 & $(13,3)$ & 6 & 1 & YES & YES & YES & $1.38$ & $(2,2)$ & NO & 1484\\
$(47,11)$ & 9 & $(15,4)$ & 6 & 1 & YES & YES & YES & $1.11$ & $(4,1)$ & -- & 1485\\
$(47,13)$ & 8 & $(17,5)$ & 6 & 1 & YES & YES & YES & $1.22$ & $(2,2)$ & NO & 1486\\
$(47,14)$ & 9 & $(17,5)$ & 6 & 1 & YES & YES & YES & $1.22$ & $(2,2)$ & NO & 1487\\
$(47,17)$ & 9 & $(17,6)$ & 7 & 1 & YES & YES & YES & $1.12$ & $(4,1)$ & NO & 1488\\
$(47,11)$ & 9 & $(18,5)$ & 6 & 1 & YES & YES & YES & $1.22$ & $(4,1)$ & -- & 1489\\
$(47,14)$ & 9 & $(24,7)$ & 7 & 1 & YES & YES & YES & $1.00$ & $(2,2)$ & NO & 1490\\
$(47,18)$ & 8 & $(29,11)$ & 7 & 1 & YES & YES & YES & $1.12$ & $(4,1)$ & 2461 & 1491\\
$(47,18)$ & 8 & $(34,13)$ & 7 & 1 & YES & YES & YES & $1.00$ & $(4,1)$ & NO & 1492\\
$(47,14)$ & 9 & $(37,11)$ & 8 & 1 & YES & YES & NO(2) & $1.30$ & $(4,1)$ & NO & 1493\\
$(47,13)$ & 8 & $(43,12)$ & 8 & 1 & YES & YES & NO(2) & $1.00$ & $(6,0)$ & NO & 1494\\
$(47,14)$ & 9 & $(47,14)$ & 9 & 47 & YES & YES & YES & $1.33$ & $(2,2)$ & NO & 1495\\
$(47,18)$ & 8 & $(47,18)$ & 8 & 47 & YES & YES & YES & $1.11$ & $(2,2)$ & NO & 1496\\
$(48,11)$ & 9 & $(5,1)$ & 4 & 1 & YES & YES & NO(2) & $1.11$ & $(6,0)$ & -- & 1497\\
$(48,11)$ & 9 & $(5,1)$ & 4 & 1 & YES & YES & NO(2) & $1.22$ & $(6,0)$ & NO & 1498\\
$(48,13)$ & 9 & $(13,3)$ & 6 & 1 & YES & YES & YES & $1.00$ & $(6,0)$ & -- & 1499\\
$(48,13)$ & 9 & $(14,3)$ & 6 & 2 & YES & YES & YES & $1.00$ & $(6,0)$ & -- & 1500\\
$(48,11)$ & 9 & $(19,4)$ & 7 & 1 & YES & YES & YES & $1.11$ & $(2,2)$ & NO & 1501\\
$(48,11)$ & 9 & $(23,5)$ & 7 & 1 & YES & YES & YES & $1.11$ & $(2,2)$ & NO & 1502\\
$(48,11)$ & 9 & $(48,11)$ & 9 & 48 & YES & YES & NO(2) & $1.00$ & $(6,0)$ & NO & 1503\\
$(49,18)$ & 8 & $(2,1)$ & 1 & 1 & YES & YES & YES & $1.11$ & $(2,2)$ & -- & 1504\\
$(49,19)$ & 8 & $(2,1)$ & 1 & 1 & YES & YES & YES & $1.11$ & $(2,2)$ & -- & 1505\\
$(49,15)$ & 9 & $(3,1)$ & 2 & 1 & YES & YES & YES & $1.33$ & $(2,2)$ & -- & 1506\\
$(49,15)$ & 9 & $(3,1)$ & 2 & 1 & YES & YES & YES & $1.44$ & $(2,2)$ & NO & 1507\\
$(49,18)$ & 8 & $(3,1)$ & 2 & 1 & YES & YES & YES & $1.00$ & $(2,2)$ & -- & 1508\\
$(49,18)$ & 8 & $(3,1)$ & 2 & 1 & YES & YES & YES & $0.88$ & $(4,1)$ & NO & 1509\\
$(49,19)$ & 8 & $(3,1)$ & 2 & 1 & YES & YES & YES & $1.11$ & $(2,2)$ & -- & 1510\\
$(49,15)$ & 9 & $(4,1)$ & 3 & 1 & YES & YES & YES & $1.22$ & $(2,2)$ & NO & 1511\\
$(49,15)$ & 9 & $(4,1)$ & 3 & 1 & YES & YES & YES & $1.22$ & $(2,2)$ & -- & 1512\\
$(49,19)$ & 8 & $(4,1)$ & 3 & 1 & YES & YES & NO(2) & $1.11$ & $(6,0)$ & NO & 1513\\
$(49,19)$ & 8 & $(4,1)$ & 3 & 1 & YES & YES & NO(2) & $1.11$ & $(6,0)$ & -- & 1514\\
$(49,20)$ & 9 & $(4,1)$ & 3 & 1 & YES & YES & NO(2) & $1.11$ & $(6,0)$ & NO & 1515\\
$(49,11)$ & 10 & $(5,2)$ & 3 & 1 & YES & YES & YES & $0.88$ & $(4,1)$ & NO & 1516\\
$(49,11)$ & 10 & $(5,2)$ & 3 & 1 & YES & YES & NO(2) & $1.20$ & $(4,1)$ & NO & 1517\\
$(49,13)$ & 9 & $(5,2)$ & 3 & 1 & YES & YES & YES & $1.11$ & $(2,2)$ & -- & 1518\\
$(49,13)$ & 9 & $(5,2)$ & 3 & 1 & YES & YES & YES & $1.22$ & $(2,2)$ & NO & 1519\\
$(49,15)$ & 9 & $(5,1)$ & 4 & 1 & YES & YES & YES & $1.00$ & $(4,1)$ & NO & 1520\\
$(49,15)$ & 9 & $(5,1)$ & 4 & 1 & YES & YES & YES & $1.00$ & $(4,1)$ & -- & 1521\\
$(49,15)$ & 9 & $(5,2)$ & 3 & 1 & YES & YES & NO(2) & $1.11$ & $(6,0)$ & NO & 1522\\
$(49,18)$ & 8 & $(5,2)$ & 3 & 1 & YES & YES & YES & $1.00$ & $(2,2)$ & -- & 1523\\
$(49,18)$ & 8 & $(5,2)$ & 3 & 1 & YES & YES & YES & $1.22$ & $(2,2)$ & NO & 1524\\
$(49,19)$ & 8 & $(5,2)$ & 3 & 1 & YES & YES & YES & $1.00$ & $(2,2)$ & -- & 1525\\
$(49,19)$ & 8 & $(5,2)$ & 3 & 1 & YES & YES & NO(2) & $1.11$ & $(4,1)$ & NO & 1526\\
$(49,20)$ & 9 & $(5,2)$ & 3 & 1 & YES & YES & NO(2) & $1.11$ & $(6,0)$ & -- & 1527\\
$(49,20)$ & 9 & $(5,2)$ & 3 & 1 & YES & YES & NO(2) & $1.11$ & $(6,0)$ & NO & 1528\\
$(49,11)$ & 10 & $(7,3)$ & 4 & 7 & YES & YES & NO(2) & $1.11$ & $(6,0)$ & -- & 1529\\
$(49,15)$ & 9 & $(7,2)$ & 4 & 7 & YES & YES & YES & $1.22$ & $(2,2)$ & NO & 1530\\
$(49,15)$ & 9 & $(7,2)$ & 4 & 7 & YES & YES & YES & $1.22$ & $(2,2)$ & -- & 1531\\
$(49,18)$ & 8 & $(7,2)$ & 4 & 7 & YES & YES & NO(2) & $1.30$ & $(4,1)$ & NO & 1532\\
$(49,18)$ & 8 & $(7,2)$ & 4 & 7 & YES & YES & YES & $1.12$ & $(2,2)$ & NO & 1533\\
$(49,18)$ & 8 & $(7,2)$ & 4 & 7 & YES & YES & YES & $1.12$ & $(2,2)$ & -- & 1534\\
$(49,18)$ & 8 & $(7,3)$ & 4 & 7 & YES & YES & NO(2) & $1.00$ & $(6,0)$ & -- & 1535\\
$(49,18)$ & 8 & $(7,3)$ & 4 & 7 & YES & YES & YES & $1.12$ & $(2,2)$ & NO & 1536\\
$(49,19)$ & 8 & $(7,2)$ & 4 & 7 & YES & YES & YES & $1.12$ & $(4,1)$ & -- & 1537\\
$(49,19)$ & 8 & $(7,2)$ & 4 & 7 & YES & YES & YES & $1.33$ & $(4,1)$ & NO & 1538\\
$(49,11)$ & 10 & $(8,3)$ & 4 & 1 & YES & YES & YES & $1.33$ & $(2,2)$ & NO & 1539\\
$(49,11)$ & 10 & $(8,3)$ & 4 & 1 & YES & YES & YES & $1.33$ & $(2,2)$ & -- & 1540\\
$(49,18)$ & 8 & $(8,3)$ & 4 & 1 & YES & YES & YES & $1.00$ & $(2,2)$ & NO & 1541\\
$(49,18)$ & 8 & $(8,3)$ & 4 & 1 & YES & YES & YES & $1.22$ & $(4,1)$ & -- & 1542\\
$(49,18)$ & 8 & $(8,3)$ & 4 & 1 & YES & YES & YES & $1.55$ & $(2,2)$ & NO & 1543\\
$(49,19)$ & 8 & $(8,3)$ & 4 & 1 & YES & YES & YES & $1.11$ & $(2,2)$ & 1685 & 1544\\
$(49,20)$ & 9 & $(8,3)$ & 4 & 1 & YES & YES & YES & $1.25$ & $(4,1)$ & NO & 1545\\
$(49,20)$ & 9 & $(9,4)$ & 5 & 1 & YES & YES & NO(2) & $1.11$ & $(6,0)$ & NO & 1546\\
$(49,18)$ & 8 & $(10,3)$ & 5 & 1 & YES & YES & NO(2) & $1.20$ & $(4,1)$ & NO & 1547\\
$(49,18)$ & 8 & $(10,3)$ & 5 & 1 & YES & YES & YES & $1.11$ & $(4,1)$ & -- & 1548\\
$(49,19)$ & 8 & $(10,3)$ & 5 & 1 & YES & YES & YES & $1.25$ & $(2,2)$ & -- & 1549\\
$(49,15)$ & 9 & $(11,3)$ & 5 & 1 & YES & YES & YES & $1.22$ & $(2,2)$ & NO & 1550\\
$(49,18)$ & 8 & $(11,3)$ & 5 & 1 & YES & YES & YES & $1.22$ & $(4,1)$ & -- & 1551\\
$(49,19)$ & 8 & $(11,3)$ & 5 & 1 & YES & YES & YES & $1.12$ & $(2,2)$ & -- & 1552\\
$(49,20)$ & 9 & $(11,2)$ & 6 & 1 & YES & YES & NO(2) & $1.11$ & $(6,0)$ & NO & 1553\\
$(49,20)$ & 9 & $(11,2)$ & 6 & 1 & YES & YES & NO(2) & $1.11$ & $(6,0)$ & -- & 1554\\
$(49,11)$ & 10 & $(13,2)$ & 7 & 1 & YES & YES & NO(2) & $1.10$ & $(4,1)$ & NO & 1555\\
$(49,18)$ & 8 & $(13,5)$ & 5 & 1 & YES & YES & YES & $1.12$ & $(2,2)$ & NO & 1556\\
$(49,19)$ & 8 & $(13,5)$ & 5 & 1 & YES & YES & YES & $1.11$ & $(2,2)$ & 1389 & 1557\\
$(49,18)$ & 8 & $(14,5)$ & 6 & 7 & YES & YES & YES & $1.11$ & $(2,2)$ & NO & 1558\\
$(49,15)$ & 9 & $(16,5)$ & 7 & 1 & YES & YES & NO(2) & $1.11$ & $(6,0)$ & NO & 1559\\
$(49,11)$ & 10 & $(17,3)$ & 7 & 1 & YES & YES & YES & $1.33$ & $(2,2)$ & NO & 1560\\
$(49,15)$ & 9 & $(17,5)$ & 6 & 1 & YES & YES & YES & $1.22$ & $(2,2)$ & 2666 & 1561\\
$(49,11)$ & 10 & $(19,4)$ & 7 & 1 & YES & YES & NO(2) & $1.00$ & $(6,0)$ & NO & 1562\\
$(49,11)$ & 10 & $(21,5)$ & 8 & 7 & YES & YES & NO(2) & $1.00$ & $(6,0)$ & NO & 1563\\
$(49,19)$ & 8 & $(21,8)$ & 6 & 7 & YES & YES & YES & $0.88$ & $(2,2)$ & NO & 1564\\
$(49,15)$ & 9 & $(23,7)$ & 7 & 1 & YES & YES & YES & $1.00$ & $(4,1)$ & 1813 & 1565\\
$(49,18)$ & 8 & $(25,9)$ & 7 & 1 & YES & YES & NO(2) & $1.00$ & $(6,0)$ & NO & 1566\\
$(49,18)$ & 8 & $(27,10)$ & 7 & 1 & YES & YES & NO(2) & $1.20$ & $(4,1)$ & NO & 1567\\
$(49,19)$ & 8 & $(28,11)$ & 8 & 7 & YES & YES & YES & $1.22$ & $(2,2)$ & NO & 1568\\
$(49,19)$ & 8 & $(31,12)$ & 7 & 1 & YES & YES & YES & $1.40$ & $(2,2)$ & NO & 1569\\
$(49,15)$ & 9 & $(36,11)$ & 8 & 1 & YES & YES & YES & $1.22$ & $(2,2)$ & NO & 1570\\
$(49,20)$ & 9 & $(39,16)$ & 8 & 1 & YES & YES & NO(2) & $1.11$ & $(6,0)$ & NO & 1571\\
$(49,18)$ & 8 & $(41,15)$ & 8 & 1 & YES & YES & YES & $1.11$ & $(2,2)$ & NO & 1572\\
$(49,19)$ & 8 & $(44,17)$ & 8 & 1 & YES & YES & YES & $1.00$ & $(2,2)$ & NO & 1573\\
$(49,15)$ & 9 & $(49,15)$ & 9 & 49 & YES & YES & YES & $1.00$ & $(4,1)$ & NO & 1574\\
$(49,18)$ & 8 & $(49,18)$ & 8 & 49 & YES & YES & YES & $1.11$ & $(2,2)$ & NO & 1575\\
$(50,21)$ & 8 & $(2,1)$ & 1 & 2 & YES & YES & YES & $1.11$ & $(2,2)$ & -- & 1576\\
$(50,19)$ & 8 & $(3,1)$ & 2 & 1 & YES & YES & YES & $1.22$ & $(2,2)$ & NO & 1577\\
$(50,19)$ & 8 & $(3,1)$ & 2 & 1 & YES & YES & YES & $1.22$ & $(2,2)$ & -- & 1578\\
$(50,21)$ & 8 & $(3,1)$ & 2 & 1 & YES & YES & YES & $1.11$ & $(2,2)$ & 1670 & 1579\\
$(50,21)$ & 8 & $(3,1)$ & 2 & 1 & YES & YES & YES & $1.11$ & $(2,2)$ & -- & 1580\\
$(50,21)$ & 8 & $(3,1)$ & 2 & 1 & YES & YES & YES & $1.11$ & $(2,2)$ & NO & 1581\\
$(50,19)$ & 8 & $(4,1)$ & 3 & 2 & YES & YES & YES & $0.88$ & $(4,1)$ & NO & 1582\\
$(50,19)$ & 8 & $(4,1)$ & 3 & 2 & YES & YES & YES & $0.88$ & $(4,1)$ & -- & 1583\\
$(50,11)$ & 10 & $(5,2)$ & 3 & 5 & YES & YES & YES & $1.11$ & $(2,2)$ & NO & 1584\\
$(50,11)$ & 10 & $(5,2)$ & 3 & 5 & YES & YES & YES & $1.22$ & $(2,2)$ & NO & 1585\\
$(50,19)$ & 8 & $(5,2)$ & 3 & 5 & YES & YES & YES & $1.11$ & $(2,2)$ & -- & 1586\\
$(50,21)$ & 8 & $(5,2)$ & 3 & 5 & YES & YES & YES & $1.12$ & $(2,2)$ & -- & 1587\\
$(50,21)$ & 8 & $(5,2)$ & 3 & 5 & YES & YES & YES & $1.11$ & $(2,2)$ & 1436 & 1588\\
$(50,11)$ & 10 & $(7,2)$ & 4 & 1 & YES & YES & YES & $0.75$ & $(4,1)$ & NO & 1589\\
$(50,19)$ & 8 & $(7,2)$ & 4 & 1 & YES & YES & YES & $1.00$ & $(2,2)$ & NO & 1590\\
$(50,19)$ & 8 & $(7,2)$ & 4 & 1 & YES & YES & YES & $0.88$ & $(4,1)$ & -- & 1591\\
$(50,19)$ & 8 & $(7,3)$ & 4 & 1 & YES & YES & YES & $1.11$ & $(2,2)$ & NO & 1592\\
$(50,21)$ & 8 & $(7,3)$ & 4 & 1 & YES & YES & YES & $1.33$ & $(4,1)$ & -- & 1593\\
$(50,21)$ & 8 & $(8,3)$ & 4 & 2 & YES & YES & YES & $1.33$ & $(2,2)$ & NO & 1594\\
$(50,19)$ & 8 & $(9,2)$ & 5 & 1 & YES & YES & YES & $0.88$ & $(2,2)$ & NO & 1595\\
$(50,21)$ & 8 & $(9,4)$ & 5 & 1 & YES & YES & NO(2) & $1.20$ & $(4,1)$ & NO & 1596\\
$(50,19)$ & 8 & $(11,4)$ & 5 & 1 & YES & YES & NO(2) & $0.88$ & $(8,-1)$ & NO & 1597\\
$(50,21)$ & 8 & $(12,5)$ & 5 & 2 & YES & YES & YES & $1.11$ & $(2,2)$ & 1367 & 1598\\
$(50,19)$ & 8 & $(13,5)$ & 5 & 1 & YES & YES & YES & $1.11$ & $(2,2)$ & NO & 1599\\
$(50,21)$ & 8 & $(17,7)$ & 6 & 1 & YES & YES & YES & $1.33$ & $(2,2)$ & NO & 1600\\
$(50,19)$ & 8 & $(18,7)$ & 6 & 2 & YES & YES & YES & $1.11$ & $(2,2)$ & NO & 1601\\
$(50,21)$ & 8 & $(26,11)$ & 7 & 2 & YES & YES & YES & $1.22$ & $(2,2)$ & NO & 1602\\
$(50,19)$ & 8 & $(29,11)$ & 7 & 1 & YES & YES & YES & $0.88$ & $(4,1)$ & NO & 1603\\
$(50,19)$ & 8 & $(34,13)$ & 7 & 2 & YES & YES & YES & $1.22$ & $(2,2)$ & NO & 1604\\
$(50,19)$ & 8 & $(37,14)$ & 8 & 1 & YES & YES & YES & $1.00$ & $(2,2)$ & NO & 1605\\
$(50,21)$ & 8 & $(43,18)$ & 8 & 1 & YES & YES & YES & $1.12$ & $(2,2)$ & NO & 1606\\
$(50,19)$ & 8 & $(50,19)$ & 8 & 50 & YES & YES & YES & $1.00$ & $(2,2)$ & NO & 1607\\
$(51,20)$ & 9 & $(3,1)$ & 2 & 3 & YES & YES & YES & $1.22$ & $(2,2)$ & -- & 1608\\
$(51,20)$ & 9 & $(3,1)$ & 2 & 3 & YES & YES & NO(2) & $1.30$ & $(4,1)$ & NO & 1609\\
$(51,14)$ & 9 & $(4,1)$ & 3 & 1 & YES & YES & YES & $1.00$ & $(4,1)$ & NO & 1610\\
$(51,14)$ & 9 & $(4,1)$ & 3 & 1 & YES & YES & NO(2) & $1.20$ & $(4,1)$ & -- & 1611\\
$(51,20)$ & 9 & $(4,1)$ & 3 & 1 & YES & YES & YES & $1.33$ & $(2,2)$ & NO & 1612\\
$(51,20)$ & 9 & $(4,1)$ & 3 & 1 & YES & YES & YES & $1.33$ & $(2,2)$ & -- & 1613\\
$(51,20)$ & 9 & $(4,1)$ & 3 & 1 & YES & YES & NO(2) & $1.11$ & $(6,0)$ & NO & 1614\\
$(51,14)$ & 9 & $(5,2)$ & 3 & 1 & YES & YES & YES & $1.12$ & $(2,2)$ & -- & 1615\\
$(51,14)$ & 9 & $(5,2)$ & 3 & 1 & YES & YES & YES & $1.25$ & $(2,2)$ & NO & 1616\\
$(51,14)$ & 9 & $(5,2)$ & 3 & 1 & YES & YES & YES & $1.00$ & $(2,2)$ & NO & 1617\\
$(51,19)$ & 10 & $(5,1)$ & 4 & 1 & YES & YES & YES & $1.12$ & $(4,1)$ & NO & 1618\\
$(51,20)$ & 9 & $(5,1)$ & 4 & 1 & YES & YES & YES & $1.22$ & $(2,2)$ & NO & 1619\\
$(51,20)$ & 9 & $(5,1)$ & 4 & 1 & YES & YES & YES & $1.22$ & $(2,2)$ & -- & 1620\\
$(51,20)$ & 9 & $(5,1)$ & 4 & 1 & YES & YES & NO(2) & $1.22$ & $(6,0)$ & NO & 1621\\
$(51,19)$ & 10 & $(6,1)$ & 5 & 3 & YES & YES & NO(2) & $1.11$ & $(6,0)$ & NO & 1622\\
$(51,19)$ & 10 & $(7,1)$ & 6 & 1 & YES & YES & YES & $1.12$ & $(4,1)$ & NO & 1623\\
$(51,20)$ & 9 & $(7,3)$ & 4 & 1 & YES & YES & YES & $1.12$ & $(4,1)$ & NO & 1624\\
$(51,20)$ & 9 & $(8,3)$ & 4 & 1 & YES & YES & YES & $1.22$ & $(2,2)$ & NO & 1625\\
$(51,14)$ & 9 & $(9,2)$ & 5 & 3 & YES & YES & YES & $1.00$ & $(2,2)$ & -- & 1626\\
$(51,14)$ & 9 & $(10,3)$ & 5 & 1 & YES & YES & YES & $1.11$ & $(2,2)$ & NO & 1627\\
$(51,14)$ & 9 & $(13,2)$ & 7 & 1 & YES & YES & NO(2) & $1.20$ & $(4,1)$ & NO & 1628\\
$(51,20)$ & 9 & $(13,5)$ & 5 & 1 & YES & YES & YES & $1.22$ & $(2,2)$ & 1081 & 1629\\
$(51,20)$ & 9 & $(18,7)$ & 6 & 3 & YES & YES & YES & $1.22$ & $(2,2)$ & NO & 1630\\
$(51,19)$ & 10 & $(19,7)$ & 6 & 1 & YES & YES & YES & $1.12$ & $(4,1)$ & NO & 1631\\
$(51,14)$ & 9 & $(25,7)$ & 7 & 1 & YES & YES & YES & $1.12$ & $(2,2)$ & NO & 1632\\
$(51,14)$ & 9 & $(26,7)$ & 7 & 1 & YES & YES & NO(2) & $1.20$ & $(4,1)$ & NO & 1633\\
$(51,19)$ & 10 & $(27,10)$ & 7 & 3 & YES & YES & NO(2) & $1.11$ & $(6,0)$ & NO & 1634\\
$(51,19)$ & 10 & $(35,13)$ & 8 & 1 & YES & YES & YES & $1.12$ & $(4,1)$ & 2342 & 1635\\
$(51,20)$ & 9 & $(41,16)$ & 8 & 1 & YES & YES & YES & $1.33$ & $(2,2)$ & NO & 1636\\
$(51,20)$ & 9 & $(51,20)$ & 9 & 51 & YES & YES & YES & $0.88$ & $(4,1)$ & NO & 1637\\
$(52,19)$ & 9 & $(4,1)$ & 3 & 4 & YES & YES & YES & $1.11$ & $(2,2)$ & NO & 1638\\
$(52,19)$ & 9 & $(8,3)$ & 4 & 4 & YES & YES & YES & $1.11$ & $(2,2)$ & NO & 1639\\
$(52,19)$ & 9 & $(52,19)$ & 9 & 52 & YES & YES & YES & $1.11$ & $(2,2)$ & NO & 1640\\
$(53,19)$ & 9 & $(3,1)$ & 2 & 1 & YES & YES & YES & $1.11$ & $(2,2)$ & -- & 1641\\
$(53,19)$ & 9 & $(3,1)$ & 2 & 1 & YES & YES & YES & $1.33$ & $(2,2)$ & NO & 1642\\
$(53,19)$ & 9 & $(4,1)$ & 3 & 1 & YES & YES & YES & $1.22$ & $(2,2)$ & -- & 1643\\
$(53,22)$ & 9 & $(4,1)$ & 3 & 1 & YES & YES & NO(2) & $0.75$ & $(8,-1)$ & -- & 1644\\
$(53,14)$ & 9 & $(5,2)$ & 3 & 1 & YES & YES & NO(2) & $1.30$ & $(4,1)$ & NO & 1645\\
$(53,19)$ & 9 & $(5,1)$ & 4 & 1 & YES & YES & YES & $0.88$ & $(4,1)$ & NO & 1646\\
$(53,19)$ & 9 & $(5,2)$ & 3 & 1 & YES & YES & YES & $1.33$ & $(2,2)$ & NO & 1647\\
$(53,19)$ & 9 & $(5,2)$ & 3 & 1 & YES & YES & YES & $1.44$ & $(2,2)$ & -- & 1648\\
$(53,11)$ & 10 & $(7,3)$ & 4 & 1 & YES & YES & NO(2) & $1.40$ & $(4,1)$ & NO & 1649\\
$(53,11)$ & 10 & $(7,3)$ & 4 & 1 & YES & YES & NO(2) & $1.40$ & $(4,1)$ & -- & 1650\\
$(53,14)$ & 9 & $(7,3)$ & 4 & 1 & YES & YES & NO(2) & $1.30$ & $(4,1)$ & -- & 1651\\
$(53,16)$ & 10 & $(7,2)$ & 4 & 1 & YES & YES & NO(2) & $1.00$ & $(6,0)$ & -- & 1652\\
$(53,23)$ & 9 & $(7,2)$ & 4 & 1 & YES & YES & NO(2) & $1.11$ & $(6,0)$ & -- & 1653\\
$(53,19)$ & 9 & $(8,3)$ & 4 & 1 & YES & YES & YES & $1.11$ & $(2,2)$ & NO & 1654\\
$(53,10)$ & 10 & $(9,4)$ & 5 & 1 & YES & YES & NO(2) & $1.11$ & $(6,0)$ & -- & 1655\\
$(53,22)$ & 9 & $(9,2)$ & 5 & 1 & YES & YES & YES & $0.88$ & $(6,0)$ & -- & 1656\\
$(53,22)$ & 9 & $(9,2)$ & 5 & 1 & YES & YES & YES & $1.00$ & $(6,0)$ & NO & 1657\\
$(53,23)$ & 9 & $(9,2)$ & 5 & 1 & YES & YES & YES & $1.00$ & $(2,2)$ & -- & 1658\\
$(53,23)$ & 9 & $(9,2)$ & 5 & 1 & YES & YES & YES & $1.25$ & $(2,2)$ & NO & 1659\\
$(53,16)$ & 10 & $(16,5)$ & 7 & 1 & YES & YES & NO(2) & $1.00$ & $(6,0)$ & NO & 1660\\
$(53,16)$ & 10 & $(17,5)$ & 6 & 1 & YES & YES & NO(2) & $1.00$ & $(6,0)$ & NO & 1661\\
$(53,19)$ & 9 & $(17,6)$ & 7 & 1 & YES & YES & YES & $1.12$ & $(4,1)$ & NO & 1662\\
$(53,19)$ & 9 & $(19,7)$ & 6 & 1 & YES & YES & YES & $1.33$ & $(2,2)$ & 2836 & 1663\\
$(53,19)$ & 9 & $(25,9)$ & 7 & 1 & YES & YES & YES & $1.11$ & $(2,2)$ & 1978 & 1664\\
$(53,19)$ & 9 & $(36,13)$ & 8 & 1 & YES & YES & YES & $1.44$ & $(2,2)$ & 2679 & 1665\\
$(53,23)$ & 9 & $(37,16)$ & 9 & 1 & YES & YES & YES & $1.12$ & $(2,2)$ & NO & 1666\\
$(53,19)$ & 9 & $(39,14)$ & 8 & 1 & YES & YES & NO(2) & $1.11$ & $(6,0)$ & NO & 1667\\
$(53,14)$ & 9 & $(42,11)$ & 9 & 1 & YES & YES & NO(2) & $1.30$ & $(4,1)$ & NO & 1668\\
$(53,19)$ & 9 & $(53,19)$ & 9 & 53 & YES & YES & YES & $1.11$ & $(2,2)$ & NO & 1669\\
$(55,21)$ & 8 & $(2,1)$ & 1 & 1 & YES & YES & YES & $1.11$ & $(2,2)$ & 1579 & 1670\\
$(55,21)$ & 8 & $(2,1)$ & 1 & 1 & YES & YES & YES & $1.11$ & $(2,2)$ & -- & 1671\\
$(55,16)$ & 9 & $(3,1)$ & 2 & 1 & YES & YES & YES & $1.00$ & $(2,2)$ & -- & 1672\\
$(55,16)$ & 9 & $(3,1)$ & 2 & 1 & YES & YES & NO(2) & $1.00$ & $(6,0)$ & NO & 1673\\
$(55,21)$ & 8 & $(3,1)$ & 2 & 1 & YES & YES & YES & $1.11$ & $(2,2)$ & NO & 1674\\
$(55,21)$ & 8 & $(3,1)$ & 2 & 1 & YES & YES & YES & $1.11$ & $(2,2)$ & -- & 1675\\
$(55,21)$ & 8 & $(3,1)$ & 2 & 1 & YES & YES & YES & $1.25$ & $(2,2)$ & NO & 1676\\
$(55,24)$ & 9 & $(3,1)$ & 2 & 1 & YES & YES & YES & $1.11$ & $(2,2)$ & -- & 1677\\
$(55,17)$ & 10 & $(4,1)$ & 3 & 1 & YES & YES & NO(2) & $1.10$ & $(8,-1)$ & -- & 1678\\
$(55,21)$ & 8 & $(4,1)$ & 3 & 1 & YES & YES & YES & $1.00$ & $(2,2)$ & -- & 1679\\
$(55,21)$ & 8 & $(4,1)$ & 3 & 1 & YES & YES & YES & $1.12$ & $(2,2)$ & NO & 1680\\
$(55,23)$ & 9 & $(4,1)$ & 3 & 1 & YES & YES & YES & $1.12$ & $(2,2)$ & -- & 1681\\
$(55,16)$ & 9 & $(5,2)$ & 3 & 5 & YES & YES & YES & $1.12$ & $(2,2)$ & 1900 & 1682\\
$(55,17)$ & 10 & $(5,1)$ & 4 & 5 & YES & YES & NO(2) & $1.11$ & $(6,0)$ & -- & 1683\\
$(55,17)$ & 10 & $(5,2)$ & 3 & 5 & YES & YES & NO(2) & $1.11$ & $(6,0)$ & NO & 1684\\
$(55,21)$ & 8 & $(5,2)$ & 3 & 5 & YES & YES & YES & $1.11$ & $(2,2)$ & 1544 & 1685\\
$(55,21)$ & 8 & $(5,2)$ & 3 & 5 & YES & YES & YES & $1.00$ & $(4,1)$ & -- & 1686\\
$(55,23)$ & 9 & $(5,1)$ & 4 & 5 & YES & YES & YES & $1.12$ & $(2,2)$ & NO & 1687\\
$(55,23)$ & 9 & $(5,2)$ & 3 & 5 & YES & YES & NO(2) & $1.11$ & $(6,0)$ & NO & 1688\\
$(55,23)$ & 9 & $(5,2)$ & 3 & 5 & YES & YES & NO(2) & $1.11$ & $(6,0)$ & -- & 1689\\
$(55,24)$ & 9 & $(5,2)$ & 3 & 5 & YES & YES & YES & $1.11$ & $(2,2)$ & 1353 & 1690\\
$(55,24)$ & 9 & $(5,2)$ & 3 & 5 & YES & YES & YES & $1.33$ & $(2,2)$ & -- & 1691\\
$(55,21)$ & 8 & $(7,2)$ & 4 & 1 & YES & YES & YES & $1.00$ & $(6,0)$ & -- & 1692\\
$(55,21)$ & 8 & $(7,3)$ & 4 & 1 & YES & YES & NO(2) & $1.11$ & $(6,0)$ & NO & 1693\\
$(55,21)$ & 8 & $(7,3)$ & 4 & 1 & YES & YES & YES & $1.22$ & $(4,1)$ & -- & 1694\\
$(55,23)$ & 9 & $(7,2)$ & 4 & 1 & YES & YES & YES & $1.33$ & $(4,1)$ & -- & 1695\\
$(55,21)$ & 8 & $(8,3)$ & 4 & 1 & YES & YES & YES & $1.00$ & $(2,2)$ & NO & 1696\\
$(55,21)$ & 8 & $(8,3)$ & 4 & 1 & YES & YES & YES & $1.12$ & $(2,2)$ & -- & 1697\\
$(55,24)$ & 9 & $(8,3)$ & 4 & 1 & YES & YES & YES & $1.33$ & $(2,2)$ & NO & 1698\\
$(55,16)$ & 9 & $(9,2)$ & 5 & 1 & YES & YES & YES & $1.22$ & $(2,2)$ & NO & 1699\\
$(55,21)$ & 8 & $(9,2)$ & 5 & 1 & YES & YES & YES & $1.22$ & $(4,1)$ & NO & 1700\\
$(55,21)$ & 8 & $(9,2)$ & 5 & 1 & YES & YES & YES & $1.22$ & $(4,1)$ & -- & 1701\\
$(55,23)$ & 9 & $(9,2)$ & 5 & 1 & YES & YES & YES & $1.22$ & $(4,1)$ & -- & 1702\\
$(55,23)$ & 9 & $(9,2)$ & 5 & 1 & YES & YES & YES & $1.33$ & $(4,1)$ & NO & 1703\\
$(55,23)$ & 9 & $(9,2)$ & 5 & 1 & YES & YES & YES & $1.33$ & $(4,1)$ & NO & 1704\\
$(55,16)$ & 9 & $(11,3)$ & 5 & 11 & YES & YES & YES & $1.22$ & $(2,2)$ & NO & 1705\\
$(55,21)$ & 8 & $(11,4)$ & 5 & 11 & YES & YES & YES & $1.33$ & $(2,2)$ & NO & 1706\\
$(55,23)$ & 9 & $(11,2)$ & 6 & 11 & YES & YES & YES & $1.33$ & $(4,1)$ & NO & 1707\\
$(55,23)$ & 9 & $(11,2)$ & 6 & 11 & YES & YES & YES & $1.33$ & $(4,1)$ & -- & 1708\\
$(55,23)$ & 9 & $(11,2)$ & 6 & 11 & YES & YES & YES & $1.50$ & $(4,1)$ & NO & 1709\\
$(55,24)$ & 9 & $(11,5)$ & 6 & 11 & YES & YES & YES & $1.00$ & $(4,1)$ & NO & 1710\\
$(55,21)$ & 8 & $(13,3)$ & 6 & 1 & YES & YES & YES & $1.00$ & $(2,2)$ & -- & 1711\\
$(55,21)$ & 8 & $(13,3)$ & 6 & 1 & YES & YES & YES & $1.25$ & $(2,2)$ & NO & 1712\\
$(55,16)$ & 9 & $(17,5)$ & 6 & 1 & YES & YES & YES & $1.00$ & $(2,2)$ & NO & 1713\\
$(55,23)$ & 9 & $(17,7)$ & 6 & 1 & YES & YES & NO(2) & $1.00$ & $(6,0)$ & NO & 1714\\
$(55,21)$ & 8 & $(18,7)$ & 6 & 1 & YES & YES & YES & $1.00$ & $(2,2)$ & NO & 1715\\
$(55,24)$ & 9 & $(23,10)$ & 7 & 1 & YES & YES & YES & $1.11$ & $(2,2)$ & 1914 & 1716\\
$(55,17)$ & 10 & $(29,9)$ & 8 & 1 & YES & YES & NO(2) & $1.11$ & $(6,0)$ & 2179 & 1717\\
$(55,21)$ & 8 & $(29,11)$ & 7 & 1 & YES & YES & YES & $1.22$ & $(2,2)$ & NO & 1718\\
$(55,23)$ & 9 & $(31,13)$ & 7 & 1 & YES & YES & YES & $1.00$ & $(2,2)$ & 2251 & 1719\\
$(55,21)$ & 8 & $(34,13)$ & 7 & 1 & YES & YES & YES & $1.33$ & $(2,2)$ & NO & 1720\\
$(55,17)$ & 10 & $(42,13)$ & 9 & 1 & YES & YES & NO(2) & $1.10$ & $(8,-1)$ & NO & 1721\\
$(55,23)$ & 9 & $(43,18)$ & 8 & 1 & YES & YES & YES & $1.00$ & $(2,2)$ & NO & 1722\\
$(55,21)$ & 8 & $(47,18)$ & 8 & 1 & YES & YES & YES & $1.22$ & $(4,1)$ & NO & 1723\\
$(55,23)$ & 9 & $(50,21)$ & 8 & 5 & YES & YES & YES & $1.33$ & $(4,1)$ & 2771 & 1724\\
$(55,21)$ & 8 & $(55,21)$ & 8 & 55 & YES & YES & YES & $1.00$ & $(2,2)$ & NO & 1725\\
$(55,24)$ & 9 & $(55,24)$ & 9 & 55 & YES & YES & NO(2) & $1.00$ & $(6,0)$ & NO & 1726\\
$(56,17)$ & 9 & $(2,1)$ & 1 & 2 & YES & YES & YES & $1.22$ & $(2,2)$ & NO & 1727\\
$(56,17)$ & 9 & $(3,1)$ & 2 & 1 & YES & YES & YES & $1.11$ & $(2,2)$ & -- & 1728\\
$(56,23)$ & 9 & $(3,1)$ & 2 & 1 & YES & YES & YES & $1.00$ & $(2,2)$ & -- & 1729\\
$(56,23)$ & 9 & $(3,1)$ & 2 & 1 & YES & YES & YES & $1.12$ & $(2,2)$ & NO & 1730\\
$(56,15)$ & 9 & $(4,1)$ & 3 & 4 & YES & YES & YES & $1.22$ & $(2,2)$ & -- & 1731\\
$(56,15)$ & 9 & $(4,1)$ & 3 & 4 & YES & YES & YES & $1.33$ & $(2,2)$ & NO & 1732\\
$(56,23)$ & 9 & $(4,1)$ & 3 & 4 & YES & YES & YES & $1.00$ & $(2,2)$ & NO & 1733\\
$(56,15)$ & 9 & $(5,2)$ & 3 & 1 & YES & YES & YES & $1.12$ & $(2,2)$ & -- & 1734\\
$(56,15)$ & 9 & $(5,2)$ & 3 & 1 & YES & YES & NO(2) & $1.11$ & $(4,1)$ & NO & 1735\\
$(56,17)$ & 9 & $(5,2)$ & 3 & 1 & YES & YES & NO(2) & $0.89$ & $(6,0)$ & -- & 1736\\
$(56,15)$ & 9 & $(7,2)$ & 4 & 7 & YES & YES & NO(2) & $1.10$ & $(4,1)$ & -- & 1737\\
$(56,17)$ & 9 & $(7,3)$ & 4 & 7 & YES & YES & NO(2) & $1.11$ & $(6,0)$ & NO & 1738\\
$(56,17)$ & 9 & $(7,3)$ & 4 & 7 & YES & YES & YES & $1.33$ & $(4,1)$ & -- & 1739\\
$(56,23)$ & 9 & $(7,2)$ & 4 & 7 & YES & YES & YES & $1.33$ & $(4,1)$ & NO & 1740\\
$(56,23)$ & 9 & $(7,2)$ & 4 & 7 & YES & YES & YES & $1.33$ & $(4,1)$ & -- & 1741\\
$(56,23)$ & 9 & $(7,3)$ & 4 & 7 & YES & YES & YES & $1.00$ & $(2,2)$ & NO & 1742\\
$(56,15)$ & 9 & $(9,2)$ & 5 & 1 & YES & YES & NO(2) & $1.20$ & $(4,1)$ & NO & 1743\\
$(56,23)$ & 9 & $(9,4)$ & 5 & 1 & YES & YES & NO(2) & $1.11$ & $(6,0)$ & NO & 1744\\
$(56,15)$ & 9 & $(10,3)$ & 5 & 2 & YES & YES & YES & $1.00$ & $(2,2)$ & NO & 1745\\
$(56,17)$ & 9 & $(13,4)$ & 6 & 1 & YES & YES & YES & $1.22$ & $(2,2)$ & NO & 1746\\
$(56,17)$ & 9 & $(17,5)$ & 6 & 1 & YES & YES & YES & $1.12$ & $(2,2)$ & NO & 1747\\
$(56,17)$ & 9 & $(23,7)$ & 7 & 1 & YES & YES & YES & $1.22$ & $(2,2)$ & NO & 1748\\
$(57,13)$ & 9 & $(2,1)$ & 1 & 1 & YES & YES & NO(2) & $1.11$ & $(6,0)$ & NO & 1749\\
$(57,25)$ & 9 & $(2,1)$ & 1 & 1 & YES & YES & YES & $1.00$ & $(4,1)$ & -- & 1750\\
$(57,16)$ & 9 & $(3,1)$ & 2 & 3 & YES & YES & NO(2) & $1.30$ & $(4,1)$ & NO & 1751\\
$(57,16)$ & 9 & $(3,1)$ & 2 & 3 & YES & YES & NO(2) & $1.30$ & $(4,1)$ & -- & 1752\\
$(57,17)$ & 10 & $(3,1)$ & 2 & 3 & YES & YES & NO(2) & $1.00$ & $(6,0)$ & -- & 1753\\
$(57,25)$ & 9 & $(3,1)$ & 2 & 3 & YES & YES & NO(2) & $1.20$ & $(4,1)$ & -- & 1754\\
$(57,25)$ & 9 & $(4,1)$ & 3 & 1 & YES & YES & NO(2) & $1.20$ & $(4,1)$ & NO & 1755\\
$(57,25)$ & 9 & $(4,1)$ & 3 & 1 & YES & YES & NO(2) & $1.20$ & $(4,1)$ & -- & 1756\\
$(57,17)$ & 10 & $(5,1)$ & 4 & 1 & YES & YES & NO(2) & $1.00$ & $(6,0)$ & NO & 1757\\
$(57,25)$ & 9 & $(5,2)$ & 3 & 1 & YES & YES & YES & $1.00$ & $(4,1)$ & -- & 1758\\
$(57,13)$ & 9 & $(7,3)$ & 4 & 1 & YES & YES & NO(2) & $1.20$ & $(4,1)$ & -- & 1759\\
$(57,17)$ & 10 & $(7,1)$ & 6 & 1 & YES & YES & NO(2) & $1.00$ & $(6,0)$ & NO & 1760\\
$(57,25)$ & 9 & $(7,3)$ & 4 & 1 & YES & YES & YES & $1.33$ & $(2,2)$ & NO & 1761\\
$(57,13)$ & 9 & $(9,2)$ & 5 & 3 & YES & YES & NO(2) & $1.00$ & $(6,0)$ & NO & 1762\\
$(57,25)$ & 9 & $(9,4)$ & 5 & 3 & YES & YES & YES & $1.44$ & $(2,2)$ & NO & 1763\\
$(57,16)$ & 9 & $(13,3)$ & 6 & 1 & YES & YES & NO(2) & $1.20$ & $(4,1)$ & NO & 1764\\
$(57,16)$ & 9 & $(15,4)$ & 6 & 3 & YES & YES & YES & $1.00$ & $(2,2)$ & NO & 1765\\
$(57,17)$ & 10 & $(17,5)$ & 6 & 1 & YES & YES & NO(2) & $1.00$ & $(6,0)$ & NO & 1766\\
$(57,16)$ & 9 & $(19,5)$ & 7 & 19 & YES & YES & NO(2) & $1.20$ & $(4,1)$ & NO & 1767\\
$(57,25)$ & 9 & $(23,10)$ & 7 & 1 & YES & YES & YES & $1.12$ & $(4,1)$ & NO & 1768\\
$(57,25)$ & 9 & $(25,11)$ & 7 & 1 & YES & YES & NO(2) & $1.20$ & $(4,1)$ & 2037 & 1769\\
$(57,17)$ & 10 & $(37,11)$ & 8 & 1 & YES & YES & NO(2) & $1.00$ & $(6,0)$ & 2540 & 1770\\
$(57,25)$ & 9 & $(41,18)$ & 8 & 1 & YES & YES & NO(2) & $1.20$ & $(4,1)$ & NO & 1771\\
$(57,16)$ & 9 & $(43,12)$ & 8 & 1 & YES & YES & NO(2) & $1.20$ & $(4,1)$ & NO & 1772\\
$(57,17)$ & 10 & $(47,14)$ & 9 & 1 & YES & YES & NO(2) & $1.30$ & $(4,1)$ & NO & 1773\\
$(57,17)$ & 10 & $(57,17)$ & 10 & 57 & YES & YES & NO(2) & $1.00$ & $(6,0)$ & NO & 1774\\
$(57,25)$ & 9 & $(57,25)$ & 9 & 57 & YES & YES & NO(2) & $1.30$ & $(4,1)$ & NO & 1775\\
$(58,17)$ & 9 & $(3,1)$ & 2 & 1 & YES & YES & YES & $1.33$ & $(2,2)$ & -- & 1776\\
$(58,17)$ & 9 & $(10,3)$ & 5 & 2 & YES & YES & YES & $1.00$ & $(2,2)$ & -- & 1777\\
$(58,17)$ & 9 & $(11,3)$ & 5 & 1 & YES & YES & YES & $1.12$ & $(2,2)$ & -- & 1778\\
$(59,18)$ & 9 & $(2,1)$ & 1 & 1 & YES & YES & YES & $1.11$ & $(2,2)$ & -- & 1779\\
$(59,18)$ & 9 & $(2,1)$ & 1 & 1 & YES & YES & YES & $1.00$ & $(4,1)$ & NO & 1780\\
$(59,24)$ & 10 & $(3,1)$ & 2 & 1 & YES & YES & NO(2) & $1.30$ & $(4,1)$ & NO & 1781\\
$(59,24)$ & 10 & $(3,1)$ & 2 & 1 & YES & YES & NO(2) & $1.30$ & $(4,1)$ & -- & 1782\\
$(59,26)$ & 9 & $(3,1)$ & 2 & 1 & YES & YES & YES & $1.22$ & $(2,2)$ & NO & 1783\\
$(59,26)$ & 9 & $(3,1)$ & 2 & 1 & YES & YES & YES & $1.22$ & $(2,2)$ & -- & 1784\\
$(59,26)$ & 9 & $(3,1)$ & 2 & 1 & YES & YES & NO(2) & $1.30$ & $(4,1)$ & NO & 1785\\
$(59,18)$ & 9 & $(4,1)$ & 3 & 1 & YES & YES & YES & $1.11$ & $(2,2)$ & 1163 & 1786\\
$(59,21)$ & 10 & $(4,1)$ & 3 & 1 & YES & YES & YES & $1.00$ & $(4,1)$ & -- & 1787\\
$(59,24)$ & 10 & $(4,1)$ & 3 & 1 & YES & YES & NO(2) & $1.30$ & $(4,1)$ & NO & 1788\\
$(59,24)$ & 10 & $(4,1)$ & 3 & 1 & YES & YES & NO(2) & $1.30$ & $(4,1)$ & NO & 1789\\
$(59,24)$ & 10 & $(4,1)$ & 3 & 1 & YES & YES & NO(2) & $1.40$ & $(4,1)$ & -- & 1790\\
$(59,25)$ & 9 & $(4,1)$ & 3 & 1 & YES & YES & YES & $1.12$ & $(2,2)$ & -- & 1791\\
$(59,25)$ & 9 & $(4,1)$ & 3 & 1 & YES & YES & YES & $1.25$ & $(2,2)$ & NO & 1792\\
$(59,26)$ & 9 & $(4,1)$ & 3 & 1 & YES & YES & YES & $1.33$ & $(2,2)$ & NO & 1793\\
$(59,26)$ & 9 & $(4,1)$ & 3 & 1 & YES & YES & YES & $1.33$ & $(2,2)$ & -- & 1794\\
$(59,18)$ & 9 & $(5,1)$ & 4 & 1 & YES & YES & YES & $1.00$ & $(4,1)$ & -- & 1795\\
$(59,18)$ & 9 & $(5,2)$ & 3 & 1 & YES & YES & NO(2) & $1.20$ & $(4,1)$ & -- & 1796\\
$(59,21)$ & 10 & $(5,1)$ & 4 & 1 & YES & YES & YES & $0.88$ & $(4,1)$ & NO & 1797\\
$(59,24)$ & 10 & $(5,1)$ & 4 & 1 & YES & YES & NO(2) & $1.30$ & $(4,1)$ & -- & 1798\\
$(59,24)$ & 10 & $(5,1)$ & 4 & 1 & YES & YES & NO(2) & $1.30$ & $(4,1)$ & NO & 1799\\
$(59,26)$ & 9 & $(5,2)$ & 3 & 1 & YES & YES & NO(2) & $1.40$ & $(4,1)$ & NO & 1800\\
$(59,26)$ & 9 & $(5,2)$ & 3 & 1 & YES & YES & NO(2) & $1.40$ & $(4,1)$ & -- & 1801\\
$(59,24)$ & 10 & $(6,1)$ & 5 & 1 & YES & YES & NO(2) & $1.30$ & $(4,1)$ & NO & 1802\\
$(59,24)$ & 10 & $(6,1)$ & 5 & 1 & YES & YES & NO(2) & $1.30$ & $(4,1)$ & NO & 1803\\
$(59,24)$ & 10 & $(7,3)$ & 4 & 1 & YES & YES & NO(2) & $1.30$ & $(4,1)$ & NO & 1804\\
$(59,26)$ & 9 & $(7,2)$ & 4 & 1 & YES & YES & YES & $1.44$ & $(2,2)$ & NO & 1805\\
$(59,26)$ & 9 & $(7,2)$ & 4 & 1 & YES & YES & YES & $1.44$ & $(2,2)$ & -- & 1806\\
$(59,26)$ & 9 & $(7,3)$ & 4 & 1 & YES & YES & YES & $1.22$ & $(2,2)$ & 1139 & 1807\\
$(59,21)$ & 10 & $(8,3)$ & 4 & 1 & YES & YES & YES & $1.00$ & $(4,1)$ & NO & 1808\\
$(59,16)$ & 10 & $(9,2)$ & 5 & 1 & YES & YES & NO(2) & $1.11$ & $(6,0)$ & 2106 & 1809\\
$(59,25)$ & 9 & $(9,4)$ & 5 & 1 & YES & YES & NO(2) & $1.11$ & $(6,0)$ & NO & 1810\\
$(59,18)$ & 9 & $(10,3)$ & 5 & 1 & YES & YES & YES & $1.11$ & $(2,2)$ & NO & 1811\\
$(59,24)$ & 10 & $(12,5)$ & 5 & 1 & YES & YES & NO(2) & $1.30$ & $(4,1)$ & NO & 1812\\
$(59,18)$ & 9 & $(13,4)$ & 6 & 1 & YES & YES & YES & $1.00$ & $(4,1)$ & 1565 & 1813\\
$(59,18)$ & 9 & $(16,5)$ & 7 & 1 & YES & YES & NO(2) & $1.20$ & $(4,1)$ & NO & 1814\\
$(59,26)$ & 9 & $(16,7)$ & 6 & 1 & YES & YES & YES & $1.33$ & $(2,2)$ & NO & 1815\\
$(59,24)$ & 10 & $(17,7)$ & 6 & 1 & YES & YES & NO(2) & $1.30$ & $(4,1)$ & 1245 & 1816\\
$(59,16)$ & 10 & $(19,5)$ & 7 & 1 & YES & YES & NO(2) & $1.11$ & $(6,0)$ & NO & 1817\\
$(59,25)$ & 9 & $(19,8)$ & 6 & 1 & YES & YES & YES & $1.12$ & $(2,2)$ & NO & 1818\\
$(59,24)$ & 10 & $(22,9)$ & 7 & 1 & YES & YES & NO(2) & $1.30$ & $(4,1)$ & NO & 1819\\
$(59,26)$ & 9 & $(23,10)$ & 7 & 1 & YES & YES & YES & $1.44$ & $(2,2)$ & NO & 1820\\
$(59,23)$ & 9 & $(28,11)$ & 8 & 1 & YES & YES & YES & $1.22$ & $(2,2)$ & 2814 & 1821\\
$(59,21)$ & 10 & $(31,11)$ & 8 & 1 & YES & YES & YES & $0.88$ & $(4,1)$ & 2323 & 1822\\
$(59,18)$ & 9 & $(33,10)$ & 8 & 1 & YES & YES & YES & $1.33$ & $(2,2)$ & NO & 1823\\
$(59,25)$ & 9 & $(59,25)$ & 9 & 59 & YES & YES & YES & $1.00$ & $(2,2)$ & NO & 1824\\
$(60,23)$ & 9 & $(5,1)$ & 4 & 5 & YES & YES & YES & $1.12$ & $(2,2)$ & NO & 1825\\
$(60,23)$ & 9 & $(6,1)$ & 5 & 6 & YES & YES & YES & $1.12$ & $(2,2)$ & NO & 1826\\
$(60,23)$ & 9 & $(6,1)$ & 5 & 6 & YES & YES & YES & $1.12$ & $(2,2)$ & NO & 1827\\
$(60,11)$ & 11 & $(7,3)$ & 4 & 1 & YES & YES & NO(2) & $1.11$ & $(6,0)$ & NO & 1828\\
$(60,23)$ & 9 & $(7,2)$ & 4 & 1 & YES & YES & YES & $1.33$ & $(4,1)$ & -- & 1829\\
$(60,11)$ & 11 & $(8,3)$ & 4 & 4 & YES & YES & YES & $1.22$ & $(2,2)$ & NO & 1830\\
$(60,11)$ & 11 & $(8,3)$ & 4 & 4 & YES & YES & YES & $1.33$ & $(2,2)$ & NO & 1831\\
$(60,23)$ & 9 & $(11,2)$ & 6 & 1 & YES & YES & YES & $1.22$ & $(4,1)$ & NO & 1832\\
$(60,11)$ & 11 & $(14,3)$ & 6 & 2 & YES & YES & NO(2) & $1.11$ & $(6,0)$ & 2815 & 1833\\
$(60,23)$ & 9 & $(21,8)$ & 6 & 3 & YES & YES & YES & $1.44$ & $(2,2)$ & NO & 1834\\
$(60,11)$ & 11 & $(23,4)$ & 8 & 1 & YES & YES & NO(2) & $1.11$ & $(6,0)$ & NO & 1835\\
$(60,23)$ & 9 & $(34,13)$ & 7 & 2 & YES & YES & YES & $1.00$ & $(2,2)$ & 2465 & 1836\\
$(60,23)$ & 9 & $(55,21)$ & 8 & 5 & YES & YES & YES & $1.22$ & $(4,1)$ & 2949 & 1837\\
$(60,23)$ & 9 & $(60,23)$ & 9 & 60 & YES & YES & YES & $1.25$ & $(2,2)$ & NO & 1838\\
$(61,18)$ & 9 & $(2,1)$ & 1 & 1 & YES & YES & YES & $1.11$ & $(2,2)$ & NO & 1839\\
$(61,18)$ & 9 & $(2,1)$ & 1 & 1 & YES & YES & YES & $1.11$ & $(2,2)$ & -- & 1840\\
$(61,18)$ & 9 & $(3,1)$ & 2 & 1 & YES & YES & YES & $1.11$ & $(2,2)$ & NO & 1841\\
$(61,18)$ & 9 & $(3,1)$ & 2 & 1 & YES & YES & YES & $1.12$ & $(2,2)$ & -- & 1842\\
$(61,18)$ & 9 & $(3,1)$ & 2 & 1 & YES & YES & YES & $1.25$ & $(2,2)$ & NO & 1843\\
$(61,19)$ & 10 & $(3,1)$ & 2 & 1 & YES & YES & NO(2) & $1.20$ & $(4,1)$ & -- & 1844\\
$(61,19)$ & 10 & $(3,1)$ & 2 & 1 & YES & YES & NO(2) & $1.40$ & $(4,1)$ & NO & 1845\\
$(61,22)$ & 9 & $(3,1)$ & 2 & 1 & YES & YES & YES & $1.11$ & $(2,2)$ & -- & 1846\\
$(61,25)$ & 9 & $(3,1)$ & 2 & 1 & YES & YES & YES & $1.00$ & $(2,2)$ & -- & 1847\\
$(61,25)$ & 9 & $(3,1)$ & 2 & 1 & YES & YES & YES & $1.12$ & $(2,2)$ & NO & 1848\\
$(61,18)$ & 9 & $(4,1)$ & 3 & 1 & YES & YES & YES & $1.22$ & $(2,2)$ & NO & 1849\\
$(61,18)$ & 9 & $(4,1)$ & 3 & 1 & YES & YES & YES & $1.33$ & $(2,2)$ & -- & 1850\\
$(61,22)$ & 9 & $(4,1)$ & 3 & 1 & YES & YES & YES & $1.33$ & $(2,2)$ & NO & 1851\\
$(61,25)$ & 9 & $(4,1)$ & 3 & 1 & YES & YES & YES & $1.00$ & $(2,2)$ & -- & 1852\\
$(61,14)$ & 10 & $(5,2)$ & 3 & 1 & YES & YES & YES & $0.88$ & $(2,2)$ & NO & 1853\\
$(61,14)$ & 10 & $(5,2)$ & 3 & 1 & YES & YES & NO(2) & $1.22$ & $(4,1)$ & -- & 1854\\
$(61,18)$ & 9 & $(5,1)$ & 4 & 1 & YES & YES & YES & $1.12$ & $(2,2)$ & -- & 1855\\
$(61,18)$ & 9 & $(5,1)$ & 4 & 1 & YES & YES & YES & $1.12$ & $(2,2)$ & NO & 1856\\
$(61,18)$ & 9 & $(5,2)$ & 3 & 1 & YES & YES & YES & $1.00$ & $(2,2)$ & -- & 1857\\
$(61,18)$ & 9 & $(5,2)$ & 3 & 1 & YES & YES & YES & $1.00$ & $(4,1)$ & NO & 1858\\
$(61,18)$ & 9 & $(5,2)$ & 3 & 1 & YES & YES & YES & $1.12$ & $(2,2)$ & NO & 1859\\
$(61,22)$ & 9 & $(5,2)$ & 3 & 1 & YES & YES & YES & $1.22$ & $(2,2)$ & NO & 1860\\
$(61,14)$ & 10 & $(7,2)$ & 4 & 1 & YES & YES & YES & $0.88$ & $(2,2)$ & NO & 1861\\
$(61,17)$ & 9 & $(7,3)$ & 4 & 1 & YES & YES & YES & $1.44$ & $(2,2)$ & NO & 1862\\
$(61,18)$ & 9 & $(7,2)$ & 4 & 1 & YES & YES & YES & $1.00$ & $(2,2)$ & NO & 1863\\
$(61,18)$ & 9 & $(7,2)$ & 4 & 1 & YES & YES & YES & $1.22$ & $(4,1)$ & -- & 1864\\
$(61,19)$ & 10 & $(7,2)$ & 4 & 1 & YES & YES & NO(2) & $1.20$ & $(4,1)$ & NO & 1865\\
$(61,22)$ & 9 & $(7,3)$ & 4 & 1 & YES & YES & YES & $1.00$ & $(2,2)$ & NO & 1866\\
$(61,25)$ & 9 & $(7,2)$ & 4 & 1 & YES & YES & YES & $1.33$ & $(4,1)$ & -- & 1867\\
$(61,25)$ & 9 & $(7,3)$ & 4 & 1 & YES & YES & YES & $1.00$ & $(2,2)$ & NO & 1868\\
$(61,22)$ & 9 & $(8,3)$ & 4 & 1 & YES & YES & YES & $1.33$ & $(2,2)$ & NO & 1869\\
$(61,18)$ & 9 & $(9,2)$ & 5 & 1 & YES & YES & YES & $1.22$ & $(4,1)$ & NO & 1870\\
$(61,18)$ & 9 & $(9,2)$ & 5 & 1 & YES & YES & YES & $1.11$ & $(4,1)$ & -- & 1871\\
$(61,18)$ & 9 & $(10,3)$ & 5 & 1 & YES & YES & YES & $1.11$ & $(2,2)$ & NO & 1872\\
$(61,19)$ & 10 & $(10,3)$ & 5 & 1 & YES & YES & NO(2) & $1.20$ & $(4,1)$ & NO & 1873\\
$(61,14)$ & 10 & $(11,2)$ & 6 & 1 & YES & YES & YES & $1.11$ & $(2,2)$ & NO & 1874\\
$(61,16)$ & 10 & $(11,2)$ & 6 & 1 & YES & YES & NO(2) & $1.00$ & $(6,0)$ & NO & 1875\\
$(61,17)$ & 9 & $(11,3)$ & 5 & 1 & YES & YES & YES & $1.00$ & $(2,2)$ & -- & 1876\\
$(61,25)$ & 9 & $(12,5)$ & 5 & 1 & YES & YES & YES & $1.00$ & $(2,2)$ & 2272 & 1877\\
$(61,14)$ & 10 & $(13,2)$ & 7 & 1 & YES & YES & NO(2) & $1.20$ & $(4,1)$ & NO & 1878\\
$(61,18)$ & 9 & $(13,4)$ & 6 & 1 & YES & YES & YES & $1.33$ & $(2,2)$ & 2385 & 1879\\
$(61,14)$ & 10 & $(14,3)$ & 6 & 1 & YES & YES & YES & $1.00$ & $(6,0)$ & -- & 1880\\
$(61,22)$ & 9 & $(14,5)$ & 6 & 1 & YES & YES & YES & $1.22$ & $(2,2)$ & NO & 1881\\
$(61,17)$ & 9 & $(17,5)$ & 6 & 1 & YES & YES & YES & $1.44$ & $(2,2)$ & NO & 1882\\
$(61,18)$ & 9 & $(27,8)$ & 7 & 1 & YES & YES & YES & $1.25$ & $(2,2)$ & 2181 & 1883\\
$(61,14)$ & 10 & $(31,7)$ & 8 & 1 & YES & YES & YES & $1.11$ & $(2,2)$ & NO & 1884\\
$(61,16)$ & 10 & $(34,9)$ & 8 & 1 & YES & YES & NO(2) & $1.11$ & $(6,0)$ & NO & 1885\\
$(61,22)$ & 9 & $(36,13)$ & 8 & 1 & YES & YES & YES & $1.22$ & $(2,2)$ & NO & 1886\\
$(61,18)$ & 9 & $(37,11)$ & 8 & 1 & YES & YES & YES & $0.88$ & $(4,1)$ & 2904 & 1887\\
$(61,25)$ & 9 & $(39,16)$ & 8 & 1 & YES & YES & YES & $1.00$ & $(2,2)$ & NO & 1888\\
$(61,14)$ & 10 & $(57,13)$ & 9 & 1 & YES & YES & NO(2) & $1.20$ & $(4,1)$ & 2974 & 1889\\
$(61,18)$ & 9 & $(61,18)$ & 9 & 61 & YES & YES & YES & $1.12$ & $(2,2)$ & NO & 1890\\
$(61,22)$ & 9 & $(61,22)$ & 9 & 61 & YES & YES & YES & $1.22$ & $(2,2)$ & NO & 1891\\
$(61,25)$ & 9 & $(61,25)$ & 9 & 61 & YES & YES & YES & $1.12$ & $(2,2)$ & NO & 1892\\
$(62,27)$ & 9 & $(2,1)$ & 1 & 2 & YES & YES & NO(2) & $1.00$ & $(6,0)$ & -- & 1893\\
$(62,23)$ & 9 & $(3,1)$ & 2 & 1 & YES & YES & NO(2) & $1.00$ & $(6,0)$ & NO & 1894\\
$(62,27)$ & 9 & $(3,1)$ & 2 & 1 & YES & YES & YES & $1.11$ & $(2,2)$ & NO & 1895\\
$(62,27)$ & 9 & $(3,1)$ & 2 & 1 & YES & YES & NO(2) & $1.20$ & $(4,1)$ & -- & 1896\\
$(62,19)$ & 10 & $(4,1)$ & 3 & 2 & YES & YES & YES & $1.22$ & $(2,2)$ & NO & 1897\\
$(62,23)$ & 9 & $(4,1)$ & 3 & 2 & YES & YES & NO(2) & $0.75$ & $(8,-1)$ & NO & 1898\\
$(62,23)$ & 9 & $(4,1)$ & 3 & 2 & YES & YES & NO(2) & $0.75$ & $(8,-1)$ & -- & 1899\\
$(62,23)$ & 9 & $(4,1)$ & 3 & 2 & YES & YES & YES & $1.12$ & $(2,2)$ & 1682 & 1900\\
$(62,27)$ & 9 & $(4,1)$ & 3 & 2 & YES & YES & NO(2) & $1.30$ & $(4,1)$ & -- & 1901\\
$(62,27)$ & 9 & $(4,1)$ & 3 & 2 & YES & YES & NO(2) & $1.30$ & $(4,1)$ & NO & 1902\\
$(62,17)$ & 10 & $(5,1)$ & 4 & 1 & YES & YES & NO(2) & $1.00$ & $(6,0)$ & NO & 1903\\
$(62,23)$ & 9 & $(5,2)$ & 3 & 1 & YES & YES & YES & $1.12$ & $(6,0)$ & -- & 1904\\
$(62,27)$ & 9 & $(5,2)$ & 3 & 1 & YES & YES & NO(2) & $1.30$ & $(4,1)$ & NO & 1905\\
$(62,27)$ & 9 & $(5,2)$ & 3 & 1 & YES & YES & NO(2) & $1.30$ & $(4,1)$ & -- & 1906\\
$(62,17)$ & 10 & $(6,1)$ & 5 & 2 & YES & YES & NO(2) & $1.30$ & $(4,1)$ & NO & 1907\\
$(62,23)$ & 9 & $(7,2)$ & 4 & 1 & YES & YES & YES & $1.22$ & $(4,1)$ & -- & 1908\\
$(62,27)$ & 9 & $(9,4)$ & 5 & 1 & YES & YES & NO(2) & $1.20$ & $(4,1)$ & 2034 & 1909\\
$(62,19)$ & 10 & $(10,3)$ & 5 & 2 & YES & YES & YES & $1.22$ & $(2,2)$ & NO & 1910\\
$(62,23)$ & 9 & $(11,4)$ & 5 & 1 & YES & YES & NO(2) & $0.88$ & $(8,-1)$ & 1154 & 1911\\
$(62,17)$ & 10 & $(15,4)$ & 6 & 1 & YES & YES & NO(2) & $1.00$ & $(6,0)$ & NO & 1912\\
$(62,19)$ & 10 & $(16,5)$ & 7 & 2 & YES & YES & NO(2) & $1.11$ & $(6,0)$ & NO & 1913\\
$(62,27)$ & 9 & $(16,7)$ & 6 & 2 & YES & YES & YES & $1.11$ & $(2,2)$ & 1716 & 1914\\
$(62,23)$ & 9 & $(19,7)$ & 6 & 1 & YES & YES & NO(2) & $1.11$ & $(6,0)$ & NO & 1915\\
$(62,17)$ & 10 & $(29,8)$ & 7 & 1 & YES & YES & NO(2) & $1.20$ & $(4,1)$ & NO & 1916\\
$(62,23)$ & 9 & $(30,11)$ & 7 & 2 & YES & YES & YES & $1.22$ & $(4,1)$ & NO & 1917\\
$(62,23)$ & 9 & $(35,13)$ & 8 & 1 & YES & YES & YES & $1.00$ & $(2,2)$ & NO & 1918\\
$(62,27)$ & 9 & $(39,17)$ & 8 & 1 & YES & YES & NO(2) & $1.30$ & $(4,1)$ & NO & 1919\\
$(62,23)$ & 9 & $(46,17)$ & 8 & 2 & YES & YES & YES & $1.00$ & $(6,0)$ & NO & 1920\\
$(62,17)$ & 10 & $(51,14)$ & 9 & 1 & YES & YES & NO(2) & $1.20$ & $(4,1)$ & NO & 1921\\
$(62,19)$ & 10 & $(62,19)$ & 10 & 62 & YES & YES & YES & $1.11$ & $(2,2)$ & NO & 1922\\
$(63,26)$ & 9 & $(2,1)$ & 1 & 1 & YES & YES & YES & $1.11$ & $(2,2)$ & -- & 1923\\
$(63,26)$ & 9 & $(2,1)$ & 1 & 1 & YES & YES & YES & $1.33$ & $(2,2)$ & NO & 1924\\
$(63,19)$ & 11 & $(3,1)$ & 2 & 3 & YES & YES & NO(2) & $1.00$ & $(6,0)$ & -- & 1925\\
$(63,26)$ & 9 & $(3,1)$ & 2 & 3 & YES & YES & YES & $1.12$ & $(2,2)$ & -- & 1926\\
$(63,26)$ & 9 & $(3,1)$ & 2 & 3 & YES & YES & YES & $1.33$ & $(2,2)$ & NO & 1927\\
$(63,26)$ & 9 & $(4,1)$ & 3 & 1 & YES & YES & YES & $1.12$ & $(2,2)$ & -- & 1928\\
$(63,26)$ & 9 & $(4,1)$ & 3 & 1 & YES & YES & YES & $1.25$ & $(2,2)$ & NO & 1929\\
$(63,19)$ & 11 & $(5,1)$ & 4 & 1 & YES & YES & NO(2) & $1.00$ & $(6,0)$ & NO & 1930\\
$(63,26)$ & 9 & $(5,1)$ & 4 & 1 & YES & YES & YES & $1.00$ & $(2,2)$ & NO & 1931\\
$(63,26)$ & 9 & $(5,1)$ & 4 & 1 & YES & YES & YES & $1.12$ & $(2,2)$ & -- & 1932\\
$(63,26)$ & 9 & $(12,5)$ & 5 & 3 & YES & YES & YES & $1.11$ & $(2,2)$ & NO & 1933\\
$(63,17)$ & 9 & $(19,5)$ & 7 & 1 & YES & YES & NO(2) & $1.20$ & $(4,1)$ & NO & 1934\\
$(63,26)$ & 9 & $(29,12)$ & 7 & 1 & YES & YES & YES & $1.00$ & $(2,2)$ & 2279 & 1935\\
$(63,19)$ & 11 & $(43,13)$ & 9 & 1 & YES & YES & NO(2) & $1.00$ & $(6,0)$ & 2722 & 1936\\
$(63,26)$ & 9 & $(46,19)$ & 8 & 1 & YES & YES & YES & $1.12$ & $(2,2)$ & NO & 1937\\
$(63,19)$ & 11 & $(63,19)$ & 11 & 63 & YES & YES & NO(2) & $1.00$ & $(6,0)$ & NO & 1938\\
$(63,26)$ & 9 & $(63,26)$ & 9 & 63 & YES & YES & YES & $1.12$ & $(2,2)$ & NO & 1939\\
$(64,23)$ & 9 & $(2,1)$ & 1 & 2 & YES & YES & YES & $1.11$ & $(2,2)$ & -- & 1940\\
$(64,27)$ & 9 & $(2,1)$ & 1 & 2 & YES & YES & YES & $1.11$ & $(2,2)$ & -- & 1941\\
$(64,27)$ & 9 & $(2,1)$ & 1 & 2 & YES & YES & YES & $1.22$ & $(2,2)$ & 1135 & 1942\\
$(64,19)$ & 9 & $(3,1)$ & 2 & 1 & YES & YES & NO(2) & $1.11$ & $(6,0)$ & NO & 1943\\
$(64,19)$ & 9 & $(3,1)$ & 2 & 1 & YES & YES & NO(2) & $1.11$ & $(6,0)$ & -- & 1944\\
$(64,19)$ & 9 & $(3,1)$ & 2 & 1 & YES & YES & YES & $1.12$ & $(2,2)$ & NO & 1945\\
$(64,23)$ & 9 & $(3,1)$ & 2 & 1 & YES & YES & YES & $1.22$ & $(2,2)$ & NO & 1946\\
$(64,23)$ & 9 & $(3,1)$ & 2 & 1 & YES & YES & YES & $1.22$ & $(2,2)$ & -- & 1947\\
$(64,25)$ & 9 & $(3,1)$ & 2 & 1 & YES & YES & YES & $1.12$ & $(2,2)$ & NO & 1948\\
$(64,25)$ & 9 & $(3,1)$ & 2 & 1 & YES & YES & YES & $1.12$ & $(2,2)$ & -- & 1949\\
$(64,27)$ & 9 & $(3,1)$ & 2 & 1 & YES & YES & YES & $1.22$ & $(2,2)$ & -- & 1950\\
$(64,27)$ & 9 & $(3,1)$ & 2 & 1 & YES & YES & YES & $1.33$ & $(2,2)$ & NO & 1951\\
$(64,27)$ & 9 & $(3,1)$ & 2 & 1 & YES & YES & YES & $1.25$ & $(2,2)$ & NO & 1952\\
$(64,19)$ & 9 & $(4,1)$ & 3 & 4 & YES & YES & YES & $1.00$ & $(2,2)$ & NO & 1953\\
$(64,25)$ & 9 & $(4,1)$ & 3 & 4 & YES & YES & YES & $1.12$ & $(2,2)$ & NO & 1954\\
$(64,25)$ & 9 & $(4,1)$ & 3 & 4 & YES & YES & YES & $1.12$ & $(2,2)$ & -- & 1955\\
$(64,25)$ & 9 & $(4,1)$ & 3 & 4 & YES & YES & YES & $1.00$ & $(4,1)$ & NO & 1956\\
$(64,27)$ & 9 & $(4,1)$ & 3 & 4 & YES & YES & YES & $1.11$ & $(2,2)$ & NO & 1957\\
$(64,19)$ & 9 & $(5,2)$ & 3 & 1 & YES & YES & YES & $1.12$ & $(2,2)$ & NO & 1958\\
$(64,19)$ & 9 & $(5,2)$ & 3 & 1 & YES & YES & YES & $1.00$ & $(4,1)$ & -- & 1959\\
$(64,19)$ & 9 & $(5,2)$ & 3 & 1 & YES & YES & YES & $1.12$ & $(4,1)$ & NO & 1960\\
$(64,23)$ & 9 & $(5,1)$ & 4 & 1 & YES & YES & YES & $1.11$ & $(2,2)$ & NO & 1961\\
$(64,23)$ & 9 & $(5,2)$ & 3 & 1 & YES & YES & YES & $1.11$ & $(2,2)$ & -- & 1962\\
$(64,27)$ & 9 & $(5,2)$ & 3 & 1 & YES & YES & YES & $1.11$ & $(2,2)$ & NO & 1963\\
$(64,27)$ & 9 & $(5,2)$ & 3 & 1 & YES & YES & NO(2) & $1.22$ & $(6,0)$ & -- & 1964\\
$(64,27)$ & 9 & $(5,2)$ & 3 & 1 & YES & YES & YES & $1.44$ & $(2,2)$ & NO & 1965\\
$(64,17)$ & 10 & $(7,2)$ & 4 & 1 & YES & YES & NO(2) & $1.10$ & $(4,1)$ & -- & 1966\\
$(64,19)$ & 9 & $(7,2)$ & 4 & 1 & YES & YES & YES & $1.22$ & $(4,1)$ & -- & 1967\\
$(64,27)$ & 9 & $(7,2)$ & 4 & 1 & YES & YES & YES & $1.12$ & $(6,0)$ & NO & 1968\\
$(64,23)$ & 9 & $(8,3)$ & 4 & 8 & YES & YES & NO(2) & $1.30$ & $(4,1)$ & NO & 1969\\
$(64,25)$ & 9 & $(8,3)$ & 4 & 8 & YES & YES & YES & $1.00$ & $(4,1)$ & NO & 1970\\
$(64,27)$ & 9 & $(8,3)$ & 4 & 8 & YES & YES & YES & $1.33$ & $(2,2)$ & NO & 1971\\
$(64,19)$ & 9 & $(9,2)$ & 5 & 1 & YES & YES & YES & $0.88$ & $(2,2)$ & NO & 1972\\
$(64,15)$ & 10 & $(11,3)$ & 5 & 1 & YES & YES & YES & $1.00$ & $(6,0)$ & -- & 1973\\
$(64,23)$ & 9 & $(11,4)$ & 5 & 1 & YES & YES & YES & $1.00$ & $(2,2)$ & NO & 1974\\
$(64,27)$ & 9 & $(12,5)$ & 5 & 4 & YES & YES & YES & $1.12$ & $(2,2)$ & NO & 1975\\
$(64,19)$ & 9 & $(13,4)$ & 6 & 1 & YES & YES & YES & $1.12$ & $(2,2)$ & NO & 1976\\
$(64,25)$ & 9 & $(13,5)$ & 5 & 1 & YES & YES & YES & $1.22$ & $(2,2)$ & 2431 & 1977\\
$(64,23)$ & 9 & $(14,5)$ & 6 & 2 & YES & YES & YES & $1.11$ & $(2,2)$ & 1664 & 1978\\
$(64,19)$ & 9 & $(17,5)$ & 6 & 1 & YES & YES & YES & $1.12$ & $(2,2)$ & NO & 1979\\
$(64,19)$ & 9 & $(24,7)$ & 7 & 8 & YES & YES & YES & $1.00$ & $(4,1)$ & NO & 1980\\
$(64,23)$ & 9 & $(25,9)$ & 7 & 1 & YES & YES & YES & $1.11$ & $(2,2)$ & NO & 1981\\
$(64,25)$ & 9 & $(28,11)$ & 8 & 4 & YES & YES & YES & $1.33$ & $(2,2)$ & NO & 1982\\
$(64,23)$ & 9 & $(36,13)$ & 8 & 4 & YES & YES & YES & $1.22$ & $(2,2)$ & NO & 1983\\
$(64,25)$ & 9 & $(41,16)$ & 8 & 1 & YES & YES & YES & $1.12$ & $(2,2)$ & NO & 1984\\
$(64,19)$ & 9 & $(44,13)$ & 8 & 4 & YES & YES & YES & $0.88$ & $(4,1)$ & NO & 1985\\
$(64,27)$ & 9 & $(45,19)$ & 8 & 1 & YES & YES & YES & $1.00$ & $(2,2)$ & NO & 1986\\
$(64,19)$ & 9 & $(47,14)$ & 9 & 1 & YES & YES & NO(2) & $1.20$ & $(4,1)$ & NO & 1987\\
$(64,25)$ & 9 & $(64,25)$ & 9 & 64 & YES & YES & YES & $1.12$ & $(2,2)$ & NO & 1988\\
$(64,27)$ & 9 & $(64,27)$ & 9 & 64 & YES & YES & YES & $1.00$ & $(2,2)$ & NO & 1989\\
$(65,19)$ & 9 & $(2,1)$ & 1 & 1 & YES & YES & YES & $1.11$ & $(2,2)$ & -- & 1990\\
$(65,19)$ & 9 & $(2,1)$ & 1 & 1 & YES & YES & NO(2) & $1.20$ & $(4,1)$ & NO & 1991\\
$(65,19)$ & 9 & $(3,1)$ & 2 & 1 & YES & YES & NO(2) & $1.20$ & $(4,1)$ & NO & 1992\\
$(65,19)$ & 9 & $(3,1)$ & 2 & 1 & YES & YES & NO(2) & $1.20$ & $(4,1)$ & -- & 1993\\
$(65,19)$ & 9 & $(3,1)$ & 2 & 1 & YES & YES & YES & $1.25$ & $(2,2)$ & NO & 1994\\
$(65,24)$ & 9 & $(3,1)$ & 2 & 1 & YES & YES & YES & $1.11$ & $(2,2)$ & -- & 1995\\
$(65,24)$ & 9 & $(3,1)$ & 2 & 1 & YES & YES & NO(2) & $1.30$ & $(4,1)$ & NO & 1996\\
$(65,27)$ & 10 & $(3,1)$ & 2 & 1 & YES & YES & NO(2) & $1.00$ & $(6,0)$ & -- & 1997\\
$(65,14)$ & 10 & $(4,1)$ & 3 & 1 & YES & YES & NO(2) & $1.30$ & $(4,1)$ & -- & 1998\\
$(65,14)$ & 10 & $(4,1)$ & 3 & 1 & YES & YES & NO(2) & $1.40$ & $(4,1)$ & NO & 1999\\
$(65,19)$ & 9 & $(4,1)$ & 3 & 1 & YES & YES & NO(2) & $1.10$ & $(4,1)$ & -- & 2000\\
$(65,19)$ & 9 & $(4,1)$ & 3 & 1 & YES & YES & YES & $1.44$ & $(2,2)$ & NO & 2001\\
$(65,14)$ & 10 & $(5,2)$ & 3 & 5 & YES & YES & YES & $1.00$ & $(2,2)$ & NO & 2002\\
$(65,14)$ & 10 & $(5,2)$ & 3 & 5 & YES & YES & YES & $1.25$ & $(2,2)$ & NO & 2003\\
$(65,14)$ & 10 & $(5,2)$ & 3 & 5 & YES & YES & NO(2) & $1.11$ & $(4,1)$ & -- & 2004\\
$(65,19)$ & 9 & $(5,2)$ & 3 & 5 & YES & YES & YES & $0.88$ & $(4,1)$ & -- & 2005\\
$(65,19)$ & 9 & $(5,2)$ & 3 & 5 & YES & YES & YES & $1.00$ & $(4,1)$ & NO & 2006\\
$(65,24)$ & 9 & $(5,1)$ & 4 & 5 & YES & YES & YES & $1.11$ & $(2,2)$ & -- & 2007\\
$(65,24)$ & 9 & $(5,2)$ & 3 & 5 & YES & YES & YES & $1.00$ & $(6,0)$ & NO & 2008\\
$(65,24)$ & 9 & $(5,2)$ & 3 & 5 & YES & YES & YES & $1.00$ & $(6,0)$ & -- & 2009\\
$(65,14)$ & 10 & $(7,2)$ & 4 & 1 & YES & YES & YES & $0.88$ & $(2,2)$ & NO & 2010\\
$(65,19)$ & 9 & $(7,2)$ & 4 & 1 & YES & YES & NO(2) & $1.20$ & $(4,1)$ & NO & 2011\\
$(65,19)$ & 9 & $(7,2)$ & 4 & 1 & YES & YES & YES & $1.11$ & $(4,1)$ & -- & 2012\\
$(65,24)$ & 9 & $(7,2)$ & 4 & 1 & YES & YES & YES & $1.22$ & $(4,1)$ & -- & 2013\\
$(65,19)$ & 9 & $(10,3)$ & 5 & 5 & YES & YES & YES & $1.44$ & $(2,2)$ & 2170 & 2014\\
$(65,24)$ & 9 & $(11,4)$ & 5 & 1 & YES & YES & YES & $1.11$ & $(2,2)$ & NO & 2015\\
$(65,14)$ & 10 & $(13,2)$ & 7 & 13 & YES & YES & NO(2) & $1.20$ & $(4,1)$ & NO & 2016\\
$(65,19)$ & 9 & $(24,7)$ & 7 & 1 & YES & YES & NO(2) & $1.20$ & $(4,1)$ & NO & 2017\\
$(65,24)$ & 9 & $(27,10)$ & 7 & 1 & YES & YES & YES & $1.11$ & $(2,2)$ & 2223 & 2018\\
$(65,27)$ & 10 & $(29,12)$ & 7 & 1 & YES & YES & NO(2) & $1.11$ & $(6,0)$ & NO & 2019\\
$(65,24)$ & 9 & $(35,13)$ & 8 & 5 & YES & YES & YES & $1.00$ & $(6,0)$ & 3002 & 2020\\
$(65,18)$ & 9 & $(40,11)$ & 8 & 5 & YES & YES & YES & $1.33$ & $(6,0)$ & 3025 & 2021\\
$(65,24)$ & 9 & $(65,24)$ & 9 & 65 & YES & YES & YES & $1.11$ & $(2,2)$ & NO & 2022\\
$(66,25)$ & 9 & $(2,1)$ & 1 & 2 & YES & YES & YES & $1.22$ & $(2,2)$ & -- & 2023\\
$(66,25)$ & 9 & $(3,1)$ & 2 & 3 & YES & YES & YES & $1.00$ & $(2,2)$ & -- & 2024\\
$(66,25)$ & 9 & $(4,1)$ & 3 & 2 & YES & YES & YES & $1.12$ & $(2,2)$ & -- & 2025\\
$(66,25)$ & 9 & $(4,1)$ & 3 & 2 & YES & YES & YES & $1.12$ & $(2,2)$ & NO & 2026\\
$(66,29)$ & 9 & $(4,1)$ & 3 & 2 & YES & YES & NO(2) & $1.10$ & $(4,1)$ & -- & 2027\\
$(66,25)$ & 9 & $(5,1)$ & 4 & 1 & YES & YES & YES & $1.11$ & $(2,2)$ & NO & 2028\\
$(66,25)$ & 9 & $(5,1)$ & 4 & 1 & YES & YES & YES & $0.88$ & $(4,1)$ & -- & 2029\\
$(66,25)$ & 9 & $(5,2)$ & 3 & 1 & YES & YES & YES & $1.12$ & $(6,0)$ & NO & 2030\\
$(66,29)$ & 9 & $(5,1)$ & 4 & 1 & YES & YES & NO(2) & $1.20$ & $(4,1)$ & NO & 2031\\
$(66,29)$ & 9 & $(5,1)$ & 4 & 1 & YES & YES & NO(2) & $1.20$ & $(4,1)$ & -- & 2032\\
$(66,29)$ & 9 & $(5,2)$ & 3 & 1 & YES & YES & YES & $1.33$ & $(2,2)$ & -- & 2033\\
$(66,29)$ & 9 & $(7,3)$ & 4 & 1 & YES & YES & NO(2) & $1.20$ & $(4,1)$ & 1909 & 2034\\
$(66,25)$ & 9 & $(8,3)$ & 4 & 2 & YES & YES & YES & $1.22$ & $(2,2)$ & 1416 & 2035\\
$(66,25)$ & 9 & $(13,5)$ & 5 & 1 & YES & YES & YES & $1.12$ & $(2,2)$ & 1207 & 2036\\
$(66,29)$ & 9 & $(16,7)$ & 6 & 2 & YES & YES & NO(2) & $1.20$ & $(4,1)$ & 1769 & 2037\\
$(66,25)$ & 9 & $(21,8)$ & 6 & 3 & YES & YES & YES & $1.12$ & $(2,2)$ & NO & 2038\\
$(66,29)$ & 9 & $(23,10)$ & 7 & 1 & YES & YES & YES & $1.33$ & $(2,2)$ & NO & 2039\\
$(66,25)$ & 9 & $(37,14)$ & 8 & 1 & YES & YES & YES & $1.25$ & $(2,2)$ & NO & 2040\\
$(66,29)$ & 9 & $(41,18)$ & 8 & 1 & YES & YES & NO(2) & $1.20$ & $(4,1)$ & NO & 2041\\
$(66,25)$ & 9 & $(50,19)$ & 8 & 2 & YES & YES & YES & $1.00$ & $(6,0)$ & NO & 2042\\
$(66,25)$ & 9 & $(66,25)$ & 9 & 66 & YES & YES & YES & $1.00$ & $(2,2)$ & NO & 2043\\
$(67,18)$ & 9 & $(2,1)$ & 1 & 1 & YES & YES & YES & $1.00$ & $(2,2)$ & NO & 2044\\
$(67,18)$ & 9 & $(2,1)$ & 1 & 1 & YES & YES & YES & $1.00$ & $(2,2)$ & -- & 2045\\
$(67,26)$ & 9 & $(2,1)$ & 1 & 1 & YES & YES & YES & $1.33$ & $(2,2)$ & NO & 2046\\
$(67,26)$ & 9 & $(2,1)$ & 1 & 1 & YES & YES & YES & $1.22$ & $(2,2)$ & -- & 2047\\
$(67,18)$ & 9 & $(3,1)$ & 2 & 1 & YES & YES & YES & $1.00$ & $(2,2)$ & 1292 & 2048\\
$(67,21)$ & 11 & $(3,1)$ & 2 & 1 & YES & YES & NO(2) & $1.20$ & $(4,1)$ & -- & 2049\\
$(67,21)$ & 11 & $(3,1)$ & 2 & 1 & YES & YES & NO(2) & $1.40$ & $(4,1)$ & NO & 2050\\
$(67,29)$ & 10 & $(3,1)$ & 2 & 1 & YES & YES & NO(2) & $1.11$ & $(6,0)$ & -- & 2051\\
$(67,30)$ & 10 & $(3,1)$ & 2 & 1 & YES & YES & NO(2) & $1.30$ & $(4,1)$ & NO & 2052\\
$(67,30)$ & 10 & $(3,1)$ & 2 & 1 & YES & YES & NO(2) & $1.30$ & $(4,1)$ & -- & 2053\\
$(67,30)$ & 10 & $(3,1)$ & 2 & 1 & YES & YES & NO(2) & $1.22$ & $(6,0)$ & NO & 2054\\
$(67,29)$ & 10 & $(4,1)$ & 3 & 1 & YES & YES & YES & $1.12$ & $(2,2)$ & -- & 2055\\
$(67,29)$ & 10 & $(4,1)$ & 3 & 1 & YES & YES & YES & $1.25$ & $(2,2)$ & NO & 2056\\
$(67,30)$ & 10 & $(4,1)$ & 3 & 1 & YES & YES & NO(2) & $1.40$ & $(4,1)$ & 753 & 2057\\
$(67,30)$ & 10 & $(4,1)$ & 3 & 1 & YES & YES & NO(2) & $1.40$ & $(4,1)$ & NO & 2058\\
$(67,30)$ & 10 & $(4,1)$ & 3 & 1 & YES & YES & NO(2) & $1.40$ & $(4,1)$ & -- & 2059\\
$(67,21)$ & 11 & $(5,1)$ & 4 & 1 & YES & YES & NO(2) & $1.20$ & $(4,1)$ & -- & 2060\\
$(67,21)$ & 11 & $(7,2)$ & 4 & 1 & YES & YES & NO(2) & $1.20$ & $(4,1)$ & NO & 2061\\
$(67,18)$ & 9 & $(9,2)$ & 5 & 1 & YES & YES & NO(2) & $1.10$ & $(4,1)$ & NO & 2062\\
$(67,18)$ & 9 & $(11,3)$ & 5 & 1 & YES & YES & YES & $1.00$ & $(2,2)$ & NO & 2063\\
$(67,30)$ & 10 & $(11,5)$ & 6 & 1 & YES & YES & NO(2) & $1.30$ & $(4,1)$ & 1217 & 2064\\
$(67,21)$ & 11 & $(13,4)$ & 6 & 1 & YES & YES & NO(2) & $1.30$ & $(4,1)$ & NO & 2065\\
$(67,29)$ & 10 & $(16,7)$ & 6 & 1 & YES & YES & YES & $1.12$ & $(2,2)$ & 1248 & 2066\\
$(67,30)$ & 10 & $(20,9)$ & 7 & 1 & YES & YES & NO(2) & $1.40$ & $(4,1)$ & NO & 2067\\
$(67,30)$ & 10 & $(29,13)$ & 8 & 1 & YES & YES & NO(2) & $1.22$ & $(6,0)$ & NO & 2068\\
$(67,18)$ & 9 & $(56,15)$ & 9 & 1 & YES & YES & NO(2) & $1.10$ & $(4,1)$ & NO & 2069\\
$(67,29)$ & 10 & $(67,29)$ & 10 & 67 & YES & YES & NO(2) & $1.22$ & $(6,0)$ & NO & 2070\\
$(67,30)$ & 10 & $(67,30)$ & 10 & 67 & YES & YES & NO(2) & $1.11$ & $(6,0)$ & NO & 2071\\
$(68,19)$ & 9 & $(3,1)$ & 2 & 1 & YES & YES & YES & $1.44$ & $(2,2)$ & NO & 2072\\
$(68,21)$ & 11 & $(3,1)$ & 2 & 1 & YES & YES & NO(2) & $1.11$ & $(6,0)$ & -- & 2073\\
$(68,25)$ & 9 & $(3,1)$ & 2 & 1 & YES & YES & YES & $0.88$ & $(4,1)$ & NO & 2074\\
$(68,25)$ & 9 & $(3,1)$ & 2 & 1 & YES & YES & YES & $1.12$ & $(2,2)$ & NO & 2075\\
$(68,25)$ & 9 & $(3,1)$ & 2 & 1 & YES & YES & YES & $1.12$ & $(2,2)$ & -- & 2076\\
$(68,25)$ & 9 & $(4,1)$ & 3 & 4 & YES & YES & YES & $1.00$ & $(4,1)$ & NO & 2077\\
$(68,25)$ & 9 & $(4,1)$ & 3 & 4 & YES & YES & YES & $1.00$ & $(2,2)$ & NO & 2078\\
$(68,25)$ & 9 & $(4,1)$ & 3 & 4 & YES & YES & YES & $1.22$ & $(2,2)$ & -- & 2079\\
$(68,19)$ & 9 & $(5,2)$ & 3 & 1 & YES & YES & YES & $1.22$ & $(4,1)$ & -- & 2080\\
$(68,19)$ & 9 & $(5,2)$ & 3 & 1 & YES & YES & YES & $1.22$ & $(4,1)$ & NO & 2081\\
$(68,25)$ & 9 & $(5,2)$ & 3 & 1 & YES & YES & YES & $1.22$ & $(4,1)$ & -- & 2082\\
$(68,25)$ & 9 & $(5,2)$ & 3 & 1 & YES & YES & YES & $1.25$ & $(6,0)$ & NO & 2083\\
$(68,25)$ & 9 & $(5,2)$ & 3 & 1 & YES & YES & YES & $1.22$ & $(2,2)$ & NO & 2084\\
$(68,21)$ & 11 & $(7,2)$ & 4 & 1 & YES & YES & NO(2) & $1.00$ & $(6,0)$ & NO & 2085\\
$(68,25)$ & 9 & $(7,2)$ & 4 & 1 & YES & YES & YES & $1.22$ & $(4,1)$ & -- & 2086\\
$(68,25)$ & 9 & $(8,3)$ & 4 & 4 & YES & YES & YES & $1.22$ & $(2,2)$ & NO & 2087\\
$(68,25)$ & 9 & $(14,5)$ & 6 & 2 & YES & YES & YES & $1.33$ & $(2,2)$ & 2560 & 2088\\
$(68,25)$ & 9 & $(19,7)$ & 6 & 1 & YES & YES & YES & $1.12$ & $(2,2)$ & NO & 2089\\
$(68,19)$ & 9 & $(29,8)$ & 7 & 1 & YES & YES & YES & $0.75$ & $(4,1)$ & NO & 2090\\
$(68,25)$ & 9 & $(30,11)$ & 7 & 2 & YES & YES & YES & $1.12$ & $(2,2)$ & 2402 & 2091\\
$(68,25)$ & 9 & $(49,18)$ & 8 & 1 & YES & YES & YES & $1.22$ & $(2,2)$ & NO & 2092\\
$(69,20)$ & 10 & $(2,1)$ & 1 & 1 & YES & YES & YES & $1.22$ & $(2,2)$ & -- & 2093\\
$(69,29)$ & 9 & $(2,1)$ & 1 & 1 & YES & YES & YES & $1.12$ & $(2,2)$ & -- & 2094\\
$(69,31)$ & 10 & $(2,1)$ & 1 & 1 & YES & YES & NO(2) & $1.11$ & $(6,0)$ & -- & 2095\\
$(69,19)$ & 9 & $(3,1)$ & 2 & 3 & YES & YES & YES & $1.00$ & $(2,2)$ & NO & 2096\\
$(69,19)$ & 9 & $(3,1)$ & 2 & 3 & YES & YES & YES & $1.11$ & $(2,2)$ & -- & 2097\\
$(69,29)$ & 9 & $(3,1)$ & 2 & 3 & YES & YES & YES & $1.12$ & $(2,2)$ & NO & 2098\\
$(69,29)$ & 9 & $(3,1)$ & 2 & 3 & YES & YES & YES & $1.12$ & $(2,2)$ & -- & 2099\\
$(69,29)$ & 9 & $(3,1)$ & 2 & 3 & YES & YES & YES & $1.25$ & $(2,2)$ & NO & 2100\\
$(69,19)$ & 9 & $(4,1)$ & 3 & 1 & YES & YES & YES & $1.00$ & $(4,1)$ & -- & 2101\\
$(69,19)$ & 9 & $(4,1)$ & 3 & 1 & YES & YES & YES & $1.12$ & $(4,1)$ & NO & 2102\\
$(69,20)$ & 10 & $(4,1)$ & 3 & 1 & YES & YES & YES & $1.22$ & $(2,2)$ & NO & 2103\\
$(69,19)$ & 9 & $(5,2)$ & 3 & 1 & YES & YES & NO(2) & $1.20$ & $(4,1)$ & -- & 2104\\
$(69,29)$ & 9 & $(5,2)$ & 3 & 1 & YES & YES & YES & $1.33$ & $(4,1)$ & -- & 2105\\
$(69,16)$ & 11 & $(7,2)$ & 4 & 1 & YES & YES & NO(2) & $1.11$ & $(6,0)$ & 1809 & 2106\\
$(69,29)$ & 9 & $(7,2)$ & 4 & 1 & YES & YES & YES & $1.22$ & $(4,1)$ & NO & 2107\\
$(69,29)$ & 9 & $(7,3)$ & 4 & 1 & YES & YES & YES & $1.12$ & $(2,2)$ & NO & 2108\\
$(69,19)$ & 9 & $(9,2)$ & 5 & 3 & YES & YES & YES & $1.00$ & $(2,2)$ & NO & 2109\\
$(69,31)$ & 10 & $(9,4)$ & 5 & 3 & YES & YES & NO(2) & $1.11$ & $(6,0)$ & NO & 2110\\
$(69,29)$ & 9 & $(11,2)$ & 6 & 1 & YES & YES & YES & $1.22$ & $(4,1)$ & NO & 2111\\
$(69,29)$ & 9 & $(11,2)$ & 6 & 1 & YES & YES & YES & $1.33$ & $(4,1)$ & -- & 2112\\
$(69,29)$ & 9 & $(11,2)$ & 6 & 1 & YES & YES & YES & $1.40$ & $(4,1)$ & NO & 2113\\
$(69,29)$ & 9 & $(12,5)$ & 5 & 3 & YES & YES & YES & $1.12$ & $(2,2)$ & NO & 2114\\
$(69,19)$ & 9 & $(15,4)$ & 6 & 3 & YES & YES & YES & $1.11$ & $(2,2)$ & NO & 2115\\
$(69,19)$ & 9 & $(19,5)$ & 7 & 1 & YES & YES & NO(2) & $1.20$ & $(4,1)$ & NO & 2116\\
$(69,31)$ & 10 & $(20,9)$ & 7 & 1 & YES & YES & NO(2) & $1.11$ & $(6,0)$ & NO & 2117\\
$(69,16)$ & 11 & $(21,5)$ & 8 & 3 & YES & YES & YES & $1.12$ & $(2,2)$ & NO & 2118\\
$(69,29)$ & 9 & $(31,13)$ & 7 & 1 & YES & YES & YES & $1.33$ & $(2,2)$ & 2466 & 2119\\
$(69,29)$ & 9 & $(43,18)$ & 8 & 1 & YES & YES & YES & $1.22$ & $(4,1)$ & 3092 & 2120\\
$(69,19)$ & 9 & $(51,14)$ & 9 & 3 & YES & YES & YES & $1.00$ & $(2,2)$ & NO & 2121\\
$(70,29)$ & 9 & $(2,1)$ & 1 & 2 & YES & YES & YES & $1.22$ & $(2,2)$ & NO & 2122\\
$(70,29)$ & 9 & $(3,1)$ & 2 & 1 & YES & YES & YES & $1.22$ & $(2,2)$ & NO & 2123\\
$(70,29)$ & 9 & $(3,1)$ & 2 & 1 & YES & YES & NO(2) & $1.20$ & $(4,1)$ & -- & 2124\\
$(70,19)$ & 11 & $(5,1)$ & 4 & 5 & YES & YES & NO(2) & $1.11$ & $(6,0)$ & NO & 2125\\
$(70,29)$ & 9 & $(5,2)$ & 3 & 5 & YES & YES & YES & $1.00$ & $(6,0)$ & -- & 2126\\
$(70,29)$ & 9 & $(7,2)$ & 4 & 7 & YES & YES & YES & $1.22$ & $(4,1)$ & NO & 2127\\
$(70,29)$ & 9 & $(7,2)$ & 4 & 7 & YES & YES & YES & $1.22$ & $(4,1)$ & -- & 2128\\
$(70,29)$ & 9 & $(7,3)$ & 4 & 7 & YES & YES & YES & $1.33$ & $(2,2)$ & NO & 2129\\
$(70,29)$ & 9 & $(8,3)$ & 4 & 2 & YES & YES & YES & $1.00$ & $(6,0)$ & NO & 2130\\
$(70,29)$ & 9 & $(9,2)$ & 5 & 1 & YES & YES & YES & $1.00$ & $(6,0)$ & NO & 2131\\
$(70,19)$ & 11 & $(15,4)$ & 6 & 5 & YES & YES & NO(2) & $1.11$ & $(6,0)$ & NO & 2132\\
$(70,29)$ & 9 & $(22,9)$ & 7 & 2 & YES & YES & YES & $1.22$ & $(4,1)$ & NO & 2133\\
$(70,29)$ & 9 & $(29,12)$ & 7 & 1 & YES & YES & YES & $1.11$ & $(2,2)$ & NO & 2134\\
$(70,29)$ & 9 & $(53,22)$ & 9 & 1 & YES & YES & YES & $1.00$ & $(6,0)$ & NO & 2135\\
$(70,29)$ & 9 & $(70,29)$ & 9 & 70 & YES & YES & YES & $1.11$ & $(2,2)$ & NO & 2136\\
$(71,21)$ & 9 & $(2,1)$ & 1 & 1 & YES & YES & YES & $1.12$ & $(2,2)$ & NO & 2137\\
$(71,21)$ & 9 & $(2,1)$ & 1 & 1 & YES & YES & YES & $1.25$ & $(2,2)$ & -- & 2138\\
$(71,22)$ & 10 & $(2,1)$ & 1 & 1 & YES & YES & YES & $1.33$ & $(2,2)$ & NO & 2139\\
$(71,27)$ & 9 & $(2,1)$ & 1 & 1 & YES & YES & YES & $1.11$ & $(2,2)$ & NO & 2140\\
$(71,30)$ & 9 & $(2,1)$ & 1 & 1 & YES & YES & YES & $1.12$ & $(2,2)$ & -- & 2141\\
$(71,31)$ & 10 & $(2,1)$ & 1 & 1 & YES & YES & YES & $1.00$ & $(4,1)$ & -- & 2142\\
$(71,31)$ & 10 & $(2,1)$ & 1 & 1 & YES & YES & YES & $1.12$ & $(4,1)$ & NO & 2143\\
$(71,21)$ & 9 & $(3,1)$ & 2 & 1 & YES & YES & YES & $1.12$ & $(2,2)$ & -- & 2144\\
$(71,21)$ & 9 & $(3,1)$ & 2 & 1 & YES & YES & YES & $1.25$ & $(2,2)$ & NO & 2145\\
$(71,26)$ & 9 & $(3,1)$ & 2 & 1 & YES & YES & YES & $1.12$ & $(2,2)$ & NO & 2146\\
$(71,26)$ & 9 & $(3,1)$ & 2 & 1 & YES & YES & YES & $1.12$ & $(2,2)$ & -- & 2147\\
$(71,26)$ & 9 & $(3,1)$ & 2 & 1 & YES & YES & NO(2) & $1.11$ & $(6,0)$ & NO & 2148\\
$(71,27)$ & 9 & $(3,1)$ & 2 & 1 & YES & YES & NO(2) & $1.10$ & $(4,1)$ & -- & 2149\\
$(71,30)$ & 9 & $(3,1)$ & 2 & 1 & YES & YES & YES & $1.00$ & $(2,2)$ & NO & 2150\\
$(71,31)$ & 10 & $(3,1)$ & 2 & 1 & YES & YES & YES & $1.00$ & $(4,1)$ & NO & 2151\\
$(71,31)$ & 10 & $(3,1)$ & 2 & 1 & YES & YES & YES & $1.00$ & $(4,1)$ & -- & 2152\\
$(71,32)$ & 10 & $(3,1)$ & 2 & 1 & YES & YES & NO(2) & $1.30$ & $(4,1)$ & -- & 2153\\
$(71,30)$ & 9 & $(4,1)$ & 3 & 1 & YES & YES & NO(2) & $1.20$ & $(4,1)$ & NO & 2154\\
$(71,30)$ & 9 & $(4,1)$ & 3 & 1 & YES & YES & NO(2) & $1.20$ & $(4,1)$ & -- & 2155\\
$(71,21)$ & 9 & $(5,1)$ & 4 & 1 & YES & YES & YES & $1.12$ & $(2,2)$ & NO & 2156\\
$(71,21)$ & 9 & $(5,1)$ & 4 & 1 & YES & YES & YES & $1.12$ & $(2,2)$ & -- & 2157\\
$(71,21)$ & 9 & $(5,2)$ & 3 & 1 & YES & YES & YES & $1.12$ & $(4,1)$ & NO & 2158\\
$(71,21)$ & 9 & $(5,2)$ & 3 & 1 & YES & YES & YES & $1.12$ & $(4,1)$ & -- & 2159\\
$(71,22)$ & 10 & $(5,1)$ & 4 & 1 & YES & YES & NO(2) & $1.00$ & $(6,0)$ & -- & 2160\\
$(71,22)$ & 10 & $(5,1)$ & 4 & 1 & YES & YES & NO(2) & $0.88$ & $(8,-1)$ & NO & 2161\\
$(71,26)$ & 9 & $(5,2)$ & 3 & 1 & YES & YES & YES & $1.12$ & $(2,2)$ & NO & 2162\\
$(71,26)$ & 9 & $(5,2)$ & 3 & 1 & YES & YES & YES & $1.12$ & $(6,0)$ & NO & 2163\\
$(71,26)$ & 9 & $(5,2)$ & 3 & 1 & YES & YES & YES & $1.12$ & $(6,0)$ & -- & 2164\\
$(71,30)$ & 9 & $(5,2)$ & 3 & 1 & YES & YES & YES & $1.12$ & $(2,2)$ & NO & 2165\\
$(71,31)$ & 10 & $(5,2)$ & 3 & 1 & YES & YES & NO(2) & $1.11$ & $(6,0)$ & NO & 2166\\
$(71,31)$ & 10 & $(5,2)$ & 3 & 1 & YES & YES & NO(2) & $1.11$ & $(6,0)$ & -- & 2167\\
$(71,32)$ & 10 & $(5,1)$ & 4 & 1 & YES & YES & NO(2) & $1.30$ & $(4,1)$ & -- & 2168\\
$(71,21)$ & 9 & $(7,2)$ & 4 & 1 & YES & YES & YES & $1.00$ & $(6,0)$ & -- & 2169\\
$(71,21)$ & 9 & $(7,2)$ & 4 & 1 & YES & YES & YES & $1.44$ & $(2,2)$ & 2014 & 2170\\
$(71,26)$ & 9 & $(7,2)$ & 4 & 1 & YES & YES & YES & $1.33$ & $(2,2)$ & NO & 2171\\
$(71,26)$ & 9 & $(7,3)$ & 4 & 1 & YES & YES & YES & $1.12$ & $(6,0)$ & NO & 2172\\
$(71,20)$ & 10 & $(9,2)$ & 5 & 1 & YES & YES & NO(2) & $1.00$ & $(6,0)$ & NO & 2173\\
$(71,21)$ & 9 & $(9,2)$ & 5 & 1 & YES & YES & YES & $1.11$ & $(4,1)$ & NO & 2174\\
$(71,31)$ & 10 & $(9,4)$ & 5 & 1 & YES & YES & NO(2) & $1.11$ & $(6,0)$ & NO & 2175\\
$(71,21)$ & 9 & $(10,3)$ & 5 & 1 & YES & YES & YES & $1.00$ & $(2,2)$ & NO & 2176\\
$(71,20)$ & 10 & $(11,3)$ & 5 & 1 & YES & YES & NO(2) & $1.00$ & $(6,0)$ & NO & 2177\\
$(71,21)$ & 9 & $(11,3)$ & 5 & 1 & YES & YES & YES & $1.00$ & $(4,1)$ & NO & 2178\\
$(71,22)$ & 10 & $(13,4)$ & 6 & 1 & YES & YES & NO(2) & $1.11$ & $(6,0)$ & 1717 & 2179\\
$(71,27)$ & 9 & $(13,5)$ & 5 & 1 & YES & YES & YES & $1.12$ & $(2,2)$ & NO & 2180\\
$(71,21)$ & 9 & $(17,5)$ & 6 & 1 & YES & YES & YES & $1.25$ & $(2,2)$ & 1883 & 2181\\
$(71,20)$ & 10 & $(18,5)$ & 6 & 1 & YES & YES & NO(2) & $1.20$ & $(4,1)$ & 1373 & 2182\\
$(71,19)$ & 10 & $(19,5)$ & 7 & 1 & YES & YES & NO(2) & $1.00$ & $(6,0)$ & NO & 2183\\
$(71,26)$ & 9 & $(19,7)$ & 6 & 1 & YES & YES & YES & $1.00$ & $(2,2)$ & NO & 2184\\
$(71,32)$ & 10 & $(20,9)$ & 7 & 1 & YES & YES & NO(2) & $1.40$ & $(4,1)$ & NO & 2185\\
$(71,21)$ & 9 & $(24,7)$ & 7 & 1 & YES & YES & YES & $1.00$ & $(4,1)$ & NO & 2186\\
$(71,19)$ & 10 & $(26,7)$ & 7 & 1 & YES & YES & YES & $1.00$ & $(2,2)$ & NO & 2187\\
$(71,21)$ & 9 & $(27,8)$ & 7 & 1 & YES & YES & YES & $1.25$ & $(2,2)$ & NO & 2188\\
$(71,26)$ & 9 & $(27,10)$ & 7 & 1 & YES & YES & YES & $1.33$ & $(2,2)$ & NO & 2189\\
$(71,22)$ & 10 & $(29,9)$ & 8 & 1 & YES & YES & YES & $1.22$ & $(2,2)$ & NO & 2190\\
$(71,32)$ & 10 & $(31,14)$ & 8 & 1 & YES & YES & NO(2) & $1.30$ & $(4,1)$ & 2496 & 2191\\
$(71,21)$ & 9 & $(37,11)$ & 8 & 1 & YES & YES & YES & $1.12$ & $(4,1)$ & NO & 2192\\
$(71,20)$ & 10 & $(39,11)$ & 9 & 1 & YES & YES & NO(2) & $1.20$ & $(4,1)$ & NO & 2193\\
$(71,26)$ & 9 & $(49,18)$ & 8 & 1 & YES & YES & YES & $1.00$ & $(6,0)$ & NO & 2194\\
$(71,21)$ & 9 & $(61,18)$ & 9 & 1 & YES & YES & YES & $1.11$ & $(4,1)$ & NO & 2195\\
$(71,26)$ & 9 & $(71,26)$ & 9 & 71 & YES & YES & NO(2) & $0.89$ & $(6,0)$ & NO & 2196\\
$(71,27)$ & 9 & $(71,27)$ & 9 & 71 & YES & YES & YES & $1.11$ & $(2,2)$ & NO & 2197\\
$(72,19)$ & 10 & $(2,1)$ & 1 & 2 & YES & YES & YES & $1.11$ & $(2,2)$ & -- & 2198\\
$(72,19)$ & 10 & $(2,1)$ & 1 & 2 & YES & YES & YES & $1.22$ & $(2,2)$ & NO & 2199\\
$(72,19)$ & 10 & $(3,1)$ & 2 & 3 & YES & YES & YES & $1.33$ & $(2,2)$ & NO & 2200\\
$(72,19)$ & 10 & $(3,1)$ & 2 & 3 & YES & YES & NO(2) & $1.20$ & $(4,1)$ & -- & 2201\\
$(72,19)$ & 10 & $(3,1)$ & 2 & 3 & YES & YES & NO(2) & $1.30$ & $(4,1)$ & NO & 2202\\
$(72,19)$ & 10 & $(4,1)$ & 3 & 4 & YES & YES & NO(2) & $1.30$ & $(4,1)$ & -- & 2203\\
$(72,19)$ & 10 & $(5,1)$ & 4 & 1 & YES & YES & YES & $1.22$ & $(2,2)$ & -- & 2204\\
$(72,19)$ & 10 & $(5,1)$ & 4 & 1 & YES & YES & NO(2) & $1.11$ & $(6,0)$ & NO & 2205\\
$(72,19)$ & 10 & $(7,2)$ & 4 & 1 & YES & YES & NO(2) & $1.20$ & $(4,1)$ & NO & 2206\\
$(72,19)$ & 10 & $(11,3)$ & 5 & 1 & YES & YES & NO(2) & $1.20$ & $(4,1)$ & NO & 2207\\
$(72,19)$ & 10 & $(15,4)$ & 6 & 3 & YES & YES & YES & $1.11$ & $(2,2)$ & NO & 2208\\
$(72,19)$ & 10 & $(23,6)$ & 8 & 1 & YES & YES & NO(2) & $1.00$ & $(6,0)$ & NO & 2209\\
$(72,17)$ & 11 & $(30,7)$ & 8 & 6 & YES & YES & YES & $1.12$ & $(2,2)$ & NO & 2210\\
$(72,19)$ & 10 & $(53,14)$ & 9 & 1 & YES & YES & NO(2) & $1.30$ & $(4,1)$ & NO & 2211\\
$(72,19)$ & 10 & $(72,19)$ & 10 & 72 & YES & YES & NO(2) & $1.00$ & $(6,0)$ & NO & 2212\\
$(73,27)$ & 9 & $(2,1)$ & 1 & 1 & YES & YES & YES & $1.11$ & $(2,2)$ & -- & 2213\\
$(73,27)$ & 9 & $(2,1)$ & 1 & 1 & YES & YES & NO(2) & $1.00$ & $(6,0)$ & NO & 2214\\
$(73,27)$ & 9 & $(3,1)$ & 2 & 1 & YES & YES & YES & $1.12$ & $(2,2)$ & NO & 2215\\
$(73,27)$ & 9 & $(3,1)$ & 2 & 1 & YES & YES & YES & $1.12$ & $(2,2)$ & -- & 2216\\
$(73,27)$ & 9 & $(4,1)$ & 3 & 1 & YES & YES & YES & $1.12$ & $(2,2)$ & -- & 2217\\
$(73,27)$ & 9 & $(4,1)$ & 3 & 1 & YES & YES & YES & $1.11$ & $(2,2)$ & NO & 2218\\
$(73,27)$ & 9 & $(5,1)$ & 4 & 1 & YES & YES & YES & $1.22$ & $(2,2)$ & -- & 2219\\
$(73,27)$ & 9 & $(5,2)$ & 3 & 1 & YES & YES & YES & $1.00$ & $(6,0)$ & -- & 2220\\
$(73,27)$ & 9 & $(5,2)$ & 3 & 1 & YES & YES & YES & $1.00$ & $(6,0)$ & NO & 2221\\
$(73,27)$ & 9 & $(13,5)$ & 5 & 1 & YES & YES & YES & $1.00$ & $(6,0)$ & NO & 2222\\
$(73,27)$ & 9 & $(19,7)$ & 6 & 1 & YES & YES & YES & $1.11$ & $(2,2)$ & 2018 & 2223\\
$(73,27)$ & 9 & $(27,10)$ & 7 & 1 & YES & YES & YES & $1.33$ & $(2,2)$ & NO & 2224\\
$(73,27)$ & 9 & $(30,11)$ & 7 & 1 & YES & YES & YES & $1.22$ & $(4,1)$ & NO & 2225\\
$(73,27)$ & 9 & $(35,13)$ & 8 & 1 & YES & YES & YES & $1.00$ & $(6,0)$ & NO & 2226\\
$(73,27)$ & 9 & $(46,17)$ & 8 & 1 & YES & YES & YES & $1.00$ & $(2,2)$ & NO & 2227\\
$(73,27)$ & 9 & $(65,24)$ & 9 & 1 & YES & YES & YES & $1.22$ & $(4,1)$ & NO & 2228\\
$(73,27)$ & 9 & $(73,27)$ & 9 & 73 & YES & YES & YES & $1.12$ & $(2,2)$ & NO & 2229\\
$(74,17)$ & 11 & $(2,1)$ & 1 & 2 & YES & YES & YES & $1.00$ & $(2,2)$ & NO & 2230\\
$(74,17)$ & 11 & $(2,1)$ & 1 & 2 & YES & YES & NO(2) & $1.20$ & $(4,1)$ & -- & 2231\\
$(74,31)$ & 9 & $(2,1)$ & 1 & 2 & YES & YES & YES & $1.12$ & $(2,2)$ & -- & 2232\\
$(74,17)$ & 11 & $(3,1)$ & 2 & 1 & YES & YES & YES & $0.88$ & $(4,1)$ & NO & 2233\\
$(74,17)$ & 11 & $(3,1)$ & 2 & 1 & YES & YES & YES & $0.88$ & $(4,1)$ & -- & 2234\\
$(74,17)$ & 11 & $(3,1)$ & 2 & 1 & YES & YES & NO(2) & $1.11$ & $(6,0)$ & NO & 2235\\
$(74,31)$ & 9 & $(3,1)$ & 2 & 1 & YES & YES & NO(2) & $1.10$ & $(4,1)$ & -- & 2236\\
$(74,31)$ & 9 & $(3,1)$ & 2 & 1 & YES & YES & NO(2) & $1.20$ & $(4,1)$ & NO & 2237\\
$(74,29)$ & 10 & $(4,1)$ & 3 & 2 & YES & YES & YES & $1.22$ & $(2,2)$ & -- & 2238\\
$(74,31)$ & 9 & $(4,1)$ & 3 & 2 & YES & YES & YES & $1.12$ & $(2,2)$ & NO & 2239\\
$(74,17)$ & 11 & $(5,1)$ & 4 & 1 & YES & YES & YES & $1.22$ & $(2,2)$ & NO & 2240\\
$(74,17)$ & 11 & $(5,2)$ & 3 & 1 & YES & YES & YES & $1.22$ & $(2,2)$ & -- & 2241\\
$(74,23)$ & 10 & $(5,1)$ & 4 & 1 & YES & YES & NO(2) & $1.00$ & $(6,0)$ & NO & 2242\\
$(74,31)$ & 9 & $(5,1)$ & 4 & 1 & YES & YES & YES & $1.00$ & $(2,2)$ & NO & 2243\\
$(74,31)$ & 9 & $(5,2)$ & 3 & 1 & YES & YES & YES & $1.33$ & $(4,1)$ & -- & 2244\\
$(74,31)$ & 9 & $(5,2)$ & 3 & 1 & YES & YES & YES & $1.12$ & $(2,2)$ & NO & 2245\\
$(74,17)$ & 11 & $(7,1)$ & 6 & 1 & YES & YES & NO(2) & $1.20$ & $(4,1)$ & NO & 2246\\
$(74,31)$ & 9 & $(7,3)$ & 4 & 1 & YES & YES & YES & $1.12$ & $(2,2)$ & 1371 & 2247\\
$(74,17)$ & 11 & $(9,2)$ & 5 & 1 & YES & YES & YES & $1.22$ & $(2,2)$ & NO & 2248\\
$(74,31)$ & 9 & $(9,2)$ & 5 & 1 & YES & YES & YES & $1.22$ & $(4,1)$ & -- & 2249\\
$(74,31)$ & 9 & $(9,2)$ & 5 & 1 & YES & YES & YES & $1.33$ & $(4,1)$ & NO & 2250\\
$(74,31)$ & 9 & $(12,5)$ & 5 & 2 & YES & YES & YES & $1.00$ & $(2,2)$ & 1719 & 2251\\
$(74,29)$ & 10 & $(13,5)$ & 5 & 1 & YES & YES & YES & $1.22$ & $(2,2)$ & NO & 2252\\
$(74,17)$ & 11 & $(14,3)$ & 6 & 2 & YES & YES & YES & $1.22$ & $(2,2)$ & NO & 2253\\
$(74,31)$ & 9 & $(19,8)$ & 6 & 1 & YES & YES & NO(2) & $1.00$ & $(6,0)$ & NO & 2254\\
$(74,23)$ & 10 & $(29,9)$ & 8 & 1 & YES & YES & NO(2) & $1.11$ & $(6,0)$ & NO & 2255\\
$(74,31)$ & 9 & $(31,13)$ & 7 & 1 & YES & YES & YES & $1.12$ & $(2,2)$ & NO & 2256\\
$(74,17)$ & 11 & $(35,8)$ & 8 & 1 & YES & YES & NO(2) & $1.20$ & $(4,1)$ & NO & 2257\\
$(74,31)$ & 9 & $(43,18)$ & 8 & 1 & YES & YES & NO(2) & $1.20$ & $(4,1)$ & NO & 2258\\
$(74,29)$ & 10 & $(51,20)$ & 9 & 1 & YES & YES & YES & $1.22$ & $(2,2)$ & NO & 2259\\
$(74,31)$ & 9 & $(55,23)$ & 9 & 1 & YES & YES & YES & $1.33$ & $(4,1)$ & NO & 2260\\
$(75,29)$ & 9 & $(2,1)$ & 1 & 1 & YES & YES & YES & $1.11$ & $(2,2)$ & NO & 2261\\
$(75,29)$ & 9 & $(2,1)$ & 1 & 1 & YES & YES & YES & $1.11$ & $(2,2)$ & -- & 2262\\
$(75,31)$ & 9 & $(2,1)$ & 1 & 1 & YES & YES & YES & $1.22$ & $(2,2)$ & 1441 & 2263\\
$(75,31)$ & 9 & $(3,1)$ & 2 & 3 & YES & YES & YES & $1.00$ & $(2,2)$ & NO & 2264\\
$(75,31)$ & 9 & $(3,1)$ & 2 & 3 & YES & YES & YES & $1.12$ & $(2,2)$ & -- & 2265\\
$(75,29)$ & 9 & $(4,1)$ & 3 & 1 & YES & YES & NO(2) & $1.10$ & $(4,1)$ & NO & 2266\\
$(75,31)$ & 9 & $(4,1)$ & 3 & 1 & YES & YES & YES & $1.00$ & $(2,2)$ & -- & 2267\\
$(75,22)$ & 10 & $(5,2)$ & 3 & 5 & YES & YES & YES & $1.00$ & $(6,0)$ & -- & 2268\\
$(75,31)$ & 9 & $(5,1)$ & 4 & 5 & YES & YES & YES & $0.88$ & $(2,2)$ & NO & 2269\\
$(75,31)$ & 9 & $(5,1)$ & 4 & 5 & YES & YES & NO(2) & $1.11$ & $(4,1)$ & -- & 2270\\
$(75,31)$ & 9 & $(5,2)$ & 3 & 5 & YES & YES & YES & $1.22$ & $(4,1)$ & -- & 2271\\
$(75,31)$ & 9 & $(5,2)$ & 3 & 5 & YES & YES & YES & $1.00$ & $(2,2)$ & 1877 & 2272\\
$(75,29)$ & 9 & $(7,2)$ & 4 & 1 & YES & YES & YES & $1.12$ & $(2,2)$ & -- & 2273\\
$(75,29)$ & 9 & $(7,3)$ & 4 & 1 & YES & YES & YES & $1.12$ & $(6,0)$ & NO & 2274\\
$(75,29)$ & 9 & $(8,3)$ & 4 & 1 & YES & YES & NO(2) & $1.20$ & $(4,1)$ & NO & 2275\\
$(75,31)$ & 9 & $(8,3)$ & 4 & 1 & YES & YES & YES & $1.22$ & $(4,1)$ & NO & 2276\\
$(75,22)$ & 10 & $(10,3)$ & 5 & 5 & YES & YES & YES & $1.12$ & $(2,2)$ & NO & 2277\\
$(75,16)$ & 11 & $(11,2)$ & 6 & 1 & YES & YES & YES & $1.00$ & $(2,2)$ & 2546 & 2278\\
$(75,31)$ & 9 & $(17,7)$ & 6 & 1 & YES & YES & YES & $1.00$ & $(2,2)$ & 1935 & 2279\\
$(75,31)$ & 9 & $(22,9)$ & 7 & 1 & YES & YES & YES & $1.22$ & $(4,1)$ & NO & 2280\\
$(75,16)$ & 11 & $(24,5)$ & 8 & 3 & YES & YES & YES & $1.00$ & $(2,2)$ & NO & 2281\\
$(75,22)$ & 10 & $(31,9)$ & 8 & 1 & YES & YES & YES & $1.00$ & $(6,0)$ & NO & 2282\\
$(75,22)$ & 10 & $(41,12)$ & 8 & 1 & YES & YES & YES & $1.11$ & $(2,2)$ & 2795 & 2283\\
$(75,29)$ & 9 & $(44,17)$ & 8 & 1 & YES & YES & NO(2) & $1.10$ & $(4,1)$ & NO & 2284\\
$(75,31)$ & 9 & $(75,31)$ & 9 & 75 & YES & YES & YES & $0.88$ & $(2,2)$ & NO & 2285\\
$(76,23)$ & 10 & $(2,1)$ & 1 & 2 & YES & YES & YES & $1.00$ & $(2,2)$ & -- & 2286\\
$(76,27)$ & 10 & $(2,1)$ & 1 & 2 & YES & YES & YES & $1.00$ & $(4,1)$ & NO & 2287\\
$(76,29)$ & 9 & $(2,1)$ & 1 & 2 & YES & YES & YES & $1.00$ & $(2,2)$ & -- & 2288\\
$(76,29)$ & 9 & $(2,1)$ & 1 & 2 & YES & YES & YES & $1.25$ & $(2,2)$ & NO & 2289\\
$(76,33)$ & 10 & $(2,1)$ & 1 & 2 & YES & YES & YES & $1.00$ & $(4,1)$ & 1032 & 2290\\
$(76,33)$ & 10 & $(2,1)$ & 1 & 2 & YES & YES & NO(2) & $1.00$ & $(6,0)$ & -- & 2291\\
$(76,23)$ & 10 & $(3,1)$ & 2 & 1 & YES & YES & YES & $1.00$ & $(2,2)$ & -- & 2292\\
$(76,23)$ & 10 & $(3,1)$ & 2 & 1 & YES & YES & YES & $1.44$ & $(2,2)$ & NO & 2293\\
$(76,29)$ & 9 & $(3,1)$ & 2 & 1 & YES & YES & YES & $1.00$ & $(4,1)$ & -- & 2294\\
$(76,29)$ & 9 & $(3,1)$ & 2 & 1 & YES & YES & YES & $1.33$ & $(6,0)$ & NO & 2295\\
$(76,29)$ & 9 & $(3,1)$ & 2 & 1 & YES & YES & YES & $1.33$ & $(2,2)$ & NO & 2296\\
$(76,33)$ & 10 & $(3,1)$ & 2 & 1 & YES & YES & NO(2) & $1.20$ & $(4,1)$ & -- & 2297\\
$(76,33)$ & 10 & $(3,1)$ & 2 & 1 & YES & YES & NO(2) & $1.00$ & $(6,0)$ & NO & 2298\\
$(76,23)$ & 10 & $(4,1)$ & 3 & 4 & YES & YES & YES & $0.88$ & $(2,2)$ & -- & 2299\\
$(76,29)$ & 9 & $(4,1)$ & 3 & 4 & YES & YES & YES & $1.00$ & $(4,1)$ & NO & 2300\\
$(76,29)$ & 9 & $(4,1)$ & 3 & 4 & YES & YES & YES & $1.33$ & $(4,1)$ & -- & 2301\\
$(76,21)$ & 9 & $(5,2)$ & 3 & 1 & YES & YES & YES & $0.75$ & $(4,1)$ & -- & 2302\\
$(76,21)$ & 9 & $(5,2)$ & 3 & 1 & YES & YES & YES & $1.00$ & $(4,1)$ & NO & 2303\\
$(76,21)$ & 9 & $(5,2)$ & 3 & 1 & YES & YES & YES & $1.00$ & $(4,1)$ & NO & 2304\\
$(76,23)$ & 10 & $(5,2)$ & 3 & 1 & YES & YES & YES & $1.22$ & $(4,1)$ & -- & 2305\\
$(76,23)$ & 10 & $(5,2)$ & 3 & 1 & YES & YES & YES & $1.12$ & $(6,0)$ & NO & 2306\\
$(76,27)$ & 10 & $(5,1)$ & 4 & 1 & YES & YES & YES & $0.88$ & $(4,1)$ & NO & 2307\\
$(76,27)$ & 10 & $(5,1)$ & 4 & 1 & YES & YES & NO(2) & $0.89$ & $(6,0)$ & -- & 2308\\
$(76,29)$ & 9 & $(5,1)$ & 4 & 1 & YES & YES & YES & $1.00$ & $(4,1)$ & NO & 2309\\
$(76,29)$ & 9 & $(5,1)$ & 4 & 1 & YES & YES & YES & $1.00$ & $(4,1)$ & NO & 2310\\
$(76,29)$ & 9 & $(5,2)$ & 3 & 1 & YES & YES & YES & $1.22$ & $(4,1)$ & -- & 2311\\
$(76,33)$ & 10 & $(5,2)$ & 3 & 1 & YES & YES & NO(2) & $1.30$ & $(4,1)$ & NO & 2312\\
$(76,33)$ & 10 & $(5,2)$ & 3 & 1 & YES & YES & NO(2) & $1.30$ & $(4,1)$ & -- & 2313\\
$(76,29)$ & 9 & $(6,1)$ & 5 & 2 & YES & YES & YES & $1.12$ & $(2,2)$ & NO & 2314\\
$(76,29)$ & 9 & $(6,1)$ & 5 & 2 & YES & YES & YES & $1.12$ & $(2,2)$ & NO & 2315\\
$(76,23)$ & 10 & $(7,2)$ & 4 & 1 & YES & YES & YES & $1.11$ & $(4,1)$ & -- & 2316\\
$(76,29)$ & 9 & $(7,2)$ & 4 & 1 & YES & YES & YES & $1.22$ & $(4,1)$ & NO & 2317\\
$(76,29)$ & 9 & $(8,3)$ & 4 & 4 & YES & YES & YES & $1.12$ & $(2,2)$ & NO & 2318\\
$(76,21)$ & 9 & $(10,3)$ & 5 & 2 & YES & YES & YES & $0.75$ & $(4,1)$ & NO & 2319\\
$(76,29)$ & 9 & $(11,2)$ & 6 & 1 & YES & YES & YES & $1.22$ & $(4,1)$ & NO & 2320\\
$(76,23)$ & 10 & $(13,4)$ & 6 & 1 & YES & YES & YES & $1.00$ & $(2,2)$ & 1372 & 2321\\
$(76,29)$ & 9 & $(13,5)$ & 5 & 1 & YES & YES & YES & $1.12$ & $(2,2)$ & NO & 2322\\
$(76,27)$ & 10 & $(14,5)$ & 6 & 2 & YES & YES & YES & $0.88$ & $(4,1)$ & 1822 & 2323\\
$(76,29)$ & 9 & $(21,8)$ & 6 & 1 & YES & YES & YES & $1.33$ & $(2,2)$ & NO & 2324\\
$(76,21)$ & 9 & $(25,7)$ & 7 & 1 & YES & YES & YES & $1.00$ & $(4,1)$ & NO & 2325\\
$(76,27)$ & 10 & $(31,11)$ & 8 & 1 & YES & YES & YES & $1.00$ & $(4,1)$ & NO & 2326\\
$(76,29)$ & 9 & $(34,13)$ & 7 & 2 & YES & YES & YES & $1.00$ & $(4,1)$ & 2636 & 2327\\
$(76,21)$ & 9 & $(40,11)$ & 8 & 4 & YES & YES & YES & $1.00$ & $(4,1)$ & NO & 2328\\
$(76,23)$ & 10 & $(43,13)$ & 9 & 1 & YES & YES & YES & $1.00$ & $(2,2)$ & NO & 2329\\
$(76,29)$ & 9 & $(47,18)$ & 8 & 1 & YES & YES & YES & $1.22$ & $(4,1)$ & 3200 & 2330\\
$(76,29)$ & 9 & $(55,21)$ & 8 & 1 & YES & YES & YES & $1.00$ & $(4,1)$ & NO & 2331\\
$(76,23)$ & 10 & $(56,17)$ & 9 & 4 & YES & YES & YES & $1.00$ & $(6,0)$ & NO & 2332\\
$(76,23)$ & 10 & $(76,23)$ & 10 & 76 & YES & YES & YES & $1.22$ & $(2,2)$ & NO & 2333\\
$(76,29)$ & 9 & $(76,29)$ & 9 & 76 & YES & YES & YES & $1.00$ & $(4,1)$ & NO & 2334\\
$(77,18)$ & 10 & $(10,3)$ & 5 & 1 & YES & YES & YES & $1.00$ & $(4,1)$ & -- & 2335\\
$(77,18)$ & 10 & $(11,3)$ & 5 & 11 & YES & YES & YES & $1.22$ & $(4,1)$ & -- & 2336\\
$(78,23)$ & 10 & $(2,1)$ & 1 & 2 & YES & YES & YES & $1.12$ & $(2,2)$ & -- & 2337\\
$(78,23)$ & 10 & $(3,1)$ & 2 & 3 & YES & YES & NO(2) & $0.89$ & $(6,0)$ & -- & 2338\\
$(78,23)$ & 10 & $(4,1)$ & 3 & 2 & YES & YES & YES & $1.12$ & $(2,2)$ & NO & 2339\\
$(78,29)$ & 10 & $(5,1)$ & 4 & 1 & YES & YES & YES & $1.00$ & $(4,1)$ & NO & 2340\\
$(78,23)$ & 10 & $(7,2)$ & 4 & 1 & YES & YES & NO(2) & $0.89$ & $(6,0)$ & NO & 2341\\
$(78,29)$ & 10 & $(8,3)$ & 4 & 2 & YES & YES & YES & $1.12$ & $(4,1)$ & 1635 & 2342\\
$(78,23)$ & 10 & $(10,3)$ & 5 & 2 & YES & YES & YES & $1.12$ & $(2,2)$ & NO & 2343\\
$(78,17)$ & 10 & $(19,4)$ & 7 & 1 & YES & YES & YES & $0.88$ & $(2,2)$ & NO & 2344\\
$(78,17)$ & 10 & $(37,8)$ & 8 & 1 & YES & YES & YES & $0.88$ & $(2,2)$ & NO & 2345\\
$(79,17)$ & 11 & $(2,1)$ & 1 & 1 & YES & YES & YES & $1.11$ & $(2,2)$ & NO & 2346\\
$(79,17)$ & 11 & $(2,1)$ & 1 & 1 & YES & YES & NO(2) & $1.20$ & $(4,1)$ & -- & 2347\\
$(79,28)$ & 10 & $(2,1)$ & 1 & 1 & YES & YES & NO(2) & $1.30$ & $(4,1)$ & NO & 2348\\
$(79,29)$ & 9 & $(2,1)$ & 1 & 1 & YES & YES & YES & $1.00$ & $(2,2)$ & NO & 2349\\
$(79,29)$ & 9 & $(2,1)$ & 1 & 1 & YES & YES & YES & $1.00$ & $(2,2)$ & -- & 2350\\
$(79,17)$ & 11 & $(3,1)$ & 2 & 1 & YES & YES & YES & $1.22$ & $(2,2)$ & -- & 2351\\
$(79,17)$ & 11 & $(3,1)$ & 2 & 1 & YES & YES & YES & $1.33$ & $(2,2)$ & NO & 2352\\
$(79,18)$ & 10 & $(3,1)$ & 2 & 1 & YES & YES & YES & $1.33$ & $(2,2)$ & NO & 2353\\
$(79,23)$ & 10 & $(3,1)$ & 2 & 1 & YES & YES & YES & $1.00$ & $(2,2)$ & 1078 & 2354\\
$(79,23)$ & 10 & $(3,1)$ & 2 & 1 & YES & YES & YES & $1.12$ & $(2,2)$ & NO & 2355\\
$(79,23)$ & 10 & $(3,1)$ & 2 & 1 & YES & YES & YES & $1.12$ & $(2,2)$ & -- & 2356\\
$(79,24)$ & 10 & $(3,1)$ & 2 & 1 & YES & YES & NO(2) & $1.00$ & $(6,0)$ & NO & 2357\\
$(79,24)$ & 10 & $(3,1)$ & 2 & 1 & YES & YES & YES & $1.22$ & $(2,2)$ & -- & 2358\\
$(79,24)$ & 10 & $(3,1)$ & 2 & 1 & YES & YES & YES & $1.25$ & $(4,1)$ & NO & 2359\\
$(79,29)$ & 9 & $(3,1)$ & 2 & 1 & YES & YES & YES & $1.00$ & $(2,2)$ & -- & 2360\\
$(79,30)$ & 9 & $(3,1)$ & 2 & 1 & YES & YES & YES & $1.00$ & $(2,2)$ & NO & 2361\\
$(79,30)$ & 9 & $(3,1)$ & 2 & 1 & YES & YES & NO(2) & $1.10$ & $(4,1)$ & -- & 2362\\
$(79,30)$ & 9 & $(3,1)$ & 2 & 1 & YES & YES & YES & $1.12$ & $(4,1)$ & NO & 2363\\
$(79,17)$ & 11 & $(4,1)$ & 3 & 1 & YES & YES & YES & $1.11$ & $(2,2)$ & NO & 2364\\
$(79,29)$ & 9 & $(4,1)$ & 3 & 1 & YES & YES & YES & $1.00$ & $(2,2)$ & -- & 2365\\
$(79,30)$ & 9 & $(4,1)$ & 3 & 1 & YES & YES & YES & $0.88$ & $(4,1)$ & NO & 2366\\
$(79,30)$ & 9 & $(4,1)$ & 3 & 1 & YES & YES & YES & $1.00$ & $(4,1)$ & -- & 2367\\
$(79,30)$ & 9 & $(4,1)$ & 3 & 1 & YES & YES & YES & $1.22$ & $(4,1)$ & NO & 2368\\
$(79,17)$ & 11 & $(5,2)$ & 3 & 1 & YES & YES & YES & $1.33$ & $(2,2)$ & -- & 2369\\
$(79,17)$ & 11 & $(5,2)$ & 3 & 1 & YES & YES & YES & $1.44$ & $(2,2)$ & NO & 2370\\
$(79,18)$ & 10 & $(5,2)$ & 3 & 1 & YES & YES & YES & $0.75$ & $(4,1)$ & NO & 2371\\
$(79,18)$ & 10 & $(5,2)$ & 3 & 1 & YES & YES & YES & $1.12$ & $(4,1)$ & -- & 2372\\
$(79,18)$ & 10 & $(5,2)$ & 3 & 1 & YES & YES & YES & $1.22$ & $(6,0)$ & NO & 2373\\
$(79,22)$ & 10 & $(5,2)$ & 3 & 1 & YES & YES & YES & $1.00$ & $(6,0)$ & -- & 2374\\
$(79,23)$ & 10 & $(5,1)$ & 4 & 1 & YES & YES & YES & $1.33$ & $(2,2)$ & NO & 2375\\
$(79,23)$ & 10 & $(5,2)$ & 3 & 1 & YES & YES & YES & $1.22$ & $(4,1)$ & -- & 2376\\
$(79,23)$ & 10 & $(5,2)$ & 3 & 1 & YES & YES & YES & $1.12$ & $(6,0)$ & NO & 2377\\
$(79,24)$ & 10 & $(5,2)$ & 3 & 1 & YES & YES & YES & $1.11$ & $(4,1)$ & -- & 2378\\
$(79,29)$ & 9 & $(5,2)$ & 3 & 1 & YES & YES & YES & $1.22$ & $(4,1)$ & NO & 2379\\
$(79,29)$ & 9 & $(5,2)$ & 3 & 1 & YES & YES & YES & $1.33$ & $(4,1)$ & -- & 2380\\
$(79,30)$ & 9 & $(5,2)$ & 3 & 1 & YES & YES & YES & $1.00$ & $(6,0)$ & NO & 2381\\
$(79,30)$ & 9 & $(5,2)$ & 3 & 1 & YES & YES & YES & $1.00$ & $(6,0)$ & -- & 2382\\
$(79,28)$ & 10 & $(6,1)$ & 5 & 1 & YES & YES & NO(2) & $1.20$ & $(4,1)$ & NO & 2383\\
$(79,17)$ & 11 & $(7,1)$ & 6 & 1 & YES & YES & NO(2) & $1.30$ & $(4,1)$ & NO & 2384\\
$(79,24)$ & 10 & $(7,2)$ & 4 & 1 & YES & YES & YES & $1.33$ & $(2,2)$ & 1879 & 2385\\
$(79,29)$ & 9 & $(7,2)$ & 4 & 1 & YES & YES & YES & $1.22$ & $(4,1)$ & -- & 2386\\
$(79,29)$ & 9 & $(7,3)$ & 4 & 1 & YES & YES & YES & $1.22$ & $(4,1)$ & 3171 & 2387\\
$(79,30)$ & 9 & $(8,3)$ & 4 & 1 & YES & YES & YES & $1.12$ & $(2,2)$ & NO & 2388\\
$(79,17)$ & 11 & $(9,2)$ & 5 & 1 & YES & YES & YES & $1.33$ & $(2,2)$ & NO & 2389\\
$(79,22)$ & 10 & $(9,2)$ & 5 & 1 & YES & YES & YES & $0.88$ & $(6,0)$ & -- & 2390\\
$(79,18)$ & 10 & $(10,3)$ & 5 & 1 & YES & YES & YES & $1.00$ & $(4,1)$ & -- & 2391\\
$(79,22)$ & 10 & $(11,3)$ & 5 & 1 & YES & YES & YES & $1.12$ & $(2,2)$ & NO & 2392\\
$(79,23)$ & 10 & $(11,3)$ & 5 & 1 & YES & YES & YES & $1.33$ & $(2,2)$ & NO & 2393\\
$(79,29)$ & 9 & $(11,4)$ & 5 & 1 & YES & YES & YES & $1.12$ & $(2,2)$ & NO & 2394\\
$(79,30)$ & 9 & $(11,4)$ & 5 & 1 & YES & YES & YES & $1.00$ & $(6,0)$ & NO & 2395\\
$(79,17)$ & 11 & $(13,3)$ & 6 & 1 & YES & YES & YES & $1.33$ & $(2,2)$ & NO & 2396\\
$(79,23)$ & 10 & $(13,4)$ & 6 & 1 & YES & YES & YES & $1.22$ & $(4,1)$ & NO & 2397\\
$(79,24)$ & 10 & $(13,4)$ & 6 & 1 & YES & YES & YES & $1.33$ & $(2,2)$ & NO & 2398\\
$(79,30)$ & 9 & $(13,5)$ & 5 & 1 & YES & YES & YES & $0.88$ & $(4,1)$ & 2626 & 2399\\
$(79,23)$ & 10 & $(17,5)$ & 6 & 1 & YES & YES & YES & $1.22$ & $(2,2)$ & NO & 2400\\
$(79,24)$ & 10 & $(17,5)$ & 6 & 1 & YES & YES & YES & $1.11$ & $(4,1)$ & NO & 2401\\
$(79,29)$ & 9 & $(19,7)$ & 6 & 1 & YES & YES & YES & $1.12$ & $(2,2)$ & 2091 & 2402\\
$(79,29)$ & 9 & $(27,10)$ & 7 & 1 & YES & YES & YES & $1.22$ & $(4,1)$ & NO & 2403\\
$(79,29)$ & 9 & $(30,11)$ & 7 & 1 & YES & YES & YES & $1.33$ & $(2,2)$ & NO & 2404\\
$(79,23)$ & 10 & $(31,9)$ & 8 & 1 & YES & YES & YES & $1.12$ & $(2,2)$ & 2563 & 2405\\
$(79,28)$ & 10 & $(31,11)$ & 8 & 1 & YES & YES & NO(2) & $1.11$ & $(6,0)$ & NO & 2406\\
$(79,22)$ & 10 & $(32,9)$ & 8 & 1 & YES & YES & YES & $1.00$ & $(6,0)$ & NO & 2407\\
$(79,24)$ & 10 & $(33,10)$ & 8 & 1 & YES & YES & YES & $1.00$ & $(2,2)$ & 2637 & 2408\\
$(79,30)$ & 9 & $(37,14)$ & 8 & 1 & YES & YES & YES & $1.00$ & $(6,0)$ & NO & 2409\\
$(79,23)$ & 10 & $(41,12)$ & 8 & 1 & YES & YES & YES & $1.12$ & $(6,0)$ & NO & 2410\\
$(79,29)$ & 9 & $(41,15)$ & 8 & 1 & YES & YES & YES & $1.00$ & $(6,0)$ & NO & 2411\\
$(79,29)$ & 9 & $(49,18)$ & 8 & 1 & YES & YES & NO(2) & $1.20$ & $(4,1)$ & NO & 2412\\
$(79,30)$ & 9 & $(50,19)$ & 8 & 1 & YES & YES & YES & $0.88$ & $(2,2)$ & NO & 2413\\
$(79,29)$ & 9 & $(68,25)$ & 9 & 1 & YES & YES & YES & $1.22$ & $(4,1)$ & NO & 2414\\
$(79,17)$ & 11 & $(79,17)$ & 11 & 79 & YES & YES & NO(2) & $1.20$ & $(4,1)$ & NO & 2415\\
$(79,22)$ & 10 & $(79,22)$ & 10 & 79 & YES & YES & YES & $1.22$ & $(2,2)$ & NO & 2416\\
$(79,24)$ & 10 & $(79,24)$ & 10 & 79 & YES & YES & YES & $1.12$ & $(2,2)$ & NO & 2417\\
$(79,30)$ & 9 & $(79,30)$ & 9 & 79 & YES & YES & YES & $1.00$ & $(4,1)$ & NO & 2418\\
$(80,19)$ & 11 & $(2,1)$ & 1 & 2 & YES & YES & YES & $1.00$ & $(2,2)$ & NO & 2419\\
$(80,19)$ & 11 & $(2,1)$ & 1 & 2 & YES & YES & NO(2) & $1.20$ & $(4,1)$ & -- & 2420\\
$(80,31)$ & 9 & $(2,1)$ & 1 & 2 & YES & YES & YES & $1.00$ & $(2,2)$ & -- & 2421\\
$(80,19)$ & 11 & $(3,1)$ & 2 & 1 & YES & YES & NO(2) & $1.11$ & $(6,0)$ & NO & 2422\\
$(80,21)$ & 11 & $(3,1)$ & 2 & 1 & YES & YES & NO(2) & $1.20$ & $(4,1)$ & -- & 2423\\
$(80,21)$ & 11 & $(3,1)$ & 2 & 1 & YES & YES & NO(2) & $1.30$ & $(4,1)$ & NO & 2424\\
$(80,31)$ & 9 & $(3,1)$ & 2 & 1 & YES & YES & YES & $1.33$ & $(4,1)$ & NO & 2425\\
$(80,31)$ & 9 & $(3,1)$ & 2 & 1 & YES & YES & YES & $1.33$ & $(4,1)$ & -- & 2426\\
$(80,31)$ & 9 & $(4,1)$ & 3 & 4 & YES & YES & YES & $0.88$ & $(4,1)$ & NO & 2427\\
$(80,31)$ & 9 & $(4,1)$ & 3 & 4 & YES & YES & YES & $1.56$ & $(4,1)$ & -- & 2428\\
$(80,19)$ & 11 & $(5,1)$ & 4 & 5 & YES & YES & YES & $1.22$ & $(2,2)$ & NO & 2429\\
$(80,21)$ & 11 & $(5,1)$ & 4 & 5 & YES & YES & NO(2) & $1.11$ & $(6,0)$ & NO & 2430\\
$(80,31)$ & 9 & $(5,2)$ & 3 & 5 & YES & YES & YES & $1.22$ & $(2,2)$ & 1977 & 2431\\
$(80,21)$ & 11 & $(7,2)$ & 4 & 1 & YES & YES & NO(2) & $1.20$ & $(4,1)$ & NO & 2432\\
$(80,31)$ & 9 & $(7,3)$ & 4 & 1 & YES & YES & YES & $1.22$ & $(4,1)$ & NO & 2433\\
$(80,31)$ & 9 & $(8,3)$ & 4 & 8 & YES & YES & YES & $1.00$ & $(4,1)$ & NO & 2434\\
$(80,19)$ & 11 & $(9,2)$ & 5 & 1 & YES & YES & NO(2) & $1.20$ & $(4,1)$ & NO & 2435\\
$(80,19)$ & 11 & $(13,3)$ & 6 & 1 & YES & YES & NO(2) & $1.20$ & $(4,1)$ & NO & 2436\\
$(80,31)$ & 9 & $(13,5)$ & 5 & 1 & YES & YES & YES & $1.00$ & $(2,2)$ & NO & 2437\\
$(80,19)$ & 11 & $(17,4)$ & 7 & 1 & YES & YES & YES & $1.22$ & $(2,2)$ & NO & 2438\\
$(80,21)$ & 11 & $(23,6)$ & 8 & 1 & YES & YES & NO(2) & $1.11$ & $(6,0)$ & NO & 2439\\
$(80,31)$ & 9 & $(31,12)$ & 7 & 1 & YES & YES & YES & $0.88$ & $(2,2)$ & NO & 2440\\
$(80,31)$ & 9 & $(49,19)$ & 8 & 1 & YES & YES & YES & $1.00$ & $(4,1)$ & NO & 2441\\
$(80,19)$ & 11 & $(80,19)$ & 11 & 80 & YES & YES & YES & $0.88$ & $(4,1)$ & NO & 2442\\
$(80,31)$ & 9 & $(80,31)$ & 9 & 80 & YES & YES & YES & $0.88$ & $(4,1)$ & NO & 2443\\
$(81,31)$ & 9 & $(2,1)$ & 1 & 1 & YES & YES & YES & $1.22$ & $(2,2)$ & -- & 2444\\
$(81,31)$ & 9 & $(2,1)$ & 1 & 1 & YES & YES & YES & $1.33$ & $(2,2)$ & NO & 2445\\
$(81,34)$ & 9 & $(2,1)$ & 1 & 1 & NO & YES & YES & $1.11$ & $(2,2)$ & -- & 2446\\
$(81,19)$ & 11 & $(3,1)$ & 2 & 3 & YES & YES & YES & $1.00$ & $(2,2)$ & NO & 2447\\
$(81,19)$ & 11 & $(3,1)$ & 2 & 3 & YES & YES & YES & $1.00$ & $(2,2)$ & -- & 2448\\
$(81,19)$ & 11 & $(3,1)$ & 2 & 3 & YES & YES & YES & $1.00$ & $(2,2)$ & NO & 2449\\
$(81,31)$ & 9 & $(3,1)$ & 2 & 3 & YES & YES & YES & $1.00$ & $(4,1)$ & -- & 2450\\
$(81,31)$ & 9 & $(3,1)$ & 2 & 3 & YES & YES & YES & $1.12$ & $(4,1)$ & NO & 2451\\
$(81,31)$ & 9 & $(3,1)$ & 2 & 3 & YES & YES & YES & $1.12$ & $(4,1)$ & NO & 2452\\
$(81,31)$ & 9 & $(4,1)$ & 3 & 1 & YES & YES & YES & $1.12$ & $(4,1)$ & -- & 2453\\
$(81,31)$ & 9 & $(4,1)$ & 3 & 1 & YES & YES & YES & $1.12$ & $(4,1)$ & NO & 2454\\
$(81,31)$ & 9 & $(5,1)$ & 4 & 1 & YES & YES & YES & $1.00$ & $(2,2)$ & NO & 2455\\
$(81,31)$ & 9 & $(5,1)$ & 4 & 1 & YES & YES & YES & $1.22$ & $(2,2)$ & -- & 2456\\
$(81,31)$ & 9 & $(5,2)$ & 3 & 1 & YES & YES & YES & $1.22$ & $(4,1)$ & -- & 2457\\
$(81,19)$ & 11 & $(6,1)$ & 5 & 3 & YES & YES & YES & $1.22$ & $(2,2)$ & NO & 2458\\
$(81,31)$ & 9 & $(7,2)$ & 4 & 1 & YES & YES & YES & $1.12$ & $(2,2)$ & -- & 2459\\
$(81,31)$ & 9 & $(7,2)$ & 4 & 1 & YES & YES & YES & $1.12$ & $(2,2)$ & NO & 2460\\
$(81,31)$ & 9 & $(8,3)$ & 4 & 1 & YES & YES & YES & $1.12$ & $(4,1)$ & 1491 & 2461\\
$(81,19)$ & 11 & $(9,2)$ & 5 & 9 & YES & YES & YES & $1.12$ & $(2,2)$ & NO & 2462\\
$(81,31)$ & 9 & $(11,4)$ & 5 & 1 & YES & YES & YES & $0.88$ & $(6,0)$ & NO & 2463\\
$(81,34)$ & 9 & $(12,5)$ & 5 & 3 & YES & YES & NO(2) & $1.00$ & $(6,0)$ & NO & 2464\\
$(81,31)$ & 9 & $(13,5)$ & 5 & 1 & YES & YES & YES & $1.00$ & $(2,2)$ & 1836 & 2465\\
$(81,34)$ & 9 & $(19,8)$ & 6 & 1 & YES & YES & YES & $1.33$ & $(2,2)$ & 2119 & 2466\\
$(81,19)$ & 11 & $(21,5)$ & 8 & 3 & YES & YES & YES & $1.00$ & $(2,2)$ & NO & 2467\\
$(81,31)$ & 9 & $(21,8)$ & 6 & 3 & YES & YES & YES & $0.88$ & $(4,1)$ & NO & 2468\\
$(81,31)$ & 9 & $(29,11)$ & 7 & 1 & YES & YES & YES & $1.12$ & $(2,2)$ & NO & 2469\\
$(81,19)$ & 11 & $(30,7)$ & 8 & 3 & YES & YES & YES & $1.22$ & $(2,2)$ & NO & 2470\\
$(81,34)$ & 9 & $(31,13)$ & 7 & 1 & YES & YES & YES & $1.12$ & $(2,2)$ & NO & 2471\\
$(81,31)$ & 9 & $(34,13)$ & 7 & 1 & YES & YES & YES & $1.22$ & $(2,2)$ & NO & 2472\\
$(81,31)$ & 9 & $(47,18)$ & 8 & 1 & YES & YES & YES & $1.33$ & $(4,1)$ & NO & 2473\\
$(81,31)$ & 9 & $(55,21)$ & 8 & 1 & YES & YES & YES & $1.12$ & $(2,2)$ & NO & 2474\\
$(81,31)$ & 9 & $(81,31)$ & 9 & 81 & YES & YES & YES & $1.33$ & $(6,0)$ & NO & 2475\\
$(82,19)$ & 12 & $(2,1)$ & 1 & 2 & YES & YES & YES & $1.00$ & $(4,1)$ & -- & 2476\\
$(82,19)$ & 12 & $(2,1)$ & 1 & 2 & YES & YES & YES & $1.12$ & $(4,1)$ & NO & 2477\\
$(82,23)$ & 10 & $(2,1)$ & 1 & 2 & YES & YES & YES & $1.00$ & $(2,2)$ & NO & 2478\\
$(82,25)$ & 10 & $(2,1)$ & 1 & 2 & YES & YES & YES & $1.33$ & $(2,2)$ & -- & 2479\\
$(82,25)$ & 10 & $(2,1)$ & 1 & 2 & YES & YES & YES & $1.00$ & $(4,1)$ & NO & 2480\\
$(82,19)$ & 12 & $(3,1)$ & 2 & 1 & YES & YES & NO(2) & $1.00$ & $(6,0)$ & -- & 2481\\
$(82,19)$ & 12 & $(3,1)$ & 2 & 1 & YES & YES & NO(2) & $1.11$ & $(6,0)$ & NO & 2482\\
$(82,25)$ & 10 & $(3,1)$ & 2 & 1 & YES & YES & NO(2) & $1.10$ & $(4,1)$ & -- & 2483\\
$(82,25)$ & 10 & $(4,1)$ & 3 & 2 & YES & YES & YES & $1.33$ & $(2,2)$ & NO & 2484\\
$(82,19)$ & 12 & $(5,1)$ & 4 & 1 & YES & YES & YES & $1.00$ & $(4,1)$ & NO & 2485\\
$(82,23)$ & 10 & $(5,1)$ & 4 & 1 & YES & YES & YES & $1.12$ & $(2,2)$ & NO & 2486\\
$(82,23)$ & 10 & $(5,2)$ & 3 & 1 & YES & YES & YES & $1.22$ & $(4,1)$ & NO & 2487\\
$(82,23)$ & 10 & $(5,2)$ & 3 & 1 & YES & YES & YES & $1.22$ & $(4,1)$ & -- & 2488\\
$(82,37)$ & 10 & $(5,1)$ & 4 & 1 & YES & YES & NO(2) & $1.20$ & $(4,1)$ & -- & 2489\\
$(82,19)$ & 12 & $(9,2)$ & 5 & 1 & YES & YES & NO(2) & $1.11$ & $(6,0)$ & NO & 2490\\
$(82,23)$ & 10 & $(10,3)$ & 5 & 2 & YES & YES & YES & $1.22$ & $(2,2)$ & NO & 2491\\
$(82,23)$ & 10 & $(11,3)$ & 5 & 1 & YES & YES & YES & $1.22$ & $(2,2)$ & NO & 2492\\
$(82,25)$ & 10 & $(13,4)$ & 6 & 1 & YES & YES & YES & $1.12$ & $(4,1)$ & NO & 2493\\
$(82,23)$ & 10 & $(15,4)$ & 6 & 1 & YES & YES & YES & $1.22$ & $(4,1)$ & NO & 2494\\
$(82,23)$ & 10 & $(18,5)$ & 6 & 2 & YES & YES & YES & $1.22$ & $(2,2)$ & NO & 2495\\
$(82,37)$ & 10 & $(20,9)$ & 7 & 2 & YES & YES & NO(2) & $1.30$ & $(4,1)$ & 2191 & 2496\\
$(82,25)$ & 10 & $(23,7)$ & 7 & 1 & YES & YES & YES & $1.12$ & $(2,2)$ & NO & 2497\\
$(82,23)$ & 10 & $(32,9)$ & 8 & 2 & YES & YES & YES & $1.12$ & $(2,2)$ & 2639 & 2498\\
$(82,23)$ & 10 & $(43,12)$ & 8 & 1 & YES & YES & YES & $1.22$ & $(4,1)$ & NO & 2499\\
$(82,25)$ & 10 & $(59,18)$ & 9 & 1 & YES & YES & YES & $1.00$ & $(2,2)$ & NO & 2500\\
$(82,23)$ & 10 & $(82,23)$ & 10 & 82 & YES & YES & YES & $1.12$ & $(2,2)$ & NO & 2501\\
$(83,22)$ & 10 & $(2,1)$ & 1 & 1 & YES & YES & YES & $1.00$ & $(2,2)$ & -- & 2502\\
$(83,22)$ & 10 & $(2,1)$ & 1 & 1 & YES & YES & YES & $1.11$ & $(2,2)$ & NO & 2503\\
$(83,23)$ & 10 & $(2,1)$ & 1 & 1 & YES & YES & YES & $1.12$ & $(2,2)$ & -- & 2504\\
$(83,23)$ & 10 & $(2,1)$ & 1 & 1 & YES & YES & YES & $1.33$ & $(2,2)$ & NO & 2505\\
$(83,18)$ & 10 & $(3,1)$ & 2 & 1 & YES & YES & YES & $1.33$ & $(2,2)$ & NO & 2506\\
$(83,22)$ & 10 & $(3,1)$ & 2 & 1 & YES & YES & YES & $1.22$ & $(2,2)$ & NO & 2507\\
$(83,22)$ & 10 & $(3,1)$ & 2 & 1 & YES & YES & YES & $1.22$ & $(2,2)$ & -- & 2508\\
$(83,22)$ & 10 & $(3,1)$ & 2 & 1 & YES & YES & NO(2) & $1.20$ & $(4,1)$ & NO & 2509\\
$(83,23)$ & 10 & $(3,1)$ & 2 & 1 & YES & YES & NO(2) & $0.78$ & $(6,0)$ & -- & 2510\\
$(83,23)$ & 10 & $(3,1)$ & 2 & 1 & YES & YES & YES & $1.12$ & $(2,2)$ & NO & 2511\\
$(83,24)$ & 11 & $(3,1)$ & 2 & 1 & YES & YES & NO(2) & $1.30$ & $(4,1)$ & NO & 2512\\
$(83,30)$ & 10 & $(3,1)$ & 2 & 1 & YES & YES & NO(2) & $1.11$ & $(6,0)$ & NO & 2513\\
$(83,35)$ & 10 & $(4,1)$ & 3 & 1 & YES & YES & YES & $1.00$ & $(6,0)$ & -- & 2514\\
$(83,22)$ & 10 & $(5,2)$ & 3 & 1 & YES & YES & YES & $1.33$ & $(2,2)$ & NO & 2515\\
$(83,22)$ & 10 & $(5,2)$ & 3 & 1 & YES & YES & YES & $1.33$ & $(2,2)$ & -- & 2516\\
$(83,23)$ & 10 & $(5,2)$ & 3 & 1 & YES & YES & YES & $1.22$ & $(4,1)$ & -- & 2517\\
$(83,30)$ & 10 & $(5,1)$ & 4 & 1 & YES & YES & NO(2) & $1.00$ & $(6,0)$ & -- & 2518\\
$(83,35)$ & 10 & $(5,1)$ & 4 & 1 & YES & YES & YES & $1.12$ & $(6,0)$ & NO & 2519\\
$(83,35)$ & 10 & $(5,1)$ & 4 & 1 & YES & YES & YES & $1.12$ & $(6,0)$ & NO & 2520\\
$(83,22)$ & 10 & $(7,2)$ & 4 & 1 & YES & YES & NO(2) & $1.10$ & $(4,1)$ & NO & 2521\\
$(83,23)$ & 10 & $(9,2)$ & 5 & 1 & YES & YES & YES & $1.22$ & $(4,1)$ & NO & 2522\\
$(83,23)$ & 10 & $(11,3)$ & 5 & 1 & YES & YES & YES & $1.12$ & $(2,2)$ & NO & 2523\\
$(83,18)$ & 10 & $(19,4)$ & 7 & 1 & YES & YES & YES & $1.00$ & $(2,2)$ & 3032 & 2524\\
$(83,22)$ & 10 & $(34,9)$ & 8 & 1 & YES & YES & YES & $1.11$ & $(2,2)$ & NO & 2525\\
$(83,35)$ & 10 & $(45,19)$ & 8 & 1 & YES & YES & YES & $1.12$ & $(6,0)$ & 2973 & 2526\\
$(83,35)$ & 10 & $(64,27)$ & 9 & 1 & YES & YES & YES & $1.25$ & $(6,0)$ & NO & 2527\\
$(83,30)$ & 10 & $(83,30)$ & 10 & 83 & YES & YES & NO(2) & $0.89$ & $(6,0)$ & NO & 2528\\
$(84,25)$ & 10 & $(2,1)$ & 1 & 2 & YES & YES & YES & $1.00$ & $(2,2)$ & -- & 2529\\
$(84,25)$ & 10 & $(2,1)$ & 1 & 2 & YES & YES & YES & $1.12$ & $(2,2)$ & NO & 2530\\
$(84,37)$ & 10 & $(2,1)$ & 1 & 2 & NO & YES & NO(2) & $1.40$ & $(4,1)$ & -- & 2531\\
$(84,25)$ & 10 & $(3,1)$ & 2 & 3 & NO & YES & NO(2) & $1.30$ & $(4,1)$ & -- & 2532\\
$(84,37)$ & 10 & $(3,1)$ & 2 & 3 & YES & YES & YES & $1.33$ & $(2,2)$ & NO & 2533\\
$(84,37)$ & 10 & $(3,1)$ & 2 & 3 & YES & YES & YES & $1.33$ & $(2,2)$ & -- & 2534\\
$(84,31)$ & 10 & $(4,1)$ & 3 & 4 & YES & YES & YES & $0.88$ & $(6,0)$ & -- & 2535\\
$(84,37)$ & 10 & $(4,1)$ & 3 & 4 & YES & YES & YES & $1.33$ & $(2,2)$ & -- & 2536\\
$(84,25)$ & 10 & $(5,1)$ & 4 & 1 & YES & YES & NO(2) & $1.00$ & $(6,0)$ & NO & 2537\\
$(84,25)$ & 10 & $(7,2)$ & 4 & 7 & YES & YES & YES & $1.00$ & $(2,2)$ & NO & 2538\\
$(84,37)$ & 10 & $(7,3)$ & 4 & 7 & YES & YES & YES & $1.44$ & $(2,2)$ & NO & 2539\\
$(84,25)$ & 10 & $(10,3)$ & 5 & 2 & YES & YES & NO(2) & $1.00$ & $(6,0)$ & 1770 & 2540\\
$(84,31)$ & 10 & $(46,17)$ & 8 & 2 & YES & YES & YES & $1.00$ & $(6,0)$ & 3008 & 2541\\
$(85,24)$ & 11 & $(2,1)$ & 1 & 1 & YES & YES & NO(2) & $1.30$ & $(4,1)$ & NO & 2542\\
$(85,26)$ & 10 & $(3,1)$ & 2 & 1 & YES & YES & NO(2) & $1.20$ & $(4,1)$ & NO & 2543\\
$(85,26)$ & 10 & $(3,1)$ & 2 & 1 & NO & YES & NO(2) & $1.11$ & $(6,0)$ & -- & 2544\\
$(85,26)$ & 10 & $(5,1)$ & 4 & 5 & YES & YES & YES & $1.11$ & $(2,2)$ & NO & 2545\\
$(85,16)$ & 12 & $(9,2)$ & 5 & 1 & YES & YES & YES & $1.00$ & $(2,2)$ & 2278 & 2546\\
$(85,23)$ & 10 & $(26,7)$ & 7 & 1 & YES & YES & YES & $0.88$ & $(2,2)$ & NO & 2547\\
$(85,26)$ & 10 & $(85,26)$ & 10 & 85 & YES & YES & YES & $1.11$ & $(2,2)$ & NO & 2548\\
$(86,25)$ & 10 & $(2,1)$ & 1 & 2 & YES & YES & YES & $1.12$ & $(2,2)$ & -- & 2549\\
$(86,25)$ & 10 & $(2,1)$ & 1 & 2 & YES & YES & YES & $1.12$ & $(2,2)$ & NO & 2550\\
$(86,31)$ & 10 & $(3,1)$ & 2 & 1 & YES & YES & NO(2) & $1.00$ & $(6,0)$ & NO & 2551\\
$(86,31)$ & 10 & $(3,1)$ & 2 & 1 & YES & YES & YES & $1.33$ & $(2,2)$ & NO & 2552\\
$(86,31)$ & 10 & $(3,1)$ & 2 & 1 & YES & YES & YES & $1.33$ & $(2,2)$ & -- & 2553\\
$(86,25)$ & 10 & $(4,1)$ & 3 & 2 & YES & YES & YES & $1.12$ & $(2,2)$ & NO & 2554\\
$(86,25)$ & 10 & $(5,1)$ & 4 & 1 & YES & YES & YES & $1.12$ & $(2,2)$ & NO & 2555\\
$(86,25)$ & 10 & $(5,2)$ & 3 & 1 & YES & YES & YES & $1.22$ & $(4,1)$ & -- & 2556\\
$(86,27)$ & 11 & $(5,1)$ & 4 & 1 & YES & YES & NO(2) & $1.00$ & $(6,0)$ & NO & 2557\\
$(86,31)$ & 10 & $(5,2)$ & 3 & 1 & YES & YES & YES & $1.22$ & $(2,2)$ & NO & 2558\\
$(86,25)$ & 10 & $(7,2)$ & 4 & 1 & YES & YES & NO(2) & $1.00$ & $(6,0)$ & NO & 2559\\
$(86,31)$ & 10 & $(8,3)$ & 4 & 2 & YES & YES & YES & $1.33$ & $(2,2)$ & 2088 & 2560\\
$(86,25)$ & 10 & $(9,2)$ & 5 & 1 & YES & YES & YES & $1.11$ & $(4,1)$ & -- & 2561\\
$(86,25)$ & 10 & $(13,4)$ & 6 & 1 & YES & YES & YES & $1.22$ & $(4,1)$ & NO & 2562\\
$(86,25)$ & 10 & $(24,7)$ & 7 & 2 & YES & YES & YES & $1.12$ & $(2,2)$ & 2405 & 2563\\
$(86,25)$ & 10 & $(31,9)$ & 8 & 1 & YES & YES & YES & $1.12$ & $(2,2)$ & NO & 2564\\
$(86,31)$ & 10 & $(36,13)$ & 8 & 2 & YES & YES & YES & $1.33$ & $(2,2)$ & 2743 & 2565\\
$(86,25)$ & 10 & $(41,12)$ & 8 & 1 & YES & YES & YES & $1.11$ & $(4,1)$ & NO & 2566\\
$(86,25)$ & 10 & $(86,25)$ & 10 & 86 & YES & YES & YES & $1.12$ & $(2,2)$ & NO & 2567\\
$(86,31)$ & 10 & $(86,31)$ & 10 & 86 & YES & YES & YES & $1.11$ & $(2,2)$ & NO & 2568\\
$(87,32)$ & 10 & $(4,1)$ & 3 & 1 & YES & YES & YES & $1.25$ & $(6,0)$ & -- & 2569\\
$(87,23)$ & 10 & $(5,1)$ & 4 & 1 & YES & YES & NO(2) & $0.89$ & $(6,0)$ & NO & 2570\\
$(87,32)$ & 10 & $(5,1)$ & 4 & 1 & YES & YES & YES & $1.22$ & $(4,1)$ & NO & 2571\\
$(87,32)$ & 10 & $(49,18)$ & 8 & 1 & YES & YES & YES & $1.50$ & $(4,1)$ & 3073 & 2572\\
$(87,32)$ & 10 & $(68,25)$ & 9 & 1 & YES & YES & YES & $1.22$ & $(4,1)$ & NO & 2573\\
$(88,21)$ & 12 & $(2,1)$ & 1 & 2 & YES & YES & YES & $1.00$ & $(4,1)$ & -- & 2574\\
$(88,21)$ & 12 & $(2,1)$ & 1 & 2 & YES & YES & YES & $1.12$ & $(4,1)$ & NO & 2575\\
$(88,21)$ & 12 & $(3,1)$ & 2 & 1 & YES & YES & NO(2) & $1.11$ & $(6,0)$ & NO & 2576\\
$(88,37)$ & 10 & $(4,1)$ & 3 & 4 & YES & YES & YES & $1.33$ & $(4,1)$ & NO & 2577\\
$(88,37)$ & 10 & $(4,1)$ & 3 & 4 & YES & YES & YES & $1.33$ & $(4,1)$ & -- & 2578\\
$(88,21)$ & 12 & $(5,1)$ & 4 & 1 & YES & YES & YES & $1.00$ & $(4,1)$ & 1091 & 2579\\
$(88,37)$ & 10 & $(5,1)$ & 4 & 1 & YES & YES & YES & $1.33$ & $(4,1)$ & NO & 2580\\
$(88,37)$ & 10 & $(5,1)$ & 4 & 1 & YES & YES & YES & $1.33$ & $(4,1)$ & NO & 2581\\
$(88,37)$ & 10 & $(5,1)$ & 4 & 1 & YES & YES & YES & $1.33$ & $(4,1)$ & -- & 2582\\
$(88,21)$ & 12 & $(25,6)$ & 9 & 1 & YES & YES & NO(2) & $1.11$ & $(6,0)$ & NO & 2583\\
$(88,37)$ & 10 & $(50,21)$ & 8 & 2 & YES & YES & YES & $1.33$ & $(4,1)$ & 3098 & 2584\\
$(88,37)$ & 10 & $(69,29)$ & 9 & 1 & YES & YES & YES & $1.33$ & $(4,1)$ & NO & 2585\\
$(88,21)$ & 12 & $(88,21)$ & 12 & 88 & YES & YES & NO(2) & $1.00$ & $(6,0)$ & NO & 2586\\
$(89,20)$ & 11 & $(2,1)$ & 1 & 1 & YES & YES & YES & $1.11$ & $(2,2)$ & -- & 2587\\
$(89,25)$ & 10 & $(2,1)$ & 1 & 1 & YES & YES & YES & $1.12$ & $(2,2)$ & NO & 2588\\
$(89,25)$ & 10 & $(2,1)$ & 1 & 1 & YES & YES & YES & $1.12$ & $(2,2)$ & -- & 2589\\
$(89,26)$ & 10 & $(2,1)$ & 1 & 1 & YES & YES & NO(2) & $1.10$ & $(4,1)$ & -- & 2590\\
$(89,27)$ & 10 & $(2,1)$ & 1 & 1 & YES & YES & YES & $1.12$ & $(2,2)$ & NO & 2591\\
$(89,27)$ & 10 & $(2,1)$ & 1 & 1 & YES & YES & YES & $1.00$ & $(2,2)$ & -- & 2592\\
$(89,34)$ & 9 & $(2,1)$ & 1 & 1 & YES & YES & YES & $1.00$ & $(4,1)$ & -- & 2593\\
$(89,40)$ & 11 & $(2,1)$ & 1 & 1 & NO & YES & YES & $1.12$ & $(4,1)$ & -- & 2594\\
$(89,21)$ & 12 & $(3,1)$ & 2 & 1 & YES & YES & NO(2) & $1.00$ & $(6,0)$ & -- & 2595\\
$(89,25)$ & 10 & $(3,1)$ & 2 & 1 & YES & YES & YES & $1.12$ & $(2,2)$ & NO & 2596\\
$(89,26)$ & 10 & $(3,1)$ & 2 & 1 & YES & YES & YES & $0.88$ & $(4,1)$ & -- & 2597\\
$(89,26)$ & 10 & $(3,1)$ & 2 & 1 & YES & YES & YES & $1.00$ & $(4,1)$ & NO & 2598\\
$(89,27)$ & 10 & $(3,1)$ & 2 & 1 & NO & YES & NO(2) & $1.20$ & $(4,1)$ & -- & 2599\\
$(89,32)$ & 10 & $(3,1)$ & 2 & 1 & YES & YES & NO(2) & $1.10$ & $(4,1)$ & -- & 2600\\
$(89,34)$ & 9 & $(3,1)$ & 2 & 1 & YES & YES & YES & $0.88$ & $(6,0)$ & -- & 2601\\
$(89,34)$ & 9 & $(3,1)$ & 2 & 1 & YES & YES & YES & $1.11$ & $(4,1)$ & NO & 2602\\
$(89,21)$ & 12 & $(4,1)$ & 3 & 1 & YES & YES & NO(2) & $1.11$ & $(6,0)$ & -- & 2603\\
$(89,27)$ & 10 & $(4,1)$ & 3 & 1 & YES & YES & NO(2) & $1.20$ & $(4,1)$ & -- & 2604\\
$(89,34)$ & 9 & $(4,1)$ & 3 & 1 & YES & YES & YES & $1.00$ & $(6,0)$ & NO & 2605\\
$(89,20)$ & 11 & $(5,1)$ & 4 & 1 & YES & YES & YES & $1.11$ & $(2,2)$ & NO & 2606\\
$(89,24)$ & 10 & $(5,2)$ & 3 & 1 & YES & YES & YES & $1.00$ & $(6,0)$ & NO & 2607\\
$(89,24)$ & 10 & $(5,2)$ & 3 & 1 & YES & YES & YES & $1.67$ & $(2,2)$ & -- & 2608\\
$(89,25)$ & 10 & $(5,1)$ & 4 & 1 & YES & YES & YES & $1.12$ & $(2,2)$ & NO & 2609\\
$(89,25)$ & 10 & $(5,2)$ & 3 & 1 & YES & YES & YES & $1.22$ & $(4,1)$ & -- & 2610\\
$(89,25)$ & 10 & $(5,2)$ & 3 & 1 & YES & YES & YES & $1.22$ & $(4,1)$ & NO & 2611\\
$(89,26)$ & 10 & $(5,1)$ & 4 & 1 & YES & YES & YES & $1.00$ & $(4,1)$ & NO & 2612\\
$(89,26)$ & 10 & $(5,2)$ & 3 & 1 & YES & YES & YES & $1.00$ & $(6,0)$ & -- & 2613\\
$(89,27)$ & 10 & $(5,2)$ & 3 & 1 & YES & YES & YES & $1.11$ & $(4,1)$ & -- & 2614\\
$(89,34)$ & 9 & $(5,1)$ & 4 & 1 & YES & YES & YES & $1.00$ & $(4,1)$ & NO & 2615\\
$(89,34)$ & 9 & $(5,1)$ & 4 & 1 & YES & YES & YES & $1.00$ & $(4,1)$ & NO & 2616\\
$(89,34)$ & 9 & $(5,2)$ & 3 & 1 & YES & YES & YES & $1.00$ & $(4,1)$ & NO & 2617\\
$(89,34)$ & 9 & $(5,2)$ & 3 & 1 & YES & YES & YES & $1.11$ & $(4,1)$ & -- & 2618\\
$(89,34)$ & 9 & $(5,2)$ & 3 & 1 & YES & YES & YES & $1.22$ & $(4,1)$ & NO & 2619\\
$(89,21)$ & 12 & $(6,1)$ & 5 & 1 & YES & YES & NO(2) & $1.11$ & $(6,0)$ & NO & 2620\\
$(89,26)$ & 10 & $(6,1)$ & 5 & 1 & YES & YES & YES & $1.00$ & $(2,2)$ & NO & 2621\\
$(89,24)$ & 10 & $(7,2)$ & 4 & 1 & YES & YES & YES & $1.11$ & $(4,1)$ & -- & 2622\\
$(89,26)$ & 10 & $(7,2)$ & 4 & 1 & YES & YES & NO(2) & $1.20$ & $(4,1)$ & NO & 2623\\
$(89,27)$ & 10 & $(7,2)$ & 4 & 1 & YES & YES & NO(2) & $1.00$ & $(6,0)$ & NO & 2624\\
$(89,34)$ & 9 & $(7,3)$ & 4 & 1 & YES & YES & YES & $1.11$ & $(4,1)$ & NO & 2625\\
$(89,34)$ & 9 & $(8,3)$ & 4 & 1 & YES & YES & YES & $0.88$ & $(4,1)$ & 2399 & 2626\\
$(89,21)$ & 12 & $(9,2)$ & 5 & 1 & YES & YES & NO(2) & $1.11$ & $(6,0)$ & NO & 2627\\
$(89,26)$ & 10 & $(10,3)$ & 5 & 1 & YES & YES & YES & $0.88$ & $(4,1)$ & NO & 2628\\
$(89,26)$ & 10 & $(11,3)$ & 5 & 1 & YES & YES & YES & $1.00$ & $(6,0)$ & NO & 2629\\
$(89,21)$ & 12 & $(13,3)$ & 6 & 1 & YES & YES & NO(2) & $1.00$ & $(6,0)$ & NO & 2630\\
$(89,34)$ & 9 & $(13,5)$ & 5 & 1 & YES & YES & YES & $0.88$ & $(4,1)$ & NO & 2631\\
$(89,24)$ & 10 & $(15,4)$ & 6 & 1 & YES & YES & YES & $1.22$ & $(2,2)$ & NO & 2632\\
$(89,25)$ & 10 & $(15,4)$ & 6 & 1 & YES & YES & YES & $1.22$ & $(4,1)$ & NO & 2633\\
$(89,24)$ & 10 & $(18,5)$ & 6 & 1 & YES & YES & YES & $1.22$ & $(4,1)$ & NO & 2634\\
$(89,34)$ & 9 & $(18,7)$ & 6 & 1 & YES & YES & YES & $1.12$ & $(2,2)$ & NO & 2635\\
$(89,34)$ & 9 & $(21,8)$ & 6 & 1 & YES & YES & YES & $1.00$ & $(4,1)$ & 2327 & 2636\\
$(89,27)$ & 10 & $(23,7)$ & 7 & 1 & YES & YES & YES & $1.00$ & $(2,2)$ & 2408 & 2637\\
$(89,26)$ & 10 & $(24,7)$ & 7 & 1 & YES & YES & YES & $1.12$ & $(2,2)$ & NO & 2638\\
$(89,25)$ & 10 & $(25,7)$ & 7 & 1 & YES & YES & YES & $1.12$ & $(2,2)$ & 2498 & 2639\\
$(89,34)$ & 9 & $(29,11)$ & 7 & 1 & YES & YES & YES & $1.25$ & $(2,2)$ & NO & 2640\\
$(89,26)$ & 10 & $(31,9)$ & 8 & 1 & YES & YES & YES & $1.00$ & $(6,0)$ & NO & 2641\\
$(89,27)$ & 10 & $(33,10)$ & 8 & 1 & YES & YES & YES & $1.22$ & $(2,2)$ & NO & 2642\\
$(89,34)$ & 9 & $(34,13)$ & 7 & 1 & YES & YES & YES & $1.00$ & $(4,1)$ & NO & 2643\\
$(89,24)$ & 10 & $(41,11)$ & 8 & 1 & YES & YES & YES & $1.22$ & $(4,1)$ & NO & 2644\\
$(89,26)$ & 10 & $(41,12)$ & 8 & 1 & YES & YES & YES & $0.88$ & $(4,1)$ & 2939 & 2645\\
$(89,34)$ & 9 & $(47,18)$ & 8 & 1 & YES & YES & YES & $1.00$ & $(2,2)$ & NO & 2646\\
$(89,24)$ & 10 & $(48,13)$ & 9 & 1 & YES & YES & YES & $1.00$ & $(6,0)$ & 3332 & 2647\\
$(89,20)$ & 11 & $(49,11)$ & 10 & 1 & YES & YES & NO(2) & $1.10$ & $(4,1)$ & NO & 2648\\
$(89,34)$ & 9 & $(55,21)$ & 8 & 1 & YES & YES & YES & $0.88$ & $(6,0)$ & NO & 2649\\
$(89,27)$ & 10 & $(56,17)$ & 9 & 1 & YES & YES & NO(2) & $1.10$ & $(4,1)$ & NO & 2650\\
$(89,34)$ & 9 & $(76,29)$ & 9 & 1 & YES & YES & YES & $1.22$ & $(4,1)$ & NO & 2651\\
$(89,27)$ & 10 & $(79,24)$ & 10 & 1 & YES & YES & YES & $0.88$ & $(6,0)$ & NO & 2652\\
$(89,26)$ & 10 & $(89,26)$ & 10 & 89 & YES & YES & YES & $0.88$ & $(4,1)$ & NO & 2653\\
$(89,34)$ & 9 & $(89,34)$ & 9 & 89 & YES & YES & YES & $0.88$ & $(6,0)$ & NO & 2654\\
$(90,19)$ & 11 & $(3,1)$ & 2 & 3 & YES & YES & YES & $1.12$ & $(2,2)$ & NO & 2655\\
$(90,19)$ & 11 & $(3,1)$ & 2 & 3 & YES & YES & YES & $1.12$ & $(2,2)$ & -- & 2656\\
$(90,19)$ & 11 & $(3,1)$ & 2 & 3 & YES & YES & YES & $1.11$ & $(2,2)$ & NO & 2657\\
$(90,19)$ & 11 & $(6,1)$ & 5 & 6 & YES & YES & YES & $1.11$ & $(2,2)$ & NO & 2658\\
$(90,19)$ & 11 & $(9,2)$ & 5 & 9 & YES & YES & YES & $1.00$ & $(2,2)$ & NO & 2659\\
$(90,19)$ & 11 & $(24,5)$ & 8 & 6 & YES & YES & YES & $1.12$ & $(2,2)$ & NO & 2660\\
$(90,19)$ & 11 & $(90,19)$ & 11 & 90 & YES & YES & YES & $1.11$ & $(2,2)$ & NO & 2661\\
$(91,27)$ & 10 & $(2,1)$ & 1 & 1 & YES & YES & YES & $1.00$ & $(2,2)$ & -- & 2662\\
$(91,25)$ & 10 & $(3,1)$ & 2 & 1 & YES & YES & NO(2) & $0.89$ & $(6,0)$ & NO & 2663\\
$(91,27)$ & 10 & $(3,1)$ & 2 & 1 & YES & YES & YES & $1.00$ & $(4,1)$ & -- & 2664\\
$(91,27)$ & 10 & $(3,1)$ & 2 & 1 & YES & YES & YES & $1.25$ & $(4,1)$ & NO & 2665\\
$(91,27)$ & 10 & $(3,1)$ & 2 & 1 & YES & YES & YES & $1.22$ & $(2,2)$ & 1561 & 2666\\
$(91,27)$ & 10 & $(5,1)$ & 4 & 1 & YES & YES & YES & $0.88$ & $(4,1)$ & NO & 2667\\
$(91,17)$ & 12 & $(6,1)$ & 5 & 1 & YES & YES & YES & $1.22$ & $(2,2)$ & NO & 2668\\
$(91,27)$ & 10 & $(7,2)$ & 4 & 7 & YES & YES & YES & $1.12$ & $(2,2)$ & -- & 2669\\
$(91,17)$ & 12 & $(11,2)$ & 6 & 1 & YES & YES & YES & $1.22$ & $(2,2)$ & NO & 2670\\
$(91,27)$ & 10 & $(13,4)$ & 6 & 13 & YES & YES & YES & $1.22$ & $(4,1)$ & NO & 2671\\
$(91,17)$ & 12 & $(17,3)$ & 7 & 1 & YES & YES & YES & $1.44$ & $(2,2)$ & NO & 2672\\
$(91,27)$ & 10 & $(17,5)$ & 6 & 1 & YES & YES & YES & $0.88$ & $(4,1)$ & NO & 2673\\
$(91,27)$ & 10 & $(37,11)$ & 8 & 1 & YES & YES & YES & $1.12$ & $(4,1)$ & 2858 & 2674\\
$(91,27)$ & 10 & $(91,27)$ & 10 & 91 & YES & YES & YES & $1.00$ & $(4,1)$ & NO & 2675\\
$(92,27)$ & 11 & $(2,1)$ & 1 & 2 & YES & YES & NO(2) & $0.89$ & $(6,0)$ & -- & 2676\\
$(92,33)$ & 10 & $(2,1)$ & 1 & 2 & YES & YES & YES & $1.33$ & $(2,2)$ & -- & 2677\\
$(92,33)$ & 10 & $(4,1)$ & 3 & 4 & YES & YES & YES & $1.33$ & $(2,2)$ & -- & 2678\\
$(92,33)$ & 10 & $(11,4)$ & 5 & 1 & YES & YES & YES & $1.44$ & $(2,2)$ & 1665 & 2679\\
$(92,33)$ & 10 & $(39,14)$ & 8 & 1 & YES & YES & YES & $1.33$ & $(2,2)$ & NO & 2680\\
$(93,25)$ & 10 & $(2,1)$ & 1 & 1 & YES & YES & YES & $0.88$ & $(2,2)$ & NO & 2681\\
$(93,25)$ & 10 & $(2,1)$ & 1 & 1 & YES & YES & YES & $1.22$ & $(2,2)$ & -- & 2682\\
$(93,26)$ & 10 & $(2,1)$ & 1 & 1 & YES & YES & YES & $1.12$ & $(2,2)$ & -- & 2683\\
$(93,25)$ & 10 & $(3,1)$ & 2 & 3 & YES & YES & YES & $1.00$ & $(2,2)$ & -- & 2684\\
$(93,25)$ & 10 & $(3,1)$ & 2 & 3 & YES & YES & YES & $1.22$ & $(2,2)$ & NO & 2685\\
$(93,26)$ & 10 & $(3,1)$ & 2 & 3 & YES & YES & YES & $0.88$ & $(4,1)$ & NO & 2686\\
$(93,26)$ & 10 & $(3,1)$ & 2 & 3 & YES & YES & YES & $1.12$ & $(4,1)$ & -- & 2687\\
$(93,25)$ & 10 & $(5,2)$ & 3 & 1 & YES & YES & YES & $1.22$ & $(4,1)$ & NO & 2688\\
$(93,25)$ & 10 & $(5,2)$ & 3 & 1 & YES & YES & YES & $1.22$ & $(4,1)$ & -- & 2689\\
$(93,25)$ & 10 & $(5,2)$ & 3 & 1 & YES & YES & YES & $1.22$ & $(4,1)$ & NO & 2690\\
$(93,26)$ & 10 & $(5,1)$ & 4 & 1 & YES & YES & YES & $1.00$ & $(4,1)$ & NO & 2691\\
$(93,26)$ & 10 & $(5,2)$ & 3 & 1 & YES & YES & YES & $1.12$ & $(2,2)$ & -- & 2692\\
$(93,26)$ & 10 & $(7,2)$ & 4 & 1 & YES & YES & YES & $1.22$ & $(2,2)$ & NO & 2693\\
$(93,25)$ & 10 & $(10,3)$ & 5 & 1 & YES & YES & YES & $1.22$ & $(4,1)$ & NO & 2694\\
$(93,26)$ & 10 & $(11,3)$ & 5 & 1 & YES & YES & YES & $1.00$ & $(4,1)$ & NO & 2695\\
$(93,25)$ & 10 & $(26,7)$ & 7 & 1 & YES & YES & YES & $0.88$ & $(2,2)$ & NO & 2696\\
$(93,26)$ & 10 & $(29,8)$ & 7 & 1 & YES & YES & YES & $1.25$ & $(2,2)$ & 3348 & 2697\\
$(93,22)$ & 11 & $(30,7)$ & 8 & 3 & YES & YES & YES & $1.00$ & $(2,2)$ & NO & 2698\\
$(93,26)$ & 10 & $(32,9)$ & 8 & 1 & YES & YES & YES & $1.33$ & $(4,1)$ & NO & 2699\\
$(93,25)$ & 10 & $(37,10)$ & 8 & 1 & YES & YES & YES & $1.22$ & $(4,1)$ & NO & 2700\\
$(93,26)$ & 10 & $(43,12)$ & 8 & 1 & YES & YES & YES & $1.00$ & $(4,1)$ & 3010 & 2701\\
$(93,26)$ & 10 & $(93,26)$ & 10 & 93 & YES & YES & YES & $0.88$ & $(4,1)$ & NO & 2702\\
$(94,41)$ & 10 & $(2,1)$ & 1 & 2 & YES & YES & YES & $1.12$ & $(2,2)$ & -- & 2703\\
$(94,39)$ & 10 & $(29,12)$ & 7 & 1 & YES & YES & YES & $1.25$ & $(6,0)$ & NO & 2704\\
$(94,41)$ & 10 & $(39,17)$ & 8 & 1 & YES & YES & YES & $1.33$ & $(2,2)$ & NO & 2705\\
$(95,39)$ & 10 & $(3,1)$ & 2 & 1 & YES & YES & YES & $1.22$ & $(4,1)$ & NO & 2706\\
$(95,39)$ & 10 & $(3,1)$ & 2 & 1 & YES & YES & YES & $1.22$ & $(4,1)$ & -- & 2707\\
$(95,36)$ & 10 & $(4,1)$ & 3 & 1 & YES & YES & YES & $1.12$ & $(6,0)$ & NO & 2708\\
$(95,36)$ & 10 & $(4,1)$ & 3 & 1 & YES & YES & YES & $1.12$ & $(6,0)$ & -- & 2709\\
$(95,39)$ & 10 & $(4,1)$ & 3 & 1 & YES & YES & YES & $1.00$ & $(6,0)$ & -- & 2710\\
$(95,39)$ & 10 & $(7,3)$ & 4 & 1 & YES & YES & YES & $1.22$ & $(4,1)$ & NO & 2711\\
$(95,39)$ & 10 & $(12,5)$ & 5 & 1 & YES & YES & YES & $1.00$ & $(6,0)$ & NO & 2712\\
$(95,39)$ & 10 & $(95,39)$ & 10 & 95 & YES & YES & YES & $1.33$ & $(4,1)$ & NO & 2713\\
$(96,17)$ & 12 & $(2,1)$ & 1 & 2 & YES & YES & YES & $0.88$ & $(4,1)$ & NO & 2714\\
$(96,17)$ & 12 & $(2,1)$ & 1 & 2 & YES & YES & NO(2) & $1.20$ & $(4,1)$ & -- & 2715\\
$(96,43)$ & 11 & $(2,1)$ & 1 & 2 & NO & YES & NO(2) & $1.11$ & $(6,0)$ & -- & 2716\\
$(96,17)$ & 12 & $(3,1)$ & 2 & 3 & YES & YES & YES & $1.22$ & $(2,2)$ & NO & 2717\\
$(96,17)$ & 12 & $(3,1)$ & 2 & 3 & YES & YES & YES & $1.22$ & $(2,2)$ & -- & 2718\\
$(96,17)$ & 12 & $(3,1)$ & 2 & 3 & YES & YES & NO(2) & $1.20$ & $(4,1)$ & NO & 2719\\
$(96,17)$ & 12 & $(5,1)$ & 4 & 1 & YES & YES & YES & $0.88$ & $(4,1)$ & NO & 2720\\
$(96,29)$ & 11 & $(5,1)$ & 4 & 1 & YES & YES & NO(2) & $1.00$ & $(6,0)$ & NO & 2721\\
$(96,29)$ & 11 & $(10,3)$ & 5 & 2 & YES & YES & NO(2) & $1.00$ & $(6,0)$ & 1936 & 2722\\
$(96,17)$ & 12 & $(11,2)$ & 6 & 1 & YES & YES & YES & $1.11$ & $(2,2)$ & NO & 2723\\
$(97,26)$ & 10 & $(2,1)$ & 1 & 1 & YES & YES & YES & $1.11$ & $(2,2)$ & -- & 2724\\
$(97,26)$ & 10 & $(2,1)$ & 1 & 1 & YES & YES & YES & $1.22$ & $(2,2)$ & NO & 2725\\
$(97,35)$ & 10 & $(2,1)$ & 1 & 1 & YES & YES & YES & $1.22$ & $(2,2)$ & -- & 2726\\
$(97,26)$ & 10 & $(3,1)$ & 2 & 1 & YES & YES & YES & $1.11$ & $(2,2)$ & NO & 2727\\
$(97,26)$ & 10 & $(3,1)$ & 2 & 1 & YES & YES & YES & $1.11$ & $(2,2)$ & -- & 2728\\
$(97,30)$ & 11 & $(3,1)$ & 2 & 1 & YES & YES & NO(2) & $1.20$ & $(4,1)$ & NO & 2729\\
$(97,36)$ & 10 & $(3,1)$ & 2 & 1 & YES & YES & YES & $1.33$ & $(4,1)$ & -- & 2730\\
$(97,36)$ & 10 & $(3,1)$ & 2 & 1 & YES & YES & YES & $1.12$ & $(6,0)$ & NO & 2731\\
$(97,26)$ & 10 & $(4,1)$ & 3 & 1 & YES & YES & NO(2) & $1.10$ & $(4,1)$ & -- & 2732\\
$(97,36)$ & 10 & $(4,1)$ & 3 & 1 & YES & YES & YES & $1.22$ & $(4,1)$ & -- & 2733\\
$(97,37)$ & 10 & $(4,1)$ & 3 & 1 & YES & YES & YES & $1.22$ & $(4,1)$ & -- & 2734\\
$(97,22)$ & 11 & $(5,2)$ & 3 & 1 & YES & YES & YES & $1.00$ & $(6,0)$ & NO & 2735\\
$(97,22)$ & 11 & $(5,2)$ & 3 & 1 & YES & YES & YES & $1.00$ & $(6,0)$ & -- & 2736\\
$(97,26)$ & 10 & $(5,2)$ & 3 & 1 & YES & YES & YES & $1.33$ & $(2,2)$ & NO & 2737\\
$(97,26)$ & 10 & $(5,2)$ & 3 & 1 & YES & YES & YES & $1.44$ & $(2,2)$ & -- & 2738\\
$(97,37)$ & 10 & $(5,1)$ & 4 & 1 & YES & YES & YES & $1.22$ & $(4,1)$ & NO & 2739\\
$(97,36)$ & 10 & $(11,4)$ & 5 & 1 & YES & YES & YES & $1.12$ & $(6,0)$ & NO & 2740\\
$(97,22)$ & 11 & $(14,3)$ & 6 & 1 & YES & YES & YES & $1.00$ & $(6,0)$ & 2852 & 2741\\
$(97,36)$ & 10 & $(19,7)$ & 6 & 1 & YES & YES & YES & $1.22$ & $(4,1)$ & 3091 & 2742\\
$(97,35)$ & 10 & $(25,9)$ & 7 & 1 & YES & YES & YES & $1.33$ & $(2,2)$ & 2565 & 2743\\
$(97,22)$ & 11 & $(40,9)$ & 9 & 1 & YES & YES & YES & $1.22$ & $(4,1)$ & NO & 2744\\
$(97,37)$ & 10 & $(55,21)$ & 8 & 1 & YES & YES & YES & $1.22$ & $(4,1)$ & 3201 & 2745\\
$(97,26)$ & 10 & $(56,15)$ & 9 & 1 & YES & YES & NO(2) & $1.10$ & $(4,1)$ & NO & 2746\\
$(97,37)$ & 10 & $(76,29)$ & 9 & 1 & YES & YES & YES & $1.22$ & $(4,1)$ & NO & 2747\\
$(97,26)$ & 10 & $(97,26)$ & 10 & 97 & YES & YES & NO(2) & $1.10$ & $(4,1)$ & NO & 2748\\
$(98,27)$ & 10 & $(2,1)$ & 1 & 2 & YES & YES & YES & $1.11$ & $(2,2)$ & -- & 2749\\
$(98,27)$ & 10 & $(2,1)$ & 1 & 2 & YES & YES & YES & $1.22$ & $(2,2)$ & NO & 2750\\
$(98,29)$ & 10 & $(2,1)$ & 1 & 2 & YES & YES & YES & $1.00$ & $(4,1)$ & -- & 2751\\
$(98,29)$ & 10 & $(2,1)$ & 1 & 2 & YES & YES & YES & $1.00$ & $(4,1)$ & NO & 2752\\
$(98,41)$ & 10 & $(2,1)$ & 1 & 2 & YES & YES & NO(2) & $1.30$ & $(4,1)$ & NO & 2753\\
$(98,27)$ & 10 & $(3,1)$ & 2 & 1 & YES & YES & YES & $1.00$ & $(4,1)$ & -- & 2754\\
$(98,27)$ & 10 & $(3,1)$ & 2 & 1 & YES & YES & YES & $1.12$ & $(4,1)$ & NO & 2755\\
$(98,29)$ & 10 & $(3,1)$ & 2 & 1 & YES & YES & YES & $1.00$ & $(6,0)$ & -- & 2756\\
$(98,29)$ & 10 & $(3,1)$ & 2 & 1 & YES & YES & YES & $1.12$ & $(6,0)$ & NO & 2757\\
$(98,41)$ & 10 & $(3,1)$ & 2 & 1 & YES & YES & YES & $1.33$ & $(4,1)$ & -- & 2758\\
$(98,29)$ & 10 & $(4,1)$ & 3 & 2 & YES & YES & YES & $1.00$ & $(4,1)$ & NO & 2759\\
$(98,41)$ & 10 & $(4,1)$ & 3 & 2 & YES & YES & YES & $1.33$ & $(4,1)$ & 3088 & 2760\\
$(98,41)$ & 10 & $(4,1)$ & 3 & 2 & YES & YES & YES & $1.33$ & $(4,1)$ & -- & 2761\\
$(98,27)$ & 10 & $(5,1)$ & 4 & 1 & YES & YES & YES & $0.88$ & $(4,1)$ & NO & 2762\\
$(98,27)$ & 10 & $(5,2)$ & 3 & 1 & YES & YES & YES & $1.12$ & $(2,2)$ & -- & 2763\\
$(98,29)$ & 10 & $(5,1)$ & 4 & 1 & YES & YES & YES & $1.00$ & $(6,0)$ & NO & 2764\\
$(98,29)$ & 10 & $(5,2)$ & 3 & 1 & YES & YES & YES & $1.22$ & $(4,1)$ & NO & 2765\\
$(98,29)$ & 10 & $(7,2)$ & 4 & 7 & YES & YES & YES & $1.33$ & $(4,1)$ & NO & 2766\\
$(98,29)$ & 10 & $(10,3)$ & 5 & 2 & YES & YES & YES & $1.00$ & $(4,1)$ & NO & 2767\\
$(98,27)$ & 10 & $(15,4)$ & 6 & 1 & YES & YES & YES & $1.11$ & $(4,1)$ & NO & 2768\\
$(98,29)$ & 10 & $(17,5)$ & 6 & 1 & YES & YES & YES & $1.00$ & $(4,1)$ & NO & 2769\\
$(98,27)$ & 10 & $(18,5)$ & 6 & 2 & YES & YES & YES & $1.00$ & $(4,1)$ & NO & 2770\\
$(98,41)$ & 10 & $(19,8)$ & 6 & 1 & YES & YES & YES & $1.33$ & $(4,1)$ & 1724 & 2771\\
$(98,27)$ & 10 & $(25,7)$ & 7 & 1 & YES & YES & YES & $1.12$ & $(2,2)$ & NO & 2772\\
$(98,41)$ & 10 & $(31,13)$ & 7 & 1 & YES & YES & YES & $1.44$ & $(4,1)$ & NO & 2773\\
$(98,27)$ & 10 & $(40,11)$ & 8 & 2 & YES & YES & YES & $0.88$ & $(4,1)$ & 2976 & 2774\\
$(98,41)$ & 10 & $(43,18)$ & 8 & 1 & YES & YES & NO(2) & $1.20$ & $(4,1)$ & NO & 2775\\
$(98,29)$ & 10 & $(44,13)$ & 8 & 2 & YES & YES & YES & $1.00$ & $(6,0)$ & 3051 & 2776\\
$(98,41)$ & 10 & $(55,23)$ & 9 & 1 & YES & YES & YES & $1.33$ & $(4,1)$ & NO & 2777\\
$(98,29)$ & 10 & $(71,21)$ & 9 & 1 & YES & YES & YES & $1.22$ & $(4,1)$ & NO & 2778\\
$(98,29)$ & 10 & $(98,29)$ & 10 & 98 & YES & YES & YES & $0.88$ & $(6,0)$ & NO & 2779\\
$(99,41)$ & 10 & $(2,1)$ & 1 & 1 & YES & YES & YES & $1.33$ & $(2,2)$ & NO & 2780\\
$(99,41)$ & 10 & $(2,1)$ & 1 & 1 & YES & YES & YES & $1.33$ & $(2,2)$ & -- & 2781\\
$(99,29)$ & 10 & $(3,1)$ & 2 & 3 & YES & YES & YES & $1.00$ & $(2,2)$ & NO & 2782\\
$(99,29)$ & 10 & $(3,1)$ & 2 & 3 & YES & YES & YES & $1.00$ & $(4,1)$ & -- & 2783\\
$(99,29)$ & 10 & $(3,1)$ & 2 & 3 & YES & YES & YES & $1.12$ & $(4,1)$ & NO & 2784\\
$(99,41)$ & 10 & $(3,1)$ & 2 & 3 & YES & YES & YES & $1.11$ & $(4,1)$ & -- & 2785\\
$(99,41)$ & 10 & $(3,1)$ & 2 & 3 & YES & YES & YES & $1.22$ & $(4,1)$ & NO & 2786\\
$(99,41)$ & 10 & $(3,1)$ & 2 & 3 & YES & YES & YES & $1.33$ & $(2,2)$ & NO & 2787\\
$(99,41)$ & 10 & $(4,1)$ & 3 & 1 & YES & YES & YES & $1.22$ & $(4,1)$ & NO & 2788\\
$(99,41)$ & 10 & $(4,1)$ & 3 & 1 & YES & YES & YES & $1.22$ & $(4,1)$ & -- & 2789\\
$(99,41)$ & 10 & $(4,1)$ & 3 & 1 & YES & YES & YES & $1.22$ & $(4,1)$ & NO & 2790\\
$(99,29)$ & 10 & $(5,2)$ & 3 & 1 & YES & YES & YES & $1.00$ & $(6,0)$ & NO & 2791\\
$(99,29)$ & 10 & $(5,2)$ & 3 & 1 & YES & YES & YES & $1.12$ & $(2,2)$ & -- & 2792\\
$(99,29)$ & 10 & $(5,2)$ & 3 & 1 & YES & YES & YES & $1.25$ & $(2,2)$ & NO & 2793\\
$(99,29)$ & 10 & $(10,3)$ & 5 & 1 & YES & YES & YES & $1.11$ & $(4,1)$ & NO & 2794\\
$(99,29)$ & 10 & $(17,5)$ & 6 & 1 & YES & YES & YES & $1.11$ & $(2,2)$ & 2283 & 2795\\
$(99,29)$ & 10 & $(24,7)$ & 7 & 3 & YES & YES & YES & $0.75$ & $(4,1)$ & NO & 2796\\
$(99,41)$ & 10 & $(41,17)$ & 8 & 1 & YES & YES & YES & $1.12$ & $(6,0)$ & 3012 & 2797\\
$(99,29)$ & 10 & $(65,19)$ & 9 & 1 & YES & YES & YES & $1.12$ & $(2,2)$ & NO & 2798\\
$(99,23)$ & 11 & $(69,16)$ & 11 & 3 & YES & YES & YES & $1.12$ & $(2,2)$ & NO & 2799\\
$(99,41)$ & 10 & $(70,29)$ & 9 & 1 & YES & YES & YES & $0.88$ & $(6,0)$ & NO & 2800\\
$(99,41)$ & 10 & $(99,41)$ & 10 & 99 & YES & YES & YES & $1.00$ & $(6,0)$ & NO & 2801\\
$(100,27)$ & 10 & $(2,1)$ & 1 & 2 & YES & YES & YES & $1.25$ & $(2,2)$ & NO & 2802\\
$(100,27)$ & 10 & $(2,1)$ & 1 & 2 & YES & YES & YES & $1.00$ & $(2,2)$ & -- & 2803\\
$(100,31)$ & 11 & $(2,1)$ & 1 & 2 & YES & YES & YES & $1.33$ & $(2,2)$ & NO & 2804\\
$(100,31)$ & 11 & $(2,1)$ & 1 & 2 & YES & YES & YES & $1.33$ & $(2,2)$ & -- & 2805\\
$(100,39)$ & 10 & $(2,1)$ & 1 & 2 & NO & YES & YES & $1.22$ & $(2,2)$ & -- & 2806\\
$(100,37)$ & 10 & $(3,1)$ & 2 & 1 & YES & YES & YES & $1.00$ & $(6,0)$ & -- & 2807\\
$(100,41)$ & 10 & $(3,1)$ & 2 & 1 & YES & YES & YES & $1.33$ & $(4,1)$ & -- & 2808\\
$(100,27)$ & 10 & $(4,1)$ & 3 & 4 & YES & YES & YES & $1.12$ & $(2,2)$ & -- & 2809\\
$(100,31)$ & 11 & $(4,1)$ & 3 & 4 & YES & YES & YES & $1.33$ & $(2,2)$ & NO & 2810\\
$(100,37)$ & 10 & $(4,1)$ & 3 & 4 & YES & YES & YES & $1.00$ & $(6,0)$ & NO & 2811\\
$(100,27)$ & 10 & $(5,2)$ & 3 & 5 & YES & YES & YES & $1.22$ & $(4,1)$ & NO & 2812\\
$(100,27)$ & 10 & $(5,2)$ & 3 & 5 & YES & YES & YES & $1.55$ & $(2,2)$ & -- & 2813\\
$(100,39)$ & 10 & $(5,2)$ & 3 & 5 & YES & YES & YES & $1.22$ & $(2,2)$ & 1821 & 2814\\
$(100,21)$ & 12 & $(6,1)$ & 5 & 2 & YES & YES & NO(2) & $1.11$ & $(6,0)$ & 1833 & 2815\\
$(100,27)$ & 10 & $(7,2)$ & 4 & 1 & YES & YES & YES & $1.22$ & $(4,1)$ & -- & 2816\\
$(100,27)$ & 10 & $(11,3)$ & 5 & 1 & YES & YES & YES & $1.00$ & $(2,2)$ & NO & 2817\\
$(100,27)$ & 10 & $(18,5)$ & 6 & 2 & YES & YES & YES & $1.22$ & $(4,1)$ & NO & 2818\\
$(100,21)$ & 12 & $(24,5)$ & 8 & 4 & YES & YES & NO(2) & $1.11$ & $(6,0)$ & NO & 2819\\
$(100,37)$ & 10 & $(46,17)$ & 8 & 2 & YES & YES & YES & $1.00$ & $(6,0)$ & 3100 & 2820\\
$(100,27)$ & 10 & $(48,13)$ & 9 & 4 & YES & YES & YES & $1.00$ & $(6,0)$ & NO & 2821\\
$(100,41)$ & 10 & $(61,25)$ & 9 & 1 & YES & YES & YES & $1.33$ & $(4,1)$ & NO & 2822\\
$(100,37)$ & 10 & $(73,27)$ & 9 & 1 & YES & YES & YES & $1.00$ & $(6,0)$ & NO & 2823\\
$(101,19)$ & 13 & $(2,1)$ & 1 & 1 & YES & YES & NO(2) & $1.00$ & $(6,0)$ & -- & 2824\\
$(101,19)$ & 13 & $(2,1)$ & 1 & 1 & YES & YES & NO(2) & $1.11$ & $(6,0)$ & NO & 2825\\
$(101,23)$ & 11 & $(2,1)$ & 1 & 1 & YES & YES & YES & $1.00$ & $(2,2)$ & -- & 2826\\
$(101,30)$ & 10 & $(2,1)$ & 1 & 1 & YES & YES & YES & $1.00$ & $(4,1)$ & NO & 2827\\
$(101,30)$ & 10 & $(2,1)$ & 1 & 1 & YES & YES & YES & $1.00$ & $(4,1)$ & -- & 2828\\
$(101,44)$ & 10 & $(2,1)$ & 1 & 1 & NO & YES & YES & $1.22$ & $(2,2)$ & -- & 2829\\
$(101,19)$ & 13 & $(3,1)$ & 2 & 1 & YES & YES & NO(2) & $1.30$ & $(4,1)$ & -- & 2830\\
$(101,19)$ & 13 & $(3,1)$ & 2 & 1 & YES & YES & NO(2) & $1.40$ & $(4,1)$ & NO & 2831\\
$(101,23)$ & 11 & $(3,1)$ & 2 & 1 & YES & YES & YES & $0.88$ & $(2,2)$ & -- & 2832\\
$(101,30)$ & 10 & $(3,1)$ & 2 & 1 & YES & YES & YES & $1.12$ & $(6,0)$ & -- & 2833\\
$(101,30)$ & 10 & $(3,1)$ & 2 & 1 & YES & YES & YES & $1.12$ & $(6,0)$ & NO & 2834\\
$(101,37)$ & 10 & $(3,1)$ & 2 & 1 & YES & YES & YES & $1.12$ & $(6,0)$ & NO & 2835\\
$(101,37)$ & 10 & $(3,1)$ & 2 & 1 & YES & YES & YES & $1.33$ & $(2,2)$ & 1663 & 2836\\
$(101,39)$ & 10 & $(3,1)$ & 2 & 1 & YES & YES & YES & $1.00$ & $(6,0)$ & -- & 2837\\
$(101,23)$ & 11 & $(4,1)$ & 3 & 1 & YES & YES & YES & $1.12$ & $(2,2)$ & NO & 2838\\
$(101,30)$ & 10 & $(4,1)$ & 3 & 1 & YES & YES & YES & $1.11$ & $(4,1)$ & -- & 2839\\
$(101,30)$ & 10 & $(4,1)$ & 3 & 1 & YES & YES & YES & $1.22$ & $(4,1)$ & NO & 2840\\
$(101,39)$ & 10 & $(4,1)$ & 3 & 1 & YES & YES & YES & $1.00$ & $(6,0)$ & -- & 2841\\
$(101,22)$ & 11 & $(5,2)$ & 3 & 1 & YES & YES & YES & $1.22$ & $(4,1)$ & NO & 2842\\
$(101,22)$ & 11 & $(5,2)$ & 3 & 1 & YES & YES & YES & $1.33$ & $(4,1)$ & -- & 2843\\
$(101,23)$ & 11 & $(5,2)$ & 3 & 1 & YES & YES & YES & $1.11$ & $(4,1)$ & -- & 2844\\
$(101,30)$ & 10 & $(5,1)$ & 4 & 1 & YES & YES & YES & $1.00$ & $(4,1)$ & NO & 2845\\
$(101,19)$ & 13 & $(6,1)$ & 5 & 1 & YES & YES & NO(2) & $1.00$ & $(6,0)$ & NO & 2846\\
$(101,23)$ & 11 & $(7,2)$ & 4 & 1 & YES & YES & YES & $1.22$ & $(4,1)$ & NO & 2847\\
$(101,30)$ & 10 & $(7,2)$ & 4 & 1 & YES & YES & YES & $1.00$ & $(4,1)$ & NO & 2848\\
$(101,23)$ & 11 & $(9,2)$ & 5 & 1 & YES & YES & YES & $1.00$ & $(2,2)$ & NO & 2849\\
$(101,19)$ & 13 & $(11,2)$ & 6 & 1 & YES & YES & NO(2) & $1.20$ & $(4,1)$ & NO & 2850\\
$(101,23)$ & 11 & $(11,2)$ & 6 & 1 & YES & YES & YES & $1.11$ & $(4,1)$ & NO & 2851\\
$(101,22)$ & 11 & $(13,3)$ & 6 & 1 & YES & YES & YES & $1.00$ & $(6,0)$ & 2741 & 2852\\
$(101,30)$ & 10 & $(13,4)$ & 6 & 1 & YES & YES & YES & $1.22$ & $(4,1)$ & NO & 2853\\
$(101,23)$ & 11 & $(14,3)$ & 6 & 1 & YES & YES & YES & $1.00$ & $(6,0)$ & NO & 2854\\
$(101,23)$ & 11 & $(17,4)$ & 7 & 1 & YES & YES & YES & $1.11$ & $(4,1)$ & NO & 2855\\
$(101,30)$ & 10 & $(17,5)$ & 6 & 1 & YES & YES & YES & $1.00$ & $(6,0)$ & 3048 & 2856\\
$(101,22)$ & 11 & $(19,4)$ & 7 & 1 & YES & YES & YES & $1.00$ & $(6,0)$ & NO & 2857\\
$(101,30)$ & 10 & $(27,8)$ & 7 & 1 & YES & YES & YES & $1.12$ & $(4,1)$ & 2674 & 2858\\
$(101,30)$ & 10 & $(37,11)$ & 8 & 1 & YES & YES & YES & $1.00$ & $(4,1)$ & NO & 2859\\
$(101,22)$ & 11 & $(41,9)$ & 9 & 1 & YES & YES & YES & $1.00$ & $(6,0)$ & NO & 2860\\
$(101,39)$ & 10 & $(57,22)$ & 9 & 1 & YES & YES & YES & $1.00$ & $(6,0)$ & NO & 2861\\
$(101,30)$ & 10 & $(64,19)$ & 9 & 1 & YES & YES & YES & $1.22$ & $(4,1)$ & NO & 2862\\
$(101,30)$ & 10 & $(101,30)$ & 10 & 101 & YES & YES & YES & $1.33$ & $(4,1)$ & NO & 2863\\
$(101,37)$ & 10 & $(101,37)$ & 10 & 101 & YES & YES & YES & $1.12$ & $(6,0)$ & NO & 2864\\
$(101,39)$ & 10 & $(101,39)$ & 10 & 101 & YES & YES & YES & $1.00$ & $(6,0)$ & NO & 2865\\
$(102,31)$ & 11 & $(3,1)$ & 2 & 3 & YES & YES & YES & $1.22$ & $(4,1)$ & -- & 2866\\
$(102,31)$ & 11 & $(3,1)$ & 2 & 3 & YES & YES & YES & $1.40$ & $(4,1)$ & NO & 2867\\
$(102,31)$ & 11 & $(5,1)$ & 4 & 1 & YES & YES & YES & $1.22$ & $(4,1)$ & NO & 2868\\
$(102,31)$ & 11 & $(6,1)$ & 5 & 6 & YES & YES & YES & $1.00$ & $(6,0)$ & NO & 2869\\
$(102,31)$ & 11 & $(56,17)$ & 9 & 2 & YES & YES & YES & $1.22$ & $(4,1)$ & 3237 & 2870\\
$(102,31)$ & 11 & $(102,31)$ & 11 & 102 & YES & YES & YES & $1.22$ & $(4,1)$ & NO & 2871\\
$(103,37)$ & 10 & $(2,1)$ & 1 & 1 & YES & YES & NO(2) & $1.20$ & $(4,1)$ & NO & 2872\\
$(103,24)$ & 11 & $(22,5)$ & 7 & 1 & YES & YES & YES & $0.88$ & $(6,0)$ & NO & 2873\\
$(103,37)$ & 10 & $(39,14)$ & 8 & 1 & YES & YES & NO(2) & $1.20$ & $(4,1)$ & NO & 2874\\
$(104,29)$ & 10 & $(2,1)$ & 1 & 2 & YES & YES & YES & $1.22$ & $(2,2)$ & NO & 2875\\
$(104,29)$ & 10 & $(3,1)$ & 2 & 1 & YES & YES & YES & $0.88$ & $(4,1)$ & -- & 2876\\
$(104,29)$ & 10 & $(3,1)$ & 2 & 1 & YES & YES & YES & $1.00$ & $(4,1)$ & NO & 2877\\
$(104,29)$ & 10 & $(5,2)$ & 3 & 1 & YES & YES & YES & $1.12$ & $(2,2)$ & NO & 2878\\
$(104,29)$ & 10 & $(5,2)$ & 3 & 1 & YES & YES & YES & $1.12$ & $(2,2)$ & -- & 2879\\
$(104,29)$ & 10 & $(9,2)$ & 5 & 1 & YES & YES & YES & $0.88$ & $(6,0)$ & NO & 2880\\
$(104,29)$ & 10 & $(11,3)$ & 5 & 1 & YES & YES & YES & $1.22$ & $(6,0)$ & NO & 2881\\
$(104,29)$ & 10 & $(25,7)$ & 7 & 1 & YES & YES & YES & $1.00$ & $(4,1)$ & NO & 2882\\
$(104,29)$ & 10 & $(79,22)$ & 10 & 1 & YES & YES & YES & $0.88$ & $(6,0)$ & NO & 2883\\
$(105,29)$ & 10 & $(2,1)$ & 1 & 1 & YES & YES & YES & $1.00$ & $(4,1)$ & -- & 2884\\
$(105,29)$ & 10 & $(2,1)$ & 1 & 1 & YES & YES & YES & $1.00$ & $(4,1)$ & NO & 2885\\
$(105,31)$ & 10 & $(2,1)$ & 1 & 1 & YES & YES & YES & $0.88$ & $(4,1)$ & NO & 2886\\
$(105,31)$ & 10 & $(2,1)$ & 1 & 1 & YES & YES & YES & $1.22$ & $(6,0)$ & -- & 2887\\
$(105,41)$ & 10 & $(2,1)$ & 1 & 1 & NO & YES & YES & $1.00$ & $(4,1)$ & -- & 2888\\
$(105,29)$ & 10 & $(3,1)$ & 2 & 3 & YES & YES & YES & $0.88$ & $(6,0)$ & -- & 2889\\
$(105,29)$ & 10 & $(3,1)$ & 2 & 3 & YES & YES & YES & $1.12$ & $(4,1)$ & NO & 2890\\
$(105,29)$ & 10 & $(3,1)$ & 2 & 3 & YES & YES & YES & $1.22$ & $(4,1)$ & NO & 2891\\
$(105,31)$ & 10 & $(3,1)$ & 2 & 3 & YES & YES & YES & $0.88$ & $(6,0)$ & -- & 2892\\
$(105,31)$ & 10 & $(3,1)$ & 2 & 3 & YES & YES & YES & $1.12$ & $(6,0)$ & NO & 2893\\
$(105,31)$ & 10 & $(3,1)$ & 2 & 3 & YES & YES & YES & $1.12$ & $(2,2)$ & NO & 2894\\
$(105,31)$ & 10 & $(4,1)$ & 3 & 1 & YES & YES & YES & $0.75$ & $(4,1)$ & -- & 2895\\
$(105,31)$ & 10 & $(4,1)$ & 3 & 1 & YES & YES & YES & $1.00$ & $(2,2)$ & NO & 2896\\
$(105,32)$ & 11 & $(4,1)$ & 3 & 1 & YES & YES & YES & $1.11$ & $(4,1)$ & -- & 2897\\
$(105,29)$ & 10 & $(5,1)$ & 4 & 5 & YES & YES & YES & $0.88$ & $(6,0)$ & NO & 2898\\
$(105,29)$ & 10 & $(5,2)$ & 3 & 5 & YES & YES & YES & $1.11$ & $(4,1)$ & NO & 2899\\
$(105,29)$ & 10 & $(5,2)$ & 3 & 5 & YES & YES & YES & $1.00$ & $(2,2)$ & -- & 2900\\
$(105,32)$ & 11 & $(6,1)$ & 5 & 3 & YES & YES & YES & $1.33$ & $(4,1)$ & NO & 2901\\
$(105,29)$ & 10 & $(7,2)$ & 4 & 7 & YES & YES & YES & $1.22$ & $(4,1)$ & NO & 2902\\
$(105,29)$ & 10 & $(10,3)$ & 5 & 5 & YES & YES & YES & $1.00$ & $(4,1)$ & NO & 2903\\
$(105,31)$ & 10 & $(10,3)$ & 5 & 5 & YES & YES & YES & $0.88$ & $(4,1)$ & 1887 & 2904\\
$(105,29)$ & 10 & $(11,3)$ & 5 & 1 & YES & YES & YES & $1.00$ & $(4,1)$ & NO & 2905\\
$(105,29)$ & 10 & $(18,5)$ & 6 & 3 & YES & YES & YES & $1.00$ & $(4,1)$ & NO & 2906\\
$(105,29)$ & 10 & $(25,7)$ & 7 & 5 & YES & YES & YES & $1.00$ & $(2,2)$ & 3299 & 2907\\
$(105,31)$ & 10 & $(27,8)$ & 7 & 3 & YES & YES & YES & $0.88$ & $(6,0)$ & NO & 2908\\
$(105,31)$ & 10 & $(44,13)$ & 8 & 1 & YES & YES & YES & $0.75$ & $(4,1)$ & NO & 2909\\
$(105,29)$ & 10 & $(47,13)$ & 8 & 1 & YES & YES & YES & $0.88$ & $(6,0)$ & 3140 & 2910\\
$(105,32)$ & 11 & $(59,18)$ & 9 & 1 & YES & YES & YES & $1.22$ & $(4,1)$ & 3274 & 2911\\
$(105,31)$ & 10 & $(61,18)$ & 9 & 1 & YES & YES & YES & $1.22$ & $(4,1)$ & NO & 2912\\
$(105,44)$ & 10 & $(74,31)$ & 9 & 1 & YES & YES & YES & $1.22$ & $(4,1)$ & NO & 2913\\
$(105,29)$ & 10 & $(105,29)$ & 10 & 105 & YES & YES & YES & $0.88$ & $(6,0)$ & NO & 2914\\
$(105,31)$ & 10 & $(105,31)$ & 10 & 105 & YES & YES & YES & $1.22$ & $(4,1)$ & NO & 2915\\
$(105,32)$ & 11 & $(105,32)$ & 11 & 105 & YES & YES & YES & $1.33$ & $(4,1)$ & NO & 2916\\
$(105,44)$ & 10 & $(105,44)$ & 10 & 105 & YES & YES & YES & $1.33$ & $(4,1)$ & NO & 2917\\
$(106,31)$ & 10 & $(2,1)$ & 1 & 2 & YES & YES & YES & $0.88$ & $(4,1)$ & NO & 2918\\
$(106,31)$ & 10 & $(2,1)$ & 1 & 2 & YES & YES & YES & $1.00$ & $(4,1)$ & -- & 2919\\
$(106,41)$ & 10 & $(2,1)$ & 1 & 2 & YES & YES & YES & $1.22$ & $(2,2)$ & NO & 2920\\
$(106,41)$ & 10 & $(2,1)$ & 1 & 2 & YES & YES & YES & $1.22$ & $(2,2)$ & -- & 2921\\
$(106,23)$ & 11 & $(3,1)$ & 2 & 1 & YES & YES & YES & $1.00$ & $(2,2)$ & -- & 2922\\
$(106,31)$ & 10 & $(3,1)$ & 2 & 1 & YES & YES & YES & $0.88$ & $(6,0)$ & -- & 2923\\
$(106,31)$ & 10 & $(3,1)$ & 2 & 1 & YES & YES & YES & $1.12$ & $(6,0)$ & NO & 2924\\
$(106,31)$ & 10 & $(4,1)$ & 3 & 2 & YES & YES & YES & $0.88$ & $(4,1)$ & NO & 2925\\
$(106,23)$ & 11 & $(5,1)$ & 4 & 1 & YES & YES & YES & $1.00$ & $(2,2)$ & NO & 2926\\
$(106,23)$ & 11 & $(5,2)$ & 3 & 1 & YES & YES & YES & $1.22$ & $(4,1)$ & NO & 2927\\
$(106,23)$ & 11 & $(5,2)$ & 3 & 1 & YES & YES & YES & $1.22$ & $(4,1)$ & -- & 2928\\
$(106,31)$ & 10 & $(5,1)$ & 4 & 1 & YES & YES & YES & $1.11$ & $(4,1)$ & NO & 2929\\
$(106,31)$ & 10 & $(5,2)$ & 3 & 1 & YES & YES & YES & $1.12$ & $(2,2)$ & -- & 2930\\
$(106,31)$ & 10 & $(5,2)$ & 3 & 1 & YES & YES & YES & $1.25$ & $(2,2)$ & NO & 2931\\
$(106,23)$ & 11 & $(7,2)$ & 4 & 1 & YES & YES & YES & $1.22$ & $(4,1)$ & NO & 2932\\
$(106,23)$ & 11 & $(9,2)$ & 5 & 1 & YES & YES & YES & $1.12$ & $(2,2)$ & NO & 2933\\
$(106,31)$ & 10 & $(10,3)$ & 5 & 2 & YES & YES & YES & $0.88$ & $(6,0)$ & NO & 2934\\
$(106,23)$ & 11 & $(11,2)$ & 6 & 1 & YES & YES & YES & $1.22$ & $(4,1)$ & NO & 2935\\
$(106,31)$ & 10 & $(11,3)$ & 5 & 1 & YES & YES & YES & $1.22$ & $(4,1)$ & NO & 2936\\
$(106,31)$ & 10 & $(17,5)$ & 6 & 1 & YES & YES & YES & $0.75$ & $(4,1)$ & NO & 2937\\
$(106,23)$ & 11 & $(19,4)$ & 7 & 1 & YES & YES & YES & $1.22$ & $(4,1)$ & NO & 2938\\
$(106,31)$ & 10 & $(24,7)$ & 7 & 2 & YES & YES & YES & $0.88$ & $(4,1)$ & 2645 & 2939\\
$(106,31)$ & 10 & $(31,9)$ & 8 & 1 & YES & YES & YES & $1.22$ & $(4,1)$ & NO & 2940\\
$(106,31)$ & 10 & $(41,12)$ & 8 & 1 & YES & YES & YES & $1.22$ & $(4,1)$ & NO & 2941\\
$(106,31)$ & 10 & $(58,17)$ & 9 & 2 & YES & YES & YES & $1.12$ & $(2,2)$ & NO & 2942\\
$(106,31)$ & 10 & $(65,19)$ & 9 & 1 & YES & YES & YES & $1.11$ & $(4,1)$ & NO & 2943\\
$(106,41)$ & 10 & $(106,41)$ & 10 & 106 & YES & YES & YES & $0.88$ & $(6,0)$ & NO & 2944\\
$(107,25)$ & 11 & $(3,1)$ & 2 & 1 & YES & YES & YES & $1.00$ & $(2,2)$ & -- & 2945\\
$(107,41)$ & 10 & $(3,1)$ & 2 & 1 & YES & YES & YES & $1.22$ & $(4,1)$ & -- & 2946\\
$(107,41)$ & 10 & $(4,1)$ & 3 & 1 & YES & YES & YES & $1.22$ & $(4,1)$ & -- & 2947\\
$(107,25)$ & 11 & $(5,2)$ & 3 & 1 & YES & YES & YES & $1.22$ & $(4,1)$ & -- & 2948\\
$(107,41)$ & 10 & $(21,8)$ & 6 & 1 & YES & YES & YES & $1.22$ & $(4,1)$ & 1837 & 2949\\
$(107,25)$ & 11 & $(30,7)$ & 8 & 1 & YES & YES & YES & $0.88$ & $(2,2)$ & NO & 2950\\
$(107,41)$ & 10 & $(34,13)$ & 7 & 1 & YES & YES & YES & $1.33$ & $(4,1)$ & NO & 2951\\
$(107,25)$ & 11 & $(43,10)$ & 9 & 1 & YES & YES & YES & $1.22$ & $(4,1)$ & NO & 2952\\
$(107,25)$ & 11 & $(47,11)$ & 9 & 1 & YES & YES & YES & $1.12$ & $(2,2)$ & 3152 & 2953\\
$(107,41)$ & 10 & $(60,23)$ & 9 & 1 & YES & YES & YES & $1.33$ & $(4,1)$ & NO & 2954\\
$(108,41)$ & 10 & $(50,19)$ & 8 & 2 & YES & YES & YES & $1.25$ & $(2,2)$ & 3187 & 2955\\
$(109,30)$ & 10 & $(2,1)$ & 1 & 1 & YES & YES & YES & $1.00$ & $(4,1)$ & -- & 2956\\
$(109,30)$ & 10 & $(2,1)$ & 1 & 1 & YES & YES & YES & $1.12$ & $(4,1)$ & NO & 2957\\
$(109,46)$ & 10 & $(2,1)$ & 1 & 1 & NO & YES & YES & $1.11$ & $(2,2)$ & -- & 2958\\
$(109,30)$ & 10 & $(3,1)$ & 2 & 1 & YES & YES & YES & $0.88$ & $(6,0)$ & -- & 2959\\
$(109,30)$ & 10 & $(3,1)$ & 2 & 1 & YES & YES & YES & $1.00$ & $(6,0)$ & NO & 2960\\
$(109,33)$ & 11 & $(3,1)$ & 2 & 1 & YES & YES & YES & $1.11$ & $(4,1)$ & -- & 2961\\
$(109,33)$ & 11 & $(3,1)$ & 2 & 1 & YES & YES & YES & $1.33$ & $(4,1)$ & NO & 2962\\
$(109,40)$ & 10 & $(3,1)$ & 2 & 1 & YES & YES & YES & $1.22$ & $(4,1)$ & NO & 2963\\
$(109,40)$ & 10 & $(3,1)$ & 2 & 1 & YES & YES & YES & $1.40$ & $(4,1)$ & -- & 2964\\
$(109,30)$ & 10 & $(4,1)$ & 3 & 1 & YES & YES & NO(2) & $1.00$ & $(6,0)$ & NO & 2965\\
$(109,30)$ & 10 & $(4,1)$ & 3 & 1 & YES & YES & YES & $0.88$ & $(4,1)$ & -- & 2966\\
$(109,40)$ & 10 & $(4,1)$ & 3 & 1 & YES & YES & YES & $1.33$ & $(4,1)$ & NO & 2967\\
$(109,46)$ & 10 & $(5,1)$ & 4 & 1 & YES & YES & YES & $1.00$ & $(6,0)$ & NO & 2968\\
$(109,46)$ & 10 & $(5,1)$ & 4 & 1 & YES & YES & YES & $1.00$ & $(6,0)$ & NO & 2969\\
$(109,30)$ & 10 & $(7,2)$ & 4 & 1 & YES & YES & YES & $1.00$ & $(4,1)$ & NO & 2970\\
$(109,30)$ & 10 & $(15,4)$ & 6 & 1 & YES & YES & YES & $1.11$ & $(4,1)$ & NO & 2971\\
$(109,30)$ & 10 & $(18,5)$ & 6 & 1 & YES & YES & YES & $0.88$ & $(6,0)$ & 3138 & 2972\\
$(109,46)$ & 10 & $(19,8)$ & 6 & 1 & YES & YES & YES & $1.12$ & $(6,0)$ & 2526 & 2973\\
$(109,25)$ & 11 & $(22,5)$ & 7 & 1 & YES & YES & NO(2) & $1.20$ & $(4,1)$ & 1889 & 2974\\
$(109,33)$ & 11 & $(23,7)$ & 7 & 1 & YES & YES & YES & $1.00$ & $(6,0)$ & NO & 2975\\
$(109,30)$ & 10 & $(29,8)$ & 7 & 1 & YES & YES & YES & $0.88$ & $(4,1)$ & 2774 & 2976\\
$(109,40)$ & 10 & $(49,18)$ & 8 & 1 & YES & YES & YES & $1.22$ & $(4,1)$ & 3178 & 2977\\
$(109,46)$ & 10 & $(64,27)$ & 9 & 1 & YES & YES & YES & $1.12$ & $(6,0)$ & NO & 2978\\
$(109,40)$ & 10 & $(79,29)$ & 9 & 1 & YES & YES & YES & $1.22$ & $(4,1)$ & NO & 2979\\
$(110,21)$ & 13 & $(3,1)$ & 2 & 1 & YES & YES & NO(2) & $1.11$ & $(6,0)$ & NO & 2980\\
$(111,31)$ & 10 & $(2,1)$ & 1 & 1 & YES & YES & YES & $1.00$ & $(4,1)$ & -- & 2981\\
$(111,31)$ & 10 & $(2,1)$ & 1 & 1 & YES & YES & YES & $1.00$ & $(4,1)$ & NO & 2982\\
$(111,41)$ & 10 & $(2,1)$ & 1 & 1 & YES & YES & YES & $1.12$ & $(6,0)$ & NO & 2983\\
$(111,43)$ & 10 & $(2,1)$ & 1 & 1 & NO & YES & YES & $1.11$ & $(2,2)$ & -- & 2984\\
$(111,46)$ & 10 & $(2,1)$ & 1 & 1 & NO & YES & NO(2) & $1.00$ & $(6,0)$ & -- & 2985\\
$(111,31)$ & 10 & $(3,1)$ & 2 & 3 & YES & YES & YES & $0.88$ & $(6,0)$ & NO & 2986\\
$(111,31)$ & 10 & $(3,1)$ & 2 & 3 & YES & YES & YES & $0.88$ & $(6,0)$ & -- & 2987\\
$(111,31)$ & 10 & $(3,1)$ & 2 & 3 & YES & YES & YES & $1.12$ & $(4,1)$ & NO & 2988\\
$(111,41)$ & 10 & $(3,1)$ & 2 & 3 & YES & YES & YES & $1.11$ & $(4,1)$ & -- & 2989\\
$(111,41)$ & 10 & $(3,1)$ & 2 & 3 & YES & YES & YES & $1.00$ & $(6,0)$ & NO & 2990\\
$(111,41)$ & 10 & $(3,1)$ & 2 & 3 & YES & YES & YES & $1.33$ & $(2,2)$ & NO & 2991\\
$(111,46)$ & 10 & $(3,1)$ & 2 & 3 & YES & YES & YES & $1.00$ & $(6,0)$ & -- & 2992\\
$(111,41)$ & 10 & $(4,1)$ & 3 & 1 & YES & YES & YES & $0.88$ & $(6,0)$ & -- & 2993\\
$(111,41)$ & 10 & $(4,1)$ & 3 & 1 & YES & YES & YES & $1.22$ & $(4,1)$ & NO & 2994\\
$(111,46)$ & 10 & $(4,1)$ & 3 & 1 & YES & YES & YES & $1.00$ & $(6,0)$ & NO & 2995\\
$(111,46)$ & 10 & $(4,1)$ & 3 & 1 & YES & YES & YES & $1.00$ & $(6,0)$ & -- & 2996\\
$(111,31)$ & 10 & $(5,2)$ & 3 & 1 & YES & YES & YES & $1.00$ & $(2,2)$ & -- & 2997\\
$(111,31)$ & 10 & $(5,2)$ & 3 & 1 & YES & YES & YES & $1.12$ & $(2,2)$ & NO & 2998\\
$(111,41)$ & 10 & $(5,1)$ & 4 & 1 & YES & YES & YES & $0.88$ & $(6,0)$ & NO & 2999\\
$(111,43)$ & 10 & $(5,1)$ & 4 & 1 & YES & YES & YES & $1.12$ & $(2,2)$ & NO & 3000\\
$(111,43)$ & 10 & $(5,1)$ & 4 & 1 & YES & YES & YES & $1.25$ & $(2,2)$ & NO & 3001\\
$(111,41)$ & 10 & $(8,3)$ & 4 & 1 & YES & YES & YES & $1.00$ & $(6,0)$ & 2020 & 3002\\
$(111,31)$ & 10 & $(10,3)$ & 5 & 1 & YES & YES & YES & $1.00$ & $(4,1)$ & NO & 3003\\
$(111,31)$ & 10 & $(11,3)$ & 5 & 1 & YES & YES & YES & $0.88$ & $(6,0)$ & NO & 3004\\
$(111,41)$ & 10 & $(11,4)$ & 5 & 1 & YES & YES & YES & $1.22$ & $(4,1)$ & NO & 3005\\
$(111,46)$ & 10 & $(12,5)$ & 5 & 3 & YES & YES & YES & $1.12$ & $(6,0)$ & NO & 3006\\
$(111,31)$ & 10 & $(18,5)$ & 6 & 3 & YES & YES & YES & $1.00$ & $(4,1)$ & NO & 3007\\
$(111,41)$ & 10 & $(19,7)$ & 6 & 1 & YES & YES & YES & $1.00$ & $(6,0)$ & 2541 & 3008\\
$(111,26)$ & 11 & $(21,5)$ & 8 & 3 & YES & YES & YES & $1.12$ & $(2,2)$ & NO & 3009\\
$(111,31)$ & 10 & $(25,7)$ & 7 & 1 & YES & YES & YES & $1.00$ & $(4,1)$ & 2701 & 3010\\
$(111,41)$ & 10 & $(27,10)$ & 7 & 3 & YES & YES & YES & $1.22$ & $(4,1)$ & NO & 3011\\
$(111,46)$ & 10 & $(29,12)$ & 7 & 1 & YES & YES & YES & $1.12$ & $(6,0)$ & 2797 & 3012\\
$(111,31)$ & 10 & $(43,12)$ & 8 & 1 & YES & YES & YES & $0.75$ & $(4,1)$ & NO & 3013\\
$(111,41)$ & 10 & $(46,17)$ & 8 & 1 & YES & YES & YES & $1.00$ & $(6,0)$ & NO & 3014\\
$(111,43)$ & 10 & $(49,19)$ & 8 & 1 & YES & YES & YES & $1.12$ & $(2,2)$ & 3188 & 3015\\
$(111,31)$ & 10 & $(61,17)$ & 9 & 1 & YES & YES & YES & $1.12$ & $(2,2)$ & NO & 3016\\
$(111,41)$ & 10 & $(65,24)$ & 9 & 1 & YES & YES & YES & $1.11$ & $(4,1)$ & NO & 3017\\
$(111,46)$ & 10 & $(70,29)$ & 9 & 1 & YES & YES & YES & $1.00$ & $(6,0)$ & NO & 3018\\
$(111,41)$ & 10 & $(111,41)$ & 10 & 111 & YES & YES & YES & $0.88$ & $(6,0)$ & NO & 3019\\
$(112,31)$ & 10 & $(2,1)$ & 1 & 2 & YES & YES & YES & $1.33$ & $(6,0)$ & NO & 3020\\
$(112,47)$ & 10 & $(2,1)$ & 1 & 2 & NO & YES & YES & $1.25$ & $(2,2)$ & -- & 3021\\
$(112,31)$ & 10 & $(3,1)$ & 2 & 1 & YES & YES & YES & $0.88$ & $(6,0)$ & -- & 3022\\
$(112,31)$ & 10 & $(3,1)$ & 2 & 1 & YES & YES & YES & $1.25$ & $(6,0)$ & NO & 3023\\
$(112,31)$ & 10 & $(7,2)$ & 4 & 7 & YES & YES & YES & $1.33$ & $(6,0)$ & NO & 3024\\
$(112,31)$ & 10 & $(11,3)$ & 5 & 1 & YES & YES & YES & $1.33$ & $(6,0)$ & 2021 & 3025\\
$(112,31)$ & 10 & $(29,8)$ & 7 & 1 & YES & YES & YES & $0.88$ & $(6,0)$ & NO & 3026\\
$(113,35)$ & 11 & $(2,1)$ & 1 & 1 & YES & YES & NO(2) & $1.10$ & $(4,1)$ & -- & 3027\\
$(113,24)$ & 11 & $(3,1)$ & 2 & 1 & YES & YES & YES & $1.00$ & $(2,2)$ & NO & 3028\\
$(113,24)$ & 11 & $(3,1)$ & 2 & 1 & YES & YES & YES & $1.00$ & $(2,2)$ & -- & 3029\\
$(113,24)$ & 11 & $(3,1)$ & 2 & 1 & YES & YES & YES & $1.12$ & $(2,2)$ & NO & 3030\\
$(113,35)$ & 11 & $(3,1)$ & 2 & 1 & YES & YES & NO(2) & $1.10$ & $(4,1)$ & -- & 3031\\
$(113,24)$ & 11 & $(9,2)$ & 5 & 1 & YES & YES & YES & $1.00$ & $(2,2)$ & 2524 & 3032\\
$(113,35)$ & 11 & $(13,4)$ & 6 & 1 & YES & YES & NO(2) & $1.20$ & $(4,1)$ & NO & 3033\\
$(113,24)$ & 11 & $(19,4)$ & 7 & 1 & YES & YES & NO(2) & $1.20$ & $(4,1)$ & NO & 3034\\
$(115,26)$ & 11 & $(2,1)$ & 1 & 1 & YES & YES & YES & $1.00$ & $(2,2)$ & -- & 3035\\
$(115,34)$ & 10 & $(2,1)$ & 1 & 1 & YES & YES & YES & $1.00$ & $(6,0)$ & -- & 3036\\
$(115,34)$ & 10 & $(2,1)$ & 1 & 1 & YES & YES & YES & $1.00$ & $(6,0)$ & NO & 3037\\
$(115,26)$ & 11 & $(3,1)$ & 2 & 1 & YES & YES & YES & $1.00$ & $(4,1)$ & -- & 3038\\
$(115,26)$ & 11 & $(3,1)$ & 2 & 1 & YES & YES & YES & $1.12$ & $(4,1)$ & NO & 3039\\
$(115,26)$ & 11 & $(3,1)$ & 2 & 1 & YES & YES & YES & $1.00$ & $(4,1)$ & NO & 3040\\
$(115,31)$ & 11 & $(3,1)$ & 2 & 1 & YES & YES & YES & $1.11$ & $(4,1)$ & -- & 3041\\
$(115,31)$ & 11 & $(3,1)$ & 2 & 1 & YES & YES & YES & $1.22$ & $(4,1)$ & NO & 3042\\
$(115,34)$ & 10 & $(3,1)$ & 2 & 1 & YES & YES & YES & $1.00$ & $(6,0)$ & -- & 3043\\
$(115,34)$ & 10 & $(4,1)$ & 3 & 1 & YES & YES & YES & $1.12$ & $(6,0)$ & NO & 3044\\
$(115,31)$ & 11 & $(5,1)$ & 4 & 5 & YES & YES & YES & $1.11$ & $(4,1)$ & NO & 3045\\
$(115,34)$ & 10 & $(5,1)$ & 4 & 5 & YES & YES & YES & $0.88$ & $(6,0)$ & NO & 3046\\
$(115,34)$ & 10 & $(7,2)$ & 4 & 1 & YES & YES & YES & $1.00$ & $(6,0)$ & NO & 3047\\
$(115,34)$ & 10 & $(10,3)$ & 5 & 5 & YES & YES & YES & $1.00$ & $(6,0)$ & 2856 & 3048\\
$(115,26)$ & 11 & $(13,3)$ & 6 & 1 & YES & YES & YES & $0.88$ & $(4,1)$ & NO & 3049\\
$(115,34)$ & 10 & $(17,5)$ & 6 & 1 & YES & YES & YES & $1.00$ & $(6,0)$ & NO & 3050\\
$(115,34)$ & 10 & $(27,8)$ & 7 & 1 & YES & YES & YES & $1.00$ & $(6,0)$ & 2776 & 3051\\
$(115,31)$ & 11 & $(37,10)$ & 8 & 1 & YES & YES & YES & $1.12$ & $(6,0)$ & NO & 3052\\
$(115,26)$ & 11 & $(40,9)$ & 9 & 5 & YES & YES & YES & $1.22$ & $(4,1)$ & NO & 3053\\
$(115,34)$ & 10 & $(44,13)$ & 8 & 1 & YES & YES & YES & $1.12$ & $(6,0)$ & NO & 3054\\
$(115,34)$ & 10 & $(71,21)$ & 9 & 1 & YES & YES & YES & $1.12$ & $(6,0)$ & NO & 3055\\
$(115,31)$ & 11 & $(115,31)$ & 11 & 115 & YES & YES & YES & $1.11$ & $(4,1)$ & NO & 3056\\
$(115,34)$ & 10 & $(115,34)$ & 10 & 115 & YES & YES & YES & $0.88$ & $(6,0)$ & NO & 3057\\
$(115,44)$ & 10 & $(115,44)$ & 10 & 115 & YES & YES & YES & $1.22$ & $(4,1)$ & NO & 3058\\
$(116,45)$ & 10 & $(3,1)$ & 2 & 1 & YES & YES & YES & $1.12$ & $(2,2)$ & -- & 3059\\
$(116,27)$ & 11 & $(5,2)$ & 3 & 1 & YES & YES & YES & $1.00$ & $(6,0)$ & NO & 3060\\
$(116,27)$ & 11 & $(5,2)$ & 3 & 1 & YES & YES & YES & $1.00$ & $(6,0)$ & -- & 3061\\
$(116,27)$ & 11 & $(47,11)$ & 9 & 1 & YES & YES & YES & $0.88$ & $(6,0)$ & NO & 3062\\
$(117,31)$ & 11 & $(2,1)$ & 1 & 1 & YES & YES & YES & $1.33$ & $(2,2)$ & NO & 3063\\
$(117,31)$ & 11 & $(2,1)$ & 1 & 1 & YES & YES & YES & $1.33$ & $(2,2)$ & -- & 3064\\
$(117,43)$ & 10 & $(2,1)$ & 1 & 1 & YES & YES & YES & $1.22$ & $(4,1)$ & -- & 3065\\
$(117,43)$ & 10 & $(2,1)$ & 1 & 1 & YES & YES & YES & $1.12$ & $(6,0)$ & NO & 3066\\
$(117,31)$ & 11 & $(3,1)$ & 2 & 3 & YES & YES & YES & $1.33$ & $(2,2)$ & NO & 3067\\
$(117,43)$ & 10 & $(3,1)$ & 2 & 3 & YES & YES & YES & $1.22$ & $(4,1)$ & -- & 3068\\
$(117,49)$ & 10 & $(3,1)$ & 2 & 3 & YES & YES & YES & $1.33$ & $(4,1)$ & -- & 3069\\
$(117,43)$ & 10 & $(5,1)$ & 4 & 1 & YES & YES & YES & $0.88$ & $(6,0)$ & NO & 3070\\
$(117,43)$ & 10 & $(5,2)$ & 3 & 1 & YES & YES & YES & $1.22$ & $(4,1)$ & NO & 3071\\
$(117,49)$ & 10 & $(12,5)$ & 5 & 3 & YES & YES & YES & $1.44$ & $(4,1)$ & NO & 3072\\
$(117,43)$ & 10 & $(19,7)$ & 6 & 1 & YES & YES & YES & $1.50$ & $(4,1)$ & 2572 & 3073\\
$(117,43)$ & 10 & $(30,11)$ & 7 & 3 & YES & YES & YES & $1.33$ & $(4,1)$ & NO & 3074\\
$(117,43)$ & 10 & $(49,18)$ & 8 & 1 & YES & YES & YES & $1.22$ & $(4,1)$ & NO & 3075\\
$(118,27)$ & 11 & $(3,1)$ & 2 & 1 & YES & YES & YES & $0.88$ & $(4,1)$ & NO & 3076\\
$(118,27)$ & 11 & $(3,1)$ & 2 & 1 & YES & YES & YES & $0.88$ & $(4,1)$ & -- & 3077\\
$(118,27)$ & 11 & $(3,1)$ & 2 & 1 & YES & YES & YES & $1.00$ & $(4,1)$ & NO & 3078\\
$(118,27)$ & 11 & $(17,4)$ & 7 & 1 & YES & YES & YES & $1.00$ & $(6,0)$ & NO & 3079\\
$(118,27)$ & 11 & $(22,5)$ & 7 & 2 & YES & YES & YES & $1.00$ & $(4,1)$ & NO & 3080\\
$(119,27)$ & 12 & $(2,1)$ & 1 & 1 & YES & YES & YES & $1.33$ & $(2,2)$ & -- & 3081\\
$(119,27)$ & 12 & $(2,1)$ & 1 & 1 & YES & YES & YES & $1.44$ & $(2,2)$ & NO & 3082\\
$(119,44)$ & 10 & $(2,1)$ & 1 & 1 & NO & YES & NO(2) & $1.11$ & $(6,0)$ & -- & 3083\\
$(119,44)$ & 10 & $(2,1)$ & 1 & 1 & YES & YES & YES & $1.12$ & $(6,0)$ & NO & 3084\\
$(119,46)$ & 10 & $(2,1)$ & 1 & 1 & NO & YES & NO(2) & $1.20$ & $(4,1)$ & -- & 3085\\
$(119,50)$ & 10 & $(2,1)$ & 1 & 1 & YES & YES & YES & $1.33$ & $(4,1)$ & -- & 3086\\
$(119,36)$ & 11 & $(3,1)$ & 2 & 1 & YES & YES & YES & $1.22$ & $(4,1)$ & -- & 3087\\
$(119,36)$ & 11 & $(3,1)$ & 2 & 1 & YES & YES & YES & $1.33$ & $(4,1)$ & 2760 & 3088\\
$(119,44)$ & 10 & $(3,1)$ & 2 & 1 & YES & YES & YES & $1.11$ & $(4,1)$ & -- & 3089\\
$(119,32)$ & 11 & $(6,1)$ & 5 & 1 & YES & YES & YES & $1.22$ & $(4,1)$ & NO & 3090\\
$(119,44)$ & 10 & $(8,3)$ & 4 & 1 & YES & YES & YES & $1.22$ & $(4,1)$ & 2742 & 3091\\
$(119,50)$ & 10 & $(12,5)$ & 5 & 1 & YES & YES & YES & $1.22$ & $(4,1)$ & 2120 & 3092\\
$(119,46)$ & 10 & $(13,5)$ & 5 & 1 & YES & YES & YES & $1.00$ & $(6,0)$ & NO & 3093\\
$(119,26)$ & 11 & $(14,3)$ & 6 & 7 & YES & YES & YES & $0.75$ & $(4,1)$ & NO & 3094\\
$(119,22)$ & 12 & $(16,3)$ & 7 & 1 & YES & YES & YES & $1.22$ & $(2,2)$ & NO & 3095\\
$(119,46)$ & 10 & $(18,7)$ & 6 & 1 & YES & YES & YES & $1.25$ & $(2,2)$ & 3186 & 3096\\
$(119,44)$ & 10 & $(19,7)$ & 6 & 1 & YES & YES & YES & $1.12$ & $(6,0)$ & NO & 3097\\
$(119,50)$ & 10 & $(19,8)$ & 6 & 1 & YES & YES & YES & $1.33$ & $(4,1)$ & 2584 & 3098\\
$(119,36)$ & 11 & $(23,7)$ & 7 & 1 & YES & YES & YES & $1.11$ & $(4,1)$ & 3280 & 3099\\
$(119,44)$ & 10 & $(27,10)$ & 7 & 1 & YES & YES & YES & $1.00$ & $(6,0)$ & 2820 & 3100\\
$(119,50)$ & 10 & $(69,29)$ & 9 & 1 & YES & YES & YES & $1.22$ & $(4,1)$ & NO & 3101\\
$(119,46)$ & 10 & $(75,29)$ & 9 & 1 & YES & YES & YES & $1.12$ & $(2,2)$ & NO & 3102\\
$(119,36)$ & 11 & $(76,23)$ & 10 & 1 & YES & YES & YES & $1.11$ & $(4,1)$ & NO & 3103\\
$(119,32)$ & 11 & $(119,32)$ & 11 & 119 & YES & YES & YES & $1.33$ & $(4,1)$ & NO & 3104\\
$(120,19)$ & 14 & $(2,1)$ & 1 & 2 & YES & YES & YES & $0.88$ & $(4,1)$ & NO & 3105\\
$(120,19)$ & 14 & $(2,1)$ & 1 & 2 & YES & YES & NO(2) & $1.20$ & $(4,1)$ & -- & 3106\\
$(120,19)$ & 14 & $(3,1)$ & 2 & 3 & YES & YES & NO(2) & $1.11$ & $(6,0)$ & -- & 3107\\
$(120,19)$ & 14 & $(3,1)$ & 2 & 3 & YES & YES & NO(2) & $1.22$ & $(6,0)$ & NO & 3108\\
$(120,19)$ & 14 & $(7,1)$ & 6 & 1 & YES & YES & YES & $1.12$ & $(4,1)$ & NO & 3109\\
$(120,19)$ & 14 & $(25,4)$ & 9 & 5 & YES & YES & NO(2) & $1.11$ & $(6,0)$ & NO & 3110\\
$(121,46)$ & 10 & $(2,1)$ & 1 & 1 & YES & YES & YES & $1.12$ & $(6,0)$ & -- & 3111\\
$(121,46)$ & 10 & $(2,1)$ & 1 & 1 & YES & YES & YES & $1.12$ & $(6,0)$ & NO & 3112\\
$(121,50)$ & 10 & $(2,1)$ & 1 & 1 & NO & YES & YES & $1.12$ & $(2,2)$ & -- & 3113\\
$(121,46)$ & 10 & $(3,1)$ & 2 & 1 & YES & YES & YES & $1.00$ & $(6,0)$ & NO & 3114\\
$(121,50)$ & 10 & $(3,1)$ & 2 & 1 & YES & YES & YES & $1.22$ & $(4,1)$ & -- & 3115\\
$(121,50)$ & 10 & $(5,2)$ & 3 & 1 & YES & YES & YES & $1.22$ & $(4,1)$ & NO & 3116\\
$(121,46)$ & 10 & $(29,11)$ & 7 & 1 & YES & YES & YES & $1.25$ & $(2,2)$ & NO & 3117\\
$(121,46)$ & 10 & $(121,46)$ & 10 & 121 & YES & YES & YES & $1.12$ & $(2,2)$ & NO & 3118\\
$(122,37)$ & 11 & $(2,1)$ & 1 & 2 & YES & YES & YES & $1.12$ & $(6,0)$ & -- & 3119\\
$(122,33)$ & 11 & $(3,1)$ & 2 & 1 & YES & YES & YES & $1.22$ & $(4,1)$ & -- & 3120\\
$(122,33)$ & 11 & $(3,1)$ & 2 & 1 & YES & YES & YES & $1.33$ & $(4,1)$ & NO & 3121\\
$(122,33)$ & 11 & $(4,1)$ & 3 & 2 & YES & YES & YES & $1.00$ & $(6,0)$ & -- & 3122\\
$(122,37)$ & 11 & $(6,1)$ & 5 & 2 & YES & YES & YES & $0.88$ & $(6,0)$ & NO & 3123\\
$(122,37)$ & 11 & $(10,3)$ & 5 & 2 & YES & YES & YES & $1.00$ & $(6,0)$ & NO & 3124\\
$(122,33)$ & 11 & $(15,4)$ & 6 & 1 & YES & YES & YES & $1.22$ & $(4,1)$ & NO & 3125\\
$(122,37)$ & 11 & $(23,7)$ & 7 & 1 & YES & YES & YES & $1.00$ & $(6,0)$ & NO & 3126\\
$(122,33)$ & 11 & $(26,7)$ & 7 & 2 & YES & YES & YES & $1.00$ & $(6,0)$ & NO & 3127\\
$(123,34)$ & 10 & $(2,1)$ & 1 & 1 & YES & YES & YES & $0.88$ & $(6,0)$ & NO & 3128\\
$(123,34)$ & 10 & $(2,1)$ & 1 & 1 & YES & YES & YES & $0.88$ & $(6,0)$ & -- & 3129\\
$(123,22)$ & 12 & $(3,1)$ & 2 & 3 & YES & YES & YES & $1.00$ & $(2,2)$ & NO & 3130\\
$(123,22)$ & 12 & $(3,1)$ & 2 & 3 & YES & YES & NO(2) & $1.11$ & $(4,1)$ & -- & 3131\\
$(123,22)$ & 12 & $(3,1)$ & 2 & 3 & YES & YES & NO(2) & $1.22$ & $(4,1)$ & NO & 3132\\
$(123,34)$ & 10 & $(3,1)$ & 2 & 3 & YES & YES & YES & $0.88$ & $(6,0)$ & NO & 3133\\
$(123,34)$ & 10 & $(3,1)$ & 2 & 3 & YES & YES & YES & $1.00$ & $(6,0)$ & -- & 3134\\
$(123,22)$ & 12 & $(4,1)$ & 3 & 1 & YES & YES & YES & $1.00$ & $(2,2)$ & NO & 3135\\
$(123,34)$ & 10 & $(5,1)$ & 4 & 1 & YES & YES & YES & $0.88$ & $(6,0)$ & NO & 3136\\
$(123,34)$ & 10 & $(7,2)$ & 4 & 1 & YES & YES & YES & $1.00$ & $(6,0)$ & NO & 3137\\
$(123,34)$ & 10 & $(11,3)$ & 5 & 1 & YES & YES & YES & $0.88$ & $(6,0)$ & 2972 & 3138\\
$(123,34)$ & 10 & $(18,5)$ & 6 & 3 & YES & YES & YES & $0.75$ & $(6,0)$ & NO & 3139\\
$(123,34)$ & 10 & $(29,8)$ & 7 & 1 & YES & YES & YES & $0.88$ & $(6,0)$ & 2910 & 3140\\
$(123,34)$ & 10 & $(47,13)$ & 8 & 1 & YES & YES & YES & $1.00$ & $(6,0)$ & NO & 3141\\
$(124,23)$ & 12 & $(2,1)$ & 1 & 2 & YES & YES & YES & $1.00$ & $(2,2)$ & NO & 3142\\
$(124,23)$ & 12 & $(2,1)$ & 1 & 2 & YES & YES & YES & $1.22$ & $(2,2)$ & -- & 3143\\
$(124,27)$ & 12 & $(2,1)$ & 1 & 2 & YES & YES & YES & $1.22$ & $(2,2)$ & -- & 3144\\
$(124,27)$ & 12 & $(2,1)$ & 1 & 2 & YES & YES & YES & $1.33$ & $(2,2)$ & NO & 3145\\
$(124,23)$ & 12 & $(5,2)$ & 3 & 1 & YES & YES & YES & $1.22$ & $(4,1)$ & NO & 3146\\
$(124,23)$ & 12 & $(5,2)$ & 3 & 1 & YES & YES & YES & $1.33$ & $(4,1)$ & -- & 3147\\
$(124,23)$ & 12 & $(6,1)$ & 5 & 2 & YES & YES & YES & $1.00$ & $(2,2)$ & NO & 3148\\
$(124,23)$ & 12 & $(9,2)$ & 5 & 1 & YES & YES & YES & $1.22$ & $(4,1)$ & NO & 3149\\
$(124,23)$ & 12 & $(11,2)$ & 6 & 1 & YES & YES & YES & $1.00$ & $(2,2)$ & NO & 3150\\
$(124,23)$ & 12 & $(16,3)$ & 7 & 4 & YES & YES & YES & $1.00$ & $(2,2)$ & NO & 3151\\
$(124,29)$ & 11 & $(30,7)$ & 8 & 2 & YES & YES & YES & $1.12$ & $(2,2)$ & 2953 & 3152\\
$(124,29)$ & 11 & $(43,10)$ & 9 & 1 & YES & YES & YES & $1.22$ & $(4,1)$ & NO & 3153\\
$(124,29)$ & 11 & $(64,15)$ & 10 & 4 & YES & YES & YES & $0.88$ & $(6,0)$ & NO & 3154\\
$(125,29)$ & 12 & $(43,10)$ & 9 & 1 & YES & YES & NO(2) & $1.00$ & $(6,0)$ & NO & 3155\\
$(127,29)$ & 11 & $(2,1)$ & 1 & 1 & YES & YES & YES & $0.88$ & $(4,1)$ & NO & 3156\\
$(127,29)$ & 11 & $(2,1)$ & 1 & 1 & YES & YES & YES & $0.88$ & $(4,1)$ & -- & 3157\\
$(127,30)$ & 12 & $(2,1)$ & 1 & 1 & YES & YES & NO(2) & $0.89$ & $(6,0)$ & -- & 3158\\
$(127,30)$ & 12 & $(2,1)$ & 1 & 1 & YES & YES & NO(2) & $1.00$ & $(6,0)$ & NO & 3159\\
$(127,29)$ & 11 & $(3,1)$ & 2 & 1 & YES & YES & YES & $0.75$ & $(6,0)$ & NO & 3160\\
$(127,29)$ & 11 & $(3,1)$ & 2 & 1 & YES & YES & YES & $1.30$ & $(4,1)$ & -- & 3161\\
$(127,30)$ & 12 & $(3,1)$ & 2 & 1 & YES & YES & NO(2) & $1.20$ & $(4,1)$ & NO & 3162\\
$(127,30)$ & 12 & $(3,1)$ & 2 & 1 & YES & YES & NO(2) & $1.20$ & $(4,1)$ & -- & 3163\\
$(127,29)$ & 11 & $(5,1)$ & 4 & 1 & YES & YES & YES & $0.88$ & $(4,1)$ & NO & 3164\\
$(127,29)$ & 11 & $(9,2)$ & 5 & 1 & YES & YES & YES & $1.11$ & $(4,1)$ & NO & 3165\\
$(127,29)$ & 11 & $(11,2)$ & 6 & 1 & YES & YES & YES & $1.11$ & $(4,1)$ & NO & 3166\\
$(127,29)$ & 11 & $(13,3)$ & 6 & 1 & YES & YES & YES & $1.00$ & $(4,1)$ & NO & 3167\\
$(127,30)$ & 12 & $(13,3)$ & 6 & 1 & YES & YES & NO(2) & $1.00$ & $(6,0)$ & NO & 3168\\
$(127,29)$ & 11 & $(22,5)$ & 7 & 1 & YES & YES & YES & $0.88$ & $(4,1)$ & NO & 3169\\
$(127,29)$ & 11 & $(79,18)$ & 10 & 1 & YES & YES & YES & $1.11$ & $(4,1)$ & 3404 & 3170\\
$(128,47)$ & 10 & $(2,1)$ & 1 & 2 & YES & YES & YES & $1.22$ & $(4,1)$ & 2387 & 3171\\
$(128,49)$ & 10 & $(3,1)$ & 2 & 1 & YES & YES & YES & $1.11$ & $(4,1)$ & -- & 3172\\
$(128,47)$ & 10 & $(4,1)$ & 3 & 4 & YES & YES & YES & $1.30$ & $(4,1)$ & -- & 3173\\
$(128,49)$ & 10 & $(4,1)$ & 3 & 4 & YES & YES & YES & $1.00$ & $(2,2)$ & -- & 3174\\
$(128,49)$ & 10 & $(4,1)$ & 3 & 4 & YES & YES & YES & $1.38$ & $(2,2)$ & NO & 3175\\
$(128,29)$ & 11 & $(13,3)$ & 6 & 1 & YES & YES & YES & $1.11$ & $(2,2)$ & NO & 3176\\
$(128,49)$ & 10 & $(13,5)$ & 5 & 1 & YES & YES & YES & $1.33$ & $(4,1)$ & NO & 3177\\
$(128,47)$ & 10 & $(30,11)$ & 7 & 2 & YES & YES & YES & $1.22$ & $(4,1)$ & 2977 & 3178\\
$(128,47)$ & 10 & $(49,18)$ & 8 & 1 & YES & YES & YES & $1.22$ & $(4,1)$ & NO & 3179\\
$(129,23)$ & 12 & $(2,1)$ & 1 & 1 & YES & YES & YES & $1.22$ & $(2,2)$ & NO & 3180\\
$(129,50)$ & 10 & $(2,1)$ & 1 & 1 & NO & YES & YES & $1.12$ & $(2,2)$ & -- & 3181\\
$(129,50)$ & 10 & $(3,1)$ & 2 & 3 & YES & YES & YES & $1.12$ & $(2,2)$ & -- & 3182\\
$(129,23)$ & 12 & $(5,1)$ & 4 & 1 & YES & YES & YES & $1.22$ & $(2,2)$ & NO & 3183\\
$(129,50)$ & 10 & $(5,1)$ & 4 & 1 & YES & YES & YES & $1.12$ & $(2,2)$ & NO & 3184\\
$(129,50)$ & 10 & $(5,1)$ & 4 & 1 & YES & YES & YES & $1.25$ & $(2,2)$ & NO & 3185\\
$(129,50)$ & 10 & $(13,5)$ & 5 & 1 & YES & YES & YES & $1.25$ & $(2,2)$ & 3096 & 3186\\
$(129,49)$ & 10 & $(29,11)$ & 7 & 1 & YES & YES & YES & $1.25$ & $(2,2)$ & 2955 & 3187\\
$(129,50)$ & 10 & $(31,12)$ & 7 & 1 & YES & YES & YES & $1.12$ & $(2,2)$ & 3015 & 3188\\
$(129,50)$ & 10 & $(49,19)$ & 8 & 1 & YES & YES & YES & $1.38$ & $(2,2)$ & NO & 3189\\
$(131,24)$ & 13 & $(2,1)$ & 1 & 1 & YES & YES & NO(2) & $1.20$ & $(4,1)$ & -- & 3190\\
$(131,40)$ & 11 & $(2,1)$ & 1 & 1 & YES & YES & YES & $1.22$ & $(4,1)$ & -- & 3191\\
$(131,50)$ & 10 & $(2,1)$ & 1 & 1 & YES & YES & YES & $1.22$ & $(4,1)$ & -- & 3192\\
$(131,30)$ & 11 & $(3,1)$ & 2 & 1 & YES & YES & YES & $0.75$ & $(4,1)$ & NO & 3193\\
$(131,30)$ & 11 & $(3,1)$ & 2 & 1 & YES & YES & YES & $1.12$ & $(4,1)$ & -- & 3194\\
$(131,50)$ & 10 & $(3,1)$ & 2 & 1 & YES & YES & YES & $1.22$ & $(4,1)$ & NO & 3195\\
$(131,24)$ & 13 & $(5,1)$ & 4 & 1 & YES & YES & NO(2) & $1.30$ & $(4,1)$ & NO & 3196\\
$(131,40)$ & 11 & $(6,1)$ & 5 & 1 & YES & YES & YES & $1.11$ & $(4,1)$ & NO & 3197\\
$(131,24)$ & 13 & $(7,1)$ & 6 & 1 & YES & YES & NO(2) & $1.20$ & $(4,1)$ & NO & 3198\\
$(131,40)$ & 11 & $(10,3)$ & 5 & 1 & YES & YES & YES & $1.11$ & $(4,1)$ & NO & 3199\\
$(131,50)$ & 10 & $(13,5)$ & 5 & 1 & YES & YES & YES & $1.22$ & $(4,1)$ & 2330 & 3200\\
$(131,50)$ & 10 & $(21,8)$ & 6 & 1 & YES & YES & YES & $1.22$ & $(4,1)$ & 2745 & 3201\\
$(131,30)$ & 11 & $(22,5)$ & 7 & 1 & YES & YES & YES & $1.22$ & $(4,1)$ & 3316 & 3202\\
$(131,40)$ & 11 & $(36,11)$ & 8 & 1 & YES & YES & YES & $1.22$ & $(4,1)$ & NO & 3203\\
$(131,50)$ & 10 & $(55,21)$ & 8 & 1 & YES & YES & YES & $1.00$ & $(2,2)$ & NO & 3204\\
$(131,30)$ & 11 & $(61,14)$ & 10 & 1 & YES & YES & YES & $0.88$ & $(6,0)$ & NO & 3205\\
$(133,31)$ & 12 & $(2,1)$ & 1 & 1 & YES & YES & YES & $1.00$ & $(6,0)$ & -- & 3206\\
$(133,31)$ & 12 & $(2,1)$ & 1 & 1 & YES & YES & YES & $1.00$ & $(6,0)$ & NO & 3207\\
$(133,58)$ & 11 & $(2,1)$ & 1 & 1 & NO & YES & YES & $1.33$ & $(2,2)$ & -- & 3208\\
$(133,30)$ & 12 & $(3,1)$ & 2 & 1 & YES & YES & YES & $1.22$ & $(4,1)$ & -- & 3209\\
$(133,30)$ & 12 & $(3,1)$ & 2 & 1 & YES & YES & YES & $1.33$ & $(4,1)$ & NO & 3210\\
$(133,31)$ & 12 & $(3,1)$ & 2 & 1 & YES & YES & YES & $1.33$ & $(4,1)$ & NO & 3211\\
$(133,31)$ & 12 & $(3,1)$ & 2 & 1 & YES & YES & YES & $1.55$ & $(2,2)$ & -- & 3212\\
$(133,36)$ & 11 & $(3,1)$ & 2 & 1 & YES & YES & YES & $1.33$ & $(4,1)$ & -- & 3213\\
$(133,36)$ & 11 & $(3,1)$ & 2 & 1 & YES & YES & YES & $1.33$ & $(4,1)$ & NO & 3214\\
$(133,39)$ & 11 & $(3,1)$ & 2 & 1 & YES & YES & YES & $1.00$ & $(4,1)$ & -- & 3215\\
$(133,39)$ & 11 & $(3,1)$ & 2 & 1 & YES & YES & YES & $1.40$ & $(4,1)$ & NO & 3216\\
$(133,30)$ & 12 & $(4,1)$ & 3 & 1 & YES & YES & YES & $1.00$ & $(6,0)$ & -- & 3217\\
$(133,31)$ & 12 & $(5,1)$ & 4 & 1 & YES & YES & YES & $1.00$ & $(6,0)$ & NO & 3218\\
$(133,31)$ & 12 & $(6,1)$ & 5 & 1 & YES & YES & YES & $1.00$ & $(6,0)$ & NO & 3219\\
$(133,31)$ & 12 & $(9,2)$ & 5 & 1 & YES & YES & YES & $1.00$ & $(6,0)$ & NO & 3220\\
$(133,31)$ & 12 & $(17,4)$ & 7 & 1 & YES & YES & YES & $1.33$ & $(4,1)$ & NO & 3221\\
$(133,30)$ & 12 & $(22,5)$ & 7 & 1 & YES & YES & YES & $1.22$ & $(4,1)$ & NO & 3222\\
$(133,36)$ & 11 & $(26,7)$ & 7 & 1 & YES & YES & YES & $1.22$ & $(4,1)$ & 3358 & 3223\\
$(133,31)$ & 12 & $(43,10)$ & 9 & 1 & YES & YES & YES & $1.00$ & $(6,0)$ & NO & 3224\\
$(134,29)$ & 11 & $(2,1)$ & 1 & 2 & YES & YES & YES & $0.88$ & $(4,1)$ & NO & 3225\\
$(134,29)$ & 11 & $(2,1)$ & 1 & 2 & YES & YES & YES & $1.00$ & $(4,1)$ & -- & 3226\\
$(134,29)$ & 11 & $(3,1)$ & 2 & 1 & YES & YES & YES & $0.88$ & $(6,0)$ & NO & 3227\\
$(134,29)$ & 11 & $(3,1)$ & 2 & 1 & YES & YES & YES & $1.12$ & $(6,0)$ & -- & 3228\\
$(134,29)$ & 11 & $(3,1)$ & 2 & 1 & YES & YES & YES & $1.33$ & $(4,1)$ & NO & 3229\\
$(134,29)$ & 11 & $(4,1)$ & 3 & 2 & YES & YES & YES & $0.88$ & $(4,1)$ & NO & 3230\\
$(134,29)$ & 11 & $(9,2)$ & 5 & 1 & YES & YES & YES & $1.22$ & $(4,1)$ & NO & 3231\\
$(134,29)$ & 11 & $(11,2)$ & 6 & 1 & YES & YES & YES & $1.11$ & $(4,1)$ & NO & 3232\\
$(135,41)$ & 11 & $(2,1)$ & 1 & 1 & YES & YES & YES & $1.22$ & $(4,1)$ & -- & 3233\\
$(135,41)$ & 11 & $(4,1)$ & 3 & 1 & YES & YES & YES & $1.00$ & $(6,0)$ & -- & 3234\\
$(135,41)$ & 11 & $(5,1)$ & 4 & 5 & YES & YES & YES & $1.22$ & $(4,1)$ & NO & 3235\\
$(135,41)$ & 11 & $(7,2)$ & 4 & 1 & YES & YES & YES & $1.22$ & $(4,1)$ & NO & 3236\\
$(135,41)$ & 11 & $(23,7)$ & 7 & 1 & YES & YES & YES & $1.22$ & $(4,1)$ & 2870 & 3237\\
$(135,41)$ & 11 & $(33,10)$ & 8 & 3 & YES & YES & YES & $1.22$ & $(4,1)$ & NO & 3238\\
$(135,41)$ & 11 & $(135,41)$ & 11 & 135 & YES & YES & YES & $1.22$ & $(4,1)$ & NO & 3239\\
$(137,32)$ & 12 & $(2,1)$ & 1 & 1 & YES & YES & YES & $1.00$ & $(6,0)$ & -- & 3240\\
$(137,32)$ & 12 & $(2,1)$ & 1 & 1 & YES & YES & YES & $1.12$ & $(6,0)$ & NO & 3241\\
$(137,37)$ & 11 & $(2,1)$ & 1 & 1 & YES & YES & YES & $1.12$ & $(6,0)$ & -- & 3242\\
$(137,30)$ & 12 & $(3,1)$ & 2 & 1 & YES & YES & YES & $1.22$ & $(4,1)$ & -- & 3243\\
$(137,30)$ & 12 & $(3,1)$ & 2 & 1 & YES & YES & YES & $1.33$ & $(4,1)$ & NO & 3244\\
$(137,32)$ & 12 & $(3,1)$ & 2 & 1 & YES & YES & YES & $1.11$ & $(4,1)$ & -- & 3245\\
$(137,32)$ & 12 & $(3,1)$ & 2 & 1 & YES & YES & YES & $1.33$ & $(4,1)$ & NO & 3246\\
$(137,32)$ & 12 & $(3,1)$ & 2 & 1 & YES & YES & YES & $1.60$ & $(4,1)$ & NO & 3247\\
$(137,30)$ & 12 & $(4,1)$ & 3 & 1 & YES & YES & YES & $1.00$ & $(6,0)$ & -- & 3248\\
$(137,32)$ & 12 & $(4,1)$ & 3 & 1 & YES & YES & YES & $1.00$ & $(6,0)$ & -- & 3249\\
$(137,32)$ & 12 & $(5,1)$ & 4 & 1 & YES & YES & YES & $1.11$ & $(4,1)$ & NO & 3250\\
$(137,32)$ & 12 & $(6,1)$ & 5 & 1 & YES & YES & YES & $1.11$ & $(4,1)$ & NO & 3251\\
$(137,32)$ & 12 & $(9,2)$ & 5 & 1 & YES & YES & YES & $1.30$ & $(4,1)$ & NO & 3252\\
$(137,37)$ & 11 & $(11,3)$ & 5 & 1 & YES & YES & YES & $1.00$ & $(6,0)$ & NO & 3253\\
$(137,32)$ & 12 & $(13,3)$ & 6 & 1 & YES & YES & YES & $1.00$ & $(6,0)$ & NO & 3254\\
$(137,32)$ & 12 & $(17,4)$ & 7 & 1 & YES & YES & YES & $1.00$ & $(6,0)$ & NO & 3255\\
$(137,30)$ & 12 & $(23,5)$ & 7 & 1 & YES & YES & YES & $1.22$ & $(4,1)$ & NO & 3256\\
$(137,37)$ & 11 & $(37,10)$ & 8 & 1 & YES & YES & YES & $1.12$ & $(6,0)$ & NO & 3257\\
$(137,32)$ & 12 & $(107,25)$ & 11 & 1 & YES & YES & YES & $0.88$ & $(6,0)$ & NO & 3258\\
$(138,37)$ & 11 & $(2,1)$ & 1 & 2 & YES & YES & YES & $1.00$ & $(6,0)$ & NO & 3259\\
$(138,37)$ & 11 & $(3,1)$ & 2 & 3 & YES & YES & YES & $1.00$ & $(6,0)$ & NO & 3260\\
$(138,37)$ & 11 & $(41,11)$ & 8 & 1 & YES & YES & YES & $1.12$ & $(6,0)$ & NO & 3261\\
$(140,41)$ & 11 & $(2,1)$ & 1 & 2 & YES & YES & YES & $1.00$ & $(4,1)$ & -- & 3262\\
$(140,41)$ & 11 & $(3,1)$ & 2 & 1 & YES & YES & YES & $1.11$ & $(4,1)$ & -- & 3263\\
$(140,41)$ & 11 & $(3,1)$ & 2 & 1 & YES & YES & YES & $1.22$ & $(4,1)$ & NO & 3264\\
$(140,41)$ & 11 & $(3,1)$ & 2 & 1 & YES & YES & YES & $1.12$ & $(2,2)$ & NO & 3265\\
$(140,41)$ & 11 & $(5,1)$ & 4 & 5 & YES & YES & YES & $1.12$ & $(2,2)$ & NO & 3266\\
$(140,41)$ & 11 & $(7,2)$ & 4 & 7 & YES & YES & YES & $1.11$ & $(4,1)$ & NO & 3267\\
$(140,41)$ & 11 & $(17,5)$ & 6 & 1 & YES & YES & YES & $1.00$ & $(6,0)$ & NO & 3268\\
$(140,41)$ & 11 & $(24,7)$ & 7 & 4 & YES & YES & YES & $1.12$ & $(2,2)$ & NO & 3269\\
$(140,41)$ & 11 & $(41,12)$ & 8 & 1 & YES & YES & YES & $1.00$ & $(6,0)$ & NO & 3270\\
$(140,41)$ & 11 & $(58,17)$ & 9 & 2 & YES & YES & YES & $1.00$ & $(4,1)$ & 3345 & 3271\\
$(141,43)$ & 11 & $(3,1)$ & 2 & 3 & YES & YES & YES & $1.22$ & $(4,1)$ & -- & 3272\\
$(141,43)$ & 11 & $(5,1)$ & 4 & 1 & YES & YES & YES & $1.11$ & $(4,1)$ & NO & 3273\\
$(141,43)$ & 11 & $(23,7)$ & 7 & 1 & YES & YES & YES & $1.22$ & $(4,1)$ & 2911 & 3274\\
$(141,43)$ & 11 & $(141,43)$ & 11 & 141 & YES & YES & YES & $1.22$ & $(4,1)$ & NO & 3275\\
$(142,33)$ & 12 & $(3,1)$ & 2 & 1 & YES & YES & YES & $0.88$ & $(6,0)$ & -- & 3276\\
$(142,33)$ & 12 & $(3,1)$ & 2 & 1 & YES & YES & YES & $1.12$ & $(6,0)$ & NO & 3277\\
$(142,33)$ & 12 & $(17,4)$ & 7 & 1 & YES & YES & YES & $0.88$ & $(6,0)$ & NO & 3278\\
$(145,44)$ & 11 & $(2,1)$ & 1 & 1 & YES & YES & YES & $1.11$ & $(4,1)$ & -- & 3279\\
$(145,44)$ & 11 & $(10,3)$ & 5 & 5 & YES & YES & YES & $1.11$ & $(4,1)$ & 3099 & 3280\\
$(145,44)$ & 11 & $(23,7)$ & 7 & 1 & YES & YES & YES & $1.11$ & $(4,1)$ & NO & 3281\\
$(146,31)$ & 12 & $(2,1)$ & 1 & 2 & YES & YES & YES & $0.88$ & $(6,0)$ & -- & 3282\\
$(146,31)$ & 12 & $(2,1)$ & 1 & 2 & YES & YES & YES & $1.00$ & $(6,0)$ & NO & 3283\\
$(146,57)$ & 11 & $(2,1)$ & 1 & 2 & NO & YES & YES & $1.00$ & $(4,1)$ & -- & 3284\\
$(146,31)$ & 12 & $(3,1)$ & 2 & 1 & YES & YES & YES & $1.11$ & $(4,1)$ & NO & 3285\\
$(146,31)$ & 12 & $(3,1)$ & 2 & 1 & YES & YES & YES & $1.22$ & $(4,1)$ & -- & 3286\\
$(146,31)$ & 12 & $(4,1)$ & 3 & 2 & YES & YES & YES & $0.88$ & $(6,0)$ & NO & 3287\\
$(146,31)$ & 12 & $(6,1)$ & 5 & 2 & YES & YES & YES & $0.88$ & $(6,0)$ & NO & 3288\\
$(146,31)$ & 12 & $(9,2)$ & 5 & 1 & YES & YES & YES & $0.88$ & $(6,0)$ & NO & 3289\\
$(146,31)$ & 12 & $(19,4)$ & 7 & 1 & YES & YES & YES & $1.22$ & $(4,1)$ & NO & 3290\\
$(146,31)$ & 12 & $(146,31)$ & 12 & 146 & YES & YES & YES & $0.88$ & $(6,0)$ & NO & 3291\\
$(147,41)$ & 11 & $(2,1)$ & 1 & 1 & YES & YES & YES & $1.11$ & $(4,1)$ & NO & 3292\\
$(147,41)$ & 11 & $(3,1)$ & 2 & 3 & YES & YES & YES & $1.00$ & $(2,2)$ & -- & 3293\\
$(147,41)$ & 11 & $(3,1)$ & 2 & 3 & YES & YES & YES & $1.38$ & $(2,2)$ & NO & 3294\\
$(147,43)$ & 11 & $(3,1)$ & 2 & 3 & YES & YES & YES & $1.11$ & $(4,1)$ & -- & 3295\\
$(147,41)$ & 11 & $(4,1)$ & 3 & 1 & YES & YES & YES & $1.11$ & $(4,1)$ & NO & 3296\\
$(147,41)$ & 11 & $(5,1)$ & 4 & 1 & YES & YES & YES & $1.00$ & $(2,2)$ & 3379 & 3297\\
$(147,43)$ & 11 & $(7,2)$ & 4 & 7 & YES & YES & YES & $1.11$ & $(4,1)$ & NO & 3298\\
$(147,41)$ & 11 & $(11,3)$ & 5 & 1 & YES & YES & YES & $1.00$ & $(2,2)$ & 2907 & 3299\\
$(147,43)$ & 11 & $(17,5)$ & 6 & 1 & YES & YES & YES & $1.12$ & $(2,2)$ & NO & 3300\\
$(147,41)$ & 11 & $(18,5)$ & 6 & 3 & YES & YES & YES & $1.11$ & $(4,1)$ & NO & 3301\\
$(147,41)$ & 11 & $(25,7)$ & 7 & 1 & YES & YES & YES & $1.12$ & $(2,2)$ & NO & 3302\\
$(147,41)$ & 11 & $(43,12)$ & 8 & 1 & YES & YES & YES & $1.11$ & $(4,1)$ & NO & 3303\\
$(147,41)$ & 11 & $(61,17)$ & 9 & 1 & YES & YES & YES & $1.00$ & $(2,2)$ & 3364 & 3304\\
$(147,41)$ & 11 & $(147,41)$ & 11 & 147 & YES & YES & YES & $1.00$ & $(2,2)$ & NO & 3305\\
$(149,34)$ & 11 & $(2,1)$ & 1 & 1 & YES & YES & YES & $1.11$ & $(4,1)$ & NO & 3306\\
$(149,40)$ & 11 & $(2,1)$ & 1 & 1 & YES & YES & YES & $1.22$ & $(4,1)$ & -- & 3307\\
$(149,34)$ & 11 & $(3,1)$ & 2 & 1 & YES & YES & YES & $1.00$ & $(6,0)$ & NO & 3308\\
$(149,34)$ & 11 & $(3,1)$ & 2 & 1 & YES & YES & YES & $1.00$ & $(6,0)$ & -- & 3309\\
$(149,34)$ & 11 & $(3,1)$ & 2 & 1 & YES & YES & YES & $1.55$ & $(2,2)$ & NO & 3310\\
$(149,40)$ & 11 & $(3,1)$ & 2 & 1 & YES & YES & YES & $1.11$ & $(4,1)$ & NO & 3311\\
$(149,44)$ & 11 & $(3,1)$ & 2 & 1 & YES & YES & YES & $1.12$ & $(2,2)$ & -- & 3312\\
$(149,40)$ & 11 & $(6,1)$ & 5 & 1 & YES & YES & YES & $1.11$ & $(4,1)$ & NO & 3313\\
$(149,34)$ & 11 & $(9,2)$ & 5 & 1 & YES & YES & YES & $1.11$ & $(4,1)$ & NO & 3314\\
$(149,40)$ & 11 & $(11,3)$ & 5 & 1 & YES & YES & YES & $1.22$ & $(4,1)$ & NO & 3315\\
$(149,34)$ & 11 & $(13,3)$ & 6 & 1 & YES & YES & YES & $1.22$ & $(4,1)$ & 3202 & 3316\\
$(149,34)$ & 11 & $(22,5)$ & 7 & 1 & YES & YES & YES & $1.11$ & $(4,1)$ & NO & 3317\\
$(149,40)$ & 11 & $(41,11)$ & 8 & 1 & YES & YES & YES & $1.33$ & $(4,1)$ & NO & 3318\\
$(151,27)$ & 13 & $(2,1)$ & 1 & 1 & YES & YES & YES & $1.22$ & $(2,2)$ & NO & 3319\\
$(151,32)$ & 12 & $(2,1)$ & 1 & 1 & YES & YES & YES & $1.22$ & $(4,1)$ & -- & 3320\\
$(151,32)$ & 12 & $(2,1)$ & 1 & 1 & YES & YES & YES & $1.33$ & $(4,1)$ & NO & 3321\\
$(151,32)$ & 12 & $(3,1)$ & 2 & 1 & YES & YES & YES & $1.22$ & $(4,1)$ & -- & 3322\\
$(151,32)$ & 12 & $(3,1)$ & 2 & 1 & YES & YES & YES & $1.22$ & $(4,1)$ & NO & 3323\\
$(151,32)$ & 12 & $(3,1)$ & 2 & 1 & YES & YES & YES & $1.44$ & $(4,1)$ & NO & 3324\\
$(151,32)$ & 12 & $(4,1)$ & 3 & 1 & YES & YES & YES & $1.22$ & $(4,1)$ & NO & 3325\\
$(151,32)$ & 12 & $(4,1)$ & 3 & 1 & YES & YES & YES & $1.00$ & $(6,0)$ & -- & 3326\\
$(151,32)$ & 12 & $(6,1)$ & 5 & 1 & YES & YES & YES & $1.22$ & $(4,1)$ & NO & 3327\\
$(151,32)$ & 12 & $(9,2)$ & 5 & 1 & YES & YES & YES & $1.22$ & $(4,1)$ & NO & 3328\\
$(151,32)$ & 12 & $(14,3)$ & 6 & 1 & YES & YES & YES & $1.00$ & $(6,0)$ & NO & 3329\\
$(151,32)$ & 12 & $(19,4)$ & 7 & 1 & YES & YES & YES & $1.00$ & $(6,0)$ & NO & 3330\\
$(152,41)$ & 11 & $(4,1)$ & 3 & 4 & YES & YES & YES & $1.00$ & $(6,0)$ & -- & 3331\\
$(152,41)$ & 11 & $(11,3)$ & 5 & 1 & YES & YES & YES & $1.00$ & $(6,0)$ & 2647 & 3332\\
$(153,35)$ & 12 & $(2,1)$ & 1 & 1 & YES & YES & YES & $1.00$ & $(6,0)$ & NO & 3333\\
$(153,35)$ & 12 & $(4,1)$ & 3 & 1 & YES & YES & YES & $0.88$ & $(6,0)$ & NO & 3334\\
$(153,35)$ & 12 & $(13,3)$ & 6 & 1 & YES & YES & YES & $1.00$ & $(6,0)$ & NO & 3335\\
$(154,29)$ & 13 & $(3,1)$ & 2 & 1 & YES & YES & NO(2) & $1.00$ & $(6,0)$ & -- & 3336\\
$(154,29)$ & 13 & $(3,1)$ & 2 & 1 & YES & YES & NO(2) & $1.11$ & $(6,0)$ & NO & 3337\\
$(154,45)$ & 11 & $(3,1)$ & 2 & 1 & YES & YES & YES & $1.12$ & $(2,2)$ & -- & 3338\\
$(154,43)$ & 11 & $(18,5)$ & 6 & 2 & YES & YES & YES & $1.12$ & $(2,2)$ & NO & 3339\\
$(154,43)$ & 11 & $(25,7)$ & 7 & 1 & YES & YES & YES & $1.25$ & $(2,2)$ & NO & 3340\\
$(155,36)$ & 12 & $(3,1)$ & 2 & 1 & YES & YES & YES & $0.88$ & $(6,0)$ & -- & 3341\\
$(155,36)$ & 12 & $(17,4)$ & 7 & 1 & YES & YES & YES & $0.88$ & $(6,0)$ & NO & 3342\\
$(157,46)$ & 11 & $(2,1)$ & 1 & 1 & YES & YES & YES & $1.11$ & $(4,1)$ & -- & 3343\\
$(157,46)$ & 11 & $(3,1)$ & 2 & 1 & YES & YES & YES & $1.11$ & $(4,1)$ & NO & 3344\\
$(157,46)$ & 11 & $(41,12)$ & 8 & 1 & YES & YES & YES & $1.00$ & $(4,1)$ & 3271 & 3345\\
$(159,44)$ & 11 & $(2,1)$ & 1 & 1 & YES & YES & YES & $1.12$ & $(2,2)$ & NO & 3346\\
$(159,44)$ & 11 & $(2,1)$ & 1 & 1 & YES & YES & YES & $1.25$ & $(2,2)$ & -- & 3347\\
$(159,44)$ & 11 & $(7,2)$ & 4 & 1 & YES & YES & YES & $1.25$ & $(2,2)$ & 2697 & 3348\\
$(159,44)$ & 11 & $(11,3)$ & 5 & 1 & YES & YES & YES & $1.25$ & $(2,2)$ & NO & 3349\\
$(162,35)$ & 12 & $(2,1)$ & 1 & 2 & YES & YES & YES & $1.00$ & $(6,0)$ & NO & 3350\\
$(162,37)$ & 12 & $(2,1)$ & 1 & 2 & YES & YES & YES & $1.30$ & $(4,1)$ & -- & 3351\\
$(162,37)$ & 12 & $(4,1)$ & 3 & 2 & YES & YES & YES & $1.22$ & $(4,1)$ & NO & 3352\\
$(162,35)$ & 12 & $(5,1)$ & 4 & 1 & YES & YES & YES & $1.00$ & $(6,0)$ & NO & 3353\\
$(162,37)$ & 12 & $(5,1)$ & 4 & 1 & YES & YES & YES & $1.11$ & $(4,1)$ & NO & 3354\\
$(162,35)$ & 12 & $(14,3)$ & 6 & 2 & YES & YES & YES & $1.00$ & $(6,0)$ & NO & 3355\\
$(162,37)$ & 12 & $(22,5)$ & 7 & 2 & YES & YES & YES & $1.11$ & $(4,1)$ & NO & 3356\\
$(163,44)$ & 11 & $(2,1)$ & 1 & 1 & YES & YES & YES & $1.33$ & $(4,1)$ & -- & 3357\\
$(163,44)$ & 11 & $(11,3)$ & 5 & 1 & YES & YES & YES & $1.22$ & $(4,1)$ & 3223 & 3358\\
$(163,44)$ & 11 & $(26,7)$ & 7 & 1 & YES & YES & YES & $1.11$ & $(4,1)$ & NO & 3359\\
$(165,46)$ & 11 & $(2,1)$ & 1 & 1 & YES & YES & YES & $1.11$ & $(4,1)$ & NO & 3360\\
$(165,46)$ & 11 & $(2,1)$ & 1 & 1 & YES & YES & YES & $1.12$ & $(2,2)$ & -- & 3361\\
$(165,46)$ & 11 & $(4,1)$ & 3 & 1 & YES & YES & YES & $1.11$ & $(4,1)$ & NO & 3362\\
$(165,46)$ & 11 & $(18,5)$ & 6 & 3 & YES & YES & YES & $1.11$ & $(4,1)$ & NO & 3363\\
$(165,46)$ & 11 & $(43,12)$ & 8 & 1 & YES & YES & YES & $1.00$ & $(2,2)$ & 3304 & 3364\\
$(165,46)$ & 11 & $(165,46)$ & 11 & 165 & YES & YES & YES & $1.11$ & $(4,1)$ & NO & 3365\\
$(169,32)$ & 13 & $(4,1)$ & 3 & 1 & YES & YES & YES & $1.22$ & $(4,1)$ & NO & 3366\\
$(171,37)$ & 12 & $(2,1)$ & 1 & 1 & YES & YES & YES & $1.22$ & $(4,1)$ & NO & 3367\\
$(171,37)$ & 12 & $(4,1)$ & 3 & 1 & YES & YES & YES & $1.11$ & $(4,1)$ & NO & 3368\\
$(171,37)$ & 12 & $(5,1)$ & 4 & 1 & YES & YES & YES & $1.22$ & $(4,1)$ & NO & 3369\\
$(171,40)$ & 12 & $(47,11)$ & 9 & 1 & YES & YES & YES & $1.22$ & $(4,1)$ & NO & 3370\\
$(171,40)$ & 12 & $(77,18)$ & 10 & 1 & YES & YES & YES & $1.11$ & $(4,1)$ & 3398 & 3371\\
$(171,37)$ & 12 & $(134,29)$ & 11 & 1 & YES & YES & YES & $1.11$ & $(4,1)$ & NO & 3372\\
$(176,41)$ & 12 & $(3,1)$ & 2 & 1 & YES & YES & YES & $1.11$ & $(4,1)$ & NO & 3373\\
$(176,41)$ & 12 & $(3,1)$ & 2 & 1 & YES & YES & YES & $1.11$ & $(4,1)$ & -- & 3374\\
$(176,41)$ & 12 & $(17,4)$ & 7 & 1 & YES & YES & YES & $1.11$ & $(4,1)$ & NO & 3375\\
$(179,75)$ & 11 & $(2,1)$ & 1 & 1 & NO & YES & YES & $1.12$ & $(6,0)$ & -- & 3376\\
$(181,41)$ & 12 & $(2,1)$ & 1 & 1 & YES & YES & YES & $0.88$ & $(6,0)$ & NO & 3377\\
$(181,41)$ & 12 & $(3,1)$ & 2 & 1 & YES & YES & YES & $1.00$ & $(2,2)$ & -- & 3378\\
$(181,41)$ & 12 & $(3,1)$ & 2 & 1 & YES & YES & YES & $1.00$ & $(2,2)$ & 3297 & 3379\\
$(181,41)$ & 12 & $(3,1)$ & 2 & 1 & YES & YES & YES & $1.38$ & $(2,2)$ & NO & 3380\\
$(181,41)$ & 12 & $(22,5)$ & 7 & 1 & YES & YES & YES & $0.75$ & $(6,0)$ & NO & 3381\\
$(181,41)$ & 12 & $(31,7)$ & 8 & 1 & YES & YES & YES & $1.12$ & $(2,2)$ & NO & 3382\\
$(184,43)$ & 12 & $(2,1)$ & 1 & 2 & YES & YES & YES & $1.11$ & $(4,1)$ & -- & 3383\\
$(184,43)$ & 12 & $(2,1)$ & 1 & 2 & YES & YES & YES & $1.22$ & $(4,1)$ & NO & 3384\\
$(184,43)$ & 12 & $(3,1)$ & 2 & 1 & YES & YES & YES & $1.11$ & $(4,1)$ & -- & 3385\\
$(184,43)$ & 12 & $(3,1)$ & 2 & 1 & YES & YES & YES & $1.33$ & $(4,1)$ & NO & 3386\\
$(184,43)$ & 12 & $(9,2)$ & 5 & 1 & YES & YES & YES & $1.11$ & $(4,1)$ & NO & 3387\\
$(190,43)$ & 12 & $(2,1)$ & 1 & 2 & YES & YES & YES & $1.25$ & $(2,2)$ & -- & 3388\\
$(190,43)$ & 12 & $(2,1)$ & 1 & 2 & YES & YES & YES & $1.25$ & $(2,2)$ & NO & 3389\\
$(190,43)$ & 12 & $(22,5)$ & 7 & 2 & YES & YES & YES & $1.25$ & $(2,2)$ & NO & 3390\\
$(193,44)$ & 12 & $(2,1)$ & 1 & 1 & YES & YES & YES & $1.12$ & $(2,2)$ & NO & 3391\\
$(193,44)$ & 12 & $(2,1)$ & 1 & 1 & YES & YES & YES & $1.25$ & $(2,2)$ & -- & 3392\\
$(196,37)$ & 13 & $(3,1)$ & 2 & 1 & YES & YES & YES & $0.88$ & $(6,0)$ & NO & 3393\\
$(196,37)$ & 13 & $(4,1)$ & 3 & 4 & YES & YES & YES & $0.88$ & $(6,0)$ & NO & 3394\\
$(197,43)$ & 12 & $(23,5)$ & 7 & 1 & YES & YES & YES & $1.12$ & $(2,2)$ & NO & 3395\\
$(201,46)$ & 12 & $(2,1)$ & 1 & 1 & YES & YES & YES & $1.00$ & $(6,0)$ & NO & 3396\\
$(201,46)$ & 12 & $(4,1)$ & 3 & 1 & YES & YES & YES & $0.88$ & $(6,0)$ & NO & 3397\\
$(201,47)$ & 12 & $(47,11)$ & 9 & 1 & YES & YES & YES & $1.11$ & $(4,1)$ & 3371 & 3398\\
$(201,47)$ & 12 & $(77,18)$ & 10 & 1 & YES & YES & YES & $1.11$ & $(4,1)$ & NO & 3399\\
$(211,46)$ & 12 & $(2,1)$ & 1 & 1 & YES & YES & YES & $1.00$ & $(6,0)$ & NO & 3400\\
$(211,40)$ & 13 & $(4,1)$ & 3 & 1 & YES & YES & YES & $1.22$ & $(4,1)$ & NO & 3401\\
$(219,50)$ & 12 & $(2,1)$ & 1 & 1 & YES & YES & YES & $1.11$ & $(4,1)$ & -- & 3402\\
$(219,50)$ & 12 & $(4,1)$ & 3 & 1 & YES & YES & YES & $1.11$ & $(4,1)$ & NO & 3403\\
$(219,50)$ & 12 & $(22,5)$ & 7 & 1 & YES & YES & YES & $1.11$ & $(4,1)$ & 3170 & 3404\\
$(a;0,0,0;3)$ & 4 & $(31,13)$ & 7 & 1 & YES & YES & NO(2) & $1.20$ & $(4,1)$ & -- & 3405\\
$(a;1,0,0;13)$ & 5 & $(11,3)$ & 5 & 1 & YES & YES & NO(2) & $1.30$ & $(4,1)$ & -- & 3406\\
$(a;1,0,0;13)$ & 5 & $(12,5)$ & 5 & 1 & YES & YES & YES & $1.00$ & $(2,2)$ & -- & 3407\\
$(a;1,0,0;13)$ & 5 & $(13,5)$ & 5 & 13 & YES & YES & YES & $1.00$ & $(2,2)$ & -- & 3408\\
$(a;1,0,0;13)$ & 5 & $(16,7)$ & 6 & 1 & YES & YES & NO(2) & $1.11$ & $(6,0)$ & -- & 3409\\
$(a;1,0,0;13)$ & 5 & $(17,7)$ & 6 & 1 & YES & YES & NO(2) & $1.30$ & $(4,1)$ & -- & 3410\\
$(a;1,0,0;13)$ & 5 & $(19,8)$ & 6 & 1 & YES & YES & YES & $1.00$ & $(2,2)$ & -- & 3411\\
$(a;1,1,0;19)$ & 6 & $(7,2)$ & 4 & 1 & YES & YES & YES & $0.88$ & $(4,1)$ & -- & 3412\\
$(a;1,1,0;19)$ & 6 & $(10,3)$ & 5 & 1 & YES & YES & YES & $1.12$ & $(4,1)$ & -- & 3413\\
$(a;1,1,0;19)$ & 6 & $(12,5)$ & 5 & 1 & YES & YES & NO(2) & $1.30$ & $(4,1)$ & -- & 3414\\
$(a;1,1,0;19)$ & 6 & $(17,5)$ & 6 & 1 & YES & YES & NO(2) & $1.20$ & $(4,1)$ & -- & 3415\\
$(a;1,1,1;4)$ & 7 & $(4,1)$ & 3 & 4 & YES & YES & YES & $1.33$ & $(2,2)$ & -- & 3416\\
$(a;2,0,0;17)$ & 6 & $(11,4)$ & 5 & 1 & YES & YES & YES & $1.00$ & $(4,1)$ & -- & 3417\\
$(a;2,0,0;17)$ & 6 & $(12,5)$ & 5 & 1 & YES & YES & NO(2) & $1.40$ & $(4,1)$ & -- & 3418\\
$(a;2,0,0;17)$ & 6 & $(13,5)$ & 5 & 1 & YES & YES & NO(2) & $1.30$ & $(4,1)$ & -- & 3419\\
$(a;2,0,0;17)$ & 6 & $(18,5)$ & 6 & 1 & YES & YES & NO(2) & $1.20$ & $(4,1)$ & -- & 3420\\
$(a;2,0,1;25)$ & 7 & $(5,2)$ & 3 & 5 & YES & YES & YES & $1.12$ & $(2,2)$ & -- & 3421\\
$(a;2,0,1;25)$ & 7 & $(7,2)$ & 4 & 1 & YES & YES & YES & $1.12$ & $(2,2)$ & -- & 3422\\
$(a;2,0,1;25)$ & 7 & $(8,3)$ & 4 & 1 & YES & YES & NO(2) & $1.11$ & $(6,0)$ & -- & 3423\\
$(a;2,1,0;5)$ & 7 & $(5,2)$ & 3 & 5 & YES & YES & YES & $1.00$ & $(2,2)$ & -- & 3424\\
$(a;2,1,0;5)$ & 7 & $(7,3)$ & 4 & 1 & YES & YES & NO(2) & $1.30$ & $(4,1)$ & -- & 3425\\
$(a;2,1,0;5)$ & 7 & $(8,3)$ & 4 & 1 & YES & YES & NO(2) & $0.89$ & $(6,0)$ & -- & 3426\\
$(a;2,1,0;5)$ & 7 & $(10,3)$ & 5 & 5 & YES & YES & NO(2) & $1.10$ & $(4,1)$ & -- & 3427\\
$(a;2,1,0;5)$ & 7 & $(17,5)$ & 6 & 1 & YES & YES & YES & $1.33$ & $(2,2)$ & -- & 3428\\
$(a;2,1,1;37)$ & 8 & $(3,1)$ & 2 & 1 & YES & YES & YES & $1.12$ & $(2,2)$ & -- & 3429\\
$(a;2,1,1;37)$ & 8 & $(5,2)$ & 3 & 1 & YES & YES & YES & $1.11$ & $(4,1)$ & -- & 3430\\
$(a;2,2,0;33)$ & 8 & $(7,2)$ & 4 & 1 & YES & YES & NO(2) & $1.11$ & $(6,0)$ & -- & 3431\\
$(a;2,2,1;49)$ & 9 & $(3,1)$ & 2 & 1 & YES & YES & YES & $1.00$ & $(4,1)$ & -- & 3432\\
$(a;3,0,0;7)$ & 7 & $(8,3)$ & 4 & 1 & YES & YES & NO(2) & $1.11$ & $(6,0)$ & -- & 3433\\
$(a;3,0,0;7)$ & 7 & $(12,5)$ & 5 & 1 & YES & YES & NO(2) & $1.22$ & $(6,0)$ & -- & 3434\\
$(a;3,0,0;7)$ & 7 & $(13,3)$ & 6 & 1 & YES & YES & NO(2) & $1.00$ & $(6,0)$ & -- & 3435\\
$(a;3,0,0;7)$ & 7 & $(14,3)$ & 6 & 7 & YES & YES & NO(2) & $1.11$ & $(6,0)$ & -- & 3436\\
$(a;3,0,0;7)$ & 7 & $(21,4)$ & 8 & 7 & YES & YES & NO(2) & $1.11$ & $(6,0)$ & -- & 3437\\
$(a;3,0,1;31)$ & 8 & $(5,2)$ & 3 & 1 & YES & YES & NO(2) & $1.11$ & $(6,0)$ & -- & 3438\\
$(a;3,0,2;41)$ & 9 & $(3,1)$ & 2 & 1 & YES & YES & NO(2) & $1.00$ & $(6,0)$ & -- & 3439\\
$(a;3,0,2;41)$ & 9 & $(4,1)$ & 3 & 1 & YES & YES & NO(2) & $1.11$ & $(6,0)$ & -- & 3440\\
$(a;3,0,2;41)$ & 9 & $(11,2)$ & 6 & 1 & YES & YES & NO(2) & $1.11$ & $(6,0)$ & -- & 3441\\
$(a;3,2,0;41)$ & 9 & $(2,1)$ & 1 & 1 & YES & YES & NO(2) & $1.00$ & $(6,0)$ & -- & 3442\\
$(a;3,2,0;41)$ & 9 & $(3,1)$ & 2 & 1 & YES & YES & NO(2) & $1.20$ & $(4,1)$ & -- & 3443\\
$(a;3,2,0;41)$ & 9 & $(4,1)$ & 3 & 1 & YES & YES & NO(2) & $1.00$ & $(6,0)$ & -- & 3444\\
$(a;4,0,1;37)$ & 9 & $(3,1)$ & 2 & 1 & YES & YES & NO(2) & $1.20$ & $(4,1)$ & -- & 3445\\
$(a;4,0,1;37)$ & 9 & $(5,1)$ & 4 & 1 & YES & YES & NO(2) & $1.20$ & $(4,1)$ & -- & 3446\\
$(b;0,0,0;14)$ & 5 & $(10,3)$ & 5 & 2 & YES & YES & YES & $1.22$ & $(2,2)$ & -- & 3447\\
$(b;0,0,0;14)$ & 5 & $(12,5)$ & 5 & 2 & YES & YES & YES & $1.00$ & $(2,2)$ & -- & 3448\\
$(b;0,0,0;14)$ & 5 & $(13,5)$ & 5 & 1 & YES & YES & YES & $1.33$ & $(2,2)$ & -- & 3449\\
$(b;0,0,0;14)$ & 5 & $(16,7)$ & 6 & 2 & YES & YES & YES & $1.33$ & $(2,2)$ & -- & 3450\\
$(b;0,0,0;14)$ & 5 & $(19,8)$ & 6 & 1 & YES & YES & YES & $1.00$ & $(2,2)$ & -- & 3451\\
$(b;0,0,0;14)$ & 5 & $(23,7)$ & 7 & 1 & YES & YES & YES & $1.33$ & $(2,2)$ & -- & 3452\\
$(b;0,0,1;4)$ & 6 & $(8,3)$ & 4 & 4 & YES & YES & YES & $1.00$ & $(4,1)$ & -- & 3453\\
$(b;0,0,1;4)$ & 6 & $(9,4)$ & 5 & 1 & YES & YES & NO(2) & $1.11$ & $(6,0)$ & -- & 3454\\
$(b;0,0,1;4)$ & 6 & $(11,4)$ & 5 & 1 & YES & YES & NO(2) & $1.11$ & $(6,0)$ & -- & 3455\\
$(b;0,0,1;4)$ & 6 & $(12,5)$ & 5 & 4 & YES & YES & YES & $1.00$ & $(2,2)$ & -- & 3456\\
$(b;0,0,2;26)$ & 7 & $(4,1)$ & 3 & 2 & YES & YES & NO(2) & $1.30$ & $(4,1)$ & -- & 3457\\
$(b;0,0,2;26)$ & 7 & $(5,2)$ & 3 & 1 & YES & YES & YES & $1.00$ & $(2,2)$ & -- & 3458\\
$(b;0,0,2;26)$ & 7 & $(7,2)$ & 4 & 1 & YES & YES & YES & $0.88$ & $(2,2)$ & -- & 3459\\
$(b;0,0,3;32)$ & 8 & $(2,1)$ & 1 & 2 & YES & YES & YES & $1.11$ & $(2,2)$ & -- & 3460\\
$(b;0,0,3;32)$ & 8 & $(3,1)$ & 2 & 1 & YES & YES & YES & $1.22$ & $(2,2)$ & -- & 3461\\
$(b;0,0,3;32)$ & 8 & $(5,1)$ & 4 & 1 & YES & YES & YES & $0.88$ & $(4,1)$ & -- & 3462\\
$(b;0,0,3;32)$ & 8 & $(5,2)$ & 3 & 1 & YES & YES & YES & $1.33$ & $(2,2)$ & -- & 3463\\
$(b;0,0,3;32)$ & 8 & $(11,2)$ & 6 & 1 & YES & YES & YES & $1.11$ & $(2,2)$ & -- & 3464\\
$(b;0,0,3;32)$ & 8 & $(16,3)$ & 7 & 16 & YES & YES & YES & $1.44$ & $(2,2)$ & -- & 3465\\
$(b;0,1,0;19)$ & 6 & $(7,2)$ & 4 & 1 & YES & YES & YES & $1.33$ & $(2,2)$ & -- & 3466\\
$(b;0,1,0;19)$ & 6 & $(7,3)$ & 4 & 1 & YES & YES & YES & $1.00$ & $(4,1)$ & -- & 3467\\
$(b;0,1,1;27)$ & 7 & $(7,3)$ & 4 & 1 & YES & YES & YES & $1.33$ & $(2,2)$ & -- & 3468\\
$(b;0,1,2;5)$ & 8 & $(3,1)$ & 2 & 1 & YES & YES & YES & $1.22$ & $(2,2)$ & -- & 3469\\
$(b;0,1,2;5)$ & 8 & $(5,2)$ & 3 & 5 & YES & YES & YES & $1.22$ & $(4,1)$ & -- & 3470\\
$(b;0,1,3;43)$ & 9 & $(2,1)$ & 1 & 1 & YES & YES & YES & $1.22$ & $(2,2)$ & -- & 3471\\
$(b;0,2,0;8)$ & 7 & $(4,1)$ & 3 & 4 & YES & YES & YES & $1.22$ & $(2,2)$ & -- & 3472\\
$(b;0,2,0;8)$ & 7 & $(7,2)$ & 4 & 1 & YES & YES & YES & $1.00$ & $(2,2)$ & -- & 3473\\
$(b;0,2,1;34)$ & 8 & $(2,1)$ & 1 & 2 & YES & YES & YES & $1.00$ & $(2,2)$ & -- & 3474\\
$(b;0,2,1;34)$ & 8 & $(3,1)$ & 2 & 1 & YES & YES & YES & $1.22$ & $(2,2)$ & -- & 3475\\
$(b;0,2,1;34)$ & 8 & $(4,1)$ & 3 & 2 & YES & YES & YES & $1.22$ & $(2,2)$ & -- & 3476\\
$(b;0,2,1;34)$ & 8 & $(5,2)$ & 3 & 1 & YES & YES & YES & $1.22$ & $(4,1)$ & -- & 3477\\
$(b;0,2,1;34)$ & 8 & $(9,2)$ & 5 & 1 & YES & YES & YES & $1.22$ & $(4,1)$ & -- & 3478\\
$(b;0,2,2;44)$ & 9 & $(3,1)$ & 2 & 1 & YES & YES & YES & $1.22$ & $(4,1)$ & -- & 3479\\
$(b;0,2,2;44)$ & 9 & $(4,1)$ & 3 & 4 & YES & YES & YES & $1.22$ & $(4,1)$ & -- & 3480\\
$(b;0,3,0;29)$ & 8 & $(3,1)$ & 2 & 1 & YES & YES & YES & $1.11$ & $(2,2)$ & -- & 3481\\
$(b;0,3,0;29)$ & 8 & $(4,1)$ & 3 & 1 & YES & YES & NO(2) & $1.11$ & $(6,0)$ & -- & 3482\\
$(b;0,3,0;29)$ & 8 & $(5,1)$ & 4 & 1 & YES & YES & YES & $1.11$ & $(2,2)$ & -- & 3483\\
$(b;0,3,0;29)$ & 8 & $(11,2)$ & 6 & 1 & YES & YES & YES & $1.33$ & $(2,2)$ & -- & 3484\\
$(b;1,0,0;5)$ & 6 & $(7,3)$ & 4 & 1 & YES & YES & YES & $1.00$ & $(2,2)$ & -- & 3485\\
$(b;1,0,0;5)$ & 6 & $(8,3)$ & 4 & 1 & YES & YES & YES & $1.12$ & $(2,2)$ & -- & 3486\\
$(b;1,0,0;5)$ & 6 & $(17,5)$ & 6 & 1 & YES & YES & YES & $1.12$ & $(2,2)$ & -- & 3487\\
$(b;1,0,1;29)$ & 7 & $(2,1)$ & 1 & 1 & YES & YES & YES & $1.00$ & $(2,2)$ & -- & 3488\\
$(b;1,0,1;29)$ & 7 & $(3,1)$ & 2 & 1 & YES & YES & YES & $1.00$ & $(2,2)$ & -- & 3489\\
$(b;1,0,1;29)$ & 7 & $(4,1)$ & 3 & 1 & YES & YES & YES & $1.22$ & $(2,2)$ & -- & 3490\\
$(b;1,0,1;29)$ & 7 & $(5,2)$ & 3 & 1 & YES & YES & YES & $0.88$ & $(2,2)$ & -- & 3491\\
$(b;1,0,1;29)$ & 7 & $(10,3)$ & 5 & 1 & YES & YES & YES & $1.11$ & $(4,1)$ & -- & 3492\\
$(b;1,0,2;19)$ & 8 & $(2,1)$ & 1 & 1 & YES & YES & YES & $1.00$ & $(2,2)$ & -- & 3493\\
$(b;1,0,2;19)$ & 8 & $(3,1)$ & 2 & 1 & YES & YES & YES & $0.88$ & $(2,2)$ & -- & 3494\\
$(b;1,0,2;19)$ & 8 & $(4,1)$ & 3 & 1 & YES & YES & YES & $1.00$ & $(2,2)$ & -- & 3495\\
$(b;1,0,2;19)$ & 8 & $(5,2)$ & 3 & 1 & YES & YES & YES & $1.11$ & $(4,1)$ & -- & 3496\\
$(b;1,0,2;19)$ & 8 & $(7,2)$ & 4 & 1 & YES & YES & YES & $1.40$ & $(4,1)$ & -- & 3497\\
$(b;1,0,2;19)$ & 8 & $(10,3)$ & 5 & 1 & YES & YES & YES & $1.11$ & $(4,1)$ & -- & 3498\\
$(b;1,1,0;27)$ & 7 & $(4,1)$ & 3 & 1 & YES & YES & YES & $1.22$ & $(2,2)$ & -- & 3499\\
$(b;1,1,0;27)$ & 7 & $(5,2)$ & 3 & 1 & YES & YES & YES & $1.12$ & $(2,2)$ & -- & 3500\\
$(b;1,1,0;27)$ & 7 & $(7,2)$ & 4 & 1 & YES & YES & YES & $1.11$ & $(2,2)$ & -- & 3501\\
$(b;1,1,0;27)$ & 7 & $(10,3)$ & 5 & 1 & YES & YES & NO(2) & $1.11$ & $(4,1)$ & -- & 3502\\
$(b;1,1,1;39)$ & 8 & $(2,1)$ & 1 & 1 & YES & YES & YES & $1.00$ & $(2,2)$ & -- & 3503\\
$(b;1,1,1;39)$ & 8 & $(3,1)$ & 2 & 3 & YES & YES & YES & $0.75$ & $(4,1)$ & -- & 3504\\
$(b;1,1,1;39)$ & 8 & $(5,2)$ & 3 & 1 & YES & YES & YES & $1.00$ & $(2,2)$ & -- & 3505\\
$(b;1,1,1;39)$ & 8 & $(7,2)$ & 4 & 1 & YES & YES & YES & $1.11$ & $(4,1)$ & -- & 3506\\
$(b;1,1,2;51)$ & 9 & $(2,1)$ & 1 & 1 & YES & YES & YES & $0.88$ & $(6,0)$ & -- & 3507\\
$(b;1,1,2;51)$ & 9 & $(4,1)$ & 3 & 1 & YES & YES & YES & $0.75$ & $(6,0)$ & -- & 3508\\
$(b;1,2,0;17)$ & 8 & $(4,1)$ & 3 & 1 & YES & YES & YES & $1.00$ & $(2,2)$ & -- & 3509\\
$(b;2,0,0;26)$ & 7 & $(5,2)$ & 3 & 1 & YES & YES & NO(2) & $1.20$ & $(4,1)$ & -- & 3510\\
$(b;2,0,1;38)$ & 8 & $(2,1)$ & 1 & 2 & YES & YES & YES & $1.22$ & $(2,2)$ & -- & 3511\\
$(b;2,0,1;38)$ & 8 & $(3,1)$ & 2 & 1 & YES & YES & NO(2) & $1.10$ & $(4,1)$ & -- & 3512\\
$(b;2,0,1;38)$ & 8 & $(4,1)$ & 3 & 2 & YES & YES & YES & $1.22$ & $(2,2)$ & -- & 3513\\
$(b;2,0,1;38)$ & 8 & $(5,2)$ & 3 & 1 & YES & YES & YES & $1.22$ & $(4,1)$ & -- & 3514\\
$(b;2,0,2;50)$ & 9 & $(2,1)$ & 1 & 2 & YES & YES & YES & $1.11$ & $(4,1)$ & -- & 3515\\
$(b;2,0,2;50)$ & 9 & $(3,1)$ & 2 & 1 & YES & YES & YES & $1.00$ & $(4,1)$ & -- & 3516\\
$(b;2,0,2;50)$ & 9 & $(4,1)$ & 3 & 2 & YES & YES & YES & $1.00$ & $(4,1)$ & -- & 3517\\
$(b;2,0,2;50)$ & 9 & $(9,2)$ & 5 & 1 & YES & YES & YES & $1.22$ & $(4,1)$ & -- & 3518\\
$(b;2,1,0;7)$ & 8 & $(3,1)$ & 2 & 1 & YES & YES & NO(2) & $1.20$ & $(4,1)$ & -- & 3519\\
$(b;3,0,1;47)$ & 9 & $(6,1)$ & 5 & 1 & YES & YES & NO(2) & $1.00$ & $(6,0)$ & -- & 3520\\
$(c;0,0,0;4)$ & 4 & $(17,7)$ & 6 & 1 & YES & YES & YES & $1.22$ & $(2,2)$ & -- & 3521\\
$(c;0,0,0;4)$ & 4 & $(18,7)$ & 6 & 2 & YES & YES & YES & $1.22$ & $(2,2)$ & -- & 3522\\
$(c;0,0,0;4)$ & 4 & $(19,8)$ & 6 & 1 & YES & YES & YES & $1.11$ & $(2,2)$ & -- & 3523\\
$(c;0,0,0;4)$ & 4 & $(21,8)$ & 6 & 1 & YES & YES & YES & $1.11$ & $(2,2)$ & -- & 3524\\
$(c;0,0,0;4)$ & 4 & $(26,11)$ & 7 & 2 & YES & YES & NO(2) & $1.11$ & $(6,0)$ & -- & 3525\\
$(c;0,0,0;4)$ & 4 & $(29,11)$ & 7 & 1 & YES & YES & YES & $1.00$ & $(4,1)$ & -- & 3526\\
$(c;0,0,0;4)$ & 4 & $(29,12)$ & 7 & 1 & YES & YES & YES & $1.12$ & $(2,2)$ & -- & 3527\\
$(c;0,0,0;4)$ & 4 & $(30,11)$ & 7 & 2 & YES & YES & NO(2) & $1.30$ & $(4,1)$ & -- & 3528\\
$(c;0,0,0;4)$ & 4 & $(31,12)$ & 7 & 1 & YES & YES & YES & $0.88$ & $(4,1)$ & -- & 3529\\
$(c;0,0,0;4)$ & 4 & $(34,13)$ & 7 & 2 & YES & YES & YES & $0.75$ & $(4,1)$ & -- & 3530\\
$(c;0,0,0;4)$ & 4 & $(37,11)$ & 8 & 1 & YES & YES & YES & $1.55$ & $(2,2)$ & -- & 3531\\
$(c;0,0,0;4)$ & 4 & $(39,16)$ & 8 & 1 & YES & YES & YES & $1.33$ & $(4,1)$ & -- & 3532\\
$(c;0,0,0;4)$ & 4 & $(40,11)$ & 8 & 4 & YES & YES & YES & $0.88$ & $(4,1)$ & -- & 3533\\
$(c;0,0,0;4)$ & 4 & $(43,12)$ & 8 & 1 & YES & YES & YES & $1.22$ & $(6,0)$ & -- & 3534\\
$(c;0,0,0;4)$ & 4 & $(43,13)$ & 9 & 1 & YES & YES & YES & $1.22$ & $(4,1)$ & -- & 3535\\
$(c;0,0,0;4)$ & 4 & $(46,19)$ & 8 & 2 & YES & YES & YES & $1.22$ & $(4,1)$ & -- & 3536\\
$(c;0,0,0;4)$ & 4 & $(61,17)$ & 9 & 1 & YES & YES & YES & $1.25$ & $(2,2)$ & -- & 3537\\
$(c;0,1,0;11)$ & 5 & $(11,3)$ & 5 & 11 & YES & YES & YES & $1.00$ & $(4,1)$ & -- & 3538\\
$(c;0,1,0;11)$ & 5 & $(11,4)$ & 5 & 11 & YES & YES & YES & $0.88$ & $(4,1)$ & -- & 3539\\
$(c;0,1,0;11)$ & 5 & $(13,3)$ & 6 & 1 & YES & YES & NO(2) & $1.00$ & $(6,0)$ & -- & 3540\\
$(c;0,1,0;11)$ & 5 & $(17,7)$ & 6 & 1 & YES & YES & YES & $1.12$ & $(2,2)$ & -- & 3541\\
$(c;0,1,0;11)$ & 5 & $(18,5)$ & 6 & 1 & YES & YES & YES & $1.11$ & $(2,2)$ & -- & 3542\\
$(c;0,1,0;11)$ & 5 & $(19,7)$ & 6 & 1 & YES & YES & YES & $1.12$ & $(2,2)$ & -- & 3543\\
$(c;0,1,0;11)$ & 5 & $(19,8)$ & 6 & 1 & YES & YES & YES & $1.00$ & $(2,2)$ & -- & 3544\\
$(c;0,1,0;11)$ & 5 & $(31,13)$ & 7 & 1 & YES & YES & YES & $1.40$ & $(4,1)$ & -- & 3545\\
$(c;0,1,1;5)$ & 6 & $(7,2)$ & 4 & 1 & YES & YES & NO(2) & $1.00$ & $(6,0)$ & -- & 3546\\
$(c;0,1,1;5)$ & 6 & $(8,3)$ & 4 & 1 & YES & YES & YES & $1.00$ & $(2,2)$ & -- & 3547\\
$(c;0,1,1;5)$ & 6 & $(9,4)$ & 5 & 1 & YES & YES & YES & $1.11$ & $(2,2)$ & -- & 3548\\
$(c;0,1,1;5)$ & 6 & $(11,4)$ & 5 & 1 & YES & YES & YES & $1.11$ & $(2,2)$ & -- & 3549\\
$(c;0,1,1;5)$ & 6 & $(17,5)$ & 6 & 1 & YES & YES & YES & $1.22$ & $(2,2)$ & -- & 3550\\
$(c;0,1,1;5)$ & 6 & $(18,5)$ & 6 & 1 & YES & YES & NO(2) & $1.20$ & $(4,1)$ & -- & 3551\\
$(c;0,2,0;7)$ & 6 & $(12,5)$ & 5 & 1 & YES & YES & YES & $1.00$ & $(2,2)$ & -- & 3552\\
$(c;0,2,0;7)$ & 6 & $(13,5)$ & 5 & 1 & YES & YES & YES & $1.00$ & $(2,2)$ & -- & 3553\\
$(c;0,2,0;7)$ & 6 & $(15,4)$ & 6 & 1 & YES & YES & YES & $1.11$ & $(2,2)$ & -- & 3554\\
$(c;0,2,0;7)$ & 6 & $(17,5)$ & 6 & 1 & YES & YES & YES & $1.33$ & $(2,2)$ & -- & 3555\\
$(c;0,2,0;7)$ & 6 & $(18,5)$ & 6 & 1 & YES & YES & YES & $1.12$ & $(2,2)$ & -- & 3556\\
$(c;0,2,0;7)$ & 6 & $(19,8)$ & 6 & 1 & YES & YES & YES & $1.40$ & $(4,1)$ & -- & 3557\\
$(c;0,2,0;7)$ & 6 & $(21,8)$ & 6 & 7 & YES & YES & YES & $1.30$ & $(4,1)$ & -- & 3558\\
$(c;0,2,0;7)$ & 6 & $(22,5)$ & 7 & 1 & YES & YES & YES & $1.12$ & $(2,2)$ & -- & 3559\\
$(c;0,2,0;7)$ & 6 & $(23,7)$ & 7 & 1 & YES & YES & YES & $1.22$ & $(4,1)$ & -- & 3560\\
$(c;0,2,0;7)$ & 6 & $(26,7)$ & 7 & 1 & YES & YES & YES & $1.22$ & $(4,1)$ & -- & 3561\\
$(c;0,2,1;19)$ & 7 & $(5,2)$ & 3 & 1 & YES & YES & YES & $1.22$ & $(2,2)$ & -- & 3562\\
$(c;0,2,1;19)$ & 7 & $(7,2)$ & 4 & 1 & YES & YES & YES & $1.22$ & $(2,2)$ & -- & 3563\\
$(c;0,2,1;19)$ & 7 & $(10,3)$ & 5 & 1 & YES & YES & YES & $1.11$ & $(2,2)$ & -- & 3564\\
$(c;0,2,1;19)$ & 7 & $(11,3)$ & 5 & 1 & YES & YES & YES & $1.22$ & $(2,2)$ & -- & 3565\\
$(c;0,2,1;19)$ & 7 & $(12,5)$ & 5 & 1 & YES & YES & YES & $1.00$ & $(6,0)$ & -- & 3566\\
$(c;0,3,1;23)$ & 8 & $(5,2)$ & 3 & 1 & YES & YES & NO(2) & $1.11$ & $(6,0)$ & -- & 3567\\
$(c;0,3,1;23)$ & 8 & $(7,2)$ & 4 & 1 & YES & YES & NO(2) & $1.10$ & $(4,1)$ & -- & 3568\\
$(d;0,0,0;5)$ & 5 & $(11,3)$ & 5 & 1 & YES & YES & YES & $0.88$ & $(4,1)$ & -- & 3569\\
$(d;0,0,0;5)$ & 5 & $(12,5)$ & 5 & 1 & YES & YES & YES & $1.11$ & $(2,2)$ & -- & 3570\\
$(d;0,0,0;5)$ & 5 & $(13,3)$ & 6 & 1 & YES & YES & NO(2) & $0.89$ & $(6,0)$ & -- & 3571\\
$(d;0,0,0;5)$ & 5 & $(13,5)$ & 5 & 1 & YES & YES & YES & $1.00$ & $(2,2)$ & -- & 3572\\
$(d;0,0,0;5)$ & 5 & $(16,7)$ & 6 & 1 & YES & YES & NO(2) & $1.30$ & $(4,1)$ & -- & 3573\\
$(d;0,0,0;5)$ & 5 & $(17,7)$ & 6 & 1 & YES & YES & YES & $1.12$ & $(2,2)$ & -- & 3574\\
$(d;0,0,0;5)$ & 5 & $(18,5)$ & 6 & 1 & YES & YES & YES & $1.11$ & $(2,2)$ & -- & 3575\\
$(d;0,0,0;5)$ & 5 & $(19,8)$ & 6 & 1 & YES & YES & YES & $0.88$ & $(2,2)$ & -- & 3576\\
$(d;0,0,0;5)$ & 5 & $(26,7)$ & 7 & 1 & YES & YES & NO(2) & $1.20$ & $(4,1)$ & -- & 3577\\
$(d;0,0,0;5)$ & 5 & $(31,13)$ & 7 & 1 & YES & YES & YES & $1.22$ & $(4,1)$ & -- & 3578\\
$(d;0,0,1;14)$ & 6 & $(7,2)$ & 4 & 7 & YES & YES & NO(2) & $1.00$ & $(6,0)$ & -- & 3579\\
$(d;0,0,1;14)$ & 6 & $(9,2)$ & 5 & 1 & YES & YES & NO(2) & $1.00$ & $(6,0)$ & -- & 3580\\
$(d;0,0,1;14)$ & 6 & $(9,4)$ & 5 & 1 & YES & YES & YES & $1.22$ & $(2,2)$ & -- & 3581\\
$(d;0,0,1;14)$ & 6 & $(11,4)$ & 5 & 1 & YES & YES & YES & $1.22$ & $(2,2)$ & -- & 3582\\
$(d;0,0,1;14)$ & 6 & $(12,5)$ & 5 & 2 & YES & YES & YES & $1.00$ & $(2,2)$ & -- & 3583\\
$(d;0,0,1;14)$ & 6 & $(17,5)$ & 6 & 1 & YES & YES & YES & $0.88$ & $(4,1)$ & -- & 3584\\
$(d;0,0,2;9)$ & 7 & $(5,2)$ & 3 & 1 & YES & YES & YES & $1.22$ & $(2,2)$ & -- & 3585\\
$(d;0,0,2;9)$ & 7 & $(7,2)$ & 4 & 1 & YES & YES & YES & $1.22$ & $(2,2)$ & -- & 3586\\
$(d;0,0,2;9)$ & 7 & $(12,5)$ & 5 & 3 & YES & YES & YES & $1.00$ & $(6,0)$ & -- & 3587\\
$(d;0,0,3;22)$ & 8 & $(8,3)$ & 4 & 2 & YES & YES & YES & $1.33$ & $(2,2)$ & -- & 3588\\
$(d;0,1,0;6)$ & 6 & $(10,3)$ & 5 & 2 & YES & YES & YES & $0.88$ & $(4,1)$ & -- & 3589\\
$(d;0,1,0;6)$ & 6 & $(12,5)$ & 5 & 6 & YES & YES & YES & $1.00$ & $(2,2)$ & -- & 3590\\
$(d;0,1,0;6)$ & 6 & $(13,5)$ & 5 & 1 & YES & YES & YES & $1.00$ & $(2,2)$ & -- & 3591\\
$(d;0,1,0;6)$ & 6 & $(23,7)$ & 7 & 1 & YES & YES & YES & $1.11$ & $(4,1)$ & -- & 3592\\
$(d;0,1,0;6)$ & 6 & $(26,7)$ & 7 & 2 & YES & YES & YES & $1.11$ & $(4,1)$ & -- & 3593\\
$(d;0,1,1;17)$ & 7 & $(7,2)$ & 4 & 1 & YES & YES & NO(2) & $1.00$ & $(6,0)$ & -- & 3594\\
$(d;0,1,1;17)$ & 7 & $(10,3)$ & 5 & 1 & YES & YES & YES & $0.88$ & $(2,2)$ & -- & 3595\\
$(d;0,1,1;17)$ & 7 & $(12,5)$ & 5 & 1 & YES & YES & YES & $0.88$ & $(6,0)$ & -- & 3596\\
$(d;0,1,2;11)$ & 8 & $(7,2)$ & 4 & 1 & YES & YES & YES & $1.22$ & $(2,2)$ & -- & 3597\\
$(d;0,2,0;7)$ & 7 & $(9,4)$ & 5 & 1 & YES & YES & NO(2) & $1.11$ & $(6,0)$ & -- & 3598\\
$(d;0,2,0;7)$ & 7 & $(10,3)$ & 5 & 1 & YES & YES & NO(2) & $1.10$ & $(4,1)$ & -- & 3599\\
$(d;0,2,1;20)$ & 8 & $(5,2)$ & 3 & 5 & YES & YES & NO(2) & $0.89$ & $(6,0)$ & -- & 3600\\
$(d;0,2,1;20)$ & 8 & $(7,2)$ & 4 & 1 & YES & YES & NO(2) & $1.20$ & $(4,1)$ & -- & 3601\\
$(d;0,2,1;20)$ & 8 & $(9,2)$ & 5 & 1 & YES & YES & NO(2) & $1.10$ & $(4,1)$ & -- & 3602\\
$(e;0,0,0;4)$ & 5 & $(10,3)$ & 5 & 2 & YES & YES & YES & $1.22$ & $(2,2)$ & -- & 3603\\
$(e;0,0,0;4)$ & 5 & $(12,5)$ & 5 & 4 & YES & YES & YES & $1.11$ & $(2,2)$ & -- & 3604\\
$(e;0,0,0;4)$ & 5 & $(13,5)$ & 5 & 1 & YES & YES & YES & $1.00$ & $(2,2)$ & -- & 3605\\
$(e;0,1,0;5)$ & 6 & $(10,3)$ & 5 & 5 & YES & YES & YES & $1.22$ & $(2,2)$ & -- & 3606\\
$(e;0,1,0;5)$ & 6 & $(22,5)$ & 7 & 1 & YES & YES & YES & $1.22$ & $(2,2)$ & -- & 3607\\
$(e;0,2,0;6)$ & 7 & $(7,2)$ & 4 & 1 & YES & YES & YES & $1.00$ & $(2,2)$ & -- & 3608\\
$(e;1,0,0;18)$ & 6 & $(7,2)$ & 4 & 1 & YES & YES & YES & $1.22$ & $(2,2)$ & -- & 3609\\
$(e;1,0,0;18)$ & 6 & $(7,3)$ & 4 & 1 & YES & YES & NO(2) & $1.11$ & $(6,0)$ & -- & 3610\\
$(e;1,0,0;18)$ & 6 & $(8,3)$ & 4 & 2 & YES & YES & YES & $1.00$ & $(2,2)$ & -- & 3611\\
$(e;1,0,0;18)$ & 6 & $(9,4)$ & 5 & 9 & YES & YES & YES & $1.22$ & $(2,2)$ & -- & 3612\\
$(e;1,0,0;18)$ & 6 & $(10,3)$ & 5 & 2 & YES & YES & YES & $1.22$ & $(2,2)$ & -- & 3613\\
$(e;1,2,0;28)$ & 8 & $(3,1)$ & 2 & 1 & YES & YES & YES & $0.88$ & $(2,2)$ & -- & 3614\\
$(e;1,2,0;28)$ & 8 & $(5,2)$ & 3 & 1 & YES & YES & YES & $0.88$ & $(6,0)$ & -- & 3615\\
$(e;1,2,0;28)$ & 8 & $(9,2)$ & 5 & 1 & YES & YES & YES & $0.88$ & $(6,0)$ & -- & 3616\\
$(e;2,0,0;24)$ & 7 & $(3,1)$ & 2 & 3 & YES & YES & YES & $1.00$ & $(2,2)$ & -- & 3617\\
$(e;2,0,0;24)$ & 7 & $(5,2)$ & 3 & 1 & YES & YES & YES & $1.25$ & $(2,2)$ & -- & 3618\\
$(e;2,1,0;31)$ & 8 & $(5,2)$ & 3 & 1 & YES & YES & YES & $0.88$ & $(6,0)$ & -- & 3619\\
$(e;2,2,0;38)$ & 9 & $(3,1)$ & 2 & 1 & YES & YES & YES & $0.88$ & $(6,0)$ & -- & 3620\\
$(e;2,2,0;38)$ & 9 & $(4,1)$ & 3 & 2 & YES & YES & YES & $0.88$ & $(6,0)$ & -- & 3621\\
$(e;3,0,0;10)$ & 8 & $(2,1)$ & 1 & 2 & YES & YES & YES & $1.22$ & $(2,2)$ & -- & 3622\\
$(e;3,0,0;10)$ & 8 & $(3,1)$ & 2 & 1 & YES & YES & NO(2) & $1.20$ & $(4,1)$ & -- & 3623\\
$(e;3,0,0;10)$ & 8 & $(5,1)$ & 4 & 5 & YES & YES & YES & $0.88$ & $(4,1)$ & -- & 3624\\
$(e;3,0,0;10)$ & 8 & $(5,2)$ & 3 & 5 & YES & YES & YES & $1.22$ & $(2,2)$ & -- & 3625\\
$(f;0,0,0;6)$ & 4 & $(25,11)$ & 7 & 1 & YES & YES & NO(2) & $1.11$ & $(6,0)$ & -- & 3626\\
$(f;0,0,0;6)$ & 4 & $(29,8)$ & 7 & 1 & YES & YES & YES & $1.11$ & $(2,2)$ & -- & 3627\\
$(f;0,0,0;6)$ & 4 & $(29,9)$ & 8 & 1 & YES & YES & NO(2) & $1.00$ & $(6,0)$ & -- & 3628\\
$(f;0,0,0;6)$ & 4 & $(29,11)$ & 7 & 1 & YES & YES & YES & $1.11$ & $(2,2)$ & -- & 3629\\
$(f;0,0,0;6)$ & 4 & $(29,12)$ & 7 & 1 & YES & YES & NO(2) & $1.00$ & $(10,-2)$ & -- & 3630\\
$(f;0,0,0;6)$ & 4 & $(29,13)$ & 8 & 1 & YES & YES & YES & $1.00$ & $(4,1)$ & -- & 3631\\
$(f;0,0,0;6)$ & 4 & $(31,11)$ & 8 & 1 & YES & YES & YES & $1.00$ & $(4,1)$ & -- & 3632\\
$(f;0,0,0;6)$ & 4 & $(31,12)$ & 7 & 1 & YES & YES & NO(2) & $1.00$ & $(6,0)$ & -- & 3633\\
$(f;0,0,0;6)$ & 4 & $(31,13)$ & 7 & 1 & YES & YES & YES & $1.00$ & $(2,2)$ & -- & 3634\\
$(f;0,0,0;6)$ & 4 & $(34,13)$ & 7 & 2 & YES & YES & YES & $1.12$ & $(2,2)$ & -- & 3635\\
$(f;0,0,0;6)$ & 4 & $(35,11)$ & 9 & 1 & YES & YES & NO(2) & $1.30$ & $(4,1)$ & -- & 3636\\
$(f;0,0,0;6)$ & 4 & $(36,11)$ & 8 & 6 & YES & YES & YES & $1.33$ & $(2,2)$ & -- & 3637\\
$(f;0,0,0;6)$ & 4 & $(40,11)$ & 8 & 2 & YES & YES & YES & $1.00$ & $(2,2)$ & -- & 3638\\
$(f;0,0,0;6)$ & 4 & $(41,12)$ & 8 & 1 & YES & YES & NO(2) & $1.20$ & $(4,1)$ & -- & 3639\\
$(f;0,0,0;6)$ & 4 & $(44,17)$ & 8 & 2 & YES & YES & YES & $1.33$ & $(2,2)$ & -- & 3640\\
$(f;0,0,0;6)$ & 4 & $(45,19)$ & 8 & 3 & YES & YES & YES & $1.12$ & $(6,0)$ & -- & 3641\\
$(f;0,0,0;6)$ & 4 & $(46,17)$ & 8 & 2 & YES & YES & YES & $1.00$ & $(6,0)$ & -- & 3642\\
$(f;0,0,0;6)$ & 4 & $(49,18)$ & 8 & 1 & YES & YES & YES & $1.00$ & $(6,0)$ & -- & 3643\\
$(f;0,0,0;6)$ & 4 & $(53,11)$ & 10 & 1 & YES & YES & NO(2) & $1.30$ & $(4,1)$ & -- & 3644\\
$(f;0,0,0;6)$ & 4 & $(56,17)$ & 9 & 2 & YES & YES & YES & $1.12$ & $(6,0)$ & -- & 3645\\
$(f;0,0,0;6)$ & 4 & $(59,18)$ & 9 & 1 & YES & YES & YES & $1.22$ & $(4,1)$ & -- & 3646\\
$(f;0,1,0;7)$ & 5 & $(26,7)$ & 7 & 1 & YES & YES & NO(2) & $1.00$ & $(6,0)$ & -- & 3647\\
$(f;0,1,0;7)$ & 5 & $(30,7)$ & 8 & 1 & YES & YES & NO(2) & $1.00$ & $(6,0)$ & -- & 3648\\
$(g;0,0,0;19)$ & 6 & $(7,3)$ & 4 & 1 & YES & YES & YES & $1.12$ & $(2,2)$ & -- & 3649\\
$(g;0,0,0;19)$ & 6 & $(8,3)$ & 4 & 1 & YES & YES & YES & $1.12$ & $(2,2)$ & -- & 3650\\
$(g;0,0,0;19)$ & 6 & $(13,4)$ & 6 & 1 & YES & YES & YES & $1.12$ & $(2,2)$ & -- & 3651\\
$(g;0,0,1;26)$ & 7 & $(3,1)$ & 2 & 1 & YES & YES & YES & $1.00$ & $(2,2)$ & -- & 3652\\
$(g;0,0,1;26)$ & 7 & $(5,2)$ & 3 & 1 & YES & YES & YES & $0.88$ & $(4,1)$ & -- & 3653\\
$(g;0,0,1;26)$ & 7 & $(7,2)$ & 4 & 1 & YES & YES & YES & $0.75$ & $(4,1)$ & -- & 3654\\
$(g;0,0,2;11)$ & 8 & $(2,1)$ & 1 & 1 & YES & YES & YES & $1.00$ & $(2,2)$ & -- & 3655\\
$(g;0,0,2;11)$ & 8 & $(3,1)$ & 2 & 1 & YES & YES & YES & $1.12$ & $(2,2)$ & -- & 3656\\
$(g;0,0,2;11)$ & 8 & $(5,1)$ & 4 & 1 & YES & YES & YES & $1.00$ & $(2,2)$ & -- & 3657\\
$(g;0,0,2;11)$ & 8 & $(5,2)$ & 3 & 1 & YES & YES & YES & $1.22$ & $(4,1)$ & -- & 3658\\
$(g;0,1,0;24)$ & 7 & $(3,1)$ & 2 & 3 & YES & YES & YES & $1.00$ & $(2,2)$ & -- & 3659\\
$(g;0,1,0;24)$ & 7 & $(5,2)$ & 3 & 1 & YES & YES & YES & $1.00$ & $(2,2)$ & -- & 3660\\
$(g;0,1,0;24)$ & 7 & $(7,2)$ & 4 & 1 & YES & YES & YES & $0.75$ & $(4,1)$ & -- & 3661\\
$(g;0,1,1;33)$ & 8 & $(3,1)$ & 2 & 3 & YES & YES & YES & $0.75$ & $(4,1)$ & -- & 3662\\
$(g;0,2,0;29)$ & 8 & $(2,1)$ & 1 & 1 & YES & YES & YES & $1.00$ & $(2,2)$ & -- & 3663\\
$(g;0,2,0;29)$ & 8 & $(5,1)$ & 4 & 1 & YES & YES & YES & $1.22$ & $(2,2)$ & -- & 3664\\
$(g;1,0,0;7)$ & 7 & $(5,2)$ & 3 & 1 & YES & YES & YES & $1.00$ & $(4,1)$ & -- & 3665\\
$(g;1,0,0;7)$ & 7 & $(10,3)$ & 5 & 1 & YES & YES & YES & $1.11$ & $(4,1)$ & -- & 3666\\
$(g;1,0,1;38)$ & 8 & $(2,1)$ & 1 & 2 & YES & YES & YES & $0.88$ & $(4,1)$ & -- & 3667\\
$(g;1,0,1;38)$ & 8 & $(3,1)$ & 2 & 1 & YES & YES & YES & $0.88$ & $(6,0)$ & -- & 3668\\
$(g;1,0,1;38)$ & 8 & $(4,1)$ & 3 & 2 & YES & YES & YES & $1.00$ & $(4,1)$ & -- & 3669\\
$(g;1,0,1;38)$ & 8 & $(7,2)$ & 4 & 1 & YES & YES & YES & $1.11$ & $(4,1)$ & -- & 3670\\
$(g;1,0,1;38)$ & 8 & $(9,2)$ & 5 & 1 & YES & YES & YES & $1.11$ & $(4,1)$ & -- & 3671\\
$(g;1,1,0;9)$ & 8 & $(2,1)$ & 1 & 1 & YES & YES & YES & $0.88$ & $(4,1)$ & -- & 3672\\
$(g;1,1,0;9)$ & 8 & $(3,1)$ & 2 & 3 & YES & YES & YES & $0.75$ & $(6,0)$ & -- & 3673\\
$(g;1,1,0;9)$ & 8 & $(5,2)$ & 3 & 1 & YES & YES & YES & $1.12$ & $(2,2)$ & -- & 3674\\
$(h;0,0,0;6)$ & 5 & $(8,3)$ & 4 & 2 & YES & YES & YES & $1.11$ & $(2,2)$ & -- & 3675\\
$(h;0,1,0;8)$ & 6 & $(7,3)$ & 4 & 1 & YES & YES & YES & $1.00$ & $(2,2)$ & -- & 3676\\
$(h;0,2,0;10)$ & 7 & $(2,1)$ & 1 & 2 & YES & YES & YES & $0.89$ & $(2,2)$ & -- & 3677\\
$(h;0,2,0;10)$ & 7 & $(5,2)$ & 3 & 5 & YES & YES & YES & $1.11$ & $(2,2)$ & -- & 3678\\
$(i;0,0,0;9)$ & 5 & $(13,5)$ & 5 & 1 & YES & YES & YES & $1.11$ & $(2,2)$ & -- & 3679\\
$(i;0,0,0;9)$ & 5 & $(17,5)$ & 6 & 1 & YES & YES & YES & $1.22$ & $(2,2)$ & -- & 3680\\
$(i;0,0,0;9)$ & 5 & $(18,5)$ & 6 & 9 & YES & YES & YES & $1.00$ & $(2,2)$ & -- & 3681\\
$(i;0,0,0;9)$ & 5 & $(22,5)$ & 7 & 1 & YES & YES & NO(2) & $1.30$ & $(4,1)$ & -- & 3682\\
$(i;0,1,0;12)$ & 6 & $(7,2)$ & 4 & 1 & YES & YES & YES & $0.88$ & $(4,1)$ & -- & 3683\\
$(i;0,1,0;12)$ & 6 & $(10,3)$ & 5 & 2 & YES & YES & YES & $1.33$ & $(2,2)$ & -- & 3684\\
$(i;0,1,0;12)$ & 6 & $(11,3)$ & 5 & 1 & YES & YES & YES & $1.12$ & $(2,2)$ & -- & 3685\\
$(i;0,1,0;12)$ & 6 & $(13,3)$ & 6 & 1 & YES & YES & NO(2) & $1.10$ & $(4,1)$ & -- & 3686\\
$(i;0,2,0;15)$ & 7 & $(5,2)$ & 3 & 5 & YES & YES & NO(2) & $1.00$ & $(6,0)$ & -- & 3687\\
$(i;0,2,0;15)$ & 7 & $(7,2)$ & 4 & 1 & YES & YES & NO(2) & $1.10$ & $(4,1)$ & -- & 3688\\
$(j;0,0,0;8)$ & 5 & $(13,5)$ & 5 & 1 & YES & YES & YES & $1.11$ & $(2,2)$ & -- & 3689\\
$(j;0,0,0;8)$ & 5 & $(18,5)$ & 6 & 2 & YES & YES & YES & $1.11$ & $(2,2)$ & -- & 3690\\
$(j;0,0,0;8)$ & 5 & $(19,8)$ & 6 & 1 & YES & YES & YES & $1.00$ & $(2,2)$ & -- & 3691\\
$(j;0,0,0;8)$ & 5 & $(21,8)$ & 6 & 1 & YES & YES & YES & $1.00$ & $(2,2)$ & -- & 3692\\
$(j;0,0,0;8)$ & 5 & $(23,7)$ & 7 & 1 & YES & YES & YES & $1.00$ & $(2,2)$ & -- & 3693\\
$(j;0,0,0;8)$ & 5 & $(24,7)$ & 7 & 8 & YES & YES & YES & $1.12$ & $(2,2)$ & -- & 3694\\
$(j;0,0,0;8)$ & 5 & $(25,7)$ & 7 & 1 & YES & YES & YES & $1.12$ & $(2,2)$ & -- & 3695\\
$(j;0,0,0;8)$ & 5 & $(29,12)$ & 7 & 1 & YES & YES & YES & $1.12$ & $(6,0)$ & -- & 3696\\
$(j;0,0,0;8)$ & 5 & $(30,7)$ & 8 & 2 & YES & YES & NO(2) & $0.89$ & $(6,0)$ & -- & 3697\\
$(j;0,0,0;8)$ & 5 & $(30,11)$ & 7 & 2 & YES & YES & YES & $1.22$ & $(4,1)$ & -- & 3698\\
$(j;0,0,0;8)$ & 5 & $(36,11)$ & 8 & 4 & YES & YES & YES & $1.22$ & $(4,1)$ & -- & 3699\\
$(j;0,0,0;8)$ & 5 & $(41,12)$ & 8 & 1 & YES & YES & YES & $1.11$ & $(4,1)$ & -- & 3700\\
$(j;0,1,0;10)$ & 6 & $(12,5)$ & 5 & 2 & YES & YES & NO(2) & $1.20$ & $(8,-1)$ & -- & 3701\\
$(j;0,1,0;10)$ & 6 & $(13,5)$ & 5 & 1 & YES & YES & YES & $0.88$ & $(4,1)$ & -- & 3702\\
$(j;0,1,0;10)$ & 6 & $(16,7)$ & 6 & 2 & YES & YES & YES & $1.12$ & $(2,2)$ & -- & 3703\\
$(j;0,2,0;12)$ & 7 & $(11,4)$ & 5 & 1 & YES & YES & NO(2) & $0.89$ & $(6,0)$ & -- & 3704\\
$(j;0,2,0;12)$ & 7 & $(13,4)$ & 6 & 1 & YES & YES & NO(2) & $1.20$ & $(4,1)$ & -- & 3705
\end{longtable}
\subsection{2 chains, $K^2 = 3$}
\begin{longtable}{|c|c|c|c|c|c|c|c|c|c|c|c|}
\hline
\multicolumn{12}{|c|}{2 chains, $K^2 = 3$}\\
\hline
$(n,a)$ & Len & $(n,a)$ & Len & GCD & Nef & $\mathbb Q$-ef & Obs 0 & $\overline c_1^2 / \overline c_2$ & $(P,K)$ & WH & Index\\
\hline
\endfirsthead

\hline
$(n,a)$ & Len & $(n,a)$ & Len & GCD & Nef & $\mathbb Q$-ef & Obs 0 & $\overline c_1^2 / \overline c_2$ & $(P,K)$ & WH & Index\\
\hline
\endhead
\hline
\endfoot

$(18,7)$ & 6 & $(17,7)$ & 6 & 1 & YES & YES & YES & $1.25$ & $(2,3)$ & -- & 3706\\
$(21,8)$ & 6 & $(19,8)$ & 6 & 1 & YES & YES & NO(2) & $1.50$ & $(2,3)$ & -- & 3707\\
$(23,10)$ & 7 & $(23,10)$ & 7 & 23 & YES & YES & YES & $1.75$ & $(2,3)$ & -- & 3708\\
$(24,7)$ & 7 & $(17,7)$ & 6 & 1 & YES & YES & NO(2) & $1.33$ & $(4,2)$ & -- & 3709\\
$(24,7)$ & 7 & $(18,5)$ & 6 & 6 & YES & YES & YES & $1.50$ & $(2,3)$ & -- & 3710\\
$(24,7)$ & 7 & $(19,8)$ & 6 & 1 & YES & YES & NO(2) & $1.73$ & $(4,2)$ & NO & 3711\\
$(25,9)$ & 7 & $(18,5)$ & 6 & 1 & YES & YES & YES & $1.43$ & $(4,2)$ & -- & 3712\\
$(25,7)$ & 7 & $(19,8)$ & 6 & 1 & YES & YES & NO(2) & $1.73$ & $(4,2)$ & NO & 3713\\
$(25,7)$ & 7 & $(19,8)$ & 6 & 1 & YES & YES & NO(2) & $1.73$ & $(4,2)$ & -- & 3714\\
$(25,7)$ & 7 & $(22,5)$ & 7 & 1 & YES & YES & NO(2) & $1.50$ & $(6,1)$ & -- & 3715\\
$(25,7)$ & 7 & $(22,9)$ & 7 & 1 & YES & YES & NO(2) & $1.60$ & $(6,1)$ & -- & 3716\\
$(25,7)$ & 7 & $(23,9)$ & 7 & 1 & YES & YES & NO(2) & $1.64$ & $(8,0)$ & -- & 3717\\
$(25,6)$ & 9 & $(24,7)$ & 7 & 1 & YES & YES & NO(2) & $1.60$ & $(6,1)$ & -- & 3718\\
$(25,6)$ & 9 & $(24,7)$ & 7 & 1 & YES & YES & NO(2) & $1.70$ & $(6,1)$ & NO & 3719\\
$(25,7)$ & 7 & $(25,7)$ & 7 & 25 & YES & YES & YES & $1.57$ & $(2,3)$ & NO & 3720\\
$(25,7)$ & 7 & $(25,7)$ & 7 & 25 & YES & YES & YES & $1.57$ & $(2,3)$ & -- & 3721\\
$(26,7)$ & 7 & $(18,7)$ & 6 & 2 & YES & YES & NO(2) & $1.40$ & $(2,3)$ & -- & 3722\\
$(26,11)$ & 7 & $(24,7)$ & 7 & 2 & YES & YES & NO(2) & $1.44$ & $(4,2)$ & NO & 3723\\
$(26,11)$ & 7 & $(25,7)$ & 7 & 1 & YES & YES & NO(2) & $1.82$ & $(4,2)$ & NO & 3724\\
$(26,11)$ & 7 & $(25,7)$ & 7 & 1 & YES & YES & NO(2) & $1.82$ & $(4,2)$ & -- & 3725\\
$(27,8)$ & 7 & $(16,7)$ & 6 & 1 & YES & YES & NO(2) & $1.70$ & $(6,1)$ & -- & 3726\\
$(27,8)$ & 7 & $(19,5)$ & 7 & 1 & YES & YES & NO(2) & $1.56$ & $(4,2)$ & -- & 3727\\
$(27,8)$ & 7 & $(19,5)$ & 7 & 1 & YES & YES & NO(2) & $1.67$ & $(4,2)$ & NO & 3728\\
$(27,8)$ & 7 & $(19,7)$ & 6 & 1 & YES & YES & NO(2) & $1.50$ & $(2,3)$ & -- & 3729\\
$(27,10)$ & 7 & $(19,8)$ & 6 & 1 & YES & YES & NO(2) & $1.73$ & $(4,2)$ & NO & 3730\\
$(27,8)$ & 7 & $(25,9)$ & 7 & 1 & YES & YES & NO(2) & $1.60$ & $(2,3)$ & -- & 3731\\
$(27,8)$ & 7 & $(25,11)$ & 7 & 1 & YES & YES & YES & $1.43$ & $(2,3)$ & -- & 3732\\
$(27,10)$ & 7 & $(25,7)$ & 7 & 1 & YES & YES & NO(2) & $1.67$ & $(4,2)$ & NO & 3733\\
$(27,10)$ & 7 & $(25,7)$ & 7 & 1 & YES & YES & NO(2) & $1.67$ & $(4,2)$ & -- & 3734\\
$(27,7)$ & 9 & $(26,11)$ & 7 & 1 & YES & YES & NO(2) & $1.80$ & $(6,1)$ & NO & 3735\\
$(27,8)$ & 7 & $(26,7)$ & 7 & 1 & YES & YES & NO(2) & $1.44$ & $(4,2)$ & -- & 3736\\
$(27,8)$ & 7 & $(26,7)$ & 7 & 1 & YES & YES & NO(2) & $1.56$ & $(4,2)$ & NO & 3737\\
$(27,8)$ & 7 & $(26,11)$ & 7 & 1 & YES & YES & YES & $1.57$ & $(2,3)$ & -- & 3738\\
$(28,11)$ & 8 & $(17,5)$ & 6 & 1 & YES & YES & YES & $1.43$ & $(4,2)$ & -- & 3739\\
$(28,11)$ & 8 & $(24,7)$ & 7 & 4 & YES & YES & NO(2) & $1.70$ & $(2,3)$ & -- & 3740\\
$(28,11)$ & 8 & $(27,10)$ & 7 & 1 & YES & YES & NO(2) & $1.78$ & $(4,2)$ & -- & 3741\\
$(29,11)$ & 7 & $(13,5)$ & 5 & 1 & YES & YES & NO(2) & $1.44$ & $(4,2)$ & -- & 3742\\
$(29,8)$ & 7 & $(17,5)$ & 6 & 1 & YES & YES & YES & $1.50$ & $(2,3)$ & NO & 3743\\
$(29,8)$ & 7 & $(17,5)$ & 6 & 1 & YES & YES & YES & $1.50$ & $(2,3)$ & -- & 3744\\
$(29,9)$ & 8 & $(17,7)$ & 6 & 1 & YES & YES & NO(2) & $1.50$ & $(6,1)$ & -- & 3745\\
$(29,8)$ & 7 & $(18,5)$ & 6 & 1 & YES & YES & NO(2) & $1.25$ & $(10,-1)$ & -- & 3746\\
$(29,9)$ & 8 & $(18,5)$ & 6 & 1 & YES & YES & YES & $1.43$ & $(4,2)$ & NO & 3747\\
$(29,9)$ & 8 & $(18,5)$ & 6 & 1 & YES & YES & YES & $1.43$ & $(4,2)$ & -- & 3748\\
$(29,12)$ & 7 & $(18,7)$ & 6 & 1 & YES & YES & NO(2) & $1.70$ & $(2,3)$ & -- & 3749\\
$(29,8)$ & 7 & $(19,5)$ & 7 & 1 & YES & YES & NO(2) & $1.56$ & $(4,2)$ & -- & 3750\\
$(29,8)$ & 7 & $(19,5)$ & 7 & 1 & YES & YES & NO(2) & $1.67$ & $(4,2)$ & NO & 3751\\
$(29,11)$ & 7 & $(19,8)$ & 6 & 1 & YES & YES & NO(2) & $1.73$ & $(4,2)$ & NO & 3752\\
$(29,11)$ & 7 & $(19,8)$ & 6 & 1 & YES & YES & NO(2) & $1.73$ & $(4,2)$ & -- & 3753\\
$(29,8)$ & 7 & $(21,8)$ & 6 & 1 & YES & YES & YES & $1.50$ & $(2,3)$ & -- & 3754\\
$(29,8)$ & 7 & $(22,9)$ & 7 & 1 & YES & YES & NO(2) & $1.73$ & $(4,2)$ & -- & 3755\\
$(29,11)$ & 7 & $(22,5)$ & 7 & 1 & YES & YES & NO(2) & $1.50$ & $(6,1)$ & -- & 3756\\
$(29,8)$ & 7 & $(23,9)$ & 7 & 1 & YES & YES & NO(2) & $1.70$ & $(6,1)$ & -- & 3757\\
$(29,8)$ & 7 & $(26,11)$ & 7 & 1 & YES & YES & NO(2) & $1.56$ & $(2,3)$ & -- & 3758\\
$(29,8)$ & 7 & $(27,10)$ & 7 & 1 & YES & YES & NO(2) & $1.38$ & $(6,1)$ & -- & 3759\\
$(29,11)$ & 7 & $(27,8)$ & 7 & 1 & YES & YES & YES & $1.90$ & $(2,3)$ & -- & 3760\\
$(29,11)$ & 7 & $(29,8)$ & 7 & 29 & YES & YES & NO(2) & $1.60$ & $(2,3)$ & NO & 3761\\
$(29,11)$ & 7 & $(29,8)$ & 7 & 29 & YES & YES & YES & $1.50$ & $(2,3)$ & -- & 3762\\
$(29,11)$ & 7 & $(29,11)$ & 7 & 29 & YES & YES & YES & $2.00$ & $(2,3)$ & -- & 3763\\
$(30,13)$ & 8 & $(21,8)$ & 6 & 3 & YES & YES & NO(2) & $1.50$ & $(10,-1)$ & -- & 3764\\
$(30,11)$ & 7 & $(24,7)$ & 7 & 6 & YES & YES & YES & $1.14$ & $(4,2)$ & -- & 3765\\
$(30,11)$ & 7 & $(25,7)$ & 7 & 5 & YES & YES & NO(2) & $1.60$ & $(2,3)$ & -- & 3766\\
$(30,11)$ & 7 & $(25,11)$ & 7 & 5 & YES & YES & NO(2) & $1.78$ & $(4,2)$ & NO & 3767\\
$(30,11)$ & 7 & $(25,11)$ & 7 & 5 & YES & YES & NO(2) & $1.78$ & $(4,2)$ & -- & 3768\\
$(30,13)$ & 8 & $(26,11)$ & 7 & 2 & YES & YES & NO(2) & $1.62$ & $(6,1)$ & -- & 3769\\
$(30,11)$ & 7 & $(28,11)$ & 8 & 2 & YES & YES & NO(2) & $1.56$ & $(8,0)$ & -- & 3770\\
$(30,11)$ & 7 & $(29,8)$ & 7 & 1 & YES & YES & NO(2) & $1.44$ & $(8,0)$ & -- & 3771\\
$(30,13)$ & 8 & $(29,8)$ & 7 & 1 & YES & YES & NO(2) & $1.56$ & $(2,3)$ & -- & 3772\\
$(30,13)$ & 8 & $(29,8)$ & 7 & 1 & YES & YES & NO(2) & $1.67$ & $(2,3)$ & NO & 3773\\
$(31,13)$ & 7 & $(13,5)$ & 5 & 1 & YES & YES & NO(2) & $1.73$ & $(4,2)$ & -- & 3774\\
$(31,12)$ & 7 & $(17,7)$ & 6 & 1 & YES & YES & NO(2) & $1.67$ & $(8,0)$ & -- & 3775\\
$(31,13)$ & 7 & $(17,5)$ & 6 & 1 & YES & YES & NO(2) & $1.64$ & $(4,2)$ & -- & 3776\\
$(31,12)$ & 7 & $(18,7)$ & 6 & 1 & YES & YES & NO(2) & $1.70$ & $(10,-1)$ & -- & 3777\\
$(31,12)$ & 7 & $(19,8)$ & 6 & 1 & YES & YES & YES & $1.43$ & $(4,2)$ & -- & 3778\\
$(31,13)$ & 7 & $(20,9)$ & 7 & 1 & YES & YES & NO(2) & $1.67$ & $(8,0)$ & -- & 3779\\
$(31,12)$ & 7 & $(21,8)$ & 6 & 1 & YES & YES & NO(2) & $1.56$ & $(8,0)$ & -- & 3780\\
$(31,13)$ & 7 & $(23,9)$ & 7 & 1 & YES & YES & NO(2) & $1.56$ & $(8,0)$ & -- & 3781\\
$(31,9)$ & 8 & $(25,9)$ & 7 & 1 & YES & YES & YES & $1.43$ & $(2,3)$ & -- & 3782\\
$(31,7)$ & 8 & $(26,11)$ & 7 & 1 & YES & YES & NO(2) & $1.70$ & $(2,3)$ & -- & 3783\\
$(31,12)$ & 7 & $(26,11)$ & 7 & 1 & YES & YES & YES & $1.75$ & $(4,2)$ & -- & 3784\\
$(31,14)$ & 8 & $(26,7)$ & 7 & 1 & YES & YES & NO(2) & $1.62$ & $(6,1)$ & -- & 3785\\
$(31,7)$ & 8 & $(27,11)$ & 8 & 1 & YES & YES & NO(2) & $1.90$ & $(2,3)$ & NO & 3786\\
$(31,12)$ & 7 & $(27,10)$ & 7 & 1 & YES & YES & NO(2) & $1.67$ & $(8,0)$ & NO & 3787\\
$(31,14)$ & 8 & $(27,8)$ & 7 & 1 & YES & YES & NO(2) & $1.50$ & $(6,1)$ & 6968 & 3788\\
$(31,7)$ & 8 & $(28,11)$ & 8 & 1 & YES & YES & NO(2) & $1.44$ & $(8,0)$ & -- & 3789\\
$(31,9)$ & 8 & $(29,9)$ & 8 & 1 & YES & YES & YES & $1.43$ & $(2,3)$ & -- & 3790\\
$(31,12)$ & 7 & $(29,8)$ & 7 & 1 & YES & YES & YES & $1.43$ & $(2,3)$ & -- & 3791\\
$(31,13)$ & 7 & $(29,11)$ & 7 & 1 & YES & YES & YES & $1.78$ & $(2,3)$ & -- & 3792\\
$(31,13)$ & 7 & $(29,12)$ & 7 & 1 & YES & YES & YES & $1.75$ & $(2,3)$ & -- & 3793\\
$(31,12)$ & 7 & $(30,11)$ & 7 & 1 & YES & YES & YES & $2.00$ & $(2,3)$ & -- & 3794\\
$(31,13)$ & 7 & $(30,11)$ & 7 & 1 & YES & YES & YES & $1.78$ & $(2,3)$ & -- & 3795\\
$(31,12)$ & 7 & $(31,12)$ & 7 & 31 & YES & YES & YES & $1.86$ & $(4,2)$ & -- & 3796\\
$(31,13)$ & 7 & $(31,9)$ & 8 & 31 & YES & YES & YES & $1.75$ & $(2,3)$ & -- & 3797\\
$(31,13)$ & 7 & $(31,12)$ & 7 & 31 & YES & YES & YES & $1.78$ & $(2,3)$ & -- & 3798\\
$(32,7)$ & 8 & $(28,11)$ & 8 & 4 & YES & YES & NO(2) & $1.70$ & $(2,3)$ & NO & 3799\\
$(32,7)$ & 8 & $(28,11)$ & 8 & 4 & YES & YES & NO(2) & $1.70$ & $(2,3)$ & -- & 3800\\
$(32,7)$ & 8 & $(29,11)$ & 7 & 1 & YES & YES & NO(2) & $1.60$ & $(2,3)$ & NO & 3801\\
$(33,10)$ & 8 & $(16,5)$ & 7 & 1 & YES & YES & NO(2) & $1.56$ & $(8,0)$ & -- & 3802\\
$(33,10)$ & 8 & $(17,7)$ & 6 & 1 & YES & YES & NO(2) & $1.67$ & $(4,2)$ & NO & 3803\\
$(33,10)$ & 8 & $(17,7)$ & 6 & 1 & YES & YES & NO(2) & $1.67$ & $(8,0)$ & -- & 3804\\
$(33,10)$ & 8 & $(19,8)$ & 6 & 1 & YES & YES & NO(2) & $1.60$ & $(6,1)$ & NO & 3805\\
$(33,14)$ & 8 & $(21,8)$ & 6 & 3 & YES & YES & YES & $1.43$ & $(2,3)$ & -- & 3806\\
$(33,10)$ & 8 & $(23,9)$ & 7 & 1 & YES & YES & NO(2) & $1.50$ & $(4,2)$ & -- & 3807\\
$(33,10)$ & 8 & $(25,7)$ & 7 & 1 & YES & YES & NO(2) & $1.56$ & $(4,2)$ & -- & 3808\\
$(33,10)$ & 8 & $(26,11)$ & 7 & 1 & YES & YES & YES & $1.57$ & $(6,1)$ & -- & 3809\\
$(33,10)$ & 8 & $(27,8)$ & 7 & 3 & YES & YES & YES & $1.86$ & $(6,1)$ & NO & 3810\\
$(33,10)$ & 8 & $(27,8)$ & 7 & 3 & YES & YES & YES & $1.86$ & $(6,1)$ & -- & 3811\\
$(33,10)$ & 8 & $(27,11)$ & 8 & 3 & YES & YES & NO(2) & $1.78$ & $(8,0)$ & NO & 3812\\
$(33,14)$ & 8 & $(27,8)$ & 7 & 3 & YES & YES & YES & $1.57$ & $(2,3)$ & -- & 3813\\
$(33,10)$ & 8 & $(28,11)$ & 8 & 1 & YES & YES & NO(2) & $1.78$ & $(8,0)$ & NO & 3814\\
$(33,10)$ & 8 & $(29,8)$ & 7 & 1 & YES & YES & YES & $1.57$ & $(6,1)$ & NO & 3815\\
$(33,10)$ & 8 & $(29,8)$ & 7 & 1 & YES & YES & YES & $1.57$ & $(6,1)$ & -- & 3816\\
$(33,10)$ & 8 & $(29,12)$ & 7 & 1 & YES & YES & YES & $1.29$ & $(4,2)$ & -- & 3817\\
$(33,10)$ & 8 & $(29,12)$ & 7 & 1 & YES & YES & YES & $1.57$ & $(2,3)$ & NO & 3818\\
$(33,10)$ & 8 & $(29,13)$ & 8 & 1 & YES & YES & NO(2) & $1.67$ & $(8,0)$ & NO & 3819\\
$(33,14)$ & 8 & $(29,8)$ & 7 & 1 & YES & YES & NO(2) & $1.56$ & $(2,3)$ & -- & 3820\\
$(33,10)$ & 8 & $(31,12)$ & 7 & 1 & YES & YES & YES & $1.62$ & $(2,3)$ & -- & 3821\\
$(34,13)$ & 7 & $(10,3)$ & 5 & 2 & YES & YES & YES & $1.50$ & $(2,3)$ & NO & 3822\\
$(34,13)$ & 7 & $(10,3)$ & 5 & 2 & YES & YES & YES & $1.50$ & $(2,3)$ & -- & 3823\\
$(34,13)$ & 7 & $(11,4)$ & 5 & 1 & YES & YES & YES & $1.50$ & $(2,3)$ & NO & 3824\\
$(34,13)$ & 7 & $(11,4)$ & 5 & 1 & YES & YES & YES & $1.50$ & $(2,3)$ & -- & 3825\\
$(34,13)$ & 7 & $(12,5)$ & 5 & 2 & YES & YES & NO(2) & $1.60$ & $(2,3)$ & -- & 3826\\
$(34,13)$ & 7 & $(13,5)$ & 5 & 1 & YES & YES & NO(2) & $1.60$ & $(2,3)$ & -- & 3827\\
$(34,13)$ & 7 & $(14,5)$ & 6 & 2 & YES & YES & NO(2) & $1.70$ & $(2,3)$ & -- & 3828\\
$(34,9)$ & 8 & $(17,5)$ & 6 & 17 & YES & YES & NO(2) & $1.56$ & $(4,2)$ & -- & 3829\\
$(34,9)$ & 8 & $(18,5)$ & 6 & 2 & YES & YES & NO(2) & $1.67$ & $(4,2)$ & -- & 3830\\
$(34,13)$ & 7 & $(18,7)$ & 6 & 2 & YES & YES & NO(2) & $1.73$ & $(4,2)$ & NO & 3831\\
$(34,15)$ & 8 & $(18,5)$ & 6 & 2 & YES & YES & NO(2) & $1.70$ & $(2,3)$ & -- & 3832\\
$(34,13)$ & 7 & $(19,7)$ & 6 & 1 & YES & YES & NO(2) & $1.60$ & $(6,1)$ & -- & 3833\\
$(34,13)$ & 7 & $(19,8)$ & 6 & 1 & YES & YES & YES & $1.43$ & $(2,3)$ & -- & 3834\\
$(34,13)$ & 7 & $(21,8)$ & 6 & 1 & YES & YES & YES & $1.29$ & $(4,2)$ & -- & 3835\\
$(34,9)$ & 8 & $(22,9)$ & 7 & 2 & YES & YES & YES & $1.43$ & $(2,3)$ & -- & 3836\\
$(34,13)$ & 7 & $(22,9)$ & 7 & 2 & YES & YES & YES & $1.43$ & $(2,3)$ & -- & 3837\\
$(34,15)$ & 8 & $(22,5)$ & 7 & 2 & YES & YES & NO(2) & $1.44$ & $(8,0)$ & NO & 3838\\
$(34,15)$ & 8 & $(22,5)$ & 7 & 2 & YES & YES & NO(2) & $1.44$ & $(8,0)$ & -- & 3839\\
$(34,13)$ & 7 & $(23,7)$ & 7 & 1 & YES & YES & YES & $2.00$ & $(6,1)$ & -- & 3840\\
$(34,13)$ & 7 & $(23,10)$ & 7 & 1 & YES & YES & YES & $1.57$ & $(4,2)$ & -- & 3841\\
$(34,15)$ & 8 & $(23,6)$ & 8 & 1 & YES & YES & NO(2) & $1.78$ & $(8,0)$ & NO & 3842\\
$(34,15)$ & 8 & $(23,6)$ & 8 & 1 & YES & YES & NO(2) & $1.78$ & $(8,0)$ & -- & 3843\\
$(34,13)$ & 7 & $(24,7)$ & 7 & 2 & YES & YES & YES & $1.90$ & $(2,3)$ & NO & 3844\\
$(34,13)$ & 7 & $(24,7)$ & 7 & 2 & YES & YES & YES & $1.90$ & $(2,3)$ & -- & 3845\\
$(34,13)$ & 7 & $(25,11)$ & 7 & 1 & YES & YES & NO(2) & $1.73$ & $(4,2)$ & NO & 3846\\
$(34,13)$ & 7 & $(26,11)$ & 7 & 2 & YES & YES & YES & $1.43$ & $(4,2)$ & -- & 3847\\
$(34,13)$ & 7 & $(27,8)$ & 7 & 1 & YES & YES & YES & $1.50$ & $(2,3)$ & -- & 3848\\
$(34,13)$ & 7 & $(27,8)$ & 7 & 1 & YES & YES & YES & $1.67$ & $(2,3)$ & NO & 3849\\
$(34,13)$ & 7 & $(29,8)$ & 7 & 1 & YES & YES & YES & $1.50$ & $(4,2)$ & -- & 3850\\
$(34,13)$ & 7 & $(30,11)$ & 7 & 2 & YES & YES & YES & $1.62$ & $(4,2)$ & -- & 3851\\
$(34,9)$ & 8 & $(31,9)$ & 8 & 1 & YES & YES & YES & $1.57$ & $(2,3)$ & -- & 3852\\
$(34,13)$ & 7 & $(31,12)$ & 7 & 1 & YES & YES & YES & $1.29$ & $(4,2)$ & -- & 3853\\
$(34,13)$ & 7 & $(31,13)$ & 7 & 1 & YES & YES & YES & $1.62$ & $(2,3)$ & -- & 3854\\
$(34,13)$ & 7 & $(33,10)$ & 8 & 1 & YES & YES & YES & $1.88$ & $(2,3)$ & -- & 3855\\
$(34,13)$ & 7 & $(34,13)$ & 7 & 34 & YES & YES & YES & $1.62$ & $(4,2)$ & -- & 3856\\
$(35,11)$ & 9 & $(17,5)$ & 6 & 1 & YES & YES & NO(2) & $1.80$ & $(6,1)$ & NO & 3857\\
$(35,11)$ & 9 & $(17,5)$ & 6 & 1 & YES & YES & NO(2) & $1.80$ & $(6,1)$ & -- & 3858\\
$(35,8)$ & 8 & $(19,5)$ & 7 & 1 & YES & YES & NO(2) & $1.60$ & $(6,1)$ & NO & 3859\\
$(35,8)$ & 8 & $(19,5)$ & 7 & 1 & YES & YES & NO(2) & $1.60$ & $(6,1)$ & -- & 3860\\
$(35,8)$ & 8 & $(19,6)$ & 8 & 1 & YES & YES & NO(2) & $1.50$ & $(6,1)$ & -- & 3861\\
$(35,8)$ & 8 & $(20,9)$ & 7 & 5 & YES & YES & NO(2) & $1.60$ & $(6,1)$ & NO & 3862\\
$(35,8)$ & 8 & $(20,9)$ & 7 & 5 & YES & YES & NO(2) & $1.60$ & $(6,1)$ & -- & 3863\\
$(35,13)$ & 8 & $(25,7)$ & 7 & 5 & YES & YES & NO(2) & $1.60$ & $(6,1)$ & -- & 3864\\
$(35,8)$ & 8 & $(26,11)$ & 7 & 1 & YES & YES & NO(2) & $1.44$ & $(4,2)$ & -- & 3865\\
$(35,8)$ & 8 & $(27,10)$ & 7 & 1 & YES & YES & NO(2) & $1.38$ & $(6,1)$ & -- & 3866\\
$(35,8)$ & 8 & $(27,11)$ & 8 & 1 & YES & YES & YES & $1.57$ & $(2,3)$ & NO & 3867\\
$(35,16)$ & 9 & $(27,8)$ & 7 & 1 & YES & YES & NO(2) & $1.50$ & $(6,1)$ & NO & 3868\\
$(35,8)$ & 8 & $(29,11)$ & 7 & 1 & YES & YES & NO(2) & $1.56$ & $(2,3)$ & -- & 3869\\
$(35,13)$ & 8 & $(33,7)$ & 8 & 1 & YES & YES & NO(2) & $1.50$ & $(4,2)$ & NO & 3870\\
$(35,8)$ & 8 & $(34,15)$ & 8 & 1 & YES & YES & NO(2) & $1.73$ & $(4,2)$ & -- & 3871\\
$(35,13)$ & 8 & $(35,8)$ & 8 & 35 & YES & YES & YES & $1.43$ & $(4,2)$ & -- & 3872\\
$(36,13)$ & 8 & $(17,7)$ & 6 & 1 & YES & YES & YES & $1.75$ & $(2,3)$ & NO & 3873\\
$(36,13)$ & 8 & $(17,7)$ & 6 & 1 & YES & YES & YES & $1.75$ & $(2,3)$ & -- & 3874\\
$(36,13)$ & 8 & $(19,5)$ & 7 & 1 & YES & YES & YES & $1.50$ & $(2,3)$ & -- & 3875\\
$(36,11)$ & 8 & $(20,9)$ & 7 & 4 & YES & YES & NO(2) & $1.67$ & $(4,2)$ & -- & 3876\\
$(36,11)$ & 8 & $(23,9)$ & 7 & 1 & YES & YES & NO(2) & $1.50$ & $(4,2)$ & -- & 3877\\
$(36,13)$ & 8 & $(23,10)$ & 7 & 1 & YES & YES & NO(2) & $1.56$ & $(8,0)$ & -- & 3878\\
$(36,11)$ & 8 & $(24,7)$ & 7 & 12 & YES & YES & NO(2) & $1.55$ & $(4,2)$ & -- & 3879\\
$(36,11)$ & 8 & $(25,7)$ & 7 & 1 & YES & YES & NO(2) & $1.64$ & $(4,2)$ & -- & 3880\\
$(36,13)$ & 8 & $(29,8)$ & 7 & 1 & YES & YES & YES & $1.57$ & $(4,2)$ & -- & 3881\\
$(36,13)$ & 8 & $(29,13)$ & 8 & 1 & YES & YES & NO(2) & $1.67$ & $(8,0)$ & NO & 3882\\
$(36,11)$ & 8 & $(34,7)$ & 10 & 2 & YES & YES & NO(2) & $1.50$ & $(6,1)$ & NO & 3883\\
$(36,11)$ & 8 & $(35,8)$ & 8 & 1 & YES & YES & NO(2) & $1.64$ & $(4,2)$ & -- & 3884\\
$(37,11)$ & 8 & $(16,7)$ & 6 & 1 & YES & YES & NO(2) & $1.56$ & $(8,0)$ & -- & 3885\\
$(37,14)$ & 8 & $(18,5)$ & 6 & 1 & YES & YES & NO(2) & $1.64$ & $(4,2)$ & -- & 3886\\
$(37,8)$ & 8 & $(19,6)$ & 8 & 1 & YES & YES & NO(2) & $1.50$ & $(6,1)$ & -- & 3887\\
$(37,11)$ & 8 & $(19,8)$ & 6 & 1 & YES & YES & YES & $1.29$ & $(2,3)$ & -- & 3888\\
$(37,11)$ & 8 & $(19,8)$ & 6 & 1 & YES & YES & NO(2) & $1.78$ & $(6,1)$ & NO & 3889\\
$(37,8)$ & 8 & $(23,9)$ & 7 & 1 & YES & YES & NO(2) & $1.56$ & $(2,3)$ & -- & 3890\\
$(37,8)$ & 8 & $(23,9)$ & 7 & 1 & YES & YES & NO(2) & $1.60$ & $(6,1)$ & NO & 3891\\
$(37,11)$ & 8 & $(23,9)$ & 7 & 1 & YES & YES & NO(2) & $1.67$ & $(2,3)$ & -- & 3892\\
$(37,14)$ & 8 & $(23,10)$ & 7 & 1 & YES & YES & NO(2) & $1.50$ & $(6,1)$ & -- & 3893\\
$(37,11)$ & 8 & $(24,7)$ & 7 & 1 & YES & YES & YES & $1.62$ & $(2,3)$ & -- & 3894\\
$(37,14)$ & 8 & $(24,7)$ & 7 & 1 & YES & YES & NO(2) & $1.80$ & $(2,3)$ & NO & 3895\\
$(37,10)$ & 8 & $(25,11)$ & 7 & 1 & YES & YES & NO(2) & $1.56$ & $(8,0)$ & -- & 3896\\
$(37,8)$ & 8 & $(26,11)$ & 7 & 1 & YES & YES & YES & $1.43$ & $(2,3)$ & -- & 3897\\
$(37,11)$ & 8 & $(26,11)$ & 7 & 1 & YES & YES & YES & $1.56$ & $(4,2)$ & -- & 3898\\
$(37,8)$ & 8 & $(27,10)$ & 7 & 1 & YES & YES & NO(2) & $1.56$ & $(4,2)$ & NO & 3899\\
$(37,10)$ & 8 & $(27,11)$ & 8 & 1 & YES & YES & NO(2) & $1.50$ & $(6,1)$ & NO & 3900\\
$(37,11)$ & 8 & $(27,8)$ & 7 & 1 & YES & YES & YES & $1.71$ & $(4,2)$ & -- & 3901\\
$(37,14)$ & 8 & $(27,5)$ & 8 & 1 & YES & YES & NO(2) & $1.67$ & $(4,2)$ & NO & 3902\\
$(37,14)$ & 8 & $(27,8)$ & 7 & 1 & YES & YES & NO(2) & $1.56$ & $(2,3)$ & -- & 3903\\
$(37,8)$ & 8 & $(29,11)$ & 7 & 1 & YES & YES & NO(2) & $1.56$ & $(2,3)$ & -- & 3904\\
$(37,8)$ & 8 & $(29,11)$ & 7 & 1 & YES & YES & NO(2) & $1.60$ & $(6,1)$ & NO & 3905\\
$(37,11)$ & 8 & $(30,7)$ & 8 & 1 & YES & YES & YES & $1.43$ & $(2,3)$ & -- & 3906\\
$(37,11)$ & 8 & $(30,11)$ & 7 & 1 & YES & YES & YES & $1.78$ & $(2,3)$ & -- & 3907\\
$(37,8)$ & 8 & $(31,14)$ & 8 & 1 & YES & YES & NO(2) & $1.62$ & $(6,1)$ & NO & 3908\\
$(37,10)$ & 8 & $(31,13)$ & 7 & 1 & YES & YES & YES & $1.62$ & $(2,3)$ & -- & 3909\\
$(37,10)$ & 8 & $(31,13)$ & 7 & 1 & YES & YES & YES & $1.78$ & $(2,3)$ & NO & 3910\\
$(37,11)$ & 8 & $(31,7)$ & 8 & 1 & YES & YES & YES & $1.50$ & $(2,3)$ & -- & 3911\\
$(37,11)$ & 8 & $(31,7)$ & 8 & 1 & YES & YES & YES & $1.90$ & $(2,3)$ & NO & 3912\\
$(37,11)$ & 8 & $(31,12)$ & 7 & 1 & YES & YES & YES & $1.89$ & $(2,3)$ & -- & 3913\\
$(37,14)$ & 8 & $(31,7)$ & 8 & 1 & YES & YES & YES & $1.62$ & $(2,3)$ & -- & 3914\\
$(37,11)$ & 8 & $(32,7)$ & 8 & 1 & YES & YES & NO(2) & $1.64$ & $(4,2)$ & NO & 3915\\
$(37,11)$ & 8 & $(32,9)$ & 8 & 1 & YES & YES & YES & $1.56$ & $(4,2)$ & -- & 3916\\
$(37,14)$ & 8 & $(32,7)$ & 8 & 1 & YES & YES & YES & $1.62$ & $(2,3)$ & -- & 3917\\
$(37,8)$ & 8 & $(33,10)$ & 8 & 1 & YES & YES & NO(2) & $1.60$ & $(2,3)$ & NO & 3918\\
$(37,10)$ & 8 & $(33,10)$ & 8 & 1 & YES & YES & YES & $1.57$ & $(6,1)$ & NO & 3919\\
$(37,10)$ & 8 & $(33,10)$ & 8 & 1 & YES & YES & YES & $1.57$ & $(6,1)$ & -- & 3920\\
$(37,11)$ & 8 & $(33,10)$ & 8 & 1 & YES & YES & YES & $1.62$ & $(4,2)$ & -- & 3921\\
$(37,10)$ & 8 & $(34,13)$ & 7 & 1 & YES & YES & YES & $1.90$ & $(2,3)$ & -- & 3922\\
$(37,11)$ & 8 & $(34,13)$ & 7 & 1 & YES & YES & YES & $1.57$ & $(2,3)$ & -- & 3923\\
$(37,14)$ & 8 & $(35,8)$ & 8 & 1 & YES & YES & YES & $1.75$ & $(4,2)$ & -- & 3924\\
$(37,14)$ & 8 & $(37,8)$ & 8 & 37 & YES & YES & YES & $1.57$ & $(4,2)$ & NO & 3925\\
$(38,11)$ & 9 & $(18,5)$ & 6 & 2 & YES & YES & NO(2) & $1.60$ & $(2,3)$ & -- & 3926\\
$(38,7)$ & 9 & $(28,11)$ & 8 & 2 & YES & YES & NO(2) & $1.44$ & $(8,0)$ & -- & 3927\\
$(38,11)$ & 9 & $(28,5)$ & 8 & 2 & YES & YES & NO(2) & $1.60$ & $(2,3)$ & NO & 3928\\
$(38,11)$ & 9 & $(30,11)$ & 7 & 2 & YES & YES & YES & $1.78$ & $(4,2)$ & -- & 3929\\
$(38,7)$ & 9 & $(31,9)$ & 8 & 1 & YES & YES & NO(2) & $1.60$ & $(2,3)$ & NO & 3930\\
$(38,11)$ & 9 & $(31,7)$ & 8 & 1 & YES & YES & NO(2) & $1.70$ & $(2,3)$ & NO & 3931\\
$(39,11)$ & 9 & $(13,5)$ & 5 & 13 & YES & YES & YES & $1.29$ & $(4,2)$ & NO & 3932\\
$(39,16)$ & 8 & $(13,5)$ & 5 & 13 & YES & YES & YES & $1.43$ & $(2,3)$ & -- & 3933\\
$(39,11)$ & 9 & $(17,7)$ & 6 & 1 & YES & YES & NO(2) & $1.80$ & $(2,3)$ & -- & 3934\\
$(39,14)$ & 8 & $(19,5)$ & 7 & 1 & YES & YES & YES & $1.50$ & $(2,3)$ & -- & 3935\\
$(39,16)$ & 8 & $(21,8)$ & 6 & 3 & YES & YES & YES & $1.91$ & $(2,3)$ & -- & 3936\\
$(39,16)$ & 8 & $(23,7)$ & 7 & 1 & YES & YES & NO(2) & $1.38$ & $(4,2)$ & -- & 3937\\
$(39,16)$ & 8 & $(23,7)$ & 7 & 1 & YES & YES & NO(2) & $1.62$ & $(4,2)$ & NO & 3938\\
$(39,17)$ & 8 & $(23,7)$ & 7 & 1 & YES & YES & NO(2) & $1.33$ & $(8,0)$ & -- & 3939\\
$(39,14)$ & 8 & $(24,7)$ & 7 & 3 & YES & YES & YES & $1.57$ & $(2,3)$ & -- & 3940\\
$(39,11)$ & 9 & $(27,10)$ & 7 & 3 & YES & YES & NO(2) & $1.56$ & $(8,0)$ & NO & 3941\\
$(39,16)$ & 8 & $(29,11)$ & 7 & 1 & YES & YES & YES & $1.78$ & $(4,2)$ & -- & 3942\\
$(39,7)$ & 9 & $(31,9)$ & 8 & 1 & YES & YES & NO(2) & $1.60$ & $(2,3)$ & -- & 3943\\
$(39,14)$ & 8 & $(31,9)$ & 8 & 1 & YES & YES & YES & $1.78$ & $(4,2)$ & -- & 3944\\
$(39,16)$ & 8 & $(31,12)$ & 7 & 1 & YES & YES & YES & $1.71$ & $(6,1)$ & NO & 3945\\
$(39,16)$ & 8 & $(31,12)$ & 7 & 1 & YES & YES & YES & $1.71$ & $(6,1)$ & -- & 3946\\
$(39,7)$ & 9 & $(37,13)$ & 9 & 1 & YES & YES & NO(2) & $1.50$ & $(6,1)$ & NO & 3947\\
$(39,11)$ & 9 & $(38,7)$ & 9 & 1 & YES & YES & YES & $1.57$ & $(2,3)$ & NO & 3948\\
$(40,11)$ & 8 & $(17,7)$ & 6 & 1 & YES & YES & NO(2) & $1.38$ & $(6,1)$ & -- & 3949\\
$(40,17)$ & 9 & $(17,5)$ & 6 & 1 & YES & YES & NO(2) & $1.80$ & $(2,3)$ & -- & 3950\\
$(40,17)$ & 9 & $(18,5)$ & 6 & 2 & YES & YES & NO(2) & $1.70$ & $(2,3)$ & -- & 3951\\
$(40,11)$ & 8 & $(19,8)$ & 6 & 1 & YES & YES & NO(2) & $1.60$ & $(2,3)$ & -- & 3952\\
$(40,11)$ & 8 & $(19,8)$ & 6 & 1 & YES & YES & NO(2) & $1.70$ & $(2,3)$ & NO & 3953\\
$(40,17)$ & 9 & $(19,7)$ & 6 & 1 & YES & YES & NO(2) & $1.75$ & $(6,1)$ & -- & 3954\\
$(40,11)$ & 8 & $(21,8)$ & 6 & 1 & YES & YES & YES & $1.29$ & $(4,2)$ & -- & 3955\\
$(40,17)$ & 9 & $(21,8)$ & 6 & 1 & YES & YES & NO(2) & $1.62$ & $(6,1)$ & -- & 3956\\
$(40,9)$ & 9 & $(23,9)$ & 7 & 1 & YES & YES & NO(2) & $1.56$ & $(4,2)$ & -- & 3957\\
$(40,9)$ & 9 & $(24,7)$ & 7 & 8 & YES & YES & YES & $1.14$ & $(4,2)$ & -- & 3958\\
$(40,11)$ & 8 & $(24,7)$ & 7 & 8 & YES & YES & YES & $1.38$ & $(4,2)$ & -- & 3959\\
$(40,11)$ & 8 & $(25,7)$ & 7 & 5 & YES & YES & YES & $1.50$ & $(2,3)$ & -- & 3960\\
$(40,11)$ & 8 & $(25,11)$ & 7 & 5 & YES & YES & YES & $1.57$ & $(2,3)$ & -- & 3961\\
$(40,11)$ & 8 & $(27,8)$ & 7 & 1 & YES & YES & YES & $1.50$ & $(2,3)$ & -- & 3962\\
$(40,11)$ & 8 & $(27,10)$ & 7 & 1 & YES & YES & YES & $1.78$ & $(2,3)$ & -- & 3963\\
$(40,11)$ & 8 & $(29,8)$ & 7 & 1 & YES & YES & YES & $1.62$ & $(2,3)$ & -- & 3964\\
$(40,11)$ & 8 & $(30,11)$ & 7 & 10 & YES & YES & YES & $1.67$ & $(4,2)$ & -- & 3965\\
$(40,11)$ & 8 & $(31,7)$ & 8 & 1 & YES & YES & YES & $1.56$ & $(2,3)$ & -- & 3966\\
$(40,11)$ & 8 & $(31,9)$ & 8 & 1 & YES & YES & YES & $1.80$ & $(2,3)$ & -- & 3967\\
$(40,11)$ & 8 & $(32,9)$ & 8 & 8 & YES & YES & YES & $1.80$ & $(2,3)$ & -- & 3968\\
$(40,11)$ & 8 & $(33,10)$ & 8 & 1 & YES & YES & YES & $1.67$ & $(2,3)$ & -- & 3969\\
$(40,11)$ & 8 & $(34,13)$ & 7 & 2 & YES & YES & YES & $1.57$ & $(2,3)$ & -- & 3970\\
$(40,11)$ & 8 & $(37,10)$ & 8 & 1 & YES & YES & YES & $1.57$ & $(2,3)$ & -- & 3971\\
$(40,11)$ & 8 & $(40,11)$ & 8 & 40 & YES & YES & YES & $1.71$ & $(2,3)$ & -- & 3972\\
$(41,17)$ & 8 & $(10,3)$ & 5 & 1 & YES & YES & NO(2) & $1.44$ & $(4,2)$ & -- & 3973\\
$(41,12)$ & 8 & $(12,5)$ & 5 & 1 & YES & YES & NO(2) & $1.50$ & $(2,3)$ & -- & 3974\\
$(41,12)$ & 8 & $(12,5)$ & 5 & 1 & YES & YES & NO(2) & $1.44$ & $(8,0)$ & NO & 3975\\
$(41,16)$ & 8 & $(12,5)$ & 5 & 1 & YES & YES & YES & $1.43$ & $(2,3)$ & -- & 3976\\
$(41,12)$ & 8 & $(13,5)$ & 5 & 1 & YES & YES & NO(2) & $1.50$ & $(6,1)$ & NO & 3977\\
$(41,12)$ & 8 & $(13,5)$ & 5 & 1 & YES & YES & NO(2) & $1.50$ & $(6,1)$ & -- & 3978\\
$(41,15)$ & 8 & $(13,5)$ & 5 & 1 & YES & YES & NO(2) & $1.60$ & $(2,3)$ & -- & 3979\\
$(41,17)$ & 8 & $(13,4)$ & 6 & 1 & YES & YES & NO(2) & $1.56$ & $(4,2)$ & NO & 3980\\
$(41,17)$ & 8 & $(13,4)$ & 6 & 1 & YES & YES & NO(2) & $1.56$ & $(4,2)$ & -- & 3981\\
$(41,11)$ & 8 & $(14,5)$ & 6 & 1 & YES & YES & NO(2) & $1.67$ & $(4,2)$ & -- & 3982\\
$(41,11)$ & 8 & $(15,4)$ & 6 & 1 & YES & YES & NO(2) & $1.60$ & $(2,3)$ & NO & 3983\\
$(41,11)$ & 8 & $(15,4)$ & 6 & 1 & YES & YES & NO(2) & $1.60$ & $(2,3)$ & -- & 3984\\
$(41,16)$ & 8 & $(15,4)$ & 6 & 1 & YES & YES & NO(2) & $1.33$ & $(4,2)$ & -- & 3985\\
$(41,12)$ & 8 & $(16,5)$ & 7 & 1 & YES & YES & YES & $1.43$ & $(4,2)$ & NO & 3986\\
$(41,12)$ & 8 & $(16,5)$ & 7 & 1 & YES & YES & YES & $1.43$ & $(4,2)$ & -- & 3987\\
$(41,12)$ & 8 & $(17,7)$ & 6 & 1 & YES & YES & NO(2) & $1.38$ & $(4,2)$ & NO & 3988\\
$(41,12)$ & 8 & $(17,7)$ & 6 & 1 & YES & YES & NO(2) & $1.38$ & $(4,2)$ & -- & 3989\\
$(41,16)$ & 8 & $(17,5)$ & 6 & 1 & YES & YES & NO(2) & $1.60$ & $(2,3)$ & -- & 3990\\
$(41,18)$ & 8 & $(17,5)$ & 6 & 1 & YES & YES & NO(2) & $1.64$ & $(4,2)$ & -- & 3991\\
$(41,18)$ & 8 & $(17,7)$ & 6 & 1 & YES & YES & NO(2) & $1.56$ & $(12,-2)$ & -- & 3992\\
$(41,15)$ & 8 & $(18,7)$ & 6 & 1 & YES & YES & NO(2) & $1.50$ & $(4,2)$ & -- & 3993\\
$(41,16)$ & 8 & $(18,5)$ & 6 & 1 & YES & YES & NO(2) & $1.56$ & $(4,2)$ & NO & 3994\\
$(41,16)$ & 8 & $(18,5)$ & 6 & 1 & YES & YES & NO(2) & $1.56$ & $(4,2)$ & -- & 3995\\
$(41,12)$ & 8 & $(19,7)$ & 6 & 1 & YES & YES & NO(2) & $1.70$ & $(6,1)$ & -- & 3996\\
$(41,15)$ & 8 & $(19,7)$ & 6 & 1 & YES & YES & NO(2) & $1.56$ & $(8,0)$ & -- & 3997\\
$(41,16)$ & 8 & $(19,8)$ & 6 & 1 & YES & YES & NO(2) & $1.38$ & $(4,2)$ & -- & 3998\\
$(41,11)$ & 8 & $(20,7)$ & 8 & 1 & YES & YES & NO(2) & $1.62$ & $(6,1)$ & -- & 3999\\
$(41,11)$ & 8 & $(21,8)$ & 6 & 1 & YES & YES & NO(2) & $1.55$ & $(4,2)$ & -- & 4000\\
$(41,16)$ & 8 & $(21,8)$ & 6 & 1 & YES & YES & YES & $1.88$ & $(4,2)$ & -- & 4001\\
$(41,18)$ & 8 & $(21,8)$ & 6 & 1 & YES & YES & YES & $1.57$ & $(4,2)$ & -- & 4002\\
$(41,11)$ & 8 & $(23,9)$ & 7 & 1 & YES & YES & NO(2) & $1.50$ & $(4,2)$ & -- & 4003\\
$(41,12)$ & 8 & $(23,9)$ & 7 & 1 & YES & YES & NO(2) & $1.56$ & $(2,3)$ & -- & 4004\\
$(41,15)$ & 8 & $(23,6)$ & 8 & 1 & YES & YES & NO(2) & $1.67$ & $(8,0)$ & NO & 4005\\
$(41,18)$ & 8 & $(23,7)$ & 7 & 1 & YES & YES & NO(2) & $1.33$ & $(8,0)$ & -- & 4006\\
$(41,12)$ & 8 & $(24,7)$ & 7 & 1 & YES & YES & YES & $1.56$ & $(2,3)$ & -- & 4007\\
$(41,15)$ & 8 & $(24,7)$ & 7 & 1 & YES & YES & YES & $1.75$ & $(2,3)$ & -- & 4008\\
$(41,15)$ & 8 & $(25,11)$ & 7 & 1 & YES & YES & YES & $1.57$ & $(2,3)$ & NO & 4009\\
$(41,17)$ & 8 & $(25,7)$ & 7 & 1 & YES & YES & YES & $1.67$ & $(2,3)$ & -- & 4010\\
$(41,12)$ & 8 & $(26,7)$ & 7 & 1 & YES & YES & YES & $1.43$ & $(2,3)$ & -- & 4011\\
$(41,15)$ & 8 & $(27,8)$ & 7 & 1 & YES & YES & YES & $1.62$ & $(4,2)$ & -- & 4012\\
$(41,16)$ & 8 & $(27,8)$ & 7 & 1 & YES & YES & YES & $1.43$ & $(4,2)$ & -- & 4013\\
$(41,11)$ & 8 & $(29,11)$ & 7 & 1 & YES & YES & YES & $1.62$ & $(2,3)$ & -- & 4014\\
$(41,12)$ & 8 & $(29,9)$ & 8 & 1 & YES & YES & YES & $1.43$ & $(4,2)$ & NO & 4015\\
$(41,12)$ & 8 & $(29,9)$ & 8 & 1 & YES & YES & NO(2) & $1.56$ & $(4,2)$ & -- & 4016\\
$(41,12)$ & 8 & $(29,12)$ & 7 & 1 & YES & YES & YES & $1.89$ & $(2,3)$ & -- & 4017\\
$(41,15)$ & 8 & $(29,8)$ & 7 & 1 & YES & YES & YES & $1.62$ & $(4,2)$ & -- & 4018\\
$(41,17)$ & 8 & $(29,8)$ & 7 & 1 & YES & YES & YES & $1.62$ & $(2,3)$ & -- & 4019\\
$(41,17)$ & 8 & $(29,8)$ & 7 & 1 & YES & YES & YES & $1.62$ & $(2,3)$ & NO & 4020\\
$(41,12)$ & 8 & $(31,9)$ & 8 & 1 & YES & YES & YES & $1.29$ & $(4,2)$ & -- & 4021\\
$(41,12)$ & 8 & $(31,12)$ & 7 & 1 & YES & YES & YES & $1.75$ & $(2,3)$ & -- & 4022\\
$(41,17)$ & 8 & $(31,12)$ & 7 & 1 & YES & YES & YES & $1.75$ & $(4,2)$ & -- & 4023\\
$(41,12)$ & 8 & $(34,13)$ & 7 & 1 & YES & YES & YES & $1.71$ & $(2,3)$ & -- & 4024\\
$(41,15)$ & 8 & $(35,8)$ & 8 & 1 & YES & YES & YES & $1.75$ & $(2,3)$ & -- & 4025\\
$(41,12)$ & 8 & $(36,11)$ & 8 & 1 & YES & YES & YES & $1.75$ & $(2,3)$ & -- & 4026\\
$(41,17)$ & 8 & $(36,13)$ & 8 & 1 & YES & YES & NO(2) & $1.50$ & $(6,1)$ & NO & 4027\\
$(41,16)$ & 8 & $(41,12)$ & 8 & 41 & YES & YES & YES & $1.89$ & $(4,2)$ & NO & 4028\\
$(41,16)$ & 8 & $(41,15)$ & 8 & 41 & YES & YES & NO(2) & $1.50$ & $(4,2)$ & NO & 4029\\
$(42,11)$ & 9 & $(17,7)$ & 6 & 1 & YES & YES & NO(2) & $1.67$ & $(4,2)$ & NO & 4030\\
$(42,11)$ & 9 & $(17,7)$ & 6 & 1 & YES & YES & NO(2) & $1.67$ & $(4,2)$ & -- & 4031\\
$(42,13)$ & 9 & $(17,6)$ & 7 & 1 & YES & YES & NO(2) & $1.78$ & $(8,0)$ & -- & 4032\\
$(42,13)$ & 9 & $(18,5)$ & 6 & 6 & YES & YES & NO(2) & $1.60$ & $(2,3)$ & -- & 4033\\
$(42,11)$ & 9 & $(19,8)$ & 6 & 1 & YES & YES & NO(2) & $1.56$ & $(4,2)$ & -- & 4034\\
$(42,11)$ & 9 & $(19,8)$ & 6 & 1 & YES & YES & NO(2) & $1.67$ & $(4,2)$ & NO & 4035\\
$(42,13)$ & 9 & $(29,12)$ & 7 & 1 & YES & YES & NO(2) & $1.50$ & $(6,1)$ & -- & 4036\\
$(42,11)$ & 9 & $(33,10)$ & 8 & 3 & YES & YES & NO(2) & $1.56$ & $(4,2)$ & NO & 4037\\
$(42,11)$ & 9 & $(36,11)$ & 8 & 6 & YES & YES & NO(2) & $1.56$ & $(4,2)$ & NO & 4038\\
$(42,11)$ & 9 & $(39,7)$ & 9 & 3 & YES & YES & YES & $1.43$ & $(4,2)$ & -- & 4039\\
$(42,13)$ & 9 & $(39,11)$ & 9 & 3 & YES & YES & NO(2) & $1.67$ & $(8,0)$ & NO & 4040\\
$(43,12)$ & 8 & $(12,5)$ & 5 & 1 & YES & YES & NO(2) & $1.56$ & $(4,2)$ & -- & 4041\\
$(43,12)$ & 8 & $(14,5)$ & 6 & 1 & YES & YES & NO(2) & $1.50$ & $(6,1)$ & -- & 4042\\
$(43,12)$ & 8 & $(17,5)$ & 6 & 1 & YES & YES & YES & $1.75$ & $(2,3)$ & NO & 4043\\
$(43,12)$ & 8 & $(17,5)$ & 6 & 1 & YES & YES & YES & $1.75$ & $(2,3)$ & -- & 4044\\
$(43,12)$ & 8 & $(17,7)$ & 6 & 1 & YES & YES & NO(2) & $1.60$ & $(6,1)$ & -- & 4045\\
$(43,19)$ & 9 & $(17,5)$ & 6 & 1 & YES & YES & NO(2) & $1.70$ & $(2,3)$ & -- & 4046\\
$(43,12)$ & 8 & $(18,7)$ & 6 & 1 & YES & YES & YES & $1.50$ & $(2,3)$ & -- & 4047\\
$(43,13)$ & 9 & $(19,7)$ & 6 & 1 & YES & YES & NO(2) & $1.62$ & $(6,1)$ & NO & 4048\\
$(43,13)$ & 9 & $(19,7)$ & 6 & 1 & YES & YES & NO(2) & $1.62$ & $(6,1)$ & -- & 4049\\
$(43,18)$ & 8 & $(19,8)$ & 6 & 1 & YES & YES & YES & $1.62$ & $(2,3)$ & -- & 4050\\
$(43,12)$ & 8 & $(21,8)$ & 6 & 1 & YES & YES & YES & $1.29$ & $(4,2)$ & -- & 4051\\
$(43,12)$ & 8 & $(21,8)$ & 6 & 1 & YES & YES & YES & $1.67$ & $(2,3)$ & NO & 4052\\
$(43,13)$ & 9 & $(21,8)$ & 6 & 1 & YES & YES & YES & $1.82$ & $(2,3)$ & -- & 4053\\
$(43,18)$ & 8 & $(21,8)$ & 6 & 1 & YES & YES & NO(2) & $1.44$ & $(4,2)$ & NO & 4054\\
$(43,18)$ & 8 & $(21,8)$ & 6 & 1 & YES & YES & YES & $1.62$ & $(2,3)$ & -- & 4055\\
$(43,19)$ & 9 & $(21,8)$ & 6 & 1 & YES & YES & NO(2) & $1.70$ & $(2,3)$ & NO & 4056\\
$(43,12)$ & 8 & $(22,9)$ & 7 & 1 & YES & YES & NO(2) & $1.67$ & $(8,0)$ & NO & 4057\\
$(43,12)$ & 8 & $(22,9)$ & 7 & 1 & YES & YES & YES & $1.43$ & $(6,1)$ & NO & 4058\\
$(43,12)$ & 8 & $(22,9)$ & 7 & 1 & YES & YES & YES & $1.43$ & $(6,1)$ & -- & 4059\\
$(43,18)$ & 8 & $(22,5)$ & 7 & 1 & YES & YES & NO(2) & $1.67$ & $(2,3)$ & -- & 4060\\
$(43,18)$ & 8 & $(22,5)$ & 7 & 1 & YES & YES & YES & $1.43$ & $(4,2)$ & NO & 4061\\
$(43,10)$ & 9 & $(23,6)$ & 8 & 1 & YES & YES & NO(2) & $1.50$ & $(6,1)$ & -- & 4062\\
$(43,12)$ & 8 & $(23,9)$ & 7 & 1 & YES & YES & NO(2) & $1.50$ & $(4,2)$ & NO & 4063\\
$(43,18)$ & 8 & $(23,10)$ & 7 & 1 & YES & YES & YES & $1.86$ & $(6,1)$ & -- & 4064\\
$(43,12)$ & 8 & $(25,7)$ & 7 & 1 & YES & YES & YES & $1.50$ & $(2,3)$ & -- & 4065\\
$(43,18)$ & 8 & $(25,9)$ & 7 & 1 & YES & YES & YES & $1.89$ & $(4,2)$ & -- & 4066\\
$(43,10)$ & 9 & $(29,12)$ & 7 & 1 & YES & YES & NO(2) & $1.56$ & $(4,2)$ & NO & 4067\\
$(43,12)$ & 8 & $(29,8)$ & 7 & 1 & YES & YES & YES & $1.67$ & $(2,3)$ & -- & 4068\\
$(43,12)$ & 8 & $(29,11)$ & 7 & 1 & YES & YES & YES & $1.78$ & $(2,3)$ & -- & 4069\\
$(43,12)$ & 8 & $(29,12)$ & 7 & 1 & YES & YES & YES & $1.62$ & $(4,2)$ & -- & 4070\\
$(43,18)$ & 8 & $(29,8)$ & 7 & 1 & YES & YES & YES & $1.75$ & $(4,2)$ & NO & 4071\\
$(43,19)$ & 9 & $(29,12)$ & 7 & 1 & YES & YES & NO(2) & $1.70$ & $(2,3)$ & NO & 4072\\
$(43,18)$ & 8 & $(30,11)$ & 7 & 1 & YES & YES & YES & $1.89$ & $(4,2)$ & -- & 4073\\
$(43,8)$ & 9 & $(31,11)$ & 8 & 1 & YES & YES & YES & $1.43$ & $(4,2)$ & NO & 4074\\
$(43,12)$ & 8 & $(31,9)$ & 8 & 1 & YES & YES & YES & $1.67$ & $(4,2)$ & -- & 4075\\
$(43,12)$ & 8 & $(33,10)$ & 8 & 1 & YES & YES & YES & $1.50$ & $(4,2)$ & -- & 4076\\
$(43,12)$ & 8 & $(34,9)$ & 8 & 1 & YES & YES & NO(2) & $1.56$ & $(4,2)$ & NO & 4077\\
$(43,12)$ & 8 & $(34,13)$ & 7 & 1 & YES & YES & YES & $1.71$ & $(2,3)$ & -- & 4078\\
$(43,18)$ & 8 & $(35,8)$ & 8 & 1 & YES & YES & YES & $1.71$ & $(2,3)$ & -- & 4079\\
$(43,10)$ & 9 & $(37,11)$ & 8 & 1 & YES & YES & YES & $1.43$ & $(2,3)$ & -- & 4080\\
$(43,13)$ & 9 & $(37,8)$ & 8 & 1 & YES & YES & YES & $1.91$ & $(2,3)$ & NO & 4081\\
$(43,18)$ & 8 & $(37,8)$ & 8 & 1 & YES & YES & YES & $1.75$ & $(2,3)$ & NO & 4082\\
$(43,18)$ & 8 & $(37,8)$ & 8 & 1 & YES & YES & YES & $1.75$ & $(2,3)$ & -- & 4083\\
$(43,18)$ & 8 & $(39,14)$ & 8 & 1 & YES & YES & YES & $1.78$ & $(4,2)$ & NO & 4084\\
$(43,13)$ & 9 & $(40,11)$ & 8 & 1 & YES & YES & NO(2) & $1.56$ & $(4,2)$ & NO & 4085\\
$(44,17)$ & 8 & $(10,3)$ & 5 & 2 & YES & YES & NO(2) & $1.44$ & $(4,2)$ & -- & 4086\\
$(44,13)$ & 8 & $(17,7)$ & 6 & 1 & YES & YES & YES & $1.43$ & $(2,3)$ & -- & 4087\\
$(44,17)$ & 8 & $(17,5)$ & 6 & 1 & YES & YES & YES & $1.50$ & $(2,3)$ & -- & 4088\\
$(44,13)$ & 8 & $(18,7)$ & 6 & 2 & YES & YES & NO(2) & $1.56$ & $(2,3)$ & -- & 4089\\
$(44,17)$ & 8 & $(18,5)$ & 6 & 2 & YES & YES & NO(2) & $1.38$ & $(4,2)$ & -- & 4090\\
$(44,17)$ & 8 & $(19,7)$ & 6 & 1 & YES & YES & YES & $1.80$ & $(2,3)$ & -- & 4091\\
$(44,17)$ & 8 & $(19,8)$ & 6 & 1 & YES & YES & YES & $1.78$ & $(2,3)$ & -- & 4092\\
$(44,13)$ & 8 & $(20,9)$ & 7 & 4 & YES & YES & NO(2) & $1.50$ & $(6,1)$ & -- & 4093\\
$(44,13)$ & 8 & $(20,9)$ & 7 & 4 & YES & YES & NO(2) & $1.50$ & $(6,1)$ & NO & 4094\\
$(44,17)$ & 8 & $(20,9)$ & 7 & 4 & YES & YES & NO(2) & $1.70$ & $(6,1)$ & -- & 4095\\
$(44,13)$ & 8 & $(21,8)$ & 6 & 1 & YES & YES & YES & $1.80$ & $(2,3)$ & NO & 4096\\
$(44,13)$ & 8 & $(21,8)$ & 6 & 1 & YES & YES & YES & $1.80$ & $(2,3)$ & -- & 4097\\
$(44,17)$ & 8 & $(21,8)$ & 6 & 1 & YES & YES & YES & $1.80$ & $(2,3)$ & -- & 4098\\
$(44,13)$ & 8 & $(23,7)$ & 7 & 1 & YES & YES & YES & $1.43$ & $(6,1)$ & -- & 4099\\
$(44,13)$ & 8 & $(23,9)$ & 7 & 1 & YES & YES & YES & $1.57$ & $(2,3)$ & NO & 4100\\
$(44,13)$ & 8 & $(23,9)$ & 7 & 1 & YES & YES & YES & $1.57$ & $(4,2)$ & -- & 4101\\
$(44,17)$ & 8 & $(23,7)$ & 7 & 1 & YES & YES & YES & $1.90$ & $(2,3)$ & NO & 4102\\
$(44,17)$ & 8 & $(23,7)$ & 7 & 1 & YES & YES & YES & $1.90$ & $(2,3)$ & -- & 4103\\
$(44,13)$ & 8 & $(24,7)$ & 7 & 4 & YES & YES & YES & $1.50$ & $(2,3)$ & -- & 4104\\
$(44,17)$ & 8 & $(24,7)$ & 7 & 4 & YES & YES & YES & $1.88$ & $(2,3)$ & -- & 4105\\
$(44,17)$ & 8 & $(25,7)$ & 7 & 1 & YES & YES & YES & $1.89$ & $(2,3)$ & NO & 4106\\
$(44,17)$ & 8 & $(25,7)$ & 7 & 1 & YES & YES & YES & $1.89$ & $(2,3)$ & -- & 4107\\
$(44,13)$ & 8 & $(26,11)$ & 7 & 2 & YES & YES & YES & $1.56$ & $(4,2)$ & -- & 4108\\
$(44,17)$ & 8 & $(26,7)$ & 7 & 2 & YES & YES & YES & $1.60$ & $(2,3)$ & -- & 4109\\
$(44,13)$ & 8 & $(27,8)$ & 7 & 1 & YES & YES & YES & $1.62$ & $(2,3)$ & -- & 4110\\
$(44,17)$ & 8 & $(27,8)$ & 7 & 1 & YES & YES & YES & $1.75$ & $(2,3)$ & -- & 4111\\
$(44,13)$ & 8 & $(29,8)$ & 7 & 1 & YES & YES & YES & $1.50$ & $(2,3)$ & -- & 4112\\
$(44,17)$ & 8 & $(29,8)$ & 7 & 1 & YES & YES & YES & $1.75$ & $(2,3)$ & -- & 4113\\
$(44,17)$ & 8 & $(31,7)$ & 8 & 1 & YES & YES & YES & $1.78$ & $(2,3)$ & -- & 4114\\
$(44,17)$ & 8 & $(31,7)$ & 8 & 1 & YES & YES & YES & $1.89$ & $(2,3)$ & NO & 4115\\
$(44,17)$ & 8 & $(37,13)$ & 9 & 1 & YES & YES & NO(2) & $1.62$ & $(6,1)$ & NO & 4116\\
$(45,17)$ & 9 & $(11,3)$ & 5 & 1 & YES & YES & YES & $1.43$ & $(4,2)$ & NO & 4117\\
$(45,17)$ & 9 & $(11,3)$ & 5 & 1 & YES & YES & YES & $1.43$ & $(4,2)$ & -- & 4118\\
$(45,19)$ & 8 & $(13,5)$ & 5 & 1 & YES & YES & NO(2) & $1.50$ & $(6,1)$ & -- & 4119\\
$(45,16)$ & 9 & $(17,6)$ & 7 & 1 & YES & YES & YES & $1.71$ & $(4,2)$ & -- & 4120\\
$(45,17)$ & 9 & $(17,5)$ & 6 & 1 & YES & YES & NO(2) & $1.70$ & $(2,3)$ & -- & 4121\\
$(45,19)$ & 8 & $(17,7)$ & 6 & 1 & YES & YES & YES & $1.43$ & $(4,2)$ & -- & 4122\\
$(45,17)$ & 9 & $(19,4)$ & 7 & 1 & YES & YES & NO(2) & $1.50$ & $(10,-1)$ & -- & 4123\\
$(45,17)$ & 9 & $(19,4)$ & 7 & 1 & YES & YES & NO(2) & $1.60$ & $(10,-1)$ & NO & 4124\\
$(45,16)$ & 9 & $(23,10)$ & 7 & 1 & YES & YES & NO(2) & $1.67$ & $(8,0)$ & NO & 4125\\
$(45,17)$ & 9 & $(24,5)$ & 8 & 3 & YES & YES & NO(2) & $1.56$ & $(8,0)$ & -- & 4126\\
$(45,19)$ & 8 & $(24,7)$ & 7 & 3 & YES & YES & YES & $1.57$ & $(4,2)$ & -- & 4127\\
$(45,19)$ & 8 & $(27,8)$ & 7 & 9 & YES & YES & YES & $1.56$ & $(4,2)$ & -- & 4128\\
$(45,16)$ & 9 & $(28,11)$ & 8 & 1 & YES & YES & NO(2) & $1.78$ & $(8,0)$ & NO & 4129\\
$(45,19)$ & 8 & $(29,9)$ & 8 & 1 & YES & YES & YES & $1.57$ & $(2,3)$ & -- & 4130\\
$(45,8)$ & 9 & $(32,9)$ & 8 & 1 & YES & YES & NO(2) & $1.44$ & $(8,0)$ & -- & 4131\\
$(45,8)$ & 9 & $(32,9)$ & 8 & 1 & YES & YES & NO(2) & $1.44$ & $(8,0)$ & NO & 4132\\
$(45,8)$ & 9 & $(35,11)$ & 9 & 5 & YES & YES & NO(2) & $1.62$ & $(6,1)$ & NO & 4133\\
$(45,8)$ & 9 & $(35,13)$ & 8 & 5 & YES & YES & NO(2) & $1.60$ & $(6,1)$ & NO & 4134\\
$(45,8)$ & 9 & $(42,11)$ & 9 & 3 & YES & YES & NO(2) & $1.56$ & $(4,2)$ & -- & 4135\\
$(45,8)$ & 9 & $(42,11)$ & 9 & 3 & YES & YES & NO(2) & $1.56$ & $(4,2)$ & NO & 4136\\
$(46,17)$ & 8 & $(12,5)$ & 5 & 2 & YES & YES & NO(2) & $1.62$ & $(4,2)$ & -- & 4137\\
$(46,17)$ & 8 & $(13,5)$ & 5 & 1 & YES & YES & NO(2) & $1.62$ & $(4,2)$ & -- & 4138\\
$(46,19)$ & 8 & $(18,7)$ & 6 & 2 & YES & YES & YES & $1.29$ & $(4,2)$ & -- & 4139\\
$(46,19)$ & 8 & $(19,7)$ & 6 & 1 & YES & YES & YES & $1.57$ & $(4,2)$ & -- & 4140\\
$(46,19)$ & 8 & $(24,7)$ & 7 & 2 & YES & YES & YES & $1.89$ & $(2,3)$ & -- & 4141\\
$(46,17)$ & 8 & $(25,7)$ & 7 & 1 & YES & YES & NO(2) & $1.44$ & $(8,0)$ & -- & 4142\\
$(46,19)$ & 8 & $(25,9)$ & 7 & 1 & YES & YES & YES & $1.75$ & $(4,2)$ & -- & 4143\\
$(46,19)$ & 8 & $(27,8)$ & 7 & 1 & YES & YES & YES & $1.56$ & $(4,2)$ & -- & 4144\\
$(46,17)$ & 8 & $(29,12)$ & 7 & 1 & YES & YES & YES & $1.75$ & $(4,2)$ & -- & 4145\\
$(46,11)$ & 10 & $(36,11)$ & 8 & 2 & YES & YES & NO(2) & $1.50$ & $(6,1)$ & NO & 4146\\
$(46,17)$ & 8 & $(39,14)$ & 8 & 1 & YES & YES & NO(2) & $1.56$ & $(4,2)$ & NO & 4147\\
$(46,19)$ & 8 & $(41,18)$ & 8 & 1 & YES & YES & YES & $1.29$ & $(4,2)$ & NO & 4148\\
$(47,18)$ & 8 & $(7,2)$ & 4 & 1 & YES & YES & YES & $1.25$ & $(2,3)$ & -- & 4149\\
$(47,13)$ & 8 & $(10,3)$ & 5 & 1 & YES & YES & NO(2) & $1.50$ & $(2,3)$ & -- & 4150\\
$(47,13)$ & 8 & $(10,3)$ & 5 & 1 & YES & YES & NO(2) & $1.70$ & $(2,3)$ & NO & 4151\\
$(47,14)$ & 9 & $(11,3)$ & 5 & 1 & YES & YES & NO(2) & $1.56$ & $(8,0)$ & NO & 4152\\
$(47,14)$ & 9 & $(11,3)$ & 5 & 1 & YES & YES & NO(2) & $1.56$ & $(8,0)$ & -- & 4153\\
$(47,18)$ & 8 & $(11,3)$ & 5 & 1 & YES & YES & NO(2) & $1.60$ & $(2,3)$ & -- & 4154\\
$(47,13)$ & 8 & $(12,5)$ & 5 & 1 & YES & YES & NO(2) & $1.56$ & $(4,2)$ & -- & 4155\\
$(47,13)$ & 8 & $(12,5)$ & 5 & 1 & YES & YES & NO(2) & $1.67$ & $(4,2)$ & NO & 4156\\
$(47,11)$ & 9 & $(13,4)$ & 6 & 1 & YES & YES & YES & $1.50$ & $(2,3)$ & NO & 4157\\
$(47,11)$ & 9 & $(13,4)$ & 6 & 1 & YES & YES & YES & $1.50$ & $(2,3)$ & -- & 4158\\
$(47,14)$ & 9 & $(13,5)$ & 5 & 1 & YES & YES & NO(2) & $1.80$ & $(6,1)$ & NO & 4159\\
$(47,14)$ & 9 & $(13,5)$ & 5 & 1 & YES & YES & NO(2) & $1.80$ & $(6,1)$ & -- & 4160\\
$(47,18)$ & 8 & $(13,3)$ & 6 & 1 & YES & YES & YES & $1.50$ & $(2,3)$ & NO & 4161\\
$(47,18)$ & 8 & $(13,3)$ & 6 & 1 & YES & YES & YES & $1.50$ & $(2,3)$ & -- & 4162\\
$(47,18)$ & 8 & $(13,5)$ & 5 & 1 & YES & YES & NO(2) & $1.56$ & $(4,2)$ & -- & 4163\\
$(47,18)$ & 8 & $(13,5)$ & 5 & 1 & YES & YES & NO(2) & $1.70$ & $(6,1)$ & NO & 4164\\
$(47,18)$ & 8 & $(16,7)$ & 6 & 1 & YES & YES & YES & $1.62$ & $(2,3)$ & -- & 4165\\
$(47,13)$ & 8 & $(17,7)$ & 6 & 1 & YES & YES & NO(2) & $1.38$ & $(6,1)$ & -- & 4166\\
$(47,14)$ & 9 & $(17,7)$ & 6 & 1 & YES & YES & NO(2) & $1.50$ & $(6,1)$ & -- & 4167\\
$(47,17)$ & 9 & $(17,5)$ & 6 & 1 & YES & YES & NO(2) & $1.67$ & $(2,3)$ & -- & 4168\\
$(47,18)$ & 8 & $(17,5)$ & 6 & 1 & YES & YES & YES & $1.90$ & $(2,3)$ & NO & 4169\\
$(47,18)$ & 8 & $(17,5)$ & 6 & 1 & YES & YES & YES & $1.90$ & $(2,3)$ & -- & 4170\\
$(47,10)$ & 9 & $(18,7)$ & 6 & 1 & YES & YES & NO(2) & $1.80$ & $(6,1)$ & -- & 4171\\
$(47,13)$ & 8 & $(18,7)$ & 6 & 1 & YES & YES & YES & $1.50$ & $(2,3)$ & -- & 4172\\
$(47,18)$ & 8 & $(18,5)$ & 6 & 1 & YES & YES & YES & $1.90$ & $(2,3)$ & NO & 4173\\
$(47,18)$ & 8 & $(18,5)$ & 6 & 1 & YES & YES & YES & $1.90$ & $(2,3)$ & -- & 4174\\
$(47,18)$ & 8 & $(18,7)$ & 6 & 1 & YES & YES & YES & $1.89$ & $(2,3)$ & -- & 4175\\
$(47,14)$ & 9 & $(19,8)$ & 6 & 1 & YES & YES & NO(2) & $1.56$ & $(4,2)$ & NO & 4176\\
$(47,18)$ & 8 & $(19,8)$ & 6 & 1 & YES & YES & YES & $1.62$ & $(4,2)$ & -- & 4177\\
$(47,13)$ & 8 & $(21,8)$ & 6 & 1 & YES & YES & NO(2) & $1.50$ & $(2,3)$ & -- & 4178\\
$(47,13)$ & 8 & $(21,8)$ & 6 & 1 & YES & YES & NO(2) & $1.60$ & $(2,3)$ & NO & 4179\\
$(47,18)$ & 8 & $(21,8)$ & 6 & 1 & YES & YES & YES & $1.78$ & $(2,3)$ & -- & 4180\\
$(47,11)$ & 9 & $(23,6)$ & 8 & 1 & YES & YES & NO(2) & $1.56$ & $(8,0)$ & -- & 4181\\
$(47,13)$ & 8 & $(23,9)$ & 7 & 1 & YES & YES & YES & $1.57$ & $(4,2)$ & -- & 4182\\
$(47,18)$ & 8 & $(23,7)$ & 7 & 1 & YES & YES & YES & $1.43$ & $(4,2)$ & -- & 4183\\
$(47,18)$ & 8 & $(23,10)$ & 7 & 1 & YES & YES & YES & $1.43$ & $(6,1)$ & -- & 4184\\
$(47,14)$ & 9 & $(24,7)$ & 7 & 1 & YES & YES & YES & $1.75$ & $(4,2)$ & -- & 4185\\
$(47,18)$ & 8 & $(24,7)$ & 7 & 1 & YES & YES & YES & $1.78$ & $(2,3)$ & -- & 4186\\
$(47,10)$ & 9 & $(25,7)$ & 7 & 1 & YES & YES & NO(2) & $1.56$ & $(4,2)$ & -- & 4187\\
$(47,11)$ & 9 & $(25,11)$ & 7 & 1 & YES & YES & NO(2) & $1.56$ & $(4,2)$ & NO & 4188\\
$(47,11)$ & 9 & $(25,11)$ & 7 & 1 & YES & YES & NO(2) & $1.56$ & $(4,2)$ & -- & 4189\\
$(47,18)$ & 8 & $(25,7)$ & 7 & 1 & YES & YES & YES & $1.62$ & $(4,2)$ & -- & 4190\\
$(47,13)$ & 8 & $(26,11)$ & 7 & 1 & YES & YES & YES & $1.62$ & $(4,2)$ & NO & 4191\\
$(47,18)$ & 8 & $(26,7)$ & 7 & 1 & YES & YES & YES & $1.71$ & $(2,3)$ & -- & 4192\\
$(47,13)$ & 8 & $(27,8)$ & 7 & 1 & YES & YES & YES & $1.50$ & $(2,3)$ & -- & 4193\\
$(47,14)$ & 9 & $(27,5)$ & 8 & 1 & YES & YES & NO(2) & $1.44$ & $(8,0)$ & NO & 4194\\
$(47,11)$ & 9 & $(29,11)$ & 7 & 1 & YES & YES & YES & $1.80$ & $(2,3)$ & -- & 4195\\
$(47,13)$ & 8 & $(29,8)$ & 7 & 1 & YES & YES & YES & $1.50$ & $(2,3)$ & -- & 4196\\
$(47,18)$ & 8 & $(29,8)$ & 7 & 1 & YES & YES & YES & $1.86$ & $(2,3)$ & -- & 4197\\
$(47,18)$ & 8 & $(29,8)$ & 7 & 1 & YES & YES & YES & $1.62$ & $(4,2)$ & NO & 4198\\
$(47,18)$ & 8 & $(30,7)$ & 8 & 1 & YES & YES & YES & $1.71$ & $(2,3)$ & -- & 4199\\
$(47,18)$ & 8 & $(31,7)$ & 8 & 1 & YES & YES & YES & $1.80$ & $(2,3)$ & -- & 4200\\
$(47,18)$ & 8 & $(31,12)$ & 7 & 1 & YES & YES & YES & $1.78$ & $(4,2)$ & -- & 4201\\
$(47,10)$ & 9 & $(32,9)$ & 8 & 1 & YES & YES & NO(2) & $1.50$ & $(6,1)$ & NO & 4202\\
$(47,11)$ & 9 & $(32,9)$ & 8 & 1 & YES & YES & YES & $1.75$ & $(2,3)$ & -- & 4203\\
$(47,18)$ & 8 & $(33,7)$ & 8 & 1 & YES & YES & YES & $1.43$ & $(4,2)$ & -- & 4204\\
$(47,11)$ & 9 & $(34,13)$ & 7 & 1 & YES & YES & YES & $1.78$ & $(2,3)$ & -- & 4205\\
$(47,13)$ & 8 & $(34,9)$ & 8 & 1 & YES & YES & NO(2) & $1.56$ & $(4,2)$ & NO & 4206\\
$(47,18)$ & 8 & $(35,8)$ & 8 & 1 & YES & YES & YES & $1.71$ & $(2,3)$ & -- & 4207\\
$(47,10)$ & 9 & $(37,11)$ & 8 & 1 & YES & YES & YES & $1.67$ & $(4,2)$ & -- & 4208\\
$(47,14)$ & 9 & $(37,8)$ & 8 & 1 & YES & YES & YES & $1.43$ & $(4,2)$ & NO & 4209\\
$(47,18)$ & 8 & $(37,8)$ & 8 & 1 & YES & YES & YES & $1.67$ & $(2,3)$ & -- & 4210\\
$(47,14)$ & 9 & $(39,7)$ & 9 & 1 & YES & YES & YES & $1.75$ & $(4,2)$ & -- & 4211\\
$(47,10)$ & 9 & $(40,11)$ & 8 & 1 & YES & YES & YES & $1.57$ & $(4,2)$ & -- & 4212\\
$(47,11)$ & 9 & $(43,12)$ & 8 & 1 & YES & YES & YES & $1.62$ & $(2,3)$ & -- & 4213\\
$(48,11)$ & 9 & $(16,5)$ & 7 & 16 & YES & YES & NO(2) & $1.67$ & $(4,2)$ & -- & 4214\\
$(48,11)$ & 9 & $(23,9)$ & 7 & 1 & YES & YES & NO(2) & $1.56$ & $(4,2)$ & -- & 4215\\
$(48,11)$ & 9 & $(27,10)$ & 7 & 3 & YES & YES & YES & $1.50$ & $(4,2)$ & -- & 4216\\
$(48,11)$ & 9 & $(27,11)$ & 8 & 3 & YES & YES & YES & $1.67$ & $(4,2)$ & -- & 4217\\
$(48,11)$ & 9 & $(29,11)$ & 7 & 1 & YES & YES & YES & $1.70$ & $(2,3)$ & -- & 4218\\
$(48,11)$ & 9 & $(31,9)$ & 8 & 1 & YES & YES & YES & $1.57$ & $(2,3)$ & -- & 4219\\
$(48,13)$ & 9 & $(31,7)$ & 8 & 1 & YES & YES & YES & $1.43$ & $(4,2)$ & -- & 4220\\
$(48,11)$ & 9 & $(32,9)$ & 8 & 16 & YES & YES & YES & $1.62$ & $(4,2)$ & -- & 4221\\
$(48,13)$ & 9 & $(32,7)$ & 8 & 16 & YES & YES & YES & $1.57$ & $(4,2)$ & -- & 4222\\
$(48,11)$ & 9 & $(37,10)$ & 8 & 1 & YES & YES & YES & $1.50$ & $(4,2)$ & -- & 4223\\
$(48,11)$ & 9 & $(39,16)$ & 8 & 3 & YES & YES & YES & $1.67$ & $(4,2)$ & -- & 4224\\
$(48,11)$ & 9 & $(40,9)$ & 9 & 8 & YES & YES & YES & $1.50$ & $(4,2)$ & -- & 4225\\
$(48,11)$ & 9 & $(40,11)$ & 8 & 8 & YES & YES & YES & $1.57$ & $(2,3)$ & -- & 4226\\
$(48,11)$ & 9 & $(41,11)$ & 8 & 1 & YES & YES & YES & $1.70$ & $(2,3)$ & -- & 4227\\
$(49,19)$ & 8 & $(11,3)$ & 5 & 1 & YES & YES & NO(2) & $1.38$ & $(10,-1)$ & NO & 4228\\
$(49,19)$ & 8 & $(11,3)$ & 5 & 1 & YES & YES & NO(2) & $1.38$ & $(10,-1)$ & -- & 4229\\
$(49,19)$ & 8 & $(12,5)$ & 5 & 1 & YES & YES & NO(2) & $1.60$ & $(2,3)$ & -- & 4230\\
$(49,13)$ & 9 & $(13,5)$ & 5 & 1 & YES & YES & YES & $1.43$ & $(4,2)$ & NO & 4231\\
$(49,13)$ & 9 & $(13,5)$ & 5 & 1 & YES & YES & YES & $1.43$ & $(4,2)$ & -- & 4232\\
$(49,15)$ & 9 & $(13,5)$ & 5 & 1 & YES & YES & NO(2) & $1.60$ & $(6,1)$ & -- & 4233\\
$(49,18)$ & 8 & $(14,5)$ & 6 & 7 & YES & YES & NO(2) & $1.50$ & $(10,-1)$ & NO & 4234\\
$(49,18)$ & 8 & $(14,5)$ & 6 & 7 & YES & YES & NO(2) & $1.50$ & $(10,-1)$ & -- & 4235\\
$(49,19)$ & 8 & $(16,7)$ & 6 & 1 & YES & YES & YES & $1.43$ & $(4,2)$ & -- & 4236\\
$(49,13)$ & 9 & $(17,5)$ & 6 & 1 & YES & YES & NO(2) & $1.56$ & $(8,0)$ & -- & 4237\\
$(49,13)$ & 9 & $(17,7)$ & 6 & 1 & YES & YES & YES & $1.62$ & $(2,3)$ & NO & 4238\\
$(49,13)$ & 9 & $(17,7)$ & 6 & 1 & YES & YES & YES & $1.62$ & $(2,3)$ & -- & 4239\\
$(49,15)$ & 9 & $(17,5)$ & 6 & 1 & YES & YES & NO(2) & $1.67$ & $(2,3)$ & -- & 4240\\
$(49,18)$ & 8 & $(17,5)$ & 6 & 1 & YES & YES & YES & $1.75$ & $(2,3)$ & -- & 4241\\
$(49,19)$ & 8 & $(17,5)$ & 6 & 1 & YES & YES & NO(2) & $1.44$ & $(2,3)$ & -- & 4242\\
$(49,19)$ & 8 & $(17,7)$ & 6 & 1 & YES & YES & YES & $1.75$ & $(2,3)$ & -- & 4243\\
$(49,13)$ & 9 & $(18,5)$ & 6 & 1 & YES & YES & NO(2) & $1.44$ & $(12,-2)$ & -- & 4244\\
$(49,18)$ & 8 & $(18,7)$ & 6 & 1 & YES & YES & YES & $1.62$ & $(2,3)$ & -- & 4245\\
$(49,19)$ & 8 & $(18,5)$ & 6 & 1 & YES & YES & YES & $1.62$ & $(2,3)$ & -- & 4246\\
$(49,19)$ & 8 & $(18,7)$ & 6 & 1 & YES & YES & YES & $1.62$ & $(4,2)$ & -- & 4247\\
$(49,13)$ & 9 & $(19,8)$ & 6 & 1 & YES & YES & YES & $1.62$ & $(2,3)$ & NO & 4248\\
$(49,13)$ & 9 & $(19,8)$ & 6 & 1 & YES & YES & YES & $1.62$ & $(2,3)$ & -- & 4249\\
$(49,18)$ & 8 & $(19,8)$ & 6 & 1 & YES & YES & YES & $1.50$ & $(2,3)$ & -- & 4250\\
$(49,19)$ & 8 & $(19,7)$ & 6 & 1 & YES & YES & YES & $1.57$ & $(4,2)$ & -- & 4251\\
$(49,19)$ & 8 & $(20,9)$ & 7 & 1 & YES & YES & NO(2) & $1.70$ & $(6,1)$ & -- & 4252\\
$(49,18)$ & 8 & $(21,8)$ & 6 & 7 & YES & YES & YES & $1.62$ & $(4,2)$ & -- & 4253\\
$(49,15)$ & 9 & $(23,6)$ & 8 & 1 & YES & YES & NO(2) & $1.60$ & $(6,1)$ & -- & 4254\\
$(49,19)$ & 8 & $(24,7)$ & 7 & 1 & YES & YES & YES & $1.78$ & $(2,3)$ & -- & 4255\\
$(49,18)$ & 8 & $(25,7)$ & 7 & 1 & YES & YES & YES & $1.56$ & $(2,3)$ & -- & 4256\\
$(49,18)$ & 8 & $(25,9)$ & 7 & 1 & YES & YES & YES & $1.86$ & $(6,1)$ & -- & 4257\\
$(49,19)$ & 8 & $(25,7)$ & 7 & 1 & YES & YES & YES & $1.71$ & $(4,2)$ & -- & 4258\\
$(49,19)$ & 8 & $(25,9)$ & 7 & 1 & YES & YES & YES & $1.75$ & $(4,2)$ & -- & 4259\\
$(49,13)$ & 9 & $(26,11)$ & 7 & 1 & YES & YES & NO(2) & $1.50$ & $(6,1)$ & NO & 4260\\
$(49,13)$ & 9 & $(27,8)$ & 7 & 1 & YES & YES & NO(2) & $1.44$ & $(4,2)$ & NO & 4261\\
$(49,15)$ & 9 & $(27,7)$ & 9 & 1 & YES & YES & NO(2) & $1.70$ & $(6,1)$ & NO & 4262\\
$(49,15)$ & 9 & $(29,8)$ & 7 & 1 & YES & YES & NO(2) & $1.70$ & $(6,1)$ & NO & 4263\\
$(49,13)$ & 9 & $(30,11)$ & 7 & 1 & YES & YES & YES & $1.57$ & $(6,1)$ & -- & 4264\\
$(49,15)$ & 9 & $(31,7)$ & 8 & 1 & YES & YES & YES & $1.75$ & $(4,2)$ & -- & 4265\\
$(49,18)$ & 8 & $(31,9)$ & 8 & 1 & YES & YES & YES & $1.67$ & $(4,2)$ & -- & 4266\\
$(49,11)$ & 10 & $(34,9)$ & 8 & 1 & YES & YES & NO(2) & $1.38$ & $(6,1)$ & -- & 4267\\
$(49,19)$ & 8 & $(35,8)$ & 8 & 7 & YES & YES & YES & $1.43$ & $(4,2)$ & -- & 4268\\
$(49,15)$ & 9 & $(41,12)$ & 8 & 1 & YES & YES & NO(2) & $1.70$ & $(6,1)$ & NO & 4269\\
$(49,13)$ & 9 & $(43,12)$ & 8 & 1 & YES & YES & NO(2) & $1.44$ & $(8,0)$ & NO & 4270\\
$(50,21)$ & 8 & $(7,3)$ & 4 & 1 & YES & YES & YES & $1.50$ & $(2,3)$ & -- & 4271\\
$(50,19)$ & 8 & $(10,3)$ & 5 & 10 & YES & YES & NO(2) & $1.50$ & $(6,1)$ & -- & 4272\\
$(50,21)$ & 8 & $(10,3)$ & 5 & 10 & YES & YES & NO(2) & $1.50$ & $(2,3)$ & -- & 4273\\
$(50,21)$ & 8 & $(10,3)$ & 5 & 10 & YES & YES & NO(2) & $1.50$ & $(6,1)$ & NO & 4274\\
$(50,21)$ & 8 & $(11,4)$ & 5 & 1 & YES & YES & NO(2) & $1.60$ & $(2,3)$ & NO & 4275\\
$(50,21)$ & 8 & $(11,4)$ & 5 & 1 & YES & YES & NO(2) & $1.82$ & $(4,2)$ & -- & 4276\\
$(50,19)$ & 8 & $(12,5)$ & 5 & 2 & YES & YES & NO(2) & $1.78$ & $(4,2)$ & -- & 4277\\
$(50,21)$ & 8 & $(12,5)$ & 5 & 2 & YES & YES & NO(2) & $1.67$ & $(4,2)$ & -- & 4278\\
$(50,21)$ & 8 & $(12,5)$ & 5 & 2 & YES & YES & NO(2) & $1.60$ & $(6,1)$ & NO & 4279\\
$(50,21)$ & 8 & $(13,5)$ & 5 & 1 & YES & YES & NO(2) & $1.56$ & $(4,2)$ & -- & 4280\\
$(50,19)$ & 8 & $(15,4)$ & 6 & 5 & YES & YES & NO(2) & $1.33$ & $(4,2)$ & -- & 4281\\
$(50,21)$ & 8 & $(16,5)$ & 7 & 2 & YES & YES & NO(2) & $1.67$ & $(8,0)$ & -- & 4282\\
$(50,21)$ & 8 & $(17,7)$ & 6 & 1 & YES & YES & YES & $1.89$ & $(2,3)$ & -- & 4283\\
$(50,19)$ & 8 & $(18,7)$ & 6 & 2 & YES & YES & YES & $1.90$ & $(2,3)$ & -- & 4284\\
$(50,21)$ & 8 & $(18,5)$ & 6 & 2 & YES & YES & YES & $1.50$ & $(2,3)$ & -- & 4285\\
$(50,21)$ & 8 & $(18,7)$ & 6 & 2 & YES & YES & YES & $1.78$ & $(4,2)$ & -- & 4286\\
$(50,21)$ & 8 & $(19,8)$ & 6 & 1 & YES & YES & YES & $1.88$ & $(2,3)$ & -- & 4287\\
$(50,21)$ & 8 & $(21,8)$ & 6 & 1 & YES & YES & YES & $1.29$ & $(4,2)$ & -- & 4288\\
$(50,21)$ & 8 & $(23,10)$ & 7 & 1 & YES & YES & YES & $1.57$ & $(6,1)$ & -- & 4289\\
$(50,19)$ & 8 & $(24,7)$ & 7 & 2 & YES & YES & YES & $1.78$ & $(2,3)$ & -- & 4290\\
$(50,21)$ & 8 & $(24,7)$ & 7 & 2 & YES & YES & YES & $1.43$ & $(2,3)$ & -- & 4291\\
$(50,19)$ & 8 & $(25,9)$ & 7 & 25 & YES & YES & YES & $1.75$ & $(4,2)$ & -- & 4292\\
$(50,21)$ & 8 & $(31,7)$ & 8 & 1 & YES & YES & YES & $1.67$ & $(4,2)$ & -- & 4293\\
$(50,21)$ & 8 & $(31,12)$ & 7 & 1 & YES & YES & YES & $1.78$ & $(4,2)$ & -- & 4294\\
$(50,21)$ & 8 & $(31,14)$ & 8 & 1 & YES & YES & NO(2) & $1.62$ & $(6,1)$ & NO & 4295\\
$(50,19)$ & 8 & $(41,16)$ & 8 & 1 & YES & YES & NO(2) & $1.78$ & $(4,2)$ & NO & 4296\\
$(50,19)$ & 8 & $(46,19)$ & 8 & 2 & YES & YES & YES & $1.78$ & $(4,2)$ & NO & 4297\\
$(51,20)$ & 9 & $(10,3)$ & 5 & 1 & YES & YES & NO(2) & $1.67$ & $(4,2)$ & -- & 4298\\
$(51,20)$ & 9 & $(13,5)$ & 5 & 1 & YES & YES & NO(2) & $1.56$ & $(8,0)$ & -- & 4299\\
$(51,23)$ & 9 & $(13,4)$ & 6 & 1 & YES & YES & NO(2) & $1.62$ & $(6,1)$ & -- & 4300\\
$(51,14)$ & 9 & $(16,5)$ & 7 & 1 & YES & YES & NO(2) & $1.67$ & $(8,0)$ & -- & 4301\\
$(51,11)$ & 9 & $(23,9)$ & 7 & 1 & YES & YES & YES & $1.62$ & $(4,2)$ & -- & 4302\\
$(51,11)$ & 9 & $(23,9)$ & 7 & 1 & YES & YES & YES & $1.78$ & $(4,2)$ & NO & 4303\\
$(51,14)$ & 9 & $(24,7)$ & 7 & 3 & YES & YES & YES & $1.62$ & $(4,2)$ & -- & 4304\\
$(51,14)$ & 9 & $(27,5)$ & 8 & 3 & YES & YES & NO(2) & $1.44$ & $(8,0)$ & NO & 4305\\
$(51,20)$ & 9 & $(29,8)$ & 7 & 1 & YES & YES & YES & $1.86$ & $(6,1)$ & NO & 4306\\
$(51,11)$ & 9 & $(31,9)$ & 8 & 1 & YES & YES & YES & $1.80$ & $(2,3)$ & -- & 4307\\
$(51,14)$ & 9 & $(31,7)$ & 8 & 1 & YES & YES & YES & $1.75$ & $(2,3)$ & -- & 4308\\
$(51,11)$ & 9 & $(32,9)$ & 8 & 1 & YES & YES & YES & $1.57$ & $(4,2)$ & -- & 4309\\
$(51,14)$ & 9 & $(32,7)$ & 8 & 1 & YES & YES & YES & $1.75$ & $(2,3)$ & -- & 4310\\
$(51,11)$ & 9 & $(36,7)$ & 11 & 3 & YES & YES & NO(2) & $1.50$ & $(6,1)$ & -- & 4311\\
$(51,14)$ & 9 & $(39,7)$ & 9 & 3 & YES & YES & YES & $1.62$ & $(4,2)$ & -- & 4312\\
$(51,11)$ & 9 & $(40,9)$ & 9 & 1 & YES & YES & YES & $1.50$ & $(4,2)$ & -- & 4313\\
$(51,20)$ & 9 & $(43,8)$ & 9 & 1 & YES & YES & YES & $1.71$ & $(6,1)$ & NO & 4314\\
$(51,20)$ & 9 & $(49,18)$ & 8 & 1 & YES & YES & YES & $1.86$ & $(6,1)$ & NO & 4315\\
$(52,19)$ & 9 & $(11,3)$ & 5 & 1 & YES & YES & NO(2) & $1.60$ & $(6,1)$ & -- & 4316\\
$(52,23)$ & 10 & $(11,4)$ & 5 & 1 & YES & YES & NO(2) & $1.78$ & $(8,0)$ & -- & 4317\\
$(52,23)$ & 10 & $(15,4)$ & 6 & 1 & YES & YES & NO(2) & $1.78$ & $(8,0)$ & NO & 4318\\
$(52,23)$ & 10 & $(15,4)$ & 6 & 1 & YES & YES & NO(2) & $1.78$ & $(8,0)$ & -- & 4319\\
$(52,19)$ & 9 & $(17,5)$ & 6 & 1 & YES & YES & NO(2) & $1.78$ & $(2,3)$ & -- & 4320\\
$(52,11)$ & 9 & $(24,7)$ & 7 & 4 & YES & YES & NO(2) & $1.56$ & $(4,2)$ & -- & 4321\\
$(52,19)$ & 9 & $(24,7)$ & 7 & 4 & YES & YES & YES & $1.75$ & $(4,2)$ & -- & 4322\\
$(52,11)$ & 9 & $(25,7)$ & 7 & 1 & YES & YES & NO(2) & $1.56$ & $(4,2)$ & -- & 4323\\
$(52,19)$ & 9 & $(25,7)$ & 7 & 1 & YES & YES & YES & $1.75$ & $(4,2)$ & -- & 4324\\
$(53,19)$ & 9 & $(10,3)$ & 5 & 1 & YES & YES & NO(2) & $1.62$ & $(6,1)$ & -- & 4325\\
$(53,23)$ & 9 & $(11,4)$ & 5 & 1 & YES & YES & YES & $1.43$ & $(2,3)$ & -- & 4326\\
$(53,22)$ & 9 & $(12,5)$ & 5 & 1 & YES & YES & NO(2) & $1.50$ & $(4,2)$ & -- & 4327\\
$(53,19)$ & 9 & $(13,5)$ & 5 & 1 & YES & YES & NO(2) & $1.67$ & $(8,0)$ & -- & 4328\\
$(53,22)$ & 9 & $(13,5)$ & 5 & 1 & YES & YES & NO(2) & $1.78$ & $(2,3)$ & -- & 4329\\
$(53,14)$ & 9 & $(17,5)$ & 6 & 1 & YES & YES & NO(2) & $1.50$ & $(10,-1)$ & -- & 4330\\
$(53,22)$ & 9 & $(17,5)$ & 6 & 1 & YES & YES & NO(2) & $1.67$ & $(2,3)$ & -- & 4331\\
$(53,12)$ & 9 & $(18,7)$ & 6 & 1 & YES & YES & NO(2) & $1.38$ & $(4,2)$ & -- & 4332\\
$(53,12)$ & 9 & $(18,7)$ & 6 & 1 & YES & YES & NO(2) & $1.56$ & $(4,2)$ & NO & 4333\\
$(53,14)$ & 9 & $(18,5)$ & 6 & 1 & YES & YES & NO(2) & $1.25$ & $(10,-1)$ & -- & 4334\\
$(53,22)$ & 9 & $(18,5)$ & 6 & 1 & YES & YES & YES & $1.57$ & $(4,2)$ & NO & 4335\\
$(53,22)$ & 9 & $(18,5)$ & 6 & 1 & YES & YES & YES & $1.57$ & $(4,2)$ & -- & 4336\\
$(53,11)$ & 10 & $(19,8)$ & 6 & 1 & YES & YES & NO(2) & $1.50$ & $(6,1)$ & NO & 4337\\
$(53,19)$ & 9 & $(21,5)$ & 8 & 1 & YES & YES & NO(2) & $1.56$ & $(8,0)$ & -- & 4338\\
$(53,22)$ & 9 & $(22,5)$ & 7 & 1 & YES & YES & YES & $1.43$ & $(4,2)$ & NO & 4339\\
$(53,22)$ & 9 & $(22,5)$ & 7 & 1 & YES & YES & YES & $1.62$ & $(4,2)$ & -- & 4340\\
$(53,11)$ & 10 & $(23,7)$ & 7 & 1 & YES & YES & NO(2) & $1.33$ & $(8,0)$ & -- & 4341\\
$(53,19)$ & 9 & $(23,10)$ & 7 & 1 & YES & YES & NO(2) & $1.50$ & $(6,1)$ & NO & 4342\\
$(53,22)$ & 9 & $(23,5)$ & 7 & 1 & YES & YES & YES & $1.71$ & $(4,2)$ & NO & 4343\\
$(53,22)$ & 9 & $(23,5)$ & 7 & 1 & YES & YES & YES & $1.71$ & $(4,2)$ & -- & 4344\\
$(53,14)$ & 9 & $(27,8)$ & 7 & 1 & YES & YES & NO(2) & $1.44$ & $(4,2)$ & NO & 4345\\
$(53,10)$ & 10 & $(29,7)$ & 10 & 1 & YES & YES & NO(2) & $1.60$ & $(6,1)$ & -- & 4346\\
$(53,12)$ & 9 & $(32,9)$ & 8 & 1 & YES & YES & YES & $1.62$ & $(2,3)$ & -- & 4347\\
$(53,22)$ & 9 & $(33,7)$ & 8 & 1 & YES & YES & YES & $1.75$ & $(4,2)$ & -- & 4348\\
$(53,14)$ & 9 & $(43,12)$ & 8 & 1 & YES & YES & NO(2) & $1.38$ & $(10,-1)$ & 5173 & 4349\\
$(53,14)$ & 9 & $(47,13)$ & 8 & 1 & YES & YES & NO(2) & $1.50$ & $(10,-1)$ & NO & 4350\\
$(53,10)$ & 10 & $(48,13)$ & 9 & 1 & YES & YES & YES & $1.75$ & $(4,2)$ & -- & 4351\\
$(54,13)$ & 11 & $(17,7)$ & 6 & 1 & YES & YES & NO(2) & $1.62$ & $(6,1)$ & NO & 4352\\
$(54,13)$ & 11 & $(17,7)$ & 6 & 1 & YES & YES & NO(2) & $1.62$ & $(6,1)$ & -- & 4353\\
$(54,13)$ & 11 & $(40,7)$ & 9 & 2 & YES & YES & NO(2) & $1.60$ & $(6,1)$ & NO & 4354\\
$(55,21)$ & 8 & $(7,2)$ & 4 & 1 & YES & YES & YES & $1.50$ & $(2,3)$ & NO & 4355\\
$(55,21)$ & 8 & $(7,2)$ & 4 & 1 & YES & YES & YES & $1.50$ & $(2,3)$ & -- & 4356\\
$(55,21)$ & 8 & $(7,3)$ & 4 & 1 & YES & YES & YES & $1.44$ & $(2,3)$ & -- & 4357\\
$(55,16)$ & 9 & $(8,3)$ & 4 & 1 & YES & YES & NO(2) & $1.60$ & $(2,3)$ & -- & 4358\\
$(55,23)$ & 9 & $(8,3)$ & 4 & 1 & YES & YES & NO(2) & $1.67$ & $(4,2)$ & -- & 4359\\
$(55,21)$ & 8 & $(10,3)$ & 5 & 5 & YES & YES & NO(2) & $1.64$ & $(4,2)$ & NO & 4360\\
$(55,21)$ & 8 & $(10,3)$ & 5 & 5 & YES & YES & NO(2) & $1.64$ & $(4,2)$ & -- & 4361\\
$(55,21)$ & 8 & $(11,4)$ & 5 & 11 & YES & YES & NO(2) & $1.56$ & $(4,2)$ & -- & 4362\\
$(55,21)$ & 8 & $(11,4)$ & 5 & 11 & YES & YES & NO(2) & $1.67$ & $(4,2)$ & NO & 4363\\
$(55,24)$ & 9 & $(11,3)$ & 5 & 11 & YES & YES & NO(2) & $1.56$ & $(4,2)$ & -- & 4364\\
$(55,21)$ & 8 & $(12,5)$ & 5 & 1 & YES & YES & NO(2) & $1.67$ & $(4,2)$ & -- & 4365\\
$(55,23)$ & 9 & $(12,5)$ & 5 & 1 & YES & YES & YES & $1.57$ & $(2,3)$ & -- & 4366\\
$(55,17)$ & 10 & $(13,5)$ & 5 & 1 & YES & YES & NO(2) & $1.43$ & $(8,0)$ & -- & 4367\\
$(55,17)$ & 10 & $(13,5)$ & 5 & 1 & YES & YES & NO(2) & $1.43$ & $(8,0)$ & NO & 4368\\
$(55,21)$ & 8 & $(13,3)$ & 6 & 1 & YES & YES & YES & $1.86$ & $(2,3)$ & NO & 4369\\
$(55,21)$ & 8 & $(13,3)$ & 6 & 1 & YES & YES & YES & $1.86$ & $(2,3)$ & -- & 4370\\
$(55,21)$ & 8 & $(13,5)$ & 5 & 1 & YES & YES & YES & $1.56$ & $(2,3)$ & -- & 4371\\
$(55,23)$ & 9 & $(13,3)$ & 6 & 1 & YES & YES & NO(2) & $1.44$ & $(8,0)$ & -- & 4372\\
$(55,23)$ & 9 & $(14,3)$ & 6 & 1 & YES & YES & NO(2) & $1.44$ & $(8,0)$ & -- & 4373\\
$(55,24)$ & 9 & $(14,5)$ & 6 & 1 & YES & YES & NO(2) & $1.56$ & $(8,0)$ & -- & 4374\\
$(55,12)$ & 9 & $(17,7)$ & 6 & 1 & YES & YES & NO(2) & $1.80$ & $(2,3)$ & NO & 4375\\
$(55,21)$ & 8 & $(17,5)$ & 6 & 1 & YES & YES & YES & $1.62$ & $(2,3)$ & -- & 4376\\
$(55,21)$ & 8 & $(17,7)$ & 6 & 1 & YES & YES & YES & $1.29$ & $(4,2)$ & -- & 4377\\
$(55,21)$ & 8 & $(17,7)$ & 6 & 1 & YES & YES & YES & $1.57$ & $(4,2)$ & NO & 4378\\
$(55,23)$ & 9 & $(17,5)$ & 6 & 1 & YES & YES & YES & $1.62$ & $(2,3)$ & -- & 4379\\
$(55,16)$ & 9 & $(18,5)$ & 6 & 1 & YES & YES & NO(2) & $1.33$ & $(8,0)$ & -- & 4380\\
$(55,21)$ & 8 & $(18,5)$ & 6 & 1 & YES & YES & YES & $1.78$ & $(2,3)$ & -- & 4381\\
$(55,21)$ & 8 & $(18,7)$ & 6 & 1 & YES & YES & YES & $1.43$ & $(4,2)$ & -- & 4382\\
$(55,23)$ & 9 & $(18,5)$ & 6 & 1 & YES & YES & YES & $1.62$ & $(2,3)$ & -- & 4383\\
$(55,23)$ & 9 & $(18,5)$ & 6 & 1 & YES & YES & YES & $1.89$ & $(2,3)$ & NO & 4384\\
$(55,24)$ & 9 & $(18,5)$ & 6 & 1 & YES & YES & YES & $1.62$ & $(4,2)$ & -- & 4385\\
$(55,21)$ & 8 & $(19,8)$ & 6 & 1 & YES & YES & YES & $1.78$ & $(4,2)$ & -- & 4386\\
$(55,23)$ & 9 & $(22,5)$ & 7 & 11 & YES & YES & YES & $1.62$ & $(2,3)$ & NO & 4387\\
$(55,23)$ & 9 & $(23,5)$ & 7 & 1 & YES & YES & YES & $1.62$ & $(2,3)$ & NO & 4388\\
$(55,21)$ & 8 & $(24,7)$ & 7 & 1 & YES & YES & YES & $1.57$ & $(2,3)$ & -- & 4389\\
$(55,17)$ & 10 & $(25,7)$ & 7 & 5 & YES & YES & YES & $1.57$ & $(6,1)$ & -- & 4390\\
$(55,23)$ & 9 & $(27,5)$ & 8 & 1 & YES & YES & YES & $2.00$ & $(2,3)$ & NO & 4391\\
$(55,23)$ & 9 & $(27,5)$ & 8 & 1 & YES & YES & YES & $2.00$ & $(2,3)$ & NO & 4392\\
$(55,13)$ & 10 & $(29,8)$ & 7 & 1 & YES & YES & YES & $1.57$ & $(4,2)$ & -- & 4393\\
$(55,21)$ & 8 & $(31,7)$ & 8 & 1 & YES & YES & YES & $1.67$ & $(4,2)$ & -- & 4394\\
$(55,23)$ & 9 & $(33,7)$ & 8 & 11 & YES & YES & YES & $1.89$ & $(4,2)$ & NO & 4395\\
$(55,23)$ & 9 & $(33,7)$ & 8 & 11 & YES & YES & YES & $1.89$ & $(4,2)$ & -- & 4396\\
$(55,13)$ & 10 & $(34,9)$ & 8 & 1 & YES & YES & NO(2) & $1.38$ & $(6,1)$ & -- & 4397\\
$(55,16)$ & 9 & $(35,8)$ & 8 & 5 & YES & YES & YES & $1.75$ & $(2,3)$ & -- & 4398\\
$(55,21)$ & 8 & $(35,8)$ & 8 & 5 & YES & YES & YES & $1.67$ & $(4,2)$ & -- & 4399\\
$(55,12)$ & 9 & $(37,11)$ & 8 & 1 & YES & YES & YES & $1.80$ & $(2,3)$ & NO & 4400\\
$(55,17)$ & 10 & $(41,12)$ & 8 & 1 & YES & YES & YES & $1.57$ & $(2,3)$ & NO & 4401\\
$(55,17)$ & 10 & $(44,13)$ & 8 & 11 & YES & YES & YES & $1.57$ & $(2,3)$ & NO & 4402\\
$(55,24)$ & 9 & $(52,23)$ & 10 & 1 & YES & YES & NO(2) & $1.78$ & $(8,0)$ & 7529 & 4403\\
$(55,12)$ & 9 & $(54,13)$ & 11 & 1 & YES & YES & NO(2) & $1.60$ & $(6,1)$ & NO & 4404\\
$(56,25)$ & 11 & $(11,4)$ & 5 & 1 & YES & YES & YES & $1.71$ & $(4,2)$ & -- & 4405\\
$(56,17)$ & 9 & $(12,5)$ & 5 & 4 & YES & YES & NO(2) & $1.44$ & $(8,0)$ & -- & 4406\\
$(56,25)$ & 11 & $(12,5)$ & 5 & 4 & YES & YES & YES & $1.86$ & $(4,2)$ & -- & 4407\\
$(56,25)$ & 11 & $(13,5)$ & 5 & 1 & YES & YES & YES & $1.71$ & $(4,2)$ & -- & 4408\\
$(56,17)$ & 9 & $(16,5)$ & 7 & 8 & YES & YES & YES & $1.62$ & $(2,3)$ & -- & 4409\\
$(56,15)$ & 9 & $(17,5)$ & 6 & 1 & YES & YES & NO(2) & $1.25$ & $(6,1)$ & -- & 4410\\
$(56,17)$ & 9 & $(18,7)$ & 6 & 2 & YES & YES & YES & $1.78$ & $(2,3)$ & NO & 4411\\
$(56,17)$ & 9 & $(18,7)$ & 6 & 2 & YES & YES & YES & $1.78$ & $(2,3)$ & -- & 4412\\
$(56,15)$ & 9 & $(21,8)$ & 6 & 7 & YES & YES & NO(2) & $1.44$ & $(4,2)$ & -- & 4413\\
$(56,17)$ & 9 & $(21,8)$ & 6 & 7 & YES & YES & YES & $2.00$ & $(2,3)$ & -- & 4414\\
$(56,15)$ & 9 & $(24,7)$ & 7 & 8 & YES & YES & NO(2) & $1.44$ & $(4,2)$ & NO & 4415\\
$(56,15)$ & 9 & $(24,7)$ & 7 & 8 & YES & YES & YES & $1.71$ & $(4,2)$ & -- & 4416\\
$(56,17)$ & 9 & $(24,7)$ & 7 & 8 & YES & YES & YES & $1.43$ & $(4,2)$ & -- & 4417\\
$(56,15)$ & 9 & $(25,7)$ & 7 & 1 & YES & YES & YES & $1.75$ & $(2,3)$ & -- & 4418\\
$(56,17)$ & 9 & $(25,7)$ & 7 & 1 & YES & YES & YES & $1.43$ & $(4,2)$ & -- & 4419\\
$(56,13)$ & 10 & $(31,12)$ & 7 & 1 & YES & YES & YES & $1.75$ & $(4,2)$ & NO & 4420\\
$(56,15)$ & 9 & $(31,7)$ & 8 & 1 & YES & YES & YES & $1.88$ & $(2,3)$ & -- & 4421\\
$(56,15)$ & 9 & $(43,12)$ & 8 & 1 & YES & YES & NO(2) & $1.38$ & $(6,1)$ & NO & 4422\\
$(57,16)$ & 9 & $(8,3)$ & 4 & 1 & YES & YES & YES & $1.50$ & $(2,3)$ & -- & 4423\\
$(57,25)$ & 9 & $(9,4)$ & 5 & 3 & YES & YES & YES & $1.75$ & $(2,3)$ & -- & 4424\\
$(57,13)$ & 9 & $(11,5)$ & 6 & 1 & YES & YES & NO(2) & $1.44$ & $(4,2)$ & -- & 4425\\
$(57,16)$ & 9 & $(11,4)$ & 5 & 1 & YES & YES & YES & $1.38$ & $(2,3)$ & -- & 4426\\
$(57,17)$ & 10 & $(13,5)$ & 5 & 1 & YES & YES & YES & $1.57$ & $(2,3)$ & -- & 4427\\
$(57,25)$ & 9 & $(13,5)$ & 5 & 1 & YES & YES & YES & $1.62$ & $(2,3)$ & -- & 4428\\
$(57,13)$ & 9 & $(17,7)$ & 6 & 1 & YES & YES & NO(2) & $1.38$ & $(6,1)$ & NO & 4429\\
$(57,17)$ & 10 & $(17,3)$ & 7 & 1 & YES & YES & NO(2) & $1.60$ & $(2,3)$ & NO & 4430\\
$(57,25)$ & 9 & $(17,5)$ & 6 & 1 & YES & YES & YES & $1.71$ & $(4,2)$ & -- & 4431\\
$(57,13)$ & 9 & $(18,7)$ & 6 & 3 & YES & YES & NO(2) & $1.38$ & $(4,2)$ & NO & 4432\\
$(57,22)$ & 9 & $(18,5)$ & 6 & 3 & YES & YES & YES & $1.89$ & $(2,3)$ & -- & 4433\\
$(57,13)$ & 9 & $(19,5)$ & 7 & 19 & YES & YES & YES & $1.57$ & $(2,3)$ & -- & 4434\\
$(57,13)$ & 9 & $(19,6)$ & 8 & 19 & YES & YES & NO(2) & $1.50$ & $(6,1)$ & NO & 4435\\
$(57,17)$ & 10 & $(22,5)$ & 7 & 1 & YES & YES & YES & $1.57$ & $(2,3)$ & NO & 4436\\
$(57,22)$ & 9 & $(22,5)$ & 7 & 1 & YES & YES & YES & $1.75$ & $(2,3)$ & -- & 4437\\
$(57,25)$ & 9 & $(22,5)$ & 7 & 1 & YES & YES & YES & $1.71$ & $(4,2)$ & NO & 4438\\
$(57,25)$ & 9 & $(22,5)$ & 7 & 1 & YES & YES & YES & $1.71$ & $(6,1)$ & -- & 4439\\
$(57,10)$ & 10 & $(23,9)$ & 7 & 1 & YES & YES & NO(2) & $1.67$ & $(8,0)$ & -- & 4440\\
$(57,13)$ & 9 & $(23,9)$ & 7 & 1 & YES & YES & YES & $1.78$ & $(4,2)$ & -- & 4441\\
$(57,16)$ & 9 & $(23,9)$ & 7 & 1 & YES & YES & YES & $1.75$ & $(4,2)$ & -- & 4442\\
$(57,22)$ & 9 & $(23,4)$ & 8 & 1 & YES & YES & NO(2) & $1.67$ & $(8,0)$ & NO & 4443\\
$(57,22)$ & 9 & $(23,4)$ & 8 & 1 & YES & YES & NO(2) & $1.67$ & $(8,0)$ & -- & 4444\\
$(57,22)$ & 9 & $(23,5)$ & 7 & 1 & YES & YES & YES & $1.62$ & $(2,3)$ & -- & 4445\\
$(57,22)$ & 9 & $(23,5)$ & 7 & 1 & YES & YES & YES & $1.78$ & $(2,3)$ & NO & 4446\\
$(57,22)$ & 9 & $(23,10)$ & 7 & 1 & YES & YES & NO(2) & $1.50$ & $(6,1)$ & NO & 4447\\
$(57,13)$ & 9 & $(24,7)$ & 7 & 3 & YES & YES & YES & $1.67$ & $(2,3)$ & -- & 4448\\
$(57,13)$ & 9 & $(25,7)$ & 7 & 1 & YES & YES & YES & $1.50$ & $(2,3)$ & -- & 4449\\
$(57,13)$ & 9 & $(27,10)$ & 7 & 3 & YES & YES & YES & $1.43$ & $(4,2)$ & -- & 4450\\
$(57,13)$ & 9 & $(30,11)$ & 7 & 3 & YES & YES & YES & $1.62$ & $(2,3)$ & -- & 4451\\
$(57,10)$ & 10 & $(31,9)$ & 8 & 1 & YES & YES & NO(2) & $1.60$ & $(6,1)$ & -- & 4452\\
$(57,13)$ & 9 & $(31,9)$ & 8 & 1 & YES & YES & YES & $1.75$ & $(2,3)$ & -- & 4453\\
$(57,22)$ & 9 & $(31,7)$ & 8 & 1 & YES & YES & YES & $1.62$ & $(4,2)$ & -- & 4454\\
$(57,13)$ & 9 & $(32,9)$ & 8 & 1 & YES & YES & YES & $1.86$ & $(2,3)$ & -- & 4455\\
$(57,25)$ & 9 & $(39,16)$ & 8 & 3 & YES & YES & YES & $1.78$ & $(4,2)$ & NO & 4456\\
$(57,16)$ & 9 & $(41,11)$ & 8 & 1 & YES & YES & NO(2) & $1.60$ & $(6,1)$ & NO & 4457\\
$(57,22)$ & 9 & $(49,19)$ & 8 & 1 & YES & YES & NO(2) & $1.40$ & $(2,3)$ & 5470 & 4458\\
$(57,17)$ & 10 & $(55,16)$ & 9 & 1 & YES & YES & YES & $1.57$ & $(2,3)$ & NO & 4459\\
$(58,17)$ & 9 & $(7,2)$ & 4 & 1 & YES & YES & NO(2) & $1.44$ & $(4,2)$ & -- & 4460\\
$(58,17)$ & 9 & $(9,4)$ & 5 & 1 & YES & YES & NO(2) & $1.44$ & $(4,2)$ & -- & 4461\\
$(58,17)$ & 9 & $(10,3)$ & 5 & 2 & YES & YES & NO(2) & $1.25$ & $(6,1)$ & -- & 4462\\
$(58,17)$ & 9 & $(12,5)$ & 5 & 2 & YES & YES & NO(2) & $1.64$ & $(4,2)$ & -- & 4463\\
$(58,17)$ & 9 & $(17,4)$ & 7 & 1 & YES & YES & YES & $1.50$ & $(2,3)$ & -- & 4464\\
$(58,17)$ & 9 & $(17,5)$ & 6 & 1 & YES & YES & YES & $1.90$ & $(2,3)$ & -- & 4465\\
$(58,17)$ & 9 & $(17,7)$ & 6 & 1 & YES & YES & YES & $1.89$ & $(2,3)$ & -- & 4466\\
$(58,17)$ & 9 & $(18,5)$ & 6 & 2 & YES & YES & YES & $1.50$ & $(2,3)$ & -- & 4467\\
$(58,17)$ & 9 & $(18,7)$ & 6 & 2 & YES & YES & YES & $1.89$ & $(2,3)$ & -- & 4468\\
$(58,17)$ & 9 & $(19,8)$ & 6 & 1 & YES & YES & YES & $1.67$ & $(2,3)$ & -- & 4469\\
$(58,11)$ & 10 & $(20,7)$ & 8 & 2 & YES & YES & NO(2) & $1.70$ & $(6,1)$ & NO & 4470\\
$(58,17)$ & 9 & $(23,5)$ & 7 & 1 & YES & YES & NO(2) & $1.56$ & $(2,3)$ & -- & 4471\\
$(58,17)$ & 9 & $(24,7)$ & 7 & 2 & YES & YES & YES & $1.50$ & $(2,3)$ & -- & 4472\\
$(58,17)$ & 9 & $(25,7)$ & 7 & 1 & YES & YES & YES & $1.62$ & $(2,3)$ & -- & 4473\\
$(58,17)$ & 9 & $(27,8)$ & 7 & 1 & YES & YES & YES & $1.57$ & $(2,3)$ & -- & 4474\\
$(58,17)$ & 9 & $(29,8)$ & 7 & 29 & YES & YES & YES & $1.57$ & $(2,3)$ & -- & 4475\\
$(58,17)$ & 9 & $(29,9)$ & 8 & 29 & YES & YES & NO(2) & $1.73$ & $(4,2)$ & NO & 4476\\
$(58,17)$ & 9 & $(31,7)$ & 8 & 1 & YES & YES & YES & $1.62$ & $(2,3)$ & -- & 4477\\
$(58,17)$ & 9 & $(32,7)$ & 8 & 2 & YES & YES & YES & $1.38$ & $(4,2)$ & -- & 4478\\
$(58,17)$ & 9 & $(35,8)$ & 8 & 1 & YES & YES & YES & $1.57$ & $(2,3)$ & -- & 4479\\
$(58,17)$ & 9 & $(37,8)$ & 8 & 1 & YES & YES & YES & $1.62$ & $(4,2)$ & NO & 4480\\
$(58,17)$ & 9 & $(55,16)$ & 9 & 1 & YES & YES & NO(2) & $1.70$ & $(2,3)$ & NO & 4481\\
$(59,18)$ & 9 & $(10,3)$ & 5 & 1 & YES & YES & NO(2) & $1.14$ & $(8,0)$ & -- & 4482\\
$(59,23)$ & 9 & $(10,3)$ & 5 & 1 & YES & YES & NO(2) & $1.60$ & $(2,3)$ & -- & 4483\\
$(59,23)$ & 9 & $(11,4)$ & 5 & 1 & YES & YES & NO(2) & $1.62$ & $(6,1)$ & -- & 4484\\
$(59,23)$ & 9 & $(11,4)$ & 5 & 1 & YES & YES & NO(2) & $1.67$ & $(8,0)$ & NO & 4485\\
$(59,18)$ & 9 & $(13,5)$ & 5 & 1 & YES & YES & NO(2) & $1.64$ & $(4,2)$ & -- & 4486\\
$(59,18)$ & 9 & $(16,5)$ & 7 & 1 & YES & YES & NO(2) & $1.56$ & $(4,2)$ & -- & 4487\\
$(59,26)$ & 9 & $(17,5)$ & 6 & 1 & YES & YES & NO(2) & $1.60$ & $(10,-1)$ & NO & 4488\\
$(59,23)$ & 9 & $(20,9)$ & 7 & 1 & YES & YES & NO(2) & $1.50$ & $(6,1)$ & -- & 4489\\
$(59,18)$ & 9 & $(23,7)$ & 7 & 1 & YES & YES & YES & $1.75$ & $(2,3)$ & -- & 4490\\
$(59,23)$ & 9 & $(24,7)$ & 7 & 1 & YES & YES & YES & $1.62$ & $(4,2)$ & -- & 4491\\
$(59,26)$ & 9 & $(25,7)$ & 7 & 1 & YES & YES & YES & $1.43$ & $(6,1)$ & -- & 4492\\
$(59,18)$ & 9 & $(26,7)$ & 7 & 1 & YES & YES & YES & $1.75$ & $(2,3)$ & -- & 4493\\
$(59,26)$ & 9 & $(29,8)$ & 7 & 1 & YES & YES & YES & $1.43$ & $(6,1)$ & -- & 4494\\
$(59,25)$ & 9 & $(32,7)$ & 8 & 1 & YES & YES & YES & $1.62$ & $(4,2)$ & NO & 4495\\
$(59,25)$ & 9 & $(32,7)$ & 8 & 1 & YES & YES & YES & $1.62$ & $(4,2)$ & -- & 4496\\
$(59,23)$ & 9 & $(50,19)$ & 8 & 1 & YES & YES & YES & $1.75$ & $(2,3)$ & NO & 4497\\
$(59,23)$ & 9 & $(55,21)$ & 8 & 1 & YES & YES & YES & $1.75$ & $(2,3)$ & NO & 4498\\
$(60,23)$ & 9 & $(8,3)$ & 4 & 4 & YES & YES & NO(2) & $1.70$ & $(6,1)$ & -- & 4499\\
$(60,23)$ & 9 & $(10,3)$ & 5 & 10 & YES & YES & NO(2) & $1.70$ & $(6,1)$ & -- & 4500\\
$(60,13)$ & 9 & $(11,5)$ & 6 & 1 & YES & YES & NO(2) & $1.44$ & $(4,2)$ & -- & 4501\\
$(60,23)$ & 9 & $(12,5)$ & 5 & 12 & YES & YES & NO(2) & $1.78$ & $(2,3)$ & -- & 4502\\
$(60,13)$ & 9 & $(14,5)$ & 6 & 2 & YES & YES & NO(2) & $1.70$ & $(2,3)$ & NO & 4503\\
$(60,13)$ & 9 & $(17,7)$ & 6 & 1 & YES & YES & NO(2) & $1.44$ & $(4,2)$ & -- & 4504\\
$(60,23)$ & 9 & $(17,5)$ & 6 & 1 & YES & YES & YES & $1.78$ & $(2,3)$ & -- & 4505\\
$(60,13)$ & 9 & $(18,7)$ & 6 & 6 & YES & YES & YES & $1.43$ & $(2,3)$ & -- & 4506\\
$(60,23)$ & 9 & $(18,5)$ & 6 & 6 & YES & YES & YES & $2.00$ & $(2,3)$ & -- & 4507\\
$(60,11)$ & 11 & $(19,5)$ & 7 & 1 & YES & YES & YES & $1.62$ & $(2,3)$ & -- & 4508\\
$(60,13)$ & 9 & $(22,9)$ & 7 & 2 & YES & YES & YES & $1.43$ & $(4,2)$ & NO & 4509\\
$(60,13)$ & 9 & $(22,9)$ & 7 & 2 & YES & YES & YES & $2.00$ & $(2,3)$ & -- & 4510\\
$(60,13)$ & 9 & $(23,9)$ & 7 & 1 & YES & YES & YES & $1.67$ & $(2,3)$ & NO & 4511\\
$(60,13)$ & 9 & $(23,9)$ & 7 & 1 & YES & YES & YES & $1.67$ & $(2,3)$ & -- & 4512\\
$(60,23)$ & 9 & $(23,5)$ & 7 & 1 & YES & YES & YES & $1.62$ & $(2,3)$ & -- & 4513\\
$(60,13)$ & 9 & $(25,7)$ & 7 & 5 & YES & YES & NO(2) & $1.44$ & $(8,0)$ & NO & 4514\\
$(60,23)$ & 9 & $(27,5)$ & 8 & 3 & YES & YES & YES & $1.70$ & $(2,3)$ & NO & 4515\\
$(60,13)$ & 9 & $(29,11)$ & 7 & 1 & YES & YES & YES & $1.62$ & $(4,2)$ & -- & 4516\\
$(60,13)$ & 9 & $(31,9)$ & 8 & 1 & YES & YES & YES & $1.67$ & $(2,3)$ & NO & 4517\\
$(60,13)$ & 9 & $(32,9)$ & 8 & 4 & YES & YES & YES & $1.43$ & $(4,2)$ & NO & 4518\\
$(60,13)$ & 9 & $(34,9)$ & 8 & 2 & YES & YES & NO(2) & $1.56$ & $(4,2)$ & NO & 4519\\
$(61,17)$ & 9 & $(11,4)$ & 5 & 1 & YES & YES & NO(2) & $1.56$ & $(4,2)$ & -- & 4520\\
$(61,17)$ & 9 & $(11,4)$ & 5 & 1 & YES & YES & NO(2) & $1.56$ & $(4,2)$ & NO & 4521\\
$(61,18)$ & 9 & $(12,5)$ & 5 & 1 & YES & YES & NO(2) & $1.44$ & $(8,0)$ & -- & 4522\\
$(61,17)$ & 9 & $(13,5)$ & 5 & 1 & YES & YES & YES & $1.29$ & $(4,2)$ & NO & 4523\\
$(61,25)$ & 9 & $(13,4)$ & 6 & 1 & YES & YES & YES & $1.57$ & $(2,3)$ & -- & 4524\\
$(61,25)$ & 9 & $(13,4)$ & 6 & 1 & YES & YES & NO(2) & $1.67$ & $(6,1)$ & NO & 4525\\
$(61,25)$ & 9 & $(13,4)$ & 6 & 1 & YES & YES & NO(2) & $1.70$ & $(10,-1)$ & NO & 4526\\
$(61,18)$ & 9 & $(15,4)$ & 6 & 1 & YES & YES & YES & $1.62$ & $(2,3)$ & -- & 4527\\
$(61,25)$ & 9 & $(15,4)$ & 6 & 1 & YES & YES & NO(2) & $1.60$ & $(6,1)$ & -- & 4528\\
$(61,17)$ & 9 & $(17,7)$ & 6 & 1 & YES & YES & YES & $1.89$ & $(2,3)$ & -- & 4529\\
$(61,18)$ & 9 & $(17,4)$ & 7 & 1 & YES & YES & YES & $1.50$ & $(2,3)$ & -- & 4530\\
$(61,18)$ & 9 & $(17,5)$ & 6 & 1 & YES & YES & YES & $1.38$ & $(2,3)$ & -- & 4531\\
$(61,18)$ & 9 & $(17,5)$ & 6 & 1 & YES & YES & YES & $1.43$ & $(4,2)$ & NO & 4532\\
$(61,18)$ & 9 & $(17,7)$ & 6 & 1 & YES & YES & YES & $1.29$ & $(4,2)$ & -- & 4533\\
$(61,18)$ & 9 & $(17,7)$ & 6 & 1 & YES & YES & YES & $1.62$ & $(4,2)$ & NO & 4534\\
$(61,22)$ & 9 & $(17,5)$ & 6 & 1 & YES & YES & YES & $1.71$ & $(4,2)$ & -- & 4535\\
$(61,25)$ & 9 & $(17,5)$ & 6 & 1 & YES & YES & YES & $1.75$ & $(2,3)$ & -- & 4536\\
$(61,17)$ & 9 & $(18,7)$ & 6 & 1 & YES & YES & YES & $1.89$ & $(2,3)$ & -- & 4537\\
$(61,17)$ & 9 & $(18,7)$ & 6 & 1 & YES & YES & YES & $1.78$ & $(2,3)$ & NO & 4538\\
$(61,18)$ & 9 & $(18,5)$ & 6 & 1 & YES & YES & YES & $1.78$ & $(2,3)$ & -- & 4539\\
$(61,18)$ & 9 & $(18,7)$ & 6 & 1 & YES & YES & YES & $1.90$ & $(2,3)$ & -- & 4540\\
$(61,22)$ & 9 & $(18,5)$ & 6 & 1 & YES & YES & YES & $1.57$ & $(4,2)$ & -- & 4541\\
$(61,25)$ & 9 & $(18,5)$ & 6 & 1 & YES & YES & YES & $1.89$ & $(2,3)$ & -- & 4542\\
$(61,17)$ & 9 & $(19,8)$ & 6 & 1 & YES & YES & YES & $1.67$ & $(2,3)$ & -- & 4543\\
$(61,25)$ & 9 & $(20,9)$ & 7 & 1 & YES & YES & NO(2) & $1.62$ & $(6,1)$ & NO & 4544\\
$(61,17)$ & 9 & $(21,8)$ & 6 & 1 & YES & YES & YES & $1.78$ & $(2,3)$ & -- & 4545\\
$(61,17)$ & 9 & $(23,9)$ & 7 & 1 & YES & YES & YES & $1.75$ & $(4,2)$ & -- & 4546\\
$(61,18)$ & 9 & $(23,7)$ & 7 & 1 & YES & YES & YES & $1.67$ & $(2,3)$ & -- & 4547\\
$(61,22)$ & 9 & $(23,7)$ & 7 & 1 & YES & YES & YES & $1.89$ & $(4,2)$ & -- & 4548\\
$(61,17)$ & 9 & $(24,7)$ & 7 & 1 & YES & YES & YES & $1.78$ & $(2,3)$ & -- & 4549\\
$(61,17)$ & 9 & $(25,7)$ & 7 & 1 & YES & YES & YES & $1.75$ & $(2,3)$ & -- & 4550\\
$(61,18)$ & 9 & $(25,7)$ & 7 & 1 & YES & YES & NO(2) & $1.82$ & $(4,2)$ & NO & 4551\\
$(61,17)$ & 9 & $(27,8)$ & 7 & 1 & YES & YES & YES & $1.57$ & $(2,3)$ & -- & 4552\\
$(61,14)$ & 10 & $(29,8)$ & 7 & 1 & YES & YES & YES & $1.67$ & $(4,2)$ & -- & 4553\\
$(61,17)$ & 9 & $(29,8)$ & 7 & 1 & YES & YES & YES & $1.57$ & $(2,3)$ & -- & 4554\\
$(61,18)$ & 9 & $(33,7)$ & 8 & 1 & YES & YES & YES & $1.43$ & $(4,2)$ & -- & 4555\\
$(61,22)$ & 9 & $(41,16)$ & 8 & 1 & YES & YES & YES & $1.89$ & $(4,2)$ & NO & 4556\\
$(61,18)$ & 9 & $(42,13)$ & 9 & 1 & YES & YES & YES & $1.57$ & $(2,3)$ & NO & 4557\\
$(61,17)$ & 9 & $(57,16)$ & 9 & 1 & YES & YES & YES & $1.62$ & $(2,3)$ & NO & 4558\\
$(62,27)$ & 9 & $(7,2)$ & 4 & 1 & YES & YES & YES & $1.29$ & $(4,2)$ & -- & 4559\\
$(62,27)$ & 9 & $(7,2)$ & 4 & 1 & YES & YES & NO(2) & $1.60$ & $(6,1)$ & NO & 4560\\
$(62,23)$ & 9 & $(10,3)$ & 5 & 2 & YES & YES & YES & $1.75$ & $(2,3)$ & -- & 4561\\
$(62,27)$ & 9 & $(10,3)$ & 5 & 2 & YES & YES & NO(2) & $1.78$ & $(4,2)$ & NO & 4562\\
$(62,17)$ & 10 & $(11,3)$ & 5 & 1 & YES & YES & NO(2) & $1.44$ & $(12,-2)$ & -- & 4563\\
$(62,17)$ & 10 & $(12,5)$ & 5 & 2 & YES & YES & YES & $1.71$ & $(2,3)$ & -- & 4564\\
$(62,23)$ & 9 & $(13,5)$ & 5 & 1 & YES & YES & NO(2) & $1.67$ & $(8,0)$ & -- & 4565\\
$(62,27)$ & 9 & $(13,5)$ & 5 & 1 & YES & YES & YES & $1.29$ & $(4,2)$ & -- & 4566\\
$(62,27)$ & 9 & $(17,7)$ & 6 & 1 & YES & YES & YES & $1.88$ & $(4,2)$ & -- & 4567\\
$(62,13)$ & 10 & $(18,7)$ & 6 & 2 & YES & YES & NO(2) & $1.56$ & $(8,0)$ & -- & 4568\\
$(62,27)$ & 9 & $(18,5)$ & 6 & 2 & YES & YES & YES & $1.57$ & $(4,2)$ & -- & 4569\\
$(62,27)$ & 9 & $(18,7)$ & 6 & 2 & YES & YES & YES & $1.88$ & $(4,2)$ & -- & 4570\\
$(62,17)$ & 10 & $(19,7)$ & 6 & 1 & YES & YES & YES & $1.71$ & $(6,1)$ & -- & 4571\\
$(62,27)$ & 9 & $(19,7)$ & 6 & 1 & YES & YES & YES & $1.75$ & $(4,2)$ & -- & 4572\\
$(62,27)$ & 9 & $(21,8)$ & 6 & 1 & YES & YES & YES & $1.29$ & $(4,2)$ & NO & 4573\\
$(62,17)$ & 10 & $(57,16)$ & 9 & 1 & YES & YES & YES & $1.71$ & $(2,3)$ & NO & 4574\\
$(63,26)$ & 9 & $(8,3)$ & 4 & 1 & YES & YES & NO(2) & $1.56$ & $(8,0)$ & -- & 4575\\
$(63,26)$ & 9 & $(10,3)$ & 5 & 1 & YES & YES & NO(2) & $1.62$ & $(4,2)$ & NO & 4576\\
$(63,26)$ & 9 & $(10,3)$ & 5 & 1 & YES & YES & NO(2) & $1.80$ & $(2,3)$ & -- & 4577\\
$(63,23)$ & 10 & $(11,4)$ & 5 & 1 & YES & YES & NO(2) & $1.78$ & $(4,2)$ & -- & 4578\\
$(63,26)$ & 9 & $(11,3)$ & 5 & 1 & YES & YES & NO(2) & $1.67$ & $(4,2)$ & -- & 4579\\
$(63,26)$ & 9 & $(11,4)$ & 5 & 1 & YES & YES & YES & $1.57$ & $(2,3)$ & -- & 4580\\
$(63,17)$ & 9 & $(12,5)$ & 5 & 3 & YES & YES & NO(2) & $1.44$ & $(8,0)$ & -- & 4581\\
$(63,23)$ & 10 & $(12,5)$ & 5 & 3 & YES & YES & NO(2) & $1.70$ & $(6,1)$ & -- & 4582\\
$(63,17)$ & 9 & $(13,5)$ & 5 & 1 & YES & YES & YES & $1.71$ & $(2,3)$ & -- & 4583\\
$(63,23)$ & 10 & $(15,4)$ & 6 & 3 & YES & YES & NO(2) & $1.67$ & $(8,0)$ & NO & 4584\\
$(63,26)$ & 9 & $(16,7)$ & 6 & 1 & YES & YES & NO(2) & $1.70$ & $(6,1)$ & -- & 4585\\
$(63,26)$ & 9 & $(17,5)$ & 6 & 1 & YES & YES & YES & $1.67$ & $(4,2)$ & NO & 4586\\
$(63,26)$ & 9 & $(17,5)$ & 6 & 1 & YES & YES & YES & $1.67$ & $(4,2)$ & -- & 4587\\
$(63,13)$ & 11 & $(18,7)$ & 6 & 9 & YES & YES & NO(2) & $1.62$ & $(6,1)$ & NO & 4588\\
$(63,17)$ & 9 & $(18,7)$ & 6 & 9 & YES & YES & YES & $1.50$ & $(2,3)$ & -- & 4589\\
$(63,26)$ & 9 & $(18,5)$ & 6 & 9 & YES & YES & YES & $1.67$ & $(4,2)$ & NO & 4590\\
$(63,26)$ & 9 & $(18,5)$ & 6 & 9 & YES & YES & YES & $1.67$ & $(4,2)$ & -- & 4591\\
$(63,19)$ & 11 & $(20,3)$ & 8 & 1 & YES & YES & NO(2) & $1.50$ & $(6,1)$ & NO & 4592\\
$(63,17)$ & 9 & $(21,8)$ & 6 & 21 & YES & YES & YES & $1.89$ & $(2,3)$ & -- & 4593\\
$(63,17)$ & 9 & $(21,8)$ & 6 & 21 & YES & YES & YES & $2.00$ & $(2,3)$ & NO & 4594\\
$(63,17)$ & 9 & $(22,9)$ & 7 & 1 & YES & YES & YES & $1.62$ & $(4,2)$ & -- & 4595\\
$(63,11)$ & 10 & $(23,6)$ & 8 & 1 & YES & YES & NO(2) & $1.56$ & $(8,0)$ & -- & 4596\\
$(63,11)$ & 10 & $(23,6)$ & 8 & 1 & YES & YES & NO(2) & $1.56$ & $(8,0)$ & NO & 4597\\
$(63,26)$ & 9 & $(23,5)$ & 7 & 1 & YES & YES & YES & $1.56$ & $(2,3)$ & NO & 4598\\
$(63,26)$ & 9 & $(23,7)$ & 7 & 1 & YES & YES & YES & $1.78$ & $(4,2)$ & -- & 4599\\
$(63,17)$ & 9 & $(24,7)$ & 7 & 3 & YES & YES & YES & $1.50$ & $(2,3)$ & -- & 4600\\
$(63,17)$ & 9 & $(25,7)$ & 7 & 1 & YES & YES & YES & $1.86$ & $(2,3)$ & -- & 4601\\
$(63,17)$ & 9 & $(27,8)$ & 7 & 9 & YES & YES & YES & $1.62$ & $(2,3)$ & -- & 4602\\
$(63,11)$ & 10 & $(29,7)$ & 10 & 1 & YES & YES & NO(2) & $1.60$ & $(6,1)$ & -- & 4603\\
$(63,11)$ & 10 & $(29,7)$ & 10 & 1 & YES & YES & NO(2) & $1.60$ & $(6,1)$ & NO & 4604\\
$(63,26)$ & 9 & $(41,17)$ & 8 & 1 & YES & YES & NO(2) & $1.50$ & $(2,3)$ & 5677 & 4605\\
$(63,17)$ & 9 & $(44,13)$ & 8 & 1 & YES & YES & YES & $1.67$ & $(2,3)$ & NO & 4606\\
$(64,19)$ & 9 & $(7,3)$ & 4 & 1 & YES & YES & NO(2) & $1.33$ & $(4,2)$ & -- & 4607\\
$(64,27)$ & 9 & $(7,2)$ & 4 & 1 & YES & YES & YES & $1.50$ & $(2,3)$ & -- & 4608\\
$(64,19)$ & 9 & $(8,3)$ & 4 & 8 & YES & YES & NO(2) & $1.60$ & $(6,1)$ & -- & 4609\\
$(64,23)$ & 9 & $(8,3)$ & 4 & 8 & YES & YES & YES & $1.29$ & $(4,2)$ & -- & 4610\\
$(64,27)$ & 9 & $(10,3)$ & 5 & 2 & YES & YES & NO(2) & $1.62$ & $(4,2)$ & -- & 4611\\
$(64,19)$ & 9 & $(11,4)$ & 5 & 1 & YES & YES & YES & $1.43$ & $(2,3)$ & -- & 4612\\
$(64,27)$ & 9 & $(11,3)$ & 5 & 1 & YES & YES & NO(2) & $1.56$ & $(2,3)$ & -- & 4613\\
$(64,17)$ & 10 & $(13,4)$ & 6 & 1 & YES & YES & YES & $1.57$ & $(2,3)$ & -- & 4614\\
$(64,27)$ & 9 & $(13,5)$ & 5 & 1 & YES & YES & YES & $1.75$ & $(2,3)$ & -- & 4615\\
$(64,19)$ & 9 & $(14,5)$ & 6 & 2 & YES & YES & YES & $1.43$ & $(4,2)$ & -- & 4616\\
$(64,27)$ & 9 & $(14,5)$ & 6 & 2 & YES & YES & NO(2) & $1.56$ & $(8,0)$ & -- & 4617\\
$(64,17)$ & 10 & $(15,4)$ & 6 & 1 & YES & YES & YES & $1.57$ & $(2,3)$ & -- & 4618\\
$(64,19)$ & 9 & $(16,7)$ & 6 & 16 & YES & YES & YES & $1.43$ & $(4,2)$ & -- & 4619\\
$(64,19)$ & 9 & $(17,5)$ & 6 & 1 & YES & YES & YES & $1.67$ & $(2,3)$ & -- & 4620\\
$(64,19)$ & 9 & $(17,7)$ & 6 & 1 & YES & YES & YES & $1.71$ & $(2,3)$ & -- & 4621\\
$(64,27)$ & 9 & $(17,5)$ & 6 & 1 & YES & YES & YES & $1.88$ & $(2,3)$ & -- & 4622\\
$(64,19)$ & 9 & $(18,7)$ & 6 & 2 & YES & YES & YES & $1.75$ & $(2,3)$ & -- & 4623\\
$(64,27)$ & 9 & $(18,5)$ & 6 & 2 & YES & YES & YES & $1.57$ & $(4,2)$ & -- & 4624\\
$(64,17)$ & 10 & $(19,7)$ & 6 & 1 & YES & YES & NO(2) & $1.70$ & $(6,1)$ & NO & 4625\\
$(64,19)$ & 9 & $(23,7)$ & 7 & 1 & YES & YES & YES & $1.70$ & $(2,3)$ & -- & 4626\\
$(64,23)$ & 9 & $(23,9)$ & 7 & 1 & YES & YES & YES & $1.57$ & $(2,3)$ & NO & 4627\\
$(64,19)$ & 9 & $(24,7)$ & 7 & 8 & YES & YES & YES & $1.70$ & $(2,3)$ & -- & 4628\\
$(64,27)$ & 9 & $(39,16)$ & 8 & 1 & YES & YES & NO(2) & $1.56$ & $(4,2)$ & NO & 4629\\
$(64,25)$ & 9 & $(47,18)$ & 8 & 1 & YES & YES & NO(2) & $1.60$ & $(6,1)$ & NO & 4630\\
$(65,19)$ & 9 & $(5,2)$ & 3 & 5 & YES & YES & NO(2) & $1.44$ & $(4,2)$ & -- & 4631\\
$(65,18)$ & 9 & $(7,3)$ & 4 & 1 & YES & YES & NO(2) & $1.33$ & $(4,2)$ & -- & 4632\\
$(65,19)$ & 9 & $(7,3)$ & 4 & 1 & YES & YES & NO(2) & $1.70$ & $(2,3)$ & -- & 4633\\
$(65,24)$ & 9 & $(8,3)$ & 4 & 1 & YES & YES & NO(2) & $1.56$ & $(4,2)$ & -- & 4634\\
$(65,24)$ & 9 & $(10,3)$ & 5 & 5 & YES & YES & NO(2) & $1.56$ & $(4,2)$ & -- & 4635\\
$(65,24)$ & 9 & $(10,3)$ & 5 & 5 & YES & YES & NO(2) & $1.50$ & $(6,1)$ & NO & 4636\\
$(65,19)$ & 9 & $(11,4)$ & 5 & 1 & YES & YES & YES & $1.29$ & $(4,2)$ & -- & 4637\\
$(65,27)$ & 10 & $(11,3)$ & 5 & 1 & YES & YES & YES & $1.43$ & $(4,2)$ & NO & 4638\\
$(65,27)$ & 10 & $(11,3)$ & 5 & 1 & YES & YES & NO(2) & $1.80$ & $(2,3)$ & -- & 4639\\
$(65,19)$ & 9 & $(12,5)$ & 5 & 1 & YES & YES & YES & $1.57$ & $(2,3)$ & -- & 4640\\
$(65,19)$ & 9 & $(12,5)$ & 5 & 1 & YES & YES & NO(2) & $1.67$ & $(2,3)$ & NO & 4641\\
$(65,18)$ & 9 & $(13,5)$ & 5 & 13 & YES & YES & NO(2) & $1.38$ & $(4,2)$ & -- & 4642\\
$(65,18)$ & 9 & $(13,5)$ & 5 & 13 & YES & YES & NO(2) & $1.70$ & $(2,3)$ & NO & 4643\\
$(65,19)$ & 9 & $(13,4)$ & 6 & 13 & YES & YES & YES & $1.29$ & $(4,2)$ & -- & 4644\\
$(65,19)$ & 9 & $(13,5)$ & 5 & 13 & YES & YES & YES & $1.50$ & $(2,3)$ & -- & 4645\\
$(65,24)$ & 9 & $(13,5)$ & 5 & 13 & YES & YES & YES & $1.67$ & $(2,3)$ & -- & 4646\\
$(65,24)$ & 9 & $(14,5)$ & 6 & 1 & YES & YES & NO(2) & $1.50$ & $(6,1)$ & -- & 4647\\
$(65,18)$ & 9 & $(15,4)$ & 6 & 5 & YES & YES & YES & $1.62$ & $(2,3)$ & -- & 4648\\
$(65,19)$ & 9 & $(15,4)$ & 6 & 5 & YES & YES & YES & $1.14$ & $(4,2)$ & -- & 4649\\
$(65,18)$ & 9 & $(17,4)$ & 7 & 1 & YES & YES & YES & $1.50$ & $(2,3)$ & -- & 4650\\
$(65,18)$ & 9 & $(17,7)$ & 6 & 1 & YES & YES & YES & $1.67$ & $(2,3)$ & -- & 4651\\
$(65,19)$ & 9 & $(17,5)$ & 6 & 1 & YES & YES & YES & $1.57$ & $(2,3)$ & -- & 4652\\
$(65,18)$ & 9 & $(18,5)$ & 6 & 1 & YES & YES & YES & $1.78$ & $(2,3)$ & -- & 4653\\
$(65,18)$ & 9 & $(18,7)$ & 6 & 1 & YES & YES & YES & $1.62$ & $(4,2)$ & -- & 4654\\
$(65,19)$ & 9 & $(18,5)$ & 6 & 1 & YES & YES & YES & $1.71$ & $(2,3)$ & -- & 4655\\
$(65,27)$ & 10 & $(19,3)$ & 8 & 1 & YES & YES & YES & $1.57$ & $(2,3)$ & -- & 4656\\
$(65,12)$ & 10 & $(22,9)$ & 7 & 1 & YES & YES & NO(2) & $1.60$ & $(6,1)$ & NO & 4657\\
$(65,12)$ & 10 & $(22,9)$ & 7 & 1 & YES & YES & NO(2) & $1.60$ & $(6,1)$ & NO & 4658\\
$(65,19)$ & 9 & $(22,5)$ & 7 & 1 & YES & YES & YES & $1.50$ & $(2,3)$ & -- & 4659\\
$(65,24)$ & 9 & $(22,9)$ & 7 & 1 & YES & YES & NO(2) & $1.62$ & $(6,1)$ & NO & 4660\\
$(65,18)$ & 9 & $(23,7)$ & 7 & 1 & YES & YES & NO(2) & $1.70$ & $(2,3)$ & NO & 4661\\
$(65,19)$ & 9 & $(24,7)$ & 7 & 1 & YES & YES & YES & $1.43$ & $(2,3)$ & -- & 4662\\
$(65,18)$ & 9 & $(25,7)$ & 7 & 5 & YES & YES & YES & $1.71$ & $(2,3)$ & -- & 4663\\
$(65,19)$ & 9 & $(25,7)$ & 7 & 5 & YES & YES & YES & $1.71$ & $(2,3)$ & -- & 4664\\
$(65,19)$ & 9 & $(35,8)$ & 8 & 5 & YES & YES & YES & $1.62$ & $(4,2)$ & NO & 4665\\
$(65,19)$ & 9 & $(35,8)$ & 8 & 5 & YES & YES & YES & $1.71$ & $(2,3)$ & -- & 4666\\
$(65,24)$ & 9 & $(63,23)$ & 10 & 1 & YES & YES & NO(2) & $1.78$ & $(4,2)$ & 8470 & 4667\\
$(66,25)$ & 9 & $(7,2)$ & 4 & 1 & YES & YES & NO(2) & $1.50$ & $(2,3)$ & -- & 4668\\
$(66,25)$ & 9 & $(8,3)$ & 4 & 2 & YES & YES & NO(2) & $1.70$ & $(2,3)$ & -- & 4669\\
$(66,25)$ & 9 & $(10,3)$ & 5 & 2 & YES & YES & NO(2) & $1.56$ & $(4,2)$ & -- & 4670\\
$(66,29)$ & 9 & $(10,3)$ & 5 & 2 & YES & YES & NO(2) & $1.60$ & $(6,1)$ & -- & 4671\\
$(66,29)$ & 9 & $(11,4)$ & 5 & 11 & YES & YES & NO(2) & $1.50$ & $(10,-1)$ & -- & 4672\\
$(66,25)$ & 9 & $(12,5)$ & 5 & 6 & YES & YES & YES & $1.57$ & $(4,2)$ & -- & 4673\\
$(66,25)$ & 9 & $(13,5)$ & 5 & 1 & YES & YES & YES & $1.88$ & $(4,2)$ & -- & 4674\\
$(66,29)$ & 9 & $(13,4)$ & 6 & 1 & YES & YES & YES & $1.57$ & $(2,3)$ & -- & 4675\\
$(66,29)$ & 9 & $(13,5)$ & 5 & 1 & YES & YES & NO(2) & $1.70$ & $(6,1)$ & NO & 4676\\
$(66,29)$ & 9 & $(13,5)$ & 5 & 1 & YES & YES & NO(2) & $1.60$ & $(10,-1)$ & -- & 4677\\
$(66,25)$ & 9 & $(16,7)$ & 6 & 2 & YES & YES & NO(2) & $1.70$ & $(6,1)$ & -- & 4678\\
$(66,25)$ & 9 & $(17,4)$ & 7 & 1 & YES & YES & NO(2) & $1.38$ & $(10,-1)$ & -- & 4679\\
$(66,25)$ & 9 & $(17,4)$ & 7 & 1 & YES & YES & NO(2) & $1.50$ & $(10,-1)$ & NO & 4680\\
$(66,25)$ & 9 & $(17,5)$ & 6 & 1 & YES & YES & YES & $1.78$ & $(4,2)$ & -- & 4681\\
$(66,29)$ & 9 & $(17,4)$ & 7 & 1 & YES & YES & NO(2) & $1.38$ & $(10,-1)$ & -- & 4682\\
$(66,25)$ & 9 & $(19,7)$ & 6 & 1 & YES & YES & YES & $1.88$ & $(4,2)$ & -- & 4683\\
$(66,25)$ & 9 & $(20,9)$ & 7 & 2 & YES & YES & NO(2) & $1.70$ & $(6,1)$ & NO & 4684\\
$(66,25)$ & 9 & $(21,5)$ & 8 & 3 & YES & YES & NO(2) & $1.44$ & $(8,0)$ & -- & 4685\\
$(66,25)$ & 9 & $(22,5)$ & 7 & 22 & YES & YES & YES & $1.75$ & $(2,3)$ & -- & 4686\\
$(66,29)$ & 9 & $(22,9)$ & 7 & 22 & YES & YES & YES & $1.43$ & $(2,3)$ & NO & 4687\\
$(66,29)$ & 9 & $(29,12)$ & 7 & 1 & YES & YES & YES & $1.43$ & $(4,2)$ & NO & 4688\\
$(66,25)$ & 9 & $(34,13)$ & 7 & 2 & YES & YES & NO(2) & $1.50$ & $(2,3)$ & NO & 4689\\
$(66,25)$ & 9 & $(55,21)$ & 8 & 11 & YES & YES & NO(2) & $1.56$ & $(4,2)$ & NO & 4690\\
$(67,18)$ & 9 & $(7,2)$ & 4 & 1 & YES & YES & NO(2) & $1.60$ & $(6,1)$ & -- & 4691\\
$(67,18)$ & 9 & $(7,2)$ & 4 & 1 & YES & YES & NO(2) & $1.70$ & $(6,1)$ & NO & 4692\\
$(67,26)$ & 9 & $(8,3)$ & 4 & 1 & YES & YES & NO(2) & $1.60$ & $(10,-1)$ & -- & 4693\\
$(67,28)$ & 10 & $(8,3)$ & 4 & 1 & YES & YES & YES & $1.50$ & $(2,3)$ & -- & 4694\\
$(67,28)$ & 10 & $(8,3)$ & 4 & 1 & YES & YES & YES & $1.62$ & $(2,3)$ & NO & 4695\\
$(67,18)$ & 9 & $(9,4)$ & 5 & 1 & YES & YES & NO(2) & $1.38$ & $(6,1)$ & -- & 4696\\
$(67,26)$ & 9 & $(9,4)$ & 5 & 1 & YES & YES & NO(2) & $1.78$ & $(8,0)$ & -- & 4697\\
$(67,26)$ & 9 & $(10,3)$ & 5 & 1 & YES & YES & NO(2) & $1.50$ & $(4,2)$ & -- & 4698\\
$(67,26)$ & 9 & $(10,3)$ & 5 & 1 & YES & YES & NO(2) & $1.50$ & $(4,2)$ & NO & 4699\\
$(67,28)$ & 10 & $(10,3)$ & 5 & 1 & YES & YES & NO(2) & $1.56$ & $(4,2)$ & -- & 4700\\
$(67,18)$ & 9 & $(11,5)$ & 6 & 1 & YES & YES & NO(2) & $1.50$ & $(6,1)$ & -- & 4701\\
$(67,18)$ & 9 & $(11,5)$ & 6 & 1 & YES & YES & NO(2) & $1.67$ & $(4,2)$ & NO & 4702\\
$(67,28)$ & 10 & $(11,3)$ & 5 & 1 & YES & YES & YES & $1.62$ & $(2,3)$ & NO & 4703\\
$(67,28)$ & 10 & $(11,3)$ & 5 & 1 & YES & YES & YES & $1.62$ & $(2,3)$ & -- & 4704\\
$(67,28)$ & 10 & $(12,5)$ & 5 & 1 & YES & YES & NO(2) & $1.62$ & $(6,1)$ & -- & 4705\\
$(67,18)$ & 9 & $(13,5)$ & 5 & 1 & YES & YES & YES & $1.43$ & $(2,3)$ & -- & 4706\\
$(67,18)$ & 9 & $(13,6)$ & 7 & 1 & YES & YES & NO(2) & $1.50$ & $(6,1)$ & NO & 4707\\
$(67,18)$ & 9 & $(17,5)$ & 6 & 1 & YES & YES & NO(2) & $1.25$ & $(6,1)$ & -- & 4708\\
$(67,18)$ & 9 & $(17,6)$ & 7 & 1 & YES & YES & NO(2) & $1.50$ & $(6,1)$ & NO & 4709\\
$(67,26)$ & 9 & $(17,5)$ & 6 & 1 & YES & YES & YES & $1.62$ & $(4,2)$ & -- & 4710\\
$(67,18)$ & 9 & $(18,7)$ & 6 & 1 & YES & YES & YES & $1.80$ & $(2,3)$ & -- & 4711\\
$(67,18)$ & 9 & $(18,7)$ & 6 & 1 & YES & YES & YES & $1.67$ & $(2,3)$ & NO & 4712\\
$(67,26)$ & 9 & $(18,5)$ & 6 & 1 & YES & YES & YES & $1.78$ & $(4,2)$ & -- & 4713\\
$(67,18)$ & 9 & $(19,6)$ & 8 & 1 & YES & YES & NO(2) & $1.50$ & $(6,1)$ & NO & 4714\\
$(67,24)$ & 10 & $(21,8)$ & 6 & 1 & YES & YES & NO(2) & $1.70$ & $(2,3)$ & NO & 4715\\
$(67,28)$ & 10 & $(21,8)$ & 6 & 1 & YES & YES & NO(2) & $1.62$ & $(6,1)$ & NO & 4716\\
$(67,26)$ & 9 & $(22,5)$ & 7 & 1 & YES & YES & YES & $1.78$ & $(4,2)$ & -- & 4717\\
$(67,14)$ & 10 & $(23,7)$ & 7 & 1 & YES & YES & NO(2) & $1.33$ & $(8,0)$ & -- & 4718\\
$(67,24)$ & 10 & $(35,13)$ & 8 & 1 & YES & YES & NO(2) & $1.50$ & $(6,1)$ & NO & 4719\\
$(67,18)$ & 9 & $(37,11)$ & 8 & 1 & YES & YES & YES & $1.67$ & $(2,3)$ & NO & 4720\\
$(67,18)$ & 9 & $(43,12)$ & 8 & 1 & YES & YES & NO(2) & $1.38$ & $(6,1)$ & NO & 4721\\
$(67,26)$ & 9 & $(50,19)$ & 8 & 1 & YES & YES & YES & $1.89$ & $(2,3)$ & NO & 4722\\
$(67,18)$ & 9 & $(61,17)$ & 9 & 1 & YES & YES & YES & $1.67$ & $(2,3)$ & NO & 4723\\
$(68,19)$ & 9 & $(7,3)$ & 4 & 1 & YES & YES & NO(2) & $1.56$ & $(8,0)$ & -- & 4724\\
$(68,25)$ & 9 & $(8,3)$ & 4 & 4 & YES & YES & NO(2) & $1.60$ & $(6,1)$ & -- & 4725\\
$(68,19)$ & 9 & $(9,4)$ & 5 & 1 & YES & YES & NO(2) & $1.56$ & $(8,0)$ & NO & 4726\\
$(68,19)$ & 9 & $(9,4)$ & 5 & 1 & YES & YES & NO(2) & $1.56$ & $(8,0)$ & -- & 4727\\
$(68,19)$ & 9 & $(10,3)$ & 5 & 2 & YES & YES & NO(2) & $1.44$ & $(4,2)$ & -- & 4728\\
$(68,19)$ & 9 & $(10,3)$ & 5 & 2 & YES & YES & YES & $1.62$ & $(2,3)$ & NO & 4729\\
$(68,25)$ & 9 & $(10,3)$ & 5 & 2 & YES & YES & NO(2) & $1.67$ & $(4,2)$ & -- & 4730\\
$(68,19)$ & 9 & $(11,4)$ & 5 & 1 & YES & YES & NO(2) & $1.38$ & $(6,1)$ & -- & 4731\\
$(68,25)$ & 9 & $(11,3)$ & 5 & 1 & YES & YES & NO(2) & $1.70$ & $(6,1)$ & -- & 4732\\
$(68,19)$ & 9 & $(12,5)$ & 5 & 4 & YES & YES & NO(2) & $1.50$ & $(2,3)$ & -- & 4733\\
$(68,19)$ & 9 & $(13,4)$ & 6 & 1 & YES & YES & YES & $1.43$ & $(4,2)$ & -- & 4734\\
$(68,19)$ & 9 & $(13,6)$ & 7 & 1 & YES & YES & NO(2) & $1.50$ & $(6,1)$ & -- & 4735\\
$(68,25)$ & 9 & $(13,5)$ & 5 & 1 & YES & YES & NO(2) & $1.44$ & $(4,2)$ & NO & 4736\\
$(68,19)$ & 9 & $(17,7)$ & 6 & 17 & YES & YES & YES & $1.75$ & $(2,3)$ & -- & 4737\\
$(68,25)$ & 9 & $(17,5)$ & 6 & 17 & YES & YES & YES & $1.86$ & $(2,3)$ & -- & 4738\\
$(68,25)$ & 9 & $(17,7)$ & 6 & 17 & YES & YES & YES & $1.62$ & $(4,2)$ & -- & 4739\\
$(68,19)$ & 9 & $(18,7)$ & 6 & 2 & YES & YES & YES & $1.78$ & $(2,3)$ & -- & 4740\\
$(68,19)$ & 9 & $(19,5)$ & 7 & 1 & YES & YES & NO(2) & $1.67$ & $(4,2)$ & NO & 4741\\
$(68,25)$ & 9 & $(21,8)$ & 6 & 1 & YES & YES & NO(2) & $1.44$ & $(4,2)$ & NO & 4742\\
$(68,25)$ & 9 & $(22,5)$ & 7 & 2 & YES & YES & YES & $1.50$ & $(4,2)$ & -- & 4743\\
$(68,19)$ & 9 & $(24,7)$ & 7 & 4 & YES & YES & YES & $1.57$ & $(2,3)$ & -- & 4744\\
$(68,21)$ & 11 & $(24,7)$ & 7 & 4 & YES & YES & YES & $1.62$ & $(2,3)$ & NO & 4745\\
$(68,19)$ & 9 & $(26,7)$ & 7 & 2 & YES & YES & NO(2) & $1.56$ & $(4,2)$ & NO & 4746\\
$(68,19)$ & 9 & $(40,11)$ & 8 & 4 & YES & YES & NO(2) & $1.56$ & $(4,2)$ & NO & 4747\\
$(68,19)$ & 9 & $(51,14)$ & 9 & 17 & YES & YES & NO(2) & $1.38$ & $(6,1)$ & NO & 4748\\
$(69,20)$ & 10 & $(7,2)$ & 4 & 1 & YES & YES & NO(2) & $1.70$ & $(6,1)$ & -- & 4749\\
$(69,29)$ & 9 & $(8,3)$ & 4 & 1 & YES & YES & NO(2) & $1.73$ & $(4,2)$ & -- & 4750\\
$(69,29)$ & 9 & $(9,4)$ & 5 & 3 & YES & YES & NO(2) & $1.90$ & $(2,3)$ & -- & 4751\\
$(69,29)$ & 9 & $(10,3)$ & 5 & 1 & YES & YES & NO(2) & $1.67$ & $(2,3)$ & -- & 4752\\
$(69,29)$ & 9 & $(10,3)$ & 5 & 1 & YES & YES & NO(2) & $1.50$ & $(4,2)$ & NO & 4753\\
$(69,20)$ & 10 & $(11,3)$ & 5 & 1 & YES & YES & NO(2) & $1.44$ & $(4,2)$ & -- & 4754\\
$(69,29)$ & 9 & $(11,3)$ & 5 & 1 & YES & YES & NO(2) & $1.44$ & $(2,3)$ & -- & 4755\\
$(69,29)$ & 9 & $(11,3)$ & 5 & 1 & YES & YES & NO(2) & $1.50$ & $(4,2)$ & NO & 4756\\
$(69,29)$ & 9 & $(11,3)$ & 5 & 1 & YES & YES & NO(2) & $1.73$ & $(4,2)$ & NO & 4757\\
$(69,25)$ & 11 & $(12,5)$ & 5 & 3 & YES & YES & YES & $1.71$ & $(4,2)$ & -- & 4758\\
$(69,29)$ & 9 & $(12,5)$ & 5 & 3 & YES & YES & YES & $1.75$ & $(2,3)$ & -- & 4759\\
$(69,19)$ & 9 & $(13,5)$ & 5 & 1 & YES & YES & NO(2) & $1.67$ & $(2,3)$ & -- & 4760\\
$(69,20)$ & 10 & $(13,3)$ & 6 & 1 & YES & YES & NO(2) & $1.44$ & $(4,2)$ & -- & 4761\\
$(69,29)$ & 9 & $(13,3)$ & 6 & 1 & YES & YES & NO(2) & $1.56$ & $(2,3)$ & -- & 4762\\
$(69,29)$ & 9 & $(13,5)$ & 5 & 1 & YES & YES & YES & $1.88$ & $(2,3)$ & -- & 4763\\
$(69,31)$ & 10 & $(13,4)$ & 6 & 1 & YES & YES & NO(2) & $1.56$ & $(8,0)$ & NO & 4764\\
$(69,19)$ & 9 & $(14,5)$ & 6 & 1 & YES & YES & YES & $1.57$ & $(4,2)$ & -- & 4765\\
$(69,31)$ & 10 & $(14,5)$ & 6 & 1 & YES & YES & NO(2) & $1.56$ & $(8,0)$ & NO & 4766\\
$(69,19)$ & 9 & $(17,5)$ & 6 & 1 & YES & YES & NO(2) & $1.25$ & $(4,2)$ & -- & 4767\\
$(69,19)$ & 9 & $(17,7)$ & 6 & 1 & YES & YES & YES & $1.57$ & $(2,3)$ & -- & 4768\\
$(69,29)$ & 9 & $(17,5)$ & 6 & 1 & YES & YES & YES & $1.75$ & $(2,3)$ & -- & 4769\\
$(69,29)$ & 9 & $(17,5)$ & 6 & 1 & YES & YES & YES & $1.88$ & $(2,3)$ & NO & 4770\\
$(69,19)$ & 9 & $(18,7)$ & 6 & 3 & YES & YES & YES & $1.75$ & $(2,3)$ & -- & 4771\\
$(69,19)$ & 9 & $(18,7)$ & 6 & 3 & YES & YES & YES & $1.88$ & $(2,3)$ & NO & 4772\\
$(69,29)$ & 9 & $(18,5)$ & 6 & 3 & YES & YES & YES & $1.67$ & $(2,3)$ & -- & 4773\\
$(69,19)$ & 9 & $(19,8)$ & 6 & 1 & YES & YES & YES & $1.67$ & $(2,3)$ & -- & 4774\\
$(69,19)$ & 9 & $(21,8)$ & 6 & 3 & YES & YES & YES & $1.67$ & $(2,3)$ & NO & 4775\\
$(69,11)$ & 11 & $(23,6)$ & 8 & 23 & YES & YES & NO(2) & $1.44$ & $(8,0)$ & NO & 4776\\
$(69,20)$ & 10 & $(23,7)$ & 7 & 23 & YES & YES & YES & $1.57$ & $(6,1)$ & -- & 4777\\
$(69,29)$ & 9 & $(23,5)$ & 7 & 23 & YES & YES & YES & $1.70$ & $(2,3)$ & -- & 4778\\
$(69,29)$ & 9 & $(23,5)$ & 7 & 23 & YES & YES & YES & $1.57$ & $(2,3)$ & NO & 4779\\
$(69,19)$ & 9 & $(24,7)$ & 7 & 3 & YES & YES & YES & $1.70$ & $(2,3)$ & -- & 4780\\
$(69,20)$ & 10 & $(27,8)$ & 7 & 3 & YES & YES & NO(2) & $1.70$ & $(6,1)$ & NO & 4781\\
$(69,19)$ & 9 & $(36,11)$ & 8 & 3 & YES & YES & YES & $1.67$ & $(2,3)$ & NO & 4782\\
$(69,19)$ & 9 & $(44,13)$ & 8 & 1 & YES & YES & YES & $1.67$ & $(2,3)$ & NO & 4783\\
$(69,20)$ & 10 & $(44,13)$ & 8 & 1 & YES & YES & NO(2) & $1.44$ & $(4,2)$ & NO & 4784\\
$(69,29)$ & 9 & $(55,23)$ & 9 & 1 & YES & YES & NO(2) & $1.67$ & $(4,2)$ & NO & 4785\\
$(69,20)$ & 10 & $(58,17)$ & 9 & 1 & YES & YES & NO(2) & $1.44$ & $(4,2)$ & 5404 & 4786\\
$(69,29)$ & 9 & $(67,28)$ & 10 & 1 & YES & YES & NO(2) & $1.44$ & $(4,2)$ & NO & 4787\\
$(69,19)$ & 9 & $(68,19)$ & 9 & 1 & YES & YES & NO(2) & $1.25$ & $(4,2)$ & NO & 4788\\
$(70,29)$ & 9 & $(8,3)$ & 4 & 2 & YES & YES & NO(2) & $1.38$ & $(6,1)$ & -- & 4789\\
$(70,29)$ & 9 & $(9,4)$ & 5 & 1 & YES & YES & NO(2) & $1.29$ & $(8,0)$ & -- & 4790\\
$(70,29)$ & 9 & $(10,3)$ & 5 & 10 & YES & YES & NO(2) & $1.44$ & $(2,3)$ & -- & 4791\\
$(70,29)$ & 9 & $(10,3)$ & 5 & 10 & YES & YES & NO(2) & $1.29$ & $(6,1)$ & NO & 4792\\
$(70,29)$ & 9 & $(12,5)$ & 5 & 2 & YES & YES & YES & $1.57$ & $(4,2)$ & -- & 4793\\
$(70,29)$ & 9 & $(13,5)$ & 5 & 1 & YES & YES & YES & $1.75$ & $(4,2)$ & -- & 4794\\
$(70,29)$ & 9 & $(16,7)$ & 6 & 2 & YES & YES & YES & $1.88$ & $(4,2)$ & -- & 4795\\
$(70,29)$ & 9 & $(17,5)$ & 6 & 1 & YES & YES & YES & $1.89$ & $(2,3)$ & -- & 4796\\
$(70,29)$ & 9 & $(18,5)$ & 6 & 2 & YES & YES & YES & $1.62$ & $(4,2)$ & -- & 4797\\
$(70,27)$ & 10 & $(20,3)$ & 8 & 10 & YES & YES & NO(2) & $1.56$ & $(8,0)$ & NO & 4798\\
$(70,27)$ & 10 & $(20,3)$ & 8 & 10 & YES & YES & NO(2) & $1.67$ & $(8,0)$ & -- & 4799\\
$(70,13)$ & 10 & $(23,9)$ & 7 & 1 & YES & YES & YES & $1.50$ & $(4,2)$ & NO & 4800\\
$(70,13)$ & 10 & $(23,9)$ & 7 & 1 & YES & YES & YES & $1.62$ & $(4,2)$ & NO & 4801\\
$(70,13)$ & 10 & $(23,9)$ & 7 & 1 & YES & YES & YES & $1.62$ & $(4,2)$ & -- & 4802\\
$(70,29)$ & 9 & $(39,17)$ & 8 & 1 & YES & YES & YES & $1.88$ & $(4,2)$ & NO & 4803\\
$(70,29)$ & 9 & $(49,20)$ & 9 & 7 & YES & YES & YES & $1.57$ & $(2,3)$ & NO & 4804\\
$(70,29)$ & 9 & $(63,26)$ & 9 & 7 & YES & YES & NO(2) & $1.38$ & $(6,1)$ & NO & 4805\\
$(71,21)$ & 9 & $(5,2)$ & 3 & 1 & YES & YES & NO(2) & $1.50$ & $(2,3)$ & -- & 4806\\
$(71,30)$ & 9 & $(5,2)$ & 3 & 1 & YES & YES & YES & $1.50$ & $(2,3)$ & -- & 4807\\
$(71,21)$ & 9 & $(7,3)$ & 4 & 1 & YES & YES & YES & $1.50$ & $(2,3)$ & NO & 4808\\
$(71,21)$ & 9 & $(7,3)$ & 4 & 1 & YES & YES & YES & $1.50$ & $(2,3)$ & -- & 4809\\
$(71,26)$ & 9 & $(7,2)$ & 4 & 1 & YES & YES & YES & $1.62$ & $(2,3)$ & -- & 4810\\
$(71,27)$ & 9 & $(7,2)$ & 4 & 1 & YES & YES & NO(2) & $1.60$ & $(2,3)$ & -- & 4811\\
$(71,30)$ & 9 & $(8,3)$ & 4 & 1 & YES & YES & NO(2) & $1.60$ & $(2,3)$ & -- & 4812\\
$(71,21)$ & 9 & $(9,4)$ & 5 & 1 & YES & YES & NO(2) & $1.70$ & $(2,3)$ & -- & 4813\\
$(71,29)$ & 10 & $(9,4)$ & 5 & 1 & YES & YES & NO(2) & $1.78$ & $(8,0)$ & -- & 4814\\
$(71,26)$ & 9 & $(10,3)$ & 5 & 1 & YES & YES & NO(2) & $1.70$ & $(2,3)$ & -- & 4815\\
$(71,27)$ & 9 & $(10,3)$ & 5 & 1 & YES & YES & NO(2) & $1.56$ & $(4,2)$ & -- & 4816\\
$(71,30)$ & 9 & $(10,3)$ & 5 & 1 & YES & YES & YES & $1.57$ & $(2,3)$ & -- & 4817\\
$(71,30)$ & 9 & $(10,3)$ & 5 & 1 & YES & YES & YES & $1.57$ & $(2,3)$ & NO & 4818\\
$(71,21)$ & 9 & $(11,3)$ & 5 & 1 & YES & YES & YES & $1.50$ & $(2,3)$ & NO & 4819\\
$(71,21)$ & 9 & $(11,5)$ & 6 & 1 & YES & YES & NO(2) & $1.67$ & $(4,2)$ & -- & 4820\\
$(71,26)$ & 9 & $(11,3)$ & 5 & 1 & YES & YES & NO(2) & $1.60$ & $(6,1)$ & -- & 4821\\
$(71,27)$ & 9 & $(11,3)$ & 5 & 1 & YES & YES & NO(2) & $1.67$ & $(2,3)$ & -- & 4822\\
$(71,27)$ & 9 & $(11,3)$ & 5 & 1 & YES & YES & YES & $1.43$ & $(4,2)$ & NO & 4823\\
$(71,27)$ & 9 & $(11,4)$ & 5 & 1 & YES & YES & NO(2) & $1.56$ & $(4,2)$ & -- & 4824\\
$(71,21)$ & 9 & $(12,5)$ & 5 & 1 & YES & YES & YES & $1.43$ & $(6,1)$ & -- & 4825\\
$(71,19)$ & 10 & $(13,4)$ & 6 & 1 & YES & YES & NO(2) & $1.50$ & $(4,2)$ & -- & 4826\\
$(71,21)$ & 9 & $(13,4)$ & 6 & 1 & YES & YES & YES & $1.38$ & $(4,2)$ & -- & 4827\\
$(71,21)$ & 9 & $(13,5)$ & 5 & 1 & YES & YES & YES & $1.62$ & $(2,3)$ & -- & 4828\\
$(71,21)$ & 9 & $(13,5)$ & 5 & 1 & YES & YES & YES & $1.67$ & $(2,3)$ & NO & 4829\\
$(71,26)$ & 9 & $(13,4)$ & 6 & 1 & YES & YES & NO(2) & $1.62$ & $(4,2)$ & -- & 4830\\
$(71,27)$ & 9 & $(13,5)$ & 5 & 1 & YES & YES & YES & $1.89$ & $(2,3)$ & -- & 4831\\
$(71,27)$ & 9 & $(13,5)$ & 5 & 1 & YES & YES & YES & $1.75$ & $(2,3)$ & NO & 4832\\
$(71,29)$ & 10 & $(13,4)$ & 6 & 1 & YES & YES & NO(2) & $1.78$ & $(8,0)$ & NO & 4833\\
$(71,27)$ & 9 & $(14,3)$ & 6 & 1 & YES & YES & NO(2) & $1.56$ & $(8,0)$ & NO & 4834\\
$(71,27)$ & 9 & $(14,3)$ & 6 & 1 & YES & YES & NO(2) & $1.56$ & $(8,0)$ & -- & 4835\\
$(71,16)$ & 10 & $(15,4)$ & 6 & 1 & YES & YES & NO(2) & $1.60$ & $(6,1)$ & -- & 4836\\
$(71,21)$ & 9 & $(15,4)$ & 6 & 1 & YES & YES & NO(2) & $1.56$ & $(4,2)$ & NO & 4837\\
$(71,21)$ & 9 & $(15,4)$ & 6 & 1 & YES & YES & YES & $1.38$ & $(4,2)$ & -- & 4838\\
$(71,27)$ & 9 & $(15,4)$ & 6 & 1 & YES & YES & YES & $1.78$ & $(2,3)$ & -- & 4839\\
$(71,30)$ & 9 & $(16,5)$ & 7 & 1 & YES & YES & NO(2) & $1.67$ & $(6,1)$ & -- & 4840\\
$(71,19)$ & 10 & $(17,5)$ & 6 & 1 & YES & YES & NO(2) & $1.56$ & $(2,3)$ & -- & 4841\\
$(71,21)$ & 9 & $(17,4)$ & 7 & 1 & YES & YES & YES & $1.38$ & $(2,3)$ & -- & 4842\\
$(71,21)$ & 9 & $(17,5)$ & 6 & 1 & YES & YES & YES & $1.38$ & $(2,3)$ & -- & 4843\\
$(71,21)$ & 9 & $(17,7)$ & 6 & 1 & YES & YES & YES & $1.78$ & $(2,3)$ & -- & 4844\\
$(71,26)$ & 9 & $(17,5)$ & 6 & 1 & YES & YES & YES & $1.71$ & $(4,2)$ & -- & 4845\\
$(71,27)$ & 9 & $(17,5)$ & 6 & 1 & YES & YES & YES & $1.50$ & $(4,2)$ & -- & 4846\\
$(71,27)$ & 9 & $(17,7)$ & 6 & 1 & YES & YES & NO(2) & $1.80$ & $(2,3)$ & NO & 4847\\
$(71,30)$ & 9 & $(17,5)$ & 6 & 1 & YES & YES & YES & $1.71$ & $(4,2)$ & -- & 4848\\
$(71,21)$ & 9 & $(18,5)$ & 6 & 1 & YES & YES & YES & $1.38$ & $(2,3)$ & -- & 4849\\
$(71,27)$ & 9 & $(18,5)$ & 6 & 1 & YES & YES & YES & $1.67$ & $(2,3)$ & -- & 4850\\
$(71,29)$ & 10 & $(18,5)$ & 6 & 1 & YES & YES & YES & $1.57$ & $(6,1)$ & -- & 4851\\
$(71,21)$ & 9 & $(19,7)$ & 6 & 1 & YES & YES & NO(2) & $1.62$ & $(6,1)$ & NO & 4852\\
$(71,26)$ & 9 & $(21,8)$ & 6 & 1 & YES & YES & YES & $1.78$ & $(4,2)$ & -- & 4853\\
$(71,30)$ & 9 & $(22,5)$ & 7 & 1 & YES & YES & YES & $1.62$ & $(4,2)$ & NO & 4854\\
$(71,26)$ & 9 & $(23,9)$ & 7 & 1 & YES & YES & NO(2) & $1.50$ & $(4,2)$ & NO & 4855\\
$(71,29)$ & 10 & $(23,10)$ & 7 & 1 & YES & YES & NO(2) & $1.67$ & $(8,0)$ & NO & 4856\\
$(71,21)$ & 9 & $(24,7)$ & 7 & 1 & YES & YES & NO(2) & $1.60$ & $(2,3)$ & NO & 4857\\
$(71,16)$ & 10 & $(25,7)$ & 7 & 1 & YES & YES & NO(2) & $1.44$ & $(4,2)$ & NO & 4858\\
$(71,30)$ & 9 & $(25,9)$ & 7 & 1 & YES & YES & YES & $1.75$ & $(4,2)$ & NO & 4859\\
$(71,16)$ & 10 & $(26,11)$ & 7 & 1 & YES & YES & YES & $1.67$ & $(4,2)$ & -- & 4860\\
$(71,19)$ & 10 & $(27,8)$ & 7 & 1 & YES & YES & NO(2) & $1.56$ & $(2,3)$ & NO & 4861\\
$(71,11)$ & 12 & $(30,7)$ & 8 & 1 & YES & YES & NO(2) & $1.38$ & $(10,-1)$ & NO & 4862\\
$(71,21)$ & 9 & $(35,11)$ & 9 & 1 & YES & YES & NO(2) & $1.62$ & $(6,1)$ & NO & 4863\\
$(71,21)$ & 9 & $(38,11)$ & 9 & 1 & YES & YES & NO(2) & $1.70$ & $(2,3)$ & NO & 4864\\
$(71,22)$ & 10 & $(41,12)$ & 8 & 1 & YES & YES & YES & $1.43$ & $(4,2)$ & NO & 4865\\
$(71,29)$ & 10 & $(41,17)$ & 8 & 1 & YES & YES & NO(2) & $1.29$ & $(8,0)$ & NO & 4866\\
$(71,30)$ & 9 & $(43,18)$ & 8 & 1 & YES & YES & NO(2) & $1.60$ & $(2,3)$ & NO & 4867\\
$(71,27)$ & 9 & $(45,17)$ & 9 & 1 & YES & YES & YES & $1.29$ & $(4,2)$ & NO & 4868\\
$(71,21)$ & 9 & $(47,14)$ & 9 & 1 & YES & YES & NO(2) & $1.44$ & $(8,0)$ & NO & 4869\\
$(71,29)$ & 10 & $(50,21)$ & 8 & 1 & YES & YES & YES & $1.57$ & $(6,1)$ & NO & 4870\\
$(71,30)$ & 9 & $(55,23)$ & 9 & 1 & YES & YES & YES & $1.43$ & $(2,3)$ & NO & 4871\\
$(71,21)$ & 9 & $(57,17)$ & 10 & 1 & YES & YES & NO(2) & $1.60$ & $(2,3)$ & 6691 & 4872\\
$(71,27)$ & 9 & $(66,25)$ & 9 & 1 & YES & YES & NO(2) & $1.70$ & $(2,3)$ & NO & 4873\\
$(71,29)$ & 10 & $(70,29)$ & 9 & 1 & YES & YES & YES & $1.43$ & $(6,1)$ & NO & 4874\\
$(73,27)$ & 9 & $(5,2)$ & 3 & 1 & YES & YES & NO(2) & $1.38$ & $(6,1)$ & -- & 4875\\
$(73,27)$ & 9 & $(7,2)$ & 4 & 1 & YES & YES & YES & $1.50$ & $(2,3)$ & -- & 4876\\
$(73,27)$ & 9 & $(7,2)$ & 4 & 1 & YES & YES & YES & $1.62$ & $(2,3)$ & NO & 4877\\
$(73,27)$ & 9 & $(7,3)$ & 4 & 1 & YES & YES & YES & $1.38$ & $(2,3)$ & -- & 4878\\
$(73,27)$ & 9 & $(8,3)$ & 4 & 1 & YES & YES & NO(2) & $1.60$ & $(6,1)$ & -- & 4879\\
$(73,27)$ & 9 & $(10,3)$ & 5 & 1 & YES & YES & YES & $1.62$ & $(2,3)$ & -- & 4880\\
$(73,27)$ & 9 & $(11,4)$ & 5 & 1 & YES & YES & YES & $1.43$ & $(2,3)$ & -- & 4881\\
$(73,30)$ & 10 & $(11,3)$ & 5 & 1 & YES & YES & YES & $1.43$ & $(2,3)$ & -- & 4882\\
$(73,32)$ & 10 & $(11,3)$ & 5 & 1 & YES & YES & NO(2) & $1.56$ & $(8,0)$ & -- & 4883\\
$(73,28)$ & 10 & $(13,3)$ & 6 & 1 & YES & YES & YES & $1.62$ & $(2,3)$ & -- & 4884\\
$(73,32)$ & 10 & $(13,3)$ & 6 & 1 & YES & YES & NO(2) & $1.67$ & $(4,2)$ & -- & 4885\\
$(73,30)$ & 10 & $(14,3)$ & 6 & 1 & YES & YES & NO(2) & $1.38$ & $(4,2)$ & -- & 4886\\
$(73,28)$ & 10 & $(17,3)$ & 7 & 1 & YES & YES & YES & $1.50$ & $(2,3)$ & -- & 4887\\
$(73,17)$ & 10 & $(18,7)$ & 6 & 1 & YES & YES & YES & $1.70$ & $(2,3)$ & -- & 4888\\
$(73,27)$ & 9 & $(18,5)$ & 6 & 1 & YES & YES & YES & $1.57$ & $(2,3)$ & -- & 4889\\
$(73,27)$ & 9 & $(19,7)$ & 6 & 1 & YES & YES & YES & $1.88$ & $(4,2)$ & -- & 4890\\
$(73,22)$ & 12 & $(20,3)$ & 8 & 1 & YES & YES & NO(2) & $1.62$ & $(6,1)$ & -- & 4891\\
$(73,27)$ & 9 & $(22,5)$ & 7 & 1 & YES & YES & YES & $1.50$ & $(4,2)$ & NO & 4892\\
$(73,17)$ & 10 & $(24,7)$ & 7 & 1 & YES & YES & YES & $1.71$ & $(2,3)$ & -- & 4893\\
$(73,17)$ & 10 & $(25,7)$ & 7 & 1 & YES & YES & YES & $1.71$ & $(2,3)$ & -- & 4894\\
$(73,27)$ & 9 & $(41,15)$ & 8 & 1 & YES & YES & NO(2) & $1.60$ & $(2,3)$ & NO & 4895\\
$(73,27)$ & 9 & $(71,26)$ & 9 & 1 & YES & YES & YES & $1.78$ & $(2,3)$ & NO & 4896\\
$(74,23)$ & 10 & $(7,2)$ & 4 & 1 & YES & YES & NO(2) & $1.62$ & $(6,1)$ & NO & 4897\\
$(74,23)$ & 10 & $(7,2)$ & 4 & 1 & YES & YES & NO(2) & $1.73$ & $(4,2)$ & -- & 4898\\
$(74,23)$ & 10 & $(8,3)$ & 4 & 2 & YES & YES & NO(2) & $1.64$ & $(4,2)$ & -- & 4899\\
$(74,31)$ & 9 & $(8,3)$ & 4 & 2 & YES & YES & NO(2) & $1.56$ & $(4,2)$ & -- & 4900\\
$(74,31)$ & 9 & $(10,3)$ & 5 & 2 & YES & YES & YES & $1.29$ & $(4,2)$ & -- & 4901\\
$(74,17)$ & 11 & $(12,5)$ & 5 & 2 & YES & YES & YES & $1.57$ & $(2,3)$ & NO & 4902\\
$(74,31)$ & 9 & $(12,5)$ & 5 & 2 & YES & YES & YES & $1.88$ & $(2,3)$ & -- & 4903\\
$(74,29)$ & 10 & $(13,4)$ & 6 & 1 & YES & YES & NO(2) & $1.78$ & $(8,0)$ & NO & 4904\\
$(74,31)$ & 9 & $(13,5)$ & 5 & 1 & YES & YES & YES & $1.75$ & $(4,2)$ & -- & 4905\\
$(74,29)$ & 10 & $(17,6)$ & 7 & 1 & YES & YES & NO(2) & $1.78$ & $(8,0)$ & NO & 4906\\
$(74,31)$ & 9 & $(17,5)$ & 6 & 1 & YES & YES & YES & $1.78$ & $(2,3)$ & -- & 4907\\
$(74,23)$ & 10 & $(27,8)$ & 7 & 1 & YES & YES & NO(2) & $1.50$ & $(10,-1)$ & NO & 4908\\
$(74,31)$ & 9 & $(31,12)$ & 7 & 1 & YES & YES & YES & $1.71$ & $(6,1)$ & NO & 4909\\
$(74,29)$ & 10 & $(55,21)$ & 8 & 1 & YES & YES & YES & $1.57$ & $(6,1)$ & NO & 4910\\
$(74,31)$ & 9 & $(69,29)$ & 9 & 1 & YES & YES & NO(2) & $1.56$ & $(4,2)$ & NO & 4911\\
$(75,29)$ & 9 & $(7,3)$ & 4 & 1 & YES & YES & NO(2) & $1.38$ & $(6,1)$ & -- & 4912\\
$(75,31)$ & 9 & $(7,2)$ & 4 & 1 & YES & YES & YES & $1.57$ & $(2,3)$ & -- & 4913\\
$(75,29)$ & 9 & $(8,3)$ & 4 & 1 & YES & YES & NO(2) & $1.75$ & $(2,3)$ & -- & 4914\\
$(75,29)$ & 9 & $(10,3)$ & 5 & 5 & YES & YES & NO(2) & $1.29$ & $(6,1)$ & NO & 4915\\
$(75,29)$ & 9 & $(10,3)$ & 5 & 5 & YES & YES & NO(2) & $1.29$ & $(6,1)$ & -- & 4916\\
$(75,31)$ & 9 & $(10,3)$ & 5 & 5 & YES & YES & NO(2) & $1.50$ & $(4,2)$ & NO & 4917\\
$(75,31)$ & 9 & $(10,3)$ & 5 & 5 & YES & YES & NO(2) & $1.50$ & $(4,2)$ & -- & 4918\\
$(75,22)$ & 10 & $(12,5)$ & 5 & 3 & YES & YES & NO(2) & $1.56$ & $(4,2)$ & NO & 4919\\
$(75,29)$ & 9 & $(12,5)$ & 5 & 3 & YES & YES & YES & $1.62$ & $(2,3)$ & -- & 4920\\
$(75,31)$ & 9 & $(12,5)$ & 5 & 3 & YES & YES & NO(2) & $1.67$ & $(4,2)$ & -- & 4921\\
$(75,22)$ & 10 & $(13,5)$ & 5 & 1 & YES & YES & NO(2) & $1.78$ & $(2,3)$ & NO & 4922\\
$(75,29)$ & 9 & $(13,4)$ & 6 & 1 & YES & YES & NO(2) & $1.50$ & $(4,2)$ & -- & 4923\\
$(75,29)$ & 9 & $(13,5)$ & 5 & 1 & YES & YES & YES & $1.43$ & $(4,2)$ & -- & 4924\\
$(75,31)$ & 9 & $(13,5)$ & 5 & 1 & YES & YES & YES & $1.29$ & $(4,2)$ & -- & 4925\\
$(75,31)$ & 9 & $(14,5)$ & 6 & 1 & YES & YES & NO(2) & $1.50$ & $(10,-1)$ & NO & 4926\\
$(75,31)$ & 9 & $(16,7)$ & 6 & 1 & YES & YES & NO(2) & $1.60$ & $(6,1)$ & -- & 4927\\
$(75,22)$ & 10 & $(17,3)$ & 7 & 1 & YES & YES & NO(2) & $1.60$ & $(2,3)$ & -- & 4928\\
$(75,29)$ & 9 & $(17,5)$ & 6 & 1 & YES & YES & YES & $1.62$ & $(2,3)$ & -- & 4929\\
$(75,31)$ & 9 & $(17,5)$ & 6 & 1 & YES & YES & YES & $1.62$ & $(2,3)$ & -- & 4930\\
$(75,17)$ & 10 & $(18,7)$ & 6 & 3 & YES & YES & YES & $1.80$ & $(2,3)$ & NO & 4931\\
$(75,22)$ & 10 & $(18,5)$ & 6 & 3 & YES & YES & YES & $1.67$ & $(2,3)$ & -- & 4932\\
$(75,29)$ & 9 & $(18,5)$ & 6 & 3 & YES & YES & YES & $1.60$ & $(2,3)$ & -- & 4933\\
$(75,31)$ & 9 & $(18,5)$ & 6 & 3 & YES & YES & YES & $1.62$ & $(4,2)$ & NO & 4934\\
$(75,17)$ & 10 & $(19,5)$ & 7 & 1 & YES & YES & YES & $1.62$ & $(4,2)$ & -- & 4935\\
$(75,22)$ & 10 & $(22,5)$ & 7 & 1 & YES & YES & YES & $1.62$ & $(2,3)$ & -- & 4936\\
$(75,17)$ & 10 & $(25,7)$ & 7 & 25 & YES & YES & YES & $1.62$ & $(2,3)$ & -- & 4937\\
$(75,22)$ & 10 & $(27,5)$ & 8 & 3 & YES & YES & YES & $1.78$ & $(2,3)$ & -- & 4938\\
$(75,22)$ & 10 & $(27,5)$ & 8 & 3 & YES & YES & YES & $1.78$ & $(2,3)$ & NO & 4939\\
$(75,17)$ & 10 & $(31,7)$ & 8 & 1 & YES & YES & YES & $1.67$ & $(2,3)$ & -- & 4940\\
$(75,29)$ & 9 & $(67,26)$ & 9 & 1 & YES & YES & NO(2) & $1.60$ & $(10,-1)$ & NO & 4941\\
$(75,29)$ & 9 & $(74,29)$ & 10 & 1 & YES & YES & YES & $1.57$ & $(6,1)$ & NO & 4942\\
$(76,21)$ & 9 & $(5,2)$ & 3 & 1 & YES & YES & YES & $1.38$ & $(2,3)$ & -- & 4943\\
$(76,21)$ & 9 & $(7,3)$ & 4 & 1 & YES & YES & NO(2) & $1.44$ & $(4,2)$ & -- & 4944\\
$(76,21)$ & 9 & $(7,3)$ & 4 & 1 & YES & YES & NO(2) & $1.56$ & $(4,2)$ & NO & 4945\\
$(76,21)$ & 9 & $(7,3)$ & 4 & 1 & YES & YES & NO(2) & $1.60$ & $(2,3)$ & NO & 4946\\
$(76,29)$ & 9 & $(7,3)$ & 4 & 1 & YES & YES & NO(2) & $1.80$ & $(6,1)$ & -- & 4947\\
$(76,29)$ & 9 & $(8,3)$ & 4 & 4 & YES & YES & NO(2) & $1.56$ & $(4,2)$ & -- & 4948\\
$(76,29)$ & 9 & $(9,2)$ & 5 & 1 & YES & YES & NO(2) & $1.56$ & $(4,2)$ & -- & 4949\\
$(76,21)$ & 9 & $(10,3)$ & 5 & 2 & YES & YES & YES & $1.50$ & $(2,3)$ & NO & 4950\\
$(76,29)$ & 9 & $(10,3)$ & 5 & 2 & YES & YES & NO(2) & $1.25$ & $(4,2)$ & -- & 4951\\
$(76,21)$ & 9 & $(11,4)$ & 5 & 1 & YES & YES & YES & $1.62$ & $(2,3)$ & -- & 4952\\
$(76,29)$ & 9 & $(12,5)$ & 5 & 4 & YES & YES & YES & $1.88$ & $(2,3)$ & -- & 4953\\
$(76,21)$ & 9 & $(13,4)$ & 6 & 1 & YES & YES & YES & $1.50$ & $(4,2)$ & -- & 4954\\
$(76,29)$ & 9 & $(13,5)$ & 5 & 1 & YES & YES & YES & $1.62$ & $(4,2)$ & -- & 4955\\
$(76,21)$ & 9 & $(15,4)$ & 6 & 1 & YES & YES & YES & $1.50$ & $(4,2)$ & -- & 4956\\
$(76,23)$ & 10 & $(15,4)$ & 6 & 1 & YES & YES & NO(2) & $1.44$ & $(8,0)$ & -- & 4957\\
$(76,23)$ & 10 & $(17,5)$ & 6 & 1 & YES & YES & YES & $1.50$ & $(2,3)$ & -- & 4958\\
$(76,29)$ & 9 & $(17,5)$ & 6 & 1 & YES & YES & YES & $1.75$ & $(2,3)$ & -- & 4959\\
$(76,21)$ & 9 & $(18,7)$ & 6 & 2 & YES & YES & YES & $1.71$ & $(2,3)$ & -- & 4960\\
$(76,21)$ & 9 & $(18,7)$ & 6 & 2 & YES & YES & YES & $1.71$ & $(2,3)$ & NO & 4961\\
$(76,29)$ & 9 & $(18,5)$ & 6 & 2 & YES & YES & YES & $1.75$ & $(2,3)$ & -- & 4962\\
$(76,21)$ & 9 & $(19,5)$ & 7 & 19 & YES & YES & NO(2) & $1.56$ & $(4,2)$ & NO & 4963\\
$(76,29)$ & 9 & $(22,5)$ & 7 & 2 & YES & YES & YES & $1.71$ & $(2,3)$ & -- & 4964\\
$(76,29)$ & 9 & $(28,11)$ & 8 & 4 & YES & YES & NO(2) & $1.70$ & $(2,3)$ & NO & 4965\\
$(76,21)$ & 9 & $(33,10)$ & 8 & 1 & YES & YES & YES & $1.67$ & $(2,3)$ & NO & 4966\\
$(76,21)$ & 9 & $(37,11)$ & 8 & 1 & YES & YES & YES & $1.67$ & $(2,3)$ & NO & 4967\\
$(76,23)$ & 10 & $(44,13)$ & 8 & 4 & YES & YES & NO(2) & $1.50$ & $(6,1)$ & NO & 4968\\
$(76,33)$ & 10 & $(59,26)$ & 9 & 1 & YES & YES & YES & $1.57$ & $(6,1)$ & NO & 4969\\
$(76,29)$ & 9 & $(60,23)$ & 9 & 4 & YES & YES & NO(2) & $1.70$ & $(6,1)$ & NO & 4970\\
$(76,21)$ & 9 & $(62,17)$ & 10 & 2 & YES & YES & NO(2) & $1.56$ & $(12,-2)$ & 7033 & 4971\\
$(76,21)$ & 9 & $(65,18)$ & 9 & 1 & YES & YES & NO(2) & $1.50$ & $(2,3)$ & NO & 4972\\
$(76,21)$ & 9 & $(68,19)$ & 9 & 4 & YES & YES & NO(2) & $1.50$ & $(2,3)$ & NO & 4973\\
$(77,30)$ & 10 & $(8,3)$ & 4 & 1 & YES & YES & NO(2) & $1.56$ & $(8,0)$ & -- & 4974\\
$(77,18)$ & 10 & $(11,4)$ & 5 & 11 & YES & YES & NO(2) & $1.56$ & $(8,0)$ & -- & 4975\\
$(77,30)$ & 10 & $(17,3)$ & 7 & 1 & YES & YES & NO(2) & $1.67$ & $(8,0)$ & NO & 4976\\
$(77,30)$ & 10 & $(17,3)$ & 7 & 1 & YES & YES & NO(2) & $1.67$ & $(8,0)$ & -- & 4977\\
$(78,35)$ & 10 & $(5,2)$ & 3 & 1 & YES & YES & NO(2) & $1.50$ & $(6,1)$ & -- & 4978\\
$(78,29)$ & 10 & $(9,2)$ & 5 & 3 & YES & YES & NO(2) & $1.44$ & $(8,0)$ & -- & 4979\\
$(78,23)$ & 10 & $(17,3)$ & 7 & 1 & YES & YES & NO(2) & $1.44$ & $(8,0)$ & NO & 4980\\
$(78,17)$ & 10 & $(19,5)$ & 7 & 1 & YES & YES & YES & $1.62$ & $(4,2)$ & -- & 4981\\
$(78,23)$ & 10 & $(23,5)$ & 7 & 1 & YES & YES & YES & $1.50$ & $(2,3)$ & -- & 4982\\
$(78,23)$ & 10 & $(29,8)$ & 7 & 1 & YES & YES & NO(2) & $1.56$ & $(2,3)$ & NO & 4983\\
$(78,23)$ & 10 & $(29,9)$ & 8 & 1 & YES & YES & YES & $1.43$ & $(2,3)$ & NO & 4984\\
$(78,29)$ & 10 & $(30,11)$ & 7 & 6 & YES & YES & NO(2) & $1.44$ & $(8,0)$ & NO & 4985\\
$(79,23)$ & 10 & $(5,2)$ & 3 & 1 & YES & YES & NO(2) & $1.70$ & $(2,3)$ & -- & 4986\\
$(79,30)$ & 9 & $(5,2)$ & 3 & 1 & YES & YES & NO(2) & $1.50$ & $(2,3)$ & NO & 4987\\
$(79,30)$ & 9 & $(5,2)$ & 3 & 1 & YES & YES & NO(2) & $1.50$ & $(2,3)$ & -- & 4988\\
$(79,23)$ & 10 & $(7,2)$ & 4 & 1 & YES & YES & NO(2) & $1.64$ & $(4,2)$ & -- & 4989\\
$(79,23)$ & 10 & $(7,2)$ & 4 & 1 & YES & YES & NO(2) & $1.64$ & $(4,2)$ & NO & 4990\\
$(79,24)$ & 10 & $(7,2)$ & 4 & 1 & YES & YES & NO(2) & $1.64$ & $(4,2)$ & -- & 4991\\
$(79,28)$ & 10 & $(7,2)$ & 4 & 1 & YES & YES & NO(2) & $1.67$ & $(8,0)$ & NO & 4992\\
$(79,30)$ & 9 & $(7,3)$ & 4 & 1 & YES & YES & NO(2) & $1.67$ & $(4,2)$ & -- & 4993\\
$(79,30)$ & 9 & $(7,3)$ & 4 & 1 & YES & YES & NO(2) & $1.50$ & $(2,3)$ & NO & 4994\\
$(79,23)$ & 10 & $(8,3)$ & 4 & 1 & YES & YES & NO(2) & $1.44$ & $(4,2)$ & -- & 4995\\
$(79,24)$ & 10 & $(8,3)$ & 4 & 1 & YES & YES & NO(2) & $1.50$ & $(6,1)$ & -- & 4996\\
$(79,29)$ & 9 & $(8,3)$ & 4 & 1 & YES & YES & NO(2) & $1.56$ & $(4,2)$ & NO & 4997\\
$(79,30)$ & 9 & $(8,3)$ & 4 & 1 & YES & YES & NO(2) & $1.60$ & $(6,1)$ & -- & 4998\\
$(79,29)$ & 9 & $(10,3)$ & 5 & 1 & YES & YES & NO(2) & $1.50$ & $(6,1)$ & NO & 4999\\
$(79,29)$ & 9 & $(10,3)$ & 5 & 1 & YES & YES & NO(2) & $1.50$ & $(6,1)$ & -- & 5000\\
$(79,30)$ & 9 & $(10,3)$ & 5 & 1 & YES & YES & NO(2) & $1.55$ & $(4,2)$ & -- & 5001\\
$(79,18)$ & 10 & $(11,3)$ & 5 & 1 & YES & YES & NO(2) & $1.56$ & $(8,0)$ & NO & 5002\\
$(79,18)$ & 10 & $(11,3)$ & 5 & 1 & YES & YES & NO(2) & $1.56$ & $(8,0)$ & -- & 5003\\
$(79,18)$ & 10 & $(11,4)$ & 5 & 1 & YES & YES & YES & $1.29$ & $(4,2)$ & -- & 5004\\
$(79,24)$ & 10 & $(11,5)$ & 6 & 1 & YES & YES & NO(2) & $1.50$ & $(6,1)$ & -- & 5005\\
$(79,29)$ & 9 & $(11,4)$ & 5 & 1 & YES & YES & YES & $1.71$ & $(4,2)$ & -- & 5006\\
$(79,30)$ & 9 & $(11,3)$ & 5 & 1 & YES & YES & YES & $1.80$ & $(2,3)$ & -- & 5007\\
$(79,17)$ & 11 & $(12,5)$ & 5 & 1 & YES & YES & YES & $1.71$ & $(2,3)$ & NO & 5008\\
$(79,17)$ & 11 & $(12,5)$ & 5 & 1 & YES & YES & YES & $1.71$ & $(2,3)$ & NO & 5009\\
$(79,18)$ & 10 & $(12,5)$ & 5 & 1 & YES & YES & YES & $1.50$ & $(2,3)$ & NO & 5010\\
$(79,22)$ & 10 & $(12,5)$ & 5 & 1 & YES & YES & YES & $1.75$ & $(4,2)$ & -- & 5011\\
$(79,23)$ & 10 & $(12,5)$ & 5 & 1 & YES & YES & YES & $1.57$ & $(2,3)$ & -- & 5012\\
$(79,29)$ & 9 & $(12,5)$ & 5 & 1 & YES & YES & YES & $1.75$ & $(2,3)$ & -- & 5013\\
$(79,28)$ & 10 & $(13,5)$ & 5 & 1 & YES & YES & NO(2) & $1.70$ & $(10,-1)$ & NO & 5014\\
$(79,29)$ & 9 & $(13,4)$ & 6 & 1 & YES & YES & YES & $1.75$ & $(2,3)$ & -- & 5015\\
$(79,30)$ & 9 & $(13,4)$ & 6 & 1 & YES & YES & YES & $1.78$ & $(2,3)$ & -- & 5016\\
$(79,23)$ & 10 & $(14,3)$ & 6 & 1 & YES & YES & NO(2) & $1.70$ & $(2,3)$ & -- & 5017\\
$(79,21)$ & 11 & $(15,4)$ & 6 & 1 & YES & YES & NO(2) & $1.70$ & $(6,1)$ & -- & 5018\\
$(79,23)$ & 10 & $(15,4)$ & 6 & 1 & YES & YES & YES & $1.57$ & $(4,2)$ & -- & 5019\\
$(79,18)$ & 10 & $(17,7)$ & 6 & 1 & YES & YES & YES & $1.50$ & $(4,2)$ & -- & 5020\\
$(79,23)$ & 10 & $(17,5)$ & 6 & 1 & YES & YES & YES & $1.56$ & $(2,3)$ & -- & 5021\\
$(79,29)$ & 9 & $(17,4)$ & 7 & 1 & YES & YES & YES & $1.57$ & $(4,2)$ & -- & 5022\\
$(79,30)$ & 9 & $(17,5)$ & 6 & 1 & YES & YES & YES & $1.70$ & $(2,3)$ & -- & 5023\\
$(79,23)$ & 10 & $(18,5)$ & 6 & 1 & YES & YES & YES & $1.67$ & $(2,3)$ & -- & 5024\\
$(79,24)$ & 10 & $(18,5)$ & 6 & 1 & YES & YES & YES & $1.43$ & $(4,2)$ & -- & 5025\\
$(79,29)$ & 9 & $(18,5)$ & 6 & 1 & YES & YES & YES & $1.57$ & $(2,3)$ & -- & 5026\\
$(79,22)$ & 10 & $(19,7)$ & 6 & 1 & YES & YES & YES & $1.75$ & $(4,2)$ & -- & 5027\\
$(79,23)$ & 10 & $(19,5)$ & 7 & 1 & YES & YES & YES & $1.62$ & $(2,3)$ & NO & 5028\\
$(79,14)$ & 11 & $(22,5)$ & 7 & 1 & YES & YES & NO(2) & $1.70$ & $(6,1)$ & NO & 5029\\
$(79,30)$ & 9 & $(22,5)$ & 7 & 1 & YES & YES & YES & $1.75$ & $(2,3)$ & -- & 5030\\
$(79,24)$ & 10 & $(29,8)$ & 7 & 1 & YES & YES & YES & $1.70$ & $(2,3)$ & NO & 5031\\
$(79,30)$ & 9 & $(31,12)$ & 7 & 1 & YES & YES & NO(2) & $1.64$ & $(4,2)$ & NO & 5032\\
$(79,30)$ & 9 & $(34,13)$ & 7 & 1 & YES & YES & NO(2) & $1.70$ & $(2,3)$ & NO & 5033\\
$(79,30)$ & 9 & $(41,16)$ & 8 & 1 & YES & YES & YES & $1.89$ & $(2,3)$ & 6203 & 5034\\
$(79,29)$ & 9 & $(52,19)$ & 9 & 1 & YES & YES & NO(2) & $1.70$ & $(6,1)$ & NO & 5035\\
$(79,18)$ & 10 & $(55,13)$ & 10 & 1 & YES & YES & YES & $1.71$ & $(2,3)$ & NO & 5036\\
$(79,18)$ & 10 & $(60,13)$ & 9 & 1 & YES & YES & YES & $1.62$ & $(4,2)$ & NO & 5037\\
$(79,22)$ & 10 & $(63,17)$ & 9 & 1 & YES & YES & YES & $1.75$ & $(4,2)$ & NO & 5038\\
$(80,33)$ & 10 & $(8,3)$ & 4 & 8 & YES & YES & NO(2) & $1.50$ & $(6,1)$ & -- & 5039\\
$(80,31)$ & 9 & $(10,3)$ & 5 & 10 & YES & YES & NO(2) & $1.56$ & $(2,3)$ & -- & 5040\\
$(80,31)$ & 9 & $(11,3)$ & 5 & 1 & YES & YES & YES & $1.62$ & $(2,3)$ & -- & 5041\\
$(80,31)$ & 9 & $(11,3)$ & 5 & 1 & YES & YES & YES & $1.62$ & $(2,3)$ & NO & 5042\\
$(80,33)$ & 10 & $(11,3)$ & 5 & 1 & YES & YES & NO(2) & $1.50$ & $(6,1)$ & NO & 5043\\
$(80,31)$ & 9 & $(12,5)$ & 5 & 4 & YES & YES & YES & $1.89$ & $(2,3)$ & -- & 5044\\
$(80,31)$ & 9 & $(13,6)$ & 7 & 1 & YES & YES & NO(2) & $1.50$ & $(6,1)$ & NO & 5045\\
$(80,31)$ & 9 & $(17,5)$ & 6 & 1 & YES & YES & YES & $1.71$ & $(2,3)$ & -- & 5046\\
$(80,31)$ & 9 & $(18,5)$ & 6 & 2 & YES & YES & YES & $1.89$ & $(2,3)$ & NO & 5047\\
$(80,31)$ & 9 & $(18,5)$ & 6 & 2 & YES & YES & YES & $1.71$ & $(2,3)$ & -- & 5048\\
$(80,17)$ & 10 & $(23,9)$ & 7 & 1 & YES & YES & YES & $1.75$ & $(4,2)$ & NO & 5049\\
$(80,31)$ & 9 & $(29,11)$ & 7 & 1 & YES & YES & NO(2) & $1.56$ & $(4,2)$ & NO & 5050\\
$(80,31)$ & 9 & $(55,21)$ & 8 & 5 & YES & YES & YES & $1.89$ & $(2,3)$ & NO & 5051\\
$(80,33)$ & 10 & $(70,29)$ & 9 & 10 & YES & YES & NO(2) & $1.38$ & $(6,1)$ & 8129 & 5052\\
$(81,34)$ & 9 & $(5,2)$ & 3 & 1 & YES & YES & NO(3) & $1.00$ & $(4,2)$ & -- & 5053\\
$(81,29)$ & 11 & $(7,3)$ & 4 & 1 & YES & YES & NO(2) & $1.80$ & $(6,1)$ & -- & 5054\\
$(81,31)$ & 9 & $(7,3)$ & 4 & 1 & YES & YES & NO(2) & $1.55$ & $(4,2)$ & NO & 5055\\
$(81,31)$ & 9 & $(7,3)$ & 4 & 1 & YES & YES & YES & $1.43$ & $(4,2)$ & -- & 5056\\
$(81,34)$ & 9 & $(7,2)$ & 4 & 1 & YES & YES & NO(2) & $1.60$ & $(2,3)$ & NO & 5057\\
$(81,34)$ & 9 & $(7,2)$ & 4 & 1 & YES & YES & NO(2) & $1.25$ & $(6,1)$ & -- & 5058\\
$(81,34)$ & 9 & $(7,3)$ & 4 & 1 & YES & YES & NO(2) & $1.67$ & $(4,2)$ & -- & 5059\\
$(81,34)$ & 9 & $(7,3)$ & 4 & 1 & YES & YES & NO(2) & $1.60$ & $(6,1)$ & NO & 5060\\
$(81,34)$ & 9 & $(8,3)$ & 4 & 1 & YES & YES & NO(2) & $1.64$ & $(4,2)$ & -- & 5061\\
$(81,31)$ & 9 & $(9,4)$ & 5 & 9 & YES & YES & YES & $1.62$ & $(2,3)$ & -- & 5062\\
$(81,34)$ & 9 & $(9,4)$ & 5 & 9 & YES & YES & NO(2) & $1.56$ & $(4,2)$ & -- & 5063\\
$(81,31)$ & 9 & $(10,3)$ & 5 & 1 & YES & YES & NO(2) & $1.56$ & $(4,2)$ & -- & 5064\\
$(81,31)$ & 9 & $(10,3)$ & 5 & 1 & YES & YES & NO(2) & $1.56$ & $(4,2)$ & NO & 5065\\
$(81,34)$ & 9 & $(10,3)$ & 5 & 1 & YES & YES & YES & $1.50$ & $(4,2)$ & -- & 5066\\
$(81,34)$ & 9 & $(11,3)$ & 5 & 1 & YES & YES & YES & $1.50$ & $(2,3)$ & -- & 5067\\
$(81,34)$ & 9 & $(11,4)$ & 5 & 1 & YES & YES & NO(2) & $1.67$ & $(4,2)$ & NO & 5068\\
$(81,31)$ & 9 & $(12,5)$ & 5 & 3 & YES & YES & YES & $1.78$ & $(2,3)$ & -- & 5069\\
$(81,34)$ & 9 & $(12,5)$ & 5 & 3 & YES & YES & YES & $1.75$ & $(2,3)$ & -- & 5070\\
$(81,31)$ & 9 & $(13,4)$ & 6 & 1 & YES & YES & YES & $1.43$ & $(4,2)$ & -- & 5071\\
$(81,31)$ & 9 & $(13,5)$ & 5 & 1 & YES & YES & YES & $1.80$ & $(2,3)$ & -- & 5072\\
$(81,34)$ & 9 & $(13,5)$ & 5 & 1 & YES & YES & NO(2) & $1.56$ & $(4,2)$ & NO & 5073\\
$(81,34)$ & 9 & $(13,5)$ & 5 & 1 & YES & YES & YES & $1.62$ & $(2,3)$ & -- & 5074\\
$(81,19)$ & 11 & $(17,4)$ & 7 & 1 & YES & YES & YES & $1.57$ & $(2,3)$ & -- & 5075\\
$(81,22)$ & 12 & $(17,3)$ & 7 & 1 & YES & YES & NO(2) & $1.50$ & $(6,1)$ & -- & 5076\\
$(81,31)$ & 9 & $(18,7)$ & 6 & 9 & YES & YES & YES & $1.89$ & $(4,2)$ & -- & 5077\\
$(81,19)$ & 11 & $(19,4)$ & 7 & 1 & YES & YES & YES & $1.43$ & $(2,3)$ & -- & 5078\\
$(81,22)$ & 12 & $(20,3)$ & 8 & 1 & YES & YES & NO(2) & $1.50$ & $(6,1)$ & -- & 5079\\
$(81,34)$ & 9 & $(20,9)$ & 7 & 1 & YES & YES & NO(2) & $1.50$ & $(6,1)$ & NO & 5080\\
$(81,34)$ & 9 & $(21,8)$ & 6 & 3 & YES & YES & YES & $2.00$ & $(2,3)$ & NO & 5081\\
$(81,34)$ & 9 & $(22,9)$ & 7 & 1 & YES & YES & NO(2) & $1.73$ & $(4,2)$ & NO & 5082\\
$(81,17)$ & 11 & $(25,7)$ & 7 & 1 & YES & YES & NO(2) & $1.50$ & $(6,1)$ & NO & 5083\\
$(81,31)$ & 9 & $(25,9)$ & 7 & 1 & YES & YES & YES & $1.62$ & $(2,3)$ & NO & 5084\\
$(81,29)$ & 11 & $(30,11)$ & 7 & 3 & YES & YES & NO(2) & $1.80$ & $(6,1)$ & NO & 5085\\
$(81,19)$ & 11 & $(32,7)$ & 8 & 1 & YES & YES & NO(2) & $1.56$ & $(4,2)$ & NO & 5086\\
$(81,19)$ & 11 & $(40,9)$ & 9 & 1 & YES & YES & NO(2) & $1.56$ & $(4,2)$ & NO & 5087\\
$(81,31)$ & 9 & $(45,17)$ & 9 & 9 & YES & YES & NO(2) & $1.70$ & $(2,3)$ & NO & 5088\\
$(81,31)$ & 9 & $(57,22)$ & 9 & 3 & YES & YES & NO(2) & $1.62$ & $(6,1)$ & NO & 5089\\
$(81,31)$ & 9 & $(76,29)$ & 9 & 1 & YES & YES & NO(2) & $1.56$ & $(4,2)$ & NO & 5090\\
$(82,23)$ & 10 & $(5,2)$ & 3 & 1 & YES & YES & YES & $1.62$ & $(2,3)$ & -- & 5091\\
$(82,23)$ & 10 & $(5,2)$ & 3 & 1 & YES & YES & YES & $1.43$ & $(4,2)$ & NO & 5092\\
$(82,31)$ & 10 & $(7,2)$ & 4 & 1 & YES & YES & NO(2) & $1.70$ & $(6,1)$ & NO & 5093\\
$(82,31)$ & 10 & $(7,2)$ & 4 & 1 & YES & YES & NO(2) & $1.70$ & $(6,1)$ & -- & 5094\\
$(82,31)$ & 10 & $(8,3)$ & 4 & 2 & YES & YES & NO(2) & $1.44$ & $(4,2)$ & -- & 5095\\
$(82,31)$ & 10 & $(10,3)$ & 5 & 2 & YES & YES & YES & $1.62$ & $(2,3)$ & -- & 5096\\
$(82,25)$ & 10 & $(12,5)$ & 5 & 2 & YES & YES & YES & $1.75$ & $(2,3)$ & -- & 5097\\
$(82,23)$ & 10 & $(13,4)$ & 6 & 1 & YES & YES & YES & $1.57$ & $(6,1)$ & NO & 5098\\
$(82,23)$ & 10 & $(13,4)$ & 6 & 1 & YES & YES & YES & $1.57$ & $(6,1)$ & -- & 5099\\
$(82,23)$ & 10 & $(13,5)$ & 5 & 1 & YES & YES & YES & $1.78$ & $(2,3)$ & -- & 5100\\
$(82,31)$ & 10 & $(14,3)$ & 6 & 2 & YES & YES & YES & $1.62$ & $(2,3)$ & NO & 5101\\
$(82,23)$ & 10 & $(15,4)$ & 6 & 1 & YES & YES & YES & $1.57$ & $(4,2)$ & -- & 5102\\
$(82,25)$ & 10 & $(23,5)$ & 7 & 1 & YES & YES & YES & $1.67$ & $(2,3)$ & NO & 5103\\
$(82,31)$ & 10 & $(47,18)$ & 8 & 1 & YES & YES & YES & $1.62$ & $(2,3)$ & NO & 5104\\
$(82,31)$ & 10 & $(79,30)$ & 9 & 1 & YES & YES & YES & $1.75$ & $(2,3)$ & NO & 5105\\
$(83,18)$ & 10 & $(9,4)$ & 5 & 1 & YES & YES & NO(2) & $1.80$ & $(2,3)$ & NO & 5106\\
$(83,18)$ & 10 & $(9,4)$ & 5 & 1 & YES & YES & NO(2) & $1.80$ & $(2,3)$ & -- & 5107\\
$(83,23)$ & 10 & $(9,4)$ & 5 & 1 & YES & YES & NO(2) & $1.80$ & $(2,3)$ & NO & 5108\\
$(83,23)$ & 10 & $(12,5)$ & 5 & 1 & YES & YES & NO(2) & $1.78$ & $(2,3)$ & NO & 5109\\
$(83,34)$ & 10 & $(14,3)$ & 6 & 1 & YES & YES & NO(2) & $1.60$ & $(6,1)$ & -- & 5110\\
$(83,19)$ & 10 & $(17,7)$ & 6 & 1 & YES & YES & YES & $1.78$ & $(2,3)$ & NO & 5111\\
$(83,19)$ & 10 & $(17,7)$ & 6 & 1 & YES & YES & YES & $1.78$ & $(2,3)$ & -- & 5112\\
$(83,22)$ & 10 & $(17,7)$ & 6 & 1 & YES & YES & YES & $2.00$ & $(2,3)$ & NO & 5113\\
$(83,23)$ & 10 & $(17,3)$ & 7 & 1 & YES & YES & NO(2) & $1.44$ & $(8,0)$ & NO & 5114\\
$(83,19)$ & 10 & $(18,7)$ & 6 & 1 & YES & YES & YES & $1.78$ & $(2,3)$ & -- & 5115\\
$(83,22)$ & 10 & $(18,7)$ & 6 & 1 & YES & YES & YES & $2.00$ & $(2,3)$ & NO & 5116\\
$(83,22)$ & 10 & $(18,7)$ & 6 & 1 & YES & YES & YES & $2.00$ & $(2,3)$ & -- & 5117\\
$(83,36)$ & 10 & $(18,5)$ & 6 & 1 & YES & YES & YES & $1.43$ & $(6,1)$ & NO & 5118\\
$(83,36)$ & 10 & $(18,5)$ & 6 & 1 & YES & YES & YES & $1.57$ & $(6,1)$ & -- & 5119\\
$(83,36)$ & 10 & $(20,9)$ & 7 & 1 & YES & YES & NO(2) & $1.38$ & $(6,1)$ & NO & 5120\\
$(83,23)$ & 10 & $(24,7)$ & 7 & 1 & YES & YES & NO(2) & $1.80$ & $(2,3)$ & NO & 5121\\
$(83,36)$ & 10 & $(25,11)$ & 7 & 1 & YES & YES & NO(2) & $1.38$ & $(6,1)$ & 6359 & 5122\\
$(83,23)$ & 10 & $(27,8)$ & 7 & 1 & YES & YES & NO(2) & $1.78$ & $(2,3)$ & NO & 5123\\
$(83,36)$ & 10 & $(59,26)$ & 9 & 1 & YES & YES & YES & $1.57$ & $(6,1)$ & 6364 & 5124\\
$(84,25)$ & 10 & $(7,2)$ & 4 & 7 & YES & YES & NO(2) & $1.38$ & $(6,1)$ & -- & 5125\\
$(84,37)$ & 10 & $(7,2)$ & 4 & 7 & YES & YES & NO(2) & $1.80$ & $(2,3)$ & -- & 5126\\
$(84,19)$ & 10 & $(9,4)$ & 5 & 3 & YES & YES & NO(2) & $1.60$ & $(6,1)$ & NO & 5127\\
$(84,19)$ & 10 & $(9,4)$ & 5 & 3 & YES & YES & NO(2) & $1.60$ & $(6,1)$ & -- & 5128\\
$(84,31)$ & 10 & $(10,3)$ & 5 & 2 & YES & YES & NO(2) & $1.60$ & $(6,1)$ & -- & 5129\\
$(84,19)$ & 10 & $(13,4)$ & 6 & 1 & YES & YES & NO(2) & $1.56$ & $(2,3)$ & -- & 5130\\
$(84,37)$ & 10 & $(13,4)$ & 6 & 1 & YES & YES & NO(2) & $1.56$ & $(8,0)$ & NO & 5131\\
$(84,37)$ & 10 & $(14,5)$ & 6 & 14 & YES & YES & NO(2) & $1.56$ & $(8,0)$ & NO & 5132\\
$(84,19)$ & 10 & $(16,5)$ & 7 & 4 & YES & YES & YES & $1.62$ & $(4,2)$ & -- & 5133\\
$(84,19)$ & 10 & $(17,7)$ & 6 & 1 & YES & YES & YES & $1.43$ & $(4,2)$ & -- & 5134\\
$(84,19)$ & 10 & $(19,5)$ & 7 & 1 & YES & YES & YES & $1.62$ & $(4,2)$ & -- & 5135\\
$(84,25)$ & 10 & $(36,11)$ & 8 & 12 & YES & YES & NO(2) & $1.56$ & $(4,2)$ & NO & 5136\\
$(84,25)$ & 10 & $(41,12)$ & 8 & 1 & YES & YES & NO(2) & $1.70$ & $(2,3)$ & NO & 5137\\
$(84,25)$ & 10 & $(61,18)$ & 9 & 1 & YES & YES & NO(2) & $1.44$ & $(8,0)$ & 5565 & 5138\\
$(85,36)$ & 10 & $(5,2)$ & 3 & 5 & YES & YES & NO(2) & $1.60$ & $(6,1)$ & -- & 5139\\
$(85,26)$ & 10 & $(7,2)$ & 4 & 1 & YES & YES & NO(2) & $1.64$ & $(4,2)$ & -- & 5140\\
$(85,33)$ & 10 & $(13,3)$ & 6 & 1 & YES & YES & YES & $1.67$ & $(2,3)$ & -- & 5141\\
$(85,37)$ & 10 & $(13,5)$ & 5 & 1 & YES & YES & NO(2) & $1.56$ & $(8,0)$ & NO & 5142\\
$(85,33)$ & 10 & $(14,3)$ & 6 & 1 & YES & YES & YES & $1.78$ & $(2,3)$ & -- & 5143\\
$(85,33)$ & 10 & $(14,3)$ & 6 & 1 & YES & YES & YES & $2.00$ & $(2,3)$ & NO & 5144\\
$(85,18)$ & 10 & $(18,7)$ & 6 & 1 & YES & YES & NO(2) & $1.60$ & $(6,1)$ & -- & 5145\\
$(85,33)$ & 10 & $(19,3)$ & 8 & 1 & YES & YES & NO(2) & $1.44$ & $(8,0)$ & NO & 5146\\
$(85,26)$ & 10 & $(24,7)$ & 7 & 1 & YES & YES & NO(2) & $1.70$ & $(2,3)$ & NO & 5147\\
$(85,26)$ & 10 & $(56,17)$ & 9 & 1 & YES & YES & NO(2) & $1.70$ & $(2,3)$ & NO & 5148\\
$(86,25)$ & 10 & $(5,2)$ & 3 & 1 & YES & YES & NO(2) & $1.73$ & $(4,2)$ & -- & 5149\\
$(86,25)$ & 10 & $(7,2)$ & 4 & 1 & YES & YES & NO(2) & $1.50$ & $(2,3)$ & -- & 5150\\
$(86,27)$ & 11 & $(7,2)$ & 4 & 1 & YES & YES & NO(2) & $1.67$ & $(8,0)$ & NO & 5151\\
$(86,23)$ & 11 & $(10,3)$ & 5 & 2 & YES & YES & NO(2) & $1.80$ & $(2,3)$ & -- & 5152\\
$(86,25)$ & 10 & $(13,4)$ & 6 & 1 & YES & YES & YES & $1.75$ & $(2,3)$ & -- & 5153\\
$(86,25)$ & 10 & $(13,5)$ & 5 & 1 & YES & YES & YES & $1.78$ & $(2,3)$ & -- & 5154\\
$(86,33)$ & 11 & $(13,2)$ & 7 & 1 & YES & YES & NO(2) & $1.67$ & $(8,0)$ & NO & 5155\\
$(86,25)$ & 10 & $(15,4)$ & 6 & 1 & YES & YES & YES & $1.62$ & $(2,3)$ & -- & 5156\\
$(86,23)$ & 11 & $(17,5)$ & 6 & 1 & YES & YES & NO(2) & $1.70$ & $(2,3)$ & NO & 5157\\
$(86,25)$ & 10 & $(19,5)$ & 7 & 1 & YES & YES & YES & $1.62$ & $(2,3)$ & NO & 5158\\
$(86,25)$ & 10 & $(58,17)$ & 9 & 2 & YES & YES & NO(2) & $1.44$ & $(4,2)$ & NO & 5159\\
$(86,25)$ & 10 & $(79,23)$ & 10 & 1 & YES & YES & NO(2) & $1.64$ & $(4,2)$ & NO & 5160\\
$(86,33)$ & 11 & $(81,31)$ & 9 & 1 & YES & YES & NO(2) & $1.67$ & $(8,0)$ & NO & 5161\\
$(87,23)$ & 10 & $(7,2)$ & 4 & 1 & YES & YES & NO(2) & $1.38$ & $(6,1)$ & -- & 5162\\
$(87,31)$ & 12 & $(7,3)$ & 4 & 1 & YES & YES & YES & $1.71$ & $(4,2)$ & -- & 5163\\
$(87,32)$ & 10 & $(7,2)$ & 4 & 1 & YES & YES & NO(2) & $1.38$ & $(10,-1)$ & -- & 5164\\
$(87,34)$ & 10 & $(7,3)$ & 4 & 1 & YES & YES & NO(2) & $1.78$ & $(8,0)$ & -- & 5165\\
$(87,34)$ & 10 & $(8,3)$ & 4 & 1 & YES & YES & NO(2) & $1.56$ & $(8,0)$ & -- & 5166\\
$(87,19)$ & 10 & $(9,4)$ & 5 & 3 & YES & YES & NO(2) & $1.60$ & $(6,1)$ & NO & 5167\\
$(87,19)$ & 10 & $(9,4)$ & 5 & 3 & YES & YES & NO(2) & $1.60$ & $(6,1)$ & -- & 5168\\
$(87,31)$ & 12 & $(12,5)$ & 5 & 3 & YES & YES & YES & $1.86$ & $(4,2)$ & NO & 5169\\
$(87,19)$ & 10 & $(13,6)$ & 7 & 1 & YES & YES & NO(2) & $1.60$ & $(6,1)$ & -- & 5170\\
$(87,19)$ & 10 & $(19,5)$ & 7 & 1 & YES & YES & YES & $1.62$ & $(4,2)$ & -- & 5171\\
$(87,34)$ & 10 & $(19,7)$ & 6 & 1 & YES & YES & NO(2) & $1.67$ & $(8,0)$ & NO & 5172\\
$(87,23)$ & 10 & $(25,7)$ & 7 & 1 & YES & YES & NO(2) & $1.38$ & $(10,-1)$ & 4349 & 5173\\
$(87,19)$ & 10 & $(29,7)$ & 10 & 29 & YES & YES & NO(2) & $1.60$ & $(6,1)$ & NO & 5174\\
$(87,34)$ & 10 & $(44,17)$ & 8 & 1 & YES & YES & NO(2) & $1.44$ & $(8,0)$ & NO & 5175\\
$(87,34)$ & 10 & $(49,19)$ & 8 & 1 & YES & YES & NO(2) & $1.67$ & $(8,0)$ & NO & 5176\\
$(87,32)$ & 10 & $(52,19)$ & 9 & 1 & YES & YES & NO(2) & $1.67$ & $(4,2)$ & NO & 5177\\
$(88,37)$ & 10 & $(8,3)$ & 4 & 8 & YES & YES & YES & $1.71$ & $(2,3)$ & -- & 5178\\
$(88,37)$ & 10 & $(11,3)$ & 5 & 11 & YES & YES & YES & $1.71$ & $(2,3)$ & NO & 5179\\
$(88,37)$ & 10 & $(11,3)$ & 5 & 11 & YES & YES & YES & $1.89$ & $(2,3)$ & -- & 5180\\
$(88,37)$ & 10 & $(11,3)$ & 5 & 11 & YES & YES & YES & $1.89$ & $(2,3)$ & NO & 5181\\
$(88,37)$ & 10 & $(14,3)$ & 6 & 2 & YES & YES & YES & $1.89$ & $(2,3)$ & NO & 5182\\
$(88,37)$ & 10 & $(14,3)$ & 6 & 2 & YES & YES & YES & $1.78$ & $(2,3)$ & -- & 5183\\
$(88,37)$ & 10 & $(16,3)$ & 7 & 8 & YES & YES & YES & $1.62$ & $(2,3)$ & -- & 5184\\
$(88,37)$ & 10 & $(74,31)$ & 9 & 2 & YES & YES & YES & $1.71$ & $(2,3)$ & 8592 & 5185\\
$(89,34)$ & 9 & $(3,1)$ & 2 & 1 & YES & YES & NO(2) & $1.50$ & $(10,-1)$ & NO & 5186\\
$(89,34)$ & 9 & $(4,1)$ & 3 & 1 & YES & YES & NO(2) & $1.60$ & $(2,3)$ & NO & 5187\\
$(89,34)$ & 9 & $(4,1)$ & 3 & 1 & YES & YES & NO(2) & $1.60$ & $(2,3)$ & -- & 5188\\
$(89,34)$ & 9 & $(4,1)$ & 3 & 1 & YES & YES & NO(2) & $1.60$ & $(2,3)$ & NO & 5189\\
$(89,25)$ & 10 & $(5,2)$ & 3 & 1 & YES & YES & NO(2) & $1.73$ & $(4,2)$ & -- & 5190\\
$(89,26)$ & 10 & $(5,2)$ & 3 & 1 & YES & YES & NO(2) & $1.50$ & $(2,3)$ & NO & 5191\\
$(89,34)$ & 9 & $(5,2)$ & 3 & 1 & YES & YES & NO(2) & $1.60$ & $(2,3)$ & NO & 5192\\
$(89,34)$ & 9 & $(5,2)$ & 3 & 1 & YES & YES & NO(2) & $1.60$ & $(2,3)$ & -- & 5193\\
$(89,26)$ & 10 & $(7,3)$ & 4 & 1 & YES & YES & NO(2) & $1.70$ & $(2,3)$ & -- & 5194\\
$(89,34)$ & 9 & $(7,2)$ & 4 & 1 & YES & YES & YES & $1.57$ & $(6,1)$ & -- & 5195\\
$(89,34)$ & 9 & $(7,3)$ & 4 & 1 & YES & YES & NO(2) & $1.73$ & $(4,2)$ & -- & 5196\\
$(89,34)$ & 9 & $(7,3)$ & 4 & 1 & YES & YES & YES & $1.44$ & $(2,3)$ & NO & 5197\\
$(89,34)$ & 9 & $(8,3)$ & 4 & 1 & YES & YES & YES & $1.67$ & $(2,3)$ & -- & 5198\\
$(89,32)$ & 10 & $(9,4)$ & 5 & 1 & YES & YES & NO(2) & $1.56$ & $(8,0)$ & -- & 5199\\
$(89,34)$ & 9 & $(9,4)$ & 5 & 1 & YES & YES & NO(2) & $1.56$ & $(4,2)$ & NO & 5200\\
$(89,34)$ & 9 & $(9,4)$ & 5 & 1 & YES & YES & NO(2) & $1.56$ & $(4,2)$ & -- & 5201\\
$(89,24)$ & 10 & $(10,3)$ & 5 & 1 & YES & YES & NO(2) & $1.50$ & $(2,3)$ & NO & 5202\\
$(89,27)$ & 10 & $(10,3)$ & 5 & 1 & YES & YES & NO(2) & $1.67$ & $(4,2)$ & NO & 5203\\
$(89,33)$ & 10 & $(10,3)$ & 5 & 1 & YES & YES & NO(2) & $1.67$ & $(8,0)$ & -- & 5204\\
$(89,34)$ & 9 & $(10,3)$ & 5 & 1 & YES & YES & YES & $1.89$ & $(2,3)$ & NO & 5205\\
$(89,34)$ & 9 & $(10,3)$ & 5 & 1 & YES & YES & YES & $1.89$ & $(2,3)$ & -- & 5206\\
$(89,24)$ & 10 & $(11,4)$ & 5 & 1 & YES & YES & NO(2) & $1.38$ & $(4,2)$ & -- & 5207\\
$(89,34)$ & 9 & $(11,3)$ & 5 & 1 & YES & YES & YES & $1.89$ & $(2,3)$ & NO & 5208\\
$(89,34)$ & 9 & $(11,3)$ & 5 & 1 & YES & YES & YES & $1.89$ & $(2,3)$ & -- & 5209\\
$(89,34)$ & 9 & $(11,4)$ & 5 & 1 & YES & YES & YES & $1.50$ & $(2,3)$ & NO & 5210\\
$(89,26)$ & 10 & $(12,5)$ & 5 & 1 & YES & YES & YES & $1.89$ & $(2,3)$ & NO & 5211\\
$(89,27)$ & 10 & $(12,5)$ & 5 & 1 & YES & YES & YES & $1.75$ & $(2,3)$ & -- & 5212\\
$(89,33)$ & 10 & $(12,5)$ & 5 & 1 & YES & YES & NO(2) & $1.50$ & $(6,1)$ & NO & 5213\\
$(89,34)$ & 9 & $(12,5)$ & 5 & 1 & YES & YES & NO(2) & $1.67$ & $(4,2)$ & NO & 5214\\
$(89,34)$ & 9 & $(12,5)$ & 5 & 1 & YES & YES & YES & $1.62$ & $(2,3)$ & -- & 5215\\
$(89,24)$ & 10 & $(13,5)$ & 5 & 1 & YES & YES & YES & $1.67$ & $(2,3)$ & -- & 5216\\
$(89,25)$ & 10 & $(13,4)$ & 6 & 1 & YES & YES & YES & $1.88$ & $(2,3)$ & -- & 5217\\
$(89,26)$ & 10 & $(13,3)$ & 6 & 1 & YES & YES & YES & $1.50$ & $(2,3)$ & -- & 5218\\
$(89,26)$ & 10 & $(13,4)$ & 6 & 1 & YES & YES & NO(2) & $1.50$ & $(2,3)$ & NO & 5219\\
$(89,26)$ & 10 & $(13,4)$ & 6 & 1 & YES & YES & YES & $1.43$ & $(4,2)$ & -- & 5220\\
$(89,26)$ & 10 & $(13,5)$ & 5 & 1 & YES & YES & YES & $1.70$ & $(2,3)$ & -- & 5221\\
$(89,27)$ & 10 & $(13,4)$ & 6 & 1 & YES & YES & YES & $2.00$ & $(2,3)$ & -- & 5222\\
$(89,27)$ & 10 & $(13,5)$ & 5 & 1 & YES & YES & YES & $1.70$ & $(2,3)$ & -- & 5223\\
$(89,37)$ & 12 & $(13,2)$ & 7 & 1 & YES & YES & NO(2) & $1.70$ & $(6,1)$ & NO & 5224\\
$(89,34)$ & 9 & $(14,5)$ & 6 & 1 & YES & YES & NO(2) & $1.83$ & $(2,3)$ & NO & 5225\\
$(89,25)$ & 10 & $(15,4)$ & 6 & 1 & YES & YES & YES & $1.62$ & $(2,3)$ & -- & 5226\\
$(89,34)$ & 9 & $(17,7)$ & 6 & 1 & YES & YES & NO(2) & $1.56$ & $(4,2)$ & NO & 5227\\
$(89,24)$ & 10 & $(18,5)$ & 6 & 1 & YES & YES & YES & $1.43$ & $(4,2)$ & -- & 5228\\
$(89,26)$ & 10 & $(18,5)$ & 6 & 1 & YES & YES & YES & $1.43$ & $(2,3)$ & -- & 5229\\
$(89,34)$ & 9 & $(19,8)$ & 6 & 1 & YES & YES & YES & $1.75$ & $(2,3)$ & NO & 5230\\
$(89,32)$ & 10 & $(21,8)$ & 6 & 1 & YES & YES & YES & $1.62$ & $(2,3)$ & NO & 5231\\
$(89,27)$ & 10 & $(22,5)$ & 7 & 1 & YES & YES & YES & $1.78$ & $(2,3)$ & -- & 5232\\
$(89,24)$ & 10 & $(23,6)$ & 8 & 1 & YES & YES & NO(2) & $1.56$ & $(8,0)$ & NO & 5233\\
$(89,26)$ & 10 & $(23,7)$ & 7 & 1 & YES & YES & NO(2) & $1.70$ & $(2,3)$ & NO & 5234\\
$(89,34)$ & 9 & $(23,9)$ & 7 & 1 & YES & YES & NO(2) & $1.56$ & $(4,2)$ & NO & 5235\\
$(89,27)$ & 10 & $(25,7)$ & 7 & 1 & YES & YES & NO(2) & $1.50$ & $(4,2)$ & NO & 5236\\
$(89,34)$ & 9 & $(30,11)$ & 7 & 1 & YES & YES & YES & $1.62$ & $(2,3)$ & NO & 5237\\
$(89,24)$ & 10 & $(32,9)$ & 8 & 1 & YES & YES & NO(2) & $1.50$ & $(6,1)$ & NO & 5238\\
$(89,33)$ & 10 & $(36,13)$ & 8 & 1 & YES & YES & NO(2) & $1.62$ & $(6,1)$ & NO & 5239\\
$(89,26)$ & 10 & $(37,11)$ & 8 & 1 & YES & YES & NO(2) & $1.64$ & $(4,2)$ & NO & 5240\\
$(89,34)$ & 9 & $(37,14)$ & 8 & 1 & YES & YES & NO(2) & $1.73$ & $(4,2)$ & NO & 5241\\
$(89,34)$ & 9 & $(41,16)$ & 8 & 1 & YES & YES & YES & $1.29$ & $(4,2)$ & NO & 5242\\
$(89,24)$ & 10 & $(43,12)$ & 8 & 1 & YES & YES & YES & $1.50$ & $(4,2)$ & NO & 5243\\
$(89,34)$ & 9 & $(47,18)$ & 8 & 1 & YES & YES & YES & $1.50$ & $(2,3)$ & NO & 5244\\
$(89,27)$ & 10 & $(49,15)$ & 9 & 1 & YES & YES & NO(2) & $1.60$ & $(6,1)$ & NO & 5245\\
$(89,34)$ & 9 & $(49,19)$ & 8 & 1 & YES & YES & YES & $1.75$ & $(4,2)$ & NO & 5246\\
$(89,26)$ & 10 & $(64,19)$ & 9 & 1 & YES & YES & YES & $1.70$ & $(2,3)$ & NO & 5247\\
$(89,34)$ & 9 & $(73,28)$ & 10 & 1 & YES & YES & YES & $1.71$ & $(2,3)$ & 7863 & 5248\\
$(89,25)$ & 10 & $(82,23)$ & 10 & 1 & YES & YES & NO(2) & $1.64$ & $(4,2)$ & NO & 5249\\
$(91,41)$ & 11 & $(7,2)$ & 4 & 7 & YES & YES & NO(2) & $1.62$ & $(6,1)$ & -- & 5250\\
$(91,25)$ & 10 & $(8,3)$ & 4 & 1 & YES & YES & NO(2) & $1.60$ & $(6,1)$ & -- & 5251\\
$(91,25)$ & 10 & $(8,3)$ & 4 & 1 & YES & YES & NO(2) & $1.70$ & $(6,1)$ & NO & 5252\\
$(91,25)$ & 10 & $(9,4)$ & 5 & 1 & YES & YES & NO(2) & $1.67$ & $(8,0)$ & NO & 5253\\
$(91,25)$ & 10 & $(9,4)$ & 5 & 1 & YES & YES & NO(2) & $1.67$ & $(8,0)$ & -- & 5254\\
$(91,27)$ & 10 & $(9,4)$ & 5 & 1 & YES & YES & YES & $1.71$ & $(2,3)$ & -- & 5255\\
$(91,25)$ & 10 & $(10,3)$ & 5 & 1 & YES & YES & NO(2) & $1.33$ & $(8,0)$ & -- & 5256\\
$(91,40)$ & 10 & $(10,3)$ & 5 & 1 & YES & YES & YES & $1.43$ & $(4,2)$ & -- & 5257\\
$(91,27)$ & 10 & $(11,3)$ & 5 & 1 & YES & YES & YES & $1.50$ & $(2,3)$ & -- & 5258\\
$(91,19)$ & 11 & $(12,5)$ & 5 & 1 & YES & YES & NO(2) & $1.70$ & $(10,-1)$ & NO & 5259\\
$(91,25)$ & 10 & $(13,5)$ & 5 & 13 & YES & YES & YES & $1.43$ & $(4,2)$ & -- & 5260\\
$(91,27)$ & 10 & $(13,4)$ & 6 & 13 & YES & YES & YES & $1.75$ & $(2,3)$ & -- & 5261\\
$(91,27)$ & 10 & $(13,5)$ & 5 & 13 & YES & YES & YES & $1.78$ & $(2,3)$ & -- & 5262\\
$(91,40)$ & 10 & $(13,3)$ & 6 & 13 & YES & YES & YES & $1.57$ & $(4,2)$ & NO & 5263\\
$(91,40)$ & 10 & $(13,3)$ & 6 & 13 & YES & YES & YES & $1.57$ & $(4,2)$ & -- & 5264\\
$(91,27)$ & 10 & $(14,3)$ & 6 & 7 & YES & YES & NO(2) & $1.44$ & $(2,3)$ & -- & 5265\\
$(91,27)$ & 10 & $(15,4)$ & 6 & 1 & YES & YES & YES & $1.71$ & $(2,3)$ & -- & 5266\\
$(91,27)$ & 10 & $(18,5)$ & 6 & 1 & YES & YES & YES & $1.57$ & $(2,3)$ & -- & 5267\\
$(91,19)$ & 11 & $(30,7)$ & 8 & 1 & YES & YES & NO(2) & $1.60$ & $(10,-1)$ & NO & 5268\\
$(91,25)$ & 10 & $(43,12)$ & 8 & 1 & YES & YES & NO(2) & $1.60$ & $(6,1)$ & NO & 5269\\
$(92,35)$ & 10 & $(3,1)$ & 2 & 1 & YES & YES & NO(2) & $1.44$ & $(4,2)$ & -- & 5270\\
$(92,35)$ & 10 & $(13,3)$ & 6 & 1 & YES & YES & YES & $1.62$ & $(4,2)$ & -- & 5271\\
$(92,35)$ & 10 & $(13,3)$ & 6 & 1 & YES & YES & YES & $1.75$ & $(4,2)$ & NO & 5272\\
$(92,27)$ & 11 & $(14,3)$ & 6 & 2 & YES & YES & YES & $1.57$ & $(4,2)$ & -- & 5273\\
$(92,27)$ & 11 & $(16,5)$ & 7 & 4 & YES & YES & YES & $1.75$ & $(2,3)$ & NO & 5274\\
$(92,35)$ & 10 & $(23,5)$ & 7 & 23 & YES & YES & YES & $1.62$ & $(4,2)$ & -- & 5275\\
$(92,33)$ & 10 & $(64,23)$ & 9 & 4 & YES & YES & YES & $1.29$ & $(4,2)$ & NO & 5276\\
$(92,39)$ & 10 & $(85,36)$ & 10 & 1 & YES & YES & NO(2) & $1.70$ & $(6,1)$ & NO & 5277\\
$(92,35)$ & 10 & $(92,35)$ & 10 & 92 & YES & YES & NO(2) & $1.44$ & $(4,2)$ & NO & 5278\\
$(93,26)$ & 10 & $(7,3)$ & 4 & 1 & YES & YES & NO(2) & $1.80$ & $(2,3)$ & NO & 5279\\
$(93,26)$ & 10 & $(7,3)$ & 4 & 1 & YES & YES & NO(2) & $1.80$ & $(2,3)$ & -- & 5280\\
$(93,34)$ & 10 & $(7,2)$ & 4 & 1 & YES & YES & NO(2) & $1.50$ & $(6,1)$ & -- & 5281\\
$(93,34)$ & 10 & $(8,3)$ & 4 & 1 & YES & YES & NO(2) & $1.44$ & $(8,0)$ & -- & 5282\\
$(93,34)$ & 10 & $(9,2)$ & 5 & 3 & YES & YES & NO(2) & $1.44$ & $(8,0)$ & -- & 5283\\
$(93,41)$ & 10 & $(10,3)$ & 5 & 1 & YES & YES & NO(2) & $1.70$ & $(10,-1)$ & NO & 5284\\
$(93,41)$ & 10 & $(11,3)$ & 5 & 1 & YES & YES & NO(2) & $1.60$ & $(10,-1)$ & -- & 5285\\
$(93,41)$ & 10 & $(11,4)$ & 5 & 1 & YES & YES & NO(2) & $1.70$ & $(10,-1)$ & NO & 5286\\
$(93,25)$ & 10 & $(13,3)$ & 6 & 1 & YES & YES & NO(2) & $1.56$ & $(2,3)$ & -- & 5287\\
$(93,25)$ & 10 & $(14,3)$ & 6 & 1 & YES & YES & YES & $1.43$ & $(2,3)$ & -- & 5288\\
$(93,22)$ & 11 & $(17,7)$ & 6 & 1 & YES & YES & YES & $1.67$ & $(4,2)$ & -- & 5289\\
$(93,22)$ & 11 & $(17,7)$ & 6 & 1 & YES & YES & YES & $1.78$ & $(4,2)$ & NO & 5290\\
$(93,41)$ & 10 & $(19,8)$ & 6 & 1 & YES & YES & NO(2) & $1.70$ & $(10,-1)$ & NO & 5291\\
$(93,22)$ & 11 & $(21,8)$ & 6 & 3 & YES & YES & YES & $1.62$ & $(4,2)$ & -- & 5292\\
$(93,26)$ & 10 & $(23,5)$ & 7 & 1 & YES & YES & YES & $1.56$ & $(2,3)$ & NO & 5293\\
$(93,34)$ & 10 & $(27,10)$ & 7 & 3 & YES & YES & NO(2) & $1.44$ & $(8,0)$ & NO & 5294\\
$(93,41)$ & 10 & $(30,13)$ & 8 & 3 & YES & YES & NO(2) & $1.70$ & $(10,-1)$ & NO & 5295\\
$(93,41)$ & 10 & $(66,29)$ & 9 & 3 & YES & YES & NO(2) & $1.70$ & $(10,-1)$ & NO & 5296\\
$(93,26)$ & 10 & $(68,19)$ & 9 & 1 & YES & YES & YES & $1.38$ & $(2,3)$ & NO & 5297\\
$(94,39)$ & 10 & $(8,3)$ & 4 & 2 & YES & YES & NO(2) & $1.56$ & $(8,0)$ & -- & 5298\\
$(94,39)$ & 10 & $(10,3)$ & 5 & 2 & YES & YES & NO(2) & $1.67$ & $(2,3)$ & -- & 5299\\
$(94,39)$ & 10 & $(13,3)$ & 6 & 1 & YES & YES & NO(2) & $1.67$ & $(2,3)$ & NO & 5300\\
$(94,39)$ & 10 & $(13,3)$ & 6 & 1 & YES & YES & NO(2) & $1.67$ & $(2,3)$ & -- & 5301\\
$(94,41)$ & 10 & $(13,5)$ & 5 & 1 & YES & YES & NO(2) & $1.56$ & $(8,0)$ & NO & 5302\\
$(94,41)$ & 10 & $(71,31)$ & 10 & 1 & YES & YES & NO(2) & $1.44$ & $(8,0)$ & NO & 5303\\
$(94,39)$ & 10 & $(75,31)$ & 9 & 1 & YES & YES & NO(2) & $1.56$ & $(2,3)$ & NO & 5304\\
$(94,39)$ & 10 & $(89,37)$ & 12 & 1 & YES & YES & NO(2) & $1.70$ & $(6,1)$ & NO & 5305\\
$(95,28)$ & 11 & $(5,1)$ & 4 & 5 & YES & YES & NO(2) & $1.70$ & $(6,1)$ & -- & 5306\\
$(95,28)$ & 11 & $(5,1)$ & 4 & 5 & YES & YES & NO(2) & $1.80$ & $(6,1)$ & NO & 5307\\
$(95,29)$ & 10 & $(5,2)$ & 3 & 5 & YES & YES & NO(2) & $1.60$ & $(2,3)$ & -- & 5308\\
$(95,36)$ & 10 & $(5,2)$ & 3 & 5 & YES & YES & NO(2) & $1.60$ & $(2,3)$ & -- & 5309\\
$(95,39)$ & 10 & $(5,2)$ & 3 & 5 & YES & YES & NO(2) & $1.50$ & $(6,1)$ & -- & 5310\\
$(95,36)$ & 10 & $(7,2)$ & 4 & 1 & YES & YES & NO(2) & $1.60$ & $(2,3)$ & -- & 5311\\
$(95,36)$ & 10 & $(7,2)$ & 4 & 1 & YES & YES & NO(2) & $1.56$ & $(8,0)$ & NO & 5312\\
$(95,39)$ & 10 & $(7,2)$ & 4 & 1 & YES & YES & NO(2) & $1.50$ & $(6,1)$ & -- & 5313\\
$(95,28)$ & 11 & $(8,3)$ & 4 & 1 & YES & YES & YES & $1.50$ & $(2,3)$ & NO & 5314\\
$(95,36)$ & 10 & $(9,4)$ & 5 & 1 & YES & YES & NO(2) & $1.50$ & $(6,1)$ & -- & 5315\\
$(95,36)$ & 10 & $(10,3)$ & 5 & 5 & YES & YES & NO(2) & $1.70$ & $(2,3)$ & NO & 5316\\
$(95,36)$ & 10 & $(10,3)$ & 5 & 5 & YES & YES & YES & $1.75$ & $(4,2)$ & -- & 5317\\
$(95,39)$ & 10 & $(10,3)$ & 5 & 5 & YES & YES & YES & $1.90$ & $(2,3)$ & -- & 5318\\
$(95,39)$ & 10 & $(12,5)$ & 5 & 1 & YES & YES & YES & $1.75$ & $(4,2)$ & -- & 5319\\
$(95,29)$ & 10 & $(13,5)$ & 5 & 1 & YES & YES & YES & $1.78$ & $(4,2)$ & -- & 5320\\
$(95,28)$ & 11 & $(16,5)$ & 7 & 1 & YES & YES & YES & $1.62$ & $(2,3)$ & NO & 5321\\
$(95,29)$ & 10 & $(16,5)$ & 7 & 1 & YES & YES & NO(2) & $1.50$ & $(6,1)$ & NO & 5322\\
$(95,29)$ & 10 & $(17,5)$ & 6 & 1 & YES & YES & NO(2) & $1.50$ & $(2,3)$ & NO & 5323\\
$(95,39)$ & 10 & $(17,5)$ & 6 & 1 & YES & YES & YES & $1.88$ & $(4,2)$ & NO & 5324\\
$(95,28)$ & 11 & $(18,5)$ & 6 & 1 & YES & YES & NO(2) & $1.70$ & $(2,3)$ & NO & 5325\\
$(95,36)$ & 10 & $(21,8)$ & 6 & 1 & YES & YES & NO(2) & $1.50$ & $(2,3)$ & NO & 5326\\
$(95,29)$ & 10 & $(22,5)$ & 7 & 1 & YES & YES & YES & $1.67$ & $(4,2)$ & -- & 5327\\
$(95,36)$ & 10 & $(23,9)$ & 7 & 1 & YES & YES & NO(2) & $1.50$ & $(4,2)$ & NO & 5328\\
$(96,17)$ & 12 & $(10,3)$ & 5 & 2 & YES & YES & NO(2) & $1.78$ & $(4,2)$ & NO & 5329\\
$(96,37)$ & 12 & $(11,2)$ & 6 & 1 & YES & YES & NO(2) & $1.70$ & $(6,1)$ & -- & 5330\\
$(96,29)$ & 11 & $(29,8)$ & 7 & 1 & YES & YES & YES & $1.71$ & $(6,1)$ & NO & 5331\\
$(97,36)$ & 10 & $(4,1)$ & 3 & 1 & YES & YES & YES & $1.75$ & $(2,3)$ & NO & 5332\\
$(97,35)$ & 10 & $(5,2)$ & 3 & 1 & YES & YES & YES & $1.43$ & $(4,2)$ & -- & 5333\\
$(97,37)$ & 10 & $(5,1)$ & 4 & 1 & YES & YES & YES & $1.57$ & $(4,2)$ & -- & 5334\\
$(97,27)$ & 11 & $(7,3)$ & 4 & 1 & YES & YES & NO(2) & $1.80$ & $(2,3)$ & NO & 5335\\
$(97,35)$ & 10 & $(7,2)$ & 4 & 1 & YES & YES & NO(2) & $1.25$ & $(10,-1)$ & -- & 5336\\
$(97,26)$ & 10 & $(8,3)$ & 4 & 1 & YES & YES & NO(2) & $1.50$ & $(6,1)$ & -- & 5337\\
$(97,36)$ & 10 & $(8,3)$ & 4 & 1 & YES & YES & NO(2) & $1.56$ & $(8,0)$ & -- & 5338\\
$(97,37)$ & 10 & $(9,4)$ & 5 & 1 & YES & YES & NO(2) & $1.70$ & $(2,3)$ & NO & 5339\\
$(97,35)$ & 10 & $(10,3)$ & 5 & 1 & YES & YES & YES & $1.43$ & $(4,2)$ & -- & 5340\\
$(97,37)$ & 10 & $(11,3)$ & 5 & 1 & YES & YES & YES & $1.62$ & $(2,3)$ & -- & 5341\\
$(97,41)$ & 10 & $(11,4)$ & 5 & 1 & YES & YES & NO(2) & $1.67$ & $(8,0)$ & NO & 5342\\
$(97,36)$ & 10 & $(12,5)$ & 5 & 1 & YES & YES & YES & $1.88$ & $(4,2)$ & -- & 5343\\
$(97,37)$ & 10 & $(13,3)$ & 6 & 1 & YES & YES & YES & $1.57$ & $(4,2)$ & -- & 5344\\
$(97,37)$ & 10 & $(13,3)$ & 6 & 1 & YES & YES & YES & $1.67$ & $(2,3)$ & NO & 5345\\
$(97,27)$ & 11 & $(14,3)$ & 6 & 1 & YES & YES & YES & $1.57$ & $(4,2)$ & -- & 5346\\
$(97,37)$ & 10 & $(14,3)$ & 6 & 1 & YES & YES & YES & $1.89$ & $(2,3)$ & -- & 5347\\
$(97,37)$ & 10 & $(16,3)$ & 7 & 1 & YES & YES & YES & $1.75$ & $(2,3)$ & -- & 5348\\
$(97,23)$ & 11 & $(17,7)$ & 6 & 1 & YES & YES & YES & $1.67$ & $(4,2)$ & -- & 5349\\
$(97,23)$ & 11 & $(18,7)$ & 6 & 1 & YES & YES & YES & $1.67$ & $(4,2)$ & -- & 5350\\
$(97,36)$ & 10 & $(18,5)$ & 6 & 1 & YES & YES & YES & $1.43$ & $(6,1)$ & NO & 5351\\
$(97,18)$ & 11 & $(21,5)$ & 8 & 1 & YES & YES & YES & $1.43$ & $(4,2)$ & -- & 5352\\
$(97,35)$ & 10 & $(21,8)$ & 6 & 1 & YES & YES & YES & $1.29$ & $(4,2)$ & NO & 5353\\
$(97,26)$ & 10 & $(25,7)$ & 7 & 1 & YES & YES & NO(2) & $1.60$ & $(6,1)$ & NO & 5354\\
$(97,35)$ & 10 & $(31,11)$ & 8 & 1 & YES & YES & YES & $1.43$ & $(4,2)$ & NO & 5355\\
$(97,35)$ & 10 & $(39,14)$ & 8 & 1 & YES & YES & YES & $1.43$ & $(4,2)$ & NO & 5356\\
$(97,26)$ & 10 & $(63,17)$ & 9 & 1 & YES & YES & NO(2) & $1.70$ & $(6,1)$ & NO & 5357\\
$(98,27)$ & 10 & $(4,1)$ & 3 & 2 & YES & YES & YES & $1.29$ & $(4,2)$ & NO & 5358\\
$(98,27)$ & 10 & $(4,1)$ & 3 & 2 & YES & YES & YES & $1.29$ & $(4,2)$ & -- & 5359\\
$(98,27)$ & 10 & $(7,3)$ & 4 & 7 & YES & YES & NO(2) & $1.56$ & $(8,0)$ & -- & 5360\\
$(98,29)$ & 10 & $(7,3)$ & 4 & 7 & YES & YES & NO(2) & $1.44$ & $(8,0)$ & -- & 5361\\
$(98,41)$ & 10 & $(7,3)$ & 4 & 7 & YES & YES & NO(2) & $1.50$ & $(4,2)$ & -- & 5362\\
$(98,41)$ & 10 & $(8,3)$ & 4 & 2 & YES & YES & YES & $1.89$ & $(4,2)$ & -- & 5363\\
$(98,27)$ & 10 & $(9,4)$ & 5 & 1 & YES & YES & YES & $1.57$ & $(2,3)$ & -- & 5364\\
$(98,29)$ & 10 & $(10,3)$ & 5 & 2 & YES & YES & YES & $1.43$ & $(2,3)$ & -- & 5365\\
$(98,41)$ & 10 & $(10,3)$ & 5 & 2 & YES & YES & YES & $1.43$ & $(4,2)$ & -- & 5366\\
$(98,43)$ & 10 & $(10,3)$ & 5 & 2 & YES & YES & YES & $1.43$ & $(4,2)$ & -- & 5367\\
$(98,27)$ & 10 & $(11,4)$ & 5 & 1 & YES & YES & YES & $1.78$ & $(2,3)$ & -- & 5368\\
$(98,41)$ & 10 & $(11,3)$ & 5 & 1 & YES & YES & YES & $1.89$ & $(2,3)$ & NO & 5369\\
$(98,41)$ & 10 & $(11,3)$ & 5 & 1 & YES & YES & YES & $1.89$ & $(2,3)$ & -- & 5370\\
$(98,27)$ & 10 & $(13,4)$ & 6 & 1 & YES & YES & YES & $1.62$ & $(4,2)$ & -- & 5371\\
$(98,43)$ & 10 & $(13,3)$ & 6 & 1 & YES & YES & NO(2) & $1.56$ & $(4,2)$ & -- & 5372\\
$(98,27)$ & 10 & $(14,3)$ & 6 & 14 & YES & YES & YES & $1.43$ & $(2,3)$ & -- & 5373\\
$(98,27)$ & 10 & $(15,4)$ & 6 & 1 & YES & YES & YES & $1.71$ & $(2,3)$ & -- & 5374\\
$(98,27)$ & 10 & $(17,4)$ & 7 & 1 & YES & YES & YES & $1.86$ & $(2,3)$ & -- & 5375\\
$(98,43)$ & 10 & $(19,8)$ & 6 & 1 & YES & YES & YES & $1.57$ & $(2,3)$ & NO & 5376\\
$(98,41)$ & 10 & $(22,9)$ & 7 & 2 & YES & YES & YES & $1.43$ & $(2,3)$ & NO & 5377\\
$(98,27)$ & 10 & $(39,11)$ & 9 & 1 & YES & YES & YES & $1.57$ & $(2,3)$ & NO & 5378\\
$(98,41)$ & 10 & $(69,29)$ & 9 & 1 & YES & YES & YES & $1.75$ & $(2,3)$ & 5981 & 5379\\
$(98,29)$ & 10 & $(71,21)$ & 9 & 1 & YES & YES & NO(2) & $1.50$ & $(2,3)$ & NO & 5380\\
$(98,41)$ & 10 & $(81,34)$ & 9 & 1 & YES & YES & YES & $2.00$ & $(2,3)$ & NO & 5381\\
$(98,43)$ & 10 & $(84,37)$ & 10 & 14 & YES & YES & NO(2) & $1.50$ & $(6,1)$ & NO & 5382\\
$(99,29)$ & 10 & $(2,1)$ & 1 & 1 & YES & YES & NO(2) & $1.33$ & $(4,2)$ & -- & 5383\\
$(99,29)$ & 10 & $(5,2)$ & 3 & 1 & YES & YES & NO(2) & $1.60$ & $(2,3)$ & -- & 5384\\
$(99,41)$ & 10 & $(5,2)$ & 3 & 1 & YES & YES & NO(2) & $1.56$ & $(4,2)$ & NO & 5385\\
$(99,41)$ & 10 & $(5,2)$ & 3 & 1 & YES & YES & NO(2) & $1.56$ & $(4,2)$ & -- & 5386\\
$(99,41)$ & 10 & $(7,2)$ & 4 & 1 & YES & YES & NO(2) & $1.38$ & $(6,1)$ & -- & 5387\\
$(99,41)$ & 10 & $(7,3)$ & 4 & 1 & YES & YES & NO(2) & $1.56$ & $(4,2)$ & -- & 5388\\
$(99,41)$ & 10 & $(8,3)$ & 4 & 1 & YES & YES & YES & $1.78$ & $(2,3)$ & -- & 5389\\
$(99,41)$ & 10 & $(9,4)$ & 5 & 9 & YES & YES & NO(2) & $1.56$ & $(4,2)$ & NO & 5390\\
$(99,29)$ & 10 & $(10,3)$ & 5 & 1 & YES & YES & NO(2) & $1.50$ & $(2,3)$ & NO & 5391\\
$(99,29)$ & 10 & $(11,4)$ & 5 & 11 & YES & YES & YES & $1.38$ & $(4,2)$ & -- & 5392\\
$(99,41)$ & 10 & $(11,3)$ & 5 & 11 & YES & YES & YES & $1.89$ & $(2,3)$ & NO & 5393\\
$(99,41)$ & 10 & $(11,3)$ & 5 & 11 & YES & YES & YES & $1.89$ & $(2,3)$ & -- & 5394\\
$(99,29)$ & 10 & $(12,5)$ & 5 & 3 & YES & YES & YES & $1.75$ & $(4,2)$ & NO & 5395\\
$(99,29)$ & 10 & $(12,5)$ & 5 & 3 & YES & YES & YES & $1.75$ & $(4,2)$ & -- & 5396\\
$(99,29)$ & 10 & $(13,4)$ & 6 & 1 & YES & YES & YES & $1.57$ & $(4,2)$ & -- & 5397\\
$(99,41)$ & 10 & $(13,5)$ & 5 & 1 & YES & YES & YES & $1.43$ & $(6,1)$ & -- & 5398\\
$(99,41)$ & 10 & $(14,5)$ & 6 & 1 & YES & YES & NO(2) & $1.62$ & $(6,1)$ & NO & 5399\\
$(99,29)$ & 10 & $(16,5)$ & 7 & 1 & YES & YES & NO(2) & $1.73$ & $(4,2)$ & NO & 5400\\
$(99,29)$ & 10 & $(17,5)$ & 6 & 1 & YES & YES & YES & $1.57$ & $(2,3)$ & -- & 5401\\
$(99,41)$ & 10 & $(22,9)$ & 7 & 11 & YES & YES & NO(2) & $1.73$ & $(4,2)$ & NO & 5402\\
$(99,29)$ & 10 & $(36,11)$ & 8 & 9 & YES & YES & YES & $1.67$ & $(2,3)$ & NO & 5403\\
$(99,29)$ & 10 & $(38,11)$ & 9 & 1 & YES & YES & NO(2) & $1.44$ & $(4,2)$ & 4786 & 5404\\
$(99,41)$ & 10 & $(39,16)$ & 8 & 3 & YES & YES & NO(2) & $1.56$ & $(4,2)$ & NO & 5405\\
$(99,29)$ & 10 & $(58,17)$ & 9 & 1 & YES & YES & NO(2) & $1.33$ & $(4,2)$ & NO & 5406\\
$(99,29)$ & 10 & $(64,19)$ & 9 & 1 & YES & YES & YES & $1.67$ & $(2,3)$ & NO & 5407\\
$(99,41)$ & 10 & $(70,29)$ & 9 & 1 & YES & YES & NO(2) & $1.60$ & $(2,3)$ & NO & 5408\\
$(99,29)$ & 10 & $(89,26)$ & 10 & 1 & YES & YES & NO(2) & $1.44$ & $(4,2)$ & NO & 5409\\
$(99,41)$ & 10 & $(99,41)$ & 10 & 99 & YES & YES & NO(2) & $1.60$ & $(2,3)$ & NO & 5410\\
$(100,27)$ & 10 & $(3,1)$ & 2 & 1 & YES & YES & NO(2) & $1.44$ & $(4,2)$ & NO & 5411\\
$(100,27)$ & 10 & $(3,1)$ & 2 & 1 & YES & YES & NO(2) & $1.44$ & $(4,2)$ & -- & 5412\\
$(100,37)$ & 10 & $(3,1)$ & 2 & 1 & YES & YES & NO(2) & $1.44$ & $(4,2)$ & NO & 5413\\
$(100,37)$ & 10 & $(3,1)$ & 2 & 1 & YES & YES & NO(2) & $1.44$ & $(4,2)$ & -- & 5414\\
$(100,37)$ & 10 & $(5,2)$ & 3 & 5 & YES & YES & NO(2) & $1.67$ & $(4,2)$ & -- & 5415\\
$(100,41)$ & 10 & $(5,2)$ & 3 & 5 & YES & YES & NO(2) & $1.60$ & $(6,1)$ & -- & 5416\\
$(100,27)$ & 10 & $(7,2)$ & 4 & 1 & YES & YES & NO(2) & $1.50$ & $(6,1)$ & -- & 5417\\
$(100,27)$ & 10 & $(7,3)$ & 4 & 1 & YES & YES & NO(2) & $1.14$ & $(8,0)$ & NO & 5418\\
$(100,27)$ & 10 & $(7,3)$ & 4 & 1 & YES & YES & NO(2) & $1.14$ & $(8,0)$ & -- & 5419\\
$(100,29)$ & 11 & $(7,3)$ & 4 & 1 & YES & YES & YES & $1.62$ & $(2,3)$ & -- & 5420\\
$(100,41)$ & 10 & $(7,3)$ & 4 & 1 & YES & YES & NO(2) & $1.67$ & $(4,2)$ & -- & 5421\\
$(100,27)$ & 10 & $(8,3)$ & 4 & 4 & YES & YES & NO(2) & $1.64$ & $(4,2)$ & -- & 5422\\
$(100,27)$ & 10 & $(8,3)$ & 4 & 4 & YES & YES & YES & $1.29$ & $(4,2)$ & NO & 5423\\
$(100,31)$ & 11 & $(8,3)$ & 4 & 4 & YES & YES & NO(2) & $1.44$ & $(8,0)$ & -- & 5424\\
$(100,37)$ & 10 & $(8,3)$ & 4 & 4 & YES & YES & NO(2) & $1.56$ & $(8,0)$ & -- & 5425\\
$(100,39)$ & 10 & $(8,3)$ & 4 & 4 & YES & YES & NO(2) & $1.44$ & $(8,0)$ & -- & 5426\\
$(100,29)$ & 11 & $(9,4)$ & 5 & 1 & YES & YES & NO(2) & $1.67$ & $(8,0)$ & NO & 5427\\
$(100,37)$ & 10 & $(10,3)$ & 5 & 10 & YES & YES & YES & $1.75$ & $(4,2)$ & NO & 5428\\
$(100,37)$ & 10 & $(10,3)$ & 5 & 10 & YES & YES & YES & $1.75$ & $(4,2)$ & -- & 5429\\
$(100,39)$ & 10 & $(10,3)$ & 5 & 10 & YES & YES & YES & $1.62$ & $(2,3)$ & -- & 5430\\
$(100,41)$ & 10 & $(10,3)$ & 5 & 10 & YES & YES & YES & $1.62$ & $(2,3)$ & -- & 5431\\
$(100,29)$ & 11 & $(11,4)$ & 5 & 1 & YES & YES & NO(2) & $1.67$ & $(8,0)$ & NO & 5432\\
$(100,39)$ & 10 & $(11,3)$ & 5 & 1 & YES & YES & YES & $1.62$ & $(2,3)$ & -- & 5433\\
$(100,41)$ & 10 & $(11,3)$ & 5 & 1 & YES & YES & YES & $1.62$ & $(2,3)$ & -- & 5434\\
$(100,41)$ & 10 & $(11,5)$ & 6 & 1 & YES & YES & NO(2) & $1.67$ & $(4,2)$ & NO & 5435\\
$(100,41)$ & 10 & $(13,3)$ & 6 & 1 & YES & YES & NO(2) & $1.56$ & $(4,2)$ & NO & 5436\\
$(100,27)$ & 10 & $(18,5)$ & 6 & 2 & YES & YES & NO(2) & $1.70$ & $(2,3)$ & NO & 5437\\
$(100,27)$ & 10 & $(19,5)$ & 7 & 1 & YES & YES & NO(2) & $1.60$ & $(6,1)$ & 6992 & 5438\\
$(100,27)$ & 10 & $(22,5)$ & 7 & 2 & YES & YES & YES & $1.70$ & $(2,3)$ & -- & 5439\\
$(100,41)$ & 10 & $(31,13)$ & 7 & 1 & YES & YES & YES & $1.62$ & $(2,3)$ & NO & 5440\\
$(100,41)$ & 10 & $(41,17)$ & 8 & 1 & YES & YES & NO(2) & $1.56$ & $(4,2)$ & NO & 5441\\
$(100,29)$ & 11 & $(86,25)$ & 10 & 2 & YES & YES & YES & $1.62$ & $(2,3)$ & NO & 5442\\
$(101,39)$ & 10 & $(2,1)$ & 1 & 1 & YES & YES & NO(2) & $1.40$ & $(2,3)$ & -- & 5443\\
$(101,30)$ & 10 & $(3,1)$ & 2 & 1 & YES & YES & YES & $1.50$ & $(2,3)$ & NO & 5444\\
$(101,30)$ & 10 & $(3,1)$ & 2 & 1 & YES & YES & YES & $1.62$ & $(2,3)$ & -- & 5445\\
$(101,39)$ & 10 & $(3,1)$ & 2 & 1 & YES & YES & NO(2) & $1.50$ & $(2,3)$ & -- & 5446\\
$(101,44)$ & 10 & $(4,1)$ & 3 & 1 & YES & YES & NO(2) & $1.50$ & $(6,1)$ & -- & 5447\\
$(101,30)$ & 10 & $(5,2)$ & 3 & 1 & YES & YES & NO(2) & $1.25$ & $(10,-1)$ & -- & 5448\\
$(101,44)$ & 10 & $(5,2)$ & 3 & 1 & YES & YES & NO(2) & $1.56$ & $(4,2)$ & -- & 5449\\
$(101,44)$ & 10 & $(5,2)$ & 3 & 1 & YES & YES & NO(2) & $1.56$ & $(4,2)$ & NO & 5450\\
$(101,22)$ & 11 & $(7,2)$ & 4 & 1 & YES & YES & NO(2) & $1.50$ & $(2,3)$ & NO & 5451\\
$(101,28)$ & 11 & $(7,3)$ & 4 & 1 & YES & YES & NO(2) & $1.80$ & $(2,3)$ & NO & 5452\\
$(101,30)$ & 10 & $(7,3)$ & 4 & 1 & YES & YES & YES & $1.43$ & $(2,3)$ & -- & 5453\\
$(101,37)$ & 10 & $(7,2)$ & 4 & 1 & YES & YES & NO(2) & $1.38$ & $(10,-1)$ & -- & 5454\\
$(101,37)$ & 10 & $(7,3)$ & 4 & 1 & YES & YES & NO(2) & $1.67$ & $(8,0)$ & -- & 5455\\
$(101,39)$ & 10 & $(7,2)$ & 4 & 1 & YES & YES & YES & $1.43$ & $(4,2)$ & NO & 5456\\
$(101,39)$ & 10 & $(7,2)$ & 4 & 1 & YES & YES & YES & $1.75$ & $(4,2)$ & -- & 5457\\
$(101,44)$ & 10 & $(7,2)$ & 4 & 1 & YES & YES & NO(2) & $1.44$ & $(4,2)$ & -- & 5458\\
$(101,39)$ & 10 & $(8,3)$ & 4 & 1 & YES & YES & YES & $1.90$ & $(2,3)$ & -- & 5459\\
$(101,30)$ & 10 & $(10,3)$ & 5 & 1 & YES & YES & YES & $1.62$ & $(2,3)$ & -- & 5460\\
$(101,30)$ & 10 & $(11,3)$ & 5 & 1 & YES & YES & NO(2) & $1.25$ & $(10,-1)$ & NO & 5461\\
$(101,30)$ & 10 & $(11,3)$ & 5 & 1 & YES & YES & YES & $1.50$ & $(2,3)$ & -- & 5462\\
$(101,44)$ & 10 & $(11,5)$ & 6 & 1 & YES & YES & NO(2) & $1.56$ & $(4,2)$ & NO & 5463\\
$(101,22)$ & 11 & $(12,5)$ & 5 & 1 & YES & YES & YES & $1.75$ & $(4,2)$ & -- & 5464\\
$(101,30)$ & 10 & $(13,4)$ & 6 & 1 & YES & YES & YES & $1.75$ & $(2,3)$ & -- & 5465\\
$(101,37)$ & 10 & $(13,3)$ & 6 & 1 & YES & YES & YES & $1.43$ & $(4,2)$ & -- & 5466\\
$(101,28)$ & 11 & $(17,5)$ & 6 & 1 & YES & YES & NO(2) & $1.80$ & $(2,3)$ & NO & 5467\\
$(101,30)$ & 10 & $(17,4)$ & 7 & 1 & YES & YES & YES & $1.43$ & $(2,3)$ & -- & 5468\\
$(101,30)$ & 10 & $(17,5)$ & 6 & 1 & YES & YES & YES & $1.50$ & $(2,3)$ & 5824 & 5469\\
$(101,39)$ & 10 & $(18,7)$ & 6 & 1 & YES & YES & NO(2) & $1.40$ & $(2,3)$ & 4458 & 5470\\
$(101,23)$ & 11 & $(21,4)$ & 8 & 1 & YES & YES & YES & $1.75$ & $(2,3)$ & -- & 5471\\
$(101,37)$ & 10 & $(21,8)$ & 6 & 1 & YES & YES & YES & $1.57$ & $(2,3)$ & NO & 5472\\
$(101,22)$ & 11 & $(26,5)$ & 9 & 1 & YES & YES & YES & $1.62$ & $(4,2)$ & -- & 5473\\
$(101,37)$ & 10 & $(36,13)$ & 8 & 1 & YES & YES & NO(2) & $1.56$ & $(8,0)$ & NO & 5474\\
$(101,44)$ & 10 & $(55,24)$ & 9 & 1 & YES & YES & NO(2) & $1.56$ & $(4,2)$ & NO & 5475\\
$(101,22)$ & 11 & $(80,17)$ & 10 & 1 & YES & YES & YES & $1.75$ & $(4,2)$ & NO & 5476\\
$(101,37)$ & 10 & $(93,34)$ & 10 & 1 & YES & YES & NO(2) & $1.44$ & $(8,0)$ & NO & 5477\\
$(102,43)$ & 11 & $(8,3)$ & 4 & 2 & YES & YES & NO(2) & $1.62$ & $(6,1)$ & -- & 5478\\
$(102,31)$ & 11 & $(10,3)$ & 5 & 2 & YES & YES & YES & $1.57$ & $(4,2)$ & -- & 5479\\
$(102,23)$ & 11 & $(11,4)$ & 5 & 1 & YES & YES & NO(2) & $1.50$ & $(4,2)$ & -- & 5480\\
$(102,31)$ & 11 & $(11,3)$ & 5 & 1 & YES & YES & YES & $1.71$ & $(4,2)$ & -- & 5481\\
$(102,43)$ & 11 & $(11,3)$ & 5 & 1 & YES & YES & NO(2) & $1.50$ & $(6,1)$ & NO & 5482\\
$(102,23)$ & 11 & $(43,10)$ & 9 & 1 & YES & YES & NO(2) & $1.56$ & $(4,2)$ & NO & 5483\\
$(103,39)$ & 10 & $(3,1)$ & 2 & 1 & YES & YES & NO(2) & $1.50$ & $(2,3)$ & -- & 5484\\
$(103,24)$ & 11 & $(4,1)$ & 3 & 1 & YES & YES & NO(2) & $1.50$ & $(6,1)$ & NO & 5485\\
$(103,39)$ & 10 & $(7,2)$ & 4 & 1 & YES & YES & YES & $1.43$ & $(2,3)$ & -- & 5486\\
$(103,37)$ & 10 & $(9,2)$ & 5 & 1 & YES & YES & NO(2) & $1.44$ & $(4,2)$ & NO & 5487\\
$(103,39)$ & 10 & $(9,2)$ & 5 & 1 & YES & YES & NO(2) & $1.50$ & $(6,1)$ & NO & 5488\\
$(103,39)$ & 10 & $(9,2)$ & 5 & 1 & YES & YES & NO(2) & $1.50$ & $(6,1)$ & -- & 5489\\
$(103,39)$ & 10 & $(9,4)$ & 5 & 1 & YES & YES & NO(2) & $1.50$ & $(6,1)$ & -- & 5490\\
$(103,37)$ & 10 & $(10,3)$ & 5 & 1 & YES & YES & YES & $1.43$ & $(4,2)$ & -- & 5491\\
$(103,39)$ & 10 & $(10,3)$ & 5 & 1 & YES & YES & YES & $1.62$ & $(4,2)$ & -- & 5492\\
$(103,39)$ & 10 & $(11,5)$ & 6 & 1 & YES & YES & NO(2) & $1.50$ & $(6,1)$ & NO & 5493\\
$(103,40)$ & 11 & $(12,5)$ & 5 & 1 & YES & YES & NO(2) & $1.80$ & $(6,1)$ & NO & 5494\\
$(103,30)$ & 11 & $(13,3)$ & 6 & 1 & YES & YES & YES & $1.29$ & $(4,2)$ & -- & 5495\\
$(103,38)$ & 11 & $(13,2)$ & 7 & 1 & YES & YES & NO(2) & $1.44$ & $(4,2)$ & -- & 5496\\
$(103,39)$ & 10 & $(18,7)$ & 6 & 1 & YES & YES & NO(2) & $1.56$ & $(8,0)$ & NO & 5497\\
$(103,37)$ & 10 & $(21,8)$ & 6 & 1 & YES & YES & YES & $1.43$ & $(4,2)$ & NO & 5498\\
$(103,39)$ & 10 & $(34,13)$ & 7 & 1 & YES & YES & YES & $1.43$ & $(2,3)$ & 7176 & 5499\\
$(103,39)$ & 10 & $(50,19)$ & 8 & 1 & YES & YES & NO(2) & $1.60$ & $(2,3)$ & NO & 5500\\
$(103,39)$ & 10 & $(55,21)$ & 8 & 1 & YES & YES & YES & $1.75$ & $(4,2)$ & NO & 5501\\
$(103,39)$ & 10 & $(95,36)$ & 10 & 1 & YES & YES & NO(2) & $1.60$ & $(2,3)$ & NO & 5502\\
$(104,43)$ & 10 & $(3,1)$ & 2 & 1 & YES & YES & NO(2) & $1.67$ & $(4,2)$ & NO & 5503\\
$(104,43)$ & 10 & $(4,1)$ & 3 & 4 & YES & YES & NO(2) & $1.60$ & $(2,3)$ & NO & 5504\\
$(104,43)$ & 10 & $(4,1)$ & 3 & 4 & YES & YES & NO(2) & $1.60$ & $(2,3)$ & -- & 5505\\
$(104,43)$ & 10 & $(4,1)$ & 3 & 4 & YES & YES & NO(2) & $1.60$ & $(2,3)$ & NO & 5506\\
$(104,43)$ & 10 & $(5,2)$ & 3 & 1 & YES & YES & NO(2) & $1.70$ & $(2,3)$ & -- & 5507\\
$(104,43)$ & 10 & $(5,2)$ & 3 & 1 & YES & YES & NO(2) & $1.60$ & $(2,3)$ & NO & 5508\\
$(104,43)$ & 10 & $(7,2)$ & 4 & 1 & YES & YES & NO(2) & $1.73$ & $(4,2)$ & -- & 5509\\
$(104,43)$ & 10 & $(7,3)$ & 4 & 1 & YES & YES & NO(2) & $1.38$ & $(4,2)$ & -- & 5510\\
$(104,29)$ & 10 & $(8,3)$ & 4 & 8 & YES & YES & NO(2) & $1.25$ & $(4,2)$ & -- & 5511\\
$(104,43)$ & 10 & $(9,2)$ & 5 & 1 & YES & YES & NO(2) & $1.73$ & $(4,2)$ & -- & 5512\\
$(104,43)$ & 10 & $(9,4)$ & 5 & 1 & YES & YES & NO(2) & $1.80$ & $(2,3)$ & NO & 5513\\
$(104,43)$ & 10 & $(10,3)$ & 5 & 2 & YES & YES & YES & $1.57$ & $(2,3)$ & -- & 5514\\
$(104,29)$ & 10 & $(11,4)$ & 5 & 1 & YES & YES & YES & $1.50$ & $(4,2)$ & -- & 5515\\
$(104,29)$ & 10 & $(12,5)$ & 5 & 4 & YES & YES & YES & $1.75$ & $(4,2)$ & NO & 5516\\
$(104,29)$ & 10 & $(12,5)$ & 5 & 4 & YES & YES & YES & $1.75$ & $(4,2)$ & -- & 5517\\
$(104,29)$ & 10 & $(13,4)$ & 6 & 13 & YES & YES & YES & $1.43$ & $(4,2)$ & -- & 5518\\
$(104,29)$ & 10 & $(13,5)$ & 5 & 13 & YES & YES & YES & $1.75$ & $(2,3)$ & -- & 5519\\
$(104,31)$ & 11 & $(13,3)$ & 6 & 13 & YES & YES & YES & $1.43$ & $(2,3)$ & NO & 5520\\
$(104,43)$ & 10 & $(13,3)$ & 6 & 13 & YES & YES & YES & $1.86$ & $(2,3)$ & -- & 5521\\
$(104,43)$ & 10 & $(14,3)$ & 6 & 2 & YES & YES & YES & $1.71$ & $(2,3)$ & -- & 5522\\
$(104,43)$ & 10 & $(16,7)$ & 6 & 8 & YES & YES & YES & $1.62$ & $(2,3)$ & NO & 5523\\
$(104,43)$ & 10 & $(19,8)$ & 6 & 1 & YES & YES & NO(2) & $1.70$ & $(2,3)$ & NO & 5524\\
$(104,43)$ & 10 & $(22,9)$ & 7 & 2 & YES & YES & NO(2) & $1.73$ & $(4,2)$ & 6333 & 5525\\
$(104,29)$ & 10 & $(25,7)$ & 7 & 1 & YES & YES & NO(2) & $1.50$ & $(6,1)$ & NO & 5526\\
$(104,43)$ & 10 & $(27,11)$ & 8 & 1 & YES & YES & YES & $1.75$ & $(2,3)$ & NO & 5527\\
$(104,31)$ & 11 & $(31,9)$ & 8 & 1 & YES & YES & YES & $1.43$ & $(2,3)$ & NO & 5528\\
$(104,29)$ & 10 & $(57,16)$ & 9 & 1 & YES & YES & NO(2) & $1.56$ & $(4,2)$ & NO & 5529\\
$(104,29)$ & 10 & $(62,17)$ & 10 & 2 & YES & YES & YES & $1.57$ & $(6,1)$ & 8392 & 5530\\
$(104,43)$ & 10 & $(63,26)$ & 9 & 1 & YES & YES & NO(2) & $1.80$ & $(2,3)$ & 7461 & 5531\\
$(105,41)$ & 10 & $(4,1)$ & 3 & 1 & YES & YES & NO(2) & $1.73$ & $(4,2)$ & NO & 5532\\
$(105,41)$ & 10 & $(4,1)$ & 3 & 1 & YES & YES & NO(2) & $1.73$ & $(4,2)$ & -- & 5533\\
$(105,41)$ & 10 & $(5,2)$ & 3 & 5 & YES & YES & NO(2) & $1.60$ & $(6,1)$ & -- & 5534\\
$(105,31)$ & 10 & $(7,2)$ & 4 & 7 & YES & YES & NO(2) & $1.55$ & $(4,2)$ & NO & 5535\\
$(105,31)$ & 10 & $(7,2)$ & 4 & 7 & YES & YES & NO(2) & $1.55$ & $(4,2)$ & -- & 5536\\
$(105,31)$ & 10 & $(7,3)$ & 4 & 7 & YES & YES & NO(2) & $1.56$ & $(4,2)$ & -- & 5537\\
$(105,38)$ & 11 & $(7,2)$ & 4 & 7 & YES & YES & YES & $1.50$ & $(2,3)$ & -- & 5538\\
$(105,43)$ & 11 & $(7,2)$ & 4 & 7 & YES & YES & NO(2) & $1.56$ & $(4,2)$ & -- & 5539\\
$(105,29)$ & 10 & $(8,3)$ & 4 & 1 & YES & YES & NO(2) & $1.56$ & $(8,0)$ & -- & 5540\\
$(105,41)$ & 10 & $(8,3)$ & 4 & 1 & YES & YES & NO(2) & $1.56$ & $(8,0)$ & -- & 5541\\
$(105,44)$ & 10 & $(8,3)$ & 4 & 1 & YES & YES & YES & $1.50$ & $(2,3)$ & -- & 5542\\
$(105,41)$ & 10 & $(9,2)$ & 5 & 3 & YES & YES & NO(2) & $1.56$ & $(8,0)$ & -- & 5543\\
$(105,41)$ & 10 & $(9,2)$ & 5 & 3 & YES & YES & NO(2) & $1.56$ & $(8,0)$ & NO & 5544\\
$(105,29)$ & 10 & $(10,3)$ & 5 & 5 & YES & YES & YES & $1.56$ & $(2,3)$ & -- & 5545\\
$(105,41)$ & 10 & $(10,3)$ & 5 & 5 & YES & YES & YES & $1.62$ & $(4,2)$ & -- & 5546\\
$(105,44)$ & 10 & $(10,3)$ & 5 & 5 & YES & YES & YES & $1.75$ & $(4,2)$ & -- & 5547\\
$(105,29)$ & 10 & $(11,3)$ & 5 & 1 & YES & YES & YES & $1.62$ & $(2,3)$ & -- & 5548\\
$(105,29)$ & 10 & $(11,4)$ & 5 & 1 & YES & YES & YES & $1.78$ & $(2,3)$ & -- & 5549\\
$(105,38)$ & 11 & $(11,3)$ & 5 & 1 & YES & YES & NO(2) & $1.62$ & $(6,1)$ & -- & 5550\\
$(105,44)$ & 10 & $(11,3)$ & 5 & 1 & YES & YES & YES & $1.78$ & $(2,3)$ & -- & 5551\\
$(105,44)$ & 10 & $(11,3)$ & 5 & 1 & YES & YES & YES & $1.80$ & $(2,3)$ & NO & 5552\\
$(105,44)$ & 10 & $(11,4)$ & 5 & 1 & YES & YES & NO(2) & $1.56$ & $(4,2)$ & NO & 5553\\
$(105,29)$ & 10 & $(12,5)$ & 5 & 3 & YES & YES & YES & $1.56$ & $(2,3)$ & -- & 5554\\
$(105,29)$ & 10 & $(13,5)$ & 5 & 1 & YES & YES & YES & $1.57$ & $(2,3)$ & -- & 5555\\
$(105,31)$ & 10 & $(13,4)$ & 6 & 1 & YES & YES & YES & $1.78$ & $(2,3)$ & -- & 5556\\
$(105,31)$ & 10 & $(13,5)$ & 5 & 1 & YES & YES & NO(2) & $1.44$ & $(4,2)$ & NO & 5557\\
$(105,31)$ & 10 & $(13,5)$ & 5 & 1 & YES & YES & YES & $1.67$ & $(2,3)$ & -- & 5558\\
$(105,43)$ & 11 & $(19,8)$ & 6 & 1 & YES & YES & NO(2) & $1.56$ & $(4,2)$ & NO & 5559\\
$(105,38)$ & 11 & $(21,8)$ & 6 & 21 & YES & YES & NO(2) & $1.62$ & $(6,1)$ & NO & 5560\\
$(105,32)$ & 11 & $(27,8)$ & 7 & 3 & YES & YES & YES & $1.57$ & $(2,3)$ & NO & 5561\\
$(105,41)$ & 10 & $(28,11)$ & 8 & 7 & YES & YES & YES & $1.43$ & $(4,2)$ & NO & 5562\\
$(105,31)$ & 10 & $(29,9)$ & 8 & 1 & YES & YES & NO(2) & $1.44$ & $(4,2)$ & NO & 5563\\
$(105,47)$ & 11 & $(31,14)$ & 8 & 1 & YES & YES & NO(2) & $1.50$ & $(6,1)$ & 7259 & 5564\\
$(105,31)$ & 10 & $(47,14)$ & 9 & 1 & YES & YES & NO(2) & $1.44$ & $(8,0)$ & 5138 & 5565\\
$(105,44)$ & 10 & $(55,23)$ & 9 & 5 & YES & YES & NO(2) & $1.44$ & $(8,0)$ & 7456 & 5566\\
$(105,29)$ & 10 & $(68,19)$ & 9 & 1 & YES & YES & YES & $1.56$ & $(2,3)$ & NO & 5567\\
$(105,32)$ & 11 & $(79,24)$ & 10 & 1 & YES & YES & NO(2) & $1.50$ & $(6,1)$ & NO & 5568\\
$(105,41)$ & 10 & $(100,39)$ & 10 & 5 & YES & YES & NO(2) & $1.56$ & $(8,0)$ & NO & 5569\\
$(106,41)$ & 10 & $(4,1)$ & 3 & 2 & YES & YES & YES & $1.43$ & $(4,2)$ & 6432 & 5570\\
$(106,41)$ & 10 & $(4,1)$ & 3 & 2 & YES & YES & YES & $1.43$ & $(4,2)$ & -- & 5571\\
$(106,41)$ & 10 & $(5,2)$ & 3 & 1 & YES & YES & NO(2) & $1.56$ & $(4,2)$ & -- & 5572\\
$(106,41)$ & 10 & $(5,2)$ & 3 & 1 & YES & YES & NO(2) & $1.67$ & $(4,2)$ & NO & 5573\\
$(106,45)$ & 11 & $(5,2)$ & 3 & 1 & YES & YES & NO(2) & $1.70$ & $(2,3)$ & -- & 5574\\
$(106,31)$ & 10 & $(7,3)$ & 4 & 1 & YES & YES & YES & $1.43$ & $(2,3)$ & -- & 5575\\
$(106,41)$ & 10 & $(7,3)$ & 4 & 1 & YES & YES & NO(2) & $1.56$ & $(4,2)$ & NO & 5576\\
$(106,41)$ & 10 & $(7,3)$ & 4 & 1 & YES & YES & YES & $1.67$ & $(2,3)$ & -- & 5577\\
$(106,31)$ & 10 & $(8,3)$ & 4 & 2 & YES & YES & YES & $1.43$ & $(2,3)$ & -- & 5578\\
$(106,41)$ & 10 & $(8,3)$ & 4 & 2 & YES & YES & YES & $1.67$ & $(2,3)$ & -- & 5579\\
$(106,31)$ & 10 & $(9,4)$ & 5 & 1 & YES & YES & YES & $1.29$ & $(4,2)$ & -- & 5580\\
$(106,41)$ & 10 & $(10,3)$ & 5 & 2 & YES & YES & YES & $1.57$ & $(4,2)$ & -- & 5581\\
$(106,41)$ & 10 & $(10,3)$ & 5 & 2 & YES & YES & YES & $1.57$ & $(4,2)$ & NO & 5582\\
$(106,31)$ & 10 & $(11,4)$ & 5 & 1 & YES & YES & YES & $1.57$ & $(2,3)$ & -- & 5583\\
$(106,41)$ & 10 & $(11,3)$ & 5 & 1 & YES & YES & YES & $1.43$ & $(4,2)$ & -- & 5584\\
$(106,41)$ & 10 & $(11,3)$ & 5 & 1 & YES & YES & YES & $1.78$ & $(2,3)$ & NO & 5585\\
$(106,41)$ & 10 & $(11,3)$ & 5 & 1 & YES & YES & YES & $1.89$ & $(2,3)$ & NO & 5586\\
$(106,41)$ & 10 & $(11,4)$ & 5 & 1 & YES & YES & NO(2) & $1.60$ & $(6,1)$ & NO & 5587\\
$(106,31)$ & 10 & $(12,5)$ & 5 & 2 & YES & YES & YES & $1.67$ & $(2,3)$ & -- & 5588\\
$(106,31)$ & 10 & $(13,5)$ & 5 & 1 & YES & YES & YES & $1.43$ & $(4,2)$ & NO & 5589\\
$(106,31)$ & 10 & $(13,5)$ & 5 & 1 & YES & YES & YES & $1.67$ & $(2,3)$ & -- & 5590\\
$(106,41)$ & 10 & $(13,5)$ & 5 & 1 & YES & YES & YES & $1.75$ & $(4,2)$ & -- & 5591\\
$(106,31)$ & 10 & $(17,4)$ & 7 & 1 & YES & YES & YES & $1.71$ & $(2,3)$ & -- & 5592\\
$(106,39)$ & 11 & $(21,8)$ & 6 & 1 & YES & YES & NO(2) & $1.62$ & $(6,1)$ & NO & 5593\\
$(106,41)$ & 10 & $(23,9)$ & 7 & 1 & YES & YES & NO(2) & $1.70$ & $(6,1)$ & NO & 5594\\
$(106,33)$ & 11 & $(36,11)$ & 8 & 2 & YES & YES & NO(2) & $1.50$ & $(10,-1)$ & NO & 5595\\
$(106,41)$ & 10 & $(44,17)$ & 8 & 2 & YES & YES & NO(2) & $1.33$ & $(4,2)$ & 5973 & 5596\\
$(106,31)$ & 10 & $(47,14)$ & 9 & 1 & YES & YES & YES & $1.43$ & $(4,2)$ & NO & 5597\\
$(106,31)$ & 10 & $(61,18)$ & 9 & 1 & YES & YES & YES & $1.29$ & $(4,2)$ & NO & 5598\\
$(106,31)$ & 10 & $(65,19)$ & 9 & 1 & YES & YES & YES & $1.62$ & $(2,3)$ & NO & 5599\\
$(106,41)$ & 10 & $(101,39)$ & 10 & 1 & YES & YES & YES & $1.78$ & $(2,3)$ & NO & 5600\\
$(107,47)$ & 10 & $(3,1)$ & 2 & 1 & YES & YES & NO(2) & $1.38$ & $(6,1)$ & -- & 5601\\
$(107,47)$ & 10 & $(5,2)$ & 3 & 1 & YES & YES & NO(2) & $1.70$ & $(2,3)$ & -- & 5602\\
$(107,47)$ & 10 & $(5,2)$ & 3 & 1 & YES & YES & NO(2) & $1.56$ & $(4,2)$ & NO & 5603\\
$(107,44)$ & 12 & $(7,3)$ & 4 & 1 & YES & YES & NO(2) & $1.67$ & $(8,0)$ & NO & 5604\\
$(107,47)$ & 10 & $(7,3)$ & 4 & 1 & YES & YES & YES & $1.71$ & $(2,3)$ & -- & 5605\\
$(107,41)$ & 10 & $(10,3)$ & 5 & 1 & YES & YES & YES & $1.75$ & $(2,3)$ & -- & 5606\\
$(107,41)$ & 10 & $(10,3)$ & 5 & 1 & YES & YES & YES & $1.57$ & $(2,3)$ & NO & 5607\\
$(107,41)$ & 10 & $(11,3)$ & 5 & 1 & YES & YES & YES & $1.29$ & $(4,2)$ & -- & 5608\\
$(107,44)$ & 12 & $(11,5)$ & 6 & 1 & YES & YES & NO(2) & $1.67$ & $(8,0)$ & NO & 5609\\
$(107,47)$ & 10 & $(11,5)$ & 6 & 1 & YES & YES & NO(2) & $1.56$ & $(4,2)$ & NO & 5610\\
$(107,41)$ & 10 & $(12,5)$ & 5 & 1 & YES & YES & YES & $1.62$ & $(4,2)$ & -- & 5611\\
$(107,44)$ & 12 & $(13,5)$ & 5 & 1 & YES & YES & YES & $1.71$ & $(4,2)$ & 5795 & 5612\\
$(107,47)$ & 10 & $(13,3)$ & 6 & 1 & YES & YES & YES & $1.57$ & $(4,2)$ & NO & 5613\\
$(107,47)$ & 10 & $(13,3)$ & 6 & 1 & YES & YES & YES & $1.57$ & $(4,2)$ & -- & 5614\\
$(107,47)$ & 10 & $(13,3)$ & 6 & 1 & YES & YES & YES & $1.75$ & $(4,2)$ & NO & 5615\\
$(107,25)$ & 11 & $(15,4)$ & 6 & 1 & YES & YES & YES & $1.75$ & $(2,3)$ & -- & 5616\\
$(107,47)$ & 10 & $(17,7)$ & 6 & 1 & YES & YES & YES & $1.57$ & $(4,2)$ & NO & 5617\\
$(107,47)$ & 10 & $(19,8)$ & 6 & 1 & YES & YES & YES & $1.43$ & $(2,3)$ & NO & 5618\\
$(107,47)$ & 10 & $(34,15)$ & 8 & 1 & YES & YES & NO(2) & $1.80$ & $(2,3)$ & NO & 5619\\
$(107,25)$ & 11 & $(37,8)$ & 8 & 1 & YES & YES & YES & $1.78$ & $(2,3)$ & NO & 5620\\
$(107,41)$ & 10 & $(66,25)$ & 9 & 1 & YES & YES & YES & $1.62$ & $(4,2)$ & NO & 5621\\
$(107,41)$ & 10 & $(76,29)$ & 9 & 1 & YES & YES & YES & $1.43$ & $(4,2)$ & 6280 & 5622\\
$(107,41)$ & 10 & $(89,34)$ & 9 & 1 & YES & YES & YES & $1.62$ & $(2,3)$ & NO & 5623\\
$(107,30)$ & 11 & $(104,29)$ & 10 & 1 & YES & YES & YES & $1.43$ & $(6,1)$ & NO & 5624\\
$(108,29)$ & 10 & $(3,1)$ & 2 & 3 & YES & YES & NO(2) & $1.60$ & $(6,1)$ & NO & 5625\\
$(108,29)$ & 10 & $(3,1)$ & 2 & 3 & YES & YES & NO(2) & $1.60$ & $(6,1)$ & -- & 5626\\
$(108,29)$ & 10 & $(4,1)$ & 3 & 4 & YES & YES & NO(2) & $1.50$ & $(6,1)$ & -- & 5627\\
$(108,41)$ & 10 & $(5,2)$ & 3 & 1 & YES & YES & NO(2) & $1.56$ & $(8,0)$ & -- & 5628\\
$(108,29)$ & 10 & $(8,3)$ & 4 & 4 & YES & YES & NO(2) & $1.44$ & $(8,0)$ & -- & 5629\\
$(108,41)$ & 10 & $(9,4)$ & 5 & 9 & YES & YES & NO(2) & $1.67$ & $(8,0)$ & NO & 5630\\
$(108,41)$ & 10 & $(10,3)$ & 5 & 2 & YES & YES & YES & $1.78$ & $(2,3)$ & -- & 5631\\
$(108,41)$ & 10 & $(11,3)$ & 5 & 1 & YES & YES & YES & $1.89$ & $(2,3)$ & -- & 5632\\
$(108,41)$ & 10 & $(13,3)$ & 6 & 1 & YES & YES & YES & $1.62$ & $(4,2)$ & -- & 5633\\
$(108,29)$ & 10 & $(17,5)$ & 6 & 1 & YES & YES & NO(2) & $1.56$ & $(8,0)$ & NO & 5634\\
$(108,23)$ & 11 & $(18,7)$ & 6 & 18 & YES & YES & YES & $1.75$ & $(4,2)$ & -- & 5635\\
$(108,41)$ & 10 & $(23,9)$ & 7 & 1 & YES & YES & NO(2) & $1.56$ & $(4,2)$ & NO & 5636\\
$(108,29)$ & 10 & $(24,7)$ & 7 & 12 & YES & YES & YES & $1.57$ & $(4,2)$ & NO & 5637\\
$(108,29)$ & 10 & $(25,7)$ & 7 & 1 & YES & YES & NO(2) & $1.44$ & $(8,0)$ & NO & 5638\\
$(108,29)$ & 10 & $(32,9)$ & 8 & 4 & YES & YES & YES & $1.43$ & $(4,2)$ & NO & 5639\\
$(108,41)$ & 10 & $(45,17)$ & 9 & 9 & YES & YES & NO(2) & $1.50$ & $(10,-1)$ & NO & 5640\\
$(108,29)$ & 10 & $(67,18)$ & 9 & 1 & YES & YES & NO(2) & $1.60$ & $(6,1)$ & NO & 5641\\
$(108,41)$ & 10 & $(71,27)$ & 9 & 1 & YES & YES & NO(2) & $1.56$ & $(4,2)$ & 7687 & 5642\\
$(108,29)$ & 10 & $(108,29)$ & 10 & 108 & YES & YES & NO(2) & $1.60$ & $(6,1)$ & NO & 5643\\
$(109,45)$ & 10 & $(2,1)$ & 1 & 1 & YES & YES & NO(2) & $1.50$ & $(2,3)$ & -- & 5644\\
$(109,30)$ & 10 & $(3,1)$ & 2 & 1 & YES & YES & NO(2) & $1.60$ & $(2,3)$ & -- & 5645\\
$(109,30)$ & 10 & $(3,1)$ & 2 & 1 & YES & YES & NO(2) & $1.70$ & $(2,3)$ & NO & 5646\\
$(109,45)$ & 10 & $(3,1)$ & 2 & 1 & YES & YES & NO(2) & $1.50$ & $(2,3)$ & -- & 5647\\
$(109,46)$ & 10 & $(3,1)$ & 2 & 1 & YES & YES & YES & $1.50$ & $(2,3)$ & -- & 5648\\
$(109,30)$ & 10 & $(4,1)$ & 3 & 1 & YES & YES & NO(2) & $1.70$ & $(2,3)$ & NO & 5649\\
$(109,30)$ & 10 & $(4,1)$ & 3 & 1 & YES & YES & NO(2) & $1.70$ & $(2,3)$ & -- & 5650\\
$(109,46)$ & 10 & $(4,1)$ & 3 & 1 & YES & YES & NO(2) & $1.44$ & $(8,0)$ & NO & 5651\\
$(109,46)$ & 10 & $(4,1)$ & 3 & 1 & YES & YES & NO(2) & $1.44$ & $(8,0)$ & -- & 5652\\
$(109,40)$ & 10 & $(5,2)$ & 3 & 1 & YES & YES & NO(2) & $1.70$ & $(6,1)$ & -- & 5653\\
$(109,46)$ & 10 & $(5,2)$ & 3 & 1 & YES & YES & NO(2) & $1.70$ & $(2,3)$ & -- & 5654\\
$(109,32)$ & 12 & $(6,1)$ & 5 & 1 & YES & YES & NO(2) & $1.60$ & $(6,1)$ & -- & 5655\\
$(109,32)$ & 12 & $(6,1)$ & 5 & 1 & YES & YES & NO(2) & $1.70$ & $(6,1)$ & NO & 5656\\
$(109,30)$ & 10 & $(7,3)$ & 4 & 1 & YES & YES & YES & $1.43$ & $(2,3)$ & -- & 5657\\
$(109,40)$ & 10 & $(7,2)$ & 4 & 1 & YES & YES & NO(2) & $1.67$ & $(4,2)$ & NO & 5658\\
$(109,40)$ & 10 & $(7,2)$ & 4 & 1 & YES & YES & NO(2) & $1.67$ & $(4,2)$ & -- & 5659\\
$(109,40)$ & 10 & $(7,3)$ & 4 & 1 & YES & YES & NO(2) & $1.50$ & $(4,2)$ & -- & 5660\\
$(109,45)$ & 10 & $(7,3)$ & 4 & 1 & YES & YES & NO(2) & $1.50$ & $(2,3)$ & NO & 5661\\
$(109,45)$ & 10 & $(7,3)$ & 4 & 1 & YES & YES & YES & $1.57$ & $(4,2)$ & -- & 5662\\
$(109,30)$ & 10 & $(8,3)$ & 4 & 1 & YES & YES & YES & $1.57$ & $(2,3)$ & -- & 5663\\
$(109,40)$ & 10 & $(8,3)$ & 4 & 1 & YES & YES & YES & $1.57$ & $(4,2)$ & -- & 5664\\
$(109,45)$ & 10 & $(8,3)$ & 4 & 1 & YES & YES & YES & $1.43$ & $(4,2)$ & -- & 5665\\
$(109,46)$ & 10 & $(8,3)$ & 4 & 1 & YES & YES & YES & $1.62$ & $(2,3)$ & -- & 5666\\
$(109,40)$ & 10 & $(9,2)$ & 5 & 1 & YES & YES & NO(2) & $1.56$ & $(2,3)$ & NO & 5667\\
$(109,30)$ & 10 & $(10,3)$ & 5 & 1 & YES & YES & NO(2) & $1.25$ & $(4,2)$ & -- & 5668\\
$(109,30)$ & 10 & $(10,3)$ & 5 & 1 & YES & YES & NO(2) & $1.33$ & $(8,0)$ & NO & 5669\\
$(109,33)$ & 11 & $(10,3)$ & 5 & 1 & YES & YES & YES & $1.29$ & $(4,2)$ & -- & 5670\\
$(109,40)$ & 10 & $(10,3)$ & 5 & 1 & YES & YES & YES & $1.75$ & $(4,2)$ & -- & 5671\\
$(109,45)$ & 10 & $(10,3)$ & 5 & 1 & YES & YES & YES & $1.62$ & $(4,2)$ & -- & 5672\\
$(109,46)$ & 10 & $(10,3)$ & 5 & 1 & YES & YES & YES & $1.71$ & $(4,2)$ & -- & 5673\\
$(109,48)$ & 11 & $(10,3)$ & 5 & 1 & YES & YES & NO(2) & $1.38$ & $(6,1)$ & -- & 5674\\
$(109,30)$ & 10 & $(11,4)$ & 5 & 1 & YES & YES & YES & $1.56$ & $(2,3)$ & -- & 5675\\
$(109,40)$ & 10 & $(12,5)$ & 5 & 1 & YES & YES & YES & $1.78$ & $(4,2)$ & -- & 5676\\
$(109,45)$ & 10 & $(12,5)$ & 5 & 1 & YES & YES & NO(2) & $1.50$ & $(2,3)$ & 4605 & 5677\\
$(109,46)$ & 10 & $(12,5)$ & 5 & 1 & YES & YES & NO(2) & $1.50$ & $(2,3)$ & NO & 5678\\
$(109,46)$ & 10 & $(12,5)$ & 5 & 1 & YES & YES & YES & $1.89$ & $(4,2)$ & -- & 5679\\
$(109,30)$ & 10 & $(13,4)$ & 6 & 1 & YES & YES & YES & $1.75$ & $(2,3)$ & -- & 5680\\
$(109,40)$ & 10 & $(13,3)$ & 6 & 1 & YES & YES & YES & $1.62$ & $(4,2)$ & -- & 5681\\
$(109,40)$ & 10 & $(13,4)$ & 6 & 1 & YES & YES & YES & $1.78$ & $(4,2)$ & -- & 5682\\
$(109,40)$ & 10 & $(13,5)$ & 5 & 1 & YES & YES & NO(2) & $1.70$ & $(2,3)$ & NO & 5683\\
$(109,40)$ & 10 & $(13,5)$ & 5 & 1 & YES & YES & YES & $1.78$ & $(4,2)$ & -- & 5684\\
$(109,30)$ & 10 & $(15,4)$ & 6 & 1 & YES & YES & NO(2) & $1.60$ & $(2,3)$ & NO & 5685\\
$(109,45)$ & 10 & $(16,7)$ & 6 & 1 & YES & YES & YES & $1.43$ & $(4,2)$ & NO & 5686\\
$(109,30)$ & 10 & $(17,4)$ & 7 & 1 & YES & YES & YES & $1.57$ & $(2,3)$ & -- & 5687\\
$(109,46)$ & 10 & $(17,7)$ & 6 & 1 & YES & YES & NO(2) & $1.38$ & $(4,2)$ & NO & 5688\\
$(109,30)$ & 10 & $(18,5)$ & 6 & 1 & YES & YES & NO(2) & $1.25$ & $(10,-1)$ & 6085 & 5689\\
$(109,33)$ & 11 & $(18,5)$ & 6 & 1 & YES & YES & YES & $1.78$ & $(4,2)$ & -- & 5690\\
$(109,30)$ & 10 & $(22,5)$ & 7 & 1 & YES & YES & YES & $1.57$ & $(2,3)$ & -- & 5691\\
$(109,45)$ & 10 & $(26,11)$ & 7 & 1 & YES & YES & YES & $1.75$ & $(4,2)$ & NO & 5692\\
$(109,45)$ & 10 & $(29,12)$ & 7 & 1 & YES & YES & NO(2) & $1.50$ & $(2,3)$ & NO & 5693\\
$(109,46)$ & 10 & $(29,12)$ & 7 & 1 & YES & YES & YES & $1.75$ & $(2,3)$ & NO & 5694\\
$(109,48)$ & 11 & $(57,25)$ & 9 & 1 & YES & YES & NO(2) & $1.50$ & $(6,1)$ & NO & 5695\\
$(109,32)$ & 12 & $(58,17)$ & 9 & 1 & YES & YES & NO(2) & $1.60$ & $(6,1)$ & NO & 5696\\
$(109,30)$ & 10 & $(61,17)$ & 9 & 1 & YES & YES & YES & $1.67$ & $(2,3)$ & NO & 5697\\
$(109,46)$ & 10 & $(64,27)$ & 9 & 1 & YES & YES & NO(2) & $1.50$ & $(2,3)$ & NO & 5698\\
$(109,30)$ & 10 & $(65,18)$ & 9 & 1 & YES & YES & NO(2) & $1.38$ & $(4,2)$ & NO & 5699\\
$(109,30)$ & 10 & $(69,19)$ & 9 & 1 & YES & YES & NO(2) & $1.80$ & $(2,3)$ & NO & 5700\\
$(109,46)$ & 10 & $(71,30)$ & 9 & 1 & YES & YES & NO(2) & $1.70$ & $(6,1)$ & NO & 5701\\
$(109,33)$ & 11 & $(73,22)$ & 12 & 1 & YES & YES & NO(2) & $1.62$ & $(6,1)$ & NO & 5702\\
$(109,30)$ & 10 & $(83,23)$ & 10 & 1 & YES & YES & YES & $1.56$ & $(2,3)$ & NO & 5703\\
$(109,33)$ & 11 & $(85,26)$ & 10 & 1 & YES & YES & YES & $1.89$ & $(4,2)$ & NO & 5704\\
$(109,48)$ & 11 & $(93,41)$ & 10 & 1 & YES & YES & NO(2) & $1.70$ & $(10,-1)$ & 7866 & 5705\\
$(109,30)$ & 10 & $(109,30)$ & 10 & 109 & YES & YES & NO(2) & $1.60$ & $(2,3)$ & NO & 5706\\
$(109,40)$ & 10 & $(109,40)$ & 10 & 109 & YES & YES & NO(2) & $1.60$ & $(2,3)$ & NO & 5707\\
$(109,46)$ & 10 & $(109,46)$ & 10 & 109 & YES & YES & NO(2) & $1.60$ & $(2,3)$ & NO & 5708\\
$(110,39)$ & 11 & $(13,5)$ & 5 & 1 & YES & YES & NO(2) & $1.56$ & $(8,0)$ & NO & 5709\\
$(111,31)$ & 10 & $(2,1)$ & 1 & 1 & YES & YES & YES & $1.38$ & $(2,3)$ & -- & 5710\\
$(111,46)$ & 10 & $(2,1)$ & 1 & 1 & YES & YES & NO(2) & $1.50$ & $(2,3)$ & -- & 5711\\
$(111,46)$ & 10 & $(3,1)$ & 2 & 3 & YES & YES & NO(2) & $1.56$ & $(4,2)$ & -- & 5712\\
$(111,46)$ & 10 & $(3,1)$ & 2 & 3 & YES & YES & NO(2) & $1.56$ & $(4,2)$ & NO & 5713\\
$(111,41)$ & 10 & $(5,2)$ & 3 & 1 & YES & YES & YES & $1.57$ & $(2,3)$ & -- & 5714\\
$(111,43)$ & 10 & $(5,2)$ & 3 & 1 & YES & YES & NO(2) & $1.67$ & $(4,2)$ & -- & 5715\\
$(111,49)$ & 11 & $(5,2)$ & 3 & 1 & YES & YES & NO(2) & $1.80$ & $(2,3)$ & -- & 5716\\
$(111,41)$ & 10 & $(7,2)$ & 4 & 1 & YES & YES & YES & $1.43$ & $(4,2)$ & -- & 5717\\
$(111,43)$ & 10 & $(7,2)$ & 4 & 1 & YES & YES & NO(2) & $1.70$ & $(2,3)$ & -- & 5718\\
$(111,43)$ & 10 & $(7,3)$ & 4 & 1 & YES & YES & YES & $1.75$ & $(2,3)$ & -- & 5719\\
$(111,46)$ & 10 & $(7,2)$ & 4 & 1 & YES & YES & NO(2) & $1.50$ & $(6,1)$ & -- & 5720\\
$(111,46)$ & 10 & $(7,2)$ & 4 & 1 & YES & YES & YES & $1.57$ & $(4,2)$ & NO & 5721\\
$(111,47)$ & 11 & $(7,2)$ & 4 & 1 & YES & YES & NO(2) & $1.56$ & $(8,0)$ & -- & 5722\\
$(111,31)$ & 10 & $(8,3)$ & 4 & 1 & YES & YES & YES & $1.43$ & $(2,3)$ & -- & 5723\\
$(111,41)$ & 10 & $(8,3)$ & 4 & 1 & YES & YES & NO(2) & $1.56$ & $(8,0)$ & -- & 5724\\
$(111,43)$ & 10 & $(8,3)$ & 4 & 1 & YES & YES & YES & $1.78$ & $(2,3)$ & -- & 5725\\
$(111,43)$ & 10 & $(8,3)$ & 4 & 1 & YES & YES & YES & $1.75$ & $(2,3)$ & NO & 5726\\
$(111,46)$ & 10 & $(8,3)$ & 4 & 1 & YES & YES & YES & $1.43$ & $(4,2)$ & -- & 5727\\
$(111,46)$ & 10 & $(9,4)$ & 5 & 3 & YES & YES & NO(2) & $1.60$ & $(6,1)$ & NO & 5728\\
$(111,40)$ & 11 & $(10,3)$ & 5 & 1 & YES & YES & YES & $2.00$ & $(2,3)$ & -- & 5729\\
$(111,41)$ & 10 & $(10,3)$ & 5 & 1 & YES & YES & NO(2) & $1.56$ & $(8,0)$ & -- & 5730\\
$(111,43)$ & 10 & $(10,3)$ & 5 & 1 & YES & YES & YES & $1.62$ & $(2,3)$ & -- & 5731\\
$(111,46)$ & 10 & $(10,3)$ & 5 & 1 & YES & YES & YES & $1.57$ & $(2,3)$ & -- & 5732\\
$(111,25)$ & 11 & $(11,4)$ & 5 & 1 & YES & YES & NO(2) & $1.50$ & $(4,2)$ & -- & 5733\\
$(111,41)$ & 10 & $(11,3)$ & 5 & 1 & YES & YES & YES & $1.62$ & $(2,3)$ & -- & 5734\\
$(111,46)$ & 10 & $(11,3)$ & 5 & 1 & YES & YES & YES & $1.75$ & $(2,3)$ & -- & 5735\\
$(111,46)$ & 10 & $(11,3)$ & 5 & 1 & YES & YES & YES & $1.78$ & $(2,3)$ & NO & 5736\\
$(111,31)$ & 10 & $(13,5)$ & 5 & 1 & YES & YES & YES & $1.29$ & $(4,2)$ & NO & 5737\\
$(111,46)$ & 10 & $(13,3)$ & 6 & 1 & YES & YES & YES & $1.78$ & $(2,3)$ & NO & 5738\\
$(111,46)$ & 10 & $(13,3)$ & 6 & 1 & YES & YES & YES & $1.78$ & $(2,3)$ & -- & 5739\\
$(111,43)$ & 10 & $(14,3)$ & 6 & 1 & YES & YES & YES & $1.60$ & $(2,3)$ & NO & 5740\\
$(111,25)$ & 11 & $(15,4)$ & 6 & 3 & YES & YES & NO(2) & $1.56$ & $(4,2)$ & NO & 5741\\
$(111,46)$ & 10 & $(16,7)$ & 6 & 1 & YES & YES & YES & $1.43$ & $(4,2)$ & NO & 5742\\
$(111,31)$ & 10 & $(17,4)$ & 7 & 1 & YES & YES & YES & $1.57$ & $(2,3)$ & -- & 5743\\
$(111,46)$ & 10 & $(17,7)$ & 6 & 1 & YES & YES & NO(2) & $1.60$ & $(2,3)$ & 6026 & 5744\\
$(111,31)$ & 10 & $(18,5)$ & 6 & 3 & YES & YES & YES & $1.38$ & $(2,3)$ & NO & 5745\\
$(111,46)$ & 10 & $(22,9)$ & 7 & 1 & YES & YES & NO(2) & $1.60$ & $(6,1)$ & 7518 & 5746\\
$(111,46)$ & 10 & $(27,11)$ & 8 & 3 & YES & YES & YES & $1.57$ & $(2,3)$ & NO & 5747\\
$(111,43)$ & 10 & $(28,11)$ & 8 & 1 & YES & YES & NO(2) & $1.44$ & $(8,0)$ & NO & 5748\\
$(111,46)$ & 10 & $(41,17)$ & 8 & 1 & YES & YES & NO(2) & $1.44$ & $(4,2)$ & NO & 5749\\
$(111,41)$ & 10 & $(62,23)$ & 9 & 1 & YES & YES & NO(2) & $1.56$ & $(8,0)$ & NO & 5750\\
$(111,46)$ & 10 & $(63,26)$ & 9 & 3 & YES & YES & YES & $1.43$ & $(4,2)$ & NO & 5751\\
$(111,46)$ & 10 & $(70,29)$ & 9 & 1 & YES & YES & NO(2) & $1.50$ & $(2,3)$ & NO & 5752\\
$(111,31)$ & 10 & $(79,22)$ & 10 & 1 & YES & YES & NO(2) & $1.38$ & $(6,1)$ & NO & 5753\\
$(111,41)$ & 10 & $(100,37)$ & 10 & 1 & YES & YES & NO(2) & $1.56$ & $(8,0)$ & NO & 5754\\
$(111,46)$ & 10 & $(111,46)$ & 10 & 111 & YES & YES & NO(2) & $1.60$ & $(2,3)$ & NO & 5755\\
$(112,47)$ & 10 & $(4,1)$ & 3 & 4 & YES & YES & YES & $1.62$ & $(2,3)$ & NO & 5756\\
$(112,47)$ & 10 & $(4,1)$ & 3 & 4 & YES & YES & YES & $1.62$ & $(2,3)$ & -- & 5757\\
$(112,33)$ & 12 & $(5,1)$ & 4 & 1 & YES & YES & NO(2) & $1.60$ & $(6,1)$ & -- & 5758\\
$(112,41)$ & 10 & $(7,3)$ & 4 & 7 & YES & YES & YES & $1.43$ & $(4,2)$ & -- & 5759\\
$(112,47)$ & 10 & $(7,2)$ & 4 & 7 & YES & YES & YES & $1.29$ & $(6,1)$ & -- & 5760\\
$(112,31)$ & 10 & $(8,3)$ & 4 & 8 & YES & YES & NO(2) & $1.38$ & $(4,2)$ & -- & 5761\\
$(112,31)$ & 10 & $(8,3)$ & 4 & 8 & YES & YES & YES & $1.38$ & $(2,3)$ & NO & 5762\\
$(112,41)$ & 10 & $(8,3)$ & 4 & 8 & YES & YES & YES & $1.67$ & $(2,3)$ & -- & 5763\\
$(112,47)$ & 10 & $(9,2)$ & 5 & 1 & YES & YES & YES & $1.62$ & $(4,2)$ & -- & 5764\\
$(112,47)$ & 10 & $(9,2)$ & 5 & 1 & YES & YES & YES & $1.78$ & $(4,2)$ & NO & 5765\\
$(112,47)$ & 10 & $(9,4)$ & 5 & 1 & YES & YES & NO(2) & $1.83$ & $(2,3)$ & NO & 5766\\
$(112,31)$ & 10 & $(11,4)$ & 5 & 1 & YES & YES & YES & $1.67$ & $(2,3)$ & -- & 5767\\
$(112,41)$ & 10 & $(11,3)$ & 5 & 1 & YES & YES & YES & $1.43$ & $(4,2)$ & -- & 5768\\
$(112,47)$ & 10 & $(11,3)$ & 5 & 1 & YES & YES & YES & $1.67$ & $(2,3)$ & NO & 5769\\
$(112,31)$ & 10 & $(13,5)$ & 5 & 1 & YES & YES & YES & $1.89$ & $(2,3)$ & NO & 5770\\
$(112,41)$ & 10 & $(13,3)$ & 6 & 1 & YES & YES & YES & $1.80$ & $(2,3)$ & -- & 5771\\
$(112,47)$ & 10 & $(13,5)$ & 5 & 1 & YES & YES & YES & $1.71$ & $(2,3)$ & NO & 5772\\
$(112,47)$ & 10 & $(17,7)$ & 6 & 1 & YES & YES & NO(2) & $1.70$ & $(2,3)$ & NO & 5773\\
$(112,31)$ & 10 & $(23,7)$ & 7 & 1 & YES & YES & YES & $1.89$ & $(2,3)$ & NO & 5774\\
$(112,47)$ & 10 & $(50,21)$ & 8 & 2 & YES & YES & NO(2) & $1.50$ & $(2,3)$ & 6275 & 5775\\
$(112,31)$ & 10 & $(69,19)$ & 9 & 1 & YES & YES & NO(2) & $1.38$ & $(4,2)$ & NO & 5776\\
$(112,47)$ & 10 & $(69,29)$ & 9 & 1 & YES & YES & NO(2) & $1.73$ & $(4,2)$ & 7862 & 5777\\
$(112,33)$ & 12 & $(78,23)$ & 10 & 2 & YES & YES & NO(2) & $1.60$ & $(6,1)$ & 7299 & 5778\\
$(112,47)$ & 10 & $(81,34)$ & 9 & 1 & YES & YES & NO(2) & $1.60$ & $(2,3)$ & NO & 5779\\
$(112,47)$ & 10 & $(112,47)$ & 10 & 112 & YES & YES & NO(2) & $1.60$ & $(2,3)$ & NO & 5780\\
$(113,35)$ & 11 & $(3,1)$ & 2 & 1 & YES & YES & NO(2) & $1.38$ & $(6,1)$ & NO & 5781\\
$(113,35)$ & 11 & $(3,1)$ & 2 & 1 & YES & YES & NO(2) & $1.38$ & $(6,1)$ & -- & 5782\\
$(113,42)$ & 11 & $(4,1)$ & 3 & 1 & YES & YES & YES & $1.43$ & $(4,2)$ & NO & 5783\\
$(113,42)$ & 11 & $(4,1)$ & 3 & 1 & YES & YES & YES & $1.43$ & $(4,2)$ & -- & 5784\\
$(113,35)$ & 11 & $(5,2)$ & 3 & 1 & YES & YES & NO(2) & $1.78$ & $(4,2)$ & NO & 5785\\
$(113,35)$ & 11 & $(5,2)$ & 3 & 1 & YES & YES & NO(2) & $1.38$ & $(6,1)$ & NO & 5786\\
$(113,35)$ & 11 & $(5,2)$ & 3 & 1 & YES & YES & NO(2) & $1.38$ & $(6,1)$ & -- & 5787\\
$(113,35)$ & 11 & $(7,2)$ & 4 & 1 & YES & YES & NO(2) & $1.60$ & $(6,1)$ & NO & 5788\\
$(113,35)$ & 11 & $(7,2)$ & 4 & 1 & YES & YES & NO(2) & $1.60$ & $(6,1)$ & -- & 5789\\
$(113,42)$ & 11 & $(8,3)$ & 4 & 1 & YES & YES & NO(2) & $1.50$ & $(6,1)$ & -- & 5790\\
$(113,21)$ & 11 & $(9,4)$ & 5 & 1 & YES & YES & NO(2) & $1.56$ & $(4,2)$ & NO & 5791\\
$(113,21)$ & 11 & $(9,4)$ & 5 & 1 & YES & YES & NO(2) & $1.56$ & $(4,2)$ & -- & 5792\\
$(113,43)$ & 11 & $(10,3)$ & 5 & 1 & YES & YES & NO(2) & $1.62$ & $(6,1)$ & NO & 5793\\
$(113,35)$ & 11 & $(11,3)$ & 5 & 1 & YES & YES & YES & $1.57$ & $(4,2)$ & -- & 5794\\
$(113,44)$ & 12 & $(12,5)$ & 5 & 1 & YES & YES & YES & $1.71$ & $(4,2)$ & 5612 & 5795\\
$(113,33)$ & 11 & $(13,3)$ & 6 & 1 & YES & YES & YES & $1.43$ & $(4,2)$ & -- & 5796\\
$(113,33)$ & 11 & $(14,3)$ & 6 & 1 & YES & YES & YES & $1.57$ & $(4,2)$ & -- & 5797\\
$(113,43)$ & 11 & $(14,3)$ & 6 & 1 & YES & YES & YES & $1.62$ & $(4,2)$ & -- & 5798\\
$(113,33)$ & 11 & $(16,3)$ & 7 & 1 & YES & YES & YES & $1.50$ & $(4,2)$ & NO & 5799\\
$(113,33)$ & 11 & $(16,3)$ & 7 & 1 & YES & YES & YES & $1.50$ & $(4,2)$ & -- & 5800\\
$(113,33)$ & 11 & $(16,5)$ & 7 & 1 & YES & YES & YES & $1.57$ & $(2,3)$ & NO & 5801\\
$(113,24)$ & 11 & $(17,4)$ & 7 & 1 & YES & YES & NO(2) & $1.56$ & $(2,3)$ & -- & 5802\\
$(113,49)$ & 11 & $(17,7)$ & 6 & 1 & YES & YES & NO(2) & $1.50$ & $(6,1)$ & NO & 5803\\
$(113,35)$ & 11 & $(19,6)$ & 8 & 1 & YES & YES & NO(2) & $1.50$ & $(6,1)$ & 7612 & 5804\\
$(113,31)$ & 11 & $(23,7)$ & 7 & 1 & YES & YES & YES & $1.71$ & $(6,1)$ & NO & 5805\\
$(113,30)$ & 11 & $(25,7)$ & 7 & 1 & YES & YES & NO(2) & $1.44$ & $(4,2)$ & NO & 5806\\
$(113,43)$ & 11 & $(45,17)$ & 9 & 1 & YES & YES & NO(2) & $1.62$ & $(6,1)$ & NO & 5807\\
$(113,42)$ & 11 & $(73,27)$ & 9 & 1 & YES & YES & YES & $1.90$ & $(2,3)$ & NO & 5808\\
$(113,42)$ & 11 & $(78,29)$ & 10 & 1 & YES & YES & YES & $1.43$ & $(4,2)$ & NO & 5809\\
$(113,33)$ & 11 & $(89,26)$ & 10 & 1 & YES & YES & NO(2) & $1.50$ & $(6,1)$ & NO & 5810\\
$(113,33)$ & 11 & $(106,31)$ & 10 & 1 & YES & YES & YES & $1.43$ & $(2,3)$ & 8064 & 5811\\
$(114,41)$ & 11 & $(10,3)$ & 5 & 2 & YES & YES & YES & $1.67$ & $(4,2)$ & -- & 5812\\
$(115,34)$ & 10 & $(3,1)$ & 2 & 1 & YES & YES & YES & $1.50$ & $(2,3)$ & -- & 5813\\
$(115,34)$ & 10 & $(5,2)$ & 3 & 5 & YES & YES & YES & $1.50$ & $(2,3)$ & -- & 5814\\
$(115,26)$ & 11 & $(7,3)$ & 4 & 1 & YES & YES & NO(2) & $1.60$ & $(2,3)$ & -- & 5815\\
$(115,26)$ & 11 & $(7,3)$ & 4 & 1 & YES & YES & NO(2) & $1.60$ & $(2,3)$ & NO & 5816\\
$(115,34)$ & 10 & $(7,2)$ & 4 & 1 & YES & YES & YES & $1.50$ & $(2,3)$ & NO & 5817\\
$(115,34)$ & 10 & $(7,3)$ & 4 & 1 & YES & YES & YES & $1.50$ & $(2,3)$ & -- & 5818\\
$(115,44)$ & 10 & $(7,2)$ & 4 & 1 & YES & YES & YES & $1.62$ & $(2,3)$ & -- & 5819\\
$(115,44)$ & 10 & $(7,3)$ & 4 & 1 & YES & YES & NO(2) & $1.78$ & $(2,3)$ & -- & 5820\\
$(115,31)$ & 11 & $(8,3)$ & 4 & 1 & YES & YES & NO(2) & $1.50$ & $(6,1)$ & NO & 5821\\
$(115,31)$ & 11 & $(8,3)$ & 4 & 1 & YES & YES & NO(2) & $1.50$ & $(6,1)$ & -- & 5822\\
$(115,44)$ & 10 & $(8,3)$ & 4 & 1 & YES & YES & YES & $1.62$ & $(2,3)$ & -- & 5823\\
$(115,34)$ & 10 & $(10,3)$ & 5 & 5 & YES & YES & YES & $1.50$ & $(2,3)$ & 5469 & 5824\\
$(115,44)$ & 10 & $(10,3)$ & 5 & 5 & YES & YES & YES & $1.43$ & $(4,2)$ & -- & 5825\\
$(115,44)$ & 10 & $(11,3)$ & 5 & 1 & YES & YES & YES & $1.50$ & $(2,3)$ & -- & 5826\\
$(115,44)$ & 10 & $(11,4)$ & 5 & 1 & YES & YES & NO(2) & $1.67$ & $(4,2)$ & NO & 5827\\
$(115,26)$ & 11 & $(15,4)$ & 6 & 5 & YES & YES & YES & $1.75$ & $(2,3)$ & -- & 5828\\
$(115,26)$ & 11 & $(17,4)$ & 7 & 1 & YES & YES & YES & $1.75$ & $(2,3)$ & -- & 5829\\
$(115,44)$ & 10 & $(47,18)$ & 8 & 1 & YES & YES & NO(2) & $1.60$ & $(2,3)$ & 6171 & 5830\\
$(115,26)$ & 11 & $(48,11)$ & 9 & 1 & YES & YES & NO(2) & $1.44$ & $(4,2)$ & NO & 5831\\
$(115,44)$ & 10 & $(50,19)$ & 8 & 5 & YES & YES & YES & $1.90$ & $(2,3)$ & NO & 5832\\
$(115,44)$ & 10 & $(73,28)$ & 10 & 1 & YES & YES & YES & $1.50$ & $(2,3)$ & NO & 5833\\
$(115,34)$ & 10 & $(78,23)$ & 10 & 1 & YES & YES & NO(2) & $1.33$ & $(8,0)$ & NO & 5834\\
$(115,44)$ & 10 & $(107,41)$ & 10 & 1 & YES & YES & YES & $1.75$ & $(2,3)$ & NO & 5835\\
$(116,49)$ & 10 & $(2,1)$ & 1 & 2 & YES & YES & YES & $1.50$ & $(2,3)$ & -- & 5836\\
$(116,49)$ & 10 & $(3,1)$ & 2 & 1 & YES & YES & NO(2) & $1.14$ & $(8,0)$ & -- & 5837\\
$(116,49)$ & 10 & $(7,2)$ & 4 & 1 & YES & YES & NO(2) & $1.44$ & $(4,2)$ & NO & 5838\\
$(116,49)$ & 10 & $(7,3)$ & 4 & 1 & YES & YES & NO(2) & $1.50$ & $(4,2)$ & -- & 5839\\
$(116,27)$ & 11 & $(8,3)$ & 4 & 4 & YES & YES & NO(2) & $1.60$ & $(2,3)$ & NO & 5840\\
$(116,45)$ & 10 & $(8,3)$ & 4 & 4 & YES & YES & YES & $1.57$ & $(4,2)$ & -- & 5841\\
$(116,45)$ & 10 & $(8,3)$ & 4 & 4 & YES & YES & YES & $1.57$ & $(4,2)$ & NO & 5842\\
$(116,35)$ & 12 & $(9,2)$ & 5 & 1 & YES & YES & NO(2) & $1.56$ & $(4,2)$ & -- & 5843\\
$(116,45)$ & 10 & $(9,4)$ & 5 & 1 & YES & YES & YES & $1.75$ & $(4,2)$ & -- & 5844\\
$(116,49)$ & 10 & $(9,2)$ & 5 & 1 & YES & YES & YES & $1.29$ & $(2,3)$ & NO & 5845\\
$(116,45)$ & 10 & $(10,3)$ & 5 & 2 & YES & YES & YES & $1.62$ & $(4,2)$ & -- & 5846\\
$(116,49)$ & 10 & $(10,3)$ & 5 & 2 & YES & YES & YES & $1.57$ & $(4,2)$ & -- & 5847\\
$(116,35)$ & 12 & $(11,2)$ & 6 & 1 & YES & YES & NO(2) & $1.44$ & $(8,0)$ & -- & 5848\\
$(116,49)$ & 10 & $(11,3)$ & 5 & 1 & YES & YES & YES & $1.57$ & $(4,2)$ & NO & 5849\\
$(116,49)$ & 10 & $(12,5)$ & 5 & 4 & YES & YES & NO(2) & $1.14$ & $(8,0)$ & NO & 5850\\
$(116,45)$ & 10 & $(13,3)$ & 6 & 1 & YES & YES & YES & $1.43$ & $(4,2)$ & -- & 5851\\
$(116,45)$ & 10 & $(13,3)$ & 6 & 1 & YES & YES & YES & $1.78$ & $(2,3)$ & NO & 5852\\
$(116,49)$ & 10 & $(17,7)$ & 6 & 1 & YES & YES & NO(2) & $1.44$ & $(4,2)$ & NO & 5853\\
$(116,43)$ & 11 & $(25,9)$ & 7 & 1 & YES & YES & NO(2) & $1.78$ & $(4,2)$ & NO & 5854\\
$(116,49)$ & 10 & $(29,12)$ & 7 & 29 & YES & YES & YES & $1.57$ & $(4,2)$ & NO & 5855\\
$(116,49)$ & 10 & $(31,13)$ & 7 & 1 & YES & YES & NO(2) & $1.44$ & $(4,2)$ & NO & 5856\\
$(116,35)$ & 12 & $(36,11)$ & 8 & 4 & YES & YES & NO(2) & $1.44$ & $(8,0)$ & NO & 5857\\
$(116,49)$ & 10 & $(43,18)$ & 8 & 1 & YES & YES & YES & $1.57$ & $(4,2)$ & NO & 5858\\
$(116,35)$ & 12 & $(56,17)$ & 9 & 4 & YES & YES & NO(2) & $1.56$ & $(4,2)$ & NO & 5859\\
$(117,34)$ & 11 & $(3,1)$ & 2 & 3 & YES & YES & NO(2) & $1.50$ & $(2,3)$ & -- & 5860\\
$(117,49)$ & 10 & $(3,1)$ & 2 & 3 & YES & YES & NO(2) & $1.50$ & $(6,1)$ & -- & 5861\\
$(117,49)$ & 10 & $(4,1)$ & 3 & 1 & YES & YES & NO(2) & $1.56$ & $(4,2)$ & -- & 5862\\
$(117,49)$ & 10 & $(5,2)$ & 3 & 1 & YES & YES & NO(2) & $1.64$ & $(4,2)$ & -- & 5863\\
$(117,31)$ & 11 & $(7,2)$ & 4 & 1 & YES & YES & NO(2) & $1.38$ & $(6,1)$ & -- & 5864\\
$(117,43)$ & 10 & $(7,2)$ & 4 & 1 & YES & YES & NO(2) & $1.64$ & $(4,2)$ & -- & 5865\\
$(117,43)$ & 10 & $(7,3)$ & 4 & 1 & YES & YES & YES & $1.80$ & $(2,3)$ & -- & 5866\\
$(117,49)$ & 10 & $(7,2)$ & 4 & 1 & YES & YES & YES & $1.29$ & $(2,3)$ & -- & 5867\\
$(117,49)$ & 10 & $(7,2)$ & 4 & 1 & YES & YES & NO(2) & $1.67$ & $(2,3)$ & NO & 5868\\
$(117,49)$ & 10 & $(7,3)$ & 4 & 1 & YES & YES & NO(2) & $1.50$ & $(4,2)$ & -- & 5869\\
$(117,49)$ & 10 & $(7,3)$ & 4 & 1 & YES & YES & NO(2) & $1.62$ & $(6,1)$ & NO & 5870\\
$(117,49)$ & 10 & $(8,3)$ & 4 & 1 & YES & YES & NO(2) & $1.60$ & $(2,3)$ & NO & 5871\\
$(117,49)$ & 10 & $(8,3)$ & 4 & 1 & YES & YES & YES & $1.57$ & $(4,2)$ & -- & 5872\\
$(117,49)$ & 10 & $(8,3)$ & 4 & 1 & YES & YES & YES & $1.57$ & $(4,2)$ & NO & 5873\\
$(117,49)$ & 10 & $(9,4)$ & 5 & 9 & YES & YES & YES & $1.43$ & $(4,2)$ & NO & 5874\\
$(117,34)$ & 11 & $(10,3)$ & 5 & 1 & YES & YES & YES & $1.56$ & $(4,2)$ & -- & 5875\\
$(117,49)$ & 10 & $(10,3)$ & 5 & 1 & YES & YES & YES & $1.67$ & $(4,2)$ & -- & 5876\\
$(117,43)$ & 10 & $(12,5)$ & 5 & 3 & YES & YES & YES & $1.57$ & $(4,2)$ & NO & 5877\\
$(117,34)$ & 11 & $(13,5)$ & 5 & 13 & YES & YES & YES & $1.89$ & $(4,2)$ & -- & 5878\\
$(117,49)$ & 10 & $(26,11)$ & 7 & 13 & YES & YES & YES & $1.43$ & $(2,3)$ & 7778 & 5879\\
$(117,43)$ & 10 & $(35,13)$ & 8 & 1 & YES & YES & NO(2) & $1.50$ & $(4,2)$ & NO & 5880\\
$(117,34)$ & 11 & $(37,11)$ & 8 & 1 & YES & YES & YES & $1.56$ & $(4,2)$ & NO & 5881\\
$(117,49)$ & 10 & $(45,19)$ & 8 & 9 & YES & YES & YES & $1.75$ & $(4,2)$ & NO & 5882\\
$(117,49)$ & 10 & $(55,23)$ & 9 & 1 & YES & YES & NO(2) & $1.44$ & $(8,0)$ & NO & 5883\\
$(117,49)$ & 10 & $(69,29)$ & 9 & 3 & YES & YES & YES & $1.57$ & $(4,2)$ & NO & 5884\\
$(117,49)$ & 10 & $(74,31)$ & 9 & 1 & YES & YES & NO(2) & $1.56$ & $(4,2)$ & NO & 5885\\
$(117,34)$ & 11 & $(79,23)$ & 10 & 1 & YES & YES & NO(2) & $1.70$ & $(2,3)$ & 8083 & 5886\\
$(117,43)$ & 10 & $(93,34)$ & 10 & 3 & YES & YES & YES & $1.78$ & $(4,2)$ & NO & 5887\\
$(117,34)$ & 11 & $(99,29)$ & 10 & 9 & YES & YES & YES & $1.78$ & $(4,2)$ & NO & 5888\\
$(118,33)$ & 11 & $(4,1)$ & 3 & 2 & YES & YES & NO(2) & $1.56$ & $(8,0)$ & NO & 5889\\
$(118,33)$ & 11 & $(4,1)$ & 3 & 2 & YES & YES & NO(2) & $1.56$ & $(8,0)$ & -- & 5890\\
$(118,45)$ & 11 & $(4,1)$ & 3 & 2 & YES & YES & NO(2) & $1.56$ & $(4,2)$ & -- & 5891\\
$(118,45)$ & 11 & $(5,1)$ & 4 & 1 & YES & YES & NO(2) & $1.80$ & $(6,1)$ & NO & 5892\\
$(118,45)$ & 11 & $(5,1)$ & 4 & 1 & YES & YES & NO(2) & $1.80$ & $(6,1)$ & -- & 5893\\
$(118,51)$ & 12 & $(5,2)$ & 3 & 1 & YES & YES & NO(2) & $1.56$ & $(8,0)$ & -- & 5894\\
$(118,45)$ & 11 & $(6,1)$ & 5 & 2 & YES & YES & NO(2) & $1.56$ & $(4,2)$ & NO & 5895\\
$(118,45)$ & 11 & $(6,1)$ & 5 & 2 & YES & YES & NO(2) & $1.56$ & $(4,2)$ & -- & 5896\\
$(118,49)$ & 11 & $(10,3)$ & 5 & 2 & YES & YES & YES & $1.88$ & $(4,2)$ & NO & 5897\\
$(118,49)$ & 11 & $(10,3)$ & 5 & 2 & YES & YES & YES & $1.88$ & $(4,2)$ & -- & 5898\\
$(118,25)$ & 11 & $(11,4)$ & 5 & 1 & YES & YES & NO(2) & $1.60$ & $(6,1)$ & -- & 5899\\
$(118,27)$ & 11 & $(11,4)$ & 5 & 1 & YES & YES & YES & $1.50$ & $(4,2)$ & -- & 5900\\
$(118,33)$ & 11 & $(13,2)$ & 7 & 1 & YES & YES & NO(2) & $1.44$ & $(8,0)$ & NO & 5901\\
$(118,35)$ & 11 & $(13,3)$ & 6 & 1 & YES & YES & YES & $1.86$ & $(2,3)$ & -- & 5902\\
$(118,35)$ & 11 & $(14,3)$ & 6 & 2 & YES & YES & YES & $1.57$ & $(2,3)$ & -- & 5903\\
$(118,35)$ & 11 & $(16,3)$ & 7 & 2 & YES & YES & YES & $1.50$ & $(4,2)$ & -- & 5904\\
$(118,35)$ & 11 & $(16,3)$ & 7 & 2 & YES & YES & YES & $1.50$ & $(4,2)$ & NO & 5905\\
$(118,27)$ & 11 & $(17,4)$ & 7 & 1 & YES & YES & YES & $1.62$ & $(2,3)$ & -- & 5906\\
$(118,45)$ & 11 & $(19,7)$ & 6 & 1 & YES & YES & NO(2) & $1.62$ & $(6,1)$ & NO & 5907\\
$(118,27)$ & 11 & $(53,12)$ & 9 & 1 & YES & YES & NO(2) & $1.44$ & $(4,2)$ & NO & 5908\\
$(118,45)$ & 11 & $(55,21)$ & 8 & 1 & YES & YES & NO(2) & $1.56$ & $(4,2)$ & NO & 5909\\
$(118,45)$ & 11 & $(76,29)$ & 9 & 2 & YES & YES & NO(2) & $1.80$ & $(6,1)$ & 7301 & 5910\\
$(118,49)$ & 11 & $(94,39)$ & 10 & 2 & YES & YES & NO(2) & $1.67$ & $(2,3)$ & NO & 5911\\
$(118,45)$ & 11 & $(97,37)$ & 10 & 1 & YES & YES & NO(2) & $1.56$ & $(4,2)$ & NO & 5912\\
$(118,49)$ & 11 & $(111,46)$ & 10 & 1 & YES & YES & YES & $1.88$ & $(4,2)$ & NO & 5913\\
$(119,46)$ & 10 & $(2,1)$ & 1 & 1 & YES & YES & NO(2) & $1.44$ & $(4,2)$ & -- & 5914\\
$(119,50)$ & 10 & $(2,1)$ & 1 & 1 & YES & YES & NO(2) & $1.50$ & $(2,3)$ & -- & 5915\\
$(119,37)$ & 11 & $(3,1)$ & 2 & 1 & YES & YES & NO(2) & $1.50$ & $(6,1)$ & NO & 5916\\
$(119,37)$ & 11 & $(3,1)$ & 2 & 1 & YES & YES & NO(2) & $1.50$ & $(6,1)$ & -- & 5917\\
$(119,44)$ & 10 & $(3,1)$ & 2 & 1 & YES & YES & NO(2) & $1.50$ & $(6,1)$ & NO & 5918\\
$(119,45)$ & 11 & $(3,1)$ & 2 & 1 & YES & YES & YES & $1.43$ & $(4,2)$ & NO & 5919\\
$(119,45)$ & 11 & $(3,1)$ & 2 & 1 & YES & YES & YES & $1.43$ & $(4,2)$ & -- & 5920\\
$(119,46)$ & 10 & $(3,1)$ & 2 & 1 & YES & YES & NO(2) & $1.70$ & $(2,3)$ & -- & 5921\\
$(119,44)$ & 10 & $(4,1)$ & 3 & 1 & YES & YES & NO(2) & $1.60$ & $(6,1)$ & NO & 5922\\
$(119,44)$ & 10 & $(4,1)$ & 3 & 1 & YES & YES & NO(2) & $1.60$ & $(6,1)$ & -- & 5923\\
$(119,44)$ & 10 & $(4,1)$ & 3 & 1 & YES & YES & NO(2) & $1.60$ & $(6,1)$ & NO & 5924\\
$(119,45)$ & 11 & $(4,1)$ & 3 & 1 & YES & YES & NO(2) & $1.50$ & $(6,1)$ & NO & 5925\\
$(119,45)$ & 11 & $(4,1)$ & 3 & 1 & YES & YES & NO(2) & $1.73$ & $(4,2)$ & -- & 5926\\
$(119,37)$ & 11 & $(5,2)$ & 3 & 1 & YES & YES & NO(2) & $1.50$ & $(6,1)$ & NO & 5927\\
$(119,37)$ & 11 & $(5,2)$ & 3 & 1 & YES & YES & NO(2) & $1.50$ & $(6,1)$ & -- & 5928\\
$(119,44)$ & 10 & $(5,2)$ & 3 & 1 & YES & YES & NO(2) & $1.62$ & $(4,2)$ & -- & 5929\\
$(119,44)$ & 10 & $(5,2)$ & 3 & 1 & YES & YES & NO(2) & $1.60$ & $(6,1)$ & NO & 5930\\
$(119,46)$ & 10 & $(5,1)$ & 4 & 1 & YES & YES & NO(2) & $1.33$ & $(4,2)$ & -- & 5931\\
$(119,46)$ & 10 & $(5,1)$ & 4 & 1 & YES & YES & NO(2) & $1.44$ & $(4,2)$ & NO & 5932\\
$(119,46)$ & 10 & $(5,2)$ & 3 & 1 & YES & YES & NO(2) & $1.38$ & $(6,1)$ & NO & 5933\\
$(119,46)$ & 10 & $(5,2)$ & 3 & 1 & YES & YES & NO(2) & $1.38$ & $(6,1)$ & -- & 5934\\
$(119,50)$ & 10 & $(5,2)$ & 3 & 1 & YES & YES & NO(2) & $1.38$ & $(6,1)$ & -- & 5935\\
$(119,27)$ & 12 & $(7,3)$ & 4 & 7 & YES & YES & NO(2) & $1.78$ & $(2,3)$ & NO & 5936\\
$(119,27)$ & 12 & $(7,3)$ & 4 & 7 & YES & YES & NO(2) & $1.56$ & $(6,1)$ & -- & 5937\\
$(119,36)$ & 11 & $(7,3)$ & 4 & 7 & YES & YES & NO(2) & $1.50$ & $(4,2)$ & -- & 5938\\
$(119,36)$ & 11 & $(7,3)$ & 4 & 7 & YES & YES & NO(2) & $1.62$ & $(4,2)$ & NO & 5939\\
$(119,46)$ & 10 & $(7,2)$ & 4 & 7 & YES & YES & YES & $1.50$ & $(2,3)$ & -- & 5940\\
$(119,46)$ & 10 & $(7,2)$ & 4 & 7 & YES & YES & YES & $1.50$ & $(2,3)$ & NO & 5941\\
$(119,46)$ & 10 & $(7,3)$ & 4 & 7 & YES & YES & YES & $1.78$ & $(2,3)$ & -- & 5942\\
$(119,50)$ & 10 & $(7,3)$ & 4 & 7 & YES & YES & YES & $1.43$ & $(4,2)$ & -- & 5943\\
$(119,36)$ & 11 & $(8,3)$ & 4 & 1 & YES & YES & YES & $1.75$ & $(4,2)$ & -- & 5944\\
$(119,44)$ & 10 & $(8,3)$ & 4 & 1 & YES & YES & YES & $1.62$ & $(2,3)$ & -- & 5945\\
$(119,46)$ & 10 & $(8,3)$ & 4 & 1 & YES & YES & YES & $1.67$ & $(2,3)$ & -- & 5946\\
$(119,50)$ & 10 & $(8,3)$ & 4 & 1 & YES & YES & YES & $1.88$ & $(4,2)$ & NO & 5947\\
$(119,50)$ & 10 & $(8,3)$ & 4 & 1 & YES & YES & YES & $1.78$ & $(2,3)$ & -- & 5948\\
$(119,37)$ & 11 & $(9,2)$ & 5 & 1 & YES & YES & NO(2) & $1.56$ & $(4,2)$ & NO & 5949\\
$(119,43)$ & 11 & $(9,4)$ & 5 & 1 & YES & YES & YES & $1.71$ & $(4,2)$ & -- & 5950\\
$(119,27)$ & 12 & $(10,3)$ & 5 & 1 & YES & YES & NO(2) & $1.67$ & $(2,3)$ & NO & 5951\\
$(119,36)$ & 11 & $(10,3)$ & 5 & 1 & YES & YES & YES & $1.62$ & $(2,3)$ & -- & 5952\\
$(119,36)$ & 11 & $(10,3)$ & 5 & 1 & YES & YES & YES & $1.75$ & $(4,2)$ & NO & 5953\\
$(119,37)$ & 11 & $(10,3)$ & 5 & 1 & YES & YES & NO(2) & $1.50$ & $(6,1)$ & 6761 & 5954\\
$(119,46)$ & 10 & $(10,3)$ & 5 & 1 & YES & YES & NO(2) & $1.56$ & $(4,2)$ & -- & 5955\\
$(119,50)$ & 10 & $(10,3)$ & 5 & 1 & YES & YES & YES & $1.56$ & $(2,3)$ & -- & 5956\\
$(119,36)$ & 11 & $(11,3)$ & 5 & 1 & YES & YES & YES & $1.75$ & $(2,3)$ & -- & 5957\\
$(119,46)$ & 10 & $(11,3)$ & 5 & 1 & YES & YES & YES & $1.78$ & $(2,3)$ & NO & 5958\\
$(119,46)$ & 10 & $(11,4)$ & 5 & 1 & YES & YES & NO(2) & $1.50$ & $(6,1)$ & NO & 5959\\
$(119,44)$ & 10 & $(13,3)$ & 6 & 1 & YES & YES & YES & $1.50$ & $(2,3)$ & -- & 5960\\
$(119,44)$ & 10 & $(13,3)$ & 6 & 1 & YES & YES & YES & $1.62$ & $(2,3)$ & NO & 5961\\
$(119,46)$ & 10 & $(13,3)$ & 6 & 1 & YES & YES & YES & $1.43$ & $(4,2)$ & -- & 5962\\
$(119,50)$ & 10 & $(13,3)$ & 6 & 1 & YES & YES & YES & $1.62$ & $(2,3)$ & NO & 5963\\
$(119,50)$ & 10 & $(13,3)$ & 6 & 1 & YES & YES & YES & $1.62$ & $(2,3)$ & -- & 5964\\
$(119,46)$ & 10 & $(18,5)$ & 6 & 1 & YES & YES & YES & $1.78$ & $(4,2)$ & NO & 5965\\
$(119,37)$ & 11 & $(19,6)$ & 8 & 1 & YES & YES & NO(2) & $1.50$ & $(6,1)$ & NO & 5966\\
$(119,44)$ & 10 & $(19,7)$ & 6 & 1 & YES & YES & NO(2) & $1.50$ & $(2,3)$ & NO & 5967\\
$(119,50)$ & 10 & $(22,9)$ & 7 & 1 & YES & YES & YES & $1.43$ & $(6,1)$ & NO & 5968\\
$(119,36)$ & 11 & $(24,7)$ & 7 & 1 & YES & YES & YES & $1.71$ & $(2,3)$ & NO & 5969\\
$(119,36)$ & 11 & $(27,8)$ & 7 & 1 & YES & YES & NO(2) & $1.50$ & $(6,1)$ & NO & 5970\\
$(119,50)$ & 10 & $(29,12)$ & 7 & 1 & YES & YES & YES & $1.43$ & $(4,2)$ & NO & 5971\\
$(119,44)$ & 10 & $(30,11)$ & 7 & 1 & YES & YES & NO(2) & $1.60$ & $(2,3)$ & NO & 5972\\
$(119,46)$ & 10 & $(31,12)$ & 7 & 1 & YES & YES & NO(2) & $1.33$ & $(4,2)$ & 5596 & 5973\\
$(119,44)$ & 10 & $(35,13)$ & 8 & 7 & YES & YES & NO(2) & $1.62$ & $(4,2)$ & NO & 5974\\
$(119,44)$ & 10 & $(41,15)$ & 8 & 1 & YES & YES & YES & $1.62$ & $(2,3)$ & NO & 5975\\
$(119,46)$ & 10 & $(41,16)$ & 8 & 1 & YES & YES & YES & $1.75$ & $(4,2)$ & NO & 5976\\
$(119,46)$ & 10 & $(44,17)$ & 8 & 1 & YES & YES & NO(2) & $1.33$ & $(4,2)$ & NO & 5977\\
$(119,44)$ & 10 & $(46,17)$ & 8 & 1 & YES & YES & NO(2) & $1.50$ & $(2,3)$ & NO & 5978\\
$(119,45)$ & 11 & $(47,18)$ & 8 & 1 & YES & YES & YES & $1.43$ & $(6,1)$ & NO & 5979\\
$(119,46)$ & 10 & $(51,20)$ & 9 & 17 & YES & YES & YES & $1.71$ & $(6,1)$ & 7894 & 5980\\
$(119,50)$ & 10 & $(55,23)$ & 9 & 1 & YES & YES & YES & $1.75$ & $(2,3)$ & 5379 & 5981\\
$(119,46)$ & 10 & $(57,22)$ & 9 & 1 & YES & YES & NO(2) & $1.70$ & $(2,3)$ & NO & 5982\\
$(119,44)$ & 10 & $(65,24)$ & 9 & 1 & YES & YES & NO(2) & $1.50$ & $(4,2)$ & NO & 5983\\
$(119,45)$ & 11 & $(66,25)$ & 9 & 1 & YES & YES & NO(2) & $1.60$ & $(10,-1)$ & NO & 5984\\
$(119,44)$ & 10 & $(73,27)$ & 9 & 1 & YES & YES & YES & $1.50$ & $(2,3)$ & NO & 5985\\
$(119,50)$ & 10 & $(74,31)$ & 9 & 1 & YES & YES & YES & $1.56$ & $(2,3)$ & NO & 5986\\
$(119,45)$ & 11 & $(82,31)$ & 10 & 1 & YES & YES & YES & $1.43$ & $(4,2)$ & NO & 5987\\
$(119,50)$ & 10 & $(112,47)$ & 10 & 7 & YES & YES & YES & $1.78$ & $(2,3)$ & NO & 5988\\
$(119,46)$ & 10 & $(119,46)$ & 10 & 119 & YES & YES & NO(2) & $1.60$ & $(2,3)$ & NO & 5989\\
$(121,50)$ & 10 & $(2,1)$ & 1 & 1 & YES & YES & NO(2) & $1.64$ & $(4,2)$ & -- & 5990\\
$(121,34)$ & 11 & $(3,1)$ & 2 & 1 & YES & YES & NO(2) & $1.82$ & $(4,2)$ & -- & 5991\\
$(121,36)$ & 11 & $(3,1)$ & 2 & 1 & YES & YES & NO(2) & $1.56$ & $(8,0)$ & NO & 5992\\
$(121,36)$ & 11 & $(3,1)$ & 2 & 1 & YES & YES & NO(2) & $1.56$ & $(8,0)$ & -- & 5993\\
$(121,46)$ & 10 & $(3,1)$ & 2 & 1 & YES & YES & NO(2) & $1.38$ & $(6,1)$ & -- & 5994\\
$(121,32)$ & 11 & $(4,1)$ & 3 & 1 & YES & YES & NO(2) & $1.50$ & $(6,1)$ & NO & 5995\\
$(121,32)$ & 11 & $(4,1)$ & 3 & 1 & YES & YES & NO(2) & $1.50$ & $(6,1)$ & -- & 5996\\
$(121,32)$ & 11 & $(5,2)$ & 3 & 1 & YES & YES & NO(2) & $1.56$ & $(4,2)$ & NO & 5997\\
$(121,32)$ & 11 & $(5,2)$ & 3 & 1 & YES & YES & NO(2) & $1.56$ & $(4,2)$ & -- & 5998\\
$(121,36)$ & 11 & $(5,2)$ & 3 & 1 & YES & YES & NO(2) & $1.44$ & $(4,2)$ & -- & 5999\\
$(121,37)$ & 11 & $(5,2)$ & 3 & 1 & YES & YES & NO(2) & $1.50$ & $(10,-1)$ & NO & 6000\\
$(121,46)$ & 10 & $(5,2)$ & 3 & 1 & YES & YES & NO(2) & $1.38$ & $(4,2)$ & -- & 6001\\
$(121,46)$ & 10 & $(5,2)$ & 3 & 1 & YES & YES & NO(2) & $1.33$ & $(4,2)$ & NO & 6002\\
$(121,50)$ & 10 & $(5,2)$ & 3 & 1 & YES & YES & NO(2) & $1.56$ & $(4,2)$ & -- & 6003\\
$(121,36)$ & 11 & $(7,2)$ & 4 & 1 & YES & YES & NO(2) & $1.60$ & $(2,3)$ & -- & 6004\\
$(121,36)$ & 11 & $(7,3)$ & 4 & 1 & YES & YES & YES & $1.43$ & $(2,3)$ & -- & 6005\\
$(121,37)$ & 11 & $(7,2)$ & 4 & 1 & YES & YES & NO(2) & $1.38$ & $(6,1)$ & -- & 6006\\
$(121,46)$ & 10 & $(7,2)$ & 4 & 1 & YES & YES & YES & $1.80$ & $(2,3)$ & -- & 6007\\
$(121,46)$ & 10 & $(7,2)$ & 4 & 1 & YES & YES & YES & $1.62$ & $(4,2)$ & NO & 6008\\
$(121,46)$ & 10 & $(7,3)$ & 4 & 1 & YES & YES & YES & $1.62$ & $(2,3)$ & -- & 6009\\
$(121,50)$ & 10 & $(7,2)$ & 4 & 1 & YES & YES & YES & $1.50$ & $(4,2)$ & -- & 6010\\
$(121,50)$ & 10 & $(7,2)$ & 4 & 1 & YES & YES & YES & $1.75$ & $(4,2)$ & NO & 6011\\
$(121,50)$ & 10 & $(7,3)$ & 4 & 1 & YES & YES & NO(2) & $1.50$ & $(4,2)$ & -- & 6012\\
$(121,50)$ & 10 & $(7,3)$ & 4 & 1 & YES & YES & NO(2) & $1.60$ & $(2,3)$ & 6234 & 6013\\
$(121,46)$ & 10 & $(8,3)$ & 4 & 1 & YES & YES & YES & $1.57$ & $(4,2)$ & -- & 6014\\
$(121,50)$ & 10 & $(8,3)$ & 4 & 1 & YES & YES & NO(2) & $1.60$ & $(6,1)$ & NO & 6015\\
$(121,50)$ & 10 & $(8,3)$ & 4 & 1 & YES & YES & YES & $1.89$ & $(2,3)$ & -- & 6016\\
$(121,46)$ & 10 & $(9,2)$ & 5 & 1 & YES & YES & YES & $1.50$ & $(2,3)$ & NO & 6017\\
$(121,50)$ & 10 & $(9,2)$ & 5 & 1 & YES & YES & NO(2) & $1.64$ & $(4,2)$ & -- & 6018\\
$(121,36)$ & 11 & $(10,3)$ & 5 & 1 & YES & YES & YES & $1.43$ & $(4,2)$ & -- & 6019\\
$(121,36)$ & 11 & $(10,3)$ & 5 & 1 & YES & YES & YES & $1.88$ & $(4,2)$ & NO & 6020\\
$(121,46)$ & 10 & $(10,3)$ & 5 & 1 & YES & YES & YES & $1.62$ & $(4,2)$ & -- & 6021\\
$(121,50)$ & 10 & $(10,3)$ & 5 & 1 & YES & YES & YES & $1.57$ & $(2,3)$ & -- & 6022\\
$(121,36)$ & 11 & $(11,3)$ & 5 & 11 & YES & YES & NO(2) & $1.44$ & $(4,2)$ & NO & 6023\\
$(121,50)$ & 10 & $(11,3)$ & 5 & 11 & YES & YES & YES & $1.43$ & $(2,3)$ & -- & 6024\\
$(121,51)$ & 12 & $(11,5)$ & 6 & 11 & YES & YES & NO(2) & $1.67$ & $(8,0)$ & NO & 6025\\
$(121,50)$ & 10 & $(12,5)$ & 5 & 1 & YES & YES & NO(2) & $1.60$ & $(2,3)$ & 5744 & 6026\\
$(121,46)$ & 10 & $(13,3)$ & 6 & 1 & YES & YES & YES & $1.62$ & $(4,2)$ & -- & 6027\\
$(121,46)$ & 10 & $(13,3)$ & 6 & 1 & YES & YES & YES & $1.75$ & $(4,2)$ & NO & 6028\\
$(121,46)$ & 10 & $(13,5)$ & 5 & 1 & YES & YES & NO(2) & $1.50$ & $(6,1)$ & NO & 6029\\
$(121,50)$ & 10 & $(13,3)$ & 6 & 1 & YES & YES & YES & $1.57$ & $(4,2)$ & NO & 6030\\
$(121,50)$ & 10 & $(13,3)$ & 6 & 1 & YES & YES & YES & $1.86$ & $(2,3)$ & -- & 6031\\
$(121,46)$ & 10 & $(18,7)$ & 6 & 1 & YES & YES & NO(2) & $1.56$ & $(4,2)$ & NO & 6032\\
$(121,50)$ & 10 & $(19,8)$ & 6 & 1 & YES & YES & YES & $1.43$ & $(4,2)$ & NO & 6033\\
$(121,50)$ & 10 & $(22,9)$ & 7 & 11 & YES & YES & NO(2) & $1.60$ & $(6,1)$ & NO & 6034\\
$(121,50)$ & 10 & $(26,11)$ & 7 & 1 & YES & YES & YES & $1.62$ & $(4,2)$ & NO & 6035\\
$(121,45)$ & 11 & $(27,10)$ & 7 & 1 & YES & YES & NO(2) & $1.73$ & $(4,2)$ & 6924 & 6036\\
$(121,34)$ & 11 & $(29,8)$ & 7 & 1 & YES & YES & NO(2) & $1.56$ & $(2,3)$ & NO & 6037\\
$(121,37)$ & 11 & $(33,10)$ & 8 & 11 & YES & YES & NO(2) & $1.38$ & $(10,-1)$ & NO & 6038\\
$(121,34)$ & 11 & $(41,11)$ & 8 & 1 & YES & YES & YES & $1.78$ & $(4,2)$ & NO & 6039\\
$(121,32)$ & 11 & $(42,11)$ & 9 & 1 & YES & YES & YES & $1.43$ & $(4,2)$ & NO & 6040\\
$(121,34)$ & 11 & $(43,12)$ & 8 & 1 & YES & YES & NO(2) & $1.38$ & $(6,1)$ & 8694 & 6041\\
$(121,34)$ & 11 & $(57,16)$ & 9 & 1 & YES & YES & NO(2) & $1.73$ & $(4,2)$ & 6598 & 6042\\
$(121,36)$ & 11 & $(61,18)$ & 9 & 1 & YES & YES & YES & $1.62$ & $(4,2)$ & NO & 6043\\
$(121,50)$ & 10 & $(63,26)$ & 9 & 1 & YES & YES & NO(2) & $1.67$ & $(4,2)$ & NO & 6044\\
$(121,46)$ & 10 & $(66,25)$ & 9 & 11 & YES & YES & YES & $1.75$ & $(2,3)$ & NO & 6045\\
$(121,36)$ & 11 & $(71,21)$ & 9 & 1 & YES & YES & YES & $1.67$ & $(2,3)$ & NO & 6046\\
$(121,46)$ & 10 & $(79,30)$ & 9 & 1 & YES & YES & YES & $1.62$ & $(2,3)$ & NO & 6047\\
$(121,50)$ & 10 & $(80,33)$ & 10 & 1 & YES & YES & NO(2) & $1.38$ & $(6,1)$ & NO & 6048\\
$(121,50)$ & 10 & $(104,43)$ & 10 & 1 & YES & YES & NO(2) & $1.73$ & $(4,2)$ & NO & 6049\\
$(122,55)$ & 11 & $(3,1)$ & 2 & 1 & YES & YES & NO(2) & $1.62$ & $(6,1)$ & -- & 6050\\
$(122,37)$ & 11 & $(4,1)$ & 3 & 2 & YES & YES & NO(2) & $1.67$ & $(4,2)$ & -- & 6051\\
$(122,33)$ & 11 & $(7,3)$ & 4 & 1 & YES & YES & NO(2) & $1.38$ & $(4,2)$ & -- & 6052\\
$(122,37)$ & 11 & $(7,2)$ & 4 & 1 & YES & YES & YES & $1.14$ & $(4,2)$ & -- & 6053\\
$(122,37)$ & 11 & $(7,3)$ & 4 & 1 & YES & YES & NO(2) & $1.67$ & $(2,3)$ & -- & 6054\\
$(122,37)$ & 11 & $(7,3)$ & 4 & 1 & YES & YES & NO(2) & $1.62$ & $(4,2)$ & NO & 6055\\
$(122,55)$ & 11 & $(7,3)$ & 4 & 1 & YES & YES & NO(2) & $1.70$ & $(6,1)$ & -- & 6056\\
$(122,55)$ & 11 & $(9,2)$ & 5 & 1 & YES & YES & NO(2) & $1.56$ & $(8,0)$ & NO & 6057\\
$(122,51)$ & 11 & $(11,2)$ & 6 & 1 & YES & YES & YES & $1.57$ & $(2,3)$ & NO & 6058\\
$(122,51)$ & 11 & $(11,2)$ & 6 & 1 & YES & YES & YES & $1.57$ & $(2,3)$ & -- & 6059\\
$(122,37)$ & 11 & $(12,5)$ & 5 & 2 & YES & YES & YES & $1.67$ & $(4,2)$ & -- & 6060\\
$(122,37)$ & 11 & $(13,5)$ & 5 & 1 & YES & YES & YES & $1.67$ & $(4,2)$ & -- & 6061\\
$(122,33)$ & 11 & $(17,4)$ & 7 & 1 & YES & YES & NO(2) & $1.50$ & $(6,1)$ & NO & 6062\\
$(122,51)$ & 11 & $(17,7)$ & 6 & 1 & YES & YES & NO(2) & $1.67$ & $(4,2)$ & NO & 6063\\
$(122,33)$ & 11 & $(25,7)$ & 7 & 1 & YES & YES & NO(2) & $1.67$ & $(2,3)$ & NO & 6064\\
$(122,33)$ & 11 & $(29,8)$ & 7 & 1 & YES & YES & NO(2) & $1.56$ & $(4,2)$ & NO & 6065\\
$(122,51)$ & 11 & $(29,12)$ & 7 & 1 & YES & YES & NO(2) & $1.50$ & $(6,1)$ & NO & 6066\\
$(122,33)$ & 11 & $(81,22)$ & 12 & 1 & YES & YES & NO(2) & $1.50$ & $(6,1)$ & NO & 6067\\
$(122,33)$ & 11 & $(89,24)$ & 10 & 1 & YES & YES & NO(2) & $1.38$ & $(4,2)$ & NO & 6068\\
$(123,47)$ & 10 & $(2,1)$ & 1 & 1 & YES & YES & YES & $1.38$ & $(2,3)$ & -- & 6069\\
$(123,34)$ & 10 & $(3,1)$ & 2 & 3 & YES & YES & NO(2) & $1.50$ & $(10,-1)$ & NO & 6070\\
$(123,34)$ & 10 & $(3,1)$ & 2 & 3 & YES & YES & NO(2) & $1.50$ & $(10,-1)$ & -- & 6071\\
$(123,47)$ & 10 & $(3,1)$ & 2 & 3 & YES & YES & NO(2) & $1.70$ & $(2,3)$ & -- & 6072\\
$(123,47)$ & 10 & $(5,2)$ & 3 & 1 & YES & YES & YES & $1.67$ & $(2,3)$ & -- & 6073\\
$(123,34)$ & 10 & $(7,3)$ & 4 & 1 & YES & YES & YES & $1.50$ & $(2,3)$ & -- & 6074\\
$(123,34)$ & 10 & $(7,3)$ & 4 & 1 & YES & YES & YES & $1.62$ & $(2,3)$ & NO & 6075\\
$(123,47)$ & 10 & $(7,2)$ & 4 & 1 & YES & YES & YES & $1.75$ & $(2,3)$ & -- & 6076\\
$(123,47)$ & 10 & $(7,3)$ & 4 & 1 & YES & YES & NO(2) & $1.38$ & $(6,1)$ & NO & 6077\\
$(123,47)$ & 10 & $(7,3)$ & 4 & 1 & YES & YES & YES & $1.75$ & $(2,3)$ & -- & 6078\\
$(123,47)$ & 10 & $(8,3)$ & 4 & 1 & YES & YES & YES & $1.89$ & $(2,3)$ & -- & 6079\\
$(123,28)$ & 12 & $(9,4)$ & 5 & 3 & YES & YES & NO(2) & $1.44$ & $(8,0)$ & -- & 6080\\
$(123,47)$ & 10 & $(9,2)$ & 5 & 3 & YES & YES & YES & $1.62$ & $(2,3)$ & -- & 6081\\
$(123,47)$ & 10 & $(9,2)$ & 5 & 3 & YES & YES & YES & $1.50$ & $(4,2)$ & NO & 6082\\
$(123,47)$ & 10 & $(9,2)$ & 5 & 3 & YES & YES & YES & $1.62$ & $(4,2)$ & NO & 6083\\
$(123,47)$ & 10 & $(10,3)$ & 5 & 1 & YES & YES & YES & $1.71$ & $(2,3)$ & -- & 6084\\
$(123,34)$ & 10 & $(11,3)$ & 5 & 1 & YES & YES & NO(2) & $1.25$ & $(10,-1)$ & 5689 & 6085\\
$(123,47)$ & 10 & $(11,3)$ & 5 & 1 & YES & YES & YES & $1.86$ & $(2,3)$ & -- & 6086\\
$(123,47)$ & 10 & $(13,3)$ & 6 & 1 & YES & YES & YES & $1.71$ & $(2,3)$ & -- & 6087\\
$(123,34)$ & 10 & $(17,5)$ & 6 & 1 & YES & YES & NO(2) & $1.56$ & $(8,0)$ & NO & 6088\\
$(123,34)$ & 10 & $(18,5)$ & 6 & 3 & YES & YES & NO(2) & $1.38$ & $(10,-1)$ & NO & 6089\\
$(123,47)$ & 10 & $(23,9)$ & 7 & 1 & YES & YES & YES & $1.75$ & $(2,3)$ & NO & 6090\\
$(123,28)$ & 12 & $(24,5)$ & 8 & 3 & YES & YES & NO(2) & $1.44$ & $(8,0)$ & NO & 6091\\
$(123,34)$ & 10 & $(32,9)$ & 8 & 1 & YES & YES & NO(2) & $1.44$ & $(8,0)$ & NO & 6092\\
$(123,47)$ & 10 & $(76,29)$ & 9 & 1 & YES & YES & NO(2) & $1.38$ & $(4,2)$ & 8307 & 6093\\
$(123,47)$ & 10 & $(89,34)$ & 9 & 1 & YES & YES & NO(2) & $1.56$ & $(4,2)$ & NO & 6094\\
$(123,47)$ & 10 & $(123,47)$ & 10 & 123 & YES & YES & YES & $1.38$ & $(2,3)$ & NO & 6095\\
$(124,29)$ & 11 & $(8,3)$ & 4 & 4 & YES & YES & NO(2) & $1.64$ & $(4,2)$ & NO & 6096\\
$(124,47)$ & 11 & $(9,4)$ & 5 & 1 & YES & YES & NO(2) & $1.78$ & $(4,2)$ & NO & 6097\\
$(124,23)$ & 12 & $(60,11)$ & 11 & 4 & YES & YES & YES & $1.62$ & $(2,3)$ & NO & 6098\\
$(125,49)$ & 11 & $(4,1)$ & 3 & 1 & YES & YES & YES & $1.29$ & $(4,2)$ & NO & 6099\\
$(125,27)$ & 11 & $(8,3)$ & 4 & 1 & YES & YES & NO(2) & $1.75$ & $(2,3)$ & NO & 6100\\
$(125,27)$ & 11 & $(8,3)$ & 4 & 1 & YES & YES & NO(2) & $1.75$ & $(2,3)$ & -- & 6101\\
$(125,37)$ & 11 & $(8,3)$ & 4 & 1 & YES & YES & YES & $1.71$ & $(2,3)$ & NO & 6102\\
$(125,37)$ & 11 & $(8,3)$ & 4 & 1 & YES & YES & YES & $1.78$ & $(2,3)$ & -- & 6103\\
$(125,49)$ & 11 & $(8,3)$ & 4 & 1 & YES & YES & YES & $1.29$ & $(4,2)$ & NO & 6104\\
$(125,27)$ & 11 & $(10,3)$ & 5 & 5 & YES & YES & NO(2) & $1.75$ & $(2,3)$ & NO & 6105\\
$(125,37)$ & 11 & $(11,2)$ & 6 & 1 & YES & YES & YES & $1.43$ & $(6,1)$ & NO & 6106\\
$(125,37)$ & 11 & $(14,3)$ & 6 & 1 & YES & YES & YES & $1.67$ & $(2,3)$ & NO & 6107\\
$(125,27)$ & 11 & $(15,4)$ & 6 & 5 & YES & YES & YES & $1.57$ & $(4,2)$ & -- & 6108\\
$(125,37)$ & 11 & $(16,5)$ & 7 & 1 & YES & YES & YES & $1.71$ & $(2,3)$ & NO & 6109\\
$(125,37)$ & 11 & $(105,31)$ & 10 & 5 & YES & YES & YES & $1.67$ & $(2,3)$ & 10612 & 6110\\
$(125,37)$ & 11 & $(115,34)$ & 10 & 5 & YES & YES & YES & $1.43$ & $(6,1)$ & 8555 & 6111\\
$(125,37)$ & 11 & $(125,37)$ & 11 & 125 & YES & YES & YES & $1.62$ & $(2,3)$ & NO & 6112\\
$(126,53)$ & 12 & $(5,1)$ & 4 & 1 & YES & YES & NO(2) & $1.75$ & $(6,1)$ & NO & 6113\\
$(126,53)$ & 12 & $(5,1)$ & 4 & 1 & YES & YES & NO(2) & $1.75$ & $(6,1)$ & -- & 6114\\
$(126,53)$ & 12 & $(6,1)$ & 5 & 6 & YES & YES & NO(2) & $1.62$ & $(6,1)$ & NO & 6115\\
$(126,53)$ & 12 & $(6,1)$ & 5 & 6 & YES & YES & NO(2) & $1.67$ & $(4,2)$ & -- & 6116\\
$(126,53)$ & 12 & $(6,1)$ & 5 & 6 & YES & YES & NO(2) & $1.67$ & $(4,2)$ & NO & 6117\\
$(126,53)$ & 12 & $(69,29)$ & 9 & 3 & YES & YES & NO(2) & $1.62$ & $(6,1)$ & NO & 6118\\
$(126,53)$ & 12 & $(88,37)$ & 10 & 2 & YES & YES & NO(2) & $1.75$ & $(6,1)$ & 7917 & 6119\\
$(127,52)$ & 12 & $(4,1)$ & 3 & 1 & YES & YES & NO(2) & $1.70$ & $(10,-1)$ & -- & 6120\\
$(127,57)$ & 11 & $(4,1)$ & 3 & 1 & YES & YES & NO(2) & $1.44$ & $(8,0)$ & -- & 6121\\
$(127,55)$ & 12 & $(5,2)$ & 3 & 1 & YES & YES & NO(2) & $1.56$ & $(8,0)$ & -- & 6122\\
$(127,56)$ & 11 & $(5,2)$ & 3 & 1 & YES & YES & YES & $1.71$ & $(2,3)$ & -- & 6123\\
$(127,47)$ & 11 & $(7,3)$ & 4 & 1 & YES & YES & NO(2) & $1.62$ & $(6,1)$ & NO & 6124\\
$(127,47)$ & 11 & $(7,3)$ & 4 & 1 & YES & YES & NO(2) & $1.62$ & $(6,1)$ & -- & 6125\\
$(127,55)$ & 12 & $(8,3)$ & 4 & 1 & YES & YES & NO(2) & $1.67$ & $(8,0)$ & NO & 6126\\
$(127,34)$ & 11 & $(11,4)$ & 5 & 1 & YES & YES & YES & $1.88$ & $(4,2)$ & -- & 6127\\
$(127,35)$ & 11 & $(13,3)$ & 6 & 1 & YES & YES & YES & $1.86$ & $(2,3)$ & -- & 6128\\
$(127,35)$ & 11 & $(14,3)$ & 6 & 1 & YES & YES & YES & $1.57$ & $(2,3)$ & -- & 6129\\
$(127,29)$ & 11 & $(15,4)$ & 6 & 1 & YES & YES & YES & $1.86$ & $(2,3)$ & -- & 6130\\
$(127,29)$ & 11 & $(17,4)$ & 7 & 1 & YES & YES & YES & $1.57$ & $(2,3)$ & -- & 6131\\
$(127,29)$ & 11 & $(33,7)$ & 8 & 1 & YES & YES & YES & $1.29$ & $(4,2)$ & NO & 6132\\
$(127,29)$ & 11 & $(37,8)$ & 8 & 1 & YES & YES & YES & $1.89$ & $(2,3)$ & NO & 6133\\
$(127,56)$ & 11 & $(41,18)$ & 8 & 1 & YES & YES & YES & $1.71$ & $(2,3)$ & 8550 & 6134\\
$(127,49)$ & 11 & $(49,19)$ & 8 & 1 & YES & YES & YES & $1.43$ & $(4,2)$ & NO & 6135\\
$(127,29)$ & 11 & $(71,16)$ & 10 & 1 & YES & YES & YES & $1.89$ & $(2,3)$ & NO & 6136\\
$(127,29)$ & 11 & $(84,19)$ & 10 & 1 & YES & YES & YES & $1.29$ & $(4,2)$ & NO & 6137\\
$(127,49)$ & 11 & $(96,37)$ & 12 & 1 & YES & YES & NO(2) & $1.70$ & $(6,1)$ & 7477 & 6138\\
$(128,49)$ & 10 & $(2,1)$ & 1 & 2 & YES & YES & YES & $1.50$ & $(2,3)$ & -- & 6139\\
$(128,47)$ & 10 & $(3,1)$ & 2 & 1 & YES & YES & NO(2) & $1.50$ & $(2,3)$ & -- & 6140\\
$(128,47)$ & 10 & $(3,1)$ & 2 & 1 & YES & YES & NO(2) & $1.70$ & $(2,3)$ & NO & 6141\\
$(128,49)$ & 10 & $(3,1)$ & 2 & 1 & YES & YES & YES & $1.50$ & $(2,3)$ & -- & 6142\\
$(128,47)$ & 10 & $(4,1)$ & 3 & 4 & YES & YES & NO(2) & $1.60$ & $(2,3)$ & NO & 6143\\
$(128,47)$ & 10 & $(5,2)$ & 3 & 1 & YES & YES & NO(2) & $1.70$ & $(6,1)$ & -- & 6144\\
$(128,49)$ & 10 & $(5,2)$ & 3 & 1 & YES & YES & YES & $1.56$ & $(2,3)$ & -- & 6145\\
$(128,53)$ & 11 & $(5,2)$ & 3 & 1 & YES & YES & NO(2) & $1.56$ & $(4,2)$ & -- & 6146\\
$(128,47)$ & 10 & $(7,3)$ & 4 & 1 & YES & YES & YES & $1.78$ & $(2,3)$ & -- & 6147\\
$(128,49)$ & 10 & $(7,2)$ & 4 & 1 & YES & YES & YES & $1.43$ & $(4,2)$ & NO & 6148\\
$(128,49)$ & 10 & $(7,2)$ & 4 & 1 & YES & YES & YES & $1.50$ & $(4,2)$ & -- & 6149\\
$(128,49)$ & 10 & $(7,3)$ & 4 & 1 & YES & YES & NO(2) & $1.67$ & $(2,3)$ & -- & 6150\\
$(128,47)$ & 10 & $(8,3)$ & 4 & 8 & YES & YES & NO(2) & $1.50$ & $(2,3)$ & NO & 6151\\
$(128,47)$ & 10 & $(8,3)$ & 4 & 8 & YES & YES & YES & $1.78$ & $(2,3)$ & -- & 6152\\
$(128,49)$ & 10 & $(8,3)$ & 4 & 8 & YES & YES & YES & $1.43$ & $(4,2)$ & -- & 6153\\
$(128,49)$ & 10 & $(8,3)$ & 4 & 8 & YES & YES & YES & $1.75$ & $(2,3)$ & NO & 6154\\
$(128,53)$ & 11 & $(8,3)$ & 4 & 8 & YES & YES & NO(2) & $1.56$ & $(4,2)$ & NO & 6155\\
$(128,47)$ & 10 & $(9,2)$ & 5 & 1 & YES & YES & NO(2) & $1.60$ & $(6,1)$ & -- & 6156\\
$(128,49)$ & 10 & $(9,2)$ & 5 & 1 & YES & YES & YES & $1.67$ & $(2,3)$ & -- & 6157\\
$(128,49)$ & 10 & $(9,4)$ & 5 & 1 & YES & YES & YES & $1.43$ & $(4,2)$ & NO & 6158\\
$(128,47)$ & 10 & $(10,3)$ & 5 & 2 & YES & YES & NO(2) & $1.62$ & $(6,1)$ & NO & 6159\\
$(128,47)$ & 10 & $(10,3)$ & 5 & 2 & YES & YES & YES & $1.29$ & $(4,2)$ & -- & 6160\\
$(128,49)$ & 10 & $(10,3)$ & 5 & 2 & YES & YES & YES & $1.57$ & $(2,3)$ & -- & 6161\\
$(128,49)$ & 10 & $(11,3)$ & 5 & 1 & YES & YES & YES & $1.86$ & $(2,3)$ & -- & 6162\\
$(128,49)$ & 10 & $(11,4)$ & 5 & 1 & YES & YES & NO(2) & $1.50$ & $(6,1)$ & NO & 6163\\
$(128,47)$ & 10 & $(12,5)$ & 5 & 4 & YES & YES & YES & $1.62$ & $(2,3)$ & NO & 6164\\
$(128,47)$ & 10 & $(13,3)$ & 6 & 1 & YES & YES & YES & $1.67$ & $(2,3)$ & -- & 6165\\
$(128,49)$ & 10 & $(13,3)$ & 6 & 1 & YES & YES & YES & $1.78$ & $(4,2)$ & -- & 6166\\
$(128,47)$ & 10 & $(18,7)$ & 6 & 2 & YES & YES & YES & $1.78$ & $(2,3)$ & NO & 6167\\
$(128,47)$ & 10 & $(19,7)$ & 6 & 1 & YES & YES & NO(2) & $1.50$ & $(2,3)$ & NO & 6168\\
$(128,49)$ & 10 & $(21,8)$ & 6 & 1 & YES & YES & NO(2) & $1.60$ & $(2,3)$ & 6533 & 6169\\
$(128,47)$ & 10 & $(27,10)$ & 7 & 1 & YES & YES & NO(2) & $1.56$ & $(4,2)$ & NO & 6170\\
$(128,49)$ & 10 & $(34,13)$ & 7 & 2 & YES & YES & NO(2) & $1.60$ & $(2,3)$ & 5830 & 6171\\
$(128,49)$ & 10 & $(55,21)$ & 8 & 1 & YES & YES & YES & $1.62$ & $(2,3)$ & NO & 6172\\
$(128,49)$ & 10 & $(76,29)$ & 9 & 4 & YES & YES & YES & $1.43$ & $(4,2)$ & NO & 6173\\
$(128,47)$ & 10 & $(109,40)$ & 10 & 1 & YES & YES & NO(2) & $1.70$ & $(6,1)$ & NO & 6174\\
$(128,47)$ & 10 & $(117,43)$ & 10 & 1 & YES & YES & YES & $1.57$ & $(4,2)$ & NO & 6175\\
$(128,47)$ & 10 & $(128,47)$ & 10 & 128 & YES & YES & NO(2) & $1.60$ & $(2,3)$ & NO & 6176\\
$(129,49)$ & 10 & $(2,1)$ & 1 & 1 & YES & YES & NO(2) & $1.60$ & $(2,3)$ & -- & 6177\\
$(129,49)$ & 10 & $(2,1)$ & 1 & 1 & YES & YES & NO(2) & $1.50$ & $(2,3)$ & NO & 6178\\
$(129,50)$ & 10 & $(2,1)$ & 1 & 1 & YES & YES & NO(2) & $1.60$ & $(6,1)$ & -- & 6179\\
$(129,49)$ & 10 & $(3,1)$ & 2 & 3 & YES & YES & NO(2) & $1.60$ & $(2,3)$ & -- & 6180\\
$(129,49)$ & 10 & $(4,1)$ & 3 & 1 & YES & YES & NO(2) & $1.44$ & $(4,2)$ & NO & 6181\\
$(129,49)$ & 10 & $(4,1)$ & 3 & 1 & YES & YES & NO(2) & $1.44$ & $(4,2)$ & -- & 6182\\
$(129,50)$ & 10 & $(4,1)$ & 3 & 1 & YES & YES & YES & $1.50$ & $(2,3)$ & -- & 6183\\
$(129,49)$ & 10 & $(5,2)$ & 3 & 1 & YES & YES & NO(2) & $1.56$ & $(4,2)$ & -- & 6184\\
$(129,50)$ & 10 & $(5,2)$ & 3 & 1 & YES & YES & YES & $1.50$ & $(2,3)$ & -- & 6185\\
$(129,49)$ & 10 & $(7,2)$ & 4 & 1 & YES & YES & NO(2) & $1.44$ & $(2,3)$ & -- & 6186\\
$(129,49)$ & 10 & $(7,2)$ & 4 & 1 & YES & YES & YES & $1.75$ & $(2,3)$ & NO & 6187\\
$(129,49)$ & 10 & $(7,3)$ & 4 & 1 & YES & YES & YES & $1.75$ & $(2,3)$ & -- & 6188\\
$(129,50)$ & 10 & $(7,2)$ & 4 & 1 & YES & YES & YES & $1.78$ & $(2,3)$ & -- & 6189\\
$(129,50)$ & 10 & $(7,2)$ & 4 & 1 & YES & YES & YES & $1.62$ & $(2,3)$ & NO & 6190\\
$(129,50)$ & 10 & $(7,3)$ & 4 & 1 & YES & YES & YES & $1.89$ & $(2,3)$ & -- & 6191\\
$(129,49)$ & 10 & $(8,3)$ & 4 & 1 & YES & YES & YES & $1.75$ & $(4,2)$ & -- & 6192\\
$(129,49)$ & 10 & $(8,3)$ & 4 & 1 & YES & YES & YES & $1.67$ & $(2,3)$ & NO & 6193\\
$(129,49)$ & 10 & $(10,3)$ & 5 & 1 & YES & YES & YES & $1.62$ & $(4,2)$ & -- & 6194\\
$(129,49)$ & 10 & $(11,3)$ & 5 & 1 & YES & YES & YES & $1.57$ & $(4,2)$ & NO & 6195\\
$(129,59)$ & 12 & $(11,2)$ & 6 & 1 & YES & YES & NO(2) & $1.70$ & $(6,1)$ & -- & 6196\\
$(129,29)$ & 12 & $(12,5)$ & 5 & 3 & YES & YES & YES & $1.78$ & $(4,2)$ & -- & 6197\\
$(129,29)$ & 12 & $(13,5)$ & 5 & 1 & YES & YES & YES & $1.78$ & $(4,2)$ & -- & 6198\\
$(129,49)$ & 10 & $(13,3)$ & 6 & 1 & YES & YES & YES & $1.62$ & $(4,2)$ & -- & 6199\\
$(129,49)$ & 10 & $(13,5)$ & 5 & 1 & YES & YES & NO(2) & $1.70$ & $(2,3)$ & NO & 6200\\
$(129,49)$ & 10 & $(18,7)$ & 6 & 3 & YES & YES & NO(2) & $1.56$ & $(2,3)$ & NO & 6201\\
$(129,50)$ & 10 & $(21,8)$ & 6 & 3 & YES & YES & YES & $1.90$ & $(2,3)$ & NO & 6202\\
$(129,49)$ & 10 & $(23,9)$ & 7 & 1 & YES & YES & YES & $1.89$ & $(2,3)$ & 5034 & 6203\\
$(129,50)$ & 10 & $(34,13)$ & 7 & 1 & YES & YES & YES & $1.67$ & $(2,3)$ & NO & 6204\\
$(129,49)$ & 10 & $(37,14)$ & 8 & 1 & YES & YES & NO(2) & $1.67$ & $(2,3)$ & NO & 6205\\
$(129,49)$ & 10 & $(47,18)$ & 8 & 1 & YES & YES & YES & $1.62$ & $(4,2)$ & NO & 6206\\
$(129,50)$ & 10 & $(57,22)$ & 9 & 3 & YES & YES & YES & $1.89$ & $(2,3)$ & NO & 6207\\
$(129,49)$ & 10 & $(71,27)$ & 9 & 1 & YES & YES & NO(2) & $1.56$ & $(4,2)$ & NO & 6208\\
$(129,49)$ & 10 & $(79,30)$ & 9 & 1 & YES & YES & NO(2) & $1.56$ & $(4,2)$ & NO & 6209\\
$(129,50)$ & 10 & $(111,43)$ & 10 & 3 & YES & YES & YES & $1.50$ & $(2,3)$ & NO & 6210\\
$(129,50)$ & 10 & $(129,50)$ & 10 & 129 & YES & YES & YES & $1.50$ & $(2,3)$ & NO & 6211\\
$(130,47)$ & 11 & $(3,1)$ & 2 & 1 & YES & YES & NO(2) & $1.80$ & $(2,3)$ & -- & 6212\\
$(130,51)$ & 11 & $(5,2)$ & 3 & 5 & YES & YES & NO(2) & $1.70$ & $(10,-1)$ & -- & 6213\\
$(130,57)$ & 11 & $(98,43)$ & 10 & 2 & YES & YES & NO(2) & $1.67$ & $(4,2)$ & NO & 6214\\
$(130,47)$ & 11 & $(130,47)$ & 11 & 130 & YES & YES & NO(2) & $1.67$ & $(4,2)$ & NO & 6215\\
$(131,50)$ & 10 & $(2,1)$ & 1 & 1 & YES & YES & NO(2) & $1.67$ & $(2,3)$ & -- & 6216\\
$(131,55)$ & 10 & $(2,1)$ & 1 & 1 & YES & YES & NO(2) & $1.50$ & $(6,1)$ & -- & 6217\\
$(131,47)$ & 11 & $(3,1)$ & 2 & 1 & YES & YES & NO(2) & $1.38$ & $(6,1)$ & -- & 6218\\
$(131,50)$ & 10 & $(3,1)$ & 2 & 1 & YES & YES & NO(2) & $1.50$ & $(2,3)$ & NO & 6219\\
$(131,50)$ & 10 & $(3,1)$ & 2 & 1 & YES & YES & YES & $1.38$ & $(2,3)$ & -- & 6220\\
$(131,55)$ & 10 & $(3,1)$ & 2 & 1 & YES & YES & NO(2) & $1.50$ & $(2,3)$ & 6511 & 6221\\
$(131,55)$ & 10 & $(3,1)$ & 2 & 1 & YES & YES & NO(2) & $1.50$ & $(2,3)$ & -- & 6222\\
$(131,36)$ & 11 & $(4,1)$ & 3 & 1 & YES & YES & NO(2) & $1.38$ & $(10,-1)$ & -- & 6223\\
$(131,36)$ & 11 & $(4,1)$ & 3 & 1 & YES & YES & NO(2) & $1.50$ & $(10,-1)$ & NO & 6224\\
$(131,36)$ & 11 & $(5,2)$ & 3 & 1 & YES & YES & NO(2) & $1.56$ & $(4,2)$ & NO & 6225\\
$(131,36)$ & 11 & $(5,2)$ & 3 & 1 & YES & YES & NO(2) & $1.56$ & $(4,2)$ & -- & 6226\\
$(131,36)$ & 11 & $(5,2)$ & 3 & 1 & YES & YES & NO(2) & $1.60$ & $(2,3)$ & NO & 6227\\
$(131,39)$ & 11 & $(5,2)$ & 3 & 1 & YES & YES & NO(2) & $1.67$ & $(4,2)$ & -- & 6228\\
$(131,40)$ & 11 & $(5,2)$ & 3 & 1 & YES & YES & NO(2) & $1.67$ & $(4,2)$ & NO & 6229\\
$(131,40)$ & 11 & $(5,2)$ & 3 & 1 & YES & YES & NO(2) & $1.67$ & $(4,2)$ & -- & 6230\\
$(131,50)$ & 10 & $(5,1)$ & 4 & 1 & YES & YES & YES & $1.50$ & $(2,3)$ & -- & 6231\\
$(131,50)$ & 10 & $(5,2)$ & 3 & 1 & YES & YES & NO(2) & $1.25$ & $(4,2)$ & -- & 6232\\
$(131,50)$ & 10 & $(5,2)$ & 3 & 1 & YES & YES & YES & $1.57$ & $(4,2)$ & NO & 6233\\
$(131,55)$ & 10 & $(5,2)$ & 3 & 1 & YES & YES & NO(2) & $1.60$ & $(2,3)$ & 6013 & 6234\\
$(131,55)$ & 10 & $(5,2)$ & 3 & 1 & YES & YES & NO(2) & $1.64$ & $(4,2)$ & -- & 6235\\
$(131,39)$ & 11 & $(7,2)$ & 4 & 1 & YES & YES & NO(2) & $1.25$ & $(6,1)$ & -- & 6236\\
$(131,50)$ & 10 & $(7,2)$ & 4 & 1 & YES & YES & YES & $1.75$ & $(2,3)$ & -- & 6237\\
$(131,50)$ & 10 & $(7,2)$ & 4 & 1 & YES & YES & YES & $1.88$ & $(6,1)$ & NO & 6238\\
$(131,50)$ & 10 & $(7,3)$ & 4 & 1 & YES & YES & NO(2) & $1.67$ & $(4,2)$ & NO & 6239\\
$(131,50)$ & 10 & $(7,3)$ & 4 & 1 & YES & YES & YES & $1.43$ & $(4,2)$ & -- & 6240\\
$(131,55)$ & 10 & $(7,2)$ & 4 & 1 & YES & YES & YES & $1.29$ & $(6,1)$ & -- & 6241\\
$(131,55)$ & 10 & $(7,2)$ & 4 & 1 & YES & YES & YES & $1.43$ & $(6,1)$ & NO & 6242\\
$(131,55)$ & 10 & $(7,2)$ & 4 & 1 & YES & YES & YES & $1.62$ & $(4,2)$ & NO & 6243\\
$(131,55)$ & 10 & $(7,3)$ & 4 & 1 & YES & YES & YES & $1.50$ & $(2,3)$ & NO & 6244\\
$(131,55)$ & 10 & $(7,3)$ & 4 & 1 & YES & YES & YES & $1.75$ & $(2,3)$ & -- & 6245\\
$(131,50)$ & 10 & $(8,3)$ & 4 & 1 & YES & YES & YES & $1.75$ & $(2,3)$ & -- & 6246\\
$(131,50)$ & 10 & $(8,3)$ & 4 & 1 & YES & YES & YES & $1.75$ & $(2,3)$ & NO & 6247\\
$(131,55)$ & 10 & $(8,3)$ & 4 & 1 & YES & YES & NO(2) & $1.67$ & $(2,3)$ & NO & 6248\\
$(131,55)$ & 10 & $(8,3)$ & 4 & 1 & YES & YES & YES & $1.43$ & $(4,2)$ & -- & 6249\\
$(131,36)$ & 11 & $(9,4)$ & 5 & 1 & YES & YES & YES & $1.78$ & $(4,2)$ & -- & 6250\\
$(131,40)$ & 11 & $(9,2)$ & 5 & 1 & YES & YES & NO(2) & $1.56$ & $(2,3)$ & NO & 6251\\
$(131,50)$ & 10 & $(9,2)$ & 5 & 1 & YES & YES & YES & $1.71$ & $(6,1)$ & NO & 6252\\
$(131,50)$ & 10 & $(9,2)$ & 5 & 1 & YES & YES & YES & $1.71$ & $(6,1)$ & -- & 6253\\
$(131,50)$ & 10 & $(9,4)$ & 5 & 1 & YES & YES & YES & $1.62$ & $(2,3)$ & NO & 6254\\
$(131,55)$ & 10 & $(9,2)$ & 5 & 1 & YES & YES & YES & $1.50$ & $(4,2)$ & 8299 & 6255\\
$(131,55)$ & 10 & $(9,2)$ & 5 & 1 & YES & YES & YES & $1.75$ & $(2,3)$ & -- & 6256\\
$(131,36)$ & 11 & $(10,3)$ & 5 & 1 & YES & YES & NO(2) & $1.60$ & $(2,3)$ & NO & 6257\\
$(131,40)$ & 11 & $(10,3)$ & 5 & 1 & YES & YES & YES & $1.67$ & $(4,2)$ & -- & 6258\\
$(131,50)$ & 10 & $(10,3)$ & 5 & 1 & YES & YES & YES & $1.62$ & $(4,2)$ & -- & 6259\\
$(131,38)$ & 12 & $(11,3)$ & 5 & 1 & YES & YES & YES & $1.57$ & $(6,1)$ & -- & 6260\\
$(131,39)$ & 11 & $(11,3)$ & 5 & 1 & YES & YES & NO(2) & $1.60$ & $(2,3)$ & NO & 6261\\
$(131,40)$ & 11 & $(11,3)$ & 5 & 1 & YES & YES & NO(2) & $1.56$ & $(2,3)$ & NO & 6262\\
$(131,50)$ & 10 & $(11,3)$ & 5 & 1 & YES & YES & YES & $1.67$ & $(2,3)$ & NO & 6263\\
$(131,55)$ & 10 & $(11,3)$ & 5 & 1 & YES & YES & YES & $1.57$ & $(2,3)$ & NO & 6264\\
$(131,50)$ & 10 & $(13,3)$ & 6 & 1 & YES & YES & YES & $1.75$ & $(2,3)$ & -- & 6265\\
$(131,50)$ & 10 & $(13,3)$ & 6 & 1 & YES & YES & YES & $1.78$ & $(2,3)$ & NO & 6266\\
$(131,55)$ & 10 & $(13,5)$ & 5 & 1 & YES & YES & YES & $1.62$ & $(2,3)$ & NO & 6267\\
$(131,50)$ & 10 & $(14,5)$ & 6 & 1 & YES & YES & YES & $1.57$ & $(4,2)$ & NO & 6268\\
$(131,40)$ & 11 & $(16,5)$ & 7 & 1 & YES & YES & NO(2) & $1.67$ & $(4,2)$ & NO & 6269\\
$(131,55)$ & 10 & $(19,8)$ & 6 & 1 & YES & YES & NO(2) & $1.50$ & $(2,3)$ & NO & 6270\\
$(131,39)$ & 11 & $(23,7)$ & 7 & 1 & YES & YES & NO(2) & $1.62$ & $(6,1)$ & NO & 6271\\
$(131,39)$ & 11 & $(24,7)$ & 7 & 1 & YES & YES & NO(2) & $1.56$ & $(4,2)$ & NO & 6272\\
$(131,36)$ & 11 & $(25,7)$ & 7 & 1 & YES & YES & NO(2) & $1.56$ & $(4,2)$ & NO & 6273\\
$(131,36)$ & 11 & $(29,8)$ & 7 & 1 & YES & YES & NO(2) & $1.50$ & $(10,-1)$ & NO & 6274\\
$(131,55)$ & 10 & $(31,13)$ & 7 & 1 & YES & YES & NO(2) & $1.50$ & $(2,3)$ & 5775 & 6275\\
$(131,40)$ & 11 & $(33,10)$ & 8 & 1 & YES & YES & NO(2) & $1.38$ & $(6,1)$ & 7785 & 6276\\
$(131,55)$ & 10 & $(43,18)$ & 8 & 1 & YES & YES & NO(2) & $1.67$ & $(2,3)$ & NO & 6277\\
$(131,50)$ & 10 & $(50,19)$ & 8 & 1 & YES & YES & YES & $1.57$ & $(4,2)$ & NO & 6278\\
$(131,55)$ & 10 & $(55,23)$ & 9 & 1 & YES & YES & YES & $1.43$ & $(4,2)$ & NO & 6279\\
$(131,50)$ & 10 & $(60,23)$ & 9 & 1 & YES & YES & YES & $1.43$ & $(4,2)$ & 5622 & 6280\\
$(131,39)$ & 11 & $(64,19)$ & 9 & 1 & YES & YES & NO(2) & $1.64$ & $(4,2)$ & NO & 6281\\
$(131,55)$ & 10 & $(69,29)$ & 9 & 1 & YES & YES & NO(2) & $1.56$ & $(2,3)$ & NO & 6282\\
$(131,50)$ & 10 & $(76,29)$ & 9 & 1 & YES & YES & NO(2) & $1.56$ & $(4,2)$ & NO & 6283\\
$(131,50)$ & 10 & $(81,31)$ & 9 & 1 & YES & YES & YES & $1.80$ & $(2,3)$ & NO & 6284\\
$(131,55)$ & 10 & $(88,37)$ & 10 & 1 & YES & YES & YES & $1.89$ & $(2,3)$ & NO & 6285\\
$(131,39)$ & 11 & $(104,31)$ & 11 & 1 & YES & YES & YES & $1.57$ & $(2,3)$ & NO & 6286\\
$(131,48)$ & 11 & $(112,41)$ & 10 & 1 & YES & YES & YES & $1.43$ & $(4,2)$ & 8759 & 6287\\
$(131,55)$ & 10 & $(112,47)$ & 10 & 1 & YES & YES & NO(2) & $1.64$ & $(4,2)$ & NO & 6288\\
$(131,50)$ & 10 & $(123,47)$ & 10 & 1 & YES & YES & YES & $1.67$ & $(2,3)$ & NO & 6289\\
$(132,49)$ & 11 & $(5,2)$ & 3 & 1 & YES & YES & NO(2) & $1.56$ & $(8,0)$ & -- & 6290\\
$(132,49)$ & 11 & $(89,33)$ & 10 & 1 & YES & YES & NO(2) & $1.67$ & $(8,0)$ & 8714 & 6291\\
$(132,35)$ & 11 & $(117,31)$ & 11 & 3 & YES & YES & NO(2) & $1.44$ & $(8,0)$ & NO & 6292\\
$(133,36)$ & 11 & $(3,1)$ & 2 & 1 & YES & YES & NO(2) & $1.56$ & $(4,2)$ & NO & 6293\\
$(133,36)$ & 11 & $(3,1)$ & 2 & 1 & YES & YES & NO(2) & $1.56$ & $(4,2)$ & -- & 6294\\
$(133,36)$ & 11 & $(3,1)$ & 2 & 1 & YES & YES & NO(2) & $1.70$ & $(2,3)$ & NO & 6295\\
$(133,39)$ & 11 & $(3,1)$ & 2 & 1 & YES & YES & NO(2) & $1.56$ & $(8,0)$ & NO & 6296\\
$(133,39)$ & 11 & $(3,1)$ & 2 & 1 & YES & YES & NO(2) & $1.56$ & $(8,0)$ & -- & 6297\\
$(133,58)$ & 11 & $(3,1)$ & 2 & 1 & YES & YES & NO(2) & $1.56$ & $(4,2)$ & -- & 6298\\
$(133,60)$ & 11 & $(3,1)$ & 2 & 1 & YES & YES & NO(2) & $1.80$ & $(6,1)$ & NO & 6299\\
$(133,60)$ & 11 & $(3,1)$ & 2 & 1 & YES & YES & NO(2) & $1.80$ & $(6,1)$ & -- & 6300\\
$(133,51)$ & 11 & $(4,1)$ & 3 & 1 & YES & YES & NO(2) & $1.67$ & $(4,2)$ & -- & 6301\\
$(133,60)$ & 11 & $(5,1)$ & 4 & 1 & YES & YES & NO(2) & $1.60$ & $(6,1)$ & NO & 6302\\
$(133,60)$ & 11 & $(5,1)$ & 4 & 1 & YES & YES & NO(2) & $1.60$ & $(6,1)$ & -- & 6303\\
$(133,39)$ & 11 & $(7,2)$ & 4 & 7 & YES & YES & YES & $1.71$ & $(6,1)$ & -- & 6304\\
$(133,39)$ & 11 & $(7,2)$ & 4 & 7 & YES & YES & YES & $1.86$ & $(6,1)$ & NO & 6305\\
$(133,58)$ & 11 & $(7,2)$ & 4 & 7 & YES & YES & YES & $1.50$ & $(4,2)$ & -- & 6306\\
$(133,39)$ & 11 & $(8,3)$ & 4 & 1 & YES & YES & YES & $1.70$ & $(2,3)$ & -- & 6307\\
$(133,48)$ & 11 & $(8,3)$ & 4 & 1 & YES & YES & YES & $1.90$ & $(2,3)$ & -- & 6308\\
$(133,51)$ & 11 & $(9,2)$ & 5 & 1 & YES & YES & YES & $1.57$ & $(4,2)$ & NO & 6309\\
$(133,51)$ & 11 & $(9,2)$ & 5 & 1 & YES & YES & YES & $1.57$ & $(6,1)$ & -- & 6310\\
$(133,30)$ & 12 & $(17,3)$ & 7 & 1 & YES & YES & NO(2) & $1.56$ & $(8,0)$ & NO & 6311\\
$(133,36)$ & 11 & $(23,6)$ & 8 & 1 & YES & YES & NO(2) & $1.56$ & $(8,0)$ & NO & 6312\\
$(133,36)$ & 11 & $(26,7)$ & 7 & 1 & YES & YES & NO(2) & $1.70$ & $(2,3)$ & 7009 & 6313\\
$(133,30)$ & 12 & $(53,12)$ & 9 & 1 & YES & YES & NO(2) & $1.44$ & $(8,0)$ & NO & 6314\\
$(133,51)$ & 11 & $(55,21)$ & 8 & 1 & YES & YES & YES & $1.43$ & $(4,2)$ & NO & 6315\\
$(133,51)$ & 11 & $(81,31)$ & 9 & 1 & YES & YES & YES & $1.57$ & $(4,2)$ & NO & 6316\\
$(133,36)$ & 11 & $(100,27)$ & 10 & 1 & YES & YES & NO(2) & $1.38$ & $(4,2)$ & NO & 6317\\
$(133,39)$ & 11 & $(106,31)$ & 10 & 1 & YES & YES & YES & $1.67$ & $(2,3)$ & NO & 6318\\
$(133,48)$ & 11 & $(111,40)$ & 11 & 1 & YES & YES & YES & $2.00$ & $(2,3)$ & NO & 6319\\
$(133,36)$ & 11 & $(133,36)$ & 11 & 133 & YES & YES & NO(2) & $1.60$ & $(2,3)$ & NO & 6320\\
$(133,48)$ & 11 & $(133,48)$ & 11 & 133 & YES & YES & NO(2) & $1.50$ & $(6,1)$ & NO & 6321\\
$(134,39)$ & 11 & $(2,1)$ & 1 & 2 & YES & YES & NO(2) & $1.50$ & $(2,3)$ & -- & 6322\\
$(134,49)$ & 11 & $(3,1)$ & 2 & 1 & YES & YES & NO(2) & $1.50$ & $(6,1)$ & -- & 6323\\
$(134,49)$ & 11 & $(5,2)$ & 3 & 1 & YES & YES & NO(2) & $1.56$ & $(8,0)$ & -- & 6324\\
$(134,49)$ & 11 & $(5,2)$ & 3 & 1 & YES & YES & NO(2) & $1.73$ & $(8,0)$ & NO & 6325\\
$(134,39)$ & 11 & $(7,2)$ & 4 & 1 & YES & YES & YES & $1.71$ & $(4,2)$ & NO & 6326\\
$(134,37)$ & 11 & $(8,3)$ & 4 & 2 & YES & YES & YES & $1.67$ & $(2,3)$ & -- & 6327\\
$(134,29)$ & 11 & $(10,3)$ & 5 & 2 & YES & YES & NO(2) & $1.25$ & $(4,2)$ & NO & 6328\\
$(134,39)$ & 11 & $(10,3)$ & 5 & 2 & YES & YES & NO(2) & $1.73$ & $(4,2)$ & NO & 6329\\
$(134,39)$ & 11 & $(10,3)$ & 5 & 2 & YES & YES & YES & $1.50$ & $(4,2)$ & -- & 6330\\
$(134,55)$ & 11 & $(10,3)$ & 5 & 2 & YES & YES & YES & $1.62$ & $(4,2)$ & -- & 6331\\
$(134,39)$ & 11 & $(12,5)$ & 5 & 2 & YES & YES & YES & $1.75$ & $(4,2)$ & NO & 6332\\
$(134,55)$ & 11 & $(12,5)$ & 5 & 2 & YES & YES & NO(2) & $1.73$ & $(4,2)$ & 5525 & 6333\\
$(134,37)$ & 11 & $(14,3)$ & 6 & 2 & YES & YES & YES & $1.75$ & $(2,3)$ & NO & 6334\\
$(134,37)$ & 11 & $(16,3)$ & 7 & 2 & YES & YES & YES & $1.57$ & $(2,3)$ & -- & 6335\\
$(134,39)$ & 11 & $(44,13)$ & 8 & 2 & YES & YES & YES & $1.89$ & $(2,3)$ & NO & 6336\\
$(134,55)$ & 11 & $(56,23)$ & 9 & 2 & YES & YES & YES & $1.43$ & $(2,3)$ & 6696 & 6337\\
$(134,49)$ & 11 & $(68,25)$ & 9 & 2 & YES & YES & YES & $1.62$ & $(4,2)$ & NO & 6338\\
$(134,37)$ & 11 & $(112,31)$ & 10 & 2 & YES & YES & YES & $1.56$ & $(2,3)$ & 10920 & 6339\\
$(134,39)$ & 11 & $(134,39)$ & 11 & 134 & YES & YES & NO(2) & $1.73$ & $(4,2)$ & NO & 6340\\
$(135,32)$ & 12 & $(5,2)$ & 3 & 5 & YES & YES & NO(2) & $1.56$ & $(4,2)$ & NO & 6341\\
$(135,41)$ & 11 & $(5,2)$ & 3 & 5 & YES & YES & NO(2) & $1.56$ & $(2,3)$ & -- & 6342\\
$(135,41)$ & 11 & $(5,2)$ & 3 & 5 & YES & YES & NO(2) & $1.50$ & $(4,2)$ & NO & 6343\\
$(135,56)$ & 11 & $(5,2)$ & 3 & 5 & YES & YES & NO(2) & $1.78$ & $(2,3)$ & -- & 6344\\
$(135,41)$ & 11 & $(7,3)$ & 4 & 1 & YES & YES & YES & $1.80$ & $(2,3)$ & -- & 6345\\
$(135,41)$ & 11 & $(8,3)$ & 4 & 1 & YES & YES & YES & $1.80$ & $(2,3)$ & -- & 6346\\
$(135,41)$ & 11 & $(10,3)$ & 5 & 5 & YES & YES & YES & $1.67$ & $(2,3)$ & -- & 6347\\
$(135,38)$ & 12 & $(11,3)$ & 5 & 1 & YES & YES & YES & $1.57$ & $(6,1)$ & -- & 6348\\
$(135,56)$ & 11 & $(19,8)$ & 6 & 1 & YES & YES & YES & $1.62$ & $(2,3)$ & NO & 6349\\
$(135,56)$ & 11 & $(26,11)$ & 7 & 1 & YES & YES & NO(2) & $1.50$ & $(6,1)$ & NO & 6350\\
$(135,32)$ & 12 & $(35,8)$ & 8 & 5 & YES & YES & YES & $1.43$ & $(4,2)$ & NO & 6351\\
$(135,41)$ & 11 & $(56,17)$ & 9 & 1 & YES & YES & NO(2) & $1.64$ & $(4,2)$ & NO & 6352\\
$(135,38)$ & 12 & $(61,17)$ & 9 & 1 & YES & YES & YES & $1.43$ & $(6,1)$ & NO & 6353\\
$(135,41)$ & 11 & $(102,31)$ & 11 & 3 & YES & YES & NO(2) & $1.44$ & $(8,0)$ & NO & 6354\\
$(136,59)$ & 11 & $(4,1)$ & 3 & 4 & YES & YES & NO(2) & $1.38$ & $(6,1)$ & -- & 6355\\
$(136,59)$ & 11 & $(5,2)$ & 3 & 1 & YES & YES & NO(2) & $1.78$ & $(4,2)$ & -- & 6356\\
$(136,31)$ & 11 & $(7,3)$ & 4 & 1 & YES & YES & NO(2) & $1.67$ & $(4,2)$ & NO & 6357\\
$(136,31)$ & 11 & $(7,3)$ & 4 & 1 & YES & YES & NO(2) & $1.67$ & $(4,2)$ & -- & 6358\\
$(136,59)$ & 11 & $(9,4)$ & 5 & 1 & YES & YES & NO(2) & $1.38$ & $(6,1)$ & 5122 & 6359\\
$(136,59)$ & 11 & $(10,3)$ & 5 & 2 & YES & YES & YES & $1.43$ & $(6,1)$ & -- & 6360\\
$(136,59)$ & 11 & $(10,3)$ & 5 & 2 & YES & YES & YES & $1.57$ & $(6,1)$ & NO & 6361\\
$(136,31)$ & 11 & $(11,4)$ & 5 & 1 & YES & YES & YES & $1.50$ & $(4,2)$ & -- & 6362\\
$(136,41)$ & 13 & $(13,2)$ & 7 & 1 & YES & YES & NO(2) & $1.38$ & $(6,1)$ & -- & 6363\\
$(136,59)$ & 11 & $(34,15)$ & 8 & 34 & YES & YES & YES & $1.57$ & $(6,1)$ & 5124 & 6364\\
$(136,41)$ & 13 & $(36,11)$ & 8 & 4 & YES & YES & NO(2) & $1.50$ & $(6,1)$ & NO & 6365\\
$(136,49)$ & 12 & $(36,13)$ & 8 & 4 & YES & YES & YES & $1.50$ & $(2,3)$ & NO & 6366\\
$(137,37)$ & 11 & $(2,1)$ & 1 & 1 & YES & YES & NO(2) & $1.50$ & $(2,3)$ & -- & 6367\\
$(137,37)$ & 11 & $(3,1)$ & 2 & 1 & YES & YES & NO(2) & $1.50$ & $(2,3)$ & -- & 6368\\
$(137,40)$ & 12 & $(3,1)$ & 2 & 1 & YES & YES & NO(2) & $1.56$ & $(4,2)$ & -- & 6369\\
$(137,43)$ & 12 & $(3,1)$ & 2 & 1 & YES & YES & NO(2) & $1.62$ & $(6,1)$ & NO & 6370\\
$(137,43)$ & 12 & $(3,1)$ & 2 & 1 & YES & YES & NO(2) & $1.67$ & $(8,0)$ & -- & 6371\\
$(137,52)$ & 11 & $(3,1)$ & 2 & 1 & YES & YES & NO(2) & $1.33$ & $(4,2)$ & -- & 6372\\
$(137,42)$ & 12 & $(5,2)$ & 3 & 1 & YES & YES & NO(2) & $1.56$ & $(12,-2)$ & -- & 6373\\
$(137,53)$ & 11 & $(5,2)$ & 3 & 1 & YES & YES & YES & $1.80$ & $(2,3)$ & -- & 6374\\
$(137,30)$ & 12 & $(7,3)$ & 4 & 1 & YES & YES & NO(2) & $1.38$ & $(4,2)$ & -- & 6375\\
$(137,53)$ & 11 & $(7,2)$ & 4 & 1 & YES & YES & YES & $1.60$ & $(2,3)$ & -- & 6376\\
$(137,31)$ & 11 & $(9,4)$ & 5 & 1 & YES & YES & YES & $1.62$ & $(4,2)$ & -- & 6377\\
$(137,52)$ & 11 & $(9,2)$ & 5 & 1 & YES & YES & YES & $1.43$ & $(4,2)$ & -- & 6378\\
$(137,53)$ & 11 & $(9,2)$ & 5 & 1 & YES & YES & YES & $1.60$ & $(2,3)$ & -- & 6379\\
$(137,31)$ & 11 & $(11,4)$ & 5 & 1 & YES & YES & YES & $1.62$ & $(4,2)$ & -- & 6380\\
$(137,37)$ & 11 & $(11,3)$ & 5 & 1 & YES & YES & YES & $1.62$ & $(4,2)$ & -- & 6381\\
$(137,53)$ & 11 & $(11,4)$ & 5 & 1 & YES & YES & NO(2) & $1.60$ & $(6,1)$ & NO & 6382\\
$(137,37)$ & 11 & $(12,5)$ & 5 & 1 & YES & YES & YES & $1.89$ & $(4,2)$ & -- & 6383\\
$(137,53)$ & 11 & $(12,5)$ & 5 & 1 & YES & YES & NO(2) & $1.67$ & $(4,2)$ & NO & 6384\\
$(137,37)$ & 11 & $(13,3)$ & 6 & 1 & YES & YES & YES & $1.50$ & $(4,2)$ & -- & 6385\\
$(137,43)$ & 12 & $(13,4)$ & 6 & 1 & YES & YES & NO(2) & $1.67$ & $(8,0)$ & NO & 6386\\
$(137,37)$ & 11 & $(15,4)$ & 6 & 1 & YES & YES & NO(2) & $1.50$ & $(2,3)$ & NO & 6387\\
$(137,37)$ & 11 & $(26,7)$ & 7 & 1 & YES & YES & NO(2) & $1.50$ & $(2,3)$ & NO & 6388\\
$(137,30)$ & 12 & $(33,7)$ & 8 & 1 & YES & YES & NO(2) & $1.38$ & $(4,2)$ & NO & 6389\\
$(137,52)$ & 11 & $(45,17)$ & 9 & 1 & YES & YES & NO(2) & $1.50$ & $(6,1)$ & NO & 6390\\
$(137,37)$ & 11 & $(56,15)$ & 9 & 1 & YES & YES & YES & $1.62$ & $(4,2)$ & NO & 6391\\
$(137,52)$ & 11 & $(129,49)$ & 10 & 1 & YES & YES & YES & $1.57$ & $(4,2)$ & 9077 & 6392\\
$(137,52)$ & 11 & $(137,52)$ & 11 & 137 & YES & YES & NO(2) & $1.33$ & $(4,2)$ & NO & 6393\\
$(138,37)$ & 11 & $(5,2)$ & 3 & 1 & YES & YES & NO(2) & $1.56$ & $(2,3)$ & -- & 6394\\
$(138,37)$ & 11 & $(5,2)$ & 3 & 1 & YES & YES & NO(2) & $1.29$ & $(6,1)$ & NO & 6395\\
$(138,37)$ & 11 & $(5,2)$ & 3 & 1 & YES & YES & NO(2) & $1.38$ & $(6,1)$ & NO & 6396\\
$(138,41)$ & 11 & $(7,3)$ & 4 & 1 & YES & YES & YES & $1.67$ & $(2,3)$ & -- & 6397\\
$(138,37)$ & 11 & $(10,3)$ & 5 & 2 & YES & YES & NO(2) & $1.56$ & $(2,3)$ & NO & 6398\\
$(138,41)$ & 11 & $(10,3)$ & 5 & 2 & YES & YES & YES & $1.86$ & $(2,3)$ & -- & 6399\\
$(138,37)$ & 11 & $(11,3)$ & 5 & 1 & YES & YES & YES & $1.57$ & $(6,1)$ & -- & 6400\\
$(138,41)$ & 11 & $(11,3)$ & 5 & 1 & YES & YES & NO(2) & $1.38$ & $(4,2)$ & NO & 6401\\
$(138,41)$ & 11 & $(16,3)$ & 7 & 2 & YES & YES & YES & $1.50$ & $(4,2)$ & NO & 6402\\
$(138,37)$ & 11 & $(23,6)$ & 8 & 23 & YES & YES & NO(2) & $1.38$ & $(6,1)$ & NO & 6403\\
$(138,41)$ & 11 & $(44,13)$ & 8 & 2 & YES & YES & NO(2) & $1.56$ & $(2,3)$ & 8935 & 6404\\
$(138,41)$ & 11 & $(91,27)$ & 10 & 1 & YES & YES & NO(2) & $1.44$ & $(2,3)$ & 8910 & 6405\\
$(138,41)$ & 11 & $(118,35)$ & 11 & 2 & YES & YES & YES & $1.67$ & $(2,3)$ & NO & 6406\\
$(139,39)$ & 11 & $(3,1)$ & 2 & 1 & YES & YES & NO(2) & $1.64$ & $(4,2)$ & -- & 6407\\
$(139,50)$ & 12 & $(3,1)$ & 2 & 1 & YES & YES & YES & $1.50$ & $(2,3)$ & -- & 6408\\
$(139,50)$ & 12 & $(3,1)$ & 2 & 1 & YES & YES & NO(2) & $1.60$ & $(6,1)$ & NO & 6409\\
$(139,61)$ & 11 & $(3,1)$ & 2 & 1 & YES & YES & NO(2) & $1.70$ & $(2,3)$ & -- & 6410\\
$(139,61)$ & 11 & $(3,1)$ & 2 & 1 & YES & YES & NO(2) & $1.73$ & $(4,2)$ & NO & 6411\\
$(139,39)$ & 11 & $(4,1)$ & 3 & 1 & YES & YES & NO(2) & $1.44$ & $(8,0)$ & NO & 6412\\
$(139,39)$ & 11 & $(4,1)$ & 3 & 1 & YES & YES & NO(2) & $1.44$ & $(8,0)$ & -- & 6413\\
$(139,61)$ & 11 & $(4,1)$ & 3 & 1 & YES & YES & NO(2) & $1.38$ & $(10,-1)$ & NO & 6414\\
$(139,59)$ & 12 & $(5,2)$ & 3 & 1 & YES & YES & NO(2) & $1.62$ & $(6,1)$ & -- & 6415\\
$(139,61)$ & 11 & $(5,2)$ & 3 & 1 & YES & YES & NO(2) & $1.80$ & $(2,3)$ & NO & 6416\\
$(139,61)$ & 11 & $(5,2)$ & 3 & 1 & YES & YES & NO(2) & $1.80$ & $(2,3)$ & -- & 6417\\
$(139,39)$ & 11 & $(7,3)$ & 4 & 1 & YES & YES & NO(2) & $1.44$ & $(4,2)$ & -- & 6418\\
$(139,41)$ & 11 & $(7,2)$ & 4 & 1 & YES & YES & YES & $1.43$ & $(6,1)$ & -- & 6419\\
$(139,51)$ & 11 & $(7,2)$ & 4 & 1 & YES & YES & NO(2) & $1.60$ & $(6,1)$ & -- & 6420\\
$(139,57)$ & 11 & $(7,2)$ & 4 & 1 & YES & YES & YES & $2.00$ & $(2,3)$ & NO & 6421\\
$(139,57)$ & 11 & $(7,2)$ & 4 & 1 & YES & YES & YES & $2.00$ & $(2,3)$ & -- & 6422\\
$(139,59)$ & 12 & $(8,3)$ & 4 & 1 & YES & YES & NO(2) & $1.50$ & $(6,1)$ & NO & 6423\\
$(139,51)$ & 11 & $(9,2)$ & 5 & 1 & YES & YES & NO(2) & $1.56$ & $(4,2)$ & -- & 6424\\
$(139,42)$ & 12 & $(10,3)$ & 5 & 1 & YES & YES & YES & $1.62$ & $(4,2)$ & -- & 6425\\
$(139,42)$ & 12 & $(27,8)$ & 7 & 1 & YES & YES & NO(2) & $1.50$ & $(6,1)$ & NO & 6426\\
$(139,39)$ & 11 & $(32,9)$ & 8 & 1 & YES & YES & NO(2) & $1.73$ & $(4,2)$ & NO & 6427\\
$(139,41)$ & 11 & $(44,13)$ & 8 & 1 & YES & YES & NO(2) & $1.50$ & $(2,3)$ & NO & 6428\\
$(139,61)$ & 11 & $(66,29)$ & 9 & 1 & YES & YES & NO(2) & $1.38$ & $(10,-1)$ & NO & 6429\\
$(139,41)$ & 11 & $(71,21)$ & 9 & 1 & YES & YES & YES & $1.43$ & $(6,1)$ & NO & 6430\\
$(139,39)$ & 11 & $(82,23)$ & 10 & 1 & YES & YES & NO(2) & $1.44$ & $(8,0)$ & NO & 6431\\
$(140,41)$ & 11 & $(2,1)$ & 1 & 2 & YES & YES & YES & $1.43$ & $(4,2)$ & 5570 & 6432\\
$(140,41)$ & 11 & $(2,1)$ & 1 & 2 & YES & YES & NO(2) & $1.56$ & $(4,2)$ & -- & 6433\\
$(140,61)$ & 11 & $(2,1)$ & 1 & 2 & YES & YES & NO(2) & $1.60$ & $(2,3)$ & -- & 6434\\
$(140,39)$ & 11 & $(3,1)$ & 2 & 1 & YES & YES & YES & $1.71$ & $(2,3)$ & -- & 6435\\
$(140,41)$ & 11 & $(3,1)$ & 2 & 1 & YES & YES & NO(2) & $1.64$ & $(4,2)$ & NO & 6436\\
$(140,41)$ & 11 & $(3,1)$ & 2 & 1 & YES & YES & NO(2) & $1.64$ & $(4,2)$ & -- & 6437\\
$(140,61)$ & 11 & $(3,1)$ & 2 & 1 & YES & YES & YES & $1.50$ & $(2,3)$ & NO & 6438\\
$(140,61)$ & 11 & $(3,1)$ & 2 & 1 & YES & YES & YES & $1.50$ & $(2,3)$ & -- & 6439\\
$(140,37)$ & 11 & $(4,1)$ & 3 & 4 & YES & YES & NO(2) & $1.38$ & $(6,1)$ & NO & 6440\\
$(140,37)$ & 11 & $(4,1)$ & 3 & 4 & YES & YES & NO(2) & $1.50$ & $(6,1)$ & -- & 6441\\
$(140,53)$ & 11 & $(4,1)$ & 3 & 4 & YES & YES & NO(2) & $1.50$ & $(6,1)$ & NO & 6442\\
$(140,53)$ & 11 & $(4,1)$ & 3 & 4 & YES & YES & NO(2) & $1.50$ & $(6,1)$ & -- & 6443\\
$(140,39)$ & 11 & $(5,1)$ & 4 & 5 & YES & YES & YES & $1.43$ & $(4,2)$ & NO & 6444\\
$(140,39)$ & 11 & $(5,1)$ & 4 & 5 & YES & YES & YES & $1.43$ & $(4,2)$ & -- & 6445\\
$(140,39)$ & 11 & $(5,2)$ & 3 & 5 & YES & YES & NO(2) & $1.60$ & $(6,1)$ & -- & 6446\\
$(140,39)$ & 11 & $(5,2)$ & 3 & 5 & YES & YES & YES & $1.62$ & $(2,3)$ & NO & 6447\\
$(140,41)$ & 11 & $(5,2)$ & 3 & 5 & YES & YES & NO(2) & $1.56$ & $(4,2)$ & -- & 6448\\
$(140,53)$ & 11 & $(5,2)$ & 3 & 5 & YES & YES & NO(2) & $1.56$ & $(8,0)$ & -- & 6449\\
$(140,61)$ & 11 & $(5,2)$ & 3 & 5 & YES & YES & NO(2) & $1.60$ & $(2,3)$ & NO & 6450\\
$(140,59)$ & 13 & $(6,1)$ & 5 & 2 & YES & YES & NO(2) & $1.62$ & $(6,1)$ & NO & 6451\\
$(140,53)$ & 11 & $(7,3)$ & 4 & 7 & YES & YES & NO(2) & $1.67$ & $(8,0)$ & NO & 6452\\
$(140,59)$ & 13 & $(7,1)$ & 6 & 7 & YES & YES & NO(2) & $1.75$ & $(6,1)$ & NO & 6453\\
$(140,59)$ & 13 & $(7,1)$ & 6 & 7 & YES & YES & NO(2) & $1.75$ & $(6,1)$ & -- & 6454\\
$(140,61)$ & 11 & $(7,3)$ & 4 & 7 & YES & YES & YES & $1.89$ & $(4,2)$ & -- & 6455\\
$(140,41)$ & 11 & $(8,3)$ & 4 & 4 & YES & YES & YES & $1.50$ & $(4,2)$ & -- & 6456\\
$(140,61)$ & 11 & $(9,2)$ & 5 & 1 & YES & YES & NO(2) & $1.56$ & $(4,2)$ & -- & 6457\\
$(140,39)$ & 11 & $(10,3)$ & 5 & 10 & YES & YES & YES & $1.78$ & $(2,3)$ & -- & 6458\\
$(140,41)$ & 11 & $(10,3)$ & 5 & 10 & YES & YES & YES & $1.75$ & $(2,3)$ & -- & 6459\\
$(140,39)$ & 11 & $(11,3)$ & 5 & 1 & YES & YES & YES & $1.50$ & $(4,2)$ & -- & 6460\\
$(140,37)$ & 11 & $(42,11)$ & 9 & 14 & YES & YES & NO(2) & $1.67$ & $(4,2)$ & NO & 6461\\
$(140,59)$ & 13 & $(64,27)$ & 9 & 4 & YES & YES & NO(2) & $1.75$ & $(6,1)$ & NO & 6462\\
$(140,53)$ & 11 & $(82,31)$ & 10 & 2 & YES & YES & NO(2) & $1.50$ & $(6,1)$ & NO & 6463\\
$(140,59)$ & 13 & $(83,35)$ & 10 & 1 & YES & YES & NO(2) & $1.62$ & $(6,1)$ & NO & 6464\\
$(140,53)$ & 11 & $(103,39)$ & 10 & 1 & YES & YES & NO(2) & $1.62$ & $(6,1)$ & NO & 6465\\
$(140,41)$ & 11 & $(133,39)$ & 11 & 7 & YES & YES & YES & $1.67$ & $(2,3)$ & NO & 6466\\
$(141,41)$ & 11 & $(2,1)$ & 1 & 1 & YES & YES & NO(2) & $1.50$ & $(2,3)$ & -- & 6467\\
$(141,41)$ & 11 & $(4,1)$ & 3 & 1 & YES & YES & NO(2) & $1.73$ & $(4,2)$ & NO & 6468\\
$(141,41)$ & 11 & $(4,1)$ & 3 & 1 & YES & YES & NO(2) & $1.73$ & $(4,2)$ & -- & 6469\\
$(141,55)$ & 11 & $(4,1)$ & 3 & 1 & YES & YES & NO(2) & $1.50$ & $(6,1)$ & NO & 6470\\
$(141,55)$ & 11 & $(4,1)$ & 3 & 1 & YES & YES & NO(2) & $1.50$ & $(6,1)$ & -- & 6471\\
$(141,55)$ & 11 & $(5,2)$ & 3 & 1 & YES & YES & NO(2) & $1.56$ & $(8,0)$ & -- & 6472\\
$(141,41)$ & 11 & $(7,3)$ & 4 & 1 & YES & YES & YES & $1.57$ & $(2,3)$ & NO & 6473\\
$(141,41)$ & 11 & $(7,3)$ & 4 & 1 & YES & YES & NO(2) & $1.44$ & $(4,2)$ & -- & 6474\\
$(141,55)$ & 11 & $(8,3)$ & 4 & 1 & YES & YES & NO(2) & $1.67$ & $(4,2)$ & NO & 6475\\
$(141,55)$ & 11 & $(10,3)$ & 5 & 1 & YES & YES & YES & $1.62$ & $(4,2)$ & -- & 6476\\
$(141,55)$ & 11 & $(11,3)$ & 5 & 1 & YES & YES & YES & $1.62$ & $(4,2)$ & -- & 6477\\
$(141,43)$ & 11 & $(24,7)$ & 7 & 3 & YES & YES & YES & $1.90$ & $(2,3)$ & NO & 6478\\
$(141,55)$ & 11 & $(28,11)$ & 8 & 1 & YES & YES & NO(2) & $1.44$ & $(8,0)$ & NO & 6479\\
$(141,38)$ & 12 & $(34,9)$ & 8 & 1 & YES & YES & NO(2) & $1.50$ & $(6,1)$ & NO & 6480\\
$(141,38)$ & 12 & $(67,18)$ & 9 & 1 & YES & YES & NO(2) & $1.50$ & $(6,1)$ & NO & 6481\\
$(142,39)$ & 11 & $(3,1)$ & 2 & 1 & YES & YES & NO(2) & $1.73$ & $(4,2)$ & -- & 6482\\
$(142,39)$ & 11 & $(3,1)$ & 2 & 1 & YES & YES & NO(2) & $1.82$ & $(4,2)$ & NO & 6483\\
$(142,51)$ & 11 & $(4,1)$ & 3 & 2 & YES & YES & NO(2) & $1.67$ & $(4,2)$ & -- & 6484\\
$(142,39)$ & 11 & $(5,2)$ & 3 & 1 & YES & YES & NO(2) & $1.56$ & $(4,2)$ & NO & 6485\\
$(142,39)$ & 11 & $(5,2)$ & 3 & 1 & YES & YES & NO(2) & $1.56$ & $(4,2)$ & -- & 6486\\
$(142,39)$ & 11 & $(5,2)$ & 3 & 1 & YES & YES & NO(2) & $1.60$ & $(2,3)$ & NO & 6487\\
$(142,55)$ & 11 & $(5,2)$ & 3 & 1 & YES & YES & NO(2) & $1.56$ & $(4,2)$ & -- & 6488\\
$(142,39)$ & 11 & $(7,3)$ & 4 & 1 & YES & YES & YES & $1.57$ & $(2,3)$ & -- & 6489\\
$(142,51)$ & 11 & $(7,2)$ & 4 & 1 & YES & YES & YES & $1.71$ & $(4,2)$ & -- & 6490\\
$(142,55)$ & 11 & $(7,3)$ & 4 & 1 & YES & YES & NO(2) & $1.50$ & $(6,1)$ & -- & 6491\\
$(142,55)$ & 11 & $(9,2)$ & 5 & 1 & YES & YES & YES & $1.75$ & $(2,3)$ & -- & 6492\\
$(142,39)$ & 11 & $(10,3)$ & 5 & 2 & YES & YES & NO(2) & $1.60$ & $(2,3)$ & NO & 6493\\
$(142,51)$ & 11 & $(13,5)$ & 5 & 1 & YES & YES & YES & $1.71$ & $(4,2)$ & NO & 6494\\
$(142,43)$ & 12 & $(16,3)$ & 7 & 2 & YES & YES & YES & $1.62$ & $(4,2)$ & -- & 6495\\
$(142,55)$ & 11 & $(23,9)$ & 7 & 1 & YES & YES & NO(2) & $1.56$ & $(4,2)$ & NO & 6496\\
$(142,39)$ & 11 & $(25,7)$ & 7 & 1 & YES & YES & NO(2) & $1.56$ & $(4,2)$ & NO & 6497\\
$(142,39)$ & 11 & $(26,7)$ & 7 & 2 & YES & YES & NO(2) & $1.50$ & $(6,1)$ & NO & 6498\\
$(142,51)$ & 11 & $(30,11)$ & 7 & 2 & YES & YES & YES & $1.89$ & $(4,2)$ & NO & 6499\\
$(142,51)$ & 11 & $(114,41)$ & 11 & 2 & YES & YES & YES & $1.78$ & $(4,2)$ & NO & 6500\\
$(143,56)$ & 13 & $(4,1)$ & 3 & 1 & YES & YES & NO(2) & $1.67$ & $(8,0)$ & -- & 6501\\
$(143,56)$ & 13 & $(5,1)$ & 4 & 1 & YES & YES & NO(2) & $1.78$ & $(8,0)$ & NO & 6502\\
$(143,56)$ & 13 & $(5,1)$ & 4 & 1 & YES & YES & NO(2) & $1.78$ & $(8,0)$ & -- & 6503\\
$(143,59)$ & 11 & $(5,2)$ & 3 & 1 & YES & YES & YES & $1.43$ & $(4,2)$ & -- & 6504\\
$(143,63)$ & 11 & $(5,2)$ & 3 & 1 & YES & YES & NO(2) & $1.80$ & $(2,3)$ & NO & 6505\\
$(143,59)$ & 11 & $(41,17)$ & 8 & 1 & YES & YES & YES & $1.57$ & $(4,2)$ & NO & 6506\\
$(143,59)$ & 11 & $(75,31)$ & 9 & 1 & YES & YES & NO(2) & $1.60$ & $(6,1)$ & NO & 6507\\
$(143,56)$ & 13 & $(97,38)$ & 11 & 1 & YES & YES & NO(2) & $1.78$ & $(8,0)$ & 8423 & 6508\\
$(143,40)$ & 12 & $(118,33)$ & 11 & 1 & YES & YES & NO(2) & $1.56$ & $(8,0)$ & NO & 6509\\
$(143,56)$ & 13 & $(120,47)$ & 12 & 1 & YES & YES & NO(2) & $1.67$ & $(8,0)$ & NO & 6510\\
$(144,55)$ & 10 & $(2,1)$ & 1 & 2 & YES & YES & NO(2) & $1.50$ & $(2,3)$ & 6221 & 6511\\
$(144,55)$ & 10 & $(2,1)$ & 1 & 2 & YES & YES & NO(2) & $1.60$ & $(2,3)$ & -- & 6512\\
$(144,55)$ & 10 & $(3,1)$ & 2 & 3 & YES & YES & NO(2) & $1.50$ & $(10,-1)$ & NO & 6513\\
$(144,55)$ & 10 & $(3,1)$ & 2 & 3 & YES & YES & NO(2) & $1.50$ & $(10,-1)$ & -- & 6514\\
$(144,55)$ & 10 & $(4,1)$ & 3 & 4 & YES & YES & YES & $1.43$ & $(6,1)$ & NO & 6515\\
$(144,55)$ & 10 & $(4,1)$ & 3 & 4 & YES & YES & YES & $1.43$ & $(6,1)$ & -- & 6516\\
$(144,55)$ & 10 & $(5,2)$ & 3 & 1 & YES & YES & YES & $1.38$ & $(2,3)$ & -- & 6517\\
$(144,55)$ & 10 & $(5,2)$ & 3 & 1 & YES & YES & YES & $1.57$ & $(4,2)$ & NO & 6518\\
$(144,65)$ & 12 & $(6,1)$ & 5 & 6 & YES & YES & NO(2) & $1.62$ & $(6,1)$ & NO & 6519\\
$(144,65)$ & 12 & $(6,1)$ & 5 & 6 & YES & YES & NO(2) & $1.62$ & $(6,1)$ & NO & 6520\\
$(144,31)$ & 12 & $(7,3)$ & 4 & 1 & YES & YES & YES & $1.57$ & $(2,3)$ & NO & 6521\\
$(144,55)$ & 10 & $(7,2)$ & 4 & 1 & YES & YES & YES & $1.62$ & $(4,2)$ & NO & 6522\\
$(144,55)$ & 10 & $(7,2)$ & 4 & 1 & YES & YES & YES & $1.50$ & $(2,3)$ & -- & 6523\\
$(144,55)$ & 10 & $(8,3)$ & 4 & 8 & YES & YES & NO(2) & $1.70$ & $(2,3)$ & NO & 6524\\
$(144,55)$ & 10 & $(8,3)$ & 4 & 8 & YES & YES & YES & $1.56$ & $(2,3)$ & -- & 6525\\
$(144,55)$ & 10 & $(9,2)$ & 5 & 9 & YES & YES & YES & $1.38$ & $(2,3)$ & NO & 6526\\
$(144,55)$ & 10 & $(9,2)$ & 5 & 9 & YES & YES & YES & $1.56$ & $(2,3)$ & -- & 6527\\
$(144,31)$ & 12 & $(11,3)$ & 5 & 1 & YES & YES & YES & $1.57$ & $(2,3)$ & NO & 6528\\
$(144,55)$ & 10 & $(12,5)$ & 5 & 12 & YES & YES & YES & $1.75$ & $(2,3)$ & NO & 6529\\
$(144,61)$ & 11 & $(12,5)$ & 5 & 12 & YES & YES & NO(2) & $1.44$ & $(8,0)$ & NO & 6530\\
$(144,55)$ & 10 & $(13,3)$ & 6 & 1 & YES & YES & YES & $1.57$ & $(2,3)$ & -- & 6531\\
$(144,55)$ & 10 & $(13,3)$ & 6 & 1 & YES & YES & YES & $1.86$ & $(2,3)$ & NO & 6532\\
$(144,55)$ & 10 & $(13,5)$ & 5 & 1 & YES & YES & NO(2) & $1.60$ & $(2,3)$ & 6169 & 6533\\
$(144,61)$ & 11 & $(13,3)$ & 6 & 1 & YES & YES & YES & $1.75$ & $(4,2)$ & NO & 6534\\
$(144,61)$ & 11 & $(13,3)$ & 6 & 1 & YES & YES & YES & $1.75$ & $(4,2)$ & -- & 6535\\
$(144,55)$ & 10 & $(18,7)$ & 6 & 18 & YES & YES & YES & $1.90$ & $(2,3)$ & NO & 6536\\
$(144,55)$ & 10 & $(23,9)$ & 7 & 1 & YES & YES & YES & $1.57$ & $(4,2)$ & NO & 6537\\
$(144,55)$ & 10 & $(29,11)$ & 7 & 1 & YES & YES & YES & $1.62$ & $(2,3)$ & NO & 6538\\
$(144,55)$ & 10 & $(31,12)$ & 7 & 1 & YES & YES & YES & $1.78$ & $(2,3)$ & NO & 6539\\
$(144,61)$ & 11 & $(43,18)$ & 8 & 1 & YES & YES & YES & $1.75$ & $(4,2)$ & NO & 6540\\
$(144,55)$ & 10 & $(47,18)$ & 8 & 1 & YES & YES & YES & $1.56$ & $(2,3)$ & NO & 6541\\
$(144,55)$ & 10 & $(50,19)$ & 8 & 2 & YES & YES & YES & $1.78$ & $(2,3)$ & NO & 6542\\
$(144,65)$ & 12 & $(51,23)$ & 9 & 3 & YES & YES & NO(2) & $1.62$ & $(6,1)$ & NO & 6543\\
$(144,55)$ & 10 & $(60,23)$ & 9 & 12 & YES & YES & YES & $1.75$ & $(2,3)$ & NO & 6544\\
$(144,55)$ & 10 & $(76,29)$ & 9 & 4 & YES & YES & YES & $1.67$ & $(2,3)$ & NO & 6545\\
$(144,65)$ & 12 & $(82,37)$ & 10 & 2 & YES & YES & NO(2) & $1.62$ & $(6,1)$ & 7919 & 6546\\
$(144,55)$ & 10 & $(89,34)$ & 9 & 1 & YES & YES & YES & $1.43$ & $(6,1)$ & NO & 6547\\
$(144,55)$ & 10 & $(97,37)$ & 10 & 1 & YES & YES & YES & $1.75$ & $(2,3)$ & NO & 6548\\
$(144,55)$ & 10 & $(123,47)$ & 10 & 3 & YES & YES & YES & $1.62$ & $(2,3)$ & NO & 6549\\
$(144,55)$ & 10 & $(144,55)$ & 10 & 144 & YES & YES & YES & $1.29$ & $(6,1)$ & NO & 6550\\
$(144,61)$ & 11 & $(144,61)$ & 11 & 144 & YES & YES & NO(2) & $1.33$ & $(8,0)$ & NO & 6551\\
$(144,65)$ & 12 & $(144,65)$ & 12 & 144 & YES & YES & NO(2) & $1.78$ & $(4,2)$ & NO & 6552\\
$(145,44)$ & 11 & $(2,1)$ & 1 & 1 & YES & YES & NO(2) & $1.60$ & $(2,3)$ & NO & 6553\\
$(145,44)$ & 11 & $(3,1)$ & 2 & 1 & YES & YES & NO(2) & $1.38$ & $(6,1)$ & -- & 6554\\
$(145,44)$ & 11 & $(3,1)$ & 2 & 1 & YES & YES & NO(2) & $1.50$ & $(6,1)$ & NO & 6555\\
$(145,52)$ & 11 & $(3,1)$ & 2 & 1 & YES & YES & YES & $1.43$ & $(4,2)$ & NO & 6556\\
$(145,53)$ & 11 & $(3,1)$ & 2 & 1 & YES & YES & NO(2) & $1.56$ & $(8,0)$ & -- & 6557\\
$(145,33)$ & 13 & $(4,1)$ & 3 & 1 & YES & YES & NO(2) & $1.56$ & $(8,0)$ & -- & 6558\\
$(145,33)$ & 13 & $(4,1)$ & 3 & 1 & YES & YES & NO(2) & $1.67$ & $(8,0)$ & NO & 6559\\
$(145,38)$ & 12 & $(4,1)$ & 3 & 1 & YES & YES & NO(2) & $1.50$ & $(6,1)$ & NO & 6560\\
$(145,44)$ & 11 & $(5,2)$ & 3 & 5 & YES & YES & NO(2) & $1.56$ & $(2,3)$ & -- & 6561\\
$(145,56)$ & 11 & $(5,2)$ & 3 & 5 & YES & YES & YES & $1.89$ & $(2,3)$ & -- & 6562\\
$(145,63)$ & 13 & $(5,2)$ & 3 & 5 & YES & YES & NO(2) & $1.67$ & $(8,0)$ & NO & 6563\\
$(145,44)$ & 11 & $(7,3)$ & 4 & 1 & YES & YES & YES & $1.80$ & $(2,3)$ & -- & 6564\\
$(145,43)$ & 12 & $(8,3)$ & 4 & 1 & YES & YES & NO(2) & $1.62$ & $(6,1)$ & NO & 6565\\
$(145,63)$ & 13 & $(8,3)$ & 4 & 1 & YES & YES & NO(2) & $1.56$ & $(8,0)$ & NO & 6566\\
$(145,43)$ & 12 & $(11,2)$ & 6 & 1 & YES & YES & YES & $1.43$ & $(4,2)$ & -- & 6567\\
$(145,44)$ & 11 & $(11,3)$ & 5 & 1 & YES & YES & YES & $1.71$ & $(2,3)$ & -- & 6568\\
$(145,44)$ & 11 & $(13,4)$ & 6 & 1 & YES & YES & NO(2) & $1.60$ & $(2,3)$ & NO & 6569\\
$(145,44)$ & 11 & $(18,5)$ & 6 & 1 & YES & YES & YES & $1.89$ & $(2,3)$ & NO & 6570\\
$(145,44)$ & 11 & $(23,7)$ & 7 & 1 & YES & YES & NO(2) & $1.73$ & $(4,2)$ & NO & 6571\\
$(145,53)$ & 11 & $(30,11)$ & 7 & 5 & YES & YES & NO(2) & $1.60$ & $(6,1)$ & 7592 & 6572\\
$(145,39)$ & 12 & $(67,18)$ & 9 & 1 & YES & YES & NO(2) & $1.25$ & $(6,1)$ & NO & 6573\\
$(145,53)$ & 11 & $(93,34)$ & 10 & 1 & YES & YES & NO(2) & $1.60$ & $(6,1)$ & NO & 6574\\
$(145,44)$ & 11 & $(102,31)$ & 11 & 1 & YES & YES & YES & $1.80$ & $(2,3)$ & NO & 6575\\
$(145,44)$ & 11 & $(122,37)$ & 11 & 1 & YES & YES & YES & $1.29$ & $(4,2)$ & NO & 6576\\
$(145,33)$ & 13 & $(123,28)$ & 12 & 1 & YES & YES & NO(2) & $1.44$ & $(4,2)$ & NO & 6577\\
$(145,43)$ & 12 & $(145,43)$ & 12 & 145 & YES & YES & YES & $1.62$ & $(2,3)$ & NO & 6578\\
$(146,41)$ & 11 & $(2,1)$ & 1 & 2 & YES & YES & NO(2) & $1.73$ & $(4,2)$ & NO & 6579\\
$(146,41)$ & 11 & $(2,1)$ & 1 & 2 & YES & YES & NO(2) & $1.73$ & $(4,2)$ & -- & 6580\\
$(146,57)$ & 11 & $(3,1)$ & 2 & 1 & YES & YES & NO(2) & $1.70$ & $(6,1)$ & -- & 6581\\
$(146,61)$ & 12 & $(3,1)$ & 2 & 1 & YES & YES & NO(2) & $1.67$ & $(4,2)$ & NO & 6582\\
$(146,61)$ & 12 & $(3,1)$ & 2 & 1 & YES & YES & NO(2) & $1.67$ & $(4,2)$ & -- & 6583\\
$(146,57)$ & 11 & $(4,1)$ & 3 & 2 & YES & YES & NO(2) & $1.50$ & $(6,1)$ & NO & 6584\\
$(146,57)$ & 11 & $(4,1)$ & 3 & 2 & YES & YES & NO(2) & $1.50$ & $(6,1)$ & -- & 6585\\
$(146,61)$ & 12 & $(4,1)$ & 3 & 2 & YES & YES & NO(2) & $1.62$ & $(6,1)$ & NO & 6586\\
$(146,61)$ & 12 & $(4,1)$ & 3 & 2 & YES & YES & NO(2) & $1.62$ & $(6,1)$ & NO & 6587\\
$(146,31)$ & 12 & $(5,2)$ & 3 & 1 & YES & YES & NO(2) & $1.44$ & $(8,0)$ & -- & 6588\\
$(146,41)$ & 11 & $(5,2)$ & 3 & 1 & YES & YES & NO(2) & $1.60$ & $(6,1)$ & NO & 6589\\
$(146,61)$ & 12 & $(6,1)$ & 5 & 2 & YES & YES & NO(2) & $1.62$ & $(6,1)$ & NO & 6590\\
$(146,57)$ & 11 & $(10,3)$ & 5 & 2 & YES & YES & YES & $1.89$ & $(4,2)$ & -- & 6591\\
$(146,57)$ & 11 & $(13,5)$ & 5 & 1 & YES & YES & NO(2) & $1.70$ & $(2,3)$ & NO & 6592\\
$(146,61)$ & 12 & $(19,8)$ & 6 & 1 & YES & YES & NO(2) & $1.75$ & $(6,1)$ & NO & 6593\\
$(146,27)$ & 13 & $(20,3)$ & 8 & 2 & YES & YES & NO(2) & $1.56$ & $(8,0)$ & NO & 6594\\
$(146,41)$ & 11 & $(25,7)$ & 7 & 1 & YES & YES & NO(2) & $1.64$ & $(4,2)$ & NO & 6595\\
$(146,57)$ & 11 & $(28,11)$ & 8 & 2 & YES & YES & NO(2) & $1.44$ & $(8,0)$ & 7360 & 6596\\
$(146,61)$ & 12 & $(31,13)$ & 7 & 1 & YES & YES & NO(2) & $1.67$ & $(8,0)$ & NO & 6597\\
$(146,41)$ & 11 & $(32,9)$ & 8 & 2 & YES & YES & NO(2) & $1.73$ & $(4,2)$ & 6042 & 6598\\
$(146,33)$ & 12 & $(40,9)$ & 9 & 2 & YES & YES & NO(2) & $1.44$ & $(8,0)$ & NO & 6599\\
$(146,61)$ & 12 & $(55,23)$ & 9 & 1 & YES & YES & NO(2) & $1.78$ & $(4,2)$ & NO & 6600\\
$(146,57)$ & 11 & $(64,25)$ & 9 & 2 & YES & YES & NO(2) & $1.60$ & $(2,3)$ & 7180 & 6601\\
$(146,57)$ & 11 & $(105,41)$ & 10 & 1 & YES & YES & NO(2) & $1.50$ & $(6,1)$ & NO & 6602\\
$(147,41)$ & 11 & $(2,1)$ & 1 & 1 & YES & YES & YES & $1.62$ & $(2,3)$ & -- & 6603\\
$(147,41)$ & 11 & $(2,1)$ & 1 & 1 & YES & YES & YES & $1.43$ & $(4,2)$ & NO & 6604\\
$(147,43)$ & 11 & $(2,1)$ & 1 & 1 & YES & YES & NO(2) & $1.60$ & $(6,1)$ & NO & 6605\\
$(147,61)$ & 11 & $(2,1)$ & 1 & 1 & YES & YES & NO(2) & $1.50$ & $(6,1)$ & -- & 6606\\
$(147,41)$ & 11 & $(3,1)$ & 2 & 3 & YES & YES & YES & $1.50$ & $(2,3)$ & NO & 6607\\
$(147,41)$ & 11 & $(3,1)$ & 2 & 3 & YES & YES & YES & $1.57$ & $(2,3)$ & -- & 6608\\
$(147,43)$ & 11 & $(3,1)$ & 2 & 3 & YES & YES & YES & $1.50$ & $(2,3)$ & NO & 6609\\
$(147,43)$ & 11 & $(3,1)$ & 2 & 3 & YES & YES & YES & $1.50$ & $(2,3)$ & -- & 6610\\
$(147,32)$ & 13 & $(4,1)$ & 3 & 1 & YES & YES & NO(2) & $1.50$ & $(6,1)$ & -- & 6611\\
$(147,32)$ & 13 & $(4,1)$ & 3 & 1 & YES & YES & NO(2) & $1.62$ & $(6,1)$ & NO & 6612\\
$(147,43)$ & 11 & $(5,2)$ & 3 & 1 & YES & YES & NO(2) & $1.38$ & $(6,1)$ & -- & 6613\\
$(147,64)$ & 11 & $(5,2)$ & 3 & 1 & YES & YES & YES & $1.57$ & $(4,2)$ & -- & 6614\\
$(147,41)$ & 11 & $(7,3)$ & 4 & 7 & YES & YES & YES & $1.89$ & $(2,3)$ & -- & 6615\\
$(147,41)$ & 11 & $(7,3)$ & 4 & 7 & YES & YES & YES & $1.89$ & $(2,3)$ & NO & 6616\\
$(147,43)$ & 11 & $(7,2)$ & 4 & 7 & YES & YES & NO(2) & $1.50$ & $(2,3)$ & NO & 6617\\
$(147,43)$ & 11 & $(7,2)$ & 4 & 7 & YES & YES & NO(2) & $1.60$ & $(2,3)$ & -- & 6618\\
$(147,43)$ & 11 & $(7,3)$ & 4 & 7 & YES & YES & NO(2) & $1.50$ & $(6,1)$ & NO & 6619\\
$(147,43)$ & 11 & $(7,3)$ & 4 & 7 & YES & YES & YES & $1.57$ & $(4,2)$ & NO & 6620\\
$(147,43)$ & 11 & $(7,3)$ & 4 & 7 & YES & YES & YES & $1.57$ & $(4,2)$ & -- & 6621\\
$(147,43)$ & 11 & $(8,3)$ & 4 & 1 & YES & YES & YES & $1.67$ & $(4,2)$ & -- & 6622\\
$(147,64)$ & 11 & $(8,3)$ & 4 & 1 & YES & YES & YES & $1.43$ & $(4,2)$ & NO & 6623\\
$(147,53)$ & 11 & $(10,3)$ & 5 & 1 & YES & YES & NO(2) & $1.60$ & $(6,1)$ & NO & 6624\\
$(147,41)$ & 11 & $(11,3)$ & 5 & 1 & YES & YES & YES & $1.62$ & $(2,3)$ & -- & 6625\\
$(147,41)$ & 11 & $(13,3)$ & 6 & 1 & YES & YES & YES & $1.67$ & $(2,3)$ & -- & 6626\\
$(147,43)$ & 11 & $(13,3)$ & 6 & 1 & YES & YES & YES & $1.43$ & $(4,2)$ & NO & 6627\\
$(147,61)$ & 11 & $(13,3)$ & 6 & 1 & YES & YES & YES & $1.75$ & $(4,2)$ & NO & 6628\\
$(147,43)$ & 11 & $(14,3)$ & 6 & 7 & YES & YES & YES & $1.71$ & $(2,3)$ & -- & 6629\\
$(147,43)$ & 11 & $(14,3)$ & 6 & 7 & YES & YES & YES & $1.70$ & $(2,3)$ & NO & 6630\\
$(147,43)$ & 11 & $(23,7)$ & 7 & 1 & YES & YES & YES & $1.43$ & $(4,2)$ & NO & 6631\\
$(147,41)$ & 11 & $(26,7)$ & 7 & 1 & YES & YES & YES & $1.70$ & $(2,3)$ & NO & 6632\\
$(147,41)$ & 11 & $(40,11)$ & 8 & 1 & YES & YES & YES & $1.70$ & $(2,3)$ & NO & 6633\\
$(147,43)$ & 11 & $(75,22)$ & 10 & 3 & YES & YES & NO(2) & $1.25$ & $(6,1)$ & NO & 6634\\
$(147,41)$ & 11 & $(93,26)$ & 10 & 3 & YES & YES & YES & $1.78$ & $(2,3)$ & NO & 6635\\
$(147,43)$ & 11 & $(113,33)$ & 11 & 1 & YES & YES & YES & $1.43$ & $(4,2)$ & NO & 6636\\
$(147,61)$ & 11 & $(135,56)$ & 11 & 3 & YES & YES & YES & $1.71$ & $(4,2)$ & NO & 6637\\
$(148,65)$ & 11 & $(3,1)$ & 2 & 1 & YES & YES & NO(2) & $1.38$ & $(10,-1)$ & -- & 6638\\
$(148,65)$ & 11 & $(4,1)$ & 3 & 4 & YES & YES & NO(2) & $1.67$ & $(4,2)$ & NO & 6639\\
$(148,65)$ & 11 & $(4,1)$ & 3 & 4 & YES & YES & NO(2) & $1.73$ & $(4,2)$ & -- & 6640\\
$(148,55)$ & 12 & $(5,2)$ & 3 & 1 & YES & YES & NO(2) & $1.62$ & $(6,1)$ & -- & 6641\\
$(148,65)$ & 11 & $(5,2)$ & 3 & 1 & YES & YES & NO(2) & $1.50$ & $(10,-1)$ & NO & 6642\\
$(148,65)$ & 11 & $(5,2)$ & 3 & 1 & YES & YES & NO(2) & $1.50$ & $(10,-1)$ & -- & 6643\\
$(148,45)$ & 13 & $(6,1)$ & 5 & 2 & YES & YES & NO(2) & $1.50$ & $(6,1)$ & NO & 6644\\
$(148,65)$ & 11 & $(98,43)$ & 10 & 2 & YES & YES & NO(2) & $1.50$ & $(6,1)$ & NO & 6645\\
$(148,41)$ & 11 & $(112,31)$ & 10 & 4 & YES & YES & NO(2) & $1.56$ & $(4,2)$ & NO & 6646\\
$(149,65)$ & 11 & $(2,1)$ & 1 & 1 & YES & YES & YES & $1.29$ & $(4,2)$ & -- & 6647\\
$(149,55)$ & 11 & $(3,1)$ & 2 & 1 & YES & YES & YES & $1.38$ & $(2,3)$ & -- & 6648\\
$(149,65)$ & 11 & $(3,1)$ & 2 & 1 & YES & YES & NO(2) & $1.56$ & $(4,2)$ & NO & 6649\\
$(149,65)$ & 11 & $(3,1)$ & 2 & 1 & YES & YES & NO(2) & $1.56$ & $(4,2)$ & -- & 6650\\
$(149,39)$ & 12 & $(4,1)$ & 3 & 1 & YES & YES & NO(2) & $1.56$ & $(4,2)$ & -- & 6651\\
$(149,42)$ & 12 & $(4,1)$ & 3 & 1 & YES & YES & NO(2) & $1.80$ & $(2,3)$ & -- & 6652\\
$(149,41)$ & 11 & $(5,2)$ & 3 & 1 & YES & YES & NO(2) & $1.25$ & $(4,2)$ & -- & 6653\\
$(149,44)$ & 11 & $(5,2)$ & 3 & 1 & YES & YES & YES & $1.62$ & $(2,3)$ & -- & 6654\\
$(149,65)$ & 11 & $(5,2)$ & 3 & 1 & YES & YES & NO(2) & $1.56$ & $(4,2)$ & NO & 6655\\
$(149,55)$ & 11 & $(7,2)$ & 4 & 1 & YES & YES & YES & $1.71$ & $(4,2)$ & -- & 6656\\
$(149,65)$ & 11 & $(7,2)$ & 4 & 1 & YES & YES & YES & $1.62$ & $(4,2)$ & NO & 6657\\
$(149,40)$ & 11 & $(8,3)$ & 4 & 1 & YES & YES & YES & $1.62$ & $(4,2)$ & -- & 6658\\
$(149,44)$ & 11 & $(8,3)$ & 4 & 1 & YES & YES & YES & $1.70$ & $(2,3)$ & -- & 6659\\
$(149,41)$ & 11 & $(10,3)$ & 5 & 1 & YES & YES & NO(2) & $1.38$ & $(4,2)$ & NO & 6660\\
$(149,44)$ & 11 & $(13,4)$ & 6 & 1 & YES & YES & NO(2) & $1.56$ & $(2,3)$ & NO & 6661\\
$(149,42)$ & 12 & $(18,5)$ & 6 & 1 & YES & YES & NO(2) & $1.70$ & $(2,3)$ & NO & 6662\\
$(149,44)$ & 11 & $(18,5)$ & 6 & 1 & YES & YES & YES & $1.88$ & $(2,3)$ & NO & 6663\\
$(149,40)$ & 11 & $(25,7)$ & 7 & 1 & YES & YES & YES & $1.62$ & $(4,2)$ & NO & 6664\\
$(149,65)$ & 11 & $(25,11)$ & 7 & 1 & YES & YES & NO(2) & $1.67$ & $(4,2)$ & NO & 6665\\
$(149,41)$ & 11 & $(29,8)$ & 7 & 1 & YES & YES & YES & $1.71$ & $(2,3)$ & NO & 6666\\
$(149,44)$ & 11 & $(37,11)$ & 8 & 1 & YES & YES & YES & $1.62$ & $(2,3)$ & NO & 6667\\
$(149,44)$ & 11 & $(47,14)$ & 9 & 1 & YES & YES & YES & $1.75$ & $(4,2)$ & NO & 6668\\
$(149,44)$ & 11 & $(64,19)$ & 9 & 1 & YES & YES & YES & $1.70$ & $(2,3)$ & NO & 6669\\
$(149,44)$ & 11 & $(71,21)$ & 9 & 1 & YES & YES & YES & $1.50$ & $(2,3)$ & NO & 6670\\
$(149,55)$ & 11 & $(73,27)$ & 9 & 1 & YES & YES & YES & $1.71$ & $(4,2)$ & NO & 6671\\
$(149,40)$ & 11 & $(93,25)$ & 10 & 1 & YES & YES & YES & $1.43$ & $(2,3)$ & 9401 & 6672\\
$(149,41)$ & 11 & $(98,27)$ & 10 & 1 & YES & YES & YES & $1.43$ & $(2,3)$ & 9421 & 6673\\
$(149,57)$ & 11 & $(128,49)$ & 10 & 1 & YES & YES & YES & $1.57$ & $(2,3)$ & 9553 & 6674\\
$(149,65)$ & 11 & $(133,58)$ & 11 & 1 & YES & YES & YES & $1.62$ & $(4,2)$ & NO & 6675\\
$(151,45)$ & 12 & $(3,1)$ & 2 & 1 & YES & YES & NO(2) & $1.60$ & $(2,3)$ & -- & 6676\\
$(151,45)$ & 12 & $(3,1)$ & 2 & 1 & YES & YES & NO(2) & $1.60$ & $(2,3)$ & NO & 6677\\
$(151,47)$ & 12 & $(3,1)$ & 2 & 1 & YES & YES & NO(2) & $1.56$ & $(4,2)$ & -- & 6678\\
$(151,62)$ & 11 & $(3,1)$ & 2 & 1 & YES & YES & NO(2) & $1.73$ & $(4,2)$ & -- & 6679\\
$(151,56)$ & 11 & $(4,1)$ & 3 & 1 & YES & YES & NO(2) & $1.50$ & $(4,2)$ & NO & 6680\\
$(151,56)$ & 11 & $(4,1)$ & 3 & 1 & YES & YES & NO(2) & $1.50$ & $(4,2)$ & -- & 6681\\
$(151,27)$ & 13 & $(5,2)$ & 3 & 1 & YES & YES & YES & $1.43$ & $(4,2)$ & -- & 6682\\
$(151,32)$ & 12 & $(5,2)$ & 3 & 1 & YES & YES & NO(2) & $1.44$ & $(8,0)$ & -- & 6683\\
$(151,62)$ & 11 & $(5,2)$ & 3 & 1 & YES & YES & NO(2) & $1.50$ & $(4,2)$ & -- & 6684\\
$(151,44)$ & 13 & $(7,1)$ & 6 & 1 & YES & YES & YES & $1.29$ & $(4,2)$ & NO & 6685\\
$(151,56)$ & 11 & $(7,2)$ & 4 & 1 & YES & YES & NO(2) & $1.50$ & $(6,1)$ & -- & 6686\\
$(151,62)$ & 11 & $(7,2)$ & 4 & 1 & YES & YES & YES & $1.88$ & $(2,3)$ & -- & 6687\\
$(151,62)$ & 11 & $(8,3)$ & 4 & 1 & YES & YES & YES & $1.62$ & $(4,2)$ & -- & 6688\\
$(151,28)$ & 13 & $(10,3)$ & 5 & 1 & YES & YES & YES & $1.43$ & $(4,2)$ & NO & 6689\\
$(151,59)$ & 11 & $(13,5)$ & 5 & 1 & YES & YES & NO(2) & $1.56$ & $(12,-2)$ & NO & 6690\\
$(151,45)$ & 12 & $(17,5)$ & 6 & 1 & YES & YES & NO(2) & $1.60$ & $(2,3)$ & 4872 & 6691\\
$(151,56)$ & 11 & $(18,7)$ & 6 & 1 & YES & YES & YES & $2.00$ & $(2,3)$ & NO & 6692\\
$(151,62)$ & 11 & $(22,9)$ & 7 & 1 & YES & YES & NO(2) & $1.73$ & $(4,2)$ & 6959 & 6693\\
$(151,32)$ & 12 & $(29,6)$ & 9 & 1 & YES & YES & NO(2) & $1.38$ & $(6,1)$ & NO & 6694\\
$(151,34)$ & 12 & $(35,8)$ & 8 & 1 & YES & YES & NO(2) & $1.44$ & $(4,2)$ & NO & 6695\\
$(151,62)$ & 11 & $(39,16)$ & 8 & 1 & YES & YES & YES & $1.43$ & $(2,3)$ & 6337 & 6696\\
$(151,32)$ & 12 & $(47,10)$ & 9 & 1 & YES & YES & NO(2) & $1.44$ & $(8,0)$ & NO & 6697\\
$(151,56)$ & 11 & $(89,33)$ & 10 & 1 & YES & YES & NO(2) & $1.62$ & $(4,2)$ & NO & 6698\\
$(151,46)$ & 13 & $(105,32)$ & 11 & 1 & YES & YES & NO(2) & $1.56$ & $(8,0)$ & 8782 & 6699\\
$(152,59)$ & 11 & $(2,1)$ & 1 & 2 & YES & YES & NO(2) & $1.56$ & $(8,0)$ & -- & 6700\\
$(152,63)$ & 11 & $(2,1)$ & 1 & 2 & YES & YES & NO(2) & $1.60$ & $(2,3)$ & -- & 6701\\
$(152,55)$ & 12 & $(3,1)$ & 2 & 1 & YES & YES & YES & $1.50$ & $(2,3)$ & -- & 6702\\
$(152,59)$ & 11 & $(3,1)$ & 2 & 1 & YES & YES & NO(2) & $1.80$ & $(6,1)$ & -- & 6703\\
$(152,59)$ & 11 & $(3,1)$ & 2 & 1 & YES & YES & NO(2) & $1.80$ & $(6,1)$ & NO & 6704\\
$(152,63)$ & 11 & $(3,1)$ & 2 & 1 & YES & YES & NO(2) & $1.50$ & $(4,2)$ & -- & 6705\\
$(152,63)$ & 11 & $(3,1)$ & 2 & 1 & YES & YES & NO(2) & $1.56$ & $(4,2)$ & NO & 6706\\
$(152,59)$ & 11 & $(5,2)$ & 3 & 1 & YES & YES & YES & $1.78$ & $(2,3)$ & -- & 6707\\
$(152,45)$ & 12 & $(6,1)$ & 5 & 2 & YES & YES & YES & $1.43$ & $(2,3)$ & NO & 6708\\
$(152,55)$ & 12 & $(7,2)$ & 4 & 1 & YES & YES & NO(2) & $1.56$ & $(8,0)$ & NO & 6709\\
$(152,55)$ & 12 & $(7,3)$ & 4 & 1 & YES & YES & YES & $1.71$ & $(4,2)$ & -- & 6710\\
$(152,63)$ & 11 & $(7,2)$ & 4 & 1 & YES & YES & YES & $1.50$ & $(2,3)$ & -- & 6711\\
$(152,63)$ & 11 & $(7,3)$ & 4 & 1 & YES & YES & NO(2) & $1.80$ & $(2,3)$ & NO & 6712\\
$(152,55)$ & 12 & $(8,3)$ & 4 & 8 & YES & YES & YES & $1.71$ & $(4,2)$ & -- & 6713\\
$(152,55)$ & 12 & $(8,3)$ & 4 & 8 & YES & YES & YES & $1.86$ & $(4,2)$ & NO & 6714\\
$(152,59)$ & 11 & $(8,3)$ & 4 & 8 & YES & YES & NO(2) & $1.80$ & $(6,1)$ & NO & 6715\\
$(152,59)$ & 11 & $(9,2)$ & 5 & 1 & YES & YES & YES & $1.43$ & $(4,2)$ & -- & 6716\\
$(152,59)$ & 11 & $(9,2)$ & 5 & 1 & YES & YES & YES & $1.89$ & $(2,3)$ & NO & 6717\\
$(152,63)$ & 11 & $(9,2)$ & 5 & 1 & YES & YES & YES & $1.75$ & $(2,3)$ & -- & 6718\\
$(152,41)$ & 11 & $(10,3)$ & 5 & 2 & YES & YES & NO(2) & $1.44$ & $(2,3)$ & NO & 6719\\
$(152,63)$ & 11 & $(10,3)$ & 5 & 2 & YES & YES & YES & $1.67$ & $(4,2)$ & -- & 6720\\
$(152,33)$ & 13 & $(11,2)$ & 6 & 1 & YES & YES & NO(2) & $1.56$ & $(8,0)$ & NO & 6721\\
$(152,41)$ & 11 & $(11,3)$ & 5 & 1 & YES & YES & YES & $1.67$ & $(4,2)$ & -- & 6722\\
$(152,63)$ & 11 & $(17,7)$ & 6 & 1 & YES & YES & NO(2) & $1.38$ & $(6,1)$ & NO & 6723\\
$(152,55)$ & 12 & $(19,7)$ & 6 & 19 & YES & YES & NO(2) & $1.56$ & $(8,0)$ & NO & 6724\\
$(152,55)$ & 12 & $(36,13)$ & 8 & 4 & YES & YES & YES & $1.43$ & $(4,2)$ & NO & 6725\\
$(152,41)$ & 11 & $(37,10)$ & 8 & 1 & YES & YES & NO(2) & $1.50$ & $(2,3)$ & NO & 6726\\
$(152,63)$ & 11 & $(39,16)$ & 8 & 1 & YES & YES & YES & $1.67$ & $(4,2)$ & NO & 6727\\
$(152,59)$ & 11 & $(44,17)$ & 8 & 4 & YES & YES & YES & $1.90$ & $(2,3)$ & NO & 6728\\
$(152,45)$ & 12 & $(71,21)$ & 9 & 1 & YES & YES & YES & $1.43$ & $(2,3)$ & NO & 6729\\
$(152,63)$ & 11 & $(99,41)$ & 10 & 1 & YES & YES & YES & $1.78$ & $(2,3)$ & 9524 & 6730\\
$(152,59)$ & 11 & $(116,45)$ & 10 & 4 & YES & YES & YES & $1.57$ & $(4,2)$ & NO & 6731\\
$(152,55)$ & 12 & $(119,43)$ & 11 & 1 & YES & YES & YES & $1.71$ & $(4,2)$ & NO & 6732\\
$(152,55)$ & 12 & $(152,55)$ & 12 & 152 & YES & YES & YES & $1.43$ & $(4,2)$ & NO & 6733\\
$(152,63)$ & 11 & $(152,63)$ & 11 & 152 & YES & YES & NO(2) & $1.67$ & $(2,3)$ & NO & 6734\\
$(153,56)$ & 11 & $(3,1)$ & 2 & 3 & YES & YES & NO(2) & $1.60$ & $(6,1)$ & -- & 6735\\
$(153,41)$ & 11 & $(5,2)$ & 3 & 1 & YES & YES & YES & $1.43$ & $(2,3)$ & -- & 6736\\
$(153,41)$ & 11 & $(5,2)$ & 3 & 1 & YES & YES & NO(2) & $1.67$ & $(6,1)$ & NO & 6737\\
$(153,56)$ & 11 & $(5,2)$ & 3 & 1 & YES & YES & YES & $1.57$ & $(4,2)$ & NO & 6738\\
$(153,56)$ & 11 & $(5,2)$ & 3 & 1 & YES & YES & YES & $1.57$ & $(4,2)$ & -- & 6739\\
$(153,64)$ & 11 & $(5,2)$ & 3 & 1 & YES & YES & YES & $1.88$ & $(2,3)$ & -- & 6740\\
$(153,41)$ & 11 & $(7,2)$ & 4 & 1 & YES & YES & YES & $1.29$ & $(2,3)$ & -- & 6741\\
$(153,64)$ & 11 & $(7,2)$ & 4 & 1 & YES & YES & YES & $1.88$ & $(2,3)$ & NO & 6742\\
$(153,64)$ & 11 & $(7,2)$ & 4 & 1 & YES & YES & YES & $1.88$ & $(2,3)$ & -- & 6743\\
$(153,35)$ & 12 & $(8,3)$ & 4 & 1 & YES & YES & YES & $1.62$ & $(4,2)$ & -- & 6744\\
$(153,35)$ & 12 & $(8,3)$ & 4 & 1 & YES & YES & YES & $1.75$ & $(4,2)$ & NO & 6745\\
$(153,35)$ & 12 & $(10,3)$ & 5 & 1 & YES & YES & YES & $1.29$ & $(4,2)$ & NO & 6746\\
$(153,41)$ & 11 & $(10,3)$ & 5 & 1 & YES & YES & YES & $1.43$ & $(2,3)$ & NO & 6747\\
$(153,35)$ & 12 & $(12,5)$ & 5 & 3 & YES & YES & YES & $1.89$ & $(4,2)$ & NO & 6748\\
$(153,32)$ & 12 & $(17,4)$ & 7 & 17 & YES & YES & NO(2) & $1.38$ & $(10,-1)$ & NO & 6749\\
$(153,59)$ & 12 & $(21,8)$ & 6 & 3 & YES & YES & NO(2) & $1.62$ & $(6,1)$ & NO & 6750\\
$(153,35)$ & 12 & $(31,7)$ & 8 & 1 & YES & YES & YES & $1.43$ & $(4,2)$ & NO & 6751\\
$(153,41)$ & 11 & $(67,18)$ & 9 & 1 & YES & YES & YES & $1.43$ & $(2,3)$ & NO & 6752\\
$(153,56)$ & 11 & $(101,37)$ & 10 & 1 & YES & YES & YES & $1.90$ & $(2,3)$ & 9585 & 6753\\
$(153,64)$ & 11 & $(141,59)$ & 11 & 3 & YES & YES & YES & $1.88$ & $(2,3)$ & NO & 6754\\
$(154,43)$ & 11 & $(2,1)$ & 1 & 2 & YES & YES & NO(2) & $1.60$ & $(6,1)$ & NO & 6755\\
$(154,47)$ & 11 & $(2,1)$ & 1 & 2 & YES & YES & NO(2) & $1.50$ & $(6,1)$ & NO & 6756\\
$(154,47)$ & 11 & $(2,1)$ & 1 & 2 & YES & YES & NO(2) & $1.50$ & $(6,1)$ & -- & 6757\\
$(154,65)$ & 11 & $(2,1)$ & 1 & 2 & YES & YES & NO(2) & $1.60$ & $(2,3)$ & -- & 6758\\
$(154,45)$ & 11 & $(3,1)$ & 2 & 1 & YES & YES & YES & $1.57$ & $(2,3)$ & -- & 6759\\
$(154,45)$ & 11 & $(3,1)$ & 2 & 1 & YES & YES & YES & $1.71$ & $(2,3)$ & NO & 6760\\
$(154,47)$ & 11 & $(3,1)$ & 2 & 1 & YES & YES & NO(2) & $1.50$ & $(6,1)$ & 5954 & 6761\\
$(154,47)$ & 11 & $(3,1)$ & 2 & 1 & YES & YES & NO(2) & $1.50$ & $(6,1)$ & -- & 6762\\
$(154,65)$ & 11 & $(3,1)$ & 2 & 1 & YES & YES & NO(2) & $1.73$ & $(4,2)$ & NO & 6763\\
$(154,65)$ & 11 & $(3,1)$ & 2 & 1 & YES & YES & NO(2) & $1.73$ & $(4,2)$ & -- & 6764\\
$(154,57)$ & 12 & $(4,1)$ & 3 & 2 & YES & YES & NO(2) & $1.44$ & $(4,2)$ & NO & 6765\\
$(154,43)$ & 11 & $(5,2)$ & 3 & 1 & YES & YES & YES & $1.56$ & $(2,3)$ & NO & 6766\\
$(154,45)$ & 11 & $(5,2)$ & 3 & 1 & YES & YES & NO(2) & $1.56$ & $(2,3)$ & -- & 6767\\
$(154,59)$ & 11 & $(5,2)$ & 3 & 1 & YES & YES & YES & $1.75$ & $(2,3)$ & -- & 6768\\
$(154,65)$ & 11 & $(5,2)$ & 3 & 1 & YES & YES & NO(2) & $1.60$ & $(2,3)$ & NO & 6769\\
$(154,65)$ & 11 & $(5,2)$ & 3 & 1 & YES & YES & NO(2) & $1.67$ & $(8,0)$ & -- & 6770\\
$(154,43)$ & 11 & $(7,3)$ & 4 & 7 & YES & YES & YES & $1.78$ & $(2,3)$ & NO & 6771\\
$(154,43)$ & 11 & $(7,3)$ & 4 & 7 & YES & YES & YES & $1.78$ & $(2,3)$ & -- & 6772\\
$(154,45)$ & 11 & $(7,2)$ & 4 & 7 & YES & YES & YES & $1.57$ & $(4,2)$ & NO & 6773\\
$(154,45)$ & 11 & $(7,2)$ & 4 & 7 & YES & YES & YES & $1.57$ & $(4,2)$ & -- & 6774\\
$(154,47)$ & 11 & $(7,2)$ & 4 & 7 & YES & YES & YES & $1.43$ & $(6,1)$ & -- & 6775\\
$(154,47)$ & 11 & $(7,3)$ & 4 & 7 & YES & YES & YES & $1.75$ & $(2,3)$ & -- & 6776\\
$(154,59)$ & 11 & $(7,2)$ & 4 & 7 & YES & YES & YES & $1.57$ & $(4,2)$ & NO & 6777\\
$(154,59)$ & 11 & $(7,2)$ & 4 & 7 & YES & YES & YES & $1.57$ & $(4,2)$ & -- & 6778\\
$(154,65)$ & 11 & $(7,2)$ & 4 & 7 & YES & YES & YES & $1.50$ & $(4,2)$ & -- & 6779\\
$(154,65)$ & 11 & $(7,2)$ & 4 & 7 & YES & YES & YES & $1.62$ & $(4,2)$ & NO & 6780\\
$(154,45)$ & 11 & $(8,3)$ & 4 & 2 & YES & YES & YES & $1.78$ & $(2,3)$ & -- & 6781\\
$(154,47)$ & 11 & $(9,2)$ & 5 & 1 & YES & YES & NO(2) & $1.67$ & $(2,3)$ & NO & 6782\\
$(154,45)$ & 11 & $(10,3)$ & 5 & 2 & YES & YES & YES & $1.57$ & $(2,3)$ & -- & 6783\\
$(154,47)$ & 11 & $(10,3)$ & 5 & 2 & YES & YES & NO(2) & $1.62$ & $(6,1)$ & NO & 6784\\
$(154,47)$ & 11 & $(10,3)$ & 5 & 2 & YES & YES & YES & $1.57$ & $(2,3)$ & -- & 6785\\
$(154,45)$ & 11 & $(11,3)$ & 5 & 11 & YES & YES & YES & $1.56$ & $(2,3)$ & -- & 6786\\
$(154,65)$ & 11 & $(17,7)$ & 6 & 1 & YES & YES & YES & $1.89$ & $(2,3)$ & NO & 6787\\
$(154,65)$ & 11 & $(26,11)$ & 7 & 2 & YES & YES & NO(2) & $1.80$ & $(6,1)$ & NO & 6788\\
$(154,43)$ & 11 & $(29,8)$ & 7 & 1 & YES & YES & YES & $1.14$ & $(4,2)$ & NO & 6789\\
$(154,45)$ & 11 & $(29,8)$ & 7 & 1 & YES & YES & YES & $1.43$ & $(6,1)$ & NO & 6790\\
$(154,45)$ & 11 & $(37,11)$ & 8 & 1 & YES & YES & YES & $1.80$ & $(2,3)$ & NO & 6791\\
$(154,45)$ & 11 & $(41,12)$ & 8 & 1 & YES & YES & YES & $1.57$ & $(2,3)$ & NO & 6792\\
$(154,45)$ & 11 & $(44,13)$ & 8 & 22 & YES & YES & YES & $1.67$ & $(2,3)$ & NO & 6793\\
$(154,59)$ & 11 & $(55,21)$ & 8 & 11 & YES & YES & YES & $1.62$ & $(2,3)$ & NO & 6794\\
$(154,47)$ & 11 & $(56,17)$ & 9 & 14 & YES & YES & YES & $1.75$ & $(2,3)$ & NO & 6795\\
$(154,45)$ & 11 & $(58,17)$ & 9 & 2 & YES & YES & NO(2) & $1.56$ & $(2,3)$ & NO & 6796\\
$(154,59)$ & 11 & $(60,23)$ & 9 & 2 & YES & YES & NO(2) & $1.56$ & $(4,2)$ & 7129 & 6797\\
$(154,65)$ & 11 & $(64,27)$ & 9 & 2 & YES & YES & NO(2) & $1.56$ & $(2,3)$ & 7306 & 6798\\
$(154,43)$ & 11 & $(68,19)$ & 9 & 2 & YES & YES & YES & $1.62$ & $(2,3)$ & 7472 & 6799\\
$(154,45)$ & 11 & $(69,20)$ & 10 & 1 & YES & YES & YES & $1.43$ & $(6,1)$ & NO & 6800\\
$(154,45)$ & 11 & $(147,43)$ & 11 & 7 & YES & YES & YES & $1.70$ & $(2,3)$ & NO & 6801\\
$(155,64)$ & 11 & $(2,1)$ & 1 & 1 & YES & YES & NO(2) & $1.64$ & $(4,2)$ & -- & 6802\\
$(155,64)$ & 11 & $(3,1)$ & 2 & 1 & YES & YES & NO(2) & $1.56$ & $(2,3)$ & NO & 6803\\
$(155,64)$ & 11 & $(3,1)$ & 2 & 1 & YES & YES & NO(2) & $1.56$ & $(2,3)$ & -- & 6804\\
$(155,64)$ & 11 & $(3,1)$ & 2 & 1 & YES & YES & NO(2) & $1.80$ & $(6,1)$ & NO & 6805\\
$(155,46)$ & 11 & $(5,1)$ & 4 & 5 & YES & YES & YES & $1.57$ & $(2,3)$ & NO & 6806\\
$(155,46)$ & 11 & $(5,1)$ & 4 & 5 & YES & YES & YES & $1.57$ & $(2,3)$ & -- & 6807\\
$(155,46)$ & 11 & $(5,2)$ & 3 & 5 & YES & YES & NO(2) & $1.56$ & $(2,3)$ & NO & 6808\\
$(155,46)$ & 11 & $(5,2)$ & 3 & 5 & YES & YES & YES & $1.67$ & $(2,3)$ & -- & 6809\\
$(155,64)$ & 11 & $(5,2)$ & 3 & 5 & YES & YES & YES & $1.43$ & $(4,2)$ & -- & 6810\\
$(155,68)$ & 11 & $(5,2)$ & 3 & 5 & YES & YES & NO(2) & $1.56$ & $(4,2)$ & -- & 6811\\
$(155,46)$ & 11 & $(7,3)$ & 4 & 1 & YES & YES & NO(2) & $1.60$ & $(6,1)$ & -- & 6812\\
$(155,47)$ & 12 & $(7,2)$ & 4 & 1 & YES & YES & NO(2) & $1.38$ & $(4,2)$ & -- & 6813\\
$(155,57)$ & 11 & $(7,2)$ & 4 & 1 & YES & YES & YES & $1.75$ & $(2,3)$ & -- & 6814\\
$(155,64)$ & 11 & $(7,2)$ & 4 & 1 & YES & YES & YES & $1.57$ & $(4,2)$ & -- & 6815\\
$(155,64)$ & 11 & $(7,2)$ & 4 & 1 & YES & YES & YES & $1.75$ & $(4,2)$ & NO & 6816\\
$(155,64)$ & 11 & $(7,3)$ & 4 & 1 & YES & YES & YES & $1.75$ & $(4,2)$ & -- & 6817\\
$(155,33)$ & 12 & $(8,3)$ & 4 & 1 & YES & YES & NO(2) & $1.33$ & $(8,0)$ & -- & 6818\\
$(155,68)$ & 11 & $(8,3)$ & 4 & 1 & YES & YES & YES & $1.62$ & $(4,2)$ & NO & 6819\\
$(155,64)$ & 11 & $(9,2)$ & 5 & 1 & YES & YES & YES & $1.43$ & $(2,3)$ & -- & 6820\\
$(155,64)$ & 11 & $(9,2)$ & 5 & 1 & YES & YES & YES & $1.71$ & $(2,3)$ & NO & 6821\\
$(155,48)$ & 12 & $(10,3)$ & 5 & 5 & YES & YES & NO(2) & $1.78$ & $(4,2)$ & NO & 6822\\
$(155,64)$ & 11 & $(10,3)$ & 5 & 5 & YES & YES & YES & $1.88$ & $(4,2)$ & NO & 6823\\
$(155,57)$ & 11 & $(12,5)$ & 5 & 1 & YES & YES & YES & $1.75$ & $(4,2)$ & NO & 6824\\
$(155,64)$ & 11 & $(12,5)$ & 5 & 1 & YES & YES & NO(2) & $1.70$ & $(6,1)$ & NO & 6825\\
$(155,47)$ & 12 & $(13,2)$ & 7 & 1 & YES & YES & NO(2) & $1.56$ & $(4,2)$ & NO & 6826\\
$(155,42)$ & 12 & $(14,3)$ & 6 & 1 & YES & YES & NO(2) & $1.50$ & $(6,1)$ & NO & 6827\\
$(155,47)$ & 12 & $(17,5)$ & 6 & 1 & YES & YES & NO(2) & $1.50$ & $(4,2)$ & NO & 6828\\
$(155,56)$ & 12 & $(19,7)$ & 6 & 1 & YES & YES & NO(2) & $1.70$ & $(6,1)$ & NO & 6829\\
$(155,64)$ & 11 & $(19,8)$ & 6 & 1 & YES & YES & YES & $1.78$ & $(2,3)$ & NO & 6830\\
$(155,33)$ & 12 & $(22,5)$ & 7 & 1 & YES & YES & NO(2) & $1.44$ & $(8,0)$ & NO & 6831\\
$(155,68)$ & 11 & $(23,10)$ & 7 & 1 & YES & YES & NO(2) & $1.56$ & $(4,2)$ & NO & 6832\\
$(155,42)$ & 12 & $(34,9)$ & 8 & 1 & YES & YES & NO(2) & $1.50$ & $(6,1)$ & NO & 6833\\
$(155,42)$ & 12 & $(56,15)$ & 9 & 1 & YES & YES & YES & $1.75$ & $(4,2)$ & NO & 6834\\
$(155,34)$ & 12 & $(60,13)$ & 9 & 5 & YES & YES & YES & $1.62$ & $(4,2)$ & NO & 6835\\
$(155,57)$ & 11 & $(68,25)$ & 9 & 1 & YES & YES & YES & $1.43$ & $(2,3)$ & NO & 6836\\
$(155,68)$ & 11 & $(73,32)$ & 10 & 1 & YES & YES & NO(2) & $1.67$ & $(4,2)$ & NO & 6837\\
$(155,57)$ & 11 & $(79,29)$ & 9 & 1 & YES & YES & YES & $1.62$ & $(2,3)$ & NO & 6838\\
$(155,42)$ & 12 & $(81,22)$ & 12 & 1 & YES & YES & NO(2) & $1.50$ & $(6,1)$ & NO & 6839\\
$(155,33)$ & 12 & $(85,18)$ & 10 & 5 & YES & YES & NO(2) & $1.60$ & $(6,1)$ & NO & 6840\\
$(155,57)$ & 11 & $(87,32)$ & 10 & 1 & YES & YES & NO(2) & $1.38$ & $(10,-1)$ & NO & 6841\\
$(155,68)$ & 11 & $(139,61)$ & 11 & 1 & YES & YES & YES & $1.62$ & $(4,2)$ & NO & 6842\\
$(155,57)$ & 11 & $(155,57)$ & 11 & 155 & YES & YES & NO(2) & $1.56$ & $(8,0)$ & NO & 6843\\
$(155,64)$ & 11 & $(155,64)$ & 11 & 155 & YES & YES & NO(2) & $1.60$ & $(2,3)$ & NO & 6844\\
$(156,59)$ & 12 & $(4,1)$ & 3 & 4 & YES & YES & NO(2) & $1.43$ & $(8,0)$ & -- & 6845\\
$(156,59)$ & 12 & $(4,1)$ & 3 & 4 & YES & YES & NO(2) & $1.56$ & $(8,0)$ & NO & 6846\\
$(156,59)$ & 12 & $(119,45)$ & 11 & 1 & YES & YES & NO(2) & $1.67$ & $(8,0)$ & NO & 6847\\
$(157,46)$ & 11 & $(2,1)$ & 1 & 1 & YES & YES & NO(2) & $1.38$ & $(6,1)$ & NO & 6848\\
$(157,58)$ & 11 & $(2,1)$ & 1 & 1 & YES & YES & YES & $1.43$ & $(4,2)$ & NO & 6849\\
$(157,58)$ & 11 & $(2,1)$ & 1 & 1 & YES & YES & NO(2) & $1.56$ & $(4,2)$ & -- & 6850\\
$(157,46)$ & 11 & $(3,1)$ & 2 & 1 & YES & YES & NO(2) & $1.80$ & $(2,3)$ & NO & 6851\\
$(157,46)$ & 11 & $(3,1)$ & 2 & 1 & YES & YES & NO(2) & $1.80$ & $(2,3)$ & -- & 6852\\
$(157,58)$ & 11 & $(3,1)$ & 2 & 1 & YES & YES & NO(2) & $1.67$ & $(4,2)$ & -- & 6853\\
$(157,58)$ & 11 & $(3,1)$ & 2 & 1 & YES & YES & NO(2) & $1.67$ & $(4,2)$ & NO & 6854\\
$(157,44)$ & 13 & $(5,1)$ & 4 & 1 & YES & YES & NO(2) & $1.44$ & $(4,2)$ & -- & 6855\\
$(157,46)$ & 11 & $(5,2)$ & 3 & 1 & YES & YES & YES & $1.29$ & $(4,2)$ & -- & 6856\\
$(157,46)$ & 11 & $(5,2)$ & 3 & 1 & YES & YES & YES & $1.43$ & $(4,2)$ & NO & 6857\\
$(157,58)$ & 11 & $(5,1)$ & 4 & 1 & YES & YES & NO(2) & $1.60$ & $(2,3)$ & -- & 6858\\
$(157,58)$ & 11 & $(5,2)$ & 3 & 1 & YES & YES & NO(2) & $1.67$ & $(4,2)$ & NO & 6859\\
$(157,58)$ & 11 & $(5,2)$ & 3 & 1 & YES & YES & YES & $1.78$ & $(2,3)$ & -- & 6860\\
$(157,69)$ & 11 & $(5,2)$ & 3 & 1 & YES & YES & YES & $1.57$ & $(2,3)$ & NO & 6861\\
$(157,46)$ & 11 & $(7,3)$ & 4 & 1 & YES & YES & YES & $1.62$ & $(4,2)$ & -- & 6862\\
$(157,46)$ & 11 & $(7,3)$ & 4 & 1 & YES & YES & YES & $1.75$ & $(4,2)$ & NO & 6863\\
$(157,69)$ & 11 & $(7,2)$ & 4 & 1 & YES & YES & YES & $1.57$ & $(4,2)$ & -- & 6864\\
$(157,46)$ & 11 & $(8,3)$ & 4 & 1 & YES & YES & YES & $1.78$ & $(2,3)$ & -- & 6865\\
$(157,58)$ & 11 & $(8,3)$ & 4 & 1 & YES & YES & NO(2) & $1.67$ & $(4,2)$ & NO & 6866\\
$(157,69)$ & 11 & $(8,3)$ & 4 & 1 & YES & YES & YES & $1.88$ & $(4,2)$ & NO & 6867\\
$(157,69)$ & 11 & $(8,3)$ & 4 & 1 & YES & YES & YES & $1.88$ & $(4,2)$ & -- & 6868\\
$(157,66)$ & 11 & $(9,2)$ & 5 & 1 & YES & YES & YES & $1.75$ & $(2,3)$ & NO & 6869\\
$(157,66)$ & 11 & $(9,2)$ & 5 & 1 & YES & YES & YES & $1.75$ & $(2,3)$ & -- & 6870\\
$(157,66)$ & 11 & $(9,2)$ & 5 & 1 & YES & YES & YES & $1.75$ & $(2,3)$ & NO & 6871\\
$(157,46)$ & 11 & $(10,3)$ & 5 & 1 & YES & YES & YES & $1.57$ & $(2,3)$ & -- & 6872\\
$(157,46)$ & 11 & $(13,3)$ & 6 & 1 & YES & YES & YES & $1.67$ & $(2,3)$ & -- & 6873\\
$(157,46)$ & 11 & $(17,5)$ & 6 & 1 & YES & YES & NO(2) & $1.70$ & $(2,3)$ & NO & 6874\\
$(157,58)$ & 11 & $(27,10)$ & 7 & 1 & YES & YES & NO(2) & $1.67$ & $(4,2)$ & NO & 6875\\
$(157,46)$ & 11 & $(37,11)$ & 8 & 1 & YES & YES & YES & $1.67$ & $(2,3)$ & NO & 6876\\
$(157,69)$ & 11 & $(41,18)$ & 8 & 1 & YES & YES & YES & $1.57$ & $(2,3)$ & NO & 6877\\
$(157,69)$ & 11 & $(57,25)$ & 9 & 1 & YES & YES & YES & $1.71$ & $(4,2)$ & NO & 6878\\
$(157,46)$ & 11 & $(58,17)$ & 9 & 1 & YES & YES & NO(2) & $1.25$ & $(6,1)$ & NO & 6879\\
$(157,46)$ & 11 & $(65,19)$ & 9 & 1 & YES & YES & YES & $1.50$ & $(2,3)$ & NO & 6880\\
$(157,66)$ & 11 & $(81,34)$ & 9 & 1 & YES & YES & YES & $1.75$ & $(2,3)$ & NO & 6881\\
$(157,46)$ & 11 & $(89,26)$ & 10 & 1 & YES & YES & YES & $1.80$ & $(2,3)$ & NO & 6882\\
$(157,46)$ & 11 & $(106,31)$ & 10 & 1 & YES & YES & YES & $1.67$ & $(2,3)$ & NO & 6883\\
$(157,69)$ & 11 & $(107,47)$ & 10 & 1 & YES & YES & YES & $1.88$ & $(4,2)$ & NO & 6884\\
$(157,58)$ & 11 & $(111,41)$ & 10 & 1 & YES & YES & NO(2) & $1.38$ & $(6,1)$ & NO & 6885\\
$(157,58)$ & 11 & $(157,58)$ & 11 & 157 & YES & YES & NO(2) & $1.60$ & $(2,3)$ & NO & 6886\\
$(157,69)$ & 11 & $(157,69)$ & 11 & 157 & YES & YES & YES & $1.57$ & $(2,3)$ & NO & 6887\\
$(158,57)$ & 11 & $(4,1)$ & 3 & 2 & YES & YES & NO(2) & $1.64$ & $(4,2)$ & -- & 6888\\
$(158,61)$ & 11 & $(5,2)$ & 3 & 1 & YES & YES & YES & $1.57$ & $(4,2)$ & -- & 6889\\
$(158,61)$ & 11 & $(7,2)$ & 4 & 1 & YES & YES & YES & $1.80$ & $(2,3)$ & -- & 6890\\
$(158,61)$ & 11 & $(7,2)$ & 4 & 1 & YES & YES & YES & $1.67$ & $(2,3)$ & NO & 6891\\
$(158,61)$ & 11 & $(7,3)$ & 4 & 1 & YES & YES & YES & $1.80$ & $(2,3)$ & -- & 6892\\
$(158,57)$ & 11 & $(13,5)$ & 5 & 1 & YES & YES & YES & $1.43$ & $(4,2)$ & NO & 6893\\
$(158,57)$ & 11 & $(14,5)$ & 6 & 2 & YES & YES & NO(2) & $1.70$ & $(2,3)$ & NO & 6894\\
$(158,61)$ & 11 & $(49,19)$ & 8 & 1 & YES & YES & YES & $1.67$ & $(2,3)$ & 9442 & 6895\\
$(158,61)$ & 11 & $(75,29)$ & 9 & 1 & YES & YES & YES & $2.00$ & $(2,3)$ & NO & 6896\\
$(158,57)$ & 11 & $(97,35)$ & 10 & 1 & YES & YES & NO(2) & $1.38$ & $(6,1)$ & NO & 6897\\
$(158,61)$ & 11 & $(127,49)$ & 11 & 1 & YES & YES & YES & $1.90$ & $(2,3)$ & NO & 6898\\
$(158,61)$ & 11 & $(145,56)$ & 11 & 1 & YES & YES & YES & $1.80$ & $(2,3)$ & NO & 6899\\
$(159,47)$ & 11 & $(3,1)$ & 2 & 3 & YES & YES & NO(2) & $1.50$ & $(2,3)$ & NO & 6900\\
$(159,59)$ & 11 & $(3,1)$ & 2 & 3 & YES & YES & NO(2) & $1.64$ & $(4,2)$ & -- & 6901\\
$(159,62)$ & 11 & $(3,1)$ & 2 & 3 & YES & YES & NO(2) & $1.55$ & $(4,2)$ & -- & 6902\\
$(159,44)$ & 11 & $(5,2)$ & 3 & 1 & YES & YES & YES & $1.50$ & $(2,3)$ & NO & 6903\\
$(159,44)$ & 11 & $(5,2)$ & 3 & 1 & YES & YES & YES & $1.67$ & $(2,3)$ & -- & 6904\\
$(159,47)$ & 11 & $(5,2)$ & 3 & 1 & YES & YES & YES & $1.62$ & $(2,3)$ & -- & 6905\\
$(159,47)$ & 11 & $(5,2)$ & 3 & 1 & YES & YES & YES & $1.71$ & $(4,2)$ & NO & 6906\\
$(159,47)$ & 11 & $(5,2)$ & 3 & 1 & YES & YES & YES & $1.78$ & $(2,3)$ & NO & 6907\\
$(159,59)$ & 11 & $(5,2)$ & 3 & 1 & YES & YES & NO(2) & $1.73$ & $(4,2)$ & NO & 6908\\
$(159,62)$ & 11 & $(5,1)$ & 4 & 1 & YES & YES & NO(2) & $1.44$ & $(4,2)$ & NO & 6909\\
$(159,62)$ & 11 & $(5,2)$ & 3 & 1 & YES & YES & YES & $2.00$ & $(2,3)$ & -- & 6910\\
$(159,61)$ & 12 & $(6,1)$ & 5 & 3 & YES & YES & NO(2) & $1.62$ & $(6,1)$ & NO & 6911\\
$(159,44)$ & 11 & $(7,3)$ & 4 & 1 & YES & YES & YES & $1.75$ & $(2,3)$ & NO & 6912\\
$(159,44)$ & 11 & $(7,3)$ & 4 & 1 & YES & YES & YES & $1.62$ & $(4,2)$ & -- & 6913\\
$(159,47)$ & 11 & $(7,2)$ & 4 & 1 & YES & YES & YES & $1.50$ & $(2,3)$ & -- & 6914\\
$(159,47)$ & 11 & $(7,3)$ & 4 & 1 & YES & YES & YES & $1.75$ & $(2,3)$ & NO & 6915\\
$(159,47)$ & 11 & $(7,3)$ & 4 & 1 & YES & YES & YES & $1.78$ & $(2,3)$ & -- & 6916\\
$(159,59)$ & 11 & $(7,3)$ & 4 & 1 & YES & YES & YES & $1.57$ & $(2,3)$ & NO & 6917\\
$(159,62)$ & 11 & $(7,2)$ & 4 & 1 & YES & YES & YES & $1.57$ & $(4,2)$ & -- & 6918\\
$(159,62)$ & 11 & $(7,3)$ & 4 & 1 & YES & YES & YES & $1.88$ & $(4,2)$ & -- & 6919\\
$(159,37)$ & 12 & $(8,3)$ & 4 & 1 & YES & YES & YES & $1.50$ & $(4,2)$ & -- & 6920\\
$(159,37)$ & 12 & $(8,3)$ & 4 & 1 & YES & YES & YES & $1.50$ & $(2,3)$ & NO & 6921\\
$(159,44)$ & 11 & $(8,3)$ & 4 & 1 & YES & YES & YES & $1.78$ & $(2,3)$ & NO & 6922\\
$(159,44)$ & 11 & $(8,3)$ & 4 & 1 & YES & YES & YES & $1.80$ & $(2,3)$ & -- & 6923\\
$(159,59)$ & 11 & $(8,3)$ & 4 & 1 & YES & YES & NO(2) & $1.73$ & $(4,2)$ & 6036 & 6924\\
$(159,47)$ & 11 & $(9,2)$ & 5 & 3 & YES & YES & YES & $1.38$ & $(2,3)$ & -- & 6925\\
$(159,47)$ & 11 & $(9,2)$ & 5 & 3 & YES & YES & YES & $1.38$ & $(4,2)$ & NO & 6926\\
$(159,47)$ & 11 & $(11,3)$ & 5 & 1 & YES & YES & YES & $1.50$ & $(2,3)$ & NO & 6927\\
$(159,62)$ & 11 & $(11,4)$ & 5 & 1 & YES & YES & YES & $1.57$ & $(4,2)$ & NO & 6928\\
$(159,44)$ & 11 & $(17,5)$ & 6 & 1 & YES & YES & YES & $1.75$ & $(2,3)$ & NO & 6929\\
$(159,47)$ & 11 & $(18,5)$ & 6 & 3 & YES & YES & YES & $1.67$ & $(2,3)$ & NO & 6930\\
$(159,61)$ & 12 & $(21,8)$ & 6 & 3 & YES & YES & NO(2) & $1.62$ & $(6,1)$ & NO & 6931\\
$(159,62)$ & 11 & $(21,8)$ & 6 & 3 & YES & YES & YES & $1.43$ & $(4,2)$ & NO & 6932\\
$(159,47)$ & 11 & $(23,7)$ & 7 & 1 & YES & YES & YES & $1.75$ & $(2,3)$ & NO & 6933\\
$(159,43)$ & 12 & $(25,7)$ & 7 & 1 & YES & YES & NO(2) & $1.50$ & $(6,1)$ & NO & 6934\\
$(159,59)$ & 11 & $(27,10)$ & 7 & 3 & YES & YES & NO(2) & $1.64$ & $(4,2)$ & NO & 6935\\
$(159,47)$ & 11 & $(37,11)$ & 8 & 1 & YES & YES & YES & $1.50$ & $(2,3)$ & 8473 & 6936\\
$(159,44)$ & 11 & $(51,14)$ & 9 & 3 & YES & YES & YES & $1.62$ & $(4,2)$ & NO & 6937\\
$(159,47)$ & 11 & $(61,18)$ & 9 & 1 & YES & YES & YES & $1.50$ & $(2,3)$ & NO & 6938\\
$(159,44)$ & 11 & $(69,19)$ & 9 & 3 & YES & YES & YES & $1.80$ & $(2,3)$ & NO & 6939\\
$(159,44)$ & 11 & $(105,29)$ & 10 & 3 & YES & YES & YES & $1.67$ & $(2,3)$ & NO & 6940\\
$(160,67)$ & 11 & $(2,1)$ & 1 & 2 & YES & YES & NO(2) & $1.60$ & $(6,1)$ & -- & 6941\\
$(160,67)$ & 11 & $(5,2)$ & 3 & 5 & YES & YES & YES & $1.67$ & $(4,2)$ & -- & 6942\\
$(160,43)$ & 11 & $(7,2)$ & 4 & 1 & YES & YES & YES & $1.29$ & $(2,3)$ & -- & 6943\\
$(160,67)$ & 11 & $(7,2)$ & 4 & 1 & YES & YES & YES & $1.67$ & $(4,2)$ & -- & 6944\\
$(160,43)$ & 11 & $(11,4)$ & 5 & 1 & YES & YES & YES & $1.75$ & $(4,2)$ & NO & 6945\\
$(160,67)$ & 11 & $(26,11)$ & 7 & 2 & YES & YES & YES & $1.67$ & $(4,2)$ & NO & 6946\\
$(160,67)$ & 11 & $(50,21)$ & 8 & 10 & YES & YES & YES & $1.67$ & $(4,2)$ & 9754 & 6947\\
$(160,67)$ & 11 & $(117,49)$ & 10 & 1 & YES & YES & NO(2) & $1.56$ & $(4,2)$ & NO & 6948\\
$(160,67)$ & 11 & $(160,67)$ & 11 & 160 & YES & YES & NO(2) & $1.56$ & $(4,2)$ & NO & 6949\\
$(161,68)$ & 11 & $(2,1)$ & 1 & 1 & YES & YES & NO(2) & $1.60$ & $(6,1)$ & -- & 6950\\
$(161,61)$ & 11 & $(3,1)$ & 2 & 1 & YES & YES & NO(2) & $1.56$ & $(4,2)$ & -- & 6951\\
$(161,66)$ & 11 & $(5,1)$ & 4 & 1 & YES & YES & NO(2) & $1.64$ & $(4,2)$ & -- & 6952\\
$(161,66)$ & 11 & $(5,2)$ & 3 & 1 & YES & YES & YES & $1.88$ & $(2,3)$ & -- & 6953\\
$(161,45)$ & 11 & $(7,2)$ & 4 & 7 & YES & YES & NO(2) & $1.73$ & $(4,2)$ & NO & 6954\\
$(161,45)$ & 11 & $(7,3)$ & 4 & 7 & YES & YES & YES & $1.67$ & $(4,2)$ & -- & 6955\\
$(161,61)$ & 11 & $(7,2)$ & 4 & 7 & YES & YES & YES & $1.29$ & $(4,2)$ & -- & 6956\\
$(161,45)$ & 11 & $(10,3)$ & 5 & 1 & YES & YES & YES & $1.86$ & $(2,3)$ & -- & 6957\\
$(161,61)$ & 11 & $(13,5)$ & 5 & 1 & YES & YES & YES & $1.71$ & $(2,3)$ & NO & 6958\\
$(161,66)$ & 11 & $(17,7)$ & 6 & 1 & YES & YES & NO(2) & $1.73$ & $(4,2)$ & 6693 & 6959\\
$(161,68)$ & 11 & $(17,7)$ & 6 & 1 & YES & YES & YES & $1.62$ & $(4,2)$ & NO & 6960\\
$(161,61)$ & 11 & $(34,13)$ & 7 & 1 & YES & YES & YES & $1.29$ & $(4,2)$ & NO & 6961\\
$(161,61)$ & 11 & $(37,14)$ & 8 & 1 & YES & YES & NO(2) & $1.64$ & $(4,2)$ & NO & 6962\\
$(161,68)$ & 11 & $(97,41)$ & 10 & 1 & YES & YES & YES & $1.57$ & $(4,2)$ & 9828 & 6963\\
$(161,45)$ & 11 & $(104,29)$ & 10 & 1 & YES & YES & YES & $1.50$ & $(4,2)$ & NO & 6964\\
$(161,68)$ & 11 & $(116,49)$ & 10 & 1 & YES & YES & NO(2) & $1.44$ & $(4,2)$ & NO & 6965\\
$(161,66)$ & 11 & $(139,57)$ & 11 & 1 & YES & YES & YES & $2.00$ & $(2,3)$ & NO & 6966\\
$(162,73)$ & 12 & $(3,1)$ & 2 & 3 & YES & YES & NO(2) & $1.62$ & $(6,1)$ & -- & 6967\\
$(162,73)$ & 12 & $(4,1)$ & 3 & 2 & YES & YES & NO(2) & $1.50$ & $(6,1)$ & 3788 & 6968\\
$(162,73)$ & 12 & $(4,1)$ & 3 & 2 & YES & YES & NO(2) & $1.67$ & $(4,2)$ & -- & 6969\\
$(162,37)$ & 12 & $(8,3)$ & 4 & 2 & YES & YES & YES & $1.89$ & $(2,3)$ & NO & 6970\\
$(162,37)$ & 12 & $(8,3)$ & 4 & 2 & YES & YES & YES & $1.89$ & $(2,3)$ & -- & 6971\\
$(162,37)$ & 12 & $(10,3)$ & 5 & 2 & YES & YES & YES & $1.75$ & $(2,3)$ & NO & 6972\\
$(162,49)$ & 12 & $(11,2)$ & 6 & 1 & YES & YES & NO(2) & $1.50$ & $(4,2)$ & NO & 6973\\
$(162,49)$ & 12 & $(16,3)$ & 7 & 2 & YES & YES & YES & $1.62$ & $(4,2)$ & NO & 6974\\
$(162,47)$ & 13 & $(24,7)$ & 7 & 6 & YES & YES & YES & $1.62$ & $(2,3)$ & NO & 6975\\
$(162,73)$ & 12 & $(91,41)$ & 11 & 1 & YES & YES & NO(2) & $1.62$ & $(6,1)$ & NO & 6976\\
$(162,47)$ & 13 & $(100,29)$ & 11 & 2 & YES & YES & YES & $1.75$ & $(2,3)$ & 8751 & 6977\\
$(162,73)$ & 12 & $(162,73)$ & 12 & 162 & YES & YES & NO(2) & $1.50$ & $(6,1)$ & NO & 6978\\
$(163,63)$ & 11 & $(2,1)$ & 1 & 1 & YES & YES & NO(2) & $1.44$ & $(4,2)$ & NO & 6979\\
$(163,63)$ & 11 & $(2,1)$ & 1 & 1 & YES & YES & NO(2) & $1.75$ & $(2,3)$ & -- & 6980\\
$(163,71)$ & 11 & $(2,1)$ & 1 & 1 & YES & YES & NO(2) & $1.56$ & $(4,2)$ & NO & 6981\\
$(163,71)$ & 11 & $(2,1)$ & 1 & 1 & YES & YES & NO(2) & $1.56$ & $(4,2)$ & -- & 6982\\
$(163,44)$ & 11 & $(3,1)$ & 2 & 1 & YES & YES & NO(2) & $1.50$ & $(6,1)$ & NO & 6983\\
$(163,44)$ & 11 & $(3,1)$ & 2 & 1 & YES & YES & NO(2) & $1.50$ & $(6,1)$ & -- & 6984\\
$(163,63)$ & 11 & $(3,1)$ & 2 & 1 & YES & YES & NO(2) & $1.64$ & $(4,2)$ & -- & 6985\\
$(163,71)$ & 11 & $(3,1)$ & 2 & 1 & YES & YES & NO(2) & $1.60$ & $(6,1)$ & NO & 6986\\
$(163,71)$ & 11 & $(3,1)$ & 2 & 1 & YES & YES & NO(2) & $1.60$ & $(6,1)$ & -- & 6987\\
$(163,39)$ & 13 & $(4,1)$ & 3 & 1 & YES & YES & NO(2) & $1.50$ & $(6,1)$ & -- & 6988\\
$(163,39)$ & 13 & $(4,1)$ & 3 & 1 & YES & YES & NO(2) & $1.62$ & $(6,1)$ & NO & 6989\\
$(163,44)$ & 11 & $(4,1)$ & 3 & 1 & YES & YES & NO(2) & $1.60$ & $(6,1)$ & NO & 6990\\
$(163,44)$ & 11 & $(4,1)$ & 3 & 1 & YES & YES & NO(2) & $1.60$ & $(6,1)$ & -- & 6991\\
$(163,44)$ & 11 & $(4,1)$ & 3 & 1 & YES & YES & NO(2) & $1.60$ & $(6,1)$ & 5438 & 6992\\
$(163,63)$ & 11 & $(4,1)$ & 3 & 1 & YES & YES & NO(2) & $1.29$ & $(6,1)$ & -- & 6993\\
$(163,63)$ & 11 & $(4,1)$ & 3 & 1 & YES & YES & YES & $1.44$ & $(2,3)$ & NO & 6994\\
$(163,62)$ & 11 & $(5,2)$ & 3 & 1 & YES & YES & YES & $1.57$ & $(6,1)$ & -- & 6995\\
$(163,63)$ & 11 & $(5,2)$ & 3 & 1 & YES & YES & NO(2) & $1.44$ & $(4,2)$ & NO & 6996\\
$(163,71)$ & 11 & $(5,2)$ & 3 & 1 & YES & YES & YES & $1.50$ & $(4,2)$ & -- & 6997\\
$(163,45)$ & 12 & $(6,1)$ & 5 & 1 & YES & YES & YES & $1.57$ & $(2,3)$ & NO & 6998\\
$(163,63)$ & 11 & $(7,2)$ & 4 & 1 & YES & YES & NO(2) & $1.67$ & $(2,3)$ & -- & 6999\\
$(163,71)$ & 11 & $(7,2)$ & 4 & 1 & YES & YES & YES & $1.62$ & $(4,2)$ & NO & 7000\\
$(163,71)$ & 11 & $(7,2)$ & 4 & 1 & YES & YES & YES & $1.62$ & $(4,2)$ & -- & 7001\\
$(163,71)$ & 11 & $(7,3)$ & 4 & 1 & YES & YES & YES & $1.57$ & $(6,1)$ & -- & 7002\\
$(163,44)$ & 11 & $(8,3)$ & 4 & 1 & YES & YES & NO(2) & $1.60$ & $(6,1)$ & NO & 7003\\
$(163,63)$ & 11 & $(8,3)$ & 4 & 1 & YES & YES & NO(2) & $1.56$ & $(4,2)$ & NO & 7004\\
$(163,63)$ & 11 & $(9,2)$ & 5 & 1 & YES & YES & YES & $1.50$ & $(4,2)$ & NO & 7005\\
$(163,63)$ & 11 & $(9,2)$ & 5 & 1 & YES & YES & YES & $1.78$ & $(2,3)$ & NO & 7006\\
$(163,63)$ & 11 & $(9,2)$ & 5 & 1 & YES & YES & YES & $1.78$ & $(2,3)$ & -- & 7007\\
$(163,44)$ & 11 & $(10,3)$ & 5 & 1 & YES & YES & YES & $1.62$ & $(2,3)$ & -- & 7008\\
$(163,44)$ & 11 & $(11,3)$ & 5 & 1 & YES & YES & NO(2) & $1.70$ & $(2,3)$ & 6313 & 7009\\
$(163,63)$ & 11 & $(13,5)$ & 5 & 1 & YES & YES & NO(2) & $1.70$ & $(2,3)$ & NO & 7010\\
$(163,63)$ & 11 & $(18,7)$ & 6 & 1 & YES & YES & NO(2) & $1.75$ & $(2,3)$ & NO & 7011\\
$(163,63)$ & 11 & $(23,9)$ & 7 & 1 & YES & YES & NO(2) & $1.67$ & $(2,3)$ & NO & 7012\\
$(163,63)$ & 11 & $(31,12)$ & 7 & 1 & YES & YES & NO(2) & $1.75$ & $(2,3)$ & NO & 7013\\
$(163,63)$ & 11 & $(49,19)$ & 8 & 1 & YES & YES & YES & $1.89$ & $(2,3)$ & 9683 & 7014\\
$(163,39)$ & 13 & $(54,13)$ & 11 & 1 & YES & YES & NO(2) & $1.50$ & $(6,1)$ & NO & 7015\\
$(163,71)$ & 11 & $(55,24)$ & 9 & 1 & YES & YES & YES & $1.62$ & $(4,2)$ & NO & 7016\\
$(163,63)$ & 11 & $(75,29)$ & 9 & 1 & YES & YES & YES & $1.50$ & $(2,3)$ & 7895 & 7017\\
$(163,62)$ & 11 & $(79,30)$ & 9 & 1 & YES & YES & YES & $1.57$ & $(2,3)$ & NO & 7018\\
$(163,71)$ & 11 & $(101,44)$ & 10 & 1 & YES & YES & YES & $1.57$ & $(2,3)$ & NO & 7019\\
$(163,59)$ & 12 & $(105,38)$ & 11 & 1 & YES & YES & YES & $1.50$ & $(2,3)$ & NO & 7020\\
$(163,63)$ & 11 & $(106,41)$ & 10 & 1 & YES & YES & YES & $1.89$ & $(2,3)$ & 9909 & 7021\\
$(163,63)$ & 11 & $(119,46)$ & 10 & 1 & YES & YES & YES & $1.44$ & $(2,3)$ & NO & 7022\\
$(163,45)$ & 12 & $(134,37)$ & 11 & 1 & YES & YES & YES & $1.57$ & $(2,3)$ & NO & 7023\\
$(163,45)$ & 12 & $(163,45)$ & 12 & 163 & YES & YES & NO(2) & $1.50$ & $(10,-1)$ & NO & 7024\\
$(163,63)$ & 11 & $(163,63)$ & 11 & 163 & YES & YES & NO(2) & $1.38$ & $(4,2)$ & NO & 7025\\
$(164,45)$ & 12 & $(3,1)$ & 2 & 1 & YES & YES & NO(2) & $1.56$ & $(12,-2)$ & -- & 7026\\
$(164,45)$ & 12 & $(3,1)$ & 2 & 1 & YES & YES & NO(2) & $1.67$ & $(12,-2)$ & NO & 7027\\
$(164,43)$ & 12 & $(4,1)$ & 3 & 4 & YES & YES & NO(2) & $1.56$ & $(8,0)$ & -- & 7028\\
$(164,75)$ & 13 & $(6,1)$ & 5 & 2 & YES & YES & NO(2) & $1.62$ & $(6,1)$ & NO & 7029\\
$(164,75)$ & 13 & $(6,1)$ & 5 & 2 & YES & YES & NO(2) & $1.62$ & $(6,1)$ & NO & 7030\\
$(164,37)$ & 13 & $(10,3)$ & 5 & 2 & YES & YES & YES & $1.43$ & $(6,1)$ & -- & 7031\\
$(164,51)$ & 12 & $(16,5)$ & 7 & 4 & YES & YES & NO(2) & $1.67$ & $(4,2)$ & NO & 7032\\
$(164,45)$ & 12 & $(18,5)$ & 6 & 2 & YES & YES & NO(2) & $1.56$ & $(12,-2)$ & 4971 & 7033\\
$(164,37)$ & 13 & $(48,11)$ & 9 & 4 & YES & YES & YES & $1.67$ & $(4,2)$ & NO & 7034\\
$(164,75)$ & 13 & $(164,75)$ & 13 & 164 & YES & YES & NO(2) & $1.62$ & $(6,1)$ & NO & 7035\\
$(165,61)$ & 11 & $(2,1)$ & 1 & 1 & YES & YES & NO(2) & $1.60$ & $(6,1)$ & NO & 7036\\
$(165,64)$ & 11 & $(2,1)$ & 1 & 1 & YES & YES & NO(2) & $1.60$ & $(6,1)$ & -- & 7037\\
$(165,61)$ & 11 & $(3,1)$ & 2 & 3 & YES & YES & NO(2) & $1.67$ & $(4,2)$ & NO & 7038\\
$(165,61)$ & 11 & $(3,1)$ & 2 & 3 & YES & YES & NO(2) & $1.67$ & $(4,2)$ & -- & 7039\\
$(165,64)$ & 11 & $(4,1)$ & 3 & 1 & YES & YES & NO(2) & $1.56$ & $(8,0)$ & NO & 7040\\
$(165,46)$ & 11 & $(5,2)$ & 3 & 5 & YES & YES & YES & $1.38$ & $(2,3)$ & -- & 7041\\
$(165,49)$ & 11 & $(5,2)$ & 3 & 5 & YES & YES & YES & $1.67$ & $(2,3)$ & -- & 7042\\
$(165,61)$ & 11 & $(5,2)$ & 3 & 5 & YES & YES & NO(2) & $1.67$ & $(4,2)$ & -- & 7043\\
$(165,64)$ & 11 & $(5,2)$ & 3 & 5 & YES & YES & YES & $1.71$ & $(4,2)$ & -- & 7044\\
$(165,64)$ & 11 & $(5,2)$ & 3 & 5 & YES & YES & YES & $1.78$ & $(2,3)$ & NO & 7045\\
$(165,46)$ & 11 & $(7,3)$ & 4 & 1 & YES & YES & YES & $1.78$ & $(2,3)$ & NO & 7046\\
$(165,46)$ & 11 & $(7,3)$ & 4 & 1 & YES & YES & YES & $1.56$ & $(2,3)$ & -- & 7047\\
$(165,49)$ & 11 & $(7,3)$ & 4 & 1 & YES & YES & YES & $1.62$ & $(4,2)$ & -- & 7048\\
$(165,61)$ & 11 & $(7,3)$ & 4 & 1 & YES & YES & NO(2) & $1.50$ & $(6,1)$ & NO & 7049\\
$(165,64)$ & 11 & $(7,2)$ & 4 & 1 & YES & YES & YES & $1.75$ & $(4,2)$ & NO & 7050\\
$(165,64)$ & 11 & $(7,2)$ & 4 & 1 & YES & YES & YES & $1.75$ & $(4,2)$ & -- & 7051\\
$(165,64)$ & 11 & $(7,3)$ & 4 & 1 & YES & YES & YES & $1.75$ & $(4,2)$ & -- & 7052\\
$(165,46)$ & 11 & $(10,3)$ & 5 & 5 & YES & YES & YES & $1.43$ & $(2,3)$ & -- & 7053\\
$(165,49)$ & 11 & $(11,3)$ & 5 & 11 & YES & YES & YES & $1.50$ & $(2,3)$ & NO & 7054\\
$(165,49)$ & 11 & $(24,7)$ & 7 & 3 & YES & YES & YES & $1.50$ & $(2,3)$ & NO & 7055\\
$(165,49)$ & 11 & $(44,13)$ & 8 & 11 & YES & YES & YES & $1.50$ & $(2,3)$ & NO & 7056\\
$(165,46)$ & 11 & $(57,16)$ & 9 & 3 & YES & YES & YES & $1.67$ & $(2,3)$ & NO & 7057\\
$(165,61)$ & 11 & $(73,27)$ & 9 & 1 & YES & YES & YES & $1.62$ & $(2,3)$ & 7830 & 7058\\
$(165,64)$ & 11 & $(85,33)$ & 10 & 5 & YES & YES & YES & $1.78$ & $(2,3)$ & 9886 & 7059\\
$(165,49)$ & 11 & $(91,27)$ & 10 & 1 & YES & YES & YES & $1.50$ & $(2,3)$ & NO & 7060\\
$(165,61)$ & 11 & $(165,61)$ & 11 & 165 & YES & YES & YES & $1.38$ & $(2,3)$ & NO & 7061\\
$(165,64)$ & 11 & $(165,64)$ & 11 & 165 & YES & YES & NO(2) & $1.56$ & $(8,0)$ & NO & 7062\\
$(166,49)$ & 11 & $(2,1)$ & 1 & 2 & YES & YES & NO(2) & $1.33$ & $(4,2)$ & -- & 7063\\
$(166,61)$ & 11 & $(3,1)$ & 2 & 1 & YES & YES & NO(2) & $1.56$ & $(4,2)$ & NO & 7064\\
$(166,61)$ & 11 & $(3,1)$ & 2 & 1 & YES & YES & NO(2) & $1.56$ & $(4,2)$ & -- & 7065\\
$(166,49)$ & 11 & $(5,2)$ & 3 & 1 & YES & YES & NO(2) & $1.38$ & $(6,1)$ & NO & 7066\\
$(166,49)$ & 11 & $(5,2)$ & 3 & 1 & YES & YES & YES & $1.62$ & $(2,3)$ & -- & 7067\\
$(166,61)$ & 11 & $(5,2)$ & 3 & 1 & YES & YES & NO(2) & $1.60$ & $(6,1)$ & -- & 7068\\
$(166,69)$ & 13 & $(6,1)$ & 5 & 2 & YES & YES & NO(2) & $1.70$ & $(6,1)$ & NO & 7069\\
$(166,49)$ & 11 & $(7,2)$ & 4 & 1 & YES & YES & YES & $1.62$ & $(2,3)$ & -- & 7070\\
$(166,49)$ & 11 & $(7,2)$ & 4 & 1 & YES & YES & YES & $1.88$ & $(2,3)$ & NO & 7071\\
$(166,49)$ & 11 & $(7,3)$ & 4 & 1 & YES & YES & YES & $1.56$ & $(4,2)$ & -- & 7072\\
$(166,49)$ & 11 & $(11,3)$ & 5 & 1 & YES & YES & YES & $1.50$ & $(2,3)$ & NO & 7073\\
$(166,69)$ & 13 & $(17,7)$ & 6 & 1 & YES & YES & NO(2) & $1.62$ & $(6,1)$ & NO & 7074\\
$(166,61)$ & 11 & $(30,11)$ & 7 & 2 & YES & YES & YES & $1.50$ & $(2,3)$ & NO & 7075\\
$(166,61)$ & 11 & $(41,15)$ & 8 & 1 & YES & YES & YES & $1.75$ & $(2,3)$ & NO & 7076\\
$(166,63)$ & 12 & $(50,19)$ & 8 & 2 & YES & YES & YES & $1.71$ & $(2,3)$ & NO & 7077\\
$(166,61)$ & 11 & $(52,19)$ & 9 & 2 & YES & YES & YES & $1.75$ & $(4,2)$ & NO & 7078\\
$(166,61)$ & 11 & $(68,25)$ & 9 & 2 & YES & YES & NO(2) & $1.70$ & $(6,1)$ & 7627 & 7079\\
$(166,49)$ & 11 & $(71,21)$ & 9 & 1 & YES & YES & YES & $1.62$ & $(2,3)$ & NO & 7080\\
$(166,61)$ & 11 & $(79,29)$ & 9 & 1 & YES & YES & YES & $1.57$ & $(4,2)$ & NO & 7081\\
$(166,61)$ & 11 & $(166,61)$ & 11 & 166 & YES & YES & YES & $1.43$ & $(4,2)$ & NO & 7082\\
$(167,49)$ & 12 & $(2,1)$ & 1 & 1 & YES & YES & NO(2) & $1.73$ & $(4,2)$ & -- & 7083\\
$(167,64)$ & 11 & $(2,1)$ & 1 & 1 & YES & YES & NO(2) & $1.56$ & $(4,2)$ & -- & 7084\\
$(167,69)$ & 11 & $(2,1)$ & 1 & 1 & YES & YES & NO(2) & $1.75$ & $(2,3)$ & NO & 7085\\
$(167,46)$ & 11 & $(3,1)$ & 2 & 1 & YES & YES & NO(2) & $1.50$ & $(6,1)$ & NO & 7086\\
$(167,46)$ & 11 & $(3,1)$ & 2 & 1 & YES & YES & NO(2) & $1.50$ & $(6,1)$ & -- & 7087\\
$(167,60)$ & 11 & $(3,1)$ & 2 & 1 & YES & YES & NO(2) & $1.60$ & $(2,3)$ & -- & 7088\\
$(167,61)$ & 12 & $(3,1)$ & 2 & 1 & YES & YES & NO(2) & $1.70$ & $(2,3)$ & -- & 7089\\
$(167,64)$ & 11 & $(3,1)$ & 2 & 1 & YES & YES & YES & $1.57$ & $(2,3)$ & -- & 7090\\
$(167,69)$ & 11 & $(3,1)$ & 2 & 1 & YES & YES & NO(2) & $1.38$ & $(6,1)$ & NO & 7091\\
$(167,69)$ & 11 & $(3,1)$ & 2 & 1 & YES & YES & NO(2) & $1.38$ & $(6,1)$ & -- & 7092\\
$(167,46)$ & 11 & $(4,1)$ & 3 & 1 & YES & YES & YES & $1.62$ & $(2,3)$ & -- & 7093\\
$(167,49)$ & 12 & $(4,1)$ & 3 & 1 & YES & YES & NO(2) & $1.56$ & $(4,2)$ & -- & 7094\\
$(167,49)$ & 12 & $(4,1)$ & 3 & 1 & YES & YES & NO(2) & $1.67$ & $(4,2)$ & NO & 7095\\
$(167,69)$ & 11 & $(4,1)$ & 3 & 1 & YES & YES & YES & $1.43$ & $(4,2)$ & NO & 7096\\
$(167,69)$ & 11 & $(4,1)$ & 3 & 1 & YES & YES & YES & $1.75$ & $(6,1)$ & -- & 7097\\
$(167,46)$ & 11 & $(5,2)$ & 3 & 1 & YES & YES & NO(2) & $1.56$ & $(2,3)$ & -- & 7098\\
$(167,51)$ & 12 & $(5,2)$ & 3 & 1 & YES & YES & NO(2) & $1.44$ & $(8,0)$ & -- & 7099\\
$(167,60)$ & 11 & $(5,2)$ & 3 & 1 & YES & YES & YES & $1.75$ & $(4,2)$ & -- & 7100\\
$(167,61)$ & 12 & $(5,2)$ & 3 & 1 & YES & YES & NO(2) & $1.70$ & $(2,3)$ & NO & 7101\\
$(167,64)$ & 11 & $(5,1)$ & 4 & 1 & YES & YES & NO(2) & $1.60$ & $(2,3)$ & NO & 7102\\
$(167,64)$ & 11 & $(5,1)$ & 4 & 1 & YES & YES & NO(2) & $1.56$ & $(4,2)$ & -- & 7103\\
$(167,64)$ & 11 & $(5,1)$ & 4 & 1 & YES & YES & NO(2) & $1.64$ & $(4,2)$ & NO & 7104\\
$(167,69)$ & 11 & $(5,2)$ & 3 & 1 & YES & YES & NO(2) & $1.67$ & $(2,3)$ & -- & 7105\\
$(167,46)$ & 11 & $(7,3)$ & 4 & 1 & YES & YES & YES & $1.78$ & $(4,2)$ & -- & 7106\\
$(167,51)$ & 12 & $(7,2)$ & 4 & 1 & YES & YES & YES & $1.75$ & $(2,3)$ & -- & 7107\\
$(167,64)$ & 11 & $(7,2)$ & 4 & 1 & YES & YES & YES & $1.75$ & $(2,3)$ & NO & 7108\\
$(167,64)$ & 11 & $(7,3)$ & 4 & 1 & YES & YES & YES & $1.89$ & $(4,2)$ & -- & 7109\\
$(167,69)$ & 11 & $(7,2)$ & 4 & 1 & YES & YES & YES & $1.57$ & $(4,2)$ & NO & 7110\\
$(167,69)$ & 11 & $(7,2)$ & 4 & 1 & YES & YES & YES & $1.75$ & $(4,2)$ & -- & 7111\\
$(167,69)$ & 11 & $(7,3)$ & 4 & 1 & YES & YES & NO(2) & $1.38$ & $(6,1)$ & NO & 7112\\
$(167,30)$ & 13 & $(8,3)$ & 4 & 1 & YES & YES & NO(2) & $1.44$ & $(8,0)$ & NO & 7113\\
$(167,30)$ & 13 & $(8,3)$ & 4 & 1 & YES & YES & NO(2) & $1.44$ & $(8,0)$ & -- & 7114\\
$(167,46)$ & 11 & $(8,3)$ & 4 & 1 & YES & YES & NO(2) & $1.60$ & $(6,1)$ & NO & 7115\\
$(167,69)$ & 11 & $(8,3)$ & 4 & 1 & YES & YES & YES & $1.57$ & $(4,2)$ & NO & 7116\\
$(167,60)$ & 11 & $(9,2)$ & 5 & 1 & YES & YES & YES & $1.62$ & $(4,2)$ & -- & 7117\\
$(167,69)$ & 11 & $(9,4)$ & 5 & 1 & YES & YES & YES & $1.43$ & $(4,2)$ & NO & 7118\\
$(167,49)$ & 12 & $(10,3)$ & 5 & 1 & YES & YES & NO(2) & $1.73$ & $(4,2)$ & NO & 7119\\
$(167,51)$ & 12 & $(11,3)$ & 5 & 1 & YES & YES & YES & $1.89$ & $(2,3)$ & NO & 7120\\
$(167,69)$ & 11 & $(12,5)$ & 5 & 1 & YES & YES & NO(2) & $1.38$ & $(6,1)$ & NO & 7121\\
$(167,30)$ & 13 & $(13,3)$ & 6 & 1 & YES & YES & NO(2) & $1.44$ & $(8,0)$ & NO & 7122\\
$(167,51)$ & 12 & $(17,5)$ & 6 & 1 & YES & YES & YES & $1.89$ & $(2,3)$ & NO & 7123\\
$(167,69)$ & 11 & $(19,8)$ & 6 & 1 & YES & YES & YES & $1.62$ & $(4,2)$ & NO & 7124\\
$(167,36)$ & 12 & $(22,5)$ & 7 & 1 & YES & YES & NO(2) & $1.56$ & $(8,0)$ & NO & 7125\\
$(167,69)$ & 11 & $(29,12)$ & 7 & 1 & YES & YES & NO(2) & $1.38$ & $(6,1)$ & NO & 7126\\
$(167,64)$ & 11 & $(34,13)$ & 7 & 1 & YES & YES & YES & $1.43$ & $(2,3)$ & 8249 & 7127\\
$(167,69)$ & 11 & $(41,17)$ & 8 & 1 & YES & YES & YES & $1.62$ & $(4,2)$ & 8906 & 7128\\
$(167,64)$ & 11 & $(47,18)$ & 8 & 1 & YES & YES & NO(2) & $1.56$ & $(4,2)$ & 6797 & 7129\\
$(167,64)$ & 11 & $(60,23)$ & 9 & 1 & YES & YES & NO(2) & $1.70$ & $(2,3)$ & NO & 7130\\
$(167,46)$ & 11 & $(69,19)$ & 9 & 1 & YES & YES & YES & $1.75$ & $(2,3)$ & NO & 7131\\
$(167,64)$ & 11 & $(73,28)$ & 10 & 1 & YES & YES & YES & $1.62$ & $(2,3)$ & NO & 7132\\
$(167,69)$ & 11 & $(104,43)$ & 10 & 1 & YES & YES & YES & $1.71$ & $(2,3)$ & 10079 & 7133\\
$(167,69)$ & 11 & $(121,50)$ & 10 & 1 & YES & YES & NO(2) & $1.64$ & $(4,2)$ & NO & 7134\\
$(167,60)$ & 11 & $(142,51)$ & 11 & 1 & YES & YES & YES & $1.75$ & $(4,2)$ & NO & 7135\\
$(167,51)$ & 12 & $(154,47)$ & 11 & 1 & YES & YES & YES & $1.75$ & $(2,3)$ & 10266 & 7136\\
$(167,69)$ & 11 & $(167,69)$ & 11 & 167 & YES & YES & NO(2) & $1.38$ & $(6,1)$ & NO & 7137\\
$(168,71)$ & 11 & $(5,2)$ & 3 & 1 & YES & YES & YES & $1.57$ & $(4,2)$ & -- & 7138\\
$(168,71)$ & 11 & $(8,3)$ & 4 & 8 & YES & YES & YES & $1.90$ & $(2,3)$ & NO & 7139\\
$(168,71)$ & 11 & $(8,3)$ & 4 & 8 & YES & YES & YES & $1.78$ & $(2,3)$ & -- & 7140\\
$(169,49)$ & 12 & $(2,1)$ & 1 & 1 & YES & YES & YES & $1.29$ & $(4,2)$ & -- & 7141\\
$(169,64)$ & 11 & $(2,1)$ & 1 & 1 & YES & YES & NO(2) & $1.73$ & $(4,2)$ & -- & 7142\\
$(169,66)$ & 11 & $(2,1)$ & 1 & 1 & YES & YES & NO(2) & $1.82$ & $(4,2)$ & -- & 7143\\
$(169,70)$ & 11 & $(2,1)$ & 1 & 1 & YES & YES & NO(2) & $1.83$ & $(2,3)$ & NO & 7144\\
$(169,71)$ & 11 & $(2,1)$ & 1 & 1 & YES & YES & NO(2) & $1.73$ & $(4,2)$ & -- & 7145\\
$(169,64)$ & 11 & $(3,1)$ & 2 & 1 & YES & YES & NO(2) & $1.70$ & $(2,3)$ & -- & 7146\\
$(169,66)$ & 11 & $(3,1)$ & 2 & 1 & YES & YES & NO(2) & $1.70$ & $(2,3)$ & -- & 7147\\
$(169,70)$ & 11 & $(3,1)$ & 2 & 1 & YES & YES & NO(2) & $1.38$ & $(6,1)$ & -- & 7148\\
$(169,71)$ & 11 & $(3,1)$ & 2 & 1 & YES & YES & NO(2) & $1.56$ & $(4,2)$ & NO & 7149\\
$(169,71)$ & 11 & $(3,1)$ & 2 & 1 & YES & YES & NO(2) & $1.56$ & $(4,2)$ & -- & 7150\\
$(169,71)$ & 11 & $(3,1)$ & 2 & 1 & YES & YES & NO(2) & $1.73$ & $(4,2)$ & NO & 7151\\
$(169,49)$ & 12 & $(4,1)$ & 3 & 1 & YES & YES & NO(2) & $1.44$ & $(4,2)$ & NO & 7152\\
$(169,49)$ & 12 & $(4,1)$ & 3 & 1 & YES & YES & NO(2) & $1.25$ & $(10,-1)$ & -- & 7153\\
$(169,49)$ & 12 & $(4,1)$ & 3 & 1 & YES & YES & NO(2) & $1.38$ & $(10,-1)$ & NO & 7154\\
$(169,66)$ & 11 & $(4,1)$ & 3 & 1 & YES & YES & NO(2) & $1.73$ & $(4,2)$ & -- & 7155\\
$(169,71)$ & 11 & $(4,1)$ & 3 & 1 & YES & YES & YES & $1.57$ & $(6,1)$ & -- & 7156\\
$(169,71)$ & 11 & $(4,1)$ & 3 & 1 & YES & YES & YES & $1.71$ & $(4,2)$ & NO & 7157\\
$(169,71)$ & 11 & $(4,1)$ & 3 & 1 & YES & YES & YES & $1.86$ & $(4,2)$ & NO & 7158\\
$(169,50)$ & 11 & $(5,2)$ & 3 & 1 & YES & YES & YES & $1.62$ & $(2,3)$ & -- & 7159\\
$(169,66)$ & 11 & $(5,1)$ & 4 & 1 & YES & YES & NO(2) & $1.73$ & $(4,2)$ & NO & 7160\\
$(169,66)$ & 11 & $(5,1)$ & 4 & 1 & YES & YES & NO(2) & $1.73$ & $(4,2)$ & NO & 7161\\
$(169,66)$ & 11 & $(5,1)$ & 4 & 1 & YES & YES & NO(2) & $1.70$ & $(6,1)$ & -- & 7162\\
$(169,70)$ & 11 & $(5,2)$ & 3 & 1 & YES & YES & YES & $1.62$ & $(2,3)$ & -- & 7163\\
$(169,70)$ & 11 & $(5,2)$ & 3 & 1 & YES & YES & YES & $1.43$ & $(4,2)$ & NO & 7164\\
$(169,71)$ & 11 & $(5,2)$ & 3 & 1 & YES & YES & YES & $1.75$ & $(2,3)$ & -- & 7165\\
$(169,32)$ & 13 & $(7,3)$ & 4 & 1 & YES & YES & NO(2) & $1.38$ & $(4,2)$ & NO & 7166\\
$(169,32)$ & 13 & $(7,3)$ & 4 & 1 & YES & YES & NO(2) & $1.62$ & $(4,2)$ & -- & 7167\\
$(169,50)$ & 11 & $(7,3)$ & 4 & 1 & YES & YES & YES & $1.43$ & $(4,2)$ & -- & 7168\\
$(169,50)$ & 11 & $(7,3)$ & 4 & 1 & YES & YES & YES & $1.57$ & $(4,2)$ & NO & 7169\\
$(169,70)$ & 11 & $(7,2)$ & 4 & 1 & YES & YES & YES & $1.62$ & $(2,3)$ & -- & 7170\\
$(169,70)$ & 11 & $(7,3)$ & 4 & 1 & YES & YES & NO(2) & $1.38$ & $(6,1)$ & NO & 7171\\
$(169,50)$ & 11 & $(8,3)$ & 4 & 1 & YES & YES & YES & $1.71$ & $(2,3)$ & -- & 7172\\
$(169,70)$ & 11 & $(8,3)$ & 4 & 1 & YES & YES & YES & $1.88$ & $(4,2)$ & -- & 7173\\
$(169,71)$ & 11 & $(8,3)$ & 4 & 1 & YES & YES & YES & $1.62$ & $(2,3)$ & NO & 7174\\
$(169,70)$ & 11 & $(10,3)$ & 5 & 1 & YES & YES & YES & $1.62$ & $(4,2)$ & -- & 7175\\
$(169,64)$ & 11 & $(13,5)$ & 5 & 13 & YES & YES & YES & $1.43$ & $(2,3)$ & 5499 & 7176\\
$(169,70)$ & 11 & $(26,11)$ & 7 & 13 & YES & YES & YES & $1.75$ & $(4,2)$ & NO & 7177\\
$(169,64)$ & 11 & $(29,11)$ & 7 & 1 & YES & YES & NO(2) & $1.60$ & $(2,3)$ & NO & 7178\\
$(169,49)$ & 12 & $(38,11)$ & 9 & 1 & YES & YES & NO(2) & $1.44$ & $(4,2)$ & NO & 7179\\
$(169,66)$ & 11 & $(41,16)$ & 8 & 1 & YES & YES & NO(2) & $1.60$ & $(2,3)$ & 6601 & 7180\\
$(169,70)$ & 11 & $(46,19)$ & 8 & 1 & YES & YES & YES & $1.43$ & $(4,2)$ & NO & 7181\\
$(169,70)$ & 11 & $(53,22)$ & 9 & 1 & YES & YES & YES & $1.71$ & $(4,2)$ & NO & 7182\\
$(169,50)$ & 11 & $(61,18)$ & 9 & 1 & YES & YES & YES & $1.50$ & $(2,3)$ & NO & 7183\\
$(169,64)$ & 11 & $(66,25)$ & 9 & 1 & YES & YES & NO(2) & $1.60$ & $(2,3)$ & NO & 7184\\
$(169,71)$ & 11 & $(69,29)$ & 9 & 1 & YES & YES & NO(2) & $1.44$ & $(2,3)$ & 7730 & 7185\\
$(169,64)$ & 11 & $(103,39)$ & 10 & 1 & YES & YES & NO(2) & $1.44$ & $(8,0)$ & NO & 7186\\
$(169,66)$ & 11 & $(105,41)$ & 10 & 1 & YES & YES & NO(2) & $1.73$ & $(4,2)$ & NO & 7187\\
$(169,70)$ & 11 & $(111,46)$ & 10 & 1 & YES & YES & YES & $1.75$ & $(2,3)$ & NO & 7188\\
$(169,71)$ & 11 & $(119,50)$ & 10 & 1 & YES & YES & NO(2) & $1.64$ & $(4,2)$ & NO & 7189\\
$(169,71)$ & 11 & $(169,71)$ & 11 & 169 & YES & YES & NO(2) & $1.73$ & $(4,2)$ & NO & 7190\\
$(170,47)$ & 11 & $(5,2)$ & 3 & 5 & YES & YES & YES & $1.50$ & $(2,3)$ & NO & 7191\\
$(170,47)$ & 11 & $(5,2)$ & 3 & 5 & YES & YES & YES & $1.38$ & $(2,3)$ & -- & 7192\\
$(170,47)$ & 11 & $(7,3)$ & 4 & 1 & YES & YES & YES & $1.43$ & $(4,2)$ & -- & 7193\\
$(170,47)$ & 11 & $(7,3)$ & 4 & 1 & YES & YES & YES & $1.78$ & $(2,3)$ & NO & 7194\\
$(170,61)$ & 12 & $(7,2)$ & 4 & 1 & YES & YES & NO(2) & $1.50$ & $(6,1)$ & -- & 7195\\
$(170,47)$ & 11 & $(8,3)$ & 4 & 2 & YES & YES & YES & $1.78$ & $(2,3)$ & NO & 7196\\
$(170,47)$ & 11 & $(10,3)$ & 5 & 10 & YES & YES & YES & $1.38$ & $(2,3)$ & NO & 7197\\
$(170,47)$ & 11 & $(16,3)$ & 7 & 2 & YES & YES & YES & $1.57$ & $(2,3)$ & NO & 7198\\
$(170,47)$ & 11 & $(25,7)$ & 7 & 5 & YES & YES & YES & $1.50$ & $(2,3)$ & NO & 7199\\
$(170,47)$ & 11 & $(40,11)$ & 8 & 10 & YES & YES & YES & $1.50$ & $(2,3)$ & 8851 & 7200\\
$(170,47)$ & 11 & $(65,18)$ & 9 & 5 & YES & YES & YES & $1.50$ & $(2,3)$ & NO & 7201\\
$(170,47)$ & 11 & $(105,29)$ & 10 & 5 & YES & YES & YES & $1.56$ & $(2,3)$ & 10197 & 7202\\
$(171,53)$ & 12 & $(2,1)$ & 1 & 1 & YES & YES & YES & $1.43$ & $(4,2)$ & NO & 7203\\
$(171,65)$ & 11 & $(2,1)$ & 1 & 1 & YES & YES & YES & $1.43$ & $(4,2)$ & NO & 7204\\
$(171,65)$ & 11 & $(2,1)$ & 1 & 1 & YES & YES & NO(2) & $1.67$ & $(4,2)$ & -- & 7205\\
$(171,71)$ & 12 & $(2,1)$ & 1 & 1 & YES & YES & YES & $1.75$ & $(2,3)$ & -- & 7206\\
$(171,53)$ & 12 & $(3,1)$ & 2 & 3 & YES & YES & NO(2) & $1.60$ & $(2,3)$ & -- & 7207\\
$(171,65)$ & 11 & $(3,1)$ & 2 & 3 & YES & YES & NO(2) & $1.56$ & $(2,3)$ & -- & 7208\\
$(171,65)$ & 11 & $(3,1)$ & 2 & 3 & YES & YES & NO(2) & $1.67$ & $(2,3)$ & NO & 7209\\
$(171,71)$ & 12 & $(3,1)$ & 2 & 3 & YES & YES & YES & $1.62$ & $(2,3)$ & NO & 7210\\
$(171,71)$ & 12 & $(3,1)$ & 2 & 3 & YES & YES & YES & $1.62$ & $(2,3)$ & -- & 7211\\
$(171,65)$ & 11 & $(4,1)$ & 3 & 1 & YES & YES & YES & $1.43$ & $(2,3)$ & -- & 7212\\
$(171,65)$ & 11 & $(4,1)$ & 3 & 1 & YES & YES & YES & $1.43$ & $(6,1)$ & NO & 7213\\
$(171,50)$ & 11 & $(5,2)$ & 3 & 1 & YES & YES & YES & $1.50$ & $(2,3)$ & -- & 7214\\
$(171,50)$ & 11 & $(5,2)$ & 3 & 1 & YES & YES & YES & $1.75$ & $(2,3)$ & NO & 7215\\
$(171,65)$ & 11 & $(5,1)$ & 4 & 1 & YES & YES & YES & $1.56$ & $(2,3)$ & NO & 7216\\
$(171,65)$ & 11 & $(5,1)$ & 4 & 1 & YES & YES & YES & $1.56$ & $(2,3)$ & -- & 7217\\
$(171,65)$ & 11 & $(5,2)$ & 3 & 1 & YES & YES & NO(2) & $1.67$ & $(4,2)$ & NO & 7218\\
$(171,65)$ & 11 & $(5,2)$ & 3 & 1 & YES & YES & YES & $1.57$ & $(4,2)$ & -- & 7219\\
$(171,50)$ & 11 & $(7,2)$ & 4 & 1 & YES & YES & NO(2) & $1.50$ & $(2,3)$ & NO & 7220\\
$(171,50)$ & 11 & $(7,2)$ & 4 & 1 & YES & YES & NO(2) & $1.67$ & $(4,2)$ & -- & 7221\\
$(171,50)$ & 11 & $(7,3)$ & 4 & 1 & YES & YES & YES & $1.62$ & $(4,2)$ & -- & 7222\\
$(171,65)$ & 11 & $(7,3)$ & 4 & 1 & YES & YES & YES & $1.89$ & $(2,3)$ & 8887 & 7223\\
$(171,65)$ & 11 & $(7,3)$ & 4 & 1 & YES & YES & YES & $1.89$ & $(2,3)$ & -- & 7224\\
$(171,50)$ & 11 & $(8,3)$ & 4 & 1 & YES & YES & YES & $1.57$ & $(2,3)$ & -- & 7225\\
$(171,65)$ & 11 & $(8,3)$ & 4 & 1 & YES & YES & NO(2) & $1.50$ & $(6,1)$ & NO & 7226\\
$(171,71)$ & 12 & $(8,3)$ & 4 & 1 & YES & YES & NO(2) & $1.50$ & $(6,1)$ & NO & 7227\\
$(171,65)$ & 11 & $(9,2)$ & 5 & 9 & YES & YES & YES & $1.62$ & $(4,2)$ & NO & 7228\\
$(171,40)$ & 12 & $(10,3)$ & 5 & 1 & YES & YES & YES & $1.75$ & $(2,3)$ & -- & 7229\\
$(171,40)$ & 12 & $(10,3)$ & 5 & 1 & YES & YES & YES & $1.29$ & $(4,2)$ & NO & 7230\\
$(171,50)$ & 11 & $(13,3)$ & 6 & 1 & YES & YES & YES & $1.43$ & $(4,2)$ & NO & 7231\\
$(171,50)$ & 11 & $(13,4)$ & 6 & 1 & YES & YES & YES & $1.50$ & $(4,2)$ & 9803 & 7232\\
$(171,65)$ & 11 & $(18,7)$ & 6 & 9 & YES & YES & YES & $1.67$ & $(2,3)$ & NO & 7233\\
$(171,65)$ & 11 & $(29,11)$ & 7 & 1 & YES & YES & NO(2) & $1.60$ & $(2,3)$ & NO & 7234\\
$(171,65)$ & 11 & $(37,14)$ & 8 & 1 & YES & YES & YES & $1.75$ & $(2,3)$ & NO & 7235\\
$(171,71)$ & 12 & $(41,17)$ & 8 & 1 & YES & YES & NO(2) & $1.56$ & $(2,3)$ & NO & 7236\\
$(171,46)$ & 13 & $(67,18)$ & 9 & 1 & YES & YES & NO(2) & $1.56$ & $(8,0)$ & NO & 7237\\
$(171,50)$ & 11 & $(75,22)$ & 10 & 3 & YES & YES & YES & $1.67$ & $(2,3)$ & NO & 7238\\
$(171,50)$ & 11 & $(89,26)$ & 10 & 1 & YES & YES & YES & $1.62$ & $(2,3)$ & NO & 7239\\
$(171,50)$ & 11 & $(99,29)$ & 10 & 9 & YES & YES & YES & $1.62$ & $(4,2)$ & NO & 7240\\
$(171,50)$ & 11 & $(113,33)$ & 11 & 1 & YES & YES & YES & $1.50$ & $(4,2)$ & NO & 7241\\
$(171,65)$ & 11 & $(121,46)$ & 10 & 1 & YES & YES & NO(2) & $1.38$ & $(4,2)$ & NO & 7242\\
$(171,50)$ & 11 & $(154,45)$ & 11 & 1 & YES & YES & YES & $1.67$ & $(2,3)$ & NO & 7243\\
$(171,71)$ & 12 & $(171,71)$ & 12 & 171 & YES & YES & NO(2) & $1.90$ & $(2,3)$ & NO & 7244\\
$(172,71)$ & 11 & $(2,1)$ & 1 & 2 & YES & YES & NO(2) & $1.60$ & $(2,3)$ & -- & 7245\\
$(172,63)$ & 11 & $(3,1)$ & 2 & 1 & YES & YES & NO(2) & $1.50$ & $(6,1)$ & -- & 7246\\
$(172,71)$ & 11 & $(3,1)$ & 2 & 1 & YES & YES & NO(2) & $1.56$ & $(4,2)$ & -- & 7247\\
$(172,63)$ & 11 & $(5,2)$ & 3 & 1 & YES & YES & NO(2) & $1.56$ & $(8,0)$ & NO & 7248\\
$(172,63)$ & 11 & $(5,2)$ & 3 & 1 & YES & YES & YES & $1.43$ & $(4,2)$ & -- & 7249\\
$(172,71)$ & 11 & $(5,2)$ & 3 & 1 & YES & YES & NO(2) & $1.60$ & $(6,1)$ & -- & 7250\\
$(172,71)$ & 11 & $(7,2)$ & 4 & 1 & YES & YES & YES & $1.89$ & $(2,3)$ & NO & 7251\\
$(172,71)$ & 11 & $(7,2)$ & 4 & 1 & YES & YES & YES & $1.89$ & $(2,3)$ & -- & 7252\\
$(172,71)$ & 11 & $(7,2)$ & 4 & 1 & YES & YES & YES & $1.43$ & $(2,3)$ & NO & 7253\\
$(172,71)$ & 11 & $(7,3)$ & 4 & 1 & YES & YES & YES & $1.43$ & $(2,3)$ & NO & 7254\\
$(172,77)$ & 12 & $(7,3)$ & 4 & 1 & YES & YES & NO(2) & $1.56$ & $(8,0)$ & NO & 7255\\
$(172,71)$ & 11 & $(9,2)$ & 5 & 1 & YES & YES & YES & $1.43$ & $(2,3)$ & NO & 7256\\
$(172,71)$ & 11 & $(9,2)$ & 5 & 1 & YES & YES & YES & $1.71$ & $(2,3)$ & NO & 7257\\
$(172,71)$ & 11 & $(9,2)$ & 5 & 1 & YES & YES & YES & $1.71$ & $(2,3)$ & -- & 7258\\
$(172,77)$ & 12 & $(11,5)$ & 6 & 1 & YES & YES & NO(2) & $1.50$ & $(6,1)$ & 5564 & 7259\\
$(172,71)$ & 11 & $(12,5)$ & 5 & 4 & YES & YES & NO(2) & $1.60$ & $(2,3)$ & NO & 7260\\
$(172,63)$ & 11 & $(19,7)$ & 6 & 1 & YES & YES & NO(2) & $1.60$ & $(6,1)$ & NO & 7261\\
$(172,77)$ & 12 & $(20,9)$ & 7 & 4 & YES & YES & NO(2) & $1.50$ & $(6,1)$ & NO & 7262\\
$(172,71)$ & 11 & $(29,12)$ & 7 & 1 & YES & YES & YES & $1.43$ & $(2,3)$ & 7944 & 7263\\
$(172,71)$ & 11 & $(41,17)$ & 8 & 1 & YES & YES & YES & $1.89$ & $(2,3)$ & 9884 & 7264\\
$(172,71)$ & 11 & $(75,31)$ & 9 & 1 & YES & YES & YES & $1.62$ & $(4,2)$ & NO & 7265\\
$(172,77)$ & 12 & $(105,47)$ & 11 & 1 & YES & YES & NO(2) & $1.50$ & $(6,1)$ & NO & 7266\\
$(172,63)$ & 11 & $(112,41)$ & 10 & 4 & YES & YES & YES & $1.80$ & $(2,3)$ & NO & 7267\\
$(172,71)$ & 11 & $(155,64)$ & 11 & 1 & YES & YES & YES & $1.29$ & $(4,2)$ & NO & 7268\\
$(172,71)$ & 11 & $(172,71)$ & 11 & 172 & YES & YES & NO(2) & $1.56$ & $(4,2)$ & NO & 7269\\
$(172,77)$ & 12 & $(172,77)$ & 12 & 172 & YES & YES & NO(2) & $1.50$ & $(6,1)$ & NO & 7270\\
$(173,66)$ & 11 & $(2,1)$ & 1 & 1 & YES & YES & NO(2) & $1.67$ & $(4,2)$ & NO & 7271\\
$(173,73)$ & 11 & $(2,1)$ & 1 & 1 & YES & YES & NO(2) & $1.60$ & $(2,3)$ & -- & 7272\\
$(173,64)$ & 11 & $(3,1)$ & 2 & 1 & YES & YES & YES & $1.29$ & $(4,2)$ & -- & 7273\\
$(173,73)$ & 11 & $(3,1)$ & 2 & 1 & YES & YES & NO(2) & $1.56$ & $(2,3)$ & -- & 7274\\
$(173,73)$ & 11 & $(3,1)$ & 2 & 1 & YES & YES & NO(2) & $1.38$ & $(6,1)$ & NO & 7275\\
$(173,76)$ & 11 & $(3,1)$ & 2 & 1 & YES & YES & YES & $1.57$ & $(2,3)$ & -- & 7276\\
$(173,66)$ & 11 & $(4,1)$ & 3 & 1 & YES & YES & YES & $1.88$ & $(6,1)$ & -- & 7277\\
$(173,71)$ & 12 & $(4,1)$ & 3 & 1 & YES & YES & NO(2) & $1.38$ & $(4,2)$ & -- & 7278\\
$(173,76)$ & 11 & $(4,1)$ & 3 & 1 & YES & YES & YES & $1.57$ & $(2,3)$ & NO & 7279\\
$(173,76)$ & 11 & $(4,1)$ & 3 & 1 & YES & YES & YES & $1.57$ & $(2,3)$ & -- & 7280\\
$(173,51)$ & 12 & $(5,1)$ & 4 & 1 & YES & YES & NO(2) & $1.50$ & $(6,1)$ & -- & 7281\\
$(173,64)$ & 11 & $(5,2)$ & 3 & 1 & YES & YES & YES & $1.56$ & $(2,3)$ & -- & 7282\\
$(173,66)$ & 11 & $(5,1)$ & 4 & 1 & YES & YES & NO(2) & $1.70$ & $(6,1)$ & NO & 7283\\
$(173,66)$ & 11 & $(5,1)$ & 4 & 1 & YES & YES & NO(2) & $1.70$ & $(6,1)$ & -- & 7284\\
$(173,66)$ & 11 & $(5,2)$ & 3 & 1 & YES & YES & YES & $1.78$ & $(2,3)$ & -- & 7285\\
$(173,73)$ & 11 & $(5,2)$ & 3 & 1 & YES & YES & NO(2) & $1.38$ & $(6,1)$ & NO & 7286\\
$(173,73)$ & 11 & $(5,2)$ & 3 & 1 & YES & YES & YES & $1.71$ & $(4,2)$ & -- & 7287\\
$(173,41)$ & 13 & $(7,2)$ & 4 & 1 & YES & YES & NO(2) & $1.56$ & $(8,0)$ & NO & 7288\\
$(173,51)$ & 12 & $(7,2)$ & 4 & 1 & YES & YES & NO(2) & $1.60$ & $(6,1)$ & NO & 7289\\
$(173,64)$ & 11 & $(7,2)$ & 4 & 1 & YES & YES & YES & $1.75$ & $(2,3)$ & -- & 7290\\
$(173,73)$ & 11 & $(8,3)$ & 4 & 1 & YES & YES & YES & $1.88$ & $(4,2)$ & -- & 7291\\
$(173,66)$ & 11 & $(9,2)$ & 5 & 1 & YES & YES & YES & $1.57$ & $(4,2)$ & NO & 7292\\
$(173,66)$ & 11 & $(9,2)$ & 5 & 1 & YES & YES & YES & $1.75$ & $(2,3)$ & -- & 7293\\
$(173,66)$ & 11 & $(9,4)$ & 5 & 1 & YES & YES & YES & $1.57$ & $(6,1)$ & NO & 7294\\
$(173,73)$ & 11 & $(9,2)$ & 5 & 1 & YES & YES & YES & $1.50$ & $(4,2)$ & -- & 7295\\
$(173,73)$ & 11 & $(9,2)$ & 5 & 1 & YES & YES & YES & $1.62$ & $(4,2)$ & NO & 7296\\
$(173,71)$ & 12 & $(12,5)$ & 5 & 1 & YES & YES & NO(2) & $1.38$ & $(4,2)$ & NO & 7297\\
$(173,73)$ & 11 & $(12,5)$ & 5 & 1 & YES & YES & NO(2) & $1.67$ & $(2,3)$ & NO & 7298\\
$(173,51)$ & 12 & $(17,5)$ & 6 & 1 & YES & YES & NO(2) & $1.60$ & $(6,1)$ & 5778 & 7299\\
$(173,66)$ & 11 & $(19,7)$ & 6 & 1 & YES & YES & YES & $1.75$ & $(4,2)$ & NO & 7300\\
$(173,66)$ & 11 & $(21,8)$ & 6 & 1 & YES & YES & NO(2) & $1.80$ & $(6,1)$ & 5910 & 7301\\
$(173,76)$ & 11 & $(25,11)$ & 7 & 1 & YES & YES & YES & $1.43$ & $(2,3)$ & NO & 7302\\
$(173,73)$ & 11 & $(26,11)$ & 7 & 1 & YES & YES & YES & $1.43$ & $(2,3)$ & 7685 & 7303\\
$(173,76)$ & 11 & $(39,17)$ & 8 & 1 & YES & YES & YES & $1.43$ & $(6,1)$ & NO & 7304\\
$(173,73)$ & 11 & $(43,18)$ & 8 & 1 & YES & YES & YES & $1.89$ & $(4,2)$ & NO & 7305\\
$(173,73)$ & 11 & $(45,19)$ & 8 & 1 & YES & YES & NO(2) & $1.56$ & $(2,3)$ & 6798 & 7306\\
$(173,64)$ & 11 & $(46,17)$ & 8 & 1 & YES & YES & YES & $1.57$ & $(4,2)$ & NO & 7307\\
$(173,66)$ & 11 & $(47,18)$ & 8 & 1 & YES & YES & YES & $1.43$ & $(4,2)$ & NO & 7308\\
$(173,73)$ & 11 & $(71,30)$ & 9 & 1 & YES & YES & YES & $1.62$ & $(4,2)$ & NO & 7309\\
$(173,66)$ & 11 & $(89,34)$ & 9 & 1 & YES & YES & YES & $1.75$ & $(2,3)$ & NO & 7310\\
$(173,73)$ & 11 & $(109,46)$ & 10 & 1 & YES & YES & NO(2) & $1.44$ & $(2,3)$ & NO & 7311\\
$(173,64)$ & 11 & $(119,44)$ & 10 & 1 & YES & YES & YES & $1.62$ & $(2,3)$ & NO & 7312\\
$(173,66)$ & 11 & $(131,50)$ & 10 & 1 & YES & YES & YES & $1.75$ & $(2,3)$ & NO & 7313\\
$(173,71)$ & 12 & $(134,55)$ & 11 & 1 & YES & YES & NO(2) & $1.38$ & $(4,2)$ & NO & 7314\\
$(173,73)$ & 11 & $(154,65)$ & 11 & 1 & YES & YES & YES & $1.62$ & $(2,3)$ & NO & 7315\\
$(173,73)$ & 11 & $(173,73)$ & 11 & 173 & YES & YES & NO(2) & $1.56$ & $(2,3)$ & NO & 7316\\
$(173,76)$ & 11 & $(173,76)$ & 11 & 173 & YES & YES & YES & $1.57$ & $(2,3)$ & NO & 7317\\
$(175,47)$ & 11 & $(2,1)$ & 1 & 1 & YES & YES & NO(2) & $1.50$ & $(2,3)$ & -- & 7318\\
$(175,47)$ & 11 & $(2,1)$ & 1 & 1 & YES & YES & NO(2) & $1.60$ & $(2,3)$ & NO & 7319\\
$(175,64)$ & 12 & $(2,1)$ & 1 & 1 & YES & YES & NO(2) & $1.64$ & $(8,0)$ & -- & 7320\\
$(175,67)$ & 11 & $(2,1)$ & 1 & 1 & YES & YES & NO(2) & $1.92$ & $(2,3)$ & -- & 7321\\
$(175,67)$ & 11 & $(3,1)$ & 2 & 1 & YES & YES & NO(2) & $1.56$ & $(2,3)$ & -- & 7322\\
$(175,67)$ & 11 & $(4,1)$ & 3 & 1 & YES & YES & YES & $1.38$ & $(2,3)$ & NO & 7323\\
$(175,67)$ & 11 & $(4,1)$ & 3 & 1 & YES & YES & YES & $1.67$ & $(2,3)$ & -- & 7324\\
$(175,47)$ & 11 & $(5,2)$ & 3 & 5 & YES & YES & NO(2) & $1.56$ & $(2,3)$ & NO & 7325\\
$(175,47)$ & 11 & $(5,2)$ & 3 & 5 & YES & YES & YES & $1.43$ & $(6,1)$ & -- & 7326\\
$(175,47)$ & 11 & $(5,2)$ & 3 & 5 & YES & YES & YES & $1.57$ & $(6,1)$ & NO & 7327\\
$(175,67)$ & 11 & $(5,2)$ & 3 & 5 & YES & YES & YES & $1.78$ & $(2,3)$ & -- & 7328\\
$(175,47)$ & 11 & $(7,2)$ & 4 & 7 & YES & YES & YES & $1.43$ & $(6,1)$ & -- & 7329\\
$(175,51)$ & 14 & $(8,1)$ & 7 & 1 & YES & YES & NO(2) & $1.38$ & $(6,1)$ & NO & 7330\\
$(175,67)$ & 11 & $(8,3)$ & 4 & 1 & YES & YES & NO(2) & $1.56$ & $(4,2)$ & NO & 7331\\
$(175,47)$ & 11 & $(10,3)$ & 5 & 5 & YES & YES & YES & $1.57$ & $(6,1)$ & NO & 7332\\
$(175,67)$ & 11 & $(11,2)$ & 6 & 1 & YES & YES & YES & $1.29$ & $(4,2)$ & NO & 7333\\
$(175,52)$ & 12 & $(13,4)$ & 6 & 1 & YES & YES & NO(2) & $1.62$ & $(4,2)$ & NO & 7334\\
$(175,47)$ & 11 & $(18,5)$ & 6 & 1 & YES & YES & YES & $1.29$ & $(6,1)$ & NO & 7335\\
$(175,67)$ & 11 & $(21,8)$ & 6 & 7 & YES & YES & NO(2) & $1.38$ & $(4,2)$ & NO & 7336\\
$(175,51)$ & 14 & $(55,16)$ & 9 & 5 & YES & YES & NO(2) & $1.38$ & $(6,1)$ & NO & 7337\\
$(175,67)$ & 11 & $(55,21)$ & 8 & 5 & YES & YES & YES & $1.78$ & $(2,3)$ & 10325 & 7338\\
$(175,47)$ & 11 & $(56,15)$ & 9 & 7 & YES & YES & NO(2) & $1.56$ & $(2,3)$ & NO & 7339\\
$(175,67)$ & 11 & $(81,31)$ & 9 & 1 & YES & YES & NO(2) & $1.56$ & $(2,3)$ & 8259 & 7340\\
$(175,47)$ & 11 & $(93,25)$ & 10 & 1 & YES & YES & NO(2) & $1.56$ & $(2,3)$ & NO & 7341\\
$(175,52)$ & 12 & $(101,30)$ & 10 & 1 & YES & YES & YES & $1.43$ & $(2,3)$ & 8911 & 7342\\
$(175,67)$ & 11 & $(115,44)$ & 10 & 5 & YES & YES & YES & $1.70$ & $(2,3)$ & 10371 & 7343\\
$(175,67)$ & 11 & $(128,49)$ & 10 & 1 & YES & YES & YES & $1.38$ & $(2,3)$ & NO & 7344\\
$(175,52)$ & 12 & $(175,52)$ & 12 & 175 & YES & YES & YES & $1.57$ & $(2,3)$ & NO & 7345\\
$(175,67)$ & 11 & $(175,67)$ & 11 & 175 & YES & YES & NO(2) & $1.56$ & $(2,3)$ & NO & 7346\\
$(176,65)$ & 11 & $(2,1)$ & 1 & 2 & YES & YES & NO(2) & $1.50$ & $(2,3)$ & -- & 7347\\
$(176,65)$ & 11 & $(3,1)$ & 2 & 1 & YES & YES & NO(2) & $1.38$ & $(6,1)$ & -- & 7348\\
$(176,65)$ & 11 & $(3,1)$ & 2 & 1 & YES & YES & YES & $1.78$ & $(4,2)$ & NO & 7349\\
$(176,69)$ & 12 & $(3,1)$ & 2 & 1 & YES & YES & NO(2) & $1.56$ & $(8,0)$ & -- & 7350\\
$(176,65)$ & 11 & $(4,1)$ & 3 & 4 & YES & YES & YES & $1.43$ & $(4,2)$ & NO & 7351\\
$(176,65)$ & 11 & $(4,1)$ & 3 & 4 & YES & YES & YES & $1.43$ & $(4,2)$ & -- & 7352\\
$(176,65)$ & 11 & $(5,2)$ & 3 & 1 & YES & YES & YES & $1.56$ & $(2,3)$ & -- & 7353\\
$(176,49)$ & 12 & $(7,2)$ & 4 & 1 & YES & YES & YES & $1.62$ & $(4,2)$ & -- & 7354\\
$(176,65)$ & 11 & $(7,2)$ & 4 & 1 & YES & YES & NO(2) & $1.50$ & $(6,1)$ & -- & 7355\\
$(176,65)$ & 11 & $(7,2)$ & 4 & 1 & YES & YES & YES & $1.62$ & $(4,2)$ & NO & 7356\\
$(176,65)$ & 11 & $(8,3)$ & 4 & 8 & YES & YES & YES & $1.50$ & $(2,3)$ & NO & 7357\\
$(176,69)$ & 12 & $(8,3)$ & 4 & 8 & YES & YES & NO(2) & $1.44$ & $(8,0)$ & NO & 7358\\
$(176,65)$ & 11 & $(11,4)$ & 5 & 11 & YES & YES & NO(2) & $1.50$ & $(6,1)$ & NO & 7359\\
$(176,69)$ & 12 & $(18,7)$ & 6 & 2 & YES & YES & NO(2) & $1.44$ & $(8,0)$ & 6596 & 7360\\
$(176,65)$ & 11 & $(27,10)$ & 7 & 1 & YES & YES & NO(2) & $1.60$ & $(2,3)$ & 7826 & 7361\\
$(176,65)$ & 11 & $(35,13)$ & 8 & 1 & YES & YES & NO(2) & $1.60$ & $(6,1)$ & 10033 & 7362\\
$(176,65)$ & 11 & $(65,24)$ & 9 & 1 & YES & YES & NO(2) & $1.56$ & $(4,2)$ & NO & 7363\\
$(176,65)$ & 11 & $(111,41)$ & 10 & 1 & YES & YES & YES & $1.43$ & $(4,2)$ & NO & 7364\\
$(176,65)$ & 11 & $(176,65)$ & 11 & 176 & YES & YES & YES & $1.43$ & $(4,2)$ & NO & 7365\\
$(177,40)$ & 13 & $(3,1)$ & 2 & 3 & YES & YES & YES & $1.43$ & $(4,2)$ & -- & 7366\\
$(177,73)$ & 12 & $(3,1)$ & 2 & 3 & YES & YES & NO(2) & $1.56$ & $(4,2)$ & -- & 7367\\
$(177,74)$ & 12 & $(3,1)$ & 2 & 3 & YES & YES & YES & $1.62$ & $(2,3)$ & NO & 7368\\
$(177,74)$ & 12 & $(3,1)$ & 2 & 3 & YES & YES & YES & $1.62$ & $(2,3)$ & -- & 7369\\
$(177,49)$ & 11 & $(5,2)$ & 3 & 1 & YES & YES & YES & $1.50$ & $(2,3)$ & -- & 7370\\
$(177,65)$ & 11 & $(5,2)$ & 3 & 1 & YES & YES & YES & $1.57$ & $(4,2)$ & -- & 7371\\
$(177,73)$ & 12 & $(5,1)$ & 4 & 1 & YES & YES & YES & $1.57$ & $(2,3)$ & NO & 7372\\
$(177,73)$ & 12 & $(5,1)$ & 4 & 1 & YES & YES & YES & $1.57$ & $(2,3)$ & -- & 7373\\
$(177,74)$ & 12 & $(5,1)$ & 4 & 1 & YES & YES & YES & $1.57$ & $(2,3)$ & -- & 7374\\
$(177,49)$ & 11 & $(7,3)$ & 4 & 1 & YES & YES & YES & $1.57$ & $(4,2)$ & NO & 7375\\
$(177,65)$ & 11 & $(7,3)$ & 4 & 1 & YES & YES & YES & $1.62$ & $(2,3)$ & NO & 7376\\
$(177,49)$ & 11 & $(8,3)$ & 4 & 1 & YES & YES & YES & $1.78$ & $(2,3)$ & NO & 7377\\
$(177,49)$ & 11 & $(10,3)$ & 5 & 1 & YES & YES & YES & $1.50$ & $(2,3)$ & NO & 7378\\
$(177,49)$ & 11 & $(13,4)$ & 6 & 1 & YES & YES & YES & $1.78$ & $(2,3)$ & NO & 7379\\
$(177,65)$ & 11 & $(13,5)$ & 5 & 1 & YES & YES & YES & $1.62$ & $(4,2)$ & NO & 7380\\
$(177,49)$ & 11 & $(17,5)$ & 6 & 1 & YES & YES & YES & $1.71$ & $(4,2)$ & NO & 7381\\
$(177,65)$ & 11 & $(27,10)$ & 7 & 3 & YES & YES & YES & $1.62$ & $(4,2)$ & NO & 7382\\
$(177,65)$ & 11 & $(30,11)$ & 7 & 3 & YES & YES & YES & $1.57$ & $(2,3)$ & NO & 7383\\
$(177,49)$ & 11 & $(40,11)$ & 8 & 1 & YES & YES & YES & $1.50$ & $(2,3)$ & 10106 & 7384\\
$(177,65)$ & 11 & $(52,19)$ & 9 & 1 & YES & YES & YES & $1.75$ & $(4,2)$ & NO & 7385\\
$(177,73)$ & 12 & $(63,26)$ & 9 & 3 & YES & YES & YES & $1.57$ & $(2,3)$ & NO & 7386\\
$(177,65)$ & 11 & $(128,47)$ & 10 & 1 & YES & YES & NO(2) & $1.60$ & $(6,1)$ & NO & 7387\\
$(177,74)$ & 12 & $(177,74)$ & 12 & 177 & YES & YES & YES & $1.43$ & $(2,3)$ & NO & 7388\\
$(178,69)$ & 11 & $(2,1)$ & 1 & 2 & YES & YES & NO(2) & $1.56$ & $(2,3)$ & -- & 7389\\
$(178,69)$ & 11 & $(2,1)$ & 1 & 2 & YES & YES & NO(2) & $1.80$ & $(6,1)$ & NO & 7390\\
$(178,69)$ & 11 & $(3,1)$ & 2 & 1 & YES & YES & NO(2) & $1.56$ & $(8,0)$ & -- & 7391\\
$(178,69)$ & 11 & $(3,1)$ & 2 & 1 & YES & YES & YES & $1.62$ & $(2,3)$ & NO & 7392\\
$(178,47)$ & 12 & $(4,1)$ & 3 & 2 & YES & YES & NO(2) & $1.60$ & $(2,3)$ & -- & 7393\\
$(178,69)$ & 11 & $(4,1)$ & 3 & 2 & YES & YES & YES & $1.90$ & $(2,3)$ & NO & 7394\\
$(178,69)$ & 11 & $(4,1)$ & 3 & 2 & YES & YES & YES & $1.90$ & $(2,3)$ & -- & 7395\\
$(178,49)$ & 11 & $(5,2)$ & 3 & 1 & YES & YES & NO(2) & $1.56$ & $(2,3)$ & NO & 7396\\
$(178,49)$ & 11 & $(5,2)$ & 3 & 1 & YES & YES & YES & $1.80$ & $(2,3)$ & -- & 7397\\
$(178,69)$ & 11 & $(5,1)$ & 4 & 1 & YES & YES & NO(2) & $1.56$ & $(2,3)$ & NO & 7398\\
$(178,69)$ & 11 & $(5,2)$ & 3 & 1 & YES & YES & NO(2) & $1.82$ & $(4,2)$ & NO & 7399\\
$(178,69)$ & 11 & $(5,2)$ & 3 & 1 & YES & YES & YES & $1.78$ & $(2,3)$ & -- & 7400\\
$(178,69)$ & 11 & $(7,2)$ & 4 & 1 & YES & YES & YES & $1.71$ & $(2,3)$ & NO & 7401\\
$(178,69)$ & 11 & $(7,2)$ & 4 & 1 & YES & YES & YES & $1.71$ & $(2,3)$ & -- & 7402\\
$(178,69)$ & 11 & $(7,3)$ & 4 & 1 & YES & YES & YES & $1.43$ & $(4,2)$ & NO & 7403\\
$(178,69)$ & 11 & $(8,3)$ & 4 & 2 & YES & YES & NO(2) & $1.38$ & $(4,2)$ & NO & 7404\\
$(178,49)$ & 11 & $(10,3)$ & 5 & 2 & YES & YES & NO(2) & $1.56$ & $(2,3)$ & NO & 7405\\
$(178,69)$ & 11 & $(13,5)$ & 5 & 1 & YES & YES & NO(2) & $1.56$ & $(4,2)$ & NO & 7406\\
$(178,47)$ & 12 & $(15,4)$ & 6 & 1 & YES & YES & YES & $1.43$ & $(4,2)$ & NO & 7407\\
$(178,63)$ & 12 & $(20,7)$ & 8 & 2 & YES & YES & NO(2) & $1.70$ & $(6,1)$ & NO & 7408\\
$(178,69)$ & 11 & $(21,8)$ & 6 & 1 & YES & YES & YES & $1.60$ & $(2,3)$ & NO & 7409\\
$(178,47)$ & 12 & $(23,6)$ & 8 & 1 & YES & YES & NO(2) & $1.50$ & $(6,1)$ & NO & 7410\\
$(178,69)$ & 11 & $(23,9)$ & 7 & 1 & YES & YES & NO(2) & $1.67$ & $(2,3)$ & NO & 7411\\
$(178,49)$ & 11 & $(25,7)$ & 7 & 1 & YES & YES & YES & $1.80$ & $(2,3)$ & NO & 7412\\
$(178,53)$ & 12 & $(27,8)$ & 7 & 1 & YES & YES & YES & $1.57$ & $(2,3)$ & NO & 7413\\
$(178,69)$ & 11 & $(31,12)$ & 7 & 1 & YES & YES & NO(2) & $1.83$ & $(2,3)$ & NO & 7414\\
$(178,69)$ & 11 & $(44,17)$ & 8 & 2 & YES & YES & YES & $1.78$ & $(2,3)$ & 9353 & 7415\\
$(178,49)$ & 11 & $(47,13)$ & 8 & 1 & YES & YES & YES & $1.56$ & $(2,3)$ & NO & 7416\\
$(178,69)$ & 11 & $(80,31)$ & 9 & 2 & YES & YES & NO(2) & $1.56$ & $(2,3)$ & 8260 & 7417\\
$(178,49)$ & 11 & $(98,27)$ & 10 & 2 & YES & YES & NO(2) & $1.56$ & $(2,3)$ & NO & 7418\\
$(178,47)$ & 12 & $(125,33)$ & 11 & 1 & YES & YES & NO(2) & $1.80$ & $(2,3)$ & NO & 7419\\
$(178,69)$ & 11 & $(129,50)$ & 10 & 1 & YES & YES & NO(2) & $1.56$ & $(2,3)$ & NO & 7420\\
$(178,55)$ & 13 & $(178,55)$ & 13 & 178 & YES & YES & YES & $1.75$ & $(2,3)$ & NO & 7421\\
$(178,69)$ & 11 & $(178,69)$ & 11 & 178 & YES & YES & NO(2) & $1.56$ & $(2,3)$ & NO & 7422\\
$(179,68)$ & 11 & $(2,1)$ & 1 & 1 & YES & YES & NO(2) & $1.70$ & $(2,3)$ & -- & 7423\\
$(179,68)$ & 11 & $(2,1)$ & 1 & 1 & YES & YES & NO(2) & $1.50$ & $(6,1)$ & NO & 7424\\
$(179,75)$ & 11 & $(2,1)$ & 1 & 1 & YES & YES & NO(2) & $1.80$ & $(2,3)$ & NO & 7425\\
$(179,75)$ & 11 & $(2,1)$ & 1 & 1 & YES & YES & NO(2) & $1.56$ & $(4,2)$ & -- & 7426\\
$(179,68)$ & 11 & $(3,1)$ & 2 & 1 & YES & YES & NO(2) & $1.70$ & $(2,3)$ & NO & 7427\\
$(179,74)$ & 11 & $(3,1)$ & 2 & 1 & YES & YES & NO(2) & $1.50$ & $(6,1)$ & -- & 7428\\
$(179,74)$ & 11 & $(3,1)$ & 2 & 1 & YES & YES & NO(2) & $1.38$ & $(10,-1)$ & NO & 7429\\
$(179,74)$ & 11 & $(3,1)$ & 2 & 1 & YES & YES & NO(2) & $1.38$ & $(10,-1)$ & NO & 7430\\
$(179,75)$ & 11 & $(3,1)$ & 2 & 1 & YES & YES & YES & $1.29$ & $(4,2)$ & -- & 7431\\
$(179,68)$ & 11 & $(4,1)$ & 3 & 1 & YES & YES & NO(2) & $1.44$ & $(4,2)$ & -- & 7432\\
$(179,74)$ & 11 & $(4,1)$ & 3 & 1 & YES & YES & NO(2) & $1.70$ & $(2,3)$ & -- & 7433\\
$(179,74)$ & 11 & $(4,1)$ & 3 & 1 & YES & YES & NO(2) & $1.56$ & $(4,2)$ & NO & 7434\\
$(179,74)$ & 11 & $(4,1)$ & 3 & 1 & YES & YES & YES & $1.57$ & $(4,2)$ & NO & 7435\\
$(179,75)$ & 11 & $(4,1)$ & 3 & 1 & YES & YES & YES & $1.29$ & $(4,2)$ & -- & 7436\\
$(179,50)$ & 11 & $(5,2)$ & 3 & 1 & YES & YES & NO(2) & $1.80$ & $(2,3)$ & NO & 7437\\
$(179,50)$ & 11 & $(5,2)$ & 3 & 1 & YES & YES & YES & $1.29$ & $(2,3)$ & -- & 7438\\
$(179,68)$ & 11 & $(5,1)$ & 4 & 1 & YES & YES & YES & $1.62$ & $(2,3)$ & NO & 7439\\
$(179,68)$ & 11 & $(5,2)$ & 3 & 1 & YES & YES & YES & $1.62$ & $(2,3)$ & NO & 7440\\
$(179,74)$ & 11 & $(5,2)$ & 3 & 1 & YES & YES & YES & $1.75$ & $(2,3)$ & -- & 7441\\
$(179,75)$ & 11 & $(5,2)$ & 3 & 1 & YES & YES & YES & $1.43$ & $(4,2)$ & -- & 7442\\
$(179,38)$ & 13 & $(6,1)$ & 5 & 1 & YES & YES & NO(2) & $1.44$ & $(8,0)$ & NO & 7443\\
$(179,69)$ & 13 & $(6,1)$ & 5 & 1 & YES & YES & NO(2) & $1.70$ & $(6,1)$ & NO & 7444\\
$(179,33)$ & 13 & $(7,2)$ & 4 & 1 & YES & YES & NO(2) & $1.44$ & $(4,2)$ & NO & 7445\\
$(179,50)$ & 11 & $(7,2)$ & 4 & 1 & YES & YES & NO(2) & $1.60$ & $(2,3)$ & NO & 7446\\
$(179,50)$ & 11 & $(7,3)$ & 4 & 1 & YES & YES & YES & $1.75$ & $(2,3)$ & -- & 7447\\
$(179,74)$ & 11 & $(7,2)$ & 4 & 1 & YES & YES & YES & $1.57$ & $(2,3)$ & -- & 7448\\
$(179,74)$ & 11 & $(7,2)$ & 4 & 1 & YES & YES & YES & $1.78$ & $(2,3)$ & NO & 7449\\
$(179,75)$ & 11 & $(7,2)$ & 4 & 1 & YES & YES & YES & $1.50$ & $(2,3)$ & -- & 7450\\
$(179,75)$ & 11 & $(7,3)$ & 4 & 1 & YES & YES & NO(2) & $1.56$ & $(4,2)$ & NO & 7451\\
$(179,75)$ & 11 & $(7,3)$ & 4 & 1 & YES & YES & YES & $1.62$ & $(4,2)$ & -- & 7452\\
$(179,74)$ & 11 & $(9,2)$ & 5 & 1 & YES & YES & YES & $1.62$ & $(2,3)$ & -- & 7453\\
$(179,74)$ & 11 & $(9,2)$ & 5 & 1 & YES & YES & YES & $1.62$ & $(2,3)$ & NO & 7454\\
$(179,75)$ & 11 & $(9,4)$ & 5 & 1 & YES & YES & YES & $1.62$ & $(4,2)$ & NO & 7455\\
$(179,75)$ & 11 & $(12,5)$ & 5 & 1 & YES & YES & NO(2) & $1.44$ & $(8,0)$ & 5566 & 7456\\
$(179,50)$ & 11 & $(13,4)$ & 6 & 1 & YES & YES & YES & $1.75$ & $(2,3)$ & NO & 7457\\
$(179,54)$ & 13 & $(13,4)$ & 6 & 1 & YES & YES & NO(2) & $1.56$ & $(8,0)$ & NO & 7458\\
$(179,50)$ & 11 & $(15,4)$ & 6 & 1 & YES & YES & YES & $1.50$ & $(4,2)$ & 10128 & 7459\\
$(179,73)$ & 12 & $(17,7)$ & 6 & 1 & YES & YES & NO(2) & $1.50$ & $(10,-1)$ & NO & 7460\\
$(179,74)$ & 11 & $(17,7)$ & 6 & 1 & YES & YES & NO(2) & $1.80$ & $(2,3)$ & 5531 & 7461\\
$(179,75)$ & 11 & $(17,7)$ & 6 & 1 & YES & YES & YES & $1.50$ & $(4,2)$ & NO & 7462\\
$(179,68)$ & 11 & $(18,7)$ & 6 & 1 & YES & YES & YES & $1.80$ & $(2,3)$ & NO & 7463\\
$(179,75)$ & 11 & $(19,8)$ & 6 & 1 & YES & YES & NO(2) & $1.78$ & $(2,3)$ & NO & 7464\\
$(179,68)$ & 11 & $(21,8)$ & 6 & 1 & YES & YES & NO(2) & $1.70$ & $(2,3)$ & NO & 7465\\
$(179,33)$ & 13 & $(22,5)$ & 7 & 1 & YES & YES & YES & $1.67$ & $(4,2)$ & NO & 7466\\
$(179,54)$ & 13 & $(23,7)$ & 7 & 1 & YES & YES & NO(2) & $1.67$ & $(8,0)$ & NO & 7467\\
$(179,50)$ & 11 & $(29,8)$ & 7 & 1 & YES & YES & YES & $1.67$ & $(2,3)$ & NO & 7468\\
$(179,68)$ & 11 & $(34,13)$ & 7 & 1 & YES & YES & YES & $1.70$ & $(2,3)$ & NO & 7469\\
$(179,41)$ & 12 & $(40,9)$ & 9 & 1 & YES & YES & YES & $1.78$ & $(2,3)$ & NO & 7470\\
$(179,74)$ & 11 & $(41,17)$ & 8 & 1 & YES & YES & YES & $1.43$ & $(4,2)$ & NO & 7471\\
$(179,50)$ & 11 & $(43,12)$ & 8 & 1 & YES & YES & YES & $1.62$ & $(2,3)$ & 6799 & 7472\\
$(179,75)$ & 11 & $(43,18)$ & 8 & 1 & YES & YES & YES & $1.43$ & $(4,2)$ & NO & 7473\\
$(179,74)$ & 11 & $(46,19)$ & 8 & 1 & YES & YES & YES & $1.62$ & $(2,3)$ & NO & 7474\\
$(179,68)$ & 11 & $(50,19)$ & 8 & 1 & YES & YES & NO(2) & $1.50$ & $(6,1)$ & NO & 7475\\
$(179,50)$ & 11 & $(57,16)$ & 9 & 1 & YES & YES & YES & $1.75$ & $(2,3)$ & NO & 7476\\
$(179,69)$ & 13 & $(57,22)$ & 9 & 1 & YES & YES & NO(2) & $1.70$ & $(6,1)$ & 6138 & 7477\\
$(179,68)$ & 11 & $(79,30)$ & 9 & 1 & YES & YES & YES & $1.80$ & $(2,3)$ & 8214 & 7478\\
$(179,74)$ & 11 & $(104,43)$ & 10 & 1 & YES & YES & YES & $1.62$ & $(2,3)$ & NO & 7479\\
$(179,75)$ & 11 & $(105,44)$ & 10 & 1 & YES & YES & YES & $1.29$ & $(4,2)$ & NO & 7480\\
$(179,75)$ & 11 & $(117,49)$ & 10 & 1 & YES & YES & YES & $1.62$ & $(4,2)$ & NO & 7481\\
$(179,74)$ & 11 & $(121,50)$ & 10 & 1 & YES & YES & YES & $1.57$ & $(4,2)$ & NO & 7482\\
$(179,50)$ & 11 & $(161,45)$ & 11 & 1 & YES & YES & YES & $1.67$ & $(4,2)$ & NO & 7483\\
$(179,74)$ & 11 & $(179,74)$ & 11 & 179 & YES & YES & YES & $1.62$ & $(2,3)$ & NO & 7484\\
$(179,75)$ & 11 & $(179,75)$ & 11 & 179 & YES & YES & YES & $1.29$ & $(4,2)$ & NO & 7485\\
$(180,41)$ & 12 & $(8,3)$ & 4 & 4 & YES & YES & YES & $1.78$ & $(2,3)$ & -- & 7486\\
$(180,41)$ & 12 & $(8,3)$ & 4 & 4 & YES & YES & YES & $1.62$ & $(4,2)$ & NO & 7487\\
$(180,41)$ & 12 & $(11,3)$ & 5 & 1 & YES & YES & YES & $1.57$ & $(2,3)$ & NO & 7488\\
$(181,70)$ & 11 & $(2,1)$ & 1 & 1 & YES & YES & NO(2) & $1.38$ & $(6,1)$ & NO & 7489\\
$(181,75)$ & 11 & $(2,1)$ & 1 & 1 & YES & YES & NO(2) & $1.56$ & $(2,3)$ & -- & 7490\\
$(181,75)$ & 11 & $(2,1)$ & 1 & 1 & YES & YES & NO(2) & $1.64$ & $(4,2)$ & NO & 7491\\
$(181,76)$ & 11 & $(2,1)$ & 1 & 1 & YES & YES & NO(2) & $1.62$ & $(10,-1)$ & NO & 7492\\
$(181,51)$ & 12 & $(3,1)$ & 2 & 1 & YES & YES & NO(2) & $1.38$ & $(10,-1)$ & -- & 7493\\
$(181,51)$ & 12 & $(3,1)$ & 2 & 1 & YES & YES & NO(2) & $1.50$ & $(10,-1)$ & NO & 7494\\
$(181,53)$ & 12 & $(3,1)$ & 2 & 1 & YES & YES & NO(2) & $1.60$ & $(2,3)$ & -- & 7495\\
$(181,65)$ & 12 & $(3,1)$ & 2 & 1 & YES & YES & YES & $1.57$ & $(2,3)$ & -- & 7496\\
$(181,65)$ & 12 & $(3,1)$ & 2 & 1 & YES & YES & NO(2) & $1.50$ & $(8,0)$ & NO & 7497\\
$(181,70)$ & 11 & $(3,1)$ & 2 & 1 & YES & YES & NO(2) & $1.38$ & $(4,2)$ & NO & 7498\\
$(181,70)$ & 11 & $(3,1)$ & 2 & 1 & YES & YES & NO(2) & $1.38$ & $(4,2)$ & -- & 7499\\
$(181,75)$ & 11 & $(3,1)$ & 2 & 1 & YES & YES & NO(2) & $1.56$ & $(2,3)$ & -- & 7500\\
$(181,75)$ & 11 & $(3,1)$ & 2 & 1 & YES & YES & NO(2) & $1.14$ & $(6,1)$ & NO & 7501\\
$(181,75)$ & 11 & $(3,1)$ & 2 & 1 & YES & YES & YES & $1.43$ & $(4,2)$ & NO & 7502\\
$(181,76)$ & 11 & $(3,1)$ & 2 & 1 & YES & YES & YES & $1.43$ & $(6,1)$ & -- & 7503\\
$(181,80)$ & 13 & $(3,1)$ & 2 & 1 & YES & YES & NO(2) & $1.78$ & $(8,0)$ & NO & 7504\\
$(181,70)$ & 11 & $(4,1)$ & 3 & 1 & YES & YES & YES & $1.71$ & $(2,3)$ & -- & 7505\\
$(181,70)$ & 11 & $(4,1)$ & 3 & 1 & YES & YES & NO(2) & $1.38$ & $(4,2)$ & NO & 7506\\
$(181,75)$ & 11 & $(4,1)$ & 3 & 1 & YES & YES & NO(2) & $1.50$ & $(6,1)$ & -- & 7507\\
$(181,76)$ & 11 & $(4,1)$ & 3 & 1 & YES & YES & YES & $1.62$ & $(4,2)$ & NO & 7508\\
$(181,76)$ & 11 & $(4,1)$ & 3 & 1 & YES & YES & YES & $1.62$ & $(4,2)$ & -- & 7509\\
$(181,41)$ & 12 & $(5,2)$ & 3 & 1 & YES & YES & NO(2) & $1.25$ & $(4,2)$ & -- & 7510\\
$(181,50)$ & 11 & $(5,2)$ & 3 & 1 & YES & YES & YES & $1.50$ & $(2,3)$ & NO & 7511\\
$(181,50)$ & 11 & $(5,2)$ & 3 & 1 & YES & YES & YES & $1.50$ & $(2,3)$ & -- & 7512\\
$(181,55)$ & 12 & $(5,2)$ & 3 & 1 & YES & YES & NO(2) & $1.38$ & $(4,2)$ & -- & 7513\\
$(181,70)$ & 11 & $(5,2)$ & 3 & 1 & YES & YES & NO(2) & $1.38$ & $(6,1)$ & NO & 7514\\
$(181,70)$ & 11 & $(5,2)$ & 3 & 1 & YES & YES & YES & $1.43$ & $(4,2)$ & -- & 7515\\
$(181,75)$ & 11 & $(5,1)$ & 4 & 1 & YES & YES & NO(2) & $1.38$ & $(4,2)$ & -- & 7516\\
$(181,75)$ & 11 & $(5,2)$ & 3 & 1 & YES & YES & NO(2) & $1.67$ & $(2,3)$ & -- & 7517\\
$(181,75)$ & 11 & $(5,2)$ & 3 & 1 & YES & YES & NO(2) & $1.60$ & $(6,1)$ & 5746 & 7518\\
$(181,76)$ & 11 & $(5,2)$ & 3 & 1 & YES & YES & NO(2) & $1.44$ & $(8,0)$ & NO & 7519\\
$(181,76)$ & 11 & $(5,2)$ & 3 & 1 & YES & YES & YES & $1.57$ & $(4,2)$ & -- & 7520\\
$(181,80)$ & 13 & $(5,2)$ & 3 & 1 & YES & YES & NO(2) & $1.90$ & $(6,1)$ & NO & 7521\\
$(181,80)$ & 13 & $(5,2)$ & 3 & 1 & YES & YES & NO(2) & $1.90$ & $(6,1)$ & -- & 7522\\
$(181,41)$ & 12 & $(7,3)$ & 4 & 1 & YES & YES & YES & $1.67$ & $(2,3)$ & NO & 7523\\
$(181,55)$ & 12 & $(7,2)$ & 4 & 1 & YES & YES & YES & $1.75$ & $(2,3)$ & -- & 7524\\
$(181,70)$ & 11 & $(7,2)$ & 4 & 1 & YES & YES & YES & $1.75$ & $(2,3)$ & -- & 7525\\
$(181,75)$ & 11 & $(7,2)$ & 4 & 1 & YES & YES & YES & $1.62$ & $(2,3)$ & -- & 7526\\
$(181,75)$ & 11 & $(7,2)$ & 4 & 1 & YES & YES & YES & $1.62$ & $(4,2)$ & NO & 7527\\
$(181,75)$ & 11 & $(7,3)$ & 4 & 1 & YES & YES & YES & $1.43$ & $(2,3)$ & NO & 7528\\
$(181,80)$ & 13 & $(7,3)$ & 4 & 1 & YES & YES & NO(2) & $1.78$ & $(8,0)$ & 4403 & 7529\\
$(181,41)$ & 12 & $(8,3)$ & 4 & 1 & YES & YES & YES & $1.78$ & $(2,3)$ & NO & 7530\\
$(181,55)$ & 12 & $(8,3)$ & 4 & 1 & YES & YES & YES & $1.75$ & $(4,2)$ & NO & 7531\\
$(181,55)$ & 12 & $(8,3)$ & 4 & 1 & YES & YES & YES & $1.75$ & $(4,2)$ & -- & 7532\\
$(181,70)$ & 11 & $(8,3)$ & 4 & 1 & YES & YES & NO(2) & $1.38$ & $(4,2)$ & NO & 7533\\
$(181,75)$ & 11 & $(8,3)$ & 4 & 1 & YES & YES & YES & $1.71$ & $(4,2)$ & NO & 7534\\
$(181,70)$ & 11 & $(9,2)$ & 5 & 1 & YES & YES & YES & $1.60$ & $(2,3)$ & NO & 7535\\
$(181,75)$ & 11 & $(9,2)$ & 5 & 1 & YES & YES & YES & $1.75$ & $(2,3)$ & NO & 7536\\
$(181,41)$ & 12 & $(10,3)$ & 5 & 1 & YES & YES & YES & $1.67$ & $(2,3)$ & NO & 7537\\
$(181,41)$ & 12 & $(11,4)$ & 5 & 1 & YES & YES & YES & $1.50$ & $(4,2)$ & -- & 7538\\
$(181,49)$ & 12 & $(11,2)$ & 6 & 1 & YES & YES & NO(2) & $1.38$ & $(4,2)$ & NO & 7539\\
$(181,76)$ & 11 & $(12,5)$ & 5 & 1 & YES & YES & NO(2) & $1.44$ & $(8,0)$ & NO & 7540\\
$(181,70)$ & 11 & $(13,3)$ & 6 & 1 & YES & YES & YES & $1.50$ & $(4,2)$ & -- & 7541\\
$(181,75)$ & 11 & $(17,7)$ & 6 & 1 & YES & YES & NO(2) & $1.44$ & $(4,2)$ & NO & 7542\\
$(181,70)$ & 11 & $(18,7)$ & 6 & 1 & YES & YES & NO(2) & $1.38$ & $(4,2)$ & NO & 7543\\
$(181,70)$ & 11 & $(21,8)$ & 6 & 1 & YES & YES & YES & $1.75$ & $(4,2)$ & NO & 7544\\
$(181,70)$ & 11 & $(23,9)$ & 7 & 1 & YES & YES & YES & $1.78$ & $(2,3)$ & NO & 7545\\
$(181,50)$ & 11 & $(25,7)$ & 7 & 1 & YES & YES & YES & $1.50$ & $(2,3)$ & NO & 7546\\
$(181,70)$ & 11 & $(44,17)$ & 8 & 1 & YES & YES & NO(2) & $1.38$ & $(4,2)$ & NO & 7547\\
$(181,70)$ & 11 & $(49,19)$ & 8 & 1 & YES & YES & YES & $1.43$ & $(4,2)$ & NO & 7548\\
$(181,65)$ & 12 & $(53,19)$ & 9 & 1 & YES & YES & NO(2) & $1.56$ & $(8,0)$ & NO & 7549\\
$(181,70)$ & 11 & $(57,22)$ & 9 & 1 & YES & YES & YES & $1.88$ & $(2,3)$ & NO & 7550\\
$(181,65)$ & 12 & $(64,23)$ & 9 & 1 & YES & YES & YES & $1.57$ & $(2,3)$ & NO & 7551\\
$(181,75)$ & 11 & $(70,29)$ & 9 & 1 & YES & YES & NO(2) & $1.44$ & $(2,3)$ & NO & 7552\\
$(181,76)$ & 11 & $(81,34)$ & 9 & 1 & YES & YES & YES & $1.62$ & $(4,2)$ & 8310 & 7553\\
$(181,55)$ & 12 & $(89,27)$ & 10 & 1 & YES & YES & YES & $1.62$ & $(2,3)$ & NO & 7554\\
$(181,75)$ & 11 & $(99,41)$ & 10 & 1 & YES & YES & YES & $1.75$ & $(2,3)$ & NO & 7555\\
$(181,70)$ & 11 & $(106,41)$ & 10 & 1 & YES & YES & YES & $1.44$ & $(2,3)$ & NO & 7556\\
$(181,75)$ & 11 & $(111,46)$ & 10 & 1 & YES & YES & YES & $1.57$ & $(4,2)$ & NO & 7557\\
$(181,41)$ & 12 & $(115,26)$ & 11 & 1 & YES & YES & YES & $1.62$ & $(4,2)$ & NO & 7558\\
$(181,70)$ & 11 & $(119,46)$ & 10 & 1 & YES & YES & YES & $1.43$ & $(4,2)$ & NO & 7559\\
$(181,76)$ & 11 & $(131,55)$ & 10 & 1 & YES & YES & YES & $1.38$ & $(2,3)$ & NO & 7560\\
$(181,70)$ & 11 & $(137,53)$ & 11 & 1 & YES & YES & YES & $1.60$ & $(2,3)$ & NO & 7561\\
$(181,75)$ & 11 & $(152,63)$ & 11 & 1 & YES & YES & YES & $1.62$ & $(4,2)$ & NO & 7562\\
$(181,53)$ & 12 & $(157,46)$ & 11 & 1 & YES & YES & YES & $1.67$ & $(2,3)$ & 10615 & 7563\\
$(181,70)$ & 11 & $(181,70)$ & 11 & 181 & YES & YES & YES & $1.62$ & $(2,3)$ & NO & 7564\\
$(181,75)$ & 11 & $(181,75)$ & 11 & 181 & YES & YES & NO(2) & $1.70$ & $(6,1)$ & NO & 7565\\
$(182,71)$ & 12 & $(5,1)$ & 4 & 1 & YES & YES & NO(2) & $1.67$ & $(8,0)$ & -- & 7566\\
$(182,53)$ & 12 & $(86,25)$ & 10 & 2 & YES & YES & YES & $1.62$ & $(2,3)$ & NO & 7567\\
$(182,43)$ & 13 & $(97,23)$ & 11 & 1 & YES & YES & YES & $1.67$ & $(4,2)$ & NO & 7568\\
$(182,71)$ & 12 & $(141,55)$ & 11 & 1 & YES & YES & NO(2) & $1.56$ & $(8,0)$ & NO & 7569\\
$(183,49)$ & 12 & $(2,1)$ & 1 & 1 & YES & YES & NO(2) & $1.50$ & $(6,1)$ & -- & 7570\\
$(183,67)$ & 11 & $(2,1)$ & 1 & 1 & YES & YES & NO(2) & $1.50$ & $(2,3)$ & -- & 7571\\
$(183,67)$ & 11 & $(3,1)$ & 2 & 3 & YES & YES & YES & $1.29$ & $(4,2)$ & NO & 7572\\
$(183,67)$ & 11 & $(3,1)$ & 2 & 3 & YES & YES & YES & $1.29$ & $(4,2)$ & -- & 7573\\
$(183,71)$ & 11 & $(3,1)$ & 2 & 3 & YES & YES & NO(2) & $1.56$ & $(2,3)$ & -- & 7574\\
$(183,71)$ & 11 & $(3,1)$ & 2 & 3 & YES & YES & NO(2) & $1.67$ & $(2,3)$ & NO & 7575\\
$(183,82)$ & 14 & $(3,1)$ & 2 & 3 & YES & YES & YES & $1.71$ & $(4,2)$ & NO & 7576\\
$(183,82)$ & 14 & $(3,1)$ & 2 & 3 & YES & YES & YES & $1.71$ & $(4,2)$ & -- & 7577\\
$(183,71)$ & 11 & $(4,1)$ & 3 & 1 & YES & YES & YES & $1.29$ & $(4,2)$ & -- & 7578\\
$(183,82)$ & 14 & $(4,1)$ & 3 & 1 & YES & YES & NO(2) & $1.67$ & $(8,0)$ & NO & 7579\\
$(183,82)$ & 14 & $(4,1)$ & 3 & 1 & YES & YES & NO(2) & $1.67$ & $(8,0)$ & -- & 7580\\
$(183,56)$ & 12 & $(5,2)$ & 3 & 1 & YES & YES & NO(2) & $1.60$ & $(6,1)$ & -- & 7581\\
$(183,67)$ & 11 & $(5,2)$ & 3 & 1 & YES & YES & YES & $1.75$ & $(4,2)$ & -- & 7582\\
$(183,71)$ & 11 & $(5,1)$ & 4 & 1 & YES & YES & NO(2) & $1.38$ & $(4,2)$ & NO & 7583\\
$(183,71)$ & 11 & $(5,1)$ & 4 & 1 & YES & YES & NO(2) & $1.50$ & $(4,2)$ & -- & 7584\\
$(183,71)$ & 11 & $(5,2)$ & 3 & 1 & YES & YES & YES & $1.75$ & $(4,2)$ & -- & 7585\\
$(183,67)$ & 11 & $(7,2)$ & 4 & 1 & YES & YES & YES & $1.57$ & $(4,2)$ & -- & 7586\\
$(183,71)$ & 11 & $(7,2)$ & 4 & 1 & YES & YES & YES & $1.57$ & $(2,3)$ & NO & 7587\\
$(183,71)$ & 11 & $(7,2)$ & 4 & 1 & YES & YES & YES & $1.43$ & $(2,3)$ & -- & 7588\\
$(183,71)$ & 11 & $(7,3)$ & 4 & 1 & YES & YES & YES & $1.62$ & $(4,2)$ & -- & 7589\\
$(183,71)$ & 11 & $(9,2)$ & 5 & 3 & YES & YES & YES & $1.62$ & $(4,2)$ & NO & 7590\\
$(183,56)$ & 12 & $(10,3)$ & 5 & 1 & YES & YES & NO(2) & $1.50$ & $(10,-1)$ & NO & 7591\\
$(183,67)$ & 11 & $(11,4)$ & 5 & 1 & YES & YES & NO(2) & $1.60$ & $(6,1)$ & 6572 & 7592\\
$(183,71)$ & 11 & $(11,4)$ & 5 & 1 & YES & YES & YES & $1.78$ & $(2,3)$ & NO & 7593\\
$(183,56)$ & 12 & $(13,4)$ & 6 & 1 & YES & YES & NO(2) & $1.60$ & $(10,-1)$ & NO & 7594\\
$(183,71)$ & 11 & $(21,8)$ & 6 & 3 & YES & YES & YES & $1.43$ & $(4,2)$ & NO & 7595\\
$(183,71)$ & 11 & $(23,9)$ & 7 & 1 & YES & YES & YES & $1.57$ & $(2,3)$ & NO & 7596\\
$(183,67)$ & 11 & $(27,10)$ & 7 & 3 & YES & YES & YES & $1.57$ & $(4,2)$ & NO & 7597\\
$(183,71)$ & 11 & $(31,12)$ & 7 & 1 & YES & YES & YES & $1.43$ & $(2,3)$ & 8253 & 7598\\
$(183,71)$ & 11 & $(44,17)$ & 8 & 1 & YES & YES & YES & $1.75$ & $(4,2)$ & 10284 & 7599\\
$(183,67)$ & 11 & $(49,18)$ & 8 & 1 & YES & YES & YES & $1.62$ & $(4,2)$ & NO & 7600\\
$(183,71)$ & 11 & $(67,26)$ & 9 & 1 & YES & YES & NO(2) & $1.50$ & $(4,2)$ & NO & 7601\\
$(183,67)$ & 11 & $(71,26)$ & 9 & 1 & YES & YES & NO(2) & $1.70$ & $(2,3)$ & NO & 7602\\
$(183,71)$ & 11 & $(80,31)$ & 9 & 1 & YES & YES & YES & $1.75$ & $(4,2)$ & NO & 7603\\
$(183,71)$ & 11 & $(85,33)$ & 10 & 1 & YES & YES & YES & $1.67$ & $(2,3)$ & NO & 7604\\
$(183,67)$ & 11 & $(101,37)$ & 10 & 1 & YES & YES & YES & $1.90$ & $(2,3)$ & NO & 7605\\
$(183,71)$ & 11 & $(116,45)$ & 10 & 1 & YES & YES & YES & $1.29$ & $(4,2)$ & NO & 7606\\
$(183,71)$ & 11 & $(165,64)$ & 11 & 3 & YES & YES & YES & $1.29$ & $(4,2)$ & NO & 7607\\
$(183,67)$ & 11 & $(183,67)$ & 11 & 183 & YES & YES & NO(2) & $1.56$ & $(4,2)$ & NO & 7608\\
$(184,57)$ & 12 & $(2,1)$ & 1 & 2 & YES & YES & NO(2) & $1.70$ & $(2,3)$ & -- & 7609\\
$(184,57)$ & 12 & $(2,1)$ & 1 & 2 & YES & YES & NO(2) & $1.78$ & $(4,2)$ & NO & 7610\\
$(184,57)$ & 12 & $(3,1)$ & 2 & 1 & YES & YES & NO(2) & $1.60$ & $(2,3)$ & -- & 7611\\
$(184,57)$ & 12 & $(3,1)$ & 2 & 1 & YES & YES & NO(2) & $1.50$ & $(6,1)$ & 5804 & 7612\\
$(184,57)$ & 12 & $(7,2)$ & 4 & 1 & YES & YES & YES & $1.57$ & $(2,3)$ & NO & 7613\\
$(184,57)$ & 12 & $(29,9)$ & 8 & 1 & YES & YES & NO(2) & $1.38$ & $(6,1)$ & NO & 7614\\
$(184,43)$ & 12 & $(167,39)$ & 13 & 1 & YES & YES & NO(2) & $1.56$ & $(4,2)$ & 10382 & 7615\\
$(185,68)$ & 11 & $(2,1)$ & 1 & 1 & YES & YES & NO(2) & $1.50$ & $(6,1)$ & -- & 7616\\
$(185,68)$ & 11 & $(3,1)$ & 2 & 1 & YES & YES & NO(2) & $1.44$ & $(8,0)$ & -- & 7617\\
$(185,68)$ & 11 & $(3,1)$ & 2 & 1 & YES & YES & YES & $1.57$ & $(4,2)$ & NO & 7618\\
$(185,68)$ & 11 & $(4,1)$ & 3 & 1 & YES & YES & NO(2) & $1.67$ & $(2,3)$ & -- & 7619\\
$(185,56)$ & 12 & $(7,3)$ & 4 & 1 & YES & YES & YES & $1.88$ & $(4,2)$ & NO & 7620\\
$(185,56)$ & 12 & $(7,3)$ & 4 & 1 & YES & YES & YES & $1.88$ & $(4,2)$ & -- & 7621\\
$(185,68)$ & 11 & $(7,3)$ & 4 & 1 & YES & YES & YES & $1.62$ & $(4,2)$ & -- & 7622\\
$(185,68)$ & 11 & $(11,4)$ & 5 & 1 & YES & YES & NO(2) & $1.50$ & $(6,1)$ & NO & 7623\\
$(185,58)$ & 13 & $(13,4)$ & 6 & 1 & YES & YES & NO(2) & $1.62$ & $(6,1)$ & NO & 7624\\
$(185,68)$ & 11 & $(19,7)$ & 6 & 1 & YES & YES & NO(2) & $1.38$ & $(10,-1)$ & NO & 7625\\
$(185,68)$ & 11 & $(30,11)$ & 7 & 5 & YES & YES & NO(2) & $1.56$ & $(8,0)$ & 8166 & 7626\\
$(185,68)$ & 11 & $(49,18)$ & 8 & 1 & YES & YES & NO(2) & $1.70$ & $(6,1)$ & 7079 & 7627\\
$(185,76)$ & 12 & $(73,30)$ & 10 & 1 & YES & YES & YES & $1.43$ & $(2,3)$ & 8065 & 7628\\
$(185,68)$ & 11 & $(185,68)$ & 11 & 185 & YES & YES & NO(2) & $1.67$ & $(2,3)$ & NO & 7629\\
$(186,71)$ & 11 & $(2,1)$ & 1 & 2 & YES & YES & NO(2) & $1.70$ & $(2,3)$ & -- & 7630\\
$(186,55)$ & 11 & $(3,1)$ & 2 & 3 & NO & YES & NO(2) & $1.50$ & $(10,-1)$ & -- & 7631\\
$(186,55)$ & 11 & $(3,1)$ & 2 & 3 & YES & YES & YES & $1.50$ & $(2,3)$ & NO & 7632\\
$(186,71)$ & 11 & $(3,1)$ & 2 & 3 & YES & YES & YES & $1.56$ & $(2,3)$ & -- & 7633\\
$(186,71)$ & 11 & $(4,1)$ & 3 & 2 & YES & YES & YES & $1.29$ & $(4,2)$ & -- & 7634\\
$(186,71)$ & 11 & $(4,1)$ & 3 & 2 & YES & YES & YES & $1.43$ & $(4,2)$ & NO & 7635\\
$(186,71)$ & 11 & $(4,1)$ & 3 & 2 & YES & YES & YES & $1.62$ & $(2,3)$ & NO & 7636\\
$(186,55)$ & 11 & $(5,2)$ & 3 & 1 & YES & YES & YES & $1.67$ & $(2,3)$ & -- & 7637\\
$(186,55)$ & 11 & $(5,2)$ & 3 & 1 & YES & YES & YES & $1.50$ & $(2,3)$ & NO & 7638\\
$(186,71)$ & 11 & $(5,2)$ & 3 & 1 & YES & YES & YES & $1.88$ & $(2,3)$ & -- & 7639\\
$(186,55)$ & 11 & $(7,2)$ & 4 & 1 & YES & YES & YES & $1.50$ & $(2,3)$ & -- & 7640\\
$(186,71)$ & 11 & $(7,2)$ & 4 & 1 & YES & YES & YES & $1.89$ & $(2,3)$ & NO & 7641\\
$(186,71)$ & 11 & $(7,2)$ & 4 & 1 & YES & YES & YES & $1.75$ & $(4,2)$ & -- & 7642\\
$(186,71)$ & 11 & $(7,3)$ & 4 & 1 & YES & YES & YES & $1.57$ & $(4,2)$ & NO & 7643\\
$(186,71)$ & 11 & $(9,2)$ & 5 & 3 & YES & YES & YES & $1.71$ & $(2,3)$ & NO & 7644\\
$(186,71)$ & 11 & $(9,2)$ & 5 & 3 & YES & YES & YES & $1.71$ & $(2,3)$ & -- & 7645\\
$(186,55)$ & 11 & $(11,3)$ & 5 & 1 & YES & YES & YES & $1.89$ & $(2,3)$ & NO & 7646\\
$(186,71)$ & 11 & $(11,4)$ & 5 & 1 & YES & YES & YES & $1.70$ & $(2,3)$ & NO & 7647\\
$(186,55)$ & 11 & $(13,4)$ & 6 & 1 & YES & YES & YES & $1.50$ & $(2,3)$ & NO & 7648\\
$(186,71)$ & 11 & $(23,9)$ & 7 & 1 & YES & YES & YES & $1.75$ & $(4,2)$ & NO & 7649\\
$(186,55)$ & 11 & $(24,7)$ & 7 & 6 & YES & YES & YES & $1.78$ & $(2,3)$ & NO & 7650\\
$(186,71)$ & 11 & $(29,11)$ & 7 & 1 & YES & YES & YES & $1.70$ & $(2,3)$ & NO & 7651\\
$(186,55)$ & 11 & $(37,11)$ & 8 & 1 & YES & YES & NO(2) & $1.78$ & $(4,2)$ & NO & 7652\\
$(186,71)$ & 11 & $(47,18)$ & 8 & 1 & YES & YES & YES & $1.90$ & $(2,3)$ & NO & 7653\\
$(186,55)$ & 11 & $(61,18)$ & 9 & 1 & YES & YES & YES & $1.89$ & $(2,3)$ & NO & 7654\\
$(186,71)$ & 11 & $(76,29)$ & 9 & 2 & YES & YES & YES & $1.62$ & $(2,3)$ & 8173 & 7655\\
$(186,71)$ & 11 & $(97,37)$ & 10 & 1 & YES & YES & YES & $1.89$ & $(2,3)$ & 10578 & 7656\\
$(186,55)$ & 11 & $(98,29)$ & 10 & 2 & YES & YES & YES & $1.62$ & $(2,3)$ & NO & 7657\\
$(186,71)$ & 11 & $(131,50)$ & 10 & 1 & YES & YES & YES & $1.29$ & $(4,2)$ & NO & 7658\\
$(187,71)$ & 11 & $(3,1)$ & 2 & 1 & YES & YES & NO(2) & $1.67$ & $(2,3)$ & -- & 7659\\
$(187,79)$ & 11 & $(3,1)$ & 2 & 1 & YES & YES & YES & $1.43$ & $(2,3)$ & NO & 7660\\
$(187,79)$ & 11 & $(3,1)$ & 2 & 1 & YES & YES & YES & $1.43$ & $(2,3)$ & -- & 7661\\
$(187,82)$ & 12 & $(3,1)$ & 2 & 1 & YES & YES & NO(2) & $1.56$ & $(8,0)$ & -- & 7662\\
$(187,71)$ & 11 & $(4,1)$ & 3 & 1 & YES & YES & NO(2) & $1.44$ & $(4,2)$ & -- & 7663\\
$(187,71)$ & 11 & $(4,1)$ & 3 & 1 & YES & YES & NO(2) & $1.56$ & $(4,2)$ & NO & 7664\\
$(187,49)$ & 13 & $(5,1)$ & 4 & 1 & YES & YES & NO(2) & $1.56$ & $(4,2)$ & -- & 7665\\
$(187,57)$ & 12 & $(5,2)$ & 3 & 1 & YES & YES & YES & $1.89$ & $(2,3)$ & -- & 7666\\
$(187,71)$ & 11 & $(5,2)$ & 3 & 1 & YES & YES & YES & $1.62$ & $(2,3)$ & -- & 7667\\
$(187,71)$ & 11 & $(5,2)$ & 3 & 1 & YES & YES & YES & $1.75$ & $(2,3)$ & NO & 7668\\
$(187,79)$ & 11 & $(5,2)$ & 3 & 1 & YES & YES & YES & $1.43$ & $(2,3)$ & NO & 7669\\
$(187,79)$ & 11 & $(5,2)$ & 3 & 1 & YES & YES & YES & $1.62$ & $(4,2)$ & -- & 7670\\
$(187,71)$ & 11 & $(7,2)$ & 4 & 1 & YES & YES & YES & $1.78$ & $(2,3)$ & -- & 7671\\
$(187,79)$ & 11 & $(7,2)$ & 4 & 1 & YES & YES & YES & $1.78$ & $(4,2)$ & -- & 7672\\
$(187,73)$ & 12 & $(8,3)$ & 4 & 1 & YES & YES & NO(2) & $1.60$ & $(6,1)$ & NO & 7673\\
$(187,79)$ & 11 & $(8,3)$ & 4 & 1 & YES & YES & YES & $1.89$ & $(4,2)$ & -- & 7674\\
$(187,71)$ & 11 & $(9,2)$ & 5 & 1 & YES & YES & YES & $1.29$ & $(4,2)$ & -- & 7675\\
$(187,49)$ & 13 & $(11,3)$ & 5 & 11 & YES & YES & NO(2) & $1.38$ & $(6,1)$ & NO & 7676\\
$(187,57)$ & 12 & $(11,3)$ & 5 & 11 & YES & YES & YES & $1.89$ & $(2,3)$ & NO & 7677\\
$(187,69)$ & 12 & $(11,4)$ & 5 & 11 & YES & YES & NO(2) & $1.44$ & $(4,2)$ & NO & 7678\\
$(187,79)$ & 11 & $(12,5)$ & 5 & 1 & YES & YES & NO(2) & $1.44$ & $(8,0)$ & 8302 & 7679\\
$(187,71)$ & 11 & $(13,5)$ & 5 & 1 & YES & YES & NO(2) & $1.67$ & $(2,3)$ & NO & 7680\\
$(187,84)$ & 12 & $(13,6)$ & 7 & 1 & YES & YES & NO(2) & $1.60$ & $(6,1)$ & NO & 7681\\
$(187,49)$ & 13 & $(15,4)$ & 6 & 1 & YES & YES & YES & $1.43$ & $(4,2)$ & NO & 7682\\
$(187,79)$ & 11 & $(17,7)$ & 6 & 17 & YES & YES & YES & $1.78$ & $(2,3)$ & 10365 & 7683\\
$(187,71)$ & 11 & $(18,7)$ & 6 & 1 & YES & YES & YES & $1.89$ & $(2,3)$ & NO & 7684\\
$(187,79)$ & 11 & $(19,8)$ & 6 & 1 & YES & YES & YES & $1.43$ & $(2,3)$ & 7303 & 7685\\
$(187,84)$ & 12 & $(20,9)$ & 7 & 1 & YES & YES & YES & $1.43$ & $(4,2)$ & NO & 7686\\
$(187,71)$ & 11 & $(21,8)$ & 6 & 1 & YES & YES & NO(2) & $1.56$ & $(4,2)$ & 5642 & 7687\\
$(187,71)$ & 11 & $(34,13)$ & 7 & 17 & YES & YES & YES & $1.50$ & $(2,3)$ & NO & 7688\\
$(187,71)$ & 11 & $(50,19)$ & 8 & 1 & YES & YES & NO(2) & $1.56$ & $(2,3)$ & NO & 7689\\
$(187,79)$ & 11 & $(71,30)$ & 9 & 1 & YES & YES & YES & $1.57$ & $(2,3)$ & NO & 7690\\
$(187,82)$ & 12 & $(73,32)$ & 10 & 1 & YES & YES & NO(2) & $1.56$ & $(8,0)$ & 8087 & 7691\\
$(187,79)$ & 11 & $(83,35)$ & 10 & 1 & YES & YES & YES & $1.89$ & $(4,2)$ & NO & 7692\\
$(187,69)$ & 12 & $(103,38)$ & 11 & 1 & YES & YES & NO(2) & $1.44$ & $(4,2)$ & NO & 7693\\
$(187,79)$ & 11 & $(116,49)$ & 10 & 1 & YES & YES & YES & $1.29$ & $(2,3)$ & NO & 7694\\
$(187,71)$ & 11 & $(129,49)$ & 10 & 1 & YES & YES & YES & $1.75$ & $(4,2)$ & NO & 7695\\
$(187,71)$ & 11 & $(137,52)$ & 11 & 1 & YES & YES & YES & $1.43$ & $(4,2)$ & NO & 7696\\
$(187,79)$ & 11 & $(161,68)$ & 11 & 1 & YES & YES & YES & $1.43$ & $(4,2)$ & NO & 7697\\
$(187,71)$ & 11 & $(187,71)$ & 11 & 187 & YES & YES & NO(2) & $1.56$ & $(2,3)$ & NO & 7698\\
$(187,84)$ & 12 & $(187,84)$ & 12 & 187 & YES & YES & NO(2) & $1.38$ & $(6,1)$ & NO & 7699\\
$(188,57)$ & 13 & $(2,1)$ & 1 & 2 & YES & YES & NO(2) & $1.56$ & $(4,2)$ & -- & 7700\\
$(188,69)$ & 11 & $(2,1)$ & 1 & 2 & YES & YES & NO(2) & $1.50$ & $(6,1)$ & -- & 7701\\
$(188,73)$ & 12 & $(2,1)$ & 1 & 2 & YES & YES & NO(2) & $1.67$ & $(8,0)$ & -- & 7702\\
$(188,79)$ & 11 & $(2,1)$ & 1 & 2 & YES & YES & NO(2) & $1.56$ & $(2,3)$ & -- & 7703\\
$(188,79)$ & 11 & $(2,1)$ & 1 & 2 & YES & YES & YES & $1.43$ & $(2,3)$ & NO & 7704\\
$(188,79)$ & 11 & $(3,1)$ & 2 & 1 & YES & YES & NO(2) & $1.50$ & $(2,3)$ & NO & 7705\\
$(188,79)$ & 11 & $(3,1)$ & 2 & 1 & YES & YES & NO(2) & $1.64$ & $(4,2)$ & -- & 7706\\
$(188,45)$ & 13 & $(4,1)$ & 3 & 4 & YES & YES & NO(2) & $1.56$ & $(8,0)$ & -- & 7707\\
$(188,45)$ & 13 & $(4,1)$ & 3 & 4 & YES & YES & NO(2) & $1.67$ & $(8,0)$ & NO & 7708\\
$(188,57)$ & 13 & $(4,1)$ & 3 & 4 & YES & YES & NO(2) & $1.56$ & $(4,2)$ & NO & 7709\\
$(188,79)$ & 11 & $(4,1)$ & 3 & 4 & YES & YES & YES & $1.43$ & $(6,1)$ & NO & 7710\\
$(188,41)$ & 12 & $(5,2)$ & 3 & 1 & YES & YES & NO(2) & $1.38$ & $(4,2)$ & NO & 7711\\
$(188,41)$ & 12 & $(5,2)$ & 3 & 1 & YES & YES & NO(2) & $1.38$ & $(4,2)$ & NO & 7712\\
$(188,55)$ & 12 & $(5,1)$ & 4 & 1 & YES & YES & YES & $1.14$ & $(4,2)$ & NO & 7713\\
$(188,69)$ & 11 & $(5,2)$ & 3 & 1 & YES & YES & YES & $1.78$ & $(2,3)$ & -- & 7714\\
$(188,79)$ & 11 & $(5,2)$ & 3 & 1 & YES & YES & YES & $1.75$ & $(2,3)$ & -- & 7715\\
$(188,69)$ & 11 & $(7,2)$ & 4 & 1 & YES & YES & YES & $1.78$ & $(2,3)$ & -- & 7716\\
$(188,79)$ & 11 & $(7,2)$ & 4 & 1 & YES & YES & YES & $1.75$ & $(2,3)$ & NO & 7717\\
$(188,79)$ & 11 & $(7,2)$ & 4 & 1 & YES & YES & YES & $1.75$ & $(2,3)$ & -- & 7718\\
$(188,79)$ & 11 & $(7,3)$ & 4 & 1 & YES & YES & YES & $1.43$ & $(2,3)$ & NO & 7719\\
$(188,79)$ & 11 & $(8,3)$ & 4 & 4 & YES & YES & YES & $1.56$ & $(2,3)$ & NO & 7720\\
$(188,69)$ & 11 & $(9,2)$ & 5 & 1 & YES & YES & NO(2) & $1.44$ & $(4,2)$ & -- & 7721\\
$(188,69)$ & 11 & $(10,3)$ & 5 & 2 & YES & YES & YES & $1.89$ & $(4,2)$ & -- & 7722\\
$(188,79)$ & 11 & $(12,5)$ & 5 & 4 & YES & YES & NO(2) & $1.50$ & $(2,3)$ & NO & 7723\\
$(188,79)$ & 11 & $(17,7)$ & 6 & 1 & YES & YES & YES & $1.56$ & $(2,3)$ & NO & 7724\\
$(188,69)$ & 11 & $(18,7)$ & 6 & 2 & YES & YES & YES & $1.89$ & $(4,2)$ & NO & 7725\\
$(188,35)$ & 13 & $(23,4)$ & 8 & 1 & YES & YES & NO(2) & $1.56$ & $(8,0)$ & NO & 7726\\
$(188,43)$ & 13 & $(23,5)$ & 7 & 1 & YES & YES & YES & $1.90$ & $(2,3)$ & NO & 7727\\
$(188,43)$ & 13 & $(40,9)$ & 9 & 4 & YES & YES & YES & $1.67$ & $(4,2)$ & NO & 7728\\
$(188,79)$ & 11 & $(43,18)$ & 8 & 1 & YES & YES & YES & $1.56$ & $(2,3)$ & 10429 & 7729\\
$(188,79)$ & 11 & $(50,21)$ & 8 & 2 & YES & YES & NO(2) & $1.44$ & $(2,3)$ & 7185 & 7730\\
$(188,57)$ & 13 & $(56,17)$ & 9 & 4 & YES & YES & YES & $1.62$ & $(2,3)$ & NO & 7731\\
$(188,55)$ & 12 & $(65,19)$ & 9 & 1 & YES & YES & YES & $1.14$ & $(4,2)$ & NO & 7732\\
$(188,69)$ & 11 & $(68,25)$ & 9 & 4 & YES & YES & YES & $1.86$ & $(2,3)$ & NO & 7733\\
$(188,79)$ & 11 & $(81,34)$ & 9 & 1 & YES & YES & YES & $1.62$ & $(2,3)$ & NO & 7734\\
$(188,73)$ & 12 & $(85,33)$ & 10 & 1 & YES & YES & NO(2) & $1.56$ & $(8,0)$ & NO & 7735\\
$(188,69)$ & 11 & $(109,40)$ & 10 & 1 & YES & YES & NO(2) & $1.44$ & $(2,3)$ & NO & 7736\\
$(188,79)$ & 11 & $(119,50)$ & 10 & 1 & YES & YES & YES & $1.43$ & $(6,1)$ & NO & 7737\\
$(188,69)$ & 11 & $(128,47)$ & 10 & 4 & YES & YES & YES & $1.67$ & $(2,3)$ & NO & 7738\\
$(188,69)$ & 11 & $(139,51)$ & 11 & 1 & YES & YES & NO(2) & $1.56$ & $(4,2)$ & NO & 7739\\
$(188,55)$ & 12 & $(147,43)$ & 11 & 1 & YES & YES & YES & $1.14$ & $(4,2)$ & NO & 7740\\
$(188,57)$ & 13 & $(155,47)$ & 12 & 1 & YES & YES & NO(2) & $1.67$ & $(4,2)$ & NO & 7741\\
$(188,79)$ & 11 & $(188,79)$ & 11 & 188 & YES & YES & NO(2) & $1.56$ & $(4,2)$ & NO & 7742\\
$(189,82)$ & 12 & $(2,1)$ & 1 & 1 & YES & YES & YES & $1.57$ & $(2,3)$ & -- & 7743\\
$(189,50)$ & 13 & $(3,1)$ & 2 & 3 & YES & YES & YES & $1.62$ & $(2,3)$ & NO & 7744\\
$(189,55)$ & 12 & $(7,2)$ & 4 & 7 & YES & YES & YES & $1.67$ & $(2,3)$ & -- & 7745\\
$(189,40)$ & 12 & $(16,3)$ & 7 & 1 & YES & YES & NO(2) & $1.56$ & $(8,0)$ & NO & 7746\\
$(189,52)$ & 12 & $(98,27)$ & 10 & 7 & YES & YES & YES & $1.78$ & $(4,2)$ & NO & 7747\\
$(190,41)$ & 12 & $(5,2)$ & 3 & 5 & YES & YES & YES & $1.43$ & $(2,3)$ & NO & 7748\\
$(190,41)$ & 12 & $(5,2)$ & 3 & 5 & YES & YES & YES & $1.43$ & $(2,3)$ & -- & 7749\\
$(190,41)$ & 12 & $(5,2)$ & 3 & 5 & YES & YES & NO(2) & $1.50$ & $(4,2)$ & NO & 7750\\
$(190,51)$ & 12 & $(5,2)$ & 3 & 5 & YES & YES & YES & $2.00$ & $(2,3)$ & -- & 7751\\
$(190,51)$ & 12 & $(10,3)$ & 5 & 10 & YES & YES & YES & $1.75$ & $(2,3)$ & NO & 7752\\
$(190,51)$ & 12 & $(18,5)$ & 6 & 2 & YES & YES & YES & $2.00$ & $(2,3)$ & NO & 7753\\
$(190,41)$ & 12 & $(60,13)$ & 9 & 10 & YES & YES & YES & $1.43$ & $(2,3)$ & 10739 & 7754\\
$(191,58)$ & 12 & $(2,1)$ & 1 & 1 & YES & YES & NO(2) & $1.38$ & $(6,1)$ & NO & 7755\\
$(191,74)$ & 11 & $(2,1)$ & 1 & 1 & YES & YES & NO(2) & $1.44$ & $(12,-2)$ & NO & 7756\\
$(191,80)$ & 11 & $(2,1)$ & 1 & 1 & YES & YES & YES & $1.43$ & $(2,3)$ & -- & 7757\\
$(191,43)$ & 13 & $(3,1)$ & 2 & 1 & YES & YES & NO(2) & $1.44$ & $(8,0)$ & -- & 7758\\
$(191,50)$ & 13 & $(3,1)$ & 2 & 1 & YES & YES & NO(2) & $1.67$ & $(4,2)$ & -- & 7759\\
$(191,50)$ & 13 & $(3,1)$ & 2 & 1 & YES & YES & NO(2) & $1.62$ & $(6,1)$ & NO & 7760\\
$(191,58)$ & 12 & $(3,1)$ & 2 & 1 & YES & YES & NO(2) & $1.56$ & $(2,3)$ & -- & 7761\\
$(191,74)$ & 11 & $(3,1)$ & 2 & 1 & YES & YES & YES & $1.90$ & $(2,3)$ & -- & 7762\\
$(191,74)$ & 11 & $(3,1)$ & 2 & 1 & YES & YES & NO(2) & $1.70$ & $(2,3)$ & NO & 7763\\
$(191,80)$ & 11 & $(3,1)$ & 2 & 1 & YES & YES & YES & $1.43$ & $(2,3)$ & -- & 7764\\
$(191,80)$ & 11 & $(3,1)$ & 2 & 1 & YES & YES & YES & $1.43$ & $(4,2)$ & NO & 7765\\
$(191,59)$ & 13 & $(4,1)$ & 3 & 1 & YES & YES & YES & $1.62$ & $(2,3)$ & NO & 7766\\
$(191,80)$ & 11 & $(4,1)$ & 3 & 1 & YES & YES & YES & $1.50$ & $(4,2)$ & NO & 7767\\
$(191,80)$ & 11 & $(4,1)$ & 3 & 1 & YES & YES & YES & $1.29$ & $(6,1)$ & -- & 7768\\
$(191,68)$ & 13 & $(5,1)$ & 4 & 1 & YES & YES & NO(2) & $1.56$ & $(8,0)$ & NO & 7769\\
$(191,74)$ & 11 & $(5,2)$ & 3 & 1 & YES & YES & YES & $1.57$ & $(4,2)$ & -- & 7770\\
$(191,74)$ & 11 & $(5,2)$ & 3 & 1 & YES & YES & YES & $1.71$ & $(4,2)$ & NO & 7771\\
$(191,80)$ & 11 & $(5,2)$ & 3 & 1 & YES & YES & YES & $1.43$ & $(4,2)$ & -- & 7772\\
$(191,80)$ & 11 & $(5,2)$ & 3 & 1 & YES & YES & YES & $1.57$ & $(6,1)$ & NO & 7773\\
$(191,50)$ & 13 & $(7,2)$ & 4 & 1 & YES & YES & NO(2) & $1.50$ & $(6,1)$ & NO & 7774\\
$(191,56)$ & 12 & $(7,2)$ & 4 & 1 & YES & YES & YES & $1.57$ & $(2,3)$ & -- & 7775\\
$(191,74)$ & 11 & $(7,2)$ & 4 & 1 & YES & YES & YES & $1.62$ & $(4,2)$ & -- & 7776\\
$(191,80)$ & 11 & $(7,2)$ & 4 & 1 & YES & YES & YES & $1.67$ & $(4,2)$ & -- & 7777\\
$(191,80)$ & 11 & $(7,3)$ & 4 & 1 & YES & YES & YES & $1.43$ & $(2,3)$ & 5879 & 7778\\
$(191,80)$ & 11 & $(8,3)$ & 4 & 1 & YES & YES & YES & $1.67$ & $(2,3)$ & NO & 7779\\
$(191,74)$ & 11 & $(9,2)$ & 5 & 1 & YES & YES & YES & $1.71$ & $(2,3)$ & NO & 7780\\
$(191,74)$ & 11 & $(9,2)$ & 5 & 1 & YES & YES & YES & $1.62$ & $(2,3)$ & -- & 7781\\
$(191,80)$ & 11 & $(9,4)$ & 5 & 1 & YES & YES & YES & $1.62$ & $(4,2)$ & NO & 7782\\
$(191,50)$ & 13 & $(11,3)$ & 5 & 1 & YES & YES & NO(2) & $1.56$ & $(8,0)$ & NO & 7783\\
$(191,74)$ & 11 & $(11,4)$ & 5 & 1 & YES & YES & YES & $1.71$ & $(2,3)$ & NO & 7784\\
$(191,58)$ & 12 & $(13,4)$ & 6 & 1 & YES & YES & NO(2) & $1.38$ & $(6,1)$ & 6276 & 7785\\
$(191,40)$ & 13 & $(16,3)$ & 7 & 1 & YES & YES & NO(2) & $1.60$ & $(6,1)$ & NO & 7786\\
$(191,80)$ & 11 & $(19,8)$ & 6 & 1 & YES & YES & NO(2) & $1.50$ & $(2,3)$ & NO & 7787\\
$(191,80)$ & 11 & $(31,13)$ & 7 & 1 & YES & YES & YES & $1.43$ & $(4,2)$ & NO & 7788\\
$(191,74)$ & 11 & $(44,17)$ & 8 & 1 & YES & YES & YES & $1.43$ & $(4,2)$ & NO & 7789\\
$(191,74)$ & 11 & $(49,19)$ & 8 & 1 & YES & YES & NO(2) & $1.44$ & $(2,3)$ & NO & 7790\\
$(191,56)$ & 12 & $(65,19)$ & 9 & 1 & YES & YES & YES & $1.50$ & $(2,3)$ & NO & 7791\\
$(191,43)$ & 13 & $(71,16)$ & 10 & 1 & YES & YES & NO(2) & $1.70$ & $(6,1)$ & NO & 7792\\
$(191,68)$ & 13 & $(73,26)$ & 11 & 1 & YES & YES & NO(2) & $1.67$ & $(8,0)$ & 8110 & 7793\\
$(191,80)$ & 11 & $(74,31)$ & 9 & 1 & YES & YES & YES & $1.14$ & $(4,2)$ & NO & 7794\\
$(191,74)$ & 11 & $(75,29)$ & 9 & 1 & YES & YES & YES & $1.62$ & $(4,2)$ & NO & 7795\\
$(191,58)$ & 12 & $(79,24)$ & 10 & 1 & YES & YES & NO(2) & $1.56$ & $(2,3)$ & 8368 & 7796\\
$(191,56)$ & 12 & $(92,27)$ & 11 & 1 & YES & YES & YES & $1.57$ & $(4,2)$ & 10686 & 7797\\
$(191,56)$ & 12 & $(99,29)$ & 10 & 1 & YES & YES & YES & $1.38$ & $(4,2)$ & NO & 7798\\
$(191,80)$ & 11 & $(105,44)$ & 10 & 1 & YES & YES & YES & $1.78$ & $(2,3)$ & NO & 7799\\
$(191,74)$ & 11 & $(111,43)$ & 10 & 1 & YES & YES & YES & $1.50$ & $(2,3)$ & NO & 7800\\
$(191,80)$ & 11 & $(117,49)$ & 10 & 1 & YES & YES & NO(2) & $1.56$ & $(2,3)$ & NO & 7801\\
$(191,74)$ & 11 & $(142,55)$ & 11 & 1 & YES & YES & YES & $1.75$ & $(2,3)$ & NO & 7802\\
$(191,58)$ & 12 & $(181,55)$ & 12 & 1 & YES & YES & YES & $1.62$ & $(4,2)$ & NO & 7803\\
$(191,59)$ & 13 & $(191,59)$ & 13 & 191 & YES & YES & YES & $1.75$ & $(2,3)$ & NO & 7804\\
$(191,74)$ & 11 & $(191,74)$ & 11 & 191 & YES & YES & YES & $1.62$ & $(2,3)$ & NO & 7805\\
$(191,80)$ & 11 & $(191,80)$ & 11 & 191 & YES & YES & YES & $1.50$ & $(4,2)$ & NO & 7806\\
$(192,71)$ & 11 & $(2,1)$ & 1 & 2 & YES & YES & NO(2) & $1.60$ & $(2,3)$ & -- & 7807\\
$(192,71)$ & 11 & $(2,1)$ & 1 & 2 & YES & YES & YES & $1.57$ & $(2,3)$ & NO & 7808\\
$(192,73)$ & 11 & $(2,1)$ & 1 & 2 & YES & YES & NO(2) & $1.29$ & $(8,0)$ & -- & 7809\\
$(192,73)$ & 11 & $(2,1)$ & 1 & 2 & YES & YES & YES & $1.43$ & $(4,2)$ & NO & 7810\\
$(192,71)$ & 11 & $(3,1)$ & 2 & 3 & YES & YES & NO(2) & $1.56$ & $(4,2)$ & NO & 7811\\
$(192,71)$ & 11 & $(3,1)$ & 2 & 3 & YES & YES & NO(2) & $1.56$ & $(4,2)$ & -- & 7812\\
$(192,73)$ & 11 & $(3,1)$ & 2 & 3 & YES & YES & NO(2) & $1.67$ & $(2,3)$ & -- & 7813\\
$(192,73)$ & 11 & $(3,1)$ & 2 & 3 & YES & YES & NO(2) & $1.50$ & $(4,2)$ & NO & 7814\\
$(192,71)$ & 11 & $(4,1)$ & 3 & 4 & YES & YES & YES & $1.57$ & $(4,2)$ & -- & 7815\\
$(192,71)$ & 11 & $(4,1)$ & 3 & 4 & YES & YES & YES & $1.62$ & $(4,2)$ & NO & 7816\\
$(192,73)$ & 11 & $(4,1)$ & 3 & 4 & YES & YES & NO(2) & $1.67$ & $(2,3)$ & -- & 7817\\
$(192,73)$ & 11 & $(4,1)$ & 3 & 4 & YES & YES & YES & $1.43$ & $(4,2)$ & NO & 7818\\
$(192,73)$ & 11 & $(4,1)$ & 3 & 4 & YES & YES & YES & $1.43$ & $(6,1)$ & NO & 7819\\
$(192,71)$ & 11 & $(5,2)$ & 3 & 1 & YES & YES & YES & $1.50$ & $(2,3)$ & NO & 7820\\
$(192,73)$ & 11 & $(7,2)$ & 4 & 1 & YES & YES & YES & $1.71$ & $(2,3)$ & NO & 7821\\
$(192,73)$ & 11 & $(7,2)$ & 4 & 1 & YES & YES & YES & $1.71$ & $(2,3)$ & -- & 7822\\
$(192,73)$ & 11 & $(9,2)$ & 5 & 3 & YES & YES & YES & $1.50$ & $(4,2)$ & -- & 7823\\
$(192,73)$ & 11 & $(9,2)$ & 5 & 3 & YES & YES & YES & $1.62$ & $(4,2)$ & NO & 7824\\
$(192,73)$ & 11 & $(13,5)$ & 5 & 1 & YES & YES & YES & $1.50$ & $(2,3)$ & NO & 7825\\
$(192,71)$ & 11 & $(19,7)$ & 6 & 1 & YES & YES & NO(2) & $1.60$ & $(2,3)$ & 7361 & 7826\\
$(192,73)$ & 11 & $(29,11)$ & 7 & 1 & YES & YES & NO(2) & $1.60$ & $(6,1)$ & 8208 & 7827\\
$(192,73)$ & 11 & $(34,13)$ & 7 & 2 & YES & YES & YES & $1.90$ & $(2,3)$ & NO & 7828\\
$(192,71)$ & 11 & $(35,13)$ & 8 & 1 & YES & YES & NO(2) & $1.60$ & $(6,1)$ & NO & 7829\\
$(192,71)$ & 11 & $(46,17)$ & 8 & 2 & YES & YES & YES & $1.62$ & $(2,3)$ & 7058 & 7830\\
$(192,73)$ & 11 & $(71,27)$ & 9 & 1 & YES & YES & NO(2) & $1.56$ & $(4,2)$ & NO & 7831\\
$(192,73)$ & 11 & $(79,30)$ & 9 & 1 & YES & YES & YES & $1.89$ & $(2,3)$ & NO & 7832\\
$(192,73)$ & 11 & $(92,35)$ & 10 & 4 & YES & YES & YES & $1.62$ & $(4,2)$ & NO & 7833\\
$(192,73)$ & 11 & $(121,46)$ & 10 & 1 & YES & YES & NO(2) & $1.56$ & $(2,3)$ & NO & 7834\\
$(192,71)$ & 11 & $(192,71)$ & 11 & 192 & YES & YES & YES & $1.57$ & $(4,2)$ & NO & 7835\\
$(192,73)$ & 11 & $(192,73)$ & 11 & 192 & YES & YES & NO(2) & $1.67$ & $(2,3)$ & NO & 7836\\
$(193,60)$ & 12 & $(2,1)$ & 1 & 1 & YES & YES & NO(2) & $1.50$ & $(10,-1)$ & -- & 7837\\
$(193,74)$ & 12 & $(2,1)$ & 1 & 1 & YES & YES & YES & $1.62$ & $(2,3)$ & NO & 7838\\
$(193,74)$ & 12 & $(2,1)$ & 1 & 1 & YES & YES & YES & $1.50$ & $(2,3)$ & -- & 7839\\
$(193,81)$ & 11 & $(2,1)$ & 1 & 1 & YES & YES & NO(2) & $1.60$ & $(6,1)$ & -- & 7840\\
$(193,57)$ & 12 & $(3,1)$ & 2 & 1 & YES & YES & YES & $1.56$ & $(2,3)$ & -- & 7841\\
$(193,60)$ & 12 & $(3,1)$ & 2 & 1 & NO & YES & NO(2) & $1.50$ & $(6,1)$ & -- & 7842\\
$(193,74)$ & 12 & $(3,1)$ & 2 & 1 & YES & YES & YES & $1.57$ & $(2,3)$ & -- & 7843\\
$(193,81)$ & 11 & $(3,1)$ & 2 & 1 & YES & YES & NO(2) & $1.67$ & $(2,3)$ & -- & 7844\\
$(193,81)$ & 11 & $(3,1)$ & 2 & 1 & YES & YES & YES & $1.43$ & $(6,1)$ & NO & 7845\\
$(193,87)$ & 12 & $(3,1)$ & 2 & 1 & YES & YES & NO(2) & $1.67$ & $(4,2)$ & -- & 7846\\
$(193,53)$ & 12 & $(4,1)$ & 3 & 1 & YES & YES & NO(2) & $1.56$ & $(2,3)$ & -- & 7847\\
$(193,73)$ & 13 & $(4,1)$ & 3 & 1 & YES & YES & NO(2) & $1.56$ & $(8,0)$ & -- & 7848\\
$(193,81)$ & 11 & $(4,1)$ & 3 & 1 & YES & YES & YES & $1.62$ & $(4,2)$ & -- & 7849\\
$(193,57)$ & 12 & $(5,2)$ & 3 & 1 & YES & YES & YES & $1.57$ & $(2,3)$ & -- & 7850\\
$(193,74)$ & 12 & $(5,1)$ & 4 & 1 & YES & YES & NO(2) & $1.56$ & $(4,2)$ & -- & 7851\\
$(193,81)$ & 11 & $(5,2)$ & 3 & 1 & YES & YES & NO(2) & $1.56$ & $(2,3)$ & NO & 7852\\
$(193,81)$ & 11 & $(5,2)$ & 3 & 1 & YES & YES & YES & $1.75$ & $(2,3)$ & -- & 7853\\
$(193,45)$ & 14 & $(7,1)$ & 6 & 1 & YES & YES & NO(2) & $1.56$ & $(4,2)$ & NO & 7854\\
$(193,81)$ & 11 & $(7,3)$ & 4 & 1 & YES & YES & NO(2) & $1.56$ & $(4,2)$ & NO & 7855\\
$(193,71)$ & 12 & $(8,3)$ & 4 & 1 & YES & YES & NO(2) & $1.44$ & $(4,2)$ & NO & 7856\\
$(193,81)$ & 11 & $(8,3)$ & 4 & 1 & YES & YES & YES & $1.75$ & $(2,3)$ & NO & 7857\\
$(193,53)$ & 12 & $(9,2)$ & 5 & 1 & YES & YES & YES & $1.43$ & $(6,1)$ & -- & 7858\\
$(193,57)$ & 12 & $(9,2)$ & 5 & 1 & YES & YES & YES & $1.50$ & $(2,3)$ & -- & 7859\\
$(193,44)$ & 12 & $(11,4)$ & 5 & 1 & YES & YES & YES & $1.67$ & $(4,2)$ & -- & 7860\\
$(193,57)$ & 12 & $(16,5)$ & 7 & 1 & YES & YES & YES & $1.29$ & $(6,1)$ & NO & 7861\\
$(193,81)$ & 11 & $(19,8)$ & 6 & 1 & YES & YES & NO(2) & $1.73$ & $(4,2)$ & 5777 & 7862\\
$(193,74)$ & 12 & $(21,8)$ & 6 & 1 & YES & YES & YES & $1.71$ & $(2,3)$ & 5248 & 7863\\
$(193,81)$ & 11 & $(26,11)$ & 7 & 1 & YES & YES & YES & $1.75$ & $(4,2)$ & NO & 7864\\
$(193,53)$ & 12 & $(29,8)$ & 7 & 1 & YES & YES & NO(2) & $1.56$ & $(2,3)$ & NO & 7865\\
$(193,85)$ & 12 & $(34,15)$ & 8 & 1 & YES & YES & NO(2) & $1.70$ & $(10,-1)$ & 5705 & 7866\\
$(193,45)$ & 14 & $(43,10)$ & 9 & 1 & YES & YES & NO(2) & $1.50$ & $(6,1)$ & NO & 7867\\
$(193,74)$ & 12 & $(47,18)$ & 8 & 1 & YES & YES & NO(2) & $1.67$ & $(4,2)$ & NO & 7868\\
$(193,81)$ & 11 & $(50,21)$ & 8 & 1 & YES & YES & YES & $1.62$ & $(4,2)$ & NO & 7869\\
$(193,85)$ & 12 & $(59,26)$ & 9 & 1 & YES & YES & NO(2) & $1.60$ & $(10,-1)$ & NO & 7870\\
$(193,57)$ & 12 & $(61,18)$ & 9 & 1 & YES & YES & YES & $1.62$ & $(2,3)$ & NO & 7871\\
$(193,53)$ & 12 & $(69,19)$ & 9 & 1 & YES & YES & YES & $1.57$ & $(6,1)$ & 11130 & 7872\\
$(193,81)$ & 11 & $(69,29)$ & 9 & 1 & YES & YES & YES & $1.75$ & $(2,3)$ & NO & 7873\\
$(193,81)$ & 11 & $(112,47)$ & 10 & 1 & YES & YES & YES & $1.29$ & $(6,1)$ & NO & 7874\\
$(193,81)$ & 11 & $(131,55)$ & 10 & 1 & YES & YES & YES & $1.50$ & $(4,2)$ & NO & 7875\\
$(193,57)$ & 12 & $(193,57)$ & 12 & 193 & YES & YES & YES & $1.57$ & $(4,2)$ & NO & 7876\\
$(193,80)$ & 12 & $(193,80)$ & 12 & 193 & YES & YES & NO(2) & $1.50$ & $(6,1)$ & NO & 7877\\
$(193,81)$ & 11 & $(193,81)$ & 11 & 193 & YES & YES & YES & $1.62$ & $(2,3)$ & NO & 7878\\
$(194,75)$ & 11 & $(2,1)$ & 1 & 2 & YES & YES & NO(2) & $1.56$ & $(2,3)$ & NO & 7879\\
$(194,75)$ & 11 & $(2,1)$ & 1 & 2 & YES & YES & NO(2) & $1.56$ & $(2,3)$ & -- & 7880\\
$(194,75)$ & 11 & $(3,1)$ & 2 & 1 & YES & YES & NO(2) & $1.56$ & $(2,3)$ & -- & 7881\\
$(194,75)$ & 11 & $(3,1)$ & 2 & 1 & YES & YES & YES & $1.50$ & $(4,2)$ & NO & 7882\\
$(194,75)$ & 11 & $(4,1)$ & 3 & 2 & YES & YES & YES & $1.57$ & $(4,2)$ & -- & 7883\\
$(194,75)$ & 11 & $(5,2)$ & 3 & 1 & YES & YES & YES & $1.43$ & $(4,2)$ & -- & 7884\\
$(194,75)$ & 11 & $(7,2)$ & 4 & 1 & YES & YES & YES & $1.71$ & $(2,3)$ & NO & 7885\\
$(194,75)$ & 11 & $(7,3)$ & 4 & 1 & YES & YES & YES & $1.71$ & $(4,2)$ & NO & 7886\\
$(194,75)$ & 11 & $(9,2)$ & 5 & 1 & YES & YES & YES & $1.62$ & $(4,2)$ & NO & 7887\\
$(194,75)$ & 11 & $(9,2)$ & 5 & 1 & YES & YES & YES & $1.67$ & $(2,3)$ & -- & 7888\\
$(194,75)$ & 11 & $(10,3)$ & 5 & 2 & YES & YES & YES & $1.78$ & $(4,2)$ & -- & 7889\\
$(194,75)$ & 11 & $(11,4)$ & 5 & 1 & YES & YES & YES & $1.57$ & $(2,3)$ & NO & 7890\\
$(194,75)$ & 11 & $(18,7)$ & 6 & 2 & YES & YES & NO(2) & $1.56$ & $(2,3)$ & NO & 7891\\
$(194,75)$ & 11 & $(21,8)$ & 6 & 1 & YES & YES & YES & $1.86$ & $(2,3)$ & NO & 7892\\
$(194,75)$ & 11 & $(23,9)$ & 7 & 1 & YES & YES & YES & $1.57$ & $(4,2)$ & NO & 7893\\
$(194,75)$ & 11 & $(28,11)$ & 8 & 2 & YES & YES & YES & $1.71$ & $(6,1)$ & 5980 & 7894\\
$(194,75)$ & 11 & $(44,17)$ & 8 & 2 & YES & YES & YES & $1.50$ & $(2,3)$ & 7017 & 7895\\
$(194,45)$ & 13 & $(47,11)$ & 9 & 1 & YES & YES & NO(2) & $1.44$ & $(4,2)$ & NO & 7896\\
$(194,75)$ & 11 & $(49,19)$ & 8 & 1 & YES & YES & YES & $1.50$ & $(4,2)$ & NO & 7897\\
$(194,75)$ & 11 & $(57,22)$ & 9 & 1 & YES & YES & YES & $1.75$ & $(2,3)$ & NO & 7898\\
$(194,75)$ & 11 & $(75,29)$ & 9 & 1 & YES & YES & YES & $1.44$ & $(2,3)$ & NO & 7899\\
$(194,75)$ & 11 & $(106,41)$ & 10 & 2 & YES & YES & YES & $1.43$ & $(4,2)$ & NO & 7900\\
$(194,75)$ & 11 & $(119,46)$ & 10 & 1 & YES & YES & YES & $1.62$ & $(2,3)$ & NO & 7901\\
$(194,75)$ & 11 & $(163,63)$ & 11 & 1 & YES & YES & YES & $1.62$ & $(4,2)$ & NO & 7902\\
$(195,82)$ & 12 & $(2,1)$ & 1 & 1 & YES & YES & NO(2) & $1.67$ & $(4,2)$ & -- & 7903\\
$(195,76)$ & 12 & $(3,1)$ & 2 & 3 & YES & YES & NO(2) & $1.50$ & $(6,1)$ & NO & 7904\\
$(195,76)$ & 12 & $(3,1)$ & 2 & 3 & YES & YES & NO(2) & $1.50$ & $(6,1)$ & -- & 7905\\
$(195,59)$ & 12 & $(5,2)$ & 3 & 5 & YES & YES & YES & $1.78$ & $(2,3)$ & -- & 7906\\
$(195,82)$ & 12 & $(5,1)$ & 4 & 5 & YES & YES & NO(2) & $1.62$ & $(6,1)$ & NO & 7907\\
$(195,82)$ & 12 & $(5,1)$ & 4 & 5 & YES & YES & NO(2) & $1.62$ & $(6,1)$ & -- & 7908\\
$(195,88)$ & 12 & $(5,1)$ & 4 & 5 & YES & YES & NO(2) & $1.62$ & $(6,1)$ & -- & 7909\\
$(195,88)$ & 12 & $(5,1)$ & 4 & 5 & YES & YES & NO(2) & $1.67$ & $(4,2)$ & NO & 7910\\
$(195,88)$ & 12 & $(5,1)$ & 4 & 5 & YES & YES & NO(2) & $1.67$ & $(8,0)$ & NO & 7911\\
$(195,44)$ & 14 & $(7,1)$ & 6 & 1 & YES & YES & NO(2) & $1.25$ & $(6,1)$ & NO & 7912\\
$(195,59)$ & 12 & $(7,3)$ & 4 & 1 & YES & YES & YES & $1.62$ & $(4,2)$ & -- & 7913\\
$(195,59)$ & 12 & $(7,3)$ & 4 & 1 & YES & YES & YES & $1.75$ & $(4,2)$ & NO & 7914\\
$(195,76)$ & 12 & $(9,4)$ & 5 & 3 & YES & YES & YES & $1.71$ & $(4,2)$ & NO & 7915\\
$(195,59)$ & 12 & $(11,3)$ & 5 & 1 & YES & YES & YES & $1.62$ & $(2,3)$ & NO & 7916\\
$(195,82)$ & 12 & $(19,8)$ & 6 & 1 & YES & YES & NO(2) & $1.75$ & $(6,1)$ & 6119 & 7917\\
$(195,76)$ & 12 & $(23,9)$ & 7 & 1 & YES & YES & NO(2) & $1.56$ & $(8,0)$ & NO & 7918\\
$(195,88)$ & 12 & $(31,14)$ & 8 & 1 & YES & YES & NO(2) & $1.62$ & $(6,1)$ & 6546 & 7919\\
$(195,59)$ & 12 & $(56,17)$ & 9 & 1 & YES & YES & YES & $2.00$ & $(2,3)$ & NO & 7920\\
$(195,58)$ & 13 & $(121,36)$ & 11 & 1 & YES & YES & YES & $1.43$ & $(2,3)$ & 9868 & 7921\\
$(195,58)$ & 13 & $(195,58)$ & 13 & 195 & YES & YES & YES & $1.57$ & $(2,3)$ & NO & 7922\\
$(196,57)$ & 12 & $(2,1)$ & 1 & 2 & YES & YES & NO(2) & $1.44$ & $(4,2)$ & NO & 7923\\
$(196,41)$ & 13 & $(3,1)$ & 2 & 1 & YES & YES & NO(2) & $1.50$ & $(6,1)$ & NO & 7924\\
$(196,41)$ & 13 & $(3,1)$ & 2 & 1 & YES & YES & NO(2) & $1.50$ & $(6,1)$ & -- & 7925\\
$(196,43)$ & 13 & $(3,1)$ & 2 & 1 & YES & YES & NO(2) & $1.44$ & $(8,0)$ & -- & 7926\\
$(196,69)$ & 13 & $(3,1)$ & 2 & 1 & YES & YES & NO(2) & $1.62$ & $(6,1)$ & -- & 7927\\
$(196,75)$ & 11 & $(3,1)$ & 2 & 1 & YES & YES & NO(2) & $1.70$ & $(6,1)$ & NO & 7928\\
$(196,75)$ & 11 & $(3,1)$ & 2 & 1 & YES & YES & YES & $1.62$ & $(2,3)$ & -- & 7929\\
$(196,75)$ & 11 & $(3,1)$ & 2 & 1 & YES & YES & YES & $1.78$ & $(4,2)$ & NO & 7930\\
$(196,81)$ & 11 & $(3,1)$ & 2 & 1 & YES & YES & YES & $1.57$ & $(6,1)$ & NO & 7931\\
$(196,81)$ & 11 & $(3,1)$ & 2 & 1 & YES & YES & YES & $1.57$ & $(6,1)$ & -- & 7932\\
$(196,81)$ & 11 & $(4,1)$ & 3 & 4 & YES & YES & YES & $1.50$ & $(2,3)$ & NO & 7933\\
$(196,81)$ & 11 & $(4,1)$ & 3 & 4 & YES & YES & YES & $1.62$ & $(2,3)$ & -- & 7934\\
$(196,75)$ & 11 & $(5,2)$ & 3 & 1 & YES & YES & YES & $1.62$ & $(2,3)$ & -- & 7935\\
$(196,75)$ & 11 & $(5,2)$ & 3 & 1 & YES & YES & YES & $1.43$ & $(4,2)$ & NO & 7936\\
$(196,81)$ & 11 & $(5,2)$ & 3 & 1 & YES & YES & YES & $1.62$ & $(2,3)$ & -- & 7937\\
$(196,55)$ & 12 & $(7,3)$ & 4 & 7 & YES & YES & YES & $1.43$ & $(6,1)$ & -- & 7938\\
$(196,75)$ & 11 & $(7,2)$ & 4 & 7 & YES & YES & YES & $1.67$ & $(2,3)$ & -- & 7939\\
$(196,81)$ & 11 & $(7,2)$ & 4 & 7 & YES & YES & YES & $1.78$ & $(4,2)$ & -- & 7940\\
$(196,41)$ & 13 & $(9,2)$ & 5 & 1 & YES & YES & NO(2) & $1.38$ & $(6,1)$ & NO & 7941\\
$(196,61)$ & 13 & $(10,3)$ & 5 & 2 & YES & YES & NO(2) & $1.50$ & $(10,-1)$ & NO & 7942\\
$(196,41)$ & 13 & $(14,3)$ & 6 & 14 & YES & YES & YES & $1.50$ & $(2,3)$ & NO & 7943\\
$(196,81)$ & 11 & $(17,7)$ & 6 & 1 & YES & YES & YES & $1.43$ & $(2,3)$ & 7263 & 7944\\
$(196,75)$ & 11 & $(18,7)$ & 6 & 2 & YES & YES & YES & $1.60$ & $(2,3)$ & NO & 7945\\
$(196,81)$ & 11 & $(19,8)$ & 6 & 1 & YES & YES & YES & $1.67$ & $(4,2)$ & NO & 7946\\
$(196,83)$ & 12 & $(19,8)$ & 6 & 1 & YES & YES & NO(2) & $1.50$ & $(4,2)$ & NO & 7947\\
$(196,75)$ & 11 & $(21,8)$ & 6 & 7 & YES & YES & YES & $1.56$ & $(2,3)$ & NO & 7948\\
$(196,75)$ & 11 & $(23,9)$ & 7 & 1 & YES & YES & YES & $1.75$ & $(4,2)$ & NO & 7949\\
$(196,55)$ & 12 & $(29,8)$ & 7 & 1 & YES & YES & YES & $1.62$ & $(2,3)$ & NO & 7950\\
$(196,69)$ & 13 & $(37,13)$ & 9 & 1 & YES & YES & NO(2) & $1.50$ & $(6,1)$ & 8850 & 7951\\
$(196,81)$ & 11 & $(41,17)$ & 8 & 1 & YES & YES & YES & $1.62$ & $(2,3)$ & NO & 7952\\
$(196,75)$ & 11 & $(47,18)$ & 8 & 1 & YES & YES & YES & $1.80$ & $(2,3)$ & NO & 7953\\
$(196,75)$ & 11 & $(60,23)$ & 9 & 4 & YES & YES & YES & $1.62$ & $(2,3)$ & NO & 7954\\
$(196,81)$ & 11 & $(63,26)$ & 9 & 7 & YES & YES & YES & $1.86$ & $(2,3)$ & NO & 7955\\
$(196,81)$ & 11 & $(75,31)$ & 9 & 1 & YES & YES & NO(2) & $1.56$ & $(4,2)$ & NO & 7956\\
$(196,81)$ & 11 & $(104,43)$ & 10 & 4 & YES & YES & YES & $1.71$ & $(2,3)$ & NO & 7957\\
$(196,75)$ & 11 & $(115,44)$ & 10 & 1 & YES & YES & YES & $1.56$ & $(2,3)$ & NO & 7958\\
$(196,81)$ & 11 & $(121,50)$ & 10 & 1 & YES & YES & YES & $1.50$ & $(4,2)$ & NO & 7959\\
$(196,75)$ & 11 & $(128,49)$ & 10 & 4 & YES & YES & YES & $1.62$ & $(2,3)$ & NO & 7960\\
$(196,55)$ & 12 & $(132,37)$ & 12 & 4 & YES & YES & YES & $1.43$ & $(6,1)$ & NO & 7961\\
$(196,83)$ & 12 & $(144,61)$ & 11 & 4 & YES & YES & YES & $1.88$ & $(4,2)$ & NO & 7962\\
$(196,75)$ & 11 & $(149,57)$ & 11 & 1 & YES & YES & YES & $1.62$ & $(4,2)$ & NO & 7963\\
$(196,81)$ & 11 & $(167,69)$ & 11 & 1 & YES & YES & YES & $1.43$ & $(4,2)$ & NO & 7964\\
$(196,81)$ & 11 & $(196,81)$ & 11 & 196 & YES & YES & YES & $1.50$ & $(4,2)$ & NO & 7965\\
$(197,77)$ & 12 & $(2,1)$ & 1 & 1 & YES & YES & NO(2) & $1.78$ & $(8,0)$ & -- & 7966\\
$(197,86)$ & 12 & $(2,1)$ & 1 & 1 & YES & YES & NO(2) & $1.50$ & $(6,1)$ & NO & 7967\\
$(197,86)$ & 12 & $(2,1)$ & 1 & 1 & YES & YES & NO(2) & $1.50$ & $(6,1)$ & -- & 7968\\
$(197,46)$ & 14 & $(3,1)$ & 2 & 1 & YES & YES & NO(2) & $1.56$ & $(8,0)$ & -- & 7969\\
$(197,46)$ & 14 & $(3,1)$ & 2 & 1 & YES & YES & NO(2) & $1.67$ & $(8,0)$ & NO & 7970\\
$(197,55)$ & 12 & $(3,1)$ & 2 & 1 & YES & YES & YES & $1.29$ & $(4,2)$ & -- & 7971\\
$(197,76)$ & 12 & $(5,1)$ & 4 & 1 & YES & YES & NO(2) & $1.44$ & $(8,0)$ & -- & 7972\\
$(197,43)$ & 12 & $(7,3)$ & 4 & 1 & YES & YES & NO(2) & $1.67$ & $(2,3)$ & -- & 7973\\
$(197,45)$ & 13 & $(8,3)$ & 4 & 1 & YES & YES & YES & $1.43$ & $(6,1)$ & -- & 7974\\
$(197,43)$ & 12 & $(11,3)$ & 5 & 1 & YES & YES & YES & $1.67$ & $(2,3)$ & NO & 7975\\
$(197,61)$ & 13 & $(13,4)$ & 6 & 1 & YES & YES & NO(2) & $1.50$ & $(6,1)$ & NO & 7976\\
$(197,77)$ & 12 & $(18,7)$ & 6 & 1 & YES & YES & NO(2) & $1.67$ & $(8,0)$ & NO & 7977\\
$(197,45)$ & 13 & $(23,5)$ & 7 & 1 & YES & YES & YES & $1.67$ & $(4,2)$ & NO & 7978\\
$(197,52)$ & 12 & $(23,6)$ & 8 & 1 & YES & YES & NO(2) & $1.44$ & $(8,0)$ & NO & 7979\\
$(197,43)$ & 12 & $(33,7)$ & 8 & 1 & YES & YES & YES & $1.57$ & $(4,2)$ & NO & 7980\\
$(197,45)$ & 13 & $(40,9)$ & 9 & 1 & YES & YES & YES & $1.67$ & $(4,2)$ & NO & 7981\\
$(197,55)$ & 12 & $(68,19)$ & 9 & 1 & YES & YES & YES & $1.29$ & $(4,2)$ & NO & 7982\\
$(197,76)$ & 12 & $(70,27)$ & 10 & 1 & YES & YES & NO(2) & $1.56$ & $(8,0)$ & NO & 7983\\
$(197,46)$ & 14 & $(77,18)$ & 10 & 1 & YES & YES & NO(2) & $1.67$ & $(4,2)$ & NO & 7984\\
$(197,46)$ & 14 & $(167,39)$ & 13 & 1 & YES & YES & NO(2) & $1.56$ & $(4,2)$ & NO & 7985\\
$(198,59)$ & 13 & $(5,1)$ & 4 & 1 & YES & YES & YES & $1.57$ & $(2,3)$ & -- & 7986\\
$(198,47)$ & 13 & $(29,7)$ & 10 & 1 & YES & YES & NO(2) & $1.60$ & $(6,1)$ & NO & 7987\\
$(198,59)$ & 13 & $(37,11)$ & 8 & 1 & YES & YES & YES & $1.57$ & $(2,3)$ & NO & 7988\\
$(199,37)$ & 13 & $(2,1)$ & 1 & 1 & YES & YES & NO(2) & $1.50$ & $(2,3)$ & NO & 7989\\
$(199,76)$ & 11 & $(2,1)$ & 1 & 1 & YES & YES & YES & $1.56$ & $(2,3)$ & -- & 7990\\
$(199,55)$ & 11 & $(3,1)$ & 2 & 1 & YES & YES & YES & $1.38$ & $(2,3)$ & -- & 7991\\
$(199,55)$ & 11 & $(3,1)$ & 2 & 1 & YES & YES & YES & $1.38$ & $(2,3)$ & NO & 7992\\
$(199,76)$ & 11 & $(3,1)$ & 2 & 1 & YES & YES & YES & $1.78$ & $(2,3)$ & -- & 7993\\
$(199,76)$ & 11 & $(3,1)$ & 2 & 1 & YES & YES & YES & $1.78$ & $(2,3)$ & NO & 7994\\
$(199,78)$ & 12 & $(3,1)$ & 2 & 1 & YES & YES & YES & $1.71$ & $(2,3)$ & -- & 7995\\
$(199,76)$ & 11 & $(4,1)$ & 3 & 1 & YES & YES & YES & $1.78$ & $(2,3)$ & NO & 7996\\
$(199,76)$ & 11 & $(4,1)$ & 3 & 1 & YES & YES & YES & $1.67$ & $(2,3)$ & -- & 7997\\
$(199,55)$ & 11 & $(5,2)$ & 3 & 1 & YES & YES & YES & $1.38$ & $(2,3)$ & -- & 7998\\
$(199,76)$ & 11 & $(5,1)$ & 4 & 1 & YES & YES & YES & $1.78$ & $(2,3)$ & NO & 7999\\
$(199,76)$ & 11 & $(5,1)$ & 4 & 1 & YES & YES & YES & $1.78$ & $(2,3)$ & NO & 8000\\
$(199,76)$ & 11 & $(5,1)$ & 4 & 1 & YES & YES & YES & $1.80$ & $(2,3)$ & -- & 8001\\
$(199,76)$ & 11 & $(5,2)$ & 3 & 1 & YES & YES & YES & $1.67$ & $(2,3)$ & -- & 8002\\
$(199,60)$ & 14 & $(7,1)$ & 6 & 1 & YES & YES & NO(2) & $1.50$ & $(6,1)$ & NO & 8003\\
$(199,76)$ & 11 & $(7,3)$ & 4 & 1 & YES & YES & YES & $1.50$ & $(4,2)$ & NO & 8004\\
$(199,76)$ & 11 & $(8,3)$ & 4 & 1 & YES & YES & NO(2) & $1.38$ & $(4,2)$ & NO & 8005\\
$(199,55)$ & 11 & $(9,2)$ & 5 & 1 & YES & YES & YES & $1.50$ & $(4,2)$ & NO & 8006\\
$(199,76)$ & 11 & $(13,5)$ & 5 & 1 & YES & YES & NO(2) & $1.38$ & $(4,2)$ & NO & 8007\\
$(199,55)$ & 11 & $(15,4)$ & 6 & 1 & YES & YES & YES & $1.38$ & $(2,3)$ & NO & 8008\\
$(199,76)$ & 11 & $(21,8)$ & 6 & 1 & YES & YES & NO(2) & $1.38$ & $(4,2)$ & NO & 8009\\
$(199,54)$ & 13 & $(26,7)$ & 7 & 1 & YES & YES & NO(2) & $1.56$ & $(8,0)$ & NO & 8010\\
$(199,47)$ & 13 & $(29,7)$ & 10 & 1 & YES & YES & NO(2) & $1.60$ & $(6,1)$ & NO & 8011\\
$(199,76)$ & 11 & $(29,11)$ & 7 & 1 & YES & YES & YES & $1.70$ & $(2,3)$ & NO & 8012\\
$(199,76)$ & 11 & $(34,13)$ & 7 & 1 & YES & YES & YES & $1.56$ & $(2,3)$ & NO & 8013\\
$(199,54)$ & 13 & $(37,10)$ & 8 & 1 & YES & YES & NO(2) & $1.56$ & $(8,0)$ & NO & 8014\\
$(199,55)$ & 11 & $(40,11)$ & 8 & 1 & YES & YES & YES & $1.50$ & $(2,3)$ & NO & 8015\\
$(199,76)$ & 11 & $(89,34)$ & 9 & 1 & YES & YES & YES & $1.89$ & $(2,3)$ & 8786 & 8016\\
$(199,55)$ & 11 & $(105,29)$ & 10 & 1 & YES & YES & YES & $1.50$ & $(2,3)$ & NO & 8017\\
$(199,78)$ & 12 & $(125,49)$ & 11 & 1 & YES & YES & YES & $1.71$ & $(2,3)$ & NO & 8018\\
$(199,76)$ & 11 & $(144,55)$ & 10 & 1 & YES & YES & YES & $1.78$ & $(2,3)$ & NO & 8019\\
$(199,76)$ & 11 & $(199,76)$ & 11 & 199 & YES & YES & YES & $1.78$ & $(2,3)$ & NO & 8020\\
$(199,84)$ & 12 & $(199,84)$ & 12 & 199 & YES & YES & YES & $1.62$ & $(2,3)$ & NO & 8021\\
$(200,53)$ & 12 & $(2,1)$ & 1 & 2 & YES & YES & YES & $1.43$ & $(4,2)$ & NO & 8022\\
$(200,61)$ & 12 & $(3,1)$ & 2 & 1 & YES & YES & NO(2) & $1.67$ & $(4,2)$ & NO & 8023\\
$(200,61)$ & 12 & $(3,1)$ & 2 & 1 & YES & YES & NO(2) & $1.67$ & $(4,2)$ & -- & 8024\\
$(200,59)$ & 12 & $(5,2)$ & 3 & 5 & YES & YES & YES & $1.67$ & $(2,3)$ & -- & 8025\\
$(200,59)$ & 12 & $(7,2)$ & 4 & 1 & YES & YES & YES & $1.71$ & $(2,3)$ & -- & 8026\\
$(200,61)$ & 12 & $(11,3)$ & 5 & 1 & YES & YES & YES & $1.62$ & $(4,2)$ & NO & 8027\\
$(200,37)$ & 15 & $(49,9)$ & 10 & 1 & YES & YES & NO(2) & $1.60$ & $(6,1)$ & NO & 8028\\
$(200,59)$ & 12 & $(71,21)$ & 9 & 1 & YES & YES & YES & $1.62$ & $(2,3)$ & NO & 8029\\
$(200,83)$ & 12 & $(94,39)$ & 10 & 2 & YES & YES & YES & $1.43$ & $(4,2)$ & 8978 & 8030\\
$(201,76)$ & 12 & $(2,1)$ & 1 & 1 & YES & YES & NO(2) & $1.56$ & $(8,0)$ & -- & 8031\\
$(201,83)$ & 12 & $(2,1)$ & 1 & 1 & YES & YES & YES & $1.57$ & $(2,3)$ & -- & 8032\\
$(201,61)$ & 12 & $(3,1)$ & 2 & 3 & NO & YES & YES & $1.50$ & $(2,3)$ & -- & 8033\\
$(201,77)$ & 12 & $(4,1)$ & 3 & 1 & YES & YES & YES & $1.29$ & $(4,2)$ & -- & 8034\\
$(201,47)$ & 12 & $(5,2)$ & 3 & 1 & YES & YES & NO(2) & $1.56$ & $(2,3)$ & NO & 8035\\
$(201,61)$ & 12 & $(5,1)$ & 4 & 1 & YES & YES & YES & $1.43$ & $(4,2)$ & NO & 8036\\
$(201,61)$ & 12 & $(5,1)$ & 4 & 1 & YES & YES & YES & $1.57$ & $(4,2)$ & -- & 8037\\
$(201,61)$ & 12 & $(5,2)$ & 3 & 1 & YES & YES & YES & $1.50$ & $(4,2)$ & -- & 8038\\
$(201,61)$ & 12 & $(7,2)$ & 4 & 1 & YES & YES & YES & $1.62$ & $(2,3)$ & -- & 8039\\
$(201,56)$ & 12 & $(15,4)$ & 6 & 3 & YES & YES & YES & $1.57$ & $(4,2)$ & NO & 8040\\
$(201,61)$ & 12 & $(17,5)$ & 6 & 1 & YES & YES & YES & $1.62$ & $(2,3)$ & NO & 8041\\
$(201,56)$ & 12 & $(29,8)$ & 7 & 1 & YES & YES & YES & $1.57$ & $(4,2)$ & NO & 8042\\
$(201,76)$ & 12 & $(82,31)$ & 10 & 1 & YES & YES & NO(2) & $1.56$ & $(4,2)$ & NO & 8043\\
$(201,56)$ & 12 & $(97,27)$ & 11 & 1 & YES & YES & YES & $1.57$ & $(4,2)$ & 10899 & 8044\\
$(201,56)$ & 12 & $(104,29)$ & 10 & 1 & YES & YES & YES & $1.50$ & $(4,2)$ & NO & 8045\\
$(201,83)$ & 12 & $(155,64)$ & 11 & 1 & YES & YES & YES & $1.57$ & $(4,2)$ & NO & 8046\\
$(201,61)$ & 12 & $(201,61)$ & 12 & 201 & YES & YES & YES & $1.57$ & $(4,2)$ & NO & 8047\\
$(202,59)$ & 12 & $(2,1)$ & 1 & 2 & YES & YES & NO(2) & $1.70$ & $(2,3)$ & -- & 8048\\
$(202,59)$ & 12 & $(3,1)$ & 2 & 1 & YES & YES & YES & $1.57$ & $(2,3)$ & -- & 8049\\
$(202,59)$ & 12 & $(4,1)$ & 3 & 2 & YES & YES & YES & $1.43$ & $(2,3)$ & -- & 8050\\
$(202,43)$ & 13 & $(5,2)$ & 3 & 1 & YES & YES & NO(2) & $1.50$ & $(4,2)$ & NO & 8051\\
$(202,43)$ & 13 & $(5,2)$ & 3 & 1 & YES & YES & NO(2) & $1.50$ & $(4,2)$ & -- & 8052\\
$(202,53)$ & 13 & $(5,1)$ & 4 & 1 & YES & YES & NO(2) & $1.56$ & $(4,2)$ & -- & 8053\\
$(202,59)$ & 12 & $(5,1)$ & 4 & 1 & YES & YES & NO(2) & $1.64$ & $(4,2)$ & -- & 8054\\
$(202,59)$ & 12 & $(5,1)$ & 4 & 1 & YES & YES & NO(2) & $1.73$ & $(4,2)$ & NO & 8055\\
$(202,59)$ & 12 & $(5,2)$ & 3 & 1 & YES & YES & YES & $1.43$ & $(4,2)$ & -- & 8056\\
$(202,83)$ & 12 & $(5,2)$ & 3 & 1 & YES & YES & YES & $1.62$ & $(4,2)$ & -- & 8057\\
$(202,59)$ & 12 & $(8,3)$ & 4 & 2 & YES & YES & YES & $1.67$ & $(4,2)$ & -- & 8058\\
$(202,59)$ & 12 & $(10,3)$ & 5 & 2 & YES & YES & NO(2) & $1.33$ & $(8,0)$ & NO & 8059\\
$(202,59)$ & 12 & $(11,3)$ & 5 & 1 & YES & YES & NO(2) & $1.67$ & $(2,3)$ & NO & 8060\\
$(202,53)$ & 13 & $(15,4)$ & 6 & 1 & YES & YES & YES & $1.43$ & $(4,2)$ & NO & 8061\\
$(202,59)$ & 12 & $(17,5)$ & 6 & 1 & YES & YES & NO(2) & $1.60$ & $(2,3)$ & NO & 8062\\
$(202,59)$ & 12 & $(27,8)$ & 7 & 1 & YES & YES & YES & $1.43$ & $(4,2)$ & NO & 8063\\
$(202,59)$ & 12 & $(41,12)$ & 8 & 1 & YES & YES & YES & $1.43$ & $(2,3)$ & 5811 & 8064\\
$(202,83)$ & 12 & $(56,23)$ & 9 & 2 & YES & YES & YES & $1.43$ & $(2,3)$ & 7628 & 8065\\
$(202,59)$ & 12 & $(65,19)$ & 9 & 1 & YES & YES & YES & $1.57$ & $(2,3)$ & NO & 8066\\
$(202,83)$ & 12 & $(95,39)$ & 10 & 1 & YES & YES & YES & $1.62$ & $(4,2)$ & NO & 8067\\
$(202,59)$ & 12 & $(99,29)$ & 10 & 1 & YES & YES & YES & $1.56$ & $(4,2)$ & NO & 8068\\
$(203,57)$ & 12 & $(2,1)$ & 1 & 1 & YES & YES & NO(2) & $1.56$ & $(4,2)$ & NO & 8069\\
$(203,57)$ & 12 & $(2,1)$ & 1 & 1 & YES & YES & NO(2) & $1.70$ & $(6,1)$ & -- & 8070\\
$(203,59)$ & 12 & $(2,1)$ & 1 & 1 & YES & YES & NO(2) & $1.80$ & $(2,3)$ & -- & 8071\\
$(203,85)$ & 12 & $(2,1)$ & 1 & 1 & YES & YES & NO(2) & $1.67$ & $(8,0)$ & -- & 8072\\
$(203,89)$ & 12 & $(2,1)$ & 1 & 1 & YES & YES & NO(2) & $1.67$ & $(8,0)$ & NO & 8073\\
$(203,57)$ & 12 & $(3,1)$ & 2 & 1 & YES & YES & NO(2) & $1.60$ & $(2,3)$ & NO & 8074\\
$(203,86)$ & 12 & $(3,1)$ & 2 & 1 & YES & YES & NO(2) & $1.44$ & $(8,0)$ & NO & 8075\\
$(203,86)$ & 12 & $(5,2)$ & 3 & 1 & YES & YES & NO(2) & $1.56$ & $(8,0)$ & NO & 8076\\
$(203,60)$ & 12 & $(6,1)$ & 5 & 1 & YES & YES & YES & $1.43$ & $(6,1)$ & NO & 8077\\
$(203,48)$ & 13 & $(7,3)$ & 4 & 7 & YES & YES & YES & $1.75$ & $(4,2)$ & -- & 8078\\
$(203,57)$ & 12 & $(7,2)$ & 4 & 7 & YES & YES & NO(2) & $1.56$ & $(4,2)$ & NO & 8079\\
$(203,59)$ & 12 & $(7,2)$ & 4 & 7 & YES & YES & YES & $1.56$ & $(4,2)$ & -- & 8080\\
$(203,48)$ & 13 & $(10,3)$ & 5 & 1 & YES & YES & YES & $1.62$ & $(4,2)$ & -- & 8081\\
$(203,48)$ & 13 & $(19,4)$ & 7 & 1 & YES & YES & YES & $1.75$ & $(4,2)$ & NO & 8082\\
$(203,59)$ & 12 & $(24,7)$ & 7 & 1 & YES & YES & NO(2) & $1.70$ & $(2,3)$ & 5886 & 8083\\
$(203,59)$ & 12 & $(27,8)$ & 7 & 1 & YES & YES & YES & $1.56$ & $(4,2)$ & NO & 8084\\
$(203,57)$ & 12 & $(32,9)$ & 8 & 1 & YES & YES & NO(2) & $1.73$ & $(4,2)$ & NO & 8085\\
$(203,60)$ & 12 & $(37,11)$ & 8 & 1 & YES & YES & YES & $1.67$ & $(2,3)$ & NO & 8086\\
$(203,89)$ & 12 & $(57,25)$ & 9 & 1 & YES & YES & NO(2) & $1.56$ & $(8,0)$ & 7691 & 8087\\
$(203,60)$ & 12 & $(71,21)$ & 9 & 1 & YES & YES & YES & $1.38$ & $(4,2)$ & NO & 8088\\
$(203,48)$ & 13 & $(97,23)$ & 11 & 1 & YES & YES & YES & $1.67$ & $(4,2)$ & NO & 8089\\
$(203,60)$ & 12 & $(159,47)$ & 11 & 1 & YES & YES & YES & $1.50$ & $(2,3)$ & NO & 8090\\
$(203,60)$ & 12 & $(203,60)$ & 12 & 203 & YES & YES & YES & $1.43$ & $(6,1)$ & NO & 8091\\
$(205,73)$ & 13 & $(2,1)$ & 1 & 1 & YES & YES & NO(2) & $1.67$ & $(8,0)$ & -- & 8092\\
$(205,84)$ & 12 & $(2,1)$ & 1 & 1 & YES & YES & NO(2) & $1.67$ & $(8,0)$ & -- & 8093\\
$(205,92)$ & 12 & $(2,1)$ & 1 & 1 & NO & YES & NO(2) & $1.56$ & $(4,2)$ & -- & 8094\\
$(205,62)$ & 13 & $(3,1)$ & 2 & 1 & YES & YES & NO(2) & $1.62$ & $(4,2)$ & -- & 8095\\
$(205,62)$ & 13 & $(3,1)$ & 2 & 1 & YES & YES & NO(2) & $1.62$ & $(4,2)$ & NO & 8096\\
$(205,61)$ & 12 & $(5,2)$ & 3 & 5 & YES & YES & YES & $1.57$ & $(4,2)$ & -- & 8097\\
$(205,61)$ & 12 & $(5,2)$ & 3 & 5 & YES & YES & YES & $1.75$ & $(4,2)$ & NO & 8098\\
$(205,62)$ & 13 & $(5,1)$ & 4 & 5 & YES & YES & NO(2) & $1.38$ & $(4,2)$ & -- & 8099\\
$(205,62)$ & 13 & $(5,1)$ & 4 & 5 & YES & YES & NO(2) & $1.50$ & $(4,2)$ & NO & 8100\\
$(205,89)$ & 12 & $(5,2)$ & 3 & 5 & YES & YES & YES & $1.62$ & $(4,2)$ & -- & 8101\\
$(205,61)$ & 12 & $(7,3)$ & 4 & 1 & YES & YES & YES & $1.75$ & $(4,2)$ & NO & 8102\\
$(205,49)$ & 14 & $(13,3)$ & 6 & 1 & YES & YES & NO(2) & $1.56$ & $(4,2)$ & NO & 8103\\
$(205,62)$ & 13 & $(13,4)$ & 6 & 1 & YES & YES & NO(2) & $1.50$ & $(4,2)$ & NO & 8104\\
$(205,49)$ & 14 & $(17,4)$ & 7 & 1 & YES & YES & NO(2) & $1.50$ & $(6,1)$ & NO & 8105\\
$(205,48)$ & 13 & $(23,5)$ & 7 & 1 & YES & YES & YES & $1.62$ & $(4,2)$ & NO & 8106\\
$(205,62)$ & 13 & $(23,7)$ & 7 & 1 & YES & YES & NO(2) & $1.50$ & $(4,2)$ & NO & 8107\\
$(205,61)$ & 12 & $(24,7)$ & 7 & 1 & YES & YES & YES & $1.57$ & $(4,2)$ & NO & 8108\\
$(205,48)$ & 13 & $(31,7)$ & 8 & 1 & YES & YES & YES & $1.62$ & $(4,2)$ & NO & 8109\\
$(205,73)$ & 13 & $(59,21)$ & 10 & 1 & YES & YES & NO(2) & $1.67$ & $(8,0)$ & 7793 & 8110\\
$(205,62)$ & 13 & $(76,23)$ & 10 & 1 & YES & YES & NO(2) & $1.44$ & $(8,0)$ & NO & 8111\\
$(205,48)$ & 13 & $(81,19)$ & 11 & 1 & YES & YES & YES & $1.43$ & $(2,3)$ & NO & 8112\\
$(205,62)$ & 13 & $(119,36)$ & 11 & 1 & YES & YES & NO(2) & $1.38$ & $(4,2)$ & 9887 & 8113\\
$(205,78)$ & 12 & $(205,78)$ & 12 & 205 & YES & YES & YES & $1.71$ & $(2,3)$ & NO & 8114\\
$(206,73)$ & 12 & $(2,1)$ & 1 & 2 & YES & YES & NO(2) & $1.50$ & $(6,1)$ & NO & 8115\\
$(206,85)$ & 12 & $(2,1)$ & 1 & 2 & YES & YES & YES & $1.43$ & $(2,3)$ & -- & 8116\\
$(206,57)$ & 12 & $(3,1)$ & 2 & 1 & YES & YES & YES & $1.56$ & $(2,3)$ & NO & 8117\\
$(206,85)$ & 12 & $(3,1)$ & 2 & 1 & YES & YES & NO(2) & $1.50$ & $(6,1)$ & NO & 8118\\
$(206,87)$ & 12 & $(3,1)$ & 2 & 1 & YES & YES & NO(2) & $1.56$ & $(4,2)$ & NO & 8119\\
$(206,87)$ & 12 & $(3,1)$ & 2 & 1 & YES & YES & NO(2) & $1.56$ & $(4,2)$ & -- & 8120\\
$(206,47)$ & 12 & $(5,2)$ & 3 & 1 & YES & YES & YES & $1.67$ & $(2,3)$ & NO & 8121\\
$(206,57)$ & 12 & $(5,2)$ & 3 & 1 & YES & YES & YES & $1.80$ & $(2,3)$ & -- & 8122\\
$(206,85)$ & 12 & $(5,2)$ & 3 & 1 & YES & YES & YES & $1.75$ & $(4,2)$ & -- & 8123\\
$(206,85)$ & 12 & $(5,2)$ & 3 & 1 & YES & YES & YES & $1.78$ & $(4,2)$ & NO & 8124\\
$(206,73)$ & 12 & $(6,1)$ & 5 & 2 & YES & YES & NO(2) & $1.62$ & $(6,1)$ & NO & 8125\\
$(206,47)$ & 12 & $(7,2)$ & 4 & 1 & YES & YES & YES & $1.62$ & $(4,2)$ & NO & 8126\\
$(206,63)$ & 12 & $(7,2)$ & 4 & 1 & YES & YES & NO(2) & $1.64$ & $(4,2)$ & NO & 8127\\
$(206,85)$ & 12 & $(7,3)$ & 4 & 1 & YES & YES & YES & $1.43$ & $(2,3)$ & NO & 8128\\
$(206,85)$ & 12 & $(12,5)$ & 5 & 2 & YES & YES & NO(2) & $1.38$ & $(6,1)$ & 5052 & 8129\\
$(206,47)$ & 12 & $(14,3)$ & 6 & 2 & YES & YES & YES & $1.38$ & $(2,3)$ & NO & 8130\\
$(206,85)$ & 12 & $(17,7)$ & 6 & 1 & YES & YES & YES & $1.57$ & $(2,3)$ & NO & 8131\\
$(206,57)$ & 12 & $(18,5)$ & 6 & 2 & YES & YES & YES & $1.50$ & $(2,3)$ & NO & 8132\\
$(206,47)$ & 12 & $(19,4)$ & 7 & 1 & YES & YES & YES & $1.43$ & $(4,2)$ & NO & 8133\\
$(206,85)$ & 12 & $(29,12)$ & 7 & 1 & YES & YES & NO(2) & $1.56$ & $(4,2)$ & NO & 8134\\
$(206,57)$ & 12 & $(40,11)$ & 8 & 2 & YES & YES & YES & $1.80$ & $(2,3)$ & NO & 8135\\
$(206,63)$ & 12 & $(49,15)$ & 9 & 1 & YES & YES & NO(2) & $1.67$ & $(4,2)$ & NO & 8136\\
$(206,57)$ & 12 & $(65,18)$ & 9 & 1 & YES & YES & YES & $1.62$ & $(2,3)$ & NO & 8137\\
$(206,47)$ & 12 & $(83,19)$ & 10 & 1 & YES & YES & YES & $1.78$ & $(2,3)$ & NO & 8138\\
$(207,55)$ & 13 & $(2,1)$ & 1 & 1 & YES & YES & YES & $1.62$ & $(2,3)$ & -- & 8139\\
$(207,55)$ & 13 & $(2,1)$ & 1 & 1 & YES & YES & YES & $1.75$ & $(2,3)$ & NO & 8140\\
$(207,79)$ & 11 & $(2,1)$ & 1 & 1 & YES & YES & NO(2) & $1.56$ & $(2,3)$ & NO & 8141\\
$(207,79)$ & 11 & $(2,1)$ & 1 & 1 & YES & YES & NO(2) & $1.25$ & $(4,2)$ & -- & 8142\\
$(207,55)$ & 13 & $(3,1)$ & 2 & 3 & YES & YES & YES & $1.62$ & $(2,3)$ & NO & 8143\\
$(207,55)$ & 13 & $(3,1)$ & 2 & 3 & YES & YES & YES & $1.62$ & $(2,3)$ & -- & 8144\\
$(207,61)$ & 13 & $(3,1)$ & 2 & 3 & YES & YES & NO(2) & $1.56$ & $(4,2)$ & NO & 8145\\
$(207,61)$ & 13 & $(3,1)$ & 2 & 3 & YES & YES & NO(2) & $1.56$ & $(4,2)$ & -- & 8146\\
$(207,76)$ & 11 & $(3,1)$ & 2 & 3 & YES & YES & YES & $1.62$ & $(2,3)$ & NO & 8147\\
$(207,76)$ & 11 & $(3,1)$ & 2 & 3 & YES & YES & YES & $1.62$ & $(4,2)$ & -- & 8148\\
$(207,79)$ & 11 & $(3,1)$ & 2 & 3 & YES & YES & YES & $1.50$ & $(4,2)$ & -- & 8149\\
$(207,79)$ & 11 & $(4,1)$ & 3 & 1 & YES & YES & YES & $1.50$ & $(2,3)$ & NO & 8150\\
$(207,79)$ & 11 & $(4,1)$ & 3 & 1 & YES & YES & YES & $1.50$ & $(2,3)$ & -- & 8151\\
$(207,64)$ & 13 & $(5,1)$ & 4 & 1 & YES & YES & YES & $1.57$ & $(2,3)$ & NO & 8152\\
$(207,76)$ & 11 & $(5,1)$ & 4 & 1 & YES & YES & YES & $1.75$ & $(2,3)$ & NO & 8153\\
$(207,76)$ & 11 & $(5,1)$ & 4 & 1 & YES & YES & YES & $1.75$ & $(2,3)$ & -- & 8154\\
$(207,76)$ & 11 & $(5,2)$ & 3 & 1 & YES & YES & YES & $1.43$ & $(4,2)$ & -- & 8155\\
$(207,76)$ & 11 & $(5,2)$ & 3 & 1 & YES & YES & YES & $1.62$ & $(2,3)$ & 8767 & 8156\\
$(207,79)$ & 11 & $(5,2)$ & 3 & 1 & YES & YES & NO(2) & $1.56$ & $(2,3)$ & NO & 8157\\
$(207,79)$ & 11 & $(5,2)$ & 3 & 1 & YES & YES & YES & $1.67$ & $(2,3)$ & -- & 8158\\
$(207,85)$ & 12 & $(5,2)$ & 3 & 1 & YES & YES & YES & $1.80$ & $(2,3)$ & -- & 8159\\
$(207,79)$ & 11 & $(7,2)$ & 4 & 1 & YES & YES & YES & $1.75$ & $(2,3)$ & NO & 8160\\
$(207,79)$ & 11 & $(7,3)$ & 4 & 1 & YES & YES & YES & $1.67$ & $(4,2)$ & NO & 8161\\
$(207,79)$ & 11 & $(8,3)$ & 4 & 1 & YES & YES & NO(2) & $1.56$ & $(2,3)$ & NO & 8162\\
$(207,79)$ & 11 & $(9,2)$ & 5 & 9 & YES & YES & YES & $1.86$ & $(2,3)$ & -- & 8163\\
$(207,79)$ & 11 & $(13,5)$ & 5 & 1 & YES & YES & YES & $1.56$ & $(2,3)$ & NO & 8164\\
$(207,80)$ & 12 & $(13,5)$ & 5 & 1 & YES & YES & NO(2) & $1.67$ & $(4,2)$ & NO & 8165\\
$(207,76)$ & 11 & $(19,7)$ & 6 & 1 & YES & YES & NO(2) & $1.56$ & $(8,0)$ & 7626 & 8166\\
$(207,76)$ & 11 & $(27,10)$ & 7 & 9 & YES & YES & YES & $1.75$ & $(2,3)$ & NO & 8167\\
$(207,79)$ & 11 & $(29,11)$ & 7 & 1 & YES & YES & YES & $1.80$ & $(2,3)$ & NO & 8168\\
$(207,76)$ & 11 & $(30,11)$ & 7 & 3 & YES & YES & NO(2) & $1.50$ & $(2,3)$ & NO & 8169\\
$(207,79)$ & 11 & $(34,13)$ & 7 & 1 & YES & YES & YES & $1.62$ & $(2,3)$ & 8781 & 8170\\
$(207,76)$ & 11 & $(41,15)$ & 8 & 1 & YES & YES & YES & $1.75$ & $(2,3)$ & NO & 8171\\
$(207,79)$ & 11 & $(47,18)$ & 8 & 1 & YES & YES & YES & $1.29$ & $(4,2)$ & 10861 & 8172\\
$(207,79)$ & 11 & $(55,21)$ & 8 & 1 & YES & YES & YES & $1.62$ & $(2,3)$ & 7655 & 8173\\
$(207,76)$ & 11 & $(68,25)$ & 9 & 1 & YES & YES & YES & $1.43$ & $(4,2)$ & NO & 8174\\
$(207,79)$ & 11 & $(76,29)$ & 9 & 1 & YES & YES & NO(2) & $1.67$ & $(2,3)$ & NO & 8175\\
$(207,79)$ & 11 & $(89,34)$ & 9 & 1 & YES & YES & YES & $1.71$ & $(2,3)$ & NO & 8176\\
$(207,64)$ & 13 & $(97,30)$ & 11 & 1 & YES & YES & YES & $1.57$ & $(2,3)$ & 9187 & 8177\\
$(207,79)$ & 11 & $(97,37)$ & 10 & 1 & YES & YES & YES & $1.57$ & $(4,2)$ & NO & 8178\\
$(207,76)$ & 11 & $(109,40)$ & 10 & 1 & YES & YES & YES & $1.62$ & $(4,2)$ & NO & 8179\\
$(207,76)$ & 11 & $(128,47)$ & 10 & 1 & YES & YES & YES & $1.62$ & $(4,2)$ & NO & 8180\\
$(207,79)$ & 11 & $(131,50)$ & 10 & 1 & YES & YES & YES & $1.62$ & $(2,3)$ & NO & 8181\\
$(207,85)$ & 12 & $(134,55)$ & 11 & 1 & YES & YES & YES & $1.75$ & $(4,2)$ & 11033 & 8182\\
$(207,79)$ & 11 & $(186,71)$ & 11 & 3 & YES & YES & YES & $1.71$ & $(2,3)$ & NO & 8183\\
$(207,64)$ & 13 & $(207,64)$ & 13 & 207 & YES & YES & YES & $1.71$ & $(2,3)$ & NO & 8184\\
$(207,79)$ & 11 & $(207,79)$ & 11 & 207 & YES & YES & YES & $1.50$ & $(2,3)$ & NO & 8185\\
$(208,79)$ & 11 & $(2,1)$ & 1 & 2 & YES & YES & YES & $1.60$ & $(2,3)$ & -- & 8186\\
$(208,45)$ & 13 & $(3,1)$ & 2 & 1 & YES & YES & YES & $1.71$ & $(2,3)$ & NO & 8187\\
$(208,61)$ & 12 & $(3,1)$ & 2 & 1 & NO & YES & NO(2) & $1.56$ & $(4,2)$ & -- & 8188\\
$(208,79)$ & 11 & $(3,1)$ & 2 & 1 & YES & YES & NO(2) & $1.56$ & $(2,3)$ & -- & 8189\\
$(208,95)$ & 13 & $(3,1)$ & 2 & 1 & YES & YES & NO(2) & $1.50$ & $(6,1)$ & NO & 8190\\
$(208,61)$ & 12 & $(4,1)$ & 3 & 4 & YES & YES & YES & $1.71$ & $(6,1)$ & NO & 8191\\
$(208,61)$ & 12 & $(4,1)$ & 3 & 4 & YES & YES & YES & $1.43$ & $(2,3)$ & -- & 8192\\
$(208,79)$ & 11 & $(4,1)$ & 3 & 4 & YES & YES & NO(2) & $1.56$ & $(2,3)$ & -- & 8193\\
$(208,79)$ & 11 & $(4,1)$ & 3 & 4 & YES & YES & YES & $1.62$ & $(4,2)$ & NO & 8194\\
$(208,61)$ & 12 & $(5,2)$ & 3 & 1 & YES & YES & YES & $1.71$ & $(2,3)$ & -- & 8195\\
$(208,79)$ & 11 & $(5,1)$ & 4 & 1 & YES & YES & YES & $1.38$ & $(2,3)$ & NO & 8196\\
$(208,79)$ & 11 & $(5,2)$ & 3 & 1 & YES & YES & YES & $1.57$ & $(4,2)$ & -- & 8197\\
$(208,79)$ & 11 & $(5,2)$ & 3 & 1 & YES & YES & YES & $1.75$ & $(2,3)$ & NO & 8198\\
$(208,45)$ & 13 & $(6,1)$ & 5 & 2 & YES & YES & YES & $1.43$ & $(2,3)$ & NO & 8199\\
$(208,61)$ & 12 & $(7,3)$ & 4 & 1 & YES & YES & YES & $1.78$ & $(4,2)$ & -- & 8200\\
$(208,79)$ & 11 & $(7,2)$ & 4 & 1 & YES & YES & YES & $1.57$ & $(2,3)$ & -- & 8201\\
$(208,79)$ & 11 & $(7,3)$ & 4 & 1 & YES & YES & YES & $1.78$ & $(2,3)$ & NO & 8202\\
$(208,79)$ & 11 & $(8,3)$ & 4 & 8 & YES & YES & NO(2) & $1.56$ & $(8,0)$ & NO & 8203\\
$(208,55)$ & 12 & $(9,2)$ & 5 & 1 & YES & YES & NO(2) & $1.44$ & $(8,0)$ & NO & 8204\\
$(208,61)$ & 12 & $(13,4)$ & 6 & 13 & YES & YES & YES & $1.62$ & $(2,3)$ & NO & 8205\\
$(208,79)$ & 11 & $(13,5)$ & 5 & 13 & YES & YES & YES & $1.62$ & $(2,3)$ & 8779 & 8206\\
$(208,61)$ & 12 & $(18,5)$ & 6 & 2 & YES & YES & YES & $1.43$ & $(6,1)$ & NO & 8207\\
$(208,79)$ & 11 & $(21,8)$ & 6 & 1 & YES & YES & NO(2) & $1.60$ & $(6,1)$ & 7827 & 8208\\
$(208,45)$ & 13 & $(22,5)$ & 7 & 2 & YES & YES & YES & $1.29$ & $(6,1)$ & NO & 8209\\
$(208,61)$ & 12 & $(24,7)$ & 7 & 8 & YES & YES & YES & $1.43$ & $(2,3)$ & NO & 8210\\
$(208,79)$ & 11 & $(34,13)$ & 7 & 2 & YES & YES & YES & $1.57$ & $(2,3)$ & NO & 8211\\
$(208,79)$ & 11 & $(37,14)$ & 8 & 1 & YES & YES & YES & $1.67$ & $(2,3)$ & NO & 8212\\
$(208,37)$ & 13 & $(39,7)$ & 9 & 13 & YES & YES & NO(2) & $1.60$ & $(2,3)$ & NO & 8213\\
$(208,79)$ & 11 & $(50,19)$ & 8 & 2 & YES & YES & YES & $1.80$ & $(2,3)$ & 7478 & 8214\\
$(208,79)$ & 11 & $(71,27)$ & 9 & 1 & YES & YES & YES & $1.62$ & $(4,2)$ & NO & 8215\\
$(208,79)$ & 11 & $(79,30)$ & 9 & 1 & YES & YES & YES & $1.80$ & $(2,3)$ & NO & 8216\\
$(208,61)$ & 12 & $(99,29)$ & 10 & 1 & YES & YES & YES & $1.89$ & $(2,3)$ & NO & 8217\\
$(208,79)$ & 11 & $(108,41)$ & 10 & 4 & YES & YES & YES & $1.62$ & $(4,2)$ & NO & 8218\\
$(208,95)$ & 13 & $(127,58)$ & 12 & 1 & YES & YES & NO(2) & $1.50$ & $(6,1)$ & NO & 8219\\
$(208,79)$ & 11 & $(129,49)$ & 10 & 1 & YES & YES & YES & $1.67$ & $(2,3)$ & NO & 8220\\
$(208,45)$ & 13 & $(171,37)$ & 12 & 1 & YES & YES & YES & $1.43$ & $(2,3)$ & NO & 8221\\
$(208,75)$ & 12 & $(208,75)$ & 12 & 208 & YES & YES & NO(2) & $1.56$ & $(8,0)$ & NO & 8222\\
$(208,79)$ & 11 & $(208,79)$ & 11 & 208 & YES & YES & YES & $1.50$ & $(2,3)$ & NO & 8223\\
$(209,80)$ & 11 & $(2,1)$ & 1 & 1 & YES & YES & YES & $1.43$ & $(2,3)$ & NO & 8224\\
$(209,81)$ & 11 & $(2,1)$ & 1 & 1 & YES & YES & YES & $1.29$ & $(2,3)$ & -- & 8225\\
$(209,50)$ & 14 & $(3,1)$ & 2 & 1 & YES & YES & NO(2) & $1.56$ & $(8,0)$ & -- & 8226\\
$(209,50)$ & 14 & $(3,1)$ & 2 & 1 & YES & YES & NO(2) & $1.67$ & $(8,0)$ & NO & 8227\\
$(209,80)$ & 11 & $(3,1)$ & 2 & 1 & YES & YES & NO(2) & $1.38$ & $(6,1)$ & NO & 8228\\
$(209,80)$ & 11 & $(3,1)$ & 2 & 1 & YES & YES & YES & $1.50$ & $(4,2)$ & -- & 8229\\
$(209,81)$ & 11 & $(3,1)$ & 2 & 1 & YES & YES & NO(2) & $1.44$ & $(2,3)$ & -- & 8230\\
$(209,81)$ & 11 & $(3,1)$ & 2 & 1 & YES & YES & YES & $1.62$ & $(2,3)$ & NO & 8231\\
$(209,50)$ & 14 & $(4,1)$ & 3 & 1 & YES & YES & NO(2) & $1.56$ & $(4,2)$ & -- & 8232\\
$(209,80)$ & 11 & $(4,1)$ & 3 & 1 & YES & YES & YES & $1.62$ & $(2,3)$ & NO & 8233\\
$(209,80)$ & 11 & $(4,1)$ & 3 & 1 & YES & YES & YES & $1.67$ & $(2,3)$ & -- & 8234\\
$(209,81)$ & 11 & $(4,1)$ & 3 & 1 & YES & YES & YES & $1.38$ & $(2,3)$ & NO & 8235\\
$(209,81)$ & 11 & $(4,1)$ & 3 & 1 & YES & YES & YES & $1.50$ & $(2,3)$ & -- & 8236\\
$(209,56)$ & 12 & $(5,2)$ & 3 & 1 & YES & YES & YES & $1.90$ & $(2,3)$ & NO & 8237\\
$(209,62)$ & 12 & $(5,2)$ & 3 & 1 & YES & YES & YES & $1.67$ & $(2,3)$ & -- & 8238\\
$(209,80)$ & 11 & $(5,2)$ & 3 & 1 & YES & YES & YES & $1.62$ & $(2,3)$ & -- & 8239\\
$(209,81)$ & 11 & $(5,1)$ & 4 & 1 & YES & YES & NO(2) & $1.67$ & $(2,3)$ & NO & 8240\\
$(209,81)$ & 11 & $(5,2)$ & 3 & 1 & YES & YES & YES & $1.57$ & $(4,2)$ & -- & 8241\\
$(209,91)$ & 12 & $(5,2)$ & 3 & 1 & YES & YES & YES & $1.43$ & $(4,2)$ & NO & 8242\\
$(209,81)$ & 11 & $(8,3)$ & 4 & 1 & YES & YES & YES & $1.50$ & $(2,3)$ & NO & 8243\\
$(209,81)$ & 11 & $(9,2)$ & 5 & 1 & YES & YES & YES & $1.71$ & $(2,3)$ & NO & 8244\\
$(209,81)$ & 11 & $(9,2)$ & 5 & 1 & YES & YES & YES & $1.71$ & $(2,3)$ & -- & 8245\\
$(209,91)$ & 12 & $(9,4)$ & 5 & 1 & YES & YES & YES & $1.57$ & $(2,3)$ & NO & 8246\\
$(209,50)$ & 14 & $(13,3)$ & 6 & 1 & YES & YES & NO(2) & $1.67$ & $(4,2)$ & NO & 8247\\
$(209,64)$ & 13 & $(13,4)$ & 6 & 1 & YES & YES & NO(2) & $1.60$ & $(10,-1)$ & NO & 8248\\
$(209,80)$ & 11 & $(13,5)$ & 5 & 1 & YES & YES & YES & $1.43$ & $(2,3)$ & 7127 & 8249\\
$(209,81)$ & 11 & $(13,5)$ & 5 & 1 & YES & YES & NO(2) & $1.67$ & $(2,3)$ & NO & 8250\\
$(209,56)$ & 12 & $(18,5)$ & 6 & 1 & YES & YES & YES & $1.78$ & $(4,2)$ & NO & 8251\\
$(209,80)$ & 11 & $(18,7)$ & 6 & 1 & YES & YES & YES & $1.90$ & $(2,3)$ & NO & 8252\\
$(209,81)$ & 11 & $(18,7)$ & 6 & 1 & YES & YES & YES & $1.43$ & $(2,3)$ & 7598 & 8253\\
$(209,80)$ & 11 & $(21,8)$ & 6 & 1 & YES & YES & YES & $1.80$ & $(2,3)$ & NO & 8254\\
$(209,81)$ & 11 & $(21,8)$ & 6 & 1 & YES & YES & YES & $1.57$ & $(2,3)$ & NO & 8255\\
$(209,81)$ & 11 & $(23,9)$ & 7 & 1 & YES & YES & YES & $1.78$ & $(4,2)$ & NO & 8256\\
$(209,80)$ & 11 & $(34,13)$ & 7 & 1 & YES & YES & YES & $1.67$ & $(2,3)$ & NO & 8257\\
$(209,81)$ & 11 & $(44,17)$ & 8 & 11 & YES & YES & YES & $1.75$ & $(2,3)$ & NO & 8258\\
$(209,80)$ & 11 & $(47,18)$ & 8 & 1 & YES & YES & NO(2) & $1.56$ & $(2,3)$ & 7340 & 8259\\
$(209,81)$ & 11 & $(49,19)$ & 8 & 1 & YES & YES & NO(2) & $1.56$ & $(2,3)$ & 7417 & 8260\\
$(209,80)$ & 11 & $(55,21)$ & 8 & 11 & YES & YES & YES & $1.86$ & $(2,3)$ & NO & 8261\\
$(209,56)$ & 12 & $(67,18)$ & 9 & 1 & YES & YES & YES & $1.57$ & $(4,2)$ & 11103 & 8262\\
$(209,81)$ & 11 & $(67,26)$ & 9 & 1 & YES & YES & YES & $1.78$ & $(4,2)$ & NO & 8263\\
$(209,81)$ & 11 & $(80,31)$ & 9 & 1 & YES & YES & YES & $1.50$ & $(2,3)$ & NO & 8264\\
$(209,62)$ & 12 & $(101,30)$ & 10 & 1 & YES & YES & YES & $1.62$ & $(2,3)$ & NO & 8265\\
$(209,80)$ & 11 & $(115,44)$ & 10 & 1 & YES & YES & YES & $1.50$ & $(2,3)$ & NO & 8266\\
$(209,81)$ & 11 & $(129,50)$ & 10 & 1 & YES & YES & YES & $1.78$ & $(2,3)$ & NO & 8267\\
$(209,62)$ & 12 & $(145,43)$ & 12 & 1 & YES & YES & YES & $1.43$ & $(4,2)$ & NO & 8268\\
$(209,80)$ & 11 & $(209,80)$ & 11 & 209 & YES & YES & YES & $1.57$ & $(2,3)$ & NO & 8269\\
$(209,81)$ & 11 & $(209,81)$ & 11 & 209 & YES & YES & YES & $1.89$ & $(2,3)$ & NO & 8270\\
$(211,56)$ & 13 & $(3,1)$ & 2 & 1 & YES & YES & YES & $1.57$ & $(2,3)$ & -- & 8271\\
$(211,89)$ & 12 & $(3,1)$ & 2 & 1 & YES & YES & YES & $1.71$ & $(2,3)$ & -- & 8272\\
$(211,93)$ & 12 & $(3,1)$ & 2 & 1 & YES & YES & NO(2) & $1.60$ & $(10,-1)$ & -- & 8273\\
$(211,32)$ & 15 & $(4,1)$ & 3 & 1 & YES & YES & NO(2) & $1.50$ & $(6,1)$ & -- & 8274\\
$(211,56)$ & 13 & $(5,2)$ & 3 & 1 & YES & YES & NO(2) & $1.67$ & $(4,2)$ & NO & 8275\\
$(211,64)$ & 12 & $(5,2)$ & 3 & 1 & YES & YES & NO(2) & $1.60$ & $(6,1)$ & -- & 8276\\
$(211,64)$ & 12 & $(5,2)$ & 3 & 1 & YES & YES & YES & $1.62$ & $(4,2)$ & NO & 8277\\
$(211,64)$ & 12 & $(7,3)$ & 4 & 1 & YES & YES & YES & $1.62$ & $(4,2)$ & NO & 8278\\
$(211,89)$ & 12 & $(45,19)$ & 8 & 1 & YES & YES & YES & $1.57$ & $(4,2)$ & NO & 8279\\
$(211,64)$ & 12 & $(79,24)$ & 10 & 1 & YES & YES & YES & $1.62$ & $(4,2)$ & NO & 8280\\
$(211,93)$ & 12 & $(93,41)$ & 10 & 1 & YES & YES & NO(2) & $1.60$ & $(10,-1)$ & 9078 & 8281\\
$(211,64)$ & 12 & $(155,47)$ & 12 & 1 & YES & YES & NO(2) & $1.56$ & $(4,2)$ & NO & 8282\\
$(211,89)$ & 12 & $(211,89)$ & 12 & 211 & YES & YES & YES & $1.43$ & $(4,2)$ & NO & 8283\\
$(212,59)$ & 13 & $(2,1)$ & 1 & 2 & YES & YES & NO(2) & $1.44$ & $(4,2)$ & -- & 8284\\
$(212,81)$ & 11 & $(2,1)$ & 1 & 2 & YES & YES & NO(2) & $1.78$ & $(2,3)$ & NO & 8285\\
$(212,81)$ & 11 & $(2,1)$ & 1 & 2 & YES & YES & YES & $1.29$ & $(2,3)$ & -- & 8286\\
$(212,89)$ & 11 & $(2,1)$ & 1 & 2 & NO & YES & NO(2) & $1.50$ & $(2,3)$ & -- & 8287\\
$(212,81)$ & 11 & $(3,1)$ & 2 & 1 & YES & YES & YES & $1.43$ & $(4,2)$ & NO & 8288\\
$(212,81)$ & 11 & $(3,1)$ & 2 & 1 & YES & YES & YES & $1.67$ & $(2,3)$ & -- & 8289\\
$(212,89)$ & 11 & $(3,1)$ & 2 & 1 & YES & YES & NO(2) & $1.25$ & $(6,1)$ & -- & 8290\\
$(212,89)$ & 11 & $(3,1)$ & 2 & 1 & YES & YES & YES & $1.62$ & $(2,3)$ & NO & 8291\\
$(212,89)$ & 11 & $(3,1)$ & 2 & 1 & YES & YES & YES & $1.62$ & $(2,3)$ & NO & 8292\\
$(212,59)$ & 13 & $(4,1)$ & 3 & 4 & YES & YES & NO(2) & $1.44$ & $(4,2)$ & NO & 8293\\
$(212,81)$ & 11 & $(4,1)$ & 3 & 4 & YES & YES & YES & $1.38$ & $(2,3)$ & -- & 8294\\
$(212,81)$ & 11 & $(4,1)$ & 3 & 4 & YES & YES & YES & $1.50$ & $(2,3)$ & NO & 8295\\
$(212,89)$ & 11 & $(4,1)$ & 3 & 4 & YES & YES & YES & $1.29$ & $(6,1)$ & NO & 8296\\
$(212,89)$ & 11 & $(4,1)$ & 3 & 4 & YES & YES & YES & $1.43$ & $(6,1)$ & -- & 8297\\
$(212,89)$ & 11 & $(4,1)$ & 3 & 4 & YES & YES & YES & $1.75$ & $(4,2)$ & NO & 8298\\
$(212,89)$ & 11 & $(5,1)$ & 4 & 1 & YES & YES & YES & $1.50$ & $(4,2)$ & 6255 & 8299\\
$(212,89)$ & 11 & $(5,2)$ & 3 & 1 & YES & YES & YES & $1.70$ & $(2,3)$ & -- & 8300\\
$(212,81)$ & 11 & $(7,3)$ & 4 & 1 & YES & YES & YES & $1.57$ & $(4,2)$ & NO & 8301\\
$(212,89)$ & 11 & $(7,3)$ & 4 & 1 & YES & YES & NO(2) & $1.44$ & $(8,0)$ & 7679 & 8302\\
$(212,81)$ & 11 & $(9,2)$ & 5 & 1 & YES & YES & YES & $1.57$ & $(2,3)$ & NO & 8303\\
$(212,89)$ & 11 & $(12,5)$ & 5 & 4 & YES & YES & YES & $1.43$ & $(6,1)$ & NO & 8304\\
$(212,81)$ & 11 & $(13,5)$ & 5 & 1 & YES & YES & NO(2) & $1.78$ & $(2,3)$ & NO & 8305\\
$(212,81)$ & 11 & $(18,7)$ & 6 & 2 & YES & YES & YES & $1.62$ & $(4,2)$ & NO & 8306\\
$(212,81)$ & 11 & $(21,8)$ & 6 & 1 & YES & YES & NO(2) & $1.38$ & $(4,2)$ & 6093 & 8307\\
$(212,89)$ & 11 & $(43,18)$ & 8 & 1 & YES & YES & YES & $1.89$ & $(2,3)$ & NO & 8308\\
$(212,81)$ & 11 & $(47,18)$ & 8 & 1 & YES & YES & YES & $1.57$ & $(2,3)$ & NO & 8309\\
$(212,89)$ & 11 & $(50,21)$ & 8 & 2 & YES & YES & YES & $1.62$ & $(4,2)$ & 7553 & 8310\\
$(212,81)$ & 11 & $(55,21)$ & 8 & 1 & YES & YES & YES & $1.78$ & $(2,3)$ & NO & 8311\\
$(212,89)$ & 11 & $(69,29)$ & 9 & 1 & YES & YES & YES & $1.57$ & $(2,3)$ & NO & 8312\\
$(212,89)$ & 11 & $(81,34)$ & 9 & 1 & YES & YES & YES & $1.62$ & $(2,3)$ & NO & 8313\\
$(212,81)$ & 11 & $(89,34)$ & 9 & 1 & YES & YES & YES & $1.67$ & $(2,3)$ & NO & 8314\\
$(212,89)$ & 11 & $(112,47)$ & 10 & 4 & YES & YES & YES & $1.56$ & $(2,3)$ & NO & 8315\\
$(212,81)$ & 11 & $(123,47)$ & 10 & 1 & YES & YES & YES & $1.75$ & $(2,3)$ & NO & 8316\\
$(212,89)$ & 11 & $(131,55)$ & 10 & 1 & YES & YES & YES & $1.29$ & $(6,1)$ & NO & 8317\\
$(212,81)$ & 11 & $(212,81)$ & 11 & 212 & YES & YES & YES & $1.50$ & $(2,3)$ & NO & 8318\\
$(212,89)$ & 11 & $(212,89)$ & 11 & 212 & YES & YES & YES & $1.29$ & $(6,1)$ & NO & 8319\\
$(213,59)$ & 12 & $(2,1)$ & 1 & 1 & YES & YES & NO(2) & $1.60$ & $(2,3)$ & NO & 8320\\
$(213,65)$ & 12 & $(2,1)$ & 1 & 1 & YES & YES & NO(2) & $1.56$ & $(4,2)$ & NO & 8321\\
$(213,88)$ & 12 & $(2,1)$ & 1 & 1 & YES & YES & NO(2) & $1.50$ & $(4,2)$ & -- & 8322\\
$(213,88)$ & 12 & $(2,1)$ & 1 & 1 & YES & YES & NO(2) & $1.56$ & $(8,0)$ & NO & 8323\\
$(213,77)$ & 12 & $(3,1)$ & 2 & 3 & YES & YES & YES & $1.43$ & $(4,2)$ & NO & 8324\\
$(213,83)$ & 12 & $(3,1)$ & 2 & 3 & YES & YES & NO(2) & $1.50$ & $(4,2)$ & -- & 8325\\
$(213,62)$ & 12 & $(4,1)$ & 3 & 1 & YES & YES & YES & $1.71$ & $(6,1)$ & NO & 8326\\
$(213,62)$ & 12 & $(4,1)$ & 3 & 1 & YES & YES & YES & $1.56$ & $(2,3)$ & NO & 8327\\
$(213,59)$ & 12 & $(5,1)$ & 4 & 1 & YES & YES & NO(2) & $1.44$ & $(8,0)$ & NO & 8328\\
$(213,59)$ & 12 & $(5,2)$ & 3 & 1 & YES & YES & YES & $1.43$ & $(4,2)$ & -- & 8329\\
$(213,62)$ & 12 & $(5,2)$ & 3 & 1 & YES & YES & YES & $1.43$ & $(4,2)$ & -- & 8330\\
$(213,65)$ & 12 & $(5,1)$ & 4 & 1 & YES & YES & YES & $1.43$ & $(6,1)$ & NO & 8331\\
$(213,65)$ & 12 & $(5,2)$ & 3 & 1 & YES & YES & YES & $1.62$ & $(2,3)$ & -- & 8332\\
$(213,79)$ & 12 & $(5,1)$ & 4 & 1 & YES & YES & NO(2) & $1.44$ & $(8,0)$ & -- & 8333\\
$(213,83)$ & 12 & $(5,1)$ & 4 & 1 & YES & YES & NO(2) & $1.50$ & $(4,2)$ & NO & 8334\\
$(213,83)$ & 12 & $(5,1)$ & 4 & 1 & YES & YES & NO(2) & $1.50$ & $(4,2)$ & NO & 8335\\
$(213,88)$ & 12 & $(5,2)$ & 3 & 1 & YES & YES & NO(2) & $1.56$ & $(8,0)$ & NO & 8336\\
$(213,59)$ & 12 & $(7,2)$ & 4 & 1 & YES & YES & NO(2) & $1.60$ & $(2,3)$ & NO & 8337\\
$(213,62)$ & 12 & $(7,2)$ & 4 & 1 & YES & YES & YES & $1.62$ & $(2,3)$ & -- & 8338\\
$(213,59)$ & 12 & $(10,3)$ & 5 & 1 & YES & YES & YES & $1.62$ & $(4,2)$ & NO & 8339\\
$(213,62)$ & 12 & $(11,3)$ & 5 & 1 & YES & YES & YES & $1.71$ & $(4,2)$ & NO & 8340\\
$(213,88)$ & 12 & $(12,5)$ & 5 & 3 & YES & YES & NO(2) & $1.38$ & $(6,1)$ & NO & 8341\\
$(213,62)$ & 12 & $(13,4)$ & 6 & 1 & YES & YES & YES & $2.00$ & $(2,3)$ & NO & 8342\\
$(213,65)$ & 12 & $(13,4)$ & 6 & 1 & YES & YES & NO(2) & $1.56$ & $(4,2)$ & NO & 8343\\
$(213,59)$ & 12 & $(15,4)$ & 6 & 3 & YES & YES & YES & $1.57$ & $(4,2)$ & NO & 8344\\
$(213,88)$ & 12 & $(17,7)$ & 6 & 1 & YES & YES & NO(2) & $1.50$ & $(6,1)$ & NO & 8345\\
$(213,59)$ & 12 & $(18,5)$ & 6 & 3 & YES & YES & NO(2) & $1.70$ & $(2,3)$ & NO & 8346\\
$(213,88)$ & 12 & $(29,12)$ & 7 & 1 & YES & YES & NO(2) & $1.38$ & $(4,2)$ & NO & 8347\\
$(213,59)$ & 12 & $(40,11)$ & 8 & 1 & YES & YES & YES & $1.67$ & $(2,3)$ & NO & 8348\\
$(213,83)$ & 12 & $(41,16)$ & 8 & 1 & YES & YES & NO(2) & $1.50$ & $(4,2)$ & 9443 & 8349\\
$(213,88)$ & 12 & $(46,19)$ & 8 & 1 & YES & YES & NO(2) & $1.56$ & $(8,0)$ & NO & 8350\\
$(213,83)$ & 12 & $(77,30)$ & 10 & 1 & YES & YES & NO(2) & $1.50$ & $(4,2)$ & NO & 8351\\
$(213,62)$ & 12 & $(86,25)$ & 10 & 1 & YES & YES & YES & $1.78$ & $(2,3)$ & NO & 8352\\
$(213,79)$ & 12 & $(89,33)$ & 10 & 1 & YES & YES & NO(2) & $1.56$ & $(8,0)$ & 8941 & 8353\\
$(213,65)$ & 12 & $(131,40)$ & 11 & 1 & YES & YES & YES & $1.62$ & $(2,3)$ & 11138 & 8354\\
$(213,83)$ & 12 & $(195,76)$ & 12 & 3 & YES & YES & YES & $1.62$ & $(4,2)$ & NO & 8355\\
$(213,65)$ & 12 & $(213,65)$ & 12 & 213 & YES & YES & YES & $1.43$ & $(6,1)$ & NO & 8356\\
$(213,77)$ & 12 & $(213,77)$ & 12 & 213 & YES & YES & YES & $1.57$ & $(2,3)$ & NO & 8357\\
$(213,79)$ & 12 & $(213,79)$ & 12 & 213 & YES & YES & NO(2) & $1.38$ & $(6,1)$ & NO & 8358\\
$(214,79)$ & 12 & $(2,1)$ & 1 & 2 & YES & YES & NO(2) & $1.50$ & $(4,2)$ & NO & 8359\\
$(214,79)$ & 12 & $(2,1)$ & 1 & 2 & YES & YES & NO(2) & $1.50$ & $(4,2)$ & -- & 8360\\
$(214,65)$ & 12 & $(3,1)$ & 2 & 1 & NO & YES & NO(2) & $1.80$ & $(2,3)$ & -- & 8361\\
$(214,65)$ & 12 & $(3,1)$ & 2 & 1 & YES & YES & YES & $1.57$ & $(4,2)$ & NO & 8362\\
$(214,65)$ & 12 & $(4,1)$ & 3 & 2 & YES & YES & NO(2) & $1.56$ & $(2,3)$ & NO & 8363\\
$(214,65)$ & 12 & $(11,3)$ & 5 & 1 & YES & YES & YES & $1.80$ & $(2,3)$ & NO & 8364\\
$(214,79)$ & 12 & $(11,4)$ & 5 & 1 & YES & YES & NO(2) & $1.50$ & $(4,2)$ & NO & 8365\\
$(214,65)$ & 12 & $(13,4)$ & 6 & 1 & YES & YES & NO(2) & $1.67$ & $(2,3)$ & NO & 8366\\
$(214,65)$ & 12 & $(33,10)$ & 8 & 1 & YES & YES & YES & $1.86$ & $(6,1)$ & 8804 & 8367\\
$(214,65)$ & 12 & $(56,17)$ & 9 & 2 & YES & YES & NO(2) & $1.56$ & $(2,3)$ & 7796 & 8368\\
$(214,65)$ & 12 & $(89,27)$ & 10 & 1 & YES & YES & YES & $1.80$ & $(2,3)$ & NO & 8369\\
$(214,83)$ & 12 & $(116,45)$ & 10 & 2 & YES & YES & YES & $1.57$ & $(4,2)$ & 9888 & 8370\\
$(215,63)$ & 12 & $(2,1)$ & 1 & 1 & YES & YES & YES & $1.56$ & $(2,3)$ & -- & 8371\\
$(215,83)$ & 12 & $(2,1)$ & 1 & 1 & YES & YES & NO(2) & $1.67$ & $(8,0)$ & NO & 8372\\
$(215,84)$ & 12 & $(2,1)$ & 1 & 1 & YES & YES & NO(2) & $1.56$ & $(8,0)$ & -- & 8373\\
$(215,63)$ & 12 & $(3,1)$ & 2 & 1 & YES & YES & YES & $1.50$ & $(2,3)$ & NO & 8374\\
$(215,63)$ & 12 & $(3,1)$ & 2 & 1 & YES & YES & YES & $1.50$ & $(2,3)$ & -- & 8375\\
$(215,84)$ & 12 & $(3,1)$ & 2 & 1 & YES & YES & NO(2) & $1.67$ & $(8,0)$ & NO & 8376\\
$(215,83)$ & 12 & $(4,1)$ & 3 & 1 & YES & YES & YES & $1.62$ & $(4,2)$ & -- & 8377\\
$(215,83)$ & 12 & $(4,1)$ & 3 & 1 & YES & YES & YES & $1.78$ & $(4,2)$ & NO & 8378\\
$(215,58)$ & 12 & $(5,2)$ & 3 & 5 & YES & YES & YES & $1.62$ & $(2,3)$ & -- & 8379\\
$(215,59)$ & 13 & $(5,2)$ & 3 & 5 & YES & YES & YES & $1.57$ & $(6,1)$ & NO & 8380\\
$(215,63)$ & 12 & $(5,2)$ & 3 & 5 & YES & YES & YES & $1.62$ & $(4,2)$ & -- & 8381\\
$(215,83)$ & 12 & $(6,1)$ & 5 & 1 & YES & YES & NO(2) & $1.56$ & $(8,0)$ & NO & 8382\\
$(215,83)$ & 12 & $(6,1)$ & 5 & 1 & YES & YES & NO(2) & $1.56$ & $(8,0)$ & -- & 8383\\
$(215,63)$ & 12 & $(7,2)$ & 4 & 1 & YES & YES & YES & $1.14$ & $(4,2)$ & NO & 8384\\
$(215,63)$ & 12 & $(11,2)$ & 6 & 1 & YES & YES & YES & $1.43$ & $(4,2)$ & NO & 8385\\
$(215,83)$ & 12 & $(13,5)$ & 5 & 1 & YES & YES & NO(2) & $1.67$ & $(8,0)$ & NO & 8386\\
$(215,63)$ & 12 & $(17,5)$ & 6 & 1 & YES & YES & NO(2) & $1.56$ & $(2,3)$ & NO & 8387\\
$(215,84)$ & 12 & $(18,7)$ & 6 & 1 & YES & YES & NO(2) & $1.44$ & $(8,0)$ & NO & 8388\\
$(215,58)$ & 12 & $(19,5)$ & 7 & 1 & YES & YES & YES & $1.62$ & $(4,2)$ & NO & 8389\\
$(215,84)$ & 12 & $(23,9)$ & 7 & 1 & YES & YES & NO(2) & $1.67$ & $(8,0)$ & NO & 8390\\
$(215,63)$ & 12 & $(24,7)$ & 7 & 1 & YES & YES & YES & $1.14$ & $(4,2)$ & NO & 8391\\
$(215,59)$ & 13 & $(25,7)$ & 7 & 5 & YES & YES & YES & $1.57$ & $(6,1)$ & 5530 & 8392\\
$(215,63)$ & 12 & $(31,9)$ & 8 & 1 & YES & YES & YES & $1.80$ & $(2,3)$ & NO & 8393\\
$(215,83)$ & 12 & $(31,12)$ & 7 & 1 & YES & YES & YES & $1.62$ & $(4,2)$ & NO & 8394\\
$(215,82)$ & 12 & $(34,13)$ & 7 & 1 & YES & YES & NO(2) & $1.44$ & $(4,2)$ & NO & 8395\\
$(215,59)$ & 13 & $(40,11)$ & 8 & 5 & YES & YES & NO(2) & $1.56$ & $(4,2)$ & NO & 8396\\
$(215,83)$ & 12 & $(57,22)$ & 9 & 1 & YES & YES & NO(2) & $1.67$ & $(8,0)$ & NO & 8397\\
$(215,63)$ & 12 & $(75,22)$ & 10 & 5 & YES & YES & YES & $1.38$ & $(4,2)$ & NO & 8398\\
$(215,82)$ & 12 & $(76,29)$ & 9 & 1 & YES & YES & YES & $1.57$ & $(2,3)$ & NO & 8399\\
$(215,51)$ & 13 & $(93,22)$ & 11 & 1 & YES & YES & YES & $1.67$ & $(4,2)$ & NO & 8400\\
$(215,63)$ & 12 & $(99,29)$ & 10 & 1 & YES & YES & YES & $1.90$ & $(2,3)$ & 9361 & 8401\\
$(215,58)$ & 12 & $(100,27)$ & 10 & 5 & YES & YES & YES & $1.50$ & $(4,2)$ & NO & 8402\\
$(215,83)$ & 12 & $(101,39)$ & 10 & 1 & YES & YES & YES & $1.62$ & $(4,2)$ & 9447 & 8403\\
$(215,63)$ & 12 & $(140,41)$ & 11 & 5 & YES & YES & YES & $1.78$ & $(2,3)$ & 11164 & 8404\\
$(215,63)$ & 12 & $(215,63)$ & 12 & 215 & YES & YES & YES & $1.78$ & $(4,2)$ & NO & 8405\\
$(215,82)$ & 12 & $(215,82)$ & 12 & 215 & YES & YES & YES & $1.71$ & $(4,2)$ & NO & 8406\\
$(217,60)$ & 12 & $(3,1)$ & 2 & 1 & YES & YES & YES & $1.50$ & $(2,3)$ & -- & 8407\\
$(217,64)$ & 12 & $(3,1)$ & 2 & 1 & NO & YES & NO(2) & $1.82$ & $(4,2)$ & -- & 8408\\
$(217,64)$ & 12 & $(4,1)$ & 3 & 1 & YES & YES & YES & $1.43$ & $(6,1)$ & -- & 8409\\
$(217,47)$ & 12 & $(5,2)$ & 3 & 1 & YES & YES & YES & $1.67$ & $(2,3)$ & NO & 8410\\
$(217,47)$ & 12 & $(5,2)$ & 3 & 1 & YES & YES & YES & $1.50$ & $(2,3)$ & -- & 8411\\
$(217,60)$ & 12 & $(5,2)$ & 3 & 1 & YES & YES & YES & $1.38$ & $(4,2)$ & -- & 8412\\
$(217,60)$ & 12 & $(5,2)$ & 3 & 1 & YES & YES & YES & $1.67$ & $(2,3)$ & NO & 8413\\
$(217,64)$ & 12 & $(5,2)$ & 3 & 1 & YES & YES & YES & $1.75$ & $(2,3)$ & -- & 8414\\
$(217,64)$ & 12 & $(5,2)$ & 3 & 1 & YES & YES & YES & $1.75$ & $(2,3)$ & NO & 8415\\
$(217,85)$ & 13 & $(5,1)$ & 4 & 1 & YES & YES & NO(2) & $1.56$ & $(8,0)$ & NO & 8416\\
$(217,85)$ & 13 & $(5,1)$ & 4 & 1 & YES & YES & NO(2) & $1.56$ & $(8,0)$ & -- & 8417\\
$(217,60)$ & 12 & $(6,1)$ & 5 & 1 & YES & YES & YES & $1.14$ & $(6,1)$ & NO & 8418\\
$(217,47)$ & 12 & $(7,2)$ & 4 & 7 & YES & YES & YES & $1.78$ & $(2,3)$ & NO & 8419\\
$(217,60)$ & 12 & $(10,3)$ & 5 & 1 & YES & YES & YES & $1.67$ & $(2,3)$ & NO & 8420\\
$(217,47)$ & 12 & $(13,3)$ & 6 & 1 & YES & YES & NO(2) & $1.56$ & $(2,3)$ & NO & 8421\\
$(217,64)$ & 12 & $(13,4)$ & 6 & 1 & YES & YES & YES & $1.75$ & $(2,3)$ & NO & 8422\\
$(217,85)$ & 13 & $(23,9)$ & 7 & 1 & YES & YES & NO(2) & $1.78$ & $(8,0)$ & 6508 & 8423\\
$(217,47)$ & 12 & $(32,7)$ & 8 & 1 & YES & YES & YES & $1.14$ & $(4,2)$ & NO & 8424\\
$(217,60)$ & 12 & $(40,11)$ & 8 & 1 & YES & YES & YES & $1.80$ & $(2,3)$ & NO & 8425\\
$(217,47)$ & 12 & $(51,11)$ & 9 & 1 & YES & YES & NO(2) & $1.56$ & $(2,3)$ & 10329 & 8426\\
$(217,60)$ & 12 & $(76,21)$ & 9 & 1 & YES & YES & YES & $1.50$ & $(4,2)$ & NO & 8427\\
$(217,46)$ & 14 & $(85,18)$ & 10 & 1 & YES & YES & NO(2) & $1.67$ & $(4,2)$ & NO & 8428\\
$(217,64)$ & 12 & $(139,41)$ & 11 & 1 & YES & YES & YES & $1.43$ & $(6,1)$ & NO & 8429\\
$(217,60)$ & 12 & $(217,60)$ & 12 & 217 & YES & YES & YES & $1.14$ & $(6,1)$ & NO & 8430\\
$(217,64)$ & 12 & $(217,64)$ & 12 & 217 & YES & YES & YES & $1.43$ & $(6,1)$ & NO & 8431\\
$(218,85)$ & 12 & $(3,1)$ & 2 & 1 & YES & YES & NO(2) & $1.60$ & $(6,1)$ & -- & 8432\\
$(218,51)$ & 13 & $(5,2)$ & 3 & 1 & YES & YES & YES & $1.75$ & $(2,3)$ & -- & 8433\\
$(218,59)$ & 12 & $(5,2)$ & 3 & 1 & YES & YES & YES & $1.67$ & $(2,3)$ & -- & 8434\\
$(218,59)$ & 12 & $(5,2)$ & 3 & 1 & YES & YES & YES & $1.78$ & $(2,3)$ & NO & 8435\\
$(218,85)$ & 12 & $(5,1)$ & 4 & 1 & YES & YES & YES & $1.75$ & $(2,3)$ & NO & 8436\\
$(218,85)$ & 12 & $(5,1)$ & 4 & 1 & YES & YES & YES & $1.75$ & $(2,3)$ & NO & 8437\\
$(218,49)$ & 13 & $(7,3)$ & 4 & 1 & YES & YES & YES & $1.89$ & $(4,2)$ & NO & 8438\\
$(218,59)$ & 12 & $(10,3)$ & 5 & 2 & YES & YES & YES & $1.62$ & $(2,3)$ & NO & 8439\\
$(218,85)$ & 12 & $(13,5)$ & 5 & 1 & YES & YES & NO(2) & $1.78$ & $(2,3)$ & NO & 8440\\
$(218,51)$ & 13 & $(22,5)$ & 7 & 2 & YES & YES & YES & $1.75$ & $(2,3)$ & NO & 8441\\
$(218,59)$ & 12 & $(63,17)$ & 9 & 1 & YES & YES & YES & $1.89$ & $(2,3)$ & NO & 8442\\
$(218,85)$ & 12 & $(100,39)$ & 10 & 2 & YES & YES & YES & $1.62$ & $(2,3)$ & 9448 & 8443\\
$(218,85)$ & 12 & $(218,85)$ & 12 & 218 & YES & YES & NO(2) & $1.67$ & $(2,3)$ & NO & 8444\\
$(219,64)$ & 12 & $(2,1)$ & 1 & 1 & YES & YES & NO(2) & $1.38$ & $(4,2)$ & NO & 8445\\
$(219,64)$ & 12 & $(2,1)$ & 1 & 1 & YES & YES & NO(2) & $1.64$ & $(2,3)$ & -- & 8446\\
$(219,67)$ & 12 & $(2,1)$ & 1 & 1 & YES & YES & NO(2) & $1.44$ & $(4,2)$ & -- & 8447\\
$(219,79)$ & 12 & $(2,1)$ & 1 & 1 & YES & YES & YES & $1.57$ & $(2,3)$ & -- & 8448\\
$(219,80)$ & 13 & $(2,1)$ & 1 & 1 & YES & YES & NO(2) & $1.50$ & $(6,1)$ & -- & 8449\\
$(219,85)$ & 12 & $(2,1)$ & 1 & 1 & YES & YES & NO(2) & $1.50$ & $(6,1)$ & -- & 8450\\
$(219,64)$ & 12 & $(3,1)$ & 2 & 3 & YES & YES & YES & $1.78$ & $(2,3)$ & -- & 8451\\
$(219,65)$ & 12 & $(3,1)$ & 2 & 3 & YES & YES & NO(2) & $1.38$ & $(4,2)$ & -- & 8452\\
$(219,80)$ & 13 & $(3,1)$ & 2 & 3 & YES & YES & NO(2) & $1.62$ & $(6,1)$ & -- & 8453\\
$(219,80)$ & 13 & $(3,1)$ & 2 & 3 & YES & YES & NO(2) & $1.67$ & $(8,0)$ & NO & 8454\\
$(219,67)$ & 12 & $(4,1)$ & 3 & 1 & YES & YES & NO(2) & $1.64$ & $(4,2)$ & -- & 8455\\
$(219,85)$ & 12 & $(4,1)$ & 3 & 1 & YES & YES & YES & $1.90$ & $(2,3)$ & NO & 8456\\
$(219,85)$ & 12 & $(4,1)$ & 3 & 1 & YES & YES & YES & $1.90$ & $(2,3)$ & -- & 8457\\
$(219,61)$ & 12 & $(5,2)$ & 3 & 1 & YES & YES & YES & $1.43$ & $(4,2)$ & NO & 8458\\
$(219,61)$ & 12 & $(5,2)$ & 3 & 1 & YES & YES & YES & $1.80$ & $(2,3)$ & -- & 8459\\
$(219,61)$ & 12 & $(5,2)$ & 3 & 1 & YES & YES & YES & $1.70$ & $(2,3)$ & NO & 8460\\
$(219,64)$ & 12 & $(5,2)$ & 3 & 1 & YES & YES & YES & $1.50$ & $(2,3)$ & -- & 8461\\
$(219,64)$ & 12 & $(5,2)$ & 3 & 1 & YES & YES & YES & $1.75$ & $(2,3)$ & NO & 8462\\
$(219,64)$ & 12 & $(5,2)$ & 3 & 1 & YES & YES & YES & $1.62$ & $(2,3)$ & NO & 8463\\
$(219,65)$ & 12 & $(5,2)$ & 3 & 1 & YES & YES & YES & $1.43$ & $(4,2)$ & -- & 8464\\
$(219,65)$ & 12 & $(5,2)$ & 3 & 1 & YES & YES & YES & $1.43$ & $(4,2)$ & NO & 8465\\
$(219,79)$ & 12 & $(5,1)$ & 4 & 1 & YES & YES & YES & $1.43$ & $(4,2)$ & NO & 8466\\
$(219,85)$ & 12 & $(5,2)$ & 3 & 1 & YES & YES & NO(2) & $1.67$ & $(8,0)$ & NO & 8467\\
$(219,77)$ & 13 & $(6,1)$ & 5 & 3 & YES & YES & NO(2) & $1.60$ & $(6,1)$ & NO & 8468\\
$(219,64)$ & 12 & $(7,2)$ & 4 & 1 & YES & YES & YES & $1.43$ & $(4,2)$ & -- & 8469\\
$(219,80)$ & 13 & $(8,3)$ & 4 & 1 & YES & YES & NO(2) & $1.78$ & $(4,2)$ & 4667 & 8470\\
$(219,65)$ & 12 & $(11,3)$ & 5 & 1 & YES & YES & YES & $1.43$ & $(4,2)$ & 11005 & 8471\\
$(219,61)$ & 12 & $(15,4)$ & 6 & 3 & YES & YES & YES & $1.50$ & $(2,3)$ & NO & 8472\\
$(219,65)$ & 12 & $(17,5)$ & 6 & 1 & YES & YES & YES & $1.50$ & $(2,3)$ & 6936 & 8473\\
$(219,85)$ & 12 & $(18,7)$ & 6 & 3 & YES & YES & NO(2) & $1.67$ & $(8,0)$ & NO & 8474\\
$(219,64)$ & 12 & $(27,8)$ & 7 & 3 & YES & YES & YES & $1.50$ & $(2,3)$ & NO & 8475\\
$(219,67)$ & 12 & $(36,11)$ & 8 & 3 & YES & YES & NO(2) & $1.56$ & $(4,2)$ & NO & 8476\\
$(219,65)$ & 12 & $(37,11)$ & 8 & 1 & YES & YES & YES & $1.62$ & $(2,3)$ & NO & 8477\\
$(219,64)$ & 12 & $(41,12)$ & 8 & 1 & YES & YES & YES & $1.43$ & $(4,2)$ & NO & 8478\\
$(219,80)$ & 13 & $(41,15)$ & 8 & 1 & YES & YES & NO(2) & $1.56$ & $(8,0)$ & NO & 8479\\
$(219,64)$ & 12 & $(58,17)$ & 9 & 1 & YES & YES & YES & $1.50$ & $(2,3)$ & NO & 8480\\
$(219,64)$ & 12 & $(89,26)$ & 10 & 1 & YES & YES & YES & $1.50$ & $(2,3)$ & 9056 & 8481\\
$(219,79)$ & 12 & $(97,35)$ & 10 & 1 & YES & YES & YES & $1.57$ & $(4,2)$ & 9325 & 8482\\
$(219,65)$ & 12 & $(101,30)$ & 10 & 1 & YES & YES & YES & $1.57$ & $(2,3)$ & NO & 8483\\
$(219,61)$ & 12 & $(104,29)$ & 10 & 1 & YES & YES & YES & $1.78$ & $(2,3)$ & NO & 8484\\
$(219,64)$ & 12 & $(106,31)$ & 10 & 1 & YES & YES & YES & $1.57$ & $(2,3)$ & NO & 8485\\
$(219,64)$ & 12 & $(113,33)$ & 11 & 1 & YES & YES & YES & $1.57$ & $(4,2)$ & 11180 & 8486\\
$(219,67)$ & 12 & $(134,41)$ & 11 & 1 & YES & YES & NO(2) & $1.64$ & $(4,2)$ & NO & 8487\\
$(219,85)$ & 12 & $(152,59)$ & 11 & 1 & YES & YES & YES & $1.78$ & $(2,3)$ & NO & 8488\\
$(219,92)$ & 12 & $(169,71)$ & 11 & 1 & YES & YES & YES & $1.62$ & $(2,3)$ & NO & 8489\\
$(219,61)$ & 12 & $(176,49)$ & 12 & 1 & YES & YES & YES & $1.78$ & $(4,2)$ & NO & 8490\\
$(219,65)$ & 12 & $(219,65)$ & 12 & 219 & YES & YES & NO(2) & $1.70$ & $(2,3)$ & NO & 8491\\
$(219,79)$ & 12 & $(219,79)$ & 12 & 219 & YES & YES & YES & $1.43$ & $(4,2)$ & NO & 8492\\
$(219,83)$ & 12 & $(219,83)$ & 12 & 219 & YES & YES & YES & $1.43$ & $(4,2)$ & NO & 8493\\
$(219,92)$ & 12 & $(219,92)$ & 12 & 219 & YES & YES & YES & $1.62$ & $(2,3)$ & NO & 8494\\
$(220,59)$ & 13 & $(5,2)$ & 3 & 5 & YES & YES & YES & $1.75$ & $(4,2)$ & -- & 8495\\
$(220,59)$ & 13 & $(5,2)$ & 3 & 5 & YES & YES & YES & $1.88$ & $(4,2)$ & NO & 8496\\
$(221,84)$ & 12 & $(2,1)$ & 1 & 1 & YES & YES & NO(2) & $1.50$ & $(6,1)$ & -- & 8497\\
$(221,84)$ & 12 & $(2,1)$ & 1 & 1 & YES & YES & NO(2) & $1.60$ & $(6,1)$ & NO & 8498\\
$(221,58)$ & 13 & $(3,1)$ & 2 & 1 & YES & YES & NO(2) & $1.56$ & $(4,2)$ & NO & 8499\\
$(221,80)$ & 13 & $(3,1)$ & 2 & 1 & YES & YES & NO(2) & $1.50$ & $(6,1)$ & NO & 8500\\
$(221,80)$ & 13 & $(3,1)$ & 2 & 1 & YES & YES & NO(2) & $1.50$ & $(6,1)$ & -- & 8501\\
$(221,62)$ & 12 & $(4,1)$ & 3 & 1 & YES & YES & YES & $1.71$ & $(4,2)$ & NO & 8502\\
$(221,62)$ & 12 & $(4,1)$ & 3 & 1 & YES & YES & YES & $1.71$ & $(6,1)$ & -- & 8503\\
$(221,58)$ & 13 & $(5,1)$ & 4 & 1 & YES & YES & NO(2) & $1.50$ & $(6,1)$ & -- & 8504\\
$(221,62)$ & 12 & $(5,2)$ & 3 & 1 & YES & YES & YES & $1.43$ & $(4,2)$ & -- & 8505\\
$(221,62)$ & 12 & $(5,2)$ & 3 & 1 & YES & YES & YES & $1.78$ & $(2,3)$ & NO & 8506\\
$(221,84)$ & 12 & $(5,1)$ & 4 & 1 & YES & YES & YES & $1.62$ & $(4,2)$ & NO & 8507\\
$(221,84)$ & 12 & $(5,1)$ & 4 & 1 & YES & YES & YES & $1.62$ & $(4,2)$ & -- & 8508\\
$(221,62)$ & 12 & $(8,3)$ & 4 & 1 & YES & YES & YES & $1.75$ & $(4,2)$ & NO & 8509\\
$(221,84)$ & 12 & $(8,3)$ & 4 & 1 & YES & YES & NO(2) & $1.50$ & $(6,1)$ & NO & 8510\\
$(221,47)$ & 13 & $(10,3)$ & 5 & 1 & YES & YES & YES & $1.75$ & $(4,2)$ & NO & 8511\\
$(221,62)$ & 12 & $(10,3)$ & 5 & 1 & YES & YES & YES & $1.57$ & $(4,2)$ & NO & 8512\\
$(221,58)$ & 13 & $(15,4)$ & 6 & 1 & YES & YES & NO(2) & $1.56$ & $(4,2)$ & NO & 8513\\
$(221,58)$ & 13 & $(23,6)$ & 8 & 1 & YES & YES & NO(2) & $1.56$ & $(8,0)$ & NO & 8514\\
$(221,62)$ & 12 & $(43,12)$ & 8 & 1 & YES & YES & YES & $1.78$ & $(2,3)$ & NO & 8515\\
$(221,84)$ & 12 & $(71,27)$ & 9 & 1 & YES & YES & YES & $1.78$ & $(2,3)$ & NO & 8516\\
$(221,80)$ & 13 & $(105,38)$ & 11 & 1 & YES & YES & NO(2) & $1.62$ & $(6,1)$ & 9637 & 8517\\
$(221,84)$ & 12 & $(121,46)$ & 10 & 1 & YES & YES & YES & $1.62$ & $(4,2)$ & 10173 & 8518\\
$(221,84)$ & 12 & $(171,65)$ & 11 & 1 & YES & YES & YES & $1.43$ & $(6,1)$ & NO & 8519\\
$(222,59)$ & 13 & $(3,1)$ & 2 & 3 & YES & YES & YES & $1.62$ & $(2,3)$ & NO & 8520\\
$(222,59)$ & 13 & $(3,1)$ & 2 & 3 & YES & YES & YES & $1.62$ & $(2,3)$ & -- & 8521\\
$(222,91)$ & 12 & $(3,1)$ & 2 & 3 & YES & YES & YES & $1.62$ & $(2,3)$ & -- & 8522\\
$(222,91)$ & 12 & $(5,1)$ & 4 & 1 & YES & YES & NO(2) & $1.56$ & $(4,2)$ & -- & 8523\\
$(222,91)$ & 12 & $(5,2)$ & 3 & 1 & YES & YES & YES & $1.78$ & $(4,2)$ & -- & 8524\\
$(222,91)$ & 12 & $(17,7)$ & 6 & 1 & YES & YES & NO(2) & $1.67$ & $(4,2)$ & NO & 8525\\
$(222,61)$ & 12 & $(25,7)$ & 7 & 1 & YES & YES & YES & $1.43$ & $(4,2)$ & NO & 8526\\
$(222,91)$ & 12 & $(27,11)$ & 8 & 3 & YES & YES & YES & $1.67$ & $(4,2)$ & NO & 8527\\
$(222,91)$ & 12 & $(61,25)$ & 9 & 1 & YES & YES & YES & $1.57$ & $(2,3)$ & NO & 8528\\
$(222,85)$ & 12 & $(81,31)$ & 9 & 3 & YES & YES & YES & $1.43$ & $(4,2)$ & NO & 8529\\
$(222,91)$ & 12 & $(100,41)$ & 10 & 2 & YES & YES & YES & $1.62$ & $(2,3)$ & 9478 & 8530\\
$(222,85)$ & 12 & $(128,49)$ & 10 & 2 & YES & YES & YES & $1.43$ & $(4,2)$ & 10372 & 8531\\
$(222,91)$ & 12 & $(161,66)$ & 11 & 1 & YES & YES & YES & $1.88$ & $(2,3)$ & NO & 8532\\
$(222,85)$ & 12 & $(175,67)$ & 11 & 1 & YES & YES & YES & $1.43$ & $(4,2)$ & NO & 8533\\
$(223,66)$ & 12 & $(2,1)$ & 1 & 1 & YES & YES & NO(2) & $1.44$ & $(8,0)$ & -- & 8534\\
$(223,68)$ & 12 & $(2,1)$ & 1 & 1 & YES & YES & NO(2) & $1.55$ & $(4,2)$ & -- & 8535\\
$(223,86)$ & 13 & $(2,1)$ & 1 & 1 & YES & YES & NO(2) & $1.56$ & $(8,0)$ & NO & 8536\\
$(223,92)$ & 12 & $(2,1)$ & 1 & 1 & YES & YES & NO(2) & $1.70$ & $(6,1)$ & -- & 8537\\
$(223,40)$ & 14 & $(3,1)$ & 2 & 1 & YES & YES & YES & $1.29$ & $(4,2)$ & NO & 8538\\
$(223,66)$ & 12 & $(3,1)$ & 2 & 1 & YES & YES & YES & $1.43$ & $(6,1)$ & -- & 8539\\
$(223,68)$ & 12 & $(3,1)$ & 2 & 1 & YES & YES & YES & $1.43$ & $(6,1)$ & -- & 8540\\
$(223,70)$ & 13 & $(3,1)$ & 2 & 1 & NO & YES & NO(2) & $1.50$ & $(6,1)$ & -- & 8541\\
$(223,80)$ & 12 & $(3,1)$ & 2 & 1 & YES & YES & NO(2) & $1.60$ & $(10,-1)$ & -- & 8542\\
$(223,98)$ & 12 & $(3,1)$ & 2 & 1 & YES & YES & YES & $1.43$ & $(4,2)$ & -- & 8543\\
$(223,66)$ & 12 & $(4,1)$ & 3 & 1 & YES & YES & YES & $1.43$ & $(6,1)$ & -- & 8544\\
$(223,68)$ & 12 & $(4,1)$ & 3 & 1 & YES & YES & NO(2) & $1.64$ & $(4,2)$ & -- & 8545\\
$(223,92)$ & 12 & $(5,2)$ & 3 & 1 & YES & YES & YES & $1.89$ & $(4,2)$ & -- & 8546\\
$(223,68)$ & 12 & $(7,2)$ & 4 & 1 & YES & YES & YES & $1.43$ & $(6,1)$ & NO & 8547\\
$(223,68)$ & 12 & $(7,2)$ & 4 & 1 & YES & YES & YES & $1.67$ & $(4,2)$ & -- & 8548\\
$(223,66)$ & 12 & $(9,2)$ & 5 & 1 & YES & YES & YES & $1.67$ & $(2,3)$ & NO & 8549\\
$(223,98)$ & 12 & $(9,4)$ & 5 & 1 & YES & YES & YES & $1.71$ & $(2,3)$ & 6134 & 8550\\
$(223,66)$ & 12 & $(11,3)$ & 5 & 1 & YES & YES & YES & $1.67$ & $(2,3)$ & NO & 8551\\
$(223,92)$ & 12 & $(17,7)$ & 6 & 1 & YES & YES & NO(2) & $1.50$ & $(10,-1)$ & NO & 8552\\
$(223,66)$ & 12 & $(24,7)$ & 7 & 1 & YES & YES & YES & $1.67$ & $(2,3)$ & NO & 8553\\
$(223,92)$ & 12 & $(29,12)$ & 7 & 1 & YES & YES & NO(2) & $1.60$ & $(6,1)$ & NO & 8554\\
$(223,66)$ & 12 & $(44,13)$ & 8 & 1 & YES & YES & YES & $1.43$ & $(6,1)$ & 6111 & 8555\\
$(223,66)$ & 12 & $(61,18)$ & 9 & 1 & YES & YES & YES & $1.67$ & $(2,3)$ & NO & 8556\\
$(223,98)$ & 12 & $(66,29)$ & 9 & 1 & YES & YES & NO(2) & $1.50$ & $(10,-1)$ & NO & 8557\\
$(223,66)$ & 12 & $(71,21)$ & 9 & 1 & YES & YES & YES & $1.43$ & $(6,1)$ & NO & 8558\\
$(223,68)$ & 12 & $(95,29)$ & 10 & 1 & YES & YES & YES & $1.67$ & $(4,2)$ & NO & 8559\\
$(223,68)$ & 12 & $(141,43)$ & 11 & 1 & YES & YES & NO(2) & $1.64$ & $(4,2)$ & NO & 8560\\
$(223,92)$ & 12 & $(206,85)$ & 12 & 1 & YES & YES & YES & $1.89$ & $(4,2)$ & NO & 8561\\
$(223,80)$ & 12 & $(223,80)$ & 12 & 223 & YES & YES & NO(2) & $1.60$ & $(10,-1)$ & NO & 8562\\
$(223,98)$ & 12 & $(223,98)$ & 12 & 223 & YES & YES & YES & $1.43$ & $(4,2)$ & NO & 8563\\
$(224,51)$ & 13 & $(3,1)$ & 2 & 1 & YES & YES & YES & $1.57$ & $(2,3)$ & -- & 8564\\
$(224,51)$ & 13 & $(19,4)$ & 7 & 1 & YES & YES & YES & $1.43$ & $(6,1)$ & NO & 8565\\
$(224,51)$ & 13 & $(61,14)$ & 10 & 1 & YES & YES & YES & $1.78$ & $(4,2)$ & NO & 8566\\
$(224,51)$ & 13 & $(224,51)$ & 13 & 224 & YES & YES & YES & $1.57$ & $(2,3)$ & NO & 8567\\
$(225,98)$ & 12 & $(2,1)$ & 1 & 1 & NO & YES & NO(2) & $1.75$ & $(2,3)$ & -- & 8568\\
$(225,62)$ & 12 & $(3,1)$ & 2 & 3 & YES & YES & NO(2) & $1.33$ & $(8,0)$ & -- & 8569\\
$(225,98)$ & 12 & $(5,2)$ & 3 & 5 & YES & YES & YES & $1.43$ & $(4,2)$ & NO & 8570\\
$(225,61)$ & 13 & $(15,4)$ & 6 & 15 & YES & YES & YES & $1.50$ & $(2,3)$ & NO & 8571\\
$(225,98)$ & 12 & $(101,44)$ & 10 & 1 & YES & YES & YES & $1.57$ & $(4,2)$ & 9554 & 8572\\
$(226,63)$ & 12 & $(2,1)$ & 1 & 2 & YES & YES & NO(2) & $1.25$ & $(4,2)$ & -- & 8573\\
$(226,95)$ & 12 & $(2,1)$ & 1 & 2 & YES & YES & YES & $1.57$ & $(2,3)$ & -- & 8574\\
$(226,95)$ & 12 & $(2,1)$ & 1 & 2 & YES & YES & NO(2) & $1.50$ & $(6,1)$ & NO & 8575\\
$(226,61)$ & 12 & $(3,1)$ & 2 & 1 & YES & YES & YES & $1.62$ & $(2,3)$ & -- & 8576\\
$(226,63)$ & 12 & $(3,1)$ & 2 & 1 & YES & YES & YES & $1.43$ & $(4,2)$ & -- & 8577\\
$(226,95)$ & 12 & $(3,1)$ & 2 & 1 & YES & YES & YES & $1.71$ & $(2,3)$ & NO & 8578\\
$(226,95)$ & 12 & $(3,1)$ & 2 & 1 & YES & YES & YES & $1.89$ & $(2,3)$ & -- & 8579\\
$(226,95)$ & 12 & $(3,1)$ & 2 & 1 & YES & YES & YES & $1.89$ & $(2,3)$ & NO & 8580\\
$(226,95)$ & 12 & $(4,1)$ & 3 & 2 & YES & YES & YES & $1.89$ & $(2,3)$ & NO & 8581\\
$(226,95)$ & 12 & $(4,1)$ & 3 & 2 & YES & YES & YES & $1.89$ & $(2,3)$ & NO & 8582\\
$(226,61)$ & 12 & $(5,1)$ & 4 & 1 & YES & YES & YES & $1.43$ & $(4,2)$ & NO & 8583\\
$(226,61)$ & 12 & $(5,2)$ & 3 & 1 & YES & YES & YES & $1.50$ & $(4,2)$ & -- & 8584\\
$(226,63)$ & 12 & $(5,2)$ & 3 & 1 & YES & YES & YES & $1.57$ & $(2,3)$ & -- & 8585\\
$(226,61)$ & 12 & $(7,2)$ & 4 & 1 & YES & YES & YES & $1.62$ & $(2,3)$ & NO & 8586\\
$(226,63)$ & 12 & $(7,2)$ & 4 & 1 & YES & YES & YES & $1.43$ & $(4,2)$ & -- & 8587\\
$(226,51)$ & 15 & $(8,1)$ & 7 & 2 & YES & YES & NO(2) & $1.60$ & $(6,1)$ & NO & 8588\\
$(226,61)$ & 12 & $(10,3)$ & 5 & 2 & YES & YES & YES & $1.57$ & $(2,3)$ & NO & 8589\\
$(226,63)$ & 12 & $(11,2)$ & 6 & 1 & YES & YES & YES & $1.75$ & $(2,3)$ & NO & 8590\\
$(226,63)$ & 12 & $(11,3)$ & 5 & 1 & YES & YES & NO(2) & $1.67$ & $(2,3)$ & NO & 8591\\
$(226,95)$ & 12 & $(12,5)$ & 5 & 2 & YES & YES & YES & $1.71$ & $(2,3)$ & 5185 & 8592\\
$(226,69)$ & 12 & $(17,5)$ & 6 & 1 & YES & YES & YES & $1.80$ & $(2,3)$ & NO & 8593\\
$(226,63)$ & 12 & $(18,5)$ & 6 & 2 & YES & YES & NO(2) & $1.67$ & $(2,3)$ & NO & 8594\\
$(226,95)$ & 12 & $(31,13)$ & 7 & 1 & YES & YES & YES & $1.89$ & $(2,3)$ & NO & 8595\\
$(226,63)$ & 12 & $(32,9)$ & 8 & 2 & YES & YES & YES & $1.57$ & $(4,2)$ & NO & 8596\\
$(226,61)$ & 12 & $(41,11)$ & 8 & 1 & YES & YES & YES & $1.57$ & $(2,3)$ & NO & 8597\\
$(226,63)$ & 12 & $(43,12)$ & 8 & 1 & YES & YES & YES & $1.50$ & $(2,3)$ & NO & 8598\\
$(226,95)$ & 12 & $(88,37)$ & 10 & 2 & YES & YES & YES & $1.89$ & $(2,3)$ & 9079 & 8599\\
$(226,95)$ & 12 & $(157,66)$ & 11 & 1 & YES & YES & YES & $1.89$ & $(2,3)$ & NO & 8600\\
$(226,63)$ & 12 & $(226,63)$ & 12 & 226 & YES & YES & NO(2) & $1.70$ & $(2,3)$ & NO & 8601\\
$(226,95)$ & 12 & $(226,95)$ & 12 & 226 & YES & YES & YES & $2.00$ & $(2,3)$ & NO & 8602\\
$(227,61)$ & 12 & $(2,1)$ & 1 & 1 & YES & YES & NO(2) & $1.56$ & $(2,3)$ & -- & 8603\\
$(227,66)$ & 12 & $(2,1)$ & 1 & 1 & YES & YES & NO(2) & $1.60$ & $(2,3)$ & -- & 8604\\
$(227,67)$ & 12 & $(2,1)$ & 1 & 1 & YES & YES & NO(2) & $1.56$ & $(2,3)$ & NO & 8605\\
$(227,67)$ & 12 & $(2,1)$ & 1 & 1 & YES & YES & YES & $1.50$ & $(2,3)$ & -- & 8606\\
$(227,83)$ & 12 & $(2,1)$ & 1 & 1 & YES & YES & NO(2) & $1.44$ & $(8,0)$ & -- & 8607\\
$(227,86)$ & 12 & $(2,1)$ & 1 & 1 & YES & YES & YES & $1.71$ & $(2,3)$ & -- & 8608\\
$(227,93)$ & 12 & $(2,1)$ & 1 & 1 & YES & YES & NO(2) & $1.70$ & $(6,1)$ & -- & 8609\\
$(227,67)$ & 12 & $(3,1)$ & 2 & 1 & YES & YES & YES & $1.62$ & $(2,3)$ & -- & 8610\\
$(227,67)$ & 12 & $(3,1)$ & 2 & 1 & YES & YES & YES & $1.71$ & $(4,2)$ & NO & 8611\\
$(227,82)$ & 14 & $(3,1)$ & 2 & 1 & YES & YES & YES & $1.71$ & $(4,2)$ & NO & 8612\\
$(227,86)$ & 12 & $(3,1)$ & 2 & 1 & YES & YES & NO(2) & $1.38$ & $(6,1)$ & NO & 8613\\
$(227,86)$ & 12 & $(3,1)$ & 2 & 1 & YES & YES & YES & $1.62$ & $(4,2)$ & NO & 8614\\
$(227,86)$ & 12 & $(3,1)$ & 2 & 1 & YES & YES & YES & $1.62$ & $(4,2)$ & -- & 8615\\
$(227,88)$ & 12 & $(3,1)$ & 2 & 1 & YES & YES & YES & $1.67$ & $(4,2)$ & -- & 8616\\
$(227,99)$ & 12 & $(3,1)$ & 2 & 1 & YES & YES & YES & $1.75$ & $(4,2)$ & -- & 8617\\
$(227,66)$ & 12 & $(4,1)$ & 3 & 1 & YES & YES & YES & $1.57$ & $(6,1)$ & -- & 8618\\
$(227,67)$ & 12 & $(4,1)$ & 3 & 1 & YES & YES & YES & $1.38$ & $(2,3)$ & NO & 8619\\
$(227,67)$ & 12 & $(4,1)$ & 3 & 1 & YES & YES & YES & $1.38$ & $(2,3)$ & -- & 8620\\
$(227,99)$ & 12 & $(4,1)$ & 3 & 1 & YES & YES & YES & $1.62$ & $(4,2)$ & NO & 8621\\
$(227,100)$ & 12 & $(4,1)$ & 3 & 1 & YES & YES & YES & $1.62$ & $(2,3)$ & -- & 8622\\
$(227,52)$ & 13 & $(5,2)$ & 3 & 1 & YES & YES & YES & $1.57$ & $(4,2)$ & -- & 8623\\
$(227,52)$ & 13 & $(5,2)$ & 3 & 1 & YES & YES & YES & $1.57$ & $(4,2)$ & NO & 8624\\
$(227,66)$ & 12 & $(5,2)$ & 3 & 1 & YES & YES & YES & $1.50$ & $(4,2)$ & -- & 8625\\
$(227,67)$ & 12 & $(5,2)$ & 3 & 1 & YES & YES & YES & $1.78$ & $(2,3)$ & -- & 8626\\
$(227,67)$ & 12 & $(5,2)$ & 3 & 1 & YES & YES & YES & $1.56$ & $(2,3)$ & NO & 8627\\
$(227,88)$ & 12 & $(5,1)$ & 4 & 1 & YES & YES & YES & $1.62$ & $(4,2)$ & NO & 8628\\
$(227,93)$ & 12 & $(5,1)$ & 4 & 1 & YES & YES & NO(2) & $1.60$ & $(6,1)$ & NO & 8629\\
$(227,93)$ & 12 & $(5,1)$ & 4 & 1 & YES & YES & NO(2) & $1.60$ & $(6,1)$ & -- & 8630\\
$(227,94)$ & 12 & $(5,2)$ & 3 & 1 & YES & YES & NO(2) & $1.50$ & $(4,2)$ & NO & 8631\\
$(227,99)$ & 12 & $(5,2)$ & 3 & 1 & YES & YES & YES & $1.50$ & $(4,2)$ & NO & 8632\\
$(227,52)$ & 13 & $(7,2)$ & 4 & 1 & YES & YES & YES & $1.67$ & $(4,2)$ & -- & 8633\\
$(227,61)$ & 12 & $(7,2)$ & 4 & 1 & YES & YES & YES & $1.71$ & $(2,3)$ & -- & 8634\\
$(227,67)$ & 12 & $(7,2)$ & 4 & 1 & YES & YES & NO(2) & $1.44$ & $(4,2)$ & NO & 8635\\
$(227,67)$ & 12 & $(7,2)$ & 4 & 1 & YES & YES & YES & $1.57$ & $(2,3)$ & -- & 8636\\
$(227,67)$ & 12 & $(7,3)$ & 4 & 1 & YES & YES & YES & $1.78$ & $(4,2)$ & -- & 8637\\
$(227,94)$ & 12 & $(9,2)$ & 5 & 1 & YES & YES & YES & $1.75$ & $(4,2)$ & -- & 8638\\
$(227,61)$ & 12 & $(10,3)$ & 5 & 1 & YES & YES & YES & $1.67$ & $(2,3)$ & NO & 8639\\
$(227,67)$ & 12 & $(13,4)$ & 6 & 1 & YES & YES & YES & $1.62$ & $(4,2)$ & NO & 8640\\
$(227,52)$ & 13 & $(14,3)$ & 6 & 1 & YES & YES & YES & $1.90$ & $(2,3)$ & NO & 8641\\
$(227,61)$ & 12 & $(15,4)$ & 6 & 1 & YES & YES & NO(2) & $1.56$ & $(2,3)$ & NO & 8642\\
$(227,67)$ & 12 & $(17,5)$ & 6 & 1 & YES & YES & NO(2) & $1.56$ & $(2,3)$ & NO & 8643\\
$(227,93)$ & 12 & $(17,7)$ & 6 & 1 & YES & YES & NO(2) & $1.60$ & $(6,1)$ & NO & 8644\\
$(227,61)$ & 12 & $(18,5)$ & 6 & 1 & YES & YES & YES & $1.67$ & $(2,3)$ & NO & 8645\\
$(227,61)$ & 12 & $(19,5)$ & 7 & 1 & YES & YES & YES & $1.62$ & $(4,2)$ & NO & 8646\\
$(227,67)$ & 12 & $(27,8)$ & 7 & 1 & YES & YES & YES & $1.62$ & $(2,3)$ & NO & 8647\\
$(227,52)$ & 13 & $(31,7)$ & 8 & 1 & YES & YES & YES & $1.57$ & $(4,2)$ & NO & 8648\\
$(227,94)$ & 12 & $(53,22)$ & 9 & 1 & YES & YES & YES & $1.75$ & $(4,2)$ & NO & 8649\\
$(227,61)$ & 12 & $(56,15)$ & 9 & 1 & YES & YES & YES & $1.57$ & $(4,2)$ & NO & 8650\\
$(227,52)$ & 13 & $(57,13)$ & 9 & 1 & YES & YES & YES & $1.78$ & $(4,2)$ & NO & 8651\\
$(227,66)$ & 12 & $(79,23)$ & 10 & 1 & YES & YES & YES & $1.67$ & $(2,3)$ & NO & 8652\\
$(227,88)$ & 12 & $(80,31)$ & 9 & 1 & YES & YES & YES & $1.78$ & $(4,2)$ & NO & 8653\\
$(227,93)$ & 12 & $(83,34)$ & 10 & 1 & YES & YES & NO(2) & $1.70$ & $(6,1)$ & NO & 8654\\
$(227,83)$ & 12 & $(93,34)$ & 10 & 1 & YES & YES & NO(2) & $1.44$ & $(8,0)$ & NO & 8655\\
$(227,86)$ & 12 & $(95,36)$ & 10 & 1 & YES & YES & YES & $1.75$ & $(4,2)$ & 9362 & 8656\\
$(227,67)$ & 12 & $(105,31)$ & 10 & 1 & YES & YES & YES & $1.38$ & $(2,3)$ & 9691 & 8657\\
$(227,66)$ & 12 & $(117,34)$ & 11 & 1 & YES & YES & YES & $1.50$ & $(4,2)$ & NO & 8658\\
$(227,69)$ & 13 & $(125,38)$ & 12 & 1 & YES & YES & NO(2) & $1.56$ & $(4,2)$ & NO & 8659\\
$(227,88)$ & 12 & $(129,50)$ & 10 & 1 & YES & YES & YES & $1.78$ & $(2,3)$ & 10432 & 8660\\
$(227,99)$ & 12 & $(133,58)$ & 11 & 1 & YES & YES & YES & $1.50$ & $(4,2)$ & NO & 8661\\
$(227,87)$ & 12 & $(154,59)$ & 11 & 1 & YES & YES & YES & $1.78$ & $(4,2)$ & 11380 & 8662\\
$(227,88)$ & 12 & $(178,69)$ & 11 & 1 & YES & YES & YES & $1.75$ & $(4,2)$ & NO & 8663\\
$(227,66)$ & 12 & $(227,66)$ & 12 & 227 & YES & YES & YES & $1.43$ & $(6,1)$ & NO & 8664\\
$(227,86)$ & 12 & $(227,86)$ & 12 & 227 & YES & YES & YES & $1.62$ & $(4,2)$ & NO & 8665\\
$(227,99)$ & 12 & $(227,99)$ & 12 & 227 & YES & YES & YES & $1.75$ & $(4,2)$ & NO & 8666\\
$(229,94)$ & 12 & $(2,1)$ & 1 & 1 & YES & YES & NO(2) & $1.38$ & $(4,2)$ & -- & 8667\\
$(229,95)$ & 12 & $(2,1)$ & 1 & 1 & YES & YES & NO(2) & $1.56$ & $(4,2)$ & NO & 8668\\
$(229,97)$ & 12 & $(2,1)$ & 1 & 1 & YES & YES & NO(2) & $1.70$ & $(6,1)$ & -- & 8669\\
$(229,64)$ & 12 & $(3,1)$ & 2 & 1 & YES & YES & YES & $1.78$ & $(2,3)$ & NO & 8670\\
$(229,64)$ & 12 & $(3,1)$ & 2 & 1 & YES & YES & YES & $1.78$ & $(2,3)$ & -- & 8671\\
$(229,68)$ & 12 & $(3,1)$ & 2 & 1 & YES & YES & NO(3) & $1.29$ & $(2,3)$ & -- & 8672\\
$(229,94)$ & 12 & $(3,1)$ & 2 & 1 & YES & YES & NO(2) & $1.56$ & $(8,0)$ & NO & 8673\\
$(229,94)$ & 12 & $(3,1)$ & 2 & 1 & YES & YES & YES & $1.90$ & $(2,3)$ & -- & 8674\\
$(229,95)$ & 12 & $(3,1)$ & 2 & 1 & YES & YES & NO(2) & $1.50$ & $(6,1)$ & NO & 8675\\
$(229,95)$ & 12 & $(3,1)$ & 2 & 1 & YES & YES & YES & $1.57$ & $(4,2)$ & -- & 8676\\
$(229,85)$ & 12 & $(4,1)$ & 3 & 1 & YES & YES & NO(2) & $1.56$ & $(8,0)$ & -- & 8677\\
$(229,94)$ & 12 & $(4,1)$ & 3 & 1 & YES & YES & YES & $1.71$ & $(4,2)$ & NO & 8678\\
$(229,94)$ & 12 & $(4,1)$ & 3 & 1 & YES & YES & YES & $1.71$ & $(4,2)$ & -- & 8679\\
$(229,62)$ & 13 & $(5,1)$ & 4 & 1 & YES & YES & NO(2) & $1.38$ & $(4,2)$ & NO & 8680\\
$(229,64)$ & 12 & $(5,1)$ & 4 & 1 & YES & YES & NO(2) & $1.67$ & $(2,3)$ & NO & 8681\\
$(229,64)$ & 12 & $(5,2)$ & 3 & 1 & YES & YES & YES & $1.57$ & $(4,2)$ & NO & 8682\\
$(229,64)$ & 12 & $(5,2)$ & 3 & 1 & YES & YES & YES & $1.89$ & $(2,3)$ & NO & 8683\\
$(229,64)$ & 12 & $(5,2)$ & 3 & 1 & YES & YES & YES & $1.89$ & $(2,3)$ & -- & 8684\\
$(229,68)$ & 12 & $(5,1)$ & 4 & 1 & YES & YES & YES & $1.62$ & $(2,3)$ & NO & 8685\\
$(229,68)$ & 12 & $(5,2)$ & 3 & 1 & YES & YES & YES & $1.57$ & $(2,3)$ & -- & 8686\\
$(229,94)$ & 12 & $(5,1)$ & 4 & 1 & YES & YES & NO(2) & $1.38$ & $(4,2)$ & NO & 8687\\
$(229,94)$ & 12 & $(5,1)$ & 4 & 1 & YES & YES & NO(2) & $1.50$ & $(4,2)$ & -- & 8688\\
$(229,94)$ & 12 & $(5,2)$ & 3 & 1 & YES & YES & YES & $1.75$ & $(4,2)$ & -- & 8689\\
$(229,95)$ & 12 & $(5,2)$ & 3 & 1 & YES & YES & YES & $1.62$ & $(4,2)$ & -- & 8690\\
$(229,97)$ & 12 & $(5,1)$ & 4 & 1 & YES & YES & NO(2) & $1.50$ & $(4,2)$ & -- & 8691\\
$(229,97)$ & 12 & $(5,1)$ & 4 & 1 & YES & YES & NO(2) & $1.62$ & $(4,2)$ & NO & 8692\\
$(229,97)$ & 12 & $(5,1)$ & 4 & 1 & YES & YES & NO(2) & $1.60$ & $(6,1)$ & NO & 8693\\
$(229,64)$ & 12 & $(7,2)$ & 4 & 1 & YES & YES & NO(2) & $1.38$ & $(6,1)$ & 6041 & 8694\\
$(229,64)$ & 12 & $(7,2)$ & 4 & 1 & YES & YES & YES & $1.86$ & $(2,3)$ & -- & 8695\\
$(229,68)$ & 12 & $(7,2)$ & 4 & 1 & YES & YES & YES & $1.86$ & $(2,3)$ & -- & 8696\\
$(229,68)$ & 12 & $(7,2)$ & 4 & 1 & YES & YES & NO(2) & $1.60$ & $(2,3)$ & NO & 8697\\
$(229,94)$ & 12 & $(7,2)$ & 4 & 1 & YES & YES & YES & $1.78$ & $(4,2)$ & -- & 8698\\
$(229,95)$ & 12 & $(7,2)$ & 4 & 1 & YES & YES & YES & $1.90$ & $(2,3)$ & NO & 8699\\
$(229,95)$ & 12 & $(7,3)$ & 4 & 1 & YES & YES & NO(2) & $1.56$ & $(4,2)$ & NO & 8700\\
$(229,64)$ & 12 & $(8,3)$ & 4 & 1 & YES & YES & YES & $1.75$ & $(4,2)$ & NO & 8701\\
$(229,64)$ & 12 & $(9,2)$ & 5 & 1 & YES & YES & YES & $1.71$ & $(2,3)$ & NO & 8702\\
$(229,64)$ & 12 & $(11,3)$ & 5 & 1 & YES & YES & YES & $1.67$ & $(2,3)$ & NO & 8703\\
$(229,68)$ & 12 & $(11,3)$ & 5 & 1 & YES & YES & YES & $1.62$ & $(4,2)$ & NO & 8704\\
$(229,62)$ & 13 & $(15,4)$ & 6 & 1 & YES & YES & NO(2) & $1.38$ & $(4,2)$ & NO & 8705\\
$(229,64)$ & 12 & $(15,4)$ & 6 & 1 & YES & YES & YES & $1.71$ & $(2,3)$ & NO & 8706\\
$(229,68)$ & 12 & $(17,5)$ & 6 & 1 & YES & YES & YES & $1.50$ & $(2,3)$ & NO & 8707\\
$(229,95)$ & 12 & $(17,7)$ & 6 & 1 & YES & YES & NO(2) & $1.78$ & $(2,3)$ & NO & 8708\\
$(229,85)$ & 12 & $(19,7)$ & 6 & 1 & YES & YES & YES & $1.57$ & $(6,1)$ & NO & 8709\\
$(229,97)$ & 12 & $(19,8)$ & 6 & 1 & YES & YES & NO(2) & $1.38$ & $(4,2)$ & NO & 8710\\
$(229,94)$ & 12 & $(22,9)$ & 7 & 1 & YES & YES & NO(2) & $1.60$ & $(6,1)$ & NO & 8711\\
$(229,64)$ & 12 & $(25,7)$ & 7 & 1 & YES & YES & NO(2) & $1.56$ & $(4,2)$ & NO & 8712\\
$(229,68)$ & 12 & $(27,8)$ & 7 & 1 & YES & YES & YES & $2.00$ & $(2,3)$ & NO & 8713\\
$(229,85)$ & 12 & $(27,10)$ & 7 & 1 & YES & YES & NO(2) & $1.67$ & $(8,0)$ & 6291 & 8714\\
$(229,64)$ & 12 & $(29,8)$ & 7 & 1 & YES & YES & YES & $1.78$ & $(2,3)$ & NO & 8715\\
$(229,64)$ & 12 & $(43,12)$ & 8 & 1 & YES & YES & YES & $1.67$ & $(2,3)$ & NO & 8716\\
$(229,68)$ & 12 & $(44,13)$ & 8 & 1 & YES & YES & YES & $1.57$ & $(2,3)$ & NO & 8717\\
$(229,95)$ & 12 & $(53,22)$ & 9 & 1 & YES & YES & YES & $1.43$ & $(4,2)$ & NO & 8718\\
$(229,94)$ & 12 & $(56,23)$ & 9 & 1 & YES & YES & YES & $2.00$ & $(2,3)$ & NO & 8719\\
$(229,64)$ & 12 & $(61,17)$ & 9 & 1 & YES & YES & YES & $1.78$ & $(2,3)$ & NO & 8720\\
$(229,100)$ & 13 & $(71,31)$ & 10 & 1 & YES & YES & NO(2) & $1.67$ & $(8,0)$ & NO & 8721\\
$(229,68)$ & 12 & $(91,27)$ & 10 & 1 & YES & YES & YES & $1.57$ & $(2,3)$ & NO & 8722\\
$(229,68)$ & 12 & $(101,30)$ & 10 & 1 & YES & YES & YES & $1.50$ & $(2,3)$ & 9599 & 8723\\
$(229,64)$ & 12 & $(111,31)$ & 10 & 1 & YES & YES & YES & $1.71$ & $(2,3)$ & NO & 8724\\
$(229,95)$ & 12 & $(135,56)$ & 11 & 1 & YES & YES & YES & $1.57$ & $(4,2)$ & NO & 8725\\
$(229,94)$ & 12 & $(151,62)$ & 11 & 1 & YES & YES & YES & $1.62$ & $(4,2)$ & NO & 8726\\
$(229,64)$ & 12 & $(229,64)$ & 12 & 229 & YES & YES & NO(2) & $1.67$ & $(2,3)$ & NO & 8727\\
$(229,85)$ & 12 & $(229,85)$ & 12 & 229 & YES & YES & YES & $1.57$ & $(6,1)$ & NO & 8728\\
$(229,95)$ & 12 & $(229,95)$ & 12 & 229 & YES & YES & YES & $1.57$ & $(4,2)$ & NO & 8729\\
$(229,97)$ & 12 & $(229,97)$ & 12 & 229 & YES & YES & NO(2) & $1.56$ & $(2,3)$ & NO & 8730\\
$(230,67)$ & 13 & $(3,1)$ & 2 & 1 & YES & YES & YES & $1.57$ & $(2,3)$ & -- & 8731\\
$(230,67)$ & 13 & $(3,1)$ & 2 & 1 & YES & YES & YES & $1.71$ & $(2,3)$ & NO & 8732\\
$(230,83)$ & 12 & $(3,1)$ & 2 & 1 & YES & YES & YES & $1.57$ & $(4,2)$ & NO & 8733\\
$(230,83)$ & 12 & $(3,1)$ & 2 & 1 & YES & YES & YES & $1.57$ & $(4,2)$ & -- & 8734\\
$(230,97)$ & 12 & $(5,1)$ & 4 & 5 & YES & YES & NO(2) & $1.50$ & $(4,2)$ & NO & 8735\\
$(230,97)$ & 12 & $(5,1)$ & 4 & 5 & YES & YES & NO(2) & $1.50$ & $(4,2)$ & -- & 8736\\
$(230,67)$ & 13 & $(7,2)$ & 4 & 1 & YES & YES & YES & $1.62$ & $(2,3)$ & NO & 8737\\
$(230,97)$ & 12 & $(7,3)$ & 4 & 1 & YES & YES & NO(2) & $1.50$ & $(4,2)$ & NO & 8738\\
$(230,67)$ & 13 & $(11,3)$ & 5 & 1 & YES & YES & NO(2) & $1.50$ & $(6,1)$ & NO & 8739\\
$(230,67)$ & 13 & $(31,9)$ & 8 & 1 & YES & YES & NO(2) & $1.56$ & $(4,2)$ & NO & 8740\\
$(230,83)$ & 12 & $(61,22)$ & 9 & 1 & YES & YES & YES & $1.57$ & $(4,2)$ & NO & 8741\\
$(230,67)$ & 13 & $(79,23)$ & 10 & 1 & YES & YES & YES & $1.57$ & $(2,3)$ & NO & 8742\\
$(231,64)$ & 12 & $(2,1)$ & 1 & 1 & YES & YES & NO(2) & $1.60$ & $(2,3)$ & NO & 8743\\
$(231,83)$ & 12 & $(2,1)$ & 1 & 1 & YES & YES & YES & $1.43$ & $(2,3)$ & -- & 8744\\
$(231,83)$ & 12 & $(3,1)$ & 2 & 3 & YES & YES & YES & $1.57$ & $(2,3)$ & -- & 8745\\
$(231,64)$ & 12 & $(5,2)$ & 3 & 1 & YES & YES & YES & $1.75$ & $(2,3)$ & -- & 8746\\
$(231,83)$ & 12 & $(5,1)$ & 4 & 1 & YES & YES & YES & $1.43$ & $(4,2)$ & NO & 8747\\
$(231,64)$ & 12 & $(7,2)$ & 4 & 7 & YES & YES & NO(2) & $1.38$ & $(6,1)$ & NO & 8748\\
$(231,64)$ & 12 & $(10,3)$ & 5 & 1 & YES & YES & YES & $1.78$ & $(4,2)$ & NO & 8749\\
$(231,64)$ & 12 & $(15,4)$ & 6 & 3 & YES & YES & YES & $1.75$ & $(2,3)$ & NO & 8750\\
$(231,67)$ & 13 & $(31,9)$ & 8 & 1 & YES & YES & YES & $1.75$ & $(2,3)$ & 6977 & 8751\\
$(231,61)$ & 13 & $(34,9)$ & 8 & 1 & YES & YES & YES & $1.43$ & $(2,3)$ & NO & 8752\\
$(231,53)$ & 13 & $(35,8)$ & 8 & 7 & YES & YES & YES & $1.43$ & $(2,3)$ & NO & 8753\\
$(231,83)$ & 12 & $(53,19)$ & 9 & 1 & YES & YES & YES & $1.57$ & $(6,1)$ & 10551 & 8754\\
$(231,83)$ & 12 & $(167,60)$ & 11 & 1 & YES & YES & YES & $1.62$ & $(4,2)$ & NO & 8755\\
$(231,83)$ & 12 & $(231,83)$ & 12 & 231 & YES & YES & YES & $1.71$ & $(4,2)$ & NO & 8756\\
$(232,85)$ & 12 & $(2,1)$ & 1 & 2 & YES & YES & NO(2) & $1.44$ & $(8,0)$ & -- & 8757\\
$(232,85)$ & 12 & $(4,1)$ & 3 & 4 & YES & YES & YES & $1.43$ & $(4,2)$ & -- & 8758\\
$(232,85)$ & 12 & $(41,15)$ & 8 & 1 & YES & YES & YES & $1.43$ & $(4,2)$ & 6287 & 8759\\
$(233,86)$ & 12 & $(2,1)$ & 1 & 1 & YES & YES & NO(2) & $1.50$ & $(4,2)$ & NO & 8760\\
$(233,89)$ & 11 & $(2,1)$ & 1 & 1 & YES & YES & YES & $1.67$ & $(2,3)$ & -- & 8761\\
$(233,89)$ & 11 & $(2,1)$ & 1 & 1 & YES & YES & YES & $1.62$ & $(4,2)$ & NO & 8762\\
$(233,84)$ & 12 & $(3,1)$ & 2 & 1 & YES & YES & YES & $1.57$ & $(4,2)$ & NO & 8763\\
$(233,84)$ & 12 & $(3,1)$ & 2 & 1 & YES & YES & YES & $1.57$ & $(4,2)$ & -- & 8764\\
$(233,89)$ & 11 & $(3,1)$ & 2 & 1 & YES & YES & YES & $1.67$ & $(2,3)$ & -- & 8765\\
$(233,89)$ & 11 & $(3,1)$ & 2 & 1 & YES & YES & YES & $1.78$ & $(2,3)$ & NO & 8766\\
$(233,89)$ & 11 & $(3,1)$ & 2 & 1 & YES & YES & YES & $1.62$ & $(2,3)$ & 8156 & 8767\\
$(233,89)$ & 11 & $(4,1)$ & 3 & 1 & YES & YES & YES & $1.62$ & $(2,3)$ & NO & 8768\\
$(233,89)$ & 11 & $(4,1)$ & 3 & 1 & YES & YES & YES & $1.62$ & $(4,2)$ & NO & 8769\\
$(233,64)$ & 12 & $(5,1)$ & 4 & 1 & YES & YES & NO(2) & $1.44$ & $(8,0)$ & NO & 8770\\
$(233,86)$ & 12 & $(5,2)$ & 3 & 1 & YES & YES & YES & $1.80$ & $(2,3)$ & NO & 8771\\
$(233,89)$ & 11 & $(5,1)$ & 4 & 1 & YES & YES & YES & $1.78$ & $(2,3)$ & NO & 8772\\
$(233,89)$ & 11 & $(5,1)$ & 4 & 1 & YES & YES & YES & $1.78$ & $(2,3)$ & -- & 8773\\
$(233,89)$ & 11 & $(5,2)$ & 3 & 1 & YES & YES & YES & $1.62$ & $(2,3)$ & -- & 8774\\
$(233,89)$ & 11 & $(5,2)$ & 3 & 1 & YES & YES & YES & $1.78$ & $(2,3)$ & NO & 8775\\
$(233,89)$ & 11 & $(5,2)$ & 3 & 1 & YES & YES & YES & $1.75$ & $(2,3)$ & NO & 8776\\
$(233,90)$ & 13 & $(5,1)$ & 4 & 1 & YES & YES & YES & $2.00$ & $(2,3)$ & NO & 8777\\
$(233,89)$ & 11 & $(7,3)$ & 4 & 1 & YES & YES & YES & $1.75$ & $(2,3)$ & NO & 8778\\
$(233,89)$ & 11 & $(8,3)$ & 4 & 1 & YES & YES & YES & $1.62$ & $(2,3)$ & 8206 & 8779\\
$(233,89)$ & 11 & $(13,5)$ & 5 & 1 & YES & YES & YES & $1.50$ & $(2,3)$ & NO & 8780\\
$(233,89)$ & 11 & $(21,8)$ & 6 & 1 & YES & YES & YES & $1.62$ & $(2,3)$ & 8170 & 8781\\
$(233,71)$ & 13 & $(23,7)$ & 7 & 1 & YES & YES & NO(2) & $1.56$ & $(8,0)$ & 6699 & 8782\\
$(233,89)$ & 11 & $(34,13)$ & 7 & 1 & YES & YES & YES & $1.67$ & $(2,3)$ & NO & 8783\\
$(233,84)$ & 12 & $(36,13)$ & 8 & 1 & YES & YES & YES & $1.57$ & $(4,2)$ & 9322 & 8784\\
$(233,89)$ & 11 & $(47,18)$ & 8 & 1 & YES & YES & YES & $1.71$ & $(2,3)$ & NO & 8785\\
$(233,89)$ & 11 & $(55,21)$ & 8 & 1 & YES & YES & YES & $1.89$ & $(2,3)$ & 8016 & 8786\\
$(233,89)$ & 11 & $(76,29)$ & 9 & 1 & YES & YES & YES & $1.75$ & $(2,3)$ & NO & 8787\\
$(233,89)$ & 11 & $(89,34)$ & 9 & 1 & YES & YES & YES & $1.78$ & $(2,3)$ & NO & 8788\\
$(233,64)$ & 12 & $(91,25)$ & 10 & 1 & YES & YES & NO(2) & $1.44$ & $(8,0)$ & NO & 8789\\
$(233,89)$ & 11 & $(123,47)$ & 10 & 1 & YES & YES & YES & $1.71$ & $(2,3)$ & NO & 8790\\
$(233,89)$ & 11 & $(144,55)$ & 10 & 1 & YES & YES & YES & $1.50$ & $(2,3)$ & NO & 8791\\
$(233,91)$ & 12 & $(146,57)$ & 11 & 1 & YES & YES & YES & $1.78$ & $(4,2)$ & 11427 & 8792\\
$(233,89)$ & 11 & $(233,89)$ & 11 & 233 & YES & YES & YES & $1.50$ & $(2,3)$ & NO & 8793\\
$(234,71)$ & 12 & $(2,1)$ & 1 & 2 & YES & YES & YES & $1.43$ & $(4,2)$ & -- & 8794\\
$(234,71)$ & 12 & $(3,1)$ & 2 & 3 & YES & YES & YES & $1.50$ & $(2,3)$ & -- & 8795\\
$(234,71)$ & 12 & $(3,1)$ & 2 & 3 & YES & YES & YES & $1.75$ & $(2,3)$ & NO & 8796\\
$(234,89)$ & 12 & $(4,1)$ & 3 & 2 & YES & YES & YES & $1.43$ & $(6,1)$ & NO & 8797\\
$(234,89)$ & 12 & $(4,1)$ & 3 & 2 & YES & YES & YES & $1.43$ & $(6,1)$ & -- & 8798\\
$(234,53)$ & 13 & $(5,2)$ & 3 & 1 & YES & YES & YES & $1.50$ & $(2,3)$ & NO & 8799\\
$(234,53)$ & 13 & $(5,2)$ & 3 & 1 & YES & YES & YES & $1.80$ & $(2,3)$ & -- & 8800\\
$(234,71)$ & 12 & $(5,2)$ & 3 & 1 & YES & YES & YES & $1.57$ & $(4,2)$ & -- & 8801\\
$(234,71)$ & 12 & $(7,2)$ & 4 & 1 & YES & YES & YES & $1.50$ & $(2,3)$ & NO & 8802\\
$(234,71)$ & 12 & $(13,4)$ & 6 & 13 & YES & YES & YES & $1.43$ & $(6,1)$ & 9185 & 8803\\
$(234,71)$ & 12 & $(23,7)$ & 7 & 1 & YES & YES & YES & $1.86$ & $(6,1)$ & 8367 & 8804\\
$(234,71)$ & 12 & $(79,24)$ & 10 & 1 & YES & YES & YES & $1.57$ & $(4,2)$ & NO & 8805\\
$(234,71)$ & 12 & $(89,27)$ & 10 & 1 & YES & YES & YES & $1.43$ & $(4,2)$ & NO & 8806\\
$(234,89)$ & 12 & $(113,43)$ & 11 & 1 & YES & YES & YES & $1.62$ & $(4,2)$ & 11399 & 8807\\
$(234,71)$ & 12 & $(122,37)$ & 11 & 2 & YES & YES & YES & $1.71$ & $(2,3)$ & NO & 8808\\
$(234,89)$ & 12 & $(163,62)$ & 11 & 1 & YES & YES & YES & $1.43$ & $(6,1)$ & NO & 8809\\
$(234,71)$ & 12 & $(201,61)$ & 12 & 3 & YES & YES & YES & $1.75$ & $(2,3)$ & NO & 8810\\
$(235,103)$ & 13 & $(2,1)$ & 1 & 1 & YES & YES & NO(2) & $1.56$ & $(8,0)$ & -- & 8811\\
$(235,66)$ & 12 & $(3,1)$ & 2 & 1 & YES & YES & YES & $1.43$ & $(6,1)$ & -- & 8812\\
$(235,89)$ & 12 & $(3,1)$ & 2 & 1 & YES & YES & NO(2) & $1.44$ & $(4,2)$ & -- & 8813\\
$(235,97)$ & 12 & $(4,1)$ & 3 & 1 & YES & YES & YES & $1.57$ & $(2,3)$ & -- & 8814\\
$(235,63)$ & 12 & $(5,2)$ & 3 & 5 & YES & YES & YES & $1.50$ & $(4,2)$ & -- & 8815\\
$(235,89)$ & 12 & $(5,2)$ & 3 & 5 & YES & YES & YES & $1.89$ & $(4,2)$ & -- & 8816\\
$(235,97)$ & 12 & $(5,2)$ & 3 & 5 & YES & YES & NO(2) & $1.70$ & $(6,1)$ & NO & 8817\\
$(235,97)$ & 12 & $(7,3)$ & 4 & 1 & YES & YES & YES & $1.78$ & $(2,3)$ & NO & 8818\\
$(235,63)$ & 12 & $(18,5)$ & 6 & 1 & YES & YES & YES & $1.50$ & $(4,2)$ & NO & 8819\\
$(235,66)$ & 12 & $(18,5)$ & 6 & 1 & YES & YES & YES & $1.43$ & $(6,1)$ & 10130 & 8820\\
$(235,63)$ & 12 & $(26,7)$ & 7 & 1 & YES & YES & YES & $1.43$ & $(2,3)$ & NO & 8821\\
$(235,97)$ & 12 & $(29,12)$ & 7 & 1 & YES & YES & YES & $1.78$ & $(2,3)$ & NO & 8822\\
$(235,63)$ & 12 & $(37,10)$ & 8 & 1 & YES & YES & YES & $1.50$ & $(4,2)$ & NO & 8823\\
$(235,97)$ & 12 & $(109,45)$ & 10 & 1 & YES & YES & YES & $1.67$ & $(2,3)$ & 9889 & 8824\\
$(235,97)$ & 12 & $(172,71)$ & 11 & 1 & YES & YES & YES & $1.43$ & $(2,3)$ & NO & 8825\\
$(236,65)$ & 12 & $(2,1)$ & 1 & 2 & YES & YES & NO(2) & $1.38$ & $(4,2)$ & -- & 8826\\
$(236,65)$ & 12 & $(2,1)$ & 1 & 2 & YES & YES & YES & $1.50$ & $(2,3)$ & NO & 8827\\
$(236,69)$ & 12 & $(2,1)$ & 1 & 2 & YES & YES & YES & $1.38$ & $(2,3)$ & -- & 8828\\
$(236,83)$ & 13 & $(2,1)$ & 1 & 2 & YES & YES & NO(2) & $1.50$ & $(6,1)$ & NO & 8829\\
$(236,65)$ & 12 & $(3,1)$ & 2 & 1 & YES & YES & YES & $1.50$ & $(2,3)$ & -- & 8830\\
$(236,65)$ & 12 & $(3,1)$ & 2 & 1 & YES & YES & YES & $1.62$ & $(2,3)$ & NO & 8831\\
$(236,69)$ & 12 & $(3,1)$ & 2 & 1 & YES & YES & YES & $1.50$ & $(2,3)$ & -- & 8832\\
$(236,69)$ & 12 & $(3,1)$ & 2 & 1 & YES & YES & YES & $1.57$ & $(4,2)$ & NO & 8833\\
$(236,73)$ & 13 & $(3,1)$ & 2 & 1 & YES & YES & YES & $1.71$ & $(2,3)$ & -- & 8834\\
$(236,103)$ & 13 & $(3,1)$ & 2 & 1 & YES & YES & NO(2) & $1.78$ & $(4,2)$ & NO & 8835\\
$(236,65)$ & 12 & $(4,1)$ & 3 & 4 & YES & YES & NO(2) & $1.56$ & $(2,3)$ & NO & 8836\\
$(236,69)$ & 12 & $(4,1)$ & 3 & 4 & YES & YES & YES & $1.90$ & $(2,3)$ & NO & 8837\\
$(236,97)$ & 13 & $(4,1)$ & 3 & 4 & YES & YES & NO(2) & $1.70$ & $(6,1)$ & NO & 8838\\
$(236,97)$ & 13 & $(4,1)$ & 3 & 4 & YES & YES & NO(2) & $1.70$ & $(6,1)$ & -- & 8839\\
$(236,69)$ & 12 & $(5,1)$ & 4 & 1 & YES & YES & YES & $1.62$ & $(2,3)$ & NO & 8840\\
$(236,69)$ & 12 & $(5,2)$ & 3 & 1 & YES & YES & YES & $1.90$ & $(2,3)$ & -- & 8841\\
$(236,65)$ & 12 & $(7,2)$ & 4 & 1 & YES & YES & YES & $1.50$ & $(2,3)$ & NO & 8842\\
$(236,69)$ & 12 & $(7,2)$ & 4 & 1 & YES & YES & YES & $1.67$ & $(4,2)$ & -- & 8843\\
$(236,69)$ & 12 & $(7,3)$ & 4 & 1 & YES & YES & YES & $1.78$ & $(4,2)$ & -- & 8844\\
$(236,69)$ & 12 & $(10,3)$ & 5 & 2 & YES & YES & YES & $1.43$ & $(2,3)$ & NO & 8845\\
$(236,69)$ & 12 & $(11,3)$ & 5 & 1 & YES & YES & YES & $1.62$ & $(4,2)$ & NO & 8846\\
$(236,69)$ & 12 & $(13,4)$ & 6 & 1 & YES & YES & YES & $1.71$ & $(2,3)$ & NO & 8847\\
$(236,49)$ & 14 & $(14,3)$ & 6 & 2 & YES & YES & NO(2) & $1.56$ & $(4,2)$ & NO & 8848\\
$(236,69)$ & 12 & $(17,5)$ & 6 & 1 & YES & YES & YES & $1.62$ & $(2,3)$ & NO & 8849\\
$(236,83)$ & 13 & $(17,6)$ & 7 & 1 & YES & YES & NO(2) & $1.50$ & $(6,1)$ & 7951 & 8850\\
$(236,65)$ & 12 & $(18,5)$ & 6 & 2 & YES & YES & YES & $1.50$ & $(2,3)$ & 7200 & 8851\\
$(236,49)$ & 14 & $(19,4)$ & 7 & 1 & YES & YES & NO(2) & $1.50$ & $(6,1)$ & NO & 8852\\
$(236,69)$ & 12 & $(31,9)$ & 8 & 1 & YES & YES & YES & $1.57$ & $(2,3)$ & NO & 8853\\
$(236,65)$ & 12 & $(40,11)$ & 8 & 4 & YES & YES & YES & $1.56$ & $(2,3)$ & NO & 8854\\
$(236,69)$ & 12 & $(41,12)$ & 8 & 1 & YES & YES & YES & $1.56$ & $(2,3)$ & NO & 8855\\
$(236,65)$ & 12 & $(51,14)$ & 9 & 1 & YES & YES & YES & $1.75$ & $(2,3)$ & NO & 8856\\
$(236,69)$ & 12 & $(58,17)$ & 9 & 2 & YES & YES & YES & $1.78$ & $(2,3)$ & 10863 & 8857\\
$(236,69)$ & 12 & $(65,19)$ & 9 & 1 & YES & YES & YES & $1.29$ & $(4,2)$ & NO & 8858\\
$(236,69)$ & 12 & $(106,31)$ & 10 & 2 & YES & YES & YES & $1.71$ & $(2,3)$ & 9834 & 8859\\
$(236,69)$ & 12 & $(147,43)$ & 11 & 1 & YES & YES & YES & $1.71$ & $(2,3)$ & 11474 & 8860\\
$(236,69)$ & 12 & $(171,50)$ & 11 & 1 & YES & YES & NO(3) & $1.29$ & $(2,3)$ & NO & 8861\\
$(236,65)$ & 12 & $(236,65)$ & 12 & 236 & YES & YES & NO(2) & $1.56$ & $(2,3)$ & NO & 8862\\
$(237,104)$ & 12 & $(2,1)$ & 1 & 1 & NO & YES & NO(2) & $1.60$ & $(2,3)$ & -- & 8863\\
$(237,100)$ & 12 & $(3,1)$ & 2 & 3 & YES & YES & YES & $1.70$ & $(2,3)$ & -- & 8864\\
$(237,104)$ & 12 & $(3,1)$ & 2 & 3 & YES & YES & YES & $1.62$ & $(4,2)$ & -- & 8865\\
$(237,104)$ & 12 & $(4,1)$ & 3 & 1 & YES & YES & NO(2) & $1.56$ & $(4,2)$ & -- & 8866\\
$(237,64)$ & 12 & $(5,2)$ & 3 & 1 & YES & YES & YES & $1.67$ & $(2,3)$ & NO & 8867\\
$(237,64)$ & 12 & $(5,2)$ & 3 & 1 & YES & YES & YES & $1.67$ & $(2,3)$ & -- & 8868\\
$(237,104)$ & 12 & $(5,2)$ & 3 & 1 & YES & YES & YES & $1.62$ & $(4,2)$ & NO & 8869\\
$(237,100)$ & 12 & $(7,3)$ & 4 & 1 & YES & YES & NO(2) & $1.70$ & $(6,1)$ & NO & 8870\\
$(237,100)$ & 12 & $(19,8)$ & 6 & 1 & YES & YES & NO(2) & $1.50$ & $(4,2)$ & NO & 8871\\
$(237,104)$ & 12 & $(57,25)$ & 9 & 3 & YES & YES & YES & $1.71$ & $(4,2)$ & NO & 8872\\
$(237,100)$ & 12 & $(109,46)$ & 10 & 1 & YES & YES & YES & $1.70$ & $(2,3)$ & 9911 & 8873\\
$(237,100)$ & 12 & $(173,73)$ & 11 & 1 & YES & YES & YES & $1.62$ & $(4,2)$ & NO & 8874\\
$(237,100)$ & 12 & $(237,100)$ & 12 & 237 & YES & YES & YES & $1.62$ & $(4,2)$ & NO & 8875\\
$(238,87)$ & 12 & $(5,1)$ & 4 & 1 & YES & YES & NO(2) & $1.38$ & $(6,1)$ & NO & 8876\\
$(239,71)$ & 12 & $(2,1)$ & 1 & 1 & YES & YES & NO(2) & $1.56$ & $(2,3)$ & -- & 8877\\
$(239,99)$ & 12 & $(2,1)$ & 1 & 1 & NO & YES & YES & $1.43$ & $(4,2)$ & -- & 8878\\
$(239,100)$ & 12 & $(2,1)$ & 1 & 1 & YES & YES & NO(2) & $1.38$ & $(4,2)$ & -- & 8879\\
$(239,101)$ & 12 & $(2,1)$ & 1 & 1 & YES & YES & YES & $1.71$ & $(2,3)$ & NO & 8880\\
$(239,105)$ & 12 & $(2,1)$ & 1 & 1 & NO & YES & NO(2) & $1.62$ & $(10,-1)$ & -- & 8881\\
$(239,66)$ & 12 & $(3,1)$ & 2 & 1 & YES & YES & NO(2) & $1.33$ & $(8,0)$ & NO & 8882\\
$(239,70)$ & 12 & $(3,1)$ & 2 & 1 & YES & YES & NO(2) & $1.44$ & $(4,2)$ & NO & 8883\\
$(239,70)$ & 12 & $(3,1)$ & 2 & 1 & YES & YES & YES & $1.50$ & $(2,3)$ & -- & 8884\\
$(239,71)$ & 12 & $(3,1)$ & 2 & 1 & YES & YES & YES & $1.14$ & $(4,2)$ & -- & 8885\\
$(239,98)$ & 12 & $(3,1)$ & 2 & 1 & YES & YES & YES & $2.00$ & $(2,3)$ & -- & 8886\\
$(239,99)$ & 12 & $(3,1)$ & 2 & 1 & YES & YES & YES & $1.89$ & $(2,3)$ & 7223 & 8887\\
$(239,99)$ & 12 & $(3,1)$ & 2 & 1 & YES & YES & YES & $1.62$ & $(4,2)$ & -- & 8888\\
$(239,100)$ & 12 & $(3,1)$ & 2 & 1 & YES & YES & YES & $1.88$ & $(2,3)$ & -- & 8889\\
$(239,64)$ & 13 & $(4,1)$ & 3 & 1 & YES & YES & NO(2) & $1.38$ & $(4,2)$ & -- & 8890\\
$(239,99)$ & 12 & $(4,1)$ & 3 & 1 & YES & YES & YES & $1.62$ & $(4,2)$ & NO & 8891\\
$(239,99)$ & 12 & $(4,1)$ & 3 & 1 & YES & YES & YES & $1.62$ & $(4,2)$ & -- & 8892\\
$(239,100)$ & 12 & $(4,1)$ & 3 & 1 & YES & YES & YES & $1.78$ & $(2,3)$ & -- & 8893\\
$(239,101)$ & 12 & $(4,1)$ & 3 & 1 & YES & YES & YES & $1.43$ & $(4,2)$ & -- & 8894\\
$(239,66)$ & 12 & $(5,2)$ & 3 & 1 & YES & YES & YES & $1.75$ & $(4,2)$ & NO & 8895\\
$(239,67)$ & 13 & $(5,1)$ & 4 & 1 & YES & YES & NO(2) & $1.44$ & $(4,2)$ & -- & 8896\\
$(239,71)$ & 12 & $(5,2)$ & 3 & 1 & YES & YES & YES & $1.57$ & $(4,2)$ & NO & 8897\\
$(239,71)$ & 12 & $(5,2)$ & 3 & 1 & YES & YES & YES & $1.50$ & $(4,2)$ & -- & 8898\\
$(239,99)$ & 12 & $(5,2)$ & 3 & 1 & YES & YES & NO(2) & $1.56$ & $(4,2)$ & NO & 8899\\
$(239,66)$ & 12 & $(7,2)$ & 4 & 1 & YES & YES & NO(2) & $1.44$ & $(8,0)$ & NO & 8900\\
$(239,99)$ & 12 & $(7,3)$ & 4 & 1 & YES & YES & YES & $1.89$ & $(2,3)$ & NO & 8901\\
$(239,100)$ & 12 & $(7,2)$ & 4 & 1 & YES & YES & YES & $1.67$ & $(4,2)$ & NO & 8902\\
$(239,66)$ & 12 & $(9,2)$ & 5 & 1 & YES & YES & YES & $1.75$ & $(2,3)$ & NO & 8903\\
$(239,66)$ & 12 & $(10,3)$ & 5 & 1 & YES & YES & YES & $1.78$ & $(2,3)$ & NO & 8904\\
$(239,70)$ & 12 & $(13,4)$ & 6 & 1 & YES & YES & YES & $1.57$ & $(4,2)$ & NO & 8905\\
$(239,99)$ & 12 & $(17,7)$ & 6 & 1 & YES & YES & YES & $1.62$ & $(4,2)$ & 7128 & 8906\\
$(239,100)$ & 12 & $(17,7)$ & 6 & 1 & YES & YES & YES & $1.67$ & $(4,2)$ & NO & 8907\\
$(239,56)$ & 13 & $(22,5)$ & 7 & 1 & YES & YES & YES & $1.78$ & $(4,2)$ & NO & 8908\\
$(239,71)$ & 12 & $(24,7)$ & 7 & 1 & YES & YES & YES & $1.70$ & $(2,3)$ & NO & 8909\\
$(239,71)$ & 12 & $(27,8)$ & 7 & 1 & YES & YES & NO(2) & $1.44$ & $(2,3)$ & 6405 & 8910\\
$(239,71)$ & 12 & $(37,11)$ & 8 & 1 & YES & YES & YES & $1.43$ & $(2,3)$ & 7342 & 8911\\
$(239,64)$ & 13 & $(41,11)$ & 8 & 1 & YES & YES & NO(2) & $1.50$ & $(4,2)$ & NO & 8912\\
$(239,99)$ & 12 & $(41,17)$ & 8 & 1 & YES & YES & YES & $1.62$ & $(4,2)$ & NO & 8913\\
$(239,56)$ & 13 & $(43,10)$ & 9 & 1 & YES & YES & YES & $1.78$ & $(4,2)$ & NO & 8914\\
$(239,85)$ & 14 & $(45,16)$ & 9 & 1 & YES & YES & YES & $1.71$ & $(4,2)$ & NO & 8915\\
$(239,70)$ & 12 & $(58,17)$ & 9 & 1 & YES & YES & YES & $1.50$ & $(2,3)$ & NO & 8916\\
$(239,98)$ & 12 & $(61,25)$ & 9 & 1 & YES & YES & YES & $1.89$ & $(2,3)$ & NO & 8917\\
$(239,66)$ & 12 & $(65,18)$ & 9 & 1 & YES & YES & YES & $1.75$ & $(2,3)$ & NO & 8918\\
$(239,56)$ & 13 & $(81,19)$ & 11 & 1 & YES & YES & YES & $1.57$ & $(2,3)$ & NO & 8919\\
$(239,67)$ & 13 & $(82,23)$ & 10 & 1 & YES & YES & YES & $1.71$ & $(2,3)$ & NO & 8920\\
$(239,99)$ & 12 & $(99,41)$ & 10 & 1 & YES & YES & YES & $1.67$ & $(2,3)$ & 9638 & 8921\\
$(239,70)$ & 12 & $(157,46)$ & 11 & 1 & YES & YES & YES & $1.78$ & $(2,3)$ & NO & 8922\\
$(239,101)$ & 12 & $(168,71)$ & 11 & 1 & YES & YES & YES & $1.43$ & $(4,2)$ & NO & 8923\\
$(239,71)$ & 12 & $(175,52)$ & 12 & 1 & YES & YES & YES & $1.43$ & $(4,2)$ & NO & 8924\\
$(239,99)$ & 12 & $(239,99)$ & 12 & 239 & YES & YES & YES & $1.75$ & $(4,2)$ & NO & 8925\\
$(240,71)$ & 12 & $(2,1)$ & 1 & 2 & YES & YES & NO(2) & $1.56$ & $(2,3)$ & NO & 8926\\
$(240,71)$ & 12 & $(2,1)$ & 1 & 2 & YES & YES & YES & $1.44$ & $(2,3)$ & -- & 8927\\
$(240,89)$ & 12 & $(2,1)$ & 1 & 2 & YES & YES & NO(2) & $1.38$ & $(6,1)$ & -- & 8928\\
$(240,71)$ & 12 & $(3,1)$ & 2 & 3 & YES & YES & NO(2) & $1.56$ & $(2,3)$ & NO & 8929\\
$(240,89)$ & 12 & $(3,1)$ & 2 & 3 & YES & YES & NO(2) & $1.60$ & $(6,1)$ & -- & 8930\\
$(240,67)$ & 13 & $(4,1)$ & 3 & 4 & YES & YES & YES & $1.62$ & $(4,2)$ & -- & 8931\\
$(240,89)$ & 12 & $(4,1)$ & 3 & 4 & YES & YES & YES & $1.43$ & $(6,1)$ & -- & 8932\\
$(240,71)$ & 12 & $(5,2)$ & 3 & 5 & YES & YES & YES & $1.50$ & $(2,3)$ & -- & 8933\\
$(240,67)$ & 13 & $(6,1)$ & 5 & 6 & YES & YES & YES & $1.75$ & $(4,2)$ & NO & 8934\\
$(240,71)$ & 12 & $(10,3)$ & 5 & 10 & YES & YES & NO(2) & $1.56$ & $(2,3)$ & 6404 & 8935\\
$(240,71)$ & 12 & $(11,3)$ & 5 & 1 & YES & YES & YES & $1.50$ & $(2,3)$ & NO & 8936\\
$(240,71)$ & 12 & $(24,7)$ & 7 & 24 & YES & YES & YES & $1.50$ & $(2,3)$ & NO & 8937\\
$(240,71)$ & 12 & $(27,8)$ & 7 & 3 & YES & YES & YES & $1.86$ & $(6,1)$ & NO & 8938\\
$(240,71)$ & 12 & $(44,13)$ & 8 & 4 & YES & YES & YES & $1.62$ & $(2,3)$ & NO & 8939\\
$(240,71)$ & 12 & $(61,18)$ & 9 & 1 & YES & YES & YES & $1.43$ & $(2,3)$ & NO & 8940\\
$(240,89)$ & 12 & $(62,23)$ & 9 & 2 & YES & YES & NO(2) & $1.56$ & $(8,0)$ & 8353 & 8941\\
$(240,71)$ & 12 & $(240,71)$ & 12 & 240 & YES & YES & YES & $1.57$ & $(2,3)$ & NO & 8942\\
$(240,89)$ & 12 & $(240,89)$ & 12 & 240 & YES & YES & YES & $1.43$ & $(6,1)$ & NO & 8943\\
$(241,89)$ & 12 & $(2,1)$ & 1 & 1 & YES & YES & YES & $1.56$ & $(2,3)$ & -- & 8944\\
$(241,65)$ & 12 & $(3,1)$ & 2 & 1 & YES & YES & YES & $1.43$ & $(4,2)$ & NO & 8945\\
$(241,89)$ & 12 & $(3,1)$ & 2 & 1 & YES & YES & YES & $1.70$ & $(2,3)$ & -- & 8946\\
$(241,94)$ & 12 & $(3,1)$ & 2 & 1 & YES & YES & NO(2) & $1.67$ & $(8,0)$ & NO & 8947\\
$(241,94)$ & 12 & $(3,1)$ & 2 & 1 & YES & YES & YES & $1.90$ & $(2,3)$ & NO & 8948\\
$(241,94)$ & 12 & $(3,1)$ & 2 & 1 & YES & YES & YES & $1.90$ & $(2,3)$ & -- & 8949\\
$(241,100)$ & 12 & $(3,1)$ & 2 & 1 & YES & YES & YES & $1.57$ & $(4,2)$ & -- & 8950\\
$(241,101)$ & 12 & $(3,1)$ & 2 & 1 & YES & YES & YES & $1.50$ & $(2,3)$ & -- & 8951\\
$(241,101)$ & 12 & $(3,1)$ & 2 & 1 & YES & YES & YES & $1.62$ & $(4,2)$ & NO & 8952\\
$(241,89)$ & 12 & $(4,1)$ & 3 & 1 & YES & YES & YES & $1.29$ & $(4,2)$ & -- & 8953\\
$(241,101)$ & 12 & $(4,1)$ & 3 & 1 & YES & YES & YES & $1.62$ & $(2,3)$ & NO & 8954\\
$(241,105)$ & 12 & $(4,1)$ & 3 & 1 & YES & YES & YES & $1.57$ & $(4,2)$ & NO & 8955\\
$(241,51)$ & 13 & $(5,2)$ & 3 & 1 & YES & YES & YES & $1.62$ & $(2,3)$ & -- & 8956\\
$(241,52)$ & 13 & $(5,2)$ & 3 & 1 & YES & YES & YES & $1.90$ & $(2,3)$ & -- & 8957\\
$(241,52)$ & 13 & $(5,2)$ & 3 & 1 & YES & YES & YES & $1.89$ & $(4,2)$ & NO & 8958\\
$(241,52)$ & 13 & $(5,2)$ & 3 & 1 & YES & YES & YES & $1.71$ & $(4,2)$ & NO & 8959\\
$(241,65)$ & 12 & $(5,2)$ & 3 & 1 & YES & YES & YES & $1.78$ & $(2,3)$ & -- & 8960\\
$(241,65)$ & 12 & $(5,2)$ & 3 & 1 & YES & YES & YES & $1.71$ & $(2,3)$ & NO & 8961\\
$(241,89)$ & 12 & $(5,1)$ & 4 & 1 & YES & YES & YES & $1.70$ & $(2,3)$ & NO & 8962\\
$(241,94)$ & 12 & $(5,1)$ & 4 & 1 & YES & YES & NO(2) & $1.44$ & $(8,0)$ & -- & 8963\\
$(241,100)$ & 12 & $(5,2)$ & 3 & 1 & YES & YES & YES & $1.90$ & $(2,3)$ & -- & 8964\\
$(241,101)$ & 12 & $(5,1)$ & 4 & 1 & YES & YES & YES & $1.57$ & $(2,3)$ & -- & 8965\\
$(241,105)$ & 12 & $(5,2)$ & 3 & 1 & YES & YES & YES & $1.43$ & $(4,2)$ & NO & 8966\\
$(241,51)$ & 13 & $(7,2)$ & 4 & 1 & YES & YES & YES & $1.62$ & $(2,3)$ & -- & 8967\\
$(241,52)$ & 13 & $(7,2)$ & 4 & 1 & YES & YES & YES & $1.57$ & $(4,2)$ & -- & 8968\\
$(241,94)$ & 12 & $(8,3)$ & 4 & 1 & YES & YES & YES & $1.75$ & $(4,2)$ & NO & 8969\\
$(241,65)$ & 12 & $(10,3)$ & 5 & 1 & YES & YES & YES & $1.71$ & $(2,3)$ & NO & 8970\\
$(241,51)$ & 13 & $(13,3)$ & 6 & 1 & YES & YES & YES & $1.62$ & $(2,3)$ & NO & 8971\\
$(241,52)$ & 13 & $(13,3)$ & 6 & 1 & YES & YES & YES & $2.00$ & $(2,3)$ & NO & 8972\\
$(241,94)$ & 12 & $(13,5)$ & 5 & 1 & YES & YES & YES & $1.75$ & $(4,2)$ & NO & 8973\\
$(241,65)$ & 12 & $(19,5)$ & 7 & 1 & YES & YES & YES & $1.62$ & $(4,2)$ & NO & 8974\\
$(241,51)$ & 13 & $(23,5)$ & 7 & 1 & YES & YES & YES & $1.62$ & $(2,3)$ & NO & 8975\\
$(241,100)$ & 12 & $(29,12)$ & 7 & 1 & YES & YES & YES & $1.62$ & $(2,3)$ & NO & 8976\\
$(241,89)$ & 12 & $(46,17)$ & 8 & 1 & YES & YES & YES & $1.56$ & $(2,3)$ & NO & 8977\\
$(241,100)$ & 12 & $(53,22)$ & 9 & 1 & YES & YES & YES & $1.43$ & $(4,2)$ & 8030 & 8978\\
$(241,52)$ & 13 & $(60,13)$ & 9 & 1 & YES & YES & YES & $1.57$ & $(4,2)$ & NO & 8979\\
$(241,100)$ & 12 & $(94,39)$ & 10 & 1 & YES & YES & YES & $1.90$ & $(2,3)$ & NO & 8980\\
$(241,105)$ & 12 & $(101,44)$ & 10 & 1 & YES & YES & YES & $1.57$ & $(4,2)$ & NO & 8981\\
$(241,89)$ & 12 & $(111,41)$ & 10 & 1 & YES & YES & YES & $1.50$ & $(4,2)$ & 10044 & 8982\\
$(241,89)$ & 12 & $(176,65)$ & 11 & 1 & YES & YES & YES & $1.80$ & $(2,3)$ & NO & 8983\\
$(241,92)$ & 12 & $(241,92)$ & 12 & 241 & YES & YES & YES & $1.62$ & $(2,3)$ & NO & 8984\\
$(241,100)$ & 12 & $(241,100)$ & 12 & 241 & YES & YES & YES & $1.57$ & $(4,2)$ & NO & 8985\\
$(241,101)$ & 12 & $(241,101)$ & 12 & 241 & YES & YES & YES & $1.50$ & $(2,3)$ & NO & 8986\\
$(242,67)$ & 12 & $(2,1)$ & 1 & 2 & YES & YES & NO(2) & $1.56$ & $(2,3)$ & -- & 8987\\
$(242,45)$ & 14 & $(3,1)$ & 2 & 1 & YES & YES & YES & $1.43$ & $(2,3)$ & NO & 8988\\
$(242,65)$ & 12 & $(3,1)$ & 2 & 1 & YES & YES & YES & $1.67$ & $(2,3)$ & NO & 8989\\
$(242,67)$ & 12 & $(3,1)$ & 2 & 1 & YES & YES & YES & $1.62$ & $(2,3)$ & NO & 8990\\
$(242,67)$ & 12 & $(3,1)$ & 2 & 1 & YES & YES & YES & $1.62$ & $(4,2)$ & -- & 8991\\
$(242,71)$ & 13 & $(3,1)$ & 2 & 1 & YES & YES & YES & $1.71$ & $(2,3)$ & -- & 8992\\
$(242,87)$ & 12 & $(3,1)$ & 2 & 1 & YES & YES & YES & $1.62$ & $(4,2)$ & -- & 8993\\
$(242,89)$ & 12 & $(3,1)$ & 2 & 1 & YES & YES & NO(2) & $1.56$ & $(4,2)$ & -- & 8994\\
$(242,45)$ & 14 & $(5,2)$ & 3 & 1 & YES & YES & YES & $1.75$ & $(4,2)$ & NO & 8995\\
$(242,45)$ & 14 & $(5,2)$ & 3 & 1 & YES & YES & YES & $1.88$ & $(4,2)$ & NO & 8996\\
$(242,45)$ & 14 & $(5,2)$ & 3 & 1 & YES & YES & YES & $1.75$ & $(4,2)$ & -- & 8997\\
$(242,65)$ & 12 & $(5,2)$ & 3 & 1 & YES & YES & YES & $1.43$ & $(4,2)$ & NO & 8998\\
$(242,65)$ & 12 & $(5,2)$ & 3 & 1 & YES & YES & YES & $1.71$ & $(2,3)$ & -- & 8999\\
$(242,67)$ & 12 & $(5,1)$ & 4 & 1 & YES & YES & YES & $1.38$ & $(2,3)$ & NO & 9000\\
$(242,67)$ & 12 & $(5,2)$ & 3 & 1 & YES & YES & YES & $1.78$ & $(2,3)$ & NO & 9001\\
$(242,67)$ & 12 & $(5,2)$ & 3 & 1 & YES & YES & YES & $1.38$ & $(4,2)$ & -- & 9002\\
$(242,71)$ & 13 & $(5,1)$ & 4 & 1 & YES & YES & YES & $1.57$ & $(2,3)$ & NO & 9003\\
$(242,65)$ & 12 & $(7,2)$ & 4 & 1 & YES & YES & YES & $1.86$ & $(2,3)$ & -- & 9004\\
$(242,87)$ & 12 & $(8,3)$ & 4 & 2 & YES & YES & YES & $1.62$ & $(4,2)$ & NO & 9005\\
$(242,67)$ & 12 & $(10,3)$ & 5 & 2 & YES & YES & YES & $1.78$ & $(2,3)$ & NO & 9006\\
$(242,67)$ & 12 & $(11,3)$ & 5 & 11 & YES & YES & YES & $1.50$ & $(2,3)$ & NO & 9007\\
$(242,65)$ & 12 & $(18,5)$ & 6 & 2 & YES & YES & YES & $1.86$ & $(2,3)$ & NO & 9008\\
$(242,67)$ & 12 & $(25,7)$ & 7 & 1 & YES & YES & YES & $1.75$ & $(2,3)$ & NO & 9009\\
$(242,67)$ & 12 & $(29,8)$ & 7 & 1 & YES & YES & YES & $1.78$ & $(2,3)$ & NO & 9010\\
$(242,65)$ & 12 & $(37,10)$ & 8 & 1 & YES & YES & YES & $1.50$ & $(4,2)$ & NO & 9011\\
$(242,67)$ & 12 & $(47,13)$ & 8 & 1 & YES & YES & YES & $1.67$ & $(2,3)$ & NO & 9012\\
$(242,65)$ & 12 & $(56,15)$ & 9 & 2 & YES & YES & YES & $1.75$ & $(2,3)$ & 10756 & 9013\\
$(242,67)$ & 12 & $(76,21)$ & 9 & 2 & YES & YES & YES & $1.57$ & $(2,3)$ & 11506 & 9014\\
$(242,65)$ & 12 & $(242,65)$ & 12 & 242 & YES & YES & YES & $1.43$ & $(6,1)$ & NO & 9015\\
$(242,67)$ & 12 & $(242,67)$ & 12 & 242 & YES & YES & YES & $1.38$ & $(2,3)$ & NO & 9016\\
$(242,87)$ & 12 & $(242,87)$ & 12 & 242 & YES & YES & YES & $1.62$ & $(4,2)$ & NO & 9017\\
$(243,71)$ & 12 & $(2,1)$ & 1 & 1 & YES & YES & NO(2) & $1.33$ & $(4,2)$ & -- & 9018\\
$(243,94)$ & 12 & $(2,1)$ & 1 & 1 & YES & YES & YES & $1.67$ & $(2,3)$ & -- & 9019\\
$(243,106)$ & 12 & $(2,1)$ & 1 & 1 & NO & YES & NO(2) & $1.64$ & $(4,2)$ & -- & 9020\\
$(243,46)$ & 15 & $(3,1)$ & 2 & 3 & YES & YES & NO(2) & $1.62$ & $(6,1)$ & -- & 9021\\
$(243,46)$ & 15 & $(3,1)$ & 2 & 3 & YES & YES & NO(2) & $1.56$ & $(8,0)$ & NO & 9022\\
$(243,55)$ & 13 & $(3,1)$ & 2 & 3 & YES & YES & YES & $1.29$ & $(4,2)$ & -- & 9023\\
$(243,71)$ & 12 & $(3,1)$ & 2 & 3 & YES & YES & YES & $1.78$ & $(2,3)$ & -- & 9024\\
$(243,89)$ & 12 & $(3,1)$ & 2 & 3 & YES & YES & YES & $1.89$ & $(2,3)$ & -- & 9025\\
$(243,94)$ & 12 & $(3,1)$ & 2 & 3 & YES & YES & NO(2) & $1.67$ & $(8,0)$ & NO & 9026\\
$(243,106)$ & 12 & $(3,1)$ & 2 & 3 & YES & YES & YES & $1.62$ & $(4,2)$ & -- & 9027\\
$(243,106)$ & 12 & $(3,1)$ & 2 & 3 & YES & YES & YES & $1.62$ & $(4,2)$ & NO & 9028\\
$(243,71)$ & 12 & $(4,1)$ & 3 & 1 & YES & YES & NO(2) & $1.56$ & $(2,3)$ & -- & 9029\\
$(243,71)$ & 12 & $(4,1)$ & 3 & 1 & YES & YES & NO(2) & $1.67$ & $(2,3)$ & NO & 9030\\
$(243,94)$ & 12 & $(4,1)$ & 3 & 1 & YES & YES & YES & $1.60$ & $(2,3)$ & -- & 9031\\
$(243,46)$ & 15 & $(5,2)$ & 3 & 1 & YES & YES & NO(2) & $1.60$ & $(6,1)$ & -- & 9032\\
$(243,53)$ & 13 & $(5,2)$ & 3 & 1 & YES & YES & YES & $1.67$ & $(2,3)$ & -- & 9033\\
$(243,58)$ & 14 & $(5,2)$ & 3 & 1 & YES & YES & NO(2) & $1.60$ & $(6,1)$ & -- & 9034\\
$(243,71)$ & 12 & $(5,1)$ & 4 & 1 & YES & YES & NO(2) & $1.67$ & $(2,3)$ & NO & 9035\\
$(243,71)$ & 12 & $(5,2)$ & 3 & 1 & YES & YES & YES & $1.50$ & $(4,2)$ & -- & 9036\\
$(243,71)$ & 12 & $(5,2)$ & 3 & 1 & YES & YES & YES & $1.57$ & $(2,3)$ & NO & 9037\\
$(243,89)$ & 12 & $(5,2)$ & 3 & 1 & YES & YES & YES & $1.75$ & $(4,2)$ & NO & 9038\\
$(243,89)$ & 12 & $(5,2)$ & 3 & 1 & YES & YES & YES & $1.75$ & $(4,2)$ & -- & 9039\\
$(243,89)$ & 12 & $(5,2)$ & 3 & 1 & YES & YES & YES & $1.67$ & $(2,3)$ & NO & 9040\\
$(243,92)$ & 12 & $(5,1)$ & 4 & 1 & YES & YES & NO(2) & $1.50$ & $(6,1)$ & NO & 9041\\
$(243,94)$ & 12 & $(5,2)$ & 3 & 1 & YES & YES & YES & $1.88$ & $(4,2)$ & NO & 9042\\
$(243,71)$ & 12 & $(7,2)$ & 4 & 1 & YES & YES & YES & $1.75$ & $(2,3)$ & -- & 9043\\
$(243,71)$ & 12 & $(9,2)$ & 5 & 9 & YES & YES & YES & $1.62$ & $(4,2)$ & NO & 9044\\
$(243,94)$ & 12 & $(9,2)$ & 5 & 9 & YES & YES & YES & $1.62$ & $(4,2)$ & -- & 9045\\
$(243,71)$ & 12 & $(10,3)$ & 5 & 1 & YES & YES & NO(2) & $1.25$ & $(4,2)$ & NO & 9046\\
$(243,46)$ & 15 & $(11,2)$ & 6 & 1 & YES & YES & NO(2) & $1.50$ & $(6,1)$ & NO & 9047\\
$(243,71)$ & 12 & $(11,3)$ & 5 & 1 & YES & YES & YES & $1.43$ & $(4,2)$ & NO & 9048\\
$(243,71)$ & 12 & $(13,4)$ & 6 & 1 & YES & YES & YES & $1.75$ & $(2,3)$ & NO & 9049\\
$(243,89)$ & 12 & $(13,5)$ & 5 & 1 & YES & YES & YES & $1.75$ & $(4,2)$ & 11490 & 9050\\
$(243,71)$ & 12 & $(17,5)$ & 6 & 1 & YES & YES & YES & $1.50$ & $(2,3)$ & NO & 9051\\
$(243,89)$ & 12 & $(27,10)$ & 7 & 27 & YES & YES & YES & $1.75$ & $(4,2)$ & NO & 9052\\
$(243,55)$ & 13 & $(35,8)$ & 8 & 1 & YES & YES & YES & $1.62$ & $(4,2)$ & NO & 9053\\
$(243,94)$ & 12 & $(49,19)$ & 8 & 1 & YES & YES & YES & $1.75$ & $(4,2)$ & NO & 9054\\
$(243,71)$ & 12 & $(58,17)$ & 9 & 1 & YES & YES & YES & $1.57$ & $(2,3)$ & 11398 & 9055\\
$(243,71)$ & 12 & $(65,19)$ & 9 & 1 & YES & YES & YES & $1.50$ & $(2,3)$ & 8481 & 9056\\
$(243,94)$ & 12 & $(75,29)$ & 9 & 3 & YES & YES & YES & $1.62$ & $(2,3)$ & NO & 9057\\
$(243,71)$ & 12 & $(89,26)$ & 10 & 1 & YES & YES & NO(2) & $1.67$ & $(2,3)$ & NO & 9058\\
$(243,71)$ & 12 & $(106,31)$ & 10 & 1 & YES & YES & YES & $1.57$ & $(2,3)$ & NO & 9059\\
$(243,94)$ & 12 & $(106,41)$ & 10 & 1 & YES & YES & YES & $1.67$ & $(2,3)$ & NO & 9060\\
$(243,71)$ & 12 & $(113,33)$ & 11 & 1 & YES & YES & YES & $1.43$ & $(4,2)$ & NO & 9061\\
$(243,94)$ & 12 & $(137,53)$ & 11 & 1 & YES & YES & YES & $1.60$ & $(2,3)$ & NO & 9062\\
$(243,106)$ & 12 & $(149,65)$ & 11 & 1 & YES & YES & YES & $1.62$ & $(4,2)$ & NO & 9063\\
$(243,94)$ & 12 & $(181,70)$ & 11 & 1 & YES & YES & YES & $1.62$ & $(4,2)$ & NO & 9064\\
$(243,71)$ & 12 & $(243,71)$ & 12 & 243 & YES & YES & NO(2) & $1.67$ & $(2,3)$ & NO & 9065\\
$(243,92)$ & 12 & $(243,92)$ & 12 & 243 & YES & YES & NO(2) & $1.50$ & $(6,1)$ & NO & 9066\\
$(243,94)$ & 12 & $(243,94)$ & 12 & 243 & YES & YES & YES & $1.50$ & $(4,2)$ & NO & 9067\\
$(244,71)$ & 13 & $(5,1)$ & 4 & 1 & YES & YES & YES & $1.29$ & $(4,2)$ & NO & 9068\\
$(244,71)$ & 13 & $(5,2)$ & 3 & 1 & YES & YES & YES & $1.90$ & $(2,3)$ & -- & 9069\\
$(244,71)$ & 13 & $(79,23)$ & 10 & 1 & YES & YES & YES & $1.57$ & $(4,2)$ & NO & 9070\\
$(245,69)$ & 13 & $(2,1)$ & 1 & 1 & YES & YES & NO(2) & $1.56$ & $(4,2)$ & NO & 9071\\
$(245,93)$ & 12 & $(3,1)$ & 2 & 1 & YES & YES & YES & $1.71$ & $(4,2)$ & -- & 9072\\
$(245,69)$ & 13 & $(5,1)$ & 4 & 5 & YES & YES & YES & $1.57$ & $(2,3)$ & NO & 9073\\
$(245,103)$ & 12 & $(5,1)$ & 4 & 5 & YES & YES & YES & $1.75$ & $(2,3)$ & NO & 9074\\
$(245,103)$ & 12 & $(5,1)$ & 4 & 5 & YES & YES & YES & $1.75$ & $(2,3)$ & -- & 9075\\
$(245,93)$ & 12 & $(11,4)$ & 5 & 1 & YES & YES & YES & $1.75$ & $(4,2)$ & NO & 9076\\
$(245,93)$ & 12 & $(50,19)$ & 8 & 5 & YES & YES & YES & $1.57$ & $(4,2)$ & 6392 & 9077\\
$(245,108)$ & 12 & $(59,26)$ & 9 & 1 & YES & YES & NO(2) & $1.60$ & $(10,-1)$ & 8281 & 9078\\
$(245,103)$ & 12 & $(69,29)$ & 9 & 1 & YES & YES & YES & $1.89$ & $(2,3)$ & 8599 & 9079\\
$(245,69)$ & 13 & $(103,29)$ & 11 & 1 & YES & YES & YES & $1.57$ & $(2,3)$ & 9835 & 9080\\
$(245,103)$ & 12 & $(157,66)$ & 11 & 1 & YES & YES & YES & $1.62$ & $(2,3)$ & NO & 9081\\
$(245,93)$ & 12 & $(245,93)$ & 12 & 245 & YES & YES & YES & $1.75$ & $(4,2)$ & NO & 9082\\
$(246,65)$ & 13 & $(2,1)$ & 1 & 2 & YES & YES & YES & $1.57$ & $(2,3)$ & NO & 9083\\
$(246,73)$ & 12 & $(2,1)$ & 1 & 2 & YES & YES & YES & $1.50$ & $(2,3)$ & -- & 9084\\
$(246,73)$ & 12 & $(2,1)$ & 1 & 2 & YES & YES & YES & $1.43$ & $(4,2)$ & NO & 9085\\
$(246,95)$ & 12 & $(2,1)$ & 1 & 2 & YES & YES & YES & $1.89$ & $(2,3)$ & -- & 9086\\
$(246,65)$ & 13 & $(3,1)$ & 2 & 3 & YES & YES & YES & $1.57$ & $(2,3)$ & NO & 9087\\
$(246,73)$ & 12 & $(3,1)$ & 2 & 3 & YES & YES & NO(2) & $1.73$ & $(4,2)$ & NO & 9088\\
$(246,73)$ & 12 & $(3,1)$ & 2 & 3 & YES & YES & YES & $1.67$ & $(2,3)$ & -- & 9089\\
$(246,91)$ & 12 & $(3,1)$ & 2 & 3 & YES & YES & NO(2) & $1.56$ & $(8,0)$ & -- & 9090\\
$(246,95)$ & 12 & $(3,1)$ & 2 & 3 & YES & YES & YES & $1.89$ & $(2,3)$ & NO & 9091\\
$(246,95)$ & 12 & $(3,1)$ & 2 & 3 & YES & YES & YES & $1.62$ & $(4,2)$ & -- & 9092\\
$(246,101)$ & 12 & $(3,1)$ & 2 & 3 & YES & YES & YES & $1.57$ & $(4,2)$ & NO & 9093\\
$(246,101)$ & 12 & $(3,1)$ & 2 & 3 & YES & YES & YES & $1.90$ & $(2,3)$ & -- & 9094\\
$(246,65)$ & 13 & $(4,1)$ & 3 & 2 & YES & YES & NO(2) & $1.78$ & $(4,2)$ & NO & 9095\\
$(246,95)$ & 12 & $(4,1)$ & 3 & 2 & YES & YES & YES & $1.78$ & $(2,3)$ & NO & 9096\\
$(246,95)$ & 12 & $(4,1)$ & 3 & 2 & YES & YES & YES & $1.70$ & $(2,3)$ & -- & 9097\\
$(246,101)$ & 12 & $(4,1)$ & 3 & 2 & YES & YES & YES & $1.88$ & $(2,3)$ & -- & 9098\\
$(246,91)$ & 12 & $(5,2)$ & 3 & 1 & YES & YES & YES & $1.75$ & $(4,2)$ & -- & 9099\\
$(246,73)$ & 12 & $(7,2)$ & 4 & 1 & YES & YES & YES & $1.50$ & $(2,3)$ & NO & 9100\\
$(246,95)$ & 12 & $(7,2)$ & 4 & 1 & YES & YES & YES & $1.78$ & $(4,2)$ & -- & 9101\\
$(246,101)$ & 12 & $(7,3)$ & 4 & 1 & YES & YES & YES & $1.62$ & $(2,3)$ & NO & 9102\\
$(246,73)$ & 12 & $(9,2)$ & 5 & 3 & YES & YES & YES & $1.62$ & $(4,2)$ & NO & 9103\\
$(246,91)$ & 12 & $(11,4)$ & 5 & 1 & YES & YES & YES & $1.62$ & $(4,2)$ & NO & 9104\\
$(246,65)$ & 13 & $(15,4)$ & 6 & 3 & YES & YES & YES & $1.43$ & $(2,3)$ & NO & 9105\\
$(246,95)$ & 12 & $(18,7)$ & 6 & 6 & YES & YES & YES & $1.50$ & $(2,3)$ & NO & 9106\\
$(246,91)$ & 12 & $(19,7)$ & 6 & 1 & YES & YES & YES & $1.89$ & $(2,3)$ & NO & 9107\\
$(246,101)$ & 12 & $(22,9)$ & 7 & 2 & YES & YES & YES & $1.89$ & $(2,3)$ & NO & 9108\\
$(246,95)$ & 12 & $(23,9)$ & 7 & 1 & YES & YES & YES & $1.75$ & $(4,2)$ & NO & 9109\\
$(246,95)$ & 12 & $(31,12)$ & 7 & 1 & YES & YES & YES & $1.60$ & $(2,3)$ & NO & 9110\\
$(246,65)$ & 13 & $(34,9)$ & 8 & 2 & YES & YES & YES & $1.57$ & $(2,3)$ & NO & 9111\\
$(246,73)$ & 12 & $(37,11)$ & 8 & 1 & YES & YES & YES & $1.62$ & $(2,3)$ & 9598 & 9112\\
$(246,101)$ & 12 & $(39,16)$ & 8 & 3 & YES & YES & YES & $2.00$ & $(2,3)$ & NO & 9113\\
$(246,73)$ & 12 & $(44,13)$ & 8 & 2 & YES & YES & YES & $1.62$ & $(2,3)$ & NO & 9114\\
$(246,95)$ & 12 & $(57,22)$ & 9 & 3 & YES & YES & YES & $1.78$ & $(2,3)$ & NO & 9115\\
$(246,101)$ & 12 & $(95,39)$ & 10 & 1 & YES & YES & YES & $1.90$ & $(2,3)$ & NO & 9116\\
$(246,95)$ & 12 & $(101,39)$ & 10 & 1 & YES & YES & YES & $1.89$ & $(2,3)$ & NO & 9117\\
$(246,73)$ & 12 & $(118,35)$ & 11 & 2 & YES & YES & YES & $1.57$ & $(2,3)$ & NO & 9118\\
$(246,101)$ & 12 & $(151,62)$ & 11 & 1 & YES & YES & YES & $1.75$ & $(2,3)$ & NO & 9119\\
$(246,91)$ & 12 & $(173,64)$ & 11 & 1 & YES & YES & YES & $1.56$ & $(2,3)$ & NO & 9120\\
$(246,73)$ & 12 & $(219,65)$ & 12 & 3 & YES & YES & YES & $1.50$ & $(2,3)$ & NO & 9121\\
$(246,91)$ & 12 & $(246,91)$ & 12 & 246 & YES & YES & NO(2) & $1.60$ & $(6,1)$ & NO & 9122\\
$(246,95)$ & 12 & $(246,95)$ & 12 & 246 & YES & YES & YES & $1.62$ & $(2,3)$ & NO & 9123\\
$(246,101)$ & 12 & $(246,101)$ & 12 & 246 & YES & YES & YES & $1.43$ & $(4,2)$ & NO & 9124\\
$(247,68)$ & 12 & $(2,1)$ & 1 & 1 & YES & YES & YES & $1.67$ & $(2,3)$ & -- & 9125\\
$(247,68)$ & 12 & $(2,1)$ & 1 & 1 & YES & YES & NO(2) & $1.80$ & $(2,3)$ & NO & 9126\\
$(247,69)$ & 12 & $(2,1)$ & 1 & 1 & YES & YES & YES & $1.70$ & $(2,3)$ & -- & 9127\\
$(247,111)$ & 13 & $(2,1)$ & 1 & 1 & NO & YES & NO(2) & $1.50$ & $(6,1)$ & -- & 9128\\
$(247,111)$ & 13 & $(2,1)$ & 1 & 1 & YES & YES & NO(2) & $1.70$ & $(6,1)$ & NO & 9129\\
$(247,68)$ & 12 & $(3,1)$ & 2 & 1 & YES & YES & YES & $1.78$ & $(2,3)$ & -- & 9130\\
$(247,68)$ & 12 & $(3,1)$ & 2 & 1 & YES & YES & NO(2) & $1.78$ & $(4,2)$ & NO & 9131\\
$(247,69)$ & 12 & $(3,1)$ & 2 & 1 & YES & YES & YES & $1.50$ & $(2,3)$ & -- & 9132\\
$(247,69)$ & 12 & $(3,1)$ & 2 & 1 & YES & YES & YES & $1.62$ & $(2,3)$ & NO & 9133\\
$(247,68)$ & 12 & $(4,1)$ & 3 & 1 & YES & YES & YES & $1.38$ & $(2,3)$ & -- & 9134\\
$(247,69)$ & 12 & $(4,1)$ & 3 & 1 & YES & YES & YES & $1.38$ & $(2,3)$ & NO & 9135\\
$(247,68)$ & 12 & $(5,2)$ & 3 & 1 & YES & YES & YES & $1.50$ & $(2,3)$ & -- & 9136\\
$(247,68)$ & 12 & $(5,2)$ & 3 & 1 & YES & YES & YES & $1.50$ & $(4,2)$ & NO & 9137\\
$(247,69)$ & 12 & $(5,1)$ & 4 & 1 & YES & YES & YES & $1.43$ & $(2,3)$ & NO & 9138\\
$(247,69)$ & 12 & $(5,2)$ & 3 & 1 & YES & YES & YES & $1.71$ & $(2,3)$ & -- & 9139\\
$(247,69)$ & 12 & $(5,2)$ & 3 & 1 & YES & YES & YES & $1.71$ & $(2,3)$ & NO & 9140\\
$(247,102)$ & 13 & $(5,1)$ & 4 & 1 & YES & YES & YES & $1.75$ & $(4,2)$ & -- & 9141\\
$(247,111)$ & 13 & $(5,1)$ & 4 & 1 & YES & YES & NO(2) & $1.60$ & $(6,1)$ & NO & 9142\\
$(247,111)$ & 13 & $(5,1)$ & 4 & 1 & YES & YES & NO(2) & $1.60$ & $(6,1)$ & -- & 9143\\
$(247,56)$ & 13 & $(7,2)$ & 4 & 1 & YES & YES & YES & $1.43$ & $(4,2)$ & NO & 9144\\
$(247,68)$ & 12 & $(7,2)$ & 4 & 1 & YES & YES & YES & $1.71$ & $(2,3)$ & -- & 9145\\
$(247,68)$ & 12 & $(7,2)$ & 4 & 1 & YES & YES & NO(2) & $1.70$ & $(2,3)$ & NO & 9146\\
$(247,68)$ & 12 & $(10,3)$ & 5 & 1 & YES & YES & YES & $1.50$ & $(2,3)$ & NO & 9147\\
$(247,69)$ & 12 & $(10,3)$ & 5 & 1 & YES & YES & YES & $1.62$ & $(4,2)$ & NO & 9148\\
$(247,69)$ & 12 & $(11,3)$ & 5 & 1 & YES & YES & YES & $1.62$ & $(2,3)$ & NO & 9149\\
$(247,68)$ & 12 & $(18,5)$ & 6 & 1 & YES & YES & YES & $1.78$ & $(2,3)$ & NO & 9150\\
$(247,69)$ & 12 & $(18,5)$ & 6 & 1 & YES & YES & YES & $1.90$ & $(2,3)$ & NO & 9151\\
$(247,68)$ & 12 & $(25,7)$ & 7 & 1 & YES & YES & YES & $1.43$ & $(2,3)$ & NO & 9152\\
$(247,68)$ & 12 & $(29,8)$ & 7 & 1 & YES & YES & YES & $1.50$ & $(2,3)$ & NO & 9153\\
$(247,69)$ & 12 & $(29,8)$ & 7 & 1 & YES & YES & YES & $1.71$ & $(2,3)$ & NO & 9154\\
$(247,69)$ & 12 & $(32,9)$ & 8 & 1 & YES & YES & YES & $1.62$ & $(4,2)$ & NO & 9155\\
$(247,68)$ & 12 & $(47,13)$ & 8 & 1 & YES & YES & YES & $1.71$ & $(2,3)$ & NO & 9156\\
$(247,68)$ & 12 & $(51,14)$ & 9 & 1 & YES & YES & YES & $1.75$ & $(2,3)$ & 10519 & 9157\\
$(247,69)$ & 12 & $(61,17)$ & 9 & 1 & YES & YES & YES & $1.75$ & $(2,3)$ & 11015 & 9158\\
$(247,111)$ & 13 & $(89,40)$ & 11 & 1 & YES & YES & NO(2) & $1.70$ & $(6,1)$ & NO & 9159\\
$(247,68)$ & 12 & $(98,27)$ & 10 & 1 & YES & YES & YES & $1.57$ & $(2,3)$ & NO & 9160\\
$(247,56)$ & 13 & $(128,29)$ & 11 & 1 & YES & YES & NO(2) & $1.67$ & $(2,3)$ & NO & 9161\\
$(247,56)$ & 13 & $(181,41)$ & 12 & 1 & YES & YES & YES & $1.50$ & $(4,2)$ & NO & 9162\\
$(247,68)$ & 12 & $(247,68)$ & 12 & 247 & YES & YES & YES & $1.50$ & $(2,3)$ & NO & 9163\\
$(247,69)$ & 12 & $(247,69)$ & 12 & 247 & YES & YES & YES & $1.38$ & $(2,3)$ & NO & 9164\\
$(247,111)$ & 13 & $(247,111)$ & 13 & 247 & YES & YES & NO(2) & $1.60$ & $(6,1)$ & NO & 9165\\
$(248,91)$ & 12 & $(2,1)$ & 1 & 2 & YES & YES & NO(2) & $1.50$ & $(4,2)$ & NO & 9166\\
$(248,109)$ & 12 & $(4,1)$ & 3 & 4 & YES & YES & YES & $1.43$ & $(4,2)$ & NO & 9167\\
$(248,109)$ & 12 & $(4,1)$ & 3 & 4 & YES & YES & YES & $1.57$ & $(4,2)$ & -- & 9168\\
$(248,109)$ & 12 & $(16,7)$ & 6 & 8 & YES & YES & YES & $1.43$ & $(4,2)$ & NO & 9169\\
$(248,111)$ & 13 & $(105,47)$ & 11 & 1 & YES & YES & NO(2) & $1.70$ & $(6,1)$ & NO & 9170\\
$(249,56)$ & 15 & $(2,1)$ & 1 & 1 & YES & YES & NO(2) & $1.56$ & $(8,0)$ & -- & 9171\\
$(249,95)$ & 12 & $(2,1)$ & 1 & 1 & YES & YES & YES & $1.80$ & $(2,3)$ & -- & 9172\\
$(249,95)$ & 12 & $(2,1)$ & 1 & 1 & YES & YES & YES & $1.57$ & $(2,3)$ & NO & 9173\\
$(249,76)$ & 12 & $(3,1)$ & 2 & 3 & YES & YES & YES & $1.43$ & $(6,1)$ & -- & 9174\\
$(249,95)$ & 12 & $(3,1)$ & 2 & 3 & YES & YES & YES & $1.62$ & $(2,3)$ & -- & 9175\\
$(249,95)$ & 12 & $(4,1)$ & 3 & 1 & YES & YES & YES & $1.71$ & $(4,2)$ & NO & 9176\\
$(249,56)$ & 15 & $(5,1)$ & 4 & 1 & YES & YES & NO(2) & $1.67$ & $(8,0)$ & NO & 9177\\
$(249,58)$ & 13 & $(5,2)$ & 3 & 1 & YES & YES & YES & $1.67$ & $(2,3)$ & -- & 9178\\
$(249,58)$ & 13 & $(5,2)$ & 3 & 1 & YES & YES & YES & $1.80$ & $(2,3)$ & NO & 9179\\
$(249,76)$ & 12 & $(5,1)$ & 4 & 1 & YES & YES & YES & $1.62$ & $(4,2)$ & NO & 9180\\
$(249,76)$ & 12 & $(5,1)$ & 4 & 1 & YES & YES & YES & $1.62$ & $(4,2)$ & -- & 9181\\
$(249,77)$ & 13 & $(5,2)$ & 3 & 1 & YES & YES & YES & $1.57$ & $(6,1)$ & NO & 9182\\
$(249,77)$ & 13 & $(5,2)$ & 3 & 1 & YES & YES & YES & $1.57$ & $(6,1)$ & -- & 9183\\
$(249,76)$ & 12 & $(7,2)$ & 4 & 1 & YES & YES & YES & $1.43$ & $(6,1)$ & 10314 & 9184\\
$(249,76)$ & 12 & $(10,3)$ & 5 & 1 & YES & YES & YES & $1.43$ & $(6,1)$ & 8803 & 9185\\
$(249,95)$ & 12 & $(34,13)$ & 7 & 1 & YES & YES & YES & $1.62$ & $(2,3)$ & NO & 9186\\
$(249,77)$ & 13 & $(55,17)$ & 10 & 1 & YES & YES & YES & $1.57$ & $(2,3)$ & 8177 & 9187\\
$(249,95)$ & 12 & $(55,21)$ & 8 & 1 & YES & YES & YES & $1.62$ & $(2,3)$ & NO & 9188\\
$(249,95)$ & 12 & $(97,37)$ & 10 & 1 & YES & YES & YES & $1.62$ & $(2,3)$ & 9666 & 9189\\
$(249,76)$ & 12 & $(154,47)$ & 11 & 1 & YES & YES & YES & $1.43$ & $(6,1)$ & NO & 9190\\
$(249,76)$ & 12 & $(249,76)$ & 12 & 249 & YES & YES & YES & $1.29$ & $(6,1)$ & NO & 9191\\
$(249,77)$ & 13 & $(249,77)$ & 13 & 249 & YES & YES & YES & $1.71$ & $(2,3)$ & NO & 9192\\
$(249,95)$ & 12 & $(249,95)$ & 12 & 249 & YES & YES & YES & $1.75$ & $(4,2)$ & NO & 9193\\
$(250,67)$ & 12 & $(2,1)$ & 1 & 2 & YES & YES & YES & $1.43$ & $(2,3)$ & NO & 9194\\
$(250,67)$ & 12 & $(3,1)$ & 2 & 1 & YES & YES & YES & $1.57$ & $(2,3)$ & NO & 9195\\
$(250,97)$ & 12 & $(4,1)$ & 3 & 2 & YES & YES & YES & $1.80$ & $(2,3)$ & NO & 9196\\
$(250,57)$ & 13 & $(5,2)$ & 3 & 5 & YES & YES & YES & $1.43$ & $(2,3)$ & -- & 9197\\
$(250,57)$ & 13 & $(5,2)$ & 3 & 5 & YES & YES & YES & $1.71$ & $(2,3)$ & NO & 9198\\
$(250,97)$ & 12 & $(5,1)$ & 4 & 5 & YES & YES & YES & $1.62$ & $(4,2)$ & NO & 9199\\
$(250,57)$ & 13 & $(48,11)$ & 9 & 2 & YES & YES & YES & $1.43$ & $(2,3)$ & NO & 9200\\
$(250,57)$ & 13 & $(61,14)$ & 10 & 1 & YES & YES & YES & $1.67$ & $(4,2)$ & NO & 9201\\
$(250,97)$ & 12 & $(116,45)$ & 10 & 2 & YES & YES & YES & $1.62$ & $(4,2)$ & 10291 & 9202\\
$(250,97)$ & 12 & $(183,71)$ & 11 & 1 & YES & YES & YES & $1.75$ & $(4,2)$ & NO & 9203\\
$(251,104)$ & 12 & $(2,1)$ & 1 & 1 & NO & YES & NO(2) & $1.60$ & $(6,1)$ & -- & 9204\\
$(251,73)$ & 13 & $(3,1)$ & 2 & 1 & YES & YES & YES & $1.62$ & $(2,3)$ & NO & 9205\\
$(251,73)$ & 13 & $(3,1)$ & 2 & 1 & YES & YES & YES & $1.62$ & $(2,3)$ & -- & 9206\\
$(251,78)$ & 13 & $(3,1)$ & 2 & 1 & NO & YES & NO(2) & $1.67$ & $(4,2)$ & -- & 9207\\
$(251,98)$ & 12 & $(3,1)$ & 2 & 1 & YES & YES & YES & $1.62$ & $(4,2)$ & NO & 9208\\
$(251,98)$ & 12 & $(3,1)$ & 2 & 1 & YES & YES & YES & $1.62$ & $(4,2)$ & -- & 9209\\
$(251,104)$ & 12 & $(3,1)$ & 2 & 1 & YES & YES & YES & $1.75$ & $(2,3)$ & -- & 9210\\
$(251,104)$ & 12 & $(3,1)$ & 2 & 1 & YES & YES & YES & $1.78$ & $(2,3)$ & NO & 9211\\
$(251,105)$ & 12 & $(3,1)$ & 2 & 1 & YES & YES & NO(2) & $1.44$ & $(4,2)$ & -- & 9212\\
$(251,105)$ & 12 & $(3,1)$ & 2 & 1 & YES & YES & YES & $1.75$ & $(2,3)$ & NO & 9213\\
$(251,70)$ & 12 & $(4,1)$ & 3 & 1 & YES & YES & YES & $1.78$ & $(4,2)$ & NO & 9214\\
$(251,104)$ & 12 & $(4,1)$ & 3 & 1 & YES & YES & YES & $1.62$ & $(2,3)$ & NO & 9215\\
$(251,104)$ & 12 & $(4,1)$ & 3 & 1 & YES & YES & YES & $1.88$ & $(2,3)$ & -- & 9216\\
$(251,70)$ & 12 & $(5,2)$ & 3 & 1 & YES & YES & YES & $1.78$ & $(2,3)$ & NO & 9217\\
$(251,70)$ & 12 & $(5,2)$ & 3 & 1 & YES & YES & YES & $1.78$ & $(2,3)$ & -- & 9218\\
$(251,73)$ & 13 & $(5,1)$ & 4 & 1 & YES & YES & YES & $1.75$ & $(2,3)$ & NO & 9219\\
$(251,73)$ & 13 & $(5,2)$ & 3 & 1 & YES & YES & YES & $1.67$ & $(4,2)$ & -- & 9220\\
$(251,104)$ & 12 & $(5,1)$ & 4 & 1 & YES & YES & YES & $1.50$ & $(2,3)$ & NO & 9221\\
$(251,104)$ & 12 & $(5,1)$ & 4 & 1 & YES & YES & YES & $1.75$ & $(2,3)$ & NO & 9222\\
$(251,104)$ & 12 & $(5,2)$ & 3 & 1 & YES & YES & NO(2) & $1.38$ & $(4,2)$ & NO & 9223\\
$(251,105)$ & 12 & $(5,2)$ & 3 & 1 & YES & YES & YES & $1.78$ & $(4,2)$ & -- & 9224\\
$(251,55)$ & 13 & $(7,3)$ & 4 & 1 & YES & YES & YES & $1.80$ & $(2,3)$ & -- & 9225\\
$(251,104)$ & 12 & $(7,3)$ & 4 & 1 & YES & YES & YES & $1.43$ & $(4,2)$ & NO & 9226\\
$(251,98)$ & 12 & $(13,5)$ & 5 & 1 & YES & YES & YES & $1.75$ & $(4,2)$ & NO & 9227\\
$(251,48)$ & 14 & $(14,3)$ & 6 & 1 & YES & YES & NO(2) & $1.60$ & $(6,1)$ & NO & 9228\\
$(251,55)$ & 13 & $(17,4)$ & 7 & 1 & YES & YES & YES & $1.90$ & $(2,3)$ & NO & 9229\\
$(251,104)$ & 12 & $(17,7)$ & 6 & 1 & YES & YES & YES & $1.50$ & $(4,2)$ & NO & 9230\\
$(251,105)$ & 12 & $(17,7)$ & 6 & 1 & YES & YES & YES & $1.67$ & $(4,2)$ & NO & 9231\\
$(251,104)$ & 12 & $(29,12)$ & 7 & 1 & YES & YES & YES & $1.78$ & $(2,3)$ & NO & 9232\\
$(251,104)$ & 12 & $(41,17)$ & 8 & 1 & YES & YES & YES & $1.78$ & $(2,3)$ & NO & 9233\\
$(251,70)$ & 12 & $(61,17)$ & 9 & 1 & YES & YES & YES & $1.38$ & $(2,3)$ & NO & 9234\\
$(251,70)$ & 12 & $(79,22)$ & 10 & 1 & YES & YES & YES & $1.43$ & $(2,3)$ & NO & 9235\\
$(251,73)$ & 13 & $(86,25)$ & 10 & 1 & YES & YES & YES & $1.75$ & $(2,3)$ & NO & 9236\\
$(251,105)$ & 12 & $(98,41)$ & 10 & 1 & YES & YES & YES & $2.00$ & $(2,3)$ & NO & 9237\\
$(251,104)$ & 12 & $(111,46)$ & 10 & 1 & YES & YES & YES & $1.75$ & $(4,2)$ & 10174 & 9238\\
$(251,105)$ & 12 & $(153,64)$ & 11 & 1 & YES & YES & YES & $1.88$ & $(2,3)$ & NO & 9239\\
$(251,104)$ & 12 & $(181,75)$ & 11 & 1 & YES & YES & YES & $1.71$ & $(4,2)$ & NO & 9240\\
$(252,55)$ & 13 & $(3,1)$ & 2 & 3 & YES & YES & YES & $1.43$ & $(4,2)$ & NO & 9241\\
$(252,55)$ & 13 & $(7,3)$ & 4 & 7 & YES & YES & YES & $1.62$ & $(4,2)$ & -- & 9242\\
$(253,74)$ & 12 & $(2,1)$ & 1 & 1 & YES & YES & YES & $1.56$ & $(2,3)$ & -- & 9243\\
$(253,106)$ & 12 & $(2,1)$ & 1 & 1 & NO & YES & YES & $1.43$ & $(4,2)$ & -- & 9244\\
$(253,114)$ & 13 & $(2,1)$ & 1 & 1 & NO & YES & NO(2) & $1.67$ & $(8,0)$ & -- & 9245\\
$(253,57)$ & 13 & $(3,1)$ & 2 & 1 & YES & YES & NO(2) & $1.56$ & $(2,3)$ & NO & 9246\\
$(253,57)$ & 13 & $(3,1)$ & 2 & 1 & YES & YES & NO(2) & $1.56$ & $(2,3)$ & -- & 9247\\
$(253,60)$ & 13 & $(3,1)$ & 2 & 1 & YES & YES & YES & $1.71$ & $(2,3)$ & NO & 9248\\
$(253,68)$ & 12 & $(3,1)$ & 2 & 1 & YES & YES & YES & $1.43$ & $(6,1)$ & -- & 9249\\
$(253,106)$ & 12 & $(3,1)$ & 2 & 1 & YES & YES & YES & $1.78$ & $(2,3)$ & -- & 9250\\
$(253,106)$ & 12 & $(3,1)$ & 2 & 1 & YES & YES & YES & $1.90$ & $(2,3)$ & NO & 9251\\
$(253,111)$ & 12 & $(3,1)$ & 2 & 1 & YES & YES & YES & $1.43$ & $(4,2)$ & -- & 9252\\
$(253,93)$ & 12 & $(4,1)$ & 3 & 1 & YES & YES & YES & $1.75$ & $(2,3)$ & -- & 9253\\
$(253,98)$ & 12 & $(4,1)$ & 3 & 1 & YES & YES & YES & $1.62$ & $(4,2)$ & NO & 9254\\
$(253,106)$ & 12 & $(4,1)$ & 3 & 1 & YES & YES & YES & $1.62$ & $(4,2)$ & NO & 9255\\
$(253,71)$ & 13 & $(5,1)$ & 4 & 1 & YES & YES & YES & $1.43$ & $(4,2)$ & NO & 9256\\
$(253,74)$ & 12 & $(5,2)$ & 3 & 1 & YES & YES & YES & $1.67$ & $(2,3)$ & -- & 9257\\
$(253,68)$ & 12 & $(7,2)$ & 4 & 1 & YES & YES & YES & $1.43$ & $(6,1)$ & NO & 9258\\
$(253,106)$ & 12 & $(7,3)$ & 4 & 1 & YES & YES & YES & $1.67$ & $(2,3)$ & NO & 9259\\
$(253,74)$ & 12 & $(9,2)$ & 5 & 1 & YES & YES & YES & $1.75$ & $(2,3)$ & NO & 9260\\
$(253,71)$ & 13 & $(10,3)$ & 5 & 1 & YES & YES & YES & $1.90$ & $(2,3)$ & NO & 9261\\
$(253,71)$ & 13 & $(11,3)$ & 5 & 11 & YES & YES & YES & $1.57$ & $(4,2)$ & NO & 9262\\
$(253,60)$ & 13 & $(13,3)$ & 6 & 1 & YES & YES & YES & $1.57$ & $(2,3)$ & NO & 9263\\
$(253,71)$ & 13 & $(18,5)$ & 6 & 1 & YES & YES & YES & $1.57$ & $(4,2)$ & NO & 9264\\
$(253,68)$ & 12 & $(19,5)$ & 7 & 1 & YES & YES & YES & $1.62$ & $(4,2)$ & NO & 9265\\
$(253,57)$ & 13 & $(22,5)$ & 7 & 11 & YES & YES & NO(2) & $1.56$ & $(2,3)$ & NO & 9266\\
$(253,111)$ & 12 & $(41,18)$ & 8 & 1 & YES & YES & YES & $1.43$ & $(4,2)$ & NO & 9267\\
$(253,74)$ & 12 & $(65,19)$ & 9 & 1 & YES & YES & YES & $1.57$ & $(2,3)$ & NO & 9268\\
$(253,71)$ & 13 & $(82,23)$ & 10 & 1 & YES & YES & YES & $1.43$ & $(4,2)$ & NO & 9269\\
$(253,106)$ & 12 & $(105,44)$ & 10 & 1 & YES & YES & YES & $1.62$ & $(4,2)$ & 9985 & 9270\\
$(253,98)$ & 12 & $(111,43)$ & 10 & 1 & YES & YES & YES & $1.78$ & $(2,3)$ & NO & 9271\\
$(253,111)$ & 12 & $(155,68)$ & 11 & 1 & YES & YES & YES & $1.62$ & $(4,2)$ & NO & 9272\\
$(253,106)$ & 12 & $(179,75)$ & 11 & 1 & YES & YES & YES & $1.50$ & $(4,2)$ & NO & 9273\\
$(253,98)$ & 12 & $(253,98)$ & 12 & 253 & YES & YES & YES & $1.67$ & $(4,2)$ & NO & 9274\\
$(253,106)$ & 12 & $(253,106)$ & 12 & 253 & YES & YES & YES & $1.50$ & $(4,2)$ & NO & 9275\\
$(254,71)$ & 12 & $(2,1)$ & 1 & 2 & YES & YES & NO(2) & $1.67$ & $(2,3)$ & NO & 9276\\
$(254,75)$ & 12 & $(2,1)$ & 1 & 2 & YES & YES & YES & $1.29$ & $(4,2)$ & -- & 9277\\
$(254,71)$ & 12 & $(3,1)$ & 2 & 1 & YES & YES & YES & $1.38$ & $(2,3)$ & -- & 9278\\
$(254,71)$ & 12 & $(3,1)$ & 2 & 1 & YES & YES & YES & $1.62$ & $(2,3)$ & NO & 9279\\
$(254,75)$ & 12 & $(3,1)$ & 2 & 1 & NO & YES & NO(2) & $1.50$ & $(6,1)$ & -- & 9280\\
$(254,75)$ & 12 & $(3,1)$ & 2 & 1 & YES & YES & YES & $1.62$ & $(4,2)$ & NO & 9281\\
$(254,93)$ & 12 & $(3,1)$ & 2 & 1 & YES & YES & YES & $1.43$ & $(4,2)$ & -- & 9282\\
$(254,93)$ & 12 & $(3,1)$ & 2 & 1 & YES & YES & YES & $1.86$ & $(4,2)$ & NO & 9283\\
$(254,105)$ & 12 & $(3,1)$ & 2 & 1 & YES & YES & YES & $1.29$ & $(4,2)$ & -- & 9284\\
$(254,105)$ & 12 & $(4,1)$ & 3 & 2 & YES & YES & YES & $2.00$ & $(2,3)$ & -- & 9285\\
$(254,105)$ & 12 & $(4,1)$ & 3 & 2 & YES & YES & YES & $1.71$ & $(2,3)$ & NO & 9286\\
$(254,71)$ & 12 & $(5,2)$ & 3 & 1 & YES & YES & YES & $1.57$ & $(2,3)$ & -- & 9287\\
$(254,71)$ & 12 & $(10,3)$ & 5 & 2 & YES & YES & YES & $1.62$ & $(2,3)$ & NO & 9288\\
$(254,71)$ & 12 & $(11,3)$ & 5 & 1 & YES & YES & YES & $1.38$ & $(2,3)$ & NO & 9289\\
$(254,75)$ & 12 & $(13,4)$ & 6 & 1 & YES & YES & YES & $1.57$ & $(4,2)$ & NO & 9290\\
$(254,71)$ & 12 & $(18,5)$ & 6 & 2 & YES & YES & NO(2) & $1.56$ & $(2,3)$ & NO & 9291\\
$(254,75)$ & 12 & $(24,7)$ & 7 & 2 & YES & YES & YES & $1.50$ & $(2,3)$ & NO & 9292\\
$(254,71)$ & 12 & $(25,7)$ & 7 & 1 & YES & YES & NO(2) & $1.64$ & $(4,2)$ & NO & 9293\\
$(254,75)$ & 12 & $(27,8)$ & 7 & 1 & YES & YES & YES & $1.56$ & $(2,3)$ & NO & 9294\\
$(254,71)$ & 12 & $(32,9)$ & 8 & 2 & YES & YES & YES & $1.50$ & $(2,3)$ & NO & 9295\\
$(254,71)$ & 12 & $(43,12)$ & 8 & 1 & YES & YES & YES & $1.62$ & $(2,3)$ & 10134 & 9296\\
$(254,105)$ & 12 & $(46,19)$ & 8 & 2 & YES & YES & YES & $1.78$ & $(2,3)$ & NO & 9297\\
$(254,71)$ & 12 & $(61,17)$ & 9 & 1 & YES & YES & YES & $1.57$ & $(2,3)$ & 11500 & 9298\\
$(254,75)$ & 12 & $(61,18)$ & 9 & 1 & YES & YES & YES & $1.38$ & $(4,2)$ & NO & 9299\\
$(254,75)$ & 12 & $(78,23)$ & 10 & 2 & YES & YES & YES & $1.50$ & $(2,3)$ & NO & 9300\\
$(254,105)$ & 12 & $(104,43)$ & 10 & 2 & YES & YES & YES & $1.29$ & $(4,2)$ & 9963 & 9301\\
$(254,93)$ & 12 & $(112,41)$ & 10 & 2 & YES & YES & YES & $1.71$ & $(4,2)$ & 10221 & 9302\\
$(254,105)$ & 12 & $(179,74)$ & 11 & 1 & YES & YES & YES & $1.90$ & $(2,3)$ & NO & 9303\\
$(254,93)$ & 12 & $(183,67)$ & 11 & 1 & YES & YES & YES & $1.75$ & $(4,2)$ & NO & 9304\\
$(254,105)$ & 12 & $(254,105)$ & 12 & 254 & YES & YES & YES & $1.78$ & $(2,3)$ & NO & 9305\\
$(255,56)$ & 15 & $(2,1)$ & 1 & 1 & YES & YES & NO(2) & $1.56$ & $(8,0)$ & -- & 9306\\
$(255,76)$ & 13 & $(2,1)$ & 1 & 1 & YES & YES & YES & $1.57$ & $(2,3)$ & NO & 9307\\
$(255,107)$ & 12 & $(2,1)$ & 1 & 1 & NO & YES & NO(2) & $1.67$ & $(4,2)$ & -- & 9308\\
$(255,112)$ & 12 & $(2,1)$ & 1 & 1 & YES & YES & YES & $1.57$ & $(4,2)$ & -- & 9309\\
$(255,92)$ & 12 & $(3,1)$ & 2 & 3 & YES & YES & YES & $1.57$ & $(4,2)$ & -- & 9310\\
$(255,97)$ & 12 & $(3,1)$ & 2 & 3 & YES & YES & YES & $1.62$ & $(2,3)$ & -- & 9311\\
$(255,112)$ & 12 & $(3,1)$ & 2 & 3 & YES & YES & YES & $1.75$ & $(4,2)$ & NO & 9312\\
$(255,112)$ & 12 & $(3,1)$ & 2 & 3 & YES & YES & YES & $1.75$ & $(4,2)$ & -- & 9313\\
$(255,71)$ & 13 & $(4,1)$ & 3 & 1 & YES & YES & YES & $1.62$ & $(4,2)$ & -- & 9314\\
$(255,97)$ & 12 & $(4,1)$ & 3 & 1 & YES & YES & YES & $1.71$ & $(2,3)$ & -- & 9315\\
$(255,92)$ & 12 & $(5,1)$ & 4 & 5 & YES & YES & YES & $1.57$ & $(4,2)$ & -- & 9316\\
$(255,97)$ & 12 & $(5,2)$ & 3 & 5 & YES & YES & YES & $1.75$ & $(4,2)$ & -- & 9317\\
$(255,107)$ & 12 & $(5,1)$ & 4 & 5 & YES & YES & YES & $1.57$ & $(2,3)$ & -- & 9318\\
$(255,107)$ & 12 & $(5,1)$ & 4 & 5 & YES & YES & YES & $1.75$ & $(2,3)$ & NO & 9319\\
$(255,97)$ & 12 & $(13,5)$ & 5 & 1 & YES & YES & YES & $1.67$ & $(2,3)$ & NO & 9320\\
$(255,92)$ & 12 & $(14,5)$ & 6 & 1 & YES & YES & YES & $1.57$ & $(4,2)$ & 9663 & 9321\\
$(255,92)$ & 12 & $(25,9)$ & 7 & 5 & YES & YES & YES & $1.57$ & $(4,2)$ & 8784 & 9322\\
$(255,97)$ & 12 & $(29,11)$ & 7 & 1 & YES & YES & YES & $1.62$ & $(2,3)$ & NO & 9323\\
$(255,47)$ & 15 & $(49,9)$ & 10 & 1 & YES & YES & NO(2) & $1.60$ & $(6,1)$ & NO & 9324\\
$(255,92)$ & 12 & $(61,22)$ & 9 & 1 & YES & YES & YES & $1.57$ & $(4,2)$ & 8482 & 9325\\
$(255,107)$ & 12 & $(81,34)$ & 9 & 3 & YES & YES & YES & $1.88$ & $(2,3)$ & NO & 9326\\
$(255,112)$ & 12 & $(107,47)$ & 10 & 1 & YES & YES & YES & $1.57$ & $(4,2)$ & NO & 9327\\
$(255,107)$ & 12 & $(112,47)$ & 10 & 1 & YES & YES & YES & $1.89$ & $(2,3)$ & NO & 9328\\
$(255,92)$ & 12 & $(158,57)$ & 11 & 1 & YES & YES & YES & $1.57$ & $(4,2)$ & NO & 9329\\
$(255,97)$ & 12 & $(163,62)$ & 11 & 1 & YES & YES & YES & $1.78$ & $(2,3)$ & NO & 9330\\
$(255,71)$ & 13 & $(176,49)$ & 12 & 1 & YES & YES & YES & $1.62$ & $(4,2)$ & NO & 9331\\
$(255,97)$ & 12 & $(255,97)$ & 12 & 255 & YES & YES & YES & $1.67$ & $(2,3)$ & NO & 9332\\
$(255,107)$ & 12 & $(255,107)$ & 12 & 255 & YES & YES & YES & $1.57$ & $(4,2)$ & NO & 9333\\
$(255,112)$ & 12 & $(255,112)$ & 12 & 255 & YES & YES & YES & $1.62$ & $(4,2)$ & NO & 9334\\
$(256,75)$ & 12 & $(2,1)$ & 1 & 2 & YES & YES & YES & $1.90$ & $(2,3)$ & NO & 9335\\
$(256,75)$ & 12 & $(2,1)$ & 1 & 2 & YES & YES & YES & $1.50$ & $(2,3)$ & -- & 9336\\
$(256,99)$ & 12 & $(2,1)$ & 1 & 2 & NO & YES & YES & $1.43$ & $(4,2)$ & -- & 9337\\
$(256,75)$ & 12 & $(3,1)$ & 2 & 1 & NO & YES & NO(2) & $1.50$ & $(6,1)$ & -- & 9338\\
$(256,75)$ & 12 & $(3,1)$ & 2 & 1 & YES & YES & YES & $1.50$ & $(4,2)$ & NO & 9339\\
$(256,97)$ & 12 & $(3,1)$ & 2 & 1 & YES & YES & YES & $1.89$ & $(2,3)$ & -- & 9340\\
$(256,99)$ & 12 & $(3,1)$ & 2 & 1 & YES & YES & YES & $1.43$ & $(4,2)$ & -- & 9341\\
$(256,99)$ & 12 & $(3,1)$ & 2 & 1 & YES & YES & YES & $1.78$ & $(2,3)$ & NO & 9342\\
$(256,99)$ & 12 & $(3,1)$ & 2 & 1 & YES & YES & YES & $1.89$ & $(2,3)$ & NO & 9343\\
$(256,97)$ & 12 & $(4,1)$ & 3 & 4 & YES & YES & YES & $1.80$ & $(2,3)$ & -- & 9344\\
$(256,97)$ & 12 & $(4,1)$ & 3 & 4 & YES & YES & YES & $1.62$ & $(4,2)$ & NO & 9345\\
$(256,99)$ & 12 & $(4,1)$ & 3 & 4 & YES & YES & YES & $1.43$ & $(4,2)$ & -- & 9346\\
$(256,99)$ & 12 & $(4,1)$ & 3 & 4 & YES & YES & YES & $1.43$ & $(4,2)$ & NO & 9347\\
$(256,97)$ & 12 & $(7,3)$ & 4 & 1 & YES & YES & YES & $1.75$ & $(4,2)$ & NO & 9348\\
$(256,99)$ & 12 & $(8,3)$ & 4 & 8 & YES & YES & YES & $1.89$ & $(2,3)$ & NO & 9349\\
$(256,75)$ & 12 & $(9,2)$ & 5 & 1 & YES & YES & YES & $1.78$ & $(2,3)$ & NO & 9350\\
$(256,97)$ & 12 & $(13,5)$ & 5 & 1 & YES & YES & YES & $1.70$ & $(2,3)$ & NO & 9351\\
$(256,99)$ & 12 & $(13,5)$ & 5 & 1 & YES & YES & YES & $1.90$ & $(2,3)$ & NO & 9352\\
$(256,99)$ & 12 & $(18,7)$ & 6 & 2 & YES & YES & YES & $1.78$ & $(2,3)$ & 7415 & 9353\\
$(256,97)$ & 12 & $(21,8)$ & 6 & 1 & YES & YES & YES & $1.43$ & $(4,2)$ & NO & 9354\\
$(256,99)$ & 12 & $(21,8)$ & 6 & 1 & YES & YES & YES & $1.62$ & $(4,2)$ & NO & 9355\\
$(256,75)$ & 12 & $(24,7)$ & 7 & 8 & YES & YES & YES & $1.67$ & $(2,3)$ & NO & 9356\\
$(256,99)$ & 12 & $(31,12)$ & 7 & 1 & YES & YES & YES & $1.67$ & $(2,3)$ & NO & 9357\\
$(256,75)$ & 12 & $(41,12)$ & 8 & 1 & YES & YES & YES & $1.50$ & $(2,3)$ & NO & 9358\\
$(256,99)$ & 12 & $(44,17)$ & 8 & 4 & YES & YES & YES & $1.89$ & $(2,3)$ & NO & 9359\\
$(256,99)$ & 12 & $(57,22)$ & 9 & 1 & YES & YES & YES & $1.62$ & $(4,2)$ & NO & 9360\\
$(256,75)$ & 12 & $(58,17)$ & 9 & 2 & YES & YES & YES & $1.90$ & $(2,3)$ & 8401 & 9361\\
$(256,97)$ & 12 & $(66,25)$ & 9 & 2 & YES & YES & YES & $1.75$ & $(4,2)$ & 8656 & 9362\\
$(256,75)$ & 12 & $(75,22)$ & 10 & 1 & YES & YES & YES & $1.62$ & $(2,3)$ & NO & 9363\\
$(256,99)$ & 12 & $(75,29)$ & 9 & 1 & YES & YES & YES & $1.78$ & $(2,3)$ & NO & 9364\\
$(256,99)$ & 12 & $(106,41)$ & 10 & 2 & YES & YES & YES & $1.43$ & $(4,2)$ & 10045 & 9365\\
$(256,99)$ & 12 & $(181,70)$ & 11 & 1 & YES & YES & YES & $1.43$ & $(4,2)$ & NO & 9366\\
$(256,99)$ & 12 & $(256,99)$ & 12 & 256 & YES & YES & YES & $1.43$ & $(4,2)$ & NO & 9367\\
$(257,69)$ & 12 & $(2,1)$ & 1 & 1 & YES & YES & YES & $1.43$ & $(2,3)$ & -- & 9368\\
$(257,69)$ & 12 & $(2,1)$ & 1 & 1 & YES & YES & YES & $1.43$ & $(2,3)$ & NO & 9369\\
$(257,76)$ & 12 & $(2,1)$ & 1 & 1 & YES & YES & YES & $1.56$ & $(2,3)$ & -- & 9370\\
$(257,76)$ & 12 & $(2,1)$ & 1 & 1 & YES & YES & YES & $1.78$ & $(2,3)$ & NO & 9371\\
$(257,106)$ & 13 & $(2,1)$ & 1 & 1 & YES & YES & NO(2) & $1.56$ & $(8,0)$ & NO & 9372\\
$(257,108)$ & 12 & $(2,1)$ & 1 & 1 & YES & YES & YES & $1.56$ & $(2,3)$ & -- & 9373\\
$(257,45)$ & 15 & $(3,1)$ & 2 & 1 & YES & YES & NO(2) & $1.38$ & $(6,1)$ & NO & 9374\\
$(257,45)$ & 15 & $(3,1)$ & 2 & 1 & YES & YES & NO(2) & $1.56$ & $(4,2)$ & -- & 9375\\
$(257,69)$ & 12 & $(3,1)$ & 2 & 1 & YES & YES & YES & $1.43$ & $(2,3)$ & NO & 9376\\
$(257,71)$ & 12 & $(3,1)$ & 2 & 1 & YES & YES & YES & $1.50$ & $(2,3)$ & NO & 9377\\
$(257,76)$ & 12 & $(3,1)$ & 2 & 1 & YES & YES & YES & $1.38$ & $(4,2)$ & -- & 9378\\
$(257,108)$ & 12 & $(3,1)$ & 2 & 1 & YES & YES & YES & $1.70$ & $(2,3)$ & NO & 9379\\
$(257,108)$ & 12 & $(3,1)$ & 2 & 1 & YES & YES & YES & $1.70$ & $(2,3)$ & -- & 9380\\
$(257,108)$ & 12 & $(3,1)$ & 2 & 1 & YES & YES & YES & $1.71$ & $(2,3)$ & NO & 9381\\
$(257,56)$ & 13 & $(4,1)$ & 3 & 1 & YES & YES & NO(2) & $1.44$ & $(4,2)$ & NO & 9382\\
$(257,69)$ & 12 & $(4,1)$ & 3 & 1 & YES & YES & NO(2) & $1.44$ & $(4,2)$ & -- & 9383\\
$(257,76)$ & 12 & $(4,1)$ & 3 & 1 & YES & YES & YES & $1.78$ & $(2,3)$ & NO & 9384\\
$(257,106)$ & 13 & $(4,1)$ & 3 & 1 & YES & YES & NO(2) & $1.44$ & $(8,0)$ & -- & 9385\\
$(257,108)$ & 12 & $(4,1)$ & 3 & 1 & YES & YES & YES & $1.67$ & $(4,2)$ & -- & 9386\\
$(257,71)$ & 12 & $(5,2)$ & 3 & 1 & YES & YES & YES & $1.75$ & $(2,3)$ & NO & 9387\\
$(257,71)$ & 12 & $(5,2)$ & 3 & 1 & YES & YES & YES & $1.67$ & $(2,3)$ & -- & 9388\\
$(257,76)$ & 12 & $(5,2)$ & 3 & 1 & YES & YES & YES & $1.57$ & $(2,3)$ & -- & 9389\\
$(257,78)$ & 13 & $(5,1)$ & 4 & 1 & YES & YES & YES & $1.75$ & $(2,3)$ & NO & 9390\\
$(257,71)$ & 12 & $(7,2)$ & 4 & 1 & YES & YES & YES & $1.50$ & $(2,3)$ & NO & 9391\\
$(257,76)$ & 12 & $(7,2)$ & 4 & 1 & YES & YES & YES & $1.38$ & $(2,3)$ & NO & 9392\\
$(257,76)$ & 12 & $(10,3)$ & 5 & 1 & YES & YES & YES & $1.75$ & $(2,3)$ & NO & 9393\\
$(257,45)$ & 15 & $(11,2)$ & 6 & 1 & YES & YES & NO(2) & $1.56$ & $(4,2)$ & NO & 9394\\
$(257,69)$ & 12 & $(11,3)$ & 5 & 1 & YES & YES & YES & $1.29$ & $(2,3)$ & NO & 9395\\
$(257,76)$ & 12 & $(11,3)$ & 5 & 1 & YES & YES & YES & $1.57$ & $(2,3)$ & NO & 9396\\
$(257,76)$ & 12 & $(13,4)$ & 6 & 1 & YES & YES & YES & $1.71$ & $(2,3)$ & NO & 9397\\
$(257,71)$ & 12 & $(15,4)$ & 6 & 1 & YES & YES & YES & $1.57$ & $(2,3)$ & NO & 9398\\
$(257,76)$ & 12 & $(17,5)$ & 6 & 1 & YES & YES & YES & $1.38$ & $(2,3)$ & NO & 9399\\
$(257,71)$ & 12 & $(25,7)$ & 7 & 1 & YES & YES & YES & $1.62$ & $(2,3)$ & NO & 9400\\
$(257,69)$ & 12 & $(26,7)$ & 7 & 1 & YES & YES & YES & $1.43$ & $(2,3)$ & 6672 & 9401\\
$(257,76)$ & 12 & $(27,8)$ & 7 & 1 & YES & YES & YES & $1.71$ & $(4,2)$ & NO & 9402\\
$(257,71)$ & 12 & $(40,11)$ & 8 & 1 & YES & YES & YES & $1.57$ & $(2,3)$ & NO & 9403\\
$(257,76)$ & 12 & $(44,13)$ & 8 & 1 & YES & YES & YES & $1.78$ & $(2,3)$ & NO & 9404\\
$(257,59)$ & 14 & $(48,11)$ & 9 & 1 & YES & YES & YES & $1.62$ & $(2,3)$ & NO & 9405\\
$(257,108)$ & 12 & $(50,21)$ & 8 & 1 & YES & YES & YES & $1.78$ & $(4,2)$ & NO & 9406\\
$(257,76)$ & 12 & $(61,18)$ & 9 & 1 & YES & YES & YES & $1.57$ & $(2,3)$ & 11051 & 9407\\
$(257,76)$ & 12 & $(71,21)$ & 9 & 1 & YES & YES & YES & $1.38$ & $(4,2)$ & NO & 9408\\
$(257,108)$ & 12 & $(119,50)$ & 10 & 1 & YES & YES & YES & $1.67$ & $(4,2)$ & 10435 & 9409\\
$(257,78)$ & 13 & $(201,61)$ & 12 & 1 & YES & YES & YES & $1.88$ & $(2,3)$ & NO & 9410\\
$(257,76)$ & 12 & $(257,76)$ & 12 & 257 & YES & YES & YES & $1.50$ & $(2,3)$ & NO & 9411\\
$(258,71)$ & 12 & $(2,1)$ & 1 & 2 & YES & YES & YES & $1.57$ & $(2,3)$ & -- & 9412\\
$(258,109)$ & 12 & $(2,1)$ & 1 & 2 & NO & YES & NO(2) & $1.80$ & $(6,1)$ & -- & 9413\\
$(258,109)$ & 12 & $(3,1)$ & 2 & 3 & YES & YES & YES & $1.57$ & $(4,2)$ & NO & 9414\\
$(258,71)$ & 12 & $(4,1)$ & 3 & 2 & YES & YES & YES & $1.43$ & $(2,3)$ & -- & 9415\\
$(258,109)$ & 12 & $(4,1)$ & 3 & 2 & YES & YES & YES & $1.78$ & $(2,3)$ & -- & 9416\\
$(258,109)$ & 12 & $(4,1)$ & 3 & 2 & YES & YES & YES & $1.50$ & $(4,2)$ & NO & 9417\\
$(258,71)$ & 12 & $(5,2)$ & 3 & 1 & YES & YES & YES & $1.43$ & $(4,2)$ & -- & 9418\\
$(258,109)$ & 12 & $(5,2)$ & 3 & 1 & YES & YES & YES & $1.62$ & $(4,2)$ & NO & 9419\\
$(258,109)$ & 12 & $(19,8)$ & 6 & 1 & YES & YES & YES & $1.62$ & $(4,2)$ & NO & 9420\\
$(258,71)$ & 12 & $(29,8)$ & 7 & 1 & YES & YES & YES & $1.43$ & $(2,3)$ & 6673 & 9421\\
$(258,109)$ & 12 & $(45,19)$ & 8 & 3 & YES & YES & YES & $1.57$ & $(4,2)$ & NO & 9422\\
$(258,71)$ & 12 & $(109,30)$ & 10 & 1 & YES & YES & YES & $1.57$ & $(2,3)$ & NO & 9423\\
$(258,109)$ & 12 & $(116,49)$ & 10 & 2 & YES & YES & YES & $1.57$ & $(4,2)$ & 10374 & 9424\\
$(259,76)$ & 13 & $(2,1)$ & 1 & 1 & YES & YES & YES & $1.57$ & $(2,3)$ & NO & 9425\\
$(259,100)$ & 12 & $(2,1)$ & 1 & 1 & YES & YES & YES & $1.78$ & $(2,3)$ & -- & 9426\\
$(259,101)$ & 12 & $(2,1)$ & 1 & 1 & YES & YES & NO(2) & $1.38$ & $(4,2)$ & -- & 9427\\
$(259,50)$ & 15 & $(3,1)$ & 2 & 1 & YES & YES & NO(2) & $1.67$ & $(4,2)$ & -- & 9428\\
$(259,50)$ & 15 & $(3,1)$ & 2 & 1 & YES & YES & NO(2) & $1.78$ & $(4,2)$ & NO & 9429\\
$(259,59)$ & 13 & $(3,1)$ & 2 & 1 & YES & YES & NO(2) & $1.56$ & $(2,3)$ & NO & 9430\\
$(259,59)$ & 13 & $(3,1)$ & 2 & 1 & YES & YES & NO(2) & $1.56$ & $(2,3)$ & -- & 9431\\
$(259,59)$ & 13 & $(3,1)$ & 2 & 1 & YES & YES & NO(2) & $1.44$ & $(8,0)$ & NO & 9432\\
$(259,79)$ & 13 & $(3,1)$ & 2 & 1 & YES & YES & NO(2) & $1.38$ & $(10,-1)$ & NO & 9433\\
$(259,100)$ & 12 & $(3,1)$ & 2 & 1 & YES & YES & YES & $1.80$ & $(2,3)$ & -- & 9434\\
$(259,100)$ & 12 & $(3,1)$ & 2 & 1 & YES & YES & YES & $1.67$ & $(2,3)$ & NO & 9435\\
$(259,100)$ & 12 & $(4,1)$ & 3 & 1 & YES & YES & YES & $1.43$ & $(4,2)$ & -- & 9436\\
$(259,100)$ & 12 & $(4,1)$ & 3 & 1 & YES & YES & YES & $1.86$ & $(2,3)$ & NO & 9437\\
$(259,59)$ & 13 & $(5,1)$ & 4 & 1 & YES & YES & NO(2) & $1.60$ & $(2,3)$ & NO & 9438\\
$(259,100)$ & 12 & $(5,2)$ & 3 & 1 & YES & YES & YES & $1.90$ & $(2,3)$ & -- & 9439\\
$(259,101)$ & 12 & $(5,1)$ & 4 & 1 & YES & YES & NO(2) & $1.67$ & $(2,3)$ & NO & 9440\\
$(259,50)$ & 15 & $(16,3)$ & 7 & 1 & YES & YES & NO(2) & $1.56$ & $(8,0)$ & NO & 9441\\
$(259,100)$ & 12 & $(18,7)$ & 6 & 1 & YES & YES & YES & $1.67$ & $(2,3)$ & 6895 & 9442\\
$(259,101)$ & 12 & $(18,7)$ & 6 & 1 & YES & YES & NO(2) & $1.50$ & $(4,2)$ & 8349 & 9443\\
$(259,50)$ & 15 & $(21,4)$ & 8 & 7 & YES & YES & NO(2) & $1.62$ & $(6,1)$ & NO & 9444\\
$(259,100)$ & 12 & $(31,12)$ & 7 & 1 & YES & YES & YES & $1.67$ & $(2,3)$ & NO & 9445\\
$(259,59)$ & 13 & $(35,8)$ & 8 & 7 & YES & YES & NO(2) & $1.67$ & $(2,3)$ & NO & 9446\\
$(259,100)$ & 12 & $(57,22)$ & 9 & 1 & YES & YES & YES & $1.62$ & $(4,2)$ & 8403 & 9447\\
$(259,101)$ & 12 & $(59,23)$ & 9 & 1 & YES & YES & YES & $1.62$ & $(2,3)$ & 8443 & 9448\\
$(259,101)$ & 12 & $(100,39)$ & 10 & 1 & YES & YES & YES & $1.62$ & $(2,3)$ & NO & 9449\\
$(259,100)$ & 12 & $(101,39)$ & 10 & 1 & YES & YES & YES & $1.60$ & $(2,3)$ & NO & 9450\\
$(259,101)$ & 12 & $(141,55)$ & 11 & 1 & YES & YES & YES & $1.50$ & $(4,2)$ & NO & 9451\\
$(259,100)$ & 12 & $(158,61)$ & 11 & 1 & YES & YES & YES & $1.43$ & $(4,2)$ & NO & 9452\\
$(259,101)$ & 12 & $(159,62)$ & 11 & 1 & YES & YES & YES & $1.57$ & $(4,2)$ & NO & 9453\\
$(259,59)$ & 13 & $(193,44)$ & 12 & 1 & YES & YES & YES & $1.67$ & $(4,2)$ & NO & 9454\\
$(259,100)$ & 12 & $(259,100)$ & 12 & 259 & YES & YES & YES & $1.62$ & $(4,2)$ & NO & 9455\\
$(259,101)$ & 12 & $(259,101)$ & 12 & 259 & YES & YES & YES & $1.71$ & $(4,2)$ & NO & 9456\\
$(260,73)$ & 13 & $(3,1)$ & 2 & 1 & YES & YES & YES & $1.75$ & $(2,3)$ & NO & 9457\\
$(260,73)$ & 13 & $(3,1)$ & 2 & 1 & YES & YES & YES & $1.62$ & $(2,3)$ & -- & 9458\\
$(260,79)$ & 13 & $(3,1)$ & 2 & 1 & YES & YES & YES & $1.71$ & $(6,1)$ & NO & 9459\\
$(260,79)$ & 13 & $(3,1)$ & 2 & 1 & YES & YES & YES & $1.71$ & $(4,2)$ & -- & 9460\\
$(260,73)$ & 13 & $(5,2)$ & 3 & 5 & YES & YES & YES & $1.67$ & $(4,2)$ & -- & 9461\\
$(260,57)$ & 13 & $(7,3)$ & 4 & 1 & YES & YES & YES & $1.78$ & $(4,2)$ & -- & 9462\\
$(260,73)$ & 13 & $(10,3)$ & 5 & 10 & YES & YES & YES & $1.56$ & $(4,2)$ & NO & 9463\\
$(260,73)$ & 13 & $(89,25)$ & 10 & 1 & YES & YES & YES & $1.88$ & $(2,3)$ & NO & 9464\\
$(261,100)$ & 12 & $(2,1)$ & 1 & 1 & YES & YES & NO(2) & $1.56$ & $(2,3)$ & -- & 9465\\
$(261,107)$ & 12 & $(2,1)$ & 1 & 1 & YES & YES & YES & $1.78$ & $(2,3)$ & -- & 9466\\
$(261,100)$ & 12 & $(3,1)$ & 2 & 3 & YES & YES & YES & $1.56$ & $(2,3)$ & -- & 9467\\
$(261,100)$ & 12 & $(3,1)$ & 2 & 3 & YES & YES & YES & $2.00$ & $(2,3)$ & NO & 9468\\
$(261,100)$ & 12 & $(4,1)$ & 3 & 1 & YES & YES & YES & $1.43$ & $(4,2)$ & -- & 9469\\
$(261,100)$ & 12 & $(4,1)$ & 3 & 1 & YES & YES & YES & $1.71$ & $(4,2)$ & NO & 9470\\
$(261,59)$ & 13 & $(5,2)$ & 3 & 1 & YES & YES & YES & $1.90$ & $(2,3)$ & -- & 9471\\
$(261,59)$ & 13 & $(5,2)$ & 3 & 1 & YES & YES & YES & $1.57$ & $(4,2)$ & NO & 9472\\
$(261,107)$ & 12 & $(17,7)$ & 6 & 1 & YES & YES & YES & $1.62$ & $(2,3)$ & NO & 9473\\
$(261,100)$ & 12 & $(21,8)$ & 6 & 3 & YES & YES & YES & $2.00$ & $(2,3)$ & NO & 9474\\
$(261,100)$ & 12 & $(34,13)$ & 7 & 1 & YES & YES & YES & $1.86$ & $(4,2)$ & NO & 9475\\
$(261,76)$ & 13 & $(55,16)$ & 9 & 1 & YES & YES & NO(2) & $1.38$ & $(4,2)$ & NO & 9476\\
$(261,100)$ & 12 & $(60,23)$ & 9 & 3 & YES & YES & YES & $2.00$ & $(2,3)$ & NO & 9477\\
$(261,107)$ & 12 & $(61,25)$ & 9 & 1 & YES & YES & YES & $1.62$ & $(2,3)$ & 8530 & 9478\\
$(261,107)$ & 12 & $(100,41)$ & 10 & 1 & YES & YES & YES & $1.62$ & $(2,3)$ & NO & 9479\\
$(261,100)$ & 12 & $(107,41)$ & 10 & 1 & YES & YES & YES & $1.89$ & $(2,3)$ & NO & 9480\\
$(262,61)$ & 13 & $(5,2)$ & 3 & 1 & YES & YES & YES & $1.43$ & $(2,3)$ & -- & 9481\\
$(262,71)$ & 13 & $(5,2)$ & 3 & 1 & YES & YES & YES & $1.75$ & $(4,2)$ & -- & 9482\\
$(262,71)$ & 13 & $(5,2)$ & 3 & 1 & YES & YES & YES & $1.88$ & $(4,2)$ & NO & 9483\\
$(262,107)$ & 13 & $(5,1)$ & 4 & 1 & YES & YES & YES & $1.78$ & $(4,2)$ & -- & 9484\\
$(262,73)$ & 13 & $(7,2)$ & 4 & 1 & YES & YES & YES & $1.67$ & $(4,2)$ & -- & 9485\\
$(262,63)$ & 15 & $(17,4)$ & 7 & 1 & YES & YES & NO(2) & $1.60$ & $(6,1)$ & NO & 9486\\
$(262,107)$ & 13 & $(17,7)$ & 6 & 1 & YES & YES & YES & $1.78$ & $(4,2)$ & NO & 9487\\
$(262,61)$ & 13 & $(47,11)$ & 9 & 1 & YES & YES & YES & $1.43$ & $(2,3)$ & NO & 9488\\
$(262,47)$ & 15 & $(50,9)$ & 10 & 2 & YES & YES & NO(2) & $1.60$ & $(6,1)$ & NO & 9489\\
$(262,107)$ & 13 & $(262,107)$ & 13 & 262 & YES & YES & YES & $1.78$ & $(4,2)$ & NO & 9490\\
$(263,109)$ & 12 & $(2,1)$ & 1 & 1 & YES & YES & YES & $1.50$ & $(4,2)$ & -- & 9491\\
$(263,111)$ & 12 & $(2,1)$ & 1 & 1 & YES & YES & YES & $1.67$ & $(2,3)$ & -- & 9492\\
$(263,60)$ & 13 & $(3,1)$ & 2 & 1 & YES & YES & YES & $1.50$ & $(4,2)$ & NO & 9493\\
$(263,60)$ & 13 & $(3,1)$ & 2 & 1 & YES & YES & YES & $1.50$ & $(4,2)$ & -- & 9494\\
$(263,71)$ & 12 & $(3,1)$ & 2 & 1 & YES & YES & YES & $1.89$ & $(2,3)$ & NO & 9495\\
$(263,100)$ & 12 & $(3,1)$ & 2 & 1 & YES & YES & YES & $1.62$ & $(4,2)$ & -- & 9496\\
$(263,109)$ & 12 & $(3,1)$ & 2 & 1 & YES & YES & YES & $1.62$ & $(4,2)$ & -- & 9497\\
$(263,109)$ & 12 & $(3,1)$ & 2 & 1 & YES & YES & YES & $1.75$ & $(2,3)$ & NO & 9498\\
$(263,111)$ & 12 & $(3,1)$ & 2 & 1 & YES & YES & YES & $1.57$ & $(4,2)$ & -- & 9499\\
$(263,111)$ & 12 & $(3,1)$ & 2 & 1 & YES & YES & YES & $1.75$ & $(4,2)$ & NO & 9500\\
$(263,100)$ & 12 & $(4,1)$ & 3 & 1 & YES & YES & YES & $1.62$ & $(4,2)$ & -- & 9501\\
$(263,109)$ & 12 & $(4,1)$ & 3 & 1 & YES & YES & YES & $1.43$ & $(4,2)$ & -- & 9502\\
$(263,109)$ & 12 & $(4,1)$ & 3 & 1 & YES & YES & YES & $1.78$ & $(2,3)$ & NO & 9503\\
$(263,111)$ & 12 & $(4,1)$ & 3 & 1 & YES & YES & YES & $1.75$ & $(2,3)$ & NO & 9504\\
$(263,111)$ & 12 & $(4,1)$ & 3 & 1 & YES & YES & YES & $1.75$ & $(2,3)$ & -- & 9505\\
$(263,57)$ & 13 & $(5,2)$ & 3 & 1 & YES & YES & YES & $1.50$ & $(2,3)$ & NO & 9506\\
$(263,57)$ & 13 & $(5,2)$ & 3 & 1 & YES & YES & YES & $1.80$ & $(2,3)$ & -- & 9507\\
$(263,60)$ & 13 & $(5,2)$ & 3 & 1 & YES & YES & YES & $1.43$ & $(4,2)$ & -- & 9508\\
$(263,60)$ & 13 & $(5,2)$ & 3 & 1 & YES & YES & YES & $1.71$ & $(2,3)$ & NO & 9509\\
$(263,71)$ & 12 & $(5,2)$ & 3 & 1 & YES & YES & YES & $1.50$ & $(4,2)$ & -- & 9510\\
$(263,71)$ & 12 & $(5,2)$ & 3 & 1 & YES & YES & YES & $1.86$ & $(2,3)$ & NO & 9511\\
$(263,100)$ & 12 & $(5,1)$ & 4 & 1 & YES & YES & YES & $1.62$ & $(4,2)$ & NO & 9512\\
$(263,102)$ & 13 & $(6,1)$ & 5 & 1 & YES & YES & YES & $1.75$ & $(4,2)$ & NO & 9513\\
$(263,57)$ & 13 & $(7,2)$ & 4 & 1 & YES & YES & YES & $1.50$ & $(2,3)$ & NO & 9514\\
$(263,109)$ & 12 & $(7,3)$ & 4 & 1 & YES & YES & YES & $1.67$ & $(4,2)$ & NO & 9515\\
$(263,57)$ & 13 & $(9,2)$ & 5 & 1 & YES & YES & YES & $1.50$ & $(2,3)$ & NO & 9516\\
$(263,71)$ & 12 & $(10,3)$ & 5 & 1 & YES & YES & YES & $1.57$ & $(2,3)$ & NO & 9517\\
$(263,111)$ & 12 & $(12,5)$ & 5 & 1 & YES & YES & YES & $1.43$ & $(6,1)$ & NO & 9518\\
$(263,100)$ & 12 & $(13,5)$ & 5 & 1 & YES & YES & YES & $1.70$ & $(2,3)$ & NO & 9519\\
$(263,71)$ & 12 & $(15,4)$ & 6 & 1 & YES & YES & YES & $1.43$ & $(6,1)$ & 9952 & 9520\\
$(263,61)$ & 14 & $(17,4)$ & 7 & 1 & YES & YES & YES & $1.62$ & $(2,3)$ & NO & 9521\\
$(263,109)$ & 12 & $(17,7)$ & 6 & 1 & YES & YES & YES & $1.78$ & $(2,3)$ & NO & 9522\\
$(263,111)$ & 12 & $(26,11)$ & 7 & 1 & YES & YES & YES & $1.57$ & $(4,2)$ & NO & 9523\\
$(263,109)$ & 12 & $(29,12)$ & 7 & 1 & YES & YES & YES & $1.78$ & $(2,3)$ & 6730 & 9524\\
$(263,60)$ & 13 & $(31,7)$ & 8 & 1 & YES & YES & YES & $1.88$ & $(2,3)$ & NO & 9525\\
$(263,60)$ & 13 & $(48,11)$ & 9 & 1 & YES & YES & YES & $1.50$ & $(4,2)$ & NO & 9526\\
$(263,100)$ & 12 & $(50,19)$ & 8 & 1 & YES & YES & YES & $1.67$ & $(2,3)$ & NO & 9527\\
$(263,57)$ & 13 & $(51,11)$ & 9 & 1 & YES & YES & YES & $1.80$ & $(2,3)$ & NO & 9528\\
$(263,109)$ & 12 & $(70,29)$ & 9 & 1 & YES & YES & YES & $1.62$ & $(2,3)$ & NO & 9529\\
$(263,71)$ & 12 & $(89,24)$ & 10 & 1 & YES & YES & YES & $1.29$ & $(4,2)$ & NO & 9530\\
$(263,60)$ & 13 & $(92,21)$ & 10 & 1 & YES & YES & YES & $1.29$ & $(6,1)$ & NO & 9531\\
$(263,109)$ & 12 & $(111,46)$ & 10 & 1 & YES & YES & YES & $1.78$ & $(2,3)$ & NO & 9532\\
$(263,100)$ & 12 & $(121,46)$ & 10 & 1 & YES & YES & YES & $1.62$ & $(4,2)$ & 10524 & 9533\\
$(263,71)$ & 12 & $(137,37)$ & 11 & 1 & YES & YES & YES & $1.50$ & $(4,2)$ & NO & 9534\\
$(263,109)$ & 12 & $(152,63)$ & 11 & 1 & YES & YES & YES & $1.43$ & $(4,2)$ & NO & 9535\\
$(263,111)$ & 12 & $(154,65)$ & 11 & 1 & YES & YES & YES & $1.75$ & $(2,3)$ & NO & 9536\\
$(263,100)$ & 12 & $(192,73)$ & 11 & 1 & YES & YES & YES & $1.62$ & $(4,2)$ & NO & 9537\\
$(263,100)$ & 12 & $(263,100)$ & 12 & 263 & YES & YES & YES & $1.62$ & $(4,2)$ & NO & 9538\\
$(263,109)$ & 12 & $(263,109)$ & 12 & 263 & YES & YES & YES & $1.62$ & $(4,2)$ & NO & 9539\\
$(263,111)$ & 12 & $(263,111)$ & 12 & 263 & YES & YES & YES & $1.43$ & $(4,2)$ & NO & 9540\\
$(264,109)$ & 12 & $(2,1)$ & 1 & 2 & YES & YES & YES & $1.67$ & $(2,3)$ & -- & 9541\\
$(264,115)$ & 12 & $(2,1)$ & 1 & 2 & YES & YES & YES & $1.62$ & $(4,2)$ & -- & 9542\\
$(264,101)$ & 12 & $(3,1)$ & 2 & 3 & YES & YES & YES & $1.62$ & $(4,2)$ & NO & 9543\\
$(264,101)$ & 12 & $(3,1)$ & 2 & 3 & YES & YES & YES & $1.71$ & $(2,3)$ & -- & 9544\\
$(264,109)$ & 12 & $(3,1)$ & 2 & 3 & YES & YES & YES & $1.57$ & $(4,2)$ & -- & 9545\\
$(264,115)$ & 12 & $(3,1)$ & 2 & 3 & YES & YES & YES & $1.57$ & $(4,2)$ & -- & 9546\\
$(264,115)$ & 12 & $(3,1)$ & 2 & 3 & YES & YES & YES & $1.62$ & $(4,2)$ & NO & 9547\\
$(264,101)$ & 12 & $(4,1)$ & 3 & 4 & YES & YES & YES & $1.89$ & $(2,3)$ & NO & 9548\\
$(264,115)$ & 12 & $(5,2)$ & 3 & 1 & YES & YES & YES & $1.43$ & $(4,2)$ & NO & 9549\\
$(264,109)$ & 12 & $(7,2)$ & 4 & 1 & YES & YES & YES & $1.62$ & $(4,2)$ & NO & 9550\\
$(264,109)$ & 12 & $(7,2)$ & 4 & 1 & YES & YES & YES & $1.62$ & $(4,2)$ & -- & 9551\\
$(264,109)$ & 12 & $(12,5)$ & 5 & 12 & YES & YES & YES & $1.62$ & $(2,3)$ & NO & 9552\\
$(264,101)$ & 12 & $(47,18)$ & 8 & 1 & YES & YES & YES & $1.57$ & $(2,3)$ & 6674 & 9553\\
$(264,115)$ & 12 & $(62,27)$ & 9 & 2 & YES & YES & YES & $1.57$ & $(4,2)$ & 8572 & 9554\\
$(264,109)$ & 12 & $(63,26)$ & 9 & 3 & YES & YES & YES & $1.43$ & $(2,3)$ & NO & 9555\\
$(264,109)$ & 12 & $(109,45)$ & 10 & 1 & YES & YES & YES & $1.67$ & $(2,3)$ & NO & 9556\\
$(264,101)$ & 12 & $(115,44)$ & 10 & 1 & YES & YES & YES & $1.67$ & $(2,3)$ & NO & 9557\\
$(265,73)$ & 12 & $(2,1)$ & 1 & 1 & YES & YES & NO(2) & $1.56$ & $(2,3)$ & -- & 9558\\
$(265,74)$ & 12 & $(2,1)$ & 1 & 1 & YES & YES & YES & $1.56$ & $(2,3)$ & NO & 9559\\
$(265,97)$ & 12 & $(2,1)$ & 1 & 1 & YES & YES & YES & $1.67$ & $(2,3)$ & -- & 9560\\
$(265,97)$ & 12 & $(2,1)$ & 1 & 1 & YES & YES & YES & $1.43$ & $(4,2)$ & NO & 9561\\
$(265,111)$ & 12 & $(2,1)$ & 1 & 1 & NO & YES & NO(2) & $1.50$ & $(6,1)$ & -- & 9562\\
$(265,73)$ & 12 & $(3,1)$ & 2 & 1 & YES & YES & YES & $1.56$ & $(2,3)$ & -- & 9563\\
$(265,74)$ & 12 & $(3,1)$ & 2 & 1 & YES & YES & YES & $1.50$ & $(2,3)$ & -- & 9564\\
$(265,97)$ & 12 & $(3,1)$ & 2 & 1 & YES & YES & YES & $1.57$ & $(4,2)$ & -- & 9565\\
$(265,98)$ & 12 & $(3,1)$ & 2 & 1 & YES & YES & YES & $1.67$ & $(2,3)$ & -- & 9566\\
$(265,73)$ & 12 & $(4,1)$ & 3 & 1 & YES & YES & NO(2) & $1.56$ & $(2,3)$ & NO & 9567\\
$(265,97)$ & 12 & $(4,1)$ & 3 & 1 & YES & YES & YES & $1.90$ & $(2,3)$ & -- & 9568\\
$(265,112)$ & 12 & $(4,1)$ & 3 & 1 & YES & YES & YES & $1.43$ & $(4,2)$ & -- & 9569\\
$(265,112)$ & 12 & $(4,1)$ & 3 & 1 & YES & YES & YES & $1.75$ & $(4,2)$ & NO & 9570\\
$(265,73)$ & 12 & $(5,2)$ & 3 & 5 & YES & YES & YES & $1.71$ & $(2,3)$ & -- & 9571\\
$(265,74)$ & 12 & $(5,2)$ & 3 & 5 & YES & YES & YES & $1.43$ & $(4,2)$ & -- & 9572\\
$(265,97)$ & 12 & $(5,2)$ & 3 & 5 & YES & YES & YES & $1.57$ & $(4,2)$ & NO & 9573\\
$(265,98)$ & 12 & $(5,1)$ & 4 & 5 & YES & YES & YES & $1.86$ & $(2,3)$ & NO & 9574\\
$(265,112)$ & 12 & $(5,2)$ & 3 & 5 & YES & YES & YES & $1.43$ & $(4,2)$ & NO & 9575\\
$(265,73)$ & 12 & $(7,2)$ & 4 & 1 & YES & YES & YES & $1.50$ & $(2,3)$ & NO & 9576\\
$(265,97)$ & 12 & $(8,3)$ & 4 & 1 & YES & YES & YES & $1.90$ & $(2,3)$ & NO & 9577\\
$(265,73)$ & 12 & $(9,2)$ & 5 & 1 & YES & YES & YES & $1.62$ & $(4,2)$ & NO & 9578\\
$(265,73)$ & 12 & $(10,3)$ & 5 & 5 & YES & YES & YES & $1.75$ & $(2,3)$ & NO & 9579\\
$(265,98)$ & 12 & $(11,4)$ & 5 & 1 & YES & YES & YES & $1.56$ & $(2,3)$ & NO & 9580\\
$(265,62)$ & 14 & $(13,3)$ & 6 & 1 & YES & YES & YES & $1.62$ & $(2,3)$ & NO & 9581\\
$(265,97)$ & 12 & $(19,7)$ & 6 & 1 & YES & YES & YES & $1.89$ & $(4,2)$ & NO & 9582\\
$(265,98)$ & 12 & $(19,7)$ & 6 & 1 & YES & YES & YES & $1.56$ & $(2,3)$ & NO & 9583\\
$(265,73)$ & 12 & $(25,7)$ & 7 & 5 & YES & YES & YES & $1.71$ & $(2,3)$ & NO & 9584\\
$(265,97)$ & 12 & $(30,11)$ & 7 & 5 & YES & YES & YES & $1.90$ & $(2,3)$ & 6753 & 9585\\
$(265,97)$ & 12 & $(112,41)$ & 10 & 1 & YES & YES & YES & $1.67$ & $(2,3)$ & NO & 9586\\
$(265,98)$ & 12 & $(119,44)$ & 10 & 1 & YES & YES & YES & $1.71$ & $(2,3)$ & 10484 & 9587\\
$(265,73)$ & 12 & $(127,35)$ & 11 & 1 & YES & YES & YES & $1.57$ & $(2,3)$ & NO & 9588\\
$(265,62)$ & 14 & $(171,40)$ & 12 & 1 & YES & YES & NO(2) & $1.56$ & $(8,0)$ & 11311 & 9589\\
$(265,98)$ & 12 & $(265,98)$ & 12 & 265 & YES & YES & YES & $1.86$ & $(2,3)$ & NO & 9590\\
$(265,111)$ & 12 & $(265,111)$ & 12 & 265 & YES & YES & YES & $1.67$ & $(4,2)$ & NO & 9591\\
$(266,79)$ & 12 & $(2,1)$ & 1 & 2 & YES & YES & YES & $1.56$ & $(2,3)$ & -- & 9592\\
$(266,79)$ & 12 & $(4,1)$ & 3 & 2 & YES & YES & YES & $1.50$ & $(2,3)$ & NO & 9593\\
$(266,101)$ & 12 & $(4,1)$ & 3 & 2 & YES & YES & YES & $1.62$ & $(2,3)$ & -- & 9594\\
$(266,101)$ & 12 & $(4,1)$ & 3 & 2 & YES & YES & YES & $1.88$ & $(2,3)$ & NO & 9595\\
$(266,101)$ & 12 & $(5,2)$ & 3 & 1 & YES & YES & YES & $1.90$ & $(2,3)$ & NO & 9596\\
$(266,79)$ & 12 & $(7,2)$ & 4 & 7 & YES & YES & YES & $1.62$ & $(2,3)$ & NO & 9597\\
$(266,79)$ & 12 & $(27,8)$ & 7 & 1 & YES & YES & YES & $1.62$ & $(2,3)$ & 9112 & 9598\\
$(266,79)$ & 12 & $(64,19)$ & 9 & 2 & YES & YES & YES & $1.50$ & $(2,3)$ & 8723 & 9599\\
$(266,79)$ & 12 & $(101,30)$ & 10 & 1 & YES & YES & YES & $1.62$ & $(2,3)$ & NO & 9600\\
$(266,101)$ & 12 & $(108,41)$ & 10 & 2 & YES & YES & YES & $1.88$ & $(2,3)$ & 10222 & 9601\\
$(266,101)$ & 12 & $(187,71)$ & 11 & 1 & YES & YES & YES & $1.62$ & $(2,3)$ & NO & 9602\\
$(267,79)$ & 12 & $(2,1)$ & 1 & 1 & YES & YES & YES & $1.62$ & $(2,3)$ & NO & 9603\\
$(267,79)$ & 12 & $(2,1)$ & 1 & 1 & YES & YES & YES & $1.62$ & $(2,3)$ & -- & 9604\\
$(267,98)$ & 12 & $(2,1)$ & 1 & 1 & YES & YES & YES & $1.57$ & $(4,2)$ & NO & 9605\\
$(267,98)$ & 12 & $(2,1)$ & 1 & 1 & YES & YES & YES & $1.71$ & $(4,2)$ & -- & 9606\\
$(267,79)$ & 12 & $(3,1)$ & 2 & 3 & YES & YES & YES & $1.50$ & $(2,3)$ & -- & 9607\\
$(267,79)$ & 12 & $(3,1)$ & 2 & 3 & YES & YES & YES & $1.50$ & $(2,3)$ & NO & 9608\\
$(267,98)$ & 12 & $(3,1)$ & 2 & 3 & YES & YES & YES & $1.57$ & $(4,2)$ & -- & 9609\\
$(267,98)$ & 12 & $(3,1)$ & 2 & 3 & YES & YES & YES & $1.78$ & $(2,3)$ & NO & 9610\\
$(267,62)$ & 14 & $(4,1)$ & 3 & 1 & YES & YES & YES & $1.43$ & $(2,3)$ & -- & 9611\\
$(267,79)$ & 12 & $(5,2)$ & 3 & 1 & YES & YES & YES & $1.62$ & $(2,3)$ & NO & 9612\\
$(267,98)$ & 12 & $(5,2)$ & 3 & 1 & YES & YES & YES & $1.62$ & $(2,3)$ & NO & 9613\\
$(267,98)$ & 12 & $(8,3)$ & 4 & 1 & YES & YES & YES & $1.62$ & $(2,3)$ & NO & 9614\\
$(267,79)$ & 12 & $(11,3)$ & 5 & 1 & YES & YES & YES & $1.71$ & $(2,3)$ & NO & 9615\\
$(267,79)$ & 12 & $(13,4)$ & 6 & 1 & YES & YES & YES & $1.62$ & $(2,3)$ & NO & 9616\\
$(267,62)$ & 14 & $(17,4)$ & 7 & 1 & YES & YES & YES & $1.57$ & $(2,3)$ & NO & 9617\\
$(267,98)$ & 12 & $(19,7)$ & 6 & 1 & YES & YES & YES & $1.62$ & $(2,3)$ & NO & 9618\\
$(267,79)$ & 12 & $(37,11)$ & 8 & 1 & YES & YES & YES & $1.50$ & $(2,3)$ & NO & 9619\\
$(267,79)$ & 12 & $(44,13)$ & 8 & 1 & YES & YES & YES & $1.38$ & $(2,3)$ & 10331 & 9620\\
$(267,98)$ & 12 & $(49,18)$ & 8 & 1 & YES & YES & YES & $1.90$ & $(2,3)$ & NO & 9621\\
$(267,98)$ & 12 & $(79,29)$ & 9 & 1 & YES & YES & YES & $1.71$ & $(4,2)$ & NO & 9622\\
$(267,98)$ & 12 & $(267,98)$ & 12 & 267 & YES & YES & YES & $1.71$ & $(2,3)$ & NO & 9623\\
$(268,111)$ & 12 & $(2,1)$ & 1 & 2 & YES & YES & YES & $1.62$ & $(4,2)$ & -- & 9624\\
$(268,111)$ & 12 & $(3,1)$ & 2 & 1 & YES & YES & YES & $1.62$ & $(2,3)$ & -- & 9625\\
$(268,111)$ & 12 & $(3,1)$ & 2 & 1 & YES & YES & YES & $1.62$ & $(2,3)$ & NO & 9626\\
$(268,111)$ & 12 & $(4,1)$ & 3 & 4 & YES & YES & YES & $1.43$ & $(6,1)$ & -- & 9627\\
$(268,97)$ & 13 & $(5,1)$ & 4 & 1 & YES & YES & NO(2) & $1.44$ & $(8,0)$ & NO & 9628\\
$(268,99)$ & 12 & $(5,2)$ & 3 & 1 & YES & YES & YES & $1.62$ & $(4,2)$ & -- & 9629\\
$(268,99)$ & 12 & $(5,2)$ & 3 & 1 & YES & YES & YES & $1.43$ & $(6,1)$ & NO & 9630\\
$(268,113)$ & 13 & $(5,2)$ & 3 & 1 & YES & YES & NO(2) & $1.44$ & $(8,0)$ & NO & 9631\\
$(268,99)$ & 12 & $(7,2)$ & 4 & 1 & YES & YES & YES & $1.62$ & $(4,2)$ & -- & 9632\\
$(268,111)$ & 12 & $(7,3)$ & 4 & 1 & YES & YES & YES & $1.56$ & $(2,3)$ & NO & 9633\\
$(268,111)$ & 12 & $(17,7)$ & 6 & 1 & YES & YES & YES & $1.43$ & $(4,2)$ & NO & 9634\\
$(268,99)$ & 12 & $(27,10)$ & 7 & 1 & YES & YES & YES & $1.57$ & $(6,1)$ & NO & 9635\\
$(268,111)$ & 12 & $(41,17)$ & 8 & 1 & YES & YES & YES & $1.67$ & $(2,3)$ & 10171 & 9636\\
$(268,97)$ & 13 & $(58,21)$ & 10 & 2 & YES & YES & NO(2) & $1.62$ & $(6,1)$ & 8517 & 9637\\
$(268,111)$ & 12 & $(70,29)$ & 9 & 2 & YES & YES & YES & $1.67$ & $(2,3)$ & 8921 & 9638\\
$(268,111)$ & 12 & $(99,41)$ & 10 & 1 & YES & YES & YES & $1.78$ & $(2,3)$ & NO & 9639\\
$(268,111)$ & 12 & $(169,70)$ & 11 & 1 & YES & YES & YES & $1.62$ & $(2,3)$ & NO & 9640\\
$(268,111)$ & 12 & $(268,111)$ & 12 & 268 & YES & YES & YES & $1.62$ & $(2,3)$ & NO & 9641\\
$(269,78)$ & 13 & $(2,1)$ & 1 & 1 & YES & YES & YES & $1.57$ & $(2,3)$ & NO & 9642\\
$(269,104)$ & 12 & $(2,1)$ & 1 & 1 & NO & YES & NO(2) & $1.60$ & $(6,1)$ & -- & 9643\\
$(269,72)$ & 13 & $(3,1)$ & 2 & 1 & YES & YES & NO(2) & $1.38$ & $(6,1)$ & NO & 9644\\
$(269,104)$ & 12 & $(3,1)$ & 2 & 1 & YES & YES & YES & $1.78$ & $(2,3)$ & -- & 9645\\
$(269,104)$ & 12 & $(3,1)$ & 2 & 1 & YES & YES & YES & $1.78$ & $(2,3)$ & NO & 9646\\
$(269,104)$ & 12 & $(4,1)$ & 3 & 1 & YES & YES & YES & $1.78$ & $(2,3)$ & -- & 9647\\
$(269,75)$ & 12 & $(5,2)$ & 3 & 1 & YES & YES & YES & $1.62$ & $(2,3)$ & -- & 9648\\
$(269,104)$ & 12 & $(5,2)$ & 3 & 1 & YES & YES & YES & $1.89$ & $(2,3)$ & NO & 9649\\
$(269,104)$ & 12 & $(8,3)$ & 4 & 1 & YES & YES & YES & $1.70$ & $(2,3)$ & NO & 9650\\
$(269,75)$ & 12 & $(9,2)$ & 5 & 1 & YES & YES & YES & $1.57$ & $(2,3)$ & NO & 9651\\
$(269,104)$ & 12 & $(18,7)$ & 6 & 1 & YES & YES & YES & $1.62$ & $(4,2)$ & NO & 9652\\
$(269,75)$ & 12 & $(25,7)$ & 7 & 1 & YES & YES & YES & $1.50$ & $(2,3)$ & NO & 9653\\
$(269,104)$ & 12 & $(44,17)$ & 8 & 1 & YES & YES & YES & $1.67$ & $(2,3)$ & NO & 9654\\
$(269,104)$ & 12 & $(119,46)$ & 10 & 1 & YES & YES & YES & $1.80$ & $(2,3)$ & 10525 & 9655\\
$(269,75)$ & 12 & $(147,41)$ & 11 & 1 & YES & YES & YES & $1.62$ & $(2,3)$ & NO & 9656\\
$(269,104)$ & 12 & $(194,75)$ & 11 & 1 & YES & YES & YES & $1.78$ & $(2,3)$ & NO & 9657\\
$(269,104)$ & 12 & $(269,104)$ & 12 & 269 & YES & YES & YES & $1.43$ & $(4,2)$ & NO & 9658\\
$(270,97)$ & 12 & $(2,1)$ & 1 & 2 & YES & YES & YES & $1.43$ & $(4,2)$ & -- & 9659\\
$(270,103)$ & 12 & $(2,1)$ & 1 & 2 & YES & YES & YES & $1.78$ & $(2,3)$ & -- & 9660\\
$(270,103)$ & 12 & $(3,1)$ & 2 & 3 & YES & YES & YES & $1.67$ & $(2,3)$ & -- & 9661\\
$(270,103)$ & 12 & $(5,1)$ & 4 & 5 & YES & YES & YES & $1.90$ & $(2,3)$ & -- & 9662\\
$(270,97)$ & 12 & $(11,4)$ & 5 & 1 & YES & YES & YES & $1.57$ & $(4,2)$ & 9321 & 9663\\
$(270,103)$ & 12 & $(11,4)$ & 5 & 1 & YES & YES & YES & $1.78$ & $(4,2)$ & NO & 9664\\
$(270,103)$ & 12 & $(13,5)$ & 5 & 1 & YES & YES & YES & $1.78$ & $(2,3)$ & NO & 9665\\
$(270,103)$ & 12 & $(76,29)$ & 9 & 2 & YES & YES & YES & $1.62$ & $(2,3)$ & 9189 & 9666\\
$(270,103)$ & 12 & $(173,66)$ & 11 & 1 & YES & YES & YES & $1.50$ & $(2,3)$ & NO & 9667\\
$(271,75)$ & 12 & $(2,1)$ & 1 & 1 & YES & YES & YES & $1.50$ & $(2,3)$ & -- & 9668\\
$(271,83)$ & 13 & $(2,1)$ & 1 & 1 & YES & YES & NO(2) & $1.50$ & $(6,1)$ & -- & 9669\\
$(271,105)$ & 12 & $(2,1)$ & 1 & 1 & YES & YES & YES & $1.70$ & $(2,3)$ & -- & 9670\\
$(271,112)$ & 12 & $(2,1)$ & 1 & 1 & YES & YES & YES & $1.89$ & $(2,3)$ & -- & 9671\\
$(271,72)$ & 14 & $(3,1)$ & 2 & 1 & YES & YES & NO(2) & $1.70$ & $(6,1)$ & -- & 9672\\
$(271,72)$ & 14 & $(3,1)$ & 2 & 1 & YES & YES & NO(2) & $1.80$ & $(6,1)$ & NO & 9673\\
$(271,75)$ & 12 & $(3,1)$ & 2 & 1 & YES & YES & YES & $1.67$ & $(2,3)$ & -- & 9674\\
$(271,80)$ & 12 & $(3,1)$ & 2 & 1 & YES & YES & YES & $1.50$ & $(2,3)$ & -- & 9675\\
$(271,82)$ & 13 & $(3,1)$ & 2 & 1 & YES & YES & YES & $1.50$ & $(2,3)$ & -- & 9676\\
$(271,82)$ & 13 & $(3,1)$ & 2 & 1 & YES & YES & YES & $1.89$ & $(2,3)$ & NO & 9677\\
$(271,105)$ & 12 & $(3,1)$ & 2 & 1 & YES & YES & YES & $1.57$ & $(4,2)$ & -- & 9678\\
$(271,112)$ & 12 & $(3,1)$ & 2 & 1 & YES & YES & YES & $1.62$ & $(2,3)$ & -- & 9679\\
$(271,112)$ & 12 & $(4,1)$ & 3 & 1 & YES & YES & YES & $1.62$ & $(2,3)$ & -- & 9680\\
$(271,82)$ & 13 & $(7,2)$ & 4 & 1 & YES & YES & YES & $2.00$ & $(2,3)$ & NO & 9681\\
$(271,112)$ & 12 & $(12,5)$ & 5 & 1 & YES & YES & YES & $1.78$ & $(2,3)$ & NO & 9682\\
$(271,105)$ & 12 & $(13,5)$ & 5 & 1 & YES & YES & YES & $1.89$ & $(2,3)$ & 7014 & 9683\\
$(271,112)$ & 12 & $(17,7)$ & 6 & 1 & YES & YES & YES & $1.62$ & $(4,2)$ & NO & 9684\\
$(271,82)$ & 13 & $(23,7)$ & 7 & 1 & YES & YES & YES & $2.00$ & $(2,3)$ & NO & 9685\\
$(271,105)$ & 12 & $(23,9)$ & 7 & 1 & YES & YES & YES & $1.62$ & $(4,2)$ & NO & 9686\\
$(271,75)$ & 12 & $(25,7)$ & 7 & 1 & YES & YES & YES & $1.71$ & $(2,3)$ & NO & 9687\\
$(271,80)$ & 12 & $(27,8)$ & 7 & 1 & YES & YES & YES & $1.50$ & $(2,3)$ & NO & 9688\\
$(271,112)$ & 12 & $(29,12)$ & 7 & 1 & YES & YES & YES & $1.89$ & $(2,3)$ & NO & 9689\\
$(271,112)$ & 12 & $(46,19)$ & 8 & 1 & YES & YES & YES & $1.43$ & $(4,2)$ & NO & 9690\\
$(271,80)$ & 12 & $(61,18)$ & 9 & 1 & YES & YES & YES & $1.38$ & $(2,3)$ & 8657 & 9691\\
$(271,75)$ & 12 & $(65,18)$ & 9 & 1 & YES & YES & YES & $1.56$ & $(2,3)$ & NO & 9692\\
$(271,82)$ & 13 & $(119,36)$ & 11 & 1 & YES & YES & YES & $1.50$ & $(2,3)$ & 10535 & 9693\\
$(271,112)$ & 12 & $(121,50)$ & 10 & 1 & YES & YES & YES & $1.62$ & $(2,3)$ & 10582 & 9694\\
$(271,83)$ & 13 & $(160,49)$ & 12 & 1 & YES & YES & NO(2) & $1.50$ & $(6,1)$ & NO & 9695\\
$(271,112)$ & 12 & $(196,81)$ & 11 & 1 & YES & YES & YES & $1.75$ & $(2,3)$ & NO & 9696\\
$(271,80)$ & 12 & $(227,67)$ & 12 & 1 & YES & YES & YES & $1.56$ & $(4,2)$ & NO & 9697\\
$(271,72)$ & 14 & $(271,72)$ & 14 & 271 & YES & YES & NO(2) & $1.70$ & $(6,1)$ & NO & 9698\\
$(272,103)$ & 12 & $(3,1)$ & 2 & 1 & YES & YES & YES & $1.78$ & $(4,2)$ & NO & 9699\\
$(272,59)$ & 13 & $(4,1)$ & 3 & 4 & YES & YES & NO(2) & $1.44$ & $(4,2)$ & -- & 9700\\
$(272,59)$ & 13 & $(4,1)$ & 3 & 4 & YES & YES & NO(2) & $1.60$ & $(2,3)$ & NO & 9701\\
$(272,83)$ & 13 & $(4,1)$ & 3 & 4 & YES & YES & YES & $1.89$ & $(2,3)$ & NO & 9702\\
$(272,83)$ & 13 & $(4,1)$ & 3 & 4 & YES & YES & YES & $1.88$ & $(2,3)$ & -- & 9703\\
$(272,103)$ & 12 & $(5,2)$ & 3 & 1 & YES & YES & YES & $1.78$ & $(4,2)$ & NO & 9704\\
$(272,83)$ & 13 & $(154,47)$ & 11 & 2 & YES & YES & YES & $1.75$ & $(2,3)$ & 11141 & 9705\\
$(273,58)$ & 13 & $(2,1)$ & 1 & 1 & YES & YES & NO(2) & $1.25$ & $(6,1)$ & NO & 9706\\
$(273,58)$ & 13 & $(3,1)$ & 2 & 3 & YES & YES & NO(2) & $1.44$ & $(4,2)$ & -- & 9707\\
$(273,80)$ & 13 & $(3,1)$ & 2 & 3 & YES & YES & YES & $1.62$ & $(4,2)$ & -- & 9708\\
$(273,80)$ & 13 & $(3,1)$ & 2 & 3 & YES & YES & YES & $1.75$ & $(4,2)$ & NO & 9709\\
$(273,100)$ & 12 & $(3,1)$ & 2 & 3 & YES & YES & YES & $1.57$ & $(4,2)$ & -- & 9710\\
$(273,100)$ & 12 & $(3,1)$ & 2 & 3 & YES & YES & YES & $1.62$ & $(2,3)$ & NO & 9711\\
$(273,58)$ & 13 & $(4,1)$ & 3 & 1 & YES & YES & NO(2) & $1.25$ & $(6,1)$ & NO & 9712\\
$(273,64)$ & 14 & $(4,1)$ & 3 & 1 & YES & YES & YES & $1.43$ & $(2,3)$ & -- & 9713\\
$(273,101)$ & 12 & $(4,1)$ & 3 & 1 & YES & YES & YES & $1.75$ & $(2,3)$ & -- & 9714\\
$(273,76)$ & 13 & $(5,1)$ & 4 & 1 & YES & YES & YES & $1.57$ & $(2,3)$ & NO & 9715\\
$(273,80)$ & 13 & $(5,1)$ & 4 & 1 & YES & YES & YES & $1.62$ & $(4,2)$ & NO & 9716\\
$(273,83)$ & 14 & $(5,1)$ & 4 & 1 & YES & YES & NO(2) & $1.60$ & $(6,1)$ & -- & 9717\\
$(273,100)$ & 12 & $(5,2)$ & 3 & 1 & YES & YES & YES & $1.57$ & $(4,2)$ & NO & 9718\\
$(273,80)$ & 13 & $(17,5)$ & 6 & 1 & YES & YES & NO(2) & $1.67$ & $(2,3)$ & NO & 9719\\
$(273,101)$ & 12 & $(19,7)$ & 6 & 1 & YES & YES & YES & $1.75$ & $(2,3)$ & NO & 9720\\
$(273,80)$ & 13 & $(24,7)$ & 7 & 3 & YES & YES & YES & $1.62$ & $(4,2)$ & NO & 9721\\
$(273,80)$ & 13 & $(41,12)$ & 8 & 1 & YES & YES & YES & $1.62$ & $(4,2)$ & NO & 9722\\
$(273,58)$ & 13 & $(47,10)$ & 9 & 1 & YES & YES & NO(2) & $1.56$ & $(4,2)$ & NO & 9723\\
$(273,64)$ & 14 & $(47,11)$ & 9 & 1 & YES & YES & YES & $1.57$ & $(2,3)$ & NO & 9724\\
$(273,80)$ & 13 & $(99,29)$ & 10 & 3 & YES & YES & YES & $1.57$ & $(4,2)$ & NO & 9725\\
$(273,80)$ & 13 & $(215,63)$ & 12 & 1 & YES & YES & YES & $1.43$ & $(4,2)$ & NO & 9726\\
$(273,76)$ & 13 & $(273,76)$ & 13 & 273 & YES & YES & YES & $1.75$ & $(4,2)$ & NO & 9727\\
$(273,101)$ & 12 & $(273,101)$ & 12 & 273 & YES & YES & YES & $1.43$ & $(2,3)$ & NO & 9728\\
$(274,81)$ & 12 & $(2,1)$ & 1 & 2 & YES & YES & NO(2) & $1.38$ & $(4,2)$ & NO & 9729\\
$(274,81)$ & 12 & $(2,1)$ & 1 & 2 & YES & YES & YES & $1.50$ & $(2,3)$ & -- & 9730\\
$(274,105)$ & 12 & $(2,1)$ & 1 & 2 & NO & YES & NO(2) & $1.73$ & $(4,2)$ & -- & 9731\\
$(274,105)$ & 12 & $(2,1)$ & 1 & 2 & YES & YES & YES & $1.89$ & $(2,3)$ & NO & 9732\\
$(274,107)$ & 12 & $(2,1)$ & 1 & 2 & NO & YES & NO(2) & $1.50$ & $(6,1)$ & -- & 9733\\
$(274,115)$ & 12 & $(2,1)$ & 1 & 2 & YES & YES & YES & $1.43$ & $(4,2)$ & -- & 9734\\
$(274,81)$ & 12 & $(3,1)$ & 2 & 1 & YES & YES & YES & $1.50$ & $(2,3)$ & -- & 9735\\
$(274,81)$ & 12 & $(3,1)$ & 2 & 1 & YES & YES & YES & $1.62$ & $(2,3)$ & NO & 9736\\
$(274,105)$ & 12 & $(3,1)$ & 2 & 1 & YES & YES & YES & $1.62$ & $(2,3)$ & NO & 9737\\
$(274,105)$ & 12 & $(3,1)$ & 2 & 1 & YES & YES & YES & $1.62$ & $(2,3)$ & -- & 9738\\
$(274,107)$ & 12 & $(3,1)$ & 2 & 1 & YES & YES & YES & $1.62$ & $(4,2)$ & -- & 9739\\
$(274,107)$ & 12 & $(3,1)$ & 2 & 1 & YES & YES & YES & $1.75$ & $(4,2)$ & NO & 9740\\
$(274,115)$ & 12 & $(3,1)$ & 2 & 1 & YES & YES & YES & $1.43$ & $(4,2)$ & -- & 9741\\
$(274,115)$ & 12 & $(3,1)$ & 2 & 1 & YES & YES & YES & $1.80$ & $(2,3)$ & NO & 9742\\
$(274,81)$ & 12 & $(4,1)$ & 3 & 2 & YES & YES & YES & $1.75$ & $(2,3)$ & -- & 9743\\
$(274,81)$ & 12 & $(5,2)$ & 3 & 1 & YES & YES & YES & $1.86$ & $(2,3)$ & -- & 9744\\
$(274,105)$ & 12 & $(5,1)$ & 4 & 1 & YES & YES & YES & $1.78$ & $(2,3)$ & -- & 9745\\
$(274,105)$ & 12 & $(5,2)$ & 3 & 1 & YES & YES & YES & $1.78$ & $(4,2)$ & -- & 9746\\
$(274,107)$ & 12 & $(5,1)$ & 4 & 1 & YES & YES & YES & $1.50$ & $(4,2)$ & NO & 9747\\
$(274,115)$ & 12 & $(5,2)$ & 3 & 1 & YES & YES & YES & $1.43$ & $(4,2)$ & NO & 9748\\
$(274,115)$ & 12 & $(5,2)$ & 3 & 1 & YES & YES & YES & $1.57$ & $(6,1)$ & -- & 9749\\
$(274,81)$ & 12 & $(7,2)$ & 4 & 1 & YES & YES & NO(2) & $1.40$ & $(6,1)$ & NO & 9750\\
$(274,81)$ & 12 & $(9,2)$ & 5 & 1 & YES & YES & YES & $1.62$ & $(2,3)$ & NO & 9751\\
$(274,81)$ & 12 & $(10,3)$ & 5 & 2 & YES & YES & NO(2) & $1.38$ & $(4,2)$ & NO & 9752\\
$(274,81)$ & 12 & $(11,3)$ & 5 & 1 & YES & YES & YES & $1.56$ & $(2,3)$ & NO & 9753\\
$(274,115)$ & 12 & $(12,5)$ & 5 & 2 & YES & YES & YES & $1.67$ & $(4,2)$ & 6947 & 9754\\
$(274,81)$ & 12 & $(13,4)$ & 6 & 1 & YES & YES & YES & $1.78$ & $(2,3)$ & NO & 9755\\
$(274,105)$ & 12 & $(18,7)$ & 6 & 2 & YES & YES & YES & $1.67$ & $(4,2)$ & NO & 9756\\
$(274,115)$ & 12 & $(19,8)$ & 6 & 1 & YES & YES & YES & $1.62$ & $(4,2)$ & NO & 9757\\
$(274,81)$ & 12 & $(24,7)$ & 7 & 2 & YES & YES & YES & $1.70$ & $(2,3)$ & NO & 9758\\
$(274,115)$ & 12 & $(31,13)$ & 7 & 1 & YES & YES & YES & $2.00$ & $(2,3)$ & NO & 9759\\
$(274,65)$ & 14 & $(38,9)$ & 9 & 2 & YES & YES & YES & $1.62$ & $(2,3)$ & NO & 9760\\
$(274,105)$ & 12 & $(47,18)$ & 8 & 1 & YES & YES & YES & $1.78$ & $(2,3)$ & NO & 9761\\
$(274,115)$ & 12 & $(50,21)$ & 8 & 2 & YES & YES & YES & $1.43$ & $(2,3)$ & NO & 9762\\
$(274,81)$ & 12 & $(71,21)$ & 9 & 1 & YES & YES & YES & $1.50$ & $(2,3)$ & NO & 9763\\
$(274,81)$ & 12 & $(159,47)$ & 11 & 1 & YES & YES & YES & $1.38$ & $(2,3)$ & NO & 9764\\
$(274,105)$ & 12 & $(167,64)$ & 11 & 1 & YES & YES & YES & $1.75$ & $(2,3)$ & NO & 9765\\
$(274,105)$ & 12 & $(227,87)$ & 12 & 1 & YES & YES & YES & $1.78$ & $(4,2)$ & NO & 9766\\
$(274,107)$ & 12 & $(233,91)$ & 12 & 1 & YES & YES & YES & $1.78$ & $(4,2)$ & NO & 9767\\
$(274,81)$ & 12 & $(274,81)$ & 12 & 274 & YES & YES & YES & $1.50$ & $(4,2)$ & NO & 9768\\
$(274,115)$ & 12 & $(274,115)$ & 12 & 274 & YES & YES & YES & $1.43$ & $(4,2)$ & NO & 9769\\
$(275,76)$ & 12 & $(2,1)$ & 1 & 1 & YES & YES & YES & $1.78$ & $(2,3)$ & -- & 9770\\
$(275,76)$ & 12 & $(2,1)$ & 1 & 1 & YES & YES & YES & $1.89$ & $(2,3)$ & NO & 9771\\
$(275,76)$ & 12 & $(3,1)$ & 2 & 1 & YES & YES & YES & $1.78$ & $(2,3)$ & NO & 9772\\
$(275,76)$ & 12 & $(3,1)$ & 2 & 1 & YES & YES & NO(2) & $1.67$ & $(4,2)$ & -- & 9773\\
$(275,73)$ & 13 & $(4,1)$ & 3 & 1 & YES & YES & NO(2) & $1.50$ & $(10,-1)$ & -- & 9774\\
$(275,76)$ & 12 & $(4,1)$ & 3 & 1 & YES & YES & YES & $1.50$ & $(4,2)$ & NO & 9775\\
$(275,76)$ & 12 & $(5,1)$ & 4 & 5 & YES & YES & YES & $1.50$ & $(2,3)$ & NO & 9776\\
$(275,76)$ & 12 & $(5,2)$ & 3 & 5 & YES & YES & YES & $1.43$ & $(2,3)$ & -- & 9777\\
$(275,104)$ & 13 & $(5,1)$ & 4 & 5 & YES & YES & NO(2) & $1.50$ & $(6,1)$ & NO & 9778\\
$(275,104)$ & 13 & $(5,1)$ & 4 & 5 & YES & YES & YES & $1.57$ & $(6,1)$ & -- & 9779\\
$(275,76)$ & 12 & $(7,2)$ & 4 & 1 & YES & YES & YES & $1.50$ & $(2,3)$ & NO & 9780\\
$(275,76)$ & 12 & $(10,3)$ & 5 & 5 & YES & YES & YES & $1.57$ & $(2,3)$ & NO & 9781\\
$(275,76)$ & 12 & $(11,3)$ & 5 & 11 & YES & YES & YES & $1.62$ & $(2,3)$ & NO & 9782\\
$(275,76)$ & 12 & $(18,5)$ & 6 & 1 & YES & YES & YES & $1.78$ & $(2,3)$ & NO & 9783\\
$(275,76)$ & 12 & $(25,7)$ & 7 & 25 & YES & YES & YES & $1.43$ & $(2,3)$ & NO & 9784\\
$(275,76)$ & 12 & $(29,8)$ & 7 & 1 & YES & YES & YES & $1.62$ & $(2,3)$ & NO & 9785\\
$(275,104)$ & 13 & $(29,11)$ & 7 & 1 & YES & YES & YES & $1.71$ & $(6,1)$ & NO & 9786\\
$(275,76)$ & 12 & $(40,11)$ & 8 & 5 & YES & YES & YES & $1.57$ & $(2,3)$ & NO & 9787\\
$(275,73)$ & 13 & $(64,17)$ & 10 & 1 & YES & YES & YES & $1.57$ & $(2,3)$ & NO & 9788\\
$(275,76)$ & 12 & $(65,18)$ & 9 & 5 & YES & YES & YES & $1.57$ & $(2,3)$ & 11282 & 9789\\
$(275,76)$ & 12 & $(76,21)$ & 9 & 1 & YES & YES & YES & $1.50$ & $(4,2)$ & NO & 9790\\
$(275,76)$ & 12 & $(275,76)$ & 12 & 275 & YES & YES & YES & $1.50$ & $(2,3)$ & NO & 9791\\
$(277,78)$ & 13 & $(2,1)$ & 1 & 1 & YES & YES & YES & $1.57$ & $(2,3)$ & -- & 9792\\
$(277,81)$ & 12 & $(2,1)$ & 1 & 1 & YES & YES & YES & $1.78$ & $(2,3)$ & -- & 9793\\
$(277,81)$ & 12 & $(2,1)$ & 1 & 1 & YES & YES & YES & $1.62$ & $(2,3)$ & NO & 9794\\
$(277,106)$ & 12 & $(2,1)$ & 1 & 1 & NO & YES & YES & $1.43$ & $(4,2)$ & -- & 9795\\
$(277,106)$ & 12 & $(2,1)$ & 1 & 1 & YES & YES & YES & $1.78$ & $(2,3)$ & NO & 9796\\
$(277,117)$ & 12 & $(2,1)$ & 1 & 1 & YES & YES & NO(2) & $1.60$ & $(6,1)$ & -- & 9797\\
$(277,60)$ & 13 & $(3,1)$ & 2 & 1 & YES & YES & YES & $1.50$ & $(4,2)$ & NO & 9798\\
$(277,60)$ & 13 & $(3,1)$ & 2 & 1 & YES & YES & YES & $1.50$ & $(4,2)$ & -- & 9799\\
$(277,60)$ & 13 & $(3,1)$ & 2 & 1 & YES & YES & YES & $1.62$ & $(2,3)$ & NO & 9800\\
$(277,78)$ & 13 & $(3,1)$ & 2 & 1 & YES & YES & YES & $1.43$ & $(4,2)$ & -- & 9801\\
$(277,78)$ & 13 & $(3,1)$ & 2 & 1 & YES & YES & YES & $1.71$ & $(4,2)$ & NO & 9802\\
$(277,81)$ & 12 & $(3,1)$ & 2 & 1 & YES & YES & YES & $1.50$ & $(4,2)$ & 7232 & 9803\\
$(277,81)$ & 12 & $(3,1)$ & 2 & 1 & YES & YES & YES & $1.50$ & $(2,3)$ & -- & 9804\\
$(277,106)$ & 12 & $(3,1)$ & 2 & 1 & YES & YES & YES & $1.62$ & $(2,3)$ & -- & 9805\\
$(277,116)$ & 12 & $(3,1)$ & 2 & 1 & YES & YES & YES & $1.57$ & $(2,3)$ & -- & 9806\\
$(277,116)$ & 12 & $(3,1)$ & 2 & 1 & YES & YES & YES & $1.86$ & $(2,3)$ & NO & 9807\\
$(277,117)$ & 12 & $(3,1)$ & 2 & 1 & YES & YES & YES & $1.57$ & $(4,2)$ & -- & 9808\\
$(277,81)$ & 12 & $(4,1)$ & 3 & 1 & YES & YES & YES & $1.78$ & $(2,3)$ & NO & 9809\\
$(277,116)$ & 12 & $(4,1)$ & 3 & 1 & YES & YES & YES & $1.62$ & $(4,2)$ & -- & 9810\\
$(277,117)$ & 12 & $(4,1)$ & 3 & 1 & YES & YES & YES & $1.70$ & $(2,3)$ & -- & 9811\\
$(277,60)$ & 13 & $(5,2)$ & 3 & 1 & YES & YES & YES & $1.29$ & $(4,2)$ & -- & 9812\\
$(277,60)$ & 13 & $(5,2)$ & 3 & 1 & YES & YES & YES & $1.43$ & $(2,3)$ & NO & 9813\\
$(277,81)$ & 12 & $(5,2)$ & 3 & 1 & YES & YES & YES & $1.57$ & $(2,3)$ & -- & 9814\\
$(277,116)$ & 12 & $(5,2)$ & 3 & 1 & YES & YES & YES & $1.67$ & $(2,3)$ & NO & 9815\\
$(277,60)$ & 13 & $(6,1)$ & 5 & 1 & YES & YES & YES & $1.57$ & $(6,1)$ & NO & 9816\\
$(277,60)$ & 13 & $(7,2)$ & 4 & 1 & YES & YES & YES & $1.86$ & $(2,3)$ & NO & 9817\\
$(277,106)$ & 12 & $(8,3)$ & 4 & 1 & YES & YES & YES & $1.89$ & $(2,3)$ & NO & 9818\\
$(277,81)$ & 12 & $(9,2)$ & 5 & 1 & YES & YES & YES & $1.71$ & $(2,3)$ & NO & 9819\\
$(277,81)$ & 12 & $(10,3)$ & 5 & 1 & YES & YES & YES & $1.50$ & $(2,3)$ & NO & 9820\\
$(277,84)$ & 13 & $(10,3)$ & 5 & 1 & YES & YES & NO(2) & $1.62$ & $(4,2)$ & NO & 9821\\
$(277,78)$ & 13 & $(11,3)$ & 5 & 1 & YES & YES & YES & $1.43$ & $(4,2)$ & NO & 9822\\
$(277,60)$ & 13 & $(13,3)$ & 6 & 1 & YES & YES & YES & $1.71$ & $(2,3)$ & NO & 9823\\
$(277,106)$ & 12 & $(18,7)$ & 6 & 1 & YES & YES & YES & $1.75$ & $(4,2)$ & NO & 9824\\
$(277,116)$ & 12 & $(19,8)$ & 6 & 1 & YES & YES & YES & $1.67$ & $(2,3)$ & NO & 9825\\
$(277,117)$ & 12 & $(19,8)$ & 6 & 1 & YES & YES & NO(2) & $1.60$ & $(6,1)$ & NO & 9826\\
$(277,78)$ & 13 & $(25,7)$ & 7 & 1 & YES & YES & YES & $1.43$ & $(4,2)$ & NO & 9827\\
$(277,117)$ & 12 & $(26,11)$ & 7 & 1 & YES & YES & YES & $1.57$ & $(4,2)$ & 6963 & 9828\\
$(277,81)$ & 12 & $(31,9)$ & 8 & 1 & YES & YES & YES & $1.75$ & $(2,3)$ & NO & 9829\\
$(277,60)$ & 13 & $(32,7)$ & 8 & 1 & YES & YES & YES & $1.75$ & $(2,3)$ & NO & 9830\\
$(277,106)$ & 12 & $(47,18)$ & 8 & 1 & YES & YES & YES & $1.62$ & $(4,2)$ & NO & 9831\\
$(277,60)$ & 13 & $(51,11)$ & 9 & 1 & YES & YES & YES & $1.43$ & $(4,2)$ & NO & 9832\\
$(277,81)$ & 12 & $(58,17)$ & 9 & 1 & YES & YES & YES & $1.57$ & $(2,3)$ & NO & 9833\\
$(277,81)$ & 12 & $(65,19)$ & 9 & 1 & YES & YES & YES & $1.71$ & $(2,3)$ & 8859 & 9834\\
$(277,78)$ & 13 & $(71,20)$ & 10 & 1 & YES & YES & YES & $1.57$ & $(2,3)$ & 9080 & 9835\\
$(277,106)$ & 12 & $(81,31)$ & 9 & 1 & YES & YES & YES & $1.56$ & $(2,3)$ & NO & 9836\\
$(277,81)$ & 12 & $(106,31)$ & 10 & 1 & YES & YES & YES & $1.50$ & $(2,3)$ & NO & 9837\\
$(277,106)$ & 12 & $(115,44)$ & 10 & 1 & YES & YES & YES & $1.50$ & $(2,3)$ & 10486 & 9838\\
$(277,117)$ & 12 & $(116,49)$ & 10 & 1 & YES & YES & YES & $1.57$ & $(4,2)$ & NO & 9839\\
$(277,81)$ & 12 & $(147,43)$ & 11 & 1 & YES & YES & YES & $1.43$ & $(2,3)$ & NO & 9840\\
$(277,106)$ & 12 & $(196,75)$ & 11 & 1 & YES & YES & YES & $1.43$ & $(4,2)$ & NO & 9841\\
$(277,106)$ & 12 & $(277,106)$ & 12 & 277 & YES & YES & YES & $1.62$ & $(2,3)$ & NO & 9842\\
$(277,117)$ & 12 & $(277,117)$ & 12 & 277 & YES & YES & YES & $1.50$ & $(2,3)$ & NO & 9843\\
$(278,121)$ & 13 & $(2,1)$ & 1 & 2 & YES & YES & NO(2) & $1.70$ & $(6,1)$ & NO & 9844\\
$(278,63)$ & 13 & $(3,1)$ & 2 & 1 & YES & YES & YES & $1.29$ & $(4,2)$ & -- & 9845\\
$(278,65)$ & 13 & $(3,1)$ & 2 & 1 & YES & YES & YES & $1.50$ & $(2,3)$ & -- & 9846\\
$(278,65)$ & 13 & $(3,1)$ & 2 & 1 & YES & YES & YES & $1.75$ & $(2,3)$ & NO & 9847\\
$(278,77)$ & 13 & $(3,1)$ & 2 & 1 & YES & YES & YES & $1.29$ & $(4,2)$ & -- & 9848\\
$(278,121)$ & 13 & $(3,1)$ & 2 & 1 & YES & YES & YES & $1.78$ & $(4,2)$ & -- & 9849\\
$(278,77)$ & 13 & $(4,1)$ & 3 & 2 & YES & YES & YES & $1.62$ & $(2,3)$ & -- & 9850\\
$(278,65)$ & 13 & $(5,2)$ & 3 & 1 & YES & YES & YES & $1.29$ & $(4,2)$ & -- & 9851\\
$(278,121)$ & 13 & $(5,2)$ & 3 & 1 & YES & YES & YES & $1.78$ & $(4,2)$ & NO & 9852\\
$(278,63)$ & 13 & $(7,2)$ & 4 & 1 & YES & YES & YES & $1.57$ & $(2,3)$ & NO & 9853\\
$(278,65)$ & 13 & $(9,2)$ & 5 & 1 & YES & YES & YES & $1.62$ & $(4,2)$ & NO & 9854\\
$(278,63)$ & 13 & $(13,3)$ & 6 & 1 & YES & YES & NO(2) & $1.56$ & $(2,3)$ & NO & 9855\\
$(278,63)$ & 13 & $(17,4)$ & 7 & 1 & YES & YES & YES & $1.71$ & $(2,3)$ & NO & 9856\\
$(278,65)$ & 13 & $(22,5)$ & 7 & 2 & YES & YES & YES & $1.71$ & $(2,3)$ & NO & 9857\\
$(278,65)$ & 13 & $(30,7)$ & 8 & 2 & YES & YES & YES & $1.43$ & $(2,3)$ & NO & 9858\\
$(278,63)$ & 13 & $(35,8)$ & 8 & 1 & YES & YES & YES & $1.75$ & $(2,3)$ & NO & 9859\\
$(278,63)$ & 13 & $(40,9)$ & 9 & 2 & YES & YES & YES & $1.50$ & $(4,2)$ & NO & 9860\\
$(278,65)$ & 13 & $(43,10)$ & 9 & 1 & YES & YES & YES & $1.57$ & $(2,3)$ & NO & 9861\\
$(278,65)$ & 13 & $(171,40)$ & 12 & 1 & YES & YES & YES & $1.75$ & $(2,3)$ & 11701 & 9862\\
$(278,77)$ & 13 & $(278,77)$ & 13 & 278 & YES & YES & YES & $1.89$ & $(2,3)$ & NO & 9863\\
$(279,74)$ & 14 & $(2,1)$ & 1 & 1 & YES & YES & NO(2) & $1.56$ & $(8,0)$ & NO & 9864\\
$(279,65)$ & 13 & $(5,2)$ & 3 & 1 & YES & YES & YES & $1.67$ & $(2,3)$ & -- & 9865\\
$(279,121)$ & 13 & $(5,1)$ & 4 & 1 & YES & YES & YES & $1.88$ & $(4,2)$ & NO & 9866\\
$(279,83)$ & 13 & $(10,3)$ & 5 & 1 & YES & YES & YES & $1.57$ & $(2,3)$ & NO & 9867\\
$(279,83)$ & 13 & $(37,11)$ & 8 & 1 & YES & YES & YES & $1.43$ & $(2,3)$ & 7921 & 9868\\
$(280,87)$ & 13 & $(3,1)$ & 2 & 1 & YES & YES & NO(2) & $1.44$ & $(4,2)$ & -- & 9869\\
$(280,107)$ & 12 & $(5,1)$ & 4 & 5 & YES & YES & YES & $1.67$ & $(2,3)$ & -- & 9870\\
$(280,107)$ & 12 & $(5,2)$ & 3 & 5 & YES & YES & YES & $1.67$ & $(2,3)$ & NO & 9871\\
$(280,67)$ & 14 & $(13,3)$ & 6 & 1 & YES & YES & NO(2) & $1.56$ & $(4,2)$ & NO & 9872\\
$(280,107)$ & 12 & $(123,47)$ & 10 & 1 & YES & YES & YES & $1.89$ & $(2,3)$ & NO & 9873\\
$(280,107)$ & 12 & $(280,107)$ & 12 & 280 & YES & YES & YES & $1.86$ & $(2,3)$ & NO & 9874\\
$(281,109)$ & 12 & $(2,1)$ & 1 & 1 & YES & YES & YES & $1.57$ & $(4,2)$ & -- & 9875\\
$(281,109)$ & 12 & $(2,1)$ & 1 & 1 & YES & YES & YES & $2.00$ & $(2,3)$ & NO & 9876\\
$(281,109)$ & 12 & $(3,1)$ & 2 & 1 & YES & YES & YES & $1.57$ & $(4,2)$ & NO & 9877\\
$(281,116)$ & 12 & $(3,1)$ & 2 & 1 & YES & YES & YES & $1.89$ & $(2,3)$ & -- & 9878\\
$(281,116)$ & 12 & $(3,1)$ & 2 & 1 & YES & YES & YES & $1.89$ & $(2,3)$ & NO & 9879\\
$(281,76)$ & 13 & $(4,1)$ & 3 & 1 & YES & YES & NO(2) & $1.38$ & $(4,2)$ & -- & 9880\\
$(281,116)$ & 12 & $(4,1)$ & 3 & 1 & YES & YES & YES & $1.75$ & $(2,3)$ & -- & 9881\\
$(281,85)$ & 13 & $(5,1)$ & 4 & 1 & YES & YES & NO(2) & $1.38$ & $(4,2)$ & -- & 9882\\
$(281,116)$ & 12 & $(5,1)$ & 4 & 1 & YES & YES & YES & $1.57$ & $(2,3)$ & NO & 9883\\
$(281,116)$ & 12 & $(12,5)$ & 5 & 1 & YES & YES & YES & $1.89$ & $(2,3)$ & 7264 & 9884\\
$(281,109)$ & 12 & $(13,5)$ & 5 & 1 & YES & YES & YES & $1.43$ & $(4,2)$ & NO & 9885\\
$(281,109)$ & 12 & $(18,7)$ & 6 & 1 & YES & YES & YES & $1.78$ & $(2,3)$ & 7059 & 9886\\
$(281,85)$ & 13 & $(43,13)$ & 9 & 1 & YES & YES & NO(2) & $1.38$ & $(4,2)$ & 8113 & 9887\\
$(281,109)$ & 12 & $(49,19)$ & 8 & 1 & YES & YES & YES & $1.57$ & $(4,2)$ & 8370 & 9888\\
$(281,116)$ & 12 & $(63,26)$ & 9 & 1 & YES & YES & YES & $1.67$ & $(2,3)$ & 8824 & 9889\\
$(281,109)$ & 12 & $(67,26)$ & 9 & 1 & YES & YES & YES & $1.78$ & $(4,2)$ & NO & 9890\\
$(281,109)$ & 12 & $(116,45)$ & 10 & 1 & YES & YES & YES & $1.43$ & $(4,2)$ & NO & 9891\\
$(281,116)$ & 12 & $(172,71)$ & 11 & 1 & YES & YES & YES & $1.43$ & $(2,3)$ & NO & 9892\\
$(281,116)$ & 12 & $(281,116)$ & 12 & 281 & YES & YES & YES & $1.62$ & $(2,3)$ & NO & 9893\\
$(282,109)$ & 12 & $(2,1)$ & 1 & 2 & YES & YES & YES & $1.43$ & $(4,2)$ & -- & 9894\\
$(282,119)$ & 12 & $(2,1)$ & 1 & 2 & YES & YES & YES & $1.75$ & $(4,2)$ & -- & 9895\\
$(282,109)$ & 12 & $(3,1)$ & 2 & 3 & YES & YES & YES & $1.75$ & $(2,3)$ & -- & 9896\\
$(282,109)$ & 12 & $(3,1)$ & 2 & 3 & YES & YES & YES & $1.75$ & $(2,3)$ & NO & 9897\\
$(282,119)$ & 12 & $(3,1)$ & 2 & 3 & YES & YES & YES & $1.57$ & $(4,2)$ & -- & 9898\\
$(282,109)$ & 12 & $(4,1)$ & 3 & 2 & YES & YES & NO(2) & $1.56$ & $(2,3)$ & -- & 9899\\
$(282,109)$ & 12 & $(4,1)$ & 3 & 2 & YES & YES & YES & $1.78$ & $(2,3)$ & NO & 9900\\
$(282,109)$ & 12 & $(4,1)$ & 3 & 2 & YES & YES & YES & $1.78$ & $(2,3)$ & NO & 9901\\
$(282,109)$ & 12 & $(5,2)$ & 3 & 1 & YES & YES & NO(2) & $1.60$ & $(6,1)$ & NO & 9902\\
$(282,119)$ & 12 & $(5,1)$ & 4 & 1 & YES & YES & YES & $1.70$ & $(2,3)$ & NO & 9903\\
$(282,119)$ & 12 & $(5,2)$ & 3 & 1 & YES & YES & YES & $1.57$ & $(4,2)$ & NO & 9904\\
$(282,109)$ & 12 & $(8,3)$ & 4 & 2 & YES & YES & YES & $1.67$ & $(2,3)$ & NO & 9905\\
$(282,119)$ & 12 & $(12,5)$ & 5 & 6 & YES & YES & YES & $1.62$ & $(4,2)$ & NO & 9906\\
$(282,109)$ & 12 & $(18,7)$ & 6 & 6 & YES & YES & YES & $1.75$ & $(2,3)$ & NO & 9907\\
$(282,119)$ & 12 & $(26,11)$ & 7 & 2 & YES & YES & YES & $1.57$ & $(4,2)$ & NO & 9908\\
$(282,109)$ & 12 & $(31,12)$ & 7 & 1 & YES & YES & YES & $1.89$ & $(2,3)$ & 7021 & 9909\\
$(282,119)$ & 12 & $(45,19)$ & 8 & 3 & YES & YES & YES & $1.62$ & $(2,3)$ & NO & 9910\\
$(282,119)$ & 12 & $(64,27)$ & 9 & 2 & YES & YES & YES & $1.70$ & $(2,3)$ & 8873 & 9911\\
$(282,109)$ & 12 & $(75,29)$ & 9 & 3 & YES & YES & YES & $1.62$ & $(2,3)$ & NO & 9912\\
$(282,119)$ & 12 & $(83,35)$ & 10 & 1 & YES & YES & YES & $1.78$ & $(4,2)$ & NO & 9913\\
$(282,109)$ & 12 & $(119,46)$ & 10 & 1 & YES & YES & YES & $1.43$ & $(4,2)$ & NO & 9914\\
$(282,109)$ & 12 & $(163,63)$ & 11 & 1 & YES & YES & YES & $1.62$ & $(4,2)$ & NO & 9915\\
$(282,119)$ & 12 & $(173,73)$ & 11 & 1 & YES & YES & YES & $1.43$ & $(4,2)$ & NO & 9916\\
$(282,109)$ & 12 & $(282,109)$ & 12 & 282 & YES & YES & YES & $1.75$ & $(4,2)$ & NO & 9917\\
$(282,119)$ & 12 & $(282,119)$ & 12 & 282 & YES & YES & YES & $1.50$ & $(4,2)$ & NO & 9918\\
$(283,64)$ & 13 & $(2,1)$ & 1 & 1 & YES & YES & NO(2) & $1.25$ & $(4,2)$ & NO & 9919\\
$(283,64)$ & 13 & $(2,1)$ & 1 & 1 & YES & YES & NO(2) & $1.38$ & $(4,2)$ & -- & 9920\\
$(283,86)$ & 13 & $(2,1)$ & 1 & 1 & YES & YES & NO(2) & $1.50$ & $(4,2)$ & NO & 9921\\
$(283,104)$ & 12 & $(2,1)$ & 1 & 1 & YES & YES & NO(2) & $1.60$ & $(6,1)$ & -- & 9922\\
$(283,108)$ & 12 & $(2,1)$ & 1 & 1 & YES & YES & YES & $1.56$ & $(2,3)$ & -- & 9923\\
$(283,117)$ & 12 & $(2,1)$ & 1 & 1 & YES & YES & YES & $1.78$ & $(2,3)$ & -- & 9924\\
$(283,119)$ & 13 & $(2,1)$ & 1 & 1 & NO & YES & NO(2) & $1.60$ & $(6,1)$ & -- & 9925\\
$(283,64)$ & 13 & $(3,1)$ & 2 & 1 & YES & YES & NO(2) & $1.56$ & $(2,3)$ & NO & 9926\\
$(283,64)$ & 13 & $(3,1)$ & 2 & 1 & YES & YES & NO(2) & $1.56$ & $(2,3)$ & -- & 9927\\
$(283,75)$ & 13 & $(3,1)$ & 2 & 1 & YES & YES & YES & $1.62$ & $(4,2)$ & -- & 9928\\
$(283,75)$ & 13 & $(3,1)$ & 2 & 1 & YES & YES & YES & $1.75$ & $(4,2)$ & NO & 9929\\
$(283,76)$ & 12 & $(3,1)$ & 2 & 1 & YES & YES & YES & $1.29$ & $(6,1)$ & -- & 9930\\
$(283,76)$ & 12 & $(3,1)$ & 2 & 1 & YES & YES & YES & $1.43$ & $(6,1)$ & NO & 9931\\
$(283,76)$ & 12 & $(3,1)$ & 2 & 1 & YES & YES & YES & $1.43$ & $(6,1)$ & NO & 9932\\
$(283,83)$ & 13 & $(3,1)$ & 2 & 1 & YES & YES & YES & $1.50$ & $(4,2)$ & -- & 9933\\
$(283,86)$ & 13 & $(3,1)$ & 2 & 1 & YES & YES & YES & $1.71$ & $(4,2)$ & NO & 9934\\
$(283,86)$ & 13 & $(3,1)$ & 2 & 1 & YES & YES & YES & $1.71$ & $(4,2)$ & -- & 9935\\
$(283,104)$ & 12 & $(3,1)$ & 2 & 1 & YES & YES & YES & $1.29$ & $(4,2)$ & -- & 9936\\
$(283,108)$ & 12 & $(3,1)$ & 2 & 1 & YES & YES & YES & $1.80$ & $(2,3)$ & -- & 9937\\
$(283,108)$ & 12 & $(3,1)$ & 2 & 1 & YES & YES & YES & $1.67$ & $(2,3)$ & NO & 9938\\
$(283,117)$ & 12 & $(3,1)$ & 2 & 1 & YES & YES & YES & $1.43$ & $(2,3)$ & NO & 9939\\
$(283,117)$ & 12 & $(3,1)$ & 2 & 1 & YES & YES & YES & $1.43$ & $(2,3)$ & -- & 9940\\
$(283,75)$ & 13 & $(4,1)$ & 3 & 1 & YES & YES & YES & $1.75$ & $(4,2)$ & -- & 9941\\
$(283,108)$ & 12 & $(4,1)$ & 3 & 1 & YES & YES & YES & $1.70$ & $(2,3)$ & -- & 9942\\
$(283,117)$ & 12 & $(4,1)$ & 3 & 1 & YES & YES & YES & $1.71$ & $(2,3)$ & -- & 9943\\
$(283,117)$ & 12 & $(4,1)$ & 3 & 1 & YES & YES & YES & $1.43$ & $(2,3)$ & NO & 9944\\
$(283,83)$ & 13 & $(5,1)$ & 4 & 1 & YES & YES & YES & $1.38$ & $(4,2)$ & NO & 9945\\
$(283,108)$ & 12 & $(6,1)$ & 5 & 1 & YES & YES & YES & $1.43$ & $(4,2)$ & NO & 9946\\
$(283,75)$ & 13 & $(7,2)$ & 4 & 1 & YES & YES & YES & $1.62$ & $(4,2)$ & NO & 9947\\
$(283,76)$ & 12 & $(7,2)$ & 4 & 1 & YES & YES & YES & $1.29$ & $(6,1)$ & 10628 & 9948\\
$(283,117)$ & 12 & $(7,3)$ & 4 & 1 & YES & YES & YES & $1.67$ & $(4,2)$ & NO & 9949\\
$(283,104)$ & 12 & $(8,3)$ & 4 & 1 & YES & YES & NO(2) & $1.60$ & $(6,1)$ & NO & 9950\\
$(283,75)$ & 13 & $(11,3)$ & 5 & 1 & YES & YES & YES & $1.62$ & $(4,2)$ & NO & 9951\\
$(283,76)$ & 12 & $(11,3)$ & 5 & 1 & YES & YES & YES & $1.43$ & $(6,1)$ & 9520 & 9952\\
$(283,117)$ & 12 & $(12,5)$ & 5 & 1 & YES & YES & YES & $1.71$ & $(2,3)$ & NO & 9953\\
$(283,108)$ & 12 & $(13,5)$ & 5 & 1 & YES & YES & YES & $1.56$ & $(2,3)$ & NO & 9954\\
$(283,64)$ & 13 & $(22,5)$ & 7 & 1 & YES & YES & NO(2) & $1.38$ & $(4,2)$ & NO & 9955\\
$(283,83)$ & 13 & $(24,7)$ & 7 & 1 & YES & YES & YES & $1.57$ & $(4,2)$ & NO & 9956\\
$(283,104)$ & 12 & $(30,11)$ & 7 & 1 & YES & YES & YES & $1.57$ & $(4,2)$ & NO & 9957\\
$(283,108)$ & 12 & $(34,13)$ & 7 & 1 & YES & YES & YES & $1.57$ & $(2,3)$ & NO & 9958\\
$(283,117)$ & 12 & $(46,19)$ & 8 & 1 & YES & YES & YES & $1.67$ & $(4,2)$ & 10580 & 9959\\
$(283,75)$ & 13 & $(49,13)$ & 9 & 1 & YES & YES & YES & $1.71$ & $(2,3)$ & NO & 9960\\
$(283,64)$ & 13 & $(53,12)$ & 9 & 1 & YES & YES & NO(2) & $1.56$ & $(2,3)$ & NO & 9961\\
$(283,108)$ & 12 & $(55,21)$ & 8 & 1 & YES & YES & YES & $1.67$ & $(4,2)$ & NO & 9962\\
$(283,117)$ & 12 & $(75,31)$ & 9 & 1 & YES & YES & YES & $1.29$ & $(4,2)$ & 9301 & 9963\\
$(283,84)$ & 13 & $(91,27)$ & 10 & 1 & YES & YES & YES & $1.71$ & $(2,3)$ & NO & 9964\\
$(283,117)$ & 12 & $(104,43)$ & 10 & 1 & YES & YES & YES & $1.43$ & $(2,3)$ & NO & 9965\\
$(283,108)$ & 12 & $(131,50)$ & 10 & 1 & YES & YES & YES & $1.78$ & $(2,3)$ & 10868 & 9966\\
$(283,83)$ & 13 & $(133,39)$ & 11 & 1 & YES & YES & YES & $1.78$ & $(2,3)$ & 10904 & 9967\\
$(283,117)$ & 12 & $(179,74)$ & 11 & 1 & YES & YES & YES & $1.67$ & $(4,2)$ & NO & 9968\\
$(283,108)$ & 12 & $(207,79)$ & 11 & 1 & YES & YES & YES & $1.86$ & $(2,3)$ & NO & 9969\\
$(283,84)$ & 13 & $(219,65)$ & 12 & 1 & YES & YES & YES & $1.86$ & $(2,3)$ & NO & 9970\\
$(283,83)$ & 13 & $(283,83)$ & 13 & 283 & YES & YES & YES & $1.78$ & $(2,3)$ & NO & 9971\\
$(283,117)$ & 12 & $(283,117)$ & 12 & 283 & YES & YES & YES & $1.43$ & $(2,3)$ & NO & 9972\\
$(284,65)$ & 13 & $(2,1)$ & 1 & 2 & YES & YES & YES & $1.38$ & $(2,3)$ & NO & 9973\\
$(284,65)$ & 13 & $(2,1)$ & 1 & 2 & YES & YES & YES & $1.38$ & $(2,3)$ & -- & 9974\\
$(284,105)$ & 12 & $(2,1)$ & 1 & 2 & YES & YES & YES & $1.57$ & $(6,1)$ & NO & 9975\\
$(284,51)$ & 16 & $(3,1)$ & 2 & 1 & YES & YES & NO(2) & $1.50$ & $(6,1)$ & NO & 9976\\
$(284,105)$ & 12 & $(3,1)$ & 2 & 1 & YES & YES & NO(2) & $1.50$ & $(6,1)$ & -- & 9977\\
$(284,105)$ & 12 & $(5,2)$ & 3 & 1 & YES & YES & YES & $1.62$ & $(4,2)$ & -- & 9978\\
$(284,105)$ & 12 & $(5,2)$ & 3 & 1 & YES & YES & YES & $1.78$ & $(2,3)$ & NO & 9979\\
$(284,65)$ & 13 & $(7,2)$ & 4 & 1 & YES & YES & YES & $1.80$ & $(2,3)$ & NO & 9980\\
$(284,105)$ & 12 & $(11,4)$ & 5 & 1 & YES & YES & YES & $1.50$ & $(4,2)$ & NO & 9981\\
$(284,119)$ & 12 & $(19,8)$ & 6 & 1 & YES & YES & YES & $1.67$ & $(4,2)$ & NO & 9982\\
$(284,65)$ & 13 & $(31,7)$ & 8 & 1 & YES & YES & YES & $1.78$ & $(2,3)$ & NO & 9983\\
$(284,65)$ & 13 & $(57,13)$ & 9 & 1 & YES & YES & YES & $1.50$ & $(4,2)$ & NO & 9984\\
$(284,119)$ & 12 & $(74,31)$ & 9 & 2 & YES & YES & YES & $1.62$ & $(4,2)$ & 9270 & 9985\\
$(284,105)$ & 12 & $(119,44)$ & 10 & 1 & YES & YES & YES & $1.50$ & $(2,3)$ & NO & 9986\\
$(284,105)$ & 12 & $(284,105)$ & 12 & 284 & YES & YES & YES & $1.50$ & $(2,3)$ & NO & 9987\\
$(285,68)$ & 15 & $(2,1)$ & 1 & 1 & YES & YES & NO(2) & $1.44$ & $(8,0)$ & -- & 9988\\
$(285,68)$ & 15 & $(2,1)$ & 1 & 1 & YES & YES & NO(2) & $1.56$ & $(8,0)$ & NO & 9989\\
$(285,53)$ & 15 & $(5,1)$ & 4 & 5 & YES & YES & NO(2) & $1.62$ & $(6,1)$ & NO & 9990\\
$(285,83)$ & 13 & $(7,2)$ & 4 & 1 & YES & YES & YES & $1.43$ & $(2,3)$ & NO & 9991\\
$(285,83)$ & 13 & $(17,5)$ & 6 & 1 & YES & YES & YES & $1.75$ & $(2,3)$ & NO & 9992\\
$(285,83)$ & 13 & $(24,7)$ & 7 & 3 & YES & YES & YES & $1.57$ & $(2,3)$ & NO & 9993\\
$(285,83)$ & 13 & $(31,9)$ & 8 & 1 & YES & YES & YES & $1.75$ & $(2,3)$ & NO & 9994\\
$(286,105)$ & 12 & $(2,1)$ & 1 & 2 & YES & YES & YES & $1.89$ & $(2,3)$ & -- & 9995\\
$(286,67)$ & 13 & $(3,1)$ & 2 & 1 & YES & YES & YES & $1.43$ & $(2,3)$ & NO & 9996\\
$(286,79)$ & 12 & $(3,1)$ & 2 & 1 & YES & YES & YES & $1.38$ & $(2,3)$ & -- & 9997\\
$(286,79)$ & 12 & $(3,1)$ & 2 & 1 & YES & YES & YES & $1.62$ & $(2,3)$ & NO & 9998\\
$(286,105)$ & 12 & $(3,1)$ & 2 & 1 & YES & YES & YES & $1.78$ & $(2,3)$ & -- & 9999\\
$(286,105)$ & 12 & $(3,1)$ & 2 & 1 & YES & YES & YES & $1.86$ & $(2,3)$ & NO & 10000\\
$(286,79)$ & 12 & $(5,2)$ & 3 & 1 & YES & YES & YES & $1.75$ & $(2,3)$ & -- & 10001\\
$(286,111)$ & 13 & $(5,1)$ & 4 & 1 & YES & YES & YES & $1.75$ & $(4,2)$ & NO & 10002\\
$(286,105)$ & 12 & $(11,4)$ & 5 & 11 & YES & YES & YES & $1.57$ & $(4,2)$ & NO & 10003\\
$(286,105)$ & 12 & $(30,11)$ & 7 & 2 & YES & YES & YES & $1.78$ & $(4,2)$ & NO & 10004\\
$(286,79)$ & 12 & $(47,13)$ & 8 & 1 & YES & YES & YES & $1.38$ & $(2,3)$ & 10632 & 10005\\
$(286,105)$ & 12 & $(128,47)$ & 10 & 2 & YES & YES & YES & $1.71$ & $(2,3)$ & 10810 & 10006\\
$(286,79)$ & 12 & $(134,37)$ & 11 & 2 & YES & YES & YES & $1.75$ & $(2,3)$ & NO & 10007\\
$(287,79)$ & 12 & $(2,1)$ & 1 & 1 & YES & YES & YES & $1.67$ & $(2,3)$ & -- & 10008\\
$(287,79)$ & 12 & $(2,1)$ & 1 & 1 & YES & YES & YES & $1.38$ & $(2,3)$ & NO & 10009\\
$(287,106)$ & 12 & $(2,1)$ & 1 & 1 & YES & YES & YES & $1.50$ & $(4,2)$ & -- & 10010\\
$(287,111)$ & 12 & $(2,1)$ & 1 & 1 & NO & YES & NO(2) & $1.38$ & $(6,1)$ & -- & 10011\\
$(287,125)$ & 13 & $(2,1)$ & 1 & 1 & NO & YES & NO(2) & $1.56$ & $(8,0)$ & -- & 10012\\
$(287,65)$ & 14 & $(3,1)$ & 2 & 1 & YES & YES & YES & $1.50$ & $(4,2)$ & -- & 10013\\
$(287,79)$ & 12 & $(3,1)$ & 2 & 1 & YES & YES & YES & $1.38$ & $(2,3)$ & NO & 10014\\
$(287,106)$ & 12 & $(3,1)$ & 2 & 1 & YES & YES & YES & $1.75$ & $(2,3)$ & -- & 10015\\
$(287,106)$ & 12 & $(3,1)$ & 2 & 1 & YES & YES & YES & $1.62$ & $(4,2)$ & NO & 10016\\
$(287,109)$ & 12 & $(3,1)$ & 2 & 1 & YES & YES & YES & $1.43$ & $(4,2)$ & -- & 10017\\
$(287,109)$ & 12 & $(3,1)$ & 2 & 1 & YES & YES & YES & $1.57$ & $(4,2)$ & NO & 10018\\
$(287,111)$ & 12 & $(3,1)$ & 2 & 1 & YES & YES & YES & $1.43$ & $(4,2)$ & -- & 10019\\
$(287,111)$ & 12 & $(3,1)$ & 2 & 1 & YES & YES & YES & $1.62$ & $(4,2)$ & NO & 10020\\
$(287,106)$ & 12 & $(4,1)$ & 3 & 1 & YES & YES & NO(2) & $1.50$ & $(6,1)$ & -- & 10021\\
$(287,109)$ & 12 & $(4,1)$ & 3 & 1 & YES & YES & YES & $1.50$ & $(4,2)$ & -- & 10022\\
$(287,80)$ & 13 & $(5,1)$ & 4 & 1 & YES & YES & YES & $1.50$ & $(4,2)$ & NO & 10023\\
$(287,106)$ & 12 & $(5,1)$ & 4 & 1 & YES & YES & YES & $1.75$ & $(2,3)$ & NO & 10024\\
$(287,106)$ & 12 & $(5,2)$ & 3 & 1 & YES & YES & YES & $1.62$ & $(2,3)$ & NO & 10025\\
$(287,109)$ & 12 & $(5,2)$ & 3 & 1 & YES & YES & YES & $1.67$ & $(4,2)$ & NO & 10026\\
$(287,111)$ & 12 & $(5,1)$ & 4 & 1 & YES & YES & YES & $1.43$ & $(4,2)$ & NO & 10027\\
$(287,111)$ & 12 & $(5,1)$ & 4 & 1 & YES & YES & YES & $1.43$ & $(4,2)$ & -- & 10028\\
$(287,111)$ & 12 & $(5,2)$ & 3 & 1 & YES & YES & NO(2) & $1.50$ & $(6,1)$ & NO & 10029\\
$(287,65)$ & 14 & $(6,1)$ & 5 & 1 & YES & YES & YES & $1.50$ & $(4,2)$ & NO & 10030\\
$(287,53)$ & 14 & $(7,2)$ & 4 & 7 & YES & YES & YES & $1.62$ & $(2,3)$ & NO & 10031\\
$(287,79)$ & 12 & $(7,2)$ & 4 & 7 & YES & YES & YES & $1.89$ & $(2,3)$ & NO & 10032\\
$(287,106)$ & 12 & $(8,3)$ & 4 & 1 & YES & YES & NO(2) & $1.60$ & $(6,1)$ & 7362 & 10033\\
$(287,111)$ & 12 & $(8,3)$ & 4 & 1 & YES & YES & YES & $1.78$ & $(2,3)$ & NO & 10034\\
$(287,106)$ & 12 & $(11,4)$ & 5 & 1 & YES & YES & YES & $1.62$ & $(4,2)$ & NO & 10035\\
$(287,109)$ & 12 & $(13,5)$ & 5 & 1 & YES & YES & YES & $1.57$ & $(2,3)$ & NO & 10036\\
$(287,111)$ & 12 & $(18,7)$ & 6 & 1 & YES & YES & YES & $1.78$ & $(2,3)$ & NO & 10037\\
$(287,109)$ & 12 & $(21,8)$ & 6 & 7 & YES & YES & YES & $1.56$ & $(2,3)$ & NO & 10038\\
$(287,111)$ & 12 & $(31,12)$ & 7 & 1 & YES & YES & YES & $1.90$ & $(2,3)$ & NO & 10039\\
$(287,80)$ & 13 & $(43,12)$ & 8 & 1 & YES & YES & YES & $1.67$ & $(2,3)$ & NO & 10040\\
$(287,111)$ & 12 & $(44,17)$ & 8 & 1 & YES & YES & YES & $1.88$ & $(2,3)$ & 10520 & 10041\\
$(287,106)$ & 12 & $(46,17)$ & 8 & 1 & YES & YES & YES & $1.75$ & $(2,3)$ & NO & 10042\\
$(287,109)$ & 12 & $(50,19)$ & 8 & 1 & YES & YES & YES & $1.90$ & $(2,3)$ & NO & 10043\\
$(287,106)$ & 12 & $(65,24)$ & 9 & 1 & YES & YES & YES & $1.50$ & $(4,2)$ & 8982 & 10044\\
$(287,111)$ & 12 & $(75,29)$ & 9 & 1 & YES & YES & YES & $1.43$ & $(4,2)$ & 9365 & 10045\\
$(287,109)$ & 12 & $(79,30)$ & 9 & 1 & YES & YES & YES & $1.78$ & $(2,3)$ & NO & 10046\\
$(287,80)$ & 13 & $(104,29)$ & 10 & 1 & YES & YES & YES & $1.43$ & $(4,2)$ & NO & 10047\\
$(287,111)$ & 12 & $(106,41)$ & 10 & 1 & YES & YES & YES & $1.57$ & $(4,2)$ & NO & 10048\\
$(287,106)$ & 12 & $(111,41)$ & 10 & 1 & YES & YES & YES & $1.62$ & $(2,3)$ & NO & 10049\\
$(287,61)$ & 13 & $(113,24)$ & 11 & 1 & YES & YES & NO(2) & $1.67$ & $(2,3)$ & NO & 10050\\
$(287,61)$ & 13 & $(146,31)$ & 12 & 1 & YES & YES & YES & $1.62$ & $(4,2)$ & NO & 10051\\
$(287,106)$ & 12 & $(176,65)$ & 11 & 1 & YES & YES & YES & $1.88$ & $(2,3)$ & NO & 10052\\
$(287,111)$ & 12 & $(181,70)$ & 11 & 1 & YES & YES & YES & $1.43$ & $(4,2)$ & NO & 10053\\
$(287,109)$ & 12 & $(208,79)$ & 11 & 1 & YES & YES & YES & $1.43$ & $(4,2)$ & NO & 10054\\
$(287,80)$ & 13 & $(226,63)$ & 12 & 1 & YES & YES & YES & $1.62$ & $(4,2)$ & NO & 10055\\
$(287,106)$ & 12 & $(287,106)$ & 12 & 287 & YES & YES & YES & $1.43$ & $(4,2)$ & NO & 10056\\
$(287,109)$ & 12 & $(287,109)$ & 12 & 287 & YES & YES & YES & $1.62$ & $(4,2)$ & NO & 10057\\
$(288,119)$ & 12 & $(2,1)$ & 1 & 2 & YES & YES & YES & $1.62$ & $(4,2)$ & -- & 10058\\
$(288,121)$ & 12 & $(2,1)$ & 1 & 2 & YES & YES & YES & $1.67$ & $(2,3)$ & -- & 10059\\
$(288,121)$ & 12 & $(2,1)$ & 1 & 2 & YES & YES & YES & $1.75$ & $(4,2)$ & NO & 10060\\
$(288,61)$ & 13 & $(3,1)$ & 2 & 3 & YES & YES & NO(2) & $1.44$ & $(4,2)$ & NO & 10061\\
$(288,61)$ & 13 & $(3,1)$ & 2 & 3 & YES & YES & NO(2) & $1.44$ & $(4,2)$ & -- & 10062\\
$(288,119)$ & 12 & $(3,1)$ & 2 & 3 & YES & YES & YES & $1.57$ & $(4,2)$ & -- & 10063\\
$(288,119)$ & 12 & $(3,1)$ & 2 & 3 & YES & YES & YES & $1.57$ & $(4,2)$ & NO & 10064\\
$(288,119)$ & 12 & $(3,1)$ & 2 & 3 & YES & YES & YES & $1.71$ & $(2,3)$ & NO & 10065\\
$(288,121)$ & 12 & $(3,1)$ & 2 & 3 & YES & YES & YES & $1.78$ & $(2,3)$ & -- & 10066\\
$(288,121)$ & 12 & $(3,1)$ & 2 & 3 & YES & YES & YES & $2.00$ & $(2,3)$ & NO & 10067\\
$(288,85)$ & 13 & $(4,1)$ & 3 & 4 & YES & YES & YES & $1.78$ & $(2,3)$ & -- & 10068\\
$(288,119)$ & 12 & $(4,1)$ & 3 & 4 & YES & YES & YES & $1.57$ & $(4,2)$ & NO & 10069\\
$(288,119)$ & 12 & $(4,1)$ & 3 & 4 & YES & YES & YES & $1.62$ & $(2,3)$ & -- & 10070\\
$(288,121)$ & 12 & $(4,1)$ & 3 & 4 & YES & YES & YES & $1.62$ & $(2,3)$ & NO & 10071\\
$(288,121)$ & 12 & $(4,1)$ & 3 & 4 & YES & YES & YES & $2.00$ & $(2,3)$ & -- & 10072\\
$(288,121)$ & 12 & $(5,2)$ & 3 & 1 & YES & YES & YES & $1.57$ & $(4,2)$ & NO & 10073\\
$(288,121)$ & 12 & $(7,3)$ & 4 & 1 & YES & YES & YES & $1.89$ & $(4,2)$ & NO & 10074\\
$(288,119)$ & 12 & $(12,5)$ & 5 & 12 & YES & YES & YES & $1.67$ & $(4,2)$ & NO & 10075\\
$(288,121)$ & 12 & $(12,5)$ & 5 & 12 & YES & YES & YES & $1.75$ & $(4,2)$ & NO & 10076\\
$(288,119)$ & 12 & $(17,7)$ & 6 & 1 & YES & YES & YES & $1.57$ & $(2,3)$ & NO & 10077\\
$(288,121)$ & 12 & $(26,11)$ & 7 & 2 & YES & YES & YES & $1.89$ & $(4,2)$ & NO & 10078\\
$(288,119)$ & 12 & $(29,12)$ & 7 & 1 & YES & YES & YES & $1.71$ & $(2,3)$ & 7133 & 10079\\
$(288,85)$ & 13 & $(44,13)$ & 8 & 4 & YES & YES & YES & $1.57$ & $(4,2)$ & NO & 10080\\
$(288,61)$ & 13 & $(52,11)$ & 9 & 4 & YES & YES & NO(2) & $1.56$ & $(4,2)$ & NO & 10081\\
$(288,119)$ & 12 & $(75,31)$ & 9 & 3 & YES & YES & YES & $1.71$ & $(2,3)$ & NO & 10082\\
$(288,121)$ & 12 & $(119,50)$ & 10 & 1 & YES & YES & YES & $1.75$ & $(4,2)$ & NO & 10083\\
$(288,119)$ & 12 & $(121,50)$ & 10 & 1 & YES & YES & YES & $1.67$ & $(2,3)$ & NO & 10084\\
$(288,85)$ & 13 & $(166,49)$ & 11 & 2 & YES & YES & YES & $1.43$ & $(4,2)$ & 11371 & 10085\\
$(288,119)$ & 12 & $(167,69)$ & 11 & 1 & YES & YES & YES & $1.80$ & $(2,3)$ & NO & 10086\\
$(288,121)$ & 12 & $(169,71)$ & 11 & 1 & YES & YES & YES & $1.75$ & $(2,3)$ & NO & 10087\\
$(288,119)$ & 12 & $(288,119)$ & 12 & 288 & YES & YES & YES & $1.62$ & $(4,2)$ & NO & 10088\\
$(288,121)$ & 12 & $(288,121)$ & 12 & 288 & YES & YES & YES & $1.62$ & $(2,3)$ & NO & 10089\\
$(289,80)$ & 12 & $(2,1)$ & 1 & 1 & YES & YES & YES & $1.50$ & $(2,3)$ & NO & 10090\\
$(289,80)$ & 12 & $(2,1)$ & 1 & 1 & YES & YES & YES & $1.50$ & $(2,3)$ & -- & 10091\\
$(289,86)$ & 13 & $(2,1)$ & 1 & 1 & YES & YES & NO(2) & $1.60$ & $(6,1)$ & -- & 10092\\
$(289,112)$ & 12 & $(2,1)$ & 1 & 1 & YES & YES & YES & $1.57$ & $(4,2)$ & -- & 10093\\
$(289,44)$ & 16 & $(3,1)$ & 2 & 1 & YES & YES & YES & $1.43$ & $(4,2)$ & NO & 10094\\
$(289,44)$ & 16 & $(3,1)$ & 2 & 1 & YES & YES & YES & $1.43$ & $(4,2)$ & -- & 10095\\
$(289,78)$ & 13 & $(3,1)$ & 2 & 1 & YES & YES & YES & $1.78$ & $(2,3)$ & -- & 10096\\
$(289,80)$ & 12 & $(3,1)$ & 2 & 1 & YES & YES & YES & $1.50$ & $(2,3)$ & -- & 10097\\
$(289,80)$ & 12 & $(3,1)$ & 2 & 1 & YES & YES & YES & $1.62$ & $(2,3)$ & NO & 10098\\
$(289,44)$ & 16 & $(4,1)$ & 3 & 1 & YES & YES & NO(2) & $1.38$ & $(6,1)$ & NO & 10099\\
$(289,112)$ & 12 & $(4,1)$ & 3 & 1 & YES & YES & YES & $1.57$ & $(2,3)$ & NO & 10100\\
$(289,112)$ & 12 & $(4,1)$ & 3 & 1 & YES & YES & YES & $1.57$ & $(2,3)$ & -- & 10101\\
$(289,44)$ & 16 & $(5,1)$ & 4 & 1 & YES & YES & NO(2) & $1.56$ & $(4,2)$ & NO & 10102\\
$(289,80)$ & 12 & $(5,2)$ & 3 & 1 & YES & YES & YES & $1.62$ & $(2,3)$ & -- & 10103\\
$(289,112)$ & 12 & $(6,1)$ & 5 & 1 & YES & YES & YES & $1.67$ & $(4,2)$ & NO & 10104\\
$(289,63)$ & 13 & $(7,3)$ & 4 & 1 & YES & YES & YES & $1.75$ & $(4,2)$ & -- & 10105\\
$(289,80)$ & 12 & $(11,3)$ & 5 & 1 & YES & YES & YES & $1.50$ & $(2,3)$ & 7384 & 10106\\
$(289,86)$ & 13 & $(11,3)$ & 5 & 1 & YES & YES & YES & $1.62$ & $(4,2)$ & NO & 10107\\
$(289,112)$ & 12 & $(13,5)$ & 5 & 1 & YES & YES & YES & $1.43$ & $(4,2)$ & NO & 10108\\
$(289,112)$ & 12 & $(18,7)$ & 6 & 1 & YES & YES & YES & $1.62$ & $(4,2)$ & NO & 10109\\
$(289,80)$ & 12 & $(29,8)$ & 7 & 1 & YES & YES & YES & $1.50$ & $(2,3)$ & NO & 10110\\
$(289,84)$ & 13 & $(31,9)$ & 8 & 1 & YES & YES & NO(2) & $1.60$ & $(6,1)$ & NO & 10111\\
$(289,112)$ & 12 & $(31,12)$ & 7 & 1 & YES & YES & YES & $1.78$ & $(2,3)$ & NO & 10112\\
$(289,80)$ & 12 & $(47,13)$ & 8 & 1 & YES & YES & YES & $1.50$ & $(4,2)$ & NO & 10113\\
$(289,112)$ & 12 & $(49,19)$ & 8 & 1 & YES & YES & YES & $1.57$ & $(4,2)$ & NO & 10114\\
$(289,112)$ & 12 & $(80,31)$ & 9 & 1 & YES & YES & YES & $1.67$ & $(4,2)$ & NO & 10115\\
$(289,112)$ & 12 & $(129,50)$ & 10 & 1 & YES & YES & YES & $1.71$ & $(2,3)$ & 10869 & 10116\\
$(289,112)$ & 12 & $(209,81)$ & 11 & 1 & YES & YES & YES & $1.57$ & $(2,3)$ & NO & 10117\\
$(289,86)$ & 13 & $(289,86)$ & 13 & 289 & YES & YES & YES & $1.50$ & $(4,2)$ & NO & 10118\\
$(290,77)$ & 13 & $(2,1)$ & 1 & 2 & YES & YES & YES & $1.43$ & $(2,3)$ & NO & 10119\\
$(290,81)$ & 12 & $(2,1)$ & 1 & 2 & YES & YES & YES & $1.78$ & $(2,3)$ & -- & 10120\\
$(290,111)$ & 12 & $(2,1)$ & 1 & 2 & NO & YES & NO(2) & $1.60$ & $(6,1)$ & -- & 10121\\
$(290,111)$ & 12 & $(2,1)$ & 1 & 2 & YES & YES & YES & $1.89$ & $(2,3)$ & NO & 10122\\
$(290,77)$ & 13 & $(3,1)$ & 2 & 1 & YES & YES & YES & $1.57$ & $(2,3)$ & NO & 10123\\
$(290,81)$ & 12 & $(3,1)$ & 2 & 1 & YES & YES & YES & $1.67$ & $(2,3)$ & NO & 10124\\
$(290,81)$ & 12 & $(3,1)$ & 2 & 1 & YES & YES & YES & $1.38$ & $(2,3)$ & -- & 10125\\
$(290,111)$ & 12 & $(3,1)$ & 2 & 1 & YES & YES & YES & $2.00$ & $(2,3)$ & NO & 10126\\
$(290,111)$ & 12 & $(3,1)$ & 2 & 1 & YES & YES & YES & $2.00$ & $(2,3)$ & -- & 10127\\
$(290,81)$ & 12 & $(4,1)$ & 3 & 2 & YES & YES & YES & $1.50$ & $(4,2)$ & 7459 & 10128\\
$(290,81)$ & 12 & $(5,2)$ & 3 & 5 & YES & YES & YES & $1.57$ & $(2,3)$ & -- & 10129\\
$(290,81)$ & 12 & $(7,2)$ & 4 & 1 & YES & YES & YES & $1.43$ & $(6,1)$ & 8820 & 10130\\
$(290,111)$ & 12 & $(8,3)$ & 4 & 2 & YES & YES & YES & $1.70$ & $(2,3)$ & NO & 10131\\
$(290,81)$ & 12 & $(10,3)$ & 5 & 10 & YES & YES & YES & $1.62$ & $(2,3)$ & NO & 10132\\
$(290,81)$ & 12 & $(18,5)$ & 6 & 2 & YES & YES & YES & $1.50$ & $(2,3)$ & NO & 10133\\
$(290,81)$ & 12 & $(25,7)$ & 7 & 5 & YES & YES & YES & $1.62$ & $(2,3)$ & 9296 & 10134\\
$(290,81)$ & 12 & $(32,9)$ & 8 & 2 & YES & YES & YES & $1.86$ & $(2,3)$ & NO & 10135\\
$(290,111)$ & 12 & $(34,13)$ & 7 & 2 & YES & YES & YES & $1.70$ & $(2,3)$ & NO & 10136\\
$(290,81)$ & 12 & $(43,12)$ & 8 & 1 & YES & YES & YES & $1.50$ & $(2,3)$ & NO & 10137\\
$(290,81)$ & 12 & $(61,17)$ & 9 & 1 & YES & YES & YES & $1.57$ & $(2,3)$ & NO & 10138\\
$(290,81)$ & 12 & $(111,31)$ & 10 & 1 & YES & YES & YES & $1.62$ & $(2,3)$ & NO & 10139\\
$(290,111)$ & 12 & $(128,49)$ & 10 & 2 & YES & YES & YES & $1.86$ & $(2,3)$ & 10841 & 10140\\
$(291,85)$ & 13 & $(3,1)$ & 2 & 3 & YES & YES & NO(2) & $1.50$ & $(4,2)$ & NO & 10141\\
$(291,85)$ & 13 & $(3,1)$ & 2 & 3 & YES & YES & YES & $1.50$ & $(4,2)$ & -- & 10142\\
$(291,89)$ & 13 & $(3,1)$ & 2 & 3 & YES & YES & YES & $1.50$ & $(4,2)$ & -- & 10143\\
$(291,85)$ & 13 & $(4,1)$ & 3 & 1 & YES & YES & YES & $1.57$ & $(4,2)$ & -- & 10144\\
$(291,68)$ & 13 & $(7,3)$ & 4 & 1 & YES & YES & YES & $1.62$ & $(4,2)$ & -- & 10145\\
$(291,89)$ & 13 & $(7,2)$ & 4 & 1 & YES & YES & YES & $1.62$ & $(4,2)$ & NO & 10146\\
$(291,85)$ & 13 & $(10,3)$ & 5 & 1 & YES & YES & YES & $1.57$ & $(4,2)$ & NO & 10147\\
$(291,62)$ & 14 & $(19,4)$ & 7 & 1 & YES & YES & YES & $1.43$ & $(2,3)$ & NO & 10148\\
$(291,68)$ & 13 & $(22,5)$ & 7 & 1 & YES & YES & YES & $1.75$ & $(2,3)$ & NO & 10149\\
$(291,85)$ & 13 & $(65,19)$ & 9 & 1 & YES & YES & YES & $1.43$ & $(4,2)$ & NO & 10150\\
$(291,85)$ & 13 & $(202,59)$ & 12 & 1 & YES & YES & YES & $1.71$ & $(4,2)$ & NO & 10151\\
$(292,111)$ & 12 & $(2,1)$ & 1 & 2 & YES & YES & YES & $1.43$ & $(4,2)$ & -- & 10152\\
$(292,121)$ & 12 & $(2,1)$ & 1 & 2 & YES & YES & YES & $1.62$ & $(2,3)$ & -- & 10153\\
$(292,85)$ & 13 & $(3,1)$ & 2 & 1 & YES & YES & YES & $1.67$ & $(2,3)$ & -- & 10154\\
$(292,111)$ & 12 & $(3,1)$ & 2 & 1 & YES & YES & YES & $1.78$ & $(2,3)$ & -- & 10155\\
$(292,111)$ & 12 & $(3,1)$ & 2 & 1 & YES & YES & YES & $1.78$ & $(2,3)$ & NO & 10156\\
$(292,121)$ & 12 & $(3,1)$ & 2 & 1 & YES & YES & YES & $1.62$ & $(2,3)$ & -- & 10157\\
$(292,121)$ & 12 & $(3,1)$ & 2 & 1 & YES & YES & YES & $1.62$ & $(4,2)$ & NO & 10158\\
$(292,85)$ & 13 & $(4,1)$ & 3 & 4 & YES & YES & YES & $1.75$ & $(2,3)$ & -- & 10159\\
$(292,111)$ & 12 & $(4,1)$ & 3 & 4 & YES & YES & YES & $1.89$ & $(2,3)$ & -- & 10160\\
$(292,121)$ & 12 & $(4,1)$ & 3 & 4 & YES & YES & YES & $1.86$ & $(2,3)$ & NO & 10161\\
$(292,121)$ & 12 & $(4,1)$ & 3 & 4 & YES & YES & YES & $2.00$ & $(2,3)$ & -- & 10162\\
$(292,67)$ & 14 & $(5,1)$ & 4 & 1 & YES & YES & NO(2) & $1.50$ & $(6,1)$ & NO & 10163\\
$(292,111)$ & 12 & $(5,1)$ & 4 & 1 & YES & YES & YES & $1.62$ & $(4,2)$ & NO & 10164\\
$(292,111)$ & 12 & $(5,1)$ & 4 & 1 & YES & YES & YES & $1.62$ & $(4,2)$ & -- & 10165\\
$(292,111)$ & 12 & $(5,2)$ & 3 & 1 & YES & YES & YES & $1.89$ & $(2,3)$ & NO & 10166\\
$(292,121)$ & 12 & $(5,1)$ & 4 & 1 & YES & YES & YES & $1.86$ & $(2,3)$ & NO & 10167\\
$(292,121)$ & 12 & $(7,3)$ & 4 & 1 & YES & YES & YES & $1.75$ & $(2,3)$ & NO & 10168\\
$(292,85)$ & 13 & $(10,3)$ & 5 & 2 & YES & YES & YES & $1.62$ & $(2,3)$ & NO & 10169\\
$(292,85)$ & 13 & $(17,5)$ & 6 & 1 & YES & YES & YES & $1.43$ & $(2,3)$ & NO & 10170\\
$(292,121)$ & 12 & $(29,12)$ & 7 & 1 & YES & YES & YES & $1.67$ & $(2,3)$ & 9636 & 10171\\
$(292,85)$ & 13 & $(31,9)$ & 8 & 1 & YES & YES & YES & $1.80$ & $(2,3)$ & NO & 10172\\
$(292,111)$ & 12 & $(50,19)$ & 8 & 2 & YES & YES & YES & $1.62$ & $(4,2)$ & 8518 & 10173\\
$(292,121)$ & 12 & $(70,29)$ & 9 & 2 & YES & YES & YES & $1.75$ & $(4,2)$ & 9238 & 10174\\
$(292,111)$ & 12 & $(71,27)$ & 9 & 1 & YES & YES & YES & $1.90$ & $(2,3)$ & NO & 10175\\
$(292,111)$ & 12 & $(92,35)$ & 10 & 4 & YES & YES & YES & $1.62$ & $(4,2)$ & NO & 10176\\
$(292,121)$ & 12 & $(111,46)$ & 10 & 1 & YES & YES & YES & $1.62$ & $(2,3)$ & NO & 10177\\
$(292,111)$ & 12 & $(121,46)$ & 10 & 1 & YES & YES & YES & $1.43$ & $(4,2)$ & NO & 10178\\
$(292,79)$ & 13 & $(122,33)$ & 11 & 2 & YES & YES & YES & $1.43$ & $(4,2)$ & 10743 & 10179\\
$(292,85)$ & 13 & $(134,39)$ & 11 & 2 & YES & YES & YES & $1.67$ & $(2,3)$ & 10953 & 10180\\
$(292,111)$ & 12 & $(171,65)$ & 11 & 1 & YES & YES & YES & $1.78$ & $(2,3)$ & NO & 10181\\
$(292,121)$ & 12 & $(181,75)$ & 11 & 1 & YES & YES & YES & $1.62$ & $(2,3)$ & NO & 10182\\
$(292,85)$ & 13 & $(292,85)$ & 13 & 292 & YES & YES & YES & $1.67$ & $(2,3)$ & NO & 10183\\
$(292,121)$ & 12 & $(292,121)$ & 12 & 292 & YES & YES & YES & $1.62$ & $(4,2)$ & NO & 10184\\
$(293,81)$ & 12 & $(2,1)$ & 1 & 1 & YES & YES & YES & $1.56$ & $(2,3)$ & -- & 10185\\
$(293,81)$ & 12 & $(2,1)$ & 1 & 1 & YES & YES & YES & $1.56$ & $(2,3)$ & NO & 10186\\
$(293,89)$ & 13 & $(2,1)$ & 1 & 1 & YES & YES & NO(2) & $1.67$ & $(2,3)$ & NO & 10187\\
$(293,89)$ & 13 & $(3,1)$ & 2 & 1 & YES & YES & YES & $1.78$ & $(2,3)$ & -- & 10188\\
$(293,123)$ & 12 & $(3,1)$ & 2 & 1 & YES & YES & YES & $1.43$ & $(2,3)$ & -- & 10189\\
$(293,123)$ & 12 & $(3,1)$ & 2 & 1 & YES & YES & YES & $1.57$ & $(2,3)$ & NO & 10190\\
$(293,79)$ & 13 & $(5,2)$ & 3 & 1 & YES & YES & YES & $1.75$ & $(4,2)$ & NO & 10191\\
$(293,79)$ & 13 & $(5,2)$ & 3 & 1 & YES & YES & YES & $1.75$ & $(4,2)$ & -- & 10192\\
$(293,64)$ & 13 & $(7,2)$ & 4 & 1 & YES & YES & YES & $1.62$ & $(4,2)$ & NO & 10193\\
$(293,81)$ & 12 & $(7,2)$ & 4 & 1 & YES & YES & YES & $1.62$ & $(2,3)$ & NO & 10194\\
$(293,89)$ & 13 & $(7,2)$ & 4 & 1 & YES & YES & YES & $1.90$ & $(2,3)$ & NO & 10195\\
$(293,64)$ & 13 & $(14,3)$ & 6 & 1 & YES & YES & NO(2) & $1.44$ & $(2,3)$ & NO & 10196\\
$(293,81)$ & 12 & $(29,8)$ & 7 & 1 & YES & YES & YES & $1.56$ & $(2,3)$ & 7202 & 10197\\
$(293,89)$ & 13 & $(33,10)$ & 8 & 1 & YES & YES & YES & $1.90$ & $(2,3)$ & NO & 10198\\
$(293,64)$ & 13 & $(55,12)$ & 9 & 1 & YES & YES & NO(2) & $1.44$ & $(2,3)$ & NO & 10199\\
$(293,79)$ & 13 & $(89,24)$ & 10 & 1 & YES & YES & NO(2) & $1.38$ & $(4,2)$ & NO & 10200\\
$(293,89)$ & 13 & $(293,89)$ & 13 & 293 & YES & YES & YES & $1.62$ & $(4,2)$ & NO & 10201\\
$(294,67)$ & 13 & $(2,1)$ & 1 & 2 & YES & YES & YES & $1.50$ & $(2,3)$ & -- & 10202\\
$(294,67)$ & 13 & $(3,1)$ & 2 & 3 & YES & YES & YES & $1.78$ & $(2,3)$ & -- & 10203\\
$(294,67)$ & 13 & $(3,1)$ & 2 & 3 & YES & YES & YES & $1.56$ & $(2,3)$ & NO & 10204\\
$(294,67)$ & 13 & $(5,2)$ & 3 & 1 & YES & YES & YES & $1.71$ & $(2,3)$ & NO & 10205\\
$(294,67)$ & 13 & $(7,2)$ & 4 & 7 & YES & YES & YES & $1.57$ & $(2,3)$ & -- & 10206\\
$(294,67)$ & 13 & $(11,2)$ & 6 & 1 & YES & YES & YES & $1.75$ & $(2,3)$ & NO & 10207\\
$(294,67)$ & 13 & $(13,3)$ & 6 & 1 & YES & YES & YES & $1.50$ & $(2,3)$ & NO & 10208\\
$(294,67)$ & 13 & $(17,4)$ & 7 & 1 & YES & YES & YES & $1.43$ & $(2,3)$ & NO & 10209\\
$(294,67)$ & 13 & $(31,7)$ & 8 & 1 & YES & YES & YES & $1.88$ & $(2,3)$ & NO & 10210\\
$(294,67)$ & 13 & $(35,8)$ & 8 & 7 & YES & YES & YES & $1.89$ & $(2,3)$ & NO & 10211\\
$(294,67)$ & 13 & $(48,11)$ & 9 & 6 & YES & YES & YES & $1.57$ & $(2,3)$ & NO & 10212\\
$(294,79)$ & 13 & $(294,79)$ & 13 & 294 & YES & YES & YES & $1.75$ & $(2,3)$ & NO & 10213\\
$(295,112)$ & 12 & $(2,1)$ & 1 & 1 & YES & YES & YES & $1.89$ & $(2,3)$ & NO & 10214\\
$(295,108)$ & 12 & $(3,1)$ & 2 & 1 & YES & YES & YES & $1.57$ & $(4,2)$ & -- & 10215\\
$(295,112)$ & 12 & $(3,1)$ & 2 & 1 & YES & YES & YES & $1.86$ & $(2,3)$ & -- & 10216\\
$(295,108)$ & 12 & $(4,1)$ & 3 & 1 & YES & YES & YES & $1.62$ & $(2,3)$ & -- & 10217\\
$(295,112)$ & 12 & $(4,1)$ & 3 & 1 & YES & YES & YES & $1.62$ & $(4,2)$ & -- & 10218\\
$(295,108)$ & 12 & $(5,1)$ & 4 & 5 & YES & YES & YES & $1.80$ & $(2,3)$ & NO & 10219\\
$(295,112)$ & 12 & $(5,1)$ & 4 & 5 & YES & YES & YES & $1.62$ & $(4,2)$ & NO & 10220\\
$(295,108)$ & 12 & $(71,26)$ & 9 & 1 & YES & YES & YES & $1.71$ & $(4,2)$ & 9302 & 10221\\
$(295,112)$ & 12 & $(79,30)$ & 9 & 1 & YES & YES & YES & $1.88$ & $(2,3)$ & 9601 & 10222\\
$(295,108)$ & 12 & $(112,41)$ & 10 & 1 & YES & YES & YES & $1.80$ & $(2,3)$ & NO & 10223\\
$(295,112)$ & 12 & $(187,71)$ & 11 & 1 & YES & YES & YES & $1.86$ & $(2,3)$ & NO & 10224\\
$(296,53)$ & 14 & $(2,1)$ & 1 & 2 & YES & YES & NO(2) & $1.50$ & $(2,3)$ & -- & 10225\\
$(296,53)$ & 14 & $(2,1)$ & 1 & 2 & YES & YES & NO(2) & $1.60$ & $(2,3)$ & NO & 10226\\
$(296,83)$ & 13 & $(4,1)$ & 3 & 4 & YES & YES & YES & $1.62$ & $(2,3)$ & -- & 10227\\
$(296,67)$ & 14 & $(6,1)$ & 5 & 2 & YES & YES & YES & $1.62$ & $(4,2)$ & NO & 10228\\
$(296,83)$ & 13 & $(7,2)$ & 4 & 1 & YES & YES & YES & $1.57$ & $(2,3)$ & NO & 10229\\
$(296,47)$ & 15 & $(9,2)$ & 5 & 1 & YES & YES & NO(2) & $1.44$ & $(4,2)$ & NO & 10230\\
$(296,53)$ & 14 & $(17,3)$ & 7 & 1 & YES & YES & NO(2) & $1.60$ & $(2,3)$ & NO & 10231\\
$(296,83)$ & 13 & $(18,5)$ & 6 & 2 & YES & YES & YES & $1.78$ & $(2,3)$ & NO & 10232\\
$(296,67)$ & 14 & $(84,19)$ & 10 & 4 & YES & YES & YES & $1.62$ & $(4,2)$ & NO & 10233\\
$(296,67)$ & 14 & $(243,55)$ & 13 & 1 & YES & YES & YES & $1.62$ & $(4,2)$ & NO & 10234\\
$(297,136)$ & 14 & $(2,1)$ & 1 & 1 & NO & YES & NO(2) & $1.70$ & $(6,1)$ & -- & 10235\\
$(297,83)$ & 13 & $(3,1)$ & 2 & 3 & YES & YES & YES & $1.70$ & $(2,3)$ & -- & 10236\\
$(297,109)$ & 12 & $(3,1)$ & 2 & 3 & YES & YES & YES & $1.62$ & $(4,2)$ & -- & 10237\\
$(297,109)$ & 12 & $(4,1)$ & 3 & 1 & YES & YES & YES & $1.62$ & $(4,2)$ & -- & 10238\\
$(297,109)$ & 12 & $(4,1)$ & 3 & 1 & YES & YES & YES & $1.78$ & $(4,2)$ & NO & 10239\\
$(297,113)$ & 13 & $(4,1)$ & 3 & 1 & YES & YES & YES & $1.62$ & $(4,2)$ & -- & 10240\\
$(297,68)$ & 13 & $(5,2)$ & 3 & 1 & YES & YES & YES & $1.78$ & $(2,3)$ & -- & 10241\\
$(297,109)$ & 12 & $(5,1)$ & 4 & 1 & YES & YES & YES & $1.62$ & $(4,2)$ & NO & 10242\\
$(297,109)$ & 12 & $(5,1)$ & 4 & 1 & YES & YES & YES & $1.67$ & $(4,2)$ & -- & 10243\\
$(297,109)$ & 12 & $(8,3)$ & 4 & 1 & YES & YES & YES & $1.62$ & $(4,2)$ & NO & 10244\\
$(297,83)$ & 13 & $(18,5)$ & 6 & 9 & YES & YES & YES & $1.78$ & $(2,3)$ & NO & 10245\\
$(297,83)$ & 13 & $(43,12)$ & 8 & 1 & YES & YES & YES & $1.67$ & $(4,2)$ & NO & 10246\\
$(297,113)$ & 13 & $(50,19)$ & 8 & 1 & YES & YES & YES & $1.62$ & $(4,2)$ & NO & 10247\\
$(297,109)$ & 12 & $(109,40)$ & 10 & 1 & YES & YES & YES & $1.75$ & $(4,2)$ & NO & 10248\\
$(297,92)$ & 13 & $(113,35)$ & 11 & 1 & YES & YES & YES & $1.57$ & $(4,2)$ & NO & 10249\\
$(297,92)$ & 13 & $(184,57)$ & 12 & 1 & YES & YES & YES & $1.57$ & $(4,2)$ & NO & 10250\\
$(297,52)$ & 16 & $(217,38)$ & 14 & 1 & YES & YES & NO(2) & $1.60$ & $(6,1)$ & 11687 & 10251\\
$(297,83)$ & 13 & $(229,64)$ & 12 & 1 & YES & YES & YES & $1.71$ & $(2,3)$ & NO & 10252\\
$(298,79)$ & 13 & $(2,1)$ & 1 & 2 & YES & YES & YES & $1.43$ & $(4,2)$ & NO & 10253\\
$(298,131)$ & 13 & $(2,1)$ & 1 & 2 & NO & YES & NO(2) & $1.67$ & $(4,2)$ & -- & 10254\\
$(298,65)$ & 14 & $(3,1)$ & 2 & 1 & YES & YES & YES & $1.62$ & $(4,2)$ & -- & 10255\\
$(298,79)$ & 13 & $(3,1)$ & 2 & 1 & YES & YES & YES & $1.43$ & $(4,2)$ & NO & 10256\\
$(298,83)$ & 13 & $(3,1)$ & 2 & 1 & YES & YES & YES & $1.43$ & $(4,2)$ & -- & 10257\\
$(298,83)$ & 13 & $(3,1)$ & 2 & 1 & YES & YES & YES & $1.57$ & $(4,2)$ & NO & 10258\\
$(298,91)$ & 13 & $(4,1)$ & 3 & 2 & YES & YES & YES & $1.89$ & $(2,3)$ & NO & 10259\\
$(298,83)$ & 13 & $(5,1)$ & 4 & 1 & YES & YES & YES & $1.50$ & $(4,2)$ & NO & 10260\\
$(298,53)$ & 15 & $(6,1)$ & 5 & 2 & YES & YES & NO(2) & $1.50$ & $(6,1)$ & NO & 10261\\
$(298,91)$ & 13 & $(10,3)$ & 5 & 2 & YES & YES & YES & $1.75$ & $(2,3)$ & NO & 10262\\
$(298,65)$ & 14 & $(14,3)$ & 6 & 2 & YES & YES & YES & $1.57$ & $(4,2)$ & NO & 10263\\
$(298,83)$ & 13 & $(25,7)$ & 7 & 1 & YES & YES & YES & $1.43$ & $(4,2)$ & NO & 10264\\
$(298,79)$ & 13 & $(49,13)$ & 9 & 1 & YES & YES & YES & $1.71$ & $(2,3)$ & NO & 10265\\
$(298,91)$ & 13 & $(59,18)$ & 9 & 1 & YES & YES & YES & $1.75$ & $(2,3)$ & 7136 & 10266\\
$(298,83)$ & 13 & $(61,17)$ & 9 & 1 & YES & YES & YES & $1.89$ & $(2,3)$ & NO & 10267\\
$(298,83)$ & 13 & $(140,39)$ & 11 & 2 & YES & YES & YES & $1.50$ & $(4,2)$ & 11055 & 10268\\
$(298,53)$ & 15 & $(208,37)$ & 13 & 2 & YES & YES & NO(2) & $1.62$ & $(6,1)$ & 11656 & 10269\\
$(298,79)$ & 13 & $(298,79)$ & 13 & 298 & YES & YES & YES & $1.29$ & $(4,2)$ & NO & 10270\\
$(298,83)$ & 13 & $(298,83)$ & 13 & 298 & YES & YES & YES & $1.78$ & $(2,3)$ & NO & 10271\\
$(299,116)$ & 12 & $(2,1)$ & 1 & 1 & YES & YES & YES & $1.57$ & $(2,3)$ & -- & 10272\\
$(299,84)$ & 13 & $(3,1)$ & 2 & 1 & YES & YES & YES & $1.75$ & $(4,2)$ & NO & 10273\\
$(299,84)$ & 13 & $(3,1)$ & 2 & 1 & YES & YES & YES & $1.75$ & $(4,2)$ & -- & 10274\\
$(299,89)$ & 13 & $(3,1)$ & 2 & 1 & YES & YES & YES & $1.75$ & $(4,2)$ & -- & 10275\\
$(299,111)$ & 13 & $(3,1)$ & 2 & 1 & YES & YES & YES & $1.88$ & $(4,2)$ & NO & 10276\\
$(299,116)$ & 12 & $(3,1)$ & 2 & 1 & YES & YES & YES & $1.62$ & $(4,2)$ & NO & 10277\\
$(299,116)$ & 12 & $(3,1)$ & 2 & 1 & YES & YES & YES & $1.62$ & $(4,2)$ & -- & 10278\\
$(299,116)$ & 12 & $(3,1)$ & 2 & 1 & YES & YES & YES & $1.57$ & $(2,3)$ & NO & 10279\\
$(299,116)$ & 12 & $(4,1)$ & 3 & 1 & YES & YES & YES & $1.29$ & $(4,2)$ & -- & 10280\\
$(299,116)$ & 12 & $(5,1)$ & 4 & 1 & YES & YES & YES & $1.50$ & $(4,2)$ & NO & 10281\\
$(299,116)$ & 12 & $(5,2)$ & 3 & 1 & YES & YES & YES & $1.75$ & $(4,2)$ & NO & 10282\\
$(299,116)$ & 12 & $(8,3)$ & 4 & 1 & YES & YES & YES & $1.78$ & $(2,3)$ & NO & 10283\\
$(299,116)$ & 12 & $(13,5)$ & 5 & 13 & YES & YES & YES & $1.75$ & $(4,2)$ & 7599 & 10284\\
$(299,84)$ & 13 & $(18,5)$ & 6 & 1 & YES & YES & YES & $1.43$ & $(6,1)$ & NO & 10285\\
$(299,58)$ & 16 & $(21,4)$ & 8 & 1 & YES & YES & NO(2) & $1.60$ & $(6,1)$ & NO & 10286\\
$(299,58)$ & 16 & $(26,5)$ & 9 & 13 & YES & YES & NO(2) & $1.50$ & $(6,1)$ & NO & 10287\\
$(299,89)$ & 13 & $(47,14)$ & 9 & 1 & YES & YES & YES & $1.75$ & $(4,2)$ & NO & 10288\\
$(299,116)$ & 12 & $(49,19)$ & 8 & 1 & YES & YES & YES & $1.29$ & $(4,2)$ & NO & 10289\\
$(299,84)$ & 13 & $(57,16)$ & 9 & 1 & YES & YES & YES & $1.62$ & $(4,2)$ & NO & 10290\\
$(299,116)$ & 12 & $(67,26)$ & 9 & 1 & YES & YES & YES & $1.62$ & $(4,2)$ & 9202 & 10291\\
$(299,116)$ & 12 & $(183,71)$ & 11 & 1 & YES & YES & YES & $1.57$ & $(2,3)$ & NO & 10292\\
$(300,91)$ & 13 & $(2,1)$ & 1 & 2 & YES & YES & NO(2) & $1.56$ & $(2,3)$ & NO & 10293\\
$(300,91)$ & 13 & $(3,1)$ & 2 & 3 & YES & YES & YES & $1.43$ & $(4,2)$ & -- & 10294\\
$(300,91)$ & 13 & $(3,1)$ & 2 & 3 & YES & YES & YES & $1.75$ & $(2,3)$ & NO & 10295\\
$(300,89)$ & 13 & $(4,1)$ & 3 & 4 & YES & YES & YES & $1.57$ & $(2,3)$ & -- & 10296\\
$(300,91)$ & 13 & $(5,2)$ & 3 & 5 & YES & YES & YES & $1.62$ & $(4,2)$ & -- & 10297\\
$(300,91)$ & 13 & $(7,2)$ & 4 & 1 & YES & YES & YES & $1.62$ & $(4,2)$ & NO & 10298\\
$(300,89)$ & 13 & $(17,5)$ & 6 & 1 & YES & YES & YES & $1.67$ & $(2,3)$ & NO & 10299\\
$(300,89)$ & 13 & $(37,11)$ & 8 & 1 & YES & YES & YES & $1.43$ & $(2,3)$ & NO & 10300\\
$(300,91)$ & 13 & $(56,17)$ & 9 & 4 & YES & YES & YES & $1.43$ & $(4,2)$ & NO & 10301\\
$(300,89)$ & 13 & $(64,19)$ & 9 & 4 & YES & YES & YES & $1.57$ & $(2,3)$ & NO & 10302\\
$(300,89)$ & 13 & $(118,35)$ & 11 & 2 & YES & YES & YES & $1.78$ & $(2,3)$ & 10732 & 10303\\
$(300,91)$ & 13 & $(122,37)$ & 11 & 2 & YES & YES & YES & $1.89$ & $(2,3)$ & 10784 & 10304\\
$(300,91)$ & 13 & $(300,91)$ & 13 & 300 & YES & YES & YES & $2.00$ & $(2,3)$ & NO & 10305\\
$(301,80)$ & 14 & $(2,1)$ & 1 & 1 & YES & YES & NO(2) & $1.56$ & $(8,0)$ & -- & 10306\\
$(301,88)$ & 13 & $(2,1)$ & 1 & 1 & YES & YES & YES & $1.67$ & $(4,2)$ & -- & 10307\\
$(301,89)$ & 12 & $(2,1)$ & 1 & 1 & YES & YES & YES & $1.50$ & $(2,3)$ & -- & 10308\\
$(301,115)$ & 12 & $(2,1)$ & 1 & 1 & YES & YES & YES & $1.43$ & $(4,2)$ & -- & 10309\\
$(301,115)$ & 12 & $(2,1)$ & 1 & 1 & YES & YES & YES & $1.57$ & $(4,2)$ & NO & 10310\\
$(301,80)$ & 14 & $(3,1)$ & 2 & 1 & YES & YES & NO(2) & $1.38$ & $(6,1)$ & NO & 10311\\
$(301,88)$ & 13 & $(3,1)$ & 2 & 1 & YES & YES & YES & $1.75$ & $(2,3)$ & -- & 10312\\
$(301,88)$ & 13 & $(3,1)$ & 2 & 1 & YES & YES & YES & $1.67$ & $(2,3)$ & NO & 10313\\
$(301,89)$ & 12 & $(3,1)$ & 2 & 1 & YES & YES & YES & $1.43$ & $(6,1)$ & 9184 & 10314\\
$(301,89)$ & 12 & $(3,1)$ & 2 & 1 & YES & YES & YES & $1.38$ & $(2,3)$ & -- & 10315\\
$(301,115)$ & 12 & $(4,1)$ & 3 & 1 & YES & YES & YES & $1.71$ & $(2,3)$ & NO & 10316\\
$(301,65)$ & 13 & $(5,2)$ & 3 & 1 & YES & YES & YES & $1.43$ & $(4,2)$ & -- & 10317\\
$(301,65)$ & 13 & $(5,2)$ & 3 & 1 & YES & YES & YES & $1.78$ & $(2,3)$ & NO & 10318\\
$(301,88)$ & 13 & $(5,1)$ & 4 & 1 & YES & YES & YES & $1.75$ & $(2,3)$ & NO & 10319\\
$(301,89)$ & 12 & $(5,1)$ & 4 & 1 & YES & YES & YES & $1.50$ & $(2,3)$ & NO & 10320\\
$(301,114)$ & 13 & $(5,1)$ & 4 & 1 & YES & YES & YES & $1.78$ & $(4,2)$ & NO & 10321\\
$(301,115)$ & 12 & $(5,2)$ & 3 & 1 & YES & YES & YES & $1.57$ & $(4,2)$ & NO & 10322\\
$(301,89)$ & 12 & $(7,2)$ & 4 & 7 & YES & YES & YES & $1.50$ & $(2,3)$ & NO & 10323\\
$(301,65)$ & 13 & $(13,3)$ & 6 & 1 & YES & YES & YES & $1.43$ & $(4,2)$ & 11592 & 10324\\
$(301,115)$ & 12 & $(13,5)$ & 5 & 1 & YES & YES & YES & $1.78$ & $(2,3)$ & 7338 & 10325\\
$(301,88)$ & 13 & $(17,5)$ & 6 & 1 & YES & YES & YES & $1.78$ & $(2,3)$ & NO & 10326\\
$(301,89)$ & 12 & $(17,5)$ & 6 & 1 & YES & YES & YES & $1.50$ & $(2,3)$ & NO & 10327\\
$(301,115)$ & 12 & $(21,8)$ & 6 & 7 & YES & YES & YES & $1.78$ & $(2,3)$ & NO & 10328\\
$(301,65)$ & 13 & $(23,5)$ & 7 & 1 & YES & YES & NO(2) & $1.56$ & $(2,3)$ & 8426 & 10329\\
$(301,88)$ & 13 & $(24,7)$ & 7 & 1 & YES & YES & YES & $1.75$ & $(4,2)$ & NO & 10330\\
$(301,89)$ & 12 & $(27,8)$ & 7 & 1 & YES & YES & YES & $1.38$ & $(2,3)$ & 9620 & 10331\\
$(301,88)$ & 13 & $(41,12)$ & 8 & 1 & YES & YES & YES & $1.29$ & $(4,2)$ & NO & 10332\\
$(301,89)$ & 12 & $(44,13)$ & 8 & 1 & YES & YES & YES & $1.50$ & $(2,3)$ & NO & 10333\\
$(301,115)$ & 12 & $(123,47)$ & 10 & 1 & YES & YES & YES & $1.71$ & $(2,3)$ & 10811 & 10334\\
$(301,88)$ & 13 & $(171,50)$ & 11 & 1 & YES & YES & YES & $1.78$ & $(2,3)$ & 11475 & 10335\\
$(301,89)$ & 12 & $(186,55)$ & 11 & 1 & YES & YES & YES & $1.50$ & $(2,3)$ & NO & 10336\\
$(301,114)$ & 13 & $(235,89)$ & 12 & 1 & YES & YES & YES & $1.89$ & $(4,2)$ & NO & 10337\\
$(301,89)$ & 12 & $(301,89)$ & 12 & 301 & YES & YES & YES & $1.50$ & $(2,3)$ & NO & 10338\\
$(301,115)$ & 12 & $(301,115)$ & 12 & 301 & YES & YES & YES & $1.57$ & $(2,3)$ & NO & 10339\\
$(302,117)$ & 12 & $(2,1)$ & 1 & 2 & YES & YES & YES & $1.43$ & $(4,2)$ & NO & 10340\\
$(302,45)$ & 16 & $(3,1)$ & 2 & 1 & YES & YES & NO(2) & $1.50$ & $(6,1)$ & NO & 10341\\
$(302,45)$ & 16 & $(3,1)$ & 2 & 1 & YES & YES & NO(2) & $1.50$ & $(6,1)$ & -- & 10342\\
$(302,111)$ & 12 & $(3,1)$ & 2 & 1 & YES & YES & YES & $1.67$ & $(4,2)$ & -- & 10343\\
$(302,117)$ & 12 & $(3,1)$ & 2 & 1 & YES & YES & YES & $1.67$ & $(2,3)$ & -- & 10344\\
$(302,117)$ & 12 & $(3,1)$ & 2 & 1 & YES & YES & YES & $1.43$ & $(2,3)$ & NO & 10345\\
$(302,45)$ & 16 & $(4,1)$ & 3 & 2 & YES & YES & NO(2) & $1.56$ & $(4,2)$ & NO & 10346\\
$(302,117)$ & 12 & $(8,3)$ & 4 & 2 & YES & YES & YES & $1.67$ & $(4,2)$ & NO & 10347\\
$(302,117)$ & 12 & $(13,5)$ & 5 & 1 & YES & YES & YES & $1.62$ & $(2,3)$ & NO & 10348\\
$(302,131)$ & 13 & $(23,10)$ & 7 & 1 & YES & YES & YES & $1.43$ & $(6,1)$ & NO & 10349\\
$(302,117)$ & 12 & $(111,43)$ & 10 & 1 & YES & YES & YES & $1.62$ & $(2,3)$ & NO & 10350\\
$(302,131)$ & 13 & $(136,59)$ & 11 & 2 & YES & YES & YES & $1.29$ & $(6,1)$ & 11019 & 10351\\
$(302,111)$ & 12 & $(185,68)$ & 11 & 1 & YES & YES & YES & $1.78$ & $(4,2)$ & NO & 10352\\
$(302,117)$ & 12 & $(191,74)$ & 11 & 1 & YES & YES & YES & $1.43$ & $(2,3)$ & NO & 10353\\
$(302,117)$ & 12 & $(302,117)$ & 12 & 302 & YES & YES & YES & $1.67$ & $(4,2)$ & NO & 10354\\
$(303,85)$ & 13 & $(2,1)$ & 1 & 1 & YES & YES & YES & $1.50$ & $(2,3)$ & -- & 10355\\
$(303,116)$ & 12 & $(2,1)$ & 1 & 1 & YES & YES & YES & $1.78$ & $(2,3)$ & -- & 10356\\
$(303,116)$ & 12 & $(2,1)$ & 1 & 1 & YES & YES & YES & $1.89$ & $(4,2)$ & NO & 10357\\
$(303,128)$ & 12 & $(2,1)$ & 1 & 1 & YES & YES & YES & $1.57$ & $(4,2)$ & -- & 10358\\
$(303,85)$ & 13 & $(3,1)$ & 2 & 3 & YES & YES & YES & $1.62$ & $(4,2)$ & -- & 10359\\
$(303,116)$ & 12 & $(3,1)$ & 2 & 3 & YES & YES & YES & $1.71$ & $(2,3)$ & -- & 10360\\
$(303,82)$ & 13 & $(4,1)$ & 3 & 1 & YES & YES & YES & $1.75$ & $(2,3)$ & -- & 10361\\
$(303,85)$ & 13 & $(4,1)$ & 3 & 1 & YES & YES & YES & $1.62$ & $(2,3)$ & -- & 10362\\
$(303,116)$ & 12 & $(4,1)$ & 3 & 1 & YES & YES & YES & $1.67$ & $(2,3)$ & -- & 10363\\
$(303,116)$ & 12 & $(5,1)$ & 4 & 1 & YES & YES & YES & $1.70$ & $(2,3)$ & NO & 10364\\
$(303,128)$ & 12 & $(5,2)$ & 3 & 1 & YES & YES & YES & $1.78$ & $(2,3)$ & 7683 & 10365\\
$(303,85)$ & 13 & $(11,3)$ & 5 & 1 & YES & YES & YES & $1.70$ & $(2,3)$ & NO & 10366\\
$(303,82)$ & 13 & $(15,4)$ & 6 & 3 & YES & YES & YES & $1.50$ & $(2,3)$ & NO & 10367\\
$(303,85)$ & 13 & $(18,5)$ & 6 & 3 & YES & YES & YES & $1.57$ & $(2,3)$ & NO & 10368\\
$(303,128)$ & 12 & $(19,8)$ & 6 & 1 & YES & YES & YES & $1.78$ & $(2,3)$ & NO & 10369\\
$(303,85)$ & 13 & $(32,9)$ & 8 & 1 & YES & YES & YES & $1.80$ & $(2,3)$ & NO & 10370\\
$(303,116)$ & 12 & $(34,13)$ & 7 & 1 & YES & YES & YES & $1.70$ & $(2,3)$ & 7343 & 10371\\
$(303,116)$ & 12 & $(47,18)$ & 8 & 1 & YES & YES & YES & $1.43$ & $(4,2)$ & 8531 & 10372\\
$(303,85)$ & 13 & $(57,16)$ & 9 & 3 & YES & YES & YES & $1.75$ & $(2,3)$ & NO & 10373\\
$(303,128)$ & 12 & $(71,30)$ & 9 & 1 & YES & YES & YES & $1.57$ & $(4,2)$ & 9424 & 10374\\
$(303,116)$ & 12 & $(81,31)$ & 9 & 3 & YES & YES & YES & $1.78$ & $(2,3)$ & NO & 10375\\
$(303,116)$ & 12 & $(128,49)$ & 10 & 1 & YES & YES & YES & $1.78$ & $(2,3)$ & NO & 10376\\
$(303,128)$ & 12 & $(187,79)$ & 11 & 1 & YES & YES & YES & $1.67$ & $(4,2)$ & NO & 10377\\
$(303,116)$ & 12 & $(303,116)$ & 12 & 303 & YES & YES & YES & $1.78$ & $(2,3)$ & NO & 10378\\
$(304,85)$ & 13 & $(3,1)$ & 2 & 1 & YES & YES & YES & $1.67$ & $(2,3)$ & -- & 10379\\
$(304,85)$ & 13 & $(11,3)$ & 5 & 1 & YES & YES & YES & $1.67$ & $(2,3)$ & NO & 10380\\
$(304,85)$ & 13 & $(68,19)$ & 9 & 4 & YES & YES & YES & $1.75$ & $(2,3)$ & NO & 10381\\
$(304,71)$ & 14 & $(77,18)$ & 10 & 1 & YES & YES & NO(2) & $1.56$ & $(4,2)$ & 7615 & 10382\\
$(305,84)$ & 13 & $(2,1)$ & 1 & 1 & YES & YES & YES & $1.78$ & $(2,3)$ & -- & 10383\\
$(305,84)$ & 13 & $(2,1)$ & 1 & 1 & YES & YES & YES & $1.70$ & $(2,3)$ & NO & 10384\\
$(305,112)$ & 12 & $(2,1)$ & 1 & 1 & YES & YES & YES & $1.80$ & $(2,3)$ & -- & 10385\\
$(305,128)$ & 12 & $(2,1)$ & 1 & 1 & YES & YES & YES & $1.67$ & $(4,2)$ & -- & 10386\\
$(305,93)$ & 13 & $(3,1)$ & 2 & 1 & YES & YES & YES & $1.67$ & $(4,2)$ & -- & 10387\\
$(305,93)$ & 13 & $(3,1)$ & 2 & 1 & YES & YES & YES & $1.78$ & $(4,2)$ & NO & 10388\\
$(305,128)$ & 12 & $(3,1)$ & 2 & 1 & YES & YES & YES & $1.56$ & $(4,2)$ & -- & 10389\\
$(305,69)$ & 13 & $(4,1)$ & 3 & 1 & YES & YES & YES & $1.38$ & $(2,3)$ & NO & 10390\\
$(305,66)$ & 13 & $(5,2)$ & 3 & 5 & YES & YES & YES & $1.62$ & $(2,3)$ & NO & 10391\\
$(305,69)$ & 13 & $(5,2)$ & 3 & 5 & YES & YES & YES & $1.57$ & $(2,3)$ & -- & 10392\\
$(305,69)$ & 13 & $(5,2)$ & 3 & 5 & YES & YES & YES & $1.71$ & $(2,3)$ & NO & 10393\\
$(305,71)$ & 13 & $(5,2)$ & 3 & 5 & YES & YES & YES & $1.67$ & $(2,3)$ & -- & 10394\\
$(305,66)$ & 13 & $(7,2)$ & 4 & 1 & YES & YES & YES & $1.75$ & $(2,3)$ & NO & 10395\\
$(305,84)$ & 13 & $(7,2)$ & 4 & 1 & YES & YES & YES & $1.70$ & $(2,3)$ & NO & 10396\\
$(305,112)$ & 12 & $(11,4)$ & 5 & 1 & YES & YES & YES & $1.78$ & $(4,2)$ & NO & 10397\\
$(305,69)$ & 13 & $(35,8)$ & 8 & 5 & YES & YES & YES & $1.57$ & $(2,3)$ & NO & 10398\\
$(305,69)$ & 13 & $(40,9)$ & 9 & 5 & YES & YES & YES & $1.50$ & $(4,2)$ & NO & 10399\\
$(305,84)$ & 13 & $(40,11)$ & 8 & 5 & YES & YES & YES & $1.67$ & $(4,2)$ & NO & 10400\\
$(305,71)$ & 13 & $(43,10)$ & 9 & 1 & YES & YES & YES & $1.50$ & $(2,3)$ & NO & 10401\\
$(305,71)$ & 13 & $(47,11)$ & 9 & 1 & YES & YES & YES & $1.67$ & $(2,3)$ & NO & 10402\\
$(305,69)$ & 13 & $(75,17)$ & 10 & 5 & YES & YES & YES & $1.62$ & $(2,3)$ & 11597 & 10403\\
$(305,84)$ & 13 & $(98,27)$ & 10 & 1 & YES & YES & YES & $1.57$ & $(2,3)$ & NO & 10404\\
$(305,128)$ & 12 & $(112,47)$ & 10 & 1 & YES & YES & YES & $1.57$ & $(2,3)$ & NO & 10405\\
$(305,84)$ & 13 & $(167,46)$ & 11 & 1 & YES & YES & YES & $1.67$ & $(4,2)$ & 11450 & 10406\\
$(305,84)$ & 13 & $(305,84)$ & 13 & 305 & YES & YES & YES & $1.57$ & $(2,3)$ & NO & 10407\\
$(306,59)$ & 15 & $(36,7)$ & 11 & 18 & YES & YES & NO(2) & $1.50$ & $(6,1)$ & NO & 10408\\
$(307,119)$ & 12 & $(2,1)$ & 1 & 1 & YES & YES & YES & $1.57$ & $(4,2)$ & NO & 10409\\
$(307,119)$ & 12 & $(2,1)$ & 1 & 1 & YES & YES & YES & $1.43$ & $(4,2)$ & -- & 10410\\
$(307,129)$ & 12 & $(2,1)$ & 1 & 1 & YES & YES & YES & $1.43$ & $(4,2)$ & -- & 10411\\
$(307,67)$ & 14 & $(3,1)$ & 2 & 1 & YES & YES & YES & $1.62$ & $(4,2)$ & -- & 10412\\
$(307,111)$ & 13 & $(3,1)$ & 2 & 1 & YES & YES & YES & $1.75$ & $(4,2)$ & -- & 10413\\
$(307,111)$ & 13 & $(3,1)$ & 2 & 1 & YES & YES & YES & $1.88$ & $(4,2)$ & NO & 10414\\
$(307,119)$ & 12 & $(3,1)$ & 2 & 1 & YES & YES & YES & $1.75$ & $(2,3)$ & -- & 10415\\
$(307,129)$ & 12 & $(3,1)$ & 2 & 1 & YES & YES & YES & $1.62$ & $(2,3)$ & NO & 10416\\
$(307,129)$ & 12 & $(3,1)$ & 2 & 1 & YES & YES & YES & $1.75$ & $(2,3)$ & -- & 10417\\
$(307,55)$ & 14 & $(4,1)$ & 3 & 1 & YES & YES & NO(2) & $1.56$ & $(2,3)$ & NO & 10418\\
$(307,85)$ & 13 & $(4,1)$ & 3 & 1 & YES & YES & YES & $1.78$ & $(2,3)$ & -- & 10419\\
$(307,119)$ & 12 & $(4,1)$ & 3 & 1 & YES & YES & YES & $1.75$ & $(2,3)$ & -- & 10420\\
$(307,126)$ & 13 & $(4,1)$ & 3 & 1 & YES & YES & YES & $1.75$ & $(4,2)$ & NO & 10421\\
$(307,126)$ & 13 & $(4,1)$ & 3 & 1 & YES & YES & YES & $1.75$ & $(4,2)$ & -- & 10422\\
$(307,129)$ & 12 & $(4,1)$ & 3 & 1 & YES & YES & YES & $1.71$ & $(2,3)$ & -- & 10423\\
$(307,129)$ & 12 & $(4,1)$ & 3 & 1 & YES & YES & YES & $1.86$ & $(2,3)$ & NO & 10424\\
$(307,119)$ & 12 & $(5,1)$ & 4 & 1 & YES & YES & YES & $1.62$ & $(4,2)$ & NO & 10425\\
$(307,55)$ & 14 & $(7,2)$ & 4 & 1 & YES & YES & YES & $1.67$ & $(4,2)$ & NO & 10426\\
$(307,129)$ & 12 & $(7,3)$ & 4 & 1 & YES & YES & YES & $1.75$ & $(4,2)$ & NO & 10427\\
$(307,126)$ & 13 & $(12,5)$ & 5 & 1 & YES & YES & YES & $1.62$ & $(4,2)$ & NO & 10428\\
$(307,129)$ & 12 & $(12,5)$ & 5 & 1 & YES & YES & YES & $1.56$ & $(2,3)$ & 7729 & 10429\\
$(307,119)$ & 12 & $(13,5)$ & 5 & 1 & YES & YES & YES & $1.78$ & $(2,3)$ & NO & 10430\\
$(307,85)$ & 13 & $(47,13)$ & 8 & 1 & YES & YES & YES & $1.57$ & $(4,2)$ & NO & 10431\\
$(307,119)$ & 12 & $(49,19)$ & 8 & 1 & YES & YES & YES & $1.78$ & $(2,3)$ & 8660 & 10432\\
$(307,129)$ & 12 & $(50,21)$ & 8 & 1 & YES & YES & YES & $2.00$ & $(2,3)$ & NO & 10433\\
$(307,65)$ & 13 & $(52,11)$ & 9 & 1 & YES & YES & NO(2) & $1.56$ & $(4,2)$ & NO & 10434\\
$(307,129)$ & 12 & $(69,29)$ & 9 & 1 & YES & YES & YES & $1.67$ & $(4,2)$ & 9409 & 10435\\
$(307,119)$ & 12 & $(80,31)$ & 9 & 1 & YES & YES & YES & $1.71$ & $(2,3)$ & NO & 10436\\
$(307,67)$ & 14 & $(87,19)$ & 10 & 1 & YES & YES & YES & $1.62$ & $(4,2)$ & NO & 10437\\
$(307,129)$ & 12 & $(119,50)$ & 10 & 1 & YES & YES & YES & $1.57$ & $(2,3)$ & NO & 10438\\
$(307,111)$ & 13 & $(177,64)$ & 12 & 1 & YES & YES & YES & $1.75$ & $(4,2)$ & NO & 10439\\
$(307,119)$ & 12 & $(178,69)$ & 11 & 1 & YES & YES & YES & $1.71$ & $(2,3)$ & NO & 10440\\
$(307,119)$ & 12 & $(307,119)$ & 12 & 307 & YES & YES & YES & $1.43$ & $(2,3)$ & NO & 10441\\
$(307,129)$ & 12 & $(307,129)$ & 12 & 307 & YES & YES & YES & $1.57$ & $(2,3)$ & NO & 10442\\
$(308,117)$ & 12 & $(4,1)$ & 3 & 4 & YES & YES & YES & $1.62$ & $(4,2)$ & -- & 10443\\
$(308,129)$ & 12 & $(7,3)$ & 4 & 7 & YES & YES & YES & $1.62$ & $(4,2)$ & NO & 10444\\
$(308,129)$ & 12 & $(8,3)$ & 4 & 4 & YES & YES & YES & $1.78$ & $(4,2)$ & NO & 10445\\
$(308,129)$ & 12 & $(17,7)$ & 6 & 1 & YES & YES & YES & $1.78$ & $(4,2)$ & NO & 10446\\
$(308,117)$ & 12 & $(21,8)$ & 6 & 7 & YES & YES & YES & $1.62$ & $(4,2)$ & NO & 10447\\
$(308,117)$ & 12 & $(129,49)$ & 10 & 1 & YES & YES & YES & $1.62$ & $(4,2)$ & NO & 10448\\
$(308,117)$ & 12 & $(308,117)$ & 12 & 308 & YES & YES & YES & $1.57$ & $(2,3)$ & NO & 10449\\
$(309,83)$ & 13 & $(2,1)$ & 1 & 1 & YES & YES & YES & $1.78$ & $(2,3)$ & -- & 10450\\
$(309,134)$ & 13 & $(2,1)$ & 1 & 1 & YES & YES & YES & $1.75$ & $(4,2)$ & -- & 10451\\
$(309,67)$ & 13 & $(5,2)$ & 3 & 1 & YES & YES & YES & $1.62$ & $(4,2)$ & -- & 10452\\
$(309,134)$ & 13 & $(5,1)$ & 4 & 1 & YES & YES & YES & $1.75$ & $(4,2)$ & NO & 10453\\
$(309,83)$ & 13 & $(11,3)$ & 5 & 1 & YES & YES & YES & $1.62$ & $(2,3)$ & NO & 10454\\
$(309,67)$ & 13 & $(13,3)$ & 6 & 1 & YES & YES & YES & $1.62$ & $(4,2)$ & NO & 10455\\
$(309,83)$ & 13 & $(41,11)$ & 8 & 1 & YES & YES & YES & $1.62$ & $(2,3)$ & NO & 10456\\
$(309,67)$ & 13 & $(60,13)$ & 9 & 3 & YES & YES & NO(2) & $1.56$ & $(2,3)$ & NO & 10457\\
$(310,83)$ & 13 & $(4,1)$ & 3 & 2 & YES & YES & NO(2) & $1.25$ & $(6,1)$ & -- & 10458\\
$(311,65)$ & 14 & $(2,1)$ & 1 & 1 & YES & YES & YES & $1.62$ & $(2,3)$ & NO & 10459\\
$(311,71)$ & 13 & $(2,1)$ & 1 & 1 & YES & YES & YES & $1.38$ & $(2,3)$ & NO & 10460\\
$(311,115)$ & 12 & $(2,1)$ & 1 & 1 & YES & YES & YES & $1.57$ & $(2,3)$ & -- & 10461\\
$(311,119)$ & 12 & $(2,1)$ & 1 & 1 & YES & YES & YES & $1.57$ & $(2,3)$ & NO & 10462\\
$(311,119)$ & 12 & $(3,1)$ & 2 & 1 & YES & YES & YES & $1.86$ & $(2,3)$ & -- & 10463\\
$(311,119)$ & 12 & $(3,1)$ & 2 & 1 & YES & YES & YES & $1.62$ & $(4,2)$ & NO & 10464\\
$(311,120)$ & 13 & $(3,1)$ & 2 & 1 & YES & YES & YES & $2.00$ & $(2,3)$ & NO & 10465\\
$(311,120)$ & 13 & $(3,1)$ & 2 & 1 & YES & YES & YES & $2.00$ & $(2,3)$ & -- & 10466\\
$(311,67)$ & 14 & $(4,1)$ & 3 & 1 & YES & YES & NO(2) & $1.38$ & $(6,1)$ & NO & 10467\\
$(311,119)$ & 12 & $(4,1)$ & 3 & 1 & YES & YES & YES & $1.86$ & $(2,3)$ & -- & 10468\\
$(311,119)$ & 12 & $(4,1)$ & 3 & 1 & YES & YES & YES & $1.43$ & $(2,3)$ & NO & 10469\\
$(311,65)$ & 14 & $(5,1)$ & 4 & 1 & YES & YES & NO(2) & $1.67$ & $(8,0)$ & NO & 10470\\
$(311,71)$ & 13 & $(5,2)$ & 3 & 1 & YES & YES & YES & $1.62$ & $(2,3)$ & -- & 10471\\
$(311,71)$ & 13 & $(5,2)$ & 3 & 1 & YES & YES & YES & $1.71$ & $(2,3)$ & NO & 10472\\
$(311,115)$ & 12 & $(5,1)$ & 4 & 1 & YES & YES & YES & $1.57$ & $(2,3)$ & NO & 10473\\
$(311,119)$ & 12 & $(5,1)$ & 4 & 1 & YES & YES & YES & $1.78$ & $(2,3)$ & NO & 10474\\
$(311,135)$ & 13 & $(5,2)$ & 3 & 1 & YES & YES & YES & $1.43$ & $(6,1)$ & NO & 10475\\
$(311,119)$ & 12 & $(8,3)$ & 4 & 1 & YES & YES & YES & $1.78$ & $(2,3)$ & NO & 10476\\
$(311,71)$ & 13 & $(17,4)$ & 7 & 1 & YES & YES & YES & $1.57$ & $(2,3)$ & NO & 10477\\
$(311,115)$ & 12 & $(19,7)$ & 6 & 1 & YES & YES & YES & $1.43$ & $(2,3)$ & NO & 10478\\
$(311,71)$ & 13 & $(31,7)$ & 8 & 1 & YES & YES & YES & $1.62$ & $(2,3)$ & NO & 10479\\
$(311,120)$ & 13 & $(31,12)$ & 7 & 1 & YES & YES & YES & $1.88$ & $(4,2)$ & NO & 10480\\
$(311,119)$ & 12 & $(47,18)$ & 8 & 1 & YES & YES & YES & $1.86$ & $(2,3)$ & 10838 & 10481\\
$(311,71)$ & 13 & $(48,11)$ & 9 & 1 & YES & YES & YES & $1.50$ & $(4,2)$ & NO & 10482\\
$(311,71)$ & 13 & $(57,13)$ & 9 & 1 & YES & YES & YES & $1.50$ & $(2,3)$ & NO & 10483\\
$(311,115)$ & 12 & $(73,27)$ & 9 & 1 & YES & YES & YES & $1.71$ & $(2,3)$ & 9587 & 10484\\
$(311,71)$ & 13 & $(79,18)$ & 10 & 1 & YES & YES & YES & $1.43$ & $(2,3)$ & NO & 10485\\
$(311,119)$ & 12 & $(81,31)$ & 9 & 1 & YES & YES & YES & $1.50$ & $(2,3)$ & 9838 & 10486\\
$(311,115)$ & 12 & $(119,44)$ & 10 & 1 & YES & YES & YES & $1.57$ & $(2,3)$ & NO & 10487\\
$(312,131)$ & 12 & $(19,8)$ & 6 & 1 & YES & YES & YES & $1.57$ & $(2,3)$ & NO & 10488\\
$(313,86)$ & 13 & $(2,1)$ & 1 & 1 & YES & YES & YES & $1.67$ & $(2,3)$ & -- & 10489\\
$(313,86)$ & 13 & $(2,1)$ & 1 & 1 & YES & YES & YES & $1.78$ & $(2,3)$ & NO & 10490\\
$(313,119)$ & 12 & $(2,1)$ & 1 & 1 & YES & YES & YES & $1.62$ & $(2,3)$ & NO & 10491\\
$(313,121)$ & 12 & $(2,1)$ & 1 & 1 & YES & YES & YES & $1.78$ & $(2,3)$ & -- & 10492\\
$(313,121)$ & 12 & $(2,1)$ & 1 & 1 & YES & YES & YES & $1.62$ & $(4,2)$ & NO & 10493\\
$(313,132)$ & 13 & $(2,1)$ & 1 & 1 & NO & YES & NO(2) & $1.67$ & $(8,0)$ & -- & 10494\\
$(313,86)$ & 13 & $(3,1)$ & 2 & 1 & YES & YES & YES & $1.43$ & $(4,2)$ & -- & 10495\\
$(313,86)$ & 13 & $(3,1)$ & 2 & 1 & YES & YES & YES & $1.50$ & $(4,2)$ & NO & 10496\\
$(313,86)$ & 13 & $(3,1)$ & 2 & 1 & YES & YES & YES & $1.62$ & $(4,2)$ & NO & 10497\\
$(313,91)$ & 13 & $(3,1)$ & 2 & 1 & YES & YES & YES & $1.75$ & $(2,3)$ & NO & 10498\\
$(313,91)$ & 13 & $(3,1)$ & 2 & 1 & YES & YES & YES & $1.67$ & $(2,3)$ & -- & 10499\\
$(313,91)$ & 13 & $(3,1)$ & 2 & 1 & YES & YES & YES & $1.86$ & $(2,3)$ & NO & 10500\\
$(313,93)$ & 13 & $(3,1)$ & 2 & 1 & YES & YES & YES & $1.56$ & $(4,2)$ & -- & 10501\\
$(313,119)$ & 12 & $(3,1)$ & 2 & 1 & YES & YES & YES & $1.50$ & $(4,2)$ & -- & 10502\\
$(313,121)$ & 12 & $(3,1)$ & 2 & 1 & YES & YES & YES & $1.62$ & $(2,3)$ & -- & 10503\\
$(313,121)$ & 12 & $(3,1)$ & 2 & 1 & YES & YES & YES & $1.50$ & $(4,2)$ & NO & 10504\\
$(313,121)$ & 12 & $(3,1)$ & 2 & 1 & YES & YES & YES & $1.71$ & $(2,3)$ & NO & 10505\\
$(313,86)$ & 13 & $(4,1)$ & 3 & 1 & YES & YES & YES & $1.75$ & $(2,3)$ & -- & 10506\\
$(313,93)$ & 13 & $(4,1)$ & 3 & 1 & YES & YES & YES & $1.50$ & $(4,2)$ & -- & 10507\\
$(313,86)$ & 13 & $(5,1)$ & 4 & 1 & YES & YES & YES & $1.67$ & $(2,3)$ & NO & 10508\\
$(313,93)$ & 13 & $(5,1)$ & 4 & 1 & YES & YES & YES & $1.62$ & $(4,2)$ & NO & 10509\\
$(313,93)$ & 13 & $(5,1)$ & 4 & 1 & YES & YES & YES & $1.62$ & $(4,2)$ & -- & 10510\\
$(313,119)$ & 12 & $(5,1)$ & 4 & 1 & YES & YES & YES & $1.62$ & $(4,2)$ & NO & 10511\\
$(313,119)$ & 12 & $(5,2)$ & 3 & 1 & YES & YES & YES & $1.75$ & $(4,2)$ & NO & 10512\\
$(313,121)$ & 12 & $(5,2)$ & 3 & 1 & YES & YES & YES & $1.50$ & $(4,2)$ & NO & 10513\\
$(313,86)$ & 13 & $(7,2)$ & 4 & 1 & YES & YES & YES & $1.89$ & $(2,3)$ & NO & 10514\\
$(313,93)$ & 13 & $(7,2)$ & 4 & 1 & YES & YES & YES & $1.62$ & $(4,2)$ & NO & 10515\\
$(313,86)$ & 13 & $(18,5)$ & 6 & 1 & YES & YES & YES & $1.43$ & $(4,2)$ & NO & 10516\\
$(313,121)$ & 12 & $(18,7)$ & 6 & 1 & YES & YES & YES & $1.71$ & $(2,3)$ & 10862 & 10517\\
$(313,91)$ & 13 & $(24,7)$ & 7 & 1 & YES & YES & YES & $1.78$ & $(2,3)$ & NO & 10518\\
$(313,86)$ & 13 & $(29,8)$ & 7 & 1 & YES & YES & YES & $1.75$ & $(2,3)$ & 9157 & 10519\\
$(313,121)$ & 12 & $(31,12)$ & 7 & 1 & YES & YES & YES & $1.88$ & $(2,3)$ & 10041 & 10520\\
$(313,121)$ & 12 & $(44,17)$ & 8 & 1 & YES & YES & YES & $1.62$ & $(2,3)$ & NO & 10521\\
$(313,86)$ & 13 & $(51,14)$ & 9 & 1 & YES & YES & YES & $1.62$ & $(4,2)$ & NO & 10522\\
$(313,91)$ & 13 & $(55,16)$ & 9 & 1 & YES & YES & YES & $1.67$ & $(2,3)$ & NO & 10523\\
$(313,119)$ & 12 & $(71,27)$ & 9 & 1 & YES & YES & YES & $1.62$ & $(4,2)$ & 9533 & 10524\\
$(313,121)$ & 12 & $(75,29)$ & 9 & 1 & YES & YES & YES & $1.80$ & $(2,3)$ & 9655 & 10525\\
$(313,91)$ & 13 & $(86,25)$ & 10 & 1 & YES & YES & YES & $1.62$ & $(2,3)$ & NO & 10526\\
$(313,121)$ & 12 & $(119,46)$ & 10 & 1 & YES & YES & YES & $1.78$ & $(2,3)$ & NO & 10527\\
$(313,119)$ & 12 & $(121,46)$ & 10 & 1 & YES & YES & YES & $1.70$ & $(2,3)$ & NO & 10528\\
$(313,86)$ & 13 & $(313,86)$ & 13 & 313 & YES & YES & YES & $1.50$ & $(4,2)$ & NO & 10529\\
$(313,121)$ & 12 & $(313,121)$ & 12 & 313 & YES & YES & YES & $1.57$ & $(2,3)$ & NO & 10530\\
$(314,95)$ & 13 & $(2,1)$ & 1 & 2 & YES & YES & YES & $1.67$ & $(2,3)$ & -- & 10531\\
$(314,129)$ & 13 & $(3,1)$ & 2 & 1 & YES & YES & YES & $1.62$ & $(4,2)$ & -- & 10532\\
$(314,87)$ & 13 & $(15,4)$ & 6 & 1 & YES & YES & YES & $1.67$ & $(4,2)$ & NO & 10533\\
$(314,87)$ & 13 & $(65,18)$ & 9 & 1 & YES & YES & YES & $1.62$ & $(4,2)$ & NO & 10534\\
$(314,95)$ & 13 & $(76,23)$ & 10 & 2 & YES & YES & YES & $1.50$ & $(2,3)$ & 9693 & 10535\\
$(314,95)$ & 13 & $(119,36)$ & 11 & 1 & YES & YES & YES & $1.75$ & $(2,3)$ & NO & 10536\\
$(315,68)$ & 13 & $(2,1)$ & 1 & 1 & YES & YES & NO(2) & $1.56$ & $(2,3)$ & NO & 10537\\
$(315,71)$ & 14 & $(2,1)$ & 1 & 1 & YES & YES & NO(2) & $1.38$ & $(4,2)$ & -- & 10538\\
$(315,88)$ & 13 & $(2,1)$ & 1 & 1 & YES & YES & YES & $1.56$ & $(2,3)$ & NO & 10539\\
$(315,88)$ & 13 & $(2,1)$ & 1 & 1 & YES & YES & YES & $1.56$ & $(2,3)$ & -- & 10540\\
$(315,71)$ & 14 & $(3,1)$ & 2 & 3 & YES & YES & YES & $1.57$ & $(4,2)$ & NO & 10541\\
$(315,88)$ & 13 & $(3,1)$ & 2 & 3 & YES & YES & YES & $1.62$ & $(4,2)$ & -- & 10542\\
$(315,113)$ & 13 & $(3,1)$ & 2 & 3 & YES & YES & YES & $1.57$ & $(6,1)$ & -- & 10543\\
$(315,88)$ & 13 & $(5,1)$ & 4 & 5 & YES & YES & YES & $1.62$ & $(2,3)$ & NO & 10544\\
$(315,92)$ & 13 & $(5,1)$ & 4 & 5 & YES & YES & YES & $1.50$ & $(4,2)$ & NO & 10545\\
$(315,92)$ & 13 & $(5,1)$ & 4 & 5 & YES & YES & YES & $1.50$ & $(4,2)$ & -- & 10546\\
$(315,88)$ & 13 & $(7,2)$ & 4 & 7 & YES & YES & YES & $1.75$ & $(2,3)$ & NO & 10547\\
$(315,68)$ & 13 & $(14,3)$ & 6 & 7 & YES & YES & YES & $1.43$ & $(2,3)$ & NO & 10548\\
$(315,92)$ & 13 & $(17,5)$ & 6 & 1 & YES & YES & YES & $1.50$ & $(4,2)$ & NO & 10549\\
$(315,88)$ & 13 & $(18,5)$ & 6 & 9 & YES & YES & YES & $1.56$ & $(2,3)$ & NO & 10550\\
$(315,113)$ & 13 & $(25,9)$ & 7 & 5 & YES & YES & YES & $1.57$ & $(6,1)$ & 8754 & 10551\\
$(315,88)$ & 13 & $(43,12)$ & 8 & 1 & YES & YES & YES & $1.56$ & $(2,3)$ & NO & 10552\\
$(315,92)$ & 13 & $(65,19)$ & 9 & 5 & YES & YES & YES & $1.62$ & $(4,2)$ & 11400 & 10553\\
$(315,92)$ & 13 & $(113,33)$ & 11 & 1 & YES & YES & YES & $1.50$ & $(4,2)$ & NO & 10554\\
$(315,88)$ & 13 & $(247,69)$ & 12 & 1 & YES & YES & YES & $1.62$ & $(2,3)$ & NO & 10555\\
$(316,69)$ & 13 & $(3,1)$ & 2 & 1 & YES & YES & YES & $1.50$ & $(2,3)$ & NO & 10556\\
$(316,69)$ & 13 & $(3,1)$ & 2 & 1 & YES & YES & YES & $1.50$ & $(2,3)$ & -- & 10557\\
$(316,69)$ & 13 & $(14,3)$ & 6 & 2 & YES & YES & YES & $1.38$ & $(2,3)$ & NO & 10558\\
$(316,87)$ & 13 & $(29,8)$ & 7 & 1 & YES & YES & YES & $1.67$ & $(4,2)$ & NO & 10559\\
$(317,121)$ & 12 & $(2,1)$ & 1 & 1 & YES & YES & YES & $1.75$ & $(2,3)$ & -- & 10560\\
$(317,121)$ & 12 & $(2,1)$ & 1 & 1 & YES & YES & YES & $1.90$ & $(2,3)$ & NO & 10561\\
$(317,131)$ & 12 & $(2,1)$ & 1 & 1 & YES & YES & YES & $1.78$ & $(2,3)$ & -- & 10562\\
$(317,117)$ & 13 & $(3,1)$ & 2 & 1 & YES & YES & YES & $1.62$ & $(4,2)$ & -- & 10563\\
$(317,121)$ & 12 & $(3,1)$ & 2 & 1 & YES & YES & YES & $1.71$ & $(2,3)$ & -- & 10564\\
$(317,131)$ & 12 & $(3,1)$ & 2 & 1 & YES & YES & YES & $1.56$ & $(4,2)$ & -- & 10565\\
$(317,85)$ & 13 & $(4,1)$ & 3 & 1 & YES & YES & NO(2) & $1.60$ & $(10,-1)$ & NO & 10566\\
$(317,121)$ & 12 & $(4,1)$ & 3 & 1 & YES & YES & YES & $1.71$ & $(2,3)$ & NO & 10567\\
$(317,131)$ & 12 & $(4,1)$ & 3 & 1 & YES & YES & YES & $1.62$ & $(2,3)$ & -- & 10568\\
$(317,121)$ & 12 & $(5,1)$ & 4 & 1 & YES & YES & YES & $1.67$ & $(2,3)$ & -- & 10569\\
$(317,121)$ & 12 & $(5,1)$ & 4 & 1 & YES & YES & YES & $1.57$ & $(2,3)$ & NO & 10570\\
$(317,121)$ & 12 & $(5,2)$ & 3 & 1 & YES & YES & YES & $1.57$ & $(4,2)$ & NO & 10571\\
$(317,131)$ & 12 & $(5,1)$ & 4 & 1 & YES & YES & YES & $1.57$ & $(2,3)$ & NO & 10572\\
$(317,138)$ & 13 & $(6,1)$ & 5 & 1 & YES & YES & YES & $1.75$ & $(4,2)$ & NO & 10573\\
$(317,138)$ & 13 & $(6,1)$ & 5 & 1 & YES & YES & YES & $1.75$ & $(4,2)$ & -- & 10574\\
$(317,121)$ & 12 & $(8,3)$ & 4 & 1 & YES & YES & YES & $1.50$ & $(2,3)$ & NO & 10575\\
$(317,121)$ & 12 & $(13,5)$ & 5 & 1 & YES & YES & YES & $1.75$ & $(2,3)$ & NO & 10576\\
$(317,131)$ & 12 & $(17,7)$ & 6 & 1 & YES & YES & YES & $1.78$ & $(2,3)$ & NO & 10577\\
$(317,121)$ & 12 & $(21,8)$ & 6 & 1 & YES & YES & YES & $1.89$ & $(2,3)$ & 7656 & 10578\\
$(317,117)$ & 13 & $(27,10)$ & 7 & 1 & YES & YES & YES & $1.62$ & $(4,2)$ & NO & 10579\\
$(317,131)$ & 12 & $(29,12)$ & 7 & 1 & YES & YES & YES & $1.67$ & $(4,2)$ & 9959 & 10580\\
$(317,121)$ & 12 & $(34,13)$ & 7 & 1 & YES & YES & YES & $1.43$ & $(2,3)$ & NO & 10581\\
$(317,131)$ & 12 & $(75,31)$ & 9 & 1 & YES & YES & YES & $1.62$ & $(2,3)$ & 9694 & 10582\\
$(317,121)$ & 12 & $(76,29)$ & 9 & 1 & YES & YES & YES & $1.71$ & $(2,3)$ & NO & 10583\\
$(317,131)$ & 12 & $(121,50)$ & 10 & 1 & YES & YES & YES & $1.57$ & $(2,3)$ & NO & 10584\\
$(317,121)$ & 12 & $(131,50)$ & 10 & 1 & YES & YES & YES & $1.62$ & $(2,3)$ & NO & 10585\\
$(317,121)$ & 12 & $(186,71)$ & 11 & 1 & YES & YES & YES & $1.71$ & $(2,3)$ & NO & 10586\\
$(317,131)$ & 12 & $(196,81)$ & 11 & 1 & YES & YES & YES & $1.67$ & $(4,2)$ & NO & 10587\\
$(317,117)$ & 13 & $(317,117)$ & 13 & 317 & YES & YES & YES & $1.62$ & $(4,2)$ & NO & 10588\\
$(317,121)$ & 12 & $(317,121)$ & 12 & 317 & YES & YES & YES & $1.62$ & $(2,3)$ & NO & 10589\\
$(317,131)$ & 12 & $(317,131)$ & 12 & 317 & YES & YES & YES & $1.50$ & $(2,3)$ & NO & 10590\\
$(317,138)$ & 13 & $(317,138)$ & 13 & 317 & YES & YES & YES & $1.75$ & $(4,2)$ & NO & 10591\\
$(318,73)$ & 15 & $(2,1)$ & 1 & 2 & YES & YES & NO(2) & $1.56$ & $(8,0)$ & -- & 10592\\
$(318,73)$ & 15 & $(5,1)$ & 4 & 1 & YES & YES & NO(2) & $1.50$ & $(6,1)$ & NO & 10593\\
$(319,86)$ & 13 & $(2,1)$ & 1 & 1 & YES & YES & NO(2) & $1.38$ & $(4,2)$ & -- & 10594\\
$(319,73)$ & 13 & $(5,2)$ & 3 & 1 & YES & YES & YES & $1.50$ & $(2,3)$ & NO & 10595\\
$(319,73)$ & 13 & $(5,2)$ & 3 & 1 & YES & YES & YES & $1.62$ & $(2,3)$ & -- & 10596\\
$(319,118)$ & 13 & $(5,1)$ & 4 & 1 & YES & YES & YES & $1.88$ & $(4,2)$ & -- & 10597\\
$(319,73)$ & 13 & $(7,2)$ & 4 & 1 & YES & YES & YES & $1.50$ & $(4,2)$ & NO & 10598\\
$(319,115)$ & 13 & $(11,4)$ & 5 & 11 & YES & YES & YES & $1.90$ & $(2,3)$ & NO & 10599\\
$(319,73)$ & 13 & $(22,5)$ & 7 & 11 & YES & YES & YES & $1.50$ & $(2,3)$ & NO & 10600\\
$(319,118)$ & 13 & $(246,91)$ & 12 & 1 & YES & YES & YES & $1.75$ & $(4,2)$ & NO & 10601\\
$(320,93)$ & 13 & $(3,1)$ & 2 & 1 & YES & YES & YES & $1.62$ & $(4,2)$ & -- & 10602\\
$(320,69)$ & 15 & $(51,11)$ & 9 & 1 & YES & YES & NO(2) & $1.50$ & $(6,1)$ & NO & 10603\\
$(321,94)$ & 13 & $(2,1)$ & 1 & 1 & YES & YES & YES & $1.89$ & $(2,3)$ & NO & 10604\\
$(321,95)$ & 13 & $(2,1)$ & 1 & 1 & YES & YES & YES & $1.57$ & $(4,2)$ & -- & 10605\\
$(321,94)$ & 13 & $(3,1)$ & 2 & 3 & YES & YES & YES & $1.67$ & $(2,3)$ & -- & 10606\\
$(321,94)$ & 13 & $(3,1)$ & 2 & 3 & YES & YES & YES & $1.57$ & $(2,3)$ & NO & 10607\\
$(321,95)$ & 13 & $(3,1)$ & 2 & 3 & YES & YES & YES & $1.89$ & $(2,3)$ & NO & 10608\\
$(321,94)$ & 13 & $(4,1)$ & 3 & 1 & YES & YES & YES & $1.43$ & $(2,3)$ & -- & 10609\\
$(321,95)$ & 13 & $(4,1)$ & 3 & 1 & YES & YES & YES & $1.56$ & $(2,3)$ & NO & 10610\\
$(321,116)$ & 13 & $(5,1)$ & 4 & 1 & YES & YES & YES & $1.90$ & $(2,3)$ & NO & 10611\\
$(321,95)$ & 13 & $(17,5)$ & 6 & 1 & YES & YES & YES & $1.67$ & $(2,3)$ & 6110 & 10612\\
$(321,62)$ & 15 & $(21,4)$ & 8 & 3 & YES & YES & NO(2) & $1.50$ & $(6,1)$ & NO & 10613\\
$(321,94)$ & 13 & $(24,7)$ & 7 & 3 & YES & YES & YES & $1.67$ & $(4,2)$ & NO & 10614\\
$(321,94)$ & 13 & $(58,17)$ & 9 & 1 & YES & YES & YES & $1.67$ & $(2,3)$ & 7563 & 10615\\
$(321,95)$ & 13 & $(71,21)$ & 9 & 1 & YES & YES & YES & $1.57$ & $(2,3)$ & NO & 10616\\
$(321,94)$ & 13 & $(99,29)$ & 10 & 3 & YES & YES & YES & $1.62$ & $(2,3)$ & NO & 10617\\
$(321,98)$ & 13 & $(131,40)$ & 11 & 1 & YES & YES & YES & $1.78$ & $(4,2)$ & 11028 & 10618\\
$(321,94)$ & 13 & $(140,41)$ & 11 & 1 & YES & YES & YES & $1.50$ & $(4,2)$ & NO & 10619\\
$(322,89)$ & 12 & $(2,1)$ & 1 & 2 & YES & YES & YES & $1.50$ & $(2,3)$ & -- & 10620\\
$(322,89)$ & 12 & $(2,1)$ & 1 & 2 & YES & YES & YES & $1.50$ & $(2,3)$ & NO & 10621\\
$(322,123)$ & 12 & $(2,1)$ & 1 & 2 & YES & YES & YES & $1.43$ & $(2,3)$ & NO & 10622\\
$(322,123)$ & 12 & $(2,1)$ & 1 & 2 & YES & YES & YES & $1.43$ & $(2,3)$ & -- & 10623\\
$(322,73)$ & 14 & $(3,1)$ & 2 & 1 & YES & YES & YES & $1.57$ & $(4,2)$ & -- & 10624\\
$(322,89)$ & 12 & $(3,1)$ & 2 & 1 & YES & YES & YES & $1.50$ & $(2,3)$ & NO & 10625\\
$(322,89)$ & 12 & $(3,1)$ & 2 & 1 & YES & YES & YES & $1.50$ & $(2,3)$ & -- & 10626\\
$(322,73)$ & 14 & $(4,1)$ & 3 & 2 & YES & YES & YES & $1.43$ & $(4,2)$ & -- & 10627\\
$(322,89)$ & 12 & $(4,1)$ & 3 & 2 & YES & YES & YES & $1.29$ & $(6,1)$ & 9948 & 10628\\
$(322,89)$ & 12 & $(7,2)$ & 4 & 7 & YES & YES & YES & $1.50$ & $(2,3)$ & NO & 10629\\
$(322,123)$ & 12 & $(8,3)$ & 4 & 2 & YES & YES & YES & $1.57$ & $(2,3)$ & NO & 10630\\
$(322,89)$ & 12 & $(18,5)$ & 6 & 2 & YES & YES & YES & $1.62$ & $(2,3)$ & NO & 10631\\
$(322,89)$ & 12 & $(29,8)$ & 7 & 1 & YES & YES & YES & $1.38$ & $(2,3)$ & 10005 & 10632\\
$(322,73)$ & 14 & $(31,7)$ & 8 & 1 & YES & YES & YES & $1.29$ & $(4,2)$ & NO & 10633\\
$(322,123)$ & 12 & $(34,13)$ & 7 & 2 & YES & YES & YES & $1.57$ & $(2,3)$ & NO & 10634\\
$(322,89)$ & 12 & $(47,13)$ & 8 & 1 & YES & YES & YES & $1.50$ & $(2,3)$ & NO & 10635\\
$(323,89)$ & 13 & $(2,1)$ & 1 & 1 & YES & YES & YES & $1.78$ & $(2,3)$ & NO & 10636\\
$(323,94)$ & 13 & $(2,1)$ & 1 & 1 & YES & YES & YES & $1.78$ & $(2,3)$ & -- & 10637\\
$(323,98)$ & 13 & $(2,1)$ & 1 & 1 & YES & YES & YES & $1.57$ & $(4,2)$ & NO & 10638\\
$(323,98)$ & 13 & $(2,1)$ & 1 & 1 & YES & YES & YES & $1.75$ & $(4,2)$ & -- & 10639\\
$(323,60)$ & 14 & $(3,1)$ & 2 & 1 & YES & YES & YES & $1.43$ & $(6,1)$ & NO & 10640\\
$(323,60)$ & 14 & $(3,1)$ & 2 & 1 & YES & YES & YES & $1.43$ & $(6,1)$ & -- & 10641\\
$(323,94)$ & 13 & $(3,1)$ & 2 & 1 & YES & YES & YES & $1.67$ & $(2,3)$ & -- & 10642\\
$(323,94)$ & 13 & $(3,1)$ & 2 & 1 & YES & YES & YES & $1.67$ & $(2,3)$ & NO & 10643\\
$(323,98)$ & 13 & $(3,1)$ & 2 & 1 & YES & YES & YES & $1.43$ & $(4,2)$ & -- & 10644\\
$(323,98)$ & 13 & $(3,1)$ & 2 & 1 & YES & YES & YES & $1.86$ & $(2,3)$ & NO & 10645\\
$(323,134)$ & 13 & $(3,1)$ & 2 & 1 & YES & YES & YES & $1.75$ & $(4,2)$ & -- & 10646\\
$(323,60)$ & 14 & $(4,1)$ & 3 & 1 & YES & YES & YES & $1.62$ & $(4,2)$ & -- & 10647\\
$(323,89)$ & 13 & $(4,1)$ & 3 & 1 & YES & YES & YES & $1.57$ & $(2,3)$ & -- & 10648\\
$(323,98)$ & 13 & $(4,1)$ & 3 & 1 & YES & YES & YES & $1.62$ & $(2,3)$ & -- & 10649\\
$(323,126)$ & 13 & $(4,1)$ & 3 & 1 & YES & YES & YES & $1.88$ & $(4,2)$ & NO & 10650\\
$(323,89)$ & 13 & $(5,1)$ & 4 & 1 & YES & YES & YES & $1.78$ & $(2,3)$ & NO & 10651\\
$(323,98)$ & 13 & $(5,1)$ & 4 & 1 & YES & YES & YES & $1.86$ & $(2,3)$ & NO & 10652\\
$(323,89)$ & 13 & $(7,2)$ & 4 & 1 & YES & YES & YES & $1.78$ & $(2,3)$ & NO & 10653\\
$(323,126)$ & 13 & $(8,3)$ & 4 & 1 & YES & YES & YES & $1.75$ & $(4,2)$ & NO & 10654\\
$(323,94)$ & 13 & $(10,3)$ & 5 & 1 & YES & YES & YES & $1.67$ & $(2,3)$ & NO & 10655\\
$(323,98)$ & 13 & $(10,3)$ & 5 & 1 & YES & YES & YES & $1.89$ & $(2,3)$ & NO & 10656\\
$(323,126)$ & 13 & $(13,5)$ & 5 & 1 & YES & YES & YES & $1.75$ & $(4,2)$ & NO & 10657\\
$(323,70)$ & 14 & $(14,3)$ & 6 & 1 & YES & YES & YES & $1.57$ & $(4,2)$ & NO & 10658\\
$(323,94)$ & 13 & $(17,5)$ & 6 & 17 & YES & YES & YES & $1.70$ & $(2,3)$ & NO & 10659\\
$(323,98)$ & 13 & $(23,7)$ & 7 & 1 & YES & YES & YES & $1.43$ & $(4,2)$ & NO & 10660\\
$(323,89)$ & 13 & $(29,8)$ & 7 & 1 & YES & YES & YES & $1.57$ & $(4,2)$ & NO & 10661\\
$(323,94)$ & 13 & $(31,9)$ & 8 & 1 & YES & YES & YES & $1.67$ & $(4,2)$ & NO & 10662\\
$(323,98)$ & 13 & $(33,10)$ & 8 & 1 & YES & YES & YES & $1.62$ & $(2,3)$ & NO & 10663\\
$(323,70)$ & 14 & $(37,8)$ & 8 & 1 & YES & YES & YES & $1.57$ & $(4,2)$ & NO & 10664\\
$(323,98)$ & 13 & $(56,17)$ & 9 & 1 & YES & YES & YES & $1.57$ & $(4,2)$ & NO & 10665\\
$(323,89)$ & 13 & $(69,19)$ & 9 & 1 & YES & YES & YES & $1.57$ & $(2,3)$ & NO & 10666\\
$(323,98)$ & 13 & $(89,27)$ & 10 & 1 & YES & YES & YES & $1.89$ & $(2,3)$ & NO & 10667\\
$(323,134)$ & 13 & $(135,56)$ & 11 & 1 & YES & YES & YES & $1.75$ & $(4,2)$ & 11095 & 10668\\
$(323,98)$ & 13 & $(145,44)$ & 11 & 1 & YES & YES & YES & $1.71$ & $(2,3)$ & 11246 & 10669\\
$(323,94)$ & 13 & $(189,55)$ & 12 & 1 & YES & YES & YES & $1.67$ & $(2,3)$ & NO & 10670\\
$(323,98)$ & 13 & $(234,71)$ & 12 & 1 & YES & YES & YES & $1.75$ & $(2,3)$ & NO & 10671\\
$(323,94)$ & 13 & $(244,71)$ & 13 & 1 & YES & YES & YES & $1.62$ & $(4,2)$ & NO & 10672\\
$(324,89)$ & 13 & $(2,1)$ & 1 & 2 & YES & YES & YES & $1.62$ & $(4,2)$ & NO & 10673\\
$(324,89)$ & 13 & $(2,1)$ & 1 & 2 & YES & YES & YES & $1.62$ & $(4,2)$ & -- & 10674\\
$(324,91)$ & 13 & $(2,1)$ & 1 & 2 & YES & YES & YES & $1.88$ & $(2,3)$ & NO & 10675\\
$(324,95)$ & 13 & $(2,1)$ & 1 & 2 & YES & YES & YES & $1.43$ & $(4,2)$ & -- & 10676\\
$(324,127)$ & 13 & $(2,1)$ & 1 & 2 & NO & YES & NO(2) & $1.78$ & $(4,2)$ & -- & 10677\\
$(324,73)$ & 14 & $(3,1)$ & 2 & 3 & YES & YES & YES & $1.50$ & $(2,3)$ & NO & 10678\\
$(324,73)$ & 14 & $(3,1)$ & 2 & 3 & YES & YES & YES & $1.89$ & $(2,3)$ & -- & 10679\\
$(324,95)$ & 13 & $(3,1)$ & 2 & 3 & YES & YES & YES & $1.67$ & $(2,3)$ & -- & 10680\\
$(324,95)$ & 13 & $(3,1)$ & 2 & 3 & YES & YES & YES & $1.71$ & $(4,2)$ & NO & 10681\\
$(324,89)$ & 13 & $(5,2)$ & 3 & 1 & YES & YES & YES & $1.78$ & $(4,2)$ & NO & 10682\\
$(324,89)$ & 13 & $(5,2)$ & 3 & 1 & YES & YES & YES & $1.78$ & $(4,2)$ & -- & 10683\\
$(324,89)$ & 13 & $(10,3)$ & 5 & 2 & YES & YES & YES & $1.78$ & $(4,2)$ & NO & 10684\\
$(324,95)$ & 13 & $(10,3)$ & 5 & 2 & YES & YES & YES & $1.43$ & $(4,2)$ & NO & 10685\\
$(324,95)$ & 13 & $(17,5)$ & 6 & 1 & YES & YES & YES & $1.57$ & $(4,2)$ & 7797 & 10686\\
$(324,95)$ & 13 & $(24,7)$ & 7 & 12 & YES & YES & YES & $1.75$ & $(2,3)$ & NO & 10687\\
$(324,95)$ & 13 & $(41,12)$ & 8 & 1 & YES & YES & YES & $1.75$ & $(2,3)$ & NO & 10688\\
$(324,89)$ & 13 & $(51,14)$ & 9 & 3 & YES & YES & YES & $1.62$ & $(4,2)$ & NO & 10689\\
$(324,133)$ & 13 & $(56,23)$ & 9 & 4 & YES & YES & YES & $1.62$ & $(4,2)$ & NO & 10690\\
$(324,95)$ & 13 & $(75,22)$ & 10 & 3 & YES & YES & YES & $1.78$ & $(2,3)$ & NO & 10691\\
$(324,91)$ & 13 & $(89,25)$ & 10 & 1 & YES & YES & YES & $1.75$ & $(2,3)$ & NO & 10692\\
$(324,89)$ & 13 & $(131,36)$ & 11 & 1 & YES & YES & YES & $1.67$ & $(4,2)$ & NO & 10693\\
$(324,133)$ & 13 & $(229,94)$ & 12 & 1 & YES & YES & YES & $1.62$ & $(4,2)$ & NO & 10694\\
$(325,74)$ & 14 & $(2,1)$ & 1 & 1 & YES & YES & YES & $1.57$ & $(2,3)$ & -- & 10695\\
$(325,141)$ & 13 & $(2,1)$ & 1 & 1 & YES & YES & YES & $1.75$ & $(4,2)$ & -- & 10696\\
$(325,76)$ & 13 & $(3,1)$ & 2 & 1 & YES & YES & YES & $1.43$ & $(6,1)$ & NO & 10697\\
$(325,141)$ & 13 & $(3,1)$ & 2 & 1 & YES & YES & YES & $1.57$ & $(6,1)$ & NO & 10698\\
$(325,141)$ & 13 & $(3,1)$ & 2 & 1 & YES & YES & YES & $1.57$ & $(6,1)$ & -- & 10699\\
$(325,141)$ & 13 & $(23,10)$ & 7 & 1 & YES & YES & YES & $1.75$ & $(4,2)$ & NO & 10700\\
$(325,76)$ & 13 & $(43,10)$ & 9 & 1 & YES & YES & YES & $1.43$ & $(4,2)$ & NO & 10701\\
$(325,76)$ & 13 & $(64,15)$ & 10 & 1 & YES & YES & YES & $1.62$ & $(4,2)$ & NO & 10702\\
$(326,99)$ & 13 & $(2,1)$ & 1 & 2 & YES & YES & NO(2) & $1.38$ & $(4,2)$ & -- & 10703\\
$(326,99)$ & 13 & $(2,1)$ & 1 & 2 & YES & YES & YES & $1.80$ & $(2,3)$ & NO & 10704\\
$(326,77)$ & 14 & $(3,1)$ & 2 & 1 & YES & YES & YES & $1.43$ & $(2,3)$ & NO & 10705\\
$(326,97)$ & 13 & $(3,1)$ & 2 & 1 & YES & YES & YES & $1.43$ & $(4,2)$ & -- & 10706\\
$(326,99)$ & 13 & $(3,1)$ & 2 & 1 & YES & YES & YES & $1.43$ & $(4,2)$ & -- & 10707\\
$(326,125)$ & 13 & $(3,1)$ & 2 & 1 & YES & YES & YES & $1.78$ & $(4,2)$ & NO & 10708\\
$(326,135)$ & 13 & $(3,1)$ & 2 & 1 & YES & YES & YES & $1.43$ & $(6,1)$ & -- & 10709\\
$(326,97)$ & 13 & $(4,1)$ & 3 & 2 & YES & YES & YES & $1.43$ & $(4,2)$ & NO & 10710\\
$(326,77)$ & 14 & $(5,2)$ & 3 & 1 & YES & YES & YES & $1.62$ & $(4,2)$ & -- & 10711\\
$(326,99)$ & 13 & $(33,10)$ & 8 & 1 & YES & YES & YES & $1.80$ & $(2,3)$ & NO & 10712\\
$(326,99)$ & 13 & $(79,24)$ & 10 & 1 & YES & YES & YES & $1.43$ & $(4,2)$ & NO & 10713\\
$(326,77)$ & 14 & $(93,22)$ & 11 & 1 & YES & YES & YES & $1.62$ & $(4,2)$ & NO & 10714\\
$(326,97)$ & 13 & $(121,36)$ & 11 & 1 & YES & YES & YES & $1.67$ & $(2,3)$ & NO & 10715\\
$(326,135)$ & 13 & $(128,53)$ & 11 & 2 & YES & YES & YES & $1.43$ & $(6,1)$ & 11020 & 10716\\
$(326,97)$ & 13 & $(326,97)$ & 13 & 326 & YES & YES & YES & $1.78$ & $(2,3)$ & NO & 10717\\
$(327,97)$ & 13 & $(2,1)$ & 1 & 1 & YES & YES & YES & $1.78$ & $(2,3)$ & NO & 10718\\
$(327,97)$ & 13 & $(2,1)$ & 1 & 1 & YES & YES & YES & $1.60$ & $(2,3)$ & -- & 10719\\
$(327,137)$ & 13 & $(2,1)$ & 1 & 1 & NO & YES & NO(2) & $1.60$ & $(6,1)$ & -- & 10720\\
$(327,97)$ & 13 & $(3,1)$ & 2 & 3 & YES & YES & YES & $1.62$ & $(2,3)$ & -- & 10721\\
$(327,97)$ & 13 & $(3,1)$ & 2 & 3 & YES & YES & YES & $2.00$ & $(2,3)$ & NO & 10722\\
$(327,97)$ & 13 & $(4,1)$ & 3 & 1 & YES & YES & YES & $1.62$ & $(2,3)$ & -- & 10723\\
$(327,121)$ & 13 & $(4,1)$ & 3 & 1 & YES & YES & YES & $1.75$ & $(4,2)$ & -- & 10724\\
$(327,52)$ & 17 & $(5,1)$ & 4 & 1 & YES & YES & NO(2) & $1.60$ & $(6,1)$ & NO & 10725\\
$(327,97)$ & 13 & $(5,1)$ & 4 & 1 & YES & YES & YES & $1.78$ & $(2,3)$ & NO & 10726\\
$(327,118)$ & 13 & $(5,2)$ & 3 & 1 & YES & YES & YES & $1.62$ & $(4,2)$ & NO & 10727\\
$(327,97)$ & 13 & $(7,2)$ & 4 & 1 & YES & YES & YES & $1.78$ & $(2,3)$ & NO & 10728\\
$(327,100)$ & 13 & $(10,3)$ & 5 & 1 & YES & YES & YES & $1.62$ & $(4,2)$ & NO & 10729\\
$(327,97)$ & 13 & $(37,11)$ & 8 & 1 & YES & YES & YES & $1.78$ & $(2,3)$ & NO & 10730\\
$(327,121)$ & 13 & $(73,27)$ & 9 & 1 & YES & YES & YES & $1.75$ & $(4,2)$ & NO & 10731\\
$(327,97)$ & 13 & $(91,27)$ & 10 & 1 & YES & YES & YES & $1.78$ & $(2,3)$ & 10303 & 10732\\
$(327,97)$ & 13 & $(327,97)$ & 13 & 327 & YES & YES & YES & $1.75$ & $(2,3)$ & NO & 10733\\
$(327,121)$ & 13 & $(327,121)$ & 13 & 327 & YES & YES & YES & $1.75$ & $(4,2)$ & NO & 10734\\
$(328,71)$ & 13 & $(2,1)$ & 1 & 2 & YES & YES & NO(2) & $1.56$ & $(2,3)$ & -- & 10735\\
$(328,71)$ & 13 & $(2,1)$ & 1 & 2 & YES & YES & NO(2) & $1.56$ & $(2,3)$ & NO & 10736\\
$(328,71)$ & 13 & $(4,1)$ & 3 & 4 & YES & YES & NO(2) & $1.67$ & $(2,3)$ & NO & 10737\\
$(328,71)$ & 13 & $(4,1)$ & 3 & 4 & YES & YES & NO(2) & $1.44$ & $(4,2)$ & -- & 10738\\
$(328,71)$ & 13 & $(14,3)$ & 6 & 2 & YES & YES & YES & $1.43$ & $(2,3)$ & 7754 & 10739\\
$(328,71)$ & 13 & $(23,5)$ & 7 & 1 & YES & YES & NO(2) & $1.67$ & $(2,3)$ & NO & 10740\\
$(329,89)$ & 13 & $(3,1)$ & 2 & 1 & YES & YES & YES & $1.50$ & $(2,3)$ & -- & 10741\\
$(329,89)$ & 13 & $(7,2)$ & 4 & 7 & YES & YES & YES & $1.62$ & $(2,3)$ & NO & 10742\\
$(329,89)$ & 13 & $(85,23)$ & 10 & 1 & YES & YES & YES & $1.43$ & $(4,2)$ & 10179 & 10743\\
$(330,89)$ & 13 & $(15,4)$ & 6 & 15 & YES & YES & YES & $1.62$ & $(4,2)$ & NO & 10744\\
$(330,89)$ & 13 & $(37,10)$ & 8 & 1 & YES & YES & YES & $1.57$ & $(2,3)$ & NO & 10745\\
$(330,89)$ & 13 & $(63,17)$ & 9 & 3 & YES & YES & YES & $1.75$ & $(2,3)$ & NO & 10746\\
$(330,89)$ & 13 & $(152,41)$ & 11 & 2 & YES & YES & YES & $1.67$ & $(4,2)$ & 11373 & 10747\\
$(331,79)$ & 15 & $(17,4)$ & 7 & 1 & YES & YES & NO(2) & $1.38$ & $(6,1)$ & NO & 10748\\
$(332,97)$ & 13 & $(2,1)$ & 1 & 2 & YES & YES & YES & $1.90$ & $(2,3)$ & NO & 10749\\
$(332,89)$ & 13 & $(3,1)$ & 2 & 1 & YES & YES & YES & $1.75$ & $(2,3)$ & -- & 10750\\
$(332,97)$ & 13 & $(3,1)$ & 2 & 1 & YES & YES & YES & $1.88$ & $(2,3)$ & NO & 10751\\
$(332,59)$ & 16 & $(6,1)$ & 5 & 2 & YES & YES & NO(2) & $1.50$ & $(6,1)$ & NO & 10752\\
$(332,97)$ & 13 & $(6,1)$ & 5 & 2 & YES & YES & YES & $1.29$ & $(4,2)$ & NO & 10753\\
$(332,59)$ & 16 & $(17,3)$ & 7 & 1 & YES & YES & NO(2) & $1.50$ & $(6,1)$ & NO & 10754\\
$(332,97)$ & 13 & $(17,5)$ & 6 & 1 & YES & YES & YES & $1.38$ & $(4,2)$ & NO & 10755\\
$(332,89)$ & 13 & $(26,7)$ & 7 & 2 & YES & YES & YES & $1.75$ & $(2,3)$ & 9013 & 10756\\
$(332,89)$ & 13 & $(56,15)$ & 9 & 4 & YES & YES & YES & $1.75$ & $(2,3)$ & NO & 10757\\
$(332,97)$ & 13 & $(89,26)$ & 10 & 1 & YES & YES & YES & $1.43$ & $(4,2)$ & NO & 10758\\
$(332,89)$ & 13 & $(138,37)$ & 11 & 2 & YES & YES & YES & $1.43$ & $(6,1)$ & 11183 & 10759\\
$(332,97)$ & 13 & $(154,45)$ & 11 & 2 & YES & YES & YES & $1.86$ & $(2,3)$ & 11402 & 10760\\
$(332,97)$ & 13 & $(243,71)$ & 12 & 1 & YES & YES & YES & $1.62$ & $(4,2)$ & NO & 10761\\
$(332,89)$ & 13 & $(332,89)$ & 13 & 332 & YES & YES & YES & $1.43$ & $(6,1)$ & NO & 10762\\
$(333,76)$ & 13 & $(2,1)$ & 1 & 1 & YES & YES & YES & $1.50$ & $(2,3)$ & -- & 10763\\
$(333,101)$ & 13 & $(2,1)$ & 1 & 1 & YES & YES & YES & $1.67$ & $(2,3)$ & -- & 10764\\
$(333,101)$ & 13 & $(2,1)$ & 1 & 1 & YES & YES & YES & $1.62$ & $(4,2)$ & NO & 10765\\
$(333,73)$ & 14 & $(3,1)$ & 2 & 3 & YES & YES & YES & $1.62$ & $(2,3)$ & NO & 10766\\
$(333,73)$ & 14 & $(3,1)$ & 2 & 3 & YES & YES & YES & $1.78$ & $(2,3)$ & -- & 10767\\
$(333,76)$ & 13 & $(3,1)$ & 2 & 3 & YES & YES & YES & $1.38$ & $(2,3)$ & NO & 10768\\
$(333,92)$ & 13 & $(3,1)$ & 2 & 3 & YES & YES & YES & $1.56$ & $(2,3)$ & -- & 10769\\
$(333,101)$ & 13 & $(3,1)$ & 2 & 3 & YES & YES & YES & $1.75$ & $(2,3)$ & NO & 10770\\
$(333,101)$ & 13 & $(3,1)$ & 2 & 3 & YES & YES & YES & $1.75$ & $(2,3)$ & -- & 10771\\
$(333,76)$ & 13 & $(4,1)$ & 3 & 1 & YES & YES & YES & $1.38$ & $(2,3)$ & NO & 10772\\
$(333,97)$ & 14 & $(4,1)$ & 3 & 1 & YES & YES & YES & $1.50$ & $(4,2)$ & -- & 10773\\
$(333,101)$ & 13 & $(4,1)$ & 3 & 1 & YES & YES & YES & $1.29$ & $(4,2)$ & -- & 10774\\
$(333,76)$ & 13 & $(5,1)$ & 4 & 1 & YES & YES & YES & $1.78$ & $(2,3)$ & NO & 10775\\
$(333,101)$ & 13 & $(5,1)$ & 4 & 1 & YES & YES & YES & $1.88$ & $(2,3)$ & NO & 10776\\
$(333,101)$ & 13 & $(7,2)$ & 4 & 1 & YES & YES & YES & $1.89$ & $(2,3)$ & NO & 10777\\
$(333,101)$ & 13 & $(23,7)$ & 7 & 1 & YES & YES & YES & $1.62$ & $(4,2)$ & NO & 10778\\
$(333,76)$ & 13 & $(35,8)$ & 8 & 1 & YES & YES & YES & $1.62$ & $(2,3)$ & NO & 10779\\
$(333,76)$ & 13 & $(48,11)$ & 9 & 3 & YES & YES & YES & $1.43$ & $(2,3)$ & NO & 10780\\
$(333,97)$ & 14 & $(55,16)$ & 9 & 1 & YES & YES & YES & $1.50$ & $(4,2)$ & NO & 10781\\
$(333,101)$ & 13 & $(56,17)$ & 9 & 1 & YES & YES & YES & $1.75$ & $(2,3)$ & 11244 & 10782\\
$(333,76)$ & 13 & $(57,13)$ & 9 & 3 & YES & YES & YES & $1.78$ & $(2,3)$ & NO & 10783\\
$(333,101)$ & 13 & $(89,27)$ & 10 & 1 & YES & YES & YES & $1.89$ & $(2,3)$ & 10304 & 10784\\
$(333,76)$ & 13 & $(127,29)$ & 11 & 1 & YES & YES & YES & $1.57$ & $(2,3)$ & NO & 10785\\
$(333,101)$ & 13 & $(211,64)$ & 12 & 1 & YES & YES & YES & $1.29$ & $(4,2)$ & NO & 10786\\
$(333,101)$ & 13 & $(333,101)$ & 13 & 333 & YES & YES & YES & $1.57$ & $(2,3)$ & NO & 10787\\
$(334,129)$ & 13 & $(5,1)$ & 4 & 1 & YES & YES & YES & $1.90$ & $(2,3)$ & -- & 10788\\
$(334,129)$ & 13 & $(5,1)$ & 4 & 1 & YES & YES & YES & $1.90$ & $(2,3)$ & NO & 10789\\
$(334,145)$ & 13 & $(5,2)$ & 3 & 1 & YES & YES & YES & $1.43$ & $(6,1)$ & NO & 10790\\
$(334,129)$ & 13 & $(13,5)$ & 5 & 1 & YES & YES & YES & $1.90$ & $(2,3)$ & NO & 10791\\
$(334,145)$ & 13 & $(30,13)$ & 8 & 2 & YES & YES & YES & $1.43$ & $(6,1)$ & NO & 10792\\
$(334,79)$ & 14 & $(203,48)$ & 13 & 1 & YES & YES & YES & $1.62$ & $(4,2)$ & 11773 & 10793\\
$(335,73)$ & 14 & $(2,1)$ & 1 & 1 & YES & YES & NO(2) & $1.38$ & $(4,2)$ & -- & 10794\\
$(335,123)$ & 12 & $(2,1)$ & 1 & 1 & YES & YES & YES & $1.57$ & $(2,3)$ & -- & 10795\\
$(335,128)$ & 12 & $(2,1)$ & 1 & 1 & YES & YES & YES & $1.67$ & $(4,2)$ & NO & 10796\\
$(335,73)$ & 14 & $(3,1)$ & 2 & 1 & YES & YES & YES & $1.50$ & $(4,2)$ & -- & 10797\\
$(335,78)$ & 14 & $(3,1)$ & 2 & 1 & YES & YES & YES & $1.75$ & $(2,3)$ & -- & 10798\\
$(335,94)$ & 13 & $(3,1)$ & 2 & 1 & YES & YES & YES & $1.78$ & $(2,3)$ & NO & 10799\\
$(335,123)$ & 12 & $(3,1)$ & 2 & 1 & YES & YES & YES & $1.62$ & $(2,3)$ & -- & 10800\\
$(335,98)$ & 13 & $(4,1)$ & 3 & 1 & YES & YES & YES & $1.75$ & $(4,2)$ & NO & 10801\\
$(335,128)$ & 12 & $(4,1)$ & 3 & 1 & YES & YES & YES & $1.57$ & $(2,3)$ & -- & 10802\\
$(335,123)$ & 12 & $(5,1)$ & 4 & 5 & YES & YES & YES & $1.57$ & $(2,3)$ & NO & 10803\\
$(335,128)$ & 12 & $(5,2)$ & 3 & 5 & YES & YES & YES & $1.86$ & $(2,3)$ & NO & 10804\\
$(335,51)$ & 17 & $(8,1)$ & 7 & 1 & YES & YES & NO(2) & $1.50$ & $(6,1)$ & NO & 10805\\
$(335,98)$ & 13 & $(17,5)$ & 6 & 1 & YES & YES & YES & $1.62$ & $(4,2)$ & NO & 10806\\
$(335,123)$ & 12 & $(19,7)$ & 6 & 1 & YES & YES & YES & $1.71$ & $(2,3)$ & NO & 10807\\
$(335,51)$ & 17 & $(20,3)$ & 8 & 5 & YES & YES & NO(2) & $1.50$ & $(6,1)$ & NO & 10808\\
$(335,73)$ & 14 & $(32,7)$ & 8 & 1 & YES & YES & YES & $1.43$ & $(4,2)$ & NO & 10809\\
$(335,123)$ & 12 & $(79,29)$ & 9 & 1 & YES & YES & YES & $1.71$ & $(2,3)$ & 10006 & 10810\\
$(335,128)$ & 12 & $(89,34)$ & 9 & 1 & YES & YES & YES & $1.71$ & $(2,3)$ & 10334 & 10811\\
$(335,128)$ & 12 & $(123,47)$ & 10 & 1 & YES & YES & YES & $1.71$ & $(2,3)$ & NO & 10812\\
$(335,98)$ & 13 & $(147,43)$ & 11 & 1 & YES & YES & YES & $1.67$ & $(4,2)$ & NO & 10813\\
$(335,128)$ & 12 & $(212,81)$ & 11 & 1 & YES & YES & YES & $1.43$ & $(2,3)$ & NO & 10814\\
$(337,73)$ & 14 & $(2,1)$ & 1 & 1 & YES & YES & YES & $1.57$ & $(4,2)$ & -- & 10815\\
$(337,76)$ & 14 & $(2,1)$ & 1 & 1 & YES & YES & NO(2) & $1.38$ & $(4,2)$ & -- & 10816\\
$(337,76)$ & 14 & $(2,1)$ & 1 & 1 & YES & YES & NO(2) & $1.50$ & $(4,2)$ & NO & 10817\\
$(337,91)$ & 13 & $(2,1)$ & 1 & 1 & YES & YES & YES & $1.62$ & $(2,3)$ & -- & 10818\\
$(337,98)$ & 13 & $(2,1)$ & 1 & 1 & YES & YES & YES & $1.67$ & $(4,2)$ & -- & 10819\\
$(337,98)$ & 13 & $(2,1)$ & 1 & 1 & YES & YES & YES & $1.80$ & $(2,3)$ & NO & 10820\\
$(337,100)$ & 13 & $(2,1)$ & 1 & 1 & YES & YES & YES & $1.75$ & $(4,2)$ & -- & 10821\\
$(337,129)$ & 12 & $(2,1)$ & 1 & 1 & YES & YES & YES & $1.62$ & $(2,3)$ & NO & 10822\\
$(337,129)$ & 12 & $(2,1)$ & 1 & 1 & YES & YES & YES & $1.78$ & $(4,2)$ & -- & 10823\\
$(337,141)$ & 13 & $(2,1)$ & 1 & 1 & YES & YES & YES & $1.43$ & $(6,1)$ & -- & 10824\\
$(337,91)$ & 13 & $(3,1)$ & 2 & 1 & YES & YES & YES & $1.80$ & $(2,3)$ & -- & 10825\\
$(337,100)$ & 13 & $(3,1)$ & 2 & 1 & YES & YES & NO(2) & $1.50$ & $(6,1)$ & NO & 10826\\
$(337,129)$ & 12 & $(3,1)$ & 2 & 1 & YES & YES & YES & $1.57$ & $(2,3)$ & -- & 10827\\
$(337,94)$ & 13 & $(4,1)$ & 3 & 1 & YES & YES & YES & $1.75$ & $(2,3)$ & -- & 10828\\
$(337,129)$ & 12 & $(5,1)$ & 4 & 1 & YES & YES & YES & $1.71$ & $(2,3)$ & NO & 10829\\
$(337,129)$ & 12 & $(5,2)$ & 3 & 1 & YES & YES & YES & $1.57$ & $(2,3)$ & NO & 10830\\
$(337,100)$ & 13 & $(6,1)$ & 5 & 1 & YES & YES & YES & $1.62$ & $(4,2)$ & NO & 10831\\
$(337,100)$ & 13 & $(7,2)$ & 4 & 1 & YES & YES & YES & $1.62$ & $(4,2)$ & NO & 10832\\
$(337,129)$ & 12 & $(8,3)$ & 4 & 1 & YES & YES & YES & $1.57$ & $(2,3)$ & NO & 10833\\
$(337,60)$ & 14 & $(11,2)$ & 6 & 1 & YES & YES & NO(2) & $1.50$ & $(2,3)$ & NO & 10834\\
$(337,98)$ & 13 & $(24,7)$ & 7 & 1 & YES & YES & YES & $1.80$ & $(2,3)$ & NO & 10835\\
$(337,91)$ & 13 & $(26,7)$ & 7 & 1 & YES & YES & YES & $1.62$ & $(2,3)$ & NO & 10836\\
$(337,100)$ & 13 & $(27,8)$ & 7 & 1 & YES & YES & YES & $1.75$ & $(4,2)$ & NO & 10837\\
$(337,129)$ & 12 & $(34,13)$ & 7 & 1 & YES & YES & YES & $1.86$ & $(2,3)$ & 10481 & 10838\\
$(337,73)$ & 14 & $(37,8)$ & 8 & 1 & YES & YES & YES & $1.57$ & $(4,2)$ & NO & 10839\\
$(337,91)$ & 13 & $(63,17)$ & 9 & 1 & YES & YES & YES & $1.86$ & $(2,3)$ & NO & 10840\\
$(337,129)$ & 12 & $(81,31)$ & 9 & 1 & YES & YES & YES & $1.86$ & $(2,3)$ & 10140 & 10841\\
$(337,94)$ & 13 & $(104,29)$ & 10 & 1 & YES & YES & YES & $1.90$ & $(2,3)$ & NO & 10842\\
$(337,91)$ & 13 & $(137,37)$ & 11 & 1 & YES & YES & YES & $1.62$ & $(4,2)$ & 11199 & 10843\\
$(337,98)$ & 13 & $(141,41)$ & 11 & 1 & YES & YES & YES & $1.80$ & $(2,3)$ & NO & 10844\\
$(338,77)$ & 14 & $(2,1)$ & 1 & 2 & YES & YES & YES & $1.50$ & $(4,2)$ & -- & 10845\\
$(338,99)$ & 13 & $(2,1)$ & 1 & 2 & YES & YES & YES & $1.43$ & $(4,2)$ & NO & 10846\\
$(338,99)$ & 13 & $(2,1)$ & 1 & 2 & YES & YES & YES & $1.50$ & $(2,3)$ & -- & 10847\\
$(338,129)$ & 12 & $(2,1)$ & 1 & 2 & YES & YES & YES & $1.29$ & $(4,2)$ & -- & 10848\\
$(338,129)$ & 12 & $(2,1)$ & 1 & 2 & YES & YES & YES & $1.71$ & $(2,3)$ & NO & 10849\\
$(338,79)$ & 14 & $(3,1)$ & 2 & 1 & YES & YES & YES & $1.75$ & $(2,3)$ & -- & 10850\\
$(338,99)$ & 13 & $(3,1)$ & 2 & 1 & YES & YES & YES & $1.62$ & $(2,3)$ & -- & 10851\\
$(338,129)$ & 12 & $(3,1)$ & 2 & 1 & YES & YES & YES & $1.62$ & $(2,3)$ & NO & 10852\\
$(338,129)$ & 12 & $(3,1)$ & 2 & 1 & YES & YES & YES & $1.62$ & $(2,3)$ & -- & 10853\\
$(338,77)$ & 14 & $(4,1)$ & 3 & 2 & YES & YES & YES & $1.75$ & $(2,3)$ & -- & 10854\\
$(338,129)$ & 12 & $(5,1)$ & 4 & 1 & YES & YES & YES & $1.43$ & $(2,3)$ & NO & 10855\\
$(338,129)$ & 12 & $(5,1)$ & 4 & 1 & YES & YES & YES & $1.71$ & $(2,3)$ & NO & 10856\\
$(338,129)$ & 12 & $(5,2)$ & 3 & 1 & YES & YES & YES & $1.89$ & $(2,3)$ & NO & 10857\\
$(338,129)$ & 12 & $(8,3)$ & 4 & 2 & YES & YES & YES & $1.62$ & $(4,2)$ & NO & 10858\\
$(338,131)$ & 12 & $(8,3)$ & 4 & 2 & YES & YES & YES & $1.57$ & $(2,3)$ & 11224 & 10859\\
$(338,99)$ & 13 & $(10,3)$ & 5 & 2 & YES & YES & YES & $1.50$ & $(2,3)$ & NO & 10860\\
$(338,129)$ & 12 & $(13,5)$ & 5 & 13 & YES & YES & YES & $1.29$ & $(4,2)$ & 8172 & 10861\\
$(338,131)$ & 12 & $(13,5)$ & 5 & 13 & YES & YES & YES & $1.71$ & $(2,3)$ & 10517 & 10862\\
$(338,99)$ & 13 & $(24,7)$ & 7 & 2 & YES & YES & YES & $1.78$ & $(2,3)$ & 8857 & 10863\\
$(338,99)$ & 13 & $(41,12)$ & 8 & 1 & YES & YES & YES & $1.75$ & $(2,3)$ & NO & 10864\\
$(338,129)$ & 12 & $(55,21)$ & 8 & 1 & YES & YES & YES & $1.86$ & $(2,3)$ & NO & 10865\\
$(338,77)$ & 14 & $(57,13)$ & 9 & 1 & YES & YES & YES & $1.50$ & $(2,3)$ & NO & 10866\\
$(338,99)$ & 13 & $(58,17)$ & 9 & 2 & YES & YES & YES & $1.62$ & $(2,3)$ & NO & 10867\\
$(338,129)$ & 12 & $(76,29)$ & 9 & 2 & YES & YES & YES & $1.78$ & $(2,3)$ & 9966 & 10868\\
$(338,131)$ & 12 & $(80,31)$ & 9 & 2 & YES & YES & YES & $1.71$ & $(2,3)$ & 10116 & 10869\\
$(338,77)$ & 14 & $(101,23)$ & 11 & 1 & YES & YES & YES & $1.89$ & $(2,3)$ & NO & 10870\\
$(338,79)$ & 14 & $(107,25)$ & 11 & 1 & YES & YES & YES & $1.75$ & $(2,3)$ & NO & 10871\\
$(338,129)$ & 12 & $(207,79)$ & 11 & 1 & YES & YES & YES & $1.75$ & $(2,3)$ & NO & 10872\\
$(338,79)$ & 14 & $(338,79)$ & 14 & 338 & YES & YES & YES & $1.75$ & $(2,3)$ & NO & 10873\\
$(339,65)$ & 15 & $(2,1)$ & 1 & 1 & YES & YES & NO(2) & $1.67$ & $(4,2)$ & NO & 10874\\
$(339,73)$ & 15 & $(2,1)$ & 1 & 1 & YES & YES & NO(2) & $1.62$ & $(6,1)$ & NO & 10875\\
$(339,80)$ & 15 & $(2,1)$ & 1 & 1 & YES & YES & NO(2) & $1.56$ & $(8,0)$ & -- & 10876\\
$(339,100)$ & 13 & $(2,1)$ & 1 & 1 & YES & YES & YES & $1.78$ & $(2,3)$ & -- & 10877\\
$(339,100)$ & 13 & $(3,1)$ & 2 & 3 & YES & YES & YES & $1.78$ & $(2,3)$ & -- & 10878\\
$(339,73)$ & 15 & $(4,1)$ & 3 & 1 & YES & YES & NO(2) & $1.44$ & $(8,0)$ & NO & 10879\\
$(339,100)$ & 13 & $(4,1)$ & 3 & 1 & YES & YES & YES & $1.78$ & $(2,3)$ & NO & 10880\\
$(339,65)$ & 15 & $(5,1)$ & 4 & 1 & YES & YES & NO(2) & $1.62$ & $(6,1)$ & NO & 10881\\
$(339,100)$ & 13 & $(7,2)$ & 4 & 1 & YES & YES & YES & $1.50$ & $(4,2)$ & NO & 10882\\
$(339,100)$ & 13 & $(27,8)$ & 7 & 3 & YES & YES & YES & $1.62$ & $(2,3)$ & NO & 10883\\
$(339,100)$ & 13 & $(139,41)$ & 11 & 1 & YES & YES & YES & $1.78$ & $(2,3)$ & NO & 10884\\
$(340,101)$ & 13 & $(2,1)$ & 1 & 2 & YES & YES & YES & $1.62$ & $(4,2)$ & -- & 10885\\
$(340,101)$ & 13 & $(2,1)$ & 1 & 2 & YES & YES & YES & $1.62$ & $(4,2)$ & NO & 10886\\
$(340,79)$ & 14 & $(4,1)$ & 3 & 4 & YES & YES & YES & $1.43$ & $(4,2)$ & -- & 10887\\
$(340,101)$ & 13 & $(4,1)$ & 3 & 4 & YES & YES & YES & $1.62$ & $(4,2)$ & NO & 10888\\
$(340,101)$ & 13 & $(4,1)$ & 3 & 4 & YES & YES & YES & $1.57$ & $(2,3)$ & -- & 10889\\
$(340,101)$ & 13 & $(7,2)$ & 4 & 1 & YES & YES & YES & $1.62$ & $(4,2)$ & NO & 10890\\
$(340,101)$ & 13 & $(27,8)$ & 7 & 1 & YES & YES & YES & $1.50$ & $(4,2)$ & NO & 10891\\
$(340,101)$ & 13 & $(101,30)$ & 10 & 1 & YES & YES & YES & $1.57$ & $(2,3)$ & NO & 10892\\
$(341,100)$ & 13 & $(2,1)$ & 1 & 1 & YES & YES & YES & $1.67$ & $(2,3)$ & -- & 10893\\
$(341,140)$ & 13 & $(2,1)$ & 1 & 1 & YES & YES & YES & $1.62$ & $(4,2)$ & -- & 10894\\
$(341,96)$ & 14 & $(3,1)$ & 2 & 1 & YES & YES & YES & $1.43$ & $(6,1)$ & -- & 10895\\
$(341,100)$ & 13 & $(3,1)$ & 2 & 1 & YES & YES & YES & $1.75$ & $(2,3)$ & -- & 10896\\
$(341,100)$ & 13 & $(4,1)$ & 3 & 1 & YES & YES & YES & $1.78$ & $(2,3)$ & NO & 10897\\
$(341,140)$ & 13 & $(17,7)$ & 6 & 1 & YES & YES & YES & $1.62$ & $(4,2)$ & NO & 10898\\
$(341,95)$ & 13 & $(18,5)$ & 6 & 1 & YES & YES & YES & $1.57$ & $(4,2)$ & 8044 & 10899\\
$(341,95)$ & 13 & $(25,7)$ & 7 & 1 & YES & YES & YES & $1.71$ & $(2,3)$ & NO & 10900\\
$(341,96)$ & 14 & $(25,7)$ & 7 & 1 & YES & YES & YES & $1.43$ & $(6,1)$ & NO & 10901\\
$(341,100)$ & 13 & $(41,12)$ & 8 & 1 & YES & YES & YES & $1.62$ & $(2,3)$ & NO & 10902\\
$(341,133)$ & 13 & $(59,23)$ & 9 & 1 & YES & YES & YES & $1.62$ & $(4,2)$ & NO & 10903\\
$(341,100)$ & 13 & $(75,22)$ & 10 & 1 & YES & YES & YES & $1.78$ & $(2,3)$ & 9967 & 10904\\
$(341,100)$ & 13 & $(133,39)$ & 11 & 1 & YES & YES & YES & $1.67$ & $(2,3)$ & NO & 10905\\
$(341,133)$ & 13 & $(341,133)$ & 13 & 341 & YES & YES & YES & $1.62$ & $(4,2)$ & NO & 10906\\
$(342,149)$ & 13 & $(2,1)$ & 1 & 2 & YES & YES & YES & $1.88$ & $(4,2)$ & -- & 10907\\
$(342,101)$ & 13 & $(3,1)$ & 2 & 3 & YES & YES & YES & $1.50$ & $(2,3)$ & NO & 10908\\
$(342,101)$ & 13 & $(4,1)$ & 3 & 2 & YES & YES & YES & $1.43$ & $(2,3)$ & -- & 10909\\
$(342,101)$ & 13 & $(193,57)$ & 12 & 1 & YES & YES & YES & $1.43$ & $(4,2)$ & NO & 10910\\
$(343,131)$ & 12 & $(3,1)$ & 2 & 1 & YES & YES & YES & $1.86$ & $(2,3)$ & -- & 10911\\
$(343,131)$ & 12 & $(8,3)$ & 4 & 1 & YES & YES & YES & $1.71$ & $(2,3)$ & NO & 10912\\
$(343,144)$ & 12 & $(19,8)$ & 6 & 1 & YES & YES & YES & $1.43$ & $(2,3)$ & NO & 10913\\
$(343,131)$ & 12 & $(21,8)$ & 6 & 7 & YES & YES & YES & $1.71$ & $(2,3)$ & NO & 10914\\
$(343,131)$ & 12 & $(144,55)$ & 10 & 1 & YES & YES & YES & $1.71$ & $(2,3)$ & NO & 10915\\
$(344,95)$ & 13 & $(2,1)$ & 1 & 2 & YES & YES & YES & $1.67$ & $(2,3)$ & -- & 10916\\
$(344,95)$ & 13 & $(2,1)$ & 1 & 2 & YES & YES & YES & $1.56$ & $(2,3)$ & NO & 10917\\
$(344,95)$ & 13 & $(3,1)$ & 2 & 1 & YES & YES & YES & $1.78$ & $(2,3)$ & NO & 10918\\
$(344,95)$ & 13 & $(7,2)$ & 4 & 1 & YES & YES & YES & $1.78$ & $(2,3)$ & NO & 10919\\
$(344,95)$ & 13 & $(18,5)$ & 6 & 2 & YES & YES & YES & $1.56$ & $(2,3)$ & 6339 & 10920\\
$(344,95)$ & 13 & $(134,37)$ & 11 & 2 & YES & YES & YES & $1.57$ & $(2,3)$ & 11185 & 10921\\
$(346,79)$ & 13 & $(2,1)$ & 1 & 2 & YES & YES & NO(2) & $1.56$ & $(2,3)$ & NO & 10922\\
$(346,125)$ & 14 & $(2,1)$ & 1 & 2 & YES & YES & YES & $1.57$ & $(4,2)$ & -- & 10923\\
$(346,79)$ & 13 & $(3,1)$ & 2 & 1 & YES & YES & YES & $1.50$ & $(2,3)$ & NO & 10924\\
$(346,93)$ & 13 & $(3,1)$ & 2 & 1 & YES & YES & YES & $1.88$ & $(2,3)$ & NO & 10925\\
$(346,93)$ & 13 & $(4,1)$ & 3 & 2 & YES & YES & YES & $1.50$ & $(4,2)$ & -- & 10926\\
$(346,79)$ & 13 & $(5,2)$ & 3 & 1 & YES & YES & YES & $1.71$ & $(2,3)$ & NO & 10927\\
$(346,79)$ & 13 & $(17,4)$ & 7 & 1 & YES & YES & YES & $1.86$ & $(2,3)$ & NO & 10928\\
$(346,79)$ & 13 & $(35,8)$ & 8 & 1 & YES & YES & NO(2) & $1.64$ & $(4,2)$ & NO & 10929\\
$(346,97)$ & 14 & $(132,37)$ & 12 & 2 & YES & YES & YES & $1.43$ & $(6,1)$ & 11156 & 10930\\
$(346,125)$ & 14 & $(155,56)$ & 12 & 1 & YES & YES & YES & $1.71$ & $(4,2)$ & NO & 10931\\
$(347,92)$ & 13 & $(2,1)$ & 1 & 1 & YES & YES & YES & $1.43$ & $(4,2)$ & NO & 10932\\
$(347,92)$ & 13 & $(2,1)$ & 1 & 1 & YES & YES & YES & $1.43$ & $(6,1)$ & -- & 10933\\
$(347,97)$ & 13 & $(2,1)$ & 1 & 1 & YES & YES & YES & $1.75$ & $(2,3)$ & NO & 10934\\
$(347,97)$ & 13 & $(2,1)$ & 1 & 1 & YES & YES & YES & $1.71$ & $(2,3)$ & -- & 10935\\
$(347,147)$ & 13 & $(2,1)$ & 1 & 1 & YES & YES & YES & $1.75$ & $(4,2)$ & -- & 10936\\
$(347,151)$ & 14 & $(2,1)$ & 1 & 1 & NO & YES & NO(2) & $1.56$ & $(8,0)$ & -- & 10937\\
$(347,92)$ & 13 & $(3,1)$ & 2 & 1 & YES & YES & YES & $1.43$ & $(4,2)$ & NO & 10938\\
$(347,92)$ & 13 & $(3,1)$ & 2 & 1 & YES & YES & YES & $1.43$ & $(6,1)$ & -- & 10939\\
$(347,93)$ & 13 & $(3,1)$ & 2 & 1 & YES & YES & YES & $1.43$ & $(4,2)$ & NO & 10940\\
$(347,101)$ & 13 & $(3,1)$ & 2 & 1 & YES & YES & YES & $1.75$ & $(2,3)$ & NO & 10941\\
$(347,101)$ & 13 & $(3,1)$ & 2 & 1 & YES & YES & YES & $1.75$ & $(2,3)$ & -- & 10942\\
$(347,101)$ & 13 & $(3,1)$ & 2 & 1 & YES & YES & YES & $1.62$ & $(2,3)$ & NO & 10943\\
$(347,93)$ & 13 & $(4,1)$ & 3 & 1 & YES & YES & YES & $1.43$ & $(4,2)$ & -- & 10944\\
$(347,97)$ & 13 & $(4,1)$ & 3 & 1 & YES & YES & YES & $1.57$ & $(2,3)$ & -- & 10945\\
$(347,101)$ & 13 & $(4,1)$ & 3 & 1 & YES & YES & YES & $1.71$ & $(4,2)$ & NO & 10946\\
$(347,101)$ & 13 & $(5,1)$ & 4 & 1 & YES & YES & YES & $1.78$ & $(2,3)$ & -- & 10947\\
$(347,147)$ & 13 & $(5,2)$ & 3 & 1 & YES & YES & YES & $1.88$ & $(4,2)$ & NO & 10948\\
$(347,97)$ & 13 & $(6,1)$ & 5 & 1 & YES & YES & YES & $1.43$ & $(4,2)$ & NO & 10949\\
$(347,101)$ & 13 & $(17,5)$ & 6 & 1 & YES & YES & YES & $1.57$ & $(2,3)$ & NO & 10950\\
$(347,93)$ & 13 & $(41,11)$ & 8 & 1 & YES & YES & YES & $1.70$ & $(2,3)$ & NO & 10951\\
$(347,93)$ & 13 & $(56,15)$ & 9 & 1 & YES & YES & YES & $1.71$ & $(4,2)$ & NO & 10952\\
$(347,101)$ & 13 & $(79,23)$ & 10 & 1 & YES & YES & YES & $1.67$ & $(2,3)$ & 10180 & 10953\\
$(347,101)$ & 13 & $(134,39)$ & 11 & 1 & YES & YES & YES & $1.57$ & $(4,2)$ & NO & 10954\\
$(347,101)$ & 13 & $(213,62)$ & 12 & 1 & YES & YES & YES & $1.62$ & $(2,3)$ & NO & 10955\\
$(347,97)$ & 13 & $(254,71)$ & 12 & 1 & YES & YES & YES & $1.57$ & $(2,3)$ & NO & 10956\\
$(347,147)$ & 13 & $(347,147)$ & 13 & 347 & YES & YES & YES & $1.62$ & $(4,2)$ & NO & 10957\\
$(348,103)$ & 13 & $(2,1)$ & 1 & 2 & YES & YES & YES & $1.71$ & $(2,3)$ & NO & 10958\\
$(349,144)$ & 13 & $(2,1)$ & 1 & 1 & YES & YES & YES & $1.75$ & $(4,2)$ & -- & 10959\\
$(349,143)$ & 13 & $(3,1)$ & 2 & 1 & YES & YES & YES & $1.78$ & $(4,2)$ & NO & 10960\\
$(349,143)$ & 13 & $(3,1)$ & 2 & 1 & YES & YES & YES & $1.78$ & $(4,2)$ & -- & 10961\\
$(349,144)$ & 13 & $(3,1)$ & 2 & 1 & YES & YES & YES & $1.75$ & $(4,2)$ & NO & 10962\\
$(349,144)$ & 13 & $(3,1)$ & 2 & 1 & YES & YES & YES & $1.75$ & $(4,2)$ & -- & 10963\\
$(349,143)$ & 13 & $(39,16)$ & 8 & 1 & YES & YES & YES & $1.89$ & $(4,2)$ & NO & 10964\\
$(349,143)$ & 13 & $(83,34)$ & 10 & 1 & YES & YES & YES & $1.89$ & $(4,2)$ & NO & 10965\\
$(349,106)$ & 13 & $(214,65)$ & 12 & 1 & YES & YES & YES & $1.50$ & $(4,2)$ & NO & 10966\\
$(349,106)$ & 13 & $(349,106)$ & 13 & 349 & YES & YES & YES & $1.50$ & $(4,2)$ & NO & 10967\\
$(350,83)$ & 14 & $(2,1)$ & 1 & 2 & YES & YES & YES & $1.57$ & $(4,2)$ & -- & 10968\\
$(350,157)$ & 14 & $(2,1)$ & 1 & 2 & NO & YES & NO(2) & $1.70$ & $(6,1)$ & -- & 10969\\
$(350,83)$ & 14 & $(17,4)$ & 7 & 1 & YES & YES & YES & $1.57$ & $(4,2)$ & NO & 10970\\
$(351,76)$ & 13 & $(2,1)$ & 1 & 1 & YES & YES & YES & $1.50$ & $(2,3)$ & NO & 10971\\
$(351,76)$ & 13 & $(2,1)$ & 1 & 1 & YES & YES & YES & $1.75$ & $(2,3)$ & -- & 10972\\
$(351,76)$ & 13 & $(5,2)$ & 3 & 1 & YES & YES & YES & $1.57$ & $(2,3)$ & -- & 10973\\
$(351,76)$ & 13 & $(5,2)$ & 3 & 1 & YES & YES & YES & $1.86$ & $(2,3)$ & NO & 10974\\
$(351,76)$ & 13 & $(9,2)$ & 5 & 9 & YES & YES & YES & $1.38$ & $(2,3)$ & NO & 10975\\
$(351,76)$ & 13 & $(13,3)$ & 6 & 13 & YES & YES & YES & $1.57$ & $(2,3)$ & NO & 10976\\
$(351,98)$ & 13 & $(111,31)$ & 10 & 3 & YES & YES & YES & $1.67$ & $(4,2)$ & NO & 10977\\
$(352,65)$ & 15 & $(3,1)$ & 2 & 1 & YES & YES & YES & $1.62$ & $(4,2)$ & NO & 10978\\
$(353,97)$ & 13 & $(2,1)$ & 1 & 1 & YES & YES & YES & $1.57$ & $(4,2)$ & -- & 10979\\
$(353,97)$ & 13 & $(2,1)$ & 1 & 1 & YES & YES & YES & $1.78$ & $(2,3)$ & NO & 10980\\
$(353,75)$ & 14 & $(3,1)$ & 2 & 1 & YES & YES & YES & $1.50$ & $(4,2)$ & -- & 10981\\
$(353,80)$ & 14 & $(3,1)$ & 2 & 1 & YES & YES & YES & $1.29$ & $(4,2)$ & -- & 10982\\
$(353,97)$ & 13 & $(3,1)$ & 2 & 1 & YES & YES & YES & $1.43$ & $(4,2)$ & NO & 10983\\
$(353,97)$ & 13 & $(3,1)$ & 2 & 1 & YES & YES & YES & $1.43$ & $(4,2)$ & -- & 10984\\
$(353,108)$ & 13 & $(3,1)$ & 2 & 1 & YES & YES & YES & $1.67$ & $(4,2)$ & -- & 10985\\
$(353,149)$ & 13 & $(3,1)$ & 2 & 1 & YES & YES & YES & $1.67$ & $(4,2)$ & -- & 10986\\
$(353,82)$ & 14 & $(4,1)$ & 3 & 1 & YES & YES & YES & $1.62$ & $(2,3)$ & -- & 10987\\
$(353,149)$ & 13 & $(4,1)$ & 3 & 1 & YES & YES & YES & $1.89$ & $(4,2)$ & NO & 10988\\
$(353,108)$ & 13 & $(5,1)$ & 4 & 1 & YES & YES & YES & $1.67$ & $(4,2)$ & NO & 10989\\
$(353,80)$ & 14 & $(6,1)$ & 5 & 1 & YES & YES & YES & $1.14$ & $(6,1)$ & NO & 10990\\
$(353,97)$ & 13 & $(7,2)$ & 4 & 1 & YES & YES & YES & $1.57$ & $(4,2)$ & NO & 10991\\
$(353,80)$ & 14 & $(13,3)$ & 6 & 1 & YES & YES & YES & $1.75$ & $(2,3)$ & NO & 10992\\
$(353,82)$ & 14 & $(17,4)$ & 7 & 1 & YES & YES & YES & $1.62$ & $(2,3)$ & NO & 10993\\
$(353,149)$ & 13 & $(26,11)$ & 7 & 1 & YES & YES & YES & $1.78$ & $(4,2)$ & NO & 10994\\
$(353,80)$ & 14 & $(31,7)$ & 8 & 1 & YES & YES & YES & $1.29$ & $(4,2)$ & NO & 10995\\
$(353,149)$ & 13 & $(109,46)$ & 10 & 1 & YES & YES & YES & $1.78$ & $(4,2)$ & NO & 10996\\
$(353,75)$ & 14 & $(113,24)$ & 11 & 1 & YES & YES & NO(2) & $1.56$ & $(2,3)$ & NO & 10997\\
$(353,149)$ & 13 & $(199,84)$ & 12 & 1 & YES & YES & YES & $1.78$ & $(4,2)$ & NO & 10998\\
$(353,80)$ & 14 & $(353,80)$ & 14 & 353 & YES & YES & YES & $1.14$ & $(6,1)$ & NO & 10999\\
$(354,83)$ & 14 & $(47,11)$ & 9 & 1 & YES & YES & YES & $1.43$ & $(2,3)$ & NO & 11000\\
$(355,154)$ & 13 & $(2,1)$ & 1 & 1 & YES & YES & YES & $1.43$ & $(6,1)$ & -- & 11001\\
$(355,77)$ & 14 & $(3,1)$ & 2 & 1 & YES & YES & YES & $1.29$ & $(4,2)$ & NO & 11002\\
$(355,81)$ & 13 & $(3,1)$ & 2 & 1 & YES & YES & YES & $1.50$ & $(4,2)$ & -- & 11003\\
$(355,99)$ & 13 & $(3,1)$ & 2 & 1 & YES & YES & YES & $1.62$ & $(2,3)$ & -- & 11004\\
$(355,99)$ & 13 & $(3,1)$ & 2 & 1 & YES & YES & YES & $1.43$ & $(4,2)$ & 8471 & 11005\\
$(355,104)$ & 13 & $(3,1)$ & 2 & 1 & YES & YES & YES & $1.86$ & $(2,3)$ & -- & 11006\\
$(355,77)$ & 14 & $(4,1)$ & 3 & 1 & YES & YES & YES & $1.29$ & $(4,2)$ & NO & 11007\\
$(355,83)$ & 14 & $(4,1)$ & 3 & 1 & YES & YES & YES & $1.62$ & $(2,3)$ & -- & 11008\\
$(355,99)$ & 13 & $(5,1)$ & 4 & 5 & YES & YES & YES & $1.62$ & $(2,3)$ & NO & 11009\\
$(355,104)$ & 13 & $(10,3)$ & 5 & 5 & YES & YES & YES & $1.86$ & $(2,3)$ & NO & 11010\\
$(355,99)$ & 13 & $(11,3)$ & 5 & 1 & YES & YES & YES & $1.62$ & $(2,3)$ & NO & 11011\\
$(355,83)$ & 14 & $(13,3)$ & 6 & 1 & YES & YES & YES & $1.78$ & $(2,3)$ & NO & 11012\\
$(355,62)$ & 15 & $(17,3)$ & 7 & 1 & YES & YES & YES & $1.50$ & $(2,3)$ & NO & 11013\\
$(355,104)$ & 13 & $(17,5)$ & 6 & 1 & YES & YES & YES & $1.86$ & $(2,3)$ & NO & 11014\\
$(355,99)$ & 13 & $(25,7)$ & 7 & 5 & YES & YES & YES & $1.75$ & $(2,3)$ & 9158 & 11015\\
$(355,83)$ & 14 & $(47,11)$ & 9 & 1 & YES & YES & YES & $1.78$ & $(2,3)$ & NO & 11016\\
$(355,84)$ & 14 & $(55,13)$ & 10 & 5 & YES & YES & YES & $1.57$ & $(4,2)$ & 11310 & 11017\\
$(355,99)$ & 13 & $(61,17)$ & 9 & 1 & YES & YES & YES & $1.62$ & $(2,3)$ & NO & 11018\\
$(355,154)$ & 13 & $(83,36)$ & 10 & 1 & YES & YES & YES & $1.29$ & $(6,1)$ & 10351 & 11019\\
$(355,147)$ & 13 & $(99,41)$ & 10 & 1 & YES & YES & YES & $1.43$ & $(6,1)$ & 10716 & 11020\\
$(355,99)$ & 13 & $(147,41)$ & 11 & 1 & YES & YES & YES & $1.67$ & $(2,3)$ & 11410 & 11021\\
$(355,83)$ & 14 & $(278,65)$ & 13 & 1 & YES & YES & YES & $1.62$ & $(2,3)$ & NO & 11022\\
$(355,99)$ & 13 & $(355,99)$ & 13 & 355 & YES & YES & YES & $1.62$ & $(2,3)$ & NO & 11023\\
$(356,147)$ & 13 & $(3,1)$ & 2 & 1 & YES & YES & YES & $1.75$ & $(4,2)$ & NO & 11024\\
$(356,147)$ & 13 & $(5,1)$ & 4 & 1 & YES & YES & YES & $1.62$ & $(4,2)$ & -- & 11025\\
$(356,105)$ & 13 & $(10,3)$ & 5 & 2 & YES & YES & YES & $1.57$ & $(2,3)$ & NO & 11026\\
$(357,109)$ & 13 & $(4,1)$ & 3 & 1 & YES & YES & YES & $1.78$ & $(4,2)$ & NO & 11027\\
$(357,109)$ & 13 & $(95,29)$ & 10 & 1 & YES & YES & YES & $1.78$ & $(4,2)$ & 10618 & 11028\\
$(358,147)$ & 13 & $(2,1)$ & 1 & 2 & YES & YES & YES & $1.62$ & $(4,2)$ & NO & 11029\\
$(358,147)$ & 13 & $(3,1)$ & 2 & 1 & YES & YES & YES & $1.75$ & $(4,2)$ & NO & 11030\\
$(358,151)$ & 13 & $(5,2)$ & 3 & 1 & YES & YES & YES & $1.43$ & $(6,1)$ & NO & 11031\\
$(358,151)$ & 13 & $(7,3)$ & 4 & 1 & YES & YES & YES & $1.88$ & $(4,2)$ & NO & 11032\\
$(358,147)$ & 13 & $(39,16)$ & 8 & 1 & YES & YES & YES & $1.75$ & $(4,2)$ & 8182 & 11033\\
$(358,147)$ & 13 & $(151,62)$ & 11 & 1 & YES & YES & YES & $1.62$ & $(4,2)$ & NO & 11034\\
$(358,147)$ & 13 & $(358,147)$ & 13 & 358 & YES & YES & YES & $1.75$ & $(4,2)$ & NO & 11035\\
$(359,100)$ & 13 & $(2,1)$ & 1 & 1 & YES & YES & YES & $1.56$ & $(2,3)$ & -- & 11036\\
$(359,100)$ & 13 & $(2,1)$ & 1 & 1 & YES & YES & YES & $1.78$ & $(2,3)$ & NO & 11037\\
$(359,106)$ & 13 & $(2,1)$ & 1 & 1 & YES & YES & YES & $1.67$ & $(4,2)$ & -- & 11038\\
$(359,106)$ & 13 & $(2,1)$ & 1 & 1 & YES & YES & YES & $1.67$ & $(4,2)$ & NO & 11039\\
$(359,140)$ & 13 & $(2,1)$ & 1 & 1 & YES & YES & YES & $1.62$ & $(4,2)$ & -- & 11040\\
$(359,100)$ & 13 & $(3,1)$ & 2 & 1 & YES & YES & YES & $1.78$ & $(2,3)$ & -- & 11041\\
$(359,100)$ & 13 & $(3,1)$ & 2 & 1 & YES & YES & YES & $1.78$ & $(2,3)$ & NO & 11042\\
$(359,106)$ & 13 & $(3,1)$ & 2 & 1 & YES & YES & YES & $1.62$ & $(2,3)$ & -- & 11043\\
$(359,106)$ & 13 & $(3,1)$ & 2 & 1 & YES & YES & YES & $1.50$ & $(4,2)$ & NO & 11044\\
$(359,140)$ & 13 & $(3,1)$ & 2 & 1 & YES & YES & YES & $1.62$ & $(4,2)$ & -- & 11045\\
$(359,105)$ & 13 & $(4,1)$ & 3 & 1 & YES & YES & YES & $1.50$ & $(2,3)$ & NO & 11046\\
$(359,100)$ & 13 & $(5,1)$ & 4 & 1 & YES & YES & YES & $1.67$ & $(2,3)$ & -- & 11047\\
$(359,106)$ & 13 & $(7,2)$ & 4 & 1 & YES & YES & YES & $1.50$ & $(2,3)$ & NO & 11048\\
$(359,100)$ & 13 & $(11,3)$ & 5 & 1 & YES & YES & YES & $1.57$ & $(2,3)$ & NO & 11049\\
$(359,105)$ & 13 & $(24,7)$ & 7 & 1 & YES & YES & YES & $1.50$ & $(2,3)$ & NO & 11050\\
$(359,106)$ & 13 & $(27,8)$ & 7 & 1 & YES & YES & YES & $1.57$ & $(2,3)$ & 9407 & 11051\\
$(359,100)$ & 13 & $(43,12)$ & 8 & 1 & YES & YES & YES & $1.43$ & $(2,3)$ & NO & 11052\\
$(359,100)$ & 13 & $(61,17)$ & 9 & 1 & YES & YES & YES & $1.78$ & $(2,3)$ & NO & 11053\\
$(359,105)$ & 13 & $(65,19)$ & 9 & 1 & YES & YES & YES & $1.43$ & $(2,3)$ & NO & 11054\\
$(359,100)$ & 13 & $(79,22)$ & 10 & 1 & YES & YES & YES & $1.50$ & $(4,2)$ & 10268 & 11055\\
$(359,106)$ & 13 & $(105,31)$ & 10 & 1 & YES & YES & YES & $1.43$ & $(2,3)$ & NO & 11056\\
$(359,105)$ & 13 & $(106,31)$ & 10 & 1 & YES & YES & YES & $1.71$ & $(2,3)$ & NO & 11057\\
$(359,100)$ & 13 & $(140,39)$ & 11 & 1 & YES & YES & YES & $1.78$ & $(2,3)$ & NO & 11058\\
$(359,106)$ & 13 & $(149,44)$ & 11 & 1 & YES & YES & YES & $1.57$ & $(2,3)$ & 11452 & 11059\\
$(360,101)$ & 13 & $(2,1)$ & 1 & 2 & YES & YES & YES & $1.71$ & $(4,2)$ & NO & 11060\\
$(360,101)$ & 13 & $(2,1)$ & 1 & 2 & YES & YES & YES & $1.67$ & $(2,3)$ & -- & 11061\\
$(360,101)$ & 13 & $(18,5)$ & 6 & 18 & YES & YES & YES & $1.71$ & $(2,3)$ & NO & 11062\\
$(360,101)$ & 13 & $(57,16)$ & 9 & 3 & YES & YES & YES & $1.43$ & $(4,2)$ & NO & 11063\\
$(361,70)$ & 16 & $(4,1)$ & 3 & 1 & YES & YES & NO(2) & $1.60$ & $(6,1)$ & -- & 11064\\
$(361,70)$ & 16 & $(21,4)$ & 8 & 1 & YES & YES & NO(2) & $1.60$ & $(6,1)$ & NO & 11065\\
$(363,98)$ & 13 & $(2,1)$ & 1 & 1 & YES & YES & YES & $1.71$ & $(2,3)$ & -- & 11066\\
$(363,98)$ & 13 & $(2,1)$ & 1 & 1 & YES & YES & YES & $1.86$ & $(2,3)$ & NO & 11067\\
$(363,100)$ & 13 & $(2,1)$ & 1 & 1 & YES & YES & YES & $1.75$ & $(2,3)$ & -- & 11068\\
$(363,100)$ & 13 & $(2,1)$ & 1 & 1 & YES & YES & YES & $1.78$ & $(2,3)$ & NO & 11069\\
$(363,65)$ & 15 & $(3,1)$ & 2 & 3 & YES & YES & YES & $1.62$ & $(4,2)$ & NO & 11070\\
$(363,98)$ & 13 & $(3,1)$ & 2 & 3 & YES & YES & YES & $1.62$ & $(2,3)$ & -- & 11071\\
$(363,98)$ & 13 & $(3,1)$ & 2 & 3 & YES & YES & YES & $1.57$ & $(2,3)$ & NO & 11072\\
$(363,100)$ & 13 & $(3,1)$ & 2 & 3 & YES & YES & YES & $1.43$ & $(2,3)$ & -- & 11073\\
$(363,100)$ & 13 & $(3,1)$ & 2 & 3 & YES & YES & YES & $1.78$ & $(2,3)$ & NO & 11074\\
$(363,65)$ & 15 & $(4,1)$ & 3 & 1 & YES & YES & YES & $1.62$ & $(4,2)$ & NO & 11075\\
$(363,100)$ & 13 & $(4,1)$ & 3 & 1 & YES & YES & YES & $1.62$ & $(4,2)$ & -- & 11076\\
$(363,133)$ & 13 & $(4,1)$ & 3 & 1 & YES & YES & YES & $1.78$ & $(4,2)$ & NO & 11077\\
$(363,98)$ & 13 & $(5,1)$ & 4 & 1 & YES & YES & YES & $1.57$ & $(2,3)$ & NO & 11078\\
$(363,100)$ & 13 & $(7,2)$ & 4 & 1 & YES & YES & YES & $1.67$ & $(2,3)$ & NO & 11079\\
$(363,100)$ & 13 & $(11,3)$ & 5 & 11 & YES & YES & YES & $1.62$ & $(4,2)$ & NO & 11080\\
$(363,98)$ & 13 & $(15,4)$ & 6 & 3 & YES & YES & YES & $1.86$ & $(2,3)$ & NO & 11081\\
$(363,100)$ & 13 & $(18,5)$ & 6 & 3 & YES & YES & YES & $1.43$ & $(2,3)$ & NO & 11082\\
$(363,98)$ & 13 & $(26,7)$ & 7 & 1 & YES & YES & YES & $1.50$ & $(2,3)$ & NO & 11083\\
$(363,100)$ & 13 & $(29,8)$ & 7 & 1 & YES & YES & YES & $1.62$ & $(4,2)$ & NO & 11084\\
$(363,98)$ & 13 & $(37,10)$ & 8 & 1 & YES & YES & YES & $1.57$ & $(2,3)$ & NO & 11085\\
$(363,100)$ & 13 & $(40,11)$ & 8 & 1 & YES & YES & YES & $1.71$ & $(2,3)$ & NO & 11086\\
$(363,98)$ & 13 & $(63,17)$ & 9 & 3 & YES & YES & YES & $1.50$ & $(2,3)$ & NO & 11087\\
$(363,100)$ & 13 & $(69,19)$ & 9 & 3 & YES & YES & YES & $1.62$ & $(2,3)$ & NO & 11088\\
$(363,100)$ & 13 & $(98,27)$ & 10 & 1 & YES & YES & YES & $1.62$ & $(4,2)$ & NO & 11089\\
$(363,149)$ & 13 & $(229,94)$ & 12 & 1 & YES & YES & YES & $1.67$ & $(4,2)$ & NO & 11090\\
$(363,100)$ & 13 & $(265,73)$ & 12 & 1 & YES & YES & YES & $1.62$ & $(4,2)$ & NO & 11091\\
$(363,100)$ & 13 & $(363,100)$ & 13 & 363 & YES & YES & YES & $1.71$ & $(2,3)$ & NO & 11092\\
$(364,151)$ & 13 & $(2,1)$ & 1 & 2 & YES & YES & YES & $1.75$ & $(4,2)$ & -- & 11093\\
$(364,151)$ & 13 & $(41,17)$ & 8 & 1 & YES & YES & YES & $1.88$ & $(4,2)$ & NO & 11094\\
$(364,151)$ & 13 & $(94,39)$ & 10 & 2 & YES & YES & YES & $1.75$ & $(4,2)$ & 10668 & 11095\\
$(365,98)$ & 13 & $(2,1)$ & 1 & 1 & YES & YES & YES & $1.80$ & $(2,3)$ & -- & 11096\\
$(365,108)$ & 13 & $(2,1)$ & 1 & 1 & YES & YES & YES & $1.71$ & $(2,3)$ & NO & 11097\\
$(365,108)$ & 13 & $(2,1)$ & 1 & 1 & YES & YES & YES & $1.71$ & $(2,3)$ & -- & 11098\\
$(365,98)$ & 13 & $(4,1)$ & 3 & 1 & YES & YES & YES & $1.57$ & $(2,3)$ & -- & 11099\\
$(365,108)$ & 13 & $(4,1)$ & 3 & 1 & YES & YES & YES & $1.57$ & $(2,3)$ & NO & 11100\\
$(365,108)$ & 13 & $(10,3)$ & 5 & 5 & YES & YES & YES & $1.71$ & $(2,3)$ & NO & 11101\\
$(365,98)$ & 13 & $(11,3)$ & 5 & 1 & YES & YES & YES & $1.86$ & $(2,3)$ & NO & 11102\\
$(365,98)$ & 13 & $(15,4)$ & 6 & 5 & YES & YES & YES & $1.57$ & $(4,2)$ & 8262 & 11103\\
$(365,108)$ & 13 & $(17,5)$ & 6 & 1 & YES & YES & YES & $1.86$ & $(2,3)$ & NO & 11104\\
$(365,98)$ & 13 & $(26,7)$ & 7 & 1 & YES & YES & YES & $1.75$ & $(2,3)$ & NO & 11105\\
$(366,83)$ & 14 & $(3,1)$ & 2 & 3 & YES & YES & YES & $1.29$ & $(4,2)$ & -- & 11106\\
$(366,97)$ & 14 & $(19,5)$ & 7 & 1 & YES & YES & YES & $1.43$ & $(6,1)$ & NO & 11107\\
$(367,78)$ & 14 & $(2,1)$ & 1 & 1 & YES & YES & YES & $1.29$ & $(4,2)$ & -- & 11108\\
$(367,84)$ & 14 & $(2,1)$ & 1 & 1 & YES & YES & YES & $1.88$ & $(2,3)$ & NO & 11109\\
$(367,112)$ & 13 & $(2,1)$ & 1 & 1 & YES & YES & YES & $1.90$ & $(2,3)$ & -- & 11110\\
$(367,78)$ & 14 & $(3,1)$ & 2 & 1 & YES & YES & YES & $1.78$ & $(2,3)$ & -- & 11111\\
$(367,80)$ & 14 & $(3,1)$ & 2 & 1 & YES & YES & YES & $1.62$ & $(4,2)$ & -- & 11112\\
$(367,83)$ & 14 & $(3,1)$ & 2 & 1 & YES & YES & YES & $1.80$ & $(2,3)$ & -- & 11113\\
$(367,83)$ & 14 & $(3,1)$ & 2 & 1 & YES & YES & YES & $1.86$ & $(2,3)$ & NO & 11114\\
$(367,83)$ & 14 & $(3,1)$ & 2 & 1 & YES & YES & YES & $1.86$ & $(2,3)$ & NO & 11115\\
$(367,99)$ & 13 & $(3,1)$ & 2 & 1 & YES & YES & YES & $1.43$ & $(4,2)$ & -- & 11116\\
$(367,101)$ & 13 & $(3,1)$ & 2 & 1 & YES & YES & YES & $1.67$ & $(2,3)$ & -- & 11117\\
$(367,109)$ & 13 & $(3,1)$ & 2 & 1 & YES & YES & YES & $1.57$ & $(2,3)$ & -- & 11118\\
$(367,109)$ & 13 & $(3,1)$ & 2 & 1 & YES & YES & YES & $1.71$ & $(2,3)$ & NO & 11119\\
$(367,112)$ & 13 & $(3,1)$ & 2 & 1 & YES & YES & YES & $1.57$ & $(2,3)$ & -- & 11120\\
$(367,101)$ & 13 & $(4,1)$ & 3 & 1 & YES & YES & YES & $1.57$ & $(4,2)$ & NO & 11121\\
$(367,101)$ & 13 & $(4,1)$ & 3 & 1 & YES & YES & YES & $1.71$ & $(2,3)$ & -- & 11122\\
$(367,83)$ & 14 & $(5,1)$ & 4 & 1 & YES & YES & YES & $1.70$ & $(2,3)$ & NO & 11123\\
$(367,101)$ & 13 & $(5,1)$ & 4 & 1 & YES & YES & YES & $1.71$ & $(2,3)$ & NO & 11124\\
$(367,111)$ & 14 & $(5,1)$ & 4 & 1 & YES & YES & YES & $1.62$ & $(4,2)$ & NO & 11125\\
$(367,83)$ & 14 & $(6,1)$ & 5 & 1 & YES & YES & YES & $1.43$ & $(4,2)$ & NO & 11126\\
$(367,84)$ & 14 & $(6,1)$ & 5 & 1 & YES & YES & YES & $1.50$ & $(4,2)$ & NO & 11127\\
$(367,109)$ & 13 & $(7,2)$ & 4 & 1 & YES & YES & YES & $1.71$ & $(2,3)$ & NO & 11128\\
$(367,112)$ & 13 & $(10,3)$ & 5 & 1 & YES & YES & YES & $1.67$ & $(2,3)$ & NO & 11129\\
$(367,101)$ & 13 & $(11,3)$ & 5 & 1 & YES & YES & YES & $1.57$ & $(6,1)$ & 7872 & 11130\\
$(367,83)$ & 14 & $(13,3)$ & 6 & 1 & YES & YES & YES & $1.57$ & $(2,3)$ & NO & 11131\\
$(367,99)$ & 13 & $(15,4)$ & 6 & 1 & YES & YES & YES & $1.29$ & $(6,1)$ & NO & 11132\\
$(367,83)$ & 14 & $(22,5)$ & 7 & 1 & YES & YES & YES & $1.62$ & $(2,3)$ & NO & 11133\\
$(367,84)$ & 14 & $(22,5)$ & 7 & 1 & YES & YES & YES & $1.71$ & $(2,3)$ & NO & 11134\\
$(367,112)$ & 13 & $(23,7)$ & 7 & 1 & YES & YES & YES & $1.80$ & $(2,3)$ & NO & 11135\\
$(367,109)$ & 13 & $(27,8)$ & 7 & 1 & YES & YES & YES & $1.71$ & $(2,3)$ & NO & 11136\\
$(367,83)$ & 14 & $(31,7)$ & 8 & 1 & YES & YES & YES & $1.43$ & $(4,2)$ & NO & 11137\\
$(367,112)$ & 13 & $(36,11)$ & 8 & 1 & YES & YES & YES & $1.62$ & $(2,3)$ & 8354 & 11138\\
$(367,101)$ & 13 & $(40,11)$ & 8 & 1 & YES & YES & YES & $1.57$ & $(4,2)$ & NO & 11139\\
$(367,84)$ & 14 & $(48,11)$ & 9 & 1 & YES & YES & YES & $1.62$ & $(4,2)$ & NO & 11140\\
$(367,112)$ & 13 & $(59,18)$ & 9 & 1 & YES & YES & YES & $1.75$ & $(2,3)$ & 9705 & 11141\\
$(367,99)$ & 13 & $(89,24)$ & 10 & 1 & YES & YES & YES & $1.43$ & $(4,2)$ & NO & 11142\\
$(367,83)$ & 14 & $(115,26)$ & 11 & 1 & YES & YES & YES & $1.75$ & $(2,3)$ & NO & 11143\\
$(367,84)$ & 14 & $(118,27)$ & 11 & 1 & YES & YES & YES & $1.62$ & $(2,3)$ & NO & 11144\\
$(367,101)$ & 13 & $(149,41)$ & 11 & 1 & YES & YES & YES & $1.71$ & $(2,3)$ & 11476 & 11145\\
$(367,84)$ & 14 & $(367,84)$ & 14 & 367 & YES & YES & YES & $1.78$ & $(2,3)$ & NO & 11146\\
$(368,107)$ & 13 & $(2,1)$ & 1 & 2 & YES & YES & YES & $1.57$ & $(2,3)$ & NO & 11147\\
$(368,107)$ & 13 & $(24,7)$ & 7 & 8 & YES & YES & YES & $1.57$ & $(2,3)$ & NO & 11148\\
$(368,107)$ & 13 & $(141,41)$ & 11 & 1 & YES & YES & YES & $1.43$ & $(2,3)$ & NO & 11149\\
$(371,87)$ & 14 & $(2,1)$ & 1 & 1 & YES & YES & NO(2) & $1.50$ & $(6,1)$ & NO & 11150\\
$(371,104)$ & 14 & $(2,1)$ & 1 & 1 & YES & YES & YES & $1.43$ & $(6,1)$ & NO & 11151\\
$(371,144)$ & 13 & $(2,1)$ & 1 & 1 & YES & YES & YES & $1.75$ & $(4,2)$ & -- & 11152\\
$(371,104)$ & 14 & $(5,1)$ & 4 & 1 & YES & YES & YES & $1.43$ & $(6,1)$ & -- & 11153\\
$(371,144)$ & 13 & $(5,1)$ & 4 & 1 & YES & YES & YES & $1.50$ & $(4,2)$ & NO & 11154\\
$(371,108)$ & 14 & $(24,7)$ & 7 & 1 & YES & YES & YES & $1.75$ & $(4,2)$ & NO & 11155\\
$(371,104)$ & 14 & $(107,30)$ & 11 & 1 & YES & YES & YES & $1.43$ & $(6,1)$ & 10930 & 11156\\
$(372,109)$ & 13 & $(2,1)$ & 1 & 2 & YES & YES & YES & $1.78$ & $(2,3)$ & NO & 11157\\
$(372,109)$ & 13 & $(3,1)$ & 2 & 3 & YES & YES & YES & $1.57$ & $(2,3)$ & -- & 11158\\
$(372,109)$ & 13 & $(3,1)$ & 2 & 3 & YES & YES & YES & $1.78$ & $(2,3)$ & NO & 11159\\
$(372,157)$ & 13 & $(3,1)$ & 2 & 3 & YES & YES & YES & $1.75$ & $(4,2)$ & NO & 11160\\
$(372,109)$ & 13 & $(4,1)$ & 3 & 4 & YES & YES & YES & $1.70$ & $(2,3)$ & -- & 11161\\
$(372,109)$ & 13 & $(10,3)$ & 5 & 2 & YES & YES & YES & $1.57$ & $(2,3)$ & NO & 11162\\
$(372,109)$ & 13 & $(17,5)$ & 6 & 1 & YES & YES & YES & $1.57$ & $(4,2)$ & NO & 11163\\
$(372,109)$ & 13 & $(41,12)$ & 8 & 1 & YES & YES & YES & $1.78$ & $(2,3)$ & 8404 & 11164\\
$(372,109)$ & 13 & $(99,29)$ & 10 & 3 & YES & YES & YES & $1.57$ & $(2,3)$ & NO & 11165\\
$(372,109)$ & 13 & $(157,46)$ & 11 & 1 & YES & YES & YES & $1.67$ & $(2,3)$ & NO & 11166\\
$(373,104)$ & 13 & $(2,1)$ & 1 & 1 & YES & YES & YES & $1.62$ & $(2,3)$ & NO & 11167\\
$(373,104)$ & 13 & $(2,1)$ & 1 & 1 & YES & YES & YES & $1.57$ & $(2,3)$ & -- & 11168\\
$(373,109)$ & 13 & $(2,1)$ & 1 & 1 & YES & YES & YES & $1.75$ & $(2,3)$ & -- & 11169\\
$(373,104)$ & 13 & $(3,1)$ & 2 & 1 & YES & YES & YES & $1.71$ & $(2,3)$ & -- & 11170\\
$(373,104)$ & 13 & $(3,1)$ & 2 & 1 & YES & YES & YES & $1.62$ & $(4,2)$ & NO & 11171\\
$(373,104)$ & 13 & $(3,1)$ & 2 & 1 & YES & YES & YES & $1.86$ & $(2,3)$ & NO & 11172\\
$(373,109)$ & 13 & $(3,1)$ & 2 & 1 & YES & YES & YES & $1.43$ & $(2,3)$ & -- & 11173\\
$(373,109)$ & 13 & $(3,1)$ & 2 & 1 & YES & YES & YES & $1.89$ & $(2,3)$ & NO & 11174\\
$(373,104)$ & 13 & $(4,1)$ & 3 & 1 & YES & YES & YES & $1.50$ & $(2,3)$ & NO & 11175\\
$(373,85)$ & 14 & $(5,1)$ & 4 & 1 & YES & YES & YES & $1.80$ & $(2,3)$ & NO & 11176\\
$(373,104)$ & 13 & $(7,2)$ & 4 & 1 & YES & YES & YES & $1.62$ & $(2,3)$ & NO & 11177\\
$(373,109)$ & 13 & $(10,3)$ & 5 & 1 & YES & YES & YES & $1.71$ & $(2,3)$ & NO & 11178\\
$(373,109)$ & 13 & $(17,5)$ & 6 & 1 & YES & YES & YES & $1.75$ & $(2,3)$ & NO & 11179\\
$(373,109)$ & 13 & $(24,7)$ & 7 & 1 & YES & YES & YES & $1.57$ & $(4,2)$ & 8486 & 11180\\
$(373,104)$ & 13 & $(25,7)$ & 7 & 1 & YES & YES & YES & $1.71$ & $(2,3)$ & NO & 11181\\
$(373,104)$ & 13 & $(43,12)$ & 8 & 1 & YES & YES & YES & $1.78$ & $(2,3)$ & NO & 11182\\
$(373,100)$ & 13 & $(97,26)$ & 10 & 1 & YES & YES & YES & $1.43$ & $(6,1)$ & 10759 & 11183\\
$(373,138)$ & 13 & $(100,37)$ & 10 & 1 & YES & YES & YES & $1.75$ & $(4,2)$ & NO & 11184\\
$(373,103)$ & 13 & $(105,29)$ & 10 & 1 & YES & YES & YES & $1.57$ & $(2,3)$ & 10921 & 11185\\
$(373,158)$ & 13 & $(144,61)$ & 11 & 1 & YES & YES & YES & $1.75$ & $(4,2)$ & NO & 11186\\
$(373,109)$ & 13 & $(154,45)$ & 11 & 1 & YES & YES & YES & $1.67$ & $(2,3)$ & NO & 11187\\
$(373,85)$ & 14 & $(294,67)$ & 13 & 1 & YES & YES & YES & $1.90$ & $(2,3)$ & NO & 11188\\
$(373,154)$ & 13 & $(373,154)$ & 13 & 373 & YES & YES & YES & $1.67$ & $(4,2)$ & NO & 11189\\
$(374,111)$ & 13 & $(2,1)$ & 1 & 2 & YES & YES & YES & $1.71$ & $(2,3)$ & -- & 11190\\
$(374,67)$ & 15 & $(3,1)$ & 2 & 1 & YES & YES & YES & $1.62$ & $(4,2)$ & NO & 11191\\
$(374,111)$ & 13 & $(3,1)$ & 2 & 1 & YES & YES & YES & $1.62$ & $(4,2)$ & NO & 11192\\
$(374,67)$ & 15 & $(4,1)$ & 3 & 2 & YES & YES & YES & $1.62$ & $(4,2)$ & NO & 11193\\
$(374,101)$ & 13 & $(4,1)$ & 3 & 2 & YES & YES & YES & $1.50$ & $(4,2)$ & -- & 11194\\
$(374,137)$ & 13 & $(4,1)$ & 3 & 2 & YES & YES & YES & $1.62$ & $(4,2)$ & -- & 11195\\
$(374,155)$ & 13 & $(7,3)$ & 4 & 1 & YES & YES & YES & $1.78$ & $(4,2)$ & NO & 11196\\
$(374,111)$ & 13 & $(10,3)$ & 5 & 2 & YES & YES & YES & $1.78$ & $(4,2)$ & NO & 11197\\
$(374,111)$ & 13 & $(91,27)$ & 10 & 1 & YES & YES & YES & $1.57$ & $(2,3)$ & NO & 11198\\
$(374,101)$ & 13 & $(100,27)$ & 10 & 2 & YES & YES & YES & $1.62$ & $(4,2)$ & 10843 & 11199\\
$(374,113)$ & 14 & $(139,42)$ & 12 & 1 & YES & YES & YES & $1.75$ & $(4,2)$ & NO & 11200\\
$(374,111)$ & 13 & $(219,65)$ & 12 & 1 & YES & YES & YES & $1.62$ & $(2,3)$ & NO & 11201\\
$(375,86)$ & 14 & $(3,1)$ & 2 & 3 & YES & YES & YES & $1.80$ & $(2,3)$ & -- & 11202\\
$(375,71)$ & 15 & $(11,2)$ & 6 & 1 & YES & YES & YES & $1.57$ & $(2,3)$ & NO & 11203\\
$(376,105)$ & 13 & $(2,1)$ & 1 & 2 & YES & YES & YES & $1.62$ & $(2,3)$ & NO & 11204\\
$(376,105)$ & 13 & $(2,1)$ & 1 & 2 & YES & YES & YES & $1.57$ & $(4,2)$ & -- & 11205\\
$(376,111)$ & 13 & $(2,1)$ & 1 & 2 & YES & YES & YES & $1.56$ & $(4,2)$ & -- & 11206\\
$(376,111)$ & 13 & $(2,1)$ & 1 & 2 & YES & YES & YES & $1.67$ & $(4,2)$ & NO & 11207\\
$(376,105)$ & 13 & $(3,1)$ & 2 & 1 & YES & YES & YES & $1.62$ & $(4,2)$ & NO & 11208\\
$(376,105)$ & 13 & $(4,1)$ & 3 & 4 & YES & YES & YES & $1.71$ & $(2,3)$ & NO & 11209\\
$(376,105)$ & 13 & $(7,2)$ & 4 & 1 & YES & YES & YES & $1.62$ & $(4,2)$ & NO & 11210\\
$(376,111)$ & 13 & $(17,5)$ & 6 & 1 & YES & YES & YES & $1.56$ & $(4,2)$ & NO & 11211\\
$(376,115)$ & 14 & $(49,15)$ & 9 & 1 & YES & YES & YES & $1.78$ & $(4,2)$ & NO & 11212\\
$(376,105)$ & 13 & $(68,19)$ & 9 & 4 & YES & YES & YES & $1.71$ & $(2,3)$ & NO & 11213\\
$(376,105)$ & 13 & $(111,31)$ & 10 & 1 & YES & YES & YES & $1.50$ & $(2,3)$ & NO & 11214\\
$(377,70)$ & 15 & $(2,1)$ & 1 & 1 & YES & YES & YES & $1.43$ & $(4,2)$ & -- & 11215\\
$(377,70)$ & 15 & $(2,1)$ & 1 & 1 & YES & YES & YES & $1.57$ & $(4,2)$ & NO & 11216\\
$(377,85)$ & 14 & $(2,1)$ & 1 & 1 & YES & YES & YES & $1.62$ & $(2,3)$ & -- & 11217\\
$(377,85)$ & 14 & $(2,1)$ & 1 & 1 & YES & YES & YES & $1.75$ & $(2,3)$ & NO & 11218\\
$(377,112)$ & 13 & $(2,1)$ & 1 & 1 & YES & YES & YES & $1.71$ & $(2,3)$ & NO & 11219\\
$(377,85)$ & 14 & $(3,1)$ & 2 & 1 & YES & YES & YES & $1.78$ & $(2,3)$ & NO & 11220\\
$(377,85)$ & 14 & $(3,1)$ & 2 & 1 & YES & YES & YES & $1.78$ & $(2,3)$ & -- & 11221\\
$(377,138)$ & 13 & $(4,1)$ & 3 & 1 & YES & YES & YES & $1.62$ & $(4,2)$ & -- & 11222\\
$(377,112)$ & 13 & $(5,1)$ & 4 & 1 & YES & YES & YES & $1.86$ & $(2,3)$ & NO & 11223\\
$(377,144)$ & 12 & $(5,2)$ & 3 & 1 & YES & YES & YES & $1.57$ & $(2,3)$ & 10859 & 11224\\
$(377,112)$ & 13 & $(7,2)$ & 4 & 1 & YES & YES & YES & $1.86$ & $(2,3)$ & NO & 11225\\
$(377,159)$ & 13 & $(7,3)$ & 4 & 1 & YES & YES & YES & $1.75$ & $(4,2)$ & NO & 11226\\
$(377,138)$ & 13 & $(8,3)$ & 4 & 1 & YES & YES & YES & $1.62$ & $(4,2)$ & NO & 11227\\
$(377,70)$ & 15 & $(11,2)$ & 6 & 1 & YES & YES & YES & $1.43$ & $(4,2)$ & NO & 11228\\
$(377,136)$ & 13 & $(11,4)$ & 5 & 1 & YES & YES & YES & $1.75$ & $(4,2)$ & NO & 11229\\
$(377,112)$ & 13 & $(37,11)$ & 8 & 1 & YES & YES & YES & $1.67$ & $(4,2)$ & NO & 11230\\
$(377,85)$ & 14 & $(40,9)$ & 9 & 1 & YES & YES & YES & $1.80$ & $(2,3)$ & NO & 11231\\
$(377,105)$ & 14 & $(61,17)$ & 9 & 1 & YES & YES & YES & $1.75$ & $(4,2)$ & NO & 11232\\
$(377,70)$ & 15 & $(97,18)$ & 11 & 1 & YES & YES & YES & $1.43$ & $(4,2)$ & NO & 11233\\
$(377,112)$ & 13 & $(138,41)$ & 11 & 1 & YES & YES & YES & $1.86$ & $(2,3)$ & NO & 11234\\
$(379,111)$ & 13 & $(2,1)$ & 1 & 1 & YES & YES & YES & $1.71$ & $(2,3)$ & -- & 11235\\
$(379,147)$ & 13 & $(2,1)$ & 1 & 1 & YES & YES & YES & $1.88$ & $(4,2)$ & NO & 11236\\
$(379,111)$ & 13 & $(3,1)$ & 2 & 1 & YES & YES & YES & $1.43$ & $(2,3)$ & -- & 11237\\
$(379,111)$ & 13 & $(4,1)$ & 3 & 1 & YES & YES & YES & $1.75$ & $(2,3)$ & NO & 11238\\
$(379,111)$ & 13 & $(5,1)$ & 4 & 1 & YES & YES & YES & $1.62$ & $(2,3)$ & NO & 11239\\
$(379,82)$ & 14 & $(9,2)$ & 5 & 1 & YES & YES & YES & $1.57$ & $(4,2)$ & NO & 11240\\
$(379,111)$ & 13 & $(10,3)$ & 5 & 1 & YES & YES & YES & $1.71$ & $(2,3)$ & NO & 11241\\
$(379,105)$ & 13 & $(11,3)$ & 5 & 1 & YES & YES & YES & $1.57$ & $(2,3)$ & NO & 11242\\
$(379,115)$ & 13 & $(23,7)$ & 7 & 1 & YES & YES & YES & $1.43$ & $(2,3)$ & NO & 11243\\
$(379,115)$ & 13 & $(33,10)$ & 8 & 1 & YES & YES & YES & $1.75$ & $(2,3)$ & 10782 & 11244\\
$(379,111)$ & 13 & $(58,17)$ & 9 & 1 & YES & YES & YES & $1.57$ & $(2,3)$ & 11492 & 11245\\
$(379,115)$ & 13 & $(89,27)$ & 10 & 1 & YES & YES & YES & $1.71$ & $(2,3)$ & 10669 & 11246\\
$(379,111)$ & 13 & $(140,41)$ & 11 & 1 & YES & YES & YES & $1.75$ & $(2,3)$ & NO & 11247\\
$(379,111)$ & 13 & $(379,111)$ & 13 & 379 & YES & YES & YES & $1.62$ & $(2,3)$ & NO & 11248\\
$(380,83)$ & 14 & $(2,1)$ & 1 & 2 & YES & YES & YES & $1.70$ & $(2,3)$ & NO & 11249\\
$(380,83)$ & 14 & $(2,1)$ & 1 & 2 & YES & YES & YES & $1.70$ & $(2,3)$ & -- & 11250\\
$(380,83)$ & 14 & $(3,1)$ & 2 & 1 & YES & YES & YES & $1.71$ & $(2,3)$ & -- & 11251\\
$(380,87)$ & 14 & $(3,1)$ & 2 & 1 & YES & YES & YES & $1.75$ & $(2,3)$ & -- & 11252\\
$(380,83)$ & 14 & $(4,1)$ & 3 & 4 & YES & YES & YES & $1.70$ & $(2,3)$ & NO & 11253\\
$(380,83)$ & 14 & $(6,1)$ & 5 & 2 & YES & YES & YES & $1.29$ & $(4,2)$ & NO & 11254\\
$(380,83)$ & 14 & $(14,3)$ & 6 & 2 & YES & YES & YES & $1.57$ & $(2,3)$ & NO & 11255\\
$(380,83)$ & 14 & $(23,5)$ & 7 & 1 & YES & YES & YES & $1.75$ & $(2,3)$ & NO & 11256\\
$(380,87)$ & 14 & $(35,8)$ & 8 & 5 & YES & YES & YES & $1.62$ & $(4,2)$ & NO & 11257\\
$(380,83)$ & 14 & $(380,83)$ & 14 & 380 & YES & YES & YES & $1.62$ & $(2,3)$ & NO & 11258\\
$(381,83)$ & 14 & $(3,1)$ & 2 & 3 & YES & YES & YES & $1.50$ & $(4,2)$ & -- & 11259\\
$(382,87)$ & 14 & $(2,1)$ & 1 & 2 & YES & YES & YES & $1.50$ & $(4,2)$ & -- & 11260\\
$(382,87)$ & 14 & $(2,1)$ & 1 & 2 & YES & YES & YES & $1.62$ & $(4,2)$ & NO & 11261\\
$(382,89)$ & 14 & $(2,1)$ & 1 & 2 & YES & YES & YES & $1.88$ & $(2,3)$ & NO & 11262\\
$(382,149)$ & 13 & $(2,1)$ & 1 & 2 & YES & YES & YES & $1.88$ & $(4,2)$ & -- & 11263\\
$(382,149)$ & 13 & $(3,1)$ & 2 & 1 & YES & YES & YES & $1.67$ & $(4,2)$ & -- & 11264\\
$(382,89)$ & 14 & $(5,1)$ & 4 & 1 & YES & YES & YES & $1.70$ & $(2,3)$ & NO & 11265\\
$(382,141)$ & 13 & $(5,1)$ & 4 & 1 & YES & YES & YES & $1.62$ & $(4,2)$ & -- & 11266\\
$(382,89)$ & 14 & $(9,2)$ & 5 & 1 & YES & YES & YES & $1.43$ & $(2,3)$ & NO & 11267\\
$(382,101)$ & 14 & $(19,5)$ & 7 & 1 & YES & YES & YES & $1.57$ & $(6,1)$ & NO & 11268\\
$(382,89)$ & 14 & $(43,10)$ & 9 & 1 & YES & YES & YES & $1.43$ & $(2,3)$ & NO & 11269\\
$(382,89)$ & 14 & $(73,17)$ & 10 & 1 & YES & YES & YES & $1.70$ & $(2,3)$ & NO & 11270\\
$(382,87)$ & 14 & $(180,41)$ & 12 & 2 & YES & YES & YES & $1.67$ & $(4,2)$ & 11657 & 11271\\
$(382,149)$ & 13 & $(382,149)$ & 13 & 382 & YES & YES & YES & $1.67$ & $(4,2)$ & NO & 11272\\
$(383,106)$ & 13 & $(2,1)$ & 1 & 1 & YES & YES & YES & $1.43$ & $(4,2)$ & -- & 11273\\
$(383,112)$ & 13 & $(2,1)$ & 1 & 1 & YES & YES & YES & $1.57$ & $(2,3)$ & -- & 11274\\
$(383,106)$ & 13 & $(3,1)$ & 2 & 1 & YES & YES & YES & $1.62$ & $(2,3)$ & NO & 11275\\
$(383,106)$ & 13 & $(3,1)$ & 2 & 1 & YES & YES & YES & $1.57$ & $(2,3)$ & -- & 11276\\
$(383,106)$ & 13 & $(3,1)$ & 2 & 1 & YES & YES & YES & $1.86$ & $(2,3)$ & NO & 11277\\
$(383,112)$ & 13 & $(3,1)$ & 2 & 1 & YES & YES & YES & $1.43$ & $(2,3)$ & -- & 11278\\
$(383,112)$ & 13 & $(4,1)$ & 3 & 1 & YES & YES & YES & $1.71$ & $(2,3)$ & NO & 11279\\
$(383,112)$ & 13 & $(17,5)$ & 6 & 1 & YES & YES & YES & $1.57$ & $(2,3)$ & NO & 11280\\
$(383,106)$ & 13 & $(18,5)$ & 6 & 1 & YES & YES & YES & $1.43$ & $(4,2)$ & NO & 11281\\
$(383,106)$ & 13 & $(29,8)$ & 7 & 1 & YES & YES & YES & $1.57$ & $(2,3)$ & 9789 & 11282\\
$(383,106)$ & 13 & $(65,18)$ & 9 & 1 & YES & YES & YES & $1.71$ & $(2,3)$ & NO & 11283\\
$(383,112)$ & 13 & $(65,19)$ & 9 & 1 & YES & YES & YES & $1.71$ & $(2,3)$ & NO & 11284\\
$(383,106)$ & 13 & $(112,31)$ & 10 & 1 & YES & YES & YES & $1.62$ & $(2,3)$ & NO & 11285\\
$(385,108)$ & 14 & $(25,7)$ & 7 & 5 & YES & YES & YES & $1.75$ & $(4,2)$ & NO & 11286\\
$(387,104)$ & 13 & $(2,1)$ & 1 & 1 & YES & YES & YES & $1.50$ & $(4,2)$ & -- & 11287\\
$(388,89)$ & 14 & $(2,1)$ & 1 & 2 & YES & YES & YES & $1.62$ & $(4,2)$ & NO & 11288\\
$(388,147)$ & 13 & $(3,1)$ & 2 & 1 & YES & YES & YES & $1.62$ & $(4,2)$ & NO & 11289\\
$(388,89)$ & 14 & $(61,14)$ & 10 & 1 & YES & YES & YES & $1.62$ & $(4,2)$ & NO & 11290\\
$(388,103)$ & 14 & $(388,103)$ & 14 & 388 & YES & YES & YES & $1.43$ & $(6,1)$ & NO & 11291\\
$(389,84)$ & 14 & $(2,1)$ & 1 & 1 & YES & YES & YES & $1.88$ & $(2,3)$ & NO & 11292\\
$(389,84)$ & 14 & $(2,1)$ & 1 & 1 & YES & YES & YES & $1.88$ & $(2,3)$ & -- & 11293\\
$(389,89)$ & 14 & $(2,1)$ & 1 & 1 & YES & YES & YES & $1.50$ & $(2,3)$ & -- & 11294\\
$(389,89)$ & 14 & $(2,1)$ & 1 & 1 & YES & YES & YES & $1.62$ & $(2,3)$ & NO & 11295\\
$(389,91)$ & 14 & $(2,1)$ & 1 & 1 & YES & YES & YES & $1.62$ & $(2,3)$ & NO & 11296\\
$(389,92)$ & 14 & $(2,1)$ & 1 & 1 & YES & YES & YES & $1.57$ & $(4,2)$ & NO & 11297\\
$(389,115)$ & 13 & $(2,1)$ & 1 & 1 & YES & YES & YES & $1.71$ & $(2,3)$ & -- & 11298\\
$(389,84)$ & 14 & $(3,1)$ & 2 & 1 & YES & YES & YES & $1.57$ & $(4,2)$ & -- & 11299\\
$(389,84)$ & 14 & $(3,1)$ & 2 & 1 & YES & YES & YES & $1.57$ & $(2,3)$ & NO & 11300\\
$(389,88)$ & 14 & $(3,1)$ & 2 & 1 & YES & YES & YES & $1.62$ & $(4,2)$ & -- & 11301\\
$(389,89)$ & 14 & $(3,1)$ & 2 & 1 & YES & YES & YES & $1.50$ & $(4,2)$ & -- & 11302\\
$(389,89)$ & 14 & $(3,1)$ & 2 & 1 & YES & YES & YES & $1.62$ & $(4,2)$ & NO & 11303\\
$(389,92)$ & 14 & $(3,1)$ & 2 & 1 & YES & YES & YES & $1.57$ & $(4,2)$ & NO & 11304\\
$(389,88)$ & 14 & $(5,1)$ & 4 & 1 & YES & YES & YES & $1.50$ & $(4,2)$ & NO & 11305\\
$(389,84)$ & 14 & $(6,1)$ & 5 & 1 & YES & YES & YES & $1.62$ & $(4,2)$ & NO & 11306\\
$(389,92)$ & 14 & $(21,5)$ & 8 & 1 & YES & YES & YES & $1.57$ & $(4,2)$ & 11482 & 11307\\
$(389,84)$ & 14 & $(23,5)$ & 7 & 1 & YES & YES & YES & $1.57$ & $(2,3)$ & NO & 11308\\
$(389,115)$ & 13 & $(27,8)$ & 7 & 1 & YES & YES & YES & $1.57$ & $(2,3)$ & NO & 11309\\
$(389,92)$ & 14 & $(38,9)$ & 9 & 1 & YES & YES & YES & $1.57$ & $(4,2)$ & 11017 & 11310\\
$(389,91)$ & 14 & $(47,11)$ & 9 & 1 & YES & YES & NO(2) & $1.56$ & $(8,0)$ & 9589 & 11311\\
$(389,84)$ & 14 & $(51,11)$ & 9 & 1 & YES & YES & YES & $1.62$ & $(4,2)$ & NO & 11312\\
$(389,88)$ & 14 & $(53,12)$ & 9 & 1 & YES & YES & YES & $1.50$ & $(4,2)$ & NO & 11313\\
$(389,89)$ & 14 & $(83,19)$ & 10 & 1 & YES & YES & YES & $1.70$ & $(2,3)$ & NO & 11314\\
$(389,88)$ & 14 & $(84,19)$ & 10 & 1 & YES & YES & YES & $1.62$ & $(4,2)$ & NO & 11315\\
$(389,84)$ & 14 & $(125,27)$ & 11 & 1 & YES & YES & YES & $1.57$ & $(4,2)$ & NO & 11316\\
$(389,118)$ & 14 & $(300,91)$ & 13 & 1 & YES & YES & YES & $1.62$ & $(4,2)$ & NO & 11317\\
$(390,89)$ & 13 & $(3,1)$ & 2 & 3 & YES & YES & YES & $1.50$ & $(2,3)$ & NO & 11318\\
$(390,89)$ & 13 & $(3,1)$ & 2 & 3 & YES & YES & YES & $1.62$ & $(2,3)$ & -- & 11319\\
$(391,91)$ & 14 & $(2,1)$ & 1 & 1 & YES & YES & YES & $1.70$ & $(2,3)$ & NO & 11320\\
$(391,105)$ & 13 & $(2,1)$ & 1 & 1 & YES & YES & YES & $1.71$ & $(2,3)$ & -- & 11321\\
$(391,108)$ & 13 & $(2,1)$ & 1 & 1 & YES & YES & YES & $1.71$ & $(2,3)$ & NO & 11322\\
$(391,109)$ & 13 & $(2,1)$ & 1 & 1 & YES & YES & YES & $1.78$ & $(2,3)$ & NO & 11323\\
$(391,91)$ & 14 & $(3,1)$ & 2 & 1 & YES & YES & YES & $1.50$ & $(2,3)$ & -- & 11324\\
$(391,91)$ & 14 & $(3,1)$ & 2 & 1 & YES & YES & YES & $1.62$ & $(2,3)$ & NO & 11325\\
$(391,91)$ & 14 & $(3,1)$ & 2 & 1 & YES & YES & YES & $1.71$ & $(2,3)$ & NO & 11326\\
$(391,105)$ & 13 & $(3,1)$ & 2 & 1 & YES & YES & YES & $1.50$ & $(2,3)$ & -- & 11327\\
$(391,105)$ & 13 & $(3,1)$ & 2 & 1 & YES & YES & YES & $1.62$ & $(2,3)$ & NO & 11328\\
$(391,105)$ & 13 & $(3,1)$ & 2 & 1 & YES & YES & YES & $2.00$ & $(2,3)$ & NO & 11329\\
$(391,108)$ & 13 & $(3,1)$ & 2 & 1 & YES & YES & YES & $1.43$ & $(2,3)$ & -- & 11330\\
$(391,108)$ & 13 & $(3,1)$ & 2 & 1 & YES & YES & YES & $1.57$ & $(2,3)$ & NO & 11331\\
$(391,109)$ & 13 & $(3,1)$ & 2 & 1 & YES & YES & YES & $1.50$ & $(4,2)$ & -- & 11332\\
$(391,109)$ & 13 & $(3,1)$ & 2 & 1 & YES & YES & YES & $1.86$ & $(2,3)$ & NO & 11333\\
$(391,91)$ & 14 & $(4,1)$ & 3 & 1 & YES & YES & YES & $1.29$ & $(4,2)$ & -- & 11334\\
$(391,105)$ & 13 & $(4,1)$ & 3 & 1 & YES & YES & YES & $1.62$ & $(2,3)$ & -- & 11335\\
$(391,109)$ & 13 & $(4,1)$ & 3 & 1 & YES & YES & YES & $1.67$ & $(4,2)$ & NO & 11336\\
$(391,109)$ & 13 & $(4,1)$ & 3 & 1 & YES & YES & YES & $1.50$ & $(2,3)$ & -- & 11337\\
$(391,59)$ & 17 & $(7,1)$ & 6 & 1 & YES & YES & NO(2) & $1.62$ & $(6,1)$ & NO & 11338\\
$(391,105)$ & 13 & $(11,3)$ & 5 & 1 & YES & YES & YES & $1.86$ & $(2,3)$ & NO & 11339\\
$(391,109)$ & 13 & $(11,3)$ & 5 & 1 & YES & YES & YES & $1.56$ & $(2,3)$ & NO & 11340\\
$(391,108)$ & 13 & $(18,5)$ & 6 & 1 & YES & YES & YES & $1.71$ & $(2,3)$ & NO & 11341\\
$(391,109)$ & 13 & $(18,5)$ & 6 & 1 & YES & YES & YES & $1.57$ & $(4,2)$ & NO & 11342\\
$(391,59)$ & 17 & $(20,3)$ & 8 & 1 & YES & YES & NO(2) & $1.62$ & $(6,1)$ & NO & 11343\\
$(391,91)$ & 14 & $(30,7)$ & 8 & 1 & YES & YES & YES & $1.70$ & $(2,3)$ & NO & 11344\\
$(391,91)$ & 14 & $(73,17)$ & 10 & 1 & YES & YES & YES & $1.43$ & $(2,3)$ & NO & 11345\\
$(391,108)$ & 13 & $(76,21)$ & 9 & 1 & YES & YES & YES & $1.57$ & $(2,3)$ & NO & 11346\\
$(391,109)$ & 13 & $(104,29)$ & 10 & 1 & YES & YES & YES & $1.43$ & $(2,3)$ & NO & 11347\\
$(391,91)$ & 14 & $(116,27)$ & 11 & 1 & YES & YES & YES & $1.43$ & $(4,2)$ & NO & 11348\\
$(391,145)$ & 13 & $(151,56)$ & 11 & 1 & YES & YES & YES & $1.62$ & $(4,2)$ & NO & 11349\\
$(392,83)$ & 14 & $(2,1)$ & 1 & 2 & YES & YES & YES & $1.67$ & $(2,3)$ & NO & 11350\\
$(392,83)$ & 14 & $(2,1)$ & 1 & 2 & YES & YES & YES & $1.67$ & $(2,3)$ & -- & 11351\\
$(392,83)$ & 14 & $(4,1)$ & 3 & 4 & YES & YES & YES & $1.50$ & $(2,3)$ & NO & 11352\\
$(392,83)$ & 14 & $(4,1)$ & 3 & 4 & YES & YES & YES & $1.75$ & $(2,3)$ & -- & 11353\\
$(392,85)$ & 14 & $(4,1)$ & 3 & 4 & YES & YES & YES & $1.80$ & $(2,3)$ & NO & 11354\\
$(392,83)$ & 14 & $(14,3)$ & 6 & 14 & YES & YES & YES & $1.89$ & $(2,3)$ & NO & 11355\\
$(393,73)$ & 15 & $(2,1)$ & 1 & 1 & YES & YES & YES & $1.43$ & $(4,2)$ & -- & 11356\\
$(393,73)$ & 15 & $(2,1)$ & 1 & 1 & YES & YES & YES & $1.57$ & $(4,2)$ & NO & 11357\\
$(393,106)$ & 13 & $(2,1)$ & 1 & 1 & YES & YES & YES & $1.56$ & $(2,3)$ & -- & 11358\\
$(393,116)$ & 13 & $(2,1)$ & 1 & 1 & YES & YES & YES & $1.71$ & $(2,3)$ & NO & 11359\\
$(393,116)$ & 13 & $(2,1)$ & 1 & 1 & YES & YES & YES & $1.60$ & $(2,3)$ & -- & 11360\\
$(393,70)$ & 15 & $(3,1)$ & 2 & 3 & YES & YES & YES & $1.57$ & $(4,2)$ & -- & 11361\\
$(393,106)$ & 13 & $(3,1)$ & 2 & 3 & YES & YES & YES & $1.50$ & $(2,3)$ & -- & 11362\\
$(393,106)$ & 13 & $(3,1)$ & 2 & 3 & YES & YES & YES & $1.89$ & $(2,3)$ & NO & 11363\\
$(393,106)$ & 13 & $(3,1)$ & 2 & 3 & YES & YES & YES & $1.62$ & $(2,3)$ & NO & 11364\\
$(393,116)$ & 13 & $(3,1)$ & 2 & 3 & YES & YES & YES & $1.80$ & $(2,3)$ & NO & 11365\\
$(393,152)$ & 13 & $(4,1)$ & 3 & 1 & YES & YES & YES & $1.62$ & $(4,2)$ & -- & 11366\\
$(393,70)$ & 15 & $(5,1)$ & 4 & 1 & YES & YES & YES & $1.57$ & $(4,2)$ & NO & 11367\\
$(393,106)$ & 13 & $(7,2)$ & 4 & 1 & YES & YES & YES & $1.62$ & $(2,3)$ & NO & 11368\\
$(393,116)$ & 13 & $(10,3)$ & 5 & 1 & YES & YES & YES & $1.57$ & $(2,3)$ & NO & 11369\\
$(393,152)$ & 13 & $(13,5)$ & 5 & 1 & YES & YES & YES & $1.62$ & $(4,2)$ & NO & 11370\\
$(393,116)$ & 13 & $(61,18)$ & 9 & 1 & YES & YES & YES & $1.43$ & $(4,2)$ & 10085 & 11371\\
$(393,106)$ & 13 & $(63,17)$ & 9 & 3 & YES & YES & YES & $1.90$ & $(2,3)$ & NO & 11372\\
$(393,106)$ & 13 & $(89,24)$ & 10 & 1 & YES & YES & YES & $1.67$ & $(4,2)$ & 10747 & 11373\\
$(393,116)$ & 13 & $(166,49)$ & 11 & 1 & YES & YES & YES & $1.80$ & $(2,3)$ & NO & 11374\\
$(393,116)$ & 13 & $(227,67)$ & 12 & 1 & YES & YES & YES & $1.57$ & $(2,3)$ & NO & 11375\\
$(394,151)$ & 13 & $(2,1)$ & 1 & 2 & YES & YES & YES & $1.78$ & $(4,2)$ & -- & 11376\\
$(394,165)$ & 13 & $(2,1)$ & 1 & 2 & YES & YES & YES & $2.00$ & $(4,2)$ & -- & 11377\\
$(394,117)$ & 13 & $(3,1)$ & 2 & 1 & YES & YES & YES & $1.78$ & $(4,2)$ & NO & 11378\\
$(394,117)$ & 13 & $(10,3)$ & 5 & 2 & YES & YES & YES & $1.78$ & $(4,2)$ & NO & 11379\\
$(394,151)$ & 13 & $(47,18)$ & 8 & 1 & YES & YES & YES & $1.78$ & $(4,2)$ & 8662 & 11380\\
$(394,165)$ & 13 & $(117,49)$ & 10 & 1 & YES & YES & YES & $1.89$ & $(4,2)$ & NO & 11381\\
$(395,73)$ & 15 & $(3,1)$ & 2 & 1 & YES & YES & NO(2) & $1.50$ & $(6,1)$ & NO & 11382\\
$(395,73)$ & 15 & $(3,1)$ & 2 & 1 & YES & YES & NO(2) & $1.50$ & $(6,1)$ & -- & 11383\\
$(395,73)$ & 15 & $(4,1)$ & 3 & 1 & YES & YES & NO(2) & $1.67$ & $(2,3)$ & NO & 11384\\
$(395,73)$ & 15 & $(6,1)$ & 5 & 1 & YES & YES & NO(2) & $1.56$ & $(8,0)$ & NO & 11385\\
$(396,109)$ & 13 & $(2,1)$ & 1 & 2 & YES & YES & YES & $1.71$ & $(2,3)$ & -- & 11386\\
$(396,109)$ & 13 & $(2,1)$ & 1 & 2 & YES & YES & YES & $1.43$ & $(2,3)$ & NO & 11387\\
$(396,109)$ & 13 & $(7,2)$ & 4 & 1 & YES & YES & YES & $1.43$ & $(2,3)$ & NO & 11388\\
$(397,116)$ & 13 & $(2,1)$ & 1 & 1 & YES & YES & YES & $1.62$ & $(4,2)$ & -- & 11389\\
$(397,116)$ & 13 & $(2,1)$ & 1 & 1 & YES & YES & YES & $1.71$ & $(2,3)$ & NO & 11390\\
$(397,116)$ & 13 & $(3,1)$ & 2 & 1 & YES & YES & YES & $1.43$ & $(2,3)$ & -- & 11391\\
$(397,93)$ & 14 & $(4,1)$ & 3 & 1 & YES & YES & YES & $1.90$ & $(2,3)$ & -- & 11392\\
$(397,116)$ & 13 & $(4,1)$ & 3 & 1 & YES & YES & YES & $1.86$ & $(2,3)$ & NO & 11393\\
$(397,151)$ & 13 & $(4,1)$ & 3 & 1 & YES & YES & YES & $1.62$ & $(4,2)$ & -- & 11394\\
$(397,90)$ & 15 & $(5,1)$ & 4 & 1 & YES & YES & YES & $1.90$ & $(2,3)$ & NO & 11395\\
$(397,93)$ & 14 & $(5,1)$ & 4 & 1 & YES & YES & YES & $1.80$ & $(2,3)$ & NO & 11396\\
$(397,146)$ & 13 & $(5,1)$ & 4 & 1 & YES & YES & YES & $1.78$ & $(4,2)$ & NO & 11397\\
$(397,116)$ & 13 & $(17,5)$ & 6 & 1 & YES & YES & YES & $1.57$ & $(2,3)$ & 9055 & 11398\\
$(397,151)$ & 13 & $(21,8)$ & 6 & 1 & YES & YES & YES & $1.62$ & $(4,2)$ & 8807 & 11399\\
$(397,116)$ & 13 & $(24,7)$ & 7 & 1 & YES & YES & YES & $1.62$ & $(4,2)$ & 10553 & 11400\\
$(397,116)$ & 13 & $(65,19)$ & 9 & 1 & YES & YES & YES & $1.86$ & $(2,3)$ & NO & 11401\\
$(397,116)$ & 13 & $(89,26)$ & 10 & 1 & YES & YES & YES & $1.86$ & $(2,3)$ & 10760 & 11402\\
$(398,111)$ & 13 & $(2,1)$ & 1 & 2 & YES & YES & YES & $1.62$ & $(2,3)$ & -- & 11403\\
$(398,111)$ & 13 & $(2,1)$ & 1 & 2 & YES & YES & YES & $1.75$ & $(2,3)$ & NO & 11404\\
$(398,111)$ & 13 & $(3,1)$ & 2 & 1 & YES & YES & YES & $1.43$ & $(2,3)$ & -- & 11405\\
$(398,111)$ & 13 & $(4,1)$ & 3 & 2 & YES & YES & YES & $1.67$ & $(2,3)$ & NO & 11406\\
$(398,111)$ & 13 & $(7,2)$ & 4 & 1 & YES & YES & YES & $1.43$ & $(4,2)$ & NO & 11407\\
$(398,111)$ & 13 & $(18,5)$ & 6 & 2 & YES & YES & YES & $1.57$ & $(2,3)$ & NO & 11408\\
$(398,111)$ & 13 & $(61,17)$ & 9 & 1 & YES & YES & YES & $1.43$ & $(2,3)$ & 11580 & 11409\\
$(398,111)$ & 13 & $(104,29)$ & 10 & 2 & YES & YES & YES & $1.67$ & $(2,3)$ & 11021 & 11410\\
$(398,93)$ & 14 & $(398,93)$ & 14 & 398 & YES & YES & YES & $1.75$ & $(2,3)$ & NO & 11411\\
$(400,147)$ & 13 & $(2,1)$ & 1 & 2 & YES & YES & YES & $1.75$ & $(4,2)$ & -- & 11412\\
$(400,117)$ & 13 & $(3,1)$ & 2 & 1 & YES & YES & YES & $1.43$ & $(2,3)$ & NO & 11413\\
$(401,87)$ & 14 & $(2,1)$ & 1 & 1 & YES & YES & YES & $1.62$ & $(4,2)$ & -- & 11414\\
$(401,87)$ & 14 & $(2,1)$ & 1 & 1 & YES & YES & YES & $1.75$ & $(4,2)$ & NO & 11415\\
$(401,112)$ & 13 & $(2,1)$ & 1 & 1 & YES & YES & YES & $1.75$ & $(2,3)$ & NO & 11416\\
$(401,112)$ & 13 & $(2,1)$ & 1 & 1 & YES & YES & YES & $1.71$ & $(2,3)$ & -- & 11417\\
$(401,111)$ & 13 & $(3,1)$ & 2 & 1 & YES & YES & YES & $1.86$ & $(2,3)$ & NO & 11418\\
$(401,112)$ & 13 & $(3,1)$ & 2 & 1 & YES & YES & YES & $1.57$ & $(2,3)$ & -- & 11419\\
$(401,155)$ & 13 & $(3,1)$ & 2 & 1 & YES & YES & YES & $1.75$ & $(4,2)$ & NO & 11420\\
$(401,70)$ & 16 & $(4,1)$ & 3 & 1 & YES & YES & NO(2) & $1.60$ & $(6,1)$ & -- & 11421\\
$(401,87)$ & 14 & $(4,1)$ & 3 & 1 & YES & YES & YES & $1.62$ & $(4,2)$ & NO & 11422\\
$(401,111)$ & 13 & $(4,1)$ & 3 & 1 & YES & YES & YES & $1.57$ & $(2,3)$ & -- & 11423\\
$(401,112)$ & 13 & $(18,5)$ & 6 & 1 & YES & YES & YES & $1.57$ & $(2,3)$ & NO & 11424\\
$(401,112)$ & 13 & $(25,7)$ & 7 & 1 & YES & YES & YES & $1.71$ & $(2,3)$ & NO & 11425\\
$(402,169)$ & 13 & $(2,1)$ & 1 & 2 & NO & YES & YES & $1.89$ & $(2,3)$ & -- & 11426\\
$(402,157)$ & 13 & $(41,16)$ & 8 & 1 & YES & YES & YES & $1.78$ & $(4,2)$ & 8792 & 11427\\
$(403,92)$ & 14 & $(2,1)$ & 1 & 1 & YES & YES & YES & $1.57$ & $(2,3)$ & -- & 11428\\
$(403,92)$ & 14 & $(2,1)$ & 1 & 1 & YES & YES & YES & $1.71$ & $(2,3)$ & NO & 11429\\
$(403,111)$ & 13 & $(2,1)$ & 1 & 1 & YES & YES & YES & $1.50$ & $(2,3)$ & NO & 11430\\
$(403,111)$ & 13 & $(2,1)$ & 1 & 1 & YES & YES & YES & $1.71$ & $(2,3)$ & -- & 11431\\
$(403,87)$ & 14 & $(3,1)$ & 2 & 1 & YES & YES & YES & $1.90$ & $(2,3)$ & -- & 11432\\
$(403,92)$ & 14 & $(3,1)$ & 2 & 1 & YES & YES & YES & $1.50$ & $(2,3)$ & -- & 11433\\
$(403,111)$ & 13 & $(3,1)$ & 2 & 1 & YES & YES & YES & $1.57$ & $(2,3)$ & -- & 11434\\
$(403,87)$ & 14 & $(4,1)$ & 3 & 1 & YES & YES & YES & $1.70$ & $(2,3)$ & -- & 11435\\
$(403,88)$ & 14 & $(4,1)$ & 3 & 1 & YES & YES & YES & $1.62$ & $(4,2)$ & NO & 11436\\
$(403,111)$ & 13 & $(4,1)$ & 3 & 1 & YES & YES & YES & $1.50$ & $(4,2)$ & NO & 11437\\
$(403,122)$ & 14 & $(4,1)$ & 3 & 1 & YES & YES & YES & $1.62$ & $(4,2)$ & -- & 11438\\
$(403,92)$ & 14 & $(5,1)$ & 4 & 1 & YES & YES & YES & $1.62$ & $(2,3)$ & NO & 11439\\
$(403,119)$ & 13 & $(7,2)$ & 4 & 1 & YES & YES & YES & $1.71$ & $(2,3)$ & NO & 11440\\
$(403,87)$ & 14 & $(9,2)$ & 5 & 1 & YES & YES & YES & $1.71$ & $(2,3)$ & NO & 11441\\
$(403,119)$ & 13 & $(10,3)$ & 5 & 1 & YES & YES & YES & $1.57$ & $(2,3)$ & NO & 11442\\
$(403,111)$ & 13 & $(11,3)$ & 5 & 1 & YES & YES & YES & $1.67$ & $(4,2)$ & NO & 11443\\
$(403,92)$ & 14 & $(13,3)$ & 6 & 13 & YES & YES & YES & $1.71$ & $(2,3)$ & NO & 11444\\
$(403,92)$ & 14 & $(22,5)$ & 7 & 1 & YES & YES & YES & $1.57$ & $(2,3)$ & NO & 11445\\
$(403,87)$ & 14 & $(23,5)$ & 7 & 1 & YES & YES & YES & $1.50$ & $(2,3)$ & NO & 11446\\
$(403,88)$ & 14 & $(23,5)$ & 7 & 1 & YES & YES & YES & $1.62$ & $(4,2)$ & NO & 11447\\
$(403,87)$ & 14 & $(37,8)$ & 8 & 1 & YES & YES & YES & $1.62$ & $(4,2)$ & NO & 11448\\
$(403,92)$ & 14 & $(57,13)$ & 9 & 1 & YES & YES & YES & $1.57$ & $(2,3)$ & NO & 11449\\
$(403,111)$ & 13 & $(69,19)$ & 9 & 1 & YES & YES & YES & $1.67$ & $(4,2)$ & 10406 & 11450\\
$(403,111)$ & 13 & $(98,27)$ & 10 & 1 & YES & YES & YES & $1.57$ & $(2,3)$ & NO & 11451\\
$(403,119)$ & 13 & $(105,31)$ & 10 & 1 & YES & YES & YES & $1.57$ & $(2,3)$ & 11059 & 11452\\
$(403,92)$ & 14 & $(127,29)$ & 11 & 1 & YES & YES & YES & $1.71$ & $(2,3)$ & NO & 11453\\
$(403,92)$ & 14 & $(403,92)$ & 14 & 403 & YES & YES & YES & $1.57$ & $(2,3)$ & NO & 11454\\
$(404,91)$ & 14 & $(2,1)$ & 1 & 2 & YES & YES & YES & $1.67$ & $(2,3)$ & NO & 11455\\
$(404,153)$ & 13 & $(2,1)$ & 1 & 2 & YES & YES & YES & $1.78$ & $(4,2)$ & -- & 11456\\
$(404,91)$ & 14 & $(3,1)$ & 2 & 1 & YES & YES & YES & $1.57$ & $(2,3)$ & NO & 11457\\
$(404,91)$ & 14 & $(3,1)$ & 2 & 1 & YES & YES & YES & $1.71$ & $(2,3)$ & -- & 11458\\
$(404,111)$ & 14 & $(4,1)$ & 3 & 4 & YES & YES & YES & $1.88$ & $(4,2)$ & NO & 11459\\
$(404,91)$ & 14 & $(31,7)$ & 8 & 1 & YES & YES & YES & $1.80$ & $(2,3)$ & NO & 11460\\
$(406,73)$ & 15 & $(3,1)$ & 2 & 1 & YES & YES & NO(2) & $1.44$ & $(4,2)$ & -- & 11461\\
$(406,149)$ & 13 & $(109,40)$ & 10 & 1 & YES & YES & YES & $1.78$ & $(4,2)$ & NO & 11462\\
$(407,112)$ & 13 & $(2,1)$ & 1 & 1 & YES & YES & YES & $1.62$ & $(4,2)$ & NO & 11463\\
$(407,112)$ & 13 & $(2,1)$ & 1 & 1 & YES & YES & YES & $1.71$ & $(2,3)$ & -- & 11464\\
$(407,119)$ & 13 & $(2,1)$ & 1 & 1 & YES & YES & YES & $1.78$ & $(4,2)$ & -- & 11465\\
$(407,119)$ & 13 & $(2,1)$ & 1 & 1 & YES & YES & YES & $1.71$ & $(2,3)$ & NO & 11466\\
$(407,171)$ & 13 & $(2,1)$ & 1 & 1 & NO & YES & YES & $1.78$ & $(2,3)$ & -- & 11467\\
$(407,172)$ & 13 & $(2,1)$ & 1 & 1 & NO & YES & YES & $2.00$ & $(2,3)$ & -- & 11468\\
$(407,119)$ & 13 & $(3,1)$ & 2 & 1 & YES & YES & YES & $1.71$ & $(2,3)$ & -- & 11469\\
$(407,119)$ & 13 & $(3,1)$ & 2 & 1 & YES & YES & YES & $1.89$ & $(2,3)$ & NO & 11470\\
$(407,119)$ & 13 & $(4,1)$ & 3 & 1 & YES & YES & YES & $1.57$ & $(2,3)$ & -- & 11471\\
$(407,119)$ & 13 & $(4,1)$ & 3 & 1 & YES & YES & YES & $1.50$ & $(4,2)$ & NO & 11472\\
$(407,119)$ & 13 & $(17,5)$ & 6 & 1 & YES & YES & YES & $1.70$ & $(2,3)$ & NO & 11473\\
$(407,119)$ & 13 & $(41,12)$ & 8 & 1 & YES & YES & YES & $1.71$ & $(2,3)$ & 8860 & 11474\\
$(407,119)$ & 13 & $(65,19)$ & 9 & 1 & YES & YES & YES & $1.78$ & $(2,3)$ & 10335 & 11475\\
$(407,112)$ & 13 & $(109,30)$ & 10 & 1 & YES & YES & YES & $1.71$ & $(2,3)$ & 11145 & 11476\\
$(407,119)$ & 13 & $(171,50)$ & 11 & 1 & YES & YES & YES & $1.71$ & $(2,3)$ & NO & 11477\\
$(409,97)$ & 14 & $(2,1)$ & 1 & 1 & YES & YES & YES & $1.57$ & $(4,2)$ & -- & 11478\\
$(409,121)$ & 13 & $(2,1)$ & 1 & 1 & YES & YES & YES & $1.71$ & $(2,3)$ & -- & 11479\\
$(409,121)$ & 13 & $(3,1)$ & 2 & 1 & YES & YES & YES & $1.57$ & $(2,3)$ & -- & 11480\\
$(409,121)$ & 13 & $(10,3)$ & 5 & 1 & YES & YES & YES & $1.62$ & $(4,2)$ & NO & 11481\\
$(409,97)$ & 14 & $(17,4)$ & 7 & 1 & YES & YES & YES & $1.57$ & $(4,2)$ & 11307 & 11482\\
$(409,121)$ & 13 & $(17,5)$ & 6 & 1 & YES & YES & YES & $1.71$ & $(2,3)$ & NO & 11483\\
$(409,121)$ & 13 & $(169,50)$ & 11 & 1 & YES & YES & YES & $1.57$ & $(2,3)$ & NO & 11484\\
$(411,97)$ & 15 & $(2,1)$ & 1 & 1 & YES & YES & NO(2) & $1.50$ & $(6,1)$ & NO & 11485\\
$(413,121)$ & 13 & $(2,1)$ & 1 & 1 & YES & YES & YES & $1.57$ & $(2,3)$ & -- & 11486\\
$(413,121)$ & 13 & $(2,1)$ & 1 & 1 & YES & YES & YES & $1.43$ & $(2,3)$ & NO & 11487\\
$(413,157)$ & 13 & $(2,1)$ & 1 & 1 & YES & YES & YES & $1.62$ & $(4,2)$ & -- & 11488\\
$(413,121)$ & 13 & $(3,1)$ & 2 & 1 & YES & YES & YES & $1.43$ & $(2,3)$ & -- & 11489\\
$(413,157)$ & 13 & $(3,1)$ & 2 & 1 & YES & YES & YES & $1.75$ & $(4,2)$ & 9050 & 11490\\
$(413,78)$ & 15 & $(21,4)$ & 8 & 7 & YES & YES & YES & $1.75$ & $(2,3)$ & NO & 11491\\
$(413,121)$ & 13 & $(41,12)$ & 8 & 1 & YES & YES & YES & $1.57$ & $(2,3)$ & 11245 & 11492\\
$(413,121)$ & 13 & $(58,17)$ & 9 & 1 & YES & YES & YES & $1.57$ & $(2,3)$ & NO & 11493\\
$(413,90)$ & 15 & $(101,22)$ & 11 & 1 & YES & YES & YES & $1.62$ & $(4,2)$ & NO & 11494\\
$(413,121)$ & 13 & $(157,46)$ & 11 & 1 & YES & YES & YES & $1.57$ & $(2,3)$ & NO & 11495\\
$(415,116)$ & 13 & $(2,1)$ & 1 & 1 & YES & YES & YES & $1.71$ & $(2,3)$ & -- & 11496\\
$(415,116)$ & 13 & $(2,1)$ & 1 & 1 & YES & YES & YES & $1.86$ & $(2,3)$ & NO & 11497\\
$(415,116)$ & 13 & $(3,1)$ & 2 & 1 & YES & YES & YES & $1.86$ & $(2,3)$ & NO & 11498\\
$(415,116)$ & 13 & $(3,1)$ & 2 & 1 & YES & YES & YES & $1.86$ & $(2,3)$ & -- & 11499\\
$(415,116)$ & 13 & $(18,5)$ & 6 & 1 & YES & YES & YES & $1.57$ & $(2,3)$ & 9298 & 11500\\
$(416,95)$ & 14 & $(2,1)$ & 1 & 2 & YES & YES & YES & $1.62$ & $(4,2)$ & NO & 11501\\
$(416,115)$ & 13 & $(2,1)$ & 1 & 2 & YES & YES & YES & $1.57$ & $(2,3)$ & -- & 11502\\
$(416,115)$ & 13 & $(2,1)$ & 1 & 2 & YES & YES & YES & $1.57$ & $(2,3)$ & NO & 11503\\
$(416,115)$ & 13 & $(3,1)$ & 2 & 1 & YES & YES & YES & $1.57$ & $(2,3)$ & NO & 11504\\
$(416,115)$ & 13 & $(3,1)$ & 2 & 1 & YES & YES & YES & $1.57$ & $(2,3)$ & -- & 11505\\
$(416,115)$ & 13 & $(18,5)$ & 6 & 2 & YES & YES & YES & $1.57$ & $(2,3)$ & 9014 & 11506\\
$(416,115)$ & 13 & $(29,8)$ & 7 & 1 & YES & YES & YES & $1.57$ & $(2,3)$ & NO & 11507\\
$(417,176)$ & 13 & $(2,1)$ & 1 & 1 & YES & YES & YES & $1.67$ & $(4,2)$ & -- & 11508\\
$(417,79)$ & 15 & $(5,2)$ & 3 & 1 & YES & YES & YES & $1.62$ & $(4,2)$ & -- & 11509\\
$(417,65)$ & 16 & $(13,2)$ & 7 & 1 & YES & YES & NO(2) & $1.56$ & $(4,2)$ & NO & 11510\\
$(417,79)$ & 15 & $(26,5)$ & 9 & 1 & YES & YES & YES & $1.62$ & $(4,2)$ & NO & 11511\\
$(418,163)$ & 13 & $(2,1)$ & 1 & 2 & YES & YES & YES & $1.88$ & $(4,2)$ & NO & 11512\\
$(418,163)$ & 13 & $(4,1)$ & 3 & 2 & YES & YES & YES & $1.78$ & $(4,2)$ & -- & 11513\\
$(418,159)$ & 13 & $(8,3)$ & 4 & 2 & YES & YES & YES & $1.62$ & $(4,2)$ & NO & 11514\\
$(419,89)$ & 14 & $(2,1)$ & 1 & 1 & YES & YES & YES & $1.56$ & $(2,3)$ & -- & 11515\\
$(419,89)$ & 14 & $(2,1)$ & 1 & 1 & YES & YES & YES & $1.80$ & $(2,3)$ & NO & 11516\\
$(419,155)$ & 13 & $(2,1)$ & 1 & 1 & YES & YES & YES & $1.88$ & $(4,2)$ & -- & 11517\\
$(419,89)$ & 14 & $(3,1)$ & 2 & 1 & YES & YES & YES & $1.57$ & $(4,2)$ & -- & 11518\\
$(419,89)$ & 14 & $(3,1)$ & 2 & 1 & YES & YES & YES & $1.67$ & $(4,2)$ & NO & 11519\\
$(419,89)$ & 14 & $(3,1)$ & 2 & 1 & YES & YES & YES & $1.78$ & $(4,2)$ & NO & 11520\\
$(419,113)$ & 14 & $(3,1)$ & 2 & 1 & YES & YES & YES & $1.67$ & $(4,2)$ & -- & 11521\\
$(419,117)$ & 13 & $(3,1)$ & 2 & 1 & YES & YES & YES & $1.67$ & $(4,2)$ & -- & 11522\\
$(419,89)$ & 14 & $(4,1)$ & 3 & 1 & YES & YES & YES & $1.90$ & $(2,3)$ & NO & 11523\\
$(419,89)$ & 14 & $(14,3)$ & 6 & 1 & YES & YES & YES & $1.43$ & $(4,2)$ & NO & 11524\\
$(419,116)$ & 13 & $(18,5)$ & 6 & 1 & YES & YES & YES & $1.67$ & $(4,2)$ & NO & 11525\\
$(419,113)$ & 14 & $(37,10)$ & 8 & 1 & YES & YES & YES & $1.78$ & $(4,2)$ & NO & 11526\\
$(421,98)$ & 14 & $(2,1)$ & 1 & 1 & YES & YES & YES & $1.71$ & $(2,3)$ & NO & 11527\\
$(421,98)$ & 14 & $(2,1)$ & 1 & 1 & YES & YES & YES & $1.71$ & $(2,3)$ & -- & 11528\\
$(421,98)$ & 14 & $(3,1)$ & 2 & 1 & YES & YES & YES & $1.71$ & $(2,3)$ & -- & 11529\\
$(421,98)$ & 14 & $(3,1)$ & 2 & 1 & YES & YES & YES & $1.71$ & $(2,3)$ & NO & 11530\\
$(421,98)$ & 14 & $(5,1)$ & 4 & 1 & YES & YES & YES & $1.57$ & $(2,3)$ & NO & 11531\\
$(421,98)$ & 14 & $(17,4)$ & 7 & 1 & YES & YES & YES & $1.71$ & $(2,3)$ & NO & 11532\\
$(421,98)$ & 14 & $(30,7)$ & 8 & 1 & YES & YES & YES & $1.75$ & $(2,3)$ & NO & 11533\\
$(421,98)$ & 14 & $(43,10)$ & 9 & 1 & YES & YES & YES & $1.43$ & $(2,3)$ & NO & 11534\\
$(421,98)$ & 14 & $(73,17)$ & 10 & 1 & YES & YES & YES & $1.71$ & $(2,3)$ & NO & 11535\\
$(423,97)$ & 14 & $(2,1)$ & 1 & 1 & YES & YES & YES & $1.50$ & $(4,2)$ & NO & 11536\\
$(423,97)$ & 14 & $(3,1)$ & 2 & 3 & YES & YES & YES & $1.70$ & $(2,3)$ & -- & 11537\\
$(423,97)$ & 14 & $(4,1)$ & 3 & 1 & YES & YES & NO(2) & $1.56$ & $(2,3)$ & NO & 11538\\
$(423,97)$ & 14 & $(61,14)$ & 10 & 1 & YES & YES & YES & $1.67$ & $(4,2)$ & 11629 & 11539\\
$(424,97)$ & 14 & $(3,1)$ & 2 & 1 & YES & YES & YES & $1.78$ & $(2,3)$ & NO & 11540\\
$(424,97)$ & 14 & $(3,1)$ & 2 & 1 & YES & YES & YES & $1.70$ & $(2,3)$ & -- & 11541\\
$(424,97)$ & 14 & $(48,11)$ & 9 & 8 & YES & YES & YES & $1.70$ & $(2,3)$ & NO & 11542\\
$(425,92)$ & 14 & $(2,1)$ & 1 & 1 & YES & YES & YES & $1.62$ & $(2,3)$ & NO & 11543\\
$(425,92)$ & 14 & $(2,1)$ & 1 & 1 & YES & YES & YES & $1.62$ & $(2,3)$ & -- & 11544\\
$(425,97)$ & 14 & $(2,1)$ & 1 & 1 & YES & YES & YES & $1.71$ & $(2,3)$ & -- & 11545\\
$(425,97)$ & 14 & $(3,1)$ & 2 & 1 & YES & YES & YES & $1.62$ & $(2,3)$ & -- & 11546\\
$(425,92)$ & 14 & $(4,1)$ & 3 & 1 & YES & YES & YES & $1.75$ & $(2,3)$ & NO & 11547\\
$(425,157)$ & 13 & $(5,1)$ & 4 & 5 & YES & YES & YES & $1.62$ & $(4,2)$ & -- & 11548\\
$(425,92)$ & 14 & $(9,2)$ & 5 & 1 & YES & YES & YES & $1.50$ & $(2,3)$ & NO & 11549\\
$(425,117)$ & 13 & $(11,3)$ & 5 & 1 & YES & YES & YES & $1.67$ & $(4,2)$ & NO & 11550\\
$(425,92)$ & 14 & $(14,3)$ & 6 & 1 & YES & YES & YES & $1.57$ & $(2,3)$ & NO & 11551\\
$(425,92)$ & 14 & $(23,5)$ & 7 & 1 & YES & YES & YES & $1.75$ & $(2,3)$ & NO & 11552\\
$(425,157)$ & 13 & $(157,58)$ & 11 & 1 & YES & YES & YES & $1.62$ & $(4,2)$ & NO & 11553\\
$(426,179)$ & 13 & $(2,1)$ & 1 & 2 & NO & YES & YES & $1.90$ & $(2,3)$ & -- & 11554\\
$(426,115)$ & 13 & $(3,1)$ & 2 & 3 & YES & YES & YES & $1.62$ & $(2,3)$ & -- & 11555\\
$(426,115)$ & 13 & $(26,7)$ & 7 & 2 & YES & YES & YES & $1.86$ & $(2,3)$ & NO & 11556\\
$(427,100)$ & 14 & $(3,1)$ & 2 & 1 & YES & YES & YES & $1.50$ & $(4,2)$ & -- & 11557\\
$(427,100)$ & 14 & $(9,2)$ & 5 & 1 & YES & YES & YES & $1.50$ & $(4,2)$ & NO & 11558\\
$(427,100)$ & 14 & $(17,4)$ & 7 & 1 & YES & YES & NO(2) & $1.56$ & $(2,3)$ & NO & 11559\\
$(429,97)$ & 14 & $(2,1)$ & 1 & 1 & YES & YES & YES & $1.57$ & $(2,3)$ & -- & 11560\\
$(429,97)$ & 14 & $(4,1)$ & 3 & 1 & YES & YES & YES & $1.57$ & $(2,3)$ & -- & 11561\\
$(429,97)$ & 14 & $(4,1)$ & 3 & 1 & YES & YES & YES & $1.57$ & $(2,3)$ & NO & 11562\\
$(429,97)$ & 14 & $(31,7)$ & 8 & 1 & YES & YES & YES & $1.75$ & $(4,2)$ & NO & 11563\\
$(429,115)$ & 14 & $(56,15)$ & 9 & 1 & YES & YES & YES & $1.78$ & $(4,2)$ & NO & 11564\\
$(430,119)$ & 13 & $(2,1)$ & 1 & 2 & YES & YES & YES & $1.67$ & $(4,2)$ & -- & 11565\\
$(431,165)$ & 13 & $(2,1)$ & 1 & 1 & YES & YES & YES & $1.89$ & $(4,2)$ & -- & 11566\\
$(432,179)$ & 13 & $(2,1)$ & 1 & 2 & NO & YES & YES & $1.57$ & $(6,1)$ & -- & 11567\\
$(432,167)$ & 13 & $(31,12)$ & 7 & 1 & YES & YES & YES & $1.67$ & $(4,2)$ & NO & 11568\\
$(432,167)$ & 13 & $(194,75)$ & 11 & 2 & YES & YES & YES & $1.67$ & $(4,2)$ & 11739 & 11569\\
$(433,80)$ & 15 & $(3,1)$ & 2 & 1 & YES & YES & YES & $1.29$ & $(6,1)$ & NO & 11570\\
$(434,121)$ & 13 & $(2,1)$ & 1 & 2 & YES & YES & YES & $1.71$ & $(2,3)$ & -- & 11571\\
$(434,121)$ & 13 & $(2,1)$ & 1 & 2 & YES & YES & YES & $1.57$ & $(2,3)$ & NO & 11572\\
$(434,121)$ & 13 & $(3,1)$ & 2 & 1 & YES & YES & YES & $1.43$ & $(2,3)$ & -- & 11573\\
$(434,121)$ & 13 & $(3,1)$ & 2 & 1 & YES & YES & YES & $1.71$ & $(2,3)$ & NO & 11574\\
$(434,121)$ & 13 & $(4,1)$ & 3 & 2 & YES & YES & YES & $1.86$ & $(2,3)$ & NO & 11575\\
$(434,101)$ & 14 & $(9,2)$ & 5 & 1 & YES & YES & YES & $1.71$ & $(2,3)$ & NO & 11576\\
$(434,101)$ & 14 & $(13,3)$ & 6 & 1 & YES & YES & YES & $1.43$ & $(4,2)$ & NO & 11577\\
$(434,101)$ & 14 & $(17,4)$ & 7 & 1 & YES & YES & YES & $1.62$ & $(2,3)$ & NO & 11578\\
$(434,121)$ & 13 & $(18,5)$ & 6 & 2 & YES & YES & YES & $1.71$ & $(2,3)$ & NO & 11579\\
$(434,121)$ & 13 & $(43,12)$ & 8 & 1 & YES & YES & YES & $1.43$ & $(2,3)$ & 11409 & 11580\\
$(434,121)$ & 13 & $(61,17)$ & 9 & 1 & YES & YES & YES & $1.43$ & $(2,3)$ & NO & 11581\\
$(434,121)$ & 13 & $(165,46)$ & 11 & 1 & YES & YES & YES & $1.43$ & $(2,3)$ & NO & 11582\\
$(436,103)$ & 15 & $(5,1)$ & 4 & 1 & YES & YES & YES & $1.90$ & $(2,3)$ & NO & 11583\\
$(436,103)$ & 15 & $(13,3)$ & 6 & 1 & YES & YES & YES & $1.75$ & $(4,2)$ & NO & 11584\\
$(437,100)$ & 14 & $(2,1)$ & 1 & 1 & YES & YES & YES & $1.62$ & $(2,3)$ & NO & 11585\\
$(437,100)$ & 14 & $(2,1)$ & 1 & 1 & YES & YES & YES & $1.75$ & $(2,3)$ & -- & 11586\\
$(437,183)$ & 13 & $(2,1)$ & 1 & 1 & NO & YES & YES & $1.89$ & $(2,3)$ & -- & 11587\\
$(437,99)$ & 14 & $(3,1)$ & 2 & 1 & YES & YES & YES & $1.62$ & $(2,3)$ & NO & 11588\\
$(437,99)$ & 14 & $(3,1)$ & 2 & 1 & YES & YES & YES & $1.62$ & $(2,3)$ & -- & 11589\\
$(437,100)$ & 14 & $(3,1)$ & 2 & 1 & YES & YES & YES & $1.57$ & $(2,3)$ & -- & 11590\\
$(437,100)$ & 14 & $(3,1)$ & 2 & 1 & YES & YES & YES & $1.78$ & $(2,3)$ & NO & 11591\\
$(437,99)$ & 14 & $(5,1)$ & 4 & 1 & YES & YES & YES & $1.43$ & $(4,2)$ & 10324 & 11592\\
$(437,100)$ & 14 & $(5,1)$ & 4 & 1 & YES & YES & YES & $1.62$ & $(4,2)$ & NO & 11593\\
$(437,100)$ & 14 & $(9,2)$ & 5 & 1 & YES & YES & YES & $1.78$ & $(2,3)$ & NO & 11594\\
$(437,99)$ & 14 & $(13,3)$ & 6 & 1 & YES & YES & YES & $1.67$ & $(2,3)$ & NO & 11595\\
$(437,100)$ & 14 & $(13,3)$ & 6 & 1 & YES & YES & YES & $1.50$ & $(4,2)$ & NO & 11596\\
$(437,99)$ & 14 & $(31,7)$ & 8 & 1 & YES & YES & YES & $1.62$ & $(2,3)$ & 10403 & 11597\\
$(437,99)$ & 14 & $(75,17)$ & 10 & 1 & YES & YES & YES & $1.75$ & $(2,3)$ & NO & 11598\\
$(437,100)$ & 14 & $(83,19)$ & 10 & 1 & YES & YES & YES & $1.78$ & $(2,3)$ & NO & 11599\\
$(437,100)$ & 14 & $(437,100)$ & 14 & 437 & YES & YES & YES & $1.62$ & $(4,2)$ & NO & 11600\\
$(438,83)$ & 15 & $(2,1)$ & 1 & 2 & YES & YES & YES & $1.50$ & $(2,3)$ & NO & 11601\\
$(438,185)$ & 13 & $(2,1)$ & 1 & 2 & NO & YES & YES & $1.67$ & $(2,3)$ & -- & 11602\\
$(438,83)$ & 15 & $(16,3)$ & 7 & 2 & YES & YES & YES & $1.75$ & $(2,3)$ & NO & 11603\\
$(439,93)$ & 14 & $(3,1)$ & 2 & 1 & YES & YES & YES & $1.67$ & $(4,2)$ & NO & 11604\\
$(439,93)$ & 14 & $(3,1)$ & 2 & 1 & YES & YES & YES & $1.67$ & $(4,2)$ & -- & 11605\\
$(439,93)$ & 14 & $(3,1)$ & 2 & 1 & YES & YES & YES & $1.78$ & $(4,2)$ & NO & 11606\\
$(441,101)$ & 14 & $(3,1)$ & 2 & 3 & YES & YES & YES & $1.62$ & $(4,2)$ & -- & 11607\\
$(441,101)$ & 14 & $(5,1)$ & 4 & 1 & YES & YES & YES & $1.62$ & $(4,2)$ & NO & 11608\\
$(441,101)$ & 14 & $(83,19)$ & 10 & 1 & YES & YES & YES & $1.57$ & $(2,3)$ & NO & 11609\\
$(443,186)$ & 13 & $(2,1)$ & 1 & 1 & NO & YES & YES & $1.57$ & $(4,2)$ & -- & 11610\\
$(444,97)$ & 14 & $(23,5)$ & 7 & 1 & YES & YES & YES & $1.62$ & $(4,2)$ & NO & 11611\\
$(445,78)$ & 15 & $(2,1)$ & 1 & 1 & YES & YES & NO(2) & $1.60$ & $(6,1)$ & NO & 11612\\
$(445,187)$ & 13 & $(2,1)$ & 1 & 1 & NO & YES & YES & $1.89$ & $(4,2)$ & -- & 11613\\
$(445,188)$ & 13 & $(2,1)$ & 1 & 1 & NO & YES & YES & $1.89$ & $(2,3)$ & -- & 11614\\
$(445,104)$ & 14 & $(3,1)$ & 2 & 1 & YES & YES & YES & $1.62$ & $(2,3)$ & -- & 11615\\
$(445,78)$ & 15 & $(57,10)$ & 10 & 1 & YES & YES & NO(2) & $1.60$ & $(6,1)$ & NO & 11616\\
$(447,80)$ & 15 & $(3,1)$ & 2 & 3 & YES & YES & YES & $1.62$ & $(4,2)$ & -- & 11617\\
$(447,80)$ & 15 & $(4,1)$ & 3 & 1 & YES & YES & YES & $1.29$ & $(6,1)$ & NO & 11618\\
$(447,80)$ & 15 & $(39,7)$ & 9 & 3 & YES & YES & YES & $1.62$ & $(4,2)$ & NO & 11619\\
$(447,121)$ & 14 & $(48,13)$ & 9 & 3 & YES & YES & YES & $1.75$ & $(4,2)$ & NO & 11620\\
$(448,97)$ & 14 & $(2,1)$ & 1 & 2 & YES & YES & YES & $1.71$ & $(2,3)$ & NO & 11621\\
$(448,97)$ & 14 & $(2,1)$ & 1 & 2 & YES & YES & YES & $1.86$ & $(2,3)$ & -- & 11622\\
$(448,97)$ & 14 & $(3,1)$ & 2 & 1 & YES & YES & YES & $1.62$ & $(2,3)$ & -- & 11623\\
$(448,97)$ & 14 & $(23,5)$ & 7 & 1 & YES & YES & YES & $1.71$ & $(2,3)$ & NO & 11624\\
$(449,105)$ & 14 & $(2,1)$ & 1 & 1 & YES & YES & YES & $1.71$ & $(2,3)$ & -- & 11625\\
$(449,105)$ & 14 & $(3,1)$ & 2 & 1 & YES & YES & YES & $1.71$ & $(2,3)$ & -- & 11626\\
$(449,126)$ & 14 & $(4,1)$ & 3 & 1 & YES & YES & YES & $1.75$ & $(4,2)$ & NO & 11627\\
$(449,103)$ & 14 & $(9,2)$ & 5 & 1 & YES & YES & YES & $1.67$ & $(4,2)$ & NO & 11628\\
$(449,103)$ & 14 & $(48,11)$ & 9 & 1 & YES & YES & YES & $1.67$ & $(4,2)$ & 11539 & 11629\\
$(449,105)$ & 14 & $(449,105)$ & 14 & 449 & YES & YES & YES & $1.62$ & $(2,3)$ & NO & 11630\\
$(451,84)$ & 15 & $(2,1)$ & 1 & 1 & YES & YES & YES & $1.50$ & $(4,2)$ & NO & 11631\\
$(451,84)$ & 15 & $(3,1)$ & 2 & 1 & YES & YES & YES & $1.70$ & $(2,3)$ & NO & 11632\\
$(451,105)$ & 14 & $(3,1)$ & 2 & 1 & YES & YES & YES & $1.62$ & $(2,3)$ & -- & 11633\\
$(451,105)$ & 14 & $(3,1)$ & 2 & 1 & YES & YES & YES & $1.62$ & $(2,3)$ & NO & 11634\\
$(451,84)$ & 15 & $(27,5)$ & 8 & 1 & YES & YES & YES & $1.78$ & $(2,3)$ & NO & 11635\\
$(451,84)$ & 15 & $(59,11)$ & 10 & 1 & YES & YES & YES & $1.67$ & $(4,2)$ & NO & 11636\\
$(452,79)$ & 15 & $(2,1)$ & 1 & 2 & YES & YES & NO(2) & $1.60$ & $(6,1)$ & NO & 11637\\
$(452,79)$ & 15 & $(17,3)$ & 7 & 1 & YES & YES & NO(2) & $1.60$ & $(6,1)$ & NO & 11638\\
$(455,69)$ & 17 & $(3,1)$ & 2 & 1 & YES & YES & NO(2) & $1.38$ & $(6,1)$ & -- & 11639\\
$(455,106)$ & 14 & $(3,1)$ & 2 & 1 & YES & YES & YES & $1.57$ & $(4,2)$ & -- & 11640\\
$(455,106)$ & 14 & $(3,1)$ & 2 & 1 & YES & YES & YES & $1.67$ & $(2,3)$ & NO & 11641\\
$(455,106)$ & 14 & $(17,4)$ & 7 & 1 & YES & YES & YES & $1.75$ & $(2,3)$ & NO & 11642\\
$(457,133)$ & 14 & $(2,1)$ & 1 & 1 & YES & YES & YES & $1.50$ & $(4,2)$ & -- & 11643\\
$(458,97)$ & 15 & $(2,1)$ & 1 & 2 & YES & YES & YES & $1.62$ & $(4,2)$ & NO & 11644\\
$(458,97)$ & 15 & $(14,3)$ & 6 & 2 & YES & YES & YES & $1.62$ & $(4,2)$ & NO & 11645\\
$(459,104)$ & 14 & $(9,2)$ & 5 & 9 & YES & YES & YES & $1.62$ & $(4,2)$ & NO & 11646\\
$(459,104)$ & 14 & $(22,5)$ & 7 & 1 & YES & YES & YES & $1.50$ & $(4,2)$ & NO & 11647\\
$(461,82)$ & 15 & $(2,1)$ & 1 & 1 & YES & YES & NO(2) & $1.56$ & $(4,2)$ & -- & 11648\\
$(461,98)$ & 14 & $(2,1)$ & 1 & 1 & YES & YES & YES & $1.43$ & $(4,2)$ & NO & 11649\\
$(461,98)$ & 14 & $(2,1)$ & 1 & 1 & YES & YES & YES & $1.70$ & $(2,3)$ & -- & 11650\\
$(461,105)$ & 14 & $(2,1)$ & 1 & 1 & YES & YES & YES & $1.67$ & $(4,2)$ & NO & 11651\\
$(461,108)$ & 14 & $(2,1)$ & 1 & 1 & YES & YES & YES & $1.70$ & $(2,3)$ & -- & 11652\\
$(461,108)$ & 14 & $(2,1)$ & 1 & 1 & YES & YES & YES & $1.80$ & $(2,3)$ & NO & 11653\\
$(461,98)$ & 14 & $(4,1)$ & 3 & 1 & YES & YES & YES & $1.43$ & $(4,2)$ & NO & 11654\\
$(461,98)$ & 14 & $(33,7)$ & 8 & 1 & YES & YES & YES & $1.43$ & $(4,2)$ & NO & 11655\\
$(461,82)$ & 15 & $(45,8)$ & 9 & 1 & YES & YES & NO(2) & $1.62$ & $(6,1)$ & 10269 & 11656\\
$(461,105)$ & 14 & $(101,23)$ & 11 & 1 & YES & YES & YES & $1.67$ & $(4,2)$ & 11271 & 11657\\
$(463,100)$ & 14 & $(2,1)$ & 1 & 1 & YES & YES & YES & $1.62$ & $(4,2)$ & -- & 11658\\
$(463,100)$ & 14 & $(2,1)$ & 1 & 1 & YES & YES & YES & $1.62$ & $(4,2)$ & NO & 11659\\
$(463,129)$ & 14 & $(3,1)$ & 2 & 1 & YES & YES & YES & $1.67$ & $(4,2)$ & -- & 11660\\
$(463,179)$ & 13 & $(3,1)$ & 2 & 1 & YES & YES & YES & $1.89$ & $(4,2)$ & NO & 11661\\
$(463,100)$ & 14 & $(4,1)$ & 3 & 1 & YES & YES & YES & $1.62$ & $(4,2)$ & NO & 11662\\
$(463,100)$ & 14 & $(9,2)$ & 5 & 1 & YES & YES & YES & $1.62$ & $(4,2)$ & NO & 11663\\
$(463,179)$ & 13 & $(13,5)$ & 5 & 1 & YES & YES & YES & $1.78$ & $(4,2)$ & NO & 11664\\
$(463,129)$ & 14 & $(18,5)$ & 6 & 1 & YES & YES & YES & $1.78$ & $(4,2)$ & NO & 11665\\
$(463,100)$ & 14 & $(88,19)$ & 10 & 1 & YES & YES & YES & $1.62$ & $(4,2)$ & NO & 11666\\
$(463,129)$ & 14 & $(262,73)$ & 13 & 1 & YES & YES & YES & $1.67$ & $(4,2)$ & NO & 11667\\
$(464,105)$ & 14 & $(2,1)$ & 1 & 2 & YES & YES & YES & $1.62$ & $(2,3)$ & NO & 11668\\
$(464,105)$ & 14 & $(2,1)$ & 1 & 2 & YES & YES & YES & $1.71$ & $(2,3)$ & -- & 11669\\
$(465,106)$ & 14 & $(136,31)$ & 11 & 1 & YES & YES & YES & $1.50$ & $(4,2)$ & NO & 11670\\
$(466,109)$ & 14 & $(13,3)$ & 6 & 1 & YES & YES & YES & $1.80$ & $(2,3)$ & NO & 11671\\
$(466,109)$ & 14 & $(30,7)$ & 8 & 2 & YES & YES & YES & $1.80$ & $(2,3)$ & NO & 11672\\
$(467,181)$ & 13 & $(2,1)$ & 1 & 1 & NO & YES & YES & $1.57$ & $(4,2)$ & -- & 11673\\
$(469,107)$ & 14 & $(3,1)$ & 2 & 1 & YES & YES & YES & $1.86$ & $(2,3)$ & -- & 11674\\
$(469,107)$ & 14 & $(9,2)$ & 5 & 1 & YES & YES & YES & $1.67$ & $(2,3)$ & NO & 11675\\
$(469,107)$ & 14 & $(22,5)$ & 7 & 1 & YES & YES & YES & $1.43$ & $(4,2)$ & NO & 11676\\
$(469,107)$ & 14 & $(35,8)$ & 8 & 7 & YES & YES & YES & $1.75$ & $(2,3)$ & NO & 11677\\
$(473,108)$ & 14 & $(2,1)$ & 1 & 1 & YES & YES & YES & $1.71$ & $(2,3)$ & NO & 11678\\
$(473,108)$ & 14 & $(2,1)$ & 1 & 1 & YES & YES & YES & $1.43$ & $(2,3)$ & -- & 11679\\
$(473,108)$ & 14 & $(3,1)$ & 2 & 1 & YES & YES & YES & $1.57$ & $(2,3)$ & -- & 11680\\
$(473,196)$ & 13 & $(3,1)$ & 2 & 1 & YES & YES & YES & $1.89$ & $(4,2)$ & NO & 11681\\
$(473,84)$ & 15 & $(4,1)$ & 3 & 1 & YES & YES & NO(2) & $1.50$ & $(6,1)$ & -- & 11682\\
$(473,84)$ & 15 & $(45,8)$ & 9 & 1 & YES & YES & NO(2) & $1.60$ & $(6,1)$ & NO & 11683\\
$(473,108)$ & 14 & $(92,21)$ & 10 & 1 & YES & YES & YES & $1.57$ & $(2,3)$ & NO & 11684\\
$(474,133)$ & 14 & $(2,1)$ & 1 & 2 & YES & YES & YES & $1.62$ & $(4,2)$ & -- & 11685\\
$(474,133)$ & 14 & $(3,1)$ & 2 & 3 & YES & YES & YES & $1.62$ & $(4,2)$ & NO & 11686\\
$(474,83)$ & 16 & $(40,7)$ & 9 & 2 & YES & YES & NO(2) & $1.60$ & $(6,1)$ & 10251 & 11687\\
$(475,111)$ & 14 & $(77,18)$ & 10 & 1 & YES & YES & YES & $1.62$ & $(2,3)$ & NO & 11688\\
$(477,104)$ & 14 & $(2,1)$ & 1 & 1 & YES & YES & YES & $1.50$ & $(2,3)$ & NO & 11689\\
$(477,104)$ & 14 & $(2,1)$ & 1 & 1 & YES & YES & YES & $1.50$ & $(2,3)$ & -- & 11690\\
$(477,88)$ & 15 & $(3,1)$ & 2 & 3 & YES & YES & YES & $1.57$ & $(2,3)$ & NO & 11691\\
$(477,88)$ & 15 & $(3,1)$ & 2 & 3 & YES & YES & YES & $1.57$ & $(2,3)$ & -- & 11692\\
$(477,104)$ & 14 & $(4,1)$ & 3 & 1 & YES & YES & YES & $1.78$ & $(2,3)$ & NO & 11693\\
$(477,104)$ & 14 & $(14,3)$ & 6 & 1 & YES & YES & YES & $1.71$ & $(2,3)$ & NO & 11694\\
$(477,104)$ & 14 & $(23,5)$ & 7 & 1 & YES & YES & YES & $1.86$ & $(2,3)$ & NO & 11695\\
$(477,88)$ & 15 & $(27,5)$ & 8 & 9 & YES & YES & YES & $1.56$ & $(2,3)$ & NO & 11696\\
$(477,104)$ & 14 & $(55,12)$ & 9 & 1 & YES & YES & YES & $1.78$ & $(2,3)$ & NO & 11697\\
$(479,112)$ & 14 & $(2,1)$ & 1 & 1 & YES & YES & YES & $1.50$ & $(2,3)$ & NO & 11698\\
$(479,112)$ & 14 & $(2,1)$ & 1 & 1 & YES & YES & YES & $1.56$ & $(2,3)$ & -- & 11699\\
$(479,112)$ & 14 & $(13,3)$ & 6 & 1 & YES & YES & YES & $1.75$ & $(2,3)$ & NO & 11700\\
$(479,112)$ & 14 & $(47,11)$ & 9 & 1 & YES & YES & YES & $1.75$ & $(2,3)$ & 9862 & 11701\\
$(482,101)$ & 15 & $(2,1)$ & 1 & 2 & YES & YES & YES & $1.75$ & $(4,2)$ & NO & 11702\\
$(482,101)$ & 15 & $(62,13)$ & 10 & 2 & YES & YES & YES & $1.75$ & $(4,2)$ & NO & 11703\\
$(484,109)$ & 15 & $(3,1)$ & 2 & 1 & YES & YES & YES & $1.78$ & $(4,2)$ & -- & 11704\\
$(484,109)$ & 15 & $(71,16)$ & 10 & 1 & YES & YES & YES & $1.67$ & $(4,2)$ & NO & 11705\\
$(484,109)$ & 15 & $(484,109)$ & 15 & 484 & YES & YES & YES & $1.67$ & $(4,2)$ & NO & 11706\\
$(487,111)$ & 14 & $(4,1)$ & 3 & 1 & YES & YES & YES & $1.71$ & $(2,3)$ & -- & 11707\\
$(487,111)$ & 14 & $(5,1)$ & 4 & 1 & YES & YES & YES & $1.71$ & $(2,3)$ & NO & 11708\\
$(489,112)$ & 14 & $(2,1)$ & 1 & 1 & YES & YES & YES & $1.75$ & $(4,2)$ & NO & 11709\\
$(489,112)$ & 14 & $(3,1)$ & 2 & 3 & YES & YES & YES & $1.57$ & $(2,3)$ & -- & 11710\\
$(489,112)$ & 14 & $(3,1)$ & 2 & 3 & YES & YES & YES & $1.86$ & $(2,3)$ & NO & 11711\\
$(489,112)$ & 14 & $(22,5)$ & 7 & 1 & YES & YES & YES & $1.57$ & $(2,3)$ & NO & 11712\\
$(489,112)$ & 14 & $(35,8)$ & 8 & 1 & YES & YES & YES & $1.50$ & $(4,2)$ & NO & 11713\\
$(491,111)$ & 15 & $(2,1)$ & 1 & 1 & YES & YES & YES & $1.62$ & $(4,2)$ & -- & 11714\\
$(491,111)$ & 15 & $(2,1)$ & 1 & 1 & YES & YES & YES & $1.75$ & $(4,2)$ & NO & 11715\\
$(491,88)$ & 15 & $(5,1)$ & 4 & 1 & YES & YES & YES & $1.67$ & $(4,2)$ & NO & 11716\\
$(491,111)$ & 15 & $(5,1)$ & 4 & 1 & YES & YES & YES & $1.62$ & $(4,2)$ & -- & 11717\\
$(491,111)$ & 15 & $(22,5)$ & 7 & 1 & YES & YES & YES & $1.75$ & $(4,2)$ & NO & 11718\\
$(491,88)$ & 15 & $(39,7)$ & 9 & 1 & YES & YES & YES & $1.75$ & $(4,2)$ & NO & 11719\\
$(491,111)$ & 15 & $(115,26)$ & 11 & 1 & YES & YES & YES & $1.75$ & $(4,2)$ & NO & 11720\\
$(494,115)$ & 14 & $(2,1)$ & 1 & 2 & YES & YES & YES & $1.57$ & $(2,3)$ & NO & 11721\\
$(494,147)$ & 14 & $(3,1)$ & 2 & 1 & YES & YES & YES & $1.78$ & $(4,2)$ & NO & 11722\\
$(494,115)$ & 14 & $(30,7)$ & 8 & 2 & YES & YES & YES & $1.71$ & $(2,3)$ & NO & 11723\\
$(495,92)$ & 15 & $(2,1)$ & 1 & 1 & YES & YES & YES & $1.56$ & $(2,3)$ & NO & 11724\\
$(495,112)$ & 14 & $(2,1)$ & 1 & 1 & YES & YES & YES & $1.57$ & $(2,3)$ & -- & 11725\\
$(495,92)$ & 15 & $(3,1)$ & 2 & 3 & YES & YES & YES & $1.78$ & $(2,3)$ & -- & 11726\\
$(495,92)$ & 15 & $(11,2)$ & 6 & 11 & YES & YES & YES & $1.57$ & $(4,2)$ & NO & 11727\\
$(495,92)$ & 15 & $(27,5)$ & 8 & 9 & YES & YES & YES & $1.43$ & $(4,2)$ & NO & 11728\\
$(497,134)$ & 14 & $(3,1)$ & 2 & 1 & YES & YES & YES & $1.62$ & $(4,2)$ & NO & 11729\\
$(497,114)$ & 15 & $(5,1)$ & 4 & 1 & YES & YES & YES & $1.67$ & $(4,2)$ & NO & 11730\\
$(498,209)$ & 13 & $(2,1)$ & 1 & 2 & NO & YES & YES & $1.71$ & $(2,3)$ & -- & 11731\\
$(501,113)$ & 15 & $(71,16)$ & 10 & 1 & YES & YES & YES & $1.67$ & $(4,2)$ & NO & 11732\\
$(504,115)$ & 14 & $(9,2)$ & 5 & 9 & YES & YES & YES & $1.71$ & $(2,3)$ & NO & 11733\\
$(504,115)$ & 14 & $(35,8)$ & 8 & 7 & YES & YES & YES & $1.57$ & $(2,3)$ & NO & 11734\\
$(505,109)$ & 14 & $(2,1)$ & 1 & 1 & YES & YES & YES & $1.57$ & $(2,3)$ & NO & 11735\\
$(505,109)$ & 14 & $(4,1)$ & 3 & 1 & YES & YES & YES & $1.71$ & $(2,3)$ & NO & 11736\\
$(505,109)$ & 14 & $(9,2)$ & 5 & 1 & YES & YES & YES & $1.71$ & $(2,3)$ & NO & 11737\\
$(505,109)$ & 14 & $(37,8)$ & 8 & 1 & YES & YES & YES & $1.71$ & $(2,3)$ & NO & 11738\\
$(507,196)$ & 13 & $(119,46)$ & 10 & 1 & YES & YES & YES & $1.67$ & $(4,2)$ & 11569 & 11739\\
$(508,111)$ & 15 & $(119,26)$ & 11 & 1 & YES & YES & YES & $1.75$ & $(4,2)$ & NO & 11740\\
$(511,155)$ & 14 & $(7,2)$ & 4 & 7 & YES & YES & YES & $1.67$ & $(4,2)$ & NO & 11741\\
$(512,97)$ & 16 & $(2,1)$ & 1 & 2 & YES & YES & YES & $1.75$ & $(4,2)$ & NO & 11742\\
$(517,117)$ & 14 & $(2,1)$ & 1 & 1 & YES & YES & YES & $1.43$ & $(2,3)$ & -- & 11743\\
$(517,157)$ & 14 & $(191,58)$ & 12 & 1 & YES & YES & YES & $1.62$ & $(4,2)$ & NO & 11744\\
$(521,121)$ & 15 & $(5,1)$ & 4 & 1 & YES & YES & YES & $1.75$ & $(4,2)$ & -- & 11745\\
$(522,119)$ & 14 & $(2,1)$ & 1 & 2 & YES & YES & YES & $1.78$ & $(2,3)$ & NO & 11746\\
$(522,119)$ & 14 & $(4,1)$ & 3 & 2 & YES & YES & YES & $1.67$ & $(2,3)$ & NO & 11747\\
$(522,119)$ & 14 & $(9,2)$ & 5 & 9 & YES & YES & YES & $1.50$ & $(4,2)$ & NO & 11748\\
$(522,119)$ & 14 & $(13,3)$ & 6 & 1 & YES & YES & YES & $1.67$ & $(2,3)$ & NO & 11749\\
$(522,97)$ & 15 & $(16,3)$ & 7 & 2 & YES & YES & YES & $1.57$ & $(2,3)$ & NO & 11750\\
$(522,119)$ & 14 & $(22,5)$ & 7 & 2 & YES & YES & YES & $1.70$ & $(2,3)$ & NO & 11751\\
$(522,119)$ & 14 & $(35,8)$ & 8 & 1 & YES & YES & YES & $1.71$ & $(2,3)$ & NO & 11752\\
$(522,119)$ & 14 & $(57,13)$ & 9 & 3 & YES & YES & YES & $1.43$ & $(4,2)$ & NO & 11753\\
$(526,123)$ & 14 & $(2,1)$ & 1 & 2 & YES & YES & YES & $1.57$ & $(2,3)$ & NO & 11754\\
$(526,123)$ & 14 & $(2,1)$ & 1 & 2 & YES & YES & YES & $1.86$ & $(2,3)$ & -- & 11755\\
$(526,123)$ & 14 & $(30,7)$ & 8 & 2 & YES & YES & YES & $1.71$ & $(2,3)$ & NO & 11756\\
$(526,123)$ & 14 & $(77,18)$ & 10 & 1 & YES & YES & YES & $1.62$ & $(2,3)$ & NO & 11757\\
$(537,100)$ & 15 & $(43,8)$ & 9 & 1 & YES & YES & NO(2) & $1.50$ & $(6,1)$ & NO & 11758\\
$(542,151)$ & 14 & $(3,1)$ & 2 & 1 & YES & YES & YES & $1.67$ & $(4,2)$ & -- & 11759\\
$(545,103)$ & 16 & $(3,1)$ & 2 & 1 & YES & YES & YES & $1.62$ & $(4,2)$ & -- & 11760\\
$(545,103)$ & 16 & $(53,10)$ & 10 & 1 & YES & YES & YES & $1.62$ & $(4,2)$ & NO & 11761\\
$(551,104)$ & 16 & $(21,4)$ & 8 & 1 & YES & YES & YES & $1.43$ & $(6,1)$ & NO & 11762\\
$(553,102)$ & 16 & $(2,1)$ & 1 & 1 & YES & YES & YES & $1.62$ & $(4,2)$ & -- & 11763\\
$(561,128)$ & 14 & $(2,1)$ & 1 & 1 & YES & YES & YES & $1.57$ & $(2,3)$ & NO & 11764\\
$(561,128)$ & 14 & $(2,1)$ & 1 & 1 & YES & YES & YES & $1.57$ & $(2,3)$ & -- & 11765\\
$(561,128)$ & 14 & $(3,1)$ & 2 & 3 & YES & YES & YES & $1.43$ & $(2,3)$ & -- & 11766\\
$(561,128)$ & 14 & $(3,1)$ & 2 & 3 & YES & YES & YES & $1.71$ & $(2,3)$ & NO & 11767\\
$(561,128)$ & 14 & $(4,1)$ & 3 & 1 & YES & YES & YES & $1.62$ & $(2,3)$ & NO & 11768\\
$(561,128)$ & 14 & $(22,5)$ & 7 & 11 & YES & YES & YES & $1.71$ & $(2,3)$ & NO & 11769\\
$(566,129)$ & 14 & $(2,1)$ & 1 & 2 & YES & YES & YES & $1.67$ & $(4,2)$ & -- & 11770\\
$(573,122)$ & 15 & $(108,23)$ & 11 & 3 & YES & YES & YES & $1.62$ & $(4,2)$ & NO & 11771\\
$(575,136)$ & 15 & $(2,1)$ & 1 & 1 & YES & YES & YES & $1.62$ & $(4,2)$ & NO & 11772\\
$(575,136)$ & 15 & $(55,13)$ & 10 & 5 & YES & YES & YES & $1.62$ & $(4,2)$ & 10793 & 11773\\
$(598,113)$ & 16 & $(3,1)$ & 2 & 1 & YES & YES & YES & $1.67$ & $(4,2)$ & -- & 11774\\
$(598,113)$ & 16 & $(3,1)$ & 2 & 1 & YES & YES & YES & $1.78$ & $(4,2)$ & NO & 11775\\
$(599,140)$ & 15 & $(2,1)$ & 1 & 1 & YES & YES & YES & $1.62$ & $(4,2)$ & NO & 11776\\
$(599,140)$ & 15 & $(13,3)$ & 6 & 1 & YES & YES & YES & $1.62$ & $(4,2)$ & NO & 11777\\
$(617,172)$ & 14 & $(3,1)$ & 2 & 1 & YES & YES & YES & $1.67$ & $(4,2)$ & -- & 11778\\
$(625,118)$ & 16 & $(4,1)$ & 3 & 1 & YES & YES & YES & $1.62$ & $(4,2)$ & -- & 11779\\
$(625,118)$ & 16 & $(5,1)$ & 4 & 5 & YES & YES & YES & $1.62$ & $(4,2)$ & NO & 11780\\
$(656,143)$ & 15 & $(23,5)$ & 7 & 1 & YES & YES & YES & $1.62$ & $(4,2)$ & NO & 11781\\
$(a;0,0,0;3)$ & 4 & $(34,13)$ & 7 & 1 & YES & YES & YES & $1.50$ & $(2,3)$ & -- & 11782\\
$(a;0,0,0;3)$ & 4 & $(47,18)$ & 8 & 1 & YES & YES & NO(2) & $1.70$ & $(6,1)$ & -- & 11783\\
$(a;0,0,0;3)$ & 4 & $(49,19)$ & 8 & 1 & YES & YES & NO(2) & $1.56$ & $(8,0)$ & -- & 11784\\
$(a;0,0,0;3)$ & 4 & $(55,21)$ & 8 & 1 & YES & YES & YES & $1.43$ & $(4,2)$ & -- & 11785\\
$(a;0,0,0;3)$ & 4 & $(59,18)$ & 9 & 1 & YES & YES & NO(2) & $1.64$ & $(4,2)$ & -- & 11786\\
$(a;0,0,0;3)$ & 4 & $(65,19)$ & 9 & 1 & YES & YES & YES & $1.29$ & $(4,2)$ & -- & 11787\\
$(a;0,0,0;3)$ & 4 & $(67,18)$ & 9 & 1 & YES & YES & NO(2) & $1.64$ & $(4,2)$ & -- & 11788\\
$(a;0,0,0;3)$ & 4 & $(68,19)$ & 9 & 1 & YES & YES & YES & $1.29$ & $(4,2)$ & -- & 11789\\
$(a;0,0,0;3)$ & 4 & $(68,25)$ & 9 & 1 & YES & YES & NO(2) & $1.67$ & $(4,2)$ & -- & 11790\\
$(a;0,0,0;3)$ & 4 & $(70,29)$ & 9 & 1 & YES & YES & YES & $1.78$ & $(2,3)$ & -- & 11791\\
$(a;0,0,0;3)$ & 4 & $(73,27)$ & 9 & 1 & YES & YES & NO(2) & $1.50$ & $(6,1)$ & -- & 11792\\
$(a;0,0,0;3)$ & 4 & $(76,21)$ & 9 & 1 & YES & YES & YES & $1.62$ & $(2,3)$ & -- & 11793\\
$(a;0,0,0;3)$ & 4 & $(79,18)$ & 10 & 1 & YES & YES & YES & $1.29$ & $(4,2)$ & -- & 11794\\
$(a;0,0,0;3)$ & 4 & $(81,31)$ & 9 & 3 & YES & YES & YES & $1.67$ & $(2,3)$ & -- & 11795\\
$(a;0,0,0;3)$ & 4 & $(89,24)$ & 10 & 1 & YES & YES & YES & $1.67$ & $(2,3)$ & -- & 11796\\
$(a;0,0,0;3)$ & 4 & $(89,34)$ & 9 & 1 & YES & YES & YES & $1.78$ & $(2,3)$ & -- & 11797\\
$(a;0,0,0;3)$ & 4 & $(101,22)$ & 11 & 1 & YES & YES & YES & $1.67$ & $(2,3)$ & -- & 11798\\
$(a;0,0,0;3)$ & 4 & $(101,23)$ & 11 & 1 & YES & YES & YES & $1.50$ & $(2,3)$ & -- & 11799\\
$(a;0,0,0;3)$ & 4 & $(119,46)$ & 10 & 1 & YES & YES & YES & $1.88$ & $(4,2)$ & -- & 11800\\
$(a;1,0,0;13)$ & 5 & $(18,7)$ & 6 & 1 & YES & YES & YES & $1.25$ & $(2,3)$ & -- & 11801\\
$(a;1,0,0;13)$ & 5 & $(24,7)$ & 7 & 1 & YES & YES & NO(2) & $1.33$ & $(4,2)$ & -- & 11802\\
$(a;1,0,0;13)$ & 5 & $(27,8)$ & 7 & 1 & YES & YES & NO(2) & $1.56$ & $(8,0)$ & -- & 11803\\
$(a;1,0,0;13)$ & 5 & $(29,12)$ & 7 & 1 & YES & YES & YES & $1.43$ & $(2,3)$ & -- & 11804\\
$(a;1,0,0;13)$ & 5 & $(30,11)$ & 7 & 1 & YES & YES & YES & $1.43$ & $(2,3)$ & -- & 11805\\
$(a;1,0,0;13)$ & 5 & $(34,13)$ & 7 & 1 & YES & YES & YES & $1.29$ & $(4,2)$ & -- & 11806\\
$(a;1,0,0;13)$ & 5 & $(36,13)$ & 8 & 1 & YES & YES & YES & $1.62$ & $(2,3)$ & -- & 11807\\
$(a;1,0,0;13)$ & 5 & $(39,11)$ & 9 & 13 & YES & YES & NO(2) & $1.70$ & $(2,3)$ & -- & 11808\\
$(a;1,0,0;13)$ & 5 & $(41,17)$ & 8 & 1 & YES & YES & NO(2) & $1.80$ & $(2,3)$ & -- & 11809\\
$(a;1,0,0;13)$ & 5 & $(43,10)$ & 9 & 1 & YES & YES & NO(2) & $1.44$ & $(4,2)$ & -- & 11810\\
$(a;1,0,0;13)$ & 5 & $(45,19)$ & 8 & 1 & YES & YES & YES & $1.43$ & $(4,2)$ & -- & 11811\\
$(a;1,0,0;13)$ & 5 & $(46,19)$ & 8 & 1 & YES & YES & YES & $1.43$ & $(2,3)$ & -- & 11812\\
$(a;1,0,0;13)$ & 5 & $(50,21)$ & 8 & 1 & YES & YES & NO(2) & $1.56$ & $(4,2)$ & -- & 11813\\
$(a;1,0,0;13)$ & 5 & $(52,19)$ & 9 & 13 & YES & YES & NO(2) & $1.78$ & $(4,2)$ & -- & 11814\\
$(a;1,0,0;13)$ & 5 & $(55,21)$ & 8 & 1 & YES & YES & YES & $1.29$ & $(4,2)$ & -- & 11815\\
$(a;1,0,0;13)$ & 5 & $(59,26)$ & 9 & 1 & YES & YES & NO(2) & $1.50$ & $(10,-1)$ & -- & 11816\\
$(a;1,0,0;13)$ & 5 & $(61,17)$ & 9 & 1 & YES & YES & YES & $1.78$ & $(2,3)$ & -- & 11817\\
$(a;1,0,0;13)$ & 5 & $(61,18)$ & 9 & 1 & YES & YES & YES & $1.29$ & $(4,2)$ & -- & 11818\\
$(a;1,0,0;13)$ & 5 & $(70,29)$ & 9 & 1 & YES & YES & YES & $2.00$ & $(2,3)$ & -- & 11819\\
$(a;1,0,0;13)$ & 5 & $(75,29)$ & 9 & 1 & YES & YES & YES & $2.00$ & $(2,3)$ & -- & 11820\\
$(a;1,0,0;13)$ & 5 & $(95,29)$ & 10 & 1 & YES & YES & YES & $1.89$ & $(2,3)$ & -- & 11821\\
$(a;1,1,0;19)$ & 6 & $(23,9)$ & 7 & 1 & YES & YES & NO(2) & $1.70$ & $(2,3)$ & -- & 11822\\
$(a;1,1,0;19)$ & 6 & $(23,10)$ & 7 & 1 & YES & YES & NO(2) & $1.70$ & $(2,3)$ & -- & 11823\\
$(a;1,1,0;19)$ & 6 & $(25,9)$ & 7 & 1 & YES & YES & YES & $1.43$ & $(2,3)$ & -- & 11824\\
$(a;1,1,0;19)$ & 6 & $(28,11)$ & 8 & 1 & YES & YES & NO(2) & $1.50$ & $(6,1)$ & -- & 11825\\
$(a;1,1,0;19)$ & 6 & $(29,11)$ & 7 & 1 & YES & YES & NO(2) & $1.67$ & $(4,2)$ & -- & 11826\\
$(a;1,1,0;19)$ & 6 & $(31,12)$ & 7 & 1 & YES & YES & YES & $1.75$ & $(4,2)$ & -- & 11827\\
$(a;1,1,0;19)$ & 6 & $(31,13)$ & 7 & 1 & YES & YES & YES & $1.43$ & $(2,3)$ & -- & 11828\\
$(a;1,1,0;19)$ & 6 & $(34,9)$ & 8 & 1 & YES & YES & YES & $1.43$ & $(2,3)$ & -- & 11829\\
$(a;1,1,0;19)$ & 6 & $(39,14)$ & 8 & 1 & YES & YES & YES & $1.78$ & $(4,2)$ & -- & 11830\\
$(a;1,1,0;19)$ & 6 & $(49,18)$ & 8 & 1 & YES & YES & YES & $1.67$ & $(4,2)$ & -- & 11831\\
$(a;2,0,0;17)$ & 6 & $(23,10)$ & 7 & 1 & YES & YES & YES & $1.62$ & $(2,3)$ & -- & 11832\\
$(a;2,0,0;17)$ & 6 & $(27,8)$ & 7 & 1 & YES & YES & NO(2) & $1.75$ & $(2,3)$ & -- & 11833\\
$(a;2,0,0;17)$ & 6 & $(34,13)$ & 7 & 17 & YES & YES & YES & $1.50$ & $(2,3)$ & -- & 11834\\
$(a;2,0,0;17)$ & 6 & $(43,18)$ & 8 & 1 & YES & YES & YES & $1.71$ & $(6,1)$ & -- & 11835\\
$(a;2,0,0;17)$ & 6 & $(56,15)$ & 9 & 1 & YES & YES & NO(2) & $1.50$ & $(6,1)$ & -- & 11836\\
$(a;2,0,1;25)$ & 7 & $(19,8)$ & 6 & 1 & YES & YES & YES & $1.43$ & $(2,3)$ & -- & 11837\\
$(a;2,0,1;25)$ & 7 & $(21,8)$ & 6 & 1 & YES & YES & YES & $1.82$ & $(2,3)$ & -- & 11838\\
$(a;2,0,1;25)$ & 7 & $(24,7)$ & 7 & 1 & YES & YES & NO(2) & $1.56$ & $(4,2)$ & -- & 11839\\
$(a;2,0,1;25)$ & 7 & $(27,8)$ & 7 & 1 & YES & YES & NO(2) & $1.67$ & $(2,3)$ & -- & 11840\\
$(a;2,0,1;25)$ & 7 & $(43,12)$ & 8 & 1 & YES & YES & YES & $1.62$ & $(4,2)$ & -- & 11841\\
$(a;2,1,0;5)$ & 7 & $(17,7)$ & 6 & 1 & YES & YES & NO(2) & $1.78$ & $(2,3)$ & -- & 11842\\
$(a;2,1,0;5)$ & 7 & $(18,7)$ & 6 & 1 & YES & YES & NO(2) & $1.56$ & $(8,0)$ & -- & 11843\\
$(a;2,1,0;5)$ & 7 & $(23,9)$ & 7 & 1 & YES & YES & NO(2) & $1.78$ & $(4,2)$ & -- & 11844\\
$(a;2,1,0;5)$ & 7 & $(29,12)$ & 7 & 1 & YES & YES & YES & $1.57$ & $(6,1)$ & -- & 11845\\
$(a;2,1,0;5)$ & 7 & $(34,13)$ & 7 & 1 & YES & YES & YES & $1.29$ & $(6,1)$ & -- & 11846\\
$(a;2,1,1;37)$ & 8 & $(9,4)$ & 5 & 1 & YES & YES & NO(2) & $1.70$ & $(2,3)$ & -- & 11847\\
$(a;2,1,1;37)$ & 8 & $(12,5)$ & 5 & 1 & YES & YES & YES & $1.43$ & $(6,1)$ & -- & 11848\\
$(a;2,1,1;37)$ & 8 & $(13,5)$ & 5 & 1 & YES & YES & YES & $1.78$ & $(2,3)$ & -- & 11849\\
$(a;2,1,1;37)$ & 8 & $(17,7)$ & 6 & 1 & YES & YES & NO(2) & $1.38$ & $(6,1)$ & -- & 11850\\
$(a;2,2,0;33)$ & 8 & $(13,5)$ & 5 & 1 & YES & YES & NO(2) & $1.56$ & $(2,3)$ & -- & 11851\\
$(a;2,2,0;33)$ & 8 & $(17,5)$ & 6 & 1 & YES & YES & NO(2) & $1.56$ & $(2,3)$ & -- & 11852\\
$(a;2,2,1;49)$ & 9 & $(8,3)$ & 4 & 1 & YES & YES & NO(2) & $1.38$ & $(4,2)$ & -- & 11853\\
$(a;2,2,1;49)$ & 9 & $(12,5)$ & 5 & 1 & YES & YES & YES & $1.88$ & $(4,2)$ & -- & 11854\\
$(a;2,2,1;49)$ & 9 & $(17,5)$ & 6 & 1 & YES & YES & YES & $1.29$ & $(6,1)$ & -- & 11855\\
$(a;3,0,1;31)$ & 8 & $(17,5)$ & 6 & 1 & YES & YES & YES & $1.57$ & $(2,3)$ & -- & 11856\\
$(a;3,0,1;31)$ & 8 & $(18,5)$ & 6 & 1 & YES & YES & YES & $1.57$ & $(2,3)$ & -- & 11857\\
$(a;3,2,1;61)$ & 10 & $(10,3)$ & 5 & 1 & YES & YES & YES & $1.29$ & $(6,1)$ & -- & 11858\\
$(a;4,0,0;25)$ & 8 & $(69,11)$ & 11 & 1 & YES & YES & NO(2) & $1.50$ & $(6,1)$ & -- & 11859\\
$(a;5,1,2;85)$ & 12 & $(5,1)$ & 4 & 5 & YES & YES & NO(2) & $1.60$ & $(6,1)$ & -- & 11860\\
$(a;5,1,2;85)$ & 12 & $(6,1)$ & 5 & 1 & YES & YES & NO(2) & $1.50$ & $(6,1)$ & -- & 11861\\
$(b;0,0,0;14)$ & 5 & $(18,7)$ & 6 & 2 & YES & YES & NO(3) & $1.25$ & $(2,3)$ & -- & 11862\\
$(b;0,0,0;14)$ & 5 & $(21,8)$ & 6 & 7 & YES & YES & NO(2) & $1.60$ & $(2,3)$ & -- & 11863\\
$(b;0,0,0;14)$ & 5 & $(29,11)$ & 7 & 1 & YES & YES & NO(2) & $1.44$ & $(4,2)$ & -- & 11864\\
$(b;0,0,0;14)$ & 5 & $(29,12)$ & 7 & 1 & YES & YES & YES & $1.29$ & $(4,2)$ & -- & 11865\\
$(b;0,0,0;14)$ & 5 & $(30,11)$ & 7 & 2 & YES & YES & NO(2) & $1.56$ & $(8,0)$ & -- & 11866\\
$(b;0,0,0;14)$ & 5 & $(34,13)$ & 7 & 2 & YES & YES & YES & $1.29$ & $(4,2)$ & -- & 11867\\
$(b;0,0,0;14)$ & 5 & $(37,11)$ & 8 & 1 & YES & YES & YES & $1.82$ & $(2,3)$ & -- & 11868\\
$(b;0,0,0;14)$ & 5 & $(40,9)$ & 9 & 2 & YES & YES & NO(2) & $1.56$ & $(2,3)$ & -- & 11869\\
$(b;0,0,0;14)$ & 5 & $(40,11)$ & 8 & 2 & YES & YES & YES & $1.29$ & $(4,2)$ & -- & 11870\\
$(b;0,0,0;14)$ & 5 & $(41,11)$ & 8 & 1 & YES & YES & YES & $1.50$ & $(2,3)$ & -- & 11871\\
$(b;0,0,0;14)$ & 5 & $(41,17)$ & 8 & 1 & YES & YES & YES & $1.89$ & $(2,3)$ & -- & 11872\\
$(b;0,0,0;14)$ & 5 & $(43,12)$ & 8 & 1 & YES & YES & YES & $1.29$ & $(4,2)$ & -- & 11873\\
$(b;0,0,0;14)$ & 5 & $(43,18)$ & 8 & 1 & YES & YES & YES & $1.57$ & $(4,2)$ & -- & 11874\\
$(b;0,0,0;14)$ & 5 & $(44,13)$ & 8 & 2 & YES & YES & YES & $1.80$ & $(2,3)$ & -- & 11875\\
$(b;0,0,0;14)$ & 5 & $(44,17)$ & 8 & 2 & YES & YES & YES & $1.75$ & $(2,3)$ & -- & 11876\\
$(b;0,0,0;14)$ & 5 & $(46,19)$ & 8 & 2 & YES & YES & YES & $1.29$ & $(4,2)$ & -- & 11877\\
$(b;0,0,0;14)$ & 5 & $(47,18)$ & 8 & 1 & YES & YES & YES & $1.71$ & $(4,2)$ & -- & 11878\\
$(b;0,0,0;14)$ & 5 & $(49,9)$ & 10 & 7 & YES & YES & NO(2) & $1.64$ & $(4,2)$ & -- & 11879\\
$(b;0,0,0;14)$ & 5 & $(49,20)$ & 9 & 7 & YES & YES & NO(2) & $1.62$ & $(6,1)$ & -- & 11880\\
$(b;0,0,0;14)$ & 5 & $(50,21)$ & 8 & 2 & YES & YES & YES & $1.57$ & $(4,2)$ & -- & 11881\\
$(b;0,0,0;14)$ & 5 & $(51,14)$ & 9 & 1 & YES & YES & YES & $1.57$ & $(4,2)$ & -- & 11882\\
$(b;0,0,0;14)$ & 5 & $(55,21)$ & 8 & 1 & YES & YES & YES & $1.75$ & $(4,2)$ & -- & 11883\\
$(b;0,0,1;4)$ & 6 & $(19,8)$ & 6 & 1 & YES & YES & NO(2) & $1.25$ & $(6,1)$ & -- & 11884\\
$(b;0,0,1;4)$ & 6 & $(21,8)$ & 6 & 1 & YES & YES & YES & $1.62$ & $(2,3)$ & -- & 11885\\
$(b;0,0,1;4)$ & 6 & $(23,9)$ & 7 & 1 & YES & YES & NO(2) & $1.70$ & $(2,3)$ & -- & 11886\\
$(b;0,0,1;4)$ & 6 & $(24,7)$ & 7 & 4 & YES & YES & YES & $1.38$ & $(2,3)$ & -- & 11887\\
$(b;0,0,1;4)$ & 6 & $(29,11)$ & 7 & 1 & YES & YES & YES & $1.67$ & $(2,3)$ & -- & 11888\\
$(b;0,0,1;4)$ & 6 & $(29,12)$ & 7 & 1 & YES & YES & YES & $1.67$ & $(2,3)$ & -- & 11889\\
$(b;0,0,1;4)$ & 6 & $(31,9)$ & 8 & 1 & YES & YES & YES & $1.67$ & $(2,3)$ & -- & 11890\\
$(b;0,0,1;4)$ & 6 & $(31,12)$ & 7 & 1 & YES & YES & YES & $1.71$ & $(4,2)$ & -- & 11891\\
$(b;0,0,2;26)$ & 7 & $(11,4)$ & 5 & 1 & YES & YES & NO(2) & $1.50$ & $(6,1)$ & -- & 11892\\
$(b;0,0,2;26)$ & 7 & $(18,7)$ & 6 & 2 & YES & YES & NO(2) & $1.50$ & $(4,2)$ & -- & 11893\\
$(b;0,0,3;32)$ & 8 & $(10,3)$ & 5 & 2 & YES & YES & NO(2) & $1.60$ & $(2,3)$ & -- & 11894\\
$(b;0,0,3;32)$ & 8 & $(11,3)$ & 5 & 1 & YES & YES & NO(2) & $1.70$ & $(2,3)$ & -- & 11895\\
$(b;0,0,4;38)$ & 9 & $(7,2)$ & 4 & 1 & YES & YES & NO(2) & $1.62$ & $(6,1)$ & -- & 11896\\
$(b;0,1,0;19)$ & 6 & $(19,8)$ & 6 & 19 & YES & YES & NO(2) & $1.56$ & $(4,2)$ & -- & 11897\\
$(b;0,1,0;19)$ & 6 & $(21,8)$ & 6 & 1 & YES & YES & YES & $1.29$ & $(4,2)$ & -- & 11898\\
$(b;0,1,0;19)$ & 6 & $(23,9)$ & 7 & 1 & YES & YES & NO(2) & $1.70$ & $(2,3)$ & -- & 11899\\
$(b;0,1,0;19)$ & 6 & $(27,8)$ & 7 & 1 & YES & YES & NO(2) & $1.56$ & $(4,2)$ & -- & 11900\\
$(b;0,1,0;19)$ & 6 & $(31,12)$ & 7 & 1 & YES & YES & YES & $1.71$ & $(4,2)$ & -- & 11901\\
$(b;0,1,0;19)$ & 6 & $(31,13)$ & 7 & 1 & YES & YES & YES & $1.78$ & $(2,3)$ & -- & 11902\\
$(b;0,1,0;19)$ & 6 & $(32,9)$ & 8 & 1 & YES & YES & YES & $1.78$ & $(4,2)$ & -- & 11903\\
$(b;0,1,0;19)$ & 6 & $(33,10)$ & 8 & 1 & YES & YES & YES & $1.78$ & $(2,3)$ & -- & 11904\\
$(b;0,1,0;19)$ & 6 & $(39,16)$ & 8 & 1 & YES & YES & YES & $1.75$ & $(4,2)$ & -- & 11905\\
$(b;0,1,0;19)$ & 6 & $(41,12)$ & 8 & 1 & YES & YES & YES & $1.67$ & $(2,3)$ & -- & 11906\\
$(b;0,1,0;19)$ & 6 & $(47,13)$ & 8 & 1 & YES & YES & YES & $1.67$ & $(2,3)$ & -- & 11907\\
$(b;0,1,0;19)$ & 6 & $(93,22)$ & 11 & 1 & YES & YES & YES & $1.88$ & $(4,2)$ & -- & 11908\\
$(b;0,1,1;27)$ & 7 & $(11,4)$ & 5 & 1 & YES & YES & NO(2) & $1.56$ & $(4,2)$ & -- & 11909\\
$(b;0,1,1;27)$ & 7 & $(13,5)$ & 5 & 1 & YES & YES & YES & $1.29$ & $(4,2)$ & -- & 11910\\
$(b;0,1,1;27)$ & 7 & $(17,7)$ & 6 & 1 & YES & YES & YES & $1.89$ & $(2,3)$ & -- & 11911\\
$(b;0,1,1;27)$ & 7 & $(18,7)$ & 6 & 9 & YES & YES & YES & $1.80$ & $(2,3)$ & -- & 11912\\
$(b;0,1,1;27)$ & 7 & $(21,8)$ & 6 & 3 & YES & YES & YES & $1.67$ & $(2,3)$ & -- & 11913\\
$(b;0,1,1;27)$ & 7 & $(27,8)$ & 7 & 27 & YES & YES & YES & $1.56$ & $(2,3)$ & -- & 11914\\
$(b;0,1,2;5)$ & 8 & $(7,2)$ & 4 & 1 & YES & YES & NO(2) & $1.50$ & $(2,3)$ & -- & 11915\\
$(b;0,1,2;5)$ & 8 & $(12,5)$ & 5 & 1 & YES & YES & YES & $1.75$ & $(4,2)$ & -- & 11916\\
$(b;0,1,2;5)$ & 8 & $(13,5)$ & 5 & 1 & YES & YES & YES & $1.75$ & $(4,2)$ & -- & 11917\\
$(b;0,1,2;5)$ & 8 & $(19,7)$ & 6 & 1 & YES & YES & YES & $1.75$ & $(4,2)$ & -- & 11918\\
$(b;0,1,3;43)$ & 9 & $(5,2)$ & 3 & 1 & YES & YES & YES & $1.43$ & $(4,2)$ & -- & 11919\\
$(b;0,1,3;43)$ & 9 & $(10,3)$ & 5 & 1 & YES & YES & YES & $1.43$ & $(4,2)$ & -- & 11920\\
$(b;0,1,4;51)$ & 10 & $(4,1)$ & 3 & 1 & YES & YES & NO(2) & $1.50$ & $(6,1)$ & -- & 11921\\
$(b;0,1,4;51)$ & 10 & $(5,2)$ & 3 & 1 & YES & YES & NO(2) & $1.50$ & $(6,1)$ & -- & 11922\\
$(b;0,2,0;8)$ & 7 & $(13,5)$ & 5 & 1 & YES & YES & NO(2) & $1.60$ & $(2,3)$ & -- & 11923\\
$(b;0,2,0;8)$ & 7 & $(15,4)$ & 6 & 1 & YES & YES & NO(2) & $1.33$ & $(4,2)$ & -- & 11924\\
$(b;0,2,0;8)$ & 7 & $(21,8)$ & 6 & 1 & YES & YES & YES & $1.88$ & $(4,2)$ & -- & 11925\\
$(b;0,2,0;8)$ & 7 & $(22,5)$ & 7 & 2 & YES & YES & NO(2) & $1.73$ & $(4,2)$ & -- & 11926\\
$(b;0,2,0;8)$ & 7 & $(27,8)$ & 7 & 1 & YES & YES & YES & $1.43$ & $(4,2)$ & -- & 11927\\
$(b;0,2,0;8)$ & 7 & $(29,8)$ & 7 & 1 & YES & YES & YES & $1.43$ & $(4,2)$ & -- & 11928\\
$(b;0,2,0;8)$ & 7 & $(29,13)$ & 8 & 1 & YES & YES & NO(2) & $1.50$ & $(6,1)$ & -- & 11929\\
$(b;0,2,0;8)$ & 7 & $(35,8)$ & 8 & 1 & YES & YES & YES & $1.75$ & $(4,2)$ & -- & 11930\\
$(b;0,2,1;34)$ & 8 & $(13,5)$ & 5 & 1 & YES & YES & YES & $1.89$ & $(2,3)$ & -- & 11931\\
$(b;0,2,1;34)$ & 8 & $(17,5)$ & 6 & 17 & YES & YES & YES & $1.57$ & $(4,2)$ & -- & 11932\\
$(b;0,2,2;44)$ & 9 & $(7,3)$ & 4 & 1 & YES & YES & NO(2) & $1.38$ & $(4,2)$ & -- & 11933\\
$(b;0,2,2;44)$ & 9 & $(8,3)$ & 4 & 4 & YES & YES & NO(2) & $1.38$ & $(4,2)$ & -- & 11934\\
$(b;0,2,4;64)$ & 11 & $(3,1)$ & 2 & 1 & YES & YES & YES & $1.43$ & $(4,2)$ & -- & 11935\\
$(b;0,2,4;64)$ & 11 & $(4,1)$ & 3 & 4 & YES & YES & NO(2) & $1.38$ & $(6,1)$ & -- & 11936\\
$(b;0,2,4;64)$ & 11 & $(5,1)$ & 4 & 1 & YES & YES & NO(2) & $1.56$ & $(4,2)$ & -- & 11937\\
$(b;0,3,0;29)$ & 8 & $(13,5)$ & 5 & 1 & YES & YES & NO(2) & $1.56$ & $(8,0)$ & -- & 11938\\
$(b;1,0,0;5)$ & 6 & $(29,11)$ & 7 & 1 & YES & YES & YES & $1.78$ & $(2,3)$ & -- & 11939\\
$(b;1,0,0;5)$ & 6 & $(29,12)$ & 7 & 1 & YES & YES & YES & $1.62$ & $(2,3)$ & -- & 11940\\
$(b;1,0,0;5)$ & 6 & $(31,9)$ & 8 & 1 & YES & YES & YES & $1.62$ & $(2,3)$ & -- & 11941\\
$(b;1,0,0;5)$ & 6 & $(31,13)$ & 7 & 1 & YES & YES & YES & $1.62$ & $(2,3)$ & -- & 11942\\
$(b;1,0,0;5)$ & 6 & $(34,13)$ & 7 & 1 & YES & YES & YES & $1.50$ & $(2,3)$ & -- & 11943\\
$(b;1,0,1;29)$ & 7 & $(12,5)$ & 5 & 1 & YES & YES & YES & $1.29$ & $(4,2)$ & -- & 11944\\
$(b;1,0,1;29)$ & 7 & $(17,7)$ & 6 & 1 & YES & YES & NO(2) & $1.50$ & $(4,2)$ & -- & 11945\\
$(b;1,0,1;29)$ & 7 & $(19,8)$ & 6 & 1 & YES & YES & YES & $1.78$ & $(2,3)$ & -- & 11946\\
$(b;1,0,1;29)$ & 7 & $(21,8)$ & 6 & 1 & YES & YES & YES & $1.29$ & $(4,2)$ & -- & 11947\\
$(b;1,0,2;19)$ & 8 & $(5,2)$ & 3 & 1 & YES & YES & NO(2) & $1.50$ & $(2,3)$ & -- & 11948\\
$(b;1,0,2;19)$ & 8 & $(11,4)$ & 5 & 1 & YES & YES & NO(2) & $1.50$ & $(4,2)$ & -- & 11949\\
$(b;1,0,2;19)$ & 8 & $(12,5)$ & 5 & 1 & YES & YES & YES & $1.91$ & $(2,3)$ & -- & 11950\\
$(b;1,0,2;19)$ & 8 & $(13,5)$ & 5 & 1 & YES & YES & YES & $1.91$ & $(2,3)$ & -- & 11951\\
$(b;1,0,3;47)$ & 9 & $(10,3)$ & 5 & 1 & YES & YES & YES & $1.43$ & $(4,2)$ & -- & 11952\\
$(b;1,1,0;27)$ & 7 & $(13,5)$ & 5 & 1 & YES & YES & NO(2) & $1.50$ & $(6,1)$ & -- & 11953\\
$(b;1,1,0;27)$ & 7 & $(17,7)$ & 6 & 1 & YES & YES & YES & $1.43$ & $(4,2)$ & -- & 11954\\
$(b;1,1,0;27)$ & 7 & $(18,7)$ & 6 & 9 & YES & YES & YES & $1.78$ & $(4,2)$ & -- & 11955\\
$(b;1,1,0;27)$ & 7 & $(21,8)$ & 6 & 3 & YES & YES & YES & $1.75$ & $(4,2)$ & -- & 11956\\
$(b;1,1,0;27)$ & 7 & $(27,8)$ & 7 & 27 & YES & YES & YES & $1.75$ & $(4,2)$ & -- & 11957\\
$(b;1,1,0;27)$ & 7 & $(31,12)$ & 7 & 1 & YES & YES & YES & $1.75$ & $(4,2)$ & -- & 11958\\
$(b;1,1,0;27)$ & 7 & $(33,10)$ & 8 & 3 & YES & YES & NO(2) & $1.50$ & $(4,2)$ & -- & 11959\\
$(b;1,1,1;39)$ & 8 & $(12,5)$ & 5 & 3 & YES & YES & YES & $1.56$ & $(4,2)$ & -- & 11960\\
$(b;1,1,1;39)$ & 8 & $(13,5)$ & 5 & 13 & YES & YES & YES & $1.70$ & $(2,3)$ & -- & 11961\\
$(b;1,1,2;51)$ & 9 & $(5,2)$ & 3 & 1 & YES & YES & YES & $1.43$ & $(4,2)$ & -- & 11962\\
$(b;1,1,2;51)$ & 9 & $(8,3)$ & 4 & 1 & YES & YES & YES & $1.50$ & $(4,2)$ & -- & 11963\\
$(b;1,1,2;51)$ & 9 & $(10,3)$ & 5 & 1 & YES & YES & YES & $1.29$ & $(4,2)$ & -- & 11964\\
$(b;1,1,2;51)$ & 9 & $(12,5)$ & 5 & 3 & YES & YES & YES & $1.67$ & $(4,2)$ & -- & 11965\\
$(b;1,1,3;63)$ & 10 & $(8,3)$ & 4 & 1 & YES & YES & YES & $1.90$ & $(2,3)$ & -- & 11966\\
$(b;1,2,0;17)$ & 8 & $(15,4)$ & 6 & 1 & YES & YES & NO(2) & $1.60$ & $(6,1)$ & -- & 11967\\
$(b;1,2,1;7)$ & 9 & $(5,2)$ & 3 & 1 & YES & YES & NO(2) & $1.55$ & $(4,2)$ & -- & 11968\\
$(b;1,2,1;7)$ & 9 & $(10,3)$ & 5 & 1 & YES & YES & YES & $1.43$ & $(4,2)$ & -- & 11969\\
$(b;1,2,1;7)$ & 9 & $(11,3)$ & 5 & 1 & YES & YES & YES & $1.67$ & $(2,3)$ & -- & 11970\\
$(b;1,3,1;59)$ & 10 & $(3,1)$ & 2 & 1 & YES & YES & NO(2) & $1.50$ & $(2,3)$ & -- & 11971\\
$(b;1,3,2;77)$ & 11 & $(5,1)$ & 4 & 1 & YES & YES & YES & $1.50$ & $(2,3)$ & -- & 11972\\
$(b;2,0,1;38)$ & 8 & $(8,3)$ & 4 & 2 & YES & YES & NO(2) & $1.38$ & $(6,1)$ & -- & 11973\\
$(b;2,0,1;38)$ & 8 & $(10,3)$ & 5 & 2 & YES & YES & NO(2) & $1.44$ & $(4,2)$ & -- & 11974\\
$(b;2,0,1;38)$ & 8 & $(12,5)$ & 5 & 2 & YES & YES & YES & $1.75$ & $(2,3)$ & -- & 11975\\
$(b;2,0,1;38)$ & 8 & $(13,5)$ & 5 & 1 & YES & YES & YES & $1.67$ & $(2,3)$ & -- & 11976\\
$(b;2,0,1;38)$ & 8 & $(17,5)$ & 6 & 1 & YES & YES & YES & $1.56$ & $(2,3)$ & -- & 11977\\
$(b;2,0,2;50)$ & 9 & $(12,5)$ & 5 & 2 & YES & YES & YES & $1.56$ & $(4,2)$ & -- & 11978\\
$(b;2,0,2;50)$ & 9 & $(13,5)$ & 5 & 1 & YES & YES & YES & $1.67$ & $(4,2)$ & -- & 11979\\
$(b;2,1,1;17)$ & 9 & $(7,2)$ & 4 & 1 & YES & YES & NO(2) & $1.25$ & $(6,1)$ & -- & 11980\\
$(b;2,1,2;67)$ & 10 & $(8,3)$ & 4 & 1 & YES & YES & YES & $1.50$ & $(4,2)$ & -- & 11981\\
$(b;2,1,2;67)$ & 10 & $(10,3)$ & 5 & 1 & YES & YES & YES & $1.62$ & $(4,2)$ & -- & 11982\\
$(b;2,2,2;84)$ & 11 & $(4,1)$ & 3 & 4 & YES & YES & YES & $1.43$ & $(2,3)$ & -- & 11983\\
$(b;3,0,0;16)$ & 8 & $(10,3)$ & 5 & 2 & YES & YES & NO(2) & $1.67$ & $(8,0)$ & -- & 11984\\
$(b;3,0,2;31)$ & 10 & $(7,2)$ & 4 & 1 & YES & YES & NO(2) & $1.56$ & $(8,0)$ & -- & 11985\\
$(b;3,1,1;63)$ & 10 & $(3,1)$ & 2 & 3 & YES & YES & NO(2) & $1.60$ & $(2,3)$ & -- & 11986\\
$(b;4,0,2;74)$ & 11 & $(3,1)$ & 2 & 1 & YES & YES & NO(2) & $1.56$ & $(8,0)$ & -- & 11987\\
$(b;4,0,2;74)$ & 11 & $(4,1)$ & 3 & 2 & YES & YES & NO(2) & $1.56$ & $(4,2)$ & -- & 11988\\
$(c;0,0,0;4)$ & 4 & $(34,13)$ & 7 & 2 & YES & YES & NO(2) & $1.50$ & $(10,-1)$ & -- & 11989\\
$(c;0,0,0;4)$ & 4 & $(41,11)$ & 8 & 1 & YES & YES & NO(2) & $1.70$ & $(6,1)$ & -- & 11990\\
$(c;0,0,0;4)$ & 4 & $(41,15)$ & 8 & 1 & YES & YES & NO(2) & $1.50$ & $(2,3)$ & -- & 11991\\
$(c;0,0,0;4)$ & 4 & $(43,18)$ & 8 & 1 & YES & YES & YES & $1.50$ & $(2,3)$ & -- & 11992\\
$(c;0,0,0;4)$ & 4 & $(44,13)$ & 8 & 4 & YES & YES & YES & $1.50$ & $(2,3)$ & -- & 11993\\
$(c;0,0,0;4)$ & 4 & $(46,19)$ & 8 & 2 & YES & YES & NO(2) & $1.60$ & $(2,3)$ & -- & 11994\\
$(c;0,0,0;4)$ & 4 & $(50,19)$ & 8 & 2 & YES & YES & NO(2) & $1.50$ & $(2,3)$ & -- & 11995\\
$(c;0,0,0;4)$ & 4 & $(50,21)$ & 8 & 2 & YES & YES & YES & $1.50$ & $(2,3)$ & -- & 11996\\
$(c;0,0,0;4)$ & 4 & $(51,20)$ & 9 & 1 & YES & YES & NO(2) & $1.56$ & $(4,2)$ & -- & 11997\\
$(c;0,0,0;4)$ & 4 & $(55,21)$ & 8 & 1 & YES & YES & NO(2) & $1.60$ & $(2,3)$ & -- & 11998\\
$(c;0,0,0;4)$ & 4 & $(55,24)$ & 9 & 1 & YES & YES & NO(2) & $1.70$ & $(2,3)$ & -- & 11999\\
$(c;0,0,0;4)$ & 4 & $(56,15)$ & 9 & 4 & YES & YES & NO(2) & $1.60$ & $(2,3)$ & -- & 12000\\
$(c;0,0,0;4)$ & 4 & $(59,18)$ & 9 & 1 & YES & YES & NO(2) & $1.50$ & $(6,1)$ & -- & 12001\\
$(c;0,0,0;4)$ & 4 & $(61,22)$ & 9 & 1 & YES & YES & NO(2) & $1.60$ & $(2,3)$ & -- & 12002\\
$(c;0,0,0;4)$ & 4 & $(64,25)$ & 9 & 4 & YES & YES & NO(2) & $1.70$ & $(6,1)$ & -- & 12003\\
$(c;0,0,0;4)$ & 4 & $(67,18)$ & 9 & 1 & YES & YES & NO(2) & $1.60$ & $(2,3)$ & -- & 12004\\
$(c;0,0,0;4)$ & 4 & $(67,26)$ & 9 & 1 & YES & YES & NO(2) & $1.67$ & $(2,3)$ & -- & 12005\\
$(c;0,0,0;4)$ & 4 & $(69,20)$ & 10 & 1 & YES & YES & NO(2) & $1.38$ & $(10,-1)$ & -- & 12006\\
$(c;0,0,0;4)$ & 4 & $(69,29)$ & 9 & 1 & YES & YES & YES & $1.29$ & $(6,1)$ & -- & 12007\\
$(c;0,0,0;4)$ & 4 & $(70,29)$ & 9 & 2 & YES & YES & NO(2) & $1.78$ & $(2,3)$ & -- & 12008\\
$(c;0,0,0;4)$ & 4 & $(71,26)$ & 9 & 1 & YES & YES & NO(2) & $1.70$ & $(2,3)$ & -- & 12009\\
$(c;0,0,0;4)$ & 4 & $(73,27)$ & 9 & 1 & YES & YES & NO(2) & $1.56$ & $(4,2)$ & -- & 12010\\
$(c;0,0,0;4)$ & 4 & $(74,31)$ & 9 & 2 & YES & YES & YES & $1.29$ & $(6,1)$ & -- & 12011\\
$(c;0,0,0;4)$ & 4 & $(75,29)$ & 9 & 1 & YES & YES & NO(2) & $1.56$ & $(4,2)$ & -- & 12012\\
$(c;0,0,0;4)$ & 4 & $(75,31)$ & 9 & 1 & YES & YES & YES & $1.62$ & $(2,3)$ & -- & 12013\\
$(c;0,0,0;4)$ & 4 & $(79,30)$ & 9 & 1 & YES & YES & YES & $1.50$ & $(2,3)$ & -- & 12014\\
$(c;0,0,0;4)$ & 4 & $(80,31)$ & 9 & 4 & YES & YES & YES & $1.50$ & $(2,3)$ & -- & 12015\\
$(c;0,0,0;4)$ & 4 & $(81,31)$ & 9 & 1 & YES & YES & YES & $1.62$ & $(2,3)$ & -- & 12016\\
$(c;0,0,0;4)$ & 4 & $(81,34)$ & 9 & 1 & YES & YES & YES & $1.29$ & $(6,1)$ & -- & 12017\\
$(c;0,0,0;4)$ & 4 & $(82,25)$ & 10 & 2 & YES & YES & YES & $1.29$ & $(6,1)$ & -- & 12018\\
$(c;0,0,0;4)$ & 4 & $(87,23)$ & 10 & 1 & YES & YES & NO(2) & $1.12$ & $(10,-1)$ & -- & 12019\\
$(c;0,0,0;4)$ & 4 & $(87,32)$ & 10 & 1 & YES & YES & YES & $1.75$ & $(2,3)$ & -- & 12020\\
$(c;0,0,0;4)$ & 4 & $(89,26)$ & 10 & 1 & YES & YES & YES & $1.70$ & $(2,3)$ & -- & 12021\\
$(c;0,0,0;4)$ & 4 & $(89,27)$ & 10 & 1 & YES & YES & YES & $1.14$ & $(6,1)$ & -- & 12022\\
$(c;0,0,0;4)$ & 4 & $(89,33)$ & 10 & 1 & YES & YES & YES & $1.43$ & $(4,2)$ & -- & 12023\\
$(c;0,0,0;4)$ & 4 & $(89,34)$ & 9 & 1 & YES & YES & YES & $1.43$ & $(6,1)$ & -- & 12024\\
$(c;0,0,0;4)$ & 4 & $(91,27)$ & 10 & 1 & YES & YES & YES & $1.50$ & $(2,3)$ & -- & 12025\\
$(c;0,0,0;4)$ & 4 & $(93,25)$ & 10 & 1 & YES & YES & YES & $1.50$ & $(4,2)$ & -- & 12026\\
$(c;0,0,0;4)$ & 4 & $(93,26)$ & 10 & 1 & YES & YES & YES & $1.78$ & $(2,3)$ & -- & 12027\\
$(c;0,0,0;4)$ & 4 & $(93,38)$ & 11 & 1 & YES & YES & NO(2) & $1.62$ & $(6,1)$ & -- & 12028\\
$(c;0,0,0;4)$ & 4 & $(94,41)$ & 10 & 2 & YES & YES & YES & $1.62$ & $(4,2)$ & -- & 12029\\
$(c;0,0,0;4)$ & 4 & $(95,29)$ & 10 & 1 & YES & YES & YES & $1.43$ & $(6,1)$ & -- & 12030\\
$(c;0,0,0;4)$ & 4 & $(95,36)$ & 10 & 1 & YES & YES & YES & $1.57$ & $(4,2)$ & -- & 12031\\
$(c;0,0,0;4)$ & 4 & $(95,39)$ & 10 & 1 & YES & YES & YES & $1.75$ & $(2,3)$ & -- & 12032\\
$(c;0,0,0;4)$ & 4 & $(99,29)$ & 10 & 1 & YES & YES & YES & $1.50$ & $(2,3)$ & -- & 12033\\
$(c;0,0,0;4)$ & 4 & $(99,41)$ & 10 & 1 & YES & YES & YES & $1.78$ & $(2,3)$ & -- & 12034\\
$(c;0,0,0;4)$ & 4 & $(100,37)$ & 10 & 4 & YES & YES & YES & $1.67$ & $(2,3)$ & -- & 12035\\
$(c;0,0,0;4)$ & 4 & $(100,41)$ & 10 & 4 & YES & YES & YES & $1.75$ & $(2,3)$ & -- & 12036\\
$(c;0,0,0;4)$ & 4 & $(101,37)$ & 10 & 1 & YES & YES & YES & $1.62$ & $(4,2)$ & -- & 12037\\
$(c;0,0,0;4)$ & 4 & $(101,39)$ & 10 & 1 & YES & YES & YES & $1.57$ & $(2,3)$ & -- & 12038\\
$(c;0,0,0;4)$ & 4 & $(104,43)$ & 10 & 4 & YES & YES & YES & $1.57$ & $(2,3)$ & -- & 12039\\
$(c;0,0,0;4)$ & 4 & $(105,31)$ & 10 & 1 & YES & YES & YES & $1.62$ & $(2,3)$ & -- & 12040\\
$(c;0,0,0;4)$ & 4 & $(105,44)$ & 10 & 1 & YES & YES & YES & $1.80$ & $(2,3)$ & -- & 12041\\
$(c;0,0,0;4)$ & 4 & $(106,31)$ & 10 & 2 & YES & YES & YES & $1.50$ & $(2,3)$ & -- & 12042\\
$(c;0,0,0;4)$ & 4 & $(106,41)$ & 10 & 2 & YES & YES & YES & $1.62$ & $(4,2)$ & -- & 12043\\
$(c;0,0,0;4)$ & 4 & $(107,30)$ & 11 & 1 & YES & YES & YES & $1.43$ & $(4,2)$ & -- & 12044\\
$(c;0,0,0;4)$ & 4 & $(108,29)$ & 10 & 4 & YES & YES & YES & $1.43$ & $(6,1)$ & -- & 12045\\
$(c;0,0,0;4)$ & 4 & $(109,40)$ & 10 & 1 & YES & YES & YES & $1.75$ & $(2,3)$ & -- & 12046\\
$(c;0,0,0;4)$ & 4 & $(109,45)$ & 10 & 1 & YES & YES & YES & $1.75$ & $(2,3)$ & -- & 12047\\
$(c;0,0,0;4)$ & 4 & $(109,46)$ & 10 & 1 & YES & YES & YES & $1.62$ & $(4,2)$ & -- & 12048\\
$(c;0,0,0;4)$ & 4 & $(111,43)$ & 10 & 1 & YES & YES & YES & $1.67$ & $(2,3)$ & -- & 12049\\
$(c;0,0,0;4)$ & 4 & $(111,46)$ & 10 & 1 & YES & YES & YES & $1.62$ & $(2,3)$ & -- & 12050\\
$(c;0,0,0;4)$ & 4 & $(115,26)$ & 11 & 1 & YES & YES & YES & $1.67$ & $(2,3)$ & -- & 12051\\
$(c;0,0,0;4)$ & 4 & $(115,31)$ & 11 & 1 & YES & YES & YES & $1.57$ & $(4,2)$ & -- & 12052\\
$(c;0,0,0;4)$ & 4 & $(115,34)$ & 10 & 1 & YES & YES & YES & $1.43$ & $(6,1)$ & -- & 12053\\
$(c;0,0,0;4)$ & 4 & $(115,44)$ & 10 & 1 & YES & YES & YES & $1.57$ & $(2,3)$ & -- & 12054\\
$(c;0,0,0;4)$ & 4 & $(117,49)$ & 10 & 1 & YES & YES & YES & $1.67$ & $(4,2)$ & -- & 12055\\
$(c;0,0,0;4)$ & 4 & $(119,32)$ & 11 & 1 & YES & YES & YES & $1.75$ & $(2,3)$ & -- & 12056\\
$(c;0,0,0;4)$ & 4 & $(119,44)$ & 10 & 1 & YES & YES & YES & $1.62$ & $(4,2)$ & -- & 12057\\
$(c;0,0,0;4)$ & 4 & $(119,50)$ & 10 & 1 & YES & YES & YES & $1.57$ & $(2,3)$ & -- & 12058\\
$(c;0,0,0;4)$ & 4 & $(122,33)$ & 11 & 2 & YES & YES & YES & $1.43$ & $(4,2)$ & -- & 12059\\
$(c;0,0,0;4)$ & 4 & $(124,23)$ & 12 & 4 & YES & YES & YES & $1.43$ & $(6,1)$ & -- & 12060\\
$(c;0,0,0;4)$ & 4 & $(131,48)$ & 11 & 1 & YES & YES & YES & $1.78$ & $(4,2)$ & -- & 12061\\
$(c;0,0,0;4)$ & 4 & $(140,41)$ & 11 & 4 & YES & YES & YES & $1.67$ & $(2,3)$ & -- & 12062\\
$(c;0,0,0;4)$ & 4 & $(144,55)$ & 10 & 4 & YES & YES & YES & $1.57$ & $(2,3)$ & -- & 12063\\
$(c;0,0,0;4)$ & 4 & $(145,44)$ & 11 & 1 & YES & YES & YES & $1.86$ & $(2,3)$ & -- & 12064\\
$(c;0,0,0;4)$ & 4 & $(146,41)$ & 11 & 2 & YES & YES & YES & $1.57$ & $(2,3)$ & -- & 12065\\
$(c;0,0,0;4)$ & 4 & $(147,43)$ & 11 & 1 & YES & YES & YES & $1.57$ & $(2,3)$ & -- & 12066\\
$(c;0,0,0;4)$ & 4 & $(153,35)$ & 12 & 1 & YES & YES & YES & $1.62$ & $(2,3)$ & -- & 12067\\
$(c;0,0,0;4)$ & 4 & $(157,46)$ & 11 & 1 & YES & YES & YES & $1.62$ & $(2,3)$ & -- & 12068\\
$(c;0,0,0;4)$ & 4 & $(179,48)$ & 12 & 1 & YES & YES & YES & $1.78$ & $(4,2)$ & -- & 12069\\
$(c;0,1,0;11)$ & 5 & $(29,11)$ & 7 & 1 & YES & YES & NO(2) & $1.44$ & $(4,2)$ & -- & 12070\\
$(c;0,1,0;11)$ & 5 & $(29,12)$ & 7 & 1 & YES & YES & NO(2) & $1.50$ & $(6,1)$ & -- & 12071\\
$(c;0,1,0;11)$ & 5 & $(31,13)$ & 7 & 1 & YES & YES & YES & $1.50$ & $(2,3)$ & -- & 12072\\
$(c;0,1,0;11)$ & 5 & $(35,11)$ & 9 & 1 & YES & YES & NO(2) & $1.80$ & $(6,1)$ & -- & 12073\\
$(c;0,1,0;11)$ & 5 & $(37,11)$ & 8 & 1 & YES & YES & NO(2) & $1.44$ & $(4,2)$ & -- & 12074\\
$(c;0,1,0;11)$ & 5 & $(37,14)$ & 8 & 1 & YES & YES & NO(2) & $1.50$ & $(6,1)$ & -- & 12075\\
$(c;0,1,0;11)$ & 5 & $(39,17)$ & 8 & 1 & YES & YES & NO(2) & $1.33$ & $(8,0)$ & -- & 12076\\
$(c;0,1,0;11)$ & 5 & $(40,17)$ & 9 & 1 & YES & YES & NO(2) & $1.44$ & $(8,0)$ & -- & 12077\\
$(c;0,1,0;11)$ & 5 & $(41,18)$ & 8 & 1 & YES & YES & NO(2) & $1.33$ & $(8,0)$ & -- & 12078\\
$(c;0,1,0;11)$ & 5 & $(43,12)$ & 8 & 1 & YES & YES & NO(2) & $1.50$ & $(6,1)$ & -- & 12079\\
$(c;0,1,0;11)$ & 5 & $(44,17)$ & 8 & 11 & YES & YES & NO(2) & $1.60$ & $(2,3)$ & -- & 12080\\
$(c;0,1,0;11)$ & 5 & $(47,13)$ & 8 & 1 & YES & YES & NO(2) & $1.50$ & $(2,3)$ & -- & 12081\\
$(c;0,1,0;11)$ & 5 & $(47,18)$ & 8 & 1 & YES & YES & YES & $1.38$ & $(2,3)$ & -- & 12082\\
$(c;0,1,0;11)$ & 5 & $(49,18)$ & 8 & 1 & YES & YES & YES & $1.62$ & $(4,2)$ & -- & 12083\\
$(c;0,1,0;11)$ & 5 & $(50,19)$ & 8 & 1 & YES & YES & YES & $1.62$ & $(2,3)$ & -- & 12084\\
$(c;0,1,0;11)$ & 5 & $(51,11)$ & 9 & 1 & YES & YES & NO(2) & $1.50$ & $(2,3)$ & -- & 12085\\
$(c;0,1,0;11)$ & 5 & $(53,11)$ & 10 & 1 & YES & YES & NO(2) & $1.33$ & $(8,0)$ & -- & 12086\\
$(c;0,1,0;11)$ & 5 & $(53,22)$ & 9 & 1 & YES & YES & NO(2) & $1.44$ & $(8,0)$ & -- & 12087\\
$(c;0,1,0;11)$ & 5 & $(55,21)$ & 8 & 11 & YES & YES & YES & $1.38$ & $(2,3)$ & -- & 12088\\
$(c;0,1,0;11)$ & 5 & $(55,24)$ & 9 & 11 & YES & YES & YES & $1.43$ & $(4,2)$ & -- & 12089\\
$(c;0,1,0;11)$ & 5 & $(57,22)$ & 9 & 1 & YES & YES & YES & $1.57$ & $(4,2)$ & -- & 12090\\
$(c;0,1,0;11)$ & 5 & $(57,25)$ & 9 & 1 & YES & YES & YES & $1.57$ & $(4,2)$ & -- & 12091\\
$(c;0,1,0;11)$ & 5 & $(58,17)$ & 9 & 1 & YES & YES & YES & $1.50$ & $(4,2)$ & -- & 12092\\
$(c;0,1,0;11)$ & 5 & $(59,23)$ & 9 & 1 & YES & YES & YES & $1.75$ & $(2,3)$ & -- & 12093\\
$(c;0,1,0;11)$ & 5 & $(59,25)$ & 9 & 1 & YES & YES & NO(2) & $1.50$ & $(4,2)$ & -- & 12094\\
$(c;0,1,0;11)$ & 5 & $(62,27)$ & 9 & 1 & YES & YES & YES & $1.29$ & $(4,2)$ & -- & 12095\\
$(c;0,1,0;11)$ & 5 & $(63,26)$ & 9 & 1 & YES & YES & YES & $1.89$ & $(2,3)$ & -- & 12096\\
$(c;0,1,0;11)$ & 5 & $(64,17)$ & 10 & 1 & YES & YES & NO(2) & $1.44$ & $(8,0)$ & -- & 12097\\
$(c;0,1,0;11)$ & 5 & $(64,27)$ & 9 & 1 & YES & YES & YES & $1.75$ & $(2,3)$ & -- & 12098\\
$(c;0,1,0;11)$ & 5 & $(65,24)$ & 9 & 1 & YES & YES & YES & $1.67$ & $(2,3)$ & -- & 12099\\
$(c;0,1,0;11)$ & 5 & $(67,14)$ & 10 & 1 & YES & YES & NO(2) & $1.33$ & $(8,0)$ & -- & 12100\\
$(c;0,1,0;11)$ & 5 & $(70,29)$ & 9 & 1 & YES & YES & YES & $1.62$ & $(2,3)$ & -- & 12101\\
$(c;0,1,0;11)$ & 5 & $(71,21)$ & 9 & 1 & YES & YES & NO(2) & $1.56$ & $(8,0)$ & -- & 12102\\
$(c;0,1,0;11)$ & 5 & $(74,31)$ & 9 & 1 & YES & YES & YES & $1.62$ & $(2,3)$ & -- & 12103\\
$(c;0,1,0;11)$ & 5 & $(75,22)$ & 10 & 1 & YES & YES & YES & $1.57$ & $(4,2)$ & -- & 12104\\
$(c;0,1,0;11)$ & 5 & $(75,29)$ & 9 & 1 & YES & YES & YES & $1.43$ & $(4,2)$ & -- & 12105\\
$(c;0,1,0;11)$ & 5 & $(75,31)$ & 9 & 1 & YES & YES & YES & $1.71$ & $(2,3)$ & -- & 12106\\
$(c;0,1,0;11)$ & 5 & $(79,22)$ & 10 & 1 & YES & YES & YES & $1.43$ & $(4,2)$ & -- & 12107\\
$(c;0,1,0;11)$ & 5 & $(79,24)$ & 10 & 1 & YES & YES & YES & $1.78$ & $(2,3)$ & -- & 12108\\
$(c;0,1,0;11)$ & 5 & $(79,29)$ & 9 & 1 & YES & YES & YES & $1.67$ & $(4,2)$ & -- & 12109\\
$(c;0,1,0;11)$ & 5 & $(79,30)$ & 9 & 1 & YES & YES & YES & $1.86$ & $(2,3)$ & -- & 12110\\
$(c;0,1,0;11)$ & 5 & $(89,24)$ & 10 & 1 & YES & YES & YES & $1.50$ & $(2,3)$ & -- & 12111\\
$(c;0,1,0;11)$ & 5 & $(89,34)$ & 9 & 1 & YES & YES & YES & $1.67$ & $(4,2)$ & -- & 12112\\
$(c;0,1,0;11)$ & 5 & $(99,29)$ & 10 & 11 & YES & YES & YES & $1.50$ & $(4,2)$ & -- & 12113\\
$(c;0,1,0;11)$ & 5 & $(101,22)$ & 11 & 1 & YES & YES & YES & $1.62$ & $(2,3)$ & -- & 12114\\
$(c;0,1,0;11)$ & 5 & $(101,23)$ & 11 & 1 & YES & YES & YES & $1.57$ & $(2,3)$ & -- & 12115\\
$(c;0,1,0;11)$ & 5 & $(104,29)$ & 10 & 1 & YES & YES & YES & $1.78$ & $(4,2)$ & -- & 12116\\
$(c;0,1,0;11)$ & 5 & $(106,31)$ & 10 & 1 & YES & YES & YES & $1.62$ & $(4,2)$ & -- & 12117\\
$(c;0,1,1;5)$ & 6 & $(26,11)$ & 7 & 1 & YES & YES & NO(2) & $1.50$ & $(6,1)$ & -- & 12118\\
$(c;0,1,1;5)$ & 6 & $(27,8)$ & 7 & 1 & YES & YES & NO(2) & $1.25$ & $(6,1)$ & -- & 12119\\
$(c;0,1,1;5)$ & 6 & $(29,11)$ & 7 & 1 & YES & YES & YES & $1.56$ & $(2,3)$ & -- & 12120\\
$(c;0,1,1;5)$ & 6 & $(41,12)$ & 8 & 1 & YES & YES & NO(2) & $1.38$ & $(6,1)$ & -- & 12121\\
$(c;0,1,1;5)$ & 6 & $(41,17)$ & 8 & 1 & YES & YES & YES & $1.75$ & $(4,2)$ & -- & 12122\\
$(c;0,1,1;5)$ & 6 & $(43,12)$ & 8 & 1 & YES & YES & NO(2) & $1.64$ & $(4,2)$ & -- & 12123\\
$(c;0,1,1;5)$ & 6 & $(46,17)$ & 8 & 1 & YES & YES & YES & $1.67$ & $(2,3)$ & -- & 12124\\
$(c;0,1,1;5)$ & 6 & $(61,17)$ & 9 & 1 & YES & YES & YES & $1.78$ & $(2,3)$ & -- & 12125\\
$(c;0,2,0;7)$ & 6 & $(24,7)$ & 7 & 1 & YES & YES & NO(2) & $1.44$ & $(4,2)$ & -- & 12126\\
$(c;0,2,0;7)$ & 6 & $(27,8)$ & 7 & 1 & YES & YES & NO(2) & $1.50$ & $(2,3)$ & -- & 12127\\
$(c;0,2,0;7)$ & 6 & $(35,11)$ & 9 & 7 & YES & YES & NO(2) & $1.75$ & $(6,1)$ & -- & 12128\\
$(c;0,2,0;7)$ & 6 & $(39,17)$ & 8 & 1 & YES & YES & NO(2) & $1.33$ & $(8,0)$ & -- & 12129\\
$(c;0,2,0;7)$ & 6 & $(40,11)$ & 8 & 1 & YES & YES & NO(2) & $1.60$ & $(2,3)$ & -- & 12130\\
$(c;0,2,0;7)$ & 6 & $(41,16)$ & 8 & 1 & YES & YES & NO(2) & $1.38$ & $(4,2)$ & -- & 12131\\
$(c;0,2,0;7)$ & 6 & $(41,18)$ & 8 & 1 & YES & YES & NO(2) & $1.33$ & $(8,0)$ & -- & 12132\\
$(c;0,2,0;7)$ & 6 & $(42,11)$ & 9 & 7 & YES & YES & NO(2) & $1.44$ & $(4,2)$ & -- & 12133\\
$(c;0,2,0;7)$ & 6 & $(44,17)$ & 8 & 1 & YES & YES & YES & $1.78$ & $(2,3)$ & -- & 12134\\
$(c;0,2,0;7)$ & 6 & $(47,18)$ & 8 & 1 & YES & YES & YES & $1.50$ & $(4,2)$ & -- & 12135\\
$(c;0,2,0;7)$ & 6 & $(48,11)$ & 9 & 1 & YES & YES & NO(2) & $1.44$ & $(4,2)$ & -- & 12136\\
$(c;0,2,0;7)$ & 6 & $(49,18)$ & 8 & 7 & YES & YES & YES & $1.50$ & $(2,3)$ & -- & 12137\\
$(c;0,2,0;7)$ & 6 & $(53,11)$ & 10 & 1 & YES & YES & NO(2) & $1.50$ & $(6,1)$ & -- & 12138\\
$(c;0,2,0;7)$ & 6 & $(64,19)$ & 9 & 1 & YES & YES & YES & $1.57$ & $(2,3)$ & -- & 12139\\
$(c;0,2,0;7)$ & 6 & $(67,14)$ & 10 & 1 & YES & YES & NO(2) & $1.33$ & $(8,0)$ & -- & 12140\\
$(c;0,2,0;7)$ & 6 & $(79,18)$ & 10 & 1 & YES & YES & YES & $1.50$ & $(4,2)$ & -- & 12141\\
$(c;0,2,1;19)$ & 7 & $(18,7)$ & 6 & 1 & YES & YES & NO(2) & $1.56$ & $(4,2)$ & -- & 12142\\
$(c;0,2,1;19)$ & 7 & $(27,7)$ & 9 & 1 & YES & YES & NO(2) & $1.50$ & $(6,1)$ & -- & 12143\\
$(c;0,2,1;19)$ & 7 & $(35,8)$ & 8 & 1 & YES & YES & NO(2) & $1.50$ & $(6,1)$ & -- & 12144\\
$(c;0,2,1;19)$ & 7 & $(41,12)$ & 8 & 1 & YES & YES & YES & $1.78$ & $(2,3)$ & -- & 12145\\
$(c;0,2,1;19)$ & 7 & $(43,10)$ & 9 & 1 & YES & YES & NO(2) & $1.56$ & $(4,2)$ & -- & 12146\\
$(c;0,2,1;19)$ & 7 & $(43,12)$ & 8 & 1 & YES & YES & YES & $1.62$ & $(4,2)$ & -- & 12147\\
$(c;0,2,2;6)$ & 8 & $(16,5)$ & 7 & 2 & YES & YES & NO(2) & $1.44$ & $(8,0)$ & -- & 12148\\
$(c;0,3,0;17)$ & 7 & $(23,7)$ & 7 & 1 & YES & YES & NO(2) & $1.56$ & $(4,2)$ & -- & 12149\\
$(c;0,3,0;17)$ & 7 & $(33,7)$ & 8 & 1 & YES & YES & NO(2) & $1.56$ & $(4,2)$ & -- & 12150\\
$(c;0,3,1;23)$ & 8 & $(17,5)$ & 6 & 1 & YES & YES & NO(2) & $1.64$ & $(4,2)$ & -- & 12151\\
$(c;0,3,1;23)$ & 8 & $(30,7)$ & 8 & 1 & YES & YES & NO(2) & $1.38$ & $(10,-1)$ & -- & 12152\\
$(c;0,3,1;23)$ & 8 & $(32,7)$ & 8 & 1 & YES & YES & YES & $1.71$ & $(2,3)$ & -- & 12153\\
$(d;0,0,0;5)$ & 5 & $(31,14)$ & 8 & 1 & YES & YES & NO(2) & $1.70$ & $(6,1)$ & -- & 12154\\
$(d;0,0,0;5)$ & 5 & $(34,9)$ & 8 & 1 & YES & YES & NO(2) & $1.60$ & $(6,1)$ & -- & 12155\\
$(d;0,0,0;5)$ & 5 & $(34,13)$ & 7 & 1 & YES & YES & YES & $1.38$ & $(4,2)$ & -- & 12156\\
$(d;0,0,0;5)$ & 5 & $(36,11)$ & 8 & 1 & YES & YES & NO(2) & $1.50$ & $(2,3)$ & -- & 12157\\
$(d;0,0,0;5)$ & 5 & $(36,13)$ & 8 & 1 & YES & YES & NO(2) & $1.70$ & $(2,3)$ & -- & 12158\\
$(d;0,0,0;5)$ & 5 & $(37,14)$ & 8 & 1 & YES & YES & NO(2) & $1.50$ & $(6,1)$ & -- & 12159\\
$(d;0,0,0;5)$ & 5 & $(39,14)$ & 8 & 1 & YES & YES & NO(2) & $1.44$ & $(4,2)$ & -- & 12160\\
$(d;0,0,0;5)$ & 5 & $(40,11)$ & 8 & 5 & YES & YES & YES & $1.50$ & $(2,3)$ & -- & 12161\\
$(d;0,0,0;5)$ & 5 & $(41,15)$ & 8 & 1 & YES & YES & NO(2) & $1.38$ & $(10,-1)$ & -- & 12162\\
$(d;0,0,0;5)$ & 5 & $(41,16)$ & 8 & 1 & YES & YES & NO(2) & $1.67$ & $(4,2)$ & -- & 12163\\
$(d;0,0,0;5)$ & 5 & $(43,19)$ & 9 & 1 & YES & YES & NO(2) & $1.33$ & $(8,0)$ & -- & 12164\\
$(d;0,0,0;5)$ & 5 & $(44,17)$ & 8 & 1 & YES & YES & NO(2) & $1.56$ & $(2,3)$ & -- & 12165\\
$(d;0,0,0;5)$ & 5 & $(45,19)$ & 8 & 5 & YES & YES & NO(2) & $1.67$ & $(2,3)$ & -- & 12166\\
$(d;0,0,0;5)$ & 5 & $(46,19)$ & 8 & 1 & YES & YES & YES & $1.50$ & $(2,3)$ & -- & 12167\\
$(d;0,0,0;5)$ & 5 & $(47,18)$ & 8 & 1 & YES & YES & NO(2) & $1.50$ & $(6,1)$ & -- & 12168\\
$(d;0,0,0;5)$ & 5 & $(49,19)$ & 8 & 1 & YES & YES & NO(2) & $1.56$ & $(2,3)$ & -- & 12169\\
$(d;0,0,0;5)$ & 5 & $(50,19)$ & 8 & 5 & YES & YES & YES & $1.38$ & $(2,3)$ & -- & 12170\\
$(d;0,0,0;5)$ & 5 & $(50,21)$ & 8 & 5 & YES & YES & YES & $1.29$ & $(6,1)$ & -- & 12171\\
$(d;0,0,0;5)$ & 5 & $(55,21)$ & 8 & 5 & YES & YES & YES & $1.50$ & $(2,3)$ & -- & 12172\\
$(d;0,0,0;5)$ & 5 & $(55,23)$ & 9 & 5 & YES & YES & NO(2) & $1.60$ & $(6,1)$ & -- & 12173\\
$(d;0,0,0;5)$ & 5 & $(57,22)$ & 9 & 1 & YES & YES & YES & $1.43$ & $(4,2)$ & -- & 12174\\
$(d;0,0,0;5)$ & 5 & $(57,25)$ & 9 & 1 & YES & YES & YES & $1.43$ & $(4,2)$ & -- & 12175\\
$(d;0,0,0;5)$ & 5 & $(61,18)$ & 9 & 1 & YES & YES & NO(2) & $1.40$ & $(6,1)$ & -- & 12176\\
$(d;0,0,0;5)$ & 5 & $(63,26)$ & 9 & 1 & YES & YES & YES & $1.70$ & $(2,3)$ & -- & 12177\\
$(d;0,0,0;5)$ & 5 & $(64,19)$ & 9 & 1 & YES & YES & YES & $1.38$ & $(2,3)$ & -- & 12178\\
$(d;0,0,0;5)$ & 5 & $(64,27)$ & 9 & 1 & YES & YES & YES & $1.43$ & $(4,2)$ & -- & 12179\\
$(d;0,0,0;5)$ & 5 & $(65,19)$ & 9 & 5 & YES & YES & NO(2) & $1.56$ & $(2,3)$ & -- & 12180\\
$(d;0,0,0;5)$ & 5 & $(65,24)$ & 9 & 5 & YES & YES & YES & $1.50$ & $(2,3)$ & -- & 12181\\
$(d;0,0,0;5)$ & 5 & $(69,29)$ & 9 & 1 & YES & YES & YES & $1.43$ & $(2,3)$ & -- & 12182\\
$(d;0,0,0;5)$ & 5 & $(70,29)$ & 9 & 5 & YES & YES & YES & $1.43$ & $(4,2)$ & -- & 12183\\
$(d;0,0,0;5)$ & 5 & $(71,21)$ & 9 & 1 & YES & YES & YES & $1.50$ & $(2,3)$ & -- & 12184\\
$(d;0,0,0;5)$ & 5 & $(71,26)$ & 9 & 1 & YES & YES & YES & $1.56$ & $(4,2)$ & -- & 12185\\
$(d;0,0,0;5)$ & 5 & $(74,31)$ & 9 & 1 & YES & YES & YES & $1.56$ & $(4,2)$ & -- & 12186\\
$(d;0,0,0;5)$ & 5 & $(75,31)$ & 9 & 5 & YES & YES & YES & $1.80$ & $(2,3)$ & -- & 12187\\
$(d;0,0,0;5)$ & 5 & $(78,23)$ & 10 & 1 & YES & YES & YES & $1.57$ & $(2,3)$ & -- & 12188\\
$(d;0,0,0;5)$ & 5 & $(79,22)$ & 10 & 1 & YES & YES & YES & $1.50$ & $(2,3)$ & -- & 12189\\
$(d;0,0,0;5)$ & 5 & $(79,23)$ & 10 & 1 & YES & YES & YES & $1.67$ & $(2,3)$ & -- & 12190\\
$(d;0,0,0;5)$ & 5 & $(79,24)$ & 10 & 1 & YES & YES & YES & $1.62$ & $(2,3)$ & -- & 12191\\
$(d;0,0,0;5)$ & 5 & $(85,26)$ & 10 & 5 & YES & YES & YES & $1.56$ & $(4,2)$ & -- & 12192\\
$(d;0,0,0;5)$ & 5 & $(104,29)$ & 10 & 1 & YES & YES & YES & $1.43$ & $(4,2)$ & -- & 12193\\
$(d;0,0,1;14)$ & 6 & $(19,8)$ & 6 & 1 & YES & YES & YES & $1.43$ & $(2,3)$ & -- & 12194\\
$(d;0,0,1;14)$ & 6 & $(23,10)$ & 7 & 1 & YES & YES & YES & $1.43$ & $(4,2)$ & -- & 12195\\
$(d;0,0,1;14)$ & 6 & $(26,11)$ & 7 & 2 & YES & YES & YES & $1.43$ & $(2,3)$ & -- & 12196\\
$(d;0,0,1;14)$ & 6 & $(27,8)$ & 7 & 1 & YES & YES & NO(2) & $1.25$ & $(6,1)$ & -- & 12197\\
$(d;0,0,1;14)$ & 6 & $(27,10)$ & 7 & 1 & YES & YES & NO(2) & $1.56$ & $(4,2)$ & -- & 12198\\
$(d;0,0,1;14)$ & 6 & $(29,11)$ & 7 & 1 & YES & YES & NO(2) & $1.60$ & $(6,1)$ & -- & 12199\\
$(d;0,0,1;14)$ & 6 & $(34,13)$ & 7 & 2 & YES & YES & NO(2) & $1.56$ & $(2,3)$ & -- & 12200\\
$(d;0,0,1;14)$ & 6 & $(37,11)$ & 8 & 1 & YES & YES & YES & $1.67$ & $(4,2)$ & -- & 12201\\
$(d;0,0,1;14)$ & 6 & $(41,12)$ & 8 & 1 & YES & YES & NO(2) & $1.50$ & $(6,1)$ & -- & 12202\\
$(d;0,0,1;14)$ & 6 & $(41,17)$ & 8 & 1 & YES & YES & YES & $1.62$ & $(4,2)$ & -- & 12203\\
$(d;0,0,1;14)$ & 6 & $(43,12)$ & 8 & 1 & YES & YES & NO(2) & $1.64$ & $(4,2)$ & -- & 12204\\
$(d;0,0,1;14)$ & 6 & $(43,18)$ & 8 & 1 & YES & YES & YES & $1.78$ & $(2,3)$ & -- & 12205\\
$(d;0,0,1;14)$ & 6 & $(44,13)$ & 8 & 2 & YES & YES & NO(2) & $1.56$ & $(2,3)$ & -- & 12206\\
$(d;0,0,1;14)$ & 6 & $(44,17)$ & 8 & 2 & YES & YES & YES & $1.89$ & $(2,3)$ & -- & 12207\\
$(d;0,0,1;14)$ & 6 & $(45,19)$ & 8 & 1 & YES & YES & YES & $1.62$ & $(4,2)$ & -- & 12208\\
$(d;0,0,1;14)$ & 6 & $(46,17)$ & 8 & 2 & YES & YES & YES & $1.67$ & $(2,3)$ & -- & 12209\\
$(d;0,0,1;14)$ & 6 & $(46,19)$ & 8 & 2 & YES & YES & YES & $1.71$ & $(2,3)$ & -- & 12210\\
$(d;0,0,1;14)$ & 6 & $(47,18)$ & 8 & 1 & YES & YES & YES & $1.50$ & $(4,2)$ & -- & 12211\\
$(d;0,0,1;14)$ & 6 & $(49,19)$ & 8 & 7 & YES & YES & YES & $1.78$ & $(2,3)$ & -- & 12212\\
$(d;0,0,1;14)$ & 6 & $(50,19)$ & 8 & 2 & YES & YES & YES & $1.62$ & $(4,2)$ & -- & 12213\\
$(d;0,0,1;14)$ & 6 & $(50,21)$ & 8 & 2 & YES & YES & YES & $1.75$ & $(2,3)$ & -- & 12214\\
$(d;0,0,1;14)$ & 6 & $(51,14)$ & 9 & 1 & YES & YES & YES & $1.50$ & $(4,2)$ & -- & 12215\\
$(d;0,0,1;14)$ & 6 & $(55,16)$ & 9 & 1 & YES & YES & YES & $1.75$ & $(2,3)$ & -- & 12216\\
$(d;0,0,1;14)$ & 6 & $(55,21)$ & 8 & 1 & YES & YES & YES & $1.57$ & $(2,3)$ & -- & 12217\\
$(d;0,0,1;14)$ & 6 & $(65,14)$ & 10 & 1 & YES & YES & YES & $1.50$ & $(4,2)$ & -- & 12218\\
$(d;0,0,2;9)$ & 7 & $(17,5)$ & 6 & 1 & YES & YES & NO(2) & $1.38$ & $(6,1)$ & -- & 12219\\
$(d;0,0,2;9)$ & 7 & $(18,7)$ & 6 & 9 & YES & YES & NO(2) & $1.70$ & $(6,1)$ & -- & 12220\\
$(d;0,0,2;9)$ & 7 & $(21,8)$ & 6 & 3 & YES & YES & NO(2) & $1.56$ & $(2,3)$ & -- & 12221\\
$(d;0,0,2;9)$ & 7 & $(34,7)$ & 10 & 1 & YES & YES & NO(2) & $1.50$ & $(6,1)$ & -- & 12222\\
$(d;0,0,2;9)$ & 7 & $(37,11)$ & 8 & 1 & YES & YES & YES & $1.78$ & $(2,3)$ & -- & 12223\\
$(d;0,0,2;9)$ & 7 & $(41,12)$ & 8 & 1 & YES & YES & YES & $1.43$ & $(4,2)$ & -- & 12224\\
$(d;0,0,3;22)$ & 8 & $(18,7)$ & 6 & 2 & YES & YES & NO(2) & $1.44$ & $(8,0)$ & -- & 12225\\
$(d;0,0,3;22)$ & 8 & $(23,7)$ & 7 & 1 & YES & YES & NO(2) & $1.56$ & $(4,2)$ & -- & 12226\\
$(d;0,1,0;6)$ & 6 & $(29,8)$ & 7 & 1 & YES & YES & YES & $1.50$ & $(2,3)$ & -- & 12227\\
$(d;0,1,0;6)$ & 6 & $(30,13)$ & 8 & 6 & YES & YES & NO(2) & $1.50$ & $(10,-1)$ & -- & 12228\\
$(d;0,1,0;6)$ & 6 & $(33,14)$ & 8 & 3 & YES & YES & YES & $1.43$ & $(2,3)$ & -- & 12229\\
$(d;0,1,0;6)$ & 6 & $(42,11)$ & 9 & 6 & YES & YES & NO(2) & $1.56$ & $(4,2)$ & -- & 12230\\
$(d;0,1,0;6)$ & 6 & $(43,18)$ & 8 & 1 & YES & YES & YES & $1.62$ & $(2,3)$ & -- & 12231\\
$(d;0,1,0;6)$ & 6 & $(44,17)$ & 8 & 2 & YES & YES & YES & $1.62$ & $(2,3)$ & -- & 12232\\
$(d;0,1,0;6)$ & 6 & $(47,18)$ & 8 & 1 & YES & YES & YES & $1.50$ & $(4,2)$ & -- & 12233\\
$(d;0,1,0;6)$ & 6 & $(49,13)$ & 9 & 1 & YES & YES & NO(2) & $1.56$ & $(4,2)$ & -- & 12234\\
$(d;0,1,0;6)$ & 6 & $(56,15)$ & 9 & 2 & YES & YES & NO(2) & $1.44$ & $(4,2)$ & -- & 12235\\
$(d;0,1,0;6)$ & 6 & $(58,17)$ & 9 & 2 & YES & YES & YES & $1.80$ & $(2,3)$ & -- & 12236\\
$(d;0,1,0;6)$ & 6 & $(61,17)$ & 9 & 1 & YES & YES & YES & $1.80$ & $(2,3)$ & -- & 12237\\
$(d;0,1,0;6)$ & 6 & $(79,18)$ & 10 & 1 & YES & YES & YES & $1.50$ & $(4,2)$ & -- & 12238\\
$(d;0,1,1;17)$ & 7 & $(34,13)$ & 7 & 17 & YES & YES & YES & $1.29$ & $(4,2)$ & -- & 12239\\
$(d;0,1,1;17)$ & 7 & $(37,11)$ & 8 & 1 & YES & YES & YES & $1.78$ & $(2,3)$ & -- & 12240\\
$(d;0,1,1;17)$ & 7 & $(41,12)$ & 8 & 1 & YES & YES & YES & $1.43$ & $(4,2)$ & -- & 12241\\
$(d;0,1,1;17)$ & 7 & $(41,17)$ & 8 & 1 & YES & YES & YES & $1.62$ & $(4,2)$ & -- & 12242\\
$(d;0,1,1;17)$ & 7 & $(47,11)$ & 9 & 1 & YES & YES & YES & $1.57$ & $(2,3)$ & -- & 12243\\
$(d;0,1,1;17)$ & 7 & $(48,11)$ & 9 & 1 & YES & YES & NO(2) & $1.50$ & $(6,1)$ & -- & 12244\\
$(d;0,1,2;11)$ & 8 & $(16,5)$ & 7 & 1 & YES & YES & NO(2) & $1.44$ & $(8,0)$ & -- & 12245\\
$(d;0,1,2;11)$ & 8 & $(19,8)$ & 6 & 1 & YES & YES & NO(2) & $1.38$ & $(4,2)$ & -- & 12246\\
$(d;0,2,0;7)$ & 7 & $(21,8)$ & 6 & 7 & YES & YES & NO(2) & $1.55$ & $(4,2)$ & -- & 12247\\
$(d;0,2,0;7)$ & 7 & $(23,7)$ & 7 & 1 & YES & YES & NO(2) & $1.67$ & $(4,2)$ & -- & 12248\\
$(d;0,2,1;20)$ & 8 & $(23,7)$ & 7 & 1 & YES & YES & NO(2) & $1.56$ & $(4,2)$ & -- & 12249\\
$(e;0,0,0;4)$ & 5 & $(19,7)$ & 6 & 1 & YES & YES & NO(2) & $1.50$ & $(2,3)$ & -- & 12250\\
$(e;0,0,0;4)$ & 5 & $(29,11)$ & 7 & 1 & YES & YES & NO(2) & $1.44$ & $(4,2)$ & -- & 12251\\
$(e;0,0,0;4)$ & 5 & $(31,12)$ & 7 & 1 & YES & YES & NO(2) & $1.44$ & $(12,-2)$ & -- & 12252\\
$(e;0,0,0;4)$ & 5 & $(32,9)$ & 8 & 4 & YES & YES & YES & $1.38$ & $(2,3)$ & -- & 12253\\
$(e;0,0,0;4)$ & 5 & $(34,13)$ & 7 & 2 & YES & YES & YES & $1.62$ & $(2,3)$ & -- & 12254\\
$(e;0,0,0;4)$ & 5 & $(37,11)$ & 8 & 1 & YES & YES & YES & $1.29$ & $(2,3)$ & -- & 12255\\
$(e;0,0,0;4)$ & 5 & $(43,12)$ & 8 & 1 & YES & YES & YES & $1.38$ & $(2,3)$ & -- & 12256\\
$(e;0,0,0;4)$ & 5 & $(46,17)$ & 8 & 2 & YES & YES & YES & $1.57$ & $(4,2)$ & -- & 12257\\
$(e;0,0,0;4)$ & 5 & $(46,19)$ & 8 & 2 & YES & YES & YES & $1.78$ & $(2,3)$ & -- & 12258\\
$(e;0,0,0;4)$ & 5 & $(49,19)$ & 8 & 1 & YES & YES & YES & $1.75$ & $(4,2)$ & -- & 12259\\
$(e;0,0,0;4)$ & 5 & $(50,21)$ & 8 & 2 & YES & YES & YES & $1.62$ & $(4,2)$ & -- & 12260\\
$(e;0,0,0;4)$ & 5 & $(55,21)$ & 8 & 1 & YES & YES & YES & $1.62$ & $(4,2)$ & -- & 12261\\
$(e;0,0,0;4)$ & 5 & $(58,17)$ & 9 & 2 & YES & YES & YES & $1.56$ & $(4,2)$ & -- & 12262\\
$(e;0,0,0;4)$ & 5 & $(61,18)$ & 9 & 1 & YES & YES & YES & $1.56$ & $(4,2)$ & -- & 12263\\
$(e;0,0,0;4)$ & 5 & $(75,29)$ & 9 & 1 & YES & YES & YES & $1.62$ & $(4,2)$ & -- & 12264\\
$(e;0,0,0;4)$ & 5 & $(79,22)$ & 10 & 1 & YES & YES & YES & $1.67$ & $(4,2)$ & -- & 12265\\
$(e;0,1,0;5)$ & 6 & $(43,18)$ & 8 & 1 & YES & YES & YES & $1.78$ & $(4,2)$ & -- & 12266\\
$(e;0,1,0;5)$ & 6 & $(44,13)$ & 8 & 1 & YES & YES & YES & $1.50$ & $(4,2)$ & -- & 12267\\
$(e;0,2,0;6)$ & 7 & $(25,7)$ & 7 & 1 & YES & YES & NO(2) & $1.44$ & $(8,0)$ & -- & 12268\\
$(e;0,2,0;6)$ & 7 & $(29,12)$ & 7 & 1 & YES & YES & YES & $1.75$ & $(4,2)$ & -- & 12269\\
$(e;0,2,0;6)$ & 7 & $(35,8)$ & 8 & 1 & YES & YES & NO(2) & $1.50$ & $(6,1)$ & -- & 12270\\
$(e;0,3,0;7)$ & 8 & $(13,5)$ & 5 & 1 & YES & YES & NO(2) & $1.56$ & $(8,0)$ & -- & 12271\\
$(e;1,0,0;18)$ & 6 & $(16,7)$ & 6 & 2 & YES & YES & NO(2) & $1.33$ & $(4,2)$ & -- & 12272\\
$(e;1,0,0;18)$ & 6 & $(19,8)$ & 6 & 1 & YES & YES & YES & $1.43$ & $(2,3)$ & -- & 12273\\
$(e;1,0,0;18)$ & 6 & $(21,8)$ & 6 & 3 & YES & YES & YES & $1.29$ & $(4,2)$ & -- & 12274\\
$(e;1,0,0;18)$ & 6 & $(23,7)$ & 7 & 1 & YES & YES & NO(2) & $1.33$ & $(4,2)$ & -- & 12275\\
$(e;1,0,0;18)$ & 6 & $(23,9)$ & 7 & 1 & YES & YES & NO(2) & $1.70$ & $(2,3)$ & -- & 12276\\
$(e;1,0,0;18)$ & 6 & $(24,7)$ & 7 & 6 & YES & YES & YES & $1.14$ & $(4,2)$ & -- & 12277\\
$(e;1,0,0;18)$ & 6 & $(31,9)$ & 8 & 1 & YES & YES & YES & $1.75$ & $(2,3)$ & -- & 12278\\
$(e;1,0,0;18)$ & 6 & $(31,12)$ & 7 & 1 & YES & YES & YES & $1.88$ & $(2,3)$ & -- & 12279\\
$(e;1,0,0;18)$ & 6 & $(31,13)$ & 7 & 1 & YES & YES & YES & $1.78$ & $(2,3)$ & -- & 12280\\
$(e;1,0,0;18)$ & 6 & $(33,10)$ & 8 & 3 & YES & YES & YES & $1.71$ & $(4,2)$ & -- & 12281\\
$(e;1,0,0;18)$ & 6 & $(34,13)$ & 7 & 2 & YES & YES & YES & $1.43$ & $(4,2)$ & -- & 12282\\
$(e;1,1,0;23)$ & 7 & $(11,4)$ & 5 & 1 & YES & YES & NO(2) & $1.56$ & $(4,2)$ & -- & 12283\\
$(e;1,1,0;23)$ & 7 & $(12,5)$ & 5 & 1 & YES & YES & YES & $1.29$ & $(4,2)$ & -- & 12284\\
$(e;1,1,0;23)$ & 7 & $(13,5)$ & 5 & 1 & YES & YES & NO(2) & $1.44$ & $(4,2)$ & -- & 12285\\
$(e;1,1,0;23)$ & 7 & $(17,7)$ & 6 & 1 & YES & YES & YES & $1.67$ & $(2,3)$ & -- & 12286\\
$(e;1,1,0;23)$ & 7 & $(18,7)$ & 6 & 1 & YES & YES & YES & $1.80$ & $(2,3)$ & -- & 12287\\
$(e;1,2,0;28)$ & 8 & $(10,3)$ & 5 & 2 & YES & YES & NO(2) & $1.50$ & $(4,2)$ & -- & 12288\\
$(e;1,2,0;28)$ & 8 & $(13,5)$ & 5 & 1 & YES & YES & YES & $1.62$ & $(4,2)$ & -- & 12289\\
$(e;1,2,0;28)$ & 8 & $(17,5)$ & 6 & 1 & YES & YES & YES & $1.50$ & $(4,2)$ & -- & 12290\\
$(e;1,2,0;28)$ & 8 & $(18,5)$ & 6 & 2 & YES & YES & YES & $1.78$ & $(2,3)$ & -- & 12291\\
$(e;1,3,0;33)$ & 9 & $(7,2)$ & 4 & 1 & YES & YES & NO(2) & $1.38$ & $(6,1)$ & -- & 12292\\
$(e;2,0,0;24)$ & 7 & $(11,4)$ & 5 & 1 & YES & YES & NO(2) & $1.56$ & $(4,2)$ & -- & 12293\\
$(e;2,0,0;24)$ & 7 & $(14,5)$ & 6 & 2 & YES & YES & NO(2) & $1.50$ & $(6,1)$ & -- & 12294\\
$(e;2,0,0;24)$ & 7 & $(16,5)$ & 7 & 8 & YES & YES & YES & $1.62$ & $(2,3)$ & -- & 12295\\
$(e;2,0,0;24)$ & 7 & $(17,7)$ & 6 & 1 & YES & YES & NO(2) & $1.44$ & $(8,0)$ & -- & 12296\\
$(e;2,0,0;24)$ & 7 & $(21,8)$ & 6 & 3 & YES & YES & YES & $1.78$ & $(2,3)$ & -- & 12297\\
$(e;2,1,0;31)$ & 8 & $(16,7)$ & 6 & 1 & YES & YES & YES & $1.88$ & $(4,2)$ & -- & 12298\\
$(e;2,1,0;31)$ & 8 & $(23,7)$ & 7 & 1 & YES & YES & YES & $1.88$ & $(4,2)$ & -- & 12299\\
$(e;2,2,0;38)$ & 9 & $(8,3)$ & 4 & 2 & YES & YES & NO(2) & $1.33$ & $(8,0)$ & -- & 12300\\
$(e;3,0,0;10)$ & 8 & $(10,3)$ & 5 & 10 & YES & YES & NO(2) & $1.50$ & $(6,1)$ & -- & 12301\\
$(e;3,0,0;10)$ & 8 & $(18,5)$ & 6 & 2 & YES & YES & YES & $1.43$ & $(4,2)$ & -- & 12302\\
$(e;3,0,0;10)$ & 8 & $(18,7)$ & 6 & 2 & YES & YES & NO(2) & $1.50$ & $(6,1)$ & -- & 12303\\
$(e;3,1,0;13)$ & 9 & $(8,3)$ & 4 & 1 & YES & YES & NO(2) & $1.33$ & $(8,0)$ & -- & 12304\\
$(e;3,2,0;16)$ & 10 & $(14,3)$ & 6 & 2 & YES & YES & NO(2) & $1.60$ & $(6,1)$ & -- & 12305\\
$(e;4,0,0;36)$ & 9 & $(7,2)$ & 4 & 1 & YES & YES & NO(2) & $1.67$ & $(8,0)$ & -- & 12306\\
$(e;5,2,0;68)$ & 12 & $(3,1)$ & 2 & 1 & YES & YES & NO(2) & $1.50$ & $(6,1)$ & -- & 12307\\
$(e;5,2,0;68)$ & 12 & $(4,1)$ & 3 & 4 & YES & YES & NO(2) & $1.60$ & $(6,1)$ & -- & 12308\\
$(f;0,0,0;6)$ & 4 & $(58,17)$ & 9 & 2 & YES & YES & NO(2) & $1.70$ & $(2,3)$ & -- & 12309\\
$(f;0,0,0;6)$ & 4 & $(61,17)$ & 9 & 1 & YES & YES & YES & $1.62$ & $(2,3)$ & -- & 12310\\
$(f;0,0,0;6)$ & 4 & $(64,19)$ & 9 & 2 & YES & YES & YES & $1.57$ & $(2,3)$ & -- & 12311\\
$(f;0,0,0;6)$ & 4 & $(64,23)$ & 9 & 2 & YES & YES & NO(2) & $1.73$ & $(4,2)$ & -- & 12312\\
$(f;0,0,0;6)$ & 4 & $(65,19)$ & 9 & 1 & YES & YES & YES & $1.62$ & $(2,3)$ & -- & 12313\\
$(f;0,0,0;6)$ & 4 & $(65,24)$ & 9 & 1 & YES & YES & NO(2) & $1.56$ & $(4,2)$ & -- & 12314\\
$(f;0,0,0;6)$ & 4 & $(70,27)$ & 10 & 2 & YES & YES & NO(2) & $1.80$ & $(2,3)$ & -- & 12315\\
$(f;0,0,0;6)$ & 4 & $(71,21)$ & 9 & 1 & YES & YES & YES & $1.50$ & $(2,3)$ & -- & 12316\\
$(f;0,0,0;6)$ & 4 & $(73,27)$ & 9 & 1 & YES & YES & NO(2) & $1.50$ & $(4,2)$ & -- & 12317\\
$(f;0,0,0;6)$ & 4 & $(73,28)$ & 10 & 1 & YES & YES & YES & $1.75$ & $(2,3)$ & -- & 12318\\
$(f;0,0,0;6)$ & 4 & $(75,22)$ & 10 & 3 & YES & YES & YES & $1.57$ & $(2,3)$ & -- & 12319\\
$(f;0,0,0;6)$ & 4 & $(75,23)$ & 11 & 3 & YES & YES & NO(2) & $1.56$ & $(12,-2)$ & -- & 12320\\
$(f;0,0,0;6)$ & 4 & $(76,21)$ & 9 & 2 & YES & YES & NO(2) & $1.60$ & $(2,3)$ & -- & 12321\\
$(f;0,0,0;6)$ & 4 & $(76,29)$ & 9 & 2 & YES & YES & NO(2) & $1.56$ & $(4,2)$ & -- & 12322\\
$(f;0,0,0;6)$ & 4 & $(79,30)$ & 9 & 1 & YES & YES & YES & $1.50$ & $(2,3)$ & -- & 12323\\
$(f;0,0,0;6)$ & 4 & $(81,29)$ & 11 & 3 & YES & YES & NO(2) & $1.70$ & $(6,1)$ & -- & 12324\\
$(f;0,0,0;6)$ & 4 & $(85,36)$ & 10 & 1 & YES & YES & NO(2) & $1.56$ & $(12,-2)$ & -- & 12325\\
$(f;0,0,0;6)$ & 4 & $(87,31)$ & 12 & 3 & YES & YES & YES & $1.71$ & $(4,2)$ & -- & 12326\\
$(f;0,0,0;6)$ & 4 & $(87,34)$ & 10 & 3 & YES & YES & NO(2) & $1.56$ & $(8,0)$ & -- & 12327\\
$(f;0,0,0;6)$ & 4 & $(89,33)$ & 10 & 1 & YES & YES & NO(2) & $1.56$ & $(8,0)$ & -- & 12328\\
$(f;0,0,0;6)$ & 4 & $(91,27)$ & 10 & 1 & YES & YES & NO(2) & $1.56$ & $(4,2)$ & -- & 12329\\
$(f;0,0,0;6)$ & 4 & $(92,27)$ & 11 & 2 & YES & YES & NO(2) & $1.67$ & $(2,3)$ & -- & 12330\\
$(f;0,0,0;6)$ & 4 & $(92,35)$ & 10 & 2 & YES & YES & NO(2) & $1.56$ & $(4,2)$ & -- & 12331\\
$(f;0,0,0;6)$ & 4 & $(95,28)$ & 11 & 1 & YES & YES & YES & $1.57$ & $(2,3)$ & -- & 12332\\
$(f;0,0,0;6)$ & 4 & $(97,37)$ & 10 & 1 & YES & YES & YES & $1.57$ & $(2,3)$ & -- & 12333\\
$(f;0,0,0;6)$ & 4 & $(97,40)$ & 11 & 1 & YES & YES & NO(2) & $1.62$ & $(6,1)$ & -- & 12334\\
$(f;0,0,0;6)$ & 4 & $(98,27)$ & 10 & 2 & YES & YES & NO(2) & $1.50$ & $(10,-1)$ & -- & 12335\\
$(f;0,0,0;6)$ & 4 & $(98,41)$ & 10 & 2 & YES & YES & NO(2) & $1.50$ & $(4,2)$ & -- & 12336\\
$(f;0,0,0;6)$ & 4 & $(99,41)$ & 10 & 3 & YES & YES & NO(2) & $1.50$ & $(4,2)$ & -- & 12337\\
$(f;0,0,0;6)$ & 4 & $(100,27)$ & 10 & 2 & YES & YES & NO(2) & $1.56$ & $(8,0)$ & -- & 12338\\
$(f;0,0,0;6)$ & 4 & $(100,39)$ & 10 & 2 & YES & YES & NO(2) & $1.44$ & $(8,0)$ & -- & 12339\\
$(f;0,0,0;6)$ & 4 & $(101,30)$ & 10 & 1 & YES & YES & NO(2) & $1.56$ & $(4,2)$ & -- & 12340\\
$(f;0,0,0;6)$ & 4 & $(105,32)$ & 11 & 3 & YES & YES & NO(2) & $1.60$ & $(6,1)$ & -- & 12341\\
$(f;0,0,0;6)$ & 4 & $(106,41)$ & 10 & 2 & YES & YES & YES & $1.78$ & $(2,3)$ & -- & 12342\\
$(f;0,0,0;6)$ & 4 & $(107,41)$ & 10 & 1 & YES & YES & NO(2) & $1.67$ & $(2,3)$ & -- & 12343\\
$(f;0,0,0;6)$ & 4 & $(107,44)$ & 12 & 1 & YES & YES & NO(2) & $1.56$ & $(8,0)$ & -- & 12344\\
$(f;0,0,0;6)$ & 4 & $(108,41)$ & 10 & 6 & YES & YES & YES & $1.62$ & $(2,3)$ & -- & 12345\\
$(f;0,0,0;6)$ & 4 & $(112,47)$ & 10 & 2 & YES & YES & YES & $1.62$ & $(2,3)$ & -- & 12346\\
$(f;0,0,0;6)$ & 4 & $(115,31)$ & 11 & 1 & YES & YES & NO(2) & $1.50$ & $(6,1)$ & -- & 12347\\
$(f;0,0,0;6)$ & 4 & $(115,44)$ & 10 & 1 & YES & YES & NO(2) & $1.67$ & $(2,3)$ & -- & 12348\\
$(f;0,0,0;6)$ & 4 & $(119,44)$ & 10 & 1 & YES & YES & YES & $1.71$ & $(4,2)$ & -- & 12349\\
$(f;0,0,0;6)$ & 4 & $(119,50)$ & 10 & 1 & YES & YES & YES & $1.89$ & $(2,3)$ & -- & 12350\\
$(f;0,0,0;6)$ & 4 & $(123,47)$ & 10 & 3 & YES & YES & YES & $1.50$ & $(4,2)$ & -- & 12351\\
$(f;0,0,0;6)$ & 4 & $(127,49)$ & 11 & 1 & YES & YES & YES & $1.80$ & $(2,3)$ & -- & 12352\\
$(f;0,0,0;6)$ & 4 & $(128,47)$ & 10 & 2 & YES & YES & YES & $1.62$ & $(2,3)$ & -- & 12353\\
$(f;0,0,0;6)$ & 4 & $(128,49)$ & 10 & 2 & YES & YES & YES & $1.71$ & $(4,2)$ & -- & 12354\\
$(f;0,0,0;6)$ & 4 & $(129,50)$ & 10 & 3 & YES & YES & YES & $1.43$ & $(4,2)$ & -- & 12355\\
$(f;0,0,0;6)$ & 4 & $(131,50)$ & 10 & 1 & YES & YES & YES & $1.43$ & $(2,3)$ & -- & 12356\\
$(f;0,0,0;6)$ & 4 & $(133,39)$ & 11 & 1 & YES & YES & YES & $1.29$ & $(4,2)$ & -- & 12357\\
$(f;0,0,0;6)$ & 4 & $(133,51)$ & 11 & 1 & YES & YES & YES & $1.89$ & $(4,2)$ & -- & 12358\\
$(f;0,0,0;6)$ & 4 & $(134,39)$ & 11 & 2 & YES & YES & NO(2) & $1.56$ & $(2,3)$ & -- & 12359\\
$(f;0,0,0;6)$ & 4 & $(140,41)$ & 11 & 2 & YES & YES & YES & $1.78$ & $(2,3)$ & -- & 12360\\
$(f;0,0,0;6)$ & 4 & $(141,59)$ & 11 & 3 & YES & YES & YES & $1.75$ & $(4,2)$ & -- & 12361\\
$(f;0,0,0;6)$ & 4 & $(144,55)$ & 10 & 6 & YES & YES & YES & $1.75$ & $(2,3)$ & -- & 12362\\
$(f;0,0,0;6)$ & 4 & $(146,31)$ & 12 & 2 & YES & YES & NO(2) & $1.38$ & $(4,2)$ & -- & 12363\\
$(f;0,0,0;6)$ & 4 & $(146,41)$ & 11 & 2 & YES & YES & YES & $1.75$ & $(2,3)$ & -- & 12364\\
$(f;0,0,0;6)$ & 4 & $(152,55)$ & 12 & 2 & YES & YES & YES & $1.71$ & $(4,2)$ & -- & 12365\\
$(f;0,0,0;6)$ & 4 & $(152,59)$ & 11 & 2 & YES & YES & YES & $1.62$ & $(4,2)$ & -- & 12366\\
$(f;0,0,0;6)$ & 4 & $(152,63)$ & 11 & 2 & YES & YES & YES & $1.88$ & $(4,2)$ & -- & 12367\\
$(f;0,0,0;6)$ & 4 & $(158,61)$ & 11 & 2 & YES & YES & YES & $1.62$ & $(4,2)$ & -- & 12368\\
$(f;0,0,0;6)$ & 4 & $(166,49)$ & 11 & 2 & YES & YES & YES & $1.56$ & $(4,2)$ & -- & 12369\\
$(f;0,0,0;6)$ & 4 & $(167,49)$ & 12 & 1 & YES & YES & YES & $1.78$ & $(4,2)$ & -- & 12370\\
$(f;0,0,0;6)$ & 4 & $(219,64)$ & 12 & 3 & YES & YES & YES & $1.67$ & $(4,2)$ & -- & 12371\\
$(f;0,0,0;6)$ & 4 & $(225,49)$ & 13 & 3 & YES & YES & YES & $1.75$ & $(4,2)$ & -- & 12372\\
$(f;0,1,0;7)$ & 5 & $(55,21)$ & 8 & 1 & YES & YES & NO(2) & $1.56$ & $(4,2)$ & -- & 12373\\
$(f;0,1,0;7)$ & 5 & $(65,19)$ & 9 & 1 & YES & YES & NO(2) & $1.50$ & $(6,1)$ & -- & 12374\\
$(f;0,1,0;7)$ & 5 & $(68,19)$ & 9 & 1 & YES & YES & NO(2) & $1.50$ & $(6,1)$ & -- & 12375\\
$(f;0,1,0;7)$ & 5 & $(84,19)$ & 10 & 7 & YES & YES & NO(2) & $1.38$ & $(6,1)$ & -- & 12376\\
$(f;0,1,0;7)$ & 5 & $(87,19)$ & 10 & 1 & YES & YES & NO(2) & $1.50$ & $(6,1)$ & -- & 12377\\
$(g;0,0,0;19)$ & 6 & $(12,5)$ & 5 & 1 & YES & YES & NO(2) & $1.33$ & $(4,2)$ & -- & 12378\\
$(g;0,0,0;19)$ & 6 & $(19,8)$ & 6 & 19 & YES & YES & NO(2) & $1.44$ & $(8,0)$ & -- & 12379\\
$(g;0,0,0;19)$ & 6 & $(21,8)$ & 6 & 1 & YES & YES & YES & $1.89$ & $(2,3)$ & -- & 12380\\
$(g;0,0,0;19)$ & 6 & $(24,7)$ & 7 & 1 & YES & YES & YES & $1.78$ & $(2,3)$ & -- & 12381\\
$(g;0,0,0;19)$ & 6 & $(29,12)$ & 7 & 1 & YES & YES & YES & $1.56$ & $(2,3)$ & -- & 12382\\
$(g;0,0,1;26)$ & 7 & $(8,3)$ & 4 & 2 & YES & YES & NO(2) & $1.50$ & $(2,3)$ & -- & 12383\\
$(g;0,0,1;26)$ & 7 & $(11,4)$ & 5 & 1 & YES & YES & YES & $1.29$ & $(4,2)$ & -- & 12384\\
$(g;0,0,1;26)$ & 7 & $(12,5)$ & 5 & 2 & YES & YES & YES & $1.38$ & $(2,3)$ & -- & 12385\\
$(g;0,0,1;26)$ & 7 & $(13,5)$ & 5 & 13 & YES & YES & NO(2) & $1.60$ & $(2,3)$ & -- & 12386\\
$(g;0,0,1;26)$ & 7 & $(17,5)$ & 6 & 1 & YES & YES & NO(2) & $1.25$ & $(4,2)$ & -- & 12387\\
$(g;0,0,1;26)$ & 7 & $(17,7)$ & 6 & 1 & YES & YES & YES & $1.71$ & $(4,2)$ & -- & 12388\\
$(g;0,0,2;11)$ & 8 & $(13,5)$ & 5 & 1 & YES & YES & YES & $1.67$ & $(2,3)$ & -- & 12389\\
$(g;0,1,0;24)$ & 7 & $(8,3)$ & 4 & 8 & YES & YES & NO(2) & $1.50$ & $(2,3)$ & -- & 12390\\
$(g;0,1,0;24)$ & 7 & $(11,4)$ & 5 & 1 & YES & YES & YES & $1.29$ & $(4,2)$ & -- & 12391\\
$(g;0,1,0;24)$ & 7 & $(13,5)$ & 5 & 1 & YES & YES & NO(2) & $1.38$ & $(4,2)$ & -- & 12392\\
$(g;0,1,0;24)$ & 7 & $(17,5)$ & 6 & 1 & YES & YES & NO(2) & $1.25$ & $(4,2)$ & -- & 12393\\
$(g;0,1,0;24)$ & 7 & $(17,7)$ & 6 & 1 & YES & YES & YES & $1.80$ & $(2,3)$ & -- & 12394\\
$(g;0,1,1;33)$ & 8 & $(12,5)$ & 5 & 3 & YES & YES & YES & $1.56$ & $(2,3)$ & -- & 12395\\
$(g;0,1,2;14)$ & 9 & $(7,3)$ & 4 & 7 & YES & YES & NO(2) & $1.60$ & $(6,1)$ & -- & 12396\\
$(g;0,1,2;14)$ & 9 & $(11,3)$ & 5 & 1 & YES & YES & YES & $1.62$ & $(4,2)$ & -- & 12397\\
$(g;0,2,0;29)$ & 8 & $(7,2)$ & 4 & 1 & YES & YES & NO(2) & $1.38$ & $(6,1)$ & -- & 12398\\
$(g;0,2,0;29)$ & 8 & $(8,3)$ & 4 & 1 & YES & YES & NO(2) & $1.56$ & $(4,2)$ & -- & 12399\\
$(g;0,2,0;29)$ & 8 & $(11,3)$ & 5 & 1 & YES & YES & NO(2) & $1.44$ & $(8,0)$ & -- & 12400\\
$(g;0,2,0;29)$ & 8 & $(22,5)$ & 7 & 1 & YES & YES & YES & $1.50$ & $(4,2)$ & -- & 12401\\
$(g;1,0,0;7)$ & 7 & $(11,4)$ & 5 & 1 & YES & YES & YES & $1.50$ & $(2,3)$ & -- & 12402\\
$(g;1,0,2;24)$ & 9 & $(5,2)$ & 3 & 1 & YES & YES & NO(2) & $1.56$ & $(4,2)$ & -- & 12403\\
$(g;1,0,2;24)$ & 9 & $(9,2)$ & 5 & 3 & YES & YES & NO(2) & $1.55$ & $(4,2)$ & -- & 12404\\
$(g;1,1,0;9)$ & 8 & $(7,3)$ & 4 & 1 & YES & YES & YES & $1.29$ & $(4,2)$ & -- & 12405\\
$(g;1,1,0;9)$ & 8 & $(8,3)$ & 4 & 1 & YES & YES & NO(2) & $1.38$ & $(4,2)$ & -- & 12406\\
$(g;1,1,0;9)$ & 8 & $(12,5)$ & 5 & 3 & YES & YES & YES & $1.89$ & $(2,3)$ & -- & 12407\\
$(g;1,1,0;9)$ & 8 & $(13,5)$ & 5 & 1 & YES & YES & YES & $1.29$ & $(4,2)$ & -- & 12408\\
$(g;1,1,1;49)$ & 9 & $(7,3)$ & 4 & 7 & YES & YES & YES & $1.89$ & $(2,3)$ & -- & 12409\\
$(g;1,1,1;49)$ & 9 & $(8,3)$ & 4 & 1 & YES & YES & YES & $1.50$ & $(4,2)$ & -- & 12410\\
$(g;1,1,2;31)$ & 10 & $(4,1)$ & 3 & 1 & YES & YES & YES & $1.29$ & $(2,3)$ & -- & 12411\\
$(g;1,2,0;11)$ & 9 & $(2,1)$ & 1 & 1 & YES & YES & NO(2) & $1.50$ & $(2,3)$ & -- & 12412\\
$(g;1,2,1;60)$ & 10 & $(3,1)$ & 2 & 3 & YES & YES & NO(2) & $1.33$ & $(8,0)$ & -- & 12413\\
$(g;2,0,0;37)$ & 8 & $(5,2)$ & 3 & 1 & YES & YES & NO(2) & $1.50$ & $(2,3)$ & -- & 12414\\
$(g;2,0,0;37)$ & 8 & $(7,2)$ & 4 & 1 & YES & YES & NO(2) & $1.60$ & $(2,3)$ & -- & 12415\\
$(g;2,0,1;10)$ & 9 & $(3,1)$ & 2 & 1 & YES & YES & NO(2) & $1.44$ & $(4,2)$ & -- & 12416\\
$(g;2,1,0;48)$ & 9 & $(10,3)$ & 5 & 2 & YES & YES & YES & $1.29$ & $(4,2)$ & -- & 12417\\
$(g;2,1,1;13)$ & 10 & $(5,2)$ & 3 & 1 & YES & YES & YES & $1.78$ & $(4,2)$ & -- & 12418\\
$(g;3,1,0;30)$ & 10 & $(8,3)$ & 4 & 2 & YES & YES & YES & $1.67$ & $(4,2)$ & -- & 12419\\
$(g;3,2,0;37)$ & 11 & $(2,1)$ & 1 & 1 & YES & YES & NO(2) & $1.56$ & $(4,2)$ & -- & 12420\\
$(g;3,2,0;37)$ & 11 & $(5,1)$ & 4 & 1 & YES & YES & NO(2) & $1.50$ & $(6,1)$ & -- & 12421\\
$(h;0,0,0;6)$ & 5 & $(17,7)$ & 6 & 1 & YES & YES & NO(2) & $1.33$ & $(4,2)$ & -- & 12422\\
$(h;0,0,0;6)$ & 5 & $(18,5)$ & 6 & 6 & YES & YES & NO(2) & $1.25$ & $(10,-1)$ & -- & 12423\\
$(h;0,0,0;6)$ & 5 & $(18,7)$ & 6 & 6 & YES & YES & NO(3) & $1.25$ & $(2,3)$ & -- & 12424\\
$(h;0,0,0;6)$ & 5 & $(25,11)$ & 7 & 1 & YES & YES & YES & $1.43$ & $(2,3)$ & -- & 12425\\
$(h;0,0,0;6)$ & 5 & $(34,13)$ & 7 & 2 & YES & YES & YES & $1.67$ & $(2,3)$ & -- & 12426\\
$(h;0,0,0;6)$ & 5 & $(37,11)$ & 8 & 1 & YES & YES & NO(2) & $1.70$ & $(4,2)$ & -- & 12427\\
$(h;0,0,0;6)$ & 5 & $(44,17)$ & 8 & 2 & YES & YES & YES & $1.67$ & $(2,3)$ & -- & 12428\\
$(h;0,1,0;8)$ & 6 & $(12,5)$ & 5 & 4 & YES & YES & NO(2) & $1.33$ & $(4,2)$ & -- & 12429\\
$(h;0,1,0;8)$ & 6 & $(21,8)$ & 6 & 1 & YES & YES & YES & $1.89$ & $(2,3)$ & -- & 12430\\
$(h;0,1,0;8)$ & 6 & $(29,12)$ & 7 & 1 & YES & YES & YES & $1.56$ & $(2,3)$ & -- & 12431\\
$(h;0,2,0;10)$ & 7 & $(8,3)$ & 4 & 2 & YES & YES & NO(2) & $1.50$ & $(2,3)$ & -- & 12432\\
$(h;0,3,0;12)$ & 8 & $(7,2)$ & 4 & 1 & YES & YES & NO(2) & $1.38$ & $(6,1)$ & -- & 12433\\
$(i;0,0,0;9)$ & 5 & $(34,13)$ & 7 & 1 & YES & YES & NO(2) & $1.60$ & $(6,1)$ & -- & 12434\\
$(i;0,0,0;9)$ & 5 & $(41,12)$ & 8 & 1 & YES & YES & NO(2) & $1.70$ & $(6,1)$ & -- & 12435\\
$(i;0,0,0;9)$ & 5 & $(41,16)$ & 8 & 1 & YES & YES & NO(2) & $1.56$ & $(8,0)$ & -- & 12436\\
$(i;0,0,0;9)$ & 5 & $(44,13)$ & 8 & 1 & YES & YES & NO(2) & $1.33$ & $(8,0)$ & -- & 12437\\
$(i;0,0,0;9)$ & 5 & $(45,16)$ & 9 & 9 & YES & YES & NO(2) & $1.44$ & $(8,0)$ & -- & 12438\\
$(i;0,0,0;9)$ & 5 & $(47,13)$ & 8 & 1 & YES & YES & NO(2) & $1.44$ & $(8,0)$ & -- & 12439\\
$(i;0,0,0;9)$ & 5 & $(57,13)$ & 9 & 3 & YES & YES & NO(2) & $1.60$ & $(2,3)$ & -- & 12440\\
$(i;0,0,0;9)$ & 5 & $(58,17)$ & 9 & 1 & YES & YES & YES & $1.57$ & $(2,3)$ & -- & 12441\\
$(i;0,0,0;9)$ & 5 & $(63,17)$ & 9 & 9 & YES & YES & NO(2) & $1.38$ & $(4,2)$ & -- & 12442\\
$(i;0,0,0;9)$ & 5 & $(79,22)$ & 10 & 1 & YES & YES & YES & $1.75$ & $(4,2)$ & -- & 12443\\
$(i;0,1,0;12)$ & 6 & $(25,7)$ & 7 & 1 & YES & YES & NO(2) & $1.60$ & $(2,3)$ & -- & 12444\\
$(i;0,1,0;12)$ & 6 & $(27,8)$ & 7 & 3 & YES & YES & NO(2) & $1.50$ & $(2,3)$ & -- & 12445\\
$(i;0,1,0;12)$ & 6 & $(29,8)$ & 7 & 1 & YES & YES & NO(2) & $1.44$ & $(8,0)$ & -- & 12446\\
$(i;0,1,0;12)$ & 6 & $(34,13)$ & 7 & 2 & YES & YES & YES & $1.50$ & $(4,2)$ & -- & 12447\\
$(i;0,1,0;12)$ & 6 & $(35,8)$ & 8 & 1 & YES & YES & NO(2) & $1.60$ & $(2,3)$ & -- & 12448\\
$(i;0,2,0;15)$ & 7 & $(13,5)$ & 5 & 1 & YES & YES & YES & $1.43$ & $(4,2)$ & -- & 12449\\
$(i;0,2,0;15)$ & 7 & $(24,7)$ & 7 & 3 & YES & YES & YES & $1.71$ & $(2,3)$ & -- & 12450\\
$(j;0,0,0;8)$ & 5 & $(29,11)$ & 7 & 1 & YES & YES & NO(2) & $1.44$ & $(4,2)$ & -- & 12451\\
$(j;0,0,0;8)$ & 5 & $(31,13)$ & 7 & 1 & YES & YES & YES & $1.50$ & $(2,3)$ & -- & 12452\\
$(j;0,0,0;8)$ & 5 & $(34,13)$ & 7 & 2 & YES & YES & NO(2) & $1.60$ & $(2,3)$ & -- & 12453\\
$(j;0,0,0;8)$ & 5 & $(47,13)$ & 8 & 1 & YES & YES & NO(2) & $1.60$ & $(2,3)$ & -- & 12454\\
$(j;0,0,0;8)$ & 5 & $(51,20)$ & 9 & 1 & YES & YES & NO(2) & $1.56$ & $(4,2)$ & -- & 12455\\
$(j;0,0,0;8)$ & 5 & $(53,19)$ & 9 & 1 & YES & YES & NO(2) & $1.56$ & $(4,2)$ & -- & 12456\\
$(j;0,0,0;8)$ & 5 & $(57,17)$ & 10 & 1 & YES & YES & YES & $1.57$ & $(2,3)$ & -- & 12457\\
$(j;0,0,0;8)$ & 5 & $(59,25)$ & 9 & 1 & YES & YES & NO(2) & $1.50$ & $(4,2)$ & -- & 12458\\
$(j;0,0,0;8)$ & 5 & $(61,19)$ & 10 & 1 & YES & YES & NO(2) & $1.56$ & $(4,2)$ & -- & 12459\\
$(j;0,0,0;8)$ & 5 & $(61,25)$ & 9 & 1 & YES & YES & NO(2) & $1.50$ & $(4,2)$ & -- & 12460\\
$(j;0,0,0;8)$ & 5 & $(63,26)$ & 9 & 1 & YES & YES & NO(2) & $1.67$ & $(2,3)$ & -- & 12461\\
$(j;0,0,0;8)$ & 5 & $(69,29)$ & 9 & 1 & YES & YES & YES & $1.62$ & $(2,3)$ & -- & 12462\\
$(j;0,0,0;8)$ & 5 & $(69,31)$ & 10 & 1 & YES & YES & NO(2) & $1.60$ & $(6,1)$ & -- & 12463\\
$(j;0,0,0;8)$ & 5 & $(76,29)$ & 9 & 4 & YES & YES & YES & $1.43$ & $(4,2)$ & -- & 12464\\
$(j;0,0,0;8)$ & 5 & $(80,31)$ & 9 & 8 & YES & YES & YES & $1.62$ & $(4,2)$ & -- & 12465\\
$(j;0,0,0;8)$ & 5 & $(81,31)$ & 9 & 1 & YES & YES & YES & $1.80$ & $(2,3)$ & -- & 12466\\
$(j;0,0,0;8)$ & 5 & $(85,37)$ & 10 & 1 & YES & YES & YES & $1.75$ & $(4,2)$ & -- & 12467\\
$(j;0,0,0;8)$ & 5 & $(89,26)$ & 10 & 1 & YES & YES & YES & $1.56$ & $(2,3)$ & -- & 12468\\
$(j;0,0,0;8)$ & 5 & $(100,37)$ & 10 & 4 & YES & YES & YES & $1.62$ & $(4,2)$ & -- & 12469\\
$(j;0,0,0;8)$ & 5 & $(105,31)$ & 10 & 1 & YES & YES & YES & $1.67$ & $(2,3)$ & -- & 12470\\
$(j;0,0,0;8)$ & 5 & $(109,40)$ & 10 & 1 & YES & YES & YES & $1.78$ & $(4,2)$ & -- & 12471\\
$(j;0,0,0;8)$ & 5 & $(117,34)$ & 11 & 1 & YES & YES & YES & $1.78$ & $(4,2)$ & -- & 12472\\
$(j;0,0,0;8)$ & 5 & $(122,37)$ & 11 & 2 & YES & YES & YES & $1.67$ & $(4,2)$ & -- & 12473\\
$(j;0,0,0;8)$ & 5 & $(129,29)$ & 12 & 1 & YES & YES & YES & $1.78$ & $(4,2)$ & -- & 12474\\
$(j;0,0,0;8)$ & 5 & $(140,41)$ & 11 & 4 & YES & YES & YES & $1.56$ & $(4,2)$ & -- & 12475\\
$(j;0,1,0;10)$ & 6 & $(25,7)$ & 7 & 5 & YES & YES & YES & $1.43$ & $(4,2)$ & -- & 12476\\
$(j;0,1,0;10)$ & 6 & $(37,11)$ & 8 & 1 & YES & YES & NO(2) & $1.70$ & $(2,3)$ & -- & 12477\\
$(j;0,1,0;10)$ & 6 & $(40,11)$ & 8 & 10 & YES & YES & NO(2) & $1.38$ & $(10,-1)$ & -- & 12478\\
$(j;0,1,0;10)$ & 6 & $(41,12)$ & 8 & 1 & YES & YES & NO(2) & $1.56$ & $(4,2)$ & -- & 12479\\
$(j;0,1,0;10)$ & 6 & $(43,12)$ & 8 & 1 & YES & YES & NO(2) & $1.64$ & $(4,2)$ & -- & 12480\\
$(j;0,1,0;10)$ & 6 & $(47,18)$ & 8 & 1 & YES & YES & YES & $1.62$ & $(2,3)$ & -- & 12481\\
$(j;0,1,0;10)$ & 6 & $(63,19)$ & 11 & 1 & YES & YES & NO(2) & $1.44$ & $(8,0)$ & -- & 12482\\
$(j;0,1,0;10)$ & 6 & $(63,26)$ & 9 & 1 & YES & YES & YES & $1.62$ & $(4,2)$ & -- & 12483\\
$(j;0,2,0;12)$ & 7 & $(21,8)$ & 6 & 3 & YES & YES & NO(2) & $1.56$ & $(4,2)$ & -- & 12484\\
$(j;0,2,0;12)$ & 7 & $(27,8)$ & 7 & 3 & YES & YES & NO(2) & $1.44$ & $(4,2)$ & -- & 12485\\
$(j;0,2,0;12)$ & 7 & $(64,15)$ & 10 & 4 & YES & YES & NO(2) & $1.60$ & $(6,1)$ & -- & 12486
\end{longtable}
\subsection{2 chains, $K^2 = 4$}
\begin{longtable}{|c|c|c|c|c|c|c|c|c|c|c|c|}
\hline
\multicolumn{12}{|c|}{2 chains, $K^2 = 4$}\\
\hline
$(n,a)$ & Len & $(n,a)$ & Len & GCD & Nef & $\mathbb Q$-ef & Obs 0 & $\overline c_1^2 / \overline c_2$ & $(P,K)$ & WH & Index\\
\hline
\endfirsthead

\hline
$(n,a)$ & Len & $(n,a)$ & Len & GCD & Nef & $\mathbb Q$-ef & Obs 0 & $\overline c_1^2 / \overline c_2$ & $(P,K)$ & WH & Index\\
\hline
\endhead
\hline
\endfoot

$(34,13)$ & 7 & $(34,13)$ & 7 & 34 & YES & YES & NO(2) & $2.00$ & $(2,4)$ & -- & 12487\\
$(37,11)$ & 8 & $(27,8)$ & 7 & 1 & YES & YES & NO(2) & $2.00$ & $(2,4)$ & -- & 12488\\
$(39,16)$ & 8 & $(18,7)$ & 6 & 3 & YES & YES & NO(3) & $1.71$ & $(2,4)$ & -- & 12489\\
$(40,11)$ & 8 & $(27,8)$ & 7 & 1 & YES & YES & NO(2) & $1.90$ & $(4,3)$ & -- & 12490\\
$(41,17)$ & 8 & $(27,8)$ & 7 & 1 & YES & YES & NO(2) & $2.08$ & $(4,3)$ & NO & 12491\\
$(43,12)$ & 8 & $(37,10)$ & 8 & 1 & YES & YES & NO(2) & $1.90$ & $(8,1)$ & -- & 12492\\
$(44,13)$ & 8 & $(27,8)$ & 7 & 1 & YES & YES & NO(2) & $1.82$ & $(6,2)$ & -- & 12493\\
$(44,13)$ & 8 & $(29,11)$ & 7 & 1 & YES & YES & NO(2) & $1.75$ & $(8,1)$ & -- & 12494\\
$(44,13)$ & 8 & $(30,11)$ & 7 & 2 & YES & YES & NO(2) & $1.89$ & $(2,4)$ & -- & 12495\\
$(44,13)$ & 8 & $(44,13)$ & 8 & 44 & YES & YES & NO(2) & $1.75$ & $(2,4)$ & -- & 12496\\
$(46,17)$ & 8 & $(21,8)$ & 6 & 1 & YES & YES & NO(2) & $1.88$ & $(4,3)$ & -- & 12497\\
$(46,17)$ & 8 & $(31,13)$ & 7 & 1 & YES & YES & NO(2) & $2.00$ & $(6,2)$ & -- & 12498\\
$(46,19)$ & 8 & $(31,12)$ & 7 & 1 & YES & YES & NO(2) & $2.00$ & $(4,3)$ & -- & 12499\\
$(47,13)$ & 8 & $(29,11)$ & 7 & 1 & YES & YES & NO(2) & $1.88$ & $(8,1)$ & -- & 12500\\
$(47,18)$ & 8 & $(29,8)$ & 7 & 1 & YES & YES & NO(2) & $1.88$ & $(8,1)$ & -- & 12501\\
$(47,18)$ & 8 & $(43,12)$ & 8 & 1 & YES & YES & YES & $2.00$ & $(2,4)$ & -- & 12502\\
$(47,18)$ & 8 & $(43,13)$ & 9 & 1 & YES & YES & YES & $1.83$ & $(4,3)$ & -- & 12503\\
$(47,14)$ & 9 & $(45,14)$ & 9 & 1 & YES & YES & NO(2) & $2.10$ & $(4,3)$ & -- & 12504\\
$(49,18)$ & 8 & $(17,7)$ & 6 & 1 & YES & YES & NO(2) & $1.75$ & $(4,3)$ & -- & 12505\\
$(49,19)$ & 8 & $(31,12)$ & 7 & 1 & YES & YES & NO(2) & $2.00$ & $(12,-1)$ & -- & 12506\\
$(49,19)$ & 8 & $(36,11)$ & 8 & 1 & YES & YES & NO(2) & $2.00$ & $(4,3)$ & -- & 12507\\
$(49,18)$ & 8 & $(44,13)$ & 8 & 1 & YES & YES & YES & $2.00$ & $(2,4)$ & -- & 12508\\
$(49,18)$ & 8 & $(45,17)$ & 9 & 1 & YES & YES & NO(2) & $2.09$ & $(6,2)$ & -- & 12509\\
$(55,21)$ & 8 & $(19,5)$ & 7 & 1 & YES & YES & NO(2) & $2.11$ & $(2,4)$ & NO & 12510\\
$(55,21)$ & 8 & $(19,5)$ & 7 & 1 & YES & YES & NO(2) & $2.11$ & $(2,4)$ & -- & 12511\\
$(55,21)$ & 8 & $(21,8)$ & 6 & 1 & YES & YES & NO(2) & $2.09$ & $(6,2)$ & -- & 12512\\
$(55,21)$ & 8 & $(24,7)$ & 7 & 1 & YES & YES & NO(2) & $1.88$ & $(4,3)$ & -- & 12513\\
$(55,21)$ & 8 & $(29,11)$ & 7 & 1 & YES & YES & NO(2) & $2.09$ & $(6,2)$ & -- & 12514\\
$(55,21)$ & 8 & $(29,12)$ & 7 & 1 & YES & YES & YES & $2.14$ & $(2,4)$ & -- & 12515\\
$(56,17)$ & 9 & $(17,7)$ & 6 & 1 & YES & YES & NO(2) & $1.75$ & $(4,3)$ & -- & 12516\\
$(57,22)$ & 9 & $(22,5)$ & 7 & 1 & YES & YES & NO(2) & $1.75$ & $(4,3)$ & -- & 12517\\
$(57,13)$ & 9 & $(25,7)$ & 7 & 1 & YES & YES & NO(2) & $1.89$ & $(2,4)$ & -- & 12518\\
$(58,17)$ & 9 & $(24,7)$ & 7 & 2 & YES & YES & NO(2) & $2.00$ & $(4,3)$ & -- & 12519\\
$(58,17)$ & 9 & $(27,10)$ & 7 & 1 & YES & YES & NO(2) & $2.10$ & $(4,3)$ & -- & 12520\\
$(58,17)$ & 9 & $(31,7)$ & 8 & 1 & YES & YES & NO(2) & $2.08$ & $(4,3)$ & -- & 12521\\
$(59,18)$ & 9 & $(37,10)$ & 8 & 1 & YES & YES & NO(2) & $2.00$ & $(6,2)$ & -- & 12522\\
$(59,18)$ & 9 & $(41,17)$ & 8 & 1 & YES & YES & YES & $2.14$ & $(2,4)$ & -- & 12523\\
$(60,13)$ & 9 & $(35,13)$ & 8 & 5 & YES & YES & NO(2) & $2.00$ & $(6,2)$ & NO & 12524\\
$(60,13)$ & 9 & $(43,18)$ & 8 & 1 & YES & YES & NO(2) & $1.90$ & $(4,3)$ & -- & 12525\\
$(61,18)$ & 9 & $(18,7)$ & 6 & 1 & YES & YES & NO(2) & $1.90$ & $(4,3)$ & -- & 12526\\
$(61,25)$ & 9 & $(21,8)$ & 6 & 1 & YES & YES & NO(2) & $2.08$ & $(4,3)$ & NO & 12527\\
$(61,25)$ & 9 & $(21,8)$ & 6 & 1 & YES & YES & NO(2) & $2.08$ & $(4,3)$ & -- & 12528\\
$(61,17)$ & 9 & $(23,9)$ & 7 & 1 & YES & YES & NO(2) & $1.86$ & $(6,2)$ & -- & 12529\\
$(61,25)$ & 9 & $(25,11)$ & 7 & 1 & YES & YES & NO(2) & $2.00$ & $(6,2)$ & -- & 12530\\
$(61,18)$ & 9 & $(31,13)$ & 7 & 1 & YES & YES & NO(2) & $2.11$ & $(2,4)$ & -- & 12531\\
$(61,18)$ & 9 & $(41,12)$ & 8 & 1 & YES & YES & YES & $2.00$ & $(2,4)$ & -- & 12532\\
$(61,25)$ & 9 & $(41,18)$ & 8 & 1 & YES & YES & NO(2) & $2.08$ & $(4,3)$ & NO & 12533\\
$(62,23)$ & 9 & $(21,8)$ & 6 & 1 & YES & YES & NO(2) & $1.89$ & $(2,4)$ & -- & 12534\\
$(62,23)$ & 9 & $(34,13)$ & 7 & 2 & YES & YES & YES & $2.12$ & $(2,4)$ & -- & 12535\\
$(62,13)$ & 10 & $(43,13)$ & 9 & 1 & YES & YES & NO(2) & $1.91$ & $(6,2)$ & -- & 12536\\
$(63,17)$ & 9 & $(39,14)$ & 8 & 3 & YES & YES & NO(2) & $2.00$ & $(6,2)$ & -- & 12537\\
$(63,11)$ & 10 & $(43,16)$ & 9 & 1 & YES & YES & NO(2) & $2.09$ & $(6,2)$ & -- & 12538\\
$(65,24)$ & 9 & $(17,5)$ & 6 & 1 & YES & YES & NO(2) & $2.00$ & $(4,3)$ & -- & 12539\\
$(65,19)$ & 9 & $(18,5)$ & 6 & 1 & YES & YES & YES & $1.83$ & $(4,3)$ & -- & 12540\\
$(65,24)$ & 9 & $(19,7)$ & 6 & 1 & YES & YES & NO(2) & $2.00$ & $(4,3)$ & -- & 12541\\
$(65,24)$ & 9 & $(21,8)$ & 6 & 1 & YES & YES & NO(2) & $2.00$ & $(4,3)$ & -- & 12542\\
$(65,17)$ & 10 & $(31,11)$ & 8 & 1 & YES & YES & NO(2) & $2.10$ & $(8,1)$ & -- & 12543\\
$(65,19)$ & 9 & $(40,9)$ & 9 & 5 & YES & YES & NO(3) & $1.67$ & $(4,3)$ & -- & 12544\\
$(65,19)$ & 9 & $(44,17)$ & 8 & 1 & YES & YES & YES & $2.17$ & $(2,4)$ & NO & 12545\\
$(65,19)$ & 9 & $(44,17)$ & 8 & 1 & YES & YES & YES & $2.17$ & $(2,4)$ & -- & 12546\\
$(65,18)$ & 9 & $(56,17)$ & 9 & 1 & YES & YES & NO(2) & $1.88$ & $(4,3)$ & NO & 12547\\
$(67,26)$ & 9 & $(49,18)$ & 8 & 1 & YES & YES & NO(2) & $1.88$ & $(4,3)$ & NO & 12548\\
$(68,19)$ & 9 & $(17,4)$ & 7 & 17 & YES & YES & NO(2) & $2.11$ & $(2,4)$ & -- & 12549\\
$(68,25)$ & 9 & $(21,8)$ & 6 & 1 & YES & YES & NO(2) & $1.86$ & $(4,3)$ & -- & 12550\\
$(70,29)$ & 9 & $(21,8)$ & 6 & 7 & YES & YES & NO(2) & $1.90$ & $(4,3)$ & -- & 12551\\
$(70,29)$ & 9 & $(29,12)$ & 7 & 1 & YES & YES & YES & $2.17$ & $(4,3)$ & -- & 12552\\
$(70,29)$ & 9 & $(32,9)$ & 8 & 2 & YES & YES & NO(2) & $2.00$ & $(10,0)$ & NO & 12553\\
$(71,21)$ & 9 & $(16,7)$ & 6 & 1 & YES & YES & NO(2) & $2.11$ & $(2,4)$ & -- & 12554\\
$(71,21)$ & 9 & $(18,5)$ & 6 & 1 & YES & YES & NO(2) & $1.90$ & $(8,1)$ & -- & 12555\\
$(71,21)$ & 9 & $(18,7)$ & 6 & 1 & YES & YES & NO(2) & $2.08$ & $(4,3)$ & -- & 12556\\
$(71,21)$ & 9 & $(31,13)$ & 7 & 1 & YES & YES & NO(2) & $2.11$ & $(2,4)$ & -- & 12557\\
$(71,27)$ & 9 & $(31,12)$ & 7 & 1 & YES & YES & NO(2) & $2.20$ & $(2,4)$ & -- & 12558\\
$(71,27)$ & 9 & $(37,8)$ & 8 & 1 & YES & YES & NO(2) & $2.00$ & $(2,4)$ & -- & 12559\\
$(71,30)$ & 9 & $(39,11)$ & 9 & 1 & YES & YES & NO(2) & $2.09$ & $(6,2)$ & -- & 12560\\
$(71,16)$ & 10 & $(43,13)$ & 9 & 1 & YES & YES & NO(2) & $1.86$ & $(4,3)$ & -- & 12561\\
$(73,27)$ & 9 & $(25,7)$ & 7 & 1 & YES & YES & NO(2) & $2.00$ & $(4,3)$ & -- & 12562\\
$(73,32)$ & 10 & $(57,10)$ & 10 & 1 & YES & YES & NO(2) & $2.00$ & $(6,2)$ & -- & 12563\\
$(74,31)$ & 9 & $(41,12)$ & 8 & 1 & YES & YES & YES & $2.17$ & $(2,4)$ & -- & 12564\\
$(74,31)$ & 9 & $(43,13)$ & 9 & 1 & YES & YES & YES & $2.00$ & $(4,3)$ & NO & 12565\\
$(75,17)$ & 10 & $(25,7)$ & 7 & 25 & YES & YES & NO(2) & $2.11$ & $(2,4)$ & -- & 12566\\
$(75,17)$ & 10 & $(37,11)$ & 8 & 1 & YES & YES & NO(2) & $2.00$ & $(6,2)$ & NO & 12567\\
$(75,29)$ & 9 & $(53,23)$ & 9 & 1 & YES & YES & NO(2) & $1.83$ & $(6,2)$ & NO & 12568\\
$(76,21)$ & 9 & $(17,5)$ & 6 & 1 & YES & YES & NO(2) & $1.90$ & $(8,1)$ & -- & 12569\\
$(76,21)$ & 9 & $(18,5)$ & 6 & 2 & YES & YES & NO(2) & $1.90$ & $(8,1)$ & -- & 12570\\
$(76,21)$ & 9 & $(29,11)$ & 7 & 1 & YES & YES & YES & $2.00$ & $(2,4)$ & -- & 12571\\
$(76,29)$ & 9 & $(32,7)$ & 8 & 4 & YES & YES & NO(2) & $2.00$ & $(2,4)$ & -- & 12572\\
$(76,21)$ & 9 & $(46,19)$ & 8 & 2 & YES & YES & YES & $2.17$ & $(2,4)$ & NO & 12573\\
$(76,21)$ & 9 & $(65,18)$ & 9 & 1 & YES & YES & YES & $2.17$ & $(2,4)$ & -- & 12574\\
$(79,29)$ & 9 & $(17,7)$ & 6 & 1 & YES & YES & NO(2) & $1.67$ & $(6,2)$ & -- & 12575\\
$(79,18)$ & 10 & $(23,9)$ & 7 & 1 & YES & YES & NO(2) & $2.10$ & $(8,1)$ & -- & 12576\\
$(79,23)$ & 10 & $(25,7)$ & 7 & 1 & YES & YES & NO(2) & $2.00$ & $(2,4)$ & -- & 12577\\
$(79,18)$ & 10 & $(27,10)$ & 7 & 1 & YES & YES & NO(2) & $2.00$ & $(4,3)$ & -- & 12578\\
$(80,17)$ & 10 & $(35,13)$ & 8 & 5 & YES & YES & NO(2) & $2.00$ & $(2,4)$ & -- & 12579\\
$(80,33)$ & 10 & $(37,8)$ & 8 & 1 & YES & YES & NO(2) & $2.00$ & $(2,4)$ & NO & 12580\\
$(80,17)$ & 10 & $(48,13)$ & 9 & 16 & YES & YES & NO(2) & $2.00$ & $(6,2)$ & NO & 12581\\
$(81,34)$ & 9 & $(10,3)$ & 5 & 1 & YES & YES & NO(2) & $2.00$ & $(2,4)$ & -- & 12582\\
$(81,31)$ & 9 & $(19,8)$ & 6 & 1 & YES & YES & NO(3) & $1.86$ & $(2,4)$ & -- & 12583\\
$(81,34)$ & 9 & $(19,8)$ & 6 & 1 & YES & YES & YES & $1.86$ & $(2,4)$ & -- & 12584\\
$(83,19)$ & 10 & $(26,11)$ & 7 & 1 & YES & YES & NO(2) & $1.90$ & $(4,3)$ & -- & 12585\\
$(83,19)$ & 10 & $(30,13)$ & 8 & 1 & YES & YES & NO(2) & $2.11$ & $(6,2)$ & NO & 12586\\
$(83,19)$ & 10 & $(30,13)$ & 8 & 1 & YES & YES & NO(2) & $2.11$ & $(6,2)$ & -- & 12587\\
$(84,19)$ & 10 & $(17,4)$ & 7 & 1 & YES & YES & NO(2) & $1.88$ & $(4,3)$ & -- & 12588\\
$(84,31)$ & 10 & $(18,5)$ & 6 & 6 & YES & YES & NO(2) & $2.00$ & $(4,3)$ & -- & 12589\\
$(84,19)$ & 10 & $(32,9)$ & 8 & 4 & YES & YES & NO(2) & $1.89$ & $(2,4)$ & NO & 12590\\
$(85,18)$ & 10 & $(30,13)$ & 8 & 5 & YES & YES & NO(2) & $1.91$ & $(6,2)$ & -- & 12591\\
$(86,27)$ & 11 & $(23,6)$ & 8 & 1 & YES & YES & NO(2) & $2.10$ & $(8,1)$ & -- & 12592\\
$(86,25)$ & 10 & $(25,7)$ & 7 & 1 & YES & YES & NO(2) & $2.00$ & $(6,2)$ & -- & 12593\\
$(86,27)$ & 11 & $(57,17)$ & 10 & 1 & YES & YES & NO(2) & $2.20$ & $(8,1)$ & NO & 12594\\
$(87,32)$ & 10 & $(27,5)$ & 8 & 3 & YES & YES & NO(2) & $2.00$ & $(6,2)$ & NO & 12595\\
$(87,23)$ & 10 & $(34,13)$ & 7 & 1 & YES & YES & NO(2) & $2.10$ & $(4,3)$ & NO & 12596\\
$(87,23)$ & 10 & $(34,13)$ & 7 & 1 & YES & YES & NO(2) & $2.10$ & $(4,3)$ & -- & 12597\\
$(88,37)$ & 10 & $(21,8)$ & 6 & 1 & YES & YES & NO(2) & $2.10$ & $(4,3)$ & -- & 12598\\
$(89,34)$ & 9 & $(11,3)$ & 5 & 1 & YES & YES & NO(2) & $1.90$ & $(12,-1)$ & -- & 12599\\
$(89,34)$ & 9 & $(16,7)$ & 6 & 1 & YES & YES & NO(2) & $2.11$ & $(2,4)$ & NO & 12600\\
$(89,33)$ & 10 & $(18,5)$ & 6 & 1 & YES & YES & NO(2) & $2.00$ & $(4,3)$ & NO & 12601\\
$(89,33)$ & 10 & $(18,5)$ & 6 & 1 & YES & YES & NO(2) & $2.00$ & $(4,3)$ & -- & 12602\\
$(91,27)$ & 10 & $(40,9)$ & 9 & 1 & YES & YES & NO(2) & $2.00$ & $(4,3)$ & NO & 12603\\
$(91,27)$ & 10 & $(43,12)$ & 8 & 1 & YES & YES & YES & $2.00$ & $(2,4)$ & -- & 12604\\
$(92,35)$ & 10 & $(17,5)$ & 6 & 1 & YES & YES & NO(2) & $2.09$ & $(2,4)$ & -- & 12605\\
$(92,33)$ & 10 & $(27,8)$ & 7 & 1 & YES & YES & NO(2) & $2.00$ & $(4,3)$ & -- & 12606\\
$(92,21)$ & 10 & $(43,18)$ & 8 & 1 & YES & YES & YES & $2.17$ & $(2,4)$ & -- & 12607\\
$(93,25)$ & 10 & $(19,8)$ & 6 & 1 & YES & YES & NO(3) & $1.83$ & $(2,4)$ & -- & 12608\\
$(93,25)$ & 10 & $(19,8)$ & 6 & 1 & YES & YES & NO(2) & $2.10$ & $(6,2)$ & NO & 12609\\
$(93,34)$ & 10 & $(35,8)$ & 8 & 1 & YES & YES & NO(2) & $2.00$ & $(4,3)$ & NO & 12610\\
$(93,25)$ & 10 & $(61,17)$ & 9 & 1 & YES & YES & NO(2) & $1.88$ & $(4,3)$ & NO & 12611\\
$(95,28)$ & 11 & $(22,5)$ & 7 & 1 & YES & YES & YES & $2.00$ & $(2,4)$ & -- & 12612\\
$(96,29)$ & 11 & $(63,17)$ & 9 & 3 & YES & YES & YES & $2.17$ & $(4,3)$ & NO & 12613\\
$(97,27)$ & 11 & $(11,4)$ & 5 & 1 & YES & YES & NO(2) & $1.89$ & $(2,4)$ & -- & 12614\\
$(97,37)$ & 10 & $(22,5)$ & 7 & 1 & YES & YES & NO(2) & $2.00$ & $(4,3)$ & NO & 12615\\
$(97,37)$ & 10 & $(22,5)$ & 7 & 1 & YES & YES & NO(2) & $2.00$ & $(4,3)$ & -- & 12616\\
$(97,37)$ & 10 & $(27,8)$ & 7 & 1 & YES & YES & NO(2) & $2.10$ & $(4,3)$ & NO & 12617\\
$(97,21)$ & 10 & $(32,9)$ & 8 & 1 & YES & YES & NO(2) & $1.90$ & $(8,1)$ & NO & 12618\\
$(97,36)$ & 10 & $(35,8)$ & 8 & 1 & YES & YES & NO(2) & $2.00$ & $(4,3)$ & -- & 12619\\
$(97,21)$ & 10 & $(47,14)$ & 9 & 1 & YES & YES & YES & $2.14$ & $(2,4)$ & NO & 12620\\
$(98,27)$ & 10 & $(11,3)$ & 5 & 1 & YES & YES & NO(2) & $2.00$ & $(2,4)$ & -- & 12621\\
$(98,27)$ & 10 & $(21,8)$ & 6 & 7 & YES & YES & NO(2) & $2.00$ & $(8,1)$ & -- & 12622\\
$(98,41)$ & 10 & $(23,7)$ & 7 & 1 & YES & YES & NO(2) & $2.00$ & $(6,2)$ & -- & 12623\\
$(98,43)$ & 10 & $(23,9)$ & 7 & 1 & YES & YES & NO(2) & $2.00$ & $(6,2)$ & -- & 12624\\
$(98,29)$ & 10 & $(86,25)$ & 10 & 2 & YES & YES & NO(2) & $2.00$ & $(6,2)$ & NO & 12625\\
$(100,29)$ & 11 & $(23,10)$ & 7 & 1 & YES & YES & YES & $2.00$ & $(4,3)$ & -- & 12626\\
$(100,39)$ & 10 & $(24,7)$ & 7 & 4 & YES & YES & NO(2) & $2.00$ & $(4,3)$ & -- & 12627\\
$(100,39)$ & 10 & $(25,9)$ & 7 & 25 & YES & YES & NO(2) & $1.83$ & $(6,2)$ & -- & 12628\\
$(100,37)$ & 10 & $(32,7)$ & 8 & 4 & YES & YES & YES & $2.00$ & $(2,4)$ & -- & 12629\\
$(100,27)$ & 10 & $(37,11)$ & 8 & 1 & YES & YES & NO(2) & $1.88$ & $(4,3)$ & NO & 12630\\
$(100,27)$ & 10 & $(68,19)$ & 9 & 4 & YES & YES & NO(2) & $1.90$ & $(8,1)$ & NO & 12631\\
$(101,28)$ & 11 & $(11,4)$ & 5 & 1 & YES & YES & NO(2) & $1.89$ & $(2,4)$ & -- & 12632\\
$(101,30)$ & 10 & $(13,5)$ & 5 & 1 & YES & YES & YES & $2.00$ & $(2,4)$ & -- & 12633\\
$(101,37)$ & 10 & $(13,5)$ & 5 & 1 & YES & YES & NO(2) & $2.08$ & $(4,3)$ & -- & 12634\\
$(101,23)$ & 11 & $(18,7)$ & 6 & 1 & YES & YES & NO(2) & $2.00$ & $(8,1)$ & -- & 12635\\
$(101,30)$ & 10 & $(19,8)$ & 6 & 1 & YES & YES & NO(2) & $2.00$ & $(2,4)$ & -- & 12636\\
$(101,44)$ & 10 & $(23,9)$ & 7 & 1 & YES & YES & NO(2) & $2.00$ & $(6,2)$ & -- & 12637\\
$(102,23)$ & 11 & $(24,7)$ & 7 & 6 & YES & YES & NO(2) & $1.90$ & $(4,3)$ & -- & 12638\\
$(102,31)$ & 11 & $(65,12)$ & 10 & 1 & YES & YES & YES & $2.29$ & $(2,4)$ & -- & 12639\\
$(103,30)$ & 11 & $(39,7)$ & 9 & 1 & YES & YES & NO(2) & $2.00$ & $(4,3)$ & -- & 12640\\
$(104,43)$ & 10 & $(10,3)$ & 5 & 2 & YES & YES & NO(2) & $2.00$ & $(2,4)$ & -- & 12641\\
$(104,43)$ & 10 & $(11,3)$ & 5 & 1 & YES & YES & NO(2) & $1.89$ & $(10,0)$ & -- & 12642\\
$(104,29)$ & 10 & $(13,5)$ & 5 & 13 & YES & YES & NO(2) & $1.89$ & $(6,2)$ & NO & 12643\\
$(104,29)$ & 10 & $(13,5)$ & 5 & 13 & YES & YES & NO(2) & $1.89$ & $(6,2)$ & -- & 12644\\
$(104,29)$ & 10 & $(19,7)$ & 6 & 1 & YES & YES & YES & $2.00$ & $(4,3)$ & NO & 12645\\
$(104,29)$ & 10 & $(19,7)$ & 6 & 1 & YES & YES & YES & $2.00$ & $(4,3)$ & -- & 12646\\
$(105,31)$ & 10 & $(19,8)$ & 6 & 1 & YES & YES & YES & $2.00$ & $(2,4)$ & -- & 12647\\
$(105,41)$ & 10 & $(75,29)$ & 9 & 15 & YES & YES & NO(2) & $2.00$ & $(4,3)$ & NO & 12648\\
$(106,31)$ & 10 & $(11,3)$ & 5 & 1 & YES & YES & YES & $1.83$ & $(4,3)$ & -- & 12649\\
$(106,41)$ & 10 & $(13,5)$ & 5 & 1 & YES & YES & NO(2) & $2.00$ & $(4,3)$ & -- & 12650\\
$(106,41)$ & 10 & $(17,5)$ & 6 & 1 & YES & YES & NO(2) & $1.90$ & $(4,3)$ & -- & 12651\\
$(107,47)$ & 10 & $(11,4)$ & 5 & 1 & YES & YES & NO(2) & $2.00$ & $(12,-1)$ & -- & 12652\\
$(107,41)$ & 10 & $(18,5)$ & 6 & 1 & YES & YES & NO(2) & $2.09$ & $(6,2)$ & -- & 12653\\
$(107,41)$ & 10 & $(25,7)$ & 7 & 1 & YES & YES & YES & $2.00$ & $(2,4)$ & -- & 12654\\
$(107,41)$ & 10 & $(27,8)$ & 7 & 1 & YES & YES & YES & $2.17$ & $(2,4)$ & -- & 12655\\
$(107,30)$ & 11 & $(56,15)$ & 9 & 1 & YES & YES & NO(2) & $1.91$ & $(6,2)$ & NO & 12656\\
$(108,29)$ & 10 & $(37,11)$ & 8 & 1 & YES & YES & NO(2) & $1.90$ & $(4,3)$ & NO & 12657\\
$(109,30)$ & 10 & $(19,7)$ & 6 & 1 & YES & YES & NO(2) & $2.00$ & $(4,3)$ & -- & 12658\\
$(109,30)$ & 10 & $(33,10)$ & 8 & 1 & YES & YES & YES & $2.14$ & $(2,4)$ & -- & 12659\\
$(111,43)$ & 10 & $(11,3)$ & 5 & 1 & YES & YES & NO(2) & $2.00$ & $(6,2)$ & NO & 12660\\
$(111,43)$ & 10 & $(11,3)$ & 5 & 1 & YES & YES & NO(2) & $2.00$ & $(6,2)$ & -- & 12661\\
$(111,31)$ & 10 & $(13,5)$ & 5 & 1 & YES & YES & NO(3) & $1.71$ & $(4,3)$ & -- & 12662\\
$(111,31)$ & 10 & $(13,5)$ & 5 & 1 & YES & YES & NO(2) & $1.89$ & $(6,2)$ & NO & 12663\\
$(111,25)$ & 11 & $(22,9)$ & 7 & 1 & YES & YES & NO(2) & $2.00$ & $(4,3)$ & -- & 12664\\
$(111,25)$ & 11 & $(22,9)$ & 7 & 1 & YES & YES & NO(2) & $2.00$ & $(6,2)$ & NO & 12665\\
$(111,41)$ & 10 & $(78,29)$ & 10 & 3 & YES & YES & NO(2) & $2.00$ & $(2,4)$ & NO & 12666\\
$(112,47)$ & 10 & $(10,3)$ & 5 & 2 & YES & YES & NO(2) & $1.89$ & $(6,2)$ & -- & 12667\\
$(112,41)$ & 10 & $(13,5)$ & 5 & 1 & YES & YES & NO(2) & $2.00$ & $(4,3)$ & -- & 12668\\
$(112,31)$ & 10 & $(16,7)$ & 6 & 16 & YES & YES & YES & $1.86$ & $(2,4)$ & -- & 12669\\
$(112,47)$ & 10 & $(19,7)$ & 6 & 1 & YES & YES & YES & $2.00$ & $(4,3)$ & -- & 12670\\
$(112,31)$ & 10 & $(23,9)$ & 7 & 1 & YES & YES & NO(2) & $2.10$ & $(4,3)$ & -- & 12671\\
$(113,42)$ & 11 & $(93,34)$ & 10 & 1 & YES & YES & NO(2) & $2.00$ & $(6,2)$ & NO & 12672\\
$(115,44)$ & 10 & $(11,3)$ & 5 & 1 & YES & YES & YES & $2.00$ & $(4,3)$ & -- & 12673\\
$(115,34)$ & 10 & $(29,12)$ & 7 & 1 & YES & YES & YES & $2.00$ & $(2,4)$ & -- & 12674\\
$(115,44)$ & 10 & $(108,41)$ & 10 & 1 & YES & YES & NO(2) & $2.00$ & $(4,3)$ & NO & 12675\\
$(116,45)$ & 10 & $(18,7)$ & 6 & 2 & YES & YES & NO(2) & $2.11$ & $(6,2)$ & -- & 12676\\
$(116,49)$ & 10 & $(23,10)$ & 7 & 1 & YES & YES & NO(2) & $2.00$ & $(6,2)$ & -- & 12677\\
$(117,43)$ & 10 & $(13,5)$ & 5 & 13 & YES & YES & NO(2) & $2.00$ & $(6,2)$ & -- & 12678\\
$(117,49)$ & 10 & $(17,5)$ & 6 & 1 & YES & YES & NO(2) & $2.00$ & $(2,4)$ & -- & 12679\\
$(117,31)$ & 11 & $(23,9)$ & 7 & 1 & YES & YES & NO(2) & $2.00$ & $(6,2)$ & -- & 12680\\
$(117,43)$ & 10 & $(39,16)$ & 8 & 39 & YES & YES & YES & $2.00$ & $(4,3)$ & NO & 12681\\
$(117,43)$ & 10 & $(59,23)$ & 9 & 1 & YES & YES & YES & $2.17$ & $(4,3)$ & NO & 12682\\
$(117,31)$ & 11 & $(115,31)$ & 11 & 1 & YES & YES & NO(2) & $2.00$ & $(6,2)$ & NO & 12683\\
$(118,33)$ & 11 & $(11,4)$ & 5 & 1 & YES & YES & NO(2) & $1.89$ & $(2,4)$ & -- & 12684\\
$(119,44)$ & 10 & $(10,3)$ & 5 & 1 & YES & YES & NO(2) & $1.90$ & $(4,3)$ & -- & 12685\\
$(119,46)$ & 10 & $(13,5)$ & 5 & 1 & YES & YES & NO(2) & $2.00$ & $(2,4)$ & -- & 12686\\
$(119,46)$ & 10 & $(17,5)$ & 6 & 17 & YES & YES & NO(2) & $1.90$ & $(4,3)$ & -- & 12687\\
$(121,46)$ & 10 & $(16,7)$ & 6 & 1 & YES & YES & YES & $2.00$ & $(2,4)$ & NO & 12688\\
$(121,32)$ & 11 & $(17,5)$ & 6 & 1 & YES & YES & NO(2) & $1.89$ & $(6,2)$ & -- & 12689\\
$(121,35)$ & 12 & $(18,5)$ & 6 & 1 & YES & YES & NO(2) & $2.10$ & $(4,3)$ & -- & 12690\\
$(122,37)$ & 11 & $(22,5)$ & 7 & 2 & YES & YES & NO(2) & $2.00$ & $(2,4)$ & -- & 12691\\
$(122,51)$ & 11 & $(24,7)$ & 7 & 2 & YES & YES & NO(2) & $2.09$ & $(6,2)$ & NO & 12692\\
$(122,33)$ & 11 & $(29,9)$ & 8 & 1 & YES & YES & YES & $2.17$ & $(4,3)$ & -- & 12693\\
$(122,33)$ & 11 & $(29,12)$ & 7 & 1 & YES & YES & YES & $2.12$ & $(2,4)$ & -- & 12694\\
$(122,33)$ & 11 & $(42,13)$ & 9 & 2 & YES & YES & NO(2) & $2.14$ & $(4,3)$ & NO & 12695\\
$(123,47)$ & 10 & $(11,4)$ & 5 & 1 & YES & YES & NO(2) & $2.09$ & $(6,2)$ & -- & 12696\\
$(123,28)$ & 12 & $(32,5)$ & 9 & 1 & YES & YES & NO(2) & $1.89$ & $(6,2)$ & NO & 12697\\
$(123,34)$ & 10 & $(44,13)$ & 8 & 1 & YES & YES & NO(2) & $1.88$ & $(6,2)$ & NO & 12698\\
$(124,27)$ & 12 & $(14,5)$ & 6 & 2 & YES & YES & NO(2) & $2.08$ & $(4,3)$ & -- & 12699\\
$(124,27)$ & 12 & $(22,9)$ & 7 & 2 & YES & YES & NO(2) & $2.00$ & $(6,2)$ & -- & 12700\\
$(124,27)$ & 12 & $(23,9)$ & 7 & 1 & YES & YES & NO(2) & $2.00$ & $(6,2)$ & -- & 12701\\
$(125,37)$ & 11 & $(21,8)$ & 6 & 1 & YES & YES & NO(2) & $2.10$ & $(4,3)$ & NO & 12702\\
$(125,29)$ & 12 & $(29,12)$ & 7 & 1 & YES & YES & YES & $2.17$ & $(4,3)$ & -- & 12703\\
$(125,37)$ & 11 & $(45,14)$ & 9 & 5 & YES & YES & NO(2) & $2.10$ & $(4,3)$ & NO & 12704\\
$(127,47)$ & 11 & $(9,2)$ & 5 & 1 & YES & YES & YES & $1.86$ & $(2,4)$ & -- & 12705\\
$(128,47)$ & 10 & $(7,3)$ & 4 & 1 & YES & YES & NO(2) & $2.00$ & $(4,3)$ & -- & 12706\\
$(128,47)$ & 10 & $(12,5)$ & 5 & 4 & YES & YES & NO(2) & $2.09$ & $(6,2)$ & -- & 12707\\
$(128,49)$ & 10 & $(21,8)$ & 6 & 1 & YES & YES & YES & $2.00$ & $(2,4)$ & -- & 12708\\
$(128,49)$ & 10 & $(23,7)$ & 7 & 1 & YES & YES & YES & $2.00$ & $(2,4)$ & -- & 12709\\
$(129,49)$ & 10 & $(12,5)$ & 5 & 3 & YES & YES & NO(2) & $1.86$ & $(6,2)$ & NO & 12710\\
$(129,49)$ & 10 & $(12,5)$ & 5 & 3 & YES & YES & NO(2) & $1.86$ & $(6,2)$ & -- & 12711\\
$(129,29)$ & 12 & $(22,9)$ & 7 & 1 & YES & YES & NO(2) & $2.00$ & $(4,3)$ & -- & 12712\\
$(129,29)$ & 12 & $(22,9)$ & 7 & 1 & YES & YES & NO(2) & $2.00$ & $(6,2)$ & NO & 12713\\
$(129,50)$ & 10 & $(30,11)$ & 7 & 3 & YES & YES & NO(2) & $2.09$ & $(6,2)$ & NO & 12714\\
$(131,55)$ & 10 & $(7,2)$ & 4 & 1 & YES & YES & NO(2) & $2.09$ & $(2,4)$ & -- & 12715\\
$(131,50)$ & 10 & $(8,3)$ & 4 & 1 & YES & YES & NO(2) & $2.00$ & $(6,2)$ & -- & 12716\\
$(131,50)$ & 10 & $(13,5)$ & 5 & 1 & YES & YES & YES & $2.00$ & $(2,4)$ & -- & 12717\\
$(131,50)$ & 10 & $(24,7)$ & 7 & 1 & YES & YES & YES & $2.12$ & $(2,4)$ & -- & 12718\\
$(131,50)$ & 10 & $(107,41)$ & 10 & 1 & YES & YES & NO(2) & $2.18$ & $(6,2)$ & NO & 12719\\
$(132,35)$ & 11 & $(32,9)$ & 8 & 4 & YES & YES & NO(2) & $1.89$ & $(2,4)$ & NO & 12720\\
$(133,31)$ & 12 & $(13,5)$ & 5 & 1 & YES & YES & YES & $1.83$ & $(4,3)$ & -- & 12721\\
$(134,37)$ & 11 & $(11,4)$ & 5 & 1 & YES & YES & NO(2) & $1.89$ & $(2,4)$ & -- & 12722\\
$(134,39)$ & 11 & $(18,5)$ & 6 & 2 & YES & YES & NO(2) & $1.71$ & $(4,3)$ & -- & 12723\\
$(134,39)$ & 11 & $(18,7)$ & 6 & 2 & YES & YES & YES & $2.00$ & $(4,3)$ & -- & 12724\\
$(134,49)$ & 11 & $(24,7)$ & 7 & 2 & YES & YES & YES & $2.00$ & $(4,3)$ & -- & 12725\\
$(134,39)$ & 11 & $(25,9)$ & 7 & 1 & YES & YES & YES & $2.29$ & $(2,4)$ & -- & 12726\\
$(134,39)$ & 11 & $(121,35)$ & 12 & 1 & YES & YES & NO(2) & $2.10$ & $(4,3)$ & NO & 12727\\
$(135,41)$ & 11 & $(32,9)$ & 8 & 1 & YES & YES & YES & $2.17$ & $(2,4)$ & -- & 12728\\
$(137,37)$ & 11 & $(11,3)$ & 5 & 1 & YES & YES & NO(2) & $1.89$ & $(2,4)$ & -- & 12729\\
$(137,40)$ & 12 & $(19,7)$ & 6 & 1 & YES & YES & YES & $2.00$ & $(4,3)$ & NO & 12730\\
$(137,30)$ & 12 & $(24,5)$ & 8 & 1 & YES & YES & NO(2) & $1.91$ & $(6,2)$ & -- & 12731\\
$(138,41)$ & 11 & $(11,4)$ & 5 & 1 & YES & YES & NO(2) & $1.88$ & $(4,3)$ & -- & 12732\\
$(138,37)$ & 11 & $(31,9)$ & 8 & 1 & YES & YES & YES & $2.33$ & $(2,4)$ & -- & 12733\\
$(139,41)$ & 11 & $(12,5)$ & 5 & 1 & YES & YES & YES & $2.00$ & $(2,4)$ & -- & 12734\\
$(139,41)$ & 11 & $(18,5)$ & 6 & 1 & YES & YES & NO(2) & $1.88$ & $(6,2)$ & -- & 12735\\
$(139,41)$ & 11 & $(84,25)$ & 10 & 1 & YES & YES & YES & $2.00$ & $(2,4)$ & NO & 12736\\
$(140,41)$ & 11 & $(10,3)$ & 5 & 10 & YES & YES & NO(2) & $2.08$ & $(4,3)$ & NO & 12737\\
$(140,41)$ & 11 & $(10,3)$ & 5 & 10 & YES & YES & NO(2) & $2.08$ & $(4,3)$ & -- & 12738\\
$(140,41)$ & 11 & $(11,3)$ & 5 & 1 & YES & YES & NO(2) & $2.00$ & $(4,3)$ & NO & 12739\\
$(140,41)$ & 11 & $(11,3)$ & 5 & 1 & YES & YES & NO(2) & $2.00$ & $(4,3)$ & -- & 12740\\
$(140,41)$ & 11 & $(12,5)$ & 5 & 4 & YES & YES & NO(2) & $1.90$ & $(4,3)$ & -- & 12741\\
$(140,41)$ & 11 & $(13,5)$ & 5 & 1 & YES & YES & NO(2) & $2.00$ & $(4,3)$ & -- & 12742\\
$(140,39)$ & 11 & $(17,5)$ & 6 & 1 & YES & YES & NO(2) & $1.86$ & $(4,3)$ & -- & 12743\\
$(141,41)$ & 11 & $(7,3)$ & 4 & 1 & YES & YES & NO(2) & $1.80$ & $(4,3)$ & -- & 12744\\
$(144,55)$ & 10 & $(17,5)$ & 6 & 1 & YES & YES & YES & $2.17$ & $(2,4)$ & -- & 12745\\
$(144,61)$ & 11 & $(25,9)$ & 7 & 1 & YES & YES & NO(2) & $1.83$ & $(6,2)$ & NO & 12746\\
$(144,61)$ & 11 & $(31,12)$ & 7 & 1 & YES & YES & NO(2) & $1.83$ & $(6,2)$ & NO & 12747\\
$(145,56)$ & 11 & $(12,5)$ & 5 & 1 & YES & YES & NO(2) & $2.00$ & $(4,3)$ & -- & 12748\\
$(145,56)$ & 11 & $(13,5)$ & 5 & 1 & YES & YES & NO(2) & $2.09$ & $(2,4)$ & -- & 12749\\
$(146,33)$ & 12 & $(8,3)$ & 4 & 2 & YES & YES & NO(2) & $1.89$ & $(2,4)$ & NO & 12750\\
$(147,61)$ & 11 & $(8,3)$ & 4 & 1 & YES & YES & NO(2) & $2.00$ & $(8,1)$ & -- & 12751\\
$(147,41)$ & 11 & $(11,3)$ & 5 & 1 & YES & YES & NO(2) & $2.00$ & $(4,3)$ & NO & 12752\\
$(147,41)$ & 11 & $(11,3)$ & 5 & 1 & YES & YES & NO(2) & $2.00$ & $(4,3)$ & -- & 12753\\
$(147,43)$ & 11 & $(11,3)$ & 5 & 1 & YES & YES & NO(2) & $2.00$ & $(6,2)$ & NO & 12754\\
$(147,43)$ & 11 & $(11,3)$ & 5 & 1 & YES & YES & NO(2) & $2.00$ & $(6,2)$ & -- & 12755\\
$(147,41)$ & 11 & $(13,5)$ & 5 & 1 & YES & YES & NO(2) & $2.00$ & $(4,3)$ & -- & 12756\\
$(147,61)$ & 11 & $(99,41)$ & 10 & 3 & YES & YES & NO(2) & $2.00$ & $(8,1)$ & NO & 12757\\
$(149,55)$ & 11 & $(8,3)$ & 4 & 1 & YES & YES & NO(2) & $2.11$ & $(10,0)$ & -- & 12758\\
$(149,41)$ & 11 & $(11,4)$ & 5 & 1 & YES & YES & NO(2) & $1.88$ & $(4,3)$ & -- & 12759\\
$(149,44)$ & 11 & $(37,8)$ & 8 & 1 & YES & YES & YES & $2.00$ & $(2,4)$ & NO & 12760\\
$(149,41)$ & 11 & $(61,17)$ & 9 & 1 & YES & YES & NO(2) & $1.88$ & $(4,3)$ & NO & 12761\\
$(149,40)$ & 11 & $(68,19)$ & 9 & 1 & YES & YES & YES & $2.12$ & $(2,4)$ & NO & 12762\\
$(151,27)$ & 13 & $(23,9)$ & 7 & 1 & YES & YES & NO(2) & $2.00$ & $(6,2)$ & NO & 12763\\
$(151,56)$ & 11 & $(34,13)$ & 7 & 1 & YES & YES & NO(2) & $2.00$ & $(8,1)$ & NO & 12764\\
$(152,63)$ & 11 & $(23,7)$ & 7 & 1 & YES & YES & YES & $2.17$ & $(4,3)$ & -- & 12765\\
$(153,56)$ & 11 & $(11,3)$ & 5 & 1 & YES & YES & NO(2) & $2.00$ & $(4,3)$ & -- & 12766\\
$(153,41)$ & 11 & $(14,5)$ & 6 & 1 & YES & YES & NO(2) & $2.00$ & $(10,0)$ & NO & 12767\\
$(154,43)$ & 11 & $(11,3)$ & 5 & 11 & YES & YES & NO(2) & $2.00$ & $(6,2)$ & NO & 12768\\
$(154,43)$ & 11 & $(11,3)$ & 5 & 11 & YES & YES & NO(2) & $2.00$ & $(6,2)$ & -- & 12769\\
$(154,43)$ & 11 & $(13,4)$ & 6 & 1 & YES & YES & NO(3) & $1.67$ & $(4,3)$ & -- & 12770\\
$(155,46)$ & 11 & $(7,3)$ & 4 & 1 & YES & YES & NO(2) & $1.89$ & $(2,4)$ & -- & 12771\\
$(155,64)$ & 11 & $(10,3)$ & 5 & 5 & YES & YES & NO(2) & $2.00$ & $(4,3)$ & -- & 12772\\
$(157,58)$ & 11 & $(7,3)$ & 4 & 1 & YES & YES & NO(2) & $2.00$ & $(4,3)$ & -- & 12773\\
$(157,58)$ & 11 & $(8,3)$ & 4 & 1 & YES & YES & NO(2) & $2.00$ & $(4,3)$ & -- & 12774\\
$(157,60)$ & 11 & $(11,3)$ & 5 & 1 & YES & YES & NO(2) & $1.71$ & $(4,3)$ & -- & 12775\\
$(157,46)$ & 11 & $(69,20)$ & 10 & 1 & YES & YES & YES & $2.00$ & $(2,4)$ & 12964 & 12776\\
$(158,57)$ & 11 & $(48,17)$ & 9 & 2 & YES & YES & NO(2) & $2.10$ & $(4,3)$ & NO & 12777\\
$(158,61)$ & 11 & $(116,45)$ & 10 & 2 & YES & YES & NO(2) & $2.00$ & $(2,4)$ & 12969 & 12778\\
$(159,47)$ & 11 & $(12,5)$ & 5 & 3 & YES & YES & NO(2) & $1.90$ & $(4,3)$ & -- & 12779\\
$(159,44)$ & 11 & $(44,13)$ & 8 & 1 & YES & YES & NO(2) & $1.89$ & $(4,3)$ & NO & 12780\\
$(160,49)$ & 12 & $(16,5)$ & 7 & 16 & YES & YES & NO(2) & $2.00$ & $(6,2)$ & -- & 12781\\
$(162,35)$ & 12 & $(13,5)$ & 5 & 1 & YES & YES & NO(2) & $2.09$ & $(2,4)$ & NO & 12782\\
$(162,35)$ & 12 & $(67,14)$ & 10 & 1 & YES & YES & NO(2) & $1.91$ & $(6,2)$ & NO & 12783\\
$(162,35)$ & 12 & $(85,18)$ & 10 & 1 & YES & YES & NO(2) & $1.91$ & $(6,2)$ & NO & 12784\\
$(164,45)$ & 12 & $(68,19)$ & 9 & 4 & YES & YES & NO(2) & $2.00$ & $(4,3)$ & NO & 12785\\
$(165,61)$ & 11 & $(7,3)$ & 4 & 1 & YES & YES & NO(2) & $2.08$ & $(4,3)$ & NO & 12786\\
$(165,61)$ & 11 & $(7,3)$ & 4 & 1 & YES & YES & NO(2) & $2.08$ & $(4,3)$ & -- & 12787\\
$(165,46)$ & 11 & $(21,8)$ & 6 & 3 & YES & YES & YES & $2.00$ & $(2,4)$ & -- & 12788\\
$(167,64)$ & 11 & $(8,3)$ & 4 & 1 & YES & YES & NO(2) & $2.09$ & $(2,4)$ & -- & 12789\\
$(168,71)$ & 11 & $(29,8)$ & 7 & 1 & YES & YES & YES & $2.14$ & $(4,3)$ & -- & 12790\\
$(169,70)$ & 11 & $(16,7)$ & 6 & 1 & YES & YES & YES & $2.00$ & $(2,4)$ & NO & 12791\\
$(169,50)$ & 11 & $(55,16)$ & 9 & 1 & YES & YES & NO(2) & $1.78$ & $(6,2)$ & NO & 12792\\
$(169,50)$ & 11 & $(61,18)$ & 9 & 1 & YES & YES & NO(2) & $1.91$ & $(2,4)$ & NO & 12793\\
$(170,47)$ & 11 & $(19,8)$ & 6 & 1 & YES & YES & YES & $2.12$ & $(2,4)$ & NO & 12794\\
$(170,47)$ & 11 & $(44,13)$ & 8 & 2 & YES & YES & YES & $2.12$ & $(2,4)$ & NO & 12795\\
$(171,50)$ & 11 & $(7,2)$ & 4 & 1 & YES & YES & NO(2) & $1.80$ & $(4,3)$ & -- & 12796\\
$(172,71)$ & 11 & $(18,5)$ & 6 & 2 & YES & YES & YES & $2.14$ & $(2,4)$ & -- & 12797\\
$(173,64)$ & 11 & $(10,3)$ & 5 & 1 & YES & YES & NO(2) & $1.88$ & $(4,3)$ & -- & 12798\\
$(173,73)$ & 11 & $(12,5)$ & 5 & 1 & YES & YES & NO(2) & $1.86$ & $(4,3)$ & -- & 12799\\
$(173,38)$ & 13 & $(13,5)$ & 5 & 1 & YES & YES & NO(2) & $1.88$ & $(8,1)$ & -- & 12800\\
$(173,51)$ & 12 & $(13,3)$ & 6 & 1 & YES & YES & YES & $2.00$ & $(2,4)$ & -- & 12801\\
$(173,71)$ & 12 & $(14,3)$ & 6 & 1 & YES & YES & NO(2) & $2.00$ & $(6,2)$ & -- & 12802\\
$(173,64)$ & 11 & $(127,47)$ & 11 & 1 & YES & YES & YES & $1.86$ & $(2,4)$ & NO & 12803\\
$(175,47)$ & 11 & $(32,9)$ & 8 & 1 & YES & YES & NO(2) & $1.90$ & $(8,1)$ & NO & 12804\\
$(175,52)$ & 12 & $(155,46)$ & 11 & 5 & YES & YES & NO(2) & $1.90$ & $(4,3)$ & 13398 & 12805\\
$(176,65)$ & 11 & $(5,2)$ & 3 & 1 & YES & YES & NO(2) & $2.00$ & $(2,4)$ & -- & 12806\\
$(176,65)$ & 11 & $(12,5)$ & 5 & 4 & YES & YES & YES & $2.00$ & $(2,4)$ & -- & 12807\\
$(176,73)$ & 12 & $(13,3)$ & 6 & 1 & YES & YES & NO(2) & $2.20$ & $(2,4)$ & NO & 12808\\
$(176,65)$ & 11 & $(62,23)$ & 9 & 2 & YES & YES & NO(2) & $1.89$ & $(2,4)$ & NO & 12809\\
$(178,47)$ & 12 & $(22,5)$ & 7 & 2 & YES & YES & NO(2) & $1.78$ & $(6,2)$ & -- & 12810\\
$(178,47)$ & 12 & $(63,17)$ & 9 & 1 & YES & YES & NO(2) & $1.89$ & $(6,2)$ & NO & 12811\\
$(179,75)$ & 11 & $(8,3)$ & 4 & 1 & YES & YES & NO(2) & $1.86$ & $(4,3)$ & -- & 12812\\
$(179,33)$ & 13 & $(13,5)$ & 5 & 1 & YES & YES & NO(2) & $2.00$ & $(2,4)$ & NO & 12813\\
$(179,75)$ & 11 & $(17,5)$ & 6 & 1 & YES & YES & YES & $2.00$ & $(2,4)$ & -- & 12814\\
$(179,28)$ & 14 & $(25,7)$ & 7 & 1 & YES & YES & NO(2) & $1.89$ & $(6,2)$ & NO & 12815\\
$(179,28)$ & 14 & $(31,7)$ & 8 & 1 & YES & YES & NO(2) & $1.89$ & $(6,2)$ & NO & 12816\\
$(181,70)$ & 11 & $(5,2)$ & 3 & 1 & YES & YES & NO(2) & $2.00$ & $(2,4)$ & -- & 12817\\
$(181,70)$ & 11 & $(7,2)$ & 4 & 1 & YES & YES & NO(2) & $1.83$ & $(6,2)$ & NO & 12818\\
$(181,70)$ & 11 & $(7,2)$ & 4 & 1 & YES & YES & NO(2) & $1.83$ & $(6,2)$ & -- & 12819\\
$(181,70)$ & 11 & $(8,3)$ & 4 & 1 & YES & YES & NO(2) & $2.00$ & $(2,4)$ & -- & 12820\\
$(181,41)$ & 12 & $(17,7)$ & 6 & 1 & YES & YES & NO(2) & $2.10$ & $(4,3)$ & -- & 12821\\
$(181,70)$ & 11 & $(57,22)$ & 9 & 1 & YES & YES & NO(2) & $1.75$ & $(4,3)$ & NO & 12822\\
$(181,76)$ & 11 & $(67,28)$ & 10 & 1 & YES & YES & NO(2) & $2.08$ & $(4,3)$ & NO & 12823\\
$(183,67)$ & 11 & $(11,4)$ & 5 & 1 & YES & YES & YES & $2.00$ & $(2,4)$ & -- & 12824\\
$(183,49)$ & 12 & $(16,5)$ & 7 & 1 & YES & YES & NO(2) & $2.00$ & $(6,2)$ & -- & 12825\\
$(183,49)$ & 12 & $(115,31)$ & 11 & 1 & YES & YES & NO(2) & $2.00$ & $(6,2)$ & NO & 12826\\
$(185,68)$ & 11 & $(5,2)$ & 3 & 5 & YES & YES & NO(2) & $2.08$ & $(6,2)$ & -- & 12827\\
$(185,68)$ & 11 & $(19,7)$ & 6 & 1 & YES & YES & YES & $2.33$ & $(2,4)$ & -- & 12828\\
$(185,68)$ & 11 & $(62,23)$ & 9 & 1 & YES & YES & NO(2) & $2.00$ & $(4,3)$ & NO & 12829\\
$(186,55)$ & 11 & $(5,2)$ & 3 & 1 & YES & YES & NO(2) & $2.00$ & $(4,3)$ & -- & 12830\\
$(186,77)$ & 13 & $(8,3)$ & 4 & 2 & YES & YES & NO(2) & $2.12$ & $(4,3)$ & -- & 12831\\
$(186,77)$ & 13 & $(11,3)$ & 5 & 1 & YES & YES & NO(2) & $2.12$ & $(4,3)$ & NO & 12832\\
$(187,79)$ & 11 & $(8,3)$ & 4 & 1 & YES & YES & NO(2) & $2.08$ & $(4,3)$ & -- & 12833\\
$(187,79)$ & 11 & $(10,3)$ & 5 & 1 & YES & YES & YES & $1.86$ & $(2,4)$ & NO & 12834\\
$(188,41)$ & 12 & $(14,5)$ & 6 & 2 & YES & YES & NO(2) & $2.10$ & $(4,3)$ & NO & 12835\\
$(188,51)$ & 13 & $(122,33)$ & 11 & 2 & YES & YES & NO(2) & $1.88$ & $(8,1)$ & NO & 12836\\
$(188,69)$ & 11 & $(185,68)$ & 11 & 1 & YES & YES & YES & $2.14$ & $(2,4)$ & NO & 12837\\
$(189,55)$ & 12 & $(11,3)$ & 5 & 1 & YES & YES & NO(2) & $2.00$ & $(4,3)$ & -- & 12838\\
$(189,55)$ & 12 & $(13,5)$ & 5 & 1 & YES & YES & NO(2) & $2.00$ & $(6,2)$ & -- & 12839\\
$(189,55)$ & 12 & $(37,11)$ & 8 & 1 & YES & YES & NO(2) & $2.00$ & $(6,2)$ & NO & 12840\\
$(191,56)$ & 12 & $(42,13)$ & 9 & 1 & YES & YES & NO(2) & $2.12$ & $(4,3)$ & NO & 12841\\
$(192,71)$ & 11 & $(5,1)$ & 4 & 1 & YES & YES & NO(2) & $2.11$ & $(2,4)$ & NO & 12842\\
$(192,71)$ & 11 & $(5,1)$ & 4 & 1 & YES & YES & NO(2) & $2.11$ & $(2,4)$ & -- & 12843\\
$(192,71)$ & 11 & $(5,1)$ & 4 & 1 & YES & YES & NO(2) & $2.11$ & $(2,4)$ & NO & 12844\\
$(192,71)$ & 11 & $(7,3)$ & 4 & 1 & YES & YES & NO(2) & $2.09$ & $(6,2)$ & NO & 12845\\
$(192,71)$ & 11 & $(7,3)$ & 4 & 1 & YES & YES & NO(2) & $2.09$ & $(6,2)$ & -- & 12846\\
$(192,71)$ & 11 & $(10,3)$ & 5 & 2 & YES & YES & NO(2) & $2.00$ & $(4,3)$ & -- & 12847\\
$(192,71)$ & 11 & $(176,65)$ & 11 & 16 & YES & YES & YES & $2.00$ & $(2,4)$ & NO & 12848\\
$(193,81)$ & 11 & $(5,2)$ & 3 & 1 & YES & YES & NO(2) & $1.89$ & $(2,4)$ & -- & 12849\\
$(193,81)$ & 11 & $(10,3)$ & 5 & 1 & YES & YES & NO(2) & $2.00$ & $(2,4)$ & -- & 12850\\
$(193,44)$ & 12 & $(37,8)$ & 8 & 1 & YES & YES & NO(2) & $2.00$ & $(2,4)$ & NO & 12851\\
$(196,43)$ & 13 & $(18,7)$ & 6 & 2 & YES & YES & NO(2) & $2.09$ & $(6,2)$ & -- & 12852\\
$(199,55)$ & 11 & $(9,2)$ & 5 & 1 & YES & YES & YES & $1.86$ & $(2,4)$ & -- & 12853\\
$(199,55)$ & 11 & $(9,2)$ & 5 & 1 & YES & YES & YES & $1.86$ & $(2,4)$ & NO & 12854\\
$(200,53)$ & 12 & $(48,13)$ & 9 & 8 & YES & YES & NO(2) & $2.00$ & $(6,2)$ & NO & 12855\\
$(200,59)$ & 12 & $(115,34)$ & 10 & 5 & YES & YES & NO(2) & $1.75$ & $(8,1)$ & NO & 12856\\
$(201,76)$ & 12 & $(8,3)$ & 4 & 1 & YES & YES & NO(2) & $2.09$ & $(10,0)$ & -- & 12857\\
$(202,59)$ & 12 & $(8,3)$ & 4 & 2 & YES & YES & NO(2) & $2.00$ & $(2,4)$ & -- & 12858\\
$(202,89)$ & 12 & $(10,3)$ & 5 & 2 & YES & YES & NO(2) & $2.10$ & $(4,3)$ & -- & 12859\\
$(202,59)$ & 12 & $(99,29)$ & 10 & 1 & YES & YES & NO(2) & $2.00$ & $(2,4)$ & NO & 12860\\
$(203,44)$ & 12 & $(35,8)$ & 8 & 7 & YES & YES & NO(2) & $2.00$ & $(2,4)$ & NO & 12861\\
$(205,62)$ & 13 & $(10,3)$ & 5 & 5 & YES & YES & NO(2) & $2.12$ & $(4,3)$ & NO & 12862\\
$(205,62)$ & 13 & $(10,3)$ & 5 & 5 & YES & YES & NO(2) & $2.12$ & $(4,3)$ & -- & 12863\\
$(205,73)$ & 13 & $(10,3)$ & 5 & 5 & YES & YES & YES & $2.00$ & $(4,3)$ & -- & 12864\\
$(205,61)$ & 12 & $(18,5)$ & 6 & 1 & YES & YES & YES & $2.17$ & $(2,4)$ & -- & 12865\\
$(206,87)$ & 12 & $(187,79)$ & 11 & 1 & YES & YES & YES & $1.86$ & $(2,4)$ & 13213 & 12866\\
$(207,79)$ & 11 & $(7,2)$ & 4 & 1 & YES & YES & YES & $1.83$ & $(2,4)$ & -- & 12867\\
$(207,79)$ & 11 & $(17,5)$ & 6 & 1 & YES & YES & YES & $2.00$ & $(2,4)$ & -- & 12868\\
$(207,76)$ & 11 & $(35,13)$ & 8 & 1 & YES & YES & NO(2) & $2.00$ & $(6,2)$ & NO & 12869\\
$(207,79)$ & 11 & $(128,49)$ & 10 & 1 & YES & YES & YES & $2.00$ & $(2,4)$ & NO & 12870\\
$(207,76)$ & 11 & $(177,65)$ & 11 & 3 & YES & YES & NO(2) & $2.00$ & $(8,1)$ & NO & 12871\\
$(208,37)$ & 13 & $(13,5)$ & 5 & 13 & YES & YES & NO(2) & $2.00$ & $(2,4)$ & NO & 12872\\
$(208,61)$ & 12 & $(18,5)$ & 6 & 2 & YES & YES & YES & $2.17$ & $(2,4)$ & -- & 12873\\
$(208,79)$ & 11 & $(208,79)$ & 11 & 208 & YES & YES & NO(2) & $1.78$ & $(2,4)$ & NO & 12874\\
$(209,80)$ & 11 & $(5,2)$ & 3 & 1 & YES & YES & NO(2) & $2.00$ & $(4,3)$ & -- & 12875\\
$(209,80)$ & 11 & $(7,2)$ & 4 & 1 & YES & YES & NO(3) & $1.83$ & $(2,4)$ & -- & 12876\\
$(209,59)$ & 13 & $(17,4)$ & 7 & 1 & YES & YES & NO(2) & $2.00$ & $(8,1)$ & NO & 12877\\
$(209,59)$ & 13 & $(23,6)$ & 8 & 1 & YES & YES & NO(2) & $2.00$ & $(8,1)$ & NO & 12878\\
$(209,81)$ & 11 & $(29,11)$ & 7 & 1 & YES & YES & NO(2) & $2.00$ & $(4,3)$ & NO & 12879\\
$(212,81)$ & 11 & $(3,1)$ & 2 & 1 & YES & YES & NO(2) & $1.89$ & $(2,4)$ & -- & 12880\\
$(212,89)$ & 11 & $(4,1)$ & 3 & 4 & YES & YES & NO(2) & $2.09$ & $(2,4)$ & NO & 12881\\
$(212,89)$ & 11 & $(4,1)$ & 3 & 4 & YES & YES & NO(2) & $2.09$ & $(2,4)$ & -- & 12882\\
$(212,87)$ & 13 & $(11,2)$ & 6 & 1 & YES & YES & NO(2) & $2.00$ & $(6,2)$ & NO & 12883\\
$(212,87)$ & 13 & $(11,2)$ & 6 & 1 & YES & YES & NO(2) & $2.00$ & $(6,2)$ & -- & 12884\\
$(212,81)$ & 11 & $(12,5)$ & 5 & 4 & YES & YES & YES & $2.17$ & $(2,4)$ & -- & 12885\\
$(212,81)$ & 11 & $(199,76)$ & 11 & 1 & YES & YES & NO(2) & $2.00$ & $(6,2)$ & NO & 12886\\
$(213,59)$ & 12 & $(123,34)$ & 10 & 3 & YES & YES & NO(2) & $1.88$ & $(8,1)$ & NO & 12887\\
$(214,65)$ & 12 & $(10,3)$ & 5 & 2 & YES & YES & NO(2) & $2.00$ & $(2,4)$ & -- & 12888\\
$(214,41)$ & 14 & $(57,10)$ & 10 & 1 & YES & YES & NO(2) & $2.00$ & $(6,2)$ & NO & 12889\\
$(217,47)$ & 12 & $(11,4)$ & 5 & 1 & YES & YES & NO(2) & $1.90$ & $(8,1)$ & -- & 12890\\
$(218,59)$ & 12 & $(43,12)$ & 8 & 1 & YES & YES & NO(2) & $2.10$ & $(4,3)$ & NO & 12891\\
$(219,61)$ & 12 & $(13,5)$ & 5 & 1 & YES & YES & YES & $2.14$ & $(2,4)$ & -- & 12892\\
$(219,61)$ & 12 & $(16,7)$ & 6 & 1 & YES & YES & YES & $2.17$ & $(4,3)$ & NO & 12893\\
$(219,61)$ & 12 & $(16,7)$ & 6 & 1 & YES & YES & YES & $2.17$ & $(4,3)$ & -- & 12894\\
$(219,64)$ & 12 & $(33,10)$ & 8 & 3 & YES & YES & NO(2) & $1.91$ & $(6,2)$ & NO & 12895\\
$(223,82)$ & 12 & $(62,23)$ & 9 & 1 & YES & YES & NO(2) & $2.00$ & $(4,3)$ & NO & 12896\\
$(226,61)$ & 12 & $(7,3)$ & 4 & 1 & YES & YES & NO(2) & $2.08$ & $(4,3)$ & NO & 12897\\
$(226,61)$ & 12 & $(7,3)$ & 4 & 1 & YES & YES & NO(2) & $2.08$ & $(4,3)$ & -- & 12898\\
$(227,61)$ & 12 & $(7,3)$ & 4 & 1 & YES & YES & NO(2) & $2.08$ & $(4,3)$ & -- & 12899\\
$(227,67)$ & 12 & $(12,5)$ & 5 & 1 & YES & YES & YES & $2.17$ & $(2,4)$ & NO & 12900\\
$(227,67)$ & 12 & $(12,5)$ & 5 & 1 & YES & YES & YES & $2.17$ & $(2,4)$ & -- & 12901\\
$(227,60)$ & 12 & $(17,5)$ & 6 & 1 & YES & YES & NO(2) & $1.89$ & $(6,2)$ & NO & 12902\\
$(227,67)$ & 12 & $(27,8)$ & 7 & 1 & YES & YES & NO(2) & $1.90$ & $(4,3)$ & NO & 12903\\
$(227,61)$ & 12 & $(43,12)$ & 8 & 1 & YES & YES & YES & $2.00$ & $(2,4)$ & NO & 12904\\
$(227,66)$ & 12 & $(44,13)$ & 8 & 1 & YES & YES & NO(2) & $1.88$ & $(6,2)$ & NO & 12905\\
$(229,68)$ & 12 & $(5,2)$ & 3 & 1 & YES & YES & NO(2) & $2.00$ & $(6,2)$ & -- & 12906\\
$(229,68)$ & 12 & $(11,3)$ & 5 & 1 & YES & YES & NO(2) & $2.00$ & $(6,2)$ & NO & 12907\\
$(231,61)$ & 13 & $(63,17)$ & 9 & 21 & YES & YES & NO(2) & $2.00$ & $(6,2)$ & NO & 12908\\
$(233,89)$ & 11 & $(5,2)$ & 3 & 1 & YES & YES & NO(2) & $1.75$ & $(2,4)$ & -- & 12909\\
$(233,89)$ & 11 & $(7,3)$ & 4 & 1 & YES & YES & NO(2) & $1.86$ & $(4,3)$ & -- & 12910\\
$(233,91)$ & 12 & $(8,3)$ & 4 & 1 & YES & YES & NO(2) & $1.88$ & $(8,1)$ & -- & 12911\\
$(233,89)$ & 11 & $(12,5)$ & 5 & 1 & YES & YES & YES & $2.14$ & $(2,4)$ & -- & 12912\\
$(233,91)$ & 12 & $(44,17)$ & 8 & 1 & YES & YES & NO(2) & $1.88$ & $(8,1)$ & NO & 12913\\
$(233,89)$ & 11 & $(76,29)$ & 9 & 1 & YES & YES & NO(2) & $2.00$ & $(6,2)$ & NO & 12914\\
$(233,89)$ & 11 & $(107,41)$ & 10 & 1 & YES & YES & YES & $2.17$ & $(2,4)$ & NO & 12915\\
$(233,89)$ & 11 & $(199,76)$ & 11 & 1 & YES & YES & NO(2) & $2.00$ & $(4,3)$ & NO & 12916\\
$(235,97)$ & 12 & $(19,8)$ & 6 & 1 & YES & YES & NO(2) & $2.00$ & $(4,3)$ & NO & 12917\\
$(235,97)$ & 12 & $(155,64)$ & 11 & 5 & YES & YES & NO(2) & $2.00$ & $(4,3)$ & 13302 & 12918\\
$(236,69)$ & 12 & $(5,2)$ & 3 & 1 & YES & YES & NO(2) & $2.00$ & $(4,3)$ & -- & 12919\\
$(236,65)$ & 12 & $(24,7)$ & 7 & 4 & YES & YES & YES & $2.33$ & $(2,4)$ & -- & 12920\\
$(236,65)$ & 12 & $(98,27)$ & 10 & 2 & YES & YES & NO(2) & $2.00$ & $(2,4)$ & 12984 & 12921\\
$(237,85)$ & 12 & $(10,3)$ & 5 & 1 & YES & YES & NO(2) & $2.00$ & $(4,3)$ & -- & 12922\\
$(237,85)$ & 12 & $(89,32)$ & 10 & 1 & YES & YES & NO(2) & $2.00$ & $(4,3)$ & NO & 12923\\
$(239,71)$ & 12 & $(7,3)$ & 4 & 1 & YES & YES & NO(2) & $2.00$ & $(4,3)$ & -- & 12924\\
$(239,66)$ & 12 & $(65,18)$ & 9 & 1 & YES & YES & NO(2) & $1.78$ & $(6,2)$ & NO & 12925\\
$(241,64)$ & 13 & $(5,2)$ & 3 & 1 & YES & YES & YES & $1.86$ & $(2,4)$ & -- & 12926\\
$(241,65)$ & 12 & $(241,65)$ & 12 & 241 & YES & YES & NO(2) & $2.00$ & $(2,4)$ & NO & 12927\\
$(242,87)$ & 12 & $(10,3)$ & 5 & 2 & YES & YES & NO(2) & $2.00$ & $(4,3)$ & -- & 12928\\
$(242,87)$ & 12 & $(92,33)$ & 10 & 2 & YES & YES & NO(2) & $2.00$ & $(4,3)$ & NO & 12929\\
$(243,94)$ & 12 & $(8,3)$ & 4 & 1 & YES & YES & NO(2) & $2.11$ & $(10,0)$ & NO & 12930\\
$(243,71)$ & 12 & $(17,5)$ & 6 & 1 & YES & YES & YES & $2.00$ & $(2,4)$ & -- & 12931\\
$(243,94)$ & 12 & $(168,65)$ & 12 & 3 & YES & YES & YES & $2.00$ & $(2,4)$ & NO & 12932\\
$(244,107)$ & 13 & $(9,2)$ & 5 & 1 & YES & YES & NO(2) & $1.91$ & $(6,2)$ & -- & 12933\\
$(244,55)$ & 13 & $(13,5)$ & 5 & 1 & YES & YES & NO(2) & $1.88$ & $(8,1)$ & -- & 12934\\
$(246,101)$ & 12 & $(7,2)$ & 4 & 1 & YES & YES & NO(2) & $2.00$ & $(4,3)$ & -- & 12935\\
$(246,91)$ & 12 & $(13,5)$ & 5 & 1 & YES & YES & NO(2) & $2.17$ & $(4,3)$ & NO & 12936\\
$(246,95)$ & 12 & $(25,9)$ & 7 & 1 & YES & YES & YES & $2.14$ & $(2,4)$ & NO & 12937\\
$(246,101)$ & 12 & $(83,34)$ & 10 & 1 & YES & YES & NO(2) & $2.00$ & $(4,3)$ & NO & 12938\\
$(247,69)$ & 12 & $(247,69)$ & 12 & 247 & YES & YES & NO(2) & $1.89$ & $(2,4)$ & NO & 12939\\
$(248,109)$ & 12 & $(53,23)$ & 9 & 1 & YES & YES & YES & $2.00$ & $(4,3)$ & NO & 12940\\
$(248,91)$ & 12 & $(128,47)$ & 10 & 8 & YES & YES & NO(2) & $2.09$ & $(6,2)$ & NO & 12941\\
$(249,76)$ & 12 & $(3,1)$ & 2 & 3 & YES & YES & NO(2) & $2.09$ & $(2,4)$ & NO & 12942\\
$(249,76)$ & 12 & $(3,1)$ & 2 & 3 & YES & YES & NO(2) & $2.09$ & $(2,4)$ & -- & 12943\\
$(249,95)$ & 12 & $(9,4)$ & 5 & 3 & YES & YES & NO(2) & $2.00$ & $(6,2)$ & -- & 12944\\
$(249,73)$ & 13 & $(13,3)$ & 6 & 1 & YES & YES & YES & $2.12$ & $(2,4)$ & -- & 12945\\
$(249,95)$ & 12 & $(89,34)$ & 9 & 1 & YES & YES & YES & $1.83$ & $(4,3)$ & NO & 12946\\
$(250,97)$ & 12 & $(5,2)$ & 3 & 5 & YES & YES & NO(2) & $1.90$ & $(12,-1)$ & -- & 12947\\
$(250,97)$ & 12 & $(11,4)$ & 5 & 1 & YES & YES & NO(2) & $1.88$ & $(4,3)$ & NO & 12948\\
$(250,97)$ & 12 & $(165,64)$ & 11 & 5 & YES & YES & NO(2) & $2.00$ & $(12,-1)$ & 13352 & 12949\\
$(251,70)$ & 12 & $(10,3)$ & 5 & 1 & YES & YES & NO(2) & $1.90$ & $(8,1)$ & NO & 12950\\
$(251,70)$ & 12 & $(165,46)$ & 11 & 1 & YES & YES & NO(2) & $2.00$ & $(6,2)$ & NO & 12951\\
$(252,73)$ & 13 & $(58,17)$ & 9 & 2 & YES & YES & NO(2) & $2.00$ & $(4,3)$ & NO & 12952\\
$(253,98)$ & 12 & $(12,5)$ & 5 & 1 & YES & YES & YES & $2.14$ & $(4,3)$ & -- & 12953\\
$(253,74)$ & 12 & $(243,71)$ & 12 & 1 & YES & YES & YES & $2.00$ & $(2,4)$ & NO & 12954\\
$(254,71)$ & 12 & $(13,5)$ & 5 & 1 & YES & YES & YES & $2.00$ & $(2,4)$ & -- & 12955\\
$(254,75)$ & 12 & $(115,34)$ & 10 & 1 & YES & YES & YES & $2.00$ & $(2,4)$ & NO & 12956\\
$(254,71)$ & 12 & $(165,46)$ & 11 & 1 & YES & YES & YES & $2.00$ & $(2,4)$ & NO & 12957\\
$(255,112)$ & 12 & $(5,2)$ & 3 & 5 & YES & YES & NO(2) & $1.90$ & $(4,3)$ & -- & 12958\\
$(255,92)$ & 12 & $(53,19)$ & 9 & 1 & YES & YES & NO(2) & $2.10$ & $(4,3)$ & NO & 12959\\
$(256,75)$ & 12 & $(2,1)$ & 1 & 2 & YES & YES & NO(2) & $1.92$ & $(4,3)$ & -- & 12960\\
$(256,99)$ & 12 & $(5,2)$ & 3 & 1 & YES & YES & NO(2) & $2.10$ & $(4,3)$ & -- & 12961\\
$(256,99)$ & 12 & $(8,3)$ & 4 & 8 & YES & YES & NO(2) & $2.00$ & $(4,3)$ & -- & 12962\\
$(256,99)$ & 12 & $(11,4)$ & 5 & 1 & YES & YES & NO(2) & $2.10$ & $(4,3)$ & NO & 12963\\
$(256,75)$ & 12 & $(38,11)$ & 9 & 2 & YES & YES & YES & $2.00$ & $(2,4)$ & 12776 & 12964\\
$(256,95)$ & 12 & $(89,33)$ & 10 & 1 & YES & YES & NO(2) & $1.86$ & $(6,2)$ & NO & 12965\\
$(256,75)$ & 12 & $(256,75)$ & 12 & 256 & YES & YES & NO(2) & $1.75$ & $(4,3)$ & NO & 12966\\
$(257,76)$ & 12 & $(27,8)$ & 7 & 1 & YES & YES & NO(2) & $1.90$ & $(4,3)$ & NO & 12967\\
$(258,71)$ & 12 & $(10,3)$ & 5 & 2 & YES & YES & NO(2) & $1.89$ & $(6,2)$ & -- & 12968\\
$(259,100)$ & 12 & $(67,26)$ & 9 & 1 & YES & YES & NO(2) & $2.00$ & $(2,4)$ & 12778 & 12969\\
$(261,100)$ & 12 & $(18,7)$ & 6 & 9 & YES & YES & YES & $2.00$ & $(2,4)$ & NO & 12970\\
$(261,76)$ & 13 & $(29,9)$ & 8 & 29 & YES & YES & YES & $2.17$ & $(4,3)$ & NO & 12971\\
$(261,100)$ & 12 & $(167,64)$ & 11 & 1 & YES & YES & NO(2) & $2.09$ & $(2,4)$ & NO & 12972\\
$(263,71)$ & 12 & $(5,1)$ & 4 & 1 & YES & YES & NO(2) & $2.00$ & $(6,2)$ & NO & 12973\\
$(263,71)$ & 12 & $(5,1)$ & 4 & 1 & YES & YES & NO(2) & $2.00$ & $(6,2)$ & -- & 12974\\
$(263,71)$ & 12 & $(5,1)$ & 4 & 1 & YES & YES & NO(2) & $2.00$ & $(6,2)$ & NO & 12975\\
$(263,71)$ & 12 & $(12,5)$ & 5 & 1 & YES & YES & NO(2) & $2.11$ & $(2,4)$ & NO & 12976\\
$(263,59)$ & 14 & $(15,4)$ & 6 & 1 & YES & YES & NO(2) & $2.10$ & $(8,1)$ & NO & 12977\\
$(263,71)$ & 12 & $(122,33)$ & 11 & 1 & YES & YES & NO(2) & $2.00$ & $(6,2)$ & NO & 12978\\
$(263,71)$ & 12 & $(137,37)$ & 11 & 1 & YES & YES & NO(2) & $1.89$ & $(2,4)$ & NO & 12979\\
$(263,100)$ & 12 & $(263,100)$ & 12 & 263 & YES & YES & NO(2) & $1.75$ & $(4,3)$ & NO & 12980\\
$(264,101)$ & 12 & $(5,2)$ & 3 & 1 & YES & YES & YES & $2.00$ & $(2,4)$ & -- & 12981\\
$(265,62)$ & 14 & $(12,5)$ & 5 & 1 & YES & YES & YES & $2.00$ & $(4,3)$ & -- & 12982\\
$(265,73)$ & 12 & $(40,11)$ & 8 & 5 & YES & YES & NO(2) & $2.00$ & $(2,4)$ & 13035 & 12983\\
$(265,73)$ & 12 & $(69,19)$ & 9 & 1 & YES & YES & NO(2) & $2.00$ & $(2,4)$ & 12921 & 12984\\
$(266,101)$ & 12 & $(10,3)$ & 5 & 2 & YES & YES & NO(2) & $2.00$ & $(4,3)$ & -- & 12985\\
$(266,101)$ & 12 & $(47,18)$ & 8 & 1 & YES & YES & NO(2) & $2.00$ & $(4,3)$ & NO & 12986\\
$(267,98)$ & 12 & $(8,3)$ & 4 & 1 & YES & YES & NO(2) & $1.88$ & $(8,1)$ & -- & 12987\\
$(268,99)$ & 12 & $(5,2)$ & 3 & 1 & YES & YES & NO(2) & $2.00$ & $(2,4)$ & NO & 12988\\
$(268,99)$ & 12 & $(7,2)$ & 4 & 1 & YES & YES & NO(2) & $1.88$ & $(6,2)$ & NO & 12989\\
$(268,99)$ & 12 & $(7,2)$ & 4 & 1 & YES & YES & NO(2) & $1.88$ & $(6,2)$ & -- & 12990\\
$(268,79)$ & 13 & $(47,14)$ & 9 & 1 & YES & YES & NO(2) & $2.10$ & $(4,3)$ & NO & 12991\\
$(269,71)$ & 13 & $(5,2)$ & 3 & 1 & YES & YES & NO(2) & $2.00$ & $(12,-1)$ & -- & 12992\\
$(269,75)$ & 12 & $(9,4)$ & 5 & 1 & YES & YES & NO(2) & $2.00$ & $(6,2)$ & NO & 12993\\
$(269,75)$ & 12 & $(11,4)$ & 5 & 1 & YES & YES & NO(2) & $2.00$ & $(6,2)$ & NO & 12994\\
$(269,75)$ & 12 & $(25,7)$ & 7 & 1 & YES & YES & NO(2) & $1.89$ & $(2,4)$ & NO & 12995\\
$(270,97)$ & 12 & $(5,2)$ & 3 & 5 & YES & YES & NO(2) & $2.00$ & $(4,3)$ & -- & 12996\\
$(270,103)$ & 12 & $(18,7)$ & 6 & 18 & YES & YES & NO(2) & $1.88$ & $(4,3)$ & NO & 12997\\
$(270,103)$ & 12 & $(34,13)$ & 7 & 2 & YES & YES & NO(2) & $1.89$ & $(6,2)$ & NO & 12998\\
$(271,75)$ & 12 & $(2,1)$ & 1 & 1 & YES & YES & NO(2) & $1.89$ & $(6,2)$ & -- & 12999\\
$(271,80)$ & 12 & $(12,5)$ & 5 & 1 & YES & YES & YES & $2.14$ & $(2,4)$ & -- & 13000\\
$(273,100)$ & 12 & $(10,3)$ & 5 & 1 & YES & YES & NO(2) & $1.88$ & $(4,3)$ & -- & 13001\\
$(273,100)$ & 12 & $(93,34)$ & 10 & 3 & YES & YES & NO(2) & $2.00$ & $(4,3)$ & NO & 13002\\
$(273,100)$ & 12 & $(273,100)$ & 12 & 273 & YES & YES & NO(2) & $1.88$ & $(4,3)$ & NO & 13003\\
$(274,107)$ & 12 & $(5,2)$ & 3 & 1 & YES & YES & NO(2) & $2.00$ & $(6,2)$ & -- & 13004\\
$(274,107)$ & 12 & $(9,2)$ & 5 & 1 & YES & YES & NO(2) & $1.90$ & $(4,3)$ & -- & 13005\\
$(275,76)$ & 12 & $(3,1)$ & 2 & 1 & YES & YES & NO(2) & $1.90$ & $(8,1)$ & -- & 13006\\
$(275,62)$ & 13 & $(97,22)$ & 11 & 1 & YES & YES & NO(2) & $1.91$ & $(6,2)$ & NO & 13007\\
$(275,76)$ & 12 & $(199,55)$ & 11 & 1 & YES & YES & YES & $1.86$ & $(2,4)$ & NO & 13008\\
$(277,81)$ & 12 & $(2,1)$ & 1 & 1 & YES & YES & NO(2) & $1.92$ & $(4,3)$ & -- & 13009\\
$(277,81)$ & 12 & $(4,1)$ & 3 & 1 & YES & YES & NO(2) & $2.00$ & $(2,4)$ & NO & 13010\\
$(277,81)$ & 12 & $(4,1)$ & 3 & 1 & YES & YES & NO(2) & $2.00$ & $(2,4)$ & -- & 13011\\
$(277,81)$ & 12 & $(4,1)$ & 3 & 1 & YES & YES & NO(2) & $2.00$ & $(4,3)$ & NO & 13012\\
$(277,81)$ & 12 & $(7,3)$ & 4 & 1 & YES & YES & NO(2) & $1.86$ & $(6,2)$ & NO & 13013\\
$(277,60)$ & 13 & $(10,3)$ & 5 & 1 & YES & YES & NO(2) & $1.71$ & $(4,3)$ & NO & 13014\\
$(277,81)$ & 12 & $(47,14)$ & 9 & 1 & YES & YES & YES & $2.14$ & $(2,4)$ & NO & 13015\\
$(277,81)$ & 12 & $(171,50)$ & 11 & 1 & YES & YES & NO(2) & $1.80$ & $(4,3)$ & NO & 13016\\
$(279,83)$ & 13 & $(7,2)$ & 4 & 1 & YES & YES & YES & $2.00$ & $(2,4)$ & -- & 13017\\
$(281,116)$ & 12 & $(11,3)$ & 5 & 1 & YES & YES & YES & $2.14$ & $(2,4)$ & NO & 13018\\
$(281,116)$ & 12 & $(13,3)$ & 6 & 1 & YES & YES & YES & $2.14$ & $(2,4)$ & NO & 13019\\
$(282,109)$ & 12 & $(106,41)$ & 10 & 2 & YES & YES & NO(2) & $1.90$ & $(4,3)$ & NO & 13020\\
$(283,75)$ & 13 & $(7,2)$ & 4 & 1 & YES & YES & NO(2) & $2.00$ & $(2,4)$ & -- & 13021\\
$(283,88)$ & 13 & $(7,3)$ & 4 & 1 & YES & YES & NO(2) & $1.90$ & $(4,3)$ & -- & 13022\\
$(283,104)$ & 12 & $(8,3)$ & 4 & 1 & YES & YES & YES & $2.29$ & $(2,4)$ & -- & 13023\\
$(283,108)$ & 12 & $(11,4)$ & 5 & 1 & YES & YES & NO(2) & $1.88$ & $(4,3)$ & NO & 13024\\
$(283,76)$ & 12 & $(17,5)$ & 6 & 1 & YES & YES & NO(2) & $1.88$ & $(6,2)$ & NO & 13025\\
$(283,86)$ & 13 & $(89,27)$ & 10 & 1 & YES & YES & NO(2) & $2.09$ & $(6,2)$ & 13428 & 13026\\
$(283,108)$ & 12 & $(89,34)$ & 9 & 1 & YES & YES & NO(2) & $2.00$ & $(8,1)$ & 13467 & 13027\\
$(287,79)$ & 12 & $(2,1)$ & 1 & 1 & YES & YES & NO(2) & $1.90$ & $(12,-1)$ & -- & 13028\\
$(287,109)$ & 12 & $(2,1)$ & 1 & 1 & YES & YES & YES & $1.86$ & $(2,4)$ & -- & 13029\\
$(287,79)$ & 12 & $(3,1)$ & 2 & 1 & YES & YES & NO(2) & $1.90$ & $(8,1)$ & -- & 13030\\
$(287,79)$ & 12 & $(3,1)$ & 2 & 1 & YES & YES & NO(2) & $2.00$ & $(8,1)$ & NO & 13031\\
$(287,106)$ & 12 & $(7,3)$ & 4 & 7 & YES & YES & YES & $2.12$ & $(2,4)$ & -- & 13032\\
$(287,109)$ & 12 & $(7,2)$ & 4 & 7 & YES & YES & YES & $2.00$ & $(2,4)$ & -- & 13033\\
$(287,79)$ & 12 & $(18,5)$ & 6 & 1 & YES & YES & NO(2) & $1.90$ & $(8,1)$ & 13113 & 13034\\
$(287,79)$ & 12 & $(29,8)$ & 7 & 1 & YES & YES & NO(2) & $2.00$ & $(2,4)$ & 12983 & 13035\\
$(288,85)$ & 13 & $(2,1)$ & 1 & 2 & YES & YES & NO(2) & $1.75$ & $(4,3)$ & -- & 13036\\
$(288,119)$ & 12 & $(5,2)$ & 3 & 1 & YES & YES & NO(2) & $2.00$ & $(12,-1)$ & -- & 13037\\
$(288,121)$ & 12 & $(5,2)$ & 3 & 1 & YES & YES & NO(2) & $2.00$ & $(2,4)$ & -- & 13038\\
$(288,85)$ & 13 & $(105,31)$ & 10 & 3 & YES & YES & NO(2) & $1.89$ & $(2,4)$ & NO & 13039\\
$(288,121)$ & 12 & $(188,79)$ & 11 & 4 & YES & YES & NO(2) & $2.00$ & $(2,4)$ & NO & 13040\\
$(289,112)$ & 12 & $(5,2)$ & 3 & 1 & YES & YES & NO(2) & $1.90$ & $(4,3)$ & -- & 13041\\
$(289,112)$ & 12 & $(7,3)$ & 4 & 1 & YES & YES & NO(2) & $2.09$ & $(6,2)$ & NO & 13042\\
$(289,84)$ & 13 & $(44,13)$ & 8 & 1 & YES & YES & NO(2) & $1.89$ & $(4,3)$ & NO & 13043\\
$(289,84)$ & 13 & $(100,29)$ & 11 & 1 & YES & YES & YES & $2.00$ & $(4,3)$ & NO & 13044\\
$(290,81)$ & 12 & $(4,1)$ & 3 & 2 & YES & YES & YES & $1.86$ & $(2,4)$ & -- & 13045\\
$(292,111)$ & 12 & $(5,2)$ & 3 & 1 & YES & YES & YES & $2.00$ & $(4,3)$ & -- & 13046\\
$(292,121)$ & 12 & $(5,2)$ & 3 & 1 & YES & YES & YES & $1.88$ & $(2,4)$ & NO & 13047\\
$(292,111)$ & 12 & $(7,2)$ & 4 & 1 & YES & YES & NO(2) & $1.88$ & $(6,2)$ & NO & 13048\\
$(292,111)$ & 12 & $(7,2)$ & 4 & 1 & YES & YES & NO(2) & $1.88$ & $(6,2)$ & -- & 13049\\
$(292,111)$ & 12 & $(8,3)$ & 4 & 4 & YES & YES & YES & $2.29$ & $(2,4)$ & -- & 13050\\
$(292,121)$ & 12 & $(22,9)$ & 7 & 2 & YES & YES & NO(2) & $2.11$ & $(4,3)$ & 13438 & 13051\\
$(293,87)$ & 13 & $(266,79)$ & 12 & 1 & YES & YES & NO(2) & $2.00$ & $(2,4)$ & 13518 & 13052\\
$(294,79)$ & 13 & $(5,2)$ & 3 & 1 & YES & YES & NO(2) & $1.67$ & $(6,2)$ & -- & 13053\\
$(294,89)$ & 13 & $(12,5)$ & 5 & 6 & YES & YES & YES & $2.17$ & $(4,3)$ & NO & 13054\\
$(295,78)$ & 13 & $(5,2)$ & 3 & 5 & YES & YES & NO(2) & $1.89$ & $(2,4)$ & -- & 13055\\
$(295,78)$ & 13 & $(7,3)$ & 4 & 1 & YES & YES & NO(2) & $1.88$ & $(8,1)$ & NO & 13056\\
$(295,78)$ & 13 & $(7,3)$ & 4 & 1 & YES & YES & NO(2) & $1.88$ & $(8,1)$ & -- & 13057\\
$(295,112)$ & 12 & $(13,5)$ & 5 & 1 & YES & YES & YES & $2.00$ & $(2,4)$ & NO & 13058\\
$(295,108)$ & 12 & $(19,7)$ & 6 & 1 & YES & YES & NO(2) & $2.00$ & $(6,2)$ & 13129 & 13059\\
$(295,78)$ & 13 & $(49,13)$ & 9 & 1 & YES & YES & NO(2) & $2.00$ & $(2,4)$ & NO & 13060\\
$(297,113)$ & 13 & $(3,1)$ & 2 & 3 & YES & YES & NO(2) & $1.89$ & $(6,2)$ & NO & 13061\\
$(297,109)$ & 12 & $(5,2)$ & 3 & 1 & YES & YES & NO(2) & $1.78$ & $(2,4)$ & NO & 13062\\
$(297,109)$ & 12 & $(7,2)$ & 4 & 1 & YES & YES & NO(2) & $1.88$ & $(2,4)$ & -- & 13063\\
$(297,109)$ & 12 & $(8,3)$ & 4 & 1 & YES & YES & NO(2) & $1.89$ & $(2,4)$ & NO & 13064\\
$(297,109)$ & 12 & $(49,18)$ & 8 & 1 & YES & YES & NO(2) & $1.89$ & $(2,4)$ & 13131 & 13065\\
$(297,109)$ & 12 & $(128,47)$ & 10 & 1 & YES & YES & NO(2) & $2.00$ & $(2,4)$ & NO & 13066\\
$(299,116)$ & 12 & $(11,3)$ & 5 & 1 & YES & YES & YES & $2.14$ & $(2,4)$ & NO & 13067\\
$(301,89)$ & 12 & $(3,1)$ & 2 & 1 & YES & YES & NO(2) & $1.82$ & $(6,2)$ & -- & 13068\\
$(301,89)$ & 12 & $(5,2)$ & 3 & 1 & YES & YES & NO(2) & $2.00$ & $(10,0)$ & NO & 13069\\
$(301,89)$ & 12 & $(5,2)$ & 3 & 1 & YES & YES & NO(2) & $2.00$ & $(10,0)$ & -- & 13070\\
$(301,115)$ & 12 & $(5,2)$ & 3 & 1 & YES & YES & NO(2) & $2.00$ & $(4,3)$ & -- & 13071\\
$(301,115)$ & 12 & $(5,2)$ & 3 & 1 & YES & YES & NO(2) & $2.00$ & $(4,3)$ & NO & 13072\\
$(301,115)$ & 12 & $(7,3)$ & 4 & 7 & YES & YES & NO(2) & $2.09$ & $(2,4)$ & NO & 13073\\
$(301,115)$ & 12 & $(29,11)$ & 7 & 1 & YES & YES & NO(2) & $2.00$ & $(4,3)$ & NO & 13074\\
$(302,111)$ & 12 & $(5,2)$ & 3 & 1 & YES & YES & NO(2) & $1.83$ & $(6,2)$ & -- & 13075\\
$(302,117)$ & 12 & $(19,7)$ & 6 & 1 & YES & YES & YES & $2.14$ & $(2,4)$ & NO & 13076\\
$(302,117)$ & 12 & $(31,12)$ & 7 & 1 & YES & YES & NO(2) & $1.78$ & $(2,4)$ & NO & 13077\\
$(303,128)$ & 12 & $(2,1)$ & 1 & 1 & YES & YES & NO(2) & $2.00$ & $(2,4)$ & -- & 13078\\
$(303,128)$ & 12 & $(5,2)$ & 3 & 1 & YES & YES & NO(2) & $2.00$ & $(4,3)$ & -- & 13079\\
$(303,82)$ & 13 & $(11,4)$ & 5 & 1 & YES & YES & YES & $2.12$ & $(2,4)$ & -- & 13080\\
$(303,116)$ & 12 & $(81,31)$ & 9 & 3 & YES & YES & YES & $2.00$ & $(2,4)$ & NO & 13081\\
$(305,112)$ & 12 & $(3,1)$ & 2 & 1 & YES & YES & NO(2) & $2.00$ & $(2,4)$ & -- & 13082\\
$(305,128)$ & 12 & $(5,2)$ & 3 & 5 & YES & YES & NO(2) & $1.89$ & $(6,2)$ & -- & 13083\\
$(305,112)$ & 12 & $(7,3)$ & 4 & 1 & YES & YES & NO(2) & $2.00$ & $(4,3)$ & NO & 13084\\
$(305,112)$ & 12 & $(109,40)$ & 10 & 1 & YES & YES & NO(2) & $2.09$ & $(2,4)$ & NO & 13085\\
$(307,129)$ & 12 & $(2,1)$ & 1 & 1 & YES & YES & NO(2) & $2.00$ & $(2,4)$ & -- & 13086\\
$(307,129)$ & 12 & $(5,2)$ & 3 & 1 & YES & YES & NO(2) & $1.89$ & $(2,4)$ & NO & 13087\\
$(307,129)$ & 12 & $(5,2)$ & 3 & 1 & YES & YES & NO(2) & $2.00$ & $(4,3)$ & -- & 13088\\
$(307,129)$ & 12 & $(7,2)$ & 4 & 1 & YES & YES & NO(2) & $1.90$ & $(4,3)$ & NO & 13089\\
$(307,57)$ & 14 & $(8,3)$ & 4 & 1 & YES & YES & NO(2) & $2.00$ & $(8,1)$ & NO & 13090\\
$(311,119)$ & 12 & $(196,75)$ & 11 & 1 & YES & YES & NO(2) & $2.00$ & $(2,4)$ & NO & 13091\\
$(313,121)$ & 12 & $(5,2)$ & 3 & 1 & YES & YES & NO(2) & $1.80$ & $(4,3)$ & -- & 13092\\
$(313,119)$ & 12 & $(9,4)$ & 5 & 1 & YES & YES & NO(2) & $2.00$ & $(6,2)$ & NO & 13093\\
$(313,91)$ & 13 & $(17,5)$ & 6 & 1 & YES & YES & NO(2) & $1.89$ & $(6,2)$ & NO & 13094\\
$(313,91)$ & 13 & $(44,13)$ & 8 & 1 & YES & YES & YES & $2.12$ & $(2,4)$ & NO & 13095\\
$(313,91)$ & 13 & $(79,23)$ & 10 & 1 & YES & YES & NO(2) & $1.88$ & $(4,3)$ & 13410 & 13096\\
$(314,87)$ & 13 & $(7,3)$ & 4 & 1 & YES & YES & NO(2) & $2.00$ & $(4,3)$ & -- & 13097\\
$(315,113)$ & 13 & $(7,2)$ & 4 & 7 & YES & YES & NO(2) & $2.00$ & $(6,2)$ & -- & 13098\\
$(317,121)$ & 12 & $(2,1)$ & 1 & 1 & YES & YES & NO(2) & $1.75$ & $(4,3)$ & -- & 13099\\
$(317,84)$ & 13 & $(7,3)$ & 4 & 1 & YES & YES & NO(2) & $1.88$ & $(8,1)$ & NO & 13100\\
$(317,121)$ & 12 & $(7,2)$ & 4 & 1 & YES & YES & NO(2) & $1.86$ & $(4,3)$ & NO & 13101\\
$(317,121)$ & 12 & $(7,2)$ & 4 & 1 & YES & YES & NO(2) & $1.86$ & $(4,3)$ & -- & 13102\\
$(317,121)$ & 12 & $(8,3)$ & 4 & 1 & YES & YES & NO(3) & $2.00$ & $(2,4)$ & -- & 13103\\
$(317,121)$ & 12 & $(13,5)$ & 5 & 1 & YES & YES & NO(2) & $1.88$ & $(4,3)$ & NO & 13104\\
$(317,121)$ & 12 & $(97,37)$ & 10 & 1 & YES & YES & NO(2) & $2.00$ & $(2,4)$ & NO & 13105\\
$(319,86)$ & 13 & $(5,2)$ & 3 & 1 & YES & YES & NO(2) & $1.91$ & $(6,2)$ & -- & 13106\\
$(319,73)$ & 13 & $(7,2)$ & 4 & 1 & YES & YES & NO(2) & $1.89$ & $(2,4)$ & NO & 13107\\
$(319,118)$ & 13 & $(41,15)$ & 8 & 1 & YES & YES & NO(2) & $2.20$ & $(4,3)$ & NO & 13108\\
$(321,94)$ & 13 & $(7,3)$ & 4 & 1 & YES & YES & YES & $2.12$ & $(2,4)$ & -- & 13109\\
$(322,89)$ & 12 & $(3,1)$ & 2 & 1 & YES & YES & NO(2) & $1.80$ & $(8,1)$ & -- & 13110\\
$(322,89)$ & 12 & $(3,1)$ & 2 & 1 & YES & YES & NO(2) & $1.90$ & $(8,1)$ & NO & 13111\\
$(322,89)$ & 12 & $(7,2)$ & 4 & 7 & YES & YES & NO(2) & $1.90$ & $(8,1)$ & NO & 13112\\
$(322,89)$ & 12 & $(11,3)$ & 5 & 1 & YES & YES & NO(2) & $1.90$ & $(8,1)$ & 13034 & 13113\\
$(322,89)$ & 12 & $(18,5)$ & 6 & 2 & YES & YES & NO(2) & $2.00$ & $(8,1)$ & NO & 13114\\
$(323,94)$ & 13 & $(4,1)$ & 3 & 1 & YES & YES & NO(2) & $1.86$ & $(6,2)$ & NO & 13115\\
$(323,94)$ & 13 & $(4,1)$ & 3 & 1 & YES & YES & NO(2) & $1.86$ & $(6,2)$ & -- & 13116\\
$(323,94)$ & 13 & $(5,2)$ & 3 & 1 & YES & YES & NO(2) & $1.90$ & $(4,3)$ & -- & 13117\\
$(323,98)$ & 13 & $(10,3)$ & 5 & 1 & YES & YES & NO(2) & $2.00$ & $(6,2)$ & -- & 13118\\
$(323,94)$ & 13 & $(213,62)$ & 12 & 1 & YES & YES & NO(2) & $1.90$ & $(4,3)$ & NO & 13119\\
$(324,89)$ & 13 & $(7,3)$ & 4 & 1 & YES & YES & NO(2) & $2.00$ & $(4,3)$ & NO & 13120\\
$(327,137)$ & 13 & $(5,2)$ & 3 & 1 & YES & YES & NO(2) & $2.11$ & $(2,4)$ & -- & 13121\\
$(333,76)$ & 13 & $(9,4)$ & 5 & 9 & YES & YES & NO(2) & $2.00$ & $(6,2)$ & -- & 13122\\
$(333,97)$ & 14 & $(10,3)$ & 5 & 1 & YES & YES & YES & $2.29$ & $(2,4)$ & -- & 13123\\
$(333,76)$ & 13 & $(11,4)$ & 5 & 1 & YES & YES & YES & $2.14$ & $(2,4)$ & -- & 13124\\
$(333,76)$ & 13 & $(83,19)$ & 10 & 1 & YES & YES & NO(2) & $1.90$ & $(4,3)$ & NO & 13125\\
$(334,93)$ & 14 & $(2,1)$ & 1 & 2 & YES & YES & NO(2) & $1.89$ & $(2,4)$ & -- & 13126\\
$(335,128)$ & 12 & $(2,1)$ & 1 & 1 & YES & YES & YES & $2.00$ & $(2,4)$ & -- & 13127\\
$(335,128)$ & 12 & $(3,1)$ & 2 & 1 & YES & YES & NO(3) & $1.71$ & $(2,4)$ & -- & 13128\\
$(335,123)$ & 12 & $(11,4)$ & 5 & 1 & YES & YES & NO(2) & $2.00$ & $(6,2)$ & 13059 & 13129\\
$(335,139)$ & 13 & $(27,11)$ & 8 & 1 & YES & YES & NO(2) & $2.09$ & $(6,2)$ & NO & 13130\\
$(335,123)$ & 12 & $(30,11)$ & 7 & 5 & YES & YES & NO(2) & $1.89$ & $(2,4)$ & 13065 & 13131\\
$(335,128)$ & 12 & $(335,128)$ & 12 & 335 & YES & YES & NO(3) & $1.71$ & $(2,4)$ & NO & 13132\\
$(336,127)$ & 13 & $(4,1)$ & 3 & 4 & YES & YES & NO(2) & $2.00$ & $(6,2)$ & -- & 13133\\
$(336,89)$ & 14 & $(132,35)$ & 11 & 12 & YES & YES & NO(2) & $2.00$ & $(12,-1)$ & NO & 13134\\
$(337,128)$ & 12 & $(3,1)$ & 2 & 1 & YES & YES & NO(2) & $2.00$ & $(2,4)$ & NO & 13135\\
$(337,129)$ & 12 & $(3,1)$ & 2 & 1 & YES & YES & NO(2) & $1.88$ & $(8,1)$ & -- & 13136\\
$(337,128)$ & 12 & $(7,2)$ & 4 & 1 & YES & YES & YES & $2.00$ & $(2,4)$ & -- & 13137\\
$(337,147)$ & 13 & $(7,3)$ & 4 & 1 & YES & YES & NO(2) & $2.08$ & $(4,3)$ & NO & 13138\\
$(337,128)$ & 12 & $(18,7)$ & 6 & 1 & YES & YES & YES & $2.00$ & $(2,4)$ & NO & 13139\\
$(337,129)$ & 12 & $(29,11)$ & 7 & 1 & YES & YES & YES & $2.12$ & $(2,4)$ & 13555 & 13140\\
$(337,100)$ & 13 & $(91,27)$ & 10 & 1 & YES & YES & NO(2) & $1.89$ & $(2,4)$ & NO & 13141\\
$(337,128)$ & 12 & $(287,109)$ & 12 & 1 & YES & YES & NO(3) & $1.86$ & $(2,4)$ & NO & 13142\\
$(338,99)$ & 13 & $(2,1)$ & 1 & 2 & YES & YES & NO(2) & $1.92$ & $(4,3)$ & -- & 13143\\
$(338,131)$ & 12 & $(3,1)$ & 2 & 1 & YES & YES & NO(2) & $2.00$ & $(2,4)$ & -- & 13144\\
$(338,131)$ & 12 & $(8,3)$ & 4 & 2 & YES & YES & NO(2) & $2.00$ & $(10,0)$ & 13238 & 13145\\
$(338,99)$ & 13 & $(17,5)$ & 6 & 1 & YES & YES & NO(2) & $2.00$ & $(4,3)$ & NO & 13146\\
$(339,139)$ & 13 & $(3,1)$ & 2 & 3 & YES & YES & NO(2) & $2.08$ & $(4,3)$ & -- & 13147\\
$(339,130)$ & 13 & $(4,1)$ & 3 & 1 & YES & YES & NO(2) & $2.00$ & $(6,2)$ & -- & 13148\\
$(339,139)$ & 13 & $(5,2)$ & 3 & 1 & YES & YES & NO(2) & $2.08$ & $(4,3)$ & NO & 13149\\
$(339,139)$ & 13 & $(5,2)$ & 3 & 1 & YES & YES & NO(2) & $2.08$ & $(4,3)$ & -- & 13150\\
$(341,140)$ & 13 & $(2,1)$ & 1 & 1 & YES & YES & NO(2) & $2.08$ & $(4,3)$ & -- & 13151\\
$(341,104)$ & 13 & $(5,2)$ & 3 & 1 & YES & YES & NO(2) & $2.00$ & $(6,2)$ & -- & 13152\\
$(341,140)$ & 13 & $(7,3)$ & 4 & 1 & YES & YES & NO(2) & $2.08$ & $(4,3)$ & NO & 13153\\
$(343,131)$ & 12 & $(4,1)$ & 3 & 1 & YES & YES & YES & $1.86$ & $(2,4)$ & -- & 13154\\
$(343,144)$ & 12 & $(5,2)$ & 3 & 1 & YES & YES & NO(2) & $1.86$ & $(4,3)$ & -- & 13155\\
$(343,81)$ & 15 & $(7,2)$ & 4 & 7 & YES & YES & NO(2) & $2.00$ & $(6,2)$ & -- & 13156\\
$(343,131)$ & 12 & $(23,9)$ & 7 & 1 & YES & YES & NO(2) & $2.10$ & $(4,3)$ & NO & 13157\\
$(345,143)$ & 13 & $(7,2)$ & 4 & 1 & YES & YES & YES & $2.00$ & $(2,4)$ & -- & 13158\\
$(346,93)$ & 13 & $(6,1)$ & 5 & 2 & YES & YES & NO(2) & $1.89$ & $(10,0)$ & -- & 13159\\
$(346,93)$ & 13 & $(6,1)$ & 5 & 2 & YES & YES & NO(2) & $2.00$ & $(10,0)$ & NO & 13160\\
$(346,143)$ & 13 & $(6,1)$ & 5 & 2 & YES & YES & NO(2) & $1.90$ & $(8,1)$ & NO & 13161\\
$(346,143)$ & 13 & $(6,1)$ & 5 & 2 & YES & YES & NO(2) & $1.90$ & $(8,1)$ & -- & 13162\\
$(346,97)$ & 14 & $(10,3)$ & 5 & 2 & YES & YES & YES & $2.14$ & $(2,4)$ & -- & 13163\\
$(346,145)$ & 13 & $(136,57)$ & 11 & 2 & YES & YES & NO(2) & $2.09$ & $(6,2)$ & 13241 & 13164\\
$(347,101)$ & 13 & $(4,1)$ & 3 & 1 & YES & YES & NO(2) & $2.00$ & $(6,2)$ & NO & 13165\\
$(347,101)$ & 13 & $(4,1)$ & 3 & 1 & YES & YES & NO(2) & $2.00$ & $(6,2)$ & -- & 13166\\
$(347,93)$ & 13 & $(5,2)$ & 3 & 1 & YES & YES & NO(2) & $2.10$ & $(2,4)$ & -- & 13167\\
$(347,101)$ & 13 & $(5,2)$ & 3 & 1 & YES & YES & NO(2) & $1.88$ & $(4,3)$ & -- & 13168\\
$(347,93)$ & 13 & $(37,10)$ & 8 & 1 & YES & YES & NO(2) & $2.00$ & $(2,4)$ & NO & 13169\\
$(347,92)$ & 13 & $(181,48)$ & 12 & 1 & YES & YES & NO(2) & $1.90$ & $(4,3)$ & NO & 13170\\
$(347,101)$ & 13 & $(189,55)$ & 12 & 1 & YES & YES & NO(2) & $1.90$ & $(4,3)$ & NO & 13171\\
$(347,147)$ & 13 & $(347,147)$ & 13 & 347 & YES & YES & NO(2) & $2.09$ & $(2,4)$ & NO & 13172\\
$(348,125)$ & 13 & $(5,2)$ & 3 & 1 & YES & YES & NO(2) & $2.00$ & $(4,3)$ & -- & 13173\\
$(348,125)$ & 13 & $(36,13)$ & 8 & 12 & YES & YES & NO(2) & $2.00$ & $(4,3)$ & NO & 13174\\
$(349,135)$ & 13 & $(5,2)$ & 3 & 1 & YES & YES & NO(2) & $2.00$ & $(4,3)$ & -- & 13175\\
$(349,143)$ & 13 & $(7,2)$ & 4 & 1 & YES & YES & YES & $2.14$ & $(2,4)$ & -- & 13176\\
$(349,153)$ & 14 & $(203,89)$ & 12 & 1 & YES & YES & NO(2) & $2.00$ & $(6,2)$ & 13453 & 13177\\
$(350,97)$ & 14 & $(2,1)$ & 1 & 2 & YES & YES & NO(2) & $1.89$ & $(2,4)$ & -- & 13178\\
$(351,76)$ & 13 & $(10,3)$ & 5 & 1 & YES & YES & NO(2) & $1.88$ & $(6,2)$ & NO & 13179\\
$(353,99)$ & 14 & $(14,3)$ & 6 & 1 & YES & YES & YES & $2.25$ & $(2,4)$ & -- & 13180\\
$(355,99)$ & 13 & $(3,1)$ & 2 & 1 & YES & YES & NO(2) & $2.00$ & $(4,3)$ & NO & 13181\\
$(355,99)$ & 13 & $(3,1)$ & 2 & 1 & YES & YES & NO(2) & $2.00$ & $(4,3)$ & -- & 13182\\
$(355,136)$ & 13 & $(7,3)$ & 4 & 1 & YES & YES & YES & $2.17$ & $(4,3)$ & -- & 13183\\
$(355,99)$ & 13 & $(11,3)$ & 5 & 1 & YES & YES & NO(2) & $2.00$ & $(4,3)$ & NO & 13184\\
$(355,147)$ & 13 & $(22,9)$ & 7 & 1 & YES & YES & NO(2) & $2.00$ & $(4,3)$ & NO & 13185\\
$(356,105)$ & 13 & $(9,2)$ & 5 & 1 & YES & YES & NO(2) & $2.00$ & $(2,4)$ & -- & 13186\\
$(356,147)$ & 13 & $(13,3)$ & 6 & 1 & YES & YES & YES & $2.29$ & $(2,4)$ & -- & 13187\\
$(357,131)$ & 13 & $(3,1)$ & 2 & 3 & YES & YES & NO(2) & $1.67$ & $(6,2)$ & -- & 13188\\
$(357,131)$ & 13 & $(79,29)$ & 9 & 1 & YES & YES & NO(2) & $1.67$ & $(6,2)$ & NO & 13189\\
$(358,147)$ & 13 & $(11,3)$ & 5 & 1 & YES & YES & YES & $2.14$ & $(4,3)$ & -- & 13190\\
$(359,100)$ & 13 & $(2,1)$ & 1 & 1 & YES & YES & NO(2) & $1.91$ & $(2,4)$ & -- & 13191\\
$(359,140)$ & 13 & $(5,1)$ & 4 & 1 & YES & YES & YES & $2.00$ & $(4,3)$ & NO & 13192\\
$(359,140)$ & 13 & $(5,1)$ & 4 & 1 & YES & YES & YES & $2.00$ & $(4,3)$ & -- & 13193\\
$(359,105)$ & 13 & $(7,2)$ & 4 & 1 & YES & YES & NO(2) & $1.89$ & $(2,4)$ & NO & 13194\\
$(360,133)$ & 13 & $(5,1)$ & 4 & 5 & YES & YES & NO(2) & $1.90$ & $(4,3)$ & -- & 13195\\
$(360,133)$ & 13 & $(5,1)$ & 4 & 5 & YES & YES & NO(2) & $2.00$ & $(4,3)$ & NO & 13196\\
$(360,109)$ & 14 & $(195,59)$ & 12 & 15 & YES & YES & YES & $2.25$ & $(2,4)$ & NO & 13197\\
$(362,107)$ & 13 & $(5,2)$ & 3 & 1 & YES & YES & YES & $2.00$ & $(2,4)$ & -- & 13198\\
$(363,152)$ & 13 & $(3,1)$ & 2 & 3 & YES & YES & YES & $1.86$ & $(2,4)$ & -- & 13199\\
$(363,107)$ & 14 & $(9,2)$ & 5 & 3 & YES & YES & NO(2) & $2.00$ & $(6,2)$ & -- & 13200\\
$(363,134)$ & 13 & $(84,31)$ & 10 & 3 & YES & YES & NO(2) & $2.00$ & $(4,3)$ & NO & 13201\\
$(363,134)$ & 13 & $(149,55)$ & 11 & 1 & YES & YES & NO(2) & $2.00$ & $(10,0)$ & NO & 13202\\
$(363,149)$ & 13 & $(173,71)$ & 12 & 1 & YES & YES & NO(2) & $2.00$ & $(6,2)$ & NO & 13203\\
$(363,134)$ & 13 & $(363,134)$ & 13 & 363 & YES & YES & NO(2) & $2.00$ & $(6,2)$ & NO & 13204\\
$(364,135)$ & 13 & $(3,1)$ & 2 & 1 & YES & YES & NO(2) & $2.00$ & $(4,3)$ & NO & 13205\\
$(364,135)$ & 13 & $(3,1)$ & 2 & 1 & YES & YES & NO(2) & $2.00$ & $(4,3)$ & -- & 13206\\
$(364,151)$ & 13 & $(27,11)$ & 8 & 1 & YES & YES & YES & $2.00$ & $(4,3)$ & NO & 13207\\
$(365,159)$ & 13 & $(163,71)$ & 11 & 1 & YES & YES & NO(2) & $1.90$ & $(4,3)$ & 13339 & 13208\\
$(365,159)$ & 13 & $(365,159)$ & 13 & 365 & YES & YES & YES & $2.00$ & $(2,4)$ & NO & 13209\\
$(366,97)$ & 14 & $(3,1)$ & 2 & 3 & YES & YES & NO(2) & $2.00$ & $(2,4)$ & -- & 13210\\
$(367,101)$ & 13 & $(2,1)$ & 1 & 1 & YES & YES & NO(2) & $2.00$ & $(12,-1)$ & -- & 13211\\
$(367,112)$ & 13 & $(5,2)$ & 3 & 1 & YES & YES & NO(2) & $1.88$ & $(4,3)$ & -- & 13212\\
$(367,155)$ & 13 & $(71,30)$ & 9 & 1 & YES & YES & YES & $1.86$ & $(2,4)$ & 12866 & 13213\\
$(367,161)$ & 13 & $(73,32)$ & 10 & 1 & YES & YES & NO(2) & $2.00$ & $(6,2)$ & NO & 13214\\
$(369,110)$ & 14 & $(4,1)$ & 3 & 1 & YES & YES & NO(2) & $1.90$ & $(4,3)$ & -- & 13215\\
$(370,163)$ & 13 & $(5,2)$ & 3 & 5 & YES & YES & NO(2) & $1.89$ & $(6,2)$ & NO & 13216\\
$(371,152)$ & 13 & $(7,2)$ & 4 & 7 & YES & YES & NO(2) & $2.00$ & $(4,3)$ & NO & 13217\\
$(371,164)$ & 14 & $(34,15)$ & 8 & 1 & YES & YES & NO(2) & $2.10$ & $(12,-1)$ & NO & 13218\\
$(373,104)$ & 13 & $(2,1)$ & 1 & 1 & YES & YES & NO(2) & $2.00$ & $(2,4)$ & -- & 13219\\
$(373,85)$ & 14 & $(3,1)$ & 2 & 1 & YES & YES & NO(2) & $1.78$ & $(6,2)$ & -- & 13220\\
$(373,138)$ & 13 & $(4,1)$ & 3 & 1 & YES & YES & NO(2) & $2.00$ & $(4,3)$ & -- & 13221\\
$(373,154)$ & 13 & $(4,1)$ & 3 & 1 & YES & YES & NO(2) & $1.86$ & $(4,3)$ & -- & 13222\\
$(373,104)$ & 13 & $(11,3)$ & 5 & 1 & YES & YES & NO(2) & $2.00$ & $(6,2)$ & NO & 13223\\
$(373,158)$ & 13 & $(229,97)$ & 12 & 1 & YES & YES & NO(2) & $2.00$ & $(10,0)$ & NO & 13224\\
$(373,154)$ & 13 & $(264,109)$ & 12 & 1 & YES & YES & NO(2) & $1.86$ & $(4,3)$ & NO & 13225\\
$(374,155)$ & 13 & $(3,1)$ & 2 & 1 & YES & YES & NO(2) & $2.10$ & $(8,1)$ & -- & 13226\\
$(374,155)$ & 13 & $(3,1)$ & 2 & 1 & YES & YES & YES & $2.00$ & $(4,3)$ & NO & 13227\\
$(374,155)$ & 13 & $(46,19)$ & 8 & 2 & YES & YES & NO(2) & $2.10$ & $(4,3)$ & NO & 13228\\
$(376,105)$ & 13 & $(3,1)$ & 2 & 1 & YES & YES & NO(2) & $1.91$ & $(6,2)$ & NO & 13229\\
$(376,105)$ & 13 & $(3,1)$ & 2 & 1 & YES & YES & NO(2) & $1.91$ & $(6,2)$ & -- & 13230\\
$(376,165)$ & 13 & $(7,3)$ & 4 & 1 & YES & YES & YES & $2.00$ & $(4,3)$ & -- & 13231\\
$(376,139)$ & 13 & $(12,5)$ & 5 & 4 & YES & YES & YES & $2.14$ & $(4,3)$ & NO & 13232\\
$(376,111)$ & 13 & $(13,4)$ & 6 & 1 & YES & YES & NO(3) & $1.67$ & $(4,3)$ & NO & 13233\\
$(377,138)$ & 13 & $(3,1)$ & 2 & 1 & YES & YES & NO(2) & $2.00$ & $(2,4)$ & -- & 13234\\
$(377,144)$ & 12 & $(3,1)$ & 2 & 1 & YES & YES & NO(2) & $1.91$ & $(6,2)$ & -- & 13235\\
$(377,85)$ & 14 & $(5,2)$ & 3 & 1 & YES & YES & NO(2) & $1.86$ & $(4,3)$ & NO & 13236\\
$(377,85)$ & 14 & $(5,2)$ & 3 & 1 & YES & YES & NO(2) & $2.00$ & $(10,0)$ & -- & 13237\\
$(377,144)$ & 12 & $(5,2)$ & 3 & 1 & YES & YES & NO(2) & $2.00$ & $(10,0)$ & 13145 & 13238\\
$(377,147)$ & 13 & $(5,2)$ & 3 & 1 & YES & YES & YES & $2.14$ & $(2,4)$ & -- & 13239\\
$(377,144)$ & 12 & $(10,3)$ & 5 & 1 & YES & YES & YES & $2.17$ & $(2,4)$ & -- & 13240\\
$(377,158)$ & 13 & $(105,44)$ & 10 & 1 & YES & YES & NO(2) & $2.09$ & $(6,2)$ & 13164 & 13241\\
$(377,138)$ & 13 & $(153,56)$ & 11 & 1 & YES & YES & NO(2) & $2.00$ & $(4,3)$ & 13331 & 13242\\
$(378,139)$ & 13 & $(87,32)$ & 10 & 3 & YES & YES & NO(2) & $2.00$ & $(6,2)$ & NO & 13243\\
$(378,139)$ & 13 & $(223,82)$ & 12 & 1 & YES & YES & NO(2) & $1.83$ & $(6,2)$ & NO & 13244\\
$(379,140)$ & 13 & $(2,1)$ & 1 & 1 & YES & YES & NO(2) & $2.00$ & $(4,3)$ & -- & 13245\\
$(379,87)$ & 16 & $(9,2)$ & 5 & 1 & YES & YES & NO(2) & $2.00$ & $(8,1)$ & -- & 13246\\
$(379,140)$ & 13 & $(19,7)$ & 6 & 1 & YES & YES & NO(2) & $2.00$ & $(4,3)$ & NO & 13247\\
$(383,106)$ & 13 & $(11,3)$ & 5 & 1 & YES & YES & NO(2) & $2.00$ & $(4,3)$ & NO & 13248\\
$(383,140)$ & 13 & $(13,5)$ & 5 & 1 & YES & YES & NO(2) & $2.00$ & $(8,1)$ & NO & 13249\\
$(383,112)$ & 13 & $(27,8)$ & 7 & 1 & YES & YES & NO(3) & $1.86$ & $(2,4)$ & NO & 13250\\
$(387,113)$ & 14 & $(4,1)$ & 3 & 1 & YES & YES & NO(2) & $2.00$ & $(4,3)$ & -- & 13251\\
$(387,113)$ & 14 & $(250,73)$ & 13 & 1 & YES & YES & NO(2) & $2.00$ & $(4,3)$ & NO & 13252\\
$(388,103)$ & 14 & $(10,3)$ & 5 & 2 & YES & YES & YES & $2.14$ & $(2,4)$ & -- & 13253\\
$(388,85)$ & 14 & $(11,4)$ & 5 & 1 & YES & YES & NO(2) & $2.00$ & $(6,2)$ & -- & 13254\\
$(388,161)$ & 13 & $(29,12)$ & 7 & 1 & YES & YES & NO(2) & $2.09$ & $(6,2)$ & 13440 & 13255\\
$(388,85)$ & 14 & $(196,43)$ & 13 & 4 & YES & YES & NO(2) & $2.00$ & $(6,2)$ & NO & 13256\\
$(389,163)$ & 13 & $(2,1)$ & 1 & 1 & YES & YES & NO(2) & $2.00$ & $(6,2)$ & -- & 13257\\
$(389,88)$ & 14 & $(5,2)$ & 3 & 1 & YES & YES & NO(2) & $1.86$ & $(4,3)$ & -- & 13258\\
$(389,92)$ & 14 & $(8,3)$ & 4 & 1 & YES & YES & NO(2) & $2.00$ & $(4,3)$ & NO & 13259\\
$(389,92)$ & 14 & $(8,3)$ & 4 & 1 & YES & YES & NO(2) & $2.00$ & $(4,3)$ & -- & 13260\\
$(389,115)$ & 13 & $(27,8)$ & 7 & 1 & YES & YES & NO(2) & $1.91$ & $(6,2)$ & NO & 13261\\
$(389,89)$ & 14 & $(57,13)$ & 9 & 1 & YES & YES & NO(2) & $1.90$ & $(4,3)$ & NO & 13262\\
$(389,88)$ & 14 & $(75,17)$ & 10 & 1 & YES & YES & NO(2) & $1.86$ & $(4,3)$ & NO & 13263\\
$(390,107)$ & 14 & $(10,3)$ & 5 & 10 & YES & YES & NO(2) & $2.10$ & $(4,3)$ & NO & 13264\\
$(391,175)$ & 14 & $(2,1)$ & 1 & 1 & YES & YES & NO(2) & $1.89$ & $(10,0)$ & -- & 13265\\
$(391,151)$ & 13 & $(9,2)$ & 5 & 1 & YES & YES & YES & $2.25$ & $(2,4)$ & NO & 13266\\
$(391,151)$ & 13 & $(9,2)$ & 5 & 1 & YES & YES & YES & $2.25$ & $(2,4)$ & -- & 13267\\
$(391,145)$ & 13 & $(97,36)$ & 10 & 1 & YES & YES & NO(2) & $2.00$ & $(4,3)$ & NO & 13268\\
$(392,85)$ & 14 & $(7,3)$ & 4 & 7 & YES & YES & NO(2) & $2.12$ & $(2,4)$ & NO & 13269\\
$(393,116)$ & 13 & $(2,1)$ & 1 & 1 & YES & YES & NO(2) & $1.80$ & $(4,3)$ & -- & 13270\\
$(393,103)$ & 14 & $(3,1)$ & 2 & 3 & YES & YES & NO(2) & $2.09$ & $(2,4)$ & -- & 13271\\
$(393,73)$ & 15 & $(5,2)$ & 3 & 1 & YES & YES & NO(2) & $1.90$ & $(4,3)$ & -- & 13272\\
$(393,152)$ & 13 & $(11,4)$ & 5 & 1 & YES & YES & YES & $2.12$ & $(2,4)$ & NO & 13273\\
$(393,116)$ & 13 & $(29,8)$ & 7 & 1 & YES & YES & YES & $2.14$ & $(4,3)$ & NO & 13274\\
$(394,165)$ & 13 & $(3,1)$ & 2 & 1 & YES & YES & NO(2) & $2.00$ & $(4,3)$ & -- & 13275\\
$(394,167)$ & 13 & $(5,2)$ & 3 & 1 & YES & YES & NO(2) & $2.00$ & $(4,3)$ & -- & 13276\\
$(394,165)$ & 13 & $(10,3)$ & 5 & 2 & YES & YES & YES & $2.33$ & $(2,4)$ & NO & 13277\\
$(397,163)$ & 13 & $(4,1)$ & 3 & 1 & YES & YES & NO(2) & $2.00$ & $(4,3)$ & -- & 13278\\
$(397,93)$ & 14 & $(7,3)$ & 4 & 1 & YES & YES & NO(2) & $2.12$ & $(2,4)$ & -- & 13279\\
$(397,163)$ & 13 & $(7,3)$ & 4 & 1 & YES & YES & YES & $2.14$ & $(4,3)$ & -- & 13280\\
$(397,154)$ & 13 & $(13,5)$ & 5 & 1 & YES & YES & NO(2) & $2.00$ & $(4,3)$ & NO & 13281\\
$(397,75)$ & 15 & $(65,12)$ & 10 & 1 & YES & YES & YES & $2.29$ & $(2,4)$ & NO & 13282\\
$(397,110)$ & 14 & $(65,18)$ & 9 & 1 & YES & YES & NO(2) & $1.89$ & $(2,4)$ & NO & 13283\\
$(397,151)$ & 13 & $(163,62)$ & 11 & 1 & YES & YES & NO(2) & $2.00$ & $(6,2)$ & NO & 13284\\
$(397,163)$ & 13 & $(246,101)$ & 12 & 1 & YES & YES & NO(2) & $2.12$ & $(4,3)$ & NO & 13285\\
$(397,163)$ & 13 & $(358,147)$ & 13 & 1 & YES & YES & YES & $2.14$ & $(4,3)$ & NO & 13286\\
$(397,151)$ & 13 & $(397,151)$ & 13 & 397 & YES & YES & NO(2) & $2.00$ & $(6,2)$ & NO & 13287\\
$(399,164)$ & 14 & $(7,2)$ & 4 & 7 & YES & YES & NO(2) & $2.00$ & $(6,2)$ & -- & 13288\\
$(400,147)$ & 13 & $(5,2)$ & 3 & 5 & YES & YES & NO(2) & $2.09$ & $(6,2)$ & NO & 13289\\
$(400,147)$ & 13 & $(8,3)$ & 4 & 8 & YES & YES & NO(2) & $2.00$ & $(6,2)$ & -- & 13290\\
$(400,147)$ & 13 & $(106,39)$ & 11 & 2 & YES & YES & NO(2) & $2.00$ & $(6,2)$ & NO & 13291\\
$(401,155)$ & 13 & $(5,2)$ & 3 & 1 & YES & YES & YES & $2.14$ & $(2,4)$ & -- & 13292\\
$(401,111)$ & 13 & $(11,3)$ & 5 & 1 & YES & YES & NO(2) & $2.00$ & $(6,2)$ & NO & 13293\\
$(402,157)$ & 13 & $(3,1)$ & 2 & 3 & YES & YES & NO(2) & $2.00$ & $(4,3)$ & -- & 13294\\
$(403,111)$ & 13 & $(2,1)$ & 1 & 1 & YES & YES & NO(2) & $2.00$ & $(10,0)$ & NO & 13295\\
$(403,92)$ & 14 & $(9,4)$ & 5 & 1 & YES & YES & NO(2) & $2.00$ & $(6,2)$ & -- & 13296\\
$(404,91)$ & 14 & $(5,2)$ & 3 & 1 & YES & YES & NO(2) & $2.00$ & $(10,0)$ & -- & 13297\\
$(404,91)$ & 14 & $(53,12)$ & 9 & 1 & YES & YES & NO(2) & $2.00$ & $(10,0)$ & NO & 13298\\
$(405,178)$ & 13 & $(3,1)$ & 2 & 3 & YES & YES & NO(2) & $2.00$ & $(4,3)$ & -- & 13299\\
$(407,168)$ & 13 & $(2,1)$ & 1 & 1 & YES & YES & NO(2) & $1.89$ & $(10,0)$ & -- & 13300\\
$(407,168)$ & 13 & $(4,1)$ & 3 & 1 & YES & YES & NO(2) & $1.90$ & $(4,3)$ & -- & 13301\\
$(407,168)$ & 13 & $(46,19)$ & 8 & 1 & YES & YES & NO(2) & $2.00$ & $(4,3)$ & 12918 & 13302\\
$(407,168)$ & 13 & $(281,116)$ & 12 & 1 & YES & YES & YES & $2.14$ & $(2,4)$ & NO & 13303\\
$(409,169)$ & 13 & $(5,2)$ & 3 & 1 & YES & YES & NO(2) & $2.00$ & $(4,3)$ & -- & 13304\\
$(409,121)$ & 13 & $(61,18)$ & 9 & 1 & YES & YES & YES & $2.00$ & $(2,4)$ & NO & 13305\\
$(410,173)$ & 13 & $(4,1)$ & 3 & 2 & YES & YES & NO(2) & $1.88$ & $(4,3)$ & -- & 13306\\
$(413,157)$ & 13 & $(2,1)$ & 1 & 1 & YES & YES & NO(2) & $2.00$ & $(4,3)$ & -- & 13307\\
$(413,157)$ & 13 & $(3,1)$ & 2 & 1 & YES & YES & NO(2) & $1.71$ & $(8,1)$ & -- & 13308\\
$(413,121)$ & 13 & $(5,1)$ & 4 & 1 & YES & YES & NO(2) & $2.00$ & $(4,3)$ & -- & 13309\\
$(413,89)$ & 14 & $(7,2)$ & 4 & 7 & YES & YES & NO(2) & $1.90$ & $(4,3)$ & -- & 13310\\
$(413,121)$ & 13 & $(7,3)$ & 4 & 7 & YES & YES & NO(2) & $2.11$ & $(2,4)$ & -- & 13311\\
$(413,158)$ & 13 & $(7,2)$ & 4 & 7 & YES & YES & YES & $2.17$ & $(2,4)$ & -- & 13312\\
$(413,157)$ & 13 & $(13,5)$ & 5 & 1 & YES & YES & NO(2) & $2.00$ & $(4,3)$ & NO & 13313\\
$(413,157)$ & 13 & $(21,8)$ & 6 & 7 & YES & YES & NO(2) & $2.00$ & $(4,3)$ & NO & 13314\\
$(413,171)$ & 14 & $(22,9)$ & 7 & 1 & YES & YES & NO(2) & $2.00$ & $(4,3)$ & NO & 13315\\
$(413,89)$ & 14 & $(60,13)$ & 9 & 1 & YES & YES & NO(2) & $1.90$ & $(4,3)$ & NO & 13316\\
$(413,160)$ & 13 & $(271,105)$ & 12 & 1 & YES & YES & NO(3) & $2.00$ & $(2,4)$ & 13748 & 13317\\
$(413,157)$ & 13 & $(292,111)$ & 12 & 1 & YES & YES & YES & $2.00$ & $(4,3)$ & NO & 13318\\
$(415,152)$ & 13 & $(7,2)$ & 4 & 1 & YES & YES & YES & $1.83$ & $(6,2)$ & NO & 13319\\
$(415,152)$ & 13 & $(7,2)$ & 4 & 1 & YES & YES & YES & $1.83$ & $(6,2)$ & -- & 13320\\
$(415,152)$ & 13 & $(273,100)$ & 12 & 1 & YES & YES & YES & $1.83$ & $(6,2)$ & NO & 13321\\
$(416,123)$ & 13 & $(5,2)$ & 3 & 1 & YES & YES & NO(2) & $2.11$ & $(2,4)$ & -- & 13322\\
$(416,123)$ & 13 & $(24,7)$ & 7 & 8 & YES & YES & YES & $2.00$ & $(2,4)$ & NO & 13323\\
$(417,175)$ & 13 & $(4,1)$ & 3 & 1 & YES & YES & NO(2) & $1.90$ & $(4,3)$ & -- & 13324\\
$(417,176)$ & 13 & $(4,1)$ & 3 & 1 & YES & YES & NO(2) & $2.00$ & $(4,3)$ & -- & 13325\\
$(417,176)$ & 13 & $(7,2)$ & 4 & 1 & YES & YES & YES & $2.00$ & $(2,4)$ & -- & 13326\\
$(417,113)$ & 14 & $(8,3)$ & 4 & 1 & YES & YES & NO(2) & $2.10$ & $(4,3)$ & -- & 13327\\
$(418,163)$ & 13 & $(3,1)$ & 2 & 1 & YES & YES & NO(2) & $2.08$ & $(4,3)$ & NO & 13328\\
$(418,163)$ & 13 & $(3,1)$ & 2 & 1 & YES & YES & NO(2) & $2.08$ & $(4,3)$ & -- & 13329\\
$(418,153)$ & 13 & $(8,3)$ & 4 & 2 & YES & YES & NO(2) & $2.00$ & $(4,3)$ & NO & 13330\\
$(418,153)$ & 13 & $(112,41)$ & 10 & 2 & YES & YES & NO(2) & $2.00$ & $(4,3)$ & 13242 & 13331\\
$(419,116)$ & 13 & $(7,3)$ & 4 & 1 & YES & YES & YES & $2.14$ & $(2,4)$ & NO & 13332\\
$(419,173)$ & 13 & $(9,2)$ & 5 & 1 & YES & YES & YES & $2.00$ & $(2,4)$ & -- & 13333\\
$(419,177)$ & 13 & $(303,128)$ & 12 & 1 & YES & YES & NO(2) & $1.90$ & $(4,3)$ & NO & 13334\\
$(422,173)$ & 13 & $(83,34)$ & 10 & 1 & YES & YES & NO(2) & $2.00$ & $(4,3)$ & NO & 13335\\
$(422,161)$ & 13 & $(422,161)$ & 13 & 422 & YES & YES & NO(2) & $2.00$ & $(4,3)$ & NO & 13336\\
$(427,186)$ & 13 & $(2,1)$ & 1 & 1 & YES & YES & NO(2) & $1.90$ & $(4,3)$ & -- & 13337\\
$(427,186)$ & 13 & $(3,1)$ & 2 & 1 & YES & YES & NO(2) & $1.90$ & $(4,3)$ & -- & 13338\\
$(427,186)$ & 13 & $(101,44)$ & 10 & 1 & YES & YES & NO(2) & $1.90$ & $(4,3)$ & 13208 & 13339\\
$(428,113)$ & 14 & $(5,2)$ & 3 & 1 & YES & YES & NO(2) & $2.10$ & $(4,3)$ & NO & 13340\\
$(428,113)$ & 14 & $(5,2)$ & 3 & 1 & YES & YES & NO(2) & $2.00$ & $(6,2)$ & -- & 13341\\
$(428,113)$ & 14 & $(49,13)$ & 9 & 1 & YES & YES & NO(2) & $2.00$ & $(6,2)$ & NO & 13342\\
$(429,127)$ & 14 & $(5,2)$ & 3 & 1 & YES & YES & NO(2) & $1.89$ & $(6,2)$ & -- & 13343\\
$(430,159)$ & 13 & $(2,1)$ & 1 & 2 & YES & YES & NO(2) & $1.86$ & $(6,2)$ & NO & 13344\\
$(430,119)$ & 13 & $(4,1)$ & 3 & 2 & YES & YES & NO(2) & $1.86$ & $(6,2)$ & -- & 13345\\
$(431,176)$ & 14 & $(7,3)$ & 4 & 1 & YES & YES & NO(2) & $2.11$ & $(6,2)$ & NO & 13346\\
$(432,181)$ & 13 & $(5,2)$ & 3 & 1 & YES & YES & NO(2) & $1.86$ & $(4,3)$ & NO & 13347\\
$(432,179)$ & 13 & $(13,3)$ & 6 & 1 & YES & YES & YES & $2.17$ & $(2,4)$ & NO & 13348\\
$(432,131)$ & 14 & $(79,24)$ & 10 & 1 & YES & YES & NO(2) & $2.00$ & $(4,3)$ & NO & 13349\\
$(433,126)$ & 14 & $(3,1)$ & 2 & 1 & YES & YES & NO(2) & $2.09$ & $(6,2)$ & NO & 13350\\
$(433,159)$ & 13 & $(5,2)$ & 3 & 1 & YES & YES & NO(2) & $2.00$ & $(4,3)$ & -- & 13351\\
$(433,168)$ & 13 & $(49,19)$ & 8 & 1 & YES & YES & NO(2) & $2.00$ & $(12,-1)$ & 12949 & 13352\\
$(434,115)$ & 14 & $(2,1)$ & 1 & 2 & YES & YES & YES & $1.86$ & $(2,4)$ & -- & 13353\\
$(434,115)$ & 14 & $(2,1)$ & 1 & 2 & YES & YES & YES & $2.00$ & $(2,4)$ & NO & 13354\\
$(434,165)$ & 13 & $(2,1)$ & 1 & 2 & YES & YES & NO(2) & $2.00$ & $(6,2)$ & -- & 13355\\
$(434,165)$ & 13 & $(3,1)$ & 2 & 1 & YES & YES & NO(2) & $1.88$ & $(6,2)$ & -- & 13356\\
$(434,121)$ & 13 & $(5,2)$ & 3 & 1 & YES & YES & YES & $1.86$ & $(2,4)$ & -- & 13357\\
$(436,129)$ & 13 & $(44,13)$ & 8 & 4 & YES & YES & NO(2) & $1.75$ & $(2,4)$ & NO & 13358\\
$(437,181)$ & 13 & $(2,1)$ & 1 & 1 & YES & YES & NO(2) & $2.09$ & $(2,4)$ & NO & 13359\\
$(437,183)$ & 13 & $(4,1)$ & 3 & 1 & YES & YES & NO(2) & $1.86$ & $(4,3)$ & NO & 13360\\
$(437,183)$ & 13 & $(4,1)$ & 3 & 1 & YES & YES & NO(2) & $1.86$ & $(4,3)$ & -- & 13361\\
$(437,167)$ & 13 & $(5,1)$ & 4 & 1 & YES & YES & NO(2) & $2.00$ & $(6,2)$ & NO & 13362\\
$(437,167)$ & 13 & $(5,1)$ & 4 & 1 & YES & YES & NO(2) & $2.00$ & $(6,2)$ & -- & 13363\\
$(437,169)$ & 13 & $(8,3)$ & 4 & 1 & YES & YES & NO(2) & $2.00$ & $(2,4)$ & NO & 13364\\
$(437,183)$ & 13 & $(19,8)$ & 6 & 19 & YES & YES & NO(2) & $2.11$ & $(2,4)$ & NO & 13365\\
$(437,183)$ & 13 & $(31,13)$ & 7 & 1 & YES & YES & NO(2) & $1.86$ & $(4,3)$ & NO & 13366\\
$(437,166)$ & 13 & $(37,14)$ & 8 & 1 & YES & YES & NO(2) & $2.10$ & $(4,3)$ & NO & 13367\\
$(437,183)$ & 13 & $(74,31)$ & 9 & 1 & YES & YES & NO(2) & $2.11$ & $(2,4)$ & 13488 & 13368\\
$(437,167)$ & 13 & $(157,60)$ & 11 & 1 & YES & YES & NO(2) & $2.11$ & $(6,2)$ & NO & 13369\\
$(437,183)$ & 13 & $(277,116)$ & 12 & 1 & YES & YES & NO(2) & $2.11$ & $(2,4)$ & NO & 13370\\
$(437,169)$ & 13 & $(287,111)$ & 12 & 1 & YES & YES & YES & $2.29$ & $(2,4)$ & NO & 13371\\
$(437,167)$ & 13 & $(403,154)$ & 13 & 1 & YES & YES & YES & $2.00$ & $(2,4)$ & NO & 13372\\
$(437,160)$ & 13 & $(437,160)$ & 13 & 437 & YES & YES & NO(2) & $1.88$ & $(8,1)$ & NO & 13373\\
$(438,181)$ & 13 & $(2,1)$ & 1 & 2 & YES & YES & NO(2) & $2.00$ & $(4,3)$ & -- & 13374\\
$(438,181)$ & 13 & $(19,8)$ & 6 & 1 & YES & YES & YES & $2.12$ & $(2,4)$ & NO & 13375\\
$(438,181)$ & 13 & $(80,33)$ & 10 & 2 & YES & YES & YES & $2.14$ & $(4,3)$ & NO & 13376\\
$(439,93)$ & 14 & $(5,2)$ & 3 & 1 & YES & YES & NO(2) & $2.00$ & $(4,3)$ & -- & 13377\\
$(441,101)$ & 14 & $(5,2)$ & 3 & 1 & YES & YES & NO(2) & $1.88$ & $(6,2)$ & -- & 13378\\
$(441,172)$ & 14 & $(13,5)$ & 5 & 1 & YES & YES & NO(2) & $2.12$ & $(2,4)$ & NO & 13379\\
$(441,101)$ & 14 & $(57,13)$ & 9 & 3 & YES & YES & NO(2) & $2.00$ & $(6,2)$ & NO & 13380\\
$(445,172)$ & 13 & $(3,1)$ & 2 & 1 & YES & YES & YES & $2.00$ & $(2,4)$ & -- & 13381\\
$(445,123)$ & 13 & $(5,2)$ & 3 & 5 & YES & YES & NO(3) & $1.86$ & $(2,4)$ & -- & 13382\\
$(445,123)$ & 13 & $(15,4)$ & 6 & 5 & YES & YES & YES & $2.00$ & $(2,4)$ & NO & 13383\\
$(445,123)$ & 13 & $(170,47)$ & 11 & 5 & YES & YES & YES & $2.00$ & $(2,4)$ & NO & 13384\\
$(445,163)$ & 13 & $(273,100)$ & 12 & 1 & YES & YES & NO(2) & $2.00$ & $(6,2)$ & NO & 13385\\
$(445,172)$ & 13 & $(445,172)$ & 13 & 445 & YES & YES & YES & $2.00$ & $(2,4)$ & NO & 13386\\
$(446,165)$ & 13 & $(5,1)$ & 4 & 1 & YES & YES & NO(2) & $1.71$ & $(6,2)$ & NO & 13387\\
$(446,165)$ & 13 & $(5,1)$ & 4 & 1 & YES & YES & NO(2) & $1.71$ & $(6,2)$ & -- & 13388\\
$(446,173)$ & 13 & $(44,17)$ & 8 & 2 & YES & YES & YES & $2.17$ & $(2,4)$ & NO & 13389\\
$(448,171)$ & 13 & $(2,1)$ & 1 & 2 & YES & YES & NO(2) & $2.00$ & $(4,3)$ & -- & 13390\\
$(448,171)$ & 13 & $(13,5)$ & 5 & 1 & YES & YES & NO(2) & $2.11$ & $(2,4)$ & NO & 13391\\
$(448,171)$ & 13 & $(21,8)$ & 6 & 7 & YES & YES & NO(2) & $2.00$ & $(4,3)$ & NO & 13392\\
$(448,137)$ & 14 & $(56,17)$ & 9 & 56 & YES & YES & NO(2) & $1.91$ & $(6,2)$ & NO & 13393\\
$(448,171)$ & 13 & $(131,50)$ & 10 & 1 & YES & YES & NO(2) & $2.00$ & $(2,4)$ & NO & 13394\\
$(448,171)$ & 13 & $(186,71)$ & 11 & 2 & YES & YES & YES & $2.00$ & $(2,4)$ & 13498 & 13395\\
$(448,171)$ & 13 & $(448,171)$ & 13 & 448 & YES & YES & YES & $2.00$ & $(2,4)$ & NO & 13396\\
$(451,134)$ & 14 & $(2,1)$ & 1 & 1 & YES & YES & NO(2) & $1.90$ & $(4,3)$ & -- & 13397\\
$(451,134)$ & 14 & $(27,8)$ & 7 & 1 & YES & YES & NO(2) & $1.90$ & $(4,3)$ & 12805 & 13398\\
$(451,122)$ & 14 & $(85,23)$ & 10 & 1 & YES & YES & NO(2) & $2.12$ & $(4,3)$ & NO & 13399\\
$(453,173)$ & 13 & $(5,1)$ & 4 & 1 & YES & YES & NO(2) & $1.86$ & $(4,3)$ & -- & 13400\\
$(453,161)$ & 14 & $(14,5)$ & 6 & 1 & YES & YES & NO(2) & $2.00$ & $(10,0)$ & NO & 13401\\
$(453,173)$ & 13 & $(199,76)$ & 11 & 1 & YES & YES & NO(2) & $1.86$ & $(4,3)$ & NO & 13402\\
$(455,188)$ & 13 & $(2,1)$ & 1 & 1 & YES & YES & NO(2) & $1.89$ & $(6,2)$ & -- & 13403\\
$(455,188)$ & 13 & $(3,1)$ & 2 & 1 & YES & YES & NO(2) & $1.89$ & $(6,2)$ & NO & 13404\\
$(455,138)$ & 14 & $(79,24)$ & 10 & 1 & YES & YES & NO(2) & $1.91$ & $(6,2)$ & NO & 13405\\
$(456,191)$ & 13 & $(4,1)$ & 3 & 4 & YES & YES & NO(2) & $2.09$ & $(2,4)$ & -- & 13406\\
$(456,191)$ & 13 & $(456,191)$ & 13 & 456 & YES & YES & YES & $1.86$ & $(2,4)$ & NO & 13407\\
$(457,133)$ & 14 & $(3,1)$ & 2 & 1 & YES & YES & NO(2) & $2.00$ & $(4,3)$ & -- & 13408\\
$(457,133)$ & 14 & $(5,1)$ & 4 & 1 & YES & YES & NO(2) & $1.90$ & $(4,3)$ & NO & 13409\\
$(457,133)$ & 14 & $(31,9)$ & 8 & 1 & YES & YES & NO(2) & $1.88$ & $(4,3)$ & 13096 & 13410\\
$(457,192)$ & 13 & $(69,29)$ & 9 & 1 & YES & YES & NO(2) & $1.90$ & $(4,3)$ & 13479 & 13411\\
$(457,133)$ & 14 & $(79,23)$ & 10 & 1 & YES & YES & NO(2) & $2.00$ & $(2,4)$ & NO & 13412\\
$(457,133)$ & 14 & $(323,94)$ & 13 & 1 & YES & YES & NO(2) & $2.00$ & $(10,0)$ & NO & 13413\\
$(458,187)$ & 14 & $(2,1)$ & 1 & 2 & YES & YES & NO(2) & $1.88$ & $(8,1)$ & -- & 13414\\
$(458,97)$ & 15 & $(10,3)$ & 5 & 2 & YES & YES & YES & $2.29$ & $(2,4)$ & NO & 13415\\
$(461,135)$ & 14 & $(3,1)$ & 2 & 1 & YES & YES & NO(2) & $1.90$ & $(4,3)$ & -- & 13416\\
$(461,200)$ & 14 & $(4,1)$ & 3 & 1 & YES & YES & NO(2) & $2.08$ & $(4,3)$ & NO & 13417\\
$(461,200)$ & 14 & $(4,1)$ & 3 & 1 & YES & YES & NO(2) & $2.08$ & $(4,3)$ & -- & 13418\\
$(461,135)$ & 14 & $(7,2)$ & 4 & 1 & YES & YES & NO(2) & $1.88$ & $(4,3)$ & NO & 13419\\
$(461,135)$ & 14 & $(24,7)$ & 7 & 1 & YES & YES & NO(2) & $1.90$ & $(4,3)$ & NO & 13420\\
$(462,179)$ & 13 & $(2,1)$ & 1 & 2 & YES & YES & NO(2) & $1.86$ & $(4,3)$ & -- & 13421\\
$(462,179)$ & 13 & $(191,74)$ & 11 & 1 & YES & YES & NO(2) & $1.86$ & $(4,3)$ & NO & 13422\\
$(463,97)$ & 15 & $(3,1)$ & 2 & 1 & YES & YES & NO(2) & $2.00$ & $(2,4)$ & NO & 13423\\
$(463,83)$ & 15 & $(5,2)$ & 3 & 1 & YES & YES & NO(2) & $1.89$ & $(10,0)$ & NO & 13424\\
$(463,98)$ & 14 & $(7,2)$ & 4 & 1 & YES & YES & NO(2) & $1.75$ & $(8,1)$ & NO & 13425\\
$(463,98)$ & 14 & $(13,3)$ & 6 & 1 & YES & YES & NO(2) & $1.75$ & $(8,1)$ & NO & 13426\\
$(464,141)$ & 14 & $(4,1)$ & 3 & 4 & YES & YES & NO(2) & $2.00$ & $(6,2)$ & -- & 13427\\
$(464,141)$ & 14 & $(33,10)$ & 8 & 1 & YES & YES & NO(2) & $2.09$ & $(6,2)$ & 13026 & 13428\\
$(466,135)$ & 15 & $(9,2)$ & 5 & 1 & YES & YES & NO(2) & $2.11$ & $(6,2)$ & NO & 13429\\
$(466,109)$ & 14 & $(43,10)$ & 9 & 1 & YES & YES & NO(2) & $1.86$ & $(4,3)$ & NO & 13430\\
$(467,129)$ & 13 & $(2,1)$ & 1 & 1 & YES & YES & NO(2) & $1.80$ & $(2,4)$ & -- & 13431\\
$(467,181)$ & 13 & $(2,1)$ & 1 & 1 & YES & YES & NO(2) & $1.89$ & $(6,2)$ & -- & 13432\\
$(467,196)$ & 13 & $(2,1)$ & 1 & 1 & YES & YES & NO(2) & $2.00$ & $(4,3)$ & -- & 13433\\
$(467,182)$ & 14 & $(8,3)$ & 4 & 1 & YES & YES & NO(2) & $2.11$ & $(4,3)$ & NO & 13434\\
$(467,177)$ & 14 & $(95,36)$ & 10 & 1 & YES & YES & NO(2) & $2.00$ & $(4,3)$ & NO & 13435\\
$(469,179)$ & 13 & $(2,1)$ & 1 & 1 & YES & YES & NO(2) & $2.00$ & $(6,2)$ & -- & 13436\\
$(473,196)$ & 13 & $(4,1)$ & 3 & 1 & YES & YES & YES & $2.00$ & $(2,4)$ & NO & 13437\\
$(473,196)$ & 13 & $(5,2)$ & 3 & 1 & YES & YES & NO(2) & $2.11$ & $(4,3)$ & 13051 & 13438\\
$(473,196)$ & 13 & $(7,3)$ & 4 & 1 & YES & YES & NO(2) & $1.90$ & $(4,3)$ & NO & 13439\\
$(473,196)$ & 13 & $(12,5)$ & 5 & 1 & YES & YES & NO(2) & $2.09$ & $(6,2)$ & 13255 & 13440\\
$(473,181)$ & 13 & $(47,18)$ & 8 & 1 & YES & YES & YES & $2.00$ & $(2,4)$ & NO & 13441\\
$(473,125)$ & 14 & $(49,13)$ & 9 & 1 & YES & YES & NO(2) & $2.11$ & $(6,2)$ & NO & 13442\\
$(474,199)$ & 13 & $(5,2)$ & 3 & 1 & YES & YES & NO(2) & $2.11$ & $(2,4)$ & NO & 13443\\
$(474,199)$ & 13 & $(31,13)$ & 7 & 1 & YES & YES & NO(2) & $2.11$ & $(2,4)$ & NO & 13444\\
$(476,109)$ & 14 & $(40,9)$ & 9 & 4 & YES & YES & YES & $2.17$ & $(2,4)$ & NO & 13445\\
$(476,109)$ & 14 & $(92,21)$ & 10 & 4 & YES & YES & YES & $2.17$ & $(2,4)$ & NO & 13446\\
$(478,201)$ & 14 & $(2,1)$ & 1 & 2 & YES & YES & NO(2) & $2.00$ & $(6,2)$ & -- & 13447\\
$(478,183)$ & 13 & $(3,1)$ & 2 & 1 & YES & YES & YES & $2.00$ & $(2,4)$ & -- & 13448\\
$(478,183)$ & 13 & $(4,1)$ & 3 & 2 & YES & YES & NO(3) & $1.86$ & $(2,4)$ & -- & 13449\\
$(478,139)$ & 14 & $(86,25)$ & 10 & 2 & YES & YES & NO(2) & $2.00$ & $(6,2)$ & NO & 13450\\
$(478,183)$ & 13 & $(303,116)$ & 12 & 1 & YES & YES & NO(3) & $1.86$ & $(2,4)$ & NO & 13451\\
$(478,139)$ & 14 & $(337,98)$ & 13 & 1 & YES & YES & NO(2) & $2.00$ & $(10,0)$ & NO & 13452\\
$(479,210)$ & 14 & $(73,32)$ & 10 & 1 & YES & YES & NO(2) & $2.00$ & $(6,2)$ & 13177 & 13453\\
$(481,203)$ & 14 & $(2,1)$ & 1 & 1 & YES & YES & NO(2) & $2.00$ & $(4,3)$ & -- & 13454\\
$(481,127)$ & 14 & $(3,1)$ & 2 & 1 & YES & YES & NO(2) & $2.00$ & $(4,3)$ & -- & 13455\\
$(481,140)$ & 14 & $(55,16)$ & 9 & 1 & YES & YES & NO(2) & $2.00$ & $(2,4)$ & NO & 13456\\
$(481,140)$ & 14 & $(213,62)$ & 12 & 1 & YES & YES & NO(2) & $1.71$ & $(4,3)$ & 13583 & 13457\\
$(482,183)$ & 13 & $(2,1)$ & 1 & 2 & YES & YES & NO(2) & $2.00$ & $(2,4)$ & -- & 13458\\
$(482,183)$ & 13 & $(3,1)$ & 2 & 1 & YES & YES & NO(2) & $1.89$ & $(6,2)$ & -- & 13459\\
$(482,183)$ & 13 & $(4,1)$ & 3 & 2 & YES & YES & NO(2) & $1.80$ & $(4,3)$ & -- & 13460\\
$(483,103)$ & 15 & $(5,2)$ & 3 & 1 & YES & YES & NO(2) & $2.00$ & $(6,2)$ & NO & 13461\\
$(484,177)$ & 14 & $(216,79)$ & 13 & 4 & YES & YES & NO(2) & $1.83$ & $(6,2)$ & NO & 13462\\
$(485,188)$ & 13 & $(44,17)$ & 8 & 1 & YES & YES & YES & $2.17$ & $(2,4)$ & NO & 13463\\
$(487,186)$ & 13 & $(2,1)$ & 1 & 1 & YES & YES & NO(2) & $2.00$ & $(8,1)$ & -- & 13464\\
$(487,144)$ & 13 & $(3,1)$ & 2 & 1 & NO & YES & NO(2) & $1.90$ & $(8,1)$ & -- & 13465\\
$(487,136)$ & 14 & $(5,2)$ & 3 & 1 & YES & YES & YES & $2.00$ & $(2,4)$ & NO & 13466\\
$(487,186)$ & 13 & $(21,8)$ & 6 & 1 & YES & YES & NO(2) & $2.00$ & $(8,1)$ & 13027 & 13467\\
$(487,185)$ & 13 & $(208,79)$ & 11 & 1 & YES & YES & YES & $2.00$ & $(2,4)$ & NO & 13468\\
$(488,143)$ & 14 & $(2,1)$ & 1 & 2 & YES & YES & YES & $2.00$ & $(2,4)$ & -- & 13469\\
$(490,181)$ & 14 & $(2,1)$ & 1 & 2 & YES & YES & NO(2) & $2.20$ & $(2,4)$ & -- & 13470\\
$(490,207)$ & 13 & $(5,2)$ & 3 & 5 & YES & YES & NO(2) & $1.90$ & $(4,3)$ & NO & 13471\\
$(491,145)$ & 14 & $(10,3)$ & 5 & 1 & YES & YES & NO(2) & $2.00$ & $(2,4)$ & NO & 13472\\
$(493,191)$ & 13 & $(7,3)$ & 4 & 1 & YES & YES & YES & $2.14$ & $(4,3)$ & -- & 13473\\
$(493,207)$ & 13 & $(493,207)$ & 13 & 493 & YES & YES & NO(2) & $2.11$ & $(2,4)$ & NO & 13474\\
$(494,113)$ & 15 & $(3,1)$ & 2 & 1 & YES & YES & NO(2) & $1.90$ & $(4,3)$ & -- & 13475\\
$(495,208)$ & 13 & $(2,1)$ & 1 & 1 & YES & YES & NO(2) & $2.09$ & $(2,4)$ & -- & 13476\\
$(495,139)$ & 14 & $(3,1)$ & 2 & 3 & YES & YES & NO(2) & $2.00$ & $(4,3)$ & NO & 13477\\
$(495,188)$ & 13 & $(3,1)$ & 2 & 3 & YES & YES & YES & $1.86$ & $(4,3)$ & -- & 13478\\
$(495,208)$ & 13 & $(50,21)$ & 8 & 5 & YES & YES & NO(2) & $1.90$ & $(4,3)$ & 13411 & 13479\\
$(497,152)$ & 14 & $(7,2)$ & 4 & 7 & YES & YES & YES & $2.00$ & $(2,4)$ & -- & 13480\\
$(498,151)$ & 15 & $(79,24)$ & 10 & 1 & YES & YES & NO(2) & $2.00$ & $(4,3)$ & NO & 13481\\
$(498,209)$ & 13 & $(193,81)$ & 11 & 1 & YES & YES & NO(2) & $2.11$ & $(2,4)$ & NO & 13482\\
$(499,191)$ & 13 & $(2,1)$ & 1 & 1 & YES & YES & NO(2) & $1.88$ & $(12,-1)$ & -- & 13483\\
$(499,209)$ & 13 & $(3,1)$ & 2 & 1 & YES & YES & NO(2) & $2.00$ & $(2,4)$ & -- & 13484\\
$(499,139)$ & 14 & $(11,3)$ & 5 & 1 & YES & YES & NO(2) & $1.89$ & $(6,2)$ & NO & 13485\\
$(499,193)$ & 14 & $(18,7)$ & 6 & 1 & YES & YES & NO(2) & $2.00$ & $(4,3)$ & NO & 13486\\
$(499,191)$ & 13 & $(21,8)$ & 6 & 1 & YES & YES & YES & $2.12$ & $(2,4)$ & NO & 13487\\
$(499,209)$ & 13 & $(43,18)$ & 8 & 1 & YES & YES & NO(2) & $2.11$ & $(2,4)$ & 13368 & 13488\\
$(500,207)$ & 15 & $(3,1)$ & 2 & 1 & YES & YES & NO(2) & $2.12$ & $(4,3)$ & NO & 13489\\
$(500,153)$ & 14 & $(4,1)$ & 3 & 4 & YES & YES & NO(2) & $1.90$ & $(4,3)$ & -- & 13490\\
$(500,153)$ & 14 & $(10,3)$ & 5 & 10 & YES & YES & NO(2) & $2.00$ & $(4,3)$ & NO & 13491\\
$(500,153)$ & 14 & $(36,11)$ & 8 & 4 & YES & YES & NO(2) & $2.00$ & $(4,3)$ & NO & 13492\\
$(500,153)$ & 14 & $(500,153)$ & 14 & 500 & YES & YES & NO(2) & $2.00$ & $(4,3)$ & NO & 13493\\
$(502,135)$ & 15 & $(2,1)$ & 1 & 2 & YES & YES & NO(2) & $2.00$ & $(6,2)$ & -- & 13494\\
$(502,181)$ & 14 & $(3,1)$ & 2 & 1 & YES & YES & NO(2) & $2.10$ & $(4,3)$ & -- & 13495\\
$(502,181)$ & 14 & $(14,5)$ & 6 & 2 & YES & YES & NO(2) & $2.10$ & $(4,3)$ & NO & 13496\\
$(502,135)$ & 15 & $(145,39)$ & 12 & 1 & YES & YES & NO(2) & $2.00$ & $(6,2)$ & NO & 13497\\
$(503,192)$ & 13 & $(131,50)$ & 10 & 1 & YES & YES & YES & $2.00$ & $(2,4)$ & 13395 & 13498\\
$(505,212)$ & 13 & $(2,1)$ & 1 & 1 & YES & YES & YES & $2.00$ & $(2,4)$ & -- & 13499\\
$(505,192)$ & 13 & $(3,1)$ & 2 & 1 & YES & YES & YES & $2.14$ & $(2,4)$ & -- & 13500\\
$(505,107)$ & 15 & $(24,5)$ & 8 & 1 & YES & YES & NO(2) & $2.00$ & $(6,2)$ & NO & 13501\\
$(505,141)$ & 14 & $(197,55)$ & 12 & 1 & YES & YES & NO(2) & $1.89$ & $(6,2)$ & 13564 & 13502\\
$(507,154)$ & 14 & $(11,3)$ & 5 & 1 & YES & YES & YES & $2.17$ & $(2,4)$ & -- & 13503\\
$(509,141)$ & 14 & $(3,1)$ & 2 & 1 & YES & YES & NO(2) & $2.00$ & $(4,3)$ & NO & 13504\\
$(512,149)$ & 14 & $(4,1)$ & 3 & 4 & YES & YES & NO(2) & $2.00$ & $(2,4)$ & NO & 13505\\
$(512,223)$ & 14 & $(512,223)$ & 14 & 512 & YES & YES & NO(2) & $2.00$ & $(6,2)$ & NO & 13506\\
$(513,212)$ & 13 & $(5,1)$ & 4 & 1 & YES & YES & NO(2) & $2.00$ & $(6,2)$ & -- & 13507\\
$(515,151)$ & 14 & $(3,1)$ & 2 & 1 & YES & YES & NO(2) & $1.86$ & $(4,3)$ & -- & 13508\\
$(517,118)$ & 15 & $(9,2)$ & 5 & 1 & YES & YES & YES & $2.00$ & $(2,4)$ & NO & 13509\\
$(517,118)$ & 15 & $(22,5)$ & 7 & 11 & YES & YES & YES & $2.00$ & $(2,4)$ & NO & 13510\\
$(519,197)$ & 14 & $(5,1)$ & 4 & 1 & YES & YES & NO(2) & $2.00$ & $(8,1)$ & -- & 13511\\
$(519,232)$ & 15 & $(7,2)$ & 4 & 1 & YES & YES & NO(2) & $2.00$ & $(6,2)$ & NO & 13512\\
$(520,201)$ & 14 & $(4,1)$ & 3 & 4 & YES & YES & YES & $2.12$ & $(2,4)$ & -- & 13513\\
$(521,154)$ & 14 & $(5,1)$ & 4 & 1 & YES & YES & NO(2) & $1.89$ & $(2,4)$ & -- & 13514\\
$(521,144)$ & 13 & $(29,8)$ & 7 & 1 & YES & YES & NO(2) & $1.88$ & $(2,4)$ & NO & 13515\\
$(521,144)$ & 13 & $(76,21)$ & 9 & 1 & YES & YES & NO(2) & $1.88$ & $(2,4)$ & NO & 13516\\
$(521,154)$ & 14 & $(362,107)$ & 13 & 1 & YES & YES & YES & $2.00$ & $(2,4)$ & NO & 13517\\
$(522,155)$ & 14 & $(101,30)$ & 10 & 1 & YES & YES & NO(2) & $2.00$ & $(2,4)$ & 13052 & 13518\\
$(522,155)$ & 14 & $(293,87)$ & 13 & 1 & YES & YES & NO(2) & $2.00$ & $(2,4)$ & NO & 13519\\
$(523,160)$ & 14 & $(3,1)$ & 2 & 1 & YES & YES & NO(2) & $2.09$ & $(2,4)$ & -- & 13520\\
$(524,145)$ & 14 & $(29,8)$ & 7 & 1 & YES & YES & NO(2) & $1.86$ & $(4,3)$ & NO & 13521\\
$(525,92)$ & 16 & $(63,11)$ & 10 & 21 & YES & YES & NO(2) & $2.00$ & $(6,2)$ & NO & 13522\\
$(526,221)$ & 14 & $(5,2)$ & 3 & 1 & YES & YES & NO(2) & $2.00$ & $(6,2)$ & -- & 13523\\
$(527,154)$ & 14 & $(27,8)$ & 7 & 1 & YES & YES & NO(3) & $2.00$ & $(2,4)$ & 13846 & 13524\\
$(528,145)$ & 14 & $(5,2)$ & 3 & 1 & YES & YES & NO(2) & $1.88$ & $(4,3)$ & -- & 13525\\
$(529,153)$ & 15 & $(4,1)$ & 3 & 1 & YES & YES & NO(2) & $2.00$ & $(4,3)$ & NO & 13526\\
$(530,191)$ & 15 & $(3,1)$ & 2 & 1 & YES & YES & NO(2) & $2.09$ & $(6,2)$ & -- & 13527\\
$(531,149)$ & 14 & $(5,1)$ & 4 & 1 & YES & YES & NO(2) & $1.90$ & $(4,3)$ & NO & 13528\\
$(531,205)$ & 14 & $(5,1)$ & 4 & 1 & YES & YES & YES & $1.83$ & $(4,3)$ & NO & 13529\\
$(533,202)$ & 14 & $(4,1)$ & 3 & 1 & YES & YES & NO(2) & $2.00$ & $(4,3)$ & NO & 13530\\
$(533,202)$ & 14 & $(219,83)$ & 12 & 1 & YES & YES & NO(2) & $2.11$ & $(4,3)$ & NO & 13531\\
$(535,158)$ & 14 & $(2,1)$ & 1 & 1 & YES & YES & NO(2) & $1.86$ & $(4,3)$ & -- & 13532\\
$(535,222)$ & 14 & $(5,1)$ & 4 & 5 & YES & YES & NO(2) & $2.00$ & $(6,2)$ & -- & 13533\\
$(535,158)$ & 14 & $(18,5)$ & 6 & 1 & YES & YES & YES & $2.14$ & $(4,3)$ & NO & 13534\\
$(535,158)$ & 14 & $(27,8)$ & 7 & 1 & YES & YES & NO(2) & $1.88$ & $(6,2)$ & NO & 13535\\
$(535,158)$ & 14 & $(105,31)$ & 10 & 5 & YES & YES & YES & $2.00$ & $(2,4)$ & 13708 & 13536\\
$(536,209)$ & 14 & $(2,1)$ & 1 & 2 & YES & YES & NO(2) & $2.18$ & $(2,4)$ & -- & 13537\\
$(536,209)$ & 14 & $(3,1)$ & 2 & 1 & YES & YES & NO(2) & $2.00$ & $(4,3)$ & -- & 13538\\
$(536,209)$ & 14 & $(41,16)$ & 8 & 1 & YES & YES & NO(2) & $2.18$ & $(2,4)$ & NO & 13539\\
$(536,209)$ & 14 & $(100,39)$ & 10 & 4 & YES & YES & NO(2) & $2.00$ & $(4,3)$ & NO & 13540\\
$(536,209)$ & 14 & $(259,101)$ & 12 & 1 & YES & YES & YES & $2.14$ & $(4,3)$ & NO & 13541\\
$(537,241)$ & 14 & $(78,35)$ & 10 & 3 & YES & YES & NO(2) & $2.00$ & $(4,3)$ & NO & 13542\\
$(538,223)$ & 14 & $(111,46)$ & 10 & 1 & YES & YES & YES & $2.14$ & $(2,4)$ & 13724 & 13543\\
$(539,123)$ & 14 & $(53,12)$ & 9 & 1 & YES & YES & YES & $2.17$ & $(2,4)$ & NO & 13544\\
$(542,127)$ & 15 & $(22,5)$ & 7 & 2 & YES & YES & NO(2) & $1.88$ & $(6,2)$ & NO & 13545\\
$(542,127)$ & 15 & $(43,10)$ & 9 & 1 & YES & YES & NO(2) & $1.88$ & $(6,2)$ & NO & 13546\\
$(544,115)$ & 15 & $(7,2)$ & 4 & 1 & YES & YES & NO(2) & $2.00$ & $(2,4)$ & -- & 13547\\
$(544,115)$ & 15 & $(85,18)$ & 10 & 17 & YES & YES & NO(2) & $2.00$ & $(2,4)$ & 13636 & 13548\\
$(544,165)$ & 14 & $(211,64)$ & 12 & 1 & YES & YES & NO(2) & $1.88$ & $(4,3)$ & NO & 13549\\
$(545,208)$ & 13 & $(5,1)$ & 4 & 5 & YES & YES & NO(3) & $1.86$ & $(2,4)$ & NO & 13550\\
$(545,207)$ & 13 & $(13,5)$ & 5 & 1 & YES & YES & YES & $2.17$ & $(2,4)$ & 13644 & 13551\\
$(545,147)$ & 14 & $(15,4)$ & 6 & 5 & YES & YES & NO(2) & $1.88$ & $(2,4)$ & NO & 13552\\
$(545,208)$ & 13 & $(207,79)$ & 11 & 1 & YES & YES & NO(3) & $1.86$ & $(2,4)$ & NO & 13553\\
$(545,207)$ & 13 & $(337,128)$ & 12 & 1 & YES & YES & YES & $2.00$ & $(2,4)$ & NO & 13554\\
$(546,209)$ & 13 & $(8,3)$ & 4 & 2 & YES & YES & YES & $2.12$ & $(2,4)$ & 13140 & 13555\\
$(546,209)$ & 13 & $(21,8)$ & 6 & 21 & YES & YES & YES & $2.17$ & $(2,4)$ & NO & 13556\\
$(547,147)$ & 14 & $(5,1)$ & 4 & 1 & YES & YES & NO(2) & $2.00$ & $(2,4)$ & NO & 13557\\
$(547,160)$ & 14 & $(5,2)$ & 3 & 1 & YES & YES & YES & $2.00$ & $(2,4)$ & NO & 13558\\
$(547,209)$ & 13 & $(5,2)$ & 3 & 1 & YES & YES & YES & $2.00$ & $(2,4)$ & -- & 13559\\
$(548,197)$ & 14 & $(2,1)$ & 1 & 2 & YES & YES & NO(2) & $2.11$ & $(6,2)$ & -- & 13560\\
$(548,209)$ & 14 & $(3,1)$ & 2 & 1 & YES & YES & NO(2) & $2.10$ & $(4,3)$ & -- & 13561\\
$(548,209)$ & 14 & $(3,1)$ & 2 & 1 & YES & YES & NO(2) & $2.10$ & $(4,3)$ & NO & 13562\\
$(548,209)$ & 14 & $(34,13)$ & 7 & 2 & YES & YES & NO(2) & $2.10$ & $(4,3)$ & NO & 13563\\
$(548,153)$ & 14 & $(154,43)$ & 11 & 2 & YES & YES & NO(2) & $1.89$ & $(6,2)$ & 13502 & 13564\\
$(548,227)$ & 14 & $(239,99)$ & 12 & 1 & YES & YES & NO(2) & $2.11$ & $(4,3)$ & NO & 13565\\
$(549,208)$ & 14 & $(2,1)$ & 1 & 1 & YES & YES & NO(2) & $2.00$ & $(4,3)$ & NO & 13566\\
$(549,212)$ & 15 & $(101,39)$ & 10 & 1 & YES & YES & NO(2) & $2.12$ & $(4,3)$ & NO & 13567\\
$(551,161)$ & 14 & $(2,1)$ & 1 & 1 & YES & YES & NO(2) & $1.90$ & $(4,3)$ & -- & 13568\\
$(551,240)$ & 14 & $(2,1)$ & 1 & 1 & YES & YES & NO(2) & $2.10$ & $(4,3)$ & -- & 13569\\
$(552,149)$ & 14 & $(5,1)$ & 4 & 1 & YES & YES & NO(2) & $1.88$ & $(2,4)$ & NO & 13570\\
$(553,229)$ & 14 & $(5,2)$ & 3 & 1 & YES & YES & YES & $2.25$ & $(2,4)$ & -- & 13571\\
$(553,146)$ & 15 & $(49,13)$ & 9 & 7 & YES & YES & NO(2) & $2.00$ & $(6,2)$ & NO & 13572\\
$(553,229)$ & 14 & $(70,29)$ & 9 & 7 & YES & YES & YES & $2.12$ & $(2,4)$ & NO & 13573\\
$(554,227)$ & 14 & $(349,143)$ & 13 & 1 & YES & YES & YES & $2.00$ & $(2,4)$ & NO & 13574\\
$(555,233)$ & 13 & $(5,2)$ & 3 & 5 & YES & YES & NO(2) & $2.11$ & $(2,4)$ & NO & 13575\\
$(555,233)$ & 13 & $(31,13)$ & 7 & 1 & YES & YES & NO(2) & $2.11$ & $(2,4)$ & NO & 13576\\
$(555,229)$ & 14 & $(206,85)$ & 12 & 1 & YES & YES & YES & $2.14$ & $(2,4)$ & NO & 13577\\
$(557,201)$ & 14 & $(2,1)$ & 1 & 1 & YES & YES & NO(2) & $2.00$ & $(12,-1)$ & -- & 13578\\
$(559,166)$ & 14 & $(3,1)$ & 2 & 1 & NO & YES & NO(2) & $1.91$ & $(6,2)$ & -- & 13579\\
$(559,165)$ & 14 & $(7,3)$ & 4 & 1 & YES & YES & YES & $2.33$ & $(2,4)$ & -- & 13580\\
$(559,165)$ & 14 & $(31,9)$ & 8 & 1 & YES & YES & YES & $2.33$ & $(2,4)$ & NO & 13581\\
$(560,163)$ & 14 & $(3,1)$ & 2 & 1 & YES & YES & YES & $1.86$ & $(2,4)$ & -- & 13582\\
$(560,163)$ & 14 & $(134,39)$ & 11 & 2 & YES & YES & NO(2) & $1.71$ & $(4,3)$ & 13457 & 13583\\
$(563,220)$ & 14 & $(18,7)$ & 6 & 1 & YES & YES & NO(2) & $1.88$ & $(8,1)$ & NO & 13584\\
$(566,175)$ & 15 & $(2,1)$ & 1 & 2 & YES & YES & NO(2) & $1.89$ & $(10,0)$ & -- & 13585\\
$(566,253)$ & 15 & $(5,2)$ & 3 & 1 & YES & YES & NO(2) & $2.00$ & $(6,2)$ & -- & 13586\\
$(569,125)$ & 16 & $(7,2)$ & 4 & 1 & YES & YES & NO(2) & $2.14$ & $(4,3)$ & NO & 13587\\
$(570,173)$ & 15 & $(2,1)$ & 1 & 2 & YES & YES & NO(2) & $1.91$ & $(6,2)$ & -- & 13588\\
$(571,158)$ & 14 & $(2,1)$ & 1 & 1 & YES & YES & NO(2) & $1.86$ & $(4,3)$ & NO & 13589\\
$(571,151)$ & 15 & $(3,1)$ & 2 & 1 & YES & YES & NO(2) & $2.00$ & $(4,3)$ & -- & 13590\\
$(571,223)$ & 14 & $(3,1)$ & 2 & 1 & YES & YES & NO(2) & $2.00$ & $(4,3)$ & -- & 13591\\
$(571,223)$ & 14 & $(18,7)$ & 6 & 1 & YES & YES & NO(2) & $1.88$ & $(8,1)$ & NO & 13592\\
$(571,151)$ & 15 & $(53,14)$ & 9 & 1 & YES & YES & NO(2) & $2.10$ & $(4,3)$ & NO & 13593\\
$(573,169)$ & 14 & $(5,1)$ & 4 & 1 & YES & YES & NO(2) & $2.00$ & $(6,2)$ & -- & 13594\\
$(575,158)$ & 14 & $(5,2)$ & 3 & 5 & YES & YES & NO(2) & $2.12$ & $(2,4)$ & -- & 13595\\
$(577,239)$ & 14 & $(2,1)$ & 1 & 1 & YES & YES & NO(2) & $2.10$ & $(4,3)$ & NO & 13596\\
$(577,207)$ & 14 & $(11,4)$ & 5 & 1 & YES & YES & NO(2) & $2.00$ & $(6,2)$ & NO & 13597\\
$(577,132)$ & 15 & $(14,3)$ & 6 & 1 & YES & YES & YES & $2.00$ & $(4,3)$ & NO & 13598\\
$(577,239)$ & 14 & $(17,7)$ & 6 & 1 & YES & YES & NO(2) & $2.10$ & $(4,3)$ & NO & 13599\\
$(577,132)$ & 15 & $(118,27)$ & 11 & 1 & YES & YES & NO(2) & $1.88$ & $(4,3)$ & NO & 13600\\
$(578,161)$ & 14 & $(2,1)$ & 1 & 2 & YES & YES & NO(2) & $1.90$ & $(8,1)$ & -- & 13601\\
$(578,219)$ & 14 & $(2,1)$ & 1 & 2 & YES & YES & NO(2) & $2.10$ & $(4,3)$ & NO & 13602\\
$(578,161)$ & 14 & $(3,1)$ & 2 & 1 & YES & YES & NO(2) & $1.86$ & $(4,3)$ & -- & 13603\\
$(578,171)$ & 14 & $(27,8)$ & 7 & 1 & YES & YES & NO(2) & $2.00$ & $(2,4)$ & NO & 13604\\
$(579,254)$ & 14 & $(16,7)$ & 6 & 1 & YES & YES & NO(2) & $2.00$ & $(6,2)$ & NO & 13605\\
$(580,251)$ & 15 & $(2,1)$ & 1 & 2 & YES & YES & YES & $2.00$ & $(4,3)$ & NO & 13606\\
$(581,221)$ & 14 & $(11,4)$ & 5 & 1 & YES & YES & YES & $2.12$ & $(2,4)$ & NO & 13607\\
$(582,239)$ & 15 & $(6,1)$ & 5 & 6 & YES & YES & NO(2) & $2.14$ & $(4,3)$ & -- & 13608\\
$(583,161)$ & 14 & $(2,1)$ & 1 & 1 & YES & YES & NO(2) & $1.86$ & $(4,3)$ & -- & 13609\\
$(583,241)$ & 14 & $(2,1)$ & 1 & 1 & YES & YES & YES & $2.14$ & $(2,4)$ & -- & 13610\\
$(583,161)$ & 14 & $(18,5)$ & 6 & 1 & YES & YES & NO(2) & $1.86$ & $(4,3)$ & NO & 13611\\
$(585,158)$ & 15 & $(2,1)$ & 1 & 1 & YES & YES & NO(2) & $1.86$ & $(4,3)$ & -- & 13612\\
$(585,158)$ & 15 & $(26,7)$ & 7 & 13 & YES & YES & NO(2) & $1.86$ & $(4,3)$ & NO & 13613\\
$(587,256)$ & 14 & $(2,1)$ & 1 & 1 & NO & YES & NO(2) & $2.09$ & $(2,4)$ & -- & 13614\\
$(587,256)$ & 14 & $(2,1)$ & 1 & 1 & YES & YES & NO(2) & $2.00$ & $(6,2)$ & NO & 13615\\
$(587,256)$ & 14 & $(3,1)$ & 2 & 1 & YES & YES & NO(2) & $1.89$ & $(6,2)$ & NO & 13616\\
$(587,175)$ & 15 & $(426,127)$ & 14 & 1 & YES & YES & NO(2) & $1.89$ & $(6,2)$ & NO & 13617\\
$(590,173)$ & 15 & $(2,1)$ & 1 & 2 & YES & YES & NO(2) & $2.00$ & $(6,2)$ & -- & 13618\\
$(590,219)$ & 14 & $(4,1)$ & 3 & 2 & YES & YES & NO(2) & $2.00$ & $(4,3)$ & -- & 13619\\
$(590,229)$ & 14 & $(219,85)$ & 12 & 1 & YES & YES & YES & $2.00$ & $(4,3)$ & NO & 13620\\
$(591,244)$ & 15 & $(109,45)$ & 10 & 1 & YES & YES & NO(2) & $2.00$ & $(8,1)$ & NO & 13621\\
$(592,229)$ & 14 & $(243,94)$ & 12 & 1 & YES & YES & NO(2) & $2.00$ & $(10,0)$ & NO & 13622\\
$(593,167)$ & 15 & $(2,1)$ & 1 & 1 & YES & YES & NO(2) & $2.00$ & $(4,3)$ & -- & 13623\\
$(593,245)$ & 14 & $(3,1)$ & 2 & 1 & YES & YES & NO(2) & $1.91$ & $(6,2)$ & -- & 13624\\
$(593,176)$ & 14 & $(10,3)$ & 5 & 1 & YES & YES & NO(2) & $2.00$ & $(2,4)$ & NO & 13625\\
$(593,231)$ & 15 & $(23,9)$ & 7 & 1 & YES & YES & NO(2) & $2.00$ & $(6,2)$ & NO & 13626\\
$(595,227)$ & 14 & $(2,1)$ & 1 & 1 & YES & YES & NO(2) & $2.00$ & $(6,2)$ & -- & 13627\\
$(595,173)$ & 14 & $(3,1)$ & 2 & 1 & YES & YES & NO(2) & $1.75$ & $(6,2)$ & -- & 13628\\
$(595,227)$ & 14 & $(3,1)$ & 2 & 1 & YES & YES & NO(2) & $2.00$ & $(4,3)$ & -- & 13629\\
$(595,227)$ & 14 & $(13,5)$ & 5 & 1 & YES & YES & NO(2) & $2.00$ & $(6,2)$ & NO & 13630\\
$(599,167)$ & 14 & $(2,1)$ & 1 & 1 & YES & YES & NO(3) & $1.86$ & $(2,4)$ & -- & 13631\\
$(599,167)$ & 14 & $(11,3)$ & 5 & 1 & YES & YES & NO(3) & $1.86$ & $(2,4)$ & NO & 13632\\
$(599,167)$ & 14 & $(43,12)$ & 8 & 1 & YES & YES & YES & $2.00$ & $(2,4)$ & NO & 13633\\
$(599,167)$ & 14 & $(147,41)$ & 11 & 1 & YES & YES & NO(3) & $2.00$ & $(2,4)$ & 13836 & 13634\\
$(599,167)$ & 14 & $(599,167)$ & 14 & 599 & YES & YES & NO(3) & $1.86$ & $(2,4)$ & NO & 13635\\
$(600,127)$ & 15 & $(71,15)$ & 10 & 1 & YES & YES & NO(2) & $2.00$ & $(2,4)$ & 13548 & 13636\\
$(601,175)$ & 15 & $(5,2)$ & 3 & 1 & YES & YES & YES & $2.25$ & $(2,4)$ & -- & 13637\\
$(602,229)$ & 14 & $(4,1)$ & 3 & 2 & YES & YES & NO(2) & $2.00$ & $(8,1)$ & -- & 13638\\
$(603,178)$ & 14 & $(7,2)$ & 4 & 1 & YES & YES & NO(3) & $1.86$ & $(2,4)$ & NO & 13639\\
$(608,189)$ & 15 & $(29,9)$ & 8 & 1 & YES & YES & NO(2) & $2.00$ & $(4,3)$ & NO & 13640\\
$(610,233)$ & 13 & $(2,1)$ & 1 & 2 & YES & YES & YES & $2.17$ & $(2,4)$ & -- & 13641\\
$(610,233)$ & 13 & $(3,1)$ & 2 & 1 & YES & YES & YES & $2.00$ & $(2,4)$ & -- & 13642\\
$(610,253)$ & 15 & $(4,1)$ & 3 & 2 & YES & YES & YES & $2.00$ & $(4,3)$ & -- & 13643\\
$(610,233)$ & 13 & $(8,3)$ & 4 & 2 & YES & YES & YES & $2.17$ & $(2,4)$ & 13551 & 13644\\
$(611,237)$ & 14 & $(2,1)$ & 1 & 1 & YES & YES & YES & $2.00$ & $(4,3)$ & -- & 13645\\
$(611,237)$ & 14 & $(13,5)$ & 5 & 13 & YES & YES & YES & $2.00$ & $(4,3)$ & NO & 13646\\
$(611,256)$ & 14 & $(105,44)$ & 10 & 1 & YES & YES & NO(2) & $2.00$ & $(4,3)$ & NO & 13647\\
$(612,179)$ & 14 & $(10,3)$ & 5 & 2 & YES & YES & YES & $2.00$ & $(2,4)$ & NO & 13648\\
$(612,179)$ & 14 & $(147,43)$ & 11 & 3 & YES & YES & YES & $2.00$ & $(2,4)$ & NO & 13649\\
$(613,186)$ & 14 & $(2,1)$ & 1 & 1 & YES & YES & YES & $2.00$ & $(2,4)$ & -- & 13650\\
$(613,221)$ & 14 & $(5,1)$ & 4 & 1 & YES & YES & NO(2) & $2.00$ & $(4,3)$ & -- & 13651\\
$(613,267)$ & 14 & $(7,3)$ & 4 & 1 & YES & YES & NO(2) & $2.00$ & $(4,3)$ & NO & 13652\\
$(613,234)$ & 14 & $(131,50)$ & 10 & 1 & YES & YES & YES & $2.00$ & $(2,4)$ & NO & 13653\\
$(613,234)$ & 14 & $(613,234)$ & 14 & 613 & YES & YES & YES & $2.00$ & $(2,4)$ & NO & 13654\\
$(614,165)$ & 14 & $(2,1)$ & 1 & 2 & YES & YES & NO(2) & $2.00$ & $(2,4)$ & -- & 13655\\
$(614,227)$ & 14 & $(2,1)$ & 1 & 2 & YES & YES & YES & $2.12$ & $(2,4)$ & -- & 13656\\
$(615,227)$ & 14 & $(3,1)$ & 2 & 3 & YES & YES & NO(2) & $2.11$ & $(2,4)$ & -- & 13657\\
$(617,172)$ & 14 & $(3,1)$ & 2 & 1 & YES & YES & YES & $2.00$ & $(4,3)$ & -- & 13658\\
$(618,239)$ & 14 & $(2,1)$ & 1 & 2 & YES & YES & NO(2) & $1.90$ & $(4,3)$ & -- & 13659\\
$(618,239)$ & 14 & $(106,41)$ & 10 & 2 & YES & YES & NO(2) & $2.11$ & $(6,2)$ & NO & 13660\\
$(619,141)$ & 15 & $(13,3)$ & 6 & 1 & YES & YES & NO(2) & $1.88$ & $(4,3)$ & NO & 13661\\
$(619,171)$ & 14 & $(25,7)$ & 7 & 1 & YES & YES & NO(3) & $2.00$ & $(2,4)$ & 13835 & 13662\\
$(620,183)$ & 14 & $(3,1)$ & 2 & 1 & YES & YES & NO(2) & $2.11$ & $(2,4)$ & NO & 13663\\
$(620,183)$ & 14 & $(4,1)$ & 3 & 4 & YES & YES & NO(3) & $1.86$ & $(2,4)$ & -- & 13664\\
$(620,227)$ & 14 & $(183,67)$ & 11 & 1 & YES & YES & YES & $2.14$ & $(2,4)$ & NO & 13665\\
$(621,140)$ & 15 & $(7,3)$ & 4 & 1 & YES & YES & YES & $2.33$ & $(2,4)$ & -- & 13666\\
$(622,241)$ & 14 & $(9,2)$ & 5 & 1 & YES & YES & YES & $2.17$ & $(2,4)$ & -- & 13667\\
$(622,181)$ & 15 & $(244,71)$ & 13 & 2 & YES & YES & NO(2) & $2.00$ & $(6,2)$ & 13722 & 13668\\
$(623,185)$ & 14 & $(7,3)$ & 4 & 7 & YES & YES & YES & $2.14$ & $(4,3)$ & -- & 13669\\
$(625,182)$ & 15 & $(31,9)$ & 8 & 1 & YES & YES & NO(2) & $2.00$ & $(6,2)$ & NO & 13670\\
$(625,271)$ & 15 & $(53,23)$ & 9 & 1 & YES & YES & NO(2) & $1.83$ & $(6,2)$ & NO & 13671\\
$(625,258)$ & 14 & $(281,116)$ & 12 & 1 & YES & YES & YES & $2.14$ & $(2,4)$ & 13763 & 13672\\
$(628,265)$ & 14 & $(3,1)$ & 2 & 1 & YES & YES & NO(2) & $2.00$ & $(4,3)$ & -- & 13673\\
$(628,265)$ & 14 & $(12,5)$ & 5 & 4 & YES & YES & NO(2) & $2.00$ & $(4,3)$ & NO & 13674\\
$(628,265)$ & 14 & $(282,119)$ & 12 & 2 & YES & YES & YES & $2.00$ & $(2,4)$ & 13765 & 13675\\
$(632,187)$ & 14 & $(3,1)$ & 2 & 1 & YES & YES & NO(2) & $1.90$ & $(4,3)$ & -- & 13676\\
$(632,137)$ & 15 & $(5,1)$ & 4 & 1 & YES & YES & NO(2) & $1.86$ & $(4,3)$ & NO & 13677\\
$(632,137)$ & 15 & $(632,137)$ & 15 & 632 & YES & YES & NO(2) & $2.00$ & $(2,4)$ & NO & 13678\\
$(633,175)$ & 14 & $(5,1)$ & 4 & 1 & YES & YES & YES & $2.00$ & $(2,4)$ & NO & 13679\\
$(633,148)$ & 15 & $(22,5)$ & 7 & 1 & YES & YES & NO(2) & $1.88$ & $(6,2)$ & NO & 13680\\
$(634,259)$ & 15 & $(12,5)$ & 5 & 2 & YES & YES & YES & $2.17$ & $(4,3)$ & NO & 13681\\
$(635,277)$ & 14 & $(3,1)$ & 2 & 1 & YES & YES & NO(2) & $1.78$ & $(6,2)$ & -- & 13682\\
$(635,243)$ & 14 & $(60,23)$ & 9 & 5 & YES & YES & YES & $2.33$ & $(2,4)$ & NO & 13683\\
$(636,179)$ & 16 & $(4,1)$ & 3 & 4 & YES & YES & YES & $2.14$ & $(2,4)$ & -- & 13684\\
$(638,175)$ & 15 & $(6,1)$ & 5 & 2 & YES & YES & NO(2) & $1.89$ & $(6,2)$ & NO & 13685\\
$(639,140)$ & 15 & $(7,3)$ & 4 & 1 & YES & YES & YES & $2.14$ & $(2,4)$ & -- & 13686\\
$(639,121)$ & 16 & $(9,2)$ & 5 & 9 & YES & YES & NO(2) & $1.91$ & $(6,2)$ & NO & 13687\\
$(640,187)$ & 14 & $(2,1)$ & 1 & 2 & YES & YES & YES & $2.00$ & $(2,4)$ & NO & 13688\\
$(640,243)$ & 14 & $(21,8)$ & 6 & 1 & YES & YES & NO(2) & $2.11$ & $(6,2)$ & NO & 13689\\
$(640,243)$ & 14 & $(108,41)$ & 10 & 4 & YES & YES & NO(2) & $2.00$ & $(4,3)$ & NO & 13690\\
$(641,146)$ & 15 & $(5,2)$ & 3 & 1 & YES & YES & NO(2) & $1.88$ & $(4,3)$ & -- & 13691\\
$(643,246)$ & 14 & $(5,2)$ & 3 & 1 & YES & YES & YES & $2.17$ & $(2,4)$ & -- & 13692\\
$(643,178)$ & 14 & $(7,2)$ & 4 & 1 & YES & YES & YES & $2.00$ & $(2,4)$ & NO & 13693\\
$(643,246)$ & 14 & $(149,57)$ & 11 & 1 & YES & YES & YES & $2.14$ & $(2,4)$ & NO & 13694\\
$(643,196)$ & 15 & $(456,139)$ & 14 & 1 & YES & YES & NO(2) & $1.86$ & $(4,3)$ & NO & 13695\\
$(643,177)$ & 14 & $(643,177)$ & 14 & 643 & YES & YES & YES & $2.00$ & $(2,4)$ & NO & 13696\\
$(647,246)$ & 14 & $(3,1)$ & 2 & 1 & YES & YES & NO(3) & $2.00$ & $(2,4)$ & -- & 13697\\
$(647,268)$ & 14 & $(12,5)$ & 5 & 1 & YES & YES & YES & $2.12$ & $(2,4)$ & NO & 13698\\
$(649,251)$ & 14 & $(5,1)$ & 4 & 1 & YES & YES & YES & $2.17$ & $(2,4)$ & NO & 13699\\
$(649,251)$ & 14 & $(5,1)$ & 4 & 1 & YES & YES & YES & $2.17$ & $(2,4)$ & -- & 13700\\
$(649,251)$ & 14 & $(106,41)$ & 10 & 1 & YES & YES & NO(2) & $2.11$ & $(6,2)$ & NO & 13701\\
$(653,149)$ & 15 & $(2,1)$ & 1 & 1 & YES & YES & NO(2) & $2.10$ & $(2,4)$ & -- & 13702\\
$(654,271)$ & 14 & $(2,1)$ & 1 & 2 & NO & YES & NO(2) & $1.86$ & $(6,2)$ & -- & 13703\\
$(655,137)$ & 16 & $(9,2)$ & 5 & 1 & YES & YES & NO(2) & $1.91$ & $(6,2)$ & NO & 13704\\
$(655,137)$ & 16 & $(67,14)$ & 10 & 1 & YES & YES & NO(2) & $2.00$ & $(6,2)$ & NO & 13705\\
$(656,183)$ & 15 & $(3,1)$ & 2 & 1 & YES & YES & NO(2) & $1.89$ & $(6,2)$ & -- & 13706\\
$(656,183)$ & 15 & $(25,7)$ & 7 & 1 & YES & YES & NO(2) & $1.89$ & $(6,2)$ & NO & 13707\\
$(657,194)$ & 14 & $(44,13)$ & 8 & 1 & YES & YES & YES & $2.00$ & $(2,4)$ & 13536 & 13708\\
$(657,148)$ & 15 & $(53,12)$ & 9 & 1 & YES & YES & YES & $2.17$ & $(2,4)$ & NO & 13709\\
$(661,200)$ & 15 & $(7,2)$ & 4 & 1 & YES & YES & NO(2) & $1.88$ & $(6,2)$ & NO & 13710\\
$(667,204)$ & 15 & $(3,1)$ & 2 & 1 & YES & YES & YES & $2.00$ & $(2,4)$ & -- & 13711\\
$(668,151)$ & 15 & $(2,1)$ & 1 & 2 & YES & YES & NO(2) & $1.90$ & $(4,3)$ & -- & 13712\\
$(668,255)$ & 14 & $(4,1)$ & 3 & 4 & YES & YES & YES & $2.00$ & $(2,4)$ & NO & 13713\\
$(668,255)$ & 14 & $(7,3)$ & 4 & 1 & YES & YES & YES & $2.17$ & $(2,4)$ & NO & 13714\\
$(673,247)$ & 14 & $(3,1)$ & 2 & 1 & YES & YES & YES & $2.14$ & $(2,4)$ & -- & 13715\\
$(673,255)$ & 14 & $(673,255)$ & 14 & 673 & YES & YES & YES & $2.00$ & $(2,4)$ & NO & 13716\\
$(676,207)$ & 16 & $(4,1)$ & 3 & 4 & YES & YES & NO(2) & $1.91$ & $(6,2)$ & -- & 13717\\
$(676,209)$ & 16 & $(10,3)$ & 5 & 2 & YES & YES & NO(2) & $2.00$ & $(6,2)$ & NO & 13718\\
$(676,207)$ & 16 & $(258,79)$ & 14 & 2 & YES & YES & NO(2) & $2.00$ & $(6,2)$ & 13756 & 13719\\
$(677,187)$ & 14 & $(7,3)$ & 4 & 1 & YES & YES & YES & $2.14$ & $(4,3)$ & -- & 13720\\
$(677,197)$ & 15 & $(9,2)$ & 5 & 1 & YES & YES & YES & $2.29$ & $(2,4)$ & NO & 13721\\
$(677,197)$ & 15 & $(189,55)$ & 12 & 1 & YES & YES & NO(2) & $2.00$ & $(6,2)$ & 13668 & 13722\\
$(677,259)$ & 14 & $(413,158)$ & 13 & 1 & YES & YES & YES & $2.17$ & $(2,4)$ & NO & 13723\\
$(678,281)$ & 14 & $(41,17)$ & 8 & 1 & YES & YES & YES & $2.14$ & $(2,4)$ & 13543 & 13724\\
$(680,287)$ & 14 & $(2,1)$ & 1 & 2 & YES & YES & YES & $2.00$ & $(2,4)$ & -- & 13725\\
$(680,287)$ & 14 & $(417,176)$ & 13 & 1 & YES & YES & YES & $2.00$ & $(2,4)$ & NO & 13726\\
$(681,125)$ & 17 & $(7,2)$ & 4 & 1 & YES & YES & NO(2) & $2.00$ & $(6,2)$ & NO & 13727\\
$(683,181)$ & 15 & $(2,1)$ & 1 & 1 & YES & YES & NO(2) & $2.10$ & $(4,3)$ & -- & 13728\\
$(683,282)$ & 14 & $(5,2)$ & 3 & 1 & YES & YES & YES & $2.00$ & $(2,4)$ & NO & 13729\\
$(684,281)$ & 15 & $(4,1)$ & 3 & 4 & YES & YES & YES & $2.25$ & $(2,4)$ & -- & 13730\\
$(685,287)$ & 14 & $(3,1)$ & 2 & 1 & YES & YES & YES & $2.17$ & $(2,4)$ & -- & 13731\\
$(691,300)$ & 15 & $(3,1)$ & 2 & 1 & YES & YES & YES & $2.14$ & $(2,4)$ & -- & 13732\\
$(691,214)$ & 15 & $(13,4)$ & 6 & 1 & YES & YES & NO(2) & $1.90$ & $(4,3)$ & NO & 13733\\
$(695,271)$ & 15 & $(3,1)$ & 2 & 1 & YES & YES & YES & $2.17$ & $(4,3)$ & -- & 13734\\
$(695,288)$ & 14 & $(3,1)$ & 2 & 1 & YES & YES & YES & $2.00$ & $(2,4)$ & -- & 13735\\
$(695,292)$ & 14 & $(407,171)$ & 13 & 1 & YES & YES & YES & $2.17$ & $(2,4)$ & NO & 13736\\
$(697,288)$ & 14 & $(2,1)$ & 1 & 1 & YES & YES & YES & $2.14$ & $(2,4)$ & -- & 13737\\
$(697,313)$ & 16 & $(29,13)$ & 8 & 1 & YES & YES & NO(2) & $2.09$ & $(6,2)$ & NO & 13738\\
$(698,267)$ & 15 & $(2,1)$ & 1 & 2 & YES & YES & NO(2) & $2.20$ & $(4,3)$ & NO & 13739\\
$(698,267)$ & 15 & $(13,5)$ & 5 & 1 & YES & YES & NO(2) & $2.20$ & $(4,3)$ & NO & 13740\\
$(701,189)$ & 15 & $(2,1)$ & 1 & 1 & YES & YES & NO(2) & $1.91$ & $(6,2)$ & -- & 13741\\
$(703,208)$ & 14 & $(2,1)$ & 1 & 1 & YES & YES & YES & $1.83$ & $(2,4)$ & NO & 13742\\
$(703,267)$ & 14 & $(3,1)$ & 2 & 1 & YES & YES & YES & $2.17$ & $(2,4)$ & -- & 13743\\
$(705,268)$ & 14 & $(5,1)$ & 4 & 5 & YES & YES & YES & $2.14$ & $(4,3)$ & -- & 13744\\
$(705,292)$ & 14 & $(12,5)$ & 5 & 3 & YES & YES & NO(2) & $2.11$ & $(2,4)$ & NO & 13745\\
$(705,268)$ & 14 & $(705,268)$ & 14 & 705 & YES & YES & YES & $2.17$ & $(2,4)$ & NO & 13746\\
$(708,191)$ & 15 & $(215,58)$ & 12 & 1 & YES & YES & NO(2) & $1.91$ & $(6,2)$ & NO & 13747\\
$(715,277)$ & 14 & $(80,31)$ & 9 & 5 & YES & YES & NO(3) & $2.00$ & $(2,4)$ & 13317 & 13748\\
$(718,263)$ & 14 & $(2,1)$ & 1 & 2 & YES & YES & YES & $2.00$ & $(2,4)$ & NO & 13749\\
$(719,273)$ & 14 & $(3,1)$ & 2 & 1 & YES & YES & YES & $2.00$ & $(2,4)$ & -- & 13750\\
$(721,281)$ & 15 & $(3,1)$ & 2 & 1 & YES & YES & YES & $2.14$ & $(4,3)$ & -- & 13751\\
$(722,201)$ & 15 & $(2,1)$ & 1 & 2 & YES & YES & NO(2) & $1.90$ & $(4,3)$ & -- & 13752\\
$(722,261)$ & 15 & $(4,1)$ & 3 & 2 & YES & YES & NO(2) & $2.00$ & $(6,2)$ & NO & 13753\\
$(725,222)$ & 16 & $(4,1)$ & 3 & 1 & YES & YES & NO(2) & $1.91$ & $(6,2)$ & -- & 13754\\
$(725,262)$ & 15 & $(4,1)$ & 3 & 1 & YES & YES & NO(2) & $2.00$ & $(6,2)$ & NO & 13755\\
$(725,222)$ & 16 & $(209,64)$ & 13 & 1 & YES & YES & NO(2) & $2.00$ & $(6,2)$ & 13719 & 13756\\
$(727,281)$ & 14 & $(445,172)$ & 13 & 1 & YES & YES & YES & $2.00$ & $(2,4)$ & NO & 13757\\
$(728,137)$ & 17 & $(2,1)$ & 1 & 2 & YES & YES & NO(2) & $1.89$ & $(10,0)$ & -- & 13758\\
$(728,137)$ & 17 & $(2,1)$ & 1 & 2 & YES & YES & NO(2) & $2.00$ & $(10,0)$ & NO & 13759\\
$(729,215)$ & 15 & $(4,1)$ & 3 & 1 & YES & YES & NO(2) & $2.00$ & $(4,3)$ & NO & 13760\\
$(733,307)$ & 14 & $(3,1)$ & 2 & 1 & YES & YES & YES & $2.17$ & $(2,4)$ & -- & 13761\\
$(734,303)$ & 14 & $(46,19)$ & 8 & 2 & YES & YES & YES & $2.17$ & $(2,4)$ & NO & 13762\\
$(734,303)$ & 14 & $(172,71)$ & 11 & 2 & YES & YES & YES & $2.14$ & $(2,4)$ & 13672 & 13763\\
$(736,219)$ & 15 & $(326,97)$ & 13 & 2 & YES & YES & YES & $2.17$ & $(2,4)$ & 13837 & 13764\\
$(737,311)$ & 14 & $(173,73)$ & 11 & 1 & YES & YES & YES & $2.00$ & $(2,4)$ & 13675 & 13765\\
$(737,311)$ & 14 & $(737,311)$ & 14 & 737 & YES & YES & YES & $2.00$ & $(2,4)$ & NO & 13766\\
$(740,321)$ & 15 & $(3,1)$ & 2 & 1 & YES & YES & YES & $2.29$ & $(2,4)$ & -- & 13767\\
$(741,175)$ & 16 & $(2,1)$ & 1 & 1 & YES & YES & NO(2) & $1.89$ & $(10,0)$ & -- & 13768\\
$(743,158)$ & 16 & $(3,1)$ & 2 & 1 & YES & YES & NO(2) & $2.00$ & $(2,4)$ & -- & 13769\\
$(745,283)$ & 14 & $(2,1)$ & 1 & 1 & YES & YES & NO(3) & $2.00$ & $(2,4)$ & -- & 13770\\
$(745,313)$ & 14 & $(5,1)$ & 4 & 5 & YES & YES & YES & $2.17$ & $(2,4)$ & -- & 13771\\
$(745,313)$ & 14 & $(457,192)$ & 13 & 1 & YES & YES & YES & $2.17$ & $(2,4)$ & NO & 13772\\
$(747,169)$ & 15 & $(75,17)$ & 10 & 3 & YES & YES & YES & $2.17$ & $(2,4)$ & NO & 13773\\
$(750,287)$ & 14 & $(21,8)$ & 6 & 3 & YES & YES & YES & $2.00$ & $(2,4)$ & NO & 13774\\
$(751,159)$ & 15 & $(2,1)$ & 1 & 1 & YES & YES & NO(2) & $1.75$ & $(8,1)$ & -- & 13775\\
$(751,159)$ & 15 & $(2,1)$ & 1 & 1 & YES & YES & NO(2) & $1.88$ & $(8,1)$ & NO & 13776\\
$(751,159)$ & 15 & $(4,1)$ & 3 & 1 & YES & YES & NO(2) & $1.75$ & $(8,1)$ & NO & 13777\\
$(752,287)$ & 14 & $(2,1)$ & 1 & 2 & YES & YES & YES & $2.12$ & $(2,4)$ & -- & 13778\\
$(752,287)$ & 14 & $(3,1)$ & 2 & 1 & YES & YES & YES & $2.00$ & $(2,4)$ & -- & 13779\\
$(752,287)$ & 14 & $(3,1)$ & 2 & 1 & YES & YES & YES & $2.12$ & $(2,4)$ & NO & 13780\\
$(752,163)$ & 15 & $(5,1)$ & 4 & 1 & YES & YES & NO(2) & $2.00$ & $(2,4)$ & NO & 13781\\
$(753,208)$ & 14 & $(105,29)$ & 10 & 3 & YES & YES & YES & $2.00$ & $(2,4)$ & NO & 13782\\
$(755,292)$ & 14 & $(2,1)$ & 1 & 1 & YES & YES & YES & $2.17$ & $(2,4)$ & -- & 13783\\
$(755,317)$ & 14 & $(3,1)$ & 2 & 1 & YES & YES & YES & $2.17$ & $(2,4)$ & -- & 13784\\
$(755,312)$ & 14 & $(12,5)$ & 5 & 1 & YES & YES & YES & $2.17$ & $(2,4)$ & NO & 13785\\
$(755,312)$ & 14 & $(29,12)$ & 7 & 1 & YES & YES & YES & $2.17$ & $(2,4)$ & NO & 13786\\
$(756,293)$ & 14 & $(31,12)$ & 7 & 1 & YES & YES & YES & $2.00$ & $(2,4)$ & NO & 13787\\
$(757,220)$ & 15 & $(24,7)$ & 7 & 1 & YES & YES & NO(2) & $1.88$ & $(8,1)$ & NO & 13788\\
$(757,311)$ & 15 & $(39,16)$ & 8 & 1 & YES & YES & YES & $2.25$ & $(2,4)$ & NO & 13789\\
$(765,292)$ & 14 & $(2,1)$ & 1 & 1 & NO & YES & NO(2) & $2.00$ & $(10,0)$ & -- & 13790\\
$(765,317)$ & 14 & $(7,3)$ & 4 & 1 & YES & YES & YES & $2.17$ & $(2,4)$ & NO & 13791\\
$(765,292)$ & 14 & $(317,121)$ & 12 & 1 & YES & YES & NO(3) & $2.00$ & $(2,4)$ & NO & 13792\\
$(773,236)$ & 15 & $(7,2)$ & 4 & 1 & YES & YES & YES & $2.17$ & $(2,4)$ & NO & 13793\\
$(775,227)$ & 15 & $(17,5)$ & 6 & 1 & YES & YES & YES & $2.12$ & $(2,4)$ & NO & 13794\\
$(781,215)$ & 15 & $(2,1)$ & 1 & 1 & YES & YES & YES & $2.00$ & $(2,4)$ & -- & 13795\\
$(781,215)$ & 15 & $(2,1)$ & 1 & 1 & YES & YES & YES & $2.14$ & $(2,4)$ & NO & 13796\\
$(782,297)$ & 14 & $(2,1)$ & 1 & 2 & YES & YES & YES & $2.00$ & $(2,4)$ & -- & 13797\\
$(782,297)$ & 14 & $(21,8)$ & 6 & 1 & YES & YES & YES & $2.00$ & $(2,4)$ & NO & 13798\\
$(782,229)$ & 15 & $(24,7)$ & 7 & 2 & YES & YES & YES & $2.14$ & $(2,4)$ & NO & 13799\\
$(788,241)$ & 15 & $(7,2)$ & 4 & 1 & YES & YES & YES & $2.17$ & $(2,4)$ & NO & 13800\\
$(790,231)$ & 15 & $(2,1)$ & 1 & 2 & YES & YES & YES & $2.17$ & $(2,4)$ & NO & 13801\\
$(790,233)$ & 15 & $(4,1)$ & 3 & 2 & YES & YES & YES & $2.12$ & $(2,4)$ & NO & 13802\\
$(791,219)$ & 15 & $(3,1)$ & 2 & 1 & YES & YES & YES & $2.00$ & $(2,4)$ & -- & 13803\\
$(792,307)$ & 14 & $(2,1)$ & 1 & 2 & YES & YES & YES & $2.00$ & $(2,4)$ & -- & 13804\\
$(793,230)$ & 16 & $(3,1)$ & 2 & 1 & YES & YES & NO(2) & $2.00$ & $(4,3)$ & NO & 13805\\
$(793,335)$ & 14 & $(19,8)$ & 6 & 1 & YES & YES & YES & $2.14$ & $(2,4)$ & NO & 13806\\
$(795,308)$ & 14 & $(2,1)$ & 1 & 1 & YES & YES & YES & $2.14$ & $(2,4)$ & NO & 13807\\
$(795,308)$ & 14 & $(302,117)$ & 12 & 1 & YES & YES & YES & $2.00$ & $(2,4)$ & NO & 13808\\
$(799,180)$ & 16 & $(3,1)$ & 2 & 1 & YES & YES & NO(2) & $1.86$ & $(4,3)$ & -- & 13809\\
$(800,243)$ & 15 & $(7,2)$ & 4 & 1 & YES & YES & YES & $2.14$ & $(2,4)$ & NO & 13810\\
$(800,243)$ & 15 & $(33,10)$ & 8 & 1 & YES & YES & YES & $2.14$ & $(2,4)$ & NO & 13811\\
$(809,240)$ & 15 & $(2,1)$ & 1 & 1 & YES & YES & YES & $2.00$ & $(2,4)$ & -- & 13812\\
$(809,226)$ & 15 & $(3,1)$ & 2 & 1 & YES & YES & YES & $2.00$ & $(2,4)$ & -- & 13813\\
$(813,317)$ & 15 & $(2,1)$ & 1 & 1 & YES & YES & NO(2) & $2.09$ & $(6,2)$ & -- & 13814\\
$(815,298)$ & 15 & $(2,1)$ & 1 & 1 & YES & YES & YES & $2.17$ & $(4,3)$ & -- & 13815\\
$(815,298)$ & 15 & $(134,49)$ & 11 & 1 & YES & YES & YES & $2.17$ & $(4,3)$ & NO & 13816\\
$(820,313)$ & 14 & $(131,50)$ & 10 & 1 & YES & YES & YES & $2.00$ & $(2,4)$ & NO & 13817\\
$(822,341)$ & 15 & $(2,1)$ & 1 & 2 & YES & YES & NO(2) & $2.00$ & $(4,3)$ & NO & 13818\\
$(823,251)$ & 15 & $(3,1)$ & 2 & 1 & YES & YES & YES & $2.00$ & $(2,4)$ & -- & 13819\\
$(823,357)$ & 15 & $(5,1)$ & 4 & 1 & YES & YES & YES & $2.29$ & $(2,4)$ & -- & 13820\\
$(832,191)$ & 17 & $(4,1)$ & 3 & 4 & YES & YES & YES & $2.14$ & $(2,4)$ & -- & 13821\\
$(833,246)$ & 15 & $(2,1)$ & 1 & 1 & YES & YES & YES & $2.00$ & $(2,4)$ & -- & 13822\\
$(838,235)$ & 16 & $(649,182)$ & 15 & 1 & YES & YES & YES & $2.25$ & $(2,4)$ & NO & 13823\\
$(842,245)$ & 16 & $(5,1)$ & 4 & 1 & YES & YES & NO(2) & $2.00$ & $(4,3)$ & -- & 13824\\
$(842,257)$ & 15 & $(10,3)$ & 5 & 2 & YES & YES & YES & $2.17$ & $(2,4)$ & NO & 13825\\
$(842,245)$ & 16 & $(299,87)$ & 14 & 1 & YES & YES & NO(2) & $2.10$ & $(4,3)$ & NO & 13826\\
$(843,323)$ & 15 & $(2,1)$ & 1 & 1 & YES & YES & YES & $2.17$ & $(4,3)$ & NO & 13827\\
$(843,233)$ & 14 & $(5,2)$ & 3 & 1 & YES & YES & YES & $2.17$ & $(2,4)$ & -- & 13828\\
$(843,322)$ & 14 & $(5,1)$ & 4 & 1 & YES & YES & YES & $2.17$ & $(2,4)$ & NO & 13829\\
$(843,233)$ & 14 & $(10,3)$ & 5 & 1 & YES & YES & YES & $2.17$ & $(2,4)$ & NO & 13830\\
$(843,322)$ & 14 & $(377,144)$ & 12 & 1 & YES & YES & YES & $2.17$ & $(2,4)$ & 13881 & 13831\\
$(848,233)$ & 15 & $(4,1)$ & 3 & 4 & YES & YES & NO(2) & $2.00$ & $(2,4)$ & -- & 13832\\
$(850,237)$ & 15 & $(4,1)$ & 3 & 2 & YES & YES & YES & $2.00$ & $(2,4)$ & -- & 13833\\
$(852,181)$ & 16 & $(80,17)$ & 10 & 4 & YES & YES & NO(2) & $2.00$ & $(2,4)$ & NO & 13834\\
$(857,239)$ & 15 & $(11,3)$ & 5 & 1 & YES & YES & NO(3) & $2.00$ & $(2,4)$ & 13662 & 13835\\
$(857,239)$ & 15 & $(61,17)$ & 9 & 1 & YES & YES & NO(3) & $2.00$ & $(2,4)$ & 13634 & 13836\\
$(857,255)$ & 15 & $(205,61)$ & 12 & 1 & YES & YES & YES & $2.17$ & $(2,4)$ & 13764 & 13837\\
$(857,239)$ & 15 & $(857,239)$ & 15 & 857 & YES & YES & NO(3) & $2.00$ & $(2,4)$ & NO & 13838\\
$(858,239)$ & 15 & $(2,1)$ & 1 & 2 & YES & YES & YES & $2.00$ & $(2,4)$ & NO & 13839\\
$(858,239)$ & 15 & $(10,3)$ & 5 & 2 & YES & YES & YES & $2.17$ & $(2,4)$ & NO & 13840\\
$(858,335)$ & 15 & $(18,7)$ & 6 & 6 & YES & YES & NO(2) & $2.00$ & $(6,2)$ & NO & 13841\\
$(860,263)$ & 15 & $(4,1)$ & 3 & 4 & YES & YES & YES & $2.17$ & $(2,4)$ & NO & 13842\\
$(860,263)$ & 15 & $(860,263)$ & 15 & 860 & YES & YES & YES & $2.17$ & $(2,4)$ & NO & 13843\\
$(865,321)$ & 15 & $(2,1)$ & 1 & 1 & YES & YES & YES & $2.14$ & $(2,4)$ & NO & 13844\\
$(867,358)$ & 15 & $(2,1)$ & 1 & 1 & YES & YES & YES & $2.29$ & $(2,4)$ & -- & 13845\\
$(867,256)$ & 15 & $(7,2)$ & 4 & 1 & YES & YES & NO(3) & $2.00$ & $(2,4)$ & 13524 & 13846\\
$(867,358)$ & 15 & $(12,5)$ & 5 & 3 & YES & YES & YES & $2.29$ & $(2,4)$ & NO & 13847\\
$(867,358)$ & 15 & $(356,147)$ & 13 & 1 & YES & YES & YES & $2.29$ & $(2,4)$ & NO & 13848\\
$(868,265)$ & 15 & $(2,1)$ & 1 & 2 & YES & YES & NO(2) & $1.91$ & $(6,2)$ & -- & 13849\\
$(874,199)$ & 16 & $(4,1)$ & 3 & 2 & YES & YES & NO(2) & $2.00$ & $(4,3)$ & NO & 13850\\
$(880,267)$ & 15 & $(2,1)$ & 1 & 2 & YES & YES & YES & $2.14$ & $(2,4)$ & -- & 13851\\
$(880,199)$ & 16 & $(4,1)$ & 3 & 4 & YES & YES & NO(2) & $2.00$ & $(2,4)$ & -- & 13852\\
$(880,199)$ & 16 & $(84,19)$ & 10 & 4 & YES & YES & NO(2) & $2.00$ & $(2,4)$ & NO & 13853\\
$(893,246)$ & 15 & $(3,1)$ & 2 & 1 & YES & YES & NO(3) & $2.00$ & $(2,4)$ & NO & 13854\\
$(893,246)$ & 15 & $(5,2)$ & 3 & 1 & YES & YES & YES & $2.33$ & $(2,4)$ & -- & 13855\\
$(893,246)$ & 15 & $(236,65)$ & 12 & 1 & YES & YES & YES & $2.33$ & $(2,4)$ & NO & 13856\\
$(899,241)$ & 15 & $(7,2)$ & 4 & 1 & YES & YES & YES & $2.17$ & $(2,4)$ & NO & 13857\\
$(903,268)$ & 15 & $(91,27)$ & 10 & 7 & YES & YES & YES & $2.00$ & $(2,4)$ & NO & 13858\\
$(908,207)$ & 16 & $(6,1)$ & 5 & 2 & YES & YES & YES & $2.17$ & $(2,4)$ & NO & 13859\\
$(921,268)$ & 16 & $(5,1)$ & 4 & 1 & YES & YES & YES & $2.29$ & $(2,4)$ & -- & 13860\\
$(923,255)$ & 15 & $(18,5)$ & 6 & 1 & YES & YES & YES & $2.17$ & $(2,4)$ & NO & 13861\\
$(924,215)$ & 16 & $(17,4)$ & 7 & 1 & YES & YES & NO(2) & $1.88$ & $(4,3)$ & NO & 13862\\
$(927,256)$ & 15 & $(2,1)$ & 1 & 1 & YES & YES & YES & $2.17$ & $(2,4)$ & NO & 13863\\
$(927,383)$ & 15 & $(2,1)$ & 1 & 1 & YES & YES & YES & $2.29$ & $(2,4)$ & -- & 13864\\
$(927,383)$ & 15 & $(380,157)$ & 13 & 1 & YES & YES & YES & $2.29$ & $(2,4)$ & NO & 13865\\
$(928,353)$ & 15 & $(3,1)$ & 2 & 1 & YES & YES & NO(2) & $2.12$ & $(4,3)$ & NO & 13866\\
$(929,256)$ & 15 & $(2,1)$ & 1 & 1 & YES & YES & YES & $2.17$ & $(2,4)$ & -- & 13867\\
$(931,218)$ & 16 & $(13,3)$ & 6 & 1 & YES & YES & NO(2) & $1.91$ & $(6,2)$ & NO & 13868\\
$(931,218)$ & 16 & $(30,7)$ & 8 & 1 & YES & YES & NO(2) & $1.91$ & $(6,2)$ & NO & 13869\\
$(937,261)$ & 15 & $(2,1)$ & 1 & 1 & YES & YES & YES & $2.17$ & $(2,4)$ & NO & 13870\\
$(944,261)$ & 15 & $(2,1)$ & 1 & 2 & YES & YES & YES & $2.17$ & $(2,4)$ & NO & 13871\\
$(944,261)$ & 15 & $(18,5)$ & 6 & 2 & YES & YES & YES & $2.17$ & $(2,4)$ & NO & 13872\\
$(945,388)$ & 15 & $(4,1)$ & 3 & 1 & YES & YES & YES & $2.33$ & $(2,4)$ & -- & 13873\\
$(950,207)$ & 16 & $(2,1)$ & 1 & 2 & YES & YES & YES & $2.00$ & $(2,4)$ & -- & 13874\\
$(955,268)$ & 16 & $(702,197)$ & 15 & 1 & YES & YES & YES & $2.29$ & $(2,4)$ & NO & 13875\\
$(958,279)$ & 16 & $(3,1)$ & 2 & 1 & YES & YES & YES & $2.33$ & $(2,4)$ & -- & 13876\\
$(966,271)$ & 16 & $(2,1)$ & 1 & 2 & YES & YES & YES & $2.25$ & $(2,4)$ & NO & 13877\\
$(966,271)$ & 16 & $(25,7)$ & 7 & 1 & YES & YES & YES & $2.29$ & $(2,4)$ & NO & 13878\\
$(979,222)$ & 16 & $(3,1)$ & 2 & 1 & YES & YES & YES & $2.00$ & $(2,4)$ & NO & 13879\\
$(987,377)$ & 14 & $(2,1)$ & 1 & 1 & YES & YES & YES & $2.17$ & $(2,4)$ & -- & 13880\\
$(987,377)$ & 14 & $(233,89)$ & 11 & 1 & YES & YES & YES & $2.17$ & $(2,4)$ & 13831 & 13881\\
$(994,227)$ & 16 & $(3,1)$ & 2 & 1 & YES & YES & YES & $2.17$ & $(2,4)$ & NO & 13882\\
$(995,227)$ & 16 & $(5,1)$ & 4 & 5 & YES & YES & NO(2) & $1.88$ & $(4,3)$ & NO & 13883\\
$(996,215)$ & 16 & $(2,1)$ & 1 & 2 & YES & YES & YES & $2.17$ & $(2,4)$ & -- & 13884\\
$(997,295)$ & 15 & $(2,1)$ & 1 & 1 & YES & YES & YES & $2.17$ & $(2,4)$ & -- & 13885\\
$(997,295)$ & 15 & $(365,108)$ & 13 & 1 & YES & YES & YES & $2.17$ & $(2,4)$ & NO & 13886\\
$(1000,297)$ & 15 & $(2,1)$ & 1 & 2 & YES & YES & YES & $2.17$ & $(2,4)$ & -- & 13887\\
$(1013,299)$ & 15 & $(2,1)$ & 1 & 1 & YES & YES & YES & $2.00$ & $(2,4)$ & -- & 13888\\
$(1043,243)$ & 16 & $(2,1)$ & 1 & 1 & YES & YES & NO(2) & $2.00$ & $(2,4)$ & NO & 13889\\
$(1049,290)$ & 15 & $(3,1)$ & 2 & 1 & YES & YES & YES & $2.00$ & $(2,4)$ & -- & 13890\\
$(1053,308)$ & 15 & $(65,19)$ & 9 & 13 & YES & YES & YES & $2.00$ & $(2,4)$ & 13902 & 13891\\
$(1055,239)$ & 16 & $(3,1)$ & 2 & 1 & YES & YES & NO(3) & $2.00$ & $(2,4)$ & -- & 13892\\
$(1063,239)$ & 17 & $(4,1)$ & 3 & 1 & YES & YES & NO(2) & $2.00$ & $(6,2)$ & NO & 13893\\
$(1063,323)$ & 16 & $(102,31)$ & 11 & 1 & YES & YES & YES & $2.29$ & $(2,4)$ & NO & 13894\\
$(1072,317)$ & 15 & $(3,1)$ & 2 & 1 & YES & YES & YES & $2.17$ & $(2,4)$ & -- & 13895\\
$(1085,237)$ & 16 & $(2,1)$ & 1 & 1 & YES & YES & NO(3) & $2.00$ & $(2,4)$ & -- & 13896\\
$(1117,432)$ & 15 & $(287,111)$ & 12 & 1 & YES & YES & YES & $2.33$ & $(2,4)$ & NO & 13897\\
$(1131,313)$ & 15 & $(65,18)$ & 9 & 13 & YES & YES & YES & $2.17$ & $(2,4)$ & 13907 & 13898\\
$(1160,343)$ & 15 & $(3,1)$ & 2 & 1 & YES & YES & YES & $2.00$ & $(2,4)$ & -- & 13899\\
$(1160,343)$ & 15 & $(673,199)$ & 14 & 1 & YES & YES & YES & $2.00$ & $(2,4)$ & NO & 13900\\
$(1165,322)$ & 15 & $(76,21)$ & 9 & 1 & YES & YES & YES & $2.00$ & $(2,4)$ & NO & 13901\\
$(1173,343)$ & 15 & $(41,12)$ & 8 & 1 & YES & YES & YES & $2.00$ & $(2,4)$ & 13891 & 13902\\
$(1189,221)$ & 17 & $(3,1)$ & 2 & 1 & YES & YES & YES & $2.00$ & $(2,4)$ & -- & 13903\\
$(1190,213)$ & 17 & $(4,1)$ & 3 & 2 & YES & YES & YES & $2.17$ & $(2,4)$ & NO & 13904\\
$(1190,349)$ & 16 & $(4,1)$ & 3 & 2 & YES & YES & YES & $2.17$ & $(2,4)$ & -- & 13905\\
$(1216,259)$ & 17 & $(2,1)$ & 1 & 2 & YES & YES & YES & $2.14$ & $(2,4)$ & NO & 13906\\
$(1221,338)$ & 15 & $(47,13)$ & 8 & 1 & YES & YES & YES & $2.17$ & $(2,4)$ & 13898 & 13907\\
$(1253,349)$ & 16 & $(2,1)$ & 1 & 1 & YES & YES & YES & $2.17$ & $(2,4)$ & NO & 13908\\
$(1258,353)$ & 16 & $(11,3)$ & 5 & 1 & YES & YES & YES & $2.17$ & $(2,4)$ & NO & 13909\\
$(1283,298)$ & 17 & $(43,10)$ & 9 & 1 & YES & YES & YES & $2.25$ & $(2,4)$ & NO & 13910\\
$(1420,393)$ & 16 & $(271,75)$ & 12 & 1 & YES & YES & YES & $2.33$ & $(2,4)$ & NO & 13911\\
$(1453,278)$ & 18 & $(115,22)$ & 12 & 1 & YES & YES & NO(2) & $1.83$ & $(6,2)$ & NO & 13912\\
$(a;1,0,0;13)$ & 5 & $(49,18)$ & 8 & 1 & YES & YES & NO(2) & $1.75$ & $(4,3)$ & -- & 13913\\
$(a;1,0,0;13)$ & 5 & $(55,21)$ & 8 & 1 & YES & YES & NO(2) & $2.11$ & $(2,4)$ & -- & 13914\\
$(a;1,0,0;13)$ & 5 & $(106,31)$ & 10 & 1 & YES & YES & YES & $1.67$ & $(4,3)$ & -- & 13915\\
$(a;1,0,0;13)$ & 5 & $(115,44)$ & 10 & 1 & YES & YES & YES & $2.00$ & $(4,3)$ & -- & 13916\\
$(a;1,0,0;13)$ & 5 & $(206,47)$ & 12 & 1 & YES & YES & YES & $2.00$ & $(2,4)$ & -- & 13917\\
$(a;1,1,0;19)$ & 6 & $(73,27)$ & 9 & 1 & YES & YES & YES & $2.12$ & $(2,4)$ & -- & 13918\\
$(a;1,1,0;19)$ & 6 & $(78,17)$ & 10 & 1 & YES & YES & NO(2) & $2.09$ & $(2,4)$ & -- & 13919\\
$(a;1,1,0;19)$ & 6 & $(79,18)$ & 10 & 1 & YES & YES & NO(2) & $1.90$ & $(4,3)$ & -- & 13920\\
$(a;1,1,1;4)$ & 7 & $(46,19)$ & 8 & 2 & YES & YES & YES & $2.14$ & $(2,4)$ & -- & 13921\\
$(a;2,0,1;25)$ & 7 & $(47,18)$ & 8 & 1 & YES & YES & YES & $1.83$ & $(4,3)$ & -- & 13922\\
$(a;2,0,1;25)$ & 7 & $(74,31)$ & 9 & 1 & YES & YES & YES & $2.00$ & $(4,3)$ & -- & 13923\\
$(b;0,0,0;14)$ & 5 & $(74,31)$ & 9 & 2 & YES & YES & NO(2) & $2.00$ & $(10,0)$ & -- & 13924\\
$(b;0,0,0;14)$ & 5 & $(75,29)$ & 9 & 1 & YES & YES & NO(2) & $2.00$ & $(2,4)$ & -- & 13925\\
$(b;0,0,0;14)$ & 5 & $(89,34)$ & 9 & 1 & YES & YES & NO(2) & $1.86$ & $(4,3)$ & -- & 13926\\
$(b;0,0,0;14)$ & 5 & $(105,44)$ & 10 & 7 & YES & YES & NO(2) & $2.00$ & $(4,3)$ & -- & 13927\\
$(b;0,0,0;14)$ & 5 & $(115,44)$ & 10 & 1 & YES & YES & YES & $2.14$ & $(2,4)$ & -- & 13928\\
$(b;0,0,0;14)$ & 5 & $(121,46)$ & 10 & 1 & YES & YES & NO(2) & $2.10$ & $(4,3)$ & -- & 13929\\
$(b;0,0,1;4)$ & 6 & $(34,13)$ & 7 & 2 & YES & YES & NO(2) & $1.86$ & $(4,3)$ & -- & 13930\\
$(b;0,0,1;4)$ & 6 & $(39,17)$ & 8 & 1 & YES & YES & NO(2) & $2.00$ & $(2,4)$ & -- & 13931\\
$(b;0,0,1;4)$ & 6 & $(49,22)$ & 9 & 1 & YES & YES & NO(2) & $2.00$ & $(8,1)$ & -- & 13932\\
$(b;0,0,1;4)$ & 6 & $(80,31)$ & 9 & 4 & YES & YES & YES & $2.14$ & $(2,4)$ & -- & 13933\\
$(b;0,0,1;4)$ & 6 & $(100,31)$ & 11 & 4 & YES & YES & NO(2) & $2.00$ & $(4,3)$ & -- & 13934\\
$(b;0,0,1;4)$ & 6 & $(140,41)$ & 11 & 4 & YES & YES & YES & $2.33$ & $(2,4)$ & -- & 13935\\
$(b;0,0,2;26)$ & 7 & $(34,13)$ & 7 & 2 & YES & YES & NO(2) & $2.00$ & $(4,3)$ & -- & 13936\\
$(b;0,0,2;26)$ & 7 & $(37,11)$ & 8 & 1 & YES & YES & NO(2) & $1.88$ & $(4,3)$ & -- & 13937\\
$(b;0,0,2;26)$ & 7 & $(44,13)$ & 8 & 2 & YES & YES & NO(2) & $2.00$ & $(4,3)$ & -- & 13938\\
$(b;0,0,2;26)$ & 7 & $(58,17)$ & 9 & 2 & YES & YES & NO(2) & $1.82$ & $(6,2)$ & -- & 13939\\
$(b;0,1,0;19)$ & 6 & $(71,27)$ & 9 & 1 & YES & YES & NO(2) & $2.10$ & $(2,4)$ & -- & 13940\\
$(b;0,1,0;19)$ & 6 & $(75,29)$ & 9 & 1 & YES & YES & YES & $2.14$ & $(2,4)$ & -- & 13941\\
$(b;0,1,1;27)$ & 7 & $(24,7)$ & 7 & 3 & YES & YES & NO(2) & $2.00$ & $(4,3)$ & -- & 13942\\
$(b;0,1,1;27)$ & 7 & $(29,11)$ & 7 & 1 & YES & YES & NO(2) & $1.88$ & $(4,3)$ & -- & 13943\\
$(b;0,1,1;27)$ & 7 & $(37,11)$ & 8 & 1 & YES & YES & NO(2) & $1.88$ & $(4,3)$ & -- & 13944\\
$(b;0,1,1;27)$ & 7 & $(43,12)$ & 8 & 1 & YES & YES & NO(2) & $1.80$ & $(8,1)$ & -- & 13945\\
$(b;0,1,3;43)$ & 9 & $(23,9)$ & 7 & 1 & YES & YES & NO(2) & $2.00$ & $(6,2)$ & -- & 13946\\
$(b;0,2,0;8)$ & 7 & $(34,9)$ & 8 & 2 & YES & YES & YES & $1.86$ & $(2,4)$ & -- & 13947\\
$(b;0,2,0;8)$ & 7 & $(76,21)$ & 9 & 4 & YES & YES & YES & $2.14$ & $(2,4)$ & -- & 13948\\
$(b;0,3,0;29)$ & 8 & $(31,12)$ & 7 & 1 & YES & YES & NO(2) & $2.10$ & $(12,-1)$ & -- & 13949\\
$(b;0,3,0;29)$ & 8 & $(65,12)$ & 10 & 1 & YES & YES & NO(2) & $2.10$ & $(12,-1)$ & -- & 13950\\
$(b;0,3,2;53)$ & 10 & $(11,3)$ & 5 & 1 & YES & YES & NO(2) & $2.00$ & $(2,4)$ & -- & 13951\\
$(b;0,3,2;53)$ & 10 & $(13,5)$ & 5 & 1 & YES & YES & NO(2) & $1.88$ & $(8,1)$ & -- & 13952\\
$(b;1,0,0;5)$ & 6 & $(79,18)$ & 10 & 1 & YES & YES & NO(2) & $1.90$ & $(4,3)$ & -- & 13953\\
$(b;1,0,1;29)$ & 7 & $(18,7)$ & 6 & 1 & YES & YES & NO(2) & $1.90$ & $(4,3)$ & -- & 13954\\
$(b;1,0,1;29)$ & 7 & $(40,9)$ & 9 & 1 & YES & YES & NO(3) & $1.67$ & $(4,3)$ & -- & 13955\\
$(b;1,0,1;29)$ & 7 & $(41,17)$ & 8 & 1 & YES & YES & NO(2) & $2.00$ & $(4,3)$ & -- & 13956\\
$(b;1,0,2;19)$ & 8 & $(12,5)$ & 5 & 1 & YES & YES & NO(2) & $1.75$ & $(4,3)$ & -- & 13957\\
$(b;1,0,2;19)$ & 8 & $(15,4)$ & 6 & 1 & YES & YES & NO(2) & $2.00$ & $(4,3)$ & -- & 13958\\
$(b;1,0,2;19)$ & 8 & $(18,7)$ & 6 & 1 & YES & YES & NO(2) & $2.00$ & $(8,1)$ & -- & 13959\\
$(b;1,1,1;39)$ & 8 & $(23,10)$ & 7 & 1 & YES & YES & NO(2) & $2.00$ & $(4,3)$ & -- & 13960\\
$(b;1,2,0;17)$ & 8 & $(25,11)$ & 7 & 1 & YES & YES & NO(2) & $2.00$ & $(6,2)$ & -- & 13961\\
$(b;1,2,0;17)$ & 8 & $(39,11)$ & 9 & 1 & YES & YES & NO(2) & $2.09$ & $(6,2)$ & -- & 13962\\
$(c;0,0,0;4)$ & 4 & $(151,62)$ & 11 & 1 & YES & YES & NO(2) & $2.00$ & $(4,3)$ & -- & 13963\\
$(c;0,0,0;4)$ & 4 & $(169,70)$ & 11 & 1 & YES & YES & NO(2) & $1.86$ & $(4,3)$ & -- & 13964\\
$(c;0,0,0;4)$ & 4 & $(173,64)$ & 11 & 1 & YES & YES & NO(2) & $1.90$ & $(4,3)$ & -- & 13965\\
$(c;0,0,0;4)$ & 4 & $(181,70)$ & 11 & 1 & YES & YES & NO(2) & $1.86$ & $(4,3)$ & -- & 13966\\
$(c;0,0,0;4)$ & 4 & $(191,74)$ & 11 & 1 & YES & YES & YES & $2.14$ & $(2,4)$ & -- & 13967\\
$(c;0,0,0;4)$ & 4 & $(208,79)$ & 11 & 4 & YES & YES & YES & $2.00$ & $(6,2)$ & -- & 13968\\
$(c;0,0,0;4)$ & 4 & $(219,65)$ & 12 & 1 & YES & YES & YES & $2.00$ & $(2,4)$ & -- & 13969\\
$(c;0,0,0;4)$ & 4 & $(223,86)$ & 13 & 1 & YES & YES & NO(2) & $2.10$ & $(8,1)$ & -- & 13970\\
$(c;0,0,0;4)$ & 4 & $(225,98)$ & 12 & 1 & YES & YES & NO(2) & $1.89$ & $(6,2)$ & -- & 13971\\
$(c;0,0,0;4)$ & 4 & $(233,89)$ & 11 & 1 & YES & YES & YES & $2.17$ & $(2,4)$ & -- & 13972\\
$(c;0,0,0;4)$ & 4 & $(243,89)$ & 12 & 1 & YES & YES & NO(2) & $2.00$ & $(6,2)$ & -- & 13973\\
$(c;0,0,0;4)$ & 4 & $(244,55)$ & 13 & 4 & YES & YES & NO(2) & $2.00$ & $(4,3)$ & -- & 13974\\
$(c;0,0,0;4)$ & 4 & $(251,104)$ & 12 & 1 & YES & YES & YES & $2.14$ & $(2,4)$ & -- & 13975\\
$(c;0,0,0;4)$ & 4 & $(274,81)$ & 12 & 2 & YES & YES & NO(2) & $2.00$ & $(6,2)$ & -- & 13976\\
$(c;0,0,0;4)$ & 4 & $(291,85)$ & 13 & 1 & YES & YES & YES & $2.14$ & $(2,4)$ & -- & 13977\\
$(c;0,0,0;4)$ & 4 & $(321,98)$ & 13 & 1 & YES & YES & YES & $2.17$ & $(2,4)$ & -- & 13978\\
$(c;0,0,0;4)$ & 4 & $(343,131)$ & 12 & 1 & YES & YES & YES & $2.17$ & $(2,4)$ & -- & 13979\\
$(c;0,0,0;4)$ & 4 & $(373,100)$ & 13 & 1 & YES & YES & YES & $2.17$ & $(2,4)$ & -- & 13980\\
$(c;0,0,0;4)$ & 4 & $(401,111)$ & 13 & 1 & YES & YES & YES & $2.00$ & $(2,4)$ & -- & 13981\\
$(c;0,0,0;4)$ & 4 & $(434,121)$ & 13 & 2 & YES & YES & YES & $2.17$ & $(2,4)$ & -- & 13982\\
$(c;0,0,0;4)$ & 4 & $(443,131)$ & 13 & 1 & YES & YES & YES & $2.00$ & $(2,4)$ & -- & 13983\\
$(c;0,0,0;4)$ & 4 & $(448,131)$ & 13 & 4 & YES & YES & YES & $2.00$ & $(2,4)$ & -- & 13984\\
$(c;0,1,0;11)$ & 5 & $(81,34)$ & 9 & 1 & YES & YES & NO(2) & $2.00$ & $(2,4)$ & -- & 13985\\
$(c;0,1,0;11)$ & 5 & $(89,34)$ & 9 & 1 & YES & YES & YES & $1.86$ & $(2,4)$ & -- & 13986\\
$(c;0,1,0;11)$ & 5 & $(99,41)$ & 10 & 11 & YES & YES & NO(2) & $2.00$ & $(2,4)$ & -- & 13987\\
$(c;0,1,0;11)$ & 5 & $(101,44)$ & 10 & 1 & YES & YES & NO(2) & $1.90$ & $(4,3)$ & -- & 13988\\
$(c;0,1,0;11)$ & 5 & $(105,44)$ & 10 & 1 & YES & YES & NO(2) & $2.00$ & $(6,2)$ & -- & 13989\\
$(c;0,1,0;11)$ & 5 & $(111,31)$ & 10 & 1 & YES & YES & NO(2) & $1.89$ & $(2,4)$ & -- & 13990\\
$(c;0,1,0;11)$ & 5 & $(111,41)$ & 10 & 1 & YES & YES & NO(2) & $1.71$ & $(4,3)$ & -- & 13991\\
$(c;0,1,0;11)$ & 5 & $(115,34)$ & 10 & 1 & YES & YES & NO(2) & $1.75$ & $(8,1)$ & -- & 13992\\
$(c;0,1,0;11)$ & 5 & $(123,34)$ & 10 & 1 & YES & YES & NO(2) & $1.71$ & $(10,0)$ & -- & 13993\\
$(c;0,1,0;11)$ & 5 & $(127,34)$ & 11 & 1 & YES & YES & NO(2) & $1.90$ & $(4,3)$ & -- & 13994\\
$(c;0,1,0;11)$ & 5 & $(129,56)$ & 11 & 1 & YES & YES & YES & $1.83$ & $(4,3)$ & -- & 13995\\
$(c;0,1,0;11)$ & 5 & $(131,50)$ & 10 & 1 & YES & YES & YES & $2.00$ & $(2,4)$ & -- & 13996\\
$(c;0,1,0;11)$ & 5 & $(149,44)$ & 11 & 1 & YES & YES & NO(2) & $2.00$ & $(2,4)$ & -- & 13997\\
$(c;0,1,0;11)$ & 5 & $(149,65)$ & 11 & 1 & YES & YES & NO(2) & $2.00$ & $(4,3)$ & -- & 13998\\
$(c;0,1,0;11)$ & 5 & $(154,43)$ & 11 & 11 & YES & YES & YES & $2.14$ & $(2,4)$ & -- & 13999\\
$(c;0,1,0;11)$ & 5 & $(171,47)$ & 12 & 1 & YES & YES & NO(2) & $1.89$ & $(10,0)$ & -- & 14000\\
$(c;0,1,0;11)$ & 5 & $(194,75)$ & 11 & 1 & YES & YES & NO(3) & $2.00$ & $(2,4)$ & -- & 14001\\
$(c;0,1,0;11)$ & 5 & $(195,76)$ & 12 & 1 & YES & YES & YES & $2.14$ & $(4,3)$ & -- & 14002\\
$(c;0,1,0;11)$ & 5 & $(236,69)$ & 12 & 1 & YES & YES & NO(2) & $2.00$ & $(4,3)$ & -- & 14003\\
$(c;0,1,0;11)$ & 5 & $(249,47)$ & 14 & 1 & YES & YES & YES & $2.14$ & $(2,4)$ & -- & 14004\\
$(c;0,1,0;11)$ & 5 & $(265,74)$ & 12 & 1 & YES & YES & YES & $2.14$ & $(2,4)$ & -- & 14005\\
$(c;0,1,0;11)$ & 5 & $(271,80)$ & 12 & 1 & YES & YES & YES & $2.00$ & $(2,4)$ & -- & 14006\\
$(c;0,1,0;11)$ & 5 & $(274,81)$ & 12 & 1 & YES & YES & YES & $2.00$ & $(2,4)$ & -- & 14007\\
$(c;0,1,1;5)$ & 6 & $(69,29)$ & 9 & 1 & YES & YES & NO(2) & $2.00$ & $(4,3)$ & -- & 14008\\
$(c;0,1,1;5)$ & 6 & $(101,39)$ & 10 & 1 & YES & YES & YES & $2.12$ & $(2,4)$ & -- & 14009\\
$(c;0,1,1;5)$ & 6 & $(106,41)$ & 10 & 1 & YES & YES & NO(2) & $2.10$ & $(2,4)$ & -- & 14010\\
$(c;0,2,1;19)$ & 7 & $(70,29)$ & 9 & 1 & YES & YES & YES & $2.17$ & $(4,3)$ & -- & 14011\\
$(d;0,0,0;5)$ & 5 & $(71,16)$ & 10 & 1 & YES & YES & NO(2) & $2.00$ & $(4,3)$ & -- & 14012\\
$(d;0,0,0;5)$ & 5 & $(85,26)$ & 10 & 5 & YES & YES & NO(2) & $1.91$ & $(2,4)$ & -- & 14013\\
$(d;0,0,0;5)$ & 5 & $(85,36)$ & 10 & 5 & YES & YES & NO(2) & $2.00$ & $(2,4)$ & -- & 14014\\
$(d;0,0,0;5)$ & 5 & $(89,34)$ & 9 & 1 & YES & YES & NO(2) & $2.00$ & $(8,1)$ & -- & 14015\\
$(d;0,0,0;5)$ & 5 & $(92,39)$ & 10 & 1 & YES & YES & NO(2) & $1.90$ & $(4,3)$ & -- & 14016\\
$(d;0,0,0;5)$ & 5 & $(105,44)$ & 10 & 5 & YES & YES & NO(2) & $2.00$ & $(2,4)$ & -- & 14017\\
$(d;0,0,0;5)$ & 5 & $(106,41)$ & 10 & 1 & YES & YES & NO(2) & $2.00$ & $(4,3)$ & -- & 14018\\
$(d;0,0,0;5)$ & 5 & $(113,48)$ & 11 & 1 & YES & YES & NO(2) & $2.11$ & $(6,2)$ & -- & 14019\\
$(d;0,0,0;5)$ & 5 & $(129,49)$ & 10 & 1 & YES & YES & NO(3) & $1.86$ & $(2,4)$ & -- & 14020\\
$(d;0,0,0;5)$ & 5 & $(129,56)$ & 11 & 1 & YES & YES & YES & $2.00$ & $(4,3)$ & -- & 14021\\
$(d;0,0,0;5)$ & 5 & $(147,62)$ & 11 & 1 & YES & YES & YES & $2.14$ & $(2,4)$ & -- & 14022\\
$(d;0,0,0;5)$ & 5 & $(154,43)$ & 11 & 1 & YES & YES & YES & $2.14$ & $(2,4)$ & -- & 14023\\
$(d;0,0,0;5)$ & 5 & $(162,35)$ & 12 & 1 & YES & YES & NO(2) & $1.88$ & $(2,4)$ & -- & 14024\\
$(d;0,0,0;5)$ & 5 & $(163,43)$ & 12 & 1 & YES & YES & NO(2) & $2.11$ & $(6,2)$ & -- & 14025\\
$(d;0,0,0;5)$ & 5 & $(165,49)$ & 11 & 5 & YES & YES & NO(3) & $1.86$ & $(2,4)$ & -- & 14026\\
$(d;0,0,0;5)$ & 5 & $(169,40)$ & 13 & 1 & YES & YES & YES & $2.00$ & $(4,3)$ & -- & 14027\\
$(d;0,0,0;5)$ & 5 & $(185,56)$ & 12 & 5 & YES & YES & YES & $2.14$ & $(2,4)$ & -- & 14028\\
$(d;0,0,3;22)$ & 8 & $(40,11)$ & 8 & 2 & YES & YES & NO(2) & $1.90$ & $(4,3)$ & -- & 14029\\
$(d;0,0,3;22)$ & 8 & $(60,23)$ & 9 & 2 & YES & YES & NO(2) & $2.00$ & $(4,3)$ & -- & 14030\\
$(d;0,0,3;22)$ & 8 & $(65,17)$ & 10 & 1 & YES & YES & NO(2) & $2.00$ & $(8,1)$ & -- & 14031\\
$(d;0,1,1;17)$ & 7 & $(115,44)$ & 10 & 1 & YES & YES & YES & $2.33$ & $(2,4)$ & -- & 14032\\
$(d;0,2,1;20)$ & 8 & $(79,23)$ & 10 & 1 & YES & YES & NO(2) & $2.00$ & $(4,3)$ & -- & 14033\\
$(e;0,0,0;4)$ & 5 & $(65,24)$ & 9 & 1 & YES & YES & NO(2) & $2.00$ & $(4,3)$ & -- & 14034\\
$(e;0,0,0;4)$ & 5 & $(81,31)$ & 9 & 1 & YES & YES & NO(2) & $1.88$ & $(12,-1)$ & -- & 14035\\
$(e;0,3,0;7)$ & 8 & $(70,13)$ & 10 & 7 & YES & YES & NO(2) & $2.10$ & $(4,3)$ & -- & 14036\\
$(e;1,0,0;18)$ & 6 & $(39,16)$ & 8 & 3 & YES & YES & YES & $1.67$ & $(4,3)$ & -- & 14037\\
$(e;1,0,0;18)$ & 6 & $(40,11)$ & 8 & 2 & YES & YES & NO(2) & $2.00$ & $(8,1)$ & -- & 14038\\
$(e;1,0,0;18)$ & 6 & $(46,19)$ & 8 & 2 & YES & YES & YES & $1.86$ & $(2,4)$ & -- & 14039\\
$(e;1,1,0;23)$ & 7 & $(24,7)$ & 7 & 1 & YES & YES & NO(2) & $2.09$ & $(6,2)$ & -- & 14040\\
$(e;1,1,0;23)$ & 7 & $(27,8)$ & 7 & 1 & YES & YES & NO(2) & $1.91$ & $(6,2)$ & -- & 14041\\
$(e;1,1,0;23)$ & 7 & $(37,11)$ & 8 & 1 & YES & YES & NO(2) & $1.88$ & $(4,3)$ & -- & 14042\\
$(e;1,2,0;28)$ & 8 & $(34,13)$ & 7 & 2 & YES & YES & NO(2) & $2.11$ & $(2,4)$ & -- & 14043\\
$(e;2,0,0;24)$ & 7 & $(17,7)$ & 6 & 1 & YES & YES & NO(2) & $1.75$ & $(4,3)$ & -- & 14044\\
$(e;2,0,0;24)$ & 7 & $(27,10)$ & 7 & 3 & YES & YES & NO(2) & $2.00$ & $(4,3)$ & -- & 14045\\
$(e;2,0,0;24)$ & 7 & $(31,13)$ & 7 & 1 & YES & YES & NO(2) & $2.00$ & $(10,0)$ & -- & 14046\\
$(e;2,0,0;24)$ & 7 & $(37,11)$ & 8 & 1 & YES & YES & NO(2) & $1.88$ & $(4,3)$ & -- & 14047\\
$(e;2,0,0;24)$ & 7 & $(41,12)$ & 8 & 1 & YES & YES & YES & $2.17$ & $(4,3)$ & -- & 14048\\
$(f;0,0,0;6)$ & 4 & $(157,58)$ & 11 & 1 & YES & YES & NO(2) & $1.88$ & $(4,3)$ & -- & 14049\\
$(f;0,0,0;6)$ & 4 & $(186,71)$ & 11 & 6 & YES & YES & NO(2) & $2.00$ & $(4,3)$ & -- & 14050\\
$(f;0,0,0;6)$ & 4 & $(257,76)$ & 12 & 1 & YES & YES & NO(2) & $2.11$ & $(2,4)$ & -- & 14051\\
$(f;0,0,0;6)$ & 4 & $(324,95)$ & 13 & 6 & YES & YES & YES & $2.12$ & $(2,4)$ & -- & 14052\\
$(g;0,0,0;19)$ & 6 & $(41,17)$ & 8 & 1 & YES & YES & NO(2) & $2.00$ & $(2,4)$ & -- & 14053\\
$(g;0,0,0;19)$ & 6 & $(44,17)$ & 8 & 1 & YES & YES & NO(2) & $2.00$ & $(2,4)$ & -- & 14054\\
$(g;0,0,0;19)$ & 6 & $(50,19)$ & 8 & 1 & YES & YES & YES & $2.00$ & $(2,4)$ & -- & 14055\\
$(g;0,0,0;19)$ & 6 & $(69,29)$ & 9 & 1 & YES & YES & YES & $2.12$ & $(2,4)$ & -- & 14056\\
$(g;0,0,1;26)$ & 7 & $(44,17)$ & 8 & 2 & YES & YES & NO(2) & $2.00$ & $(6,2)$ & -- & 14057\\
$(g;0,0,1;26)$ & 7 & $(47,18)$ & 8 & 1 & YES & YES & YES & $2.12$ & $(2,4)$ & -- & 14058\\
$(g;0,0,1;26)$ & 7 & $(50,21)$ & 8 & 2 & YES & YES & YES & $2.17$ & $(2,4)$ & -- & 14059\\
$(g;0,1,1;33)$ & 8 & $(19,7)$ & 6 & 1 & YES & YES & NO(2) & $1.86$ & $(4,3)$ & -- & 14060\\
$(g;1,0,1;38)$ & 8 & $(18,7)$ & 6 & 2 & YES & YES & NO(2) & $2.00$ & $(6,2)$ & -- & 14061\\
$(g;1,0,1;38)$ & 8 & $(23,10)$ & 7 & 1 & YES & YES & NO(2) & $2.00$ & $(6,2)$ & -- & 14062\\
$(g;1,0,2;24)$ & 9 & $(13,5)$ & 5 & 1 & YES & YES & NO(2) & $2.00$ & $(2,4)$ & -- & 14063\\
$(g;1,2,0;11)$ & 9 & $(17,5)$ & 6 & 1 & YES & YES & NO(2) & $1.88$ & $(6,2)$ & -- & 14064\\
$(g;3,0,0;23)$ & 9 & $(13,5)$ & 5 & 1 & YES & YES & NO(2) & $2.00$ & $(6,2)$ & -- & 14065\\
$(h;0,0,0;6)$ & 5 & $(34,13)$ & 7 & 2 & YES & YES & YES & $1.71$ & $(2,4)$ & -- & 14066\\
$(h;0,0,0;6)$ & 5 & $(57,22)$ & 9 & 3 & YES & YES & NO(2) & $2.00$ & $(2,4)$ & -- & 14067\\
$(h;0,0,0;6)$ & 5 & $(74,31)$ & 9 & 2 & YES & YES & NO(2) & $2.11$ & $(2,4)$ & -- & 14068\\
$(h;0,0,0;6)$ & 5 & $(79,23)$ & 10 & 1 & YES & YES & NO(2) & $1.86$ & $(4,3)$ & -- & 14069\\
$(h;0,0,0;6)$ & 5 & $(109,46)$ & 10 & 1 & YES & YES & NO(2) & $2.00$ & $(2,4)$ & -- & 14070\\
$(h;0,0,0;6)$ & 5 & $(169,70)$ & 11 & 1 & YES & YES & YES & $2.17$ & $(2,4)$ & -- & 14071\\
$(h;0,0,0;6)$ & 5 & $(171,50)$ & 11 & 3 & YES & YES & YES & $2.00$ & $(2,4)$ & -- & 14072\\
$(h;0,1,0;8)$ & 6 & $(19,7)$ & 6 & 1 & YES & YES & NO(2) & $1.67$ & $(6,2)$ & -- & 14073\\
$(h;0,1,0;8)$ & 6 & $(50,19)$ & 8 & 2 & YES & YES & YES & $2.00$ & $(2,4)$ & -- & 14074\\
$(h;0,2,0;10)$ & 7 & $(31,9)$ & 8 & 1 & YES & YES & YES & $1.86$ & $(2,4)$ & -- & 14075\\
$(j;0,0,0;8)$ & 5 & $(141,43)$ & 11 & 1 & YES & YES & NO(2) & $1.88$ & $(12,-1)$ & -- & 14076\\
$(j;0,0,0;8)$ & 5 & $(163,63)$ & 11 & 1 & YES & YES & YES & $2.00$ & $(4,3)$ & -- & 14077\\
$(j;0,0,0;8)$ & 5 & $(207,85)$ & 12 & 1 & YES & YES & YES & $2.25$ & $(2,4)$ & -- & 14078\\
$(j;0,1,0;10)$ & 6 & $(65,24)$ & 9 & 5 & YES & YES & NO(2) & $1.90$ & $(4,3)$ & -- & 14079\\
$(j;0,1,0;10)$ & 6 & $(134,49)$ & 11 & 2 & YES & YES & NO(2) & $2.00$ & $(6,2)$ & -- & 14080
\end{longtable}
\subsection{2 chains, $K^2 = 5$}
\begin{longtable}{|c|c|c|c|c|c|c|c|c|c|c|c|}
\hline
\multicolumn{12}{|c|}{2 chains, $K^2 = 5$}\\
\hline
$(n,a)$ & Len & $(n,a)$ & Len & GCD & Nef & $\mathbb Q$-ef & Obs 0 & $\overline c_1^2 / \overline c_2$ & $(P,K)$ & WH & Index\\
\hline
\endfirsthead

\hline
$(n,a)$ & Len & $(n,a)$ & Len & GCD & Nef & $\mathbb Q$-ef & Obs 0 & $\overline c_1^2 / \overline c_2$ & $(P,K)$ & WH & Index\\
\hline
\endhead
\hline
\endfoot

$(75,29)$ & 9 & $(44,17)$ & 8 & 1 & YES & YES & NO(2) & $2.38$ & $(4,4)$ & -- & 14081\\
$(82,25)$ & 10 & $(59,18)$ & 9 & 1 & YES & YES & NO(2) & $2.33$ & $(6,3)$ & -- & 14082\\
$(89,34)$ & 9 & $(26,11)$ & 7 & 1 & YES & YES & NO(2) & $2.30$ & $(2,5)$ & -- & 14083\\
$(89,34)$ & 9 & $(37,11)$ & 8 & 1 & YES & YES & NO(2) & $2.22$ & $(4,4)$ & -- & 14084\\
$(111,31)$ & 10 & $(26,11)$ & 7 & 1 & YES & YES & NO(2) & $2.30$ & $(2,5)$ & -- & 14085\\
$(111,31)$ & 10 & $(37,11)$ & 8 & 37 & YES & YES & NO(2) & $2.22$ & $(4,4)$ & -- & 14086\\
$(256,75)$ & 12 & $(17,5)$ & 6 & 1 & YES & YES & NO(2) & $2.33$ & $(2,5)$ & -- & 14087\\
$(700,257)$ & 14 & $(3,1)$ & 2 & 1 & YES & YES & NO(3) & $2.33$ & $(4,4)$ & -- & 14088\\
$(700,257)$ & 14 & $(493,181)$ & 13 & 1 & YES & YES & NO(2) & $2.33$ & $(6,3)$ & NO & 14089\\
$(700,257)$ & 14 & $(700,257)$ & 14 & 700 & YES & YES & NO(3) & $2.33$ & $(4,4)$ & NO & 14090\\
$(a;1,1,0;19)$ & 6 & $(89,34)$ & 9 & 1 & YES & YES & NO(2) & $2.30$ & $(2,5)$ & -- & 14091\\
$(a;1,1,0;19)$ & 6 & $(111,31)$ & 10 & 1 & YES & YES & NO(2) & $2.30$ & $(2,5)$ & -- & 14092\\
$(b;0,0,1;4)$ & 6 & $(74,31)$ & 9 & 2 & YES & YES & NO(3) & $2.29$ & $(2,5)$ & -- & 14093\\
$(b;0,0,1;4)$ & 6 & $(89,34)$ & 9 & 1 & YES & YES & NO(2) & $2.22$ & $(2,5)$ & -- & 14094\\
$(b;0,0,1;4)$ & 6 & $(106,31)$ & 10 & 2 & YES & YES & NO(2) & $2.33$ & $(2,5)$ & -- & 14095\\
$(b;0,0,1;4)$ & 6 & $(111,31)$ & 10 & 1 & YES & YES & NO(2) & $2.22$ & $(2,5)$ & -- & 14096\\
$(d;0,1,0;6)$ & 6 & $(169,50)$ & 11 & 1 & YES & YES & NO(2) & $2.20$ & $(6,3)$ & -- & 14097\\
$(e;1,0,0;18)$ & 6 & $(75,29)$ & 9 & 3 & YES & YES & NO(2) & $2.29$ & $(6,3)$ & -- & 14098\\
$(f;0,0,0;6)$ & 4 & $(436,129)$ & 13 & 2 & YES & YES & NO(2) & $2.30$ & $(6,3)$ & -- & 14099
\end{longtable}





%%%%%%%%%%%%%%%%%%%%%%%%%%%%%%%%%%%%%%%%%%%
\section{$4I_3$}

Hesse configuration. Let $\zeta$ be a primitive third root of unity. Base curves:
\begin{itemize}
  \item $L_x = x$.
  \item $L_y = y$.
  \item $L_z = z$.
  \item $L_{i,j} = X + \zeta^i Y + \zeta^j z$
\end{itemize}
Fibration given by pencil
\[F_\lambda = L_x L_y L_z + \lambda L_{0,1} L_{1,0} L_{2,2}\]

Singular fibers are as follows:
\begin{itemize}
  \item $I_3$ fiber given by $L_x$, $L_y$, $L_z$.
  \item $I_3$ fiber given by $L_{0,1}$, $L_{1,0}$, $L_{2,2}$.
  \item $I_3$ fiber given by $L_{0,2}$, $L_{1,1}$, $L_{2,0}$.
  \item $I_3$ fiber given by $L_{0,0}$, $L_{1,2}$, $L_{2,1}$.
\end{itemize}
Special curves:

Input:
\lstinputlisting[language=config]{../Tests/3333.txt}
Result:
%\usepackage{longtable}
\subsection{1 chain, $K^2 = 3$}
\begin{longtable}{|c|c|c|c|c|c|c|c|}
\hline
\multicolumn{8}{|c|}{1 chain, $K^2 = 3$}\\
\hline
$(n,a)$ & Len & Nef & $\mathbb Q$-ef & Obs 0 & $\overline c_1^2 / \overline c_2$ & $(P,K)$ & Index\\
\hline
\endfirsthead

\hline
$(n,a)$ & Len & Nef & $\mathbb Q$-ef & Obs 0 & $\overline c_1^2 / \overline c_2$ & $(P,K)$ & Index\\
\hline
\endhead
\hline
\endfoot

$(193,81)$ & 11 & YES & YES & YES & $1.79$ & $(1,3)$ & 1\\
$(208,79)$ & 11 & YES & YES & YES & $1.73$ & $(3,2)$ & 2\\
$(212,81)$ & 11 & YES & YES & NO(2) & $1.82$ & $(1,3)$ & 3\\
$(233,89)$ & 11 & YES & YES & YES & $1.40$ & $(5,1)$ & 4\\
$(234,71)$ & 12 & YES & YES & YES & $1.71$ & $(1,3)$ & 5\\
$(242,65)$ & 12 & YES & YES & YES & $1.79$ & $(1,3)$ & 6\\
$(256,75)$ & 12 & YES & YES & NO(2) & $1.76$ & $(1,3)$ & 7\\
$(256,99)$ & 12 & YES & YES & YES & $1.67$ & $(1,3)$ & 8\\
$(257,76)$ & 12 & YES & YES & YES & $1.67$ & $(1,3)$ & 9\\
$(263,71)$ & 12 & YES & YES & YES & $1.79$ & $(1,3)$ & 10\\
$(269,104)$ & 12 & YES & YES & YES & $1.67$ & $(1,3)$ & 11\\
$(277,60)$ & 13 & YES & YES & YES & $1.79$ & $(1,3)$ & 12\\
$(277,81)$ & 12 & YES & YES & YES & $1.75$ & $(1,3)$ & 13\\
$(277,106)$ & 12 & YES & YES & YES & $1.55$ & $(3,2)$ & 14\\
$(281,109)$ & 12 & YES & YES & YES & $1.50$ & $(5,1)$ & 15\\
$(283,108)$ & 12 & YES & YES & YES & $1.64$ & $(3,2)$ & 16\\
$(287,79)$ & 12 & YES & YES & YES & $1.67$ & $(1,3)$ & 17\\
$(287,111)$ & 12 & YES & YES & YES & $1.40$ & $(5,1)$ & 18\\
$(287,109)$ & 12 & YES & YES & YES & $1.64$ & $(3,2)$ & 19\\
$(290,81)$ & 12 & YES & YES & YES & $1.50$ & $(5,1)$ & 20\\
$(292,121)$ & 12 & YES & YES & YES & $1.25$ & $(5,1)$ & 21\\
$(301,89)$ & 12 & YES & YES & YES & $1.50$ & $(1,3)$ & 22\\
$(301,115)$ & 12 & YES & YES & YES & $1.64$ & $(3,2)$ & 23\\
$(302,117)$ & 12 & YES & YES & YES & $1.64$ & $(3,2)$ & 24\\
$(303,116)$ & 12 & YES & YES & YES & $1.64$ & $(3,2)$ & 25\\
$(307,119)$ & 12 & YES & YES & YES & $1.44$ & $(3,2)$ & 26\\
$(313,121)$ & 12 & YES & YES & YES & $1.64$ & $(3,2)$ & 27\\
$(317,121)$ & 12 & YES & YES & YES & $1.55$ & $(3,2)$ & 28\\
$(322,89)$ & 12 & YES & YES & NO(2) & $1.64$ & $(3,2)$ & 29\\
$(333,76)$ & 13 & YES & YES & YES & $1.60$ & $(5,1)$ & 30\\
$(337,129)$ & 12 & YES & YES & YES & $1.44$ & $(3,2)$ & 31\\
$(338,129)$ & 12 & YES & YES & YES & $1.44$ & $(3,2)$ & 32\\
$(351,76)$ & 13 & YES & YES & YES & $1.75$ & $(1,3)$ & 33\\
$(359,105)$ & 13 & YES & YES & YES & $1.64$ & $(3,2)$ & 34\\
$(359,106)$ & 13 & YES & YES & YES & $1.55$ & $(3,2)$ & 35\\
$(363,98)$ & 13 & YES & YES & YES & $1.12$ & $(5,1)$ & 36\\
$(365,108)$ & 13 & YES & YES & YES & $1.56$ & $(3,2)$ & 37\\
$(372,109)$ & 13 & YES & YES & YES & $1.64$ & $(3,2)$ & 38\\
$(373,104)$ & 13 & YES & YES & YES & $1.58$ & $(1,3)$ & 39\\
$(373,109)$ & 13 & YES & YES & YES & $1.64$ & $(3,2)$ & 40\\
$(376,105)$ & 13 & YES & YES & YES & $1.58$ & $(1,3)$ & 41\\
$(379,111)$ & 13 & YES & YES & YES & $1.64$ & $(3,2)$ & 42\\
$(383,106)$ & 13 & YES & YES & YES & $1.45$ & $(3,2)$ & 43\\
$(383,112)$ & 13 & YES & YES & YES & $1.44$ & $(3,2)$ & 44\\
$(391,108)$ & 13 & YES & YES & YES & $1.44$ & $(3,2)$ & 45\\
$(393,116)$ & 13 & YES & YES & YES & $1.33$ & $(3,2)$ & 46\\
$(397,116)$ & 13 & YES & YES & YES & $1.22$ & $(3,2)$ & 47\\
$(398,111)$ & 13 & YES & YES & YES & $1.55$ & $(3,2)$ & 48\\
$(401,112)$ & 13 & YES & YES & YES & $1.44$ & $(3,2)$ & 49\\
$(403,92)$ & 14 & YES & YES & YES & $1.25$ & $(5,1)$ & 50\\
$(407,119)$ & 13 & YES & YES & YES & $1.56$ & $(3,2)$ & 51\\
$(409,121)$ & 13 & YES & YES & YES & $1.44$ & $(3,2)$ & 52\\
$(413,121)$ & 13 & YES & YES & YES & $1.33$ & $(3,2)$ & 53\\
$(421,98)$ & 14 & YES & YES & YES & $1.38$ & $(5,1)$ & 54\\
$(425,92)$ & 14 & YES & YES & YES & $1.25$ & $(5,1)$ & 55\\
$(473,108)$ & 14 & YES & YES & YES & $1.44$ & $(3,2)$ & 56\\
$(487,111)$ & 14 & YES & YES & YES & $1.33$ & $(3,2)$ & 57\\
$(522,119)$ & 14 & YES & YES & YES & $1.45$ & $(3,2)$ & 58\\
$(561,128)$ & 14 & YES & YES & YES & $1.22$ & $(3,2)$ & 59
\end{longtable}
\subsection{1 chain, $K^2 = 4$}
\begin{longtable}{|c|c|c|c|c|c|c|c|}
\hline
\multicolumn{8}{|c|}{1 chain, $K^2 = 4$}\\
\hline
$(n,a)$ & Len & Nef & $\mathbb Q$-ef & Obs 0 & $\overline c_1^2 / \overline c_2$ & $(P,K)$ & Index\\
\hline
\endfirsthead

\hline
$(n,a)$ & Len & Nef & $\mathbb Q$-ef & Obs 0 & $\overline c_1^2 / \overline c_2$ & $(P,K)$ & Index\\
\hline
\endhead
\hline
\endfoot

$(437,181)$ & 13 & YES & YES & NO(2) & $1.93$ & $(3,3)$ & 60\\
$(467,181)$ & 13 & YES & YES & NO(2) & $1.92$ & $(1,4)$ & 61\\
$(482,187)$ & 13 & YES & YES & NO(2) & $1.82$ & $(3,3)$ & 62\\
$(618,181)$ & 14 & YES & YES & NO(2) & $1.83$ & $(1,4)$ & 63\\
$(665,258)$ & 14 & YES & YES & NO(2) & $1.75$ & $(5,2)$ & 64\\
$(718,273)$ & 14 & YES & YES & NO(2) & $1.90$ & $(1,4)$ & 65\\
$(725,212)$ & 14 & YES & YES & YES & $1.89$ & $(1,4)$ & 66\\
$(737,311)$ & 14 & YES & YES & NO(2) & $1.43$ & $(7,1)$ & 67\\
$(857,239)$ & 15 & YES & YES & NO(3) & $1.71$ & $(1,4)$ & 68\\
$(875,241)$ & 15 & YES & YES & NO(2) & $1.90$ & $(1,4)$ & 69\\
$(970,267)$ & 15 & YES & YES & NO(2) & $1.89$ & $(3,3)$ & 70\\
$(972,271)$ & 15 & YES & YES & NO(3) & $1.71$ & $(1,4)$ & 71\\
$(992,277)$ & 15 & YES & YES & YES & $1.89$ & $(1,4)$ & 72\\
$(1053,241)$ & 16 & YES & YES & NO(2) & $1.90$ & $(1,4)$ & 73\\
$(1072,317)$ & 15 & YES & YES & YES & $1.86$ & $(1,4)$ & 74\\
$(1193,273)$ & 16 & YES & YES & NO(2) & $1.75$ & $(5,2)$ & 75\\
$(1295,293)$ & 16 & YES & YES & YES & $1.86$ & $(1,4)$ & 76\\
$(1495,337)$ & 17 & YES & YES & YES & $1.86$ & $(1,4)$ & 77
\end{longtable}
\subsection{2 chains, $K^2 = 2$}
\begin{longtable}{|c|c|c|c|c|c|c|c|c|c|c|c|}
\hline
\multicolumn{12}{|c|}{2 chains, $K^2 = 2$}\\
\hline
$(n,a)$ & Len & $(n,a)$ & Len & GCD & Nef & $\mathbb Q$-ef & Obs 0 & $\overline c_1^2 / \overline c_2$ & $(P,K)$ & WH & Index\\
\hline
\endfirsthead

\hline
$(n,a)$ & Len & $(n,a)$ & Len & GCD & Nef & $\mathbb Q$-ef & Obs 0 & $\overline c_1^2 / \overline c_2$ & $(P,K)$ & WH & Index\\
\hline
\endhead
\hline
\endfoot

$(21,8)$ & 6 & $(17,5)$ & 6 & 1 & YES & YES & YES & $1.69$ & $(2,2)$ & -- & 78\\
$(21,8)$ & 6 & $(17,5)$ & 6 & 1 & YES & YES & YES & $1.62$ & $(4,1)$ & NO & 79\\
$(21,8)$ & 6 & $(18,5)$ & 6 & 3 & YES & YES & YES & $1.20$ & $(6,0)$ & -- & 80\\
$(21,8)$ & 6 & $(18,5)$ & 6 & 3 & YES & YES & YES & $1.30$ & $(6,0)$ & NO & 81\\
$(24,7)$ & 7 & $(17,5)$ & 6 & 1 & YES & YES & YES & $1.46$ & $(4,1)$ & -- & 82\\
$(24,7)$ & 7 & $(18,5)$ & 6 & 6 & YES & YES & YES & $1.27$ & $(4,1)$ & -- & 83\\
$(24,7)$ & 7 & $(22,5)$ & 7 & 2 & YES & YES & YES & $1.45$ & $(4,1)$ & -- & 84\\
$(25,7)$ & 7 & $(17,5)$ & 6 & 1 & YES & YES & YES & $1.57$ & $(2,2)$ & -- & 85\\
$(25,7)$ & 7 & $(18,5)$ & 6 & 1 & YES & YES & YES & $1.36$ & $(4,1)$ & -- & 86\\
$(25,7)$ & 7 & $(22,5)$ & 7 & 1 & YES & YES & YES & $1.45$ & $(4,1)$ & -- & 87\\
$(27,8)$ & 7 & $(13,5)$ & 5 & 1 & YES & YES & YES & $1.40$ & $(6,0)$ & -- & 88\\
$(27,8)$ & 7 & $(17,5)$ & 6 & 1 & YES & YES & YES & $0.88$ & $(6,0)$ & -- & 89\\
$(27,8)$ & 7 & $(18,5)$ & 6 & 9 & YES & YES & YES & $0.88$ & $(6,0)$ & -- & 90\\
$(29,8)$ & 7 & $(13,5)$ & 5 & 1 & YES & YES & YES & $1.57$ & $(2,2)$ & -- & 91\\
$(29,8)$ & 7 & $(17,5)$ & 6 & 1 & YES & YES & YES & $0.88$ & $(6,0)$ & -- & 92\\
$(29,8)$ & 7 & $(18,5)$ & 6 & 1 & YES & YES & YES & $0.75$ & $(6,0)$ & -- & 93\\
$(29,8)$ & 7 & $(22,5)$ & 7 & 1 & YES & YES & YES & $1.64$ & $(2,2)$ & -- & 94\\
$(34,13)$ & 7 & $(10,3)$ & 5 & 2 & YES & YES & YES & $1.38$ & $(4,1)$ & -- & 95\\
$(34,13)$ & 7 & $(10,3)$ & 5 & 2 & YES & YES & YES & $1.62$ & $(4,1)$ & NO & 96\\
$(34,13)$ & 7 & $(11,3)$ & 5 & 1 & YES & YES & YES & $1.36$ & $(4,1)$ & -- & 97\\
$(34,13)$ & 7 & $(11,3)$ & 5 & 1 & YES & YES & YES & $1.50$ & $(6,0)$ & NO & 98\\
$(35,13)$ & 8 & $(13,3)$ & 6 & 1 & YES & YES & YES & $1.42$ & $(6,0)$ & -- & 99\\
$(35,13)$ & 8 & $(14,3)$ & 6 & 7 & YES & YES & YES & $1.42$ & $(6,0)$ & -- & 100\\
$(35,13)$ & 8 & $(14,3)$ & 6 & 7 & YES & YES & YES & $1.69$ & $(2,2)$ & NO & 101\\
$(37,11)$ & 8 & $(10,3)$ & 5 & 1 & YES & YES & YES & $1.69$ & $(2,2)$ & -- & 102\\
$(37,11)$ & 8 & $(11,3)$ & 5 & 1 & YES & YES & YES & $1.50$ & $(2,2)$ & -- & 103\\
$(37,11)$ & 8 & $(13,3)$ & 6 & 1 & YES & YES & YES & $1.45$ & $(4,1)$ & -- & 104\\
$(37,11)$ & 8 & $(14,3)$ & 6 & 1 & YES & YES & YES & $1.71$ & $(2,2)$ & -- & 105\\
$(40,11)$ & 8 & $(10,3)$ & 5 & 10 & YES & YES & YES & $1.64$ & $(2,2)$ & -- & 106\\
$(40,11)$ & 8 & $(11,3)$ & 5 & 1 & YES & YES & YES & $1.46$ & $(4,1)$ & -- & 107\\
$(40,11)$ & 8 & $(13,3)$ & 6 & 1 & YES & YES & YES & $1.36$ & $(4,1)$ & -- & 108\\
$(40,11)$ & 8 & $(14,3)$ & 6 & 2 & YES & YES & YES & $1.64$ & $(2,2)$ & -- & 109\\
$(41,12)$ & 8 & $(10,3)$ & 5 & 1 & YES & YES & YES & $1.20$ & $(6,0)$ & -- & 110\\
$(41,12)$ & 8 & $(11,3)$ & 5 & 1 & YES & YES & YES & $1.57$ & $(2,2)$ & -- & 111\\
$(43,12)$ & 8 & $(10,3)$ & 5 & 1 & YES & YES & YES & $1.69$ & $(2,2)$ & -- & 112\\
$(43,12)$ & 8 & $(11,3)$ & 5 & 1 & YES & YES & YES & $1.64$ & $(2,2)$ & -- & 113\\
$(44,13)$ & 8 & $(8,3)$ & 4 & 4 & YES & YES & YES & $1.64$ & $(2,2)$ & -- & 114\\
$(44,13)$ & 8 & $(8,3)$ & 4 & 4 & YES & YES & YES & $1.55$ & $(4,1)$ & NO & 115\\
$(44,13)$ & 8 & $(10,3)$ & 5 & 2 & YES & YES & YES & $0.75$ & $(6,0)$ & -- & 116\\
$(44,13)$ & 8 & $(11,3)$ & 5 & 11 & YES & YES & YES & $1.12$ & $(6,0)$ & -- & 117\\
$(44,13)$ & 8 & $(13,3)$ & 6 & 1 & YES & YES & YES & $1.71$ & $(2,2)$ & -- & 118\\
$(46,17)$ & 8 & $(7,3)$ & 4 & 1 & YES & YES & YES & $1.40$ & $(6,0)$ & -- & 119\\
$(46,17)$ & 8 & $(8,3)$ & 4 & 2 & YES & YES & YES & $1.42$ & $(6,0)$ & -- & 120\\
$(46,17)$ & 8 & $(8,3)$ & 4 & 2 & YES & YES & YES & $1.30$ & $(6,0)$ & NO & 121\\
$(46,17)$ & 8 & $(13,3)$ & 6 & 1 & YES & YES & YES & $1.20$ & $(6,0)$ & -- & 122\\
$(47,18)$ & 8 & $(7,2)$ & 4 & 1 & YES & YES & YES & $1.20$ & $(6,0)$ & NO & 123\\
$(47,18)$ & 8 & $(7,2)$ & 4 & 1 & YES & YES & YES & $1.20$ & $(6,0)$ & -- & 124\\
$(47,18)$ & 8 & $(9,2)$ & 5 & 1 & YES & YES & YES & $1.20$ & $(6,0)$ & NO & 125\\
$(47,18)$ & 8 & $(9,2)$ & 5 & 1 & YES & YES & YES & $1.20$ & $(6,0)$ & -- & 126\\
$(47,18)$ & 8 & $(9,2)$ & 5 & 1 & YES & YES & YES & $1.40$ & $(6,0)$ & NO & 127\\
$(47,13)$ & 8 & $(10,3)$ & 5 & 1 & YES & YES & YES & $0.75$ & $(6,0)$ & -- & 128\\
$(47,13)$ & 8 & $(11,3)$ & 5 & 1 & YES & YES & YES & $0.88$ & $(6,0)$ & -- & 129\\
$(49,19)$ & 8 & $(7,2)$ & 4 & 7 & YES & YES & YES & $1.27$ & $(4,1)$ & -- & 130\\
$(49,19)$ & 8 & $(7,2)$ & 4 & 7 & YES & YES & YES & $1.69$ & $(4,1)$ & NO & 131\\
$(50,19)$ & 8 & $(7,2)$ & 4 & 1 & YES & YES & YES & $1.30$ & $(6,0)$ & -- & 132\\
$(55,21)$ & 8 & $(5,2)$ & 3 & 5 & YES & YES & YES & $1.30$ & $(6,0)$ & -- & 133\\
$(55,21)$ & 8 & $(7,2)$ & 4 & 1 & YES & YES & YES & $1.00$ & $(6,0)$ & -- & 134\\
$(55,21)$ & 8 & $(9,2)$ & 5 & 1 & YES & YES & YES & $1.20$ & $(6,0)$ & NO & 135\\
$(55,21)$ & 8 & $(9,2)$ & 5 & 1 & YES & YES & YES & $1.20$ & $(6,0)$ & -- & 136\\
$(55,21)$ & 8 & $(47,18)$ & 8 & 1 & YES & YES & YES & $1.20$ & $(6,0)$ & NO & 137\\
$(56,13)$ & 10 & $(13,3)$ & 6 & 1 & YES & YES & YES & $1.69$ & $(2,2)$ & -- & 138\\
$(56,13)$ & 10 & $(14,3)$ & 6 & 14 & YES & YES & YES & $1.69$ & $(2,2)$ & -- & 139\\
$(61,18)$ & 9 & $(5,2)$ & 3 & 1 & YES & YES & YES & $1.45$ & $(4,1)$ & -- & 140\\
$(61,18)$ & 9 & $(5,2)$ & 3 & 1 & YES & YES & YES & $1.45$ & $(4,1)$ & NO & 141\\
$(61,18)$ & 9 & $(7,2)$ & 4 & 1 & YES & YES & YES & $1.20$ & $(6,0)$ & -- & 142\\
$(61,18)$ & 9 & $(9,2)$ & 5 & 1 & YES & YES & YES & $1.20$ & $(6,0)$ & -- & 143\\
$(61,18)$ & 9 & $(9,2)$ & 5 & 1 & YES & YES & YES & $1.30$ & $(6,0)$ & NO & 144\\
$(61,18)$ & 9 & $(37,11)$ & 8 & 1 & YES & YES & YES & $1.71$ & $(2,2)$ & 296 & 145\\
$(62,23)$ & 9 & $(5,2)$ & 3 & 1 & YES & YES & YES & $1.20$ & $(6,0)$ & -- & 146\\
$(62,23)$ & 9 & $(5,2)$ & 3 & 1 & YES & YES & YES & $1.75$ & $(2,2)$ & NO & 147\\
$(62,23)$ & 9 & $(46,17)$ & 8 & 2 & YES & YES & YES & $1.42$ & $(6,0)$ & NO & 148\\
$(64,19)$ & 9 & $(5,2)$ & 3 & 1 & YES & YES & YES & $1.45$ & $(4,1)$ & -- & 149\\
$(64,19)$ & 9 & $(5,2)$ & 3 & 1 & YES & YES & YES & $1.55$ & $(4,1)$ & NO & 150\\
$(64,19)$ & 9 & $(24,7)$ & 7 & 8 & YES & YES & YES & $1.54$ & $(4,1)$ & NO & 151\\
$(64,19)$ & 9 & $(44,13)$ & 8 & 4 & YES & YES & YES & $1.75$ & $(2,2)$ & NO & 152\\
$(65,19)$ & 9 & $(5,2)$ & 3 & 5 & YES & YES & YES & $1.36$ & $(4,1)$ & -- & 153\\
$(65,19)$ & 9 & $(5,2)$ & 3 & 5 & YES & YES & YES & $1.75$ & $(2,2)$ & NO & 154\\
$(65,24)$ & 9 & $(5,2)$ & 3 & 5 & YES & YES & YES & $1.10$ & $(6,0)$ & -- & 155\\
$(65,24)$ & 9 & $(5,2)$ & 3 & 5 & YES & YES & YES & $1.30$ & $(6,0)$ & NO & 156\\
$(65,24)$ & 9 & $(35,13)$ & 8 & 5 & YES & YES & YES & $1.30$ & $(6,0)$ & 333 & 157\\
$(65,18)$ & 9 & $(40,11)$ & 8 & 5 & YES & YES & YES & $1.64$ & $(2,2)$ & 344 & 158\\
$(68,19)$ & 9 & $(29,8)$ & 7 & 1 & YES & YES & YES & $1.36$ & $(4,1)$ & NO & 159\\
$(70,29)$ & 9 & $(5,2)$ & 3 & 5 & YES & YES & YES & $1.42$ & $(6,0)$ & -- & 160\\
$(70,29)$ & 9 & $(8,3)$ & 4 & 2 & YES & YES & YES & $1.42$ & $(6,0)$ & NO & 161\\
$(71,21)$ & 9 & $(7,2)$ & 4 & 1 & YES & YES & YES & $0.62$ & $(6,0)$ & -- & 162\\
$(71,21)$ & 9 & $(9,2)$ & 5 & 1 & YES & YES & YES & $1.20$ & $(6,0)$ & NO & 163\\
$(71,21)$ & 9 & $(11,3)$ & 5 & 1 & YES & YES & YES & $1.36$ & $(4,1)$ & NO & 164\\
$(71,21)$ & 9 & $(24,7)$ & 7 & 1 & YES & YES & YES & $1.36$ & $(4,1)$ & NO & 165\\
$(71,21)$ & 9 & $(37,11)$ & 8 & 1 & YES & YES & YES & $1.45$ & $(4,1)$ & NO & 166\\
$(71,21)$ & 9 & $(61,18)$ & 9 & 1 & YES & YES & YES & $1.20$ & $(6,0)$ & NO & 167\\
$(73,27)$ & 9 & $(5,2)$ & 3 & 1 & YES & YES & YES & $1.10$ & $(6,0)$ & -- & 168\\
$(73,27)$ & 9 & $(35,13)$ & 8 & 1 & YES & YES & YES & $1.42$ & $(6,0)$ & NO & 169\\
$(75,29)$ & 9 & $(7,3)$ & 4 & 1 & YES & YES & YES & $1.42$ & $(6,0)$ & NO & 170\\
$(76,29)$ & 9 & $(3,1)$ & 2 & 1 & YES & YES & YES & $1.54$ & $(4,1)$ & -- & 171\\
$(76,29)$ & 9 & $(3,1)$ & 2 & 1 & YES & YES & YES & $1.40$ & $(6,0)$ & NO & 172\\
$(76,29)$ & 9 & $(4,1)$ & 3 & 4 & YES & YES & YES & $1.54$ & $(4,1)$ & NO & 173\\
$(76,29)$ & 9 & $(4,1)$ & 3 & 4 & YES & YES & YES & $1.54$ & $(4,1)$ & -- & 174\\
$(76,29)$ & 9 & $(4,1)$ & 3 & 4 & YES & YES & YES & $1.71$ & $(2,2)$ & NO & 175\\
$(76,21)$ & 9 & $(5,2)$ & 3 & 1 & YES & YES & YES & $1.36$ & $(4,1)$ & -- & 176\\
$(76,21)$ & 9 & $(5,2)$ & 3 & 1 & YES & YES & YES & $1.36$ & $(4,1)$ & NO & 177\\
$(76,21)$ & 9 & $(5,2)$ & 3 & 1 & YES & YES & YES & $1.36$ & $(4,1)$ & NO & 178\\
$(76,29)$ & 9 & $(5,1)$ & 4 & 1 & YES & YES & YES & $1.30$ & $(6,0)$ & NO & 179\\
$(76,29)$ & 9 & $(5,1)$ & 4 & 1 & YES & YES & YES & $1.45$ & $(4,1)$ & NO & 180\\
$(76,21)$ & 9 & $(10,3)$ & 5 & 2 & YES & YES & YES & $1.27$ & $(4,1)$ & NO & 181\\
$(76,21)$ & 9 & $(25,7)$ & 7 & 1 & YES & YES & YES & $1.45$ & $(4,1)$ & NO & 182\\
$(76,29)$ & 9 & $(34,13)$ & 7 & 2 & YES & YES & YES & $1.10$ & $(6,0)$ & 228 & 183\\
$(76,21)$ & 9 & $(40,11)$ & 8 & 4 & YES & YES & YES & $1.57$ & $(2,2)$ & NO & 184\\
$(76,29)$ & 9 & $(55,21)$ & 8 & 1 & YES & YES & YES & $1.30$ & $(6,0)$ & NO & 185\\
$(76,29)$ & 9 & $(76,29)$ & 9 & 76 & YES & YES & YES & $1.54$ & $(4,1)$ & NO & 186\\
$(79,30)$ & 9 & $(3,1)$ & 2 & 1 & YES & YES & YES & $1.46$ & $(4,1)$ & -- & 187\\
$(79,30)$ & 9 & $(3,1)$ & 2 & 1 & YES & YES & YES & $1.64$ & $(2,2)$ & NO & 188\\
$(79,30)$ & 9 & $(4,1)$ & 3 & 1 & YES & YES & YES & $1.46$ & $(4,1)$ & NO & 189\\
$(79,30)$ & 9 & $(4,1)$ & 3 & 1 & YES & YES & YES & $1.64$ & $(2,2)$ & -- & 190\\
$(79,30)$ & 9 & $(4,1)$ & 3 & 1 & YES & YES & YES & $1.62$ & $(4,1)$ & NO & 191\\
$(79,18)$ & 10 & $(5,2)$ & 3 & 1 & YES & YES & YES & $1.36$ & $(4,1)$ & NO & 192\\
$(79,18)$ & 10 & $(5,2)$ & 3 & 1 & YES & YES & YES & $1.64$ & $(2,2)$ & NO & 193\\
$(79,30)$ & 9 & $(13,5)$ & 5 & 1 & YES & YES & YES & $1.64$ & $(2,2)$ & 225 & 194\\
$(79,30)$ & 9 & $(50,19)$ & 8 & 1 & YES & YES & YES & $1.46$ & $(4,1)$ & NO & 195\\
$(79,30)$ & 9 & $(79,30)$ & 9 & 79 & YES & YES & YES & $1.36$ & $(4,1)$ & NO & 196\\
$(80,31)$ & 9 & $(3,1)$ & 2 & 1 & YES & YES & YES & $1.57$ & $(2,2)$ & -- & 197\\
$(80,31)$ & 9 & $(4,1)$ & 3 & 4 & YES & YES & YES & $1.27$ & $(4,1)$ & NO & 198\\
$(80,31)$ & 9 & $(4,1)$ & 3 & 4 & YES & YES & YES & $1.62$ & $(4,1)$ & -- & 199\\
$(80,31)$ & 9 & $(4,1)$ & 3 & 4 & YES & YES & YES & $1.45$ & $(4,1)$ & NO & 200\\
$(80,31)$ & 9 & $(8,3)$ & 4 & 8 & YES & YES & YES & $1.64$ & $(2,2)$ & NO & 201\\
$(80,31)$ & 9 & $(49,19)$ & 8 & 1 & YES & YES & YES & $1.27$ & $(4,1)$ & NO & 202\\
$(80,31)$ & 9 & $(80,31)$ & 9 & 80 & YES & YES & YES & $1.55$ & $(4,1)$ & NO & 203\\
$(81,31)$ & 9 & $(3,1)$ & 2 & 3 & YES & YES & YES & $1.46$ & $(4,1)$ & -- & 204\\
$(81,31)$ & 9 & $(3,1)$ & 2 & 3 & YES & YES & YES & $1.54$ & $(4,1)$ & NO & 205\\
$(81,31)$ & 9 & $(4,1)$ & 3 & 1 & YES & YES & YES & $1.40$ & $(6,0)$ & NO & 206\\
$(81,31)$ & 9 & $(21,8)$ & 6 & 3 & YES & YES & YES & $1.46$ & $(4,1)$ & NO & 207\\
$(81,31)$ & 9 & $(47,18)$ & 8 & 1 & YES & YES & YES & $1.20$ & $(6,0)$ & NO & 208\\
$(84,31)$ & 10 & $(4,1)$ & 3 & 4 & YES & YES & YES & $1.40$ & $(6,0)$ & -- & 209\\
$(84,31)$ & 10 & $(5,1)$ & 4 & 1 & YES & YES & YES & $1.10$ & $(6,0)$ & NO & 210\\
$(84,31)$ & 10 & $(46,17)$ & 8 & 2 & YES & YES & YES & $1.40$ & $(6,0)$ & 336 & 211\\
$(84,31)$ & 10 & $(65,24)$ & 9 & 1 & YES & YES & YES & $1.10$ & $(6,0)$ & NO & 212\\
$(89,34)$ & 9 & $(2,1)$ & 1 & 1 & YES & YES & YES & $1.38$ & $(4,1)$ & -- & 213\\
$(89,26)$ & 10 & $(3,1)$ & 2 & 1 & YES & YES & YES & $1.69$ & $(2,2)$ & -- & 214\\
$(89,26)$ & 10 & $(3,1)$ & 2 & 1 & YES & YES & YES & $1.36$ & $(4,1)$ & NO & 215\\
$(89,34)$ & 9 & $(3,1)$ & 2 & 1 & YES & YES & YES & $1.38$ & $(4,1)$ & -- & 216\\
$(89,34)$ & 9 & $(3,1)$ & 2 & 1 & YES & YES & YES & $0.88$ & $(6,0)$ & NO & 217\\
$(89,34)$ & 9 & $(4,1)$ & 3 & 1 & YES & YES & YES & $1.00$ & $(6,0)$ & NO & 218\\
$(89,24)$ & 10 & $(5,2)$ & 3 & 1 & YES & YES & YES & $1.20$ & $(6,0)$ & NO & 219\\
$(89,24)$ & 10 & $(5,2)$ & 3 & 1 & YES & YES & YES & $1.20$ & $(6,0)$ & -- & 220\\
$(89,26)$ & 10 & $(5,1)$ & 4 & 1 & YES & YES & YES & $1.69$ & $(2,2)$ & NO & 221\\
$(89,34)$ & 9 & $(5,1)$ & 4 & 1 & YES & YES & YES & $1.45$ & $(4,1)$ & NO & 222\\
$(89,34)$ & 9 & $(5,1)$ & 4 & 1 & YES & YES & YES & $1.45$ & $(4,1)$ & NO & 223\\
$(89,34)$ & 9 & $(5,2)$ & 3 & 1 & YES & YES & YES & $1.38$ & $(4,1)$ & NO & 224\\
$(89,34)$ & 9 & $(8,3)$ & 4 & 1 & YES & YES & YES & $1.64$ & $(2,2)$ & 194 & 225\\
$(89,26)$ & 10 & $(10,3)$ & 5 & 1 & YES & YES & YES & $1.62$ & $(2,2)$ & NO & 226\\
$(89,34)$ & 9 & $(13,5)$ & 5 & 1 & YES & YES & YES & $1.57$ & $(2,2)$ & NO & 227\\
$(89,34)$ & 9 & $(21,8)$ & 6 & 1 & YES & YES & YES & $1.10$ & $(6,0)$ & 183 & 228\\
$(89,34)$ & 9 & $(34,13)$ & 7 & 1 & YES & YES & YES & $1.27$ & $(4,1)$ & NO & 229\\
$(89,26)$ & 10 & $(41,12)$ & 8 & 1 & YES & YES & YES & $1.50$ & $(2,2)$ & 311 & 230\\
$(89,34)$ & 9 & $(55,21)$ & 8 & 1 & YES & YES & YES & $0.88$ & $(6,0)$ & NO & 231\\
$(89,26)$ & 10 & $(89,26)$ & 10 & 89 & YES & YES & YES & $1.57$ & $(2,2)$ & NO & 232\\
$(89,34)$ & 9 & $(89,34)$ & 9 & 89 & YES & YES & YES & $0.88$ & $(6,0)$ & NO & 233\\
$(91,27)$ & 10 & $(3,1)$ & 2 & 1 & YES & YES & YES & $1.50$ & $(2,2)$ & -- & 234\\
$(91,27)$ & 10 & $(3,1)$ & 2 & 1 & YES & YES & YES & $1.45$ & $(4,1)$ & NO & 235\\
$(91,27)$ & 10 & $(5,1)$ & 4 & 1 & YES & YES & YES & $1.75$ & $(2,2)$ & NO & 236\\
$(91,27)$ & 10 & $(17,5)$ & 6 & 1 & YES & YES & YES & $1.54$ & $(4,1)$ & NO & 237\\
$(91,27)$ & 10 & $(37,11)$ & 8 & 1 & YES & YES & YES & $1.54$ & $(4,1)$ & 276 & 238\\
$(91,27)$ & 10 & $(91,27)$ & 10 & 91 & YES & YES & YES & $1.54$ & $(4,1)$ & NO & 239\\
$(93,26)$ & 10 & $(3,1)$ & 2 & 3 & YES & YES & YES & $1.36$ & $(4,1)$ & -- & 240\\
$(93,26)$ & 10 & $(3,1)$ & 2 & 3 & YES & YES & YES & $1.45$ & $(4,1)$ & NO & 241\\
$(93,26)$ & 10 & $(5,1)$ & 4 & 1 & YES & YES & YES & $1.36$ & $(4,1)$ & NO & 242\\
$(93,26)$ & 10 & $(11,3)$ & 5 & 1 & YES & YES & YES & $1.46$ & $(4,1)$ & NO & 243\\
$(93,26)$ & 10 & $(43,12)$ & 8 & 1 & YES & YES & YES & $1.57$ & $(2,2)$ & 337 & 244\\
$(93,26)$ & 10 & $(93,26)$ & 10 & 93 & YES & YES & YES & $1.45$ & $(4,1)$ & NO & 245\\
$(97,22)$ & 11 & $(14,3)$ & 6 & 1 & YES & YES & YES & $1.69$ & $(2,2)$ & 274 & 246\\
$(98,29)$ & 10 & $(2,1)$ & 1 & 2 & YES & YES & YES & $1.46$ & $(4,1)$ & -- & 247\\
$(98,29)$ & 10 & $(2,1)$ & 1 & 2 & YES & YES & YES & $1.62$ & $(4,1)$ & NO & 248\\
$(98,27)$ & 10 & $(3,1)$ & 2 & 1 & YES & YES & YES & $1.27$ & $(4,1)$ & -- & 249\\
$(98,29)$ & 10 & $(3,1)$ & 2 & 1 & YES & YES & YES & $1.40$ & $(6,0)$ & -- & 250\\
$(98,29)$ & 10 & $(3,1)$ & 2 & 1 & YES & YES & YES & $1.12$ & $(6,0)$ & NO & 251\\
$(98,29)$ & 10 & $(4,1)$ & 3 & 2 & YES & YES & YES & $1.46$ & $(4,1)$ & NO & 252\\
$(98,27)$ & 10 & $(5,1)$ & 4 & 1 & YES & YES & YES & $1.46$ & $(4,1)$ & NO & 253\\
$(98,29)$ & 10 & $(5,1)$ & 4 & 1 & YES & YES & YES & $0.88$ & $(6,0)$ & NO & 254\\
$(98,29)$ & 10 & $(7,2)$ & 4 & 7 & YES & YES & YES & $1.20$ & $(6,0)$ & NO & 255\\
$(98,29)$ & 10 & $(10,3)$ & 5 & 2 & YES & YES & YES & $1.45$ & $(4,1)$ & NO & 256\\
$(98,29)$ & 10 & $(17,5)$ & 6 & 1 & YES & YES & YES & $1.46$ & $(4,1)$ & NO & 257\\
$(98,27)$ & 10 & $(18,5)$ & 6 & 2 & YES & YES & YES & $1.27$ & $(4,1)$ & NO & 258\\
$(98,27)$ & 10 & $(40,11)$ & 8 & 2 & YES & YES & YES & $1.46$ & $(4,1)$ & 320 & 259\\
$(98,29)$ & 10 & $(44,13)$ & 8 & 2 & YES & YES & YES & $0.88$ & $(6,0)$ & 358 & 260\\
$(98,29)$ & 10 & $(98,29)$ & 10 & 98 & YES & YES & YES & $0.88$ & $(6,0)$ & NO & 261\\
$(99,29)$ & 10 & $(3,1)$ & 2 & 3 & YES & YES & YES & $1.36$ & $(4,1)$ & -- & 262\\
$(99,29)$ & 10 & $(3,1)$ & 2 & 3 & YES & YES & YES & $1.30$ & $(6,0)$ & NO & 263\\
$(99,29)$ & 10 & $(10,3)$ & 5 & 1 & YES & YES & YES & $1.30$ & $(6,0)$ & NO & 264\\
$(99,29)$ & 10 & $(24,7)$ & 7 & 3 & YES & YES & YES & $1.57$ & $(2,2)$ & NO & 265\\
$(101,30)$ & 10 & $(2,1)$ & 1 & 1 & YES & YES & YES & $1.38$ & $(4,1)$ & -- & 266\\
$(101,30)$ & 10 & $(2,1)$ & 1 & 1 & YES & YES & YES & $1.46$ & $(4,1)$ & NO & 267\\
$(101,30)$ & 10 & $(3,1)$ & 2 & 1 & YES & YES & YES & $0.88$ & $(6,0)$ & -- & 268\\
$(101,30)$ & 10 & $(3,1)$ & 2 & 1 & YES & YES & YES & $1.00$ & $(6,0)$ & NO & 269\\
$(101,30)$ & 10 & $(4,1)$ & 3 & 1 & YES & YES & YES & $1.30$ & $(6,0)$ & NO & 270\\
$(101,22)$ & 11 & $(5,2)$ & 3 & 1 & YES & YES & YES & $1.20$ & $(6,0)$ & -- & 271\\
$(101,30)$ & 10 & $(5,1)$ & 4 & 1 & YES & YES & YES & $1.46$ & $(4,1)$ & NO & 272\\
$(101,30)$ & 10 & $(7,2)$ & 4 & 1 & YES & YES & YES & $1.46$ & $(4,1)$ & NO & 273\\
$(101,22)$ & 11 & $(13,3)$ & 6 & 1 & YES & YES & YES & $1.69$ & $(2,2)$ & 246 & 274\\
$(101,30)$ & 10 & $(17,5)$ & 6 & 1 & YES & YES & YES & $0.88$ & $(6,0)$ & 355 & 275\\
$(101,30)$ & 10 & $(27,8)$ & 7 & 1 & YES & YES & YES & $1.54$ & $(4,1)$ & 238 & 276\\
$(101,30)$ & 10 & $(37,11)$ & 8 & 1 & YES & YES & YES & $1.69$ & $(2,2)$ & NO & 277\\
$(101,30)$ & 10 & $(101,30)$ & 10 & 101 & YES & YES & YES & $1.40$ & $(6,0)$ & NO & 278\\
$(103,24)$ & 11 & $(56,13)$ & 10 & 1 & YES & YES & YES & $1.69$ & $(2,2)$ & 413 & 279\\
$(104,29)$ & 10 & $(3,1)$ & 2 & 1 & YES & YES & YES & $1.36$ & $(4,1)$ & -- & 280\\
$(104,29)$ & 10 & $(3,1)$ & 2 & 1 & YES & YES & YES & $1.45$ & $(4,1)$ & NO & 281\\
$(104,29)$ & 10 & $(11,3)$ & 5 & 1 & YES & YES & YES & $1.64$ & $(2,2)$ & NO & 282\\
$(104,29)$ & 10 & $(25,7)$ & 7 & 1 & YES & YES & YES & $1.36$ & $(4,1)$ & NO & 283\\
$(105,29)$ & 10 & $(2,1)$ & 1 & 1 & YES & YES & YES & $1.00$ & $(6,0)$ & -- & 284\\
$(105,29)$ & 10 & $(2,1)$ & 1 & 1 & YES & YES & YES & $1.10$ & $(6,0)$ & NO & 285\\
$(105,31)$ & 10 & $(2,1)$ & 1 & 1 & YES & YES & YES & $1.71$ & $(2,2)$ & -- & 286\\
$(105,31)$ & 10 & $(2,1)$ & 1 & 1 & YES & YES & YES & $1.71$ & $(2,2)$ & NO & 287\\
$(105,29)$ & 10 & $(3,1)$ & 2 & 3 & YES & YES & YES & $0.88$ & $(6,0)$ & -- & 288\\
$(105,29)$ & 10 & $(3,1)$ & 2 & 3 & YES & YES & YES & $1.54$ & $(4,1)$ & NO & 289\\
$(105,29)$ & 10 & $(3,1)$ & 2 & 3 & YES & YES & YES & $1.30$ & $(6,0)$ & NO & 290\\
$(105,31)$ & 10 & $(3,1)$ & 2 & 3 & YES & YES & YES & $0.75$ & $(6,0)$ & -- & 291\\
$(105,31)$ & 10 & $(3,1)$ & 2 & 3 & YES & YES & YES & $0.88$ & $(6,0)$ & NO & 292\\
$(105,31)$ & 10 & $(4,1)$ & 3 & 1 & YES & YES & YES & $1.27$ & $(4,1)$ & -- & 293\\
$(105,29)$ & 10 & $(5,1)$ & 4 & 5 & YES & YES & YES & $0.88$ & $(6,0)$ & NO & 294\\
$(105,29)$ & 10 & $(7,2)$ & 4 & 7 & YES & YES & YES & $1.20$ & $(6,0)$ & NO & 295\\
$(105,31)$ & 10 & $(10,3)$ & 5 & 5 & YES & YES & YES & $1.71$ & $(2,2)$ & 145 & 296\\
$(105,29)$ & 10 & $(11,3)$ & 5 & 1 & YES & YES & YES & $1.50$ & $(2,2)$ & NO & 297\\
$(105,29)$ & 10 & $(18,5)$ & 6 & 3 & YES & YES & YES & $1.20$ & $(6,0)$ & NO & 298\\
$(105,31)$ & 10 & $(27,8)$ & 7 & 3 & YES & YES & YES & $0.75$ & $(6,0)$ & NO & 299\\
$(105,31)$ & 10 & $(44,13)$ & 8 & 1 & YES & YES & YES & $1.27$ & $(4,1)$ & NO & 300\\
$(105,29)$ & 10 & $(47,13)$ & 8 & 1 & YES & YES & YES & $0.88$ & $(6,0)$ & 376 & 301\\
$(105,31)$ & 10 & $(61,18)$ & 9 & 1 & YES & YES & YES & $1.20$ & $(6,0)$ & NO & 302\\
$(105,29)$ & 10 & $(105,29)$ & 10 & 105 & YES & YES & YES & $0.88$ & $(6,0)$ & NO & 303\\
$(106,31)$ & 10 & $(2,1)$ & 1 & 2 & YES & YES & YES & $1.18$ & $(4,1)$ & -- & 304\\
$(106,31)$ & 10 & $(2,1)$ & 1 & 2 & YES & YES & YES & $1.27$ & $(4,1)$ & NO & 305\\
$(106,31)$ & 10 & $(3,1)$ & 2 & 1 & YES & YES & YES & $0.75$ & $(6,0)$ & -- & 306\\
$(106,31)$ & 10 & $(3,1)$ & 2 & 1 & YES & YES & YES & $1.25$ & $(6,0)$ & NO & 307\\
$(106,31)$ & 10 & $(4,1)$ & 3 & 2 & YES & YES & YES & $1.50$ & $(2,2)$ & NO & 308\\
$(106,31)$ & 10 & $(10,3)$ & 5 & 2 & YES & YES & YES & $0.62$ & $(6,0)$ & NO & 309\\
$(106,31)$ & 10 & $(17,5)$ & 6 & 1 & YES & YES & YES & $1.36$ & $(4,1)$ & NO & 310\\
$(106,31)$ & 10 & $(24,7)$ & 7 & 2 & YES & YES & YES & $1.50$ & $(2,2)$ & 230 & 311\\
$(106,31)$ & 10 & $(41,12)$ & 8 & 1 & YES & YES & YES & $1.30$ & $(6,0)$ & NO & 312\\
$(109,30)$ & 10 & $(2,1)$ & 1 & 1 & YES & YES & YES & $1.46$ & $(4,1)$ & NO & 313\\
$(109,30)$ & 10 & $(2,1)$ & 1 & 1 & YES & YES & YES & $1.45$ & $(4,1)$ & -- & 314\\
$(109,30)$ & 10 & $(3,1)$ & 2 & 1 & YES & YES & YES & $0.88$ & $(6,0)$ & -- & 315\\
$(109,30)$ & 10 & $(3,1)$ & 2 & 1 & YES & YES & YES & $1.00$ & $(6,0)$ & NO & 316\\
$(109,30)$ & 10 & $(4,1)$ & 3 & 1 & YES & YES & YES & $1.27$ & $(4,1)$ & -- & 317\\
$(109,30)$ & 10 & $(7,2)$ & 4 & 1 & YES & YES & YES & $1.27$ & $(4,1)$ & NO & 318\\
$(109,30)$ & 10 & $(18,5)$ & 6 & 1 & YES & YES & YES & $0.88$ & $(6,0)$ & 374 & 319\\
$(109,30)$ & 10 & $(29,8)$ & 7 & 1 & YES & YES & YES & $1.46$ & $(4,1)$ & 259 & 320\\
$(109,30)$ & 10 & $(40,11)$ & 8 & 1 & YES & YES & YES & $1.64$ & $(2,2)$ & NO & 321\\
$(111,31)$ & 10 & $(2,1)$ & 1 & 1 & YES & YES & YES & $1.38$ & $(4,1)$ & -- & 322\\
$(111,31)$ & 10 & $(2,1)$ & 1 & 1 & YES & YES & YES & $1.46$ & $(4,1)$ & NO & 323\\
$(111,41)$ & 10 & $(2,1)$ & 1 & 1 & YES & YES & YES & $1.00$ & $(6,0)$ & -- & 324\\
$(111,41)$ & 10 & $(2,1)$ & 1 & 1 & YES & YES & YES & $1.62$ & $(2,2)$ & NO & 325\\
$(111,31)$ & 10 & $(3,1)$ & 2 & 3 & YES & YES & YES & $0.88$ & $(6,0)$ & NO & 326\\
$(111,31)$ & 10 & $(3,1)$ & 2 & 3 & YES & YES & YES & $0.88$ & $(6,0)$ & -- & 327\\
$(111,31)$ & 10 & $(3,1)$ & 2 & 3 & YES & YES & YES & $1.54$ & $(4,1)$ & NO & 328\\
$(111,41)$ & 10 & $(4,1)$ & 3 & 1 & YES & YES & YES & $1.20$ & $(6,0)$ & -- & 329\\
$(111,41)$ & 10 & $(4,1)$ & 3 & 1 & YES & YES & YES & $1.69$ & $(2,2)$ & NO & 330\\
$(111,31)$ & 10 & $(5,1)$ & 4 & 1 & YES & YES & YES & $1.69$ & $(2,2)$ & NO & 331\\
$(111,41)$ & 10 & $(5,1)$ & 4 & 1 & YES & YES & YES & $1.10$ & $(6,0)$ & NO & 332\\
$(111,41)$ & 10 & $(8,3)$ & 4 & 1 & YES & YES & YES & $1.30$ & $(6,0)$ & 157 & 333\\
$(111,31)$ & 10 & $(11,3)$ & 5 & 1 & YES & YES & YES & $1.00$ & $(6,0)$ & NO & 334\\
$(111,31)$ & 10 & $(18,5)$ & 6 & 3 & YES & YES & YES & $1.30$ & $(6,0)$ & NO & 335\\
$(111,41)$ & 10 & $(19,7)$ & 6 & 1 & YES & YES & YES & $1.40$ & $(6,0)$ & 211 & 336\\
$(111,31)$ & 10 & $(25,7)$ & 7 & 1 & YES & YES & YES & $1.57$ & $(2,2)$ & 244 & 337\\
$(111,31)$ & 10 & $(43,12)$ & 8 & 1 & YES & YES & YES & $1.36$ & $(4,1)$ & NO & 338\\
$(111,41)$ & 10 & $(46,17)$ & 8 & 1 & YES & YES & YES & $1.20$ & $(6,0)$ & NO & 339\\
$(112,31)$ & 10 & $(2,1)$ & 1 & 2 & YES & YES & YES & $1.75$ & $(2,2)$ & NO & 340\\
$(112,31)$ & 10 & $(3,1)$ & 2 & 1 & YES & YES & YES & $0.62$ & $(6,0)$ & -- & 341\\
$(112,31)$ & 10 & $(3,1)$ & 2 & 1 & YES & YES & YES & $1.00$ & $(6,0)$ & NO & 342\\
$(112,31)$ & 10 & $(7,2)$ & 4 & 7 & YES & YES & YES & $1.75$ & $(2,2)$ & NO & 343\\
$(112,31)$ & 10 & $(11,3)$ & 5 & 1 & YES & YES & YES & $1.64$ & $(2,2)$ & 158 & 344\\
$(112,31)$ & 10 & $(29,8)$ & 7 & 1 & YES & YES & YES & $0.88$ & $(6,0)$ & NO & 345\\
$(115,34)$ & 10 & $(2,1)$ & 1 & 1 & YES & YES & YES & $0.88$ & $(6,0)$ & -- & 346\\
$(115,34)$ & 10 & $(2,1)$ & 1 & 1 & YES & YES & YES & $1.00$ & $(6,0)$ & NO & 347\\
$(115,26)$ & 11 & $(3,1)$ & 2 & 1 & YES & YES & YES & $1.27$ & $(4,1)$ & NO & 348\\
$(115,26)$ & 11 & $(3,1)$ & 2 & 1 & YES & YES & YES & $1.36$ & $(4,1)$ & -- & 349\\
$(115,26)$ & 11 & $(3,1)$ & 2 & 1 & YES & YES & YES & $1.69$ & $(4,1)$ & NO & 350\\
$(115,34)$ & 10 & $(3,1)$ & 2 & 1 & YES & YES & YES & $0.62$ & $(6,0)$ & -- & 351\\
$(115,34)$ & 10 & $(4,1)$ & 3 & 1 & YES & YES & YES & $0.88$ & $(6,0)$ & NO & 352\\
$(115,34)$ & 10 & $(5,1)$ & 4 & 5 & YES & YES & YES & $0.88$ & $(6,0)$ & NO & 353\\
$(115,34)$ & 10 & $(7,2)$ & 4 & 1 & YES & YES & YES & $0.88$ & $(6,0)$ & NO & 354\\
$(115,34)$ & 10 & $(10,3)$ & 5 & 5 & YES & YES & YES & $0.88$ & $(6,0)$ & 275 & 355\\
$(115,26)$ & 11 & $(13,3)$ & 6 & 1 & YES & YES & YES & $1.27$ & $(4,1)$ & NO & 356\\
$(115,34)$ & 10 & $(17,5)$ & 6 & 1 & YES & YES & YES & $0.88$ & $(6,0)$ & NO & 357\\
$(115,34)$ & 10 & $(27,8)$ & 7 & 1 & YES & YES & YES & $0.88$ & $(6,0)$ & 260 & 358\\
$(115,34)$ & 10 & $(44,13)$ & 8 & 1 & YES & YES & YES & $0.88$ & $(6,0)$ & NO & 359\\
$(115,34)$ & 10 & $(71,21)$ & 9 & 1 & YES & YES & YES & $0.62$ & $(6,0)$ & NO & 360\\
$(115,34)$ & 10 & $(115,34)$ & 10 & 115 & YES & YES & YES & $0.88$ & $(6,0)$ & NO & 361\\
$(118,27)$ & 11 & $(3,1)$ & 2 & 1 & YES & YES & YES & $1.57$ & $(2,2)$ & NO & 362\\
$(118,27)$ & 11 & $(3,1)$ & 2 & 1 & YES & YES & YES & $1.57$ & $(2,2)$ & -- & 363\\
$(118,27)$ & 11 & $(3,1)$ & 2 & 1 & YES & YES & YES & $1.69$ & $(4,1)$ & NO & 364\\
$(118,27)$ & 11 & $(22,5)$ & 7 & 2 & YES & YES & YES & $1.64$ & $(2,2)$ & NO & 365\\
$(119,26)$ & 11 & $(3,1)$ & 2 & 1 & YES & YES & YES & $1.64$ & $(2,2)$ & NO & 366\\
$(119,26)$ & 11 & $(14,3)$ & 6 & 7 & YES & YES & YES & $1.27$ & $(4,1)$ & NO & 367\\
$(123,34)$ & 10 & $(2,1)$ & 1 & 1 & YES & YES & YES & $0.88$ & $(6,0)$ & -- & 368\\
$(123,34)$ & 10 & $(2,1)$ & 1 & 1 & YES & YES & YES & $0.88$ & $(6,0)$ & NO & 369\\
$(123,34)$ & 10 & $(3,1)$ & 2 & 3 & YES & YES & YES & $0.75$ & $(6,0)$ & NO & 370\\
$(123,34)$ & 10 & $(3,1)$ & 2 & 3 & YES & YES & YES & $0.75$ & $(6,0)$ & -- & 371\\
$(123,34)$ & 10 & $(5,1)$ & 4 & 1 & YES & YES & YES & $0.88$ & $(6,0)$ & NO & 372\\
$(123,34)$ & 10 & $(7,2)$ & 4 & 1 & YES & YES & YES & $0.75$ & $(6,0)$ & NO & 373\\
$(123,34)$ & 10 & $(11,3)$ & 5 & 1 & YES & YES & YES & $0.88$ & $(6,0)$ & 319 & 374\\
$(123,34)$ & 10 & $(18,5)$ & 6 & 3 & YES & YES & YES & $0.75$ & $(6,0)$ & NO & 375\\
$(123,34)$ & 10 & $(29,8)$ & 7 & 1 & YES & YES & YES & $0.88$ & $(6,0)$ & 301 & 376\\
$(123,34)$ & 10 & $(47,13)$ & 8 & 1 & YES & YES & YES & $0.75$ & $(6,0)$ & NO & 377\\
$(127,29)$ & 11 & $(2,1)$ & 1 & 1 & YES & YES & YES & $1.27$ & $(4,1)$ & -- & 378\\
$(127,29)$ & 11 & $(2,1)$ & 1 & 1 & YES & YES & YES & $1.36$ & $(4,1)$ & NO & 379\\
$(127,29)$ & 11 & $(3,1)$ & 2 & 1 & YES & YES & YES & $1.00$ & $(6,0)$ & NO & 380\\
$(127,29)$ & 11 & $(3,1)$ & 2 & 1 & YES & YES & YES & $1.00$ & $(6,0)$ & -- & 381\\
$(127,29)$ & 11 & $(5,1)$ & 4 & 1 & YES & YES & YES & $1.45$ & $(4,1)$ & NO & 382\\
$(127,29)$ & 11 & $(9,2)$ & 5 & 1 & YES & YES & YES & $1.20$ & $(6,0)$ & NO & 383\\
$(127,29)$ & 11 & $(13,3)$ & 6 & 1 & YES & YES & YES & $1.45$ & $(4,1)$ & NO & 384\\
$(127,29)$ & 11 & $(22,5)$ & 7 & 1 & YES & YES & YES & $1.45$ & $(4,1)$ & NO & 385\\
$(131,30)$ & 11 & $(3,1)$ & 2 & 1 & YES & YES & YES & $1.64$ & $(2,2)$ & NO & 386\\
$(131,30)$ & 11 & $(3,1)$ & 2 & 1 & YES & YES & YES & $1.64$ & $(2,2)$ & -- & 387\\
$(131,30)$ & 11 & $(22,5)$ & 7 & 1 & YES & YES & YES & $1.71$ & $(2,2)$ & 409 & 388\\
$(133,31)$ & 12 & $(2,1)$ & 1 & 1 & YES & YES & YES & $1.33$ & $(6,0)$ & -- & 389\\
$(133,31)$ & 12 & $(2,1)$ & 1 & 1 & YES & YES & YES & $1.42$ & $(6,0)$ & NO & 390\\
$(133,31)$ & 12 & $(5,1)$ & 4 & 1 & YES & YES & YES & $1.20$ & $(6,0)$ & NO & 391\\
$(133,31)$ & 12 & $(13,3)$ & 6 & 1 & YES & YES & YES & $1.69$ & $(2,2)$ & NO & 392\\
$(134,29)$ & 11 & $(2,1)$ & 1 & 2 & YES & YES & YES & $1.27$ & $(4,1)$ & -- & 393\\
$(134,29)$ & 11 & $(2,1)$ & 1 & 2 & YES & YES & YES & $1.36$ & $(4,1)$ & NO & 394\\
$(134,29)$ & 11 & $(3,1)$ & 2 & 1 & YES & YES & YES & $0.88$ & $(6,0)$ & NO & 395\\
$(134,29)$ & 11 & $(3,1)$ & 2 & 1 & YES & YES & YES & $1.00$ & $(6,0)$ & -- & 396\\
$(134,29)$ & 11 & $(3,1)$ & 2 & 1 & YES & YES & YES & $1.30$ & $(6,0)$ & NO & 397\\
$(134,29)$ & 11 & $(4,1)$ & 3 & 2 & YES & YES & YES & $1.36$ & $(4,1)$ & NO & 398\\
$(134,29)$ & 11 & $(9,2)$ & 5 & 1 & YES & YES & YES & $1.20$ & $(6,0)$ & NO & 399\\
$(134,29)$ & 11 & $(14,3)$ & 6 & 2 & YES & YES & YES & $1.75$ & $(2,2)$ & NO & 400\\
$(146,31)$ & 12 & $(2,1)$ & 1 & 2 & YES & YES & YES & $1.33$ & $(6,0)$ & -- & 401\\
$(146,31)$ & 12 & $(2,1)$ & 1 & 2 & YES & YES & YES & $1.42$ & $(6,0)$ & NO & 402\\
$(146,31)$ & 12 & $(4,1)$ & 3 & 2 & YES & YES & YES & $1.20$ & $(6,0)$ & NO & 403\\
$(146,31)$ & 12 & $(4,1)$ & 3 & 2 & YES & YES & YES & $1.69$ & $(2,2)$ & -- & 404\\
$(146,31)$ & 12 & $(14,3)$ & 6 & 2 & YES & YES & YES & $1.69$ & $(2,2)$ & NO & 405\\
$(149,34)$ & 11 & $(2,1)$ & 1 & 1 & YES & YES & YES & $1.71$ & $(2,2)$ & NO & 406\\
$(149,34)$ & 11 & $(3,1)$ & 2 & 1 & YES & YES & YES & $0.75$ & $(6,0)$ & NO & 407\\
$(149,34)$ & 11 & $(3,1)$ & 2 & 1 & YES & YES & YES & $1.00$ & $(6,0)$ & -- & 408\\
$(149,34)$ & 11 & $(13,3)$ & 6 & 1 & YES & YES & YES & $1.71$ & $(2,2)$ & 388 & 409\\
$(169,70)$ & 11 & $(2,1)$ & 1 & 1 & NO & YES & YES & $1.69$ & $(2,2)$ & -- & 410\\
$(176,41)$ & 12 & $(2,1)$ & 1 & 2 & YES & YES & YES & $1.10$ & $(6,0)$ & -- & 411\\
$(176,41)$ & 12 & $(2,1)$ & 1 & 2 & YES & YES & YES & $1.20$ & $(6,0)$ & NO & 412\\
$(176,41)$ & 12 & $(13,3)$ & 6 & 1 & YES & YES & YES & $1.69$ & $(2,2)$ & 279 & 413\\
$(b;0,1,2;5)$ & 8 & $(5,2)$ & 3 & 5 & YES & YES & YES & $1.10$ & $(6,0)$ & -- & 414\\
$(b;0,1,2;5)$ & 8 & $(13,3)$ & 6 & 1 & YES & YES & YES & $1.69$ & $(2,2)$ & -- & 415\\
$(b;1,0,1;29)$ & 7 & $(5,2)$ & 3 & 1 & YES & YES & YES & $1.69$ & $(2,2)$ & -- & 416\\
$(b;1,0,1;29)$ & 7 & $(10,3)$ & 5 & 1 & YES & YES & YES & $1.40$ & $(6,0)$ & -- & 417\\
$(b;1,1,1;39)$ & 8 & $(3,1)$ & 2 & 3 & YES & YES & YES & $1.50$ & $(2,2)$ & -- & 418\\
$(b;1,1,1;39)$ & 8 & $(7,2)$ & 4 & 1 & YES & YES & YES & $1.64$ & $(2,2)$ & -- & 419\\
$(c;0,0,0;4)$ & 4 & $(29,11)$ & 7 & 1 & YES & YES & YES & $1.46$ & $(4,1)$ & -- & 420\\
$(c;0,0,0;4)$ & 4 & $(31,12)$ & 7 & 1 & YES & YES & YES & $1.36$ & $(4,1)$ & -- & 421\\
$(c;0,0,0;4)$ & 4 & $(34,13)$ & 7 & 2 & YES & YES & YES & $1.50$ & $(2,2)$ & -- & 422\\
$(c;0,0,0;4)$ & 4 & $(40,11)$ & 8 & 4 & YES & YES & YES & $1.27$ & $(4,1)$ & -- & 423\\
$(c;0,0,0;4)$ & 4 & $(43,12)$ & 8 & 1 & YES & YES & YES & $1.57$ & $(2,2)$ & -- & 424\\
$(c;0,1,0;11)$ & 5 & $(21,8)$ & 6 & 1 & YES & YES & YES & $1.50$ & $(6,0)$ & -- & 425\\
$(d;0,0,0;5)$ & 5 & $(21,8)$ & 6 & 1 & YES & YES & YES & $1.40$ & $(6,0)$ & -- & 426\\
$(e;2,1,0;31)$ & 8 & $(7,3)$ & 4 & 1 & YES & YES & YES & $1.75$ & $(2,2)$ & -- & 427\\
$(f;0,0,0;6)$ & 4 & $(46,17)$ & 8 & 2 & YES & YES & YES & $1.40$ & $(6,0)$ & -- & 428\\
$(g;0,0,1;26)$ & 7 & $(5,2)$ & 3 & 1 & YES & YES & YES & $1.46$ & $(4,1)$ & -- & 429\\
$(g;0,0,1;26)$ & 7 & $(7,2)$ & 4 & 1 & YES & YES & YES & $1.27$ & $(4,1)$ & -- & 430\\
$(g;0,1,0;24)$ & 7 & $(5,2)$ & 3 & 1 & YES & YES & YES & $1.46$ & $(4,1)$ & -- & 431\\
$(g;0,1,0;24)$ & 7 & $(7,2)$ & 4 & 1 & YES & YES & YES & $1.36$ & $(4,1)$ & -- & 432\\
$(g;0,1,1;33)$ & 8 & $(3,1)$ & 2 & 3 & YES & YES & YES & $1.27$ & $(4,1)$ & -- & 433\\
$(g;1,0,0;7)$ & 7 & $(5,2)$ & 3 & 1 & YES & YES & YES & $1.27$ & $(4,1)$ & -- & 434\\
$(g;1,0,0;7)$ & 7 & $(10,3)$ & 5 & 1 & YES & YES & YES & $1.30$ & $(6,0)$ & -- & 435\\
$(g;1,0,1;38)$ & 8 & $(2,1)$ & 1 & 2 & YES & YES & YES & $1.38$ & $(4,1)$ & -- & 436\\
$(g;1,0,1;38)$ & 8 & $(3,1)$ & 2 & 1 & YES & YES & YES & $0.75$ & $(6,0)$ & -- & 437\\
$(g;1,0,1;38)$ & 8 & $(4,1)$ & 3 & 2 & YES & YES & YES & $1.36$ & $(4,1)$ & -- & 438\\
$(g;1,0,1;38)$ & 8 & $(7,2)$ & 4 & 1 & YES & YES & YES & $1.45$ & $(4,1)$ & -- & 439\\
$(g;1,1,0;9)$ & 8 & $(2,1)$ & 1 & 1 & YES & YES & YES & $1.00$ & $(6,0)$ & -- & 440\\
$(g;1,1,0;9)$ & 8 & $(3,1)$ & 2 & 3 & YES & YES & YES & $0.88$ & $(6,0)$ & -- & 441
\end{longtable}
\subsection{2 chains, $K^2 = 3$}
\begin{longtable}{|c|c|c|c|c|c|c|c|c|c|c|c|}
\hline
\multicolumn{12}{|c|}{2 chains, $K^2 = 3$}\\
\hline
$(n,a)$ & Len & $(n,a)$ & Len & GCD & Nef & $\mathbb Q$-ef & Obs 0 & $\overline c_1^2 / \overline c_2$ & $(P,K)$ & WH & Index\\
\hline
\endfirsthead

\hline
$(n,a)$ & Len & $(n,a)$ & Len & GCD & Nef & $\mathbb Q$-ef & Obs 0 & $\overline c_1^2 / \overline c_2$ & $(P,K)$ & WH & Index\\
\hline
\endhead
\hline
\endfoot

$(29,11)$ & 7 & $(17,7)$ & 6 & 1 & YES & YES & NO(2) & $1.86$ & $(2,3)$ & -- & 442\\
$(29,12)$ & 7 & $(24,7)$ & 7 & 1 & YES & YES & YES & $1.82$ & $(2,3)$ & -- & 443\\
$(29,11)$ & 7 & $(29,8)$ & 7 & 29 & YES & YES & YES & $1.38$ & $(4,2)$ & -- & 444\\
$(29,12)$ & 7 & $(29,11)$ & 7 & 29 & YES & YES & YES & $1.78$ & $(2,3)$ & -- & 445\\
$(31,12)$ & 7 & $(21,8)$ & 6 & 1 & YES & YES & NO(2) & $1.93$ & $(2,3)$ & -- & 446\\
$(31,12)$ & 7 & $(24,7)$ & 7 & 1 & YES & YES & NO(2) & $1.86$ & $(2,3)$ & -- & 447\\
$(31,9)$ & 8 & $(29,11)$ & 7 & 1 & YES & YES & YES & $1.78$ & $(2,3)$ & -- & 448\\
$(31,12)$ & 7 & $(30,11)$ & 7 & 1 & YES & YES & YES & $1.78$ & $(2,3)$ & -- & 449\\
$(31,12)$ & 7 & $(31,9)$ & 8 & 31 & YES & YES & YES & $2.00$ & $(2,3)$ & -- & 450\\
$(31,12)$ & 7 & $(31,9)$ & 8 & 31 & YES & YES & YES & $2.00$ & $(2,3)$ & NO & 451\\
$(33,10)$ & 8 & $(31,12)$ & 7 & 1 & YES & YES & YES & $2.00$ & $(2,3)$ & -- & 452\\
$(34,13)$ & 7 & $(17,7)$ & 6 & 17 & YES & YES & NO(2) & $1.85$ & $(4,2)$ & -- & 453\\
$(34,13)$ & 7 & $(21,8)$ & 6 & 1 & YES & YES & YES & $1.50$ & $(4,2)$ & -- & 454\\
$(35,13)$ & 8 & $(35,8)$ & 8 & 35 & YES & YES & YES & $1.62$ & $(4,2)$ & -- & 455\\
$(37,11)$ & 8 & $(23,7)$ & 7 & 1 & YES & YES & NO(2) & $1.85$ & $(4,2)$ & -- & 456\\
$(37,11)$ & 8 & $(24,7)$ & 7 & 1 & YES & YES & YES & $1.38$ & $(4,2)$ & -- & 457\\
$(37,11)$ & 8 & $(30,11)$ & 7 & 1 & YES & YES & YES & $1.78$ & $(2,3)$ & -- & 458\\
$(37,8)$ & 8 & $(35,13)$ & 8 & 1 & YES & YES & YES & $1.50$ & $(4,2)$ & NO & 459\\
$(40,11)$ & 8 & $(21,8)$ & 6 & 1 & YES & YES & YES & $1.82$ & $(2,3)$ & -- & 460\\
$(40,11)$ & 8 & $(24,7)$ & 7 & 8 & YES & YES & YES & $1.50$ & $(4,2)$ & -- & 461\\
$(40,11)$ & 8 & $(25,7)$ & 7 & 5 & YES & YES & YES & $1.38$ & $(4,2)$ & -- & 462\\
$(40,11)$ & 8 & $(27,8)$ & 7 & 1 & YES & YES & NO(2) & $1.73$ & $(4,2)$ & -- & 463\\
$(40,11)$ & 8 & $(35,8)$ & 8 & 5 & YES & YES & YES & $1.67$ & $(2,3)$ & -- & 464\\
$(40,11)$ & 8 & $(40,11)$ & 8 & 40 & YES & YES & YES & $1.71$ & $(2,3)$ & -- & 465\\
$(41,12)$ & 8 & $(17,7)$ & 6 & 1 & YES & YES & YES & $1.82$ & $(2,3)$ & -- & 466\\
$(41,17)$ & 8 & $(17,5)$ & 6 & 1 & YES & YES & NO(2) & $1.82$ & $(4,2)$ & -- & 467\\
$(41,12)$ & 8 & $(24,7)$ & 7 & 1 & YES & YES & YES & $1.38$ & $(4,2)$ & -- & 468\\
$(41,12)$ & 8 & $(27,8)$ & 7 & 1 & YES & YES & YES & $1.73$ & $(2,3)$ & -- & 469\\
$(41,12)$ & 8 & $(30,11)$ & 7 & 1 & YES & YES & YES & $1.67$ & $(2,3)$ & -- & 470\\
$(41,12)$ & 8 & $(31,12)$ & 7 & 1 & YES & YES & YES & $1.57$ & $(2,3)$ & -- & 471\\
$(43,18)$ & 8 & $(18,5)$ & 6 & 1 & YES & YES & NO(2) & $1.70$ & $(6,1)$ & -- & 472\\
$(43,12)$ & 8 & $(21,8)$ & 6 & 1 & YES & YES & YES & $1.91$ & $(2,3)$ & NO & 473\\
$(43,12)$ & 8 & $(21,8)$ & 6 & 1 & YES & YES & YES & $1.91$ & $(2,3)$ & -- & 474\\
$(43,12)$ & 8 & $(27,8)$ & 7 & 1 & YES & YES & YES & $1.73$ & $(2,3)$ & -- & 475\\
$(43,18)$ & 8 & $(37,8)$ & 8 & 1 & YES & YES & YES & $1.71$ & $(2,3)$ & NO & 476\\
$(43,18)$ & 8 & $(37,8)$ & 8 & 1 & YES & YES & YES & $1.71$ & $(2,3)$ & -- & 477\\
$(44,13)$ & 8 & $(24,7)$ & 7 & 4 & YES & YES & YES & $1.67$ & $(2,3)$ & -- & 478\\
$(44,13)$ & 8 & $(25,7)$ & 7 & 1 & YES & YES & YES & $1.67$ & $(2,3)$ & -- & 479\\
$(44,13)$ & 8 & $(27,8)$ & 7 & 1 & YES & YES & YES & $1.67$ & $(2,3)$ & -- & 480\\
$(44,17)$ & 8 & $(27,8)$ & 7 & 1 & YES & YES & YES & $1.57$ & $(2,3)$ & -- & 481\\
$(44,13)$ & 8 & $(29,8)$ & 7 & 1 & YES & YES & YES & $1.67$ & $(2,3)$ & -- & 482\\
$(46,17)$ & 8 & $(17,7)$ & 6 & 1 & YES & YES & YES & $1.75$ & $(4,2)$ & -- & 483\\
$(46,19)$ & 8 & $(17,5)$ & 6 & 1 & YES & YES & NO(2) & $1.69$ & $(4,2)$ & -- & 484\\
$(46,17)$ & 8 & $(31,13)$ & 7 & 1 & YES & YES & YES & $1.75$ & $(4,2)$ & NO & 485\\
$(47,18)$ & 8 & $(12,5)$ & 5 & 1 & YES & YES & YES & $1.82$ & $(2,3)$ & -- & 486\\
$(47,18)$ & 8 & $(17,5)$ & 6 & 1 & YES & YES & YES & $1.82$ & $(2,3)$ & -- & 487\\
$(47,18)$ & 8 & $(18,5)$ & 6 & 1 & YES & YES & YES & $1.67$ & $(2,3)$ & -- & 488\\
$(47,13)$ & 8 & $(21,8)$ & 6 & 1 & YES & YES & YES & $1.67$ & $(2,3)$ & -- & 489\\
$(47,18)$ & 8 & $(22,5)$ & 7 & 1 & YES & YES & YES & $1.67$ & $(2,3)$ & -- & 490\\
$(47,13)$ & 8 & $(24,7)$ & 7 & 1 & YES & YES & YES & $1.56$ & $(2,3)$ & -- & 491\\
$(47,13)$ & 8 & $(27,8)$ & 7 & 1 & YES & YES & YES & $1.56$ & $(2,3)$ & -- & 492\\
$(47,18)$ & 8 & $(29,8)$ & 7 & 1 & YES & YES & YES & $1.57$ & $(2,3)$ & -- & 493\\
$(47,18)$ & 8 & $(35,8)$ & 8 & 1 & YES & YES & YES & $1.57$ & $(2,3)$ & -- & 494\\
$(48,11)$ & 9 & $(25,7)$ & 7 & 1 & YES & YES & YES & $1.38$ & $(4,2)$ & -- & 495\\
$(48,11)$ & 9 & $(31,9)$ & 8 & 1 & YES & YES & YES & $1.78$ & $(2,3)$ & -- & 496\\
$(48,11)$ & 9 & $(32,9)$ & 8 & 16 & YES & YES & YES & $1.78$ & $(2,3)$ & -- & 497\\
$(48,13)$ & 9 & $(32,7)$ & 8 & 16 & YES & YES & YES & $1.50$ & $(4,2)$ & -- & 498\\
$(48,11)$ & 9 & $(40,11)$ & 8 & 8 & YES & YES & YES & $1.57$ & $(2,3)$ & -- & 499\\
$(49,19)$ & 8 & $(18,5)$ & 6 & 1 & YES & YES & YES & $1.38$ & $(4,2)$ & -- & 500\\
$(50,19)$ & 8 & $(13,5)$ & 5 & 1 & YES & YES & NO(2) & $1.79$ & $(2,3)$ & -- & 501\\
$(51,11)$ & 9 & $(41,9)$ & 9 & 1 & YES & YES & YES & $1.89$ & $(2,3)$ & -- & 502\\
$(55,21)$ & 8 & $(11,4)$ & 5 & 11 & YES & YES & YES & $1.73$ & $(2,3)$ & -- & 503\\
$(55,21)$ & 8 & $(13,5)$ & 5 & 1 & YES & YES & YES & $1.78$ & $(2,3)$ & -- & 504\\
$(55,21)$ & 8 & $(17,5)$ & 6 & 1 & YES & YES & YES & $1.56$ & $(2,3)$ & -- & 505\\
$(55,21)$ & 8 & $(18,5)$ & 6 & 1 & YES & YES & YES & $1.67$ & $(2,3)$ & -- & 506\\
$(55,21)$ & 8 & $(19,8)$ & 6 & 1 & YES & YES & YES & $1.78$ & $(2,3)$ & -- & 507\\
$(56,23)$ & 9 & $(17,5)$ & 6 & 1 & YES & YES & YES & $1.62$ & $(4,2)$ & -- & 508\\
$(56,17)$ & 9 & $(18,7)$ & 6 & 2 & YES & YES & YES & $1.78$ & $(2,3)$ & -- & 509\\
$(56,13)$ & 10 & $(32,7)$ & 8 & 8 & YES & YES & YES & $1.62$ & $(4,2)$ & -- & 510\\
$(57,22)$ & 9 & $(23,5)$ & 7 & 1 & YES & YES & YES & $1.62$ & $(4,2)$ & -- & 511\\
$(57,13)$ & 9 & $(25,7)$ & 7 & 1 & YES & YES & YES & $1.67$ & $(2,3)$ & -- & 512\\
$(57,13)$ & 9 & $(31,9)$ & 8 & 1 & YES & YES & YES & $1.78$ & $(2,3)$ & -- & 513\\
$(58,17)$ & 9 & $(17,5)$ & 6 & 1 & YES & YES & YES & $1.82$ & $(2,3)$ & -- & 514\\
$(58,17)$ & 9 & $(18,5)$ & 6 & 2 & YES & YES & YES & $1.50$ & $(4,2)$ & -- & 515\\
$(58,17)$ & 9 & $(19,7)$ & 6 & 1 & YES & YES & YES & $1.67$ & $(2,3)$ & -- & 516\\
$(58,17)$ & 9 & $(19,8)$ & 6 & 1 & YES & YES & YES & $1.78$ & $(2,3)$ & -- & 517\\
$(58,17)$ & 9 & $(24,7)$ & 7 & 2 & YES & YES & YES & $1.86$ & $(2,3)$ & -- & 518\\
$(58,17)$ & 9 & $(25,7)$ & 7 & 1 & YES & YES & YES & $1.43$ & $(2,3)$ & -- & 519\\
$(58,17)$ & 9 & $(31,7)$ & 8 & 1 & YES & YES & YES & $1.57$ & $(2,3)$ & -- & 520\\
$(60,23)$ & 9 & $(13,5)$ & 5 & 1 & YES & YES & YES & $1.89$ & $(2,3)$ & -- & 521\\
$(60,23)$ & 9 & $(17,5)$ & 6 & 1 & YES & YES & YES & $1.89$ & $(2,3)$ & -- & 522\\
$(60,23)$ & 9 & $(18,5)$ & 6 & 6 & YES & YES & YES & $1.67$ & $(2,3)$ & -- & 523\\
$(60,13)$ & 9 & $(22,9)$ & 7 & 2 & YES & YES & YES & $1.50$ & $(4,2)$ & NO & 524\\
$(60,13)$ & 9 & $(22,9)$ & 7 & 2 & YES & YES & YES & $1.75$ & $(4,2)$ & -- & 525\\
$(60,13)$ & 9 & $(32,9)$ & 8 & 4 & YES & YES & YES & $1.50$ & $(4,2)$ & NO & 526\\
$(61,18)$ & 9 & $(15,4)$ & 6 & 1 & YES & YES & YES & $1.56$ & $(2,3)$ & -- & 527\\
$(61,17)$ & 9 & $(17,7)$ & 6 & 1 & YES & YES & YES & $1.89$ & $(2,3)$ & -- & 528\\
$(61,18)$ & 9 & $(17,5)$ & 6 & 1 & YES & YES & YES & $1.56$ & $(2,3)$ & -- & 529\\
$(61,18)$ & 9 & $(22,5)$ & 7 & 1 & YES & YES & YES & $1.56$ & $(2,3)$ & -- & 530\\
$(61,17)$ & 9 & $(25,7)$ & 7 & 1 & YES & YES & YES & $1.43$ & $(2,3)$ & -- & 531\\
$(62,23)$ & 9 & $(13,5)$ & 5 & 1 & YES & YES & YES & $1.75$ & $(4,2)$ & -- & 532\\
$(63,26)$ & 9 & $(10,3)$ & 5 & 1 & YES & YES & NO(2) & $1.77$ & $(4,2)$ & -- & 533\\
$(63,26)$ & 9 & $(13,5)$ & 5 & 1 & YES & YES & YES & $1.89$ & $(2,3)$ & -- & 534\\
$(63,17)$ & 9 & $(24,7)$ & 7 & 3 & YES & YES & YES & $1.43$ & $(2,3)$ & -- & 535\\
$(64,19)$ & 9 & $(13,5)$ & 5 & 1 & YES & YES & YES & $1.82$ & $(2,3)$ & -- & 536\\
$(64,19)$ & 9 & $(18,7)$ & 6 & 2 & YES & YES & YES & $1.71$ & $(2,3)$ & -- & 537\\
$(65,18)$ & 9 & $(12,5)$ & 5 & 1 & YES & YES & YES & $1.82$ & $(2,3)$ & NO & 538\\
$(65,19)$ & 9 & $(12,5)$ & 5 & 1 & YES & YES & YES & $1.14$ & $(6,1)$ & -- & 539\\
$(65,19)$ & 9 & $(13,4)$ & 6 & 13 & YES & YES & YES & $1.82$ & $(2,3)$ & -- & 540\\
$(65,19)$ & 9 & $(13,5)$ & 5 & 13 & YES & YES & YES & $1.38$ & $(4,2)$ & -- & 541\\
$(65,19)$ & 9 & $(17,5)$ & 6 & 1 & YES & YES & NO(2) & $1.73$ & $(4,2)$ & -- & 542\\
$(65,18)$ & 9 & $(18,5)$ & 6 & 1 & YES & YES & YES & $1.56$ & $(2,3)$ & -- & 543\\
$(65,18)$ & 9 & $(22,5)$ & 7 & 1 & YES & YES & YES & $1.67$ & $(2,3)$ & -- & 544\\
$(65,19)$ & 9 & $(64,19)$ & 9 & 1 & YES & YES & YES & $1.82$ & $(2,3)$ & NO & 545\\
$(67,26)$ & 9 & $(14,3)$ & 6 & 1 & YES & YES & NO(2) & $1.79$ & $(2,3)$ & NO & 546\\
$(67,26)$ & 9 & $(18,5)$ & 6 & 1 & YES & YES & YES & $1.78$ & $(2,3)$ & NO & 547\\
$(67,26)$ & 9 & $(30,11)$ & 7 & 1 & YES & YES & YES & $1.78$ & $(2,3)$ & NO & 548\\
$(67,18)$ & 9 & $(37,11)$ & 8 & 1 & YES & YES & YES & $2.00$ & $(2,3)$ & NO & 549\\
$(68,19)$ & 9 & $(11,4)$ & 5 & 1 & YES & YES & YES & $1.91$ & $(2,3)$ & NO & 550\\
$(68,19)$ & 9 & $(11,4)$ & 5 & 1 & YES & YES & YES & $1.91$ & $(2,3)$ & -- & 551\\
$(68,19)$ & 9 & $(18,5)$ & 6 & 2 & YES & YES & YES & $1.67$ & $(2,3)$ & -- & 552\\
$(69,19)$ & 9 & $(17,7)$ & 6 & 1 & YES & YES & YES & $2.00$ & $(2,3)$ & -- & 553\\
$(69,29)$ & 9 & $(17,5)$ & 6 & 1 & YES & YES & YES & $1.71$ & $(2,3)$ & -- & 554\\
$(69,19)$ & 9 & $(18,7)$ & 6 & 3 & YES & YES & YES & $1.71$ & $(2,3)$ & -- & 555\\
$(71,27)$ & 9 & $(7,3)$ & 4 & 1 & YES & YES & NO(2) & $1.86$ & $(2,3)$ & -- & 556\\
$(71,21)$ & 9 & $(11,4)$ & 5 & 1 & YES & YES & NO(2) & $1.77$ & $(4,2)$ & NO & 557\\
$(71,21)$ & 9 & $(11,4)$ & 5 & 1 & YES & YES & NO(2) & $1.77$ & $(4,2)$ & -- & 558\\
$(71,21)$ & 9 & $(13,5)$ & 5 & 1 & YES & YES & YES & $1.67$ & $(2,3)$ & -- & 559\\
$(71,21)$ & 9 & $(17,7)$ & 6 & 1 & YES & YES & YES & $1.78$ & $(2,3)$ & -- & 560\\
$(71,21)$ & 9 & $(18,5)$ & 6 & 1 & YES & YES & YES & $1.56$ & $(2,3)$ & -- & 561\\
$(71,27)$ & 9 & $(18,5)$ & 6 & 1 & YES & YES & YES & $2.00$ & $(2,3)$ & -- & 562\\
$(71,21)$ & 9 & $(22,5)$ & 7 & 1 & YES & YES & YES & $1.67$ & $(2,3)$ & -- & 563\\
$(71,27)$ & 9 & $(22,5)$ & 7 & 1 & YES & YES & YES & $1.67$ & $(2,3)$ & -- & 564\\
$(74,31)$ & 9 & $(10,3)$ & 5 & 2 & YES & YES & YES & $1.73$ & $(2,3)$ & -- & 565\\
$(74,31)$ & 9 & $(11,4)$ & 5 & 1 & YES & YES & YES & $1.50$ & $(4,2)$ & -- & 566\\
$(74,31)$ & 9 & $(12,5)$ & 5 & 2 & YES & YES & YES & $1.50$ & $(4,2)$ & -- & 567\\
$(75,29)$ & 9 & $(10,3)$ & 5 & 5 & YES & YES & YES & $1.60$ & $(4,2)$ & -- & 568\\
$(75,31)$ & 9 & $(10,3)$ & 5 & 5 & YES & YES & NO(2) & $1.73$ & $(4,2)$ & -- & 569\\
$(75,31)$ & 9 & $(11,3)$ & 5 & 1 & YES & YES & YES & $1.67$ & $(2,3)$ & -- & 570\\
$(75,29)$ & 9 & $(13,5)$ & 5 & 1 & YES & YES & YES & $1.89$ & $(2,3)$ & -- & 571\\
$(75,29)$ & 9 & $(17,5)$ & 6 & 1 & YES & YES & YES & $1.71$ & $(2,3)$ & -- & 572\\
$(75,31)$ & 9 & $(17,5)$ & 6 & 1 & YES & YES & YES & $1.43$ & $(2,3)$ & -- & 573\\
$(75,17)$ & 10 & $(25,7)$ & 7 & 25 & YES & YES & YES & $1.43$ & $(2,3)$ & -- & 574\\
$(76,29)$ & 9 & $(10,3)$ & 5 & 2 & YES & YES & YES & $1.62$ & $(4,2)$ & -- & 575\\
$(76,21)$ & 9 & $(11,4)$ & 5 & 1 & YES & YES & YES & $1.14$ & $(6,1)$ & -- & 576\\
$(76,21)$ & 9 & $(17,7)$ & 6 & 1 & YES & YES & YES & $2.00$ & $(2,3)$ & -- & 577\\
$(76,21)$ & 9 & $(17,7)$ & 6 & 1 & YES & YES & YES & $2.00$ & $(2,3)$ & NO & 578\\
$(76,29)$ & 9 & $(23,5)$ & 7 & 1 & YES & YES & YES & $1.78$ & $(2,3)$ & -- & 579\\
$(76,21)$ & 9 & $(37,11)$ & 8 & 1 & YES & YES & YES & $2.00$ & $(2,3)$ & NO & 580\\
$(78,23)$ & 10 & $(23,5)$ & 7 & 1 & YES & YES & YES & $1.57$ & $(2,3)$ & -- & 581\\
$(79,30)$ & 9 & $(7,3)$ & 4 & 1 & YES & YES & NO(2) & $1.77$ & $(4,2)$ & -- & 582\\
$(79,30)$ & 9 & $(8,3)$ & 4 & 1 & YES & YES & YES & $1.38$ & $(4,2)$ & -- & 583\\
$(79,30)$ & 9 & $(10,3)$ & 5 & 1 & YES & YES & YES & $1.50$ & $(4,2)$ & -- & 584\\
$(79,18)$ & 10 & $(11,4)$ & 5 & 1 & YES & YES & YES & $1.91$ & $(2,3)$ & -- & 585\\
$(79,18)$ & 10 & $(11,4)$ & 5 & 1 & YES & YES & YES & $1.91$ & $(2,3)$ & NO & 586\\
$(79,18)$ & 10 & $(12,5)$ & 5 & 1 & YES & YES & YES & $1.14$ & $(6,1)$ & NO & 587\\
$(79,18)$ & 10 & $(13,5)$ & 5 & 1 & YES & YES & YES & $1.91$ & $(2,3)$ & NO & 588\\
$(79,23)$ & 10 & $(13,5)$ & 5 & 1 & YES & YES & YES & $1.78$ & $(2,3)$ & -- & 589\\
$(79,30)$ & 9 & $(13,3)$ & 6 & 1 & YES & YES & NO(2) & $1.86$ & $(2,3)$ & NO & 590\\
$(79,30)$ & 9 & $(13,3)$ & 6 & 1 & YES & YES & NO(2) & $1.86$ & $(2,3)$ & -- & 591\\
$(79,23)$ & 10 & $(15,4)$ & 6 & 1 & YES & YES & YES & $1.75$ & $(4,2)$ & -- & 592\\
$(79,23)$ & 10 & $(18,5)$ & 6 & 1 & YES & YES & YES & $1.78$ & $(2,3)$ & -- & 593\\
$(80,31)$ & 9 & $(55,21)$ & 8 & 5 & YES & YES & YES & $1.78$ & $(2,3)$ & NO & 594\\
$(81,31)$ & 9 & $(7,3)$ & 4 & 1 & YES & YES & YES & $1.82$ & $(2,3)$ & -- & 595\\
$(81,31)$ & 9 & $(10,3)$ & 5 & 1 & YES & YES & YES & $1.82$ & $(2,3)$ & -- & 596\\
$(81,34)$ & 9 & $(11,3)$ & 5 & 1 & YES & YES & NO(2) & $1.73$ & $(4,2)$ & NO & 597\\
$(81,34)$ & 9 & $(13,5)$ & 5 & 1 & YES & YES & YES & $1.50$ & $(4,2)$ & -- & 598\\
$(81,31)$ & 9 & $(50,19)$ & 8 & 1 & YES & YES & NO(2) & $1.79$ & $(2,3)$ & NO & 599\\
$(82,23)$ & 10 & $(69,19)$ & 9 & 1 & YES & YES & YES & $2.00$ & $(2,3)$ & NO & 600\\
$(82,23)$ & 10 & $(76,21)$ & 9 & 2 & YES & YES & YES & $2.00$ & $(2,3)$ & NO & 601\\
$(89,34)$ & 9 & $(5,2)$ & 3 & 1 & YES & YES & NO(2) & $1.93$ & $(2,3)$ & -- & 602\\
$(89,34)$ & 9 & $(7,3)$ & 4 & 1 & YES & YES & YES & $1.82$ & $(2,3)$ & -- & 603\\
$(89,34)$ & 9 & $(10,3)$ & 5 & 1 & YES & YES & YES & $1.67$ & $(2,3)$ & -- & 604\\
$(89,26)$ & 10 & $(14,3)$ & 6 & 1 & YES & YES & NO(2) & $1.79$ & $(2,3)$ & NO & 605\\
$(89,26)$ & 10 & $(18,5)$ & 6 & 1 & YES & YES & YES & $1.86$ & $(2,3)$ & -- & 606\\
$(91,27)$ & 10 & $(10,3)$ & 5 & 1 & YES & YES & YES & $1.82$ & $(2,3)$ & -- & 607\\
$(91,27)$ & 10 & $(13,4)$ & 6 & 13 & YES & YES & YES & $1.78$ & $(2,3)$ & -- & 608\\
$(94,39)$ & 10 & $(8,3)$ & 4 & 2 & YES & YES & YES & $1.50$ & $(4,2)$ & -- & 609\\
$(94,39)$ & 10 & $(11,3)$ & 5 & 1 & YES & YES & YES & $1.62$ & $(4,2)$ & -- & 610\\
$(94,39)$ & 10 & $(11,3)$ & 5 & 1 & YES & YES & YES & $1.62$ & $(4,2)$ & NO & 611\\
$(97,37)$ & 10 & $(11,3)$ & 5 & 1 & YES & YES & YES & $1.67$ & $(2,3)$ & -- & 612\\
$(97,37)$ & 10 & $(14,3)$ & 6 & 1 & YES & YES & YES & $1.67$ & $(2,3)$ & -- & 613\\
$(97,37)$ & 10 & $(14,3)$ & 6 & 1 & YES & YES & YES & $1.67$ & $(2,3)$ & NO & 614\\
$(98,29)$ & 10 & $(7,3)$ & 4 & 7 & YES & YES & YES & $1.82$ & $(2,3)$ & -- & 615\\
$(98,29)$ & 10 & $(10,3)$ & 5 & 2 & YES & YES & YES & $1.67$ & $(2,3)$ & -- & 616\\
$(98,41)$ & 10 & $(10,3)$ & 5 & 2 & YES & YES & YES & $1.50$ & $(4,2)$ & -- & 617\\
$(98,27)$ & 10 & $(11,4)$ & 5 & 1 & YES & YES & YES & $1.75$ & $(4,2)$ & -- & 618\\
$(98,29)$ & 10 & $(11,3)$ & 5 & 1 & YES & YES & YES & $1.67$ & $(2,3)$ & -- & 619\\
$(98,27)$ & 10 & $(13,4)$ & 6 & 1 & YES & YES & YES & $1.67$ & $(2,3)$ & -- & 620\\
$(98,29)$ & 10 & $(13,3)$ & 6 & 1 & YES & YES & YES & $1.67$ & $(2,3)$ & -- & 621\\
$(98,29)$ & 10 & $(14,3)$ & 6 & 14 & YES & YES & YES & $1.56$ & $(2,3)$ & -- & 622\\
$(98,41)$ & 10 & $(69,29)$ & 9 & 1 & YES & YES & YES & $1.50$ & $(4,2)$ & 695 & 623\\
$(99,29)$ & 10 & $(7,3)$ & 4 & 1 & YES & YES & YES & $1.82$ & $(2,3)$ & -- & 624\\
$(99,41)$ & 10 & $(7,2)$ & 4 & 1 & YES & YES & NO(2) & $1.73$ & $(4,2)$ & -- & 625\\
$(99,29)$ & 10 & $(10,3)$ & 5 & 1 & YES & YES & YES & $1.82$ & $(2,3)$ & -- & 626\\
$(99,29)$ & 10 & $(17,5)$ & 6 & 1 & YES & YES & YES & $1.57$ & $(2,3)$ & -- & 627\\
$(101,39)$ & 10 & $(5,2)$ & 3 & 1 & YES & YES & NO(2) & $1.93$ & $(2,3)$ & -- & 628\\
$(101,30)$ & 10 & $(10,3)$ & 5 & 1 & YES & YES & NO(2) & $1.73$ & $(4,2)$ & -- & 629\\
$(101,39)$ & 10 & $(10,3)$ & 5 & 1 & YES & YES & YES & $2.00$ & $(2,3)$ & NO & 630\\
$(101,39)$ & 10 & $(10,3)$ & 5 & 1 & YES & YES & YES & $2.00$ & $(2,3)$ & -- & 631\\
$(101,30)$ & 10 & $(11,3)$ & 5 & 1 & YES & YES & NO(2) & $1.64$ & $(4,2)$ & -- & 632\\
$(101,30)$ & 10 & $(11,4)$ & 5 & 1 & YES & YES & YES & $1.67$ & $(2,3)$ & -- & 633\\
$(101,30)$ & 10 & $(13,4)$ & 6 & 1 & YES & YES & YES & $1.89$ & $(2,3)$ & -- & 634\\
$(101,30)$ & 10 & $(78,23)$ & 10 & 1 & YES & YES & YES & $1.67$ & $(2,3)$ & NO & 635\\
$(102,31)$ & 11 & $(10,3)$ & 5 & 2 & YES & YES & YES & $1.62$ & $(4,2)$ & -- & 636\\
$(102,31)$ & 11 & $(11,3)$ & 5 & 1 & YES & YES & YES & $1.62$ & $(4,2)$ & -- & 637\\
$(104,43)$ & 10 & $(5,2)$ & 3 & 1 & YES & YES & NO(2) & $1.85$ & $(4,2)$ & -- & 638\\
$(104,43)$ & 10 & $(9,2)$ & 5 & 1 & YES & YES & YES & $1.91$ & $(2,3)$ & NO & 639\\
$(104,43)$ & 10 & $(9,2)$ & 5 & 1 & YES & YES & YES & $1.91$ & $(2,3)$ & -- & 640\\
$(104,29)$ & 10 & $(13,5)$ & 5 & 13 & YES & YES & YES & $1.67$ & $(2,3)$ & -- & 641\\
$(104,43)$ & 10 & $(63,26)$ & 9 & 1 & YES & YES & NO(2) & $1.69$ & $(4,2)$ & 886 & 642\\
$(105,31)$ & 10 & $(7,3)$ & 4 & 7 & YES & YES & YES & $1.14$ & $(6,1)$ & -- & 643\\
$(105,44)$ & 10 & $(7,2)$ & 4 & 7 & YES & YES & NO(2) & $1.64$ & $(4,2)$ & -- & 644\\
$(105,31)$ & 10 & $(8,3)$ & 4 & 1 & YES & YES & YES & $1.82$ & $(2,3)$ & NO & 645\\
$(105,31)$ & 10 & $(8,3)$ & 4 & 1 & YES & YES & YES & $1.82$ & $(2,3)$ & -- & 646\\
$(105,44)$ & 10 & $(8,3)$ & 4 & 1 & YES & YES & YES & $1.62$ & $(4,2)$ & -- & 647\\
$(105,31)$ & 10 & $(10,3)$ & 5 & 5 & YES & YES & YES & $1.73$ & $(2,3)$ & -- & 648\\
$(105,29)$ & 10 & $(11,4)$ & 5 & 1 & YES & YES & YES & $1.62$ & $(4,2)$ & -- & 649\\
$(105,31)$ & 10 & $(11,4)$ & 5 & 1 & YES & YES & YES & $1.67$ & $(2,3)$ & -- & 650\\
$(105,31)$ & 10 & $(64,19)$ & 9 & 1 & YES & YES & YES & $1.82$ & $(2,3)$ & NO & 651\\
$(106,31)$ & 10 & $(7,3)$ & 4 & 1 & YES & YES & YES & $1.82$ & $(2,3)$ & -- & 652\\
$(106,31)$ & 10 & $(10,3)$ & 5 & 2 & YES & YES & NO(2) & $1.73$ & $(4,2)$ & -- & 653\\
$(107,41)$ & 10 & $(7,3)$ & 4 & 1 & YES & YES & YES & $1.62$ & $(4,2)$ & -- & 654\\
$(107,41)$ & 10 & $(10,3)$ & 5 & 1 & YES & YES & YES & $1.89$ & $(2,3)$ & -- & 655\\
$(107,41)$ & 10 & $(89,34)$ & 9 & 1 & YES & YES & YES & $1.78$ & $(2,3)$ & NO & 656\\
$(108,29)$ & 10 & $(13,4)$ & 6 & 1 & YES & YES & YES & $1.50$ & $(4,2)$ & -- & 657\\
$(108,29)$ & 10 & $(24,7)$ & 7 & 12 & YES & YES & YES & $1.62$ & $(4,2)$ & NO & 658\\
$(108,29)$ & 10 & $(32,9)$ & 8 & 4 & YES & YES & YES & $1.50$ & $(4,2)$ & NO & 659\\
$(109,45)$ & 10 & $(5,2)$ & 3 & 1 & YES & YES & NO(2) & $1.73$ & $(4,2)$ & -- & 660\\
$(109,30)$ & 10 & $(7,3)$ & 4 & 1 & YES & YES & NO(2) & $1.69$ & $(4,2)$ & NO & 661\\
$(109,40)$ & 10 & $(10,3)$ & 5 & 1 & YES & YES & YES & $1.78$ & $(2,3)$ & -- & 662\\
$(109,30)$ & 10 & $(11,3)$ & 5 & 1 & YES & YES & NO(2) & $1.91$ & $(4,2)$ & -- & 663\\
$(109,45)$ & 10 & $(11,3)$ & 5 & 1 & YES & YES & YES & $1.78$ & $(2,3)$ & NO & 664\\
$(109,45)$ & 10 & $(11,3)$ & 5 & 1 & YES & YES & YES & $1.78$ & $(2,3)$ & -- & 665\\
$(109,30)$ & 10 & $(13,4)$ & 6 & 1 & YES & YES & YES & $1.67$ & $(2,3)$ & -- & 666\\
$(109,45)$ & 10 & $(31,13)$ & 7 & 1 & YES & YES & YES & $1.75$ & $(4,2)$ & NO & 667\\
$(109,45)$ & 10 & $(41,17)$ & 8 & 1 & YES & YES & NO(2) & $1.82$ & $(4,2)$ & NO & 668\\
$(111,31)$ & 10 & $(7,3)$ & 4 & 1 & YES & YES & YES & $1.82$ & $(2,3)$ & -- & 669\\
$(111,43)$ & 10 & $(10,3)$ & 5 & 1 & YES & YES & YES & $1.71$ & $(2,3)$ & -- & 670\\
$(111,46)$ & 10 & $(11,3)$ & 5 & 1 & YES & YES & YES & $1.43$ & $(2,3)$ & -- & 671\\
$(111,31)$ & 10 & $(40,11)$ & 8 & 1 & YES & YES & YES & $1.82$ & $(2,3)$ & NO & 672\\
$(112,47)$ & 10 & $(8,3)$ & 4 & 8 & YES & YES & YES & $1.62$ & $(4,2)$ & -- & 673\\
$(112,31)$ & 10 & $(11,4)$ & 5 & 1 & YES & YES & YES & $1.78$ & $(2,3)$ & -- & 674\\
$(115,34)$ & 10 & $(7,3)$ & 4 & 1 & YES & YES & YES & $1.14$ & $(6,1)$ & -- & 675\\
$(115,44)$ & 10 & $(7,2)$ & 4 & 1 & YES & YES & YES & $1.78$ & $(2,3)$ & -- & 676\\
$(115,34)$ & 10 & $(10,3)$ & 5 & 5 & YES & YES & YES & $1.56$ & $(2,3)$ & -- & 677\\
$(115,34)$ & 10 & $(11,3)$ & 5 & 1 & YES & YES & YES & $1.56$ & $(2,3)$ & -- & 678\\
$(115,44)$ & 10 & $(11,3)$ & 5 & 1 & YES & YES & YES & $1.86$ & $(2,3)$ & -- & 679\\
$(116,45)$ & 10 & $(5,2)$ & 3 & 1 & YES & YES & NO(2) & $1.79$ & $(2,3)$ & -- & 680\\
$(116,45)$ & 10 & $(21,8)$ & 6 & 1 & YES & YES & NO(2) & $1.71$ & $(2,3)$ & NO & 681\\
$(117,49)$ & 10 & $(5,2)$ & 3 & 1 & YES & YES & NO(2) & $1.73$ & $(4,2)$ & -- & 682\\
$(117,49)$ & 10 & $(7,2)$ & 4 & 1 & YES & YES & YES & $1.82$ & $(2,3)$ & NO & 683\\
$(117,43)$ & 10 & $(12,5)$ & 5 & 3 & YES & YES & YES & $1.62$ & $(4,2)$ & NO & 684\\
$(117,49)$ & 10 & $(105,44)$ & 10 & 3 & YES & YES & NO(2) & $1.73$ & $(4,2)$ & NO & 685\\
$(118,27)$ & 11 & $(17,4)$ & 7 & 1 & YES & YES & YES & $1.67$ & $(2,3)$ & -- & 686\\
$(118,27)$ & 11 & $(71,16)$ & 10 & 1 & YES & YES & YES & $2.00$ & $(2,3)$ & NO & 687\\
$(119,46)$ & 10 & $(5,2)$ & 3 & 1 & YES & YES & YES & $1.62$ & $(4,2)$ & -- & 688\\
$(119,50)$ & 10 & $(7,3)$ & 4 & 7 & YES & YES & YES & $1.75$ & $(4,2)$ & -- & 689\\
$(119,46)$ & 10 & $(9,2)$ & 5 & 1 & YES & YES & YES & $1.60$ & $(4,2)$ & NO & 690\\
$(119,46)$ & 10 & $(9,2)$ & 5 & 1 & YES & YES & YES & $1.60$ & $(4,2)$ & -- & 691\\
$(119,50)$ & 10 & $(10,3)$ & 5 & 1 & YES & YES & YES & $1.78$ & $(2,3)$ & -- & 692\\
$(119,50)$ & 10 & $(11,3)$ & 5 & 1 & YES & YES & YES & $1.50$ & $(4,2)$ & -- & 693\\
$(119,46)$ & 10 & $(49,19)$ & 8 & 7 & YES & YES & YES & $1.78$ & $(2,3)$ & NO & 694\\
$(119,50)$ & 10 & $(55,23)$ & 9 & 1 & YES & YES & YES & $1.50$ & $(4,2)$ & 623 & 695\\
$(119,50)$ & 10 & $(74,31)$ & 9 & 1 & YES & YES & YES & $1.78$ & $(2,3)$ & NO & 696\\
$(121,46)$ & 10 & $(5,2)$ & 3 & 1 & YES & YES & YES & $1.67$ & $(2,3)$ & -- & 697\\
$(121,50)$ & 10 & $(5,2)$ & 3 & 1 & YES & YES & YES & $1.91$ & $(2,3)$ & -- & 698\\
$(121,46)$ & 10 & $(7,2)$ & 4 & 1 & YES & YES & YES & $1.50$ & $(4,2)$ & -- & 699\\
$(121,50)$ & 10 & $(7,2)$ & 4 & 1 & YES & YES & NO(2) & $1.73$ & $(4,2)$ & NO & 700\\
$(121,46)$ & 10 & $(8,3)$ & 4 & 1 & YES & YES & YES & $1.78$ & $(2,3)$ & -- & 701\\
$(121,50)$ & 10 & $(9,2)$ & 5 & 1 & YES & YES & YES & $1.64$ & $(2,3)$ & NO & 702\\
$(121,50)$ & 10 & $(19,8)$ & 6 & 1 & YES & YES & NO(2) & $1.82$ & $(4,2)$ & NO & 703\\
$(121,46)$ & 10 & $(79,30)$ & 9 & 1 & YES & YES & YES & $1.78$ & $(2,3)$ & NO & 704\\
$(121,50)$ & 10 & $(104,43)$ & 10 & 1 & YES & YES & YES & $1.91$ & $(2,3)$ & NO & 705\\
$(122,37)$ & 11 & $(7,3)$ & 4 & 1 & YES & YES & YES & $1.62$ & $(4,2)$ & -- & 706\\
$(122,33)$ & 11 & $(8,3)$ & 4 & 2 & YES & YES & YES & $1.50$ & $(4,2)$ & -- & 707\\
$(122,33)$ & 11 & $(8,3)$ & 4 & 2 & YES & YES & YES & $1.62$ & $(4,2)$ & NO & 708\\
$(122,37)$ & 11 & $(18,5)$ & 6 & 2 & YES & YES & YES & $1.78$ & $(2,3)$ & NO & 709\\
$(122,37)$ & 11 & $(102,31)$ & 11 & 2 & YES & YES & YES & $1.62$ & $(4,2)$ & NO & 710\\
$(123,34)$ & 10 & $(7,3)$ & 4 & 1 & YES & YES & YES & $1.14$ & $(6,1)$ & NO & 711\\
$(123,34)$ & 10 & $(7,3)$ & 4 & 1 & YES & YES & YES & $1.14$ & $(6,1)$ & -- & 712\\
$(123,47)$ & 10 & $(7,2)$ & 4 & 1 & YES & YES & YES & $1.78$ & $(2,3)$ & -- & 713\\
$(123,47)$ & 10 & $(9,2)$ & 5 & 3 & YES & YES & YES & $1.67$ & $(2,3)$ & -- & 714\\
$(123,47)$ & 10 & $(123,47)$ & 10 & 123 & YES & YES & NO(2) & $1.86$ & $(2,3)$ & NO & 715\\
$(125,37)$ & 11 & $(14,3)$ & 6 & 1 & YES & YES & YES & $1.67$ & $(2,3)$ & NO & 716\\
$(125,37)$ & 11 & $(105,31)$ & 10 & 5 & YES & YES & YES & $1.67$ & $(2,3)$ & 1557 & 717\\
$(128,49)$ & 10 & $(5,2)$ & 3 & 1 & YES & YES & YES & $1.67$ & $(2,3)$ & -- & 718\\
$(128,49)$ & 10 & $(7,2)$ & 4 & 1 & YES & YES & YES & $1.78$ & $(2,3)$ & -- & 719\\
$(128,53)$ & 11 & $(7,2)$ & 4 & 1 & YES & YES & YES & $1.62$ & $(4,2)$ & -- & 720\\
$(128,53)$ & 11 & $(9,2)$ & 5 & 1 & YES & YES & YES & $1.75$ & $(4,2)$ & -- & 721\\
$(128,47)$ & 10 & $(10,3)$ & 5 & 2 & YES & YES & YES & $1.50$ & $(4,2)$ & -- & 722\\
$(128,49)$ & 10 & $(11,3)$ & 5 & 1 & YES & YES & YES & $1.57$ & $(2,3)$ & -- & 723\\
$(128,47)$ & 10 & $(35,13)$ & 8 & 1 & YES & YES & YES & $1.50$ & $(4,2)$ & NO & 724\\
$(128,49)$ & 10 & $(115,44)$ & 10 & 1 & YES & YES & YES & $1.78$ & $(2,3)$ & NO & 725\\
$(129,50)$ & 10 & $(5,2)$ & 3 & 1 & YES & YES & NO(2) & $1.79$ & $(2,3)$ & -- & 726\\
$(129,49)$ & 10 & $(7,2)$ & 4 & 1 & YES & YES & YES & $1.73$ & $(2,3)$ & NO & 727\\
$(129,49)$ & 10 & $(7,2)$ & 4 & 1 & YES & YES & YES & $1.73$ & $(2,3)$ & -- & 728\\
$(129,49)$ & 10 & $(7,3)$ & 4 & 1 & YES & YES & NO(2) & $1.82$ & $(4,2)$ & NO & 729\\
$(129,50)$ & 10 & $(7,2)$ & 4 & 1 & YES & YES & NO(2) & $1.85$ & $(4,2)$ & NO & 730\\
$(129,50)$ & 10 & $(9,2)$ & 5 & 3 & YES & YES & YES & $1.50$ & $(4,2)$ & NO & 731\\
$(129,50)$ & 10 & $(11,4)$ & 5 & 1 & YES & YES & NO(2) & $1.77$ & $(4,2)$ & 1037 & 732\\
$(129,50)$ & 10 & $(21,8)$ & 6 & 3 & YES & YES & YES & $1.50$ & $(4,2)$ & NO & 733\\
$(129,50)$ & 10 & $(111,43)$ & 10 & 3 & YES & YES & NO(2) & $1.75$ & $(2,3)$ & NO & 734\\
$(131,55)$ & 10 & $(5,2)$ & 3 & 1 & YES & YES & NO(2) & $1.60$ & $(6,1)$ & -- & 735\\
$(131,50)$ & 10 & $(7,3)$ & 4 & 1 & YES & YES & YES & $1.82$ & $(2,3)$ & NO & 736\\
$(131,50)$ & 10 & $(8,3)$ & 4 & 1 & YES & YES & YES & $1.71$ & $(2,3)$ & -- & 737\\
$(131,55)$ & 10 & $(8,3)$ & 4 & 1 & YES & YES & YES & $1.50$ & $(4,2)$ & -- & 738\\
$(131,55)$ & 10 & $(8,3)$ & 4 & 1 & YES & YES & NO(2) & $1.70$ & $(6,1)$ & NO & 739\\
$(131,50)$ & 10 & $(13,3)$ & 6 & 1 & YES & YES & YES & $1.78$ & $(2,3)$ & NO & 740\\
$(131,55)$ & 10 & $(17,7)$ & 6 & 1 & YES & YES & NO(2) & $1.82$ & $(4,2)$ & NO & 741\\
$(131,55)$ & 10 & $(43,18)$ & 8 & 1 & YES & YES & NO(2) & $1.70$ & $(6,1)$ & NO & 742\\
$(131,55)$ & 10 & $(55,23)$ & 9 & 1 & YES & YES & YES & $1.50$ & $(4,2)$ & NO & 743\\
$(133,39)$ & 11 & $(8,3)$ & 4 & 1 & YES & YES & YES & $1.67$ & $(2,3)$ & -- & 744\\
$(133,31)$ & 12 & $(23,5)$ & 7 & 1 & YES & YES & YES & $1.62$ & $(4,2)$ & NO & 745\\
$(133,39)$ & 11 & $(89,26)$ & 10 & 1 & YES & YES & YES & $1.78$ & $(2,3)$ & 792 & 746\\
$(134,39)$ & 11 & $(8,3)$ & 4 & 2 & YES & YES & YES & $1.67$ & $(2,3)$ & -- & 747\\
$(134,37)$ & 11 & $(112,31)$ & 10 & 2 & YES & YES & YES & $1.78$ & $(2,3)$ & 1625 & 748\\
$(135,56)$ & 11 & $(5,2)$ & 3 & 5 & YES & YES & YES & $1.62$ & $(4,2)$ & -- & 749\\
$(135,41)$ & 11 & $(7,3)$ & 4 & 1 & YES & YES & YES & $1.62$ & $(4,2)$ & -- & 750\\
$(135,56)$ & 11 & $(7,2)$ & 4 & 1 & YES & YES & YES & $1.62$ & $(4,2)$ & NO & 751\\
$(135,56)$ & 11 & $(7,2)$ & 4 & 1 & YES & YES & YES & $1.75$ & $(4,2)$ & -- & 752\\
$(137,37)$ & 11 & $(7,3)$ & 4 & 1 & YES & YES & YES & $1.62$ & $(4,2)$ & -- & 753\\
$(138,41)$ & 11 & $(118,35)$ & 11 & 2 & YES & YES & YES & $1.67$ & $(2,3)$ & NO & 754\\
$(139,41)$ & 11 & $(13,3)$ & 6 & 1 & YES & YES & YES & $1.78$ & $(2,3)$ & -- & 755\\
$(140,41)$ & 11 & $(7,3)$ & 4 & 7 & YES & YES & YES & $1.78$ & $(2,3)$ & -- & 756\\
$(140,41)$ & 11 & $(10,3)$ & 5 & 10 & YES & YES & YES & $1.57$ & $(2,3)$ & -- & 757\\
$(144,55)$ & 10 & $(5,2)$ & 3 & 1 & YES & YES & YES & $1.67$ & $(2,3)$ & -- & 758\\
$(144,55)$ & 10 & $(7,2)$ & 4 & 1 & YES & YES & YES & $1.67$ & $(2,3)$ & -- & 759\\
$(144,55)$ & 10 & $(7,3)$ & 4 & 1 & YES & YES & YES & $1.78$ & $(2,3)$ & -- & 760\\
$(144,55)$ & 10 & $(8,3)$ & 4 & 8 & YES & YES & NO(2) & $1.86$ & $(2,3)$ & NO & 761\\
$(144,55)$ & 10 & $(8,3)$ & 4 & 8 & YES & YES & YES & $1.62$ & $(4,2)$ & -- & 762\\
$(144,55)$ & 10 & $(9,2)$ & 5 & 9 & YES & YES & YES & $1.56$ & $(2,3)$ & NO & 763\\
$(144,55)$ & 10 & $(18,7)$ & 6 & 18 & YES & YES & YES & $1.50$ & $(4,2)$ & NO & 764\\
$(144,55)$ & 10 & $(29,11)$ & 7 & 1 & YES & YES & YES & $1.38$ & $(4,2)$ & NO & 765\\
$(144,55)$ & 10 & $(47,18)$ & 8 & 1 & YES & YES & YES & $1.67$ & $(2,3)$ & NO & 766\\
$(144,55)$ & 10 & $(76,29)$ & 9 & 4 & YES & YES & YES & $1.78$ & $(2,3)$ & NO & 767\\
$(144,55)$ & 10 & $(97,37)$ & 10 & 1 & YES & YES & YES & $1.78$ & $(2,3)$ & NO & 768\\
$(144,55)$ & 10 & $(123,47)$ & 10 & 3 & YES & YES & YES & $1.67$ & $(2,3)$ & NO & 769\\
$(145,44)$ & 11 & $(7,3)$ & 4 & 1 & YES & YES & YES & $1.62$ & $(4,2)$ & -- & 770\\
$(145,56)$ & 11 & $(7,2)$ & 4 & 1 & YES & YES & YES & $1.62$ & $(4,2)$ & NO & 771\\
$(145,44)$ & 11 & $(102,31)$ & 11 & 1 & YES & YES & YES & $1.62$ & $(4,2)$ & NO & 772\\
$(147,43)$ & 11 & $(9,2)$ & 5 & 3 & YES & YES & YES & $1.38$ & $(4,2)$ & NO & 773\\
$(147,41)$ & 11 & $(11,3)$ & 5 & 1 & YES & YES & YES & $1.43$ & $(2,3)$ & -- & 774\\
$(147,43)$ & 11 & $(14,3)$ & 6 & 7 & YES & YES & YES & $1.57$ & $(2,3)$ & -- & 775\\
$(147,43)$ & 11 & $(89,26)$ & 10 & 1 & YES & YES & YES & $1.38$ & $(4,2)$ & 1249 & 776\\
$(147,61)$ & 11 & $(135,56)$ & 11 & 3 & YES & YES & YES & $1.75$ & $(4,2)$ & NO & 777\\
$(149,44)$ & 11 & $(5,2)$ & 3 & 1 & YES & YES & YES & $1.14$ & $(6,1)$ & -- & 778\\
$(149,40)$ & 11 & $(7,3)$ & 4 & 1 & YES & YES & YES & $1.75$ & $(4,2)$ & -- & 779\\
$(149,41)$ & 11 & $(13,4)$ & 6 & 1 & YES & YES & YES & $1.78$ & $(2,3)$ & NO & 780\\
$(149,44)$ & 11 & $(37,11)$ & 8 & 1 & YES & YES & YES & $1.38$ & $(4,2)$ & NO & 781\\
$(152,63)$ & 11 & $(5,2)$ & 3 & 1 & YES & YES & YES & $1.62$ & $(4,2)$ & -- & 782\\
$(152,63)$ & 11 & $(7,2)$ & 4 & 1 & YES & YES & YES & $1.78$ & $(2,3)$ & -- & 783\\
$(152,63)$ & 11 & $(8,3)$ & 4 & 8 & YES & YES & YES & $1.62$ & $(4,2)$ & NO & 784\\
$(154,45)$ & 11 & $(5,2)$ & 3 & 1 & YES & YES & NO(2) & $1.83$ & $(2,3)$ & -- & 785\\
$(154,59)$ & 11 & $(5,2)$ & 3 & 1 & YES & YES & YES & $1.67$ & $(2,3)$ & -- & 786\\
$(154,45)$ & 11 & $(7,3)$ & 4 & 7 & YES & YES & YES & $1.78$ & $(2,3)$ & -- & 787\\
$(154,45)$ & 11 & $(10,3)$ & 5 & 2 & YES & YES & YES & $1.71$ & $(2,3)$ & -- & 788\\
$(154,45)$ & 11 & $(11,3)$ & 5 & 11 & YES & YES & YES & $1.67$ & $(2,3)$ & -- & 789\\
$(154,45)$ & 11 & $(23,7)$ & 7 & 1 & YES & YES & YES & $2.00$ & $(2,3)$ & NO & 790\\
$(154,43)$ & 11 & $(29,8)$ & 7 & 1 & YES & YES & YES & $1.50$ & $(4,2)$ & NO & 791\\
$(154,45)$ & 11 & $(75,22)$ & 10 & 1 & YES & YES & YES & $1.78$ & $(2,3)$ & 746 & 792\\
$(155,64)$ & 11 & $(5,2)$ & 3 & 5 & YES & YES & YES & $1.75$ & $(4,2)$ & -- & 793\\
$(155,64)$ & 11 & $(7,2)$ & 4 & 1 & YES & YES & YES & $1.67$ & $(2,3)$ & -- & 794\\
$(155,46)$ & 11 & $(24,7)$ & 7 & 1 & YES & YES & YES & $1.91$ & $(2,3)$ & NO & 795\\
$(157,46)$ & 11 & $(5,2)$ & 3 & 1 & YES & YES & YES & $1.67$ & $(2,3)$ & -- & 796\\
$(157,46)$ & 11 & $(5,2)$ & 3 & 1 & YES & YES & YES & $1.78$ & $(2,3)$ & NO & 797\\
$(157,58)$ & 11 & $(7,2)$ & 4 & 1 & YES & YES & YES & $1.78$ & $(2,3)$ & -- & 798\\
$(157,58)$ & 11 & $(9,2)$ & 5 & 1 & YES & YES & YES & $1.78$ & $(2,3)$ & -- & 799\\
$(157,46)$ & 11 & $(10,3)$ & 5 & 1 & YES & YES & YES & $1.57$ & $(2,3)$ & -- & 800\\
$(157,46)$ & 11 & $(37,11)$ & 8 & 1 & YES & YES & YES & $1.78$ & $(2,3)$ & NO & 801\\
$(157,46)$ & 11 & $(140,41)$ & 11 & 1 & YES & YES & NO(2) & $1.79$ & $(2,3)$ & NO & 802\\
$(158,61)$ & 11 & $(75,29)$ & 9 & 1 & YES & YES & YES & $1.89$ & $(2,3)$ & NO & 803\\
$(158,61)$ & 11 & $(145,56)$ & 11 & 1 & YES & YES & YES & $1.89$ & $(2,3)$ & NO & 804\\
$(159,47)$ & 11 & $(5,2)$ & 3 & 1 & YES & YES & YES & $1.62$ & $(4,2)$ & -- & 805\\
$(159,47)$ & 11 & $(7,2)$ & 4 & 1 & YES & YES & YES & $1.67$ & $(2,3)$ & -- & 806\\
$(159,47)$ & 11 & $(9,2)$ & 5 & 3 & YES & YES & YES & $1.56$ & $(2,3)$ & -- & 807\\
$(159,44)$ & 11 & $(40,11)$ & 8 & 1 & YES & YES & YES & $1.50$ & $(4,2)$ & NO & 808\\
$(159,47)$ & 11 & $(98,29)$ & 10 & 1 & YES & YES & YES & $1.56$ & $(2,3)$ & 1370 & 809\\
$(161,68)$ & 11 & $(4,1)$ & 3 & 1 & YES & YES & YES & $1.73$ & $(2,3)$ & -- & 810\\
$(161,68)$ & 11 & $(5,1)$ & 4 & 1 & YES & YES & YES & $1.82$ & $(2,3)$ & NO & 811\\
$(161,68)$ & 11 & $(5,1)$ & 4 & 1 & YES & YES & YES & $1.82$ & $(2,3)$ & NO & 812\\
$(163,63)$ & 11 & $(75,29)$ & 9 & 1 & YES & YES & YES & $1.60$ & $(4,2)$ & 968 & 813\\
$(163,63)$ & 11 & $(119,46)$ & 10 & 1 & YES & YES & YES & $1.60$ & $(4,2)$ & NO & 814\\
$(165,64)$ & 11 & $(4,1)$ & 3 & 1 & YES & YES & NO(2) & $1.71$ & $(2,3)$ & NO & 815\\
$(165,61)$ & 11 & $(5,2)$ & 3 & 5 & YES & YES & YES & $1.67$ & $(2,3)$ & -- & 816\\
$(165,64)$ & 11 & $(165,64)$ & 11 & 165 & YES & YES & NO(2) & $1.71$ & $(2,3)$ & NO & 817\\
$(166,49)$ & 11 & $(5,2)$ & 3 & 1 & YES & YES & YES & $1.67$ & $(2,3)$ & -- & 818\\
$(166,49)$ & 11 & $(7,2)$ & 4 & 1 & YES & YES & YES & $1.73$ & $(2,3)$ & -- & 819\\
$(166,49)$ & 11 & $(37,11)$ & 8 & 1 & YES & YES & YES & $1.73$ & $(2,3)$ & 1350 & 820\\
$(166,49)$ & 11 & $(71,21)$ & 9 & 1 & YES & YES & YES & $1.67$ & $(2,3)$ & NO & 821\\
$(167,69)$ & 11 & $(2,1)$ & 1 & 1 & YES & YES & YES & $1.82$ & $(2,3)$ & -- & 822\\
$(167,69)$ & 11 & $(3,1)$ & 2 & 1 & YES & YES & YES & $1.82$ & $(2,3)$ & -- & 823\\
$(167,69)$ & 11 & $(5,1)$ & 4 & 1 & YES & YES & YES & $1.73$ & $(2,3)$ & NO & 824\\
$(167,69)$ & 11 & $(5,2)$ & 3 & 1 & YES & YES & YES & $1.75$ & $(4,2)$ & -- & 825\\
$(167,69)$ & 11 & $(29,12)$ & 7 & 1 & YES & YES & NO(2) & $1.77$ & $(4,2)$ & NO & 826\\
$(167,69)$ & 11 & $(167,69)$ & 11 & 167 & YES & YES & YES & $1.82$ & $(2,3)$ & NO & 827\\
$(169,70)$ & 11 & $(3,1)$ & 2 & 1 & YES & YES & NO(2) & $1.79$ & $(2,3)$ & NO & 828\\
$(169,50)$ & 11 & $(5,2)$ & 3 & 1 & YES & YES & YES & $1.38$ & $(4,2)$ & -- & 829\\
$(169,70)$ & 11 & $(5,2)$ & 3 & 1 & YES & YES & YES & $1.62$ & $(4,2)$ & -- & 830\\
$(169,71)$ & 11 & $(7,2)$ & 4 & 1 & YES & YES & YES & $1.67$ & $(2,3)$ & NO & 831\\
$(169,71)$ & 11 & $(17,7)$ & 6 & 1 & YES & YES & YES & $1.89$ & $(2,3)$ & NO & 832\\
$(169,71)$ & 11 & $(31,13)$ & 7 & 1 & YES & YES & NO(2) & $1.55$ & $(4,2)$ & NO & 833\\
$(169,50)$ & 11 & $(61,18)$ & 9 & 1 & YES & YES & YES & $1.67$ & $(2,3)$ & NO & 834\\
$(169,50)$ & 11 & $(115,34)$ & 10 & 1 & YES & YES & NO(2) & $1.82$ & $(4,2)$ & NO & 835\\
$(169,70)$ & 11 & $(169,70)$ & 11 & 169 & YES & YES & NO(2) & $1.79$ & $(2,3)$ & NO & 836\\
$(170,47)$ & 11 & $(2,1)$ & 1 & 2 & YES & YES & YES & $1.64$ & $(2,3)$ & -- & 837\\
$(170,47)$ & 11 & $(5,2)$ & 3 & 5 & YES & YES & YES & $1.38$ & $(4,2)$ & -- & 838\\
$(170,47)$ & 11 & $(5,2)$ & 3 & 5 & YES & YES & YES & $1.38$ & $(4,2)$ & NO & 839\\
$(170,47)$ & 11 & $(7,3)$ & 4 & 1 & YES & YES & YES & $1.38$ & $(4,2)$ & -- & 840\\
$(170,47)$ & 11 & $(10,3)$ & 5 & 10 & YES & YES & YES & $1.62$ & $(4,2)$ & NO & 841\\
$(170,47)$ & 11 & $(25,7)$ & 7 & 5 & YES & YES & YES & $1.38$ & $(4,2)$ & NO & 842\\
$(170,47)$ & 11 & $(40,11)$ & 8 & 10 & YES & YES & YES & $1.50$ & $(4,2)$ & 1145 & 843\\
$(171,50)$ & 11 & $(2,1)$ & 1 & 1 & YES & YES & NO(2) & $1.71$ & $(2,3)$ & -- & 844\\
$(171,65)$ & 11 & $(4,1)$ & 3 & 1 & YES & YES & YES & $1.78$ & $(2,3)$ & NO & 845\\
$(171,65)$ & 11 & $(5,2)$ & 3 & 1 & YES & YES & YES & $1.67$ & $(2,3)$ & -- & 846\\
$(171,37)$ & 12 & $(7,3)$ & 4 & 1 & YES & YES & YES & $1.75$ & $(4,2)$ & -- & 847\\
$(171,50)$ & 11 & $(9,2)$ & 5 & 9 & YES & YES & YES & $1.50$ & $(4,2)$ & NO & 848\\
$(171,50)$ & 11 & $(27,8)$ & 7 & 9 & YES & YES & YES & $1.56$ & $(2,3)$ & NO & 849\\
$(171,50)$ & 11 & $(58,17)$ & 9 & 1 & YES & YES & YES & $1.78$ & $(2,3)$ & NO & 850\\
$(171,65)$ & 11 & $(71,27)$ & 9 & 1 & YES & YES & NO(2) & $1.79$ & $(2,3)$ & 957 & 851\\
$(171,50)$ & 11 & $(99,29)$ & 10 & 9 & YES & YES & YES & $1.67$ & $(2,3)$ & NO & 852\\
$(171,50)$ & 11 & $(147,43)$ & 11 & 3 & YES & YES & NO(2) & $1.67$ & $(2,3)$ & NO & 853\\
$(171,50)$ & 11 & $(154,45)$ & 11 & 1 & YES & YES & YES & $1.67$ & $(2,3)$ & NO & 854\\
$(171,65)$ & 11 & $(171,65)$ & 11 & 171 & YES & YES & YES & $1.62$ & $(4,2)$ & NO & 855\\
$(173,64)$ & 11 & $(2,1)$ & 1 & 1 & YES & YES & NO(2) & $1.93$ & $(2,3)$ & -- & 856\\
$(173,73)$ & 11 & $(3,1)$ & 2 & 1 & YES & YES & YES & $1.91$ & $(2,3)$ & -- & 857\\
$(173,64)$ & 11 & $(5,2)$ & 3 & 1 & YES & YES & YES & $1.67$ & $(2,3)$ & -- & 858\\
$(173,66)$ & 11 & $(7,3)$ & 4 & 1 & YES & YES & YES & $1.62$ & $(4,2)$ & NO & 859\\
$(173,64)$ & 11 & $(30,11)$ & 7 & 1 & YES & YES & YES & $1.78$ & $(2,3)$ & NO & 860\\
$(175,67)$ & 11 & $(2,1)$ & 1 & 1 & YES & YES & NO(2) & $1.79$ & $(2,3)$ & -- & 861\\
$(175,67)$ & 11 & $(3,1)$ & 2 & 1 & YES & YES & YES & $1.82$ & $(2,3)$ & -- & 862\\
$(175,67)$ & 11 & $(4,1)$ & 3 & 1 & YES & YES & NO(2) & $1.86$ & $(2,3)$ & -- & 863\\
$(175,67)$ & 11 & $(8,3)$ & 4 & 1 & YES & YES & NO(2) & $1.93$ & $(2,3)$ & NO & 864\\
$(175,67)$ & 11 & $(81,31)$ & 9 & 1 & YES & YES & YES & $1.82$ & $(2,3)$ & 1041 & 865\\
$(175,67)$ & 11 & $(128,49)$ & 10 & 1 & YES & YES & YES & $1.62$ & $(4,2)$ & NO & 866\\
$(176,65)$ & 11 & $(5,2)$ & 3 & 1 & YES & YES & YES & $1.67$ & $(2,3)$ & -- & 867\\
$(176,65)$ & 11 & $(7,2)$ & 4 & 1 & YES & YES & YES & $1.56$ & $(2,3)$ & -- & 868\\
$(176,65)$ & 11 & $(7,3)$ & 4 & 1 & YES & YES & YES & $1.50$ & $(4,2)$ & NO & 869\\
$(176,65)$ & 11 & $(9,2)$ & 5 & 1 & YES & YES & YES & $1.78$ & $(2,3)$ & NO & 870\\
$(176,65)$ & 11 & $(157,58)$ & 11 & 1 & YES & YES & YES & $1.78$ & $(2,3)$ & NO & 871\\
$(177,65)$ & 11 & $(109,40)$ & 10 & 1 & YES & YES & YES & $1.67$ & $(2,3)$ & 1509 & 872\\
$(178,69)$ & 11 & $(2,1)$ & 1 & 2 & YES & YES & YES & $1.67$ & $(2,3)$ & -- & 873\\
$(178,69)$ & 11 & $(5,1)$ & 4 & 1 & YES & YES & YES & $1.73$ & $(2,3)$ & NO & 874\\
$(178,69)$ & 11 & $(129,50)$ & 10 & 1 & YES & YES & YES & $1.62$ & $(4,2)$ & NO & 875\\
$(179,74)$ & 11 & $(3,1)$ & 2 & 1 & YES & YES & NO(2) & $1.69$ & $(4,2)$ & NO & 876\\
$(179,74)$ & 11 & $(3,1)$ & 2 & 1 & YES & YES & YES & $1.73$ & $(2,3)$ & -- & 877\\
$(179,74)$ & 11 & $(4,1)$ & 3 & 1 & YES & YES & NO(2) & $1.69$ & $(4,2)$ & NO & 878\\
$(179,50)$ & 11 & $(5,2)$ & 3 & 1 & YES & YES & YES & $1.56$ & $(2,3)$ & -- & 879\\
$(179,68)$ & 11 & $(5,1)$ & 4 & 1 & YES & YES & YES & $1.82$ & $(2,3)$ & NO & 880\\
$(179,68)$ & 11 & $(5,1)$ & 4 & 1 & YES & YES & YES & $1.82$ & $(2,3)$ & NO & 881\\
$(179,68)$ & 11 & $(5,1)$ & 4 & 1 & YES & YES & YES & $1.82$ & $(2,3)$ & -- & 882\\
$(179,68)$ & 11 & $(5,2)$ & 3 & 1 & YES & YES & YES & $1.67$ & $(2,3)$ & -- & 883\\
$(179,75)$ & 11 & $(5,2)$ & 3 & 1 & YES & YES & YES & $1.78$ & $(2,3)$ & -- & 884\\
$(179,75)$ & 11 & $(8,3)$ & 4 & 1 & YES & YES & YES & $1.78$ & $(2,3)$ & NO & 885\\
$(179,74)$ & 11 & $(17,7)$ & 6 & 1 & YES & YES & NO(2) & $1.69$ & $(4,2)$ & 642 & 886\\
$(179,74)$ & 11 & $(46,19)$ & 8 & 1 & YES & YES & NO(2) & $1.69$ & $(4,2)$ & NO & 887\\
$(179,74)$ & 11 & $(75,31)$ & 9 & 1 & YES & YES & YES & $1.82$ & $(2,3)$ & NO & 888\\
$(179,68)$ & 11 & $(79,30)$ & 9 & 1 & YES & YES & YES & $1.82$ & $(2,3)$ & 1030 & 889\\
$(179,75)$ & 11 & $(105,44)$ & 10 & 1 & YES & YES & NO(2) & $1.64$ & $(4,2)$ & NO & 890\\
$(179,74)$ & 11 & $(179,74)$ & 11 & 179 & YES & YES & YES & $1.82$ & $(2,3)$ & NO & 891\\
$(180,41)$ & 12 & $(7,3)$ & 4 & 1 & YES & YES & YES & $1.62$ & $(4,2)$ & -- & 892\\
$(180,41)$ & 12 & $(8,3)$ & 4 & 4 & YES & YES & YES & $1.78$ & $(2,3)$ & NO & 893\\
$(180,41)$ & 12 & $(8,3)$ & 4 & 4 & YES & YES & YES & $1.78$ & $(2,3)$ & -- & 894\\
$(181,75)$ & 11 & $(2,1)$ & 1 & 1 & YES & YES & NO(2) & $1.55$ & $(4,2)$ & -- & 895\\
$(181,76)$ & 11 & $(2,1)$ & 1 & 1 & YES & YES & NO(2) & $1.73$ & $(4,2)$ & -- & 896\\
$(181,70)$ & 11 & $(3,1)$ & 2 & 1 & YES & YES & YES & $1.60$ & $(4,2)$ & -- & 897\\
$(181,75)$ & 11 & $(3,1)$ & 2 & 1 & YES & YES & YES & $1.82$ & $(2,3)$ & -- & 898\\
$(181,76)$ & 11 & $(3,1)$ & 2 & 1 & YES & YES & NO(2) & $1.73$ & $(4,2)$ & NO & 899\\
$(181,75)$ & 11 & $(5,2)$ & 3 & 1 & YES & YES & YES & $1.62$ & $(4,2)$ & -- & 900\\
$(181,76)$ & 11 & $(5,2)$ & 3 & 1 & YES & YES & YES & $1.38$ & $(4,2)$ & -- & 901\\
$(181,75)$ & 11 & $(7,2)$ & 4 & 1 & YES & YES & YES & $1.78$ & $(2,3)$ & -- & 902\\
$(181,75)$ & 11 & $(7,2)$ & 4 & 1 & YES & YES & YES & $1.78$ & $(2,3)$ & NO & 903\\
$(181,55)$ & 12 & $(11,3)$ & 5 & 1 & YES & YES & YES & $1.62$ & $(4,2)$ & NO & 904\\
$(181,76)$ & 11 & $(31,13)$ & 7 & 1 & YES & YES & NO(2) & $1.73$ & $(4,2)$ & NO & 905\\
$(181,41)$ & 12 & $(71,16)$ & 10 & 1 & YES & YES & YES & $2.00$ & $(2,3)$ & NO & 906\\
$(181,70)$ & 11 & $(106,41)$ & 10 & 1 & YES & YES & YES & $1.60$ & $(4,2)$ & NO & 907\\
$(181,75)$ & 11 & $(152,63)$ & 11 & 1 & YES & YES & YES & $1.67$ & $(2,3)$ & NO & 908\\
$(183,71)$ & 11 & $(11,4)$ & 5 & 1 & YES & YES & YES & $1.67$ & $(2,3)$ & NO & 909\\
$(183,71)$ & 11 & $(67,26)$ & 9 & 1 & YES & YES & NO(2) & $1.86$ & $(2,3)$ & NO & 910\\
$(186,71)$ & 11 & $(2,1)$ & 1 & 2 & YES & YES & NO(2) & $1.69$ & $(4,2)$ & -- & 911\\
$(186,71)$ & 11 & $(4,1)$ & 3 & 2 & YES & YES & YES & $1.78$ & $(2,3)$ & NO & 912\\
$(186,55)$ & 11 & $(5,2)$ & 3 & 1 & YES & YES & YES & $1.67$ & $(2,3)$ & -- & 913\\
$(186,71)$ & 11 & $(5,2)$ & 3 & 1 & YES & YES & YES & $1.57$ & $(2,3)$ & -- & 914\\
$(186,55)$ & 11 & $(7,2)$ & 4 & 1 & YES & YES & YES & $1.56$ & $(2,3)$ & -- & 915\\
$(186,71)$ & 11 & $(7,2)$ & 4 & 1 & YES & YES & YES & $1.67$ & $(2,3)$ & NO & 916\\
$(186,55)$ & 11 & $(9,2)$ & 5 & 3 & YES & YES & YES & $1.44$ & $(2,3)$ & NO & 917\\
$(186,55)$ & 11 & $(11,3)$ & 5 & 1 & YES & YES & YES & $1.56$ & $(2,3)$ & NO & 918\\
$(186,55)$ & 11 & $(24,7)$ & 7 & 6 & YES & YES & YES & $1.67$ & $(2,3)$ & NO & 919\\
$(186,55)$ & 11 & $(61,18)$ & 9 & 1 & YES & YES & YES & $1.56$ & $(2,3)$ & NO & 920\\
$(186,71)$ & 11 & $(76,29)$ & 9 & 2 & YES & YES & YES & $1.78$ & $(2,3)$ & 1015 & 921\\
$(186,71)$ & 11 & $(97,37)$ & 10 & 1 & YES & YES & YES & $1.67$ & $(2,3)$ & 1545 & 922\\
$(186,55)$ & 11 & $(98,29)$ & 10 & 2 & YES & YES & YES & $1.67$ & $(2,3)$ & NO & 923\\
$(186,55)$ & 11 & $(159,47)$ & 11 & 3 & YES & YES & YES & $1.44$ & $(2,3)$ & NO & 924\\
$(187,79)$ & 11 & $(2,1)$ & 1 & 1 & YES & YES & YES & $1.82$ & $(2,3)$ & -- & 925\\
$(187,71)$ & 11 & $(4,1)$ & 3 & 1 & YES & YES & NO(2) & $1.86$ & $(2,3)$ & NO & 926\\
$(187,71)$ & 11 & $(5,2)$ & 3 & 1 & YES & YES & YES & $1.78$ & $(2,3)$ & -- & 927\\
$(187,71)$ & 11 & $(5,2)$ & 3 & 1 & YES & YES & YES & $2.00$ & $(2,3)$ & NO & 928\\
$(187,71)$ & 11 & $(34,13)$ & 7 & 17 & YES & YES & YES & $1.67$ & $(2,3)$ & NO & 929\\
$(187,71)$ & 11 & $(79,30)$ & 9 & 1 & YES & YES & NO(2) & $1.86$ & $(2,3)$ & NO & 930\\
$(187,71)$ & 11 & $(129,49)$ & 10 & 1 & YES & YES & YES & $2.00$ & $(2,3)$ & NO & 931\\
$(188,79)$ & 11 & $(2,1)$ & 1 & 2 & YES & YES & NO(2) & $1.73$ & $(4,2)$ & -- & 932\\
$(188,69)$ & 11 & $(5,2)$ & 3 & 1 & YES & YES & YES & $1.78$ & $(2,3)$ & -- & 933\\
$(188,79)$ & 11 & $(5,1)$ & 4 & 1 & YES & YES & NO(2) & $1.69$ & $(4,2)$ & NO & 934\\
$(188,79)$ & 11 & $(5,2)$ & 3 & 1 & YES & YES & YES & $1.78$ & $(2,3)$ & -- & 935\\
$(188,55)$ & 12 & $(6,1)$ & 5 & 2 & YES & YES & NO(2) & $1.77$ & $(4,2)$ & NO & 936\\
$(188,79)$ & 11 & $(7,2)$ & 4 & 1 & YES & YES & YES & $1.67$ & $(2,3)$ & NO & 937\\
$(188,79)$ & 11 & $(12,5)$ & 5 & 4 & YES & YES & NO(2) & $1.64$ & $(4,2)$ & NO & 938\\
$(188,69)$ & 11 & $(27,10)$ & 7 & 1 & YES & YES & YES & $1.78$ & $(2,3)$ & NO & 939\\
$(188,55)$ & 12 & $(106,31)$ & 10 & 2 & YES & YES & NO(2) & $1.77$ & $(4,2)$ & 1250 & 940\\
$(189,55)$ & 12 & $(5,2)$ & 3 & 1 & YES & YES & YES & $1.89$ & $(2,3)$ & -- & 941\\
$(190,43)$ & 12 & $(5,2)$ & 3 & 5 & YES & YES & YES & $1.67$ & $(2,3)$ & -- & 942\\
$(191,74)$ & 11 & $(2,1)$ & 1 & 1 & YES & YES & NO(2) & $1.71$ & $(2,3)$ & -- & 943\\
$(191,74)$ & 11 & $(3,1)$ & 2 & 1 & YES & YES & YES & $1.50$ & $(4,2)$ & -- & 944\\
$(191,80)$ & 11 & $(3,1)$ & 2 & 1 & YES & YES & NO(2) & $1.69$ & $(4,2)$ & -- & 945\\
$(191,80)$ & 11 & $(3,1)$ & 2 & 1 & YES & YES & YES & $1.82$ & $(2,3)$ & NO & 946\\
$(191,74)$ & 11 & $(4,1)$ & 3 & 1 & YES & YES & YES & $1.62$ & $(4,2)$ & -- & 947\\
$(191,56)$ & 12 & $(5,2)$ & 3 & 1 & YES & YES & YES & $1.62$ & $(4,2)$ & NO & 948\\
$(191,74)$ & 11 & $(13,5)$ & 5 & 1 & YES & YES & NO(2) & $1.86$ & $(2,3)$ & NO & 949\\
$(191,80)$ & 11 & $(31,13)$ & 7 & 1 & YES & YES & NO(2) & $1.82$ & $(4,2)$ & NO & 950\\
$(191,80)$ & 11 & $(50,21)$ & 8 & 1 & YES & YES & YES & $1.50$ & $(4,2)$ & NO & 951\\
$(191,56)$ & 12 & $(65,19)$ & 9 & 1 & YES & YES & YES & $1.43$ & $(2,3)$ & NO & 952\\
$(192,73)$ & 11 & $(2,1)$ & 1 & 2 & YES & YES & NO(2) & $1.86$ & $(2,3)$ & -- & 953\\
$(192,73)$ & 11 & $(4,1)$ & 3 & 4 & YES & YES & YES & $1.50$ & $(4,2)$ & NO & 954\\
$(192,73)$ & 11 & $(13,5)$ & 5 & 1 & YES & YES & YES & $1.38$ & $(4,2)$ & NO & 955\\
$(192,73)$ & 11 & $(34,13)$ & 7 & 2 & YES & YES & YES & $1.67$ & $(2,3)$ & NO & 956\\
$(192,73)$ & 11 & $(50,19)$ & 8 & 2 & YES & YES & NO(2) & $1.79$ & $(2,3)$ & 851 & 957\\
$(192,73)$ & 11 & $(71,27)$ & 9 & 1 & YES & YES & NO(2) & $1.93$ & $(2,3)$ & NO & 958\\
$(193,57)$ & 12 & $(3,1)$ & 2 & 1 & YES & YES & YES & $1.56$ & $(2,3)$ & -- & 959\\
$(193,57)$ & 12 & $(5,1)$ & 4 & 1 & YES & YES & YES & $1.56$ & $(2,3)$ & NO & 960\\
$(193,81)$ & 11 & $(5,2)$ & 3 & 1 & YES & YES & YES & $1.62$ & $(4,2)$ & -- & 961\\
$(193,81)$ & 11 & $(81,34)$ & 9 & 1 & YES & YES & NO(2) & $1.73$ & $(4,2)$ & NO & 962\\
$(193,81)$ & 11 & $(131,55)$ & 10 & 1 & YES & YES & YES & $1.78$ & $(2,3)$ & NO & 963\\
$(194,75)$ & 11 & $(2,1)$ & 1 & 2 & YES & YES & YES & $1.67$ & $(2,3)$ & -- & 964\\
$(194,75)$ & 11 & $(3,1)$ & 2 & 1 & YES & YES & YES & $1.60$ & $(4,2)$ & -- & 965\\
$(194,75)$ & 11 & $(4,1)$ & 3 & 2 & YES & YES & YES & $1.78$ & $(2,3)$ & NO & 966\\
$(194,75)$ & 11 & $(5,2)$ & 3 & 1 & YES & YES & YES & $1.62$ & $(4,2)$ & -- & 967\\
$(194,75)$ & 11 & $(44,17)$ & 8 & 2 & YES & YES & YES & $1.60$ & $(4,2)$ & 813 & 968\\
$(194,75)$ & 11 & $(57,22)$ & 9 & 1 & YES & YES & YES & $1.62$ & $(4,2)$ & NO & 969\\
$(194,75)$ & 11 & $(75,29)$ & 9 & 1 & YES & YES & YES & $1.62$ & $(4,2)$ & NO & 970\\
$(194,75)$ & 11 & $(119,46)$ & 10 & 1 & YES & YES & YES & $1.50$ & $(4,2)$ & NO & 971\\
$(194,75)$ & 11 & $(194,75)$ & 11 & 194 & YES & YES & YES & $1.62$ & $(4,2)$ & NO & 972\\
$(196,81)$ & 11 & $(2,1)$ & 1 & 2 & YES & YES & YES & $1.73$ & $(2,3)$ & -- & 973\\
$(196,75)$ & 11 & $(3,1)$ & 2 & 1 & YES & YES & YES & $1.67$ & $(2,3)$ & -- & 974\\
$(196,81)$ & 11 & $(3,1)$ & 2 & 1 & YES & YES & YES & $1.64$ & $(2,3)$ & -- & 975\\
$(196,81)$ & 11 & $(3,1)$ & 2 & 1 & YES & YES & NO(2) & $1.69$ & $(4,2)$ & NO & 976\\
$(196,55)$ & 12 & $(5,2)$ & 3 & 1 & YES & YES & YES & $1.89$ & $(2,3)$ & -- & 977\\
$(196,75)$ & 11 & $(5,2)$ & 3 & 1 & YES & YES & YES & $1.71$ & $(2,3)$ & -- & 978\\
$(196,81)$ & 11 & $(5,1)$ & 4 & 1 & YES & YES & YES & $1.82$ & $(2,3)$ & NO & 979\\
$(196,81)$ & 11 & $(5,2)$ & 3 & 1 & YES & YES & YES & $1.67$ & $(2,3)$ & -- & 980\\
$(196,75)$ & 11 & $(128,49)$ & 10 & 4 & YES & YES & YES & $1.57$ & $(2,3)$ & NO & 981\\
$(199,76)$ & 11 & $(2,1)$ & 1 & 1 & YES & YES & NO(2) & $1.73$ & $(4,2)$ & -- & 982\\
$(199,76)$ & 11 & $(3,1)$ & 2 & 1 & YES & YES & YES & $1.82$ & $(2,3)$ & -- & 983\\
$(199,76)$ & 11 & $(4,1)$ & 3 & 1 & YES & YES & YES & $1.67$ & $(2,3)$ & NO & 984\\
$(199,55)$ & 11 & $(5,2)$ & 3 & 1 & YES & YES & YES & $1.56$ & $(2,3)$ & -- & 985\\
$(199,55)$ & 11 & $(5,2)$ & 3 & 1 & YES & YES & YES & $1.67$ & $(2,3)$ & NO & 986\\
$(199,76)$ & 11 & $(5,1)$ & 4 & 1 & YES & YES & YES & $1.56$ & $(2,3)$ & NO & 987\\
$(199,76)$ & 11 & $(8,3)$ & 4 & 1 & YES & YES & YES & $1.82$ & $(2,3)$ & NO & 988\\
$(199,55)$ & 11 & $(10,3)$ & 5 & 1 & YES & YES & YES & $1.56$ & $(2,3)$ & NO & 989\\
$(199,76)$ & 11 & $(21,8)$ & 6 & 1 & YES & YES & YES & $1.82$ & $(2,3)$ & NO & 990\\
$(199,55)$ & 11 & $(25,7)$ & 7 & 1 & YES & YES & YES & $1.67$ & $(2,3)$ & NO & 991\\
$(199,55)$ & 11 & $(65,18)$ & 9 & 1 & YES & YES & YES & $1.67$ & $(2,3)$ & NO & 992\\
$(199,76)$ & 11 & $(89,34)$ & 9 & 1 & YES & YES & YES & $1.67$ & $(2,3)$ & 1124 & 993\\
$(199,55)$ & 11 & $(105,29)$ & 10 & 1 & YES & YES & YES & $1.67$ & $(2,3)$ & NO & 994\\
$(199,76)$ & 11 & $(144,55)$ & 10 & 1 & YES & YES & YES & $1.56$ & $(2,3)$ & NO & 995\\
$(199,76)$ & 11 & $(199,76)$ & 11 & 199 & YES & YES & YES & $1.67$ & $(2,3)$ & NO & 996\\
$(200,59)$ & 12 & $(71,21)$ & 9 & 1 & YES & YES & YES & $1.57$ & $(2,3)$ & NO & 997\\
$(201,61)$ & 12 & $(5,1)$ & 4 & 1 & YES & YES & YES & $1.73$ & $(2,3)$ & NO & 998\\
$(201,77)$ & 12 & $(5,1)$ & 4 & 1 & YES & YES & YES & $1.62$ & $(4,2)$ & NO & 999\\
$(201,56)$ & 12 & $(15,4)$ & 6 & 3 & YES & YES & YES & $1.62$ & $(4,2)$ & NO & 1000\\
$(201,83)$ & 12 & $(155,64)$ & 11 & 1 & YES & YES & YES & $1.75$ & $(4,2)$ & NO & 1001\\
$(203,60)$ & 12 & $(37,11)$ & 8 & 1 & YES & YES & YES & $1.78$ & $(2,3)$ & NO & 1002\\
$(205,61)$ & 12 & $(27,8)$ & 7 & 1 & YES & YES & YES & $1.73$ & $(2,3)$ & NO & 1003\\
$(206,47)$ & 12 & $(2,1)$ & 1 & 2 & YES & YES & NO(2) & $1.86$ & $(2,3)$ & -- & 1004\\
$(206,57)$ & 12 & $(5,1)$ & 4 & 1 & YES & YES & YES & $1.56$ & $(2,3)$ & NO & 1005\\
$(206,47)$ & 12 & $(23,5)$ & 7 & 1 & YES & YES & YES & $1.78$ & $(2,3)$ & NO & 1006\\
$(207,79)$ & 11 & $(2,1)$ & 1 & 1 & YES & YES & YES & $1.82$ & $(2,3)$ & -- & 1007\\
$(207,79)$ & 11 & $(3,1)$ & 2 & 3 & YES & YES & YES & $1.82$ & $(2,3)$ & -- & 1008\\
$(207,85)$ & 12 & $(3,1)$ & 2 & 3 & YES & YES & YES & $1.62$ & $(4,2)$ & -- & 1009\\
$(207,79)$ & 11 & $(4,1)$ & 3 & 1 & YES & YES & YES & $1.82$ & $(2,3)$ & NO & 1010\\
$(207,79)$ & 11 & $(4,1)$ & 3 & 1 & YES & YES & YES & $1.67$ & $(2,3)$ & -- & 1011\\
$(207,79)$ & 11 & $(5,2)$ & 3 & 1 & YES & YES & YES & $1.78$ & $(2,3)$ & -- & 1012\\
$(207,79)$ & 11 & $(7,2)$ & 4 & 1 & YES & YES & YES & $1.67$ & $(2,3)$ & NO & 1013\\
$(207,79)$ & 11 & $(34,13)$ & 7 & 1 & YES & YES & YES & $1.67$ & $(2,3)$ & 1122 & 1014\\
$(207,79)$ & 11 & $(55,21)$ & 8 & 1 & YES & YES & YES & $1.78$ & $(2,3)$ & 921 & 1015\\
$(207,85)$ & 12 & $(95,39)$ & 10 & 1 & YES & YES & YES & $1.62$ & $(4,2)$ & 1209 & 1016\\
$(207,79)$ & 11 & $(97,37)$ & 10 & 1 & YES & YES & YES & $1.78$ & $(2,3)$ & NO & 1017\\
$(207,79)$ & 11 & $(131,50)$ & 10 & 1 & YES & YES & YES & $1.67$ & $(2,3)$ & NO & 1018\\
$(207,79)$ & 11 & $(207,79)$ & 11 & 207 & YES & YES & YES & $1.67$ & $(2,3)$ & NO & 1019\\
$(207,85)$ & 12 & $(207,85)$ & 12 & 207 & YES & YES & YES & $1.75$ & $(4,2)$ & NO & 1020\\
$(208,79)$ & 11 & $(2,1)$ & 1 & 2 & YES & YES & YES & $1.82$ & $(2,3)$ & -- & 1021\\
$(208,79)$ & 11 & $(3,1)$ & 2 & 1 & YES & YES & YES & $1.38$ & $(4,2)$ & -- & 1022\\
$(208,79)$ & 11 & $(3,1)$ & 2 & 1 & YES & YES & YES & $1.82$ & $(2,3)$ & NO & 1023\\
$(208,79)$ & 11 & $(4,1)$ & 3 & 4 & YES & YES & YES & $1.82$ & $(2,3)$ & NO & 1024\\
$(208,79)$ & 11 & $(4,1)$ & 3 & 4 & YES & YES & YES & $1.82$ & $(2,3)$ & -- & 1025\\
$(208,61)$ & 12 & $(5,2)$ & 3 & 1 & YES & YES & YES & $1.50$ & $(4,2)$ & NO & 1026\\
$(208,61)$ & 12 & $(5,2)$ & 3 & 1 & YES & YES & YES & $1.78$ & $(2,3)$ & -- & 1027\\
$(208,79)$ & 11 & $(5,2)$ & 3 & 1 & YES & YES & YES & $1.38$ & $(4,2)$ & NO & 1028\\
$(208,79)$ & 11 & $(13,5)$ & 5 & 13 & YES & YES & YES & $1.73$ & $(2,3)$ & 1120 & 1029\\
$(208,79)$ & 11 & $(50,19)$ & 8 & 2 & YES & YES & YES & $1.82$ & $(2,3)$ & 889 & 1030\\
$(208,79)$ & 11 & $(79,30)$ & 9 & 1 & YES & YES & YES & $1.38$ & $(4,2)$ & NO & 1031\\
$(208,79)$ & 11 & $(129,49)$ & 10 & 1 & YES & YES & YES & $1.82$ & $(2,3)$ & NO & 1032\\
$(208,79)$ & 11 & $(208,79)$ & 11 & 208 & YES & YES & YES & $1.82$ & $(2,3)$ & NO & 1033\\
$(209,80)$ & 11 & $(2,1)$ & 1 & 1 & YES & YES & YES & $1.64$ & $(2,3)$ & -- & 1034\\
$(209,81)$ & 11 & $(2,1)$ & 1 & 1 & YES & YES & YES & $1.73$ & $(2,3)$ & -- & 1035\\
$(209,80)$ & 11 & $(3,1)$ & 2 & 1 & YES & YES & YES & $1.67$ & $(2,3)$ & -- & 1036\\
$(209,81)$ & 11 & $(3,1)$ & 2 & 1 & YES & YES & NO(2) & $1.77$ & $(4,2)$ & 732 & 1037\\
$(209,80)$ & 11 & $(5,2)$ & 3 & 1 & YES & YES & YES & $1.86$ & $(2,3)$ & -- & 1038\\
$(209,80)$ & 11 & $(21,8)$ & 6 & 1 & YES & YES & YES & $1.67$ & $(2,3)$ & NO & 1039\\
$(209,80)$ & 11 & $(34,13)$ & 7 & 1 & YES & YES & YES & $1.67$ & $(2,3)$ & NO & 1040\\
$(209,80)$ & 11 & $(47,18)$ & 8 & 1 & YES & YES & YES & $1.82$ & $(2,3)$ & 865 & 1041\\
$(209,62)$ & 12 & $(101,30)$ & 10 & 1 & YES & YES & YES & $1.43$ & $(2,3)$ & NO & 1042\\
$(209,80)$ & 11 & $(115,44)$ & 10 & 1 & YES & YES & YES & $1.86$ & $(2,3)$ & NO & 1043\\
$(209,80)$ & 11 & $(209,80)$ & 11 & 209 & YES & YES & YES & $1.78$ & $(2,3)$ & NO & 1044\\
$(211,64)$ & 12 & $(5,2)$ & 3 & 1 & YES & YES & YES & $1.89$ & $(2,3)$ & -- & 1045\\
$(212,89)$ & 11 & $(2,1)$ & 1 & 2 & YES & YES & NO(2) & $1.73$ & $(4,2)$ & -- & 1046\\
$(212,81)$ & 11 & $(3,1)$ & 2 & 1 & YES & YES & YES & $1.67$ & $(2,3)$ & -- & 1047\\
$(212,81)$ & 11 & $(5,2)$ & 3 & 1 & YES & YES & YES & $1.82$ & $(2,3)$ & NO & 1048\\
$(212,89)$ & 11 & $(8,3)$ & 4 & 4 & YES & YES & YES & $1.62$ & $(4,2)$ & NO & 1049\\
$(212,81)$ & 11 & $(13,5)$ & 5 & 1 & YES & YES & YES & $1.82$ & $(2,3)$ & NO & 1050\\
$(212,89)$ & 11 & $(17,7)$ & 6 & 1 & YES & YES & YES & $2.00$ & $(2,3)$ & NO & 1051\\
$(212,89)$ & 11 & $(43,18)$ & 8 & 1 & YES & YES & YES & $1.62$ & $(4,2)$ & NO & 1052\\
$(212,81)$ & 11 & $(55,21)$ & 8 & 1 & YES & YES & YES & $1.56$ & $(2,3)$ & NO & 1053\\
$(212,89)$ & 11 & $(112,47)$ & 10 & 4 & YES & YES & YES & $1.78$ & $(2,3)$ & NO & 1054\\
$(212,81)$ & 11 & $(123,47)$ & 10 & 1 & YES & YES & YES & $1.78$ & $(2,3)$ & NO & 1055\\
$(213,59)$ & 12 & $(5,2)$ & 3 & 1 & YES & YES & YES & $1.50$ & $(4,2)$ & -- & 1056\\
$(213,59)$ & 12 & $(5,2)$ & 3 & 1 & YES & YES & YES & $1.67$ & $(2,3)$ & NO & 1057\\
$(213,62)$ & 12 & $(5,2)$ & 3 & 1 & YES & YES & YES & $1.67$ & $(2,3)$ & -- & 1058\\
$(213,62)$ & 12 & $(7,2)$ & 4 & 1 & YES & YES & YES & $1.67$ & $(2,3)$ & -- & 1059\\
$(213,59)$ & 12 & $(15,4)$ & 6 & 3 & YES & YES & YES & $1.62$ & $(4,2)$ & NO & 1060\\
$(213,65)$ & 12 & $(23,7)$ & 7 & 1 & YES & YES & NO(2) & $1.85$ & $(4,2)$ & NO & 1061\\
$(213,59)$ & 12 & $(40,11)$ & 8 & 1 & YES & YES & YES & $1.67$ & $(2,3)$ & NO & 1062\\
$(213,88)$ & 12 & $(167,69)$ & 11 & 1 & YES & YES & YES & $1.75$ & $(4,2)$ & NO & 1063\\
$(214,79)$ & 12 & $(4,1)$ & 3 & 2 & YES & YES & YES & $1.62$ & $(4,2)$ & -- & 1064\\
$(214,79)$ & 12 & $(46,17)$ & 8 & 2 & YES & YES & YES & $1.62$ & $(4,2)$ & NO & 1065\\
$(214,83)$ & 12 & $(214,83)$ & 12 & 214 & YES & YES & YES & $1.78$ & $(2,3)$ & NO & 1066\\
$(215,63)$ & 12 & $(2,1)$ & 1 & 1 & YES & YES & YES & $1.38$ & $(4,2)$ & -- & 1067\\
$(215,63)$ & 12 & $(3,1)$ & 2 & 1 & YES & YES & YES & $1.38$ & $(4,2)$ & -- & 1068\\
$(215,63)$ & 12 & $(3,1)$ & 2 & 1 & YES & YES & YES & $1.78$ & $(2,3)$ & NO & 1069\\
$(215,63)$ & 12 & $(10,3)$ & 5 & 5 & YES & YES & YES & $1.50$ & $(4,2)$ & NO & 1070\\
$(215,63)$ & 12 & $(99,29)$ & 10 & 1 & YES & YES & YES & $1.50$ & $(4,2)$ & 1273 & 1071\\
$(215,58)$ & 12 & $(100,27)$ & 10 & 5 & YES & YES & YES & $1.50$ & $(4,2)$ & NO & 1072\\
$(219,64)$ & 12 & $(2,1)$ & 1 & 1 & YES & YES & YES & $1.62$ & $(4,2)$ & NO & 1073\\
$(219,65)$ & 12 & $(2,1)$ & 1 & 1 & YES & YES & NO(2) & $1.73$ & $(4,2)$ & -- & 1074\\
$(219,64)$ & 12 & $(4,1)$ & 3 & 1 & YES & YES & NO(2) & $1.79$ & $(2,3)$ & NO & 1075\\
$(219,64)$ & 12 & $(5,2)$ & 3 & 1 & YES & YES & YES & $1.71$ & $(2,3)$ & -- & 1076\\
$(219,64)$ & 12 & $(5,2)$ & 3 & 1 & YES & YES & YES & $1.86$ & $(2,3)$ & NO & 1077\\
$(219,64)$ & 12 & $(27,8)$ & 7 & 3 & YES & YES & YES & $1.57$ & $(2,3)$ & NO & 1078\\
$(219,64)$ & 12 & $(41,12)$ & 8 & 1 & YES & YES & YES & $1.62$ & $(4,2)$ & NO & 1079\\
$(219,64)$ & 12 & $(58,17)$ & 9 & 1 & YES & YES & YES & $1.71$ & $(2,3)$ & NO & 1080\\
$(219,61)$ & 12 & $(104,29)$ & 10 & 1 & YES & YES & YES & $1.67$ & $(2,3)$ & NO & 1081\\
$(219,64)$ & 12 & $(219,64)$ & 12 & 219 & YES & YES & NO(2) & $1.83$ & $(2,3)$ & NO & 1082\\
$(226,63)$ & 12 & $(2,1)$ & 1 & 2 & YES & YES & YES & $1.38$ & $(4,2)$ & -- & 1083\\
$(226,63)$ & 12 & $(3,1)$ & 2 & 1 & YES & YES & YES & $1.38$ & $(4,2)$ & -- & 1084\\
$(226,61)$ & 12 & $(5,2)$ & 3 & 1 & YES & YES & YES & $1.50$ & $(4,2)$ & -- & 1085\\
$(226,63)$ & 12 & $(11,3)$ & 5 & 1 & YES & YES & YES & $1.62$ & $(4,2)$ & NO & 1086\\
$(226,63)$ & 12 & $(25,7)$ & 7 & 1 & YES & YES & YES & $1.38$ & $(4,2)$ & NO & 1087\\
$(226,63)$ & 12 & $(226,63)$ & 12 & 226 & YES & YES & YES & $1.62$ & $(4,2)$ & NO & 1088\\
$(227,67)$ & 12 & $(2,1)$ & 1 & 1 & YES & YES & YES & $1.25$ & $(4,2)$ & -- & 1089\\
$(227,67)$ & 12 & $(3,1)$ & 2 & 1 & YES & YES & NO(2) & $1.73$ & $(4,2)$ & -- & 1090\\
$(227,88)$ & 12 & $(4,1)$ & 3 & 1 & YES & YES & YES & $1.78$ & $(2,3)$ & NO & 1091\\
$(227,67)$ & 12 & $(5,2)$ & 3 & 1 & YES & YES & YES & $1.78$ & $(2,3)$ & -- & 1092\\
$(227,88)$ & 12 & $(5,1)$ & 4 & 1 & YES & YES & YES & $1.78$ & $(2,3)$ & NO & 1093\\
$(227,61)$ & 12 & $(7,2)$ & 4 & 1 & YES & YES & YES & $1.89$ & $(2,3)$ & -- & 1094\\
$(227,67)$ & 12 & $(7,2)$ & 4 & 1 & YES & YES & YES & $1.71$ & $(2,3)$ & -- & 1095\\
$(227,61)$ & 12 & $(10,3)$ & 5 & 1 & YES & YES & YES & $1.89$ & $(2,3)$ & NO & 1096\\
$(227,67)$ & 12 & $(27,8)$ & 7 & 1 & YES & YES & NO(2) & $1.73$ & $(4,2)$ & NO & 1097\\
$(227,67)$ & 12 & $(44,13)$ & 8 & 1 & YES & YES & YES & $1.62$ & $(4,2)$ & NO & 1098\\
$(227,94)$ & 12 & $(128,53)$ & 11 & 1 & YES & YES & YES & $1.62$ & $(4,2)$ & NO & 1099\\
$(227,88)$ & 12 & $(129,50)$ & 10 & 1 & YES & YES & YES & $1.78$ & $(2,3)$ & 1516 & 1100\\
$(227,67)$ & 12 & $(166,49)$ & 11 & 1 & YES & YES & YES & $1.62$ & $(4,2)$ & NO & 1101\\
$(227,88)$ & 12 & $(178,69)$ & 11 & 1 & YES & YES & YES & $1.78$ & $(2,3)$ & NO & 1102\\
$(229,95)$ & 12 & $(2,1)$ & 1 & 1 & YES & YES & YES & $1.62$ & $(4,2)$ & -- & 1103\\
$(229,94)$ & 12 & $(3,1)$ & 2 & 1 & YES & YES & YES & $1.75$ & $(4,2)$ & -- & 1104\\
$(229,95)$ & 12 & $(3,1)$ & 2 & 1 & YES & YES & YES & $1.62$ & $(4,2)$ & NO & 1105\\
$(229,95)$ & 12 & $(4,1)$ & 3 & 1 & YES & YES & YES & $1.75$ & $(4,2)$ & NO & 1106\\
$(229,95)$ & 12 & $(4,1)$ & 3 & 1 & YES & YES & YES & $1.75$ & $(4,2)$ & -- & 1107\\
$(229,68)$ & 12 & $(5,1)$ & 4 & 1 & YES & YES & YES & $1.82$ & $(2,3)$ & NO & 1108\\
$(229,95)$ & 12 & $(94,39)$ & 10 & 1 & YES & YES & YES & $1.62$ & $(4,2)$ & NO & 1109\\
$(229,68)$ & 12 & $(101,30)$ & 10 & 1 & YES & YES & NO(2) & $1.64$ & $(4,2)$ & 1320 & 1110\\
$(229,94)$ & 12 & $(134,55)$ & 11 & 1 & YES & YES & YES & $1.75$ & $(4,2)$ & NO & 1111\\
$(229,68)$ & 12 & $(229,68)$ & 12 & 229 & YES & YES & NO(2) & $1.73$ & $(4,2)$ & NO & 1112\\
$(229,95)$ & 12 & $(229,95)$ & 12 & 229 & YES & YES & YES & $1.75$ & $(4,2)$ & NO & 1113\\
$(233,89)$ & 11 & $(2,1)$ & 1 & 1 & YES & YES & YES & $1.56$ & $(2,3)$ & -- & 1114\\
$(233,89)$ & 11 & $(3,1)$ & 2 & 1 & YES & YES & YES & $1.64$ & $(2,3)$ & -- & 1115\\
$(233,89)$ & 11 & $(4,1)$ & 3 & 1 & YES & YES & YES & $1.67$ & $(2,3)$ & NO & 1116\\
$(233,89)$ & 11 & $(5,1)$ & 4 & 1 & YES & YES & YES & $1.44$ & $(2,3)$ & NO & 1117\\
$(233,89)$ & 11 & $(5,2)$ & 3 & 1 & YES & YES & YES & $1.57$ & $(2,3)$ & -- & 1118\\
$(233,89)$ & 11 & $(7,3)$ & 4 & 1 & YES & YES & YES & $1.57$ & $(2,3)$ & NO & 1119\\
$(233,89)$ & 11 & $(8,3)$ & 4 & 1 & YES & YES & YES & $1.73$ & $(2,3)$ & 1029 & 1120\\
$(233,89)$ & 11 & $(13,5)$ & 5 & 1 & YES & YES & YES & $1.67$ & $(2,3)$ & NO & 1121\\
$(233,89)$ & 11 & $(21,8)$ & 6 & 1 & YES & YES & YES & $1.67$ & $(2,3)$ & 1014 & 1122\\
$(233,89)$ & 11 & $(47,18)$ & 8 & 1 & YES & YES & YES & $1.57$ & $(2,3)$ & NO & 1123\\
$(233,89)$ & 11 & $(55,21)$ & 8 & 1 & YES & YES & YES & $1.67$ & $(2,3)$ & 993 & 1124\\
$(233,89)$ & 11 & $(89,34)$ & 9 & 1 & YES & YES & YES & $1.67$ & $(2,3)$ & NO & 1125\\
$(233,89)$ & 11 & $(144,55)$ & 10 & 1 & YES & YES & YES & $1.67$ & $(2,3)$ & NO & 1126\\
$(233,89)$ & 11 & $(233,89)$ & 11 & 233 & YES & YES & YES & $1.56$ & $(2,3)$ & NO & 1127\\
$(234,53)$ & 13 & $(35,8)$ & 8 & 1 & YES & YES & YES & $1.62$ & $(4,2)$ & 1237 & 1128\\
$(235,97)$ & 12 & $(3,1)$ & 2 & 1 & YES & YES & YES & $1.78$ & $(2,3)$ & -- & 1129\\
$(235,97)$ & 12 & $(29,12)$ & 7 & 1 & YES & YES & YES & $1.67$ & $(2,3)$ & NO & 1130\\
$(235,97)$ & 12 & $(109,45)$ & 10 & 1 & YES & YES & YES & $1.78$ & $(2,3)$ & 1406 & 1131\\
$(236,65)$ & 12 & $(2,1)$ & 1 & 2 & YES & YES & YES & $1.50$ & $(4,2)$ & -- & 1132\\
$(236,65)$ & 12 & $(2,1)$ & 1 & 2 & YES & YES & YES & $1.62$ & $(4,2)$ & NO & 1133\\
$(236,69)$ & 12 & $(2,1)$ & 1 & 2 & YES & YES & YES & $1.67$ & $(2,3)$ & -- & 1134\\
$(236,65)$ & 12 & $(3,1)$ & 2 & 1 & YES & YES & YES & $1.62$ & $(4,2)$ & -- & 1135\\
$(236,65)$ & 12 & $(3,1)$ & 2 & 1 & YES & YES & YES & $1.62$ & $(4,2)$ & NO & 1136\\
$(236,69)$ & 12 & $(3,1)$ & 2 & 1 & YES & YES & YES & $1.38$ & $(4,2)$ & -- & 1137\\
$(236,69)$ & 12 & $(3,1)$ & 2 & 1 & YES & YES & YES & $1.82$ & $(2,3)$ & NO & 1138\\
$(236,69)$ & 12 & $(4,1)$ & 3 & 4 & YES & YES & YES & $1.38$ & $(4,2)$ & NO & 1139\\
$(236,69)$ & 12 & $(5,1)$ & 4 & 1 & YES & YES & YES & $1.67$ & $(2,3)$ & NO & 1140\\
$(236,69)$ & 12 & $(5,2)$ & 3 & 1 & YES & YES & YES & $1.78$ & $(2,3)$ & -- & 1141\\
$(236,69)$ & 12 & $(10,3)$ & 5 & 2 & YES & YES & YES & $1.56$ & $(2,3)$ & NO & 1142\\
$(236,69)$ & 12 & $(11,3)$ & 5 & 1 & YES & YES & YES & $1.89$ & $(2,3)$ & NO & 1143\\
$(236,69)$ & 12 & $(17,5)$ & 6 & 1 & YES & YES & YES & $1.82$ & $(2,3)$ & NO & 1144\\
$(236,65)$ & 12 & $(18,5)$ & 6 & 2 & YES & YES & YES & $1.50$ & $(4,2)$ & 843 & 1145\\
$(236,69)$ & 12 & $(31,9)$ & 8 & 1 & YES & YES & YES & $1.78$ & $(2,3)$ & NO & 1146\\
$(236,69)$ & 12 & $(41,12)$ & 8 & 1 & YES & YES & YES & $1.50$ & $(4,2)$ & NO & 1147\\
$(236,69)$ & 12 & $(147,43)$ & 11 & 1 & YES & YES & YES & $1.57$ & $(2,3)$ & 1719 & 1148\\
$(236,69)$ & 12 & $(236,69)$ & 12 & 236 & YES & YES & YES & $1.67$ & $(2,3)$ & NO & 1149\\
$(239,70)$ & 12 & $(3,1)$ & 2 & 1 & YES & YES & YES & $1.75$ & $(4,2)$ & NO & 1150\\
$(239,71)$ & 12 & $(3,1)$ & 2 & 1 & YES & YES & NO(2) & $1.82$ & $(4,2)$ & NO & 1151\\
$(239,99)$ & 12 & $(3,1)$ & 2 & 1 & YES & YES & YES & $1.78$ & $(2,3)$ & -- & 1152\\
$(239,99)$ & 12 & $(4,1)$ & 3 & 1 & YES & YES & YES & $1.78$ & $(2,3)$ & NO & 1153\\
$(239,70)$ & 12 & $(5,2)$ & 3 & 1 & YES & YES & YES & $1.62$ & $(4,2)$ & NO & 1154\\
$(239,99)$ & 12 & $(7,3)$ & 4 & 1 & YES & YES & YES & $1.78$ & $(2,3)$ & NO & 1155\\
$(239,70)$ & 12 & $(58,17)$ & 9 & 1 & YES & YES & YES & $1.50$ & $(4,2)$ & NO & 1156\\
$(239,71)$ & 12 & $(64,19)$ & 9 & 1 & YES & YES & NO(2) & $1.73$ & $(4,2)$ & NO & 1157\\
$(240,71)$ & 12 & $(2,1)$ & 1 & 2 & YES & YES & YES & $1.73$ & $(2,3)$ & -- & 1158\\
$(240,71)$ & 12 & $(3,1)$ & 2 & 3 & YES & YES & YES & $1.67$ & $(2,3)$ & -- & 1159\\
$(240,71)$ & 12 & $(5,1)$ & 4 & 5 & YES & YES & NO(2) & $1.82$ & $(4,2)$ & NO & 1160\\
$(240,71)$ & 12 & $(5,2)$ & 3 & 5 & YES & YES & YES & $1.43$ & $(2,3)$ & -- & 1161\\
$(240,71)$ & 12 & $(7,2)$ & 4 & 1 & YES & YES & YES & $1.78$ & $(2,3)$ & NO & 1162\\
$(240,71)$ & 12 & $(11,3)$ & 5 & 1 & YES & YES & YES & $1.71$ & $(2,3)$ & NO & 1163\\
$(240,71)$ & 12 & $(17,5)$ & 6 & 1 & YES & YES & YES & $1.78$ & $(2,3)$ & NO & 1164\\
$(240,71)$ & 12 & $(24,7)$ & 7 & 24 & YES & YES & YES & $1.71$ & $(2,3)$ & NO & 1165\\
$(240,71)$ & 12 & $(44,13)$ & 8 & 4 & YES & YES & YES & $1.67$ & $(2,3)$ & NO & 1166\\
$(240,71)$ & 12 & $(98,29)$ & 10 & 2 & YES & YES & YES & $1.67$ & $(2,3)$ & 1333 & 1167\\
$(240,71)$ & 12 & $(240,71)$ & 12 & 240 & YES & YES & YES & $1.78$ & $(2,3)$ & NO & 1168\\
$(241,100)$ & 12 & $(3,1)$ & 2 & 1 & YES & YES & YES & $1.62$ & $(4,2)$ & NO & 1169\\
$(241,100)$ & 12 & $(3,1)$ & 2 & 1 & YES & YES & YES & $1.62$ & $(4,2)$ & -- & 1170\\
$(241,101)$ & 12 & $(3,1)$ & 2 & 1 & YES & YES & YES & $1.62$ & $(4,2)$ & -- & 1171\\
$(241,100)$ & 12 & $(4,1)$ & 3 & 1 & YES & YES & YES & $1.62$ & $(4,2)$ & NO & 1172\\
$(241,100)$ & 12 & $(4,1)$ & 3 & 1 & YES & YES & YES & $1.62$ & $(4,2)$ & NO & 1173\\
$(241,65)$ & 12 & $(5,2)$ & 3 & 1 & YES & YES & YES & $1.78$ & $(2,3)$ & NO & 1174\\
$(241,65)$ & 12 & $(5,2)$ & 3 & 1 & YES & YES & YES & $1.78$ & $(2,3)$ & -- & 1175\\
$(241,100)$ & 12 & $(94,39)$ & 10 & 1 & YES & YES & YES & $1.62$ & $(4,2)$ & NO & 1176\\
$(241,101)$ & 12 & $(105,44)$ & 10 & 1 & YES & YES & YES & $1.62$ & $(4,2)$ & NO & 1177\\
$(241,100)$ & 12 & $(241,100)$ & 12 & 241 & YES & YES & YES & $1.62$ & $(4,2)$ & NO & 1178\\
$(241,101)$ & 12 & $(241,101)$ & 12 & 241 & YES & YES & YES & $1.62$ & $(4,2)$ & NO & 1179\\
$(242,67)$ & 12 & $(2,1)$ & 1 & 2 & YES & YES & YES & $1.38$ & $(4,2)$ & -- & 1180\\
$(242,67)$ & 12 & $(2,1)$ & 1 & 2 & YES & YES & NO(2) & $1.83$ & $(2,3)$ & NO & 1181\\
$(242,67)$ & 12 & $(3,1)$ & 2 & 1 & YES & YES & YES & $1.38$ & $(4,2)$ & -- & 1182\\
$(242,67)$ & 12 & $(242,67)$ & 12 & 242 & YES & YES & YES & $1.82$ & $(2,3)$ & NO & 1183\\
$(243,71)$ & 12 & $(3,1)$ & 2 & 3 & YES & YES & YES & $1.91$ & $(2,3)$ & -- & 1184\\
$(243,94)$ & 12 & $(3,1)$ & 2 & 3 & YES & YES & YES & $1.62$ & $(4,2)$ & NO & 1185\\
$(243,94)$ & 12 & $(4,1)$ & 3 & 1 & YES & YES & YES & $1.78$ & $(2,3)$ & NO & 1186\\
$(243,71)$ & 12 & $(7,2)$ & 4 & 1 & YES & YES & YES & $1.67$ & $(2,3)$ & -- & 1187\\
$(243,71)$ & 12 & $(13,4)$ & 6 & 1 & YES & YES & YES & $1.78$ & $(2,3)$ & NO & 1188\\
$(243,53)$ & 13 & $(19,4)$ & 7 & 1 & YES & YES & YES & $1.50$ & $(4,2)$ & NO & 1189\\
$(243,53)$ & 13 & $(37,8)$ & 8 & 1 & YES & YES & YES & $1.50$ & $(4,2)$ & 1307 & 1190\\
$(243,71)$ & 12 & $(89,26)$ & 10 & 1 & YES & YES & NO(2) & $1.79$ & $(2,3)$ & NO & 1191\\
$(244,71)$ & 13 & $(3,1)$ & 2 & 1 & YES & YES & YES & $1.62$ & $(4,2)$ & NO & 1192\\
$(244,71)$ & 13 & $(3,1)$ & 2 & 1 & YES & YES & YES & $1.62$ & $(4,2)$ & -- & 1193\\
$(244,55)$ & 13 & $(5,2)$ & 3 & 1 & YES & YES & YES & $1.67$ & $(2,3)$ & -- & 1194\\
$(244,71)$ & 13 & $(5,1)$ & 4 & 1 & YES & YES & YES & $1.75$ & $(4,2)$ & NO & 1195\\
$(244,71)$ & 13 & $(79,23)$ & 10 & 1 & YES & YES & YES & $1.75$ & $(4,2)$ & NO & 1196\\
$(246,91)$ & 12 & $(2,1)$ & 1 & 2 & YES & YES & YES & $1.62$ & $(4,2)$ & NO & 1197\\
$(246,91)$ & 12 & $(2,1)$ & 1 & 2 & YES & YES & YES & $1.62$ & $(4,2)$ & -- & 1198\\
$(246,73)$ & 12 & $(3,1)$ & 2 & 3 & YES & YES & YES & $1.67$ & $(2,3)$ & -- & 1199\\
$(246,91)$ & 12 & $(3,1)$ & 2 & 3 & YES & YES & YES & $1.89$ & $(2,3)$ & -- & 1200\\
$(246,95)$ & 12 & $(3,1)$ & 2 & 3 & YES & YES & YES & $1.62$ & $(4,2)$ & NO & 1201\\
$(246,95)$ & 12 & $(3,1)$ & 2 & 3 & YES & YES & YES & $2.00$ & $(2,3)$ & NO & 1202\\
$(246,95)$ & 12 & $(3,1)$ & 2 & 3 & YES & YES & YES & $2.00$ & $(2,3)$ & -- & 1203\\
$(246,73)$ & 12 & $(4,1)$ & 3 & 2 & YES & YES & YES & $1.91$ & $(2,3)$ & NO & 1204\\
$(246,91)$ & 12 & $(4,1)$ & 3 & 2 & YES & YES & YES & $1.78$ & $(2,3)$ & NO & 1205\\
$(246,101)$ & 12 & $(7,3)$ & 4 & 1 & YES & YES & YES & $1.75$ & $(4,2)$ & NO & 1206\\
$(246,73)$ & 12 & $(17,5)$ & 6 & 1 & YES & YES & NO(2) & $1.73$ & $(4,2)$ & NO & 1207\\
$(246,73)$ & 12 & $(44,13)$ & 8 & 2 & YES & YES & YES & $1.67$ & $(2,3)$ & NO & 1208\\
$(246,101)$ & 12 & $(56,23)$ & 9 & 2 & YES & YES & YES & $1.62$ & $(4,2)$ & 1016 & 1209\\
$(246,101)$ & 12 & $(95,39)$ & 10 & 1 & YES & YES & YES & $1.62$ & $(4,2)$ & NO & 1210\\
$(246,101)$ & 12 & $(151,62)$ & 11 & 1 & YES & YES & YES & $1.62$ & $(4,2)$ & NO & 1211\\
$(246,73)$ & 12 & $(219,65)$ & 12 & 3 & YES & YES & YES & $1.71$ & $(2,3)$ & NO & 1212\\
$(246,101)$ & 12 & $(246,101)$ & 12 & 246 & YES & YES & YES & $1.62$ & $(4,2)$ & NO & 1213\\
$(247,68)$ & 12 & $(2,1)$ & 1 & 1 & YES & YES & YES & $1.62$ & $(4,2)$ & -- & 1214\\
$(247,69)$ & 12 & $(2,1)$ & 1 & 1 & YES & YES & YES & $1.67$ & $(2,3)$ & -- & 1215\\
$(247,69)$ & 12 & $(2,1)$ & 1 & 1 & YES & YES & YES & $1.78$ & $(2,3)$ & NO & 1216\\
$(247,68)$ & 12 & $(3,1)$ & 2 & 1 & YES & YES & YES & $1.56$ & $(2,3)$ & -- & 1217\\
$(247,69)$ & 12 & $(3,1)$ & 2 & 1 & YES & YES & YES & $1.67$ & $(2,3)$ & -- & 1218\\
$(247,68)$ & 12 & $(5,1)$ & 4 & 1 & YES & YES & YES & $1.82$ & $(2,3)$ & NO & 1219\\
$(247,68)$ & 12 & $(5,2)$ & 3 & 1 & YES & YES & YES & $1.57$ & $(2,3)$ & -- & 1220\\
$(247,69)$ & 12 & $(5,2)$ & 3 & 1 & YES & YES & YES & $1.71$ & $(2,3)$ & -- & 1221\\
$(247,68)$ & 12 & $(7,2)$ & 4 & 1 & YES & YES & YES & $1.57$ & $(2,3)$ & -- & 1222\\
$(247,68)$ & 12 & $(10,3)$ & 5 & 1 & YES & YES & YES & $1.71$ & $(2,3)$ & NO & 1223\\
$(247,69)$ & 12 & $(10,3)$ & 5 & 1 & YES & YES & YES & $1.78$ & $(2,3)$ & NO & 1224\\
$(247,69)$ & 12 & $(11,3)$ & 5 & 1 & YES & YES & YES & $1.67$ & $(2,3)$ & NO & 1225\\
$(247,68)$ & 12 & $(18,5)$ & 6 & 1 & YES & YES & YES & $1.56$ & $(2,3)$ & NO & 1226\\
$(247,69)$ & 12 & $(18,5)$ & 6 & 1 & YES & YES & YES & $1.67$ & $(2,3)$ & NO & 1227\\
$(247,68)$ & 12 & $(29,8)$ & 7 & 1 & YES & YES & YES & $1.62$ & $(4,2)$ & NO & 1228\\
$(247,69)$ & 12 & $(29,8)$ & 7 & 1 & YES & YES & YES & $1.57$ & $(2,3)$ & NO & 1229\\
$(247,69)$ & 12 & $(32,9)$ & 8 & 1 & YES & YES & YES & $1.78$ & $(2,3)$ & NO & 1230\\
$(247,68)$ & 12 & $(47,13)$ & 8 & 1 & YES & YES & YES & $1.57$ & $(2,3)$ & NO & 1231\\
$(247,69)$ & 12 & $(61,17)$ & 9 & 1 & YES & YES & YES & $1.43$ & $(2,3)$ & 1645 & 1232\\
$(247,69)$ & 12 & $(247,69)$ & 12 & 247 & YES & YES & YES & $1.67$ & $(2,3)$ & NO & 1233\\
$(249,95)$ & 12 & $(3,1)$ & 2 & 3 & YES & YES & YES & $1.67$ & $(2,3)$ & -- & 1234\\
$(249,95)$ & 12 & $(173,66)$ & 11 & 1 & YES & YES & YES & $1.67$ & $(2,3)$ & NO & 1235\\
$(250,57)$ & 13 & $(5,2)$ & 3 & 5 & YES & YES & YES & $1.67$ & $(2,3)$ & -- & 1236\\
$(250,57)$ & 13 & $(31,7)$ & 8 & 1 & YES & YES & YES & $1.62$ & $(4,2)$ & 1128 & 1237\\
$(250,57)$ & 13 & $(48,11)$ & 9 & 2 & YES & YES & YES & $1.78$ & $(2,3)$ & NO & 1238\\
$(251,104)$ & 12 & $(3,1)$ & 2 & 1 & YES & YES & YES & $1.62$ & $(4,2)$ & -- & 1239\\
$(251,104)$ & 12 & $(4,1)$ & 3 & 1 & YES & YES & YES & $1.57$ & $(2,3)$ & NO & 1240\\
$(251,104)$ & 12 & $(5,1)$ & 4 & 1 & YES & YES & YES & $1.71$ & $(2,3)$ & NO & 1241\\
$(253,74)$ & 12 & $(2,1)$ & 1 & 1 & YES & YES & YES & $1.38$ & $(4,2)$ & -- & 1242\\
$(253,68)$ & 12 & $(3,1)$ & 2 & 1 & YES & YES & NO(2) & $1.85$ & $(4,2)$ & NO & 1243\\
$(253,74)$ & 12 & $(3,1)$ & 2 & 1 & YES & YES & YES & $1.73$ & $(2,3)$ & -- & 1244\\
$(253,74)$ & 12 & $(3,1)$ & 2 & 1 & YES & YES & YES & $1.78$ & $(2,3)$ & NO & 1245\\
$(253,74)$ & 12 & $(7,2)$ & 4 & 1 & YES & YES & NO(2) & $1.77$ & $(4,2)$ & NO & 1246\\
$(253,74)$ & 12 & $(10,3)$ & 5 & 1 & YES & YES & YES & $1.73$ & $(2,3)$ & NO & 1247\\
$(253,68)$ & 12 & $(11,3)$ & 5 & 11 & YES & YES & NO(2) & $1.85$ & $(4,2)$ & NO & 1248\\
$(253,74)$ & 12 & $(24,7)$ & 7 & 1 & YES & YES & YES & $1.38$ & $(4,2)$ & 776 & 1249\\
$(253,74)$ & 12 & $(41,12)$ & 8 & 1 & YES & YES & NO(2) & $1.77$ & $(4,2)$ & 940 & 1250\\
$(253,74)$ & 12 & $(65,19)$ & 9 & 1 & YES & YES & YES & $1.73$ & $(2,3)$ & NO & 1251\\
$(253,106)$ & 12 & $(74,31)$ & 9 & 1 & YES & YES & YES & $1.50$ & $(4,2)$ & NO & 1252\\
$(254,71)$ & 12 & $(2,1)$ & 1 & 2 & YES & YES & NO(2) & $1.83$ & $(2,3)$ & NO & 1253\\
$(254,75)$ & 12 & $(2,1)$ & 1 & 2 & YES & YES & YES & $1.38$ & $(4,2)$ & -- & 1254\\
$(254,71)$ & 12 & $(3,1)$ & 2 & 1 & YES & YES & YES & $1.82$ & $(2,3)$ & -- & 1255\\
$(254,75)$ & 12 & $(3,1)$ & 2 & 1 & YES & YES & YES & $1.64$ & $(2,3)$ & -- & 1256\\
$(254,75)$ & 12 & $(7,2)$ & 4 & 1 & YES & YES & YES & $1.38$ & $(4,2)$ & NO & 1257\\
$(254,75)$ & 12 & $(27,8)$ & 7 & 1 & YES & YES & YES & $1.56$ & $(2,3)$ & NO & 1258\\
$(254,71)$ & 12 & $(43,12)$ & 8 & 1 & YES & YES & YES & $1.78$ & $(2,3)$ & 1451 & 1259\\
$(254,75)$ & 12 & $(78,23)$ & 10 & 2 & YES & YES & YES & $1.57$ & $(2,3)$ & NO & 1260\\
$(254,75)$ & 12 & $(105,31)$ & 10 & 1 & YES & YES & YES & $1.67$ & $(2,3)$ & NO & 1261\\
$(255,107)$ & 12 & $(3,1)$ & 2 & 3 & YES & YES & YES & $1.62$ & $(4,2)$ & NO & 1262\\
$(255,107)$ & 12 & $(3,1)$ & 2 & 3 & YES & YES & YES & $1.62$ & $(4,2)$ & -- & 1263\\
$(255,107)$ & 12 & $(5,1)$ & 4 & 5 & YES & YES & YES & $1.62$ & $(4,2)$ & -- & 1264\\
$(255,107)$ & 12 & $(112,47)$ & 10 & 1 & YES & YES & YES & $1.62$ & $(4,2)$ & NO & 1265\\
$(255,107)$ & 12 & $(255,107)$ & 12 & 255 & YES & YES & YES & $1.62$ & $(4,2)$ & NO & 1266\\
$(256,75)$ & 12 & $(2,1)$ & 1 & 2 & YES & YES & YES & $1.56$ & $(2,3)$ & -- & 1267\\
$(256,75)$ & 12 & $(2,1)$ & 1 & 2 & YES & YES & YES & $1.62$ & $(4,2)$ & NO & 1268\\
$(256,75)$ & 12 & $(3,1)$ & 2 & 1 & YES & YES & YES & $1.56$ & $(2,3)$ & -- & 1269\\
$(256,75)$ & 12 & $(4,1)$ & 3 & 4 & YES & YES & YES & $1.50$ & $(4,2)$ & NO & 1270\\
$(256,75)$ & 12 & $(10,3)$ & 5 & 2 & YES & YES & YES & $1.67$ & $(2,3)$ & NO & 1271\\
$(256,75)$ & 12 & $(24,7)$ & 7 & 8 & YES & YES & YES & $1.67$ & $(2,3)$ & NO & 1272\\
$(256,75)$ & 12 & $(58,17)$ & 9 & 2 & YES & YES & YES & $1.50$ & $(4,2)$ & 1071 & 1273\\
$(256,75)$ & 12 & $(99,29)$ & 10 & 1 & YES & YES & YES & $1.82$ & $(2,3)$ & NO & 1274\\
$(257,71)$ & 12 & $(2,1)$ & 1 & 1 & YES & YES & YES & $1.67$ & $(2,3)$ & -- & 1275\\
$(257,76)$ & 12 & $(2,1)$ & 1 & 1 & YES & YES & NO(2) & $1.64$ & $(4,2)$ & -- & 1276\\
$(257,76)$ & 12 & $(2,1)$ & 1 & 1 & YES & YES & YES & $1.78$ & $(2,3)$ & NO & 1277\\
$(257,76)$ & 12 & $(3,1)$ & 2 & 1 & YES & YES & YES & $1.67$ & $(2,3)$ & -- & 1278\\
$(257,76)$ & 12 & $(4,1)$ & 3 & 1 & YES & YES & YES & $1.67$ & $(2,3)$ & NO & 1279\\
$(257,71)$ & 12 & $(5,2)$ & 3 & 1 & YES & YES & YES & $1.57$ & $(2,3)$ & NO & 1280\\
$(257,76)$ & 12 & $(5,1)$ & 4 & 1 & YES & YES & YES & $1.56$ & $(2,3)$ & NO & 1281\\
$(257,76)$ & 12 & $(7,2)$ & 4 & 1 & YES & YES & YES & $1.67$ & $(2,3)$ & NO & 1282\\
$(257,76)$ & 12 & $(17,5)$ & 6 & 1 & YES & YES & YES & $1.67$ & $(2,3)$ & NO & 1283\\
$(257,71)$ & 12 & $(25,7)$ & 7 & 1 & YES & YES & YES & $1.57$ & $(2,3)$ & NO & 1284\\
$(257,76)$ & 12 & $(27,8)$ & 7 & 1 & YES & YES & NO(2) & $1.73$ & $(4,2)$ & NO & 1285\\
$(257,76)$ & 12 & $(44,13)$ & 8 & 1 & YES & YES & YES & $1.67$ & $(2,3)$ & NO & 1286\\
$(257,71)$ & 12 & $(47,13)$ & 8 & 1 & YES & YES & YES & $1.67$ & $(2,3)$ & NO & 1287\\
$(257,76)$ & 12 & $(115,34)$ & 10 & 1 & YES & YES & YES & $1.56$ & $(2,3)$ & 1498 & 1288\\
$(259,100)$ & 12 & $(3,1)$ & 2 & 1 & YES & YES & YES & $1.62$ & $(4,2)$ & NO & 1289\\
$(259,100)$ & 12 & $(3,1)$ & 2 & 1 & YES & YES & YES & $1.62$ & $(4,2)$ & -- & 1290\\
$(259,100)$ & 12 & $(3,1)$ & 2 & 1 & YES & YES & YES & $1.75$ & $(4,2)$ & NO & 1291\\
$(260,79)$ & 13 & $(3,1)$ & 2 & 1 & YES & YES & YES & $1.62$ & $(4,2)$ & -- & 1292\\
$(260,79)$ & 13 & $(7,2)$ & 4 & 1 & YES & YES & YES & $1.62$ & $(4,2)$ & NO & 1293\\
$(261,100)$ & 12 & $(2,1)$ & 1 & 1 & YES & YES & YES & $1.62$ & $(4,2)$ & -- & 1294\\
$(261,77)$ & 13 & $(5,1)$ & 4 & 1 & YES & YES & YES & $1.50$ & $(4,2)$ & -- & 1295\\
$(261,100)$ & 12 & $(21,8)$ & 6 & 3 & YES & YES & YES & $1.67$ & $(2,3)$ & NO & 1296\\
$(261,61)$ & 13 & $(22,5)$ & 7 & 1 & YES & YES & YES & $1.67$ & $(2,3)$ & NO & 1297\\
$(262,73)$ & 13 & $(5,1)$ & 4 & 1 & YES & YES & YES & $1.62$ & $(4,2)$ & NO & 1298\\
$(263,109)$ & 12 & $(2,1)$ & 1 & 1 & YES & YES & YES & $1.78$ & $(2,3)$ & -- & 1299\\
$(263,57)$ & 13 & $(5,2)$ & 3 & 1 & YES & YES & YES & $1.86$ & $(2,3)$ & NO & 1300\\
$(263,57)$ & 13 & $(5,2)$ & 3 & 1 & YES & YES & YES & $1.86$ & $(2,3)$ & -- & 1301\\
$(263,60)$ & 13 & $(5,2)$ & 3 & 1 & YES & YES & YES & $1.56$ & $(2,3)$ & -- & 1302\\
$(263,109)$ & 12 & $(5,2)$ & 3 & 1 & YES & YES & YES & $1.62$ & $(4,2)$ & NO & 1303\\
$(263,57)$ & 13 & $(7,2)$ & 4 & 1 & YES & YES & YES & $1.86$ & $(2,3)$ & NO & 1304\\
$(263,60)$ & 13 & $(14,3)$ & 6 & 1 & YES & YES & YES & $1.67$ & $(2,3)$ & NO & 1305\\
$(263,109)$ & 12 & $(17,7)$ & 6 & 1 & YES & YES & YES & $2.00$ & $(2,3)$ & NO & 1306\\
$(263,57)$ & 13 & $(32,7)$ & 8 & 1 & YES & YES & YES & $1.50$ & $(4,2)$ & 1190 & 1307\\
$(263,109)$ & 12 & $(263,109)$ & 12 & 263 & YES & YES & YES & $1.78$ & $(2,3)$ & NO & 1308\\
$(265,73)$ & 12 & $(3,1)$ & 2 & 1 & YES & YES & YES & $1.67$ & $(2,3)$ & -- & 1309\\
$(265,74)$ & 12 & $(3,1)$ & 2 & 1 & YES & YES & YES & $1.56$ & $(2,3)$ & -- & 1310\\
$(265,73)$ & 12 & $(10,3)$ & 5 & 5 & YES & YES & YES & $1.43$ & $(2,3)$ & NO & 1311\\
$(265,73)$ & 12 & $(18,5)$ & 6 & 1 & YES & YES & YES & $1.67$ & $(2,3)$ & NO & 1312\\
$(265,73)$ & 12 & $(40,11)$ & 8 & 5 & YES & YES & YES & $1.67$ & $(2,3)$ & 1428 & 1313\\
$(266,101)$ & 12 & $(2,1)$ & 1 & 2 & YES & YES & YES & $1.78$ & $(2,3)$ & -- & 1314\\
$(266,101)$ & 12 & $(2,1)$ & 1 & 2 & YES & YES & YES & $1.89$ & $(2,3)$ & NO & 1315\\
$(266,79)$ & 12 & $(3,1)$ & 2 & 1 & YES & YES & NO(2) & $1.73$ & $(4,2)$ & -- & 1316\\
$(266,101)$ & 12 & $(3,1)$ & 2 & 1 & YES & YES & YES & $1.78$ & $(2,3)$ & -- & 1317\\
$(266,79)$ & 12 & $(4,1)$ & 3 & 2 & YES & YES & NO(2) & $1.64$ & $(4,2)$ & NO & 1318\\
$(266,101)$ & 12 & $(21,8)$ & 6 & 7 & YES & YES & YES & $2.00$ & $(2,3)$ & NO & 1319\\
$(266,79)$ & 12 & $(64,19)$ & 9 & 2 & YES & YES & NO(2) & $1.64$ & $(4,2)$ & 1110 & 1320\\
$(266,79)$ & 12 & $(101,30)$ & 10 & 1 & YES & YES & NO(2) & $1.73$ & $(4,2)$ & NO & 1321\\
$(267,79)$ & 12 & $(2,1)$ & 1 & 1 & YES & YES & YES & $1.67$ & $(2,3)$ & -- & 1322\\
$(267,79)$ & 12 & $(2,1)$ & 1 & 1 & YES & YES & YES & $1.67$ & $(2,3)$ & NO & 1323\\
$(267,79)$ & 12 & $(3,1)$ & 2 & 3 & YES & YES & YES & $1.82$ & $(2,3)$ & -- & 1324\\
$(267,79)$ & 12 & $(5,1)$ & 4 & 1 & YES & YES & YES & $1.82$ & $(2,3)$ & NO & 1325\\
$(267,79)$ & 12 & $(5,2)$ & 3 & 1 & YES & YES & YES & $1.78$ & $(2,3)$ & NO & 1326\\
$(267,79)$ & 12 & $(10,3)$ & 5 & 1 & YES & YES & NO(2) & $1.73$ & $(4,2)$ & NO & 1327\\
$(267,79)$ & 12 & $(11,3)$ & 5 & 1 & YES & YES & YES & $1.57$ & $(2,3)$ & NO & 1328\\
$(267,79)$ & 12 & $(13,4)$ & 6 & 1 & YES & YES & YES & $1.78$ & $(2,3)$ & NO & 1329\\
$(267,79)$ & 12 & $(17,5)$ & 6 & 1 & YES & YES & YES & $1.67$ & $(2,3)$ & NO & 1330\\
$(267,79)$ & 12 & $(37,11)$ & 8 & 1 & YES & YES & YES & $1.78$ & $(2,3)$ & NO & 1331\\
$(267,79)$ & 12 & $(44,13)$ & 8 & 1 & YES & YES & YES & $1.67$ & $(2,3)$ & 1496 & 1332\\
$(267,79)$ & 12 & $(71,21)$ & 9 & 1 & YES & YES & YES & $1.67$ & $(2,3)$ & 1167 & 1333\\
$(267,79)$ & 12 & $(98,29)$ & 10 & 1 & YES & YES & YES & $1.67$ & $(2,3)$ & NO & 1334\\
$(268,111)$ & 12 & $(4,1)$ & 3 & 4 & YES & YES & YES & $1.62$ & $(4,2)$ & -- & 1335\\
$(268,99)$ & 12 & $(157,58)$ & 11 & 1 & YES & YES & YES & $1.78$ & $(2,3)$ & NO & 1336\\
$(268,111)$ & 12 & $(268,111)$ & 12 & 268 & YES & YES & YES & $1.57$ & $(2,3)$ & NO & 1337\\
$(269,75)$ & 12 & $(2,1)$ & 1 & 1 & YES & YES & YES & $1.67$ & $(2,3)$ & -- & 1338\\
$(269,75)$ & 12 & $(3,1)$ & 2 & 1 & YES & YES & YES & $1.67$ & $(2,3)$ & NO & 1339\\
$(269,75)$ & 12 & $(3,1)$ & 2 & 1 & YES & YES & YES & $1.67$ & $(2,3)$ & -- & 1340\\
$(269,75)$ & 12 & $(5,2)$ & 3 & 1 & YES & YES & YES & $1.57$ & $(2,3)$ & -- & 1341\\
$(269,75)$ & 12 & $(11,3)$ & 5 & 1 & YES & YES & YES & $1.67$ & $(2,3)$ & NO & 1342\\
$(269,75)$ & 12 & $(25,7)$ & 7 & 1 & YES & YES & YES & $1.67$ & $(2,3)$ & NO & 1343\\
$(269,75)$ & 12 & $(147,41)$ & 11 & 1 & YES & YES & YES & $1.43$ & $(2,3)$ & NO & 1344\\
$(271,80)$ & 12 & $(2,1)$ & 1 & 1 & YES & YES & YES & $1.73$ & $(2,3)$ & -- & 1345\\
$(271,112)$ & 12 & $(2,1)$ & 1 & 1 & YES & YES & YES & $1.56$ & $(2,3)$ & -- & 1346\\
$(271,80)$ & 12 & $(3,1)$ & 2 & 1 & YES & YES & YES & $1.73$ & $(2,3)$ & -- & 1347\\
$(271,80)$ & 12 & $(3,1)$ & 2 & 1 & YES & YES & YES & $1.82$ & $(2,3)$ & NO & 1348\\
$(271,80)$ & 12 & $(4,1)$ & 3 & 1 & YES & YES & YES & $1.56$ & $(2,3)$ & -- & 1349\\
$(271,80)$ & 12 & $(10,3)$ & 5 & 1 & YES & YES & YES & $1.73$ & $(2,3)$ & 820 & 1350\\
$(271,80)$ & 12 & $(27,8)$ & 7 & 1 & YES & YES & YES & $1.56$ & $(2,3)$ & NO & 1351\\
$(271,80)$ & 12 & $(44,13)$ & 8 & 1 & YES & YES & YES & $1.56$ & $(2,3)$ & NO & 1352\\
$(271,75)$ & 12 & $(65,18)$ & 9 & 1 & YES & YES & YES & $1.56$ & $(2,3)$ & NO & 1353\\
$(271,112)$ & 12 & $(121,50)$ & 10 & 1 & YES & YES & YES & $1.43$ & $(2,3)$ & 1547 & 1354\\
$(271,80)$ & 12 & $(166,49)$ & 11 & 1 & YES & YES & YES & $1.82$ & $(2,3)$ & NO & 1355\\
$(271,112)$ & 12 & $(196,81)$ & 11 & 1 & YES & YES & YES & $1.56$ & $(2,3)$ & NO & 1356\\
$(271,80)$ & 12 & $(271,80)$ & 12 & 271 & YES & YES & YES & $1.91$ & $(2,3)$ & NO & 1357\\
$(273,101)$ & 12 & $(3,1)$ & 2 & 3 & YES & YES & YES & $1.78$ & $(2,3)$ & -- & 1358\\
$(273,101)$ & 12 & $(4,1)$ & 3 & 1 & YES & YES & YES & $1.78$ & $(2,3)$ & NO & 1359\\
$(273,101)$ & 12 & $(273,101)$ & 12 & 273 & YES & YES & YES & $1.67$ & $(2,3)$ & NO & 1360\\
$(274,81)$ & 12 & $(2,1)$ & 1 & 2 & YES & YES & NO(2) & $1.55$ & $(4,2)$ & -- & 1361\\
$(274,115)$ & 12 & $(2,1)$ & 1 & 2 & YES & YES & YES & $1.50$ & $(4,2)$ & -- & 1362\\
$(274,81)$ & 12 & $(3,1)$ & 2 & 1 & YES & YES & YES & $1.56$ & $(2,3)$ & -- & 1363\\
$(274,81)$ & 12 & $(3,1)$ & 2 & 1 & YES & YES & YES & $1.78$ & $(2,3)$ & NO & 1364\\
$(274,81)$ & 12 & $(4,1)$ & 3 & 2 & YES & YES & YES & $1.56$ & $(2,3)$ & -- & 1365\\
$(274,81)$ & 12 & $(7,2)$ & 4 & 1 & YES & YES & YES & $1.82$ & $(2,3)$ & NO & 1366\\
$(274,81)$ & 12 & $(9,2)$ & 5 & 1 & YES & YES & YES & $1.78$ & $(2,3)$ & NO & 1367\\
$(274,81)$ & 12 & $(17,5)$ & 6 & 1 & YES & YES & YES & $1.67$ & $(2,3)$ & NO & 1368\\
$(274,115)$ & 12 & $(19,8)$ & 6 & 1 & YES & YES & YES & $1.78$ & $(2,3)$ & NO & 1369\\
$(274,81)$ & 12 & $(27,8)$ & 7 & 1 & YES & YES & YES & $1.56$ & $(2,3)$ & 809 & 1370\\
$(274,81)$ & 12 & $(159,47)$ & 11 & 1 & YES & YES & YES & $1.67$ & $(2,3)$ & NO & 1371\\
$(274,115)$ & 12 & $(274,115)$ & 12 & 274 & YES & YES & YES & $1.50$ & $(4,2)$ & NO & 1372\\
$(275,76)$ & 12 & $(2,1)$ & 1 & 1 & YES & YES & YES & $1.44$ & $(2,3)$ & -- & 1373\\
$(275,76)$ & 12 & $(2,1)$ & 1 & 1 & YES & YES & YES & $1.67$ & $(2,3)$ & NO & 1374\\
$(275,76)$ & 12 & $(3,1)$ & 2 & 1 & YES & YES & YES & $1.44$ & $(2,3)$ & -- & 1375\\
$(275,76)$ & 12 & $(3,1)$ & 2 & 1 & YES & YES & YES & $1.78$ & $(2,3)$ & NO & 1376\\
$(275,76)$ & 12 & $(5,1)$ & 4 & 5 & YES & YES & YES & $1.56$ & $(2,3)$ & NO & 1377\\
$(275,76)$ & 12 & $(7,2)$ & 4 & 1 & YES & YES & YES & $1.78$ & $(2,3)$ & NO & 1378\\
$(275,76)$ & 12 & $(11,3)$ & 5 & 11 & YES & YES & YES & $1.78$ & $(2,3)$ & NO & 1379\\
$(275,76)$ & 12 & $(18,5)$ & 6 & 1 & YES & YES & YES & $1.56$ & $(2,3)$ & NO & 1380\\
$(275,76)$ & 12 & $(29,8)$ & 7 & 1 & YES & YES & YES & $1.78$ & $(2,3)$ & NO & 1381\\
$(275,76)$ & 12 & $(40,11)$ & 8 & 5 & YES & YES & YES & $1.57$ & $(2,3)$ & NO & 1382\\
$(275,76)$ & 12 & $(47,13)$ & 8 & 1 & YES & YES & YES & $1.56$ & $(2,3)$ & NO & 1383\\
$(275,76)$ & 12 & $(275,76)$ & 12 & 275 & YES & YES & YES & $1.56$ & $(2,3)$ & NO & 1384\\
$(277,81)$ & 12 & $(2,1)$ & 1 & 1 & YES & YES & YES & $1.73$ & $(2,3)$ & -- & 1385\\
$(277,81)$ & 12 & $(3,1)$ & 2 & 1 & YES & YES & YES & $1.56$ & $(2,3)$ & -- & 1386\\
$(277,106)$ & 12 & $(3,1)$ & 2 & 1 & YES & YES & YES & $1.57$ & $(2,3)$ & -- & 1387\\
$(277,106)$ & 12 & $(3,1)$ & 2 & 1 & YES & YES & YES & $1.57$ & $(2,3)$ & NO & 1388\\
$(277,81)$ & 12 & $(4,1)$ & 3 & 1 & YES & YES & YES & $1.78$ & $(2,3)$ & NO & 1389\\
$(277,60)$ & 13 & $(5,2)$ & 3 & 1 & YES & YES & YES & $1.67$ & $(2,3)$ & -- & 1390\\
$(277,81)$ & 12 & $(10,3)$ & 5 & 1 & YES & YES & YES & $1.56$ & $(2,3)$ & NO & 1391\\
$(277,81)$ & 12 & $(17,5)$ & 6 & 1 & YES & YES & YES & $1.56$ & $(2,3)$ & NO & 1392\\
$(277,81)$ & 12 & $(31,9)$ & 8 & 1 & YES & YES & YES & $1.78$ & $(2,3)$ & NO & 1393\\
$(277,81)$ & 12 & $(41,12)$ & 8 & 1 & YES & YES & YES & $1.56$ & $(2,3)$ & NO & 1394\\
$(277,81)$ & 12 & $(106,31)$ & 10 & 1 & YES & YES & NO(2) & $1.73$ & $(4,2)$ & NO & 1395\\
$(277,106)$ & 12 & $(115,44)$ & 10 & 1 & YES & YES & YES & $1.86$ & $(2,3)$ & 1525 & 1396\\
$(277,106)$ & 12 & $(277,106)$ & 12 & 277 & YES & YES & YES & $1.71$ & $(2,3)$ & NO & 1397\\
$(278,63)$ & 13 & $(3,1)$ & 2 & 1 & YES & YES & YES & $1.82$ & $(2,3)$ & -- & 1398\\
$(278,65)$ & 13 & $(17,4)$ & 7 & 1 & YES & YES & YES & $1.73$ & $(2,3)$ & NO & 1399\\
$(279,65)$ & 13 & $(5,2)$ & 3 & 1 & YES & YES & YES & $1.67$ & $(2,3)$ & NO & 1400\\
$(280,107)$ & 12 & $(3,1)$ & 2 & 1 & YES & YES & YES & $1.62$ & $(4,2)$ & NO & 1401\\
$(281,116)$ & 12 & $(2,1)$ & 1 & 1 & YES & YES & YES & $1.56$ & $(2,3)$ & -- & 1402\\
$(281,64)$ & 13 & $(4,1)$ & 3 & 1 & YES & YES & NO(2) & $1.77$ & $(4,2)$ & -- & 1403\\
$(281,116)$ & 12 & $(4,1)$ & 3 & 1 & YES & YES & YES & $1.56$ & $(2,3)$ & -- & 1404\\
$(281,116)$ & 12 & $(7,3)$ & 4 & 1 & YES & YES & YES & $1.78$ & $(2,3)$ & NO & 1405\\
$(281,116)$ & 12 & $(63,26)$ & 9 & 1 & YES & YES & YES & $1.78$ & $(2,3)$ & 1131 & 1406\\
$(281,64)$ & 13 & $(180,41)$ & 12 & 1 & YES & YES & NO(2) & $1.85$ & $(4,2)$ & NO & 1407\\
$(281,116)$ & 12 & $(281,116)$ & 12 & 281 & YES & YES & YES & $1.43$ & $(2,3)$ & NO & 1408\\
$(282,109)$ & 12 & $(3,1)$ & 2 & 3 & YES & YES & YES & $1.57$ & $(2,3)$ & -- & 1409\\
$(282,109)$ & 12 & $(18,7)$ & 6 & 6 & YES & YES & YES & $1.71$ & $(2,3)$ & NO & 1410\\
$(282,109)$ & 12 & $(75,29)$ & 9 & 3 & YES & YES & YES & $1.57$ & $(2,3)$ & NO & 1411\\
$(283,86)$ & 13 & $(3,1)$ & 2 & 1 & YES & YES & YES & $1.62$ & $(4,2)$ & NO & 1412\\
$(283,86)$ & 13 & $(3,1)$ & 2 & 1 & YES & YES & YES & $1.62$ & $(4,2)$ & -- & 1413\\
$(283,86)$ & 13 & $(7,2)$ & 4 & 1 & YES & YES & YES & $1.62$ & $(4,2)$ & NO & 1414\\
$(283,108)$ & 12 & $(131,50)$ & 10 & 1 & YES & YES & YES & $1.78$ & $(2,3)$ & 1613 & 1415\\
$(284,83)$ & 13 & $(41,12)$ & 8 & 1 & YES & YES & YES & $1.78$ & $(2,3)$ & NO & 1416\\
$(284,65)$ & 13 & $(48,11)$ & 9 & 4 & YES & YES & YES & $1.38$ & $(4,2)$ & NO & 1417\\
$(286,79)$ & 12 & $(2,1)$ & 1 & 2 & YES & YES & YES & $1.56$ & $(2,3)$ & -- & 1418\\
$(286,105)$ & 12 & $(3,1)$ & 2 & 1 & YES & YES & YES & $1.67$ & $(2,3)$ & -- & 1419\\
$(286,79)$ & 12 & $(5,1)$ & 4 & 1 & YES & YES & NO(2) & $1.64$ & $(4,2)$ & NO & 1420\\
$(286,79)$ & 12 & $(11,3)$ & 5 & 11 & YES & YES & NO(2) & $1.64$ & $(4,2)$ & NO & 1421\\
$(286,105)$ & 12 & $(30,11)$ & 7 & 2 & YES & YES & YES & $1.78$ & $(2,3)$ & NO & 1422\\
$(287,79)$ & 12 & $(2,1)$ & 1 & 1 & YES & YES & NO(2) & $1.64$ & $(4,2)$ & -- & 1423\\
$(287,79)$ & 12 & $(2,1)$ & 1 & 1 & YES & YES & NO(2) & $1.73$ & $(4,2)$ & NO & 1424\\
$(287,106)$ & 12 & $(2,1)$ & 1 & 1 & YES & YES & YES & $1.62$ & $(4,2)$ & -- & 1425\\
$(287,79)$ & 12 & $(4,1)$ & 3 & 1 & YES & YES & YES & $1.56$ & $(2,3)$ & -- & 1426\\
$(287,106)$ & 12 & $(5,2)$ & 3 & 1 & YES & YES & YES & $1.86$ & $(2,3)$ & NO & 1427\\
$(287,79)$ & 12 & $(29,8)$ & 7 & 1 & YES & YES & YES & $1.67$ & $(2,3)$ & 1313 & 1428\\
$(287,111)$ & 12 & $(31,12)$ & 7 & 1 & YES & YES & YES & $1.78$ & $(2,3)$ & NO & 1429\\
$(287,79)$ & 12 & $(40,11)$ & 8 & 1 & YES & YES & NO(2) & $1.73$ & $(4,2)$ & NO & 1430\\
$(287,79)$ & 12 & $(109,30)$ & 10 & 1 & YES & YES & NO(2) & $1.82$ & $(4,2)$ & NO & 1431\\
$(288,121)$ & 12 & $(3,1)$ & 2 & 3 & YES & YES & YES & $1.67$ & $(2,3)$ & NO & 1432\\
$(288,121)$ & 12 & $(3,1)$ & 2 & 3 & YES & YES & YES & $1.67$ & $(2,3)$ & -- & 1433\\
$(288,121)$ & 12 & $(4,1)$ & 3 & 4 & YES & YES & YES & $1.57$ & $(2,3)$ & NO & 1434\\
$(288,121)$ & 12 & $(4,1)$ & 3 & 4 & YES & YES & YES & $1.57$ & $(2,3)$ & -- & 1435\\
$(288,121)$ & 12 & $(169,71)$ & 11 & 1 & YES & YES & YES & $1.67$ & $(2,3)$ & NO & 1436\\
$(288,121)$ & 12 & $(288,121)$ & 12 & 288 & YES & YES & YES & $1.57$ & $(2,3)$ & NO & 1437\\
$(289,63)$ & 13 & $(2,1)$ & 1 & 1 & YES & YES & NO(2) & $1.93$ & $(2,3)$ & NO & 1438\\
$(289,80)$ & 12 & $(2,1)$ & 1 & 1 & YES & YES & YES & $1.44$ & $(2,3)$ & -- & 1439\\
$(289,80)$ & 12 & $(3,1)$ & 2 & 1 & YES & YES & YES & $1.44$ & $(2,3)$ & -- & 1440\\
$(289,80)$ & 12 & $(5,2)$ & 3 & 1 & YES & YES & YES & $1.67$ & $(2,3)$ & -- & 1441\\
$(289,80)$ & 12 & $(29,8)$ & 7 & 1 & YES & YES & YES & $1.67$ & $(2,3)$ & NO & 1442\\
$(289,63)$ & 13 & $(41,9)$ & 9 & 1 & YES & YES & YES & $1.89$ & $(2,3)$ & NO & 1443\\
$(289,80)$ & 12 & $(47,13)$ & 8 & 1 & YES & YES & YES & $1.67$ & $(2,3)$ & NO & 1444\\
$(290,81)$ & 12 & $(2,1)$ & 1 & 2 & YES & YES & YES & $1.67$ & $(2,3)$ & -- & 1445\\
$(290,81)$ & 12 & $(3,1)$ & 2 & 1 & YES & YES & YES & $1.56$ & $(2,3)$ & -- & 1446\\
$(290,81)$ & 12 & $(5,1)$ & 4 & 5 & YES & YES & YES & $1.82$ & $(2,3)$ & NO & 1447\\
$(290,81)$ & 12 & $(10,3)$ & 5 & 10 & YES & YES & YES & $1.57$ & $(2,3)$ & NO & 1448\\
$(290,81)$ & 12 & $(11,3)$ & 5 & 1 & YES & YES & YES & $1.67$ & $(2,3)$ & NO & 1449\\
$(290,81)$ & 12 & $(18,5)$ & 6 & 2 & YES & YES & YES & $1.56$ & $(2,3)$ & NO & 1450\\
$(290,81)$ & 12 & $(25,7)$ & 7 & 5 & YES & YES & YES & $1.78$ & $(2,3)$ & 1259 & 1451\\
$(290,81)$ & 12 & $(43,12)$ & 8 & 1 & YES & YES & YES & $1.82$ & $(2,3)$ & NO & 1452\\
$(291,68)$ & 13 & $(3,1)$ & 2 & 3 & YES & YES & NO(2) & $1.77$ & $(4,2)$ & NO & 1453\\
$(292,111)$ & 12 & $(2,1)$ & 1 & 2 & YES & YES & YES & $1.67$ & $(2,3)$ & -- & 1454\\
$(292,121)$ & 12 & $(2,1)$ & 1 & 2 & YES & YES & YES & $1.71$ & $(2,3)$ & -- & 1455\\
$(292,111)$ & 12 & $(3,1)$ & 2 & 1 & YES & YES & YES & $1.89$ & $(2,3)$ & -- & 1456\\
$(292,121)$ & 12 & $(3,1)$ & 2 & 1 & YES & YES & YES & $1.57$ & $(2,3)$ & -- & 1457\\
$(292,121)$ & 12 & $(7,3)$ & 4 & 1 & YES & YES & YES & $1.71$ & $(2,3)$ & NO & 1458\\
$(292,111)$ & 12 & $(13,5)$ & 5 & 1 & YES & YES & YES & $1.67$ & $(2,3)$ & NO & 1459\\
$(292,111)$ & 12 & $(71,27)$ & 9 & 1 & YES & YES & YES & $1.67$ & $(2,3)$ & NO & 1460\\
$(292,121)$ & 12 & $(111,46)$ & 10 & 1 & YES & YES & YES & $1.57$ & $(2,3)$ & NO & 1461\\
$(292,111)$ & 12 & $(121,46)$ & 10 & 1 & YES & YES & YES & $1.78$ & $(2,3)$ & NO & 1462\\
$(292,85)$ & 13 & $(134,39)$ & 11 & 2 & YES & YES & YES & $1.78$ & $(2,3)$ & 1632 & 1463\\
$(292,121)$ & 12 & $(181,75)$ & 11 & 1 & YES & YES & YES & $1.67$ & $(2,3)$ & NO & 1464\\
$(292,85)$ & 13 & $(292,85)$ & 13 & 292 & YES & YES & YES & $1.78$ & $(2,3)$ & NO & 1465\\
$(292,111)$ & 12 & $(292,111)$ & 12 & 292 & YES & YES & YES & $1.78$ & $(2,3)$ & NO & 1466\\
$(293,89)$ & 13 & $(4,1)$ & 3 & 1 & YES & YES & YES & $1.78$ & $(2,3)$ & NO & 1467\\
$(293,89)$ & 13 & $(7,2)$ & 4 & 1 & YES & YES & YES & $1.78$ & $(2,3)$ & NO & 1468\\
$(293,89)$ & 13 & $(33,10)$ & 8 & 1 & YES & YES & YES & $1.78$ & $(2,3)$ & NO & 1469\\
$(293,89)$ & 13 & $(56,17)$ & 9 & 1 & YES & YES & YES & $1.78$ & $(2,3)$ & NO & 1470\\
$(294,67)$ & 13 & $(3,1)$ & 2 & 3 & YES & YES & YES & $1.82$ & $(2,3)$ & NO & 1471\\
$(294,67)$ & 13 & $(48,11)$ & 9 & 6 & YES & YES & YES & $1.57$ & $(2,3)$ & NO & 1472\\
$(297,68)$ & 13 & $(2,1)$ & 1 & 1 & YES & YES & YES & $1.73$ & $(2,3)$ & NO & 1473\\
$(297,83)$ & 13 & $(2,1)$ & 1 & 1 & YES & YES & YES & $1.75$ & $(4,2)$ & -- & 1474\\
$(297,68)$ & 13 & $(3,1)$ & 2 & 3 & YES & YES & YES & $1.73$ & $(2,3)$ & NO & 1475\\
$(297,68)$ & 13 & $(3,1)$ & 2 & 3 & YES & YES & YES & $1.73$ & $(2,3)$ & -- & 1476\\
$(297,68)$ & 13 & $(4,1)$ & 3 & 1 & YES & YES & YES & $1.73$ & $(2,3)$ & NO & 1477\\
$(297,68)$ & 13 & $(5,2)$ & 3 & 1 & YES & YES & YES & $1.78$ & $(2,3)$ & -- & 1478\\
$(297,68)$ & 13 & $(9,2)$ & 5 & 9 & YES & YES & YES & $1.91$ & $(2,3)$ & NO & 1479\\
$(297,68)$ & 13 & $(22,5)$ & 7 & 11 & YES & YES & YES & $1.67$ & $(2,3)$ & NO & 1480\\
$(299,116)$ & 12 & $(8,3)$ & 4 & 1 & YES & YES & YES & $1.67$ & $(2,3)$ & NO & 1481\\
$(300,91)$ & 13 & $(2,1)$ & 1 & 2 & YES & YES & YES & $1.67$ & $(2,3)$ & -- & 1482\\
$(300,91)$ & 13 & $(5,1)$ & 4 & 5 & YES & YES & YES & $1.67$ & $(2,3)$ & NO & 1483\\
$(300,91)$ & 13 & $(7,2)$ & 4 & 1 & YES & YES & YES & $1.67$ & $(2,3)$ & NO & 1484\\
$(300,91)$ & 13 & $(23,7)$ & 7 & 1 & YES & YES & YES & $1.89$ & $(2,3)$ & NO & 1485\\
$(300,91)$ & 13 & $(56,17)$ & 9 & 4 & YES & YES & YES & $1.67$ & $(2,3)$ & NO & 1486\\
$(300,89)$ & 13 & $(118,35)$ & 11 & 2 & YES & YES & YES & $1.67$ & $(2,3)$ & 1580 & 1487\\
$(301,88)$ & 13 & $(2,1)$ & 1 & 1 & YES & YES & YES & $1.89$ & $(2,3)$ & NO & 1488\\
$(301,89)$ & 12 & $(2,1)$ & 1 & 1 & YES & YES & YES & $1.44$ & $(2,3)$ & -- & 1489\\
$(301,89)$ & 12 & $(3,1)$ & 2 & 1 & YES & YES & YES & $1.56$ & $(2,3)$ & -- & 1490\\
$(301,115)$ & 12 & $(3,1)$ & 2 & 1 & YES & YES & YES & $1.67$ & $(2,3)$ & -- & 1491\\
$(301,88)$ & 13 & $(4,1)$ & 3 & 1 & YES & YES & YES & $1.78$ & $(2,3)$ & NO & 1492\\
$(301,89)$ & 12 & $(4,1)$ & 3 & 1 & YES & YES & YES & $1.67$ & $(2,3)$ & NO & 1493\\
$(301,88)$ & 13 & $(5,1)$ & 4 & 1 & YES & YES & YES & $1.78$ & $(2,3)$ & NO & 1494\\
$(301,89)$ & 12 & $(7,2)$ & 4 & 7 & YES & YES & YES & $1.67$ & $(2,3)$ & NO & 1495\\
$(301,89)$ & 12 & $(27,8)$ & 7 & 1 & YES & YES & YES & $1.67$ & $(2,3)$ & 1332 & 1496\\
$(301,89)$ & 12 & $(44,13)$ & 8 & 1 & YES & YES & YES & $1.67$ & $(2,3)$ & NO & 1497\\
$(301,89)$ & 12 & $(71,21)$ & 9 & 1 & YES & YES & YES & $1.56$ & $(2,3)$ & 1288 & 1498\\
$(301,89)$ & 12 & $(115,34)$ & 10 & 1 & YES & YES & YES & $1.67$ & $(2,3)$ & NO & 1499\\
$(302,117)$ & 12 & $(13,5)$ & 5 & 1 & YES & YES & YES & $1.86$ & $(2,3)$ & NO & 1500\\
$(302,117)$ & 12 & $(111,43)$ & 10 & 1 & YES & YES & YES & $1.71$ & $(2,3)$ & NO & 1501\\
$(303,85)$ & 13 & $(2,1)$ & 1 & 1 & YES & YES & YES & $1.62$ & $(4,2)$ & -- & 1502\\
$(305,84)$ & 13 & $(2,1)$ & 1 & 1 & YES & YES & YES & $1.75$ & $(4,2)$ & -- & 1503\\
$(305,69)$ & 13 & $(3,1)$ & 2 & 1 & YES & YES & YES & $1.67$ & $(2,3)$ & NO & 1504\\
$(305,69)$ & 13 & $(3,1)$ & 2 & 1 & YES & YES & YES & $1.67$ & $(2,3)$ & -- & 1505\\
$(305,112)$ & 12 & $(4,1)$ & 3 & 1 & YES & YES & YES & $1.89$ & $(2,3)$ & -- & 1506\\
$(305,69)$ & 13 & $(5,2)$ & 3 & 5 & YES & YES & YES & $1.43$ & $(2,3)$ & -- & 1507\\
$(305,69)$ & 13 & $(5,2)$ & 3 & 5 & YES & YES & YES & $1.57$ & $(2,3)$ & NO & 1508\\
$(305,112)$ & 12 & $(30,11)$ & 7 & 5 & YES & YES & YES & $1.67$ & $(2,3)$ & 872 & 1509\\
$(305,69)$ & 13 & $(35,8)$ & 8 & 5 & YES & YES & YES & $1.57$ & $(2,3)$ & NO & 1510\\
$(305,69)$ & 13 & $(75,17)$ & 10 & 5 & YES & YES & YES & $1.43$ & $(2,3)$ & 1738 & 1511\\
$(305,84)$ & 13 & $(305,84)$ & 13 & 305 & YES & YES & YES & $1.67$ & $(2,3)$ & NO & 1512\\
$(307,129)$ & 12 & $(2,1)$ & 1 & 1 & YES & YES & YES & $1.78$ & $(2,3)$ & -- & 1513\\
$(307,119)$ & 12 & $(3,1)$ & 2 & 1 & YES & YES & YES & $1.57$ & $(2,3)$ & -- & 1514\\
$(307,129)$ & 12 & $(3,1)$ & 2 & 1 & YES & YES & YES & $1.57$ & $(2,3)$ & NO & 1515\\
$(307,119)$ & 12 & $(49,19)$ & 8 & 1 & YES & YES & YES & $1.78$ & $(2,3)$ & 1100 & 1516\\
$(307,129)$ & 12 & $(50,21)$ & 8 & 1 & YES & YES & YES & $1.50$ & $(4,2)$ & NO & 1517\\
$(311,71)$ & 13 & $(3,1)$ & 2 & 1 & YES & YES & YES & $1.67$ & $(2,3)$ & -- & 1518\\
$(311,119)$ & 12 & $(3,1)$ & 2 & 1 & YES & YES & YES & $1.57$ & $(2,3)$ & NO & 1519\\
$(311,119)$ & 12 & $(3,1)$ & 2 & 1 & YES & YES & YES & $1.57$ & $(2,3)$ & -- & 1520\\
$(311,71)$ & 13 & $(5,2)$ & 3 & 1 & YES & YES & YES & $1.43$ & $(2,3)$ & -- & 1521\\
$(311,71)$ & 13 & $(31,7)$ & 8 & 1 & YES & YES & YES & $1.57$ & $(2,3)$ & NO & 1522\\
$(311,119)$ & 12 & $(47,18)$ & 8 & 1 & YES & YES & YES & $1.57$ & $(2,3)$ & 1601 & 1523\\
$(311,71)$ & 13 & $(57,13)$ & 9 & 1 & YES & YES & YES & $1.67$ & $(2,3)$ & NO & 1524\\
$(311,119)$ & 12 & $(81,31)$ & 9 & 1 & YES & YES & YES & $1.86$ & $(2,3)$ & 1396 & 1525\\
$(313,119)$ & 12 & $(2,1)$ & 1 & 1 & YES & YES & YES & $1.89$ & $(2,3)$ & NO & 1526\\
$(313,121)$ & 12 & $(3,1)$ & 2 & 1 & YES & YES & YES & $1.57$ & $(2,3)$ & -- & 1527\\
$(313,121)$ & 12 & $(8,3)$ & 4 & 1 & YES & YES & YES & $2.00$ & $(2,3)$ & NO & 1528\\
$(313,121)$ & 12 & $(13,5)$ & 5 & 1 & YES & YES & YES & $1.62$ & $(4,2)$ & NO & 1529\\
$(313,121)$ & 12 & $(44,17)$ & 8 & 1 & YES & YES & YES & $1.71$ & $(2,3)$ & NO & 1530\\
$(315,71)$ & 14 & $(3,1)$ & 2 & 3 & YES & YES & YES & $1.50$ & $(4,2)$ & NO & 1531\\
$(315,88)$ & 13 & $(5,1)$ & 4 & 5 & YES & YES & YES & $1.57$ & $(2,3)$ & NO & 1532\\
$(315,88)$ & 13 & $(18,5)$ & 6 & 9 & YES & YES & YES & $1.67$ & $(2,3)$ & NO & 1533\\
$(315,88)$ & 13 & $(247,69)$ & 12 & 1 & YES & YES & YES & $1.71$ & $(2,3)$ & NO & 1534\\
$(316,69)$ & 13 & $(2,1)$ & 1 & 2 & YES & YES & YES & $1.38$ & $(4,2)$ & NO & 1535\\
$(316,69)$ & 13 & $(3,1)$ & 2 & 1 & YES & YES & YES & $1.78$ & $(2,3)$ & NO & 1536\\
$(316,69)$ & 13 & $(4,1)$ & 3 & 4 & YES & YES & YES & $1.38$ & $(4,2)$ & NO & 1537\\
$(316,69)$ & 13 & $(5,2)$ & 3 & 1 & YES & YES & YES & $1.78$ & $(2,3)$ & -- & 1538\\
$(317,121)$ & 12 & $(2,1)$ & 1 & 1 & YES & YES & YES & $1.57$ & $(2,3)$ & -- & 1539\\
$(317,131)$ & 12 & $(2,1)$ & 1 & 1 & YES & YES & YES & $1.56$ & $(2,3)$ & -- & 1540\\
$(317,121)$ & 12 & $(4,1)$ & 3 & 1 & YES & YES & YES & $1.78$ & $(2,3)$ & -- & 1541\\
$(317,121)$ & 12 & $(4,1)$ & 3 & 1 & YES & YES & YES & $1.71$ & $(2,3)$ & NO & 1542\\
$(317,131)$ & 12 & $(4,1)$ & 3 & 1 & YES & YES & YES & $1.56$ & $(2,3)$ & -- & 1543\\
$(317,121)$ & 12 & $(13,5)$ & 5 & 1 & YES & YES & YES & $1.57$ & $(2,3)$ & NO & 1544\\
$(317,121)$ & 12 & $(21,8)$ & 6 & 1 & YES & YES & YES & $1.67$ & $(2,3)$ & 922 & 1545\\
$(317,121)$ & 12 & $(34,13)$ & 7 & 1 & YES & YES & YES & $1.71$ & $(2,3)$ & NO & 1546\\
$(317,131)$ & 12 & $(75,31)$ & 9 & 1 & YES & YES & YES & $1.43$ & $(2,3)$ & 1354 & 1547\\
$(317,121)$ & 12 & $(131,50)$ & 10 & 1 & YES & YES & YES & $1.71$ & $(2,3)$ & NO & 1548\\
$(317,121)$ & 12 & $(317,121)$ & 12 & 317 & YES & YES & YES & $1.78$ & $(2,3)$ & NO & 1549\\
$(317,131)$ & 12 & $(317,131)$ & 12 & 317 & YES & YES & YES & $1.43$ & $(2,3)$ & NO & 1550\\
$(319,73)$ & 13 & $(2,1)$ & 1 & 1 & YES & YES & YES & $1.62$ & $(4,2)$ & NO & 1551\\
$(319,73)$ & 13 & $(5,2)$ & 3 & 1 & YES & YES & YES & $1.71$ & $(2,3)$ & NO & 1552\\
$(321,94)$ & 13 & $(2,1)$ & 1 & 1 & YES & YES & YES & $1.67$ & $(2,3)$ & -- & 1553\\
$(321,95)$ & 13 & $(2,1)$ & 1 & 1 & YES & YES & YES & $1.78$ & $(2,3)$ & -- & 1554\\
$(321,95)$ & 13 & $(4,1)$ & 3 & 1 & YES & YES & YES & $1.89$ & $(2,3)$ & NO & 1555\\
$(321,95)$ & 13 & $(5,1)$ & 4 & 1 & YES & YES & YES & $1.50$ & $(4,2)$ & -- & 1556\\
$(321,95)$ & 13 & $(17,5)$ & 6 & 1 & YES & YES & YES & $1.67$ & $(2,3)$ & 717 & 1557\\
$(321,94)$ & 13 & $(99,29)$ & 10 & 3 & YES & YES & YES & $1.67$ & $(2,3)$ & NO & 1558\\
$(322,73)$ & 14 & $(2,1)$ & 1 & 2 & YES & YES & YES & $1.62$ & $(4,2)$ & NO & 1559\\
$(322,89)$ & 12 & $(2,1)$ & 1 & 2 & YES & YES & YES & $1.56$ & $(2,3)$ & NO & 1560\\
$(322,89)$ & 12 & $(2,1)$ & 1 & 2 & YES & YES & YES & $1.56$ & $(2,3)$ & -- & 1561\\
$(322,89)$ & 12 & $(3,1)$ & 2 & 1 & YES & YES & YES & $1.44$ & $(2,3)$ & -- & 1562\\
$(322,89)$ & 12 & $(7,2)$ & 4 & 7 & YES & YES & YES & $1.56$ & $(2,3)$ & NO & 1563\\
$(322,73)$ & 14 & $(31,7)$ & 8 & 1 & YES & YES & YES & $1.62$ & $(4,2)$ & NO & 1564\\
$(322,89)$ & 12 & $(47,13)$ & 8 & 1 & YES & YES & YES & $1.56$ & $(2,3)$ & NO & 1565\\
$(323,94)$ & 13 & $(3,1)$ & 2 & 1 & YES & YES & YES & $1.78$ & $(2,3)$ & NO & 1566\\
$(323,94)$ & 13 & $(3,1)$ & 2 & 1 & YES & YES & YES & $1.78$ & $(2,3)$ & -- & 1567\\
$(323,94)$ & 13 & $(79,23)$ & 10 & 1 & YES & YES & YES & $1.89$ & $(2,3)$ & NO & 1568\\
$(323,89)$ & 13 & $(98,27)$ & 10 & 1 & YES & YES & YES & $1.75$ & $(4,2)$ & NO & 1569\\
$(324,95)$ & 13 & $(3,1)$ & 2 & 3 & YES & YES & YES & $1.75$ & $(4,2)$ & NO & 1570\\
$(324,95)$ & 13 & $(24,7)$ & 7 & 12 & YES & YES & YES & $1.43$ & $(2,3)$ & NO & 1571\\
$(324,95)$ & 13 & $(41,12)$ & 8 & 1 & YES & YES & YES & $1.43$ & $(2,3)$ & NO & 1572\\
$(326,99)$ & 13 & $(191,58)$ & 12 & 1 & YES & YES & YES & $2.00$ & $(2,3)$ & NO & 1573\\
$(327,97)$ & 13 & $(2,1)$ & 1 & 1 & YES & YES & YES & $1.67$ & $(2,3)$ & -- & 1574\\
$(327,97)$ & 13 & $(2,1)$ & 1 & 1 & YES & YES & YES & $1.78$ & $(2,3)$ & NO & 1575\\
$(327,97)$ & 13 & $(3,1)$ & 2 & 3 & YES & YES & YES & $1.57$ & $(2,3)$ & -- & 1576\\
$(327,97)$ & 13 & $(3,1)$ & 2 & 3 & YES & YES & YES & $1.86$ & $(2,3)$ & NO & 1577\\
$(327,97)$ & 13 & $(4,1)$ & 3 & 1 & YES & YES & YES & $1.67$ & $(2,3)$ & -- & 1578\\
$(327,97)$ & 13 & $(7,2)$ & 4 & 1 & YES & YES & YES & $1.89$ & $(2,3)$ & NO & 1579\\
$(327,97)$ & 13 & $(91,27)$ & 10 & 1 & YES & YES & YES & $1.67$ & $(2,3)$ & 1487 & 1580\\
$(327,97)$ & 13 & $(327,97)$ & 13 & 327 & YES & YES & YES & $1.57$ & $(2,3)$ & NO & 1581\\
$(333,92)$ & 13 & $(2,1)$ & 1 & 1 & YES & YES & YES & $1.62$ & $(4,2)$ & -- & 1582\\
$(333,101)$ & 13 & $(2,1)$ & 1 & 1 & YES & YES & YES & $1.67$ & $(2,3)$ & -- & 1583\\
$(333,101)$ & 13 & $(3,1)$ & 2 & 3 & YES & YES & YES & $1.57$ & $(2,3)$ & -- & 1584\\
$(333,101)$ & 13 & $(3,1)$ & 2 & 3 & YES & YES & YES & $1.86$ & $(2,3)$ & NO & 1585\\
$(333,76)$ & 13 & $(5,1)$ & 4 & 1 & YES & YES & YES & $1.67$ & $(2,3)$ & NO & 1586\\
$(333,76)$ & 13 & $(9,2)$ & 5 & 9 & YES & YES & YES & $1.56$ & $(2,3)$ & NO & 1587\\
$(333,76)$ & 13 & $(22,5)$ & 7 & 1 & YES & YES & YES & $1.56$ & $(2,3)$ & NO & 1588\\
$(333,76)$ & 13 & $(35,8)$ & 8 & 1 & YES & YES & YES & $1.67$ & $(2,3)$ & NO & 1589\\
$(333,76)$ & 13 & $(48,11)$ & 9 & 3 & YES & YES & YES & $1.71$ & $(2,3)$ & NO & 1590\\
$(333,101)$ & 13 & $(56,17)$ & 9 & 1 & YES & YES & YES & $1.67$ & $(2,3)$ & 1684 & 1591\\
$(333,76)$ & 13 & $(57,13)$ & 9 & 3 & YES & YES & YES & $1.67$ & $(2,3)$ & NO & 1592\\
$(333,101)$ & 13 & $(122,37)$ & 11 & 1 & YES & YES & YES & $2.00$ & $(2,3)$ & NO & 1593\\
$(335,73)$ & 14 & $(2,1)$ & 1 & 1 & YES & YES & YES & $1.62$ & $(4,2)$ & NO & 1594\\
$(335,73)$ & 14 & $(3,1)$ & 2 & 1 & YES & YES & YES & $1.50$ & $(4,2)$ & -- & 1595\\
$(335,123)$ & 12 & $(3,1)$ & 2 & 1 & YES & YES & YES & $1.43$ & $(2,3)$ & -- & 1596\\
$(335,73)$ & 14 & $(32,7)$ & 8 & 1 & YES & YES & YES & $1.75$ & $(4,2)$ & NO & 1597\\
$(337,129)$ & 12 & $(2,1)$ & 1 & 1 & YES & YES & YES & $1.57$ & $(2,3)$ & NO & 1598\\
$(337,91)$ & 13 & $(3,1)$ & 2 & 1 & YES & YES & YES & $1.89$ & $(2,3)$ & -- & 1599\\
$(337,91)$ & 13 & $(5,1)$ & 4 & 1 & YES & YES & YES & $1.78$ & $(2,3)$ & NO & 1600\\
$(337,129)$ & 12 & $(34,13)$ & 7 & 1 & YES & YES & YES & $1.57$ & $(2,3)$ & 1523 & 1601\\
$(338,129)$ & 12 & $(2,1)$ & 1 & 2 & YES & YES & YES & $1.56$ & $(2,3)$ & -- & 1602\\
$(338,77)$ & 14 & $(3,1)$ & 2 & 1 & YES & YES & YES & $1.50$ & $(4,2)$ & NO & 1603\\
$(338,77)$ & 14 & $(3,1)$ & 2 & 1 & YES & YES & YES & $1.50$ & $(4,2)$ & -- & 1604\\
$(338,99)$ & 13 & $(3,1)$ & 2 & 1 & YES & YES & YES & $1.43$ & $(2,3)$ & -- & 1605\\
$(338,129)$ & 12 & $(3,1)$ & 2 & 1 & YES & YES & YES & $1.43$ & $(2,3)$ & -- & 1606\\
$(338,129)$ & 12 & $(3,1)$ & 2 & 1 & YES & YES & YES & $1.57$ & $(2,3)$ & NO & 1607\\
$(338,77)$ & 14 & $(5,1)$ & 4 & 1 & YES & YES & YES & $1.62$ & $(4,2)$ & NO & 1608\\
$(338,99)$ & 13 & $(7,2)$ & 4 & 1 & YES & YES & YES & $1.78$ & $(2,3)$ & NO & 1609\\
$(338,99)$ & 13 & $(10,3)$ & 5 & 2 & YES & YES & YES & $1.57$ & $(2,3)$ & NO & 1610\\
$(338,77)$ & 14 & $(13,3)$ & 6 & 13 & YES & YES & YES & $1.62$ & $(4,2)$ & NO & 1611\\
$(338,99)$ & 13 & $(58,17)$ & 9 & 2 & YES & YES & YES & $1.57$ & $(2,3)$ & NO & 1612\\
$(338,129)$ & 12 & $(76,29)$ & 9 & 2 & YES & YES & YES & $1.78$ & $(2,3)$ & 1415 & 1613\\
$(338,129)$ & 12 & $(207,79)$ & 11 & 1 & YES & YES & YES & $1.56$ & $(2,3)$ & NO & 1614\\
$(339,100)$ & 13 & $(2,1)$ & 1 & 1 & YES & YES & YES & $1.50$ & $(4,2)$ & -- & 1615\\
$(339,100)$ & 13 & $(3,1)$ & 2 & 3 & YES & YES & YES & $1.43$ & $(2,3)$ & -- & 1616\\
$(339,100)$ & 13 & $(27,8)$ & 7 & 3 & YES & YES & YES & $1.43$ & $(2,3)$ & NO & 1617\\
$(339,100)$ & 13 & $(139,41)$ & 11 & 1 & YES & YES & YES & $1.78$ & $(2,3)$ & NO & 1618\\
$(341,100)$ & 13 & $(2,1)$ & 1 & 1 & YES & YES & YES & $1.67$ & $(2,3)$ & -- & 1619\\
$(341,100)$ & 13 & $(2,1)$ & 1 & 1 & YES & YES & YES & $1.78$ & $(2,3)$ & NO & 1620\\
$(341,100)$ & 13 & $(3,1)$ & 2 & 1 & YES & YES & YES & $1.57$ & $(2,3)$ & -- & 1621\\
$(341,100)$ & 13 & $(41,12)$ & 8 & 1 & YES & YES & YES & $1.67$ & $(2,3)$ & NO & 1622\\
$(344,95)$ & 13 & $(2,1)$ & 1 & 2 & YES & YES & YES & $1.78$ & $(2,3)$ & NO & 1623\\
$(344,95)$ & 13 & $(5,1)$ & 4 & 1 & YES & YES & YES & $1.62$ & $(4,2)$ & -- & 1624\\
$(344,95)$ & 13 & $(18,5)$ & 6 & 2 & YES & YES & YES & $1.78$ & $(2,3)$ & 748 & 1625\\
$(344,95)$ & 13 & $(105,29)$ & 10 & 1 & YES & YES & YES & $1.62$ & $(4,2)$ & NO & 1626\\
$(347,101)$ & 13 & $(2,1)$ & 1 & 1 & YES & YES & YES & $1.67$ & $(2,3)$ & -- & 1627\\
$(347,101)$ & 13 & $(2,1)$ & 1 & 1 & YES & YES & YES & $1.78$ & $(2,3)$ & NO & 1628\\
$(347,101)$ & 13 & $(3,1)$ & 2 & 1 & YES & YES & YES & $1.57$ & $(2,3)$ & -- & 1629\\
$(347,101)$ & 13 & $(3,1)$ & 2 & 1 & YES & YES & YES & $1.71$ & $(2,3)$ & NO & 1630\\
$(347,101)$ & 13 & $(55,16)$ & 9 & 1 & YES & YES & YES & $1.78$ & $(2,3)$ & NO & 1631\\
$(347,101)$ & 13 & $(79,23)$ & 10 & 1 & YES & YES & YES & $1.78$ & $(2,3)$ & 1463 & 1632\\
$(347,101)$ & 13 & $(213,62)$ & 12 & 1 & YES & YES & YES & $1.67$ & $(2,3)$ & NO & 1633\\
$(351,76)$ & 13 & $(3,1)$ & 2 & 3 & YES & YES & YES & $1.56$ & $(2,3)$ & -- & 1634\\
$(351,76)$ & 13 & $(14,3)$ & 6 & 1 & YES & YES & NO(2) & $1.82$ & $(4,2)$ & NO & 1635\\
$(351,76)$ & 13 & $(37,8)$ & 8 & 1 & YES & YES & NO(2) & $1.82$ & $(4,2)$ & NO & 1636\\
$(353,75)$ & 14 & $(3,1)$ & 2 & 1 & YES & YES & YES & $1.62$ & $(4,2)$ & -- & 1637\\
$(353,75)$ & 14 & $(3,1)$ & 2 & 1 & YES & YES & YES & $1.75$ & $(4,2)$ & NO & 1638\\
$(353,75)$ & 14 & $(9,2)$ & 5 & 1 & YES & YES & YES & $1.75$ & $(4,2)$ & NO & 1639\\
$(355,99)$ & 13 & $(2,1)$ & 1 & 1 & YES & YES & YES & $2.00$ & $(2,3)$ & NO & 1640\\
$(355,77)$ & 14 & $(3,1)$ & 2 & 1 & YES & YES & YES & $1.50$ & $(4,2)$ & NO & 1641\\
$(355,99)$ & 13 & $(3,1)$ & 2 & 1 & YES & YES & YES & $1.43$ & $(2,3)$ & -- & 1642\\
$(355,99)$ & 13 & $(5,1)$ & 4 & 5 & YES & YES & YES & $1.57$ & $(2,3)$ & NO & 1643\\
$(355,99)$ & 13 & $(11,3)$ & 5 & 1 & YES & YES & YES & $1.43$ & $(2,3)$ & NO & 1644\\
$(355,99)$ & 13 & $(25,7)$ & 7 & 5 & YES & YES & YES & $1.43$ & $(2,3)$ & 1232 & 1645\\
$(355,99)$ & 13 & $(61,17)$ & 9 & 1 & YES & YES & YES & $1.43$ & $(2,3)$ & NO & 1646\\
$(355,99)$ & 13 & $(355,99)$ & 13 & 355 & YES & YES & YES & $1.57$ & $(2,3)$ & NO & 1647\\
$(359,100)$ & 13 & $(2,1)$ & 1 & 1 & YES & YES & YES & $1.50$ & $(4,2)$ & -- & 1648\\
$(359,100)$ & 13 & $(3,1)$ & 2 & 1 & YES & YES & YES & $1.89$ & $(2,3)$ & NO & 1649\\
$(359,106)$ & 13 & $(3,1)$ & 2 & 1 & YES & YES & YES & $1.43$ & $(2,3)$ & -- & 1650\\
$(359,105)$ & 13 & $(4,1)$ & 3 & 1 & YES & YES & YES & $1.71$ & $(2,3)$ & NO & 1651\\
$(359,106)$ & 13 & $(7,2)$ & 4 & 1 & YES & YES & YES & $1.71$ & $(2,3)$ & NO & 1652\\
$(359,105)$ & 13 & $(24,7)$ & 7 & 1 & YES & YES & YES & $1.86$ & $(2,3)$ & NO & 1653\\
$(360,101)$ & 13 & $(2,1)$ & 1 & 2 & YES & YES & YES & $1.50$ & $(4,2)$ & -- & 1654\\
$(363,100)$ & 13 & $(2,1)$ & 1 & 1 & YES & YES & YES & $1.43$ & $(2,3)$ & -- & 1655\\
$(363,98)$ & 13 & $(3,1)$ & 2 & 3 & YES & YES & YES & $1.43$ & $(2,3)$ & -- & 1656\\
$(363,100)$ & 13 & $(3,1)$ & 2 & 3 & YES & YES & YES & $1.71$ & $(2,3)$ & -- & 1657\\
$(363,100)$ & 13 & $(3,1)$ & 2 & 3 & YES & YES & YES & $1.86$ & $(2,3)$ & NO & 1658\\
$(363,100)$ & 13 & $(18,5)$ & 6 & 3 & YES & YES & YES & $1.71$ & $(2,3)$ & NO & 1659\\
$(363,98)$ & 13 & $(26,7)$ & 7 & 1 & YES & YES & YES & $1.86$ & $(2,3)$ & NO & 1660\\
$(363,98)$ & 13 & $(63,17)$ & 9 & 3 & YES & YES & YES & $1.86$ & $(2,3)$ & NO & 1661\\
$(363,100)$ & 13 & $(69,19)$ & 9 & 3 & YES & YES & YES & $1.43$ & $(2,3)$ & NO & 1662\\
$(363,100)$ & 13 & $(363,100)$ & 13 & 363 & YES & YES & YES & $1.71$ & $(2,3)$ & NO & 1663\\
$(365,108)$ & 13 & $(2,1)$ & 1 & 1 & YES & YES & YES & $1.86$ & $(2,3)$ & NO & 1664\\
$(365,98)$ & 13 & $(3,1)$ & 2 & 1 & YES & YES & YES & $1.89$ & $(2,3)$ & NO & 1665\\
$(367,83)$ & 14 & $(2,1)$ & 1 & 1 & YES & YES & YES & $1.62$ & $(4,2)$ & -- & 1666\\
$(367,101)$ & 13 & $(2,1)$ & 1 & 1 & YES & YES & YES & $1.78$ & $(2,3)$ & -- & 1667\\
$(367,84)$ & 14 & $(118,27)$ & 11 & 1 & YES & YES & YES & $1.67$ & $(2,3)$ & NO & 1668\\
$(372,109)$ & 13 & $(3,1)$ & 2 & 3 & YES & YES & YES & $1.57$ & $(2,3)$ & -- & 1669\\
$(372,109)$ & 13 & $(99,29)$ & 10 & 3 & YES & YES & YES & $1.57$ & $(2,3)$ & NO & 1670\\
$(373,104)$ & 13 & $(2,1)$ & 1 & 1 & YES & YES & YES & $1.57$ & $(2,3)$ & NO & 1671\\
$(373,109)$ & 13 & $(2,1)$ & 1 & 1 & YES & YES & YES & $1.43$ & $(2,3)$ & -- & 1672\\
$(373,104)$ & 13 & $(3,1)$ & 2 & 1 & YES & YES & YES & $1.57$ & $(2,3)$ & -- & 1673\\
$(373,109)$ & 13 & $(3,1)$ & 2 & 1 & YES & YES & YES & $1.71$ & $(2,3)$ & -- & 1674\\
$(373,104)$ & 13 & $(4,1)$ & 3 & 1 & YES & YES & YES & $1.57$ & $(2,3)$ & NO & 1675\\
$(373,109)$ & 13 & $(17,5)$ & 6 & 1 & YES & YES & YES & $1.43$ & $(2,3)$ & NO & 1676\\
$(374,101)$ & 13 & $(2,1)$ & 1 & 2 & YES & YES & YES & $1.78$ & $(2,3)$ & -- & 1677\\
$(374,111)$ & 13 & $(219,65)$ & 12 & 1 & YES & YES & YES & $1.57$ & $(2,3)$ & NO & 1678\\
$(376,105)$ & 13 & $(2,1)$ & 1 & 2 & YES & YES & YES & $1.57$ & $(2,3)$ & NO & 1679\\
$(376,105)$ & 13 & $(111,31)$ & 10 & 1 & YES & YES & YES & $1.57$ & $(2,3)$ & NO & 1680\\
$(379,111)$ & 13 & $(4,1)$ & 3 & 1 & YES & YES & YES & $1.57$ & $(2,3)$ & NO & 1681\\
$(379,111)$ & 13 & $(5,1)$ & 4 & 1 & YES & YES & YES & $1.57$ & $(2,3)$ & NO & 1682\\
$(379,111)$ & 13 & $(7,2)$ & 4 & 1 & YES & YES & YES & $1.67$ & $(2,3)$ & NO & 1683\\
$(379,115)$ & 13 & $(33,10)$ & 8 & 1 & YES & YES & YES & $1.67$ & $(2,3)$ & 1591 & 1684\\
$(379,111)$ & 13 & $(41,12)$ & 8 & 1 & YES & YES & YES & $1.67$ & $(2,3)$ & NO & 1685\\
$(379,111)$ & 13 & $(140,41)$ & 11 & 1 & YES & YES & YES & $1.57$ & $(2,3)$ & NO & 1686\\
$(379,111)$ & 13 & $(379,111)$ & 13 & 379 & YES & YES & YES & $1.57$ & $(2,3)$ & NO & 1687\\
$(380,83)$ & 14 & $(4,1)$ & 3 & 4 & YES & YES & YES & $1.62$ & $(4,2)$ & -- & 1688\\
$(382,89)$ & 14 & $(2,1)$ & 1 & 2 & YES & YES & YES & $1.62$ & $(4,2)$ & NO & 1689\\
$(382,89)$ & 14 & $(2,1)$ & 1 & 2 & YES & YES & YES & $1.62$ & $(4,2)$ & -- & 1690\\
$(383,106)$ & 13 & $(3,1)$ & 2 & 1 & YES & YES & YES & $1.43$ & $(2,3)$ & NO & 1691\\
$(389,84)$ & 14 & $(51,11)$ & 9 & 1 & YES & YES & YES & $2.00$ & $(2,3)$ & NO & 1692\\
$(389,84)$ & 14 & $(389,84)$ & 14 & 389 & YES & YES & YES & $2.00$ & $(2,3)$ & NO & 1693\\
$(390,89)$ & 13 & $(2,1)$ & 1 & 2 & YES & YES & YES & $1.56$ & $(2,3)$ & NO & 1694\\
$(391,105)$ & 13 & $(3,1)$ & 2 & 1 & YES & YES & YES & $1.57$ & $(2,3)$ & -- & 1695\\
$(391,105)$ & 13 & $(3,1)$ & 2 & 1 & YES & YES & YES & $1.71$ & $(2,3)$ & NO & 1696\\
$(391,108)$ & 13 & $(3,1)$ & 2 & 1 & YES & YES & YES & $1.86$ & $(2,3)$ & NO & 1697\\
$(391,108)$ & 13 & $(3,1)$ & 2 & 1 & YES & YES & YES & $1.86$ & $(2,3)$ & -- & 1698\\
$(391,105)$ & 13 & $(4,1)$ & 3 & 1 & YES & YES & YES & $1.56$ & $(2,3)$ & -- & 1699\\
$(391,109)$ & 13 & $(11,3)$ & 5 & 1 & YES & YES & YES & $1.78$ & $(2,3)$ & NO & 1700\\
$(393,116)$ & 13 & $(2,1)$ & 1 & 1 & YES & YES & YES & $1.71$ & $(2,3)$ & NO & 1701\\
$(393,106)$ & 13 & $(3,1)$ & 2 & 3 & YES & YES & YES & $1.67$ & $(2,3)$ & NO & 1702\\
$(393,106)$ & 13 & $(3,1)$ & 2 & 3 & YES & YES & YES & $1.57$ & $(2,3)$ & NO & 1703\\
$(393,106)$ & 13 & $(3,1)$ & 2 & 3 & YES & YES & YES & $1.57$ & $(2,3)$ & -- & 1704\\
$(393,106)$ & 13 & $(7,2)$ & 4 & 1 & YES & YES & YES & $1.67$ & $(2,3)$ & NO & 1705\\
$(393,116)$ & 13 & $(10,3)$ & 5 & 1 & YES & YES & YES & $1.71$ & $(2,3)$ & NO & 1706\\
$(393,116)$ & 13 & $(227,67)$ & 12 & 1 & YES & YES & YES & $1.71$ & $(2,3)$ & NO & 1707\\
$(394,165)$ & 13 & $(2,1)$ & 1 & 2 & NO & YES & YES & $1.89$ & $(2,3)$ & -- & 1708\\
$(397,116)$ & 13 & $(3,1)$ & 2 & 1 & YES & YES & YES & $1.71$ & $(2,3)$ & -- & 1709\\
$(398,111)$ & 13 & $(2,1)$ & 1 & 2 & YES & YES & YES & $1.43$ & $(2,3)$ & -- & 1710\\
$(398,111)$ & 13 & $(2,1)$ & 1 & 2 & YES & YES & YES & $1.57$ & $(2,3)$ & NO & 1711\\
$(398,111)$ & 13 & $(3,1)$ & 2 & 1 & YES & YES & YES & $1.71$ & $(2,3)$ & -- & 1712\\
$(401,112)$ & 13 & $(2,1)$ & 1 & 1 & YES & YES & YES & $1.56$ & $(2,3)$ & NO & 1713\\
$(403,92)$ & 14 & $(5,1)$ & 4 & 1 & YES & YES & YES & $1.57$ & $(2,3)$ & NO & 1714\\
$(403,87)$ & 14 & $(23,5)$ & 7 & 1 & YES & YES & YES & $1.57$ & $(2,3)$ & NO & 1715\\
$(403,87)$ & 14 & $(51,11)$ & 9 & 1 & YES & YES & YES & $2.00$ & $(2,3)$ & NO & 1716\\
$(407,119)$ & 13 & $(3,1)$ & 2 & 1 & YES & YES & YES & $1.57$ & $(2,3)$ & -- & 1717\\
$(407,119)$ & 13 & $(4,1)$ & 3 & 1 & YES & YES & YES & $1.57$ & $(2,3)$ & -- & 1718\\
$(407,119)$ & 13 & $(41,12)$ & 8 & 1 & YES & YES & YES & $1.57$ & $(2,3)$ & 1148 & 1719\\
$(409,121)$ & 13 & $(2,1)$ & 1 & 1 & YES & YES & YES & $1.57$ & $(2,3)$ & -- & 1720\\
$(413,121)$ & 13 & $(3,1)$ & 2 & 1 & YES & YES & YES & $1.57$ & $(2,3)$ & -- & 1721\\
$(413,121)$ & 13 & $(157,46)$ & 11 & 1 & YES & YES & YES & $1.57$ & $(2,3)$ & NO & 1722\\
$(421,98)$ & 14 & $(30,7)$ & 8 & 1 & YES & YES & YES & $1.43$ & $(2,3)$ & NO & 1723\\
$(425,92)$ & 14 & $(2,1)$ & 1 & 1 & YES & YES & YES & $1.57$ & $(2,3)$ & NO & 1724\\
$(425,92)$ & 14 & $(2,1)$ & 1 & 1 & YES & YES & YES & $1.57$ & $(2,3)$ & -- & 1725\\
$(425,97)$ & 14 & $(3,1)$ & 2 & 1 & YES & YES & YES & $1.78$ & $(2,3)$ & -- & 1726\\
$(425,92)$ & 14 & $(4,1)$ & 3 & 1 & YES & YES & YES & $1.43$ & $(2,3)$ & NO & 1727\\
$(425,92)$ & 14 & $(9,2)$ & 5 & 1 & YES & YES & YES & $1.86$ & $(2,3)$ & NO & 1728\\
$(425,92)$ & 14 & $(23,5)$ & 7 & 1 & YES & YES & YES & $1.43$ & $(2,3)$ & NO & 1729\\
$(426,115)$ & 13 & $(3,1)$ & 2 & 3 & YES & YES & YES & $1.56$ & $(2,3)$ & -- & 1730\\
$(434,101)$ & 14 & $(3,1)$ & 2 & 1 & YES & YES & YES & $1.78$ & $(2,3)$ & NO & 1731\\
$(434,101)$ & 14 & $(3,1)$ & 2 & 1 & YES & YES & YES & $1.78$ & $(2,3)$ & -- & 1732\\
$(434,101)$ & 14 & $(17,4)$ & 7 & 1 & YES & YES & YES & $1.67$ & $(2,3)$ & NO & 1733\\
$(437,100)$ & 14 & $(2,1)$ & 1 & 1 & YES & YES & YES & $1.57$ & $(2,3)$ & NO & 1734\\
$(437,99)$ & 14 & $(3,1)$ & 2 & 1 & YES & YES & YES & $1.43$ & $(2,3)$ & NO & 1735\\
$(437,100)$ & 14 & $(3,1)$ & 2 & 1 & YES & YES & YES & $1.57$ & $(2,3)$ & -- & 1736\\
$(437,99)$ & 14 & $(9,2)$ & 5 & 1 & YES & YES & YES & $1.78$ & $(2,3)$ & NO & 1737\\
$(437,99)$ & 14 & $(31,7)$ & 8 & 1 & YES & YES & YES & $1.43$ & $(2,3)$ & 1511 & 1738\\
$(437,99)$ & 14 & $(75,17)$ & 10 & 1 & YES & YES & YES & $1.43$ & $(2,3)$ & NO & 1739\\
$(448,97)$ & 14 & $(3,1)$ & 2 & 1 & YES & YES & YES & $1.56$ & $(2,3)$ & -- & 1740\\
$(449,105)$ & 14 & $(449,105)$ & 14 & 449 & YES & YES & YES & $1.78$ & $(2,3)$ & NO & 1741\\
$(464,105)$ & 14 & $(2,1)$ & 1 & 2 & YES & YES & YES & $1.57$ & $(2,3)$ & NO & 1742\\
$(473,108)$ & 14 & $(3,1)$ & 2 & 1 & YES & YES & YES & $1.71$ & $(2,3)$ & -- & 1743\\
$(477,104)$ & 14 & $(2,1)$ & 1 & 1 & YES & YES & YES & $1.71$ & $(2,3)$ & NO & 1744\\
$(477,104)$ & 14 & $(14,3)$ & 6 & 1 & YES & YES & YES & $1.57$ & $(2,3)$ & NO & 1745\\
$(526,123)$ & 14 & $(77,18)$ & 10 & 1 & YES & YES & YES & $1.56$ & $(2,3)$ & NO & 1746\\
$(561,128)$ & 14 & $(3,1)$ & 2 & 3 & YES & YES & YES & $1.57$ & $(2,3)$ & -- & 1747\\
$(561,128)$ & 14 & $(4,1)$ & 3 & 1 & YES & YES & YES & $1.67$ & $(2,3)$ & NO & 1748\\
$(a;0,0,0;3)$ & 4 & $(55,21)$ & 8 & 1 & YES & YES & YES & $1.64$ & $(2,3)$ & -- & 1749\\
$(a;0,0,0;3)$ & 4 & $(61,18)$ & 9 & 1 & YES & YES & YES & $1.56$ & $(2,3)$ & -- & 1750\\
$(a;0,0,0;3)$ & 4 & $(68,19)$ & 9 & 1 & YES & YES & YES & $1.82$ & $(2,3)$ & -- & 1751\\
$(a;0,0,0;3)$ & 4 & $(76,21)$ & 9 & 1 & YES & YES & YES & $1.14$ & $(6,1)$ & -- & 1752\\
$(a;0,0,0;3)$ & 4 & $(79,18)$ & 10 & 1 & YES & YES & YES & $1.82$ & $(2,3)$ & -- & 1753\\
$(a;1,0,0;13)$ & 5 & $(29,11)$ & 7 & 1 & YES & YES & NO(2) & $1.93$ & $(2,3)$ & -- & 1754\\
$(a;1,0,0;13)$ & 5 & $(46,17)$ & 8 & 1 & YES & YES & YES & $1.62$ & $(4,2)$ & -- & 1755\\
$(a;1,0,0;13)$ & 5 & $(46,19)$ & 8 & 1 & YES & YES & YES & $1.50$ & $(4,2)$ & -- & 1756\\
$(a;1,0,0;13)$ & 5 & $(89,27)$ & 10 & 1 & YES & YES & YES & $1.50$ & $(4,2)$ & -- & 1757\\
$(a;1,1,0;19)$ & 6 & $(21,8)$ & 6 & 1 & YES & YES & YES & $1.82$ & $(2,3)$ & -- & 1758\\
$(a;1,1,0;19)$ & 6 & $(29,11)$ & 7 & 1 & YES & YES & YES & $1.78$ & $(2,3)$ & -- & 1759\\
$(a;1,1,0;19)$ & 6 & $(31,12)$ & 7 & 1 & YES & YES & YES & $2.00$ & $(2,3)$ & -- & 1760\\
$(b;0,0,0;14)$ & 5 & $(31,12)$ & 7 & 1 & YES & YES & NO(2) & $1.93$ & $(2,3)$ & -- & 1761\\
$(b;0,0,0;14)$ & 5 & $(34,13)$ & 7 & 2 & YES & YES & YES & $1.38$ & $(4,2)$ & -- & 1762\\
$(b;0,0,0;14)$ & 5 & $(40,11)$ & 8 & 2 & YES & YES & YES & $1.38$ & $(4,2)$ & -- & 1763\\
$(b;0,0,0;14)$ & 5 & $(43,12)$ & 8 & 1 & YES & YES & YES & $1.38$ & $(4,2)$ & -- & 1764\\
$(b;0,0,1;4)$ & 6 & $(21,8)$ & 6 & 1 & YES & YES & YES & $1.25$ & $(4,2)$ & -- & 1765\\
$(b;0,0,1;4)$ & 6 & $(24,7)$ & 7 & 4 & YES & YES & YES & $1.56$ & $(2,3)$ & -- & 1766\\
$(b;0,0,1;4)$ & 6 & $(29,11)$ & 7 & 1 & YES & YES & YES & $1.78$ & $(2,3)$ & -- & 1767\\
$(b;0,0,1;4)$ & 6 & $(29,12)$ & 7 & 1 & YES & YES & YES & $1.67$ & $(2,3)$ & -- & 1768\\
$(b;0,0,1;4)$ & 6 & $(31,9)$ & 8 & 1 & YES & YES & YES & $1.67$ & $(2,3)$ & -- & 1769\\
$(b;0,1,0;19)$ & 6 & $(21,8)$ & 6 & 1 & YES & YES & NO(2) & $1.93$ & $(2,3)$ & -- & 1770\\
$(b;0,1,0;19)$ & 6 & $(24,7)$ & 7 & 1 & YES & YES & NO(2) & $1.86$ & $(2,3)$ & -- & 1771\\
$(b;0,1,0;19)$ & 6 & $(29,12)$ & 7 & 1 & YES & YES & YES & $1.78$ & $(2,3)$ & -- & 1772\\
$(b;0,1,0;19)$ & 6 & $(31,9)$ & 8 & 1 & YES & YES & YES & $1.78$ & $(2,3)$ & -- & 1773\\
$(b;0,1,0;19)$ & 6 & $(33,10)$ & 8 & 1 & YES & YES & YES & $1.89$ & $(2,3)$ & -- & 1774\\
$(b;0,1,0;19)$ & 6 & $(47,13)$ & 8 & 1 & YES & YES & YES & $1.78$ & $(2,3)$ & -- & 1775\\
$(b;0,1,1;27)$ & 7 & $(19,8)$ & 6 & 1 & YES & YES & YES & $1.78$ & $(2,3)$ & -- & 1776\\
$(b;0,1,2;5)$ & 8 & $(13,5)$ & 5 & 1 & YES & YES & YES & $1.62$ & $(4,2)$ & -- & 1777\\
$(b;0,2,1;34)$ & 8 & $(17,5)$ & 6 & 17 & YES & YES & YES & $1.38$ & $(4,2)$ & -- & 1778\\
$(b;0,2,1;34)$ & 8 & $(18,5)$ & 6 & 2 & YES & YES & YES & $2.00$ & $(2,3)$ & -- & 1779\\
$(b;1,0,1;29)$ & 7 & $(11,4)$ & 5 & 1 & YES & YES & YES & $1.73$ & $(2,3)$ & -- & 1780\\
$(c;0,0,0;4)$ & 4 & $(66,25)$ & 9 & 2 & YES & YES & YES & $1.82$ & $(2,3)$ & -- & 1781\\
$(c;0,0,0;4)$ & 4 & $(71,27)$ & 9 & 1 & YES & YES & YES & $1.50$ & $(4,2)$ & -- & 1782\\
$(c;0,0,0;4)$ & 4 & $(75,31)$ & 9 & 1 & YES & YES & NO(2) & $1.73$ & $(4,2)$ & -- & 1783\\
$(c;0,0,0;4)$ & 4 & $(76,29)$ & 9 & 4 & YES & YES & YES & $1.78$ & $(2,3)$ & -- & 1784\\
$(c;0,0,0;4)$ & 4 & $(79,30)$ & 9 & 1 & YES & YES & NO(2) & $1.73$ & $(4,2)$ & -- & 1785\\
$(c;0,0,0;4)$ & 4 & $(81,31)$ & 9 & 1 & YES & YES & YES & $1.67$ & $(2,3)$ & -- & 1786\\
$(c;0,0,0;4)$ & 4 & $(89,33)$ & 10 & 1 & YES & YES & YES & $1.50$ & $(4,2)$ & -- & 1787\\
$(c;0,0,0;4)$ & 4 & $(89,34)$ & 9 & 1 & YES & YES & YES & $1.67$ & $(2,3)$ & -- & 1788\\
$(c;0,0,0;4)$ & 4 & $(99,29)$ & 10 & 1 & YES & YES & YES & $1.67$ & $(2,3)$ & -- & 1789\\
$(c;0,0,0;4)$ & 4 & $(100,37)$ & 10 & 4 & YES & YES & YES & $1.78$ & $(2,3)$ & -- & 1790\\
$(c;0,0,0;4)$ & 4 & $(105,29)$ & 10 & 1 & YES & YES & YES & $1.56$ & $(2,3)$ & -- & 1791\\
$(c;0,0,0;4)$ & 4 & $(105,31)$ & 10 & 1 & YES & YES & YES & $1.67$ & $(2,3)$ & -- & 1792\\
$(c;0,0,0;4)$ & 4 & $(106,31)$ & 10 & 2 & YES & YES & YES & $1.67$ & $(2,3)$ & -- & 1793\\
$(c;0,0,0;4)$ & 4 & $(107,30)$ & 11 & 1 & YES & YES & YES & $1.50$ & $(4,2)$ & -- & 1794\\
$(c;0,0,0;4)$ & 4 & $(109,40)$ & 10 & 1 & YES & YES & YES & $1.78$ & $(2,3)$ & -- & 1795\\
$(c;0,0,0;4)$ & 4 & $(109,45)$ & 10 & 1 & YES & YES & YES & $1.71$ & $(2,3)$ & -- & 1796\\
$(c;0,0,0;4)$ & 4 & $(111,31)$ & 10 & 1 & YES & YES & YES & $1.78$ & $(2,3)$ & -- & 1797\\
$(c;0,0,0;4)$ & 4 & $(111,46)$ & 10 & 1 & YES & YES & YES & $1.67$ & $(2,3)$ & -- & 1798\\
$(c;0,0,0;4)$ & 4 & $(122,33)$ & 11 & 2 & YES & YES & YES & $1.62$ & $(4,2)$ & -- & 1799\\
$(c;0,0,0;4)$ & 4 & $(123,34)$ & 10 & 1 & YES & YES & YES & $1.67$ & $(2,3)$ & -- & 1800\\
$(c;0,0,0;4)$ & 4 & $(127,29)$ & 11 & 1 & YES & YES & YES & $1.78$ & $(2,3)$ & -- & 1801\\
$(c;0,0,0;4)$ & 4 & $(153,35)$ & 12 & 1 & YES & YES & YES & $1.67$ & $(2,3)$ & -- & 1802\\
$(c;0,0,0;4)$ & 4 & $(157,46)$ & 11 & 1 & YES & YES & YES & $1.67$ & $(2,3)$ & -- & 1803\\
$(c;0,1,0;11)$ & 5 & $(47,18)$ & 8 & 1 & YES & YES & YES & $1.62$ & $(4,2)$ & -- & 1804\\
$(c;0,1,0;11)$ & 5 & $(55,21)$ & 8 & 11 & YES & YES & YES & $1.78$ & $(2,3)$ & -- & 1805\\
$(c;0,1,0;11)$ & 5 & $(62,23)$ & 9 & 1 & YES & YES & YES & $1.62$ & $(4,2)$ & -- & 1806\\
$(c;0,1,0;11)$ & 5 & $(79,24)$ & 10 & 1 & YES & YES & YES & $1.78$ & $(2,3)$ & -- & 1807\\
$(c;0,1,0;11)$ & 5 & $(79,30)$ & 9 & 1 & YES & YES & YES & $1.71$ & $(2,3)$ & -- & 1808\\
$(d;0,0,0;5)$ & 5 & $(47,18)$ & 8 & 1 & YES & YES & YES & $1.73$ & $(2,3)$ & -- & 1809\\
$(d;0,0,0;5)$ & 5 & $(58,17)$ & 9 & 1 & YES & YES & NO(2) & $1.79$ & $(2,3)$ & -- & 1810\\
$(d;0,0,0;5)$ & 5 & $(61,18)$ & 9 & 1 & YES & YES & YES & $1.82$ & $(2,3)$ & -- & 1811\\
$(d;0,0,0;5)$ & 5 & $(68,19)$ & 9 & 1 & YES & YES & YES & $1.82$ & $(2,3)$ & -- & 1812\\
$(d;0,0,0;5)$ & 5 & $(71,21)$ & 9 & 1 & YES & YES & NO(2) & $1.73$ & $(4,2)$ & -- & 1813\\
$(d;0,0,1;14)$ & 6 & $(29,11)$ & 7 & 1 & YES & YES & YES & $1.60$ & $(4,2)$ & -- & 1814\\
$(d;0,0,1;14)$ & 6 & $(31,13)$ & 7 & 1 & YES & YES & NO(2) & $1.70$ & $(6,1)$ & -- & 1815\\
$(d;0,0,1;14)$ & 6 & $(39,16)$ & 8 & 1 & YES & YES & YES & $1.62$ & $(4,2)$ & -- & 1816\\
$(d;0,0,1;14)$ & 6 & $(43,18)$ & 8 & 1 & YES & YES & YES & $1.78$ & $(2,3)$ & -- & 1817\\
$(d;0,0,1;14)$ & 6 & $(50,21)$ & 8 & 2 & YES & YES & YES & $1.57$ & $(2,3)$ & -- & 1818\\
$(e;0,0,0;4)$ & 5 & $(56,13)$ & 10 & 4 & YES & YES & YES & $1.62$ & $(4,2)$ & -- & 1819\\
$(e;1,0,0;18)$ & 6 & $(21,8)$ & 6 & 3 & YES & YES & YES & $1.82$ & $(2,3)$ & -- & 1820\\
$(e;1,0,0;18)$ & 6 & $(24,7)$ & 7 & 6 & YES & YES & NO(2) & $1.83$ & $(2,3)$ & -- & 1821\\
$(e;1,0,0;18)$ & 6 & $(31,12)$ & 7 & 1 & YES & YES & YES & $1.43$ & $(2,3)$ & -- & 1822\\
$(e;1,1,0;23)$ & 7 & $(17,7)$ & 6 & 1 & YES & YES & YES & $2.00$ & $(2,3)$ & -- & 1823\\
$(e;1,1,0;23)$ & 7 & $(27,8)$ & 7 & 1 & YES & YES & YES & $1.78$ & $(2,3)$ & -- & 1824\\
$(f;0,0,0;6)$ & 4 & $(80,31)$ & 9 & 2 & YES & YES & NO(2) & $1.73$ & $(4,2)$ & -- & 1825\\
$(f;0,0,0;6)$ & 4 & $(89,34)$ & 9 & 1 & YES & YES & YES & $1.82$ & $(2,3)$ & -- & 1826\\
$(f;0,0,0;6)$ & 4 & $(111,46)$ & 10 & 3 & YES & YES & YES & $1.50$ & $(4,2)$ & -- & 1827\\
$(f;0,0,0;6)$ & 4 & $(144,55)$ & 10 & 6 & YES & YES & YES & $1.57$ & $(2,3)$ & -- & 1828\\
$(f;0,0,0;6)$ & 4 & $(171,37)$ & 12 & 3 & YES & YES & YES & $1.62$ & $(4,2)$ & -- & 1829\\
$(g;0,0,0;19)$ & 6 & $(17,7)$ & 6 & 1 & YES & YES & NO(2) & $1.73$ & $(4,2)$ & -- & 1830\\
$(g;0,0,0;19)$ & 6 & $(21,8)$ & 6 & 1 & YES & YES & YES & $1.78$ & $(2,3)$ & -- & 1831\\
$(g;0,0,0;19)$ & 6 & $(24,7)$ & 7 & 1 & YES & YES & YES & $1.56$ & $(2,3)$ & -- & 1832\\
$(g;0,0,1;26)$ & 7 & $(11,4)$ & 5 & 1 & YES & YES & YES & $1.82$ & $(2,3)$ & -- & 1833\\
$(g;1,1,0;9)$ & 8 & $(8,3)$ & 4 & 1 & YES & YES & YES & $1.73$ & $(2,3)$ & -- & 1834\\
$(h;0,0,0;6)$ & 5 & $(21,8)$ & 6 & 3 & YES & YES & NO(2) & $1.79$ & $(2,3)$ & -- & 1835\\
$(h;0,0,0;6)$ & 5 & $(34,13)$ & 7 & 2 & YES & YES & YES & $1.89$ & $(2,3)$ & -- & 1836\\
$(h;0,0,0;6)$ & 5 & $(44,17)$ & 8 & 2 & YES & YES & YES & $1.78$ & $(2,3)$ & -- & 1837\\
$(h;0,1,0;8)$ & 6 & $(17,7)$ & 6 & 1 & YES & YES & NO(2) & $1.64$ & $(4,2)$ & -- & 1838\\
$(h;0,1,0;8)$ & 6 & $(21,8)$ & 6 & 1 & YES & YES & YES & $1.78$ & $(2,3)$ & -- & 1839\\
$(h;0,1,0;8)$ & 6 & $(24,7)$ & 7 & 8 & YES & YES & YES & $1.56$ & $(2,3)$ & -- & 1840\\
$(j;0,0,0;8)$ & 5 & $(75,29)$ & 9 & 1 & YES & YES & YES & $1.89$ & $(2,3)$ & -- & 1841
\end{longtable}
\subsection{2 chains, $K^2 = 4$}
\begin{longtable}{|c|c|c|c|c|c|c|c|c|c|c|c|}
\hline
\multicolumn{12}{|c|}{2 chains, $K^2 = 4$}\\
\hline
$(n,a)$ & Len & $(n,a)$ & Len & GCD & Nef & $\mathbb Q$-ef & Obs 0 & $\overline c_1^2 / \overline c_2$ & $(P,K)$ & WH & Index\\
\hline
\endfirsthead

\hline
$(n,a)$ & Len & $(n,a)$ & Len & GCD & Nef & $\mathbb Q$-ef & Obs 0 & $\overline c_1^2 / \overline c_2$ & $(P,K)$ & WH & Index\\
\hline
\endhead
\hline
\endfoot

$(58,17)$ & 9 & $(44,13)$ & 8 & 2 & YES & YES & NO(3) & $1.83$ & $(2,4)$ & -- & 1842\\
$(67,26)$ & 9 & $(44,13)$ & 8 & 1 & YES & YES & NO(2) & $2.00$ & $(4,3)$ & -- & 1843\\
$(89,34)$ & 9 & $(17,7)$ & 6 & 1 & YES & YES & YES & $1.71$ & $(4,3)$ & -- & 1844\\
$(101,30)$ & 10 & $(31,12)$ & 7 & 1 & YES & YES & NO(2) & $2.00$ & $(4,3)$ & -- & 1845\\
$(109,45)$ & 10 & $(25,7)$ & 7 & 1 & YES & YES & NO(2) & $2.12$ & $(4,3)$ & NO & 1846\\
$(116,45)$ & 10 & $(29,8)$ & 7 & 29 & YES & YES & NO(2) & $2.00$ & $(4,3)$ & -- & 1847\\
$(123,47)$ & 10 & $(37,8)$ & 8 & 1 & YES & YES & YES & $2.17$ & $(2,4)$ & -- & 1848\\
$(137,37)$ & 11 & $(37,11)$ & 8 & 1 & YES & YES & NO(2) & $2.12$ & $(4,3)$ & NO & 1849\\
$(138,37)$ & 11 & $(32,9)$ & 8 & 2 & YES & YES & YES & $2.17$ & $(2,4)$ & -- & 1850\\
$(157,46)$ & 11 & $(18,5)$ & 6 & 1 & YES & YES & NO(3) & $1.83$ & $(2,4)$ & -- & 1851\\
$(211,64)$ & 12 & $(29,8)$ & 7 & 1 & YES & YES & NO(2) & $2.00$ & $(4,3)$ & NO & 1852\\
$(246,91)$ & 12 & $(10,3)$ & 5 & 2 & YES & YES & NO(2) & $2.00$ & $(4,3)$ & -- & 1853\\
$(257,76)$ & 12 & $(10,3)$ & 5 & 1 & YES & YES & YES & $2.00$ & $(2,4)$ & -- & 1854\\
$(263,109)$ & 12 & $(7,3)$ & 4 & 1 & YES & YES & NO(2) & $2.00$ & $(4,3)$ & -- & 1855\\
$(278,65)$ & 13 & $(37,8)$ & 8 & 1 & YES & YES & YES & $2.17$ & $(2,4)$ & NO & 1856\\
$(292,85)$ & 13 & $(44,13)$ & 8 & 4 & YES & YES & NO(2) & $2.00$ & $(4,3)$ & NO & 1857\\
$(312,131)$ & 12 & $(17,7)$ & 6 & 1 & YES & YES & NO(2) & $2.00$ & $(4,3)$ & NO & 1858\\
$(317,131)$ & 12 & $(11,3)$ & 5 & 1 & YES & YES & YES & $2.17$ & $(2,4)$ & NO & 1859\\
$(374,155)$ & 13 & $(7,3)$ & 4 & 1 & YES & YES & NO(2) & $2.00$ & $(2,4)$ & NO & 1860\\
$(377,112)$ & 13 & $(41,12)$ & 8 & 1 & YES & YES & YES & $2.00$ & $(2,4)$ & NO & 1861\\
$(379,147)$ & 13 & $(5,2)$ & 3 & 1 & YES & YES & NO(2) & $2.00$ & $(4,3)$ & -- & 1862\\
$(391,165)$ & 13 & $(5,2)$ & 3 & 1 & YES & YES & NO(2) & $1.86$ & $(6,2)$ & -- & 1863\\
$(401,119)$ & 13 & $(5,2)$ & 3 & 1 & YES & YES & NO(2) & $2.00$ & $(6,2)$ & -- & 1864\\
$(408,169)$ & 13 & $(309,128)$ & 13 & 3 & YES & YES & NO(2) & $2.00$ & $(4,3)$ & NO & 1865\\
$(413,157)$ & 13 & $(7,2)$ & 4 & 7 & YES & YES & YES & $2.17$ & $(2,4)$ & -- & 1866\\
$(415,116)$ & 13 & $(10,3)$ & 5 & 5 & YES & YES & YES & $2.00$ & $(2,4)$ & -- & 1867\\
$(438,181)$ & 13 & $(2,1)$ & 1 & 2 & YES & YES & NO(2) & $2.00$ & $(2,4)$ & -- & 1868\\
$(445,187)$ & 13 & $(307,129)$ & 12 & 1 & YES & YES & YES & $2.17$ & $(2,4)$ & NO & 1869\\
$(462,179)$ & 13 & $(2,1)$ & 1 & 2 & YES & YES & NO(2) & $2.00$ & $(2,4)$ & -- & 1870\\
$(466,177)$ & 13 & $(3,1)$ & 2 & 1 & YES & YES & YES & $1.88$ & $(2,4)$ & -- & 1871\\
$(467,181)$ & 13 & $(3,1)$ & 2 & 1 & YES & YES & NO(2) & $2.00$ & $(2,4)$ & -- & 1872\\
$(485,188)$ & 13 & $(44,17)$ & 8 & 1 & YES & YES & YES & $2.17$ & $(2,4)$ & NO & 1873\\
$(507,214)$ & 14 & $(507,214)$ & 14 & 507 & YES & YES & NO(2) & $1.86$ & $(6,2)$ & NO & 1874\\
$(515,153)$ & 14 & $(340,101)$ & 13 & 5 & YES & YES & NO(2) & $2.00$ & $(4,3)$ & 1920 & 1875\\
$(522,119)$ & 14 & $(5,2)$ & 3 & 1 & YES & YES & NO(2) & $1.88$ & $(4,3)$ & NO & 1876\\
$(527,154)$ & 14 & $(27,8)$ & 7 & 1 & YES & YES & YES & $2.17$ & $(2,4)$ & 1916 & 1877\\
$(559,165)$ & 14 & $(2,1)$ & 1 & 1 & YES & YES & NO(2) & $2.00$ & $(4,3)$ & -- & 1878\\
$(559,214)$ & 14 & $(5,2)$ & 3 & 1 & YES & YES & NO(2) & $2.12$ & $(4,3)$ & NO & 1879\\
$(573,160)$ & 14 & $(5,2)$ & 3 & 1 & YES & YES & YES & $2.17$ & $(2,4)$ & NO & 1880\\
$(579,239)$ & 14 & $(3,1)$ & 2 & 3 & YES & YES & NO(2) & $2.12$ & $(4,3)$ & NO & 1881\\
$(610,233)$ & 13 & $(2,1)$ & 1 & 2 & YES & YES & YES & $2.00$ & $(2,4)$ & -- & 1882\\
$(611,179)$ & 14 & $(3,1)$ & 2 & 1 & YES & YES & NO(3) & $1.83$ & $(2,4)$ & -- & 1883\\
$(611,179)$ & 14 & $(157,46)$ & 11 & 1 & YES & YES & NO(3) & $1.83$ & $(2,4)$ & NO & 1884\\
$(619,181)$ & 14 & $(10,3)$ & 5 & 1 & YES & YES & YES & $2.00$ & $(2,4)$ & NO & 1885\\
$(633,266)$ & 14 & $(2,1)$ & 1 & 1 & YES & YES & NO(2) & $2.11$ & $(2,4)$ & -- & 1886\\
$(642,265)$ & 14 & $(4,1)$ & 3 & 2 & YES & YES & NO(2) & $1.88$ & $(4,3)$ & -- & 1887\\
$(643,188)$ & 14 & $(2,1)$ & 1 & 1 & YES & YES & NO(2) & $2.11$ & $(2,4)$ & -- & 1888\\
$(657,148)$ & 15 & $(53,12)$ & 9 & 1 & YES & YES & YES & $2.17$ & $(2,4)$ & NO & 1889\\
$(665,258)$ & 14 & $(67,26)$ & 9 & 1 & YES & YES & NO(2) & $2.12$ & $(4,3)$ & NO & 1890\\
$(673,199)$ & 14 & $(2,1)$ & 1 & 1 & YES & YES & YES & $2.00$ & $(2,4)$ & -- & 1891\\
$(683,265)$ & 14 & $(116,45)$ & 10 & 1 & YES & YES & NO(2) & $2.00$ & $(4,3)$ & 1903 & 1892\\
$(684,265)$ & 14 & $(8,3)$ & 4 & 4 & YES & YES & YES & $2.00$ & $(2,4)$ & NO & 1893\\
$(691,264)$ & 14 & $(390,149)$ & 13 & 1 & YES & YES & YES & $2.17$ & $(2,4)$ & NO & 1894\\
$(695,192)$ & 14 & $(18,5)$ & 6 & 1 & YES & YES & NO(3) & $1.83$ & $(2,4)$ & NO & 1895\\
$(698,265)$ & 14 & $(2,1)$ & 1 & 2 & YES & YES & NO(2) & $2.00$ & $(4,3)$ & NO & 1896\\
$(725,212)$ & 14 & $(277,81)$ & 12 & 1 & YES & YES & YES & $2.00$ & $(2,4)$ & NO & 1897\\
$(747,169)$ & 15 & $(5,2)$ & 3 & 1 & YES & YES & YES & $2.17$ & $(2,4)$ & NO & 1898\\
$(747,169)$ & 15 & $(5,2)$ & 3 & 1 & YES & YES & YES & $2.17$ & $(2,4)$ & -- & 1899\\
$(747,169)$ & 15 & $(75,17)$ & 10 & 3 & YES & YES & YES & $2.17$ & $(2,4)$ & NO & 1900\\
$(755,312)$ & 14 & $(3,1)$ & 2 & 1 & YES & YES & YES & $2.17$ & $(2,4)$ & NO & 1901\\
$(759,290)$ & 14 & $(2,1)$ & 1 & 1 & YES & YES & YES & $2.17$ & $(2,4)$ & NO & 1902\\
$(781,303)$ & 14 & $(67,26)$ & 9 & 1 & YES & YES & NO(2) & $2.00$ & $(4,3)$ & 1892 & 1903\\
$(788,301)$ & 14 & $(2,1)$ & 1 & 2 & YES & YES & YES & $2.17$ & $(2,4)$ & -- & 1904\\
$(797,232)$ & 15 & $(2,1)$ & 1 & 1 & YES & YES & NO(2) & $2.11$ & $(2,4)$ & -- & 1905\\
$(802,337)$ & 14 & $(2,1)$ & 1 & 2 & YES & YES & YES & $2.00$ & $(2,4)$ & -- & 1906\\
$(802,337)$ & 14 & $(307,129)$ & 12 & 1 & YES & YES & YES & $2.00$ & $(2,4)$ & NO & 1907\\
$(827,232)$ & 15 & $(2,1)$ & 1 & 1 & YES & YES & NO(2) & $2.11$ & $(2,4)$ & -- & 1908\\
$(830,343)$ & 14 & $(2,1)$ & 1 & 2 & YES & YES & YES & $2.00$ & $(2,4)$ & -- & 1909\\
$(830,343)$ & 14 & $(317,131)$ & 12 & 1 & YES & YES & YES & $2.00$ & $(2,4)$ & NO & 1910\\
$(833,246)$ & 15 & $(3,1)$ & 2 & 1 & YES & YES & YES & $2.17$ & $(2,4)$ & -- & 1911\\
$(833,246)$ & 15 & $(27,8)$ & 7 & 1 & YES & YES & YES & $2.17$ & $(2,4)$ & NO & 1912\\
$(835,248)$ & 15 & $(2,1)$ & 1 & 1 & YES & YES & YES & $2.17$ & $(2,4)$ & -- & 1913\\
$(835,248)$ & 15 & $(367,109)$ & 13 & 1 & YES & YES & YES & $2.17$ & $(2,4)$ & NO & 1914\\
$(842,257)$ & 15 & $(10,3)$ & 5 & 2 & YES & YES & YES & $2.17$ & $(2,4)$ & NO & 1915\\
$(867,256)$ & 15 & $(7,2)$ & 4 & 1 & YES & YES & YES & $2.17$ & $(2,4)$ & 1877 & 1916\\
$(882,337)$ & 14 & $(3,1)$ & 2 & 3 & YES & YES & YES & $2.17$ & $(2,4)$ & -- & 1917\\
$(882,337)$ & 14 & $(123,47)$ & 10 & 3 & YES & YES & YES & $2.17$ & $(2,4)$ & NO & 1918\\
$(885,338)$ & 14 & $(2,1)$ & 1 & 1 & YES & YES & YES & $2.17$ & $(2,4)$ & -- & 1919\\
$(892,265)$ & 15 & $(101,30)$ & 10 & 1 & YES & YES & NO(2) & $2.00$ & $(4,3)$ & 1875 & 1920\\
$(907,265)$ & 15 & $(3,1)$ & 2 & 1 & YES & YES & NO(2) & $2.00$ & $(4,3)$ & NO & 1921\\
$(907,264)$ & 15 & $(347,101)$ & 13 & 1 & YES & YES & YES & $2.00$ & $(2,4)$ & NO & 1922\\
$(941,264)$ & 15 & $(2,1)$ & 1 & 1 & YES & YES & YES & $2.17$ & $(2,4)$ & NO & 1923\\
$(949,265)$ & 15 & $(43,12)$ & 8 & 1 & YES & YES & YES & $2.00$ & $(2,4)$ & NO & 1924\\
$(955,392)$ & 15 & $(955,392)$ & 15 & 955 & YES & YES & YES & $2.17$ & $(2,4)$ & NO & 1925\\
$(995,279)$ & 16 & $(3,1)$ & 2 & 1 & YES & YES & YES & $2.17$ & $(2,4)$ & -- & 1926\\
$(995,279)$ & 16 & $(57,16)$ & 9 & 1 & YES & YES & YES & $2.17$ & $(2,4)$ & NO & 1927\\
$(1027,305)$ & 15 & $(37,11)$ & 8 & 1 & YES & YES & YES & $2.00$ & $(2,4)$ & NO & 1928\\
$(1037,278)$ & 16 & $(3,1)$ & 2 & 1 & YES & YES & YES & $2.17$ & $(2,4)$ & -- & 1929\\
$(1043,243)$ & 16 & $(13,3)$ & 6 & 1 & YES & YES & NO(2) & $2.00$ & $(4,3)$ & NO & 1930\\
$(1171,265)$ & 16 & $(53,12)$ & 9 & 1 & YES & YES & YES & $2.00$ & $(2,4)$ & NO & 1931\\
$(1348,305)$ & 16 & $(2,1)$ & 1 & 2 & YES & YES & NO(2) & $1.88$ & $(4,3)$ & -- & 1932\\
$(b;0,0,0;14)$ & 5 & $(118,35)$ & 11 & 2 & YES & YES & NO(2) & $2.00$ & $(4,3)$ & -- & 1933\\
$(c;0,0,0;4)$ & 4 & $(236,69)$ & 12 & 4 & YES & YES & NO(3) & $1.83$ & $(2,4)$ & -- & 1934\\
$(e;0,0,0;4)$ & 5 & $(118,35)$ & 11 & 2 & YES & YES & NO(2) & $2.00$ & $(4,3)$ & -- & 1935\\
$(e;1,0,0;18)$ & 6 & $(31,12)$ & 7 & 1 & YES & YES & NO(2) & $2.00$ & $(2,4)$ & -- & 1936\\
$(e;1,0,0;18)$ & 6 & $(65,18)$ & 9 & 1 & YES & YES & NO(3) & $1.83$ & $(2,4)$ & -- & 1937\\
$(f;0,0,0;6)$ & 4 & $(219,64)$ & 12 & 3 & YES & YES & NO(2) & $2.00$ & $(2,4)$ & -- & 1938\\
$(f;0,0,0;6)$ & 4 & $(308,129)$ & 12 & 2 & YES & YES & YES & $2.00$ & $(2,4)$ & -- & 1939
\end{longtable}


%%%%%%%%%%%%%%%%%%%%%%%%%%%%%%%%%%%%%%%%%%%
\section{$II^* + 2I_1$}

Base curves:
\begin{itemize}
  \item $A = z$.
  \item $F_1 = y^2z - x^3 - x^2z$.
  \item $F_2 = y^2z - x^3 - x^2z +\frac{4}{27}z^3$.
\end{itemize}
Pencil given by
\[F_\lambda = y^2z - x^3 - x^2z - \lambda z^3\]

All nine blowups are done at $[0,1,0]$.

Singular fibers are as follows:
\begin{itemize}
  \item $\lambda = \infty$: $II^*$ fiber given by $A$ and $E_1$ - $E_8$
  \item $\lambda = 0$: $I_1$ fiber given by $F_1$ with node at $[0,0,1]$.
  \item $\lambda = -4/27$: $I_1$ fiber given by $F_2$ with node at $[-2,0,3]$.
\end{itemize}

Special curves:
\begin{itemize}
  \item $R_1 = x$, double section through $[0,1,0]$ and $[0,0,1]$.
  \item $R_2 = 3x + 2z$, double section through $[0,1,0]$ and $[-2,0,3]$.
  \item $T = y$, triple section through $[0,0,1]$ and $[-2,0,3]$.
\end{itemize}
Input:
\lstinputlisting[language=config]{../Tests/IIs11.txt}
Result:
%\usepackage{longtable}
\subsection{1 chain, $K^2 = 1$}
\begin{longtable}{|c|c|c|c|c|c|c|c|}
\hline
\multicolumn{8}{|c|}{1 chain, $K^2 = 1$}\\
\hline
$(n,a)$ & Len & Nef & $\mathbb Q$-ef & Obs 0 & $\overline c_1^2 / \overline c_2$ & $(P,K)$ & Index\\
\hline
\endfirsthead

\hline
$(n,a)$ & Len & Nef & $\mathbb Q$-ef & Obs 0 & $\overline c_1^2 / \overline c_2$ & $(P,K)$ & Index\\
\hline
\endhead
\hline
\endfoot

$(13,4)$ & 6 & YES & YES & YES & $0.67$ & $(3,0)$ & 1\\
$(16,7)$ & 6 & YES & YES & YES & $1.00$ & $(1,1)$ & 2\\
$(30,7)$ & 8 & YES & YES & YES & $0.67$ & $(1,1)$ & 3\\
$(j;0,1,0;10)$ & 6 & YES & YES & YES & $0.67$ & $(1,1)$ & 4
\end{longtable}
\subsection{1 chain, $K^2 = 4$}
\begin{longtable}{|c|c|c|c|c|c|c|c|}
\hline
\multicolumn{8}{|c|}{1 chain, $K^2 = 4$}\\
\hline
$(n,a)$ & Len & Nef & $\mathbb Q$-ef & Obs 0 & $\overline c_1^2 / \overline c_2$ & $(P,K)$ & Index\\
\hline
\endfirsthead

\hline
$(n,a)$ & Len & Nef & $\mathbb Q$-ef & Obs 0 & $\overline c_1^2 / \overline c_2$ & $(P,K)$ & Index\\
\hline
\endhead
\hline
\endfoot

$(343,52)$ & 19 & YES & YES & YES & $1.57$ & $(7,1)$ & 5
\end{longtable}
\subsection{2 chains, $K^2 = 1$}
\begin{longtable}{|c|c|c|c|c|c|c|c|c|c|c|c|}
\hline
\multicolumn{12}{|c|}{2 chains, $K^2 = 1$}\\
\hline
$(n,a)$ & Len & $(n,a)$ & Len & GCD & Nef & $\mathbb Q$-ef & Obs 0 & $\overline c_1^2 / \overline c_2$ & $(P,K)$ & WH & Index\\
\hline
\endfirsthead

\hline
$(n,a)$ & Len & $(n,a)$ & Len & GCD & Nef & $\mathbb Q$-ef & Obs 0 & $\overline c_1^2 / \overline c_2$ & $(P,K)$ & WH & Index\\
\hline
\endhead
\hline
\endfoot

$(8,3)$ & 4 & $(7,3)$ & 4 & 1 & YES & YES & YES & $0.82$ & $(2,1)$ & NO & 6\\
$(8,3)$ & 4 & $(7,3)$ & 4 & 1 & YES & YES & YES & $0.82$ & $(2,1)$ & -- & 7\\
$(11,3)$ & 5 & $(2,1)$ & 1 & 1 & YES & YES & YES & $0.60$ & $(4,0)$ & -- & 8\\
$(11,3)$ & 5 & $(4,1)$ & 3 & 1 & YES & YES & YES & $0.60$ & $(4,0)$ & NO & 9\\
$(11,3)$ & 5 & $(4,1)$ & 3 & 1 & YES & YES & YES & $0.60$ & $(4,0)$ & -- & 10\\
$(11,3)$ & 5 & $(4,1)$ & 3 & 1 & YES & YES & YES & $0.60$ & $(4,0)$ & NO & 11\\
$(11,4)$ & 5 & $(8,3)$ & 4 & 1 & YES & YES & YES & $0.82$ & $(2,1)$ & NO & 12\\
$(13,3)$ & 6 & $(2,1)$ & 1 & 1 & YES & YES & YES & $0.60$ & $(4,0)$ & -- & 13\\
$(13,3)$ & 6 & $(4,1)$ & 3 & 1 & YES & YES & YES & $0.60$ & $(4,0)$ & NO & 14\\
$(13,3)$ & 6 & $(4,1)$ & 3 & 1 & YES & YES & YES & $0.60$ & $(4,0)$ & -- & 15\\
$(15,4)$ & 6 & $(4,1)$ & 3 & 1 & NO & YES & YES & $0.60$ & $(4,0)$ & -- & 16\\
$(16,7)$ & 6 & $(5,1)$ & 4 & 1 & YES & YES & YES & $0.82$ & $(2,1)$ & NO & 17\\
$(16,7)$ & 6 & $(5,1)$ & 4 & 1 & YES & YES & YES & $0.82$ & $(2,1)$ & -- & 18\\
$(17,4)$ & 7 & $(4,1)$ & 3 & 1 & NO & YES & YES & $0.60$ & $(4,0)$ & NO & 19\\
$(17,4)$ & 7 & $(4,1)$ & 3 & 1 & NO & YES & YES & $0.60$ & $(4,0)$ & -- & 20\\
$(c;0,1,1;5)$ & 6 & $(2,1)$ & 1 & 1 & YES & YES & YES & $0.73$ & $(2,1)$ & -- & 21\\
$(f;0,0,0;6)$ & 4 & $(4,1)$ & 3 & 2 & YES & YES & YES & $0.44$ & $(4,0)$ & -- & 22\\
$(f;0,0,0;6)$ & 4 & $(9,2)$ & 5 & 3 & YES & YES & YES & $0.82$ & $(2,1)$ & -- & 23\\
$(f;0,1,0;7)$ & 5 & $(4,1)$ & 3 & 1 & YES & YES & YES & $0.82$ & $(2,1)$ & -- & 24
\end{longtable}
\subsection{2 chains, $K^2 = 2$}
\begin{longtable}{|c|c|c|c|c|c|c|c|c|c|c|c|}
\hline
\multicolumn{12}{|c|}{2 chains, $K^2 = 2$}\\
\hline
$(n,a)$ & Len & $(n,a)$ & Len & GCD & Nef & $\mathbb Q$-ef & Obs 0 & $\overline c_1^2 / \overline c_2$ & $(P,K)$ & WH & Index\\
\hline
\endfirsthead

\hline
$(n,a)$ & Len & $(n,a)$ & Len & GCD & Nef & $\mathbb Q$-ef & Obs 0 & $\overline c_1^2 / \overline c_2$ & $(P,K)$ & WH & Index\\
\hline
\endhead
\hline
\endfoot

$(11,2)$ & 6 & $(10,3)$ & 5 & 1 & YES & YES & YES & $0.89$ & $(6,0)$ & NO & 25\\
$(11,2)$ & 6 & $(10,3)$ & 5 & 1 & YES & YES & YES & $0.89$ & $(6,0)$ & -- & 26\\
$(17,6)$ & 7 & $(13,3)$ & 6 & 1 & YES & YES & YES & $1.33$ & $(2,2)$ & NO & 27\\
$(20,7)$ & 8 & $(7,2)$ & 4 & 1 & YES & YES & YES & $1.00$ & $(4,1)$ & -- & 28\\
$(20,3)$ & 8 & $(17,6)$ & 7 & 1 & YES & YES & YES & $1.33$ & $(2,2)$ & NO & 29\\
$(25,9)$ & 7 & $(13,3)$ & 6 & 1 & YES & YES & YES & $1.25$ & $(2,2)$ & NO & 30\\
$(27,5)$ & 8 & $(3,1)$ & 2 & 3 & YES & YES & YES & $0.89$ & $(6,0)$ & NO & 31\\
$(27,5)$ & 8 & $(3,1)$ & 2 & 3 & YES & YES & YES & $0.89$ & $(6,0)$ & -- & 32\\
$(27,5)$ & 8 & $(6,1)$ & 5 & 3 & YES & YES & YES & $0.89$ & $(6,0)$ & NO & 33\\
$(27,5)$ & 8 & $(6,1)$ & 5 & 3 & YES & YES & YES & $0.89$ & $(6,0)$ & -- & 34\\
$(27,5)$ & 8 & $(6,1)$ & 5 & 3 & YES & YES & YES & $0.89$ & $(6,0)$ & NO & 35\\
$(27,10)$ & 7 & $(17,6)$ & 7 & 1 & YES & YES & YES & $1.33$ & $(2,2)$ & NO & 36\\
$(28,5)$ & 8 & $(3,1)$ & 2 & 1 & YES & YES & YES & $0.89$ & $(6,0)$ & NO & 37\\
$(28,5)$ & 8 & $(3,1)$ & 2 & 1 & YES & YES & YES & $0.89$ & $(6,0)$ & -- & 38\\
$(28,5)$ & 8 & $(5,1)$ & 4 & 1 & YES & YES & YES & $0.89$ & $(6,0)$ & NO & 39\\
$(28,5)$ & 8 & $(5,1)$ & 4 & 1 & YES & YES & YES & $0.89$ & $(6,0)$ & NO & 40\\
$(28,5)$ & 8 & $(5,1)$ & 4 & 1 & YES & YES & YES & $0.89$ & $(6,0)$ & -- & 41\\
$(28,5)$ & 8 & $(11,2)$ & 6 & 1 & YES & YES & YES & $0.89$ & $(6,0)$ & NO & 42\\
$(28,5)$ & 8 & $(14,5)$ & 6 & 14 & YES & YES & YES & $1.25$ & $(2,2)$ & -- & 43\\
$(32,7)$ & 8 & $(5,1)$ & 4 & 1 & YES & YES & YES & $0.89$ & $(4,1)$ & NO & 44\\
$(32,7)$ & 8 & $(5,1)$ & 4 & 1 & YES & YES & YES & $0.89$ & $(4,1)$ & -- & 45\\
$(33,13)$ & 9 & $(5,1)$ & 4 & 1 & YES & YES & YES & $1.33$ & $(2,2)$ & -- & 46\\
$(33,13)$ & 9 & $(23,9)$ & 7 & 1 & YES & YES & YES & $1.33$ & $(2,2)$ & 53 & 47\\
$(43,8)$ & 9 & $(5,1)$ & 4 & 1 & NO & YES & YES & $0.89$ & $(6,0)$ & -- & 48\\
$(46,13)$ & 10 & $(7,2)$ & 4 & 1 & YES & YES & YES & $1.00$ & $(4,1)$ & NO & 49\\
$(48,17)$ & 9 & $(17,6)$ & 7 & 1 & YES & YES & YES & $1.33$ & $(2,2)$ & NO & 50\\
$(49,9)$ & 10 & $(28,5)$ & 8 & 7 & YES & YES & YES & $1.25$ & $(2,2)$ & NO & 51\\
$(51,20)$ & 9 & $(5,1)$ & 4 & 1 & YES & YES & YES & $1.25$ & $(2,2)$ & -- & 52\\
$(51,20)$ & 9 & $(5,2)$ & 3 & 1 & YES & YES & YES & $1.33$ & $(2,2)$ & 47 & 53\\
$(53,19)$ & 9 & $(3,1)$ & 2 & 1 & YES & YES & YES & $1.33$ & $(2,2)$ & NO & 54\\
$(53,19)$ & 9 & $(25,9)$ & 7 & 1 & YES & YES & YES & $1.25$ & $(2,2)$ & 56 & 55\\
$(64,23)$ & 9 & $(14,5)$ & 6 & 2 & YES & YES & YES & $1.25$ & $(2,2)$ & 55 & 56\\
$(72,13)$ & 12 & $(5,1)$ & 4 & 1 & YES & YES & YES & $1.25$ & $(2,2)$ & NO & 57\\
$(c;0,0,0;4)$ & 4 & $(13,6)$ & 7 & 1 & YES & YES & YES & $1.00$ & $(4,1)$ & -- & 58\\
$(c;0,0,0;4)$ & 4 & $(17,6)$ & 7 & 1 & YES & YES & YES & $1.00$ & $(4,1)$ & -- & 59\\
$(c;0,2,1;19)$ & 7 & $(11,3)$ & 5 & 1 & YES & YES & YES & $1.25$ & $(2,2)$ & -- & 60\\
$(i;0,0,0;9)$ & 5 & $(6,1)$ & 5 & 3 & YES & YES & YES & $1.00$ & $(4,1)$ & -- & 61\\
$(i;0,3,0;18)$ & 8 & $(5,1)$ & 4 & 1 & YES & YES & YES & $1.25$ & $(2,2)$ & -- & 62
\end{longtable}
\subsection{2 chains, $K^2 = 3$}
\begin{longtable}{|c|c|c|c|c|c|c|c|c|c|c|c|}
\hline
\multicolumn{12}{|c|}{2 chains, $K^2 = 3$}\\
\hline
$(n,a)$ & Len & $(n,a)$ & Len & GCD & Nef & $\mathbb Q$-ef & Obs 0 & $\overline c_1^2 / \overline c_2$ & $(P,K)$ & WH & Index\\
\hline
\endfirsthead

\hline
$(n,a)$ & Len & $(n,a)$ & Len & GCD & Nef & $\mathbb Q$-ef & Obs 0 & $\overline c_1^2 / \overline c_2$ & $(P,K)$ & WH & Index\\
\hline
\endhead
\hline
\endfoot

$(54,17)$ & 10 & $(5,1)$ & 4 & 1 & YES & YES & YES & $1.38$ & $(6,1)$ & NO & 63\\
$(59,8)$ & 11 & $(2,1)$ & 1 & 1 & YES & YES & YES & $1.14$ & $(8,0)$ & NO & 64\\
$(59,8)$ & 11 & $(8,1)$ & 7 & 1 & YES & YES & YES & $1.14$ & $(8,0)$ & 69 & 65\\
$(59,8)$ & 11 & $(8,1)$ & 7 & 1 & YES & YES & YES & $1.14$ & $(8,0)$ & -- & 66\\
$(59,8)$ & 11 & $(8,1)$ & 7 & 1 & YES & YES & YES & $1.14$ & $(8,0)$ & NO & 67\\
$(61,8)$ & 11 & $(2,1)$ & 1 & 1 & YES & YES & YES & $1.14$ & $(8,0)$ & NO & 68\\
$(61,8)$ & 11 & $(7,1)$ & 6 & 1 & YES & YES & YES & $1.14$ & $(8,0)$ & 65 & 69\\
$(61,8)$ & 11 & $(7,1)$ & 6 & 1 & YES & YES & YES & $1.14$ & $(8,0)$ & NO & 70\\
$(61,8)$ & 11 & $(7,1)$ & 6 & 1 & YES & YES & YES & $1.14$ & $(8,0)$ & -- & 71\\
$(73,11)$ & 11 & $(6,1)$ & 5 & 1 & YES & YES & YES & $1.14$ & $(6,1)$ & NO & 72\\
$(73,11)$ & 11 & $(6,1)$ & 5 & 1 & YES & YES & YES & $1.14$ & $(6,1)$ & NO & 73\\
$(73,11)$ & 11 & $(6,1)$ & 5 & 1 & YES & YES & YES & $1.14$ & $(6,1)$ & -- & 74\\
$(87,23)$ & 10 & $(2,1)$ & 1 & 1 & YES & YES & YES & $1.25$ & $(6,1)$ & -- & 75\\
$(96,13)$ & 12 & $(7,1)$ & 6 & 1 & NO & YES & YES & $1.14$ & $(8,0)$ & -- & 76\\
$(115,18)$ & 12 & $(6,1)$ & 5 & 1 & NO & YES & YES & $1.14$ & $(6,1)$ & -- & 77\\
$(145,51)$ & 12 & $(2,1)$ & 1 & 1 & NO & YES & YES & $1.14$ & $(8,0)$ & -- & 78\\
$(151,53)$ & 12 & $(2,1)$ & 1 & 1 & NO & YES & YES & $1.14$ & $(8,0)$ & -- & 79
\end{longtable}
\subsection{2 chains, $K^2 = 4$}
\begin{longtable}{|c|c|c|c|c|c|c|c|c|c|c|c|}
\hline
\multicolumn{12}{|c|}{2 chains, $K^2 = 4$}\\
\hline
$(n,a)$ & Len & $(n,a)$ & Len & GCD & Nef & $\mathbb Q$-ef & Obs 0 & $\overline c_1^2 / \overline c_2$ & $(P,K)$ & WH & Index\\
\hline
\endfirsthead

\hline
$(n,a)$ & Len & $(n,a)$ & Len & GCD & Nef & $\mathbb Q$-ef & Obs 0 & $\overline c_1^2 / \overline c_2$ & $(P,K)$ & WH & Index\\
\hline
\endhead
\hline
\endfoot

$(34,11)$ & 13 & $(33,5)$ & 9 & 1 & YES & YES & YES & $1.83$ & $(8,1)$ & -- & 80\\
$(42,5)$ & 11 & $(34,11)$ & 13 & 2 & YES & YES & YES & $1.83$ & $(8,1)$ & -- & 81\\
$(73,21)$ & 14 & $(22,3)$ & 9 & 1 & YES & YES & YES & $1.83$ & $(6,2)$ & -- & 82\\
$(89,28)$ & 11 & $(17,6)$ & 7 & 1 & YES & YES & YES & $1.83$ & $(8,1)$ & NO & 83\\
$(89,28)$ & 11 & $(34,11)$ & 13 & 1 & YES & YES & YES & $1.83$ & $(8,1)$ & NO & 84\\
$(95,23)$ & 13 & $(25,7)$ & 7 & 5 & YES & YES & YES & $2.00$ & $(6,2)$ & NO & 85\\
$(109,28)$ & 14 & $(11,3)$ & 5 & 1 & YES & YES & YES & $2.00$ & $(6,2)$ & -- & 86\\
$(109,38)$ & 12 & $(13,4)$ & 6 & 1 & YES & YES & YES & $1.83$ & $(8,1)$ & NO & 87\\
$(109,28)$ & 14 & $(29,8)$ & 7 & 1 & YES & YES & YES & $2.00$ & $(6,2)$ & NO & 88\\
$(116,37)$ & 13 & $(34,11)$ & 13 & 2 & YES & YES & YES & $1.83$ & $(8,1)$ & NO & 89\\
$(128,37)$ & 12 & $(32,9)$ & 8 & 32 & YES & YES & YES & $1.83$ & $(6,2)$ & NO & 90\\
$(128,37)$ & 12 & $(73,21)$ & 14 & 1 & YES & YES & YES & $1.83$ & $(6,2)$ & NO & 91\\
$(208,59)$ & 16 & $(3,1)$ & 2 & 1 & YES & YES & YES & $2.00$ & $(6,2)$ & -- & 92\\
$(208,59)$ & 16 & $(60,17)$ & 12 & 4 & YES & YES & YES & $2.00$ & $(6,2)$ & NO & 93\\
$(244,33)$ & 16 & $(163,22)$ & 14 & 1 & YES & YES & YES & $1.83$ & $(6,2)$ & NO & 94\\
$(267,59)$ & 17 & $(3,1)$ & 2 & 3 & YES & YES & YES & $2.00$ & $(6,2)$ & NO & 95\\
$(296,81)$ & 16 & $(11,3)$ & 5 & 1 & YES & YES & YES & $2.00$ & $(6,2)$ & NO & 96\\
$(350,53)$ & 19 & $(2,1)$ & 1 & 2 & YES & YES & YES & $2.00$ & $(6,2)$ & -- & 97\\
$(350,53)$ & 19 & $(33,5)$ & 9 & 1 & YES & YES & YES & $1.83$ & $(8,1)$ & NO & 98\\
$(392,53)$ & 20 & $(8,1)$ & 7 & 8 & YES & YES & YES & $1.83$ & $(6,2)$ & NO & 99\\
$(445,53)$ & 21 & $(42,5)$ & 11 & 1 & YES & YES & YES & $1.83$ & $(8,1)$ & NO & 100\\
$(b;0,0,4;38)$ & 9 & $(7,1)$ & 6 & 1 & YES & YES & YES & $1.83$ & $(8,1)$ & -- & 101\\
$(d;0,0,6;17)$ & 11 & $(25,7)$ & 7 & 1 & YES & YES & YES & $2.00$ & $(6,2)$ & -- & 102
\end{longtable}







%%%%%%%%%%%%%%%%%%%%%%%%%%%%%%%%%%%%%%%%%%%
\section{$I_4^* + 2I_1$}

Base curves:
\begin{itemize}
  \item $A = z$.
  \item $B = y$.
\end{itemize}
Pencil given by
\[F_\lambda = x^2y + z^3 + y^2z + \lambda yz^2\]

Nine exceptionals are as follows:
\begin{itemize}
  \item $E_1$ - $E_5$ at $A \cap B = [1,0,0]$.
  \item $E_6$ - $E_9$ at $A \cap x = [0,1,0]$.
\end{itemize}
Singular fibers are as follows:
\begin{itemize}
  \item $\lambda = 0$: $I_4^*$ fiber given by $A$, $B$ and $E_1$ - $E_4$, and $E_5$ - $E_8$.
  \item $\lambda = 2$: $I_1$ fiber called $F_1$ with node at $[0,-1,1]$.
  \item $\lambda = -2$: $I_1$ fiber called $F_2$ with node at $[0,1,1]$.
\end{itemize}

Special curves:
\begin{itemize}
  \item $H = x$, double section through $[0,1,1]$ and $[0,-1,1]$ and $[0,0,1]$.
  \item $V = y + z$, double section through $[1,0,0]$ and $[0,-1,1]$.
  \item $V = y - z$, double section through $[1,0,0]$ and $[0,1,1]$.
\end{itemize}
Input:
\lstinputlisting[language=config]{../Tests/4s11.txt}
Result:
%\usepackage{longtable}
\subsection{1 chain, $K^2 = 1$}
\begin{longtable}{|c|c|c|c|c|c|}
\hline
\multicolumn{6}{|c|}{1 chain, $K^2 = 1$}\\
\hline
$(n,a)$ & Length & Nef & $\mathbb Q$-ef & Obstruction 0 & Index\\
\hline
\endfirsthead

\hline
$(n,a)$ & Length & Nef & $\mathbb Q$-ef & Obstruction 0 & Index\\
\hline
\endhead
\hline
\endfoot

$(16, 7)$ & 6 & YES & YES & NO(2) & 1\\
$(30, 7)$ & 8 & YES & YES & NO(2) & 2\\
$(j; 0, 1, 0; 10)$ & 6 & YES & YES & NO(2) & 3
\end{longtable}
\subsection{2 chains, $K^2 = 1$}
\begin{longtable}{|c|c|c|c|c|c|c|c|c|c|}
\hline
\multicolumn{10}{|c|}{2 chains, $K^2 = 1$}\\
\hline
$(n,a)$ & Length & $(n,a)$ & Length & GCD & Nef & $\mathbb Q$-ef & Obstruction 0 & WH & Index\\
\hline
\endfirsthead

\hline
$(n,a)$ & Length & $(n,a)$ & Length & GCD & Nef & $\mathbb Q$-ef & Obstruction 0 & WH & Index\\
\hline
\endhead
\hline
\endfoot

$(7, 3)$ & 4 & $(5, 1)$ & 4 & 1 & YES & YES & YES & NO & 4\\
$(7, 3)$ & 4 & $(5, 1)$ & 4 & 1 & YES & YES & YES & NO & 5\\
$(8, 3)$ & 4 & $(7, 3)$ & 4 & 1 & YES & YES & NO(2) & -- & 6\\
$(8, 3)$ & 4 & $(7, 3)$ & 4 & 1 & YES & YES & NO(2) & NO & 7\\
$(11, 4)$ & 5 & $(8, 3)$ & 4 & 1 & YES & YES & NO(2) & NO & 8\\
$(16, 7)$ & 6 & $(5, 1)$ & 4 & 1 & YES & YES & NO(2) & -- & 9\\
$(16, 7)$ & 6 & $(5, 1)$ & 4 & 1 & YES & YES & NO(2) & NO & 10\\
$(c; 0, 1, 1; 5)$ & 6 & $(2, 1)$ & 1 & 1 & YES & YES & NO(2) & -- & 11\\
$(f; 0, 0, 0; 6)$ & 4 & $(4, 1)$ & 3 & 2 & YES & YES & YES & -- & 12\\
$(f; 0, 0, 0; 6)$ & 4 & $(9, 2)$ & 5 & 3 & YES & YES & NO(2) & -- & 13\\
$(f; 0, 1, 0; 7)$ & 5 & $(4, 1)$ & 3 & 1 & YES & YES & NO(2) & -- & 14
\end{longtable}
\subsection{2 chains, $K^2 = 2$}
\begin{longtable}{|c|c|c|c|c|c|c|c|c|c|}
\hline
\multicolumn{10}{|c|}{2 chains, $K^2 = 2$}\\
\hline
$(n,a)$ & Length & $(n,a)$ & Length & GCD & Nef & $\mathbb Q$-ef & Obstruction 0 & WH & Index\\
\hline
\endfirsthead

\hline
$(n,a)$ & Length & $(n,a)$ & Length & GCD & Nef & $\mathbb Q$-ef & Obstruction 0 & WH & Index\\
\hline
\endhead
\hline
\endfoot

$(13, 6)$ & 7 & $(10, 3)$ & 5 & 1 & YES & YES & YES & -- & 15\\
$(13, 6)$ & 7 & $(10, 3)$ & 5 & 1 & YES & YES & YES & NO & 16\\
$(13, 6)$ & 7 & $(13, 3)$ & 6 & 13 & YES & YES & YES & NO & 17\\
$(17, 6)$ & 7 & $(13, 3)$ & 6 & 1 & YES & YES & NO(2) & NO & 18\\
$(17, 7)$ & 6 & $(13, 6)$ & 7 & 1 & YES & YES & YES & NO & 19\\
$(20, 9)$ & 7 & $(8, 3)$ & 4 & 4 & YES & YES & YES & -- & 20\\
$(20, 9)$ & 7 & $(11, 3)$ & 5 & 1 & YES & YES & YES & NO & 21\\
$(20, 3)$ & 8 & $(13, 6)$ & 7 & 1 & YES & YES & YES & NO & 22\\
$(20, 9)$ & 7 & $(13, 3)$ & 6 & 1 & YES & YES & YES & NO & 23\\
$(20, 3)$ & 8 & $(17, 6)$ & 7 & 1 & YES & YES & NO(2) & NO & 24\\
$(24, 11)$ & 8 & $(5, 2)$ & 3 & 1 & YES & YES & YES & -- & 25\\
$(24, 11)$ & 8 & $(20, 9)$ & 7 & 4 & YES & YES & YES & NO & 26\\
$(25, 9)$ & 7 & $(13, 3)$ & 6 & 1 & YES & YES & NO(2) & NO & 27\\
$(27, 10)$ & 7 & $(17, 6)$ & 7 & 1 & YES & YES & NO(2) & NO & 28\\
$(28, 5)$ & 8 & $(14, 5)$ & 6 & 14 & YES & YES & NO(2) & -- & 29\\
$(29, 9)$ & 8 & $(8, 3)$ & 4 & 1 & YES & YES & YES & NO & 30\\
$(31, 14)$ & 8 & $(13, 6)$ & 7 & 1 & YES & YES & YES & NO & 31\\
$(32, 7)$ & 8 & $(5, 1)$ & 4 & 1 & YES & YES & NO(2) & -- & 32\\
$(32, 7)$ & 8 & $(5, 1)$ & 4 & 1 & YES & YES & NO(2) & NO & 33\\
$(32, 7)$ & 8 & $(9, 2)$ & 5 & 1 & YES & YES & NO(2) & NO & 34\\
$(33, 13)$ & 9 & $(5, 1)$ & 4 & 1 & YES & YES & NO(2) & -- & 35\\
$(33, 13)$ & 9 & $(23, 9)$ & 7 & 1 & YES & YES & NO(2) & 54 & 36\\
$(35, 16)$ & 9 & $(9, 4)$ & 5 & 1 & YES & YES & YES & NO & 37\\
$(35, 6)$ & 10 & $(20, 3)$ & 8 & 5 & YES & YES & YES & NO & 38\\
$(35, 6)$ & 10 & $(22, 3)$ & 9 & 1 & YES & YES & YES & NO & 39\\
$(37, 17)$ & 9 & $(3, 1)$ & 2 & 1 & YES & YES & YES & NO & 40\\
$(37, 13)$ & 9 & $(8, 3)$ & 4 & 1 & YES & YES & YES & NO & 41\\
$(37, 17)$ & 9 & $(13, 6)$ & 7 & 1 & YES & YES & YES & NO & 42\\
$(41, 13)$ & 10 & $(2, 1)$ & 1 & 1 & YES & YES & YES & NO & 43\\
$(41, 13)$ & 10 & $(4, 1)$ & 3 & 1 & YES & YES & YES & NO & 44\\
$(42, 19)$ & 9 & $(3, 1)$ & 2 & 3 & YES & YES & YES & NO & 45\\
$(42, 19)$ & 9 & $(9, 4)$ & 5 & 3 & YES & YES & YES & NO & 46\\
$(48, 17)$ & 9 & $(3, 1)$ & 2 & 3 & YES & YES & YES & NO & 47\\
$(48, 17)$ & 9 & $(11, 4)$ & 5 & 1 & YES & YES & YES & 59 & 48\\
$(48, 17)$ & 9 & $(17, 6)$ & 7 & 1 & YES & YES & NO(2) & NO & 49\\
$(49, 22)$ & 9 & $(3, 1)$ & 2 & 1 & YES & YES & YES & NO & 50\\
$(49, 22)$ & 9 & $(20, 9)$ & 7 & 1 & YES & YES & YES & NO & 51\\
$(49, 9)$ & 10 & $(28, 5)$ & 8 & 7 & YES & YES & NO(2) & NO & 52\\
$(51, 20)$ & 9 & $(5, 1)$ & 4 & 1 & YES & YES & NO(2) & -- & 53\\
$(51, 20)$ & 9 & $(5, 2)$ & 3 & 1 & YES & YES & NO(2) & 36 & 54\\
$(51, 23)$ & 9 & $(9, 4)$ & 5 & 3 & YES & YES & YES & NO & 55\\
$(51, 23)$ & 9 & $(20, 9)$ & 7 & 1 & YES & YES & YES & NO & 56\\
$(53, 19)$ & 9 & $(3, 1)$ & 2 & 1 & YES & YES & NO(2) & NO & 57\\
$(53, 19)$ & 9 & $(25, 9)$ & 7 & 1 & YES & YES & NO(2) & 60 & 58\\
$(64, 23)$ & 9 & $(3, 1)$ & 2 & 1 & YES & YES & YES & 48 & 59\\
$(64, 23)$ & 9 & $(14, 5)$ & 6 & 2 & YES & YES & NO(2) & 58 & 60\\
$(72, 13)$ & 12 & $(5, 1)$ & 4 & 1 & YES & YES & NO(2) & NO & 61\\
$(72, 13)$ & 12 & $(11, 2)$ & 6 & 1 & YES & YES & YES & NO & 62\\
$(76, 13)$ & 12 & $(5, 1)$ & 4 & 1 & YES & YES & YES & NO & 63\\
$(76, 13)$ & 12 & $(35, 6)$ & 10 & 1 & YES & YES & YES & NO & 64\\
$(99, 17)$ & 12 & $(2, 1)$ & 1 & 1 & YES & YES & YES & NO & 65\\
$(99, 17)$ & 12 & $(35, 6)$ & 10 & 1 & YES & YES & YES & NO & 66\\
$(a; 4, 0, 0; 25)$ & 8 & $(3, 1)$ & 2 & 1 & YES & YES & YES & -- & 67\\
$(c; 0, 2, 1; 19)$ & 7 & $(11, 3)$ & 5 & 1 & YES & YES & NO(2) & -- & 68\\
$(c; 0, 3, 0; 17)$ & 7 & $(8, 3)$ & 4 & 1 & YES & YES & YES & -- & 69\\
$(i; 0, 0, 0; 9)$ & 5 & $(6, 1)$ & 5 & 3 & YES & YES & NO(2) & -- & 70\\
$(i; 0, 3, 0; 18)$ & 8 & $(5, 1)$ & 4 & 1 & YES & YES & NO(2) & -- & 71
\end{longtable}
\subsection{2 chains, $K^2 = 3$}
\begin{longtable}{|c|c|c|c|c|c|c|c|c|c|}
\hline
\multicolumn{10}{|c|}{2 chains, $K^2 = 3$}\\
\hline
$(n,a)$ & Length & $(n,a)$ & Length & GCD & Nef & $\mathbb Q$-ef & Obstruction 0 & WH & Index\\
\hline
\endfirsthead

\hline
$(n,a)$ & Length & $(n,a)$ & Length & GCD & Nef & $\mathbb Q$-ef & Obstruction 0 & WH & Index\\
\hline
\endhead
\hline
\endfoot

$(79, 28)$ & 10 & $(4, 1)$ & 3 & 1 & YES & YES & NO(2) & -- & 72\\
$(79, 25)$ & 12 & $(79, 25)$ & 12 & 79 & YES & YES & NO(2) & NO & 73\\
$(83, 24)$ & 11 & $(2, 1)$ & 1 & 1 & YES & YES & NO(2) & -- & 74\\
$(85, 24)$ & 11 & $(2, 1)$ & 1 & 1 & YES & YES & NO(2) & -- & 75\\
$(105, 43)$ & 11 & $(83, 34)$ & 10 & 1 & YES & YES & NO(2) & NO & 76\\
$(143, 59)$ & 11 & $(5, 2)$ & 3 & 1 & YES & YES & NO(2) & NO & 77\\
$(144, 61)$ & 11 & $(2, 1)$ & 1 & 2 & YES & YES & NO(2) & -- & 78\\
$(169, 64)$ & 11 & $(2, 1)$ & 1 & 1 & YES & YES & NO(2) & NO & 79\\
$(234, 43)$ & 14 & $(2, 1)$ & 1 & 2 & YES & YES & NO(2) & -- & 80\\
$(f; 0, 0, 0; 6)$ & 4 & $(83, 18)$ & 10 & 1 & YES & YES & NO(2) & -- & 81
\end{longtable}
\subsection{2 chains, $K^2 = 4$}
\begin{longtable}{|c|c|c|c|c|c|c|c|c|c|}
\hline
\multicolumn{10}{|c|}{2 chains, $K^2 = 4$}\\
\hline
$(n,a)$ & Length & $(n,a)$ & Length & GCD & Nef & $\mathbb Q$-ef & Obstruction 0 & WH & Index\\
\hline
\endfirsthead

\hline
$(n,a)$ & Length & $(n,a)$ & Length & GCD & Nef & $\mathbb Q$-ef & Obstruction 0 & WH & Index\\
\hline
\endhead
\hline
\endfoot

$(138, 61)$ & 12 & $(7, 2)$ & 4 & 1 & YES & YES & NO(2) & -- & 82\\
$(186, 41)$ & 14 & $(9, 4)$ & 5 & 3 & YES & YES & NO(2) & NO & 83\\
$(249, 104)$ & 14 & $(9, 4)$ & 5 & 3 & YES & YES & NO(2) & NO & 84\\
$(292, 77)$ & 14 & $(7, 2)$ & 4 & 1 & YES & YES & NO(2) & NO & 85\\
$(346, 95)$ & 15 & $(11, 3)$ & 5 & 1 & YES & YES & NO(2) & NO & 86\\
$(404, 107)$ & 15 & $(3, 1)$ & 2 & 1 & YES & YES & NO(2) & -- & 87\\
$(445, 68)$ & 18 & $(5, 1)$ & 4 & 5 & YES & YES & NO(2) & NO & 88\\
$(640, 87)$ & 19 & $(37, 5)$ & 10 & 1 & YES & YES & NO(2) & NO & 89\\
$(b; 0, 0, 4; 38)$ & 9 & $(11, 5)$ & 6 & 1 & YES & YES & NO(2) & -- & 90\\
$(b; 0, 4, 4; 90)$ & 13 & $(5, 1)$ & 4 & 5 & YES & YES & NO(2) & -- & 91
\end{longtable}


%%%%%%%%%%%%%%%%%%%%%%%%%%%%%%%%%%%%%%%%%%%
\section{$III^* + I_2 + I_1$}
Input:
\lstinputlisting[language=config]{../Tests/IIIs21.txt}
Result:
%\usepackage{longtable}
\subsection{1 chain, $K^2 = 1$}
\begin{longtable}{|c|c|c|c|c|c|c|c|}
\hline
\multicolumn{8}{|c|}{1 chain, $K^2 = 1$}\\
\hline
$(n,a)$ & Len & Nef & $\mathbb Q$-ef & Obs 0 & $\overline c_1^2 / \overline c_2$ & $(P,K)$ & Index\\
\hline
\endfirsthead

\hline
$(n,a)$ & Len & Nef & $\mathbb Q$-ef & Obs 0 & $\overline c_1^2 / \overline c_2$ & $(P,K)$ & Index\\
\hline
\endhead
\hline
\endfoot

$(b;0,0,0;14)$ & 5 & YES & YES & YES & $0.50$ & $(3,0)$ & 1
\end{longtable}
\subsection{1 chain, $K^2 = 2$}
\begin{longtable}{|c|c|c|c|c|c|c|c|}
\hline
\multicolumn{8}{|c|}{1 chain, $K^2 = 2$}\\
\hline
$(n,a)$ & Len & Nef & $\mathbb Q$-ef & Obs 0 & $\overline c_1^2 / \overline c_2$ & $(P,K)$ & Index\\
\hline
\endfirsthead

\hline
$(n,a)$ & Len & Nef & $\mathbb Q$-ef & Obs 0 & $\overline c_1^2 / \overline c_2$ & $(P,K)$ & Index\\
\hline
\endhead
\hline
\endfoot

$(50,19)$ & 8 & YES & YES & YES & $1.25$ & $(1,2)$ & 2\\
$(56,13)$ & 10 & YES & YES & YES & $0.78$ & $(5,0)$ & 3
\end{longtable}
\subsection{1 chain, $K^2 = 3$}
\begin{longtable}{|c|c|c|c|c|c|c|c|}
\hline
\multicolumn{8}{|c|}{1 chain, $K^2 = 3$}\\
\hline
$(n,a)$ & Len & Nef & $\mathbb Q$-ef & Obs 0 & $\overline c_1^2 / \overline c_2$ & $(P,K)$ & Index\\
\hline
\endfirsthead

\hline
$(n,a)$ & Len & Nef & $\mathbb Q$-ef & Obs 0 & $\overline c_1^2 / \overline c_2$ & $(P,K)$ & Index\\
\hline
\endhead
\hline
\endfoot

$(97,18)$ & 11 & YES & YES & YES & $1.12$ & $(5,1)$ & 4\\
$(113,24)$ & 11 & YES & YES & YES & $1.30$ & $(3,2)$ & 5\\
$(133,60)$ & 11 & YES & YES & YES & $1.25$ & $(5,1)$ & 6\\
$(135,32)$ & 12 & YES & YES & YES & $1.33$ & $(3,2)$ & 7\\
$(152,67)$ & 11 & YES & YES & YES & $1.36$ & $(1,3)$ & 8\\
$(161,48)$ & 12 & YES & YES & YES & $1.45$ & $(1,3)$ & 9
\end{longtable}
\subsection{2 chains, $K^2 = 1$}
\begin{longtable}{|c|c|c|c|c|c|c|c|c|c|c|c|}
\hline
\multicolumn{12}{|c|}{2 chains, $K^2 = 1$}\\
\hline
$(n,a)$ & Len & $(n,a)$ & Len & GCD & Nef & $\mathbb Q$-ef & Obs 0 & $\overline c_1^2 / \overline c_2$ & $(P,K)$ & WH & Index\\
\hline
\endfirsthead

\hline
$(n,a)$ & Len & $(n,a)$ & Len & GCD & Nef & $\mathbb Q$-ef & Obs 0 & $\overline c_1^2 / \overline c_2$ & $(P,K)$ & WH & Index\\
\hline
\endhead
\hline
\endfoot

$(7,3)$ & 4 & $(7,2)$ & 4 & 7 & YES & YES & YES & $0.56$ & $(4,0)$ & NO & 10\\
$(7,3)$ & 4 & $(7,2)$ & 4 & 7 & YES & YES & YES & $0.56$ & $(4,0)$ & -- & 11\\
$(8,3)$ & 4 & $(5,2)$ & 3 & 1 & YES & YES & YES & $0.44$ & $(4,0)$ & -- & 12\\
$(8,3)$ & 4 & $(7,2)$ & 4 & 1 & YES & YES & YES & $0.44$ & $(4,0)$ & NO & 13\\
$(8,3)$ & 4 & $(7,2)$ & 4 & 1 & YES & YES & YES & $0.44$ & $(4,0)$ & -- & 14\\
$(8,3)$ & 4 & $(7,3)$ & 4 & 1 & YES & YES & YES & $0.44$ & $(4,0)$ & -- & 15\\
$(9,4)$ & 5 & $(9,2)$ & 5 & 9 & YES & YES & YES & $0.91$ & $(2,1)$ & NO & 16\\
$(11,4)$ & 5 & $(4,1)$ & 3 & 1 & YES & YES & YES & $0.56$ & $(4,0)$ & NO & 17\\
$(11,4)$ & 5 & $(4,1)$ & 3 & 1 & YES & YES & YES & $0.56$ & $(4,0)$ & -- & 18\\
$(11,4)$ & 5 & $(8,3)$ & 4 & 1 & YES & YES & YES & $0.44$ & $(4,0)$ & NO & 19\\
$(12,5)$ & 5 & $(4,1)$ & 3 & 4 & YES & YES & YES & $0.56$ & $(4,0)$ & NO & 20\\
$(12,5)$ & 5 & $(4,1)$ & 3 & 4 & YES & YES & YES & $0.56$ & $(4,0)$ & -- & 21\\
$(12,5)$ & 5 & $(4,1)$ & 3 & 4 & YES & YES & YES & $0.56$ & $(4,0)$ & NO & 22\\
$(12,5)$ & 5 & $(5,2)$ & 3 & 1 & YES & YES & YES & $0.91$ & $(2,1)$ & NO & 23\\
$(12,5)$ & 5 & $(5,2)$ & 3 & 1 & YES & YES & YES & $0.91$ & $(2,1)$ & -- & 24\\
$(12,5)$ & 5 & $(7,2)$ & 4 & 1 & YES & YES & YES & $0.44$ & $(4,0)$ & -- & 25\\
$(12,5)$ & 5 & $(7,2)$ & 4 & 1 & YES & YES & YES & $0.44$ & $(4,0)$ & NO & 26\\
$(12,5)$ & 5 & $(9,2)$ & 5 & 3 & YES & YES & YES & $0.82$ & $(2,1)$ & -- & 27\\
$(12,5)$ & 5 & $(9,4)$ & 5 & 3 & YES & YES & YES & $0.91$ & $(2,1)$ & NO & 28\\
$(13,5)$ & 5 & $(3,1)$ & 2 & 1 & YES & YES & YES & $0.56$ & $(4,0)$ & NO & 29\\
$(13,5)$ & 5 & $(13,5)$ & 5 & 13 & YES & YES & YES & $0.44$ & $(4,0)$ & NO & 30\\
$(14,5)$ & 6 & $(2,1)$ & 1 & 2 & YES & YES & YES & $0.82$ & $(2,1)$ & -- & 31\\
$(15,4)$ & 6 & $(9,2)$ & 5 & 3 & YES & YES & YES & $0.82$ & $(2,1)$ & NO & 32\\
$(16,7)$ & 6 & $(4,1)$ & 3 & 4 & YES & YES & YES & $0.91$ & $(2,1)$ & NO & 33\\
$(16,7)$ & 6 & $(4,1)$ & 3 & 4 & YES & YES & YES & $0.91$ & $(2,1)$ & -- & 34\\
$(16,7)$ & 6 & $(5,2)$ & 3 & 1 & YES & YES & YES & $0.91$ & $(2,1)$ & NO & 35\\
$(16,7)$ & 6 & $(16,7)$ & 6 & 16 & YES & YES & YES & $0.91$ & $(2,1)$ & NO & 36\\
$(17,7)$ & 6 & $(12,5)$ & 5 & 1 & YES & YES & YES & $0.82$ & $(2,1)$ & NO & 37\\
$(17,7)$ & 6 & $(17,7)$ & 6 & 17 & YES & YES & YES & $0.91$ & $(2,1)$ & NO & 38\\
$(19,8)$ & 6 & $(4,1)$ & 3 & 1 & YES & YES & YES & $0.56$ & $(4,0)$ & -- & 39\\
$(19,5)$ & 7 & $(5,1)$ & 4 & 1 & YES & YES & YES & $0.91$ & $(2,1)$ & NO & 40\\
$(19,5)$ & 7 & $(6,1)$ & 5 & 1 & YES & YES & YES & $0.82$ & $(2,1)$ & NO & 41\\
$(19,5)$ & 7 & $(15,4)$ & 6 & 1 & YES & YES & YES & $0.82$ & $(2,1)$ & NO & 42\\
$(19,5)$ & 7 & $(19,5)$ & 7 & 19 & YES & YES & YES & $0.91$ & $(2,1)$ & NO & 43\\
$(19,8)$ & 6 & $(19,8)$ & 6 & 19 & YES & YES & YES & $0.91$ & $(2,1)$ & NO & 44\\
$(26,7)$ & 7 & $(26,7)$ & 7 & 26 & YES & YES & YES & $0.82$ & $(2,1)$ & NO & 45\\
$(30,7)$ & 8 & $(3,1)$ & 2 & 3 & YES & YES & YES & $0.73$ & $(2,1)$ & NO & 46\\
$(30,7)$ & 8 & $(9,2)$ & 5 & 3 & YES & YES & YES & $0.73$ & $(2,1)$ & NO & 47\\
$(a;1,0,0;13)$ & 5 & $(5,2)$ & 3 & 1 & YES & YES & YES & $0.91$ & $(2,1)$ & -- & 48\\
$(b;0,0,0;14)$ & 5 & $(2,1)$ & 1 & 2 & YES & YES & YES & $0.91$ & $(2,1)$ & -- & 49\\
$(d;0,0,0;5)$ & 5 & $(2,1)$ & 1 & 1 & YES & YES & YES & $0.44$ & $(4,0)$ & -- & 50\\
$(d;0,0,0;5)$ & 5 & $(3,1)$ & 2 & 1 & YES & YES & YES & $0.44$ & $(4,0)$ & -- & 51\\
$(f;0,0,0;6)$ & 4 & $(5,2)$ & 3 & 1 & YES & YES & YES & $0.56$ & $(4,0)$ & -- & 52\\
$(f;0,0,0;6)$ & 4 & $(7,2)$ & 4 & 1 & YES & YES & YES & $0.56$ & $(4,0)$ & -- & 53\\
$(j;0,0,0;8)$ & 5 & $(3,1)$ & 2 & 1 & YES & YES & YES & $0.44$ & $(4,0)$ & -- & 54\\
$(j;0,1,0;10)$ & 6 & $(3,1)$ & 2 & 1 & YES & YES & YES & $0.73$ & $(2,1)$ & -- & 55\\
$(j;0,1,0;10)$ & 6 & $(4,1)$ & 3 & 2 & YES & YES & YES & $0.73$ & $(2,1)$ & -- & 56
\end{longtable}
\subsection{2 chains, $K^2 = 2$}
\begin{longtable}{|c|c|c|c|c|c|c|c|c|c|c|c|}
\hline
\multicolumn{12}{|c|}{2 chains, $K^2 = 2$}\\
\hline
$(n,a)$ & Len & $(n,a)$ & Len & GCD & Nef & $\mathbb Q$-ef & Obs 0 & $\overline c_1^2 / \overline c_2$ & $(P,K)$ & WH & Index\\
\hline
\endfirsthead

\hline
$(n,a)$ & Len & $(n,a)$ & Len & GCD & Nef & $\mathbb Q$-ef & Obs 0 & $\overline c_1^2 / \overline c_2$ & $(P,K)$ & WH & Index\\
\hline
\endhead
\hline
\endfoot

$(13,5)$ & 5 & $(9,2)$ & 5 & 1 & YES & YES & YES & $0.89$ & $(4,1)$ & -- & 57\\
$(14,3)$ & 6 & $(10,3)$ & 5 & 2 & YES & YES & YES & $0.88$ & $(6,0)$ & NO & 58\\
$(14,3)$ & 6 & $(10,3)$ & 5 & 2 & YES & YES & YES & $0.88$ & $(6,0)$ & -- & 59\\
$(14,3)$ & 6 & $(11,3)$ & 5 & 1 & YES & YES & YES & $0.89$ & $(4,1)$ & -- & 60\\
$(14,3)$ & 6 & $(11,3)$ & 5 & 1 & YES & YES & YES & $1.00$ & $(4,1)$ & NO & 61\\
$(18,7)$ & 6 & $(9,2)$ & 5 & 9 & YES & YES & YES & $1.36$ & $(2,2)$ & NO & 62\\
$(18,7)$ & 6 & $(9,2)$ & 5 & 9 & YES & YES & YES & $1.36$ & $(2,2)$ & -- & 63\\
$(19,7)$ & 6 & $(7,2)$ & 4 & 1 & YES & YES & YES & $1.00$ & $(4,1)$ & NO & 64\\
$(19,7)$ & 6 & $(9,2)$ & 5 & 1 & YES & YES & YES & $1.00$ & $(4,1)$ & NO & 65\\
$(19,7)$ & 6 & $(9,2)$ & 5 & 1 & YES & YES & YES & $1.00$ & $(4,1)$ & -- & 66\\
$(19,7)$ & 6 & $(9,2)$ & 5 & 1 & YES & YES & YES & $1.00$ & $(4,1)$ & NO & 67\\
$(21,8)$ & 6 & $(5,1)$ & 4 & 1 & YES & YES & YES & $1.11$ & $(4,1)$ & NO & 68\\
$(21,8)$ & 6 & $(6,1)$ & 5 & 3 & YES & YES & YES & $1.11$ & $(4,1)$ & NO & 69\\
$(21,8)$ & 6 & $(6,1)$ & 5 & 3 & YES & YES & YES & $1.11$ & $(4,1)$ & NO & 70\\
$(21,8)$ & 6 & $(6,1)$ & 5 & 3 & YES & YES & YES & $1.11$ & $(4,1)$ & -- & 71\\
$(21,4)$ & 8 & $(7,3)$ & 4 & 7 & YES & YES & YES & $0.88$ & $(6,0)$ & NO & 72\\
$(21,4)$ & 8 & $(7,3)$ & 4 & 7 & YES & YES & YES & $0.88$ & $(6,0)$ & -- & 73\\
$(24,7)$ & 7 & $(4,1)$ & 3 & 4 & YES & YES & YES & $1.00$ & $(4,1)$ & -- & 74\\
$(24,7)$ & 7 & $(4,1)$ & 3 & 4 & YES & YES & YES & $1.11$ & $(4,1)$ & NO & 75\\
$(24,7)$ & 7 & $(5,1)$ & 4 & 1 & YES & YES & YES & $1.11$ & $(4,1)$ & NO & 76\\
$(24,7)$ & 7 & $(6,1)$ & 5 & 6 & YES & YES & YES & $1.00$ & $(4,1)$ & NO & 77\\
$(24,7)$ & 7 & $(6,1)$ & 5 & 6 & YES & YES & YES & $1.00$ & $(4,1)$ & -- & 78\\
$(24,7)$ & 7 & $(6,1)$ & 5 & 6 & YES & YES & YES & $1.11$ & $(4,1)$ & NO & 79\\
$(26,7)$ & 7 & $(5,2)$ & 3 & 1 & YES & YES & YES & $1.00$ & $(4,1)$ & NO & 80\\
$(26,7)$ & 7 & $(5,2)$ & 3 & 1 & YES & YES & YES & $1.00$ & $(4,1)$ & -- & 81\\
$(26,11)$ & 7 & $(5,1)$ & 4 & 1 & YES & YES & YES & $1.36$ & $(2,2)$ & NO & 82\\
$(29,11)$ & 7 & $(4,1)$ & 3 & 1 & YES & YES & YES & $1.27$ & $(2,2)$ & -- & 83\\
$(29,12)$ & 7 & $(5,1)$ & 4 & 1 & YES & YES & YES & $1.00$ & $(4,1)$ & NO & 84\\
$(29,12)$ & 7 & $(5,1)$ & 4 & 1 & YES & YES & YES & $1.00$ & $(4,1)$ & -- & 85\\
$(30,11)$ & 7 & $(5,1)$ & 4 & 5 & YES & YES & YES & $1.11$ & $(4,1)$ & NO & 86\\
$(30,11)$ & 7 & $(5,1)$ & 4 & 5 & YES & YES & YES & $1.11$ & $(4,1)$ & -- & 87\\
$(30,11)$ & 7 & $(5,1)$ & 4 & 5 & YES & YES & YES & $1.11$ & $(4,1)$ & NO & 88\\
$(31,7)$ & 8 & $(2,1)$ & 1 & 1 & YES & YES & YES & $1.00$ & $(4,1)$ & -- & 89\\
$(31,7)$ & 8 & $(5,1)$ & 4 & 1 & YES & YES & YES & $1.00$ & $(4,1)$ & NO & 90\\
$(31,7)$ & 8 & $(5,1)$ & 4 & 1 & YES & YES & YES & $1.00$ & $(4,1)$ & -- & 91\\
$(32,7)$ & 8 & $(2,1)$ & 1 & 2 & YES & YES & YES & $0.89$ & $(4,1)$ & NO & 92\\
$(32,7)$ & 8 & $(3,1)$ & 2 & 1 & YES & YES & YES & $0.89$ & $(4,1)$ & -- & 93\\
$(32,7)$ & 8 & $(3,1)$ & 2 & 1 & YES & YES & YES & $1.00$ & $(4,1)$ & NO & 94\\
$(32,7)$ & 8 & $(14,3)$ & 6 & 2 & YES & YES & YES & $0.75$ & $(6,0)$ & 102 & 95\\
$(32,7)$ & 8 & $(32,7)$ & 8 & 32 & YES & YES & YES & $0.89$ & $(4,1)$ & NO & 96\\
$(37,10)$ & 8 & $(3,1)$ & 2 & 1 & YES & YES & YES & $0.89$ & $(4,1)$ & NO & 97\\
$(37,10)$ & 8 & $(3,1)$ & 2 & 1 & YES & YES & YES & $0.89$ & $(4,1)$ & -- & 98\\
$(37,8)$ & 8 & $(5,1)$ & 4 & 1 & YES & YES & YES & $1.00$ & $(4,1)$ & -- & 99\\
$(37,8)$ & 8 & $(5,1)$ & 4 & 1 & YES & YES & YES & $1.11$ & $(4,1)$ & NO & 100\\
$(37,10)$ & 8 & $(7,2)$ & 4 & 1 & YES & YES & YES & $0.89$ & $(4,1)$ & NO & 101\\
$(37,8)$ & 8 & $(9,2)$ & 5 & 1 & YES & YES & YES & $0.75$ & $(6,0)$ & 95 & 102\\
$(37,8)$ & 8 & $(14,3)$ & 6 & 1 & YES & YES & YES & $1.00$ & $(4,1)$ & NO & 103\\
$(39,14)$ & 8 & $(3,1)$ & 2 & 3 & YES & YES & YES & $0.88$ & $(4,1)$ & NO & 104\\
$(39,14)$ & 8 & $(3,1)$ & 2 & 3 & YES & YES & YES & $0.88$ & $(4,1)$ & -- & 105\\
$(40,11)$ & 8 & $(2,1)$ & 1 & 2 & YES & YES & YES & $1.00$ & $(4,1)$ & -- & 106\\
$(41,11)$ & 8 & $(2,1)$ & 1 & 1 & YES & YES & YES & $1.27$ & $(2,2)$ & NO & 107\\
$(42,13)$ & 9 & $(5,1)$ & 4 & 1 & YES & YES & YES & $0.88$ & $(4,1)$ & NO & 108\\
$(42,13)$ & 9 & $(5,1)$ & 4 & 1 & YES & YES & YES & $0.88$ & $(4,1)$ & -- & 109\\
$(43,10)$ & 9 & $(5,1)$ & 4 & 1 & YES & YES & YES & $1.00$ & $(4,1)$ & -- & 110\\
$(43,10)$ & 9 & $(5,1)$ & 4 & 1 & YES & YES & YES & $1.11$ & $(4,1)$ & NO & 111\\
$(47,14)$ & 9 & $(4,1)$ & 3 & 1 & YES & YES & YES & $1.00$ & $(4,1)$ & -- & 112\\
$(49,15)$ & 9 & $(5,1)$ & 4 & 1 & YES & YES & YES & $1.27$ & $(2,2)$ & -- & 113\\
$(49,15)$ & 9 & $(5,1)$ & 4 & 1 & YES & YES & YES & $1.36$ & $(2,2)$ & NO & 114\\
$(50,19)$ & 8 & $(2,1)$ & 1 & 2 & YES & YES & YES & $0.88$ & $(4,1)$ & NO & 115\\
$(56,15)$ & 9 & $(4,1)$ & 3 & 4 & YES & YES & YES & $0.88$ & $(4,1)$ & NO & 116\\
$(68,25)$ & 9 & $(2,1)$ & 1 & 2 & NO & YES & YES & $0.89$ & $(4,1)$ & -- & 117\\
$(74,31)$ & 9 & $(2,1)$ & 1 & 2 & NO & YES & YES & $0.88$ & $(4,1)$ & -- & 118\\
$(79,30)$ & 9 & $(2,1)$ & 1 & 1 & NO & YES & YES & $1.27$ & $(2,2)$ & -- & 119\\
$(c;0,0,0;4)$ & 4 & $(15,4)$ & 6 & 1 & YES & YES & YES & $0.89$ & $(4,1)$ & -- & 120\\
$(c;0,2,1;19)$ & 7 & $(5,1)$ & 4 & 1 & YES & YES & YES & $1.00$ & $(4,1)$ & -- & 121\\
$(f;0,0,0;6)$ & 4 & $(21,4)$ & 8 & 3 & YES & YES & YES & $0.88$ & $(6,0)$ & -- & 122\\
$(i;0,0,0;9)$ & 5 & $(9,2)$ & 5 & 9 & YES & YES & YES & $1.00$ & $(4,1)$ & -- & 123\\
$(i;0,0,0;9)$ & 5 & $(10,3)$ & 5 & 1 & YES & YES & YES & $0.88$ & $(4,1)$ & -- & 124\\
$(i;0,1,0;12)$ & 6 & $(5,1)$ & 4 & 1 & YES & YES & YES & $1.27$ & $(2,2)$ & -- & 125
\end{longtable}
\subsection{2 chains, $K^2 = 3$}
\begin{longtable}{|c|c|c|c|c|c|c|c|c|c|c|c|}
\hline
\multicolumn{12}{|c|}{2 chains, $K^2 = 3$}\\
\hline
$(n,a)$ & Len & $(n,a)$ & Len & GCD & Nef & $\mathbb Q$-ef & Obs 0 & $\overline c_1^2 / \overline c_2$ & $(P,K)$ & WH & Index\\
\hline
\endfirsthead

\hline
$(n,a)$ & Len & $(n,a)$ & Len & GCD & Nef & $\mathbb Q$-ef & Obs 0 & $\overline c_1^2 / \overline c_2$ & $(P,K)$ & WH & Index\\
\hline
\endhead
\hline
\endfoot

$(21,8)$ & 6 & $(13,2)$ & 7 & 1 & YES & YES & YES & $1.14$ & $(6,1)$ & NO & 126\\
$(21,8)$ & 6 & $(13,2)$ & 7 & 1 & YES & YES & YES & $1.14$ & $(6,1)$ & -- & 127\\
$(21,8)$ & 6 & $(16,7)$ & 6 & 1 & YES & YES & YES & $1.25$ & $(4,2)$ & -- & 128\\
$(23,8)$ & 9 & $(23,5)$ & 7 & 23 & YES & YES & YES & $1.50$ & $(4,2)$ & NO & 129\\
$(25,8)$ & 10 & $(23,5)$ & 7 & 1 & YES & YES & YES & $1.50$ & $(4,2)$ & NO & 130\\
$(25,7)$ & 7 & $(24,11)$ & 8 & 1 & YES & YES & YES & $1.50$ & $(4,2)$ & -- & 131\\
$(29,9)$ & 8 & $(19,8)$ & 6 & 1 & YES & YES & YES & $1.44$ & $(2,3)$ & -- & 132\\
$(29,8)$ & 7 & $(25,8)$ & 10 & 1 & YES & YES & YES & $1.50$ & $(4,2)$ & NO & 133\\
$(31,14)$ & 8 & $(10,3)$ & 5 & 1 & YES & YES & YES & $1.38$ & $(4,2)$ & -- & 134\\
$(32,5)$ & 9 & $(10,3)$ & 5 & 2 & YES & YES & YES & $1.14$ & $(6,1)$ & -- & 135\\
$(32,5)$ & 9 & $(10,3)$ & 5 & 2 & YES & YES & YES & $1.29$ & $(6,1)$ & NO & 136\\
$(33,10)$ & 8 & $(31,13)$ & 7 & 1 & YES & NO & YES & $1.86$ & $(4,2)$ & -- & 137\\
$(37,8)$ & 8 & $(12,5)$ & 5 & 1 & YES & YES & YES & $1.38$ & $(4,2)$ & -- & 138\\
$(37,8)$ & 8 & $(15,7)$ & 8 & 1 & YES & YES & YES & $1.43$ & $(6,1)$ & NO & 139\\
$(37,8)$ & 8 & $(20,7)$ & 8 & 1 & YES & YES & YES & $1.70$ & $(2,3)$ & NO & 140\\
$(37,8)$ & 8 & $(22,7)$ & 9 & 1 & YES & YES & YES & $1.70$ & $(2,3)$ & NO & 141\\
$(38,7)$ & 9 & $(15,7)$ & 8 & 1 & YES & YES & YES & $1.50$ & $(4,2)$ & NO & 142\\
$(38,7)$ & 9 & $(22,7)$ & 9 & 2 & YES & YES & YES & $1.50$ & $(4,2)$ & -- & 143\\
$(38,7)$ & 9 & $(22,7)$ & 9 & 2 & YES & YES & YES & $1.70$ & $(2,3)$ & NO & 144\\
$(38,7)$ & 9 & $(27,7)$ & 9 & 1 & YES & YES & YES & $1.70$ & $(2,3)$ & NO & 145\\
$(39,7)$ & 9 & $(22,7)$ & 9 & 1 & YES & YES & YES & $1.38$ & $(4,2)$ & -- & 146\\
$(41,17)$ & 8 & $(16,7)$ & 6 & 1 & YES & YES & YES & $1.50$ & $(4,2)$ & -- & 147\\
$(44,13)$ & 8 & $(13,6)$ & 7 & 1 & YES & YES & YES & $1.70$ & $(2,3)$ & -- & 148\\
$(45,14)$ & 9 & $(10,3)$ & 5 & 5 & YES & YES & YES & $1.60$ & $(2,3)$ & -- & 149\\
$(45,19)$ & 8 & $(16,7)$ & 6 & 1 & YES & YES & YES & $1.38$ & $(4,2)$ & -- & 150\\
$(46,7)$ & 10 & $(15,7)$ & 8 & 1 & YES & YES & YES & $1.50$ & $(4,2)$ & NO & 151\\
$(47,13)$ & 8 & $(13,6)$ & 7 & 1 & YES & YES & YES & $1.70$ & $(2,3)$ & -- & 152\\
$(47,13)$ & 8 & $(17,6)$ & 7 & 1 & YES & YES & YES & $1.70$ & $(2,3)$ & NO & 153\\
$(47,13)$ & 8 & $(22,7)$ & 9 & 1 & YES & YES & YES & $1.70$ & $(2,3)$ & NO & 154\\
$(50,21)$ & 8 & $(11,5)$ & 6 & 1 & YES & YES & YES & $1.43$ & $(6,1)$ & -- & 155\\
$(50,21)$ & 8 & $(15,7)$ & 8 & 5 & YES & YES & YES & $1.43$ & $(6,1)$ & NO & 156\\
$(52,15)$ & 11 & $(17,3)$ & 7 & 1 & YES & YES & YES & $1.60$ & $(2,3)$ & -- & 157\\
$(53,14)$ & 9 & $(12,5)$ & 5 & 1 & YES & YES & YES & $1.56$ & $(2,3)$ & -- & 158\\
$(53,8)$ & 10 & $(20,7)$ & 8 & 1 & YES & YES & YES & $1.50$ & $(4,2)$ & -- & 159\\
$(53,14)$ & 9 & $(23,5)$ & 7 & 1 & YES & YES & YES & $1.44$ & $(2,3)$ & -- & 160\\
$(53,14)$ & 9 & $(23,7)$ & 7 & 1 & YES & YES & YES & $1.56$ & $(2,3)$ & NO & 161\\
$(54,17)$ & 10 & $(7,2)$ & 4 & 1 & YES & YES & YES & $1.29$ & $(6,1)$ & NO & 162\\
$(54,17)$ & 10 & $(13,5)$ & 5 & 1 & YES & YES & YES & $1.29$ & $(6,1)$ & NO & 163\\
$(55,24)$ & 9 & $(12,5)$ & 5 & 1 & YES & YES & YES & $1.38$ & $(4,2)$ & -- & 164\\
$(55,24)$ & 9 & $(15,7)$ & 8 & 5 & YES & YES & YES & $1.50$ & $(4,2)$ & NO & 165\\
$(55,24)$ & 9 & $(24,11)$ & 8 & 1 & YES & YES & YES & $1.38$ & $(4,2)$ & NO & 166\\
$(57,10)$ & 10 & $(5,2)$ & 3 & 1 & YES & YES & YES & $1.56$ & $(4,2)$ & NO & 167\\
$(57,10)$ & 10 & $(5,2)$ & 3 & 1 & YES & YES & YES & $1.56$ & $(4,2)$ & -- & 168\\
$(57,10)$ & 10 & $(9,2)$ & 5 & 3 & YES & YES & YES & $1.56$ & $(4,2)$ & NO & 169\\
$(57,10)$ & 10 & $(9,2)$ & 5 & 3 & YES & YES & YES & $1.56$ & $(4,2)$ & -- & 170\\
$(58,17)$ & 9 & $(22,7)$ & 9 & 2 & YES & YES & YES & $1.70$ & $(2,3)$ & NO & 171\\
$(59,8)$ & 11 & $(15,7)$ & 8 & 1 & YES & YES & YES & $1.43$ & $(6,1)$ & NO & 172\\
$(61,23)$ & 11 & $(13,2)$ & 7 & 1 & YES & YES & YES & $1.50$ & $(4,2)$ & NO & 173\\
$(61,23)$ & 11 & $(13,2)$ & 7 & 1 & YES & YES & YES & $1.70$ & $(2,3)$ & -- & 174\\
$(62,11)$ & 10 & $(9,2)$ & 5 & 1 & YES & YES & YES & $1.14$ & $(6,1)$ & NO & 175\\
$(62,11)$ & 10 & $(9,2)$ & 5 & 1 & YES & YES & YES & $1.14$ & $(6,1)$ & NO & 176\\
$(62,11)$ & 10 & $(9,2)$ & 5 & 1 & YES & YES & YES & $1.14$ & $(6,1)$ & -- & 177\\
$(66,29)$ & 9 & $(12,5)$ & 5 & 6 & YES & YES & YES & $1.38$ & $(4,2)$ & -- & 178\\
$(66,25)$ & 9 & $(61,23)$ & 11 & 1 & YES & YES & YES & $1.50$ & $(4,2)$ & NO & 179\\
$(67,18)$ & 9 & $(53,14)$ & 9 & 1 & YES & YES & YES & $1.44$ & $(2,3)$ & NO & 180\\
$(71,20)$ & 10 & $(15,4)$ & 6 & 1 & YES & YES & YES & $1.70$ & $(2,3)$ & NO & 181\\
$(71,32)$ & 10 & $(15,7)$ & 8 & 1 & YES & YES & YES & $1.50$ & $(4,2)$ & NO & 182\\
$(72,19)$ & 10 & $(7,2)$ & 4 & 1 & YES & YES & YES & $1.70$ & $(2,3)$ & -- & 183\\
$(83,13)$ & 11 & $(3,1)$ & 2 & 1 & YES & YES & YES & $1.14$ & $(6,1)$ & -- & 184\\
$(83,13)$ & 11 & $(3,1)$ & 2 & 1 & YES & YES & YES & $1.29$ & $(6,1)$ & NO & 185\\
$(83,13)$ & 11 & $(13,2)$ & 7 & 1 & YES & YES & YES & $1.14$ & $(6,1)$ & NO & 186\\
$(84,13)$ & 13 & $(7,2)$ & 4 & 7 & YES & YES & YES & $1.50$ & $(4,2)$ & NO & 187\\
$(84,13)$ & 13 & $(53,8)$ & 10 & 1 & YES & YES & YES & $1.50$ & $(4,2)$ & NO & 188\\
$(85,39)$ & 11 & $(5,2)$ & 3 & 5 & YES & YES & YES & $1.50$ & $(4,2)$ & -- & 189\\
$(86,27)$ & 11 & $(2,1)$ & 1 & 2 & YES & YES & YES & $1.50$ & $(4,2)$ & -- & 190\\
$(87,40)$ & 11 & $(5,2)$ & 3 & 1 & YES & YES & YES & $1.50$ & $(4,2)$ & -- & 191\\
$(87,40)$ & 11 & $(20,9)$ & 7 & 1 & YES & YES & YES & $1.38$ & $(4,2)$ & NO & 192\\
$(91,24)$ & 11 & $(7,2)$ & 4 & 7 & YES & YES & YES & $1.56$ & $(2,3)$ & -- & 193\\
$(94,41)$ & 10 & $(7,3)$ & 4 & 1 & YES & YES & YES & $1.38$ & $(4,2)$ & -- & 194\\
$(94,43)$ & 11 & $(15,7)$ & 8 & 1 & YES & YES & YES & $1.43$ & $(6,1)$ & 255 & 195\\
$(97,21)$ & 10 & $(7,3)$ & 4 & 1 & YES & YES & YES & $1.14$ & $(6,1)$ & -- & 196\\
$(97,28)$ & 12 & $(11,2)$ & 6 & 1 & YES & YES & YES & $1.50$ & $(4,2)$ & -- & 197\\
$(97,28)$ & 12 & $(55,16)$ & 9 & 1 & YES & YES & YES & $1.50$ & $(4,2)$ & NO & 198\\
$(97,28)$ & 12 & $(69,20)$ & 10 & 1 & YES & YES & YES & $1.70$ & $(2,3)$ & NO & 199\\
$(100,31)$ & 11 & $(22,7)$ & 9 & 2 & YES & YES & YES & $1.38$ & $(4,2)$ & NO & 200\\
$(100,29)$ & 11 & $(52,15)$ & 11 & 4 & YES & YES & YES & $1.60$ & $(2,3)$ & NO & 201\\
$(101,30)$ & 10 & $(23,7)$ & 7 & 1 & YES & YES & YES & $1.56$ & $(2,3)$ & NO & 202\\
$(103,32)$ & 11 & $(3,1)$ & 2 & 1 & YES & YES & YES & $1.70$ & $(2,3)$ & NO & 203\\
$(103,32)$ & 11 & $(3,1)$ & 2 & 1 & YES & YES & YES & $1.70$ & $(2,3)$ & -- & 204\\
$(103,32)$ & 11 & $(22,7)$ & 9 & 1 & YES & YES & YES & $1.50$ & $(4,2)$ & NO & 205\\
$(106,37)$ & 12 & $(23,8)$ & 9 & 1 & YES & YES & YES & $1.50$ & $(4,2)$ & NO & 206\\
$(110,51)$ & 12 & $(3,1)$ & 2 & 1 & YES & YES & YES & $1.50$ & $(4,2)$ & NO & 207\\
$(111,32)$ & 13 & $(31,9)$ & 8 & 1 & YES & YES & YES & $1.43$ & $(6,1)$ & NO & 208\\
$(111,32)$ & 13 & $(45,13)$ & 10 & 3 & YES & YES & YES & $1.43$ & $(6,1)$ & NO & 209\\
$(111,32)$ & 13 & $(52,15)$ & 11 & 1 & YES & YES & YES & $1.50$ & $(4,2)$ & NO & 210\\
$(113,35)$ & 11 & $(3,1)$ & 2 & 1 & YES & YES & YES & $1.50$ & $(4,2)$ & NO & 211\\
$(113,35)$ & 11 & $(3,1)$ & 2 & 1 & YES & YES & YES & $1.50$ & $(4,2)$ & -- & 212\\
$(113,48)$ & 11 & $(3,1)$ & 2 & 1 & YES & YES & YES & $1.56$ & $(2,3)$ & -- & 213\\
$(113,35)$ & 11 & $(16,5)$ & 7 & 1 & YES & YES & YES & $1.70$ & $(2,3)$ & 225 & 214\\
$(113,32)$ & 13 & $(46,13)$ & 10 & 1 & YES & YES & YES & $1.43$ & $(6,1)$ & NO & 215\\
$(113,32)$ & 13 & $(53,15)$ & 11 & 1 & YES & YES & YES & $1.50$ & $(4,2)$ & NO & 216\\
$(114,53)$ & 12 & $(3,1)$ & 2 & 3 & YES & YES & YES & $1.43$ & $(6,1)$ & NO & 217\\
$(114,53)$ & 12 & $(7,1)$ & 6 & 1 & YES & YES & YES & $1.43$ & $(6,1)$ & NO & 218\\
$(114,53)$ & 12 & $(11,5)$ & 6 & 1 & YES & YES & YES & $1.43$ & $(6,1)$ & 261 & 219\\
$(114,43)$ & 12 & $(29,11)$ & 7 & 1 & YES & YES & YES & $1.50$ & $(4,2)$ & NO & 220\\
$(119,37)$ & 11 & $(2,1)$ & 1 & 1 & YES & YES & YES & $1.70$ & $(2,3)$ & -- & 221\\
$(119,37)$ & 11 & $(2,1)$ & 1 & 1 & YES & YES & YES & $1.70$ & $(2,3)$ & NO & 222\\
$(119,37)$ & 11 & $(3,1)$ & 2 & 1 & YES & YES & YES & $1.43$ & $(6,1)$ & NO & 223\\
$(119,37)$ & 11 & $(3,1)$ & 2 & 1 & YES & YES & YES & $1.43$ & $(6,1)$ & -- & 224\\
$(119,37)$ & 11 & $(13,4)$ & 6 & 1 & YES & YES & YES & $1.70$ & $(2,3)$ & 214 & 225\\
$(120,53)$ & 11 & $(5,2)$ & 3 & 5 & YES & YES & YES & $1.56$ & $(2,3)$ & NO & 226\\
$(123,23)$ & 14 & $(5,2)$ & 3 & 1 & YES & YES & YES & $1.50$ & $(4,2)$ & -- & 227\\
$(123,23)$ & 14 & $(5,2)$ & 3 & 1 & YES & YES & YES & $1.50$ & $(4,2)$ & NO & 228\\
$(123,23)$ & 14 & $(13,2)$ & 7 & 1 & YES & YES & YES & $1.50$ & $(4,2)$ & NO & 229\\
$(123,23)$ & 14 & $(38,7)$ & 9 & 1 & YES & YES & YES & $1.50$ & $(4,2)$ & NO & 230\\
$(124,57)$ & 12 & $(3,1)$ & 2 & 1 & YES & YES & YES & $1.70$ & $(2,3)$ & -- & 231\\
$(124,35)$ & 12 & $(10,3)$ & 5 & 2 & YES & YES & YES & $1.60$ & $(2,3)$ & NO & 232\\
$(127,45)$ & 11 & $(5,2)$ & 3 & 1 & YES & YES & YES & $1.38$ & $(4,2)$ & -- & 233\\
$(127,45)$ & 11 & $(20,7)$ & 8 & 1 & YES & YES & YES & $1.50$ & $(4,2)$ & NO & 234\\
$(130,23)$ & 14 & $(39,7)$ & 9 & 13 & YES & YES & YES & $1.38$ & $(4,2)$ & NO & 235\\
$(134,47)$ & 12 & $(5,2)$ & 3 & 1 & YES & YES & YES & $1.50$ & $(4,2)$ & NO & 236\\
$(135,32)$ & 12 & $(38,9)$ & 9 & 1 & YES & YES & YES & $1.70$ & $(2,3)$ & NO & 237\\
$(137,43)$ & 12 & $(3,1)$ & 2 & 1 & NO & YES & YES & $1.50$ & $(4,2)$ & -- & 238\\
$(137,43)$ & 12 & $(5,2)$ & 3 & 1 & YES & YES & YES & $1.38$ & $(4,2)$ & -- & 239\\
$(137,43)$ & 12 & $(5,2)$ & 3 & 1 & YES & YES & YES & $1.50$ & $(4,2)$ & NO & 240\\
$(137,63)$ & 12 & $(24,11)$ & 8 & 1 & YES & YES & YES & $1.70$ & $(2,3)$ & 263 & 241\\
$(142,65)$ & 12 & $(5,1)$ & 4 & 1 & YES & YES & YES & $1.38$ & $(4,2)$ & -- & 242\\
$(143,45)$ & 12 & $(3,1)$ & 2 & 1 & NO & YES & YES & $1.43$ & $(6,1)$ & -- & 243\\
$(145,51)$ & 12 & $(3,1)$ & 2 & 1 & YES & YES & YES & $1.50$ & $(4,2)$ & NO & 244\\
$(145,51)$ & 12 & $(20,7)$ & 8 & 5 & YES & YES & YES & $1.70$ & $(2,3)$ & 252 & 245\\
$(145,41)$ & 13 & $(145,41)$ & 13 & 145 & YES & YES & YES & $1.70$ & $(2,3)$ & NO & 246\\
$(146,23)$ & 15 & $(5,2)$ & 3 & 1 & YES & YES & YES & $1.38$ & $(4,2)$ & NO & 247\\
$(151,53)$ & 12 & $(2,1)$ & 1 & 1 & YES & YES & YES & $1.50$ & $(4,2)$ & -- & 248\\
$(151,53)$ & 12 & $(3,1)$ & 2 & 1 & YES & YES & YES & $1.29$ & $(6,1)$ & -- & 249\\
$(151,53)$ & 12 & $(3,1)$ & 2 & 1 & YES & YES & YES & $1.43$ & $(6,1)$ & NO & 250\\
$(151,53)$ & 12 & $(5,1)$ & 4 & 1 & YES & YES & YES & $1.50$ & $(4,2)$ & NO & 251\\
$(151,53)$ & 12 & $(17,6)$ & 7 & 1 & YES & YES & YES & $1.70$ & $(2,3)$ & 245 & 252\\
$(152,67)$ & 11 & $(3,1)$ & 2 & 1 & YES & YES & YES & $1.44$ & $(2,3)$ & -- & 253\\
$(153,64)$ & 11 & $(2,1)$ & 1 & 1 & YES & YES & YES & $1.56$ & $(2,3)$ & NO & 254\\
$(153,70)$ & 12 & $(2,1)$ & 1 & 1 & YES & YES & YES & $1.43$ & $(6,1)$ & 195 & 255\\
$(153,40)$ & 12 & $(5,2)$ & 3 & 1 & YES & YES & YES & $1.50$ & $(4,2)$ & NO & 256\\
$(153,40)$ & 12 & $(5,2)$ & 3 & 1 & YES & YES & YES & $1.50$ & $(4,2)$ & -- & 257\\
$(153,40)$ & 12 & $(10,3)$ & 5 & 1 & YES & YES & YES & $1.50$ & $(4,2)$ & NO & 258\\
$(153,64)$ & 11 & $(12,5)$ & 5 & 3 & YES & YES & YES & $1.56$ & $(2,3)$ & NO & 259\\
$(153,70)$ & 12 & $(13,6)$ & 7 & 1 & YES & YES & YES & $1.70$ & $(2,3)$ & NO & 260\\
$(159,73)$ & 12 & $(2,1)$ & 1 & 1 & YES & YES & YES & $1.43$ & $(6,1)$ & 219 & 261\\
$(159,61)$ & 12 & $(5,1)$ & 4 & 1 & YES & YES & YES & $1.60$ & $(2,3)$ & -- & 262\\
$(159,73)$ & 12 & $(13,6)$ & 7 & 1 & YES & YES & YES & $1.70$ & $(2,3)$ & 241 & 263\\
$(161,51)$ & 13 & $(2,1)$ & 1 & 1 & YES & YES & YES & $1.50$ & $(4,2)$ & NO & 264\\
$(161,51)$ & 13 & $(2,1)$ & 1 & 1 & YES & YES & YES & $1.38$ & $(4,2)$ & -- & 265\\
$(161,48)$ & 12 & $(3,1)$ & 2 & 1 & YES & YES & YES & $1.56$ & $(2,3)$ & NO & 266\\
$(161,51)$ & 13 & $(3,1)$ & 2 & 1 & YES & YES & YES & $1.29$ & $(6,1)$ & -- & 267\\
$(163,43)$ & 12 & $(11,3)$ & 5 & 1 & YES & YES & YES & $1.44$ & $(2,3)$ & NO & 268\\
$(163,43)$ & 12 & $(91,24)$ & 11 & 1 & YES & YES & YES & $1.56$ & $(2,3)$ & NO & 269\\
$(167,38)$ & 14 & $(2,1)$ & 1 & 1 & YES & YES & YES & $1.50$ & $(4,2)$ & -- & 270\\
$(173,45)$ & 13 & $(2,1)$ & 1 & 1 & YES & YES & YES & $1.70$ & $(2,3)$ & -- & 271\\
$(173,78)$ & 12 & $(2,1)$ & 1 & 1 & YES & YES & YES & $1.70$ & $(2,3)$ & NO & 272\\
$(173,45)$ & 13 & $(173,45)$ & 13 & 173 & YES & YES & YES & $1.50$ & $(4,2)$ & NO & 273\\
$(175,62)$ & 12 & $(5,2)$ & 3 & 5 & YES & YES & YES & $1.70$ & $(2,3)$ & NO & 274\\
$(177,80)$ & 12 & $(2,1)$ & 1 & 1 & YES & YES & YES & $1.38$ & $(4,2)$ & NO & 275\\
$(177,46)$ & 13 & $(4,1)$ & 3 & 1 & YES & YES & YES & $1.43$ & $(6,1)$ & NO & 276\\
$(177,65)$ & 11 & $(5,2)$ & 3 & 1 & YES & NO & YES & $1.86$ & $(4,2)$ & -- & 277\\
$(177,46)$ & 13 & $(27,7)$ & 9 & 3 & YES & YES & YES & $1.70$ & $(2,3)$ & NO & 278\\
$(181,47)$ & 13 & $(2,1)$ & 1 & 1 & YES & YES & YES & $1.50$ & $(4,2)$ & NO & 279\\
$(181,28)$ & 15 & $(5,2)$ & 3 & 1 & YES & YES & YES & $1.38$ & $(4,2)$ & NO & 280\\
$(181,47)$ & 13 & $(77,20)$ & 11 & 1 & YES & YES & YES & $1.50$ & $(4,2)$ & NO & 281\\
$(191,46)$ & 14 & $(2,1)$ & 1 & 1 & YES & YES & YES & $1.50$ & $(4,2)$ & -- & 282\\
$(207,37)$ & 15 & $(17,3)$ & 7 & 1 & YES & YES & YES & $1.70$ & $(2,3)$ & NO & 283\\
$(207,37)$ & 15 & $(28,5)$ & 8 & 1 & YES & YES & YES & $1.50$ & $(4,2)$ & NO & 284\\
$(209,82)$ & 12 & $(2,1)$ & 1 & 1 & NO & YES & YES & $1.70$ & $(2,3)$ & -- & 285\\
$(213,38)$ & 15 & $(2,1)$ & 1 & 1 & YES & YES & YES & $1.60$ & $(2,3)$ & -- & 286\\
$(239,32)$ & 17 & $(2,1)$ & 1 & 1 & YES & YES & YES & $1.38$ & $(4,2)$ & -- & 287\\
$(239,32)$ & 17 & $(2,1)$ & 1 & 1 & YES & YES & YES & $1.50$ & $(4,2)$ & NO & 288\\
$(239,32)$ & 17 & $(3,1)$ & 2 & 1 & YES & YES & YES & $1.29$ & $(6,1)$ & NO & 289\\
$(241,63)$ & 13 & $(3,1)$ & 2 & 1 & YES & YES & YES & $1.70$ & $(2,3)$ & NO & 290\\
$(243,38)$ & 16 & $(2,1)$ & 1 & 1 & YES & YES & YES & $1.70$ & $(2,3)$ & NO & 291\\
$(243,43)$ & 15 & $(3,1)$ & 2 & 3 & YES & YES & YES & $1.50$ & $(4,2)$ & -- & 292\\
$(243,43)$ & 15 & $(8,1)$ & 7 & 1 & YES & YES & YES & $1.38$ & $(4,2)$ & NO & 293\\
$(243,43)$ & 15 & $(11,2)$ & 6 & 1 & YES & YES & YES & $1.38$ & $(4,2)$ & NO & 294\\
$(243,38)$ & 16 & $(13,2)$ & 7 & 1 & YES & YES & YES & $1.70$ & $(2,3)$ & NO & 295\\
$(243,38)$ & 16 & $(32,5)$ & 9 & 1 & YES & YES & YES & $1.43$ & $(6,1)$ & NO & 296\\
$(271,48)$ & 14 & $(3,1)$ & 2 & 1 & YES & YES & YES & $1.44$ & $(2,3)$ & NO & 297\\
$(286,105)$ & 12 & $(2,1)$ & 1 & 2 & YES & NO & YES & $1.86$ & $(4,2)$ & -- & 298\\
$(a;5,1,1;64)$ & 11 & $(5,1)$ & 4 & 1 & YES & YES & YES & $1.50$ & $(4,2)$ & -- & 299\\
$(b;0,1,5;59)$ & 11 & $(2,1)$ & 1 & 1 & YES & YES & YES & $1.70$ & $(2,3)$ & -- & 300\\
$(b;1,0,5;65)$ & 11 & $(2,1)$ & 1 & 1 & YES & YES & YES & $1.38$ & $(4,2)$ & -- & 301\\
$(c;0,0,0;4)$ & 4 & $(41,19)$ & 10 & 1 & YES & YES & YES & $1.50$ & $(4,2)$ & -- & 302\\
$(c;0,2,0;7)$ & 6 & $(52,11)$ & 9 & 1 & YES & YES & YES & $1.56$ & $(2,3)$ & -- & 303\\
$(c;0,3,0;17)$ & 7 & $(19,8)$ & 6 & 1 & YES & YES & YES & $1.44$ & $(2,3)$ & -- & 304\\
$(e;1,5,0;43)$ & 11 & $(7,1)$ & 6 & 1 & YES & YES & YES & $1.50$ & $(4,2)$ & -- & 305\\
$(h;0,0,0;6)$ & 5 & $(33,14)$ & 8 & 3 & YES & NO & YES & $1.86$ & $(4,2)$ & -- & 306\\
$(h;0,4,0;14)$ & 9 & $(4,1)$ & 3 & 2 & YES & YES & YES & $1.29$ & $(6,1)$ & -- & 307\\
$(j;0,3,0;14)$ & 8 & $(31,7)$ & 8 & 1 & YES & YES & YES & $1.38$ & $(4,2)$ & -- & 308
\end{longtable}
\subsection{2 chains, $K^2 = 4$}
\begin{longtable}{|c|c|c|c|c|c|c|c|c|c|c|c|}
\hline
\multicolumn{12}{|c|}{2 chains, $K^2 = 4$}\\
\hline
$(n,a)$ & Len & $(n,a)$ & Len & GCD & Nef & $\mathbb Q$-ef & Obs 0 & $\overline c_1^2 / \overline c_2$ & $(P,K)$ & WH & Index\\
\hline
\endfirsthead

\hline
$(n,a)$ & Len & $(n,a)$ & Len & GCD & Nef & $\mathbb Q$-ef & Obs 0 & $\overline c_1^2 / \overline c_2$ & $(P,K)$ & WH & Index\\
\hline
\endhead
\hline
\endfoot

$(41,11)$ & 8 & $(23,9)$ & 7 & 1 & YES & YES & YES & $1.86$ & $(4,3)$ & -- & 309\\
$(47,20)$ & 10 & $(29,8)$ & 7 & 1 & YES & YES & YES & $2.11$ & $(2,4)$ & -- & 310\\
$(192,71)$ & 11 & $(9,4)$ & 5 & 3 & YES & YES & YES & $2.00$ & $(6,2)$ & -- & 311\\
$(263,72)$ & 15 & $(113,31)$ & 11 & 1 & YES & YES & YES & $2.00$ & $(4,3)$ & NO & 312\\
$(289,101)$ & 15 & $(6,1)$ & 5 & 1 & YES & YES & YES & $2.00$ & $(4,3)$ & -- & 313\\
$(289,101)$ & 15 & $(14,5)$ & 6 & 1 & YES & YES & YES & $2.00$ & $(4,3)$ & NO & 314\\
$(329,74)$ & 17 & $(169,38)$ & 13 & 1 & YES & YES & YES & $2.00$ & $(4,3)$ & NO & 315\\
$(480,133)$ & 15 & $(8,1)$ & 7 & 8 & YES & YES & YES & $2.11$ & $(2,4)$ & NO & 316\\
$(503,132)$ & 15 & $(7,2)$ & 4 & 1 & YES & YES & YES & $2.11$ & $(2,4)$ & NO & 317\\
$(635,132)$ & 16 & $(3,1)$ & 2 & 1 & YES & YES & YES & $2.11$ & $(2,4)$ & NO & 318\\
$(635,132)$ & 16 & $(4,1)$ & 3 & 1 & YES & YES & YES & $2.00$ & $(4,3)$ & NO & 319\\
$(722,113)$ & 18 & $(8,1)$ & 7 & 2 & YES & YES & YES & $2.11$ & $(2,4)$ & NO & 320\\
$(b;1,3,0;41)$ & 9 & $(13,3)$ & 6 & 1 & YES & YES & YES & $2.00$ & $(4,3)$ & -- & 321\\
$(b;6,0,3;122)$ & 14 & $(8,1)$ & 7 & 2 & YES & YES & YES & $2.00$ & $(4,3)$ & -- & 322\\
$(e;1,3,0;33)$ & 9 & $(13,3)$ & 6 & 1 & YES & YES & YES & $2.00$ & $(4,3)$ & -- & 323
\end{longtable}




%%%%%%%%%%%%%%%%%%%%%%%%%%%%%%%%%%%%%%%%%%%
\section{$IV^* + IV$}
Input:
\lstinputlisting[language=config]{../Tests/IVsIV.txt}
Result:
%\usepackage{longtable}
\subsection{2 chains, $K^2 = 3$}
\begin{longtable}{|c|c|c|c|c|c|c|c|c|c|c|c|}
\hline
\multicolumn{12}{|c|}{2 chains, $K^2 = 3$}\\
\hline
$(n,a)$ & Len & $(n,a)$ & Len & GCD & Nef & $\mathbb Q$-ef & Obs 0 & $\overline c_1^2 / \overline c_2$ & $(P,K)$ & WH & Index\\
\hline
\endfirsthead

\hline
$(n,a)$ & Len & $(n,a)$ & Len & GCD & Nef & $\mathbb Q$-ef & Obs 0 & $\overline c_1^2 / \overline c_2$ & $(P,K)$ & WH & Index\\
\hline
\endhead
\hline
\endfoot

$(68,19)$ & 9 & $(27,8)$ & 7 & 1 & YES & NO & YES & $2.12$ & $(6,1)$ & NO & 1\\
$(68,19)$ & 9 & $(27,8)$ & 7 & 1 & YES & NO & YES & $2.12$ & $(6,1)$ & -- & 2\\
$(71,21)$ & 9 & $(25,7)$ & 7 & 1 & YES & NO & YES & $2.12$ & $(6,1)$ & -- & 3\\
$(76,21)$ & 9 & $(25,7)$ & 7 & 1 & YES & NO & YES & $2.12$ & $(6,1)$ & -- & 4\\
$(257,76)$ & 12 & $(7,2)$ & 4 & 1 & YES & NO & YES & $2.12$ & $(6,1)$ & -- & 5\\
$(289,80)$ & 12 & $(5,2)$ & 3 & 1 & YES & NO & YES & $2.12$ & $(6,1)$ & NO & 6\\
$(301,89)$ & 12 & $(9,2)$ & 5 & 1 & YES & NO & YES & $2.12$ & $(6,1)$ & NO & 7\\
$(301,89)$ & 12 & $(257,76)$ & 12 & 1 & YES & NO & YES & $2.12$ & $(6,1)$ & NO & 8\\
$(322,89)$ & 12 & $(5,2)$ & 3 & 1 & YES & NO & YES & $2.12$ & $(6,1)$ & NO & 9\\
$(322,89)$ & 12 & $(25,7)$ & 7 & 1 & YES & NO & YES & $2.12$ & $(6,1)$ & NO & 10\\
$(338,129)$ & 12 & $(4,1)$ & 3 & 2 & YES & NO & YES & $2.00$ & $(6,1)$ & NO & 11\\
$(338,129)$ & 12 & $(4,1)$ & 3 & 2 & YES & NO & YES & $2.00$ & $(6,1)$ & -- & 12\\
$(383,112)$ & 13 & $(3,1)$ & 2 & 1 & YES & NO & YES & $2.12$ & $(6,1)$ & -- & 13\\
$(389,115)$ & 13 & $(3,1)$ & 2 & 1 & YES & NO & YES & $2.12$ & $(6,1)$ & -- & 14\\
$(401,112)$ & 13 & $(3,1)$ & 2 & 1 & YES & NO & YES & $2.12$ & $(6,1)$ & -- & 15\\
$(403,119)$ & 13 & $(3,1)$ & 2 & 1 & YES & NO & YES & $2.12$ & $(6,1)$ & -- & 16\\
$(403,119)$ & 13 & $(27,8)$ & 7 & 1 & YES & NO & YES & $2.12$ & $(6,1)$ & NO & 17\\
$(413,121)$ & 13 & $(3,1)$ & 2 & 1 & YES & NO & YES & $2.12$ & $(6,1)$ & -- & 18\\
$(416,115)$ & 13 & $(3,1)$ & 2 & 1 & YES & NO & YES & $2.12$ & $(6,1)$ & -- & 19\\
$(416,123)$ & 13 & $(3,1)$ & 2 & 1 & YES & NO & YES & $2.12$ & $(6,1)$ & -- & 20\\
$(416,123)$ & 13 & $(301,89)$ & 12 & 1 & YES & NO & YES & $2.12$ & $(6,1)$ & NO & 21\\
$(433,128)$ & 13 & $(433,128)$ & 13 & 433 & YES & NO & YES & $2.12$ & $(6,1)$ & NO & 22\\
$(434,121)$ & 13 & $(3,1)$ & 2 & 1 & YES & NO & YES & $2.12$ & $(6,1)$ & NO & 23\\
$(434,121)$ & 13 & $(3,1)$ & 2 & 1 & YES & NO & YES & $2.12$ & $(6,1)$ & -- & 24\\
$(434,121)$ & 13 & $(25,7)$ & 7 & 1 & YES & NO & YES & $2.12$ & $(6,1)$ & 32 & 25\\
$(436,129)$ & 13 & $(3,1)$ & 2 & 1 & YES & NO & YES & $2.12$ & $(6,1)$ & -- & 26\\
$(436,129)$ & 13 & $(4,1)$ & 3 & 4 & YES & NO & YES & $2.12$ & $(6,1)$ & -- & 27\\
$(467,129)$ & 13 & $(3,1)$ & 2 & 1 & YES & NO & YES & $2.12$ & $(6,1)$ & -- & 28\\
$(467,129)$ & 13 & $(4,1)$ & 3 & 1 & YES & NO & YES & $2.12$ & $(6,1)$ & -- & 29\\
$(467,129)$ & 13 & $(4,1)$ & 3 & 1 & YES & NO & YES & $2.12$ & $(6,1)$ & NO & 30\\
$(469,131)$ & 13 & $(3,1)$ & 2 & 1 & YES & NO & YES & $2.12$ & $(6,1)$ & NO & 31\\
$(469,131)$ & 13 & $(18,5)$ & 6 & 1 & YES & NO & YES & $2.12$ & $(6,1)$ & 25 & 32\\
$(469,131)$ & 13 & $(68,19)$ & 9 & 1 & YES & NO & YES & $2.12$ & $(6,1)$ & NO & 33\\
$(487,111)$ & 14 & $(35,8)$ & 8 & 1 & YES & NO & YES & $2.12$ & $(6,1)$ & NO & 34\\
$(495,112)$ & 14 & $(3,1)$ & 2 & 3 & YES & NO & YES & $2.12$ & $(6,1)$ & NO & 35\\
$(495,112)$ & 14 & $(13,3)$ & 6 & 1 & YES & NO & YES & $2.12$ & $(6,1)$ & NO & 36\\
$(c;0,0,0;4)$ & 4 & $(144,55)$ & 10 & 4 & YES & NO & YES & $2.12$ & $(6,1)$ & -- & 37\\
$(c;0,0,0;4)$ & 4 & $(159,47)$ & 11 & 1 & YES & NO & YES & $2.12$ & $(6,1)$ & -- & 38\\
$(c;0,0,0;4)$ & 4 & $(171,50)$ & 11 & 1 & YES & NO & YES & $2.12$ & $(6,1)$ & -- & 39
\end{longtable}



%%%%%%%%%%%%%%%%%%%%%%%%%%%%%%%%%%%%%%%%%%%
\section{$IV^* + I_3 + I_1$}
Input:
\lstinputlisting[language=config]{../Tests/IVs31.txt}
Result:
%\usepackage{longtable}
\subsection{1 chain, $K^2 = 2$}
\begin{longtable}{|c|c|c|c|c|c|}
\hline
\multicolumn{6}{|c|}{1 chain, $K^2 = 2$}\\
\hline
$(n,a)$ & Length & Nef & $\mathbb Q$-ef & Obstruction 0 & Index\\
\hline
\endfirsthead

\hline
$(n,a)$ & Length & Nef & $\mathbb Q$-ef & Obstruction 0 & Index\\
\hline
\endhead
\hline
\endfoot

$(65, 19)$ & 9 & YES & YES & YES & 1\\
$(70, 29)$ & 9 & YES & YES & YES & 2\\
$(71, 26)$ & 9 & YES & YES & YES & 3\\
$(79, 24)$ & 10 & YES & YES & YES & 4\\
$(85, 26)$ & 10 & YES & YES & YES & 5
\end{longtable}
\subsection{1 chain, $K^2 = 3$}
\begin{longtable}{|c|c|c|c|c|c|}
\hline
\multicolumn{6}{|c|}{1 chain, $K^2 = 3$}\\
\hline
$(n,a)$ & Length & Nef & $\mathbb Q$-ef & Obstruction 0 & Index\\
\hline
\endfirsthead

\hline
$(n,a)$ & Length & Nef & $\mathbb Q$-ef & Obstruction 0 & Index\\
\hline
\endhead
\hline
\endfoot

$(173, 76)$ & 11 & YES & YES & YES & 6\\
$(191, 80)$ & 11 & YES & YES & YES & 7\\
$(193, 81)$ & 11 & YES & YES & YES & 8\\
$(265, 62)$ & 14 & YES & YES & YES & 9\\
$(267, 98)$ & 12 & YES & YES & YES & 10\\
$(274, 65)$ & 14 & YES & YES & YES & 11\\
$(274, 115)$ & 12 & YES & YES & YES & 12\\
$(307, 65)$ & 13 & YES & YES & YES & 13\\
$(311, 65)$ & 14 & YES & YES & YES & 14\\
$(355, 62)$ & 15 & YES & YES & YES & 15\\
$(367, 112)$ & 13 & YES & YES & YES & 16\\
$(389, 91)$ & 14 & YES & YES & YES & 17
\end{longtable}
\subsection{1 chain, $K^2 = 4$}
\begin{longtable}{|c|c|c|c|c|c|}
\hline
\multicolumn{6}{|c|}{1 chain, $K^2 = 4$}\\
\hline
$(n,a)$ & Length & Nef & $\mathbb Q$-ef & Obstruction 0 & Index\\
\hline
\endfirsthead

\hline
$(n,a)$ & Length & Nef & $\mathbb Q$-ef & Obstruction 0 & Index\\
\hline
\endhead
\hline
\endfoot

$(945, 388)$ & 15 & YES & YES & YES & 18
\end{longtable}
\subsection{2 chains, $K^2 = 1$}
\begin{longtable}{|c|c|c|c|c|c|c|c|c|c|}
\hline
\multicolumn{10}{|c|}{2 chains, $K^2 = 1$}\\
\hline
$(n,a)$ & Length & $(n,a)$ & Length & GCD & Nef & $\mathbb Q$-ef & Obstruction 0 & WH & Index\\
\hline
\endfirsthead

\hline
$(n,a)$ & Length & $(n,a)$ & Length & GCD & Nef & $\mathbb Q$-ef & Obstruction 0 & WH & Index\\
\hline
\endhead
\hline
\endfoot

$(11, 3)$ & 5 & $(7, 3)$ & 4 & 1 & YES & YES & YES & -- & 19\\
$(11, 4)$ & 5 & $(7, 2)$ & 4 & 1 & YES & YES & YES & -- & 20\\
$(12, 5)$ & 5 & $(5, 2)$ & 3 & 1 & YES & YES & YES & -- & 21\\
$(12, 5)$ & 5 & $(7, 2)$ & 4 & 1 & YES & YES & YES & -- & 22\\
$(12, 5)$ & 5 & $(7, 2)$ & 4 & 1 & YES & YES & YES & NO & 23\\
$(13, 5)$ & 5 & $(7, 2)$ & 4 & 1 & YES & YES & YES & NO & 24\\
$(13, 5)$ & 5 & $(11, 4)$ & 5 & 1 & YES & YES & YES & NO & 25\\
$(17, 7)$ & 6 & $(3, 1)$ & 2 & 1 & YES & YES & YES & NO & 26\\
$(17, 7)$ & 6 & $(4, 1)$ & 3 & 1 & YES & YES & YES & -- & 27\\
$(17, 7)$ & 6 & $(4, 1)$ & 3 & 1 & YES & YES & YES & NO & 28\\
$(17, 7)$ & 6 & $(4, 1)$ & 3 & 1 & YES & YES & YES & NO & 29\\
$(17, 7)$ & 6 & $(5, 1)$ & 4 & 1 & YES & YES & YES & NO & 30\\
$(17, 7)$ & 6 & $(5, 1)$ & 4 & 1 & YES & YES & YES & NO & 31\\
$(17, 7)$ & 6 & $(7, 3)$ & 4 & 1 & YES & YES & YES & 45 & 32\\
$(17, 7)$ & 6 & $(12, 5)$ & 5 & 1 & YES & YES & YES & NO & 33\\
$(17, 7)$ & 6 & $(17, 7)$ & 6 & 17 & YES & YES & YES & NO & 34\\
$(18, 7)$ & 6 & $(3, 1)$ & 2 & 3 & YES & YES & YES & -- & 35\\
$(18, 7)$ & 6 & $(3, 1)$ & 2 & 3 & YES & YES & YES & NO & 36\\
$(18, 7)$ & 6 & $(18, 7)$ & 6 & 18 & YES & YES & YES & NO & 37\\
$(19, 7)$ & 6 & $(3, 1)$ & 2 & 1 & YES & YES & YES & -- & 38\\
$(19, 7)$ & 6 & $(3, 1)$ & 2 & 1 & YES & YES & YES & NO & 39\\
$(19, 8)$ & 6 & $(3, 1)$ & 2 & 1 & YES & YES & YES & -- & 40\\
$(19, 8)$ & 6 & $(3, 1)$ & 2 & 1 & YES & YES & YES & 48 & 41\\
$(19, 8)$ & 6 & $(5, 1)$ & 4 & 1 & YES & YES & YES & NO & 42\\
$(19, 8)$ & 6 & $(5, 1)$ & 4 & 1 & YES & YES & YES & NO & 43\\
$(19, 8)$ & 6 & $(5, 2)$ & 3 & 1 & YES & YES & YES & -- & 44\\
$(19, 8)$ & 6 & $(5, 2)$ & 3 & 1 & YES & YES & YES & 32 & 45\\
$(19, 7)$ & 6 & $(19, 7)$ & 6 & 19 & YES & YES & YES & NO & 46\\
$(19, 8)$ & 6 & $(19, 8)$ & 6 & 19 & YES & YES & YES & NO & 47\\
$(21, 8)$ & 6 & $(2, 1)$ & 1 & 1 & YES & YES & YES & 41 & 48\\
$(21, 8)$ & 6 & $(3, 1)$ & 2 & 3 & YES & YES & YES & NO & 49\\
$(24, 7)$ & 7 & $(3, 1)$ & 2 & 3 & YES & YES & YES & NO & 50\\
$(24, 7)$ & 7 & $(7, 2)$ & 4 & 1 & YES & YES & YES & NO & 51\\
$(25, 7)$ & 7 & $(2, 1)$ & 1 & 1 & YES & YES & YES & NO & 52\\
$(25, 7)$ & 7 & $(3, 1)$ & 2 & 1 & YES & YES & YES & NO & 53\\
$(31, 7)$ & 8 & $(2, 1)$ & 1 & 1 & YES & YES & YES & NO & 54\\
$(31, 13)$ & 7 & $(2, 1)$ & 1 & 1 & NO & YES & YES & -- & 55\\
$(31, 7)$ & 8 & $(4, 1)$ & 3 & 1 & YES & YES & YES & NO & 56\\
$(31, 7)$ & 8 & $(31, 7)$ & 8 & 31 & YES & YES & YES & NO & 57\\
$(a; 1, 0, 0; 13)$ & 5 & $(3, 1)$ & 2 & 1 & YES & YES & YES & -- & 58\\
$(a; 1, 0, 0; 13)$ & 5 & $(4, 1)$ & 3 & 1 & YES & YES & YES & -- & 59\\
$(a; 1, 0, 0; 13)$ & 5 & $(5, 2)$ & 3 & 1 & YES & YES & YES & -- & 60\\
$(a; 1, 0, 0; 13)$ & 5 & $(7, 2)$ & 4 & 1 & YES & YES & YES & -- & 61\\
$(a; 1, 1, 0; 19)$ & 6 & $(4, 1)$ & 3 & 1 & YES & YES & YES & -- & 62\\
$(b; 0, 0, 0; 14)$ & 5 & $(2, 1)$ & 1 & 2 & YES & YES & YES & -- & 63\\
$(b; 0, 0, 0; 14)$ & 5 & $(3, 1)$ & 2 & 1 & YES & YES & YES & -- & 64\\
$(b; 0, 0, 0; 14)$ & 5 & $(5, 2)$ & 3 & 1 & YES & YES & YES & -- & 65\\
$(c; 0, 0, 0; 4)$ & 4 & $(7, 3)$ & 4 & 1 & YES & YES & YES & -- & 66\\
$(c; 0, 0, 0; 4)$ & 4 & $(8, 3)$ & 4 & 4 & YES & YES & YES & -- & 67\\
$(c; 0, 1, 1; 5)$ & 6 & $(2, 1)$ & 1 & 1 & YES & YES & YES & -- & 68\\
$(c; 0, 1, 1; 5)$ & 6 & $(4, 1)$ & 3 & 1 & YES & YES & YES & -- & 69\\
$(d; 0, 0, 0; 5)$ & 5 & $(5, 2)$ & 3 & 5 & YES & YES & YES & -- & 70\\
$(d; 0, 0, 0; 5)$ & 5 & $(7, 2)$ & 4 & 1 & YES & YES & YES & -- & 71\\
$(d; 0, 0, 1; 14)$ & 6 & $(2, 1)$ & 1 & 2 & YES & YES & YES & -- & 72\\
$(d; 0, 0, 1; 14)$ & 6 & $(4, 1)$ & 3 & 2 & YES & YES & YES & -- & 73\\
$(e; 0, 0, 0; 4)$ & 5 & $(2, 1)$ & 1 & 2 & YES & YES & YES & -- & 74\\
$(i; 0, 0, 0; 9)$ & 5 & $(3, 1)$ & 2 & 3 & YES & YES & YES & -- & 75\\
$(i; 0, 0, 0; 9)$ & 5 & $(4, 1)$ & 3 & 1 & YES & YES & YES & -- & 76
\end{longtable}
\subsection{2 chains, $K^2 = 2$}
\begin{longtable}{|c|c|c|c|c|c|c|c|c|c|}
\hline
\multicolumn{10}{|c|}{2 chains, $K^2 = 2$}\\
\hline
$(n,a)$ & Length & $(n,a)$ & Length & GCD & Nef & $\mathbb Q$-ef & Obstruction 0 & WH & Index\\
\hline
\endfirsthead

\hline
$(n,a)$ & Length & $(n,a)$ & Length & GCD & Nef & $\mathbb Q$-ef & Obstruction 0 & WH & Index\\
\hline
\endhead
\hline
\endfoot

$(17, 7)$ & 6 & $(13, 5)$ & 5 & 1 & YES & YES & YES & -- & 77\\
$(19, 8)$ & 6 & $(10, 3)$ & 5 & 1 & YES & YES & YES & -- & 78\\
$(19, 8)$ & 6 & $(10, 3)$ & 5 & 1 & YES & YES & YES & NO & 79\\
$(19, 5)$ & 7 & $(18, 5)$ & 6 & 1 & YES & YES & YES & -- & 80\\
$(21, 8)$ & 6 & $(7, 2)$ & 4 & 7 & YES & YES & YES & -- & 81\\
$(21, 8)$ & 6 & $(7, 2)$ & 4 & 7 & YES & YES & YES & NO & 82\\
$(21, 8)$ & 6 & $(9, 2)$ & 5 & 3 & YES & YES & YES & -- & 83\\
$(21, 8)$ & 6 & $(9, 2)$ & 5 & 3 & YES & YES & YES & NO & 84\\
$(21, 8)$ & 6 & $(9, 2)$ & 5 & 3 & YES & YES & YES & NO & 85\\
$(21, 8)$ & 6 & $(10, 3)$ & 5 & 1 & YES & YES & YES & -- & 86\\
$(22, 5)$ & 7 & $(19, 8)$ & 6 & 1 & YES & YES & YES & NO & 87\\
$(23, 7)$ & 7 & $(8, 3)$ & 4 & 1 & YES & YES & YES & -- & 88\\
$(23, 7)$ & 7 & $(8, 3)$ & 4 & 1 & YES & YES & YES & NO & 89\\
$(23, 7)$ & 7 & $(11, 3)$ & 5 & 1 & YES & YES & YES & -- & 90\\
$(23, 7)$ & 7 & $(11, 3)$ & 5 & 1 & YES & YES & YES & NO & 91\\
$(23, 5)$ & 7 & $(16, 5)$ & 7 & 1 & YES & YES & YES & NO & 92\\
$(23, 5)$ & 7 & $(17, 7)$ & 6 & 1 & YES & YES & YES & NO & 93\\
$(24, 7)$ & 7 & $(7, 2)$ & 4 & 1 & YES & YES & YES & -- & 94\\
$(24, 7)$ & 7 & $(7, 2)$ & 4 & 1 & YES & YES & YES & NO & 95\\
$(24, 7)$ & 7 & $(13, 5)$ & 5 & 1 & YES & NO & YES & -- & 96\\
$(25, 7)$ & 7 & $(7, 2)$ & 4 & 1 & YES & YES & YES & -- & 97\\
$(25, 11)$ & 7 & $(8, 3)$ & 4 & 1 & YES & YES & YES & -- & 98\\
$(25, 7)$ & 7 & $(9, 2)$ & 5 & 1 & YES & YES & YES & -- & 99\\
$(25, 7)$ & 7 & $(9, 2)$ & 5 & 1 & YES & YES & YES & NO & 100\\
$(25, 9)$ & 7 & $(11, 3)$ & 5 & 1 & YES & YES & YES & -- & 101\\
$(25, 7)$ & 7 & $(12, 5)$ & 5 & 1 & YES & YES & YES & -- & 102\\
$(25, 11)$ & 7 & $(17, 7)$ & 6 & 1 & YES & YES & YES & NO & 103\\
$(27, 10)$ & 7 & $(7, 2)$ & 4 & 1 & YES & YES & YES & -- & 104\\
$(27, 10)$ & 7 & $(7, 2)$ & 4 & 1 & YES & YES & YES & NO & 105\\
$(27, 8)$ & 7 & $(9, 2)$ & 5 & 9 & YES & YES & YES & -- & 106\\
$(27, 8)$ & 7 & $(9, 2)$ & 5 & 9 & YES & YES & YES & NO & 107\\
$(27, 8)$ & 7 & $(9, 2)$ & 5 & 9 & YES & YES & YES & NO & 108\\
$(27, 8)$ & 7 & $(10, 3)$ & 5 & 1 & YES & YES & YES & -- & 109\\
$(27, 8)$ & 7 & $(10, 3)$ & 5 & 1 & YES & YES & YES & NO & 110\\
$(27, 8)$ & 7 & $(13, 5)$ & 5 & 1 & YES & YES & YES & NO & 111\\
$(27, 5)$ & 8 & $(19, 5)$ & 7 & 1 & YES & YES & YES & -- & 112\\
$(27, 5)$ & 8 & $(19, 5)$ & 7 & 1 & YES & YES & YES & NO & 113\\
$(28, 5)$ & 8 & $(16, 5)$ & 7 & 4 & YES & YES & YES & NO & 114\\
$(28, 5)$ & 8 & $(19, 5)$ & 7 & 1 & YES & YES & YES & NO & 115\\
$(29, 8)$ & 7 & $(5, 2)$ & 3 & 1 & YES & YES & YES & -- & 116\\
$(29, 8)$ & 7 & $(5, 2)$ & 3 & 1 & YES & YES & YES & NO & 117\\
$(29, 8)$ & 7 & $(9, 2)$ & 5 & 1 & YES & YES & YES & -- & 118\\
$(29, 8)$ & 7 & $(9, 2)$ & 5 & 1 & YES & YES & YES & NO & 119\\
$(29, 11)$ & 7 & $(9, 2)$ & 5 & 1 & YES & YES & YES & -- & 120\\
$(29, 11)$ & 7 & $(9, 2)$ & 5 & 1 & YES & YES & YES & NO & 121\\
$(29, 12)$ & 7 & $(10, 3)$ & 5 & 1 & YES & YES & YES & -- & 122\\
$(29, 8)$ & 7 & $(11, 4)$ & 5 & 1 & YES & YES & YES & NO & 123\\
$(29, 8)$ & 7 & $(16, 5)$ & 7 & 1 & YES & YES & YES & NO & 124\\
$(29, 8)$ & 7 & $(22, 5)$ & 7 & 1 & YES & YES & YES & NO & 125\\
$(29, 8)$ & 7 & $(25, 7)$ & 7 & 1 & YES & YES & YES & NO & 126\\
$(30, 11)$ & 7 & $(7, 2)$ & 4 & 1 & YES & YES & YES & -- & 127\\
$(30, 11)$ & 7 & $(7, 2)$ & 4 & 1 & YES & YES & YES & NO & 128\\
$(31, 13)$ & 7 & $(7, 2)$ & 4 & 1 & YES & YES & YES & -- & 129\\
$(31, 13)$ & 7 & $(7, 2)$ & 4 & 1 & YES & YES & YES & NO & 130\\
$(31, 9)$ & 8 & $(17, 3)$ & 7 & 1 & YES & YES & YES & NO & 131\\
$(32, 9)$ & 8 & $(17, 3)$ & 7 & 1 & YES & YES & YES & NO & 132\\
$(34, 13)$ & 7 & $(4, 1)$ & 3 & 2 & YES & YES & YES & NO & 133\\
$(34, 13)$ & 7 & $(5, 1)$ & 4 & 1 & YES & YES & YES & -- & 134\\
$(34, 13)$ & 7 & $(5, 1)$ & 4 & 1 & YES & YES & YES & NO & 135\\
$(34, 13)$ & 7 & $(5, 1)$ & 4 & 1 & YES & YES & YES & NO & 136\\
$(34, 13)$ & 7 & $(5, 2)$ & 3 & 1 & YES & YES & YES & -- & 137\\
$(34, 13)$ & 7 & $(7, 2)$ & 4 & 1 & YES & YES & YES & -- & 138\\
$(34, 13)$ & 7 & $(7, 2)$ & 4 & 1 & YES & YES & YES & NO & 139\\
$(34, 13)$ & 7 & $(9, 2)$ & 5 & 1 & YES & YES & YES & -- & 140\\
$(34, 13)$ & 7 & $(29, 11)$ & 7 & 1 & YES & YES & YES & NO & 141\\
$(35, 13)$ & 8 & $(5, 1)$ & 4 & 5 & YES & YES & YES & -- & 142\\
$(35, 13)$ & 8 & $(5, 1)$ & 4 & 5 & YES & YES & YES & NO & 143\\
$(35, 13)$ & 8 & $(5, 1)$ & 4 & 5 & YES & YES & YES & 208 & 144\\
$(35, 8)$ & 8 & $(9, 4)$ & 5 & 1 & YES & YES & YES & NO & 145\\
$(37, 11)$ & 8 & $(2, 1)$ & 1 & 1 & YES & YES & YES & -- & 146\\
$(37, 11)$ & 8 & $(3, 1)$ & 2 & 1 & YES & YES & YES & -- & 147\\
$(37, 11)$ & 8 & $(4, 1)$ & 3 & 1 & YES & YES & YES & -- & 148\\
$(37, 11)$ & 8 & $(4, 1)$ & 3 & 1 & YES & YES & YES & NO & 149\\
$(37, 11)$ & 8 & $(7, 2)$ & 4 & 1 & YES & YES & YES & NO & 150\\
$(37, 8)$ & 8 & $(11, 4)$ & 5 & 1 & YES & YES & YES & NO & 151\\
$(37, 11)$ & 8 & $(11, 2)$ & 6 & 1 & YES & YES & YES & -- & 152\\
$(37, 11)$ & 8 & $(11, 2)$ & 6 & 1 & YES & YES & YES & NO & 153\\
$(39, 14)$ & 8 & $(7, 2)$ & 4 & 1 & YES & YES & YES & -- & 154\\
$(39, 14)$ & 8 & $(13, 5)$ & 5 & 13 & YES & YES & YES & 229 & 155\\
$(40, 11)$ & 8 & $(2, 1)$ & 1 & 2 & YES & YES & YES & -- & 156\\
$(40, 11)$ & 8 & $(2, 1)$ & 1 & 2 & YES & YES & YES & NO & 157\\
$(40, 11)$ & 8 & $(4, 1)$ & 3 & 4 & YES & YES & YES & -- & 158\\
$(40, 11)$ & 8 & $(4, 1)$ & 3 & 4 & YES & YES & YES & NO & 159\\
$(40, 11)$ & 8 & $(7, 2)$ & 4 & 1 & YES & YES & YES & NO & 160\\
$(40, 11)$ & 8 & $(13, 3)$ & 6 & 1 & YES & YES & YES & NO & 161\\
$(40, 11)$ & 8 & $(29, 8)$ & 7 & 1 & YES & YES & YES & NO & 162\\
$(41, 12)$ & 8 & $(4, 1)$ & 3 & 1 & YES & YES & YES & -- & 163\\
$(41, 12)$ & 8 & $(4, 1)$ & 3 & 1 & YES & YES & YES & NO & 164\\
$(41, 16)$ & 8 & $(4, 1)$ & 3 & 1 & YES & YES & YES & NO & 165\\
$(41, 12)$ & 8 & $(24, 7)$ & 7 & 1 & YES & YES & YES & NO & 166\\
$(43, 12)$ & 8 & $(25, 7)$ & 7 & 1 & YES & YES & YES & NO & 167\\
$(44, 17)$ & 8 & $(4, 1)$ & 3 & 4 & YES & YES & YES & -- & 168\\
$(44, 13)$ & 8 & $(5, 2)$ & 3 & 1 & YES & YES & YES & -- & 169\\
$(44, 13)$ & 8 & $(5, 2)$ & 3 & 1 & YES & YES & YES & NO & 170\\
$(44, 13)$ & 8 & $(8, 3)$ & 4 & 4 & YES & YES & YES & NO & 171\\
$(44, 13)$ & 8 & $(16, 5)$ & 7 & 4 & YES & YES & YES & NO & 172\\
$(45, 19)$ & 8 & $(5, 2)$ & 3 & 5 & YES & YES & YES & -- & 173\\
$(45, 19)$ & 8 & $(5, 2)$ & 3 & 5 & YES & YES & YES & NO & 174\\
$(45, 19)$ & 8 & $(31, 13)$ & 7 & 1 & YES & YES & YES & NO & 175\\
$(46, 17)$ & 8 & $(3, 1)$ & 2 & 1 & YES & YES & YES & -- & 176\\
$(46, 17)$ & 8 & $(3, 1)$ & 2 & 1 & YES & YES & YES & NO & 177\\
$(46, 19)$ & 8 & $(4, 1)$ & 3 & 2 & YES & YES & YES & -- & 178\\
$(46, 19)$ & 8 & $(4, 1)$ & 3 & 2 & YES & YES & YES & NO & 179\\
$(46, 19)$ & 8 & $(7, 2)$ & 4 & 1 & YES & YES & YES & -- & 180\\
$(46, 19)$ & 8 & $(9, 4)$ & 5 & 1 & YES & YES & YES & 255 & 181\\
$(46, 17)$ & 8 & $(27, 10)$ & 7 & 1 & YES & YES & YES & NO & 182\\
$(47, 13)$ & 8 & $(8, 3)$ & 4 & 1 & YES & YES & YES & -- & 183\\
$(47, 13)$ & 8 & $(8, 3)$ & 4 & 1 & YES & YES & YES & NO & 184\\
$(47, 13)$ & 8 & $(19, 5)$ & 7 & 1 & YES & YES & YES & NO & 185\\
$(47, 13)$ & 8 & $(32, 9)$ & 8 & 1 & YES & YES & YES & NO & 186\\
$(47, 18)$ & 8 & $(34, 13)$ & 7 & 1 & YES & YES & YES & NO & 187\\
$(49, 18)$ & 8 & $(3, 1)$ & 2 & 1 & YES & YES & YES & -- & 188\\
$(49, 18)$ & 8 & $(3, 1)$ & 2 & 1 & YES & YES & YES & NO & 189\\
$(49, 18)$ & 8 & $(4, 1)$ & 3 & 1 & YES & YES & YES & -- & 190\\
$(49, 18)$ & 8 & $(4, 1)$ & 3 & 1 & YES & YES & YES & NO & 191\\
$(49, 19)$ & 8 & $(4, 1)$ & 3 & 1 & YES & YES & YES & -- & 192\\
$(49, 19)$ & 8 & $(4, 1)$ & 3 & 1 & YES & YES & YES & NO & 193\\
$(49, 15)$ & 9 & $(5, 1)$ & 4 & 1 & YES & YES & YES & -- & 194\\
$(49, 15)$ & 9 & $(5, 1)$ & 4 & 1 & YES & YES & YES & NO & 195\\
$(49, 15)$ & 9 & $(23, 7)$ & 7 & 1 & YES & YES & YES & 217 & 196\\
$(50, 19)$ & 8 & $(3, 1)$ & 2 & 1 & YES & YES & YES & -- & 197\\
$(50, 19)$ & 8 & $(4, 1)$ & 3 & 2 & YES & YES & YES & -- & 198\\
$(50, 19)$ & 8 & $(4, 1)$ & 3 & 2 & YES & YES & YES & NO & 199\\
$(50, 19)$ & 8 & $(29, 11)$ & 7 & 1 & YES & YES & YES & NO & 200\\
$(50, 19)$ & 8 & $(50, 19)$ & 8 & 50 & YES & YES & YES & NO & 201\\
$(51, 14)$ & 9 & $(4, 1)$ & 3 & 1 & YES & YES & YES & -- & 202\\
$(51, 14)$ & 9 & $(4, 1)$ & 3 & 1 & YES & YES & YES & NO & 203\\
$(51, 20)$ & 9 & $(18, 7)$ & 6 & 3 & YES & YES & YES & NO & 204\\
$(53, 19)$ & 9 & $(5, 1)$ & 4 & 1 & YES & YES & YES & -- & 205\\
$(53, 19)$ & 9 & $(5, 1)$ & 4 & 1 & YES & YES & YES & NO & 206\\
$(53, 19)$ & 9 & $(25, 9)$ & 7 & 1 & YES & YES & YES & 232 & 207\\
$(56, 13)$ & 10 & $(2, 1)$ & 1 & 2 & YES & YES & YES & 144 & 208\\
$(56, 17)$ & 9 & $(2, 1)$ & 1 & 2 & YES & YES & YES & NO & 209\\
$(56, 17)$ & 9 & $(4, 1)$ & 3 & 4 & YES & YES & YES & NO & 210\\
$(56, 17)$ & 9 & $(23, 7)$ & 7 & 1 & YES & YES & YES & NO & 211\\
$(59, 18)$ & 9 & $(2, 1)$ & 1 & 1 & YES & YES & YES & -- & 212\\
$(59, 18)$ & 9 & $(2, 1)$ & 1 & 1 & YES & YES & YES & NO & 213\\
$(59, 18)$ & 9 & $(5, 1)$ & 4 & 1 & YES & YES & YES & -- & 214\\
$(59, 26)$ & 9 & $(5, 2)$ & 3 & 1 & YES & YES & YES & NO & 215\\
$(59, 18)$ & 9 & $(10, 3)$ & 5 & 1 & YES & YES & YES & NO & 216\\
$(59, 18)$ & 9 & $(13, 4)$ & 6 & 1 & YES & YES & YES & 196 & 217\\
$(62, 17)$ & 10 & $(3, 1)$ & 2 & 1 & YES & YES & YES & -- & 218\\
$(62, 17)$ & 10 & $(3, 1)$ & 2 & 1 & YES & YES & YES & NO & 219\\
$(62, 17)$ & 10 & $(5, 1)$ & 4 & 1 & YES & YES & YES & NO & 220\\
$(62, 17)$ & 10 & $(18, 5)$ & 6 & 2 & YES & YES & YES & NO & 221\\
$(62, 17)$ & 10 & $(40, 11)$ & 8 & 2 & YES & YES & YES & 274 & 222\\
$(63, 26)$ & 9 & $(7, 3)$ & 4 & 7 & YES & YES & YES & NO & 223\\
$(63, 26)$ & 9 & $(29, 12)$ & 7 & 1 & YES & YES & YES & 256 & 224\\
$(64, 23)$ & 9 & $(2, 1)$ & 1 & 2 & YES & YES & YES & -- & 225\\
$(64, 27)$ & 9 & $(3, 1)$ & 2 & 1 & YES & YES & YES & -- & 226\\
$(64, 23)$ & 9 & $(4, 1)$ & 3 & 4 & YES & YES & YES & NO & 227\\
$(64, 23)$ & 9 & $(5, 1)$ & 4 & 1 & YES & YES & YES & NO & 228\\
$(64, 23)$ & 9 & $(5, 2)$ & 3 & 1 & YES & YES & YES & 155 & 229\\
$(64, 23)$ & 9 & $(11, 4)$ & 5 & 1 & YES & YES & YES & NO & 230\\
$(64, 27)$ & 9 & $(12, 5)$ & 5 & 4 & YES & YES & YES & NO & 231\\
$(64, 23)$ & 9 & $(14, 5)$ & 6 & 2 & YES & YES & YES & 207 & 232\\
$(64, 23)$ & 9 & $(39, 14)$ & 8 & 1 & YES & YES & YES & NO & 233\\
$(65, 14)$ & 10 & $(4, 1)$ & 3 & 1 & YES & YES & YES & -- & 234\\
$(65, 19)$ & 9 & $(7, 2)$ & 4 & 1 & YES & YES & YES & NO & 235\\
$(69, 19)$ & 9 & $(4, 1)$ & 3 & 1 & YES & YES & YES & -- & 236\\
$(69, 19)$ & 9 & $(4, 1)$ & 3 & 1 & YES & YES & YES & NO & 237\\
$(71, 27)$ & 9 & $(3, 1)$ & 2 & 1 & YES & NO & YES & -- & 238\\
$(71, 20)$ & 10 & $(11, 3)$ & 5 & 1 & YES & YES & YES & NO & 239\\
$(71, 27)$ & 9 & $(13, 5)$ & 5 & 1 & YES & NO & YES & NO & 240\\
$(71, 20)$ & 10 & $(25, 7)$ & 7 & 1 & YES & YES & YES & NO & 241\\
$(72, 19)$ & 10 & $(5, 1)$ & 4 & 1 & YES & YES & YES & -- & 242\\
$(72, 19)$ & 10 & $(5, 1)$ & 4 & 1 & YES & YES & YES & NO & 243\\
$(72, 19)$ & 10 & $(15, 4)$ & 6 & 3 & YES & YES & YES & NO & 244\\
$(72, 19)$ & 10 & $(34, 9)$ & 8 & 2 & YES & YES & YES & 269 & 245\\
$(73, 27)$ & 9 & $(11, 4)$ & 5 & 1 & YES & YES & YES & 262 & 246\\
$(74, 23)$ & 10 & $(2, 1)$ & 1 & 2 & YES & YES & YES & -- & 247\\
$(74, 23)$ & 10 & $(2, 1)$ & 1 & 2 & YES & YES & YES & NO & 248\\
$(74, 31)$ & 9 & $(2, 1)$ & 1 & 2 & YES & YES & YES & -- & 249\\
$(74, 31)$ & 9 & $(2, 1)$ & 1 & 2 & YES & YES & YES & NO & 250\\
$(74, 31)$ & 9 & $(3, 1)$ & 2 & 1 & YES & YES & YES & -- & 251\\
$(74, 23)$ & 10 & $(5, 1)$ & 4 & 1 & YES & YES & YES & NO & 252\\
$(74, 31)$ & 9 & $(19, 8)$ & 6 & 1 & YES & YES & YES & NO & 253\\
$(74, 23)$ & 10 & $(29, 9)$ & 8 & 1 & YES & YES & YES & NO & 254\\
$(75, 31)$ & 9 & $(2, 1)$ & 1 & 1 & YES & YES & YES & 181 & 255\\
$(75, 31)$ & 9 & $(17, 7)$ & 6 & 1 & YES & YES & YES & 224 & 256\\
$(76, 29)$ & 9 & $(2, 1)$ & 1 & 2 & NO & YES & YES & -- & 257\\
$(76, 29)$ & 9 & $(3, 1)$ & 2 & 1 & YES & NO & YES & -- & 258\\
$(77, 18)$ & 10 & $(4, 1)$ & 3 & 1 & YES & YES & YES & -- & 259\\
$(79, 17)$ & 11 & $(3, 1)$ & 2 & 1 & YES & YES & YES & NO & 260\\
$(79, 29)$ & 9 & $(3, 1)$ & 2 & 1 & YES & YES & YES & NO & 261\\
$(79, 29)$ & 9 & $(8, 3)$ & 4 & 1 & YES & YES & YES & 246 & 262\\
$(79, 29)$ & 9 & $(11, 4)$ & 5 & 1 & YES & YES & YES & NO & 263\\
$(83, 22)$ & 10 & $(19, 5)$ & 7 & 1 & YES & YES & YES & NO & 264\\
$(84, 25)$ & 10 & $(5, 1)$ & 4 & 1 & YES & YES & YES & NO & 265\\
$(86, 25)$ & 10 & $(7, 2)$ & 4 & 1 & YES & YES & YES & NO & 266\\
$(86, 25)$ & 10 & $(31, 9)$ & 8 & 1 & YES & YES & YES & NO & 267\\
$(87, 23)$ & 10 & $(2, 1)$ & 1 & 1 & YES & YES & YES & NO & 268\\
$(87, 23)$ & 10 & $(19, 5)$ & 7 & 1 & YES & YES & YES & 245 & 269\\
$(89, 27)$ & 10 & $(2, 1)$ & 1 & 1 & YES & YES & YES & -- & 270\\
$(89, 27)$ & 10 & $(7, 2)$ & 4 & 1 & YES & YES & YES & NO & 271\\
$(91, 25)$ & 10 & $(2, 1)$ & 1 & 1 & YES & YES & YES & -- & 272\\
$(91, 17)$ & 12 & $(7, 1)$ & 6 & 7 & YES & YES & YES & NO & 273\\
$(91, 25)$ & 10 & $(11, 3)$ & 5 & 1 & YES & YES & YES & 222 & 274\\
$(91, 17)$ & 12 & $(27, 5)$ & 8 & 1 & YES & YES & YES & NO & 275\\
$(96, 17)$ & 12 & $(2, 1)$ & 1 & 2 & YES & YES & YES & -- & 276\\
$(96, 17)$ & 12 & $(2, 1)$ & 1 & 2 & YES & YES & YES & NO & 277\\
$(96, 17)$ & 12 & $(7, 1)$ & 6 & 1 & YES & YES & YES & NO & 278\\
$(97, 26)$ & 10 & $(3, 1)$ & 2 & 1 & YES & YES & YES & -- & 279\\
$(97, 26)$ & 10 & $(3, 1)$ & 2 & 1 & YES & YES & YES & NO & 280\\
$(97, 26)$ & 10 & $(7, 2)$ & 4 & 1 & YES & YES & YES & NO & 281\\
$(100, 39)$ & 10 & $(2, 1)$ & 1 & 2 & NO & YES & YES & -- & 282\\
$(105, 41)$ & 10 & $(2, 1)$ & 1 & 1 & NO & YES & YES & -- & 283\\
$(109, 30)$ & 10 & $(2, 1)$ & 1 & 1 & YES & YES & YES & -- & 284\\
$(109, 30)$ & 10 & $(4, 1)$ & 3 & 1 & YES & YES & YES & NO & 285\\
$(116, 43)$ & 11 & $(2, 1)$ & 1 & 2 & NO & YES & YES & -- & 286\\
$(a; 0, 0, 0; 3)$ & 4 & $(19, 8)$ & 6 & 1 & YES & YES & YES & -- & 287\\
$(a; 1, 0, 0; 13)$ & 5 & $(17, 5)$ & 6 & 1 & YES & YES & YES & -- & 288\\
$(a; 1, 1, 0; 19)$ & 6 & $(7, 2)$ & 4 & 1 & YES & YES & YES & -- & 289\\
$(b; 0, 0, 1; 4)$ & 6 & $(9, 4)$ & 5 & 1 & YES & YES & YES & -- & 290\\
$(b; 0, 0, 2; 26)$ & 7 & $(4, 1)$ & 3 & 2 & YES & YES & YES & -- & 291\\
$(b; 0, 0, 3; 32)$ & 8 & $(2, 1)$ & 1 & 2 & YES & YES & YES & -- & 292\\
$(b; 0, 0, 3; 32)$ & 8 & $(3, 1)$ & 2 & 1 & YES & YES & YES & -- & 293\\
$(b; 0, 1, 0; 19)$ & 6 & $(7, 3)$ & 4 & 1 & YES & YES & YES & -- & 294\\
$(b; 0, 2, 0; 8)$ & 7 & $(4, 1)$ & 3 & 4 & YES & YES & YES & -- & 295\\
$(b; 0, 3, 0; 29)$ & 8 & $(5, 1)$ & 4 & 1 & YES & YES & YES & -- & 296\\
$(b; 3, 0, 1; 47)$ & 9 & $(6, 1)$ & 5 & 1 & YES & YES & YES & -- & 297\\
$(c; 0, 0, 0; 4)$ & 4 & $(26, 11)$ & 7 & 2 & YES & YES & YES & -- & 298\\
$(c; 0, 0, 0; 4)$ & 4 & $(30, 11)$ & 7 & 2 & YES & YES & YES & -- & 299\\
$(c; 0, 1, 0; 11)$ & 5 & $(11, 3)$ & 5 & 11 & YES & YES & YES & -- & 300\\
$(c; 0, 1, 0; 11)$ & 5 & $(18, 7)$ & 6 & 1 & YES & YES & YES & -- & 301\\
$(c; 0, 1, 1; 5)$ & 6 & $(7, 2)$ & 4 & 1 & YES & YES & YES & -- & 302\\
$(d; 0, 0, 0; 5)$ & 5 & $(11, 3)$ & 5 & 1 & YES & YES & YES & -- & 303\\
$(d; 0, 0, 0; 5)$ & 5 & $(16, 7)$ & 6 & 1 & YES & YES & YES & -- & 304\\
$(d; 0, 0, 0; 5)$ & 5 & $(18, 7)$ & 6 & 1 & YES & NO & YES & -- & 305\\
$(d; 0, 0, 0; 5)$ & 5 & $(26, 7)$ & 7 & 1 & YES & YES & YES & -- & 306\\
$(d; 0, 0, 1; 14)$ & 6 & $(7, 2)$ & 4 & 7 & YES & YES & YES & -- & 307\\
$(d; 0, 0, 2; 9)$ & 7 & $(7, 2)$ & 4 & 1 & YES & YES & YES & -- & 308\\
$(d; 0, 1, 0; 6)$ & 6 & $(10, 3)$ & 5 & 2 & YES & YES & YES & -- & 309\\
$(d; 0, 1, 1; 17)$ & 7 & $(7, 2)$ & 4 & 1 & YES & YES & YES & -- & 310\\
$(e; 2, 0, 0; 24)$ & 7 & $(4, 1)$ & 3 & 4 & YES & YES & YES & -- & 311\\
$(e; 3, 0, 0; 10)$ & 8 & $(2, 1)$ & 1 & 2 & YES & YES & YES & -- & 312\\
$(e; 3, 0, 0; 10)$ & 8 & $(3, 1)$ & 2 & 1 & YES & YES & YES & -- & 313\\
$(f; 0, 0, 0; 6)$ & 4 & $(37, 11)$ & 8 & 1 & YES & YES & YES & -- & 314\\
$(g; 0, 2, 0; 29)$ & 8 & $(3, 1)$ & 2 & 1 & YES & YES & YES & -- & 315\\
$(h; 0, 3, 0; 12)$ & 8 & $(2, 1)$ & 1 & 2 & YES & YES & YES & -- & 316\\
$(h; 0, 3, 0; 12)$ & 8 & $(5, 1)$ & 4 & 1 & YES & YES & YES & -- & 317\\
$(i; 0, 0, 0; 9)$ & 5 & $(17, 5)$ & 6 & 1 & YES & YES & YES & -- & 318\\
$(i; 0, 1, 0; 12)$ & 6 & $(7, 2)$ & 4 & 1 & YES & YES & YES & -- & 319
\end{longtable}
\subsection{2 chains, $K^2 = 3$}
\begin{longtable}{|c|c|c|c|c|c|c|c|c|c|}
\hline
\multicolumn{10}{|c|}{2 chains, $K^2 = 3$}\\
\hline
$(n,a)$ & Length & $(n,a)$ & Length & GCD & Nef & $\mathbb Q$-ef & Obstruction 0 & WH & Index\\
\hline
\endfirsthead

\hline
$(n,a)$ & Length & $(n,a)$ & Length & GCD & Nef & $\mathbb Q$-ef & Obstruction 0 & WH & Index\\
\hline
\endhead
\hline
\endfoot

$(25, 7)$ & 7 & $(19, 5)$ & 7 & 1 & YES & YES & YES & -- & 320\\
$(29, 8)$ & 7 & $(16, 7)$ & 6 & 1 & YES & YES & YES & -- & 321\\
$(29, 8)$ & 7 & $(16, 7)$ & 6 & 1 & YES & YES & YES & NO & 322\\
$(29, 8)$ & 7 & $(24, 11)$ & 8 & 1 & YES & YES & YES & -- & 323\\
$(31, 7)$ & 8 & $(24, 11)$ & 8 & 1 & YES & YES & YES & -- & 324\\
$(31, 13)$ & 7 & $(27, 8)$ & 7 & 1 & YES & YES & YES & -- & 325\\
$(31, 13)$ & 7 & $(27, 8)$ & 7 & 1 & YES & YES & YES & NO & 326\\
$(33, 7)$ & 8 & $(13, 5)$ & 5 & 1 & YES & YES & YES & -- & 327\\
$(33, 7)$ & 8 & $(13, 5)$ & 5 & 1 & YES & YES & YES & NO & 328\\
$(33, 14)$ & 8 & $(18, 5)$ & 6 & 3 & YES & YES & YES & -- & 329\\
$(33, 14)$ & 8 & $(23, 5)$ & 7 & 1 & YES & YES & YES & NO & 330\\
$(33, 14)$ & 8 & $(27, 5)$ & 8 & 3 & YES & YES & YES & NO & 331\\
$(35, 8)$ & 8 & $(24, 11)$ & 8 & 1 & YES & YES & YES & NO & 332\\
$(35, 8)$ & 8 & $(31, 14)$ & 8 & 1 & YES & YES & YES & NO & 333\\
$(36, 11)$ & 8 & $(19, 8)$ & 6 & 1 & YES & YES & YES & -- & 334\\
$(37, 13)$ & 9 & $(18, 5)$ & 6 & 1 & YES & YES & YES & -- & 335\\
$(37, 8)$ & 8 & $(24, 11)$ & 8 & 1 & YES & YES & YES & NO & 336\\
$(37, 13)$ & 9 & $(27, 5)$ & 8 & 1 & YES & YES & YES & NO & 337\\
$(37, 13)$ & 9 & $(34, 13)$ & 7 & 1 & YES & YES & YES & NO & 338\\
$(38, 7)$ & 9 & $(27, 7)$ & 9 & 1 & YES & YES & YES & -- & 339\\
$(39, 7)$ & 9 & $(27, 7)$ & 9 & 3 & YES & YES & YES & -- & 340\\
$(41, 12)$ & 8 & $(13, 4)$ & 6 & 1 & YES & YES & YES & NO & 341\\
$(41, 17)$ & 8 & $(16, 7)$ & 6 & 1 & YES & YES & YES & -- & 342\\
$(41, 17)$ & 8 & $(17, 5)$ & 6 & 1 & YES & NO & YES & -- & 343\\
$(41, 17)$ & 8 & $(17, 5)$ & 6 & 1 & YES & NO & YES & NO & 344\\
$(41, 12)$ & 8 & $(34, 13)$ & 7 & 1 & YES & NO & YES & -- & 345\\
$(42, 19)$ & 9 & $(17, 5)$ & 6 & 1 & YES & YES & YES & -- & 346\\
$(42, 19)$ & 9 & $(22, 5)$ & 7 & 2 & YES & YES & YES & NO & 347\\
$(43, 8)$ & 9 & $(20, 7)$ & 8 & 1 & YES & YES & YES & NO & 348\\
$(43, 8)$ & 9 & $(22, 7)$ & 9 & 1 & YES & YES & YES & NO & 349\\
$(43, 8)$ & 9 & $(24, 11)$ & 8 & 1 & YES & YES & YES & NO & 350\\
$(43, 18)$ & 8 & $(24, 11)$ & 8 & 1 & YES & YES & YES & NO & 351\\
$(43, 8)$ & 9 & $(27, 7)$ & 9 & 1 & YES & YES & YES & -- & 352\\
$(43, 12)$ & 8 & $(34, 13)$ & 7 & 1 & YES & NO & YES & -- & 353\\
$(43, 8)$ & 9 & $(42, 11)$ & 9 & 1 & YES & YES & YES & NO & 354\\
$(45, 8)$ & 9 & $(22, 7)$ & 9 & 1 & YES & YES & YES & NO & 355\\
$(45, 16)$ & 9 & $(22, 5)$ & 7 & 1 & YES & YES & YES & -- & 356\\
$(45, 8)$ & 9 & $(27, 7)$ & 9 & 9 & YES & YES & YES & -- & 357\\
$(45, 8)$ & 9 & $(27, 7)$ & 9 & 9 & YES & YES & YES & NO & 358\\
$(45, 8)$ & 9 & $(29, 7)$ & 10 & 1 & YES & YES & YES & NO & 359\\
$(46, 19)$ & 8 & $(24, 11)$ & 8 & 2 & YES & YES & YES & NO & 360\\
$(47, 18)$ & 8 & $(21, 8)$ & 6 & 1 & YES & NO & YES & -- & 361\\
$(47, 18)$ & 8 & $(21, 8)$ & 6 & 1 & YES & NO & YES & NO & 362\\
$(49, 13)$ & 9 & $(19, 8)$ & 6 & 1 & YES & YES & YES & -- & 363\\
$(49, 19)$ & 8 & $(24, 7)$ & 7 & 1 & YES & NO & YES & -- & 364\\
$(50, 21)$ & 8 & $(10, 3)$ & 5 & 10 & YES & YES & YES & -- & 365\\
$(50, 21)$ & 8 & $(10, 3)$ & 5 & 10 & YES & YES & YES & NO & 366\\
$(50, 13)$ & 10 & $(13, 5)$ & 5 & 1 & YES & YES & YES & -- & 367\\
$(50, 21)$ & 8 & $(24, 7)$ & 7 & 2 & YES & NO & YES & -- & 368\\
$(50, 21)$ & 8 & $(24, 7)$ & 7 & 2 & YES & NO & YES & NO & 369\\
$(50, 21)$ & 8 & $(24, 11)$ & 8 & 2 & YES & YES & YES & NO & 370\\
$(52, 15)$ & 11 & $(13, 3)$ & 6 & 13 & YES & YES & YES & -- & 371\\
$(52, 15)$ & 11 & $(14, 3)$ & 6 & 2 & YES & YES & YES & -- & 372\\
$(53, 12)$ & 9 & $(13, 6)$ & 7 & 1 & YES & YES & YES & -- & 373\\
$(53, 12)$ & 9 & $(13, 6)$ & 7 & 1 & YES & YES & YES & NO & 374\\
$(53, 15)$ & 11 & $(13, 3)$ & 6 & 1 & YES & YES & YES & -- & 375\\
$(53, 20)$ & 10 & $(16, 3)$ & 7 & 1 & YES & YES & YES & -- & 376\\
$(53, 8)$ & 10 & $(20, 7)$ & 8 & 1 & YES & YES & YES & NO & 377\\
$(53, 8)$ & 10 & $(36, 7)$ & 11 & 1 & YES & YES & YES & NO & 378\\
$(55, 24)$ & 9 & $(12, 5)$ & 5 & 1 & YES & YES & YES & -- & 379\\
$(55, 21)$ & 8 & $(24, 7)$ & 7 & 1 & YES & NO & YES & -- & 380\\
$(56, 15)$ & 9 & $(36, 11)$ & 8 & 4 & YES & NO & YES & NO & 381\\
$(57, 13)$ & 9 & $(13, 6)$ & 7 & 1 & YES & YES & YES & -- & 382\\
$(57, 13)$ & 9 & $(13, 6)$ & 7 & 1 & YES & YES & YES & NO & 383\\
$(58, 17)$ & 9 & $(27, 8)$ & 7 & 1 & YES & NO & YES & -- & 384\\
$(58, 17)$ & 9 & $(29, 8)$ & 7 & 29 & YES & NO & YES & -- & 385\\
$(59, 14)$ & 10 & $(10, 3)$ & 5 & 1 & YES & YES & YES & -- & 386\\
$(59, 14)$ & 10 & $(11, 3)$ & 5 & 1 & YES & YES & YES & -- & 387\\
$(59, 14)$ & 10 & $(11, 3)$ & 5 & 1 & YES & YES & YES & NO & 388\\
$(59, 8)$ & 11 & $(27, 7)$ & 9 & 1 & YES & YES & YES & NO & 389\\
$(59, 8)$ & 11 & $(36, 7)$ & 11 & 1 & YES & YES & YES & NO & 390\\
$(60, 13)$ & 9 & $(13, 6)$ & 7 & 1 & YES & YES & YES & -- & 391\\
$(60, 13)$ & 9 & $(13, 6)$ & 7 & 1 & YES & YES & YES & NO & 392\\
$(60, 13)$ & 9 & $(19, 6)$ & 8 & 1 & YES & YES & YES & NO & 393\\
$(60, 13)$ & 9 & $(23, 6)$ & 8 & 1 & YES & YES & YES & NO & 394\\
$(60, 13)$ & 9 & $(29, 7)$ & 10 & 1 & YES & YES & YES & NO & 395\\
$(61, 18)$ & 9 & $(25, 7)$ & 7 & 1 & YES & NO & YES & -- & 396\\
$(61, 17)$ & 9 & $(27, 8)$ & 7 & 1 & YES & NO & YES & -- & 397\\
$(61, 17)$ & 9 & $(35, 8)$ & 8 & 1 & YES & NO & YES & -- & 398\\
$(61, 22)$ & 9 & $(37, 13)$ & 9 & 1 & YES & YES & YES & NO & 399\\
$(62, 27)$ & 9 & $(7, 2)$ & 4 & 1 & YES & YES & YES & -- & 400\\
$(62, 27)$ & 9 & $(7, 2)$ & 4 & 1 & YES & YES & YES & NO & 401\\
$(62, 27)$ & 9 & $(24, 11)$ & 8 & 2 & YES & YES & YES & 498 & 402\\
$(63, 13)$ & 11 & $(17, 5)$ & 6 & 1 & YES & YES & YES & NO & 403\\
$(64, 27)$ & 9 & $(12, 5)$ & 5 & 4 & YES & YES & YES & -- & 404\\
$(64, 19)$ & 9 & $(24, 7)$ & 7 & 8 & YES & NO & YES & -- & 405\\
$(65, 19)$ & 9 & $(22, 7)$ & 9 & 1 & YES & YES & YES & NO & 406\\
$(65, 18)$ & 9 & $(25, 7)$ & 7 & 5 & YES & NO & YES & -- & 407\\
$(65, 18)$ & 9 & $(25, 7)$ & 7 & 5 & YES & NO & YES & NO & 408\\
$(65, 18)$ & 9 & $(32, 7)$ & 8 & 1 & YES & NO & YES & -- & 409\\
$(65, 19)$ & 9 & $(35, 8)$ & 8 & 5 & YES & NO & YES & -- & 410\\
$(67, 28)$ & 10 & $(11, 3)$ & 5 & 1 & YES & YES & YES & -- & 411\\
$(67, 18)$ & 9 & $(12, 5)$ & 5 & 1 & YES & YES & YES & -- & 412\\
$(67, 18)$ & 9 & $(23, 7)$ & 7 & 1 & YES & YES & YES & NO & 413\\
$(68, 19)$ & 9 & $(11, 5)$ & 6 & 1 & YES & YES & YES & -- & 414\\
$(68, 19)$ & 9 & $(27, 8)$ & 7 & 1 & YES & NO & YES & -- & 415\\
$(68, 19)$ & 9 & $(35, 8)$ & 8 & 1 & YES & NO & YES & -- & 416\\
$(70, 29)$ & 9 & $(12, 5)$ & 5 & 2 & YES & YES & YES & -- & 417\\
$(71, 20)$ & 10 & $(7, 2)$ & 4 & 1 & YES & YES & YES & -- & 418\\
$(71, 21)$ & 9 & $(14, 5)$ & 6 & 1 & YES & YES & YES & NO & 419\\
$(71, 22)$ & 10 & $(17, 5)$ & 6 & 1 & YES & NO & YES & -- & 420\\
$(71, 26)$ & 9 & $(17, 5)$ & 6 & 1 & YES & YES & YES & NO & 421\\
$(71, 21)$ & 9 & $(22, 7)$ & 9 & 1 & YES & YES & YES & NO & 422\\
$(71, 21)$ & 9 & $(25, 7)$ & 7 & 1 & YES & NO & YES & -- & 423\\
$(73, 33)$ & 10 & $(10, 3)$ & 5 & 1 & YES & YES & YES & -- & 424\\
$(73, 13)$ & 10 & $(13, 6)$ & 7 & 1 & YES & YES & YES & -- & 425\\
$(73, 13)$ & 10 & $(36, 7)$ & 11 & 1 & YES & YES & YES & NO & 426\\
$(73, 33)$ & 10 & $(71, 32)$ & 10 & 1 & YES & YES & YES & NO & 427\\
$(74, 23)$ & 10 & $(7, 2)$ & 4 & 1 & YES & YES & YES & NO & 428\\
$(74, 23)$ & 10 & $(13, 5)$ & 5 & 1 & YES & YES & YES & NO & 429\\
$(74, 23)$ & 10 & $(41, 13)$ & 10 & 1 & YES & YES & YES & NO & 430\\
$(75, 31)$ & 9 & $(9, 4)$ & 5 & 3 & YES & YES & YES & -- & 431\\
$(75, 31)$ & 9 & $(9, 4)$ & 5 & 3 & YES & YES & YES & NO & 432\\
$(75, 31)$ & 9 & $(12, 5)$ & 5 & 3 & YES & YES & YES & -- & 433\\
$(75, 31)$ & 9 & $(13, 6)$ & 7 & 1 & YES & YES & YES & NO & 434\\
$(76, 21)$ & 9 & $(11, 5)$ & 6 & 1 & YES & YES & YES & NO & 435\\
$(76, 29)$ & 9 & $(12, 5)$ & 5 & 4 & YES & NO & YES & -- & 436\\
$(76, 21)$ & 9 & $(14, 5)$ & 6 & 2 & YES & YES & YES & NO & 437\\
$(76, 21)$ & 9 & $(18, 7)$ & 6 & 2 & YES & NO & YES & -- & 438\\
$(76, 21)$ & 9 & $(19, 6)$ & 8 & 19 & YES & YES & YES & NO & 439\\
$(76, 21)$ & 9 & $(24, 7)$ & 7 & 4 & YES & NO & YES & -- & 440\\
$(76, 21)$ & 9 & $(24, 7)$ & 7 & 4 & YES & NO & YES & NO & 441\\
$(76, 21)$ & 9 & $(25, 7)$ & 7 & 1 & YES & NO & YES & -- & 442\\
$(77, 20)$ & 11 & $(10, 3)$ & 5 & 1 & YES & YES & YES & -- & 443\\
$(77, 20)$ & 11 & $(11, 3)$ & 5 & 11 & YES & YES & YES & -- & 444\\
$(79, 30)$ & 9 & $(7, 2)$ & 4 & 1 & YES & YES & YES & -- & 445\\
$(79, 18)$ & 10 & $(24, 7)$ & 7 & 1 & YES & NO & YES & -- & 446\\
$(80, 31)$ & 9 & $(17, 5)$ & 6 & 1 & YES & NO & YES & -- & 447\\
$(81, 34)$ & 9 & $(7, 2)$ & 4 & 1 & YES & YES & YES & -- & 448\\
$(81, 34)$ & 9 & $(9, 4)$ & 5 & 9 & YES & YES & YES & -- & 449\\
$(81, 34)$ & 9 & $(9, 4)$ & 5 & 9 & YES & YES & YES & NO & 450\\
$(81, 34)$ & 9 & $(12, 5)$ & 5 & 3 & YES & YES & YES & -- & 451\\
$(81, 34)$ & 9 & $(13, 6)$ & 7 & 1 & YES & YES & YES & NO & 452\\
$(81, 34)$ & 9 & $(33, 14)$ & 8 & 3 & YES & YES & YES & NO & 453\\
$(82, 37)$ & 10 & $(13, 3)$ & 6 & 1 & YES & YES & YES & -- & 454\\
$(82, 37)$ & 10 & $(13, 3)$ & 6 & 1 & YES & YES & YES & NO & 455\\
$(83, 18)$ & 10 & $(6, 1)$ & 5 & 1 & YES & YES & YES & -- & 456\\
$(83, 18)$ & 10 & $(6, 1)$ & 5 & 1 & YES & YES & YES & NO & 457\\
$(83, 13)$ & 11 & $(13, 6)$ & 7 & 1 & YES & YES & YES & NO & 458\\
$(83, 24)$ & 11 & $(14, 3)$ & 6 & 1 & YES & YES & YES & NO & 459\\
$(84, 25)$ & 10 & $(7, 2)$ & 4 & 7 & YES & YES & YES & -- & 460\\
$(85, 36)$ & 10 & $(7, 2)$ & 4 & 1 & YES & YES & YES & -- & 461\\
$(87, 23)$ & 10 & $(7, 2)$ & 4 & 1 & YES & YES & YES & -- & 462\\
$(87, 20)$ & 12 & $(8, 3)$ & 4 & 1 & YES & YES & YES & -- & 463\\
$(87, 20)$ & 12 & $(8, 3)$ & 4 & 1 & YES & YES & YES & NO & 464\\
$(87, 20)$ & 12 & $(17, 3)$ & 7 & 1 & YES & YES & YES & NO & 465\\
$(87, 23)$ & 10 & $(50, 13)$ & 10 & 1 & YES & YES & YES & NO & 466\\
$(87, 20)$ & 12 & $(53, 12)$ & 9 & 1 & YES & YES & YES & NO & 467\\
$(89, 34)$ & 9 & $(5, 2)$ & 3 & 1 & YES & YES & YES & -- & 468\\
$(89, 35)$ & 11 & $(7, 2)$ & 4 & 1 & YES & YES & YES & -- & 469\\
$(89, 34)$ & 9 & $(11, 5)$ & 6 & 1 & YES & YES & YES & NO & 470\\
$(89, 39)$ & 11 & $(13, 2)$ & 7 & 1 & YES & YES & YES & -- & 471\\
$(89, 26)$ & 10 & $(18, 5)$ & 6 & 1 & YES & NO & YES & -- & 472\\
$(91, 40)$ & 10 & $(5, 1)$ & 4 & 1 & YES & YES & YES & NO & 473\\
$(91, 40)$ & 10 & $(7, 3)$ & 4 & 7 & YES & YES & YES & -- & 474\\
$(91, 40)$ & 10 & $(7, 3)$ & 4 & 7 & YES & YES & YES & NO & 475\\
$(91, 27)$ & 10 & $(18, 5)$ & 6 & 1 & YES & NO & YES & -- & 476\\
$(92, 39)$ & 10 & $(7, 2)$ & 4 & 1 & YES & YES & YES & -- & 477\\
$(92, 21)$ & 10 & $(13, 4)$ & 6 & 1 & YES & YES & YES & NO & 478\\
$(93, 34)$ & 10 & $(7, 2)$ & 4 & 1 & YES & YES & YES & -- & 479\\
$(93, 34)$ & 10 & $(49, 18)$ & 8 & 1 & YES & YES & YES & NO & 480\\
$(95, 29)$ & 10 & $(16, 5)$ & 7 & 1 & YES & YES & YES & NO & 481\\
$(97, 18)$ & 11 & $(3, 1)$ & 2 & 1 & YES & YES & YES & -- & 482\\
$(97, 18)$ & 11 & $(3, 1)$ & 2 & 1 & YES & YES & YES & NO & 483\\
$(97, 35)$ & 10 & $(20, 7)$ & 8 & 1 & YES & YES & YES & NO & 484\\
$(97, 27)$ & 11 & $(25, 7)$ & 7 & 1 & YES & YES & YES & NO & 485\\
$(98, 45)$ & 11 & $(5, 2)$ & 3 & 1 & YES & YES & YES & -- & 486\\
$(98, 29)$ & 10 & $(17, 5)$ & 6 & 1 & YES & NO & YES & -- & 487\\
$(98, 29)$ & 10 & $(17, 5)$ & 6 & 1 & YES & NO & YES & NO & 488\\
$(98, 45)$ & 11 & $(20, 9)$ & 7 & 2 & YES & YES & YES & NO & 489\\
$(98, 29)$ & 10 & $(36, 11)$ & 8 & 2 & YES & NO & YES & NO & 490\\
$(98, 37)$ & 11 & $(66, 25)$ & 9 & 2 & YES & YES & YES & NO & 491\\
$(100, 27)$ & 10 & $(7, 2)$ & 4 & 1 & YES & YES & YES & -- & 492\\
$(100, 27)$ & 10 & $(7, 2)$ & 4 & 1 & YES & YES & YES & NO & 493\\
$(100, 27)$ & 10 & $(19, 5)$ & 7 & 1 & YES & YES & YES & 654 & 494\\
$(101, 44)$ & 10 & $(4, 1)$ & 3 & 1 & YES & YES & YES & -- & 495\\
$(101, 44)$ & 10 & $(7, 3)$ & 4 & 1 & YES & YES & YES & -- & 496\\
$(101, 30)$ & 10 & $(13, 5)$ & 5 & 1 & YES & NO & YES & -- & 497\\
$(101, 44)$ & 10 & $(13, 6)$ & 7 & 1 & YES & YES & YES & 402 & 498\\
$(101, 30)$ & 10 & $(22, 5)$ & 7 & 1 & YES & NO & YES & -- & 499\\
$(103, 37)$ & 10 & $(10, 3)$ & 5 & 1 & YES & YES & YES & NO & 500\\
$(103, 19)$ & 11 & $(36, 7)$ & 11 & 1 & YES & YES & YES & NO & 501\\
$(103, 39)$ & 10 & $(53, 20)$ & 10 & 1 & YES & YES & YES & NO & 502\\
$(104, 29)$ & 10 & $(17, 5)$ & 6 & 1 & YES & NO & YES & -- & 503\\
$(105, 44)$ & 10 & $(7, 3)$ & 4 & 7 & YES & YES & YES & -- & 504\\
$(107, 20)$ & 13 & $(8, 3)$ & 4 & 1 & YES & YES & YES & NO & 505\\
$(108, 29)$ & 10 & $(7, 3)$ & 4 & 1 & YES & YES & YES & -- & 506\\
$(108, 29)$ & 10 & $(7, 3)$ & 4 & 1 & YES & YES & YES & NO & 507\\
$(108, 29)$ & 10 & $(13, 4)$ & 6 & 1 & YES & YES & YES & NO & 508\\
$(108, 29)$ & 10 & $(19, 5)$ & 7 & 1 & YES & YES & YES & NO & 509\\
$(108, 29)$ & 10 & $(29, 8)$ & 7 & 1 & YES & YES & YES & NO & 510\\
$(108, 29)$ & 10 & $(49, 13)$ & 9 & 1 & YES & YES & YES & NO & 511\\
$(108, 29)$ & 10 & $(63, 17)$ & 9 & 9 & YES & YES & YES & NO & 512\\
$(109, 46)$ & 10 & $(33, 14)$ & 8 & 1 & YES & YES & YES & NO & 513\\
$(111, 46)$ & 10 & $(7, 3)$ & 4 & 1 & YES & YES & YES & -- & 514\\
$(113, 24)$ & 11 & $(3, 1)$ & 2 & 1 & YES & YES & YES & -- & 515\\
$(113, 24)$ & 11 & $(3, 1)$ & 2 & 1 & YES & YES & YES & NO & 516\\
$(113, 35)$ & 11 & $(3, 1)$ & 2 & 1 & YES & YES & YES & -- & 517\\
$(113, 35)$ & 11 & $(3, 1)$ & 2 & 1 & YES & YES & YES & NO & 518\\
$(113, 35)$ & 11 & $(7, 2)$ & 4 & 1 & YES & YES & YES & NO & 519\\
$(113, 35)$ & 11 & $(10, 3)$ & 5 & 1 & YES & NO & YES & -- & 520\\
$(113, 24)$ & 11 & $(47, 10)$ & 9 & 1 & YES & YES & YES & 552 & 521\\
$(115, 34)$ & 10 & $(23, 7)$ & 7 & 23 & YES & YES & YES & NO & 522\\
$(117, 49)$ & 10 & $(11, 3)$ & 5 & 1 & YES & NO & YES & NO & 523\\
$(117, 31)$ & 11 & $(27, 7)$ & 9 & 9 & YES & YES & YES & NO & 524\\
$(119, 37)$ & 11 & $(3, 1)$ & 2 & 1 & YES & YES & YES & -- & 525\\
$(119, 37)$ & 11 & $(3, 1)$ & 2 & 1 & YES & YES & YES & NO & 526\\
$(119, 44)$ & 10 & $(4, 1)$ & 3 & 1 & YES & YES & YES & -- & 527\\
$(119, 44)$ & 10 & $(4, 1)$ & 3 & 1 & YES & YES & YES & NO & 528\\
$(119, 44)$ & 10 & $(4, 1)$ & 3 & 1 & YES & YES & YES & NO & 529\\
$(119, 37)$ & 11 & $(10, 3)$ & 5 & 1 & YES & YES & YES & 632 & 530\\
$(119, 50)$ & 10 & $(112, 47)$ & 10 & 7 & YES & NO & YES & NO & 531\\
$(121, 32)$ & 11 & $(4, 1)$ & 3 & 1 & YES & YES & YES & -- & 532\\
$(121, 32)$ & 11 & $(4, 1)$ & 3 & 1 & YES & YES & YES & NO & 533\\
$(121, 35)$ & 12 & $(11, 3)$ & 5 & 11 & YES & YES & YES & NO & 534\\
$(121, 32)$ & 11 & $(13, 3)$ & 6 & 1 & YES & YES & YES & NO & 535\\
$(121, 34)$ & 11 & $(17, 5)$ & 6 & 1 & YES & YES & YES & NO & 536\\
$(121, 32)$ & 11 & $(27, 7)$ & 9 & 1 & YES & YES & YES & NO & 537\\
$(121, 35)$ & 12 & $(52, 15)$ & 11 & 1 & YES & YES & YES & 729 & 538\\
$(122, 55)$ & 11 & $(7, 2)$ & 4 & 1 & YES & YES & YES & -- & 539\\
$(122, 55)$ & 11 & $(7, 2)$ & 4 & 1 & YES & YES & YES & NO & 540\\
$(122, 55)$ & 11 & $(13, 6)$ & 7 & 1 & YES & YES & YES & NO & 541\\
$(122, 55)$ & 11 & $(42, 19)$ & 9 & 2 & YES & YES & YES & NO & 542\\
$(123, 47)$ & 10 & $(10, 3)$ & 5 & 1 & YES & NO & YES & -- & 543\\
$(123, 47)$ & 10 & $(13, 3)$ & 6 & 1 & YES & NO & YES & -- & 544\\
$(123, 47)$ & 10 & $(89, 34)$ & 9 & 1 & YES & YES & YES & NO & 545\\
$(124, 37)$ & 12 & $(84, 25)$ & 10 & 4 & YES & YES & YES & NO & 546\\
$(125, 27)$ & 11 & $(3, 1)$ & 2 & 1 & YES & YES & YES & -- & 547\\
$(126, 37)$ & 13 & $(126, 37)$ & 13 & 126 & YES & YES & YES & NO & 548\\
$(127, 45)$ & 11 & $(5, 2)$ & 3 & 1 & YES & YES & YES & -- & 549\\
$(127, 57)$ & 11 & $(13, 6)$ & 7 & 1 & YES & YES & YES & NO & 550\\
$(127, 45)$ & 11 & $(20, 7)$ & 8 & 1 & YES & YES & YES & NO & 551\\
$(127, 27)$ & 11 & $(33, 7)$ & 8 & 1 & YES & YES & YES & 521 & 552\\
$(127, 45)$ & 11 & $(45, 16)$ & 9 & 1 & YES & YES & YES & NO & 553\\
$(128, 49)$ & 10 & $(10, 3)$ & 5 & 2 & YES & NO & YES & -- & 554\\
$(128, 37)$ & 12 & $(11, 3)$ & 5 & 1 & YES & YES & YES & NO & 555\\
$(128, 37)$ & 12 & $(52, 15)$ & 11 & 4 & YES & YES & YES & NO & 556\\
$(129, 49)$ & 10 & $(3, 1)$ & 2 & 3 & YES & YES & YES & -- & 557\\
$(129, 49)$ & 10 & $(10, 3)$ & 5 & 1 & YES & NO & YES & -- & 558\\
$(129, 49)$ & 10 & $(11, 3)$ & 5 & 1 & YES & NO & YES & -- & 559\\
$(131, 24)$ & 13 & $(8, 3)$ & 4 & 1 & YES & YES & YES & NO & 560\\
$(131, 24)$ & 13 & $(10, 3)$ & 5 & 1 & YES & YES & YES & NO & 561\\
$(131, 55)$ & 10 & $(11, 3)$ & 5 & 1 & YES & NO & YES & NO & 562\\
$(131, 37)$ & 12 & $(53, 15)$ & 11 & 1 & YES & YES & YES & NO & 563\\
$(132, 35)$ & 11 & $(10, 3)$ & 5 & 2 & YES & YES & YES & -- & 564\\
$(132, 35)$ & 11 & $(27, 7)$ & 9 & 3 & YES & YES & YES & NO & 565\\
$(132, 35)$ & 11 & $(42, 11)$ & 9 & 6 & YES & YES & YES & NO & 566\\
$(135, 32)$ & 12 & $(3, 1)$ & 2 & 3 & YES & YES & YES & -- & 567\\
$(135, 32)$ & 12 & $(3, 1)$ & 2 & 3 & YES & YES & YES & NO & 568\\
$(135, 62)$ & 12 & $(5, 1)$ & 4 & 5 & YES & YES & YES & -- & 569\\
$(135, 62)$ & 12 & $(5, 1)$ & 4 & 5 & YES & YES & YES & NO & 570\\
$(135, 62)$ & 12 & $(5, 2)$ & 3 & 5 & YES & YES & YES & -- & 571\\
$(135, 62)$ & 12 & $(5, 2)$ & 3 & 5 & YES & YES & YES & NO & 572\\
$(135, 32)$ & 12 & $(10, 3)$ & 5 & 5 & YES & YES & YES & NO & 573\\
$(135, 32)$ & 12 & $(11, 3)$ & 5 & 1 & YES & YES & YES & NO & 574\\
$(135, 32)$ & 12 & $(59, 14)$ & 10 & 1 & YES & YES & YES & 635 & 575\\
$(138, 41)$ & 11 & $(11, 3)$ & 5 & 1 & YES & NO & YES & -- & 576\\
$(138, 41)$ & 11 & $(13, 3)$ & 6 & 1 & YES & NO & YES & -- & 577\\
$(140, 37)$ & 11 & $(4, 1)$ & 3 & 4 & YES & YES & YES & -- & 578\\
$(140, 37)$ & 11 & $(4, 1)$ & 3 & 4 & YES & YES & YES & NO & 579\\
$(140, 37)$ & 11 & $(13, 3)$ & 6 & 1 & YES & YES & YES & NO & 580\\
$(140, 37)$ & 11 & $(27, 7)$ & 9 & 1 & YES & YES & YES & NO & 581\\
$(141, 59)$ & 11 & $(3, 1)$ & 2 & 3 & YES & YES & YES & -- & 582\\
$(141, 59)$ & 11 & $(19, 8)$ & 6 & 1 & YES & YES & YES & NO & 583\\
$(141, 59)$ & 11 & $(50, 21)$ & 8 & 1 & YES & NO & YES & NO & 584\\
$(143, 54)$ & 12 & $(4, 1)$ & 3 & 1 & YES & YES & YES & -- & 585\\
$(143, 54)$ & 12 & $(29, 11)$ & 7 & 1 & YES & YES & YES & NO & 586\\
$(143, 60)$ & 11 & $(143, 60)$ & 11 & 143 & YES & YES & YES & NO & 587\\
$(144, 61)$ & 11 & $(3, 1)$ & 2 & 3 & YES & YES & YES & -- & 588\\
$(144, 55)$ & 10 & $(13, 3)$ & 6 & 1 & YES & NO & YES & -- & 589\\
$(144, 61)$ & 11 & $(144, 61)$ & 11 & 144 & YES & YES & YES & NO & 590\\
$(145, 51)$ & 12 & $(2, 1)$ & 1 & 1 & YES & YES & YES & -- & 591\\
$(145, 44)$ & 11 & $(3, 1)$ & 2 & 1 & YES & YES & YES & -- & 592\\
$(145, 44)$ & 11 & $(3, 1)$ & 2 & 1 & YES & YES & YES & NO & 593\\
$(145, 51)$ & 12 & $(3, 1)$ & 2 & 1 & YES & YES & YES & -- & 594\\
$(145, 51)$ & 12 & $(4, 1)$ & 3 & 1 & YES & YES & YES & NO & 595\\
$(145, 51)$ & 12 & $(5, 1)$ & 4 & 5 & YES & YES & YES & NO & 596\\
$(145, 51)$ & 12 & $(5, 2)$ & 3 & 5 & YES & YES & YES & NO & 597\\
$(145, 51)$ & 12 & $(20, 7)$ & 8 & 5 & YES & YES & YES & 613 & 598\\
$(146, 67)$ & 12 & $(3, 1)$ & 2 & 1 & YES & YES & YES & -- & 599\\
$(146, 67)$ & 12 & $(5, 2)$ & 3 & 1 & YES & YES & YES & NO & 600\\
$(147, 43)$ & 11 & $(11, 3)$ & 5 & 1 & YES & NO & YES & -- & 601\\
$(148, 65)$ & 11 & $(12, 5)$ & 5 & 4 & YES & NO & YES & NO & 602\\
$(150, 59)$ & 12 & $(4, 1)$ & 3 & 2 & YES & YES & YES & NO & 603\\
$(150, 59)$ & 12 & $(89, 35)$ & 11 & 1 & YES & YES & YES & NO & 604\\
$(151, 53)$ & 12 & $(2, 1)$ & 1 & 1 & YES & YES & YES & -- & 605\\
$(151, 53)$ & 12 & $(3, 1)$ & 2 & 1 & YES & YES & YES & -- & 606\\
$(151, 53)$ & 12 & $(3, 1)$ & 2 & 1 & YES & YES & YES & NO & 607\\
$(151, 64)$ & 11 & $(4, 1)$ & 3 & 1 & YES & YES & YES & NO & 608\\
$(151, 53)$ & 12 & $(5, 1)$ & 4 & 1 & YES & YES & YES & NO & 609\\
$(151, 53)$ & 12 & $(5, 2)$ & 3 & 1 & YES & YES & YES & NO & 610\\
$(151, 53)$ & 12 & $(6, 1)$ & 5 & 1 & YES & YES & YES & NO & 611\\
$(151, 53)$ & 12 & $(14, 5)$ & 6 & 1 & YES & YES & YES & 740 & 612\\
$(151, 53)$ & 12 & $(17, 6)$ & 7 & 1 & YES & YES & YES & 598 & 613\\
$(151, 27)$ & 13 & $(73, 13)$ & 10 & 1 & YES & YES & YES & NO & 614\\
$(151, 64)$ & 11 & $(92, 39)$ & 10 & 1 & YES & YES & YES & NO & 615\\
$(151, 64)$ & 11 & $(151, 64)$ & 11 & 151 & YES & YES & YES & NO & 616\\
$(153, 64)$ & 11 & $(3, 1)$ & 2 & 3 & YES & YES & YES & -- & 617\\
$(153, 70)$ & 12 & $(3, 1)$ & 2 & 3 & YES & YES & YES & -- & 618\\
$(153, 64)$ & 11 & $(4, 1)$ & 3 & 1 & YES & YES & YES & NO & 619\\
$(153, 70)$ & 12 & $(5, 1)$ & 4 & 1 & YES & YES & YES & -- & 620\\
$(153, 70)$ & 12 & $(5, 2)$ & 3 & 1 & YES & YES & YES & NO & 621\\
$(153, 70)$ & 12 & $(9, 4)$ & 5 & 9 & YES & YES & YES & NO & 622\\
$(153, 43)$ & 12 & $(18, 5)$ & 6 & 9 & YES & YES & YES & NO & 623\\
$(153, 64)$ & 11 & $(19, 8)$ & 6 & 1 & YES & YES & YES & NO & 624\\
$(153, 64)$ & 11 & $(31, 13)$ & 7 & 1 & YES & YES & YES & 706 & 625\\
$(153, 43)$ & 12 & $(57, 16)$ & 9 & 3 & YES & YES & YES & NO & 626\\
$(153, 64)$ & 11 & $(98, 41)$ & 10 & 1 & YES & YES & YES & NO & 627\\
$(153, 64)$ & 11 & $(153, 64)$ & 11 & 153 & YES & YES & YES & NO & 628\\
$(154, 47)$ & 11 & $(2, 1)$ & 1 & 2 & YES & YES & YES & -- & 629\\
$(154, 47)$ & 11 & $(2, 1)$ & 1 & 2 & YES & YES & YES & NO & 630\\
$(154, 47)$ & 11 & $(3, 1)$ & 2 & 1 & YES & YES & YES & -- & 631\\
$(154, 47)$ & 11 & $(3, 1)$ & 2 & 1 & YES & YES & YES & 530 & 632\\
$(154, 47)$ & 11 & $(10, 3)$ & 5 & 2 & YES & YES & YES & NO & 633\\
$(156, 37)$ & 12 & $(11, 3)$ & 5 & 1 & YES & YES & YES & NO & 634\\
$(156, 37)$ & 12 & $(38, 9)$ & 9 & 2 & YES & YES & YES & 575 & 635\\
$(156, 37)$ & 12 & $(59, 14)$ & 10 & 1 & YES & YES & YES & NO & 636\\
$(157, 46)$ & 11 & $(4, 1)$ & 3 & 1 & YES & YES & YES & -- & 637\\
$(157, 46)$ & 11 & $(4, 1)$ & 3 & 1 & YES & YES & YES & NO & 638\\
$(157, 46)$ & 11 & $(10, 3)$ & 5 & 1 & YES & NO & YES & -- & 639\\
$(159, 73)$ & 12 & $(5, 2)$ & 3 & 1 & YES & YES & YES & NO & 640\\
$(159, 61)$ & 12 & $(6, 1)$ & 5 & 3 & YES & YES & YES & NO & 641\\
$(159, 73)$ & 12 & $(9, 4)$ & 5 & 3 & YES & YES & YES & NO & 642\\
$(161, 51)$ & 13 & $(2, 1)$ & 1 & 1 & YES & YES & YES & -- & 643\\
$(161, 66)$ & 11 & $(2, 1)$ & 1 & 1 & YES & YES & YES & -- & 644\\
$(161, 57)$ & 12 & $(3, 1)$ & 2 & 1 & YES & YES & YES & -- & 645\\
$(161, 51)$ & 13 & $(4, 1)$ & 3 & 1 & YES & YES & YES & NO & 646\\
$(161, 51)$ & 13 & $(5, 1)$ & 4 & 1 & YES & YES & YES & NO & 647\\
$(161, 57)$ & 12 & $(5, 2)$ & 3 & 1 & YES & YES & YES & NO & 648\\
$(161, 66)$ & 11 & $(7, 3)$ & 4 & 7 & YES & YES & YES & NO & 649\\
$(162, 71)$ & 12 & $(89, 39)$ & 11 & 1 & YES & YES & YES & NO & 650\\
$(162, 71)$ & 12 & $(162, 71)$ & 12 & 162 & YES & YES & YES & NO & 651\\
$(163, 37)$ & 14 & $(2, 1)$ & 1 & 1 & YES & YES & YES & -- & 652\\
$(163, 44)$ & 11 & $(4, 1)$ & 3 & 1 & YES & YES & YES & -- & 653\\
$(163, 44)$ & 11 & $(4, 1)$ & 3 & 1 & YES & YES & YES & 494 & 654\\
$(163, 44)$ & 11 & $(4, 1)$ & 3 & 1 & YES & YES & YES & NO & 655\\
$(163, 37)$ & 14 & $(13, 3)$ & 6 & 1 & YES & YES & YES & NO & 656\\
$(165, 46)$ & 11 & $(10, 3)$ & 5 & 5 & YES & NO & YES & -- & 657\\
$(167, 38)$ & 14 & $(2, 1)$ & 1 & 1 & YES & YES & YES & -- & 658\\
$(167, 38)$ & 14 & $(13, 3)$ & 6 & 1 & YES & YES & YES & NO & 659\\
$(169, 66)$ & 11 & $(4, 1)$ & 3 & 1 & YES & YES & YES & NO & 660\\
$(169, 71)$ & 11 & $(5, 2)$ & 3 & 1 & YES & NO & YES & -- & 661\\
$(170, 37)$ & 14 & $(2, 1)$ & 1 & 2 & YES & YES & YES & -- & 662\\
$(170, 37)$ & 14 & $(4, 1)$ & 3 & 2 & YES & YES & YES & NO & 663\\
$(170, 37)$ & 14 & $(14, 3)$ & 6 & 2 & YES & YES & YES & NO & 664\\
$(171, 74)$ & 12 & $(3, 1)$ & 2 & 3 & YES & YES & YES & -- & 665\\
$(173, 45)$ & 13 & $(2, 1)$ & 1 & 1 & YES & YES & YES & NO & 666\\
$(173, 45)$ & 13 & $(3, 1)$ & 2 & 1 & YES & YES & YES & NO & 667\\
$(173, 53)$ & 12 & $(3, 1)$ & 2 & 1 & YES & YES & YES & -- & 668\\
$(173, 78)$ & 12 & $(3, 1)$ & 2 & 1 & YES & YES & YES & -- & 669\\
$(173, 53)$ & 12 & $(7, 2)$ & 4 & 1 & YES & YES & YES & NO & 670\\
$(173, 78)$ & 12 & $(173, 78)$ & 12 & 173 & YES & YES & YES & NO & 671\\
$(175, 38)$ & 14 & $(2, 1)$ & 1 & 1 & YES & YES & YES & -- & 672\\
$(175, 38)$ & 14 & $(4, 1)$ & 3 & 1 & YES & YES & YES & NO & 673\\
$(175, 62)$ & 12 & $(5, 1)$ & 4 & 5 & YES & YES & YES & -- & 674\\
$(175, 62)$ & 12 & $(5, 2)$ & 3 & 5 & YES & YES & YES & NO & 675\\
$(177, 46)$ & 13 & $(3, 1)$ & 2 & 3 & YES & YES & YES & -- & 676\\
$(177, 46)$ & 13 & $(3, 1)$ & 2 & 3 & YES & YES & YES & NO & 677\\
$(177, 46)$ & 13 & $(5, 1)$ & 4 & 1 & YES & YES & YES & -- & 678\\
$(177, 80)$ & 12 & $(42, 19)$ & 9 & 3 & YES & YES & YES & NO & 679\\
$(177, 74)$ & 12 & $(67, 28)$ & 10 & 1 & YES & YES & YES & 703 & 680\\
$(177, 80)$ & 12 & $(73, 33)$ & 10 & 1 & YES & YES & YES & NO & 681\\
$(177, 46)$ & 13 & $(77, 20)$ & 11 & 1 & YES & YES & YES & 731 & 682\\
$(177, 46)$ & 13 & $(177, 46)$ & 13 & 177 & YES & YES & YES & NO & 683\\
$(178, 69)$ & 11 & $(7, 2)$ & 4 & 1 & YES & NO & YES & -- & 684\\
$(178, 33)$ & 14 & $(17, 3)$ & 7 & 1 & YES & YES & YES & NO & 685\\
$(178, 33)$ & 14 & $(38, 7)$ & 9 & 2 & YES & YES & YES & NO & 686\\
$(181, 76)$ & 11 & $(6, 1)$ & 5 & 1 & YES & YES & YES & -- & 687\\
$(181, 76)$ & 11 & $(6, 1)$ & 5 & 1 & YES & YES & YES & NO & 688\\
$(181, 76)$ & 11 & $(181, 76)$ & 11 & 181 & YES & YES & YES & NO & 689\\
$(183, 71)$ & 11 & $(7, 2)$ & 4 & 1 & YES & NO & YES & -- & 690\\
$(185, 33)$ & 14 & $(39, 7)$ & 9 & 1 & YES & YES & YES & NO & 691\\
$(187, 45)$ & 14 & $(2, 1)$ & 1 & 1 & YES & YES & YES & -- & 692\\
$(187, 45)$ & 14 & $(2, 1)$ & 1 & 1 & YES & YES & YES & NO & 693\\
$(187, 84)$ & 12 & $(2, 1)$ & 1 & 1 & YES & YES & YES & NO & 694\\
$(187, 45)$ & 14 & $(3, 1)$ & 2 & 1 & YES & YES & YES & NO & 695\\
$(187, 84)$ & 12 & $(3, 1)$ & 2 & 1 & YES & YES & YES & -- & 696\\
$(187, 45)$ & 14 & $(29, 7)$ & 10 & 1 & YES & YES & YES & NO & 697\\
$(188, 85)$ & 12 & $(4, 1)$ & 3 & 4 & YES & YES & YES & NO & 698\\
$(188, 85)$ & 12 & $(20, 9)$ & 7 & 4 & YES & YES & YES & NO & 699\\
$(189, 50)$ & 13 & $(3, 1)$ & 2 & 3 & YES & YES & YES & -- & 700\\
$(189, 67)$ & 12 & $(4, 1)$ & 3 & 1 & YES & YES & YES & -- & 701\\
$(189, 50)$ & 13 & $(11, 3)$ & 5 & 1 & YES & YES & YES & NO & 702\\
$(189, 79)$ & 12 & $(55, 23)$ & 9 & 1 & YES & YES & YES & 680 & 703\\
$(191, 46)$ & 14 & $(2, 1)$ & 1 & 1 & YES & YES & YES & -- & 704\\
$(191, 59)$ & 13 & $(10, 3)$ & 5 & 1 & YES & YES & YES & NO & 705\\
$(191, 80)$ & 11 & $(12, 5)$ & 5 & 1 & YES & YES & YES & 625 & 706\\
$(191, 46)$ & 14 & $(25, 6)$ & 9 & 1 & YES & YES & YES & NO & 707\\
$(193, 87)$ & 12 & $(2, 1)$ & 1 & 1 & YES & YES & YES & NO & 708\\
$(193, 60)$ & 12 & $(3, 1)$ & 2 & 1 & NO & YES & YES & -- & 709\\
$(193, 87)$ & 12 & $(4, 1)$ & 3 & 1 & YES & YES & YES & -- & 710\\
$(193, 87)$ & 12 & $(4, 1)$ & 3 & 1 & YES & YES & YES & NO & 711\\
$(193, 87)$ & 12 & $(31, 14)$ & 8 & 1 & YES & YES & YES & 755 & 712\\
$(193, 81)$ & 11 & $(131, 55)$ & 10 & 1 & YES & NO & YES & NO & 713\\
$(195, 88)$ & 12 & $(2, 1)$ & 1 & 1 & YES & YES & YES & -- & 714\\
$(195, 88)$ & 12 & $(2, 1)$ & 1 & 1 & YES & YES & YES & NO & 715\\
$(195, 88)$ & 12 & $(82, 37)$ & 10 & 1 & YES & YES & YES & NO & 716\\
$(196, 51)$ & 13 & $(2, 1)$ & 1 & 2 & YES & YES & YES & NO & 717\\
$(196, 51)$ & 13 & $(3, 1)$ & 2 & 1 & YES & YES & YES & NO & 718\\
$(196, 51)$ & 13 & $(27, 7)$ & 9 & 1 & YES & YES & YES & 730 & 719\\
$(197, 54)$ & 13 & $(197, 54)$ & 13 & 197 & YES & YES & YES & NO & 720\\
$(201, 59)$ & 13 & $(5, 1)$ & 4 & 1 & YES & YES & YES & -- & 721\\
$(201, 59)$ & 13 & $(92, 27)$ & 11 & 1 & YES & YES & YES & NO & 722\\
$(204, 53)$ & 13 & $(2, 1)$ & 1 & 2 & YES & YES & YES & -- & 723\\
$(204, 53)$ & 13 & $(2, 1)$ & 1 & 2 & YES & YES & YES & NO & 724\\
$(204, 59)$ & 13 & $(2, 1)$ & 1 & 2 & YES & YES & YES & -- & 725\\
$(204, 53)$ & 13 & $(3, 1)$ & 2 & 3 & YES & YES & YES & NO & 726\\
$(204, 53)$ & 13 & $(5, 1)$ & 4 & 1 & YES & YES & YES & -- & 727\\
$(204, 53)$ & 13 & $(7, 1)$ & 6 & 1 & YES & YES & YES & NO & 728\\
$(204, 59)$ & 13 & $(7, 2)$ & 4 & 1 & YES & YES & YES & 538 & 729\\
$(204, 53)$ & 13 & $(23, 6)$ & 8 & 1 & YES & YES & YES & 719 & 730\\
$(204, 53)$ & 13 & $(50, 13)$ & 10 & 2 & YES & YES & YES & 682 & 731\\
$(204, 53)$ & 13 & $(77, 20)$ & 11 & 1 & YES & YES & YES & NO & 732\\
$(205, 92)$ & 12 & $(2, 1)$ & 1 & 1 & YES & YES & YES & NO & 733\\
$(205, 38)$ & 15 & $(3, 1)$ & 2 & 1 & YES & YES & YES & -- & 734\\
$(205, 38)$ & 15 & $(3, 1)$ & 2 & 1 & YES & YES & YES & NO & 735\\
$(205, 92)$ & 12 & $(3, 1)$ & 2 & 1 & YES & YES & YES & NO & 736\\
$(205, 38)$ & 15 & $(7, 1)$ & 6 & 1 & YES & YES & YES & NO & 737\\
$(205, 38)$ & 15 & $(43, 8)$ & 9 & 1 & YES & YES & YES & NO & 738\\
$(206, 73)$ & 12 & $(2, 1)$ & 1 & 2 & YES & YES & YES & -- & 739\\
$(206, 73)$ & 12 & $(3, 1)$ & 2 & 1 & YES & YES & YES & 612 & 740\\
$(206, 73)$ & 12 & $(14, 5)$ & 6 & 2 & YES & YES & YES & NO & 741\\
$(207, 76)$ & 11 & $(2, 1)$ & 1 & 1 & YES & YES & YES & NO & 742\\
$(207, 37)$ & 15 & $(3, 1)$ & 2 & 3 & YES & YES & YES & -- & 743\\
$(207, 61)$ & 13 & $(5, 1)$ & 4 & 1 & YES & YES & YES & -- & 744\\
$(207, 37)$ & 15 & $(7, 1)$ & 6 & 1 & YES & YES & YES & NO & 745\\
$(208, 87)$ & 12 & $(4, 1)$ & 3 & 4 & YES & NO & YES & NO & 746\\
$(208, 87)$ & 12 & $(19, 8)$ & 6 & 1 & YES & NO & YES & NO & 747\\
$(212, 59)$ & 13 & $(5, 1)$ & 4 & 1 & YES & YES & YES & -- & 748\\
$(213, 38)$ & 15 & $(3, 1)$ & 2 & 3 & YES & YES & YES & -- & 749\\
$(213, 38)$ & 15 & $(7, 1)$ & 6 & 1 & YES & YES & YES & NO & 750\\
$(213, 38)$ & 15 & $(45, 8)$ & 9 & 3 & YES & YES & YES & NO & 751\\
$(215, 64)$ & 12 & $(3, 1)$ & 2 & 1 & YES & YES & YES & NO & 752\\
$(215, 97)$ & 12 & $(3, 1)$ & 2 & 1 & YES & YES & YES & -- & 753\\
$(215, 82)$ & 12 & $(5, 1)$ & 4 & 5 & YES & YES & YES & -- & 754\\
$(215, 97)$ & 12 & $(20, 9)$ & 7 & 5 & YES & YES & YES & 712 & 755\\
$(222, 59)$ & 13 & $(11, 3)$ & 5 & 1 & YES & YES & YES & NO & 756\\
$(225, 98)$ & 12 & $(3, 1)$ & 2 & 3 & YES & NO & YES & -- & 757\\
$(225, 98)$ & 12 & $(3, 1)$ & 2 & 3 & YES & NO & YES & NO & 758\\
$(226, 61)$ & 12 & $(3, 1)$ & 2 & 1 & NO & YES & YES & -- & 759\\
$(229, 68)$ & 12 & $(5, 2)$ & 3 & 1 & YES & NO & YES & -- & 760\\
$(229, 68)$ & 12 & $(24, 7)$ & 7 & 1 & YES & NO & YES & NO & 761\\
$(231, 106)$ & 13 & $(2, 1)$ & 1 & 1 & NO & YES & YES & -- & 762\\
$(231, 83)$ & 12 & $(3, 1)$ & 2 & 3 & YES & NO & YES & -- & 763\\
$(231, 62)$ & 13 & $(4, 1)$ & 3 & 1 & YES & YES & YES & NO & 764\\
$(231, 62)$ & 13 & $(108, 29)$ & 10 & 3 & YES & YES & YES & NO & 765\\
$(233, 107)$ & 13 & $(2, 1)$ & 1 & 1 & NO & YES & YES & -- & 766\\
$(233, 89)$ & 11 & $(47, 18)$ & 8 & 1 & YES & NO & YES & NO & 767\\
$(236, 69)$ & 12 & $(7, 2)$ & 4 & 1 & YES & NO & YES & -- & 768\\
$(237, 37)$ & 16 & $(5, 1)$ & 4 & 1 & YES & YES & YES & NO & 769\\
$(239, 71)$ & 12 & $(5, 2)$ & 3 & 1 & YES & NO & YES & -- & 770\\
$(239, 71)$ & 12 & $(7, 2)$ & 4 & 1 & YES & NO & YES & -- & 771\\
$(239, 71)$ & 12 & $(7, 2)$ & 4 & 1 & YES & NO & YES & NO & 772\\
$(239, 71)$ & 12 & $(24, 7)$ & 7 & 1 & YES & NO & YES & NO & 773\\
$(239, 71)$ & 12 & $(44, 13)$ & 8 & 1 & YES & NO & YES & NO & 774\\
$(240, 71)$ & 12 & $(5, 2)$ & 3 & 5 & YES & NO & YES & -- & 775\\
$(240, 71)$ & 12 & $(61, 18)$ & 9 & 1 & YES & NO & YES & NO & 776\\
$(242, 67)$ & 12 & $(5, 2)$ & 3 & 1 & YES & NO & YES & -- & 777\\
$(242, 67)$ & 12 & $(76, 21)$ & 9 & 2 & YES & NO & YES & 955 & 778\\
$(243, 38)$ & 16 & $(3, 1)$ & 2 & 3 & YES & YES & YES & NO & 779\\
$(243, 38)$ & 16 & $(5, 1)$ & 4 & 1 & YES & YES & YES & NO & 780\\
$(243, 71)$ & 12 & $(5, 2)$ & 3 & 1 & YES & NO & YES & -- & 781\\
$(243, 71)$ & 12 & $(7, 2)$ & 4 & 1 & YES & NO & YES & -- & 782\\
$(243, 71)$ & 12 & $(58, 17)$ & 9 & 1 & YES & NO & YES & 929 & 783\\
$(243, 71)$ & 12 & $(106, 31)$ & 10 & 1 & YES & NO & YES & NO & 784\\
$(244, 37)$ & 16 & $(3, 1)$ & 2 & 1 & YES & YES & YES & -- & 785\\
$(244, 37)$ & 16 & $(3, 1)$ & 2 & 1 & YES & YES & YES & NO & 786\\
$(244, 37)$ & 16 & $(5, 1)$ & 4 & 1 & YES & YES & YES & NO & 787\\
$(246, 73)$ & 12 & $(101, 30)$ & 10 & 1 & YES & NO & YES & NO & 788\\
$(249, 95)$ & 12 & $(3, 1)$ & 2 & 3 & YES & NO & YES & -- & 789\\
$(249, 95)$ & 12 & $(34, 13)$ & 7 & 1 & YES & NO & YES & NO & 790\\
$(251, 38)$ & 16 & $(3, 1)$ & 2 & 1 & YES & YES & YES & NO & 791\\
$(254, 71)$ & 12 & $(5, 2)$ & 3 & 1 & YES & NO & YES & -- & 792\\
$(257, 71)$ & 12 & $(5, 2)$ & 3 & 1 & YES & NO & YES & NO & 793\\
$(257, 76)$ & 12 & $(5, 2)$ & 3 & 1 & YES & NO & YES & -- & 794\\
$(257, 76)$ & 12 & $(11, 3)$ & 5 & 1 & YES & NO & YES & NO & 795\\
$(257, 76)$ & 12 & $(24, 7)$ & 7 & 1 & YES & NO & YES & NO & 796\\
$(257, 76)$ & 12 & $(61, 18)$ & 9 & 1 & YES & NO & YES & 885 & 797\\
$(257, 71)$ & 12 & $(65, 18)$ & 9 & 1 & YES & NO & YES & NO & 798\\
$(265, 111)$ & 12 & $(3, 1)$ & 2 & 1 & YES & NO & YES & -- & 799\\
$(265, 111)$ & 12 & $(3, 1)$ & 2 & 1 & YES & NO & YES & NO & 800\\
$(266, 79)$ & 12 & $(5, 2)$ & 3 & 1 & YES & NO & YES & -- & 801\\
$(266, 79)$ & 12 & $(11, 3)$ & 5 & 1 & YES & NO & YES & NO & 802\\
$(266, 79)$ & 12 & $(91, 27)$ & 10 & 7 & YES & NO & YES & NO & 803\\
$(266, 79)$ & 12 & $(138, 41)$ & 11 & 2 & YES & NO & YES & NO & 804\\
$(267, 98)$ & 12 & $(3, 1)$ & 2 & 3 & YES & NO & YES & -- & 805\\
$(267, 98)$ & 12 & $(5, 2)$ & 3 & 1 & YES & NO & YES & NO & 806\\
$(273, 76)$ & 13 & $(3, 1)$ & 2 & 3 & YES & NO & YES & -- & 807\\
$(273, 74)$ & 13 & $(4, 1)$ & 3 & 1 & YES & YES & YES & NO & 808\\
$(274, 107)$ & 12 & $(2, 1)$ & 1 & 2 & NO & YES & YES & -- & 809\\
$(274, 115)$ & 12 & $(2, 1)$ & 1 & 2 & YES & NO & YES & -- & 810\\
$(274, 81)$ & 12 & $(5, 2)$ & 3 & 1 & YES & NO & YES & -- & 811\\
$(274, 115)$ & 12 & $(5, 2)$ & 3 & 1 & YES & NO & YES & NO & 812\\
$(274, 115)$ & 12 & $(50, 21)$ & 8 & 2 & YES & NO & YES & NO & 813\\
$(274, 81)$ & 12 & $(98, 29)$ & 10 & 2 & YES & NO & YES & NO & 814\\
$(275, 76)$ & 12 & $(5, 2)$ & 3 & 5 & YES & NO & YES & -- & 815\\
$(275, 76)$ & 12 & $(5, 2)$ & 3 & 5 & YES & NO & YES & NO & 816\\
$(275, 76)$ & 12 & $(10, 3)$ & 5 & 5 & YES & NO & YES & NO & 817\\
$(277, 81)$ & 12 & $(5, 2)$ & 3 & 1 & YES & NO & YES & -- & 818\\
$(277, 81)$ & 12 & $(58, 17)$ & 9 & 1 & YES & NO & YES & NO & 819\\
$(277, 81)$ & 12 & $(147, 43)$ & 11 & 1 & YES & NO & YES & NO & 820\\
$(281, 38)$ & 17 & $(2, 1)$ & 1 & 1 & YES & YES & YES & NO & 821\\
$(281, 38)$ & 17 & $(244, 33)$ & 16 & 1 & YES & YES & YES & NO & 822\\
$(285, 53)$ & 15 & $(2, 1)$ & 1 & 1 & YES & YES & YES & NO & 823\\
$(285, 53)$ & 15 & $(43, 8)$ & 9 & 1 & YES & YES & YES & NO & 824\\
$(287, 79)$ & 12 & $(98, 27)$ & 10 & 7 & YES & NO & YES & NO & 825\\
$(290, 81)$ & 12 & $(5, 2)$ & 3 & 5 & YES & NO & YES & -- & 826\\
$(290, 81)$ & 12 & $(5, 2)$ & 3 & 5 & YES & NO & YES & NO & 827\\
$(290, 111)$ & 12 & $(128, 49)$ & 10 & 2 & YES & NO & YES & 870 & 828\\
$(293, 123)$ & 12 & $(3, 1)$ & 2 & 1 & YES & NO & YES & -- & 829\\
$(293, 123)$ & 12 & $(3, 1)$ & 2 & 1 & YES & NO & YES & NO & 830\\
$(294, 67)$ & 13 & $(5, 2)$ & 3 & 1 & YES & NO & YES & -- & 831\\
$(295, 87)$ & 13 & $(3, 1)$ & 2 & 1 & YES & NO & YES & NO & 832\\
$(295, 112)$ & 12 & $(4, 1)$ & 3 & 1 & YES & NO & YES & -- & 833\\
$(295, 112)$ & 12 & $(295, 112)$ & 12 & 295 & YES & NO & YES & NO & 834\\
$(297, 92)$ & 13 & $(3, 1)$ & 2 & 3 & YES & NO & YES & -- & 835\\
$(297, 92)$ & 13 & $(113, 35)$ & 11 & 1 & YES & NO & YES & NO & 836\\
$(298, 53)$ & 15 & $(2, 1)$ & 1 & 2 & YES & YES & YES & -- & 837\\
$(298, 53)$ & 15 & $(28, 5)$ & 8 & 2 & YES & YES & YES & NO & 838\\
$(298, 53)$ & 15 & $(45, 8)$ & 9 & 1 & YES & YES & YES & NO & 839\\
$(299, 116)$ & 12 & $(31, 12)$ & 7 & 1 & YES & NO & YES & NO & 840\\
$(299, 116)$ & 12 & $(183, 71)$ & 11 & 1 & YES & NO & YES & NO & 841\\
$(301, 89)$ & 12 & $(5, 2)$ & 3 & 1 & YES & NO & YES & -- & 842\\
$(301, 115)$ & 12 & $(123, 47)$ & 10 & 1 & YES & NO & YES & 866 & 843\\
$(307, 119)$ & 12 & $(3, 1)$ & 2 & 1 & YES & NO & YES & -- & 844\\
$(307, 119)$ & 12 & $(3, 1)$ & 2 & 1 & YES & NO & YES & NO & 845\\
$(307, 85)$ & 13 & $(4, 1)$ & 3 & 1 & YES & NO & YES & -- & 846\\
$(307, 85)$ & 13 & $(4, 1)$ & 3 & 1 & YES & NO & YES & NO & 847\\
$(307, 119)$ & 12 & $(307, 119)$ & 12 & 307 & YES & NO & YES & NO & 848\\
$(308, 129)$ & 12 & $(117, 49)$ & 10 & 1 & YES & NO & YES & NO & 849\\
$(311, 50)$ & 16 & $(3, 1)$ & 2 & 1 & YES & YES & YES & -- & 850\\
$(311, 71)$ & 13 & $(5, 2)$ & 3 & 1 & YES & NO & YES & -- & 851\\
$(311, 50)$ & 16 & $(13, 2)$ & 7 & 1 & YES & YES & YES & NO & 852\\
$(313, 121)$ & 12 & $(3, 1)$ & 2 & 1 & YES & NO & YES & -- & 853\\
$(313, 121)$ & 12 & $(18, 7)$ & 6 & 1 & YES & NO & YES & 874 & 854\\
$(317, 121)$ & 12 & $(34, 13)$ & 7 & 1 & YES & NO & YES & NO & 855\\
$(320, 93)$ & 13 & $(24, 7)$ & 7 & 8 & YES & NO & YES & NO & 856\\
$(322, 123)$ & 12 & $(2, 1)$ & 1 & 2 & YES & NO & YES & NO & 857\\
$(322, 89)$ & 12 & $(5, 2)$ & 3 & 1 & YES & NO & YES & -- & 858\\
$(322, 89)$ & 12 & $(5, 2)$ & 3 & 1 & YES & NO & YES & NO & 859\\
$(322, 89)$ & 12 & $(25, 7)$ & 7 & 1 & YES & NO & YES & NO & 860\\
$(322, 89)$ & 12 & $(65, 18)$ & 9 & 1 & YES & NO & YES & NO & 861\\
$(322, 123)$ & 12 & $(144, 55)$ & 10 & 2 & YES & NO & YES & 907 & 862\\
$(329, 61)$ & 15 & $(6, 1)$ & 5 & 1 & YES & YES & YES & NO & 863\\
$(333, 76)$ & 13 & $(5, 2)$ & 3 & 1 & YES & NO & YES & -- & 864\\
$(335, 128)$ & 12 & $(2, 1)$ & 1 & 1 & YES & NO & YES & -- & 865\\
$(335, 128)$ & 12 & $(89, 34)$ & 9 & 1 & YES & NO & YES & 843 & 866\\
$(335, 128)$ & 12 & $(123, 47)$ & 10 & 1 & YES & NO & YES & NO & 867\\
$(337, 129)$ & 12 & $(3, 1)$ & 2 & 1 & YES & NO & YES & -- & 868\\
$(337, 60)$ & 14 & $(6, 1)$ & 5 & 1 & YES & YES & YES & NO & 869\\
$(337, 129)$ & 12 & $(81, 31)$ & 9 & 1 & YES & NO & YES & 828 & 870\\
$(338, 131)$ & 12 & $(3, 1)$ & 2 & 1 & YES & NO & YES & -- & 871\\
$(338, 129)$ & 12 & $(5, 1)$ & 4 & 1 & YES & NO & YES & NO & 872\\
$(338, 131)$ & 12 & $(5, 1)$ & 4 & 1 & YES & NO & YES & NO & 873\\
$(338, 131)$ & 12 & $(13, 5)$ & 5 & 13 & YES & NO & YES & 854 & 874\\
$(340, 101)$ & 13 & $(17, 5)$ & 6 & 17 & YES & NO & YES & NO & 875\\
$(340, 101)$ & 13 & $(64, 19)$ & 9 & 4 & YES & NO & YES & NO & 876\\
$(340, 101)$ & 13 & $(138, 41)$ & 11 & 2 & YES & NO & YES & 908 & 877\\
$(343, 131)$ & 12 & $(3, 1)$ & 2 & 1 & YES & NO & YES & -- & 878\\
$(347, 97)$ & 13 & $(5, 1)$ & 4 & 1 & YES & NO & YES & NO & 879\\
$(347, 97)$ & 13 & $(11, 3)$ & 5 & 1 & YES & NO & YES & NO & 880\\
$(351, 76)$ & 13 & $(13, 3)$ & 6 & 13 & YES & NO & YES & NO & 881\\
$(359, 105)$ & 13 & $(3, 1)$ & 2 & 1 & YES & NO & YES & -- & 882\\
$(359, 106)$ & 13 & $(3, 1)$ & 2 & 1 & YES & NO & YES & -- & 883\\
$(359, 106)$ & 13 & $(3, 1)$ & 2 & 1 & YES & NO & YES & NO & 884\\
$(359, 106)$ & 13 & $(27, 8)$ & 7 & 1 & YES & NO & YES & 797 & 885\\
$(359, 106)$ & 13 & $(359, 106)$ & 13 & 359 & YES & NO & YES & NO & 886\\
$(367, 109)$ & 13 & $(3, 1)$ & 2 & 1 & YES & NO & YES & -- & 887\\
$(367, 109)$ & 13 & $(3, 1)$ & 2 & 1 & YES & NO & YES & NO & 888\\
$(367, 112)$ & 13 & $(3, 1)$ & 2 & 1 & YES & NO & YES & -- & 889\\
$(367, 109)$ & 13 & $(4, 1)$ & 3 & 1 & YES & NO & YES & -- & 890\\
$(367, 109)$ & 13 & $(4, 1)$ & 3 & 1 & YES & NO & YES & NO & 891\\
$(367, 109)$ & 13 & $(4, 1)$ & 3 & 1 & YES & NO & YES & NO & 892\\
$(367, 109)$ & 13 & $(7, 2)$ & 4 & 1 & YES & NO & YES & NO & 893\\
$(367, 112)$ & 13 & $(10, 3)$ & 5 & 1 & YES & NO & YES & NO & 894\\
$(367, 109)$ & 13 & $(27, 8)$ & 7 & 1 & YES & NO & YES & NO & 895\\
$(367, 109)$ & 13 & $(64, 19)$ & 9 & 1 & YES & NO & YES & NO & 896\\
$(367, 101)$ & 13 & $(69, 19)$ & 9 & 1 & YES & NO & YES & NO & 897\\
$(368, 107)$ & 13 & $(24, 7)$ & 7 & 8 & YES & NO & YES & NO & 898\\
$(373, 104)$ & 13 & $(373, 104)$ & 13 & 373 & YES & NO & YES & NO & 899\\
$(374, 111)$ & 13 & $(3, 1)$ & 2 & 1 & YES & NO & YES & -- & 900\\
$(374, 111)$ & 13 & $(17, 5)$ & 6 & 17 & YES & NO & YES & NO & 901\\
$(374, 111)$ & 13 & $(91, 27)$ & 10 & 1 & YES & NO & YES & NO & 902\\
$(376, 105)$ & 13 & $(3, 1)$ & 2 & 1 & YES & NO & YES & -- & 903\\
$(376, 105)$ & 13 & $(68, 19)$ & 9 & 4 & YES & NO & YES & NO & 904\\
$(377, 112)$ & 13 & $(4, 1)$ & 3 & 1 & YES & NO & YES & -- & 905\\
$(377, 144)$ & 12 & $(5, 1)$ & 4 & 1 & YES & NO & YES & NO & 906\\
$(377, 144)$ & 12 & $(89, 34)$ & 9 & 1 & YES & NO & YES & 862 & 907\\
$(377, 112)$ & 13 & $(101, 30)$ & 10 & 1 & YES & NO & YES & 877 & 908\\
$(377, 144)$ & 12 & $(144, 55)$ & 10 & 1 & YES & NO & YES & NO & 909\\
$(379, 111)$ & 13 & $(10, 3)$ & 5 & 1 & YES & NO & YES & NO & 910\\
$(379, 111)$ & 13 & $(58, 17)$ & 9 & 1 & YES & NO & YES & 945 & 911\\
$(383, 106)$ & 13 & $(3, 1)$ & 2 & 1 & YES & NO & YES & -- & 912\\
$(383, 106)$ & 13 & $(3, 1)$ & 2 & 1 & YES & NO & YES & NO & 913\\
$(383, 112)$ & 13 & $(3, 1)$ & 2 & 1 & YES & NO & YES & -- & 914\\
$(383, 112)$ & 13 & $(3, 1)$ & 2 & 1 & YES & NO & YES & NO & 915\\
$(383, 112)$ & 13 & $(4, 1)$ & 3 & 1 & YES & NO & YES & -- & 916\\
$(383, 112)$ & 13 & $(4, 1)$ & 3 & 1 & YES & NO & YES & NO & 917\\
$(383, 106)$ & 13 & $(65, 18)$ & 9 & 1 & YES & NO & YES & NO & 918\\
$(389, 115)$ & 13 & $(3, 1)$ & 2 & 1 & YES & NO & YES & -- & 919\\
$(389, 115)$ & 13 & $(10, 3)$ & 5 & 1 & YES & NO & YES & NO & 920\\
$(391, 108)$ & 13 & $(3, 1)$ & 2 & 1 & YES & NO & YES & -- & 921\\
$(391, 108)$ & 13 & $(3, 1)$ & 2 & 1 & YES & NO & YES & NO & 922\\
$(391, 109)$ & 13 & $(3, 1)$ & 2 & 1 & YES & NO & YES & -- & 923\\
$(391, 109)$ & 13 & $(104, 29)$ & 10 & 1 & YES & NO & YES & NO & 924\\
$(396, 109)$ & 13 & $(3, 1)$ & 2 & 3 & YES & NO & YES & -- & 925\\
$(397, 116)$ & 13 & $(2, 1)$ & 1 & 1 & YES & NO & YES & -- & 926\\
$(397, 116)$ & 13 & $(3, 1)$ & 2 & 1 & YES & NO & YES & -- & 927\\
$(397, 116)$ & 13 & $(4, 1)$ & 3 & 1 & YES & NO & YES & -- & 928\\
$(397, 116)$ & 13 & $(17, 5)$ & 6 & 1 & YES & NO & YES & 783 & 929\\
$(397, 116)$ & 13 & $(397, 116)$ & 13 & 397 & YES & NO & YES & NO & 930\\
$(398, 111)$ & 13 & $(3, 1)$ & 2 & 1 & YES & NO & YES & -- & 931\\
$(398, 111)$ & 13 & $(61, 17)$ & 9 & 1 & YES & NO & YES & 963 & 932\\
$(401, 112)$ & 13 & $(3, 1)$ & 2 & 1 & YES & NO & YES & -- & 933\\
$(401, 112)$ & 13 & $(3, 1)$ & 2 & 1 & YES & NO & YES & NO & 934\\
$(401, 112)$ & 13 & $(5, 1)$ & 4 & 1 & YES & NO & YES & NO & 935\\
$(401, 112)$ & 13 & $(25, 7)$ & 7 & 1 & YES & NO & YES & NO & 936\\
$(403, 119)$ & 13 & $(2, 1)$ & 1 & 1 & YES & NO & YES & -- & 937\\
$(403, 111)$ & 13 & $(3, 1)$ & 2 & 1 & YES & NO & YES & -- & 938\\
$(403, 119)$ & 13 & $(10, 3)$ & 5 & 1 & YES & NO & YES & NO & 939\\
$(403, 111)$ & 13 & $(98, 27)$ & 10 & 1 & YES & NO & YES & NO & 940\\
$(409, 121)$ & 13 & $(17, 5)$ & 6 & 1 & YES & NO & YES & NO & 941\\
$(413, 121)$ & 13 & $(2, 1)$ & 1 & 1 & YES & NO & YES & -- & 942\\
$(413, 121)$ & 13 & $(3, 1)$ & 2 & 1 & YES & NO & YES & -- & 943\\
$(413, 121)$ & 13 & $(3, 1)$ & 2 & 1 & YES & NO & YES & NO & 944\\
$(413, 121)$ & 13 & $(41, 12)$ & 8 & 1 & YES & NO & YES & 911 & 945\\
$(413, 121)$ & 13 & $(58, 17)$ & 9 & 1 & YES & NO & YES & NO & 946\\
$(415, 116)$ & 13 & $(2, 1)$ & 1 & 1 & YES & NO & YES & NO & 947\\
$(415, 116)$ & 13 & $(3, 1)$ & 2 & 1 & YES & NO & YES & NO & 948\\
$(415, 116)$ & 13 & $(4, 1)$ & 3 & 1 & YES & NO & YES & -- & 949\\
$(415, 116)$ & 13 & $(68, 19)$ & 9 & 1 & YES & NO & YES & NO & 950\\
$(416, 115)$ & 13 & $(3, 1)$ & 2 & 1 & YES & NO & YES & -- & 951\\
$(416, 115)$ & 13 & $(3, 1)$ & 2 & 1 & YES & NO & YES & NO & 952\\
$(416, 115)$ & 13 & $(3, 1)$ & 2 & 1 & YES & NO & YES & NO & 953\\
$(416, 123)$ & 13 & $(17, 5)$ & 6 & 1 & YES & NO & YES & NO & 954\\
$(416, 115)$ & 13 & $(18, 5)$ & 6 & 2 & YES & NO & YES & 778 & 955\\
$(416, 115)$ & 13 & $(29, 8)$ & 7 & 1 & YES & NO & YES & NO & 956\\
$(416, 123)$ & 13 & $(71, 21)$ & 9 & 1 & YES & NO & YES & NO & 957\\
$(416, 115)$ & 13 & $(76, 21)$ & 9 & 4 & YES & NO & YES & NO & 958\\
$(429, 97)$ & 14 & $(53, 12)$ & 9 & 1 & YES & NO & YES & NO & 959\\
$(434, 121)$ & 13 & $(3, 1)$ & 2 & 1 & YES & NO & YES & -- & 960\\
$(434, 121)$ & 13 & $(3, 1)$ & 2 & 1 & YES & NO & YES & NO & 961\\
$(434, 121)$ & 13 & $(25, 7)$ & 7 & 1 & YES & NO & YES & 971 & 962\\
$(434, 121)$ & 13 & $(43, 12)$ & 8 & 1 & YES & NO & YES & 932 & 963\\
$(445, 123)$ & 13 & $(2, 1)$ & 1 & 1 & YES & NO & YES & -- & 964\\
$(445, 123)$ & 13 & $(2, 1)$ & 1 & 1 & YES & NO & YES & NO & 965\\
$(445, 123)$ & 13 & $(3, 1)$ & 2 & 1 & YES & NO & YES & -- & 966\\
$(445, 123)$ & 13 & $(3, 1)$ & 2 & 1 & YES & NO & YES & NO & 967\\
$(445, 123)$ & 13 & $(18, 5)$ & 6 & 1 & YES & NO & YES & NO & 968\\
$(469, 131)$ & 13 & $(2, 1)$ & 1 & 1 & YES & NO & YES & -- & 969\\
$(469, 131)$ & 13 & $(3, 1)$ & 2 & 1 & YES & NO & YES & -- & 970\\
$(469, 131)$ & 13 & $(18, 5)$ & 6 & 1 & YES & NO & YES & 962 & 971\\
$(469, 131)$ & 13 & $(68, 19)$ & 9 & 1 & YES & NO & YES & NO & 972\\
$(473, 108)$ & 14 & $(2, 1)$ & 1 & 1 & YES & NO & YES & NO & 973\\
$(473, 108)$ & 14 & $(22, 5)$ & 7 & 11 & YES & NO & YES & NO & 974\\
$(473, 108)$ & 14 & $(473, 108)$ & 14 & 473 & YES & NO & YES & NO & 975\\
$(474, 199)$ & 13 & $(2, 1)$ & 1 & 2 & NO & NO & YES & -- & 976\\
$(495, 112)$ & 14 & $(5, 1)$ & 4 & 5 & YES & NO & YES & NO & 977\\
$(504, 115)$ & 14 & $(35, 8)$ & 8 & 7 & YES & NO & YES & NO & 978\\
$(517, 117)$ & 14 & $(3, 1)$ & 2 & 1 & YES & NO & YES & -- & 979\\
$(522, 119)$ & 14 & $(35, 8)$ & 8 & 1 & YES & NO & YES & NO & 980\\
$(561, 128)$ & 14 & $(2, 1)$ & 1 & 1 & YES & NO & YES & NO & 981\\
$(561, 128)$ & 14 & $(22, 5)$ & 7 & 11 & YES & NO & YES & NO & 982\\
$(561, 128)$ & 14 & $(35, 8)$ & 8 & 1 & YES & NO & YES & NO & 983\\
$(643, 177)$ & 14 & $(4, 1)$ & 3 & 1 & NO & NO & YES & -- & 984\\
$(a; 4, 1, 1; 55)$ & 10 & $(5, 2)$ & 3 & 5 & YES & YES & YES & -- & 985\\
$(b; 0, 0, 0; 14)$ & 5 & $(19, 5)$ & 7 & 1 & YES & YES & YES & -- & 986\\
$(b; 0, 1, 3; 43)$ & 9 & $(11, 3)$ & 5 & 1 & YES & YES & YES & -- & 987\\
$(b; 0, 1, 5; 59)$ & 11 & $(2, 1)$ & 1 & 1 & YES & YES & YES & -- & 988\\
$(b; 0, 1, 5; 59)$ & 11 & $(7, 1)$ & 6 & 1 & YES & YES & YES & -- & 989\\
$(b; 1, 0, 5; 65)$ & 11 & $(2, 1)$ & 1 & 1 & YES & YES & YES & -- & 990\\
$(b; 3, 1, 0; 43)$ & 9 & $(10, 3)$ & 5 & 1 & YES & YES & YES & -- & 991\\
$(c; 0, 0, 0; 4)$ & 4 & $(59, 18)$ & 9 & 1 & YES & YES & YES & -- & 992\\
$(c; 0, 0, 0; 4)$ & 4 & $(71, 30)$ & 9 & 1 & YES & YES & YES & -- & 993\\
$(c; 0, 0, 0; 4)$ & 4 & $(71, 32)$ & 10 & 1 & YES & YES & YES & -- & 994\\
$(c; 0, 0, 0; 4)$ & 4 & $(105, 44)$ & 10 & 1 & YES & NO & YES & -- & 995\\
$(c; 0, 0, 0; 4)$ & 4 & $(113, 35)$ & 11 & 1 & YES & YES & YES & -- & 996\\
$(c; 0, 0, 0; 4)$ & 4 & $(115, 44)$ & 10 & 1 & YES & NO & YES & -- & 997\\
$(c; 0, 0, 0; 4)$ & 4 & $(119, 46)$ & 10 & 1 & YES & NO & YES & -- & 998\\
$(c; 0, 0, 0; 4)$ & 4 & $(123, 28)$ & 12 & 1 & YES & NO & YES & -- & 999\\
$(c; 0, 0, 0; 4)$ & 4 & $(138, 41)$ & 11 & 2 & YES & NO & YES & -- & 1000\\
$(c; 0, 0, 0; 4)$ & 4 & $(144, 55)$ & 10 & 4 & YES & NO & YES & -- & 1001\\
$(c; 0, 0, 0; 4)$ & 4 & $(146, 41)$ & 11 & 2 & YES & NO & YES & -- & 1002\\
$(c; 0, 0, 0; 4)$ & 4 & $(159, 44)$ & 11 & 1 & YES & NO & YES & -- & 1003\\
$(c; 0, 0, 0; 4)$ & 4 & $(171, 50)$ & 11 & 1 & YES & NO & YES & -- & 1004\\
$(c; 0, 1, 0; 11)$ & 5 & $(64, 27)$ & 9 & 1 & YES & NO & YES & -- & 1005\\
$(c; 0, 1, 0; 11)$ & 5 & $(81, 31)$ & 9 & 1 & YES & NO & YES & -- & 1006\\
$(c; 0, 2, 0; 7)$ & 6 & $(56, 15)$ & 9 & 7 & YES & NO & YES & -- & 1007\\
$(c; 0, 2, 0; 7)$ & 6 & $(60, 13)$ & 9 & 1 & YES & YES & YES & -- & 1008\\
$(d; 0, 0, 0; 5)$ & 5 & $(69, 29)$ & 9 & 1 & YES & NO & YES & -- & 1009\\
$(d; 0, 0, 0; 5)$ & 5 & $(76, 29)$ & 9 & 1 & YES & NO & YES & -- & 1010\\
$(d; 0, 0, 1; 14)$ & 6 & $(55, 21)$ & 8 & 1 & YES & NO & YES & -- & 1011\\
$(e; 5, 1, 0; 55)$ & 11 & $(2, 1)$ & 1 & 1 & YES & YES & YES & -- & 1012\\
$(e; 5, 1, 0; 55)$ & 11 & $(3, 1)$ & 2 & 1 & YES & YES & YES & -- & 1013\\
$(f; 0, 0, 0; 6)$ & 4 & $(65, 19)$ & 9 & 1 & YES & YES & YES & -- & 1014\\
$(f; 0, 0, 0; 6)$ & 4 & $(68, 19)$ & 9 & 2 & YES & YES & YES & -- & 1015\\
$(f; 0, 0, 0; 6)$ & 4 & $(99, 41)$ & 10 & 3 & YES & YES & YES & -- & 1016\\
$(f; 0, 1, 0; 7)$ & 5 & $(37, 11)$ & 8 & 1 & YES & YES & YES & -- & 1017\\
$(g; 0, 0, 0; 19)$ & 6 & $(31, 12)$ & 7 & 1 & YES & NO & YES & -- & 1018\\
$(g; 0, 2, 0; 29)$ & 8 & $(7, 2)$ & 4 & 1 & YES & YES & YES & -- & 1019\\
$(g; 0, 2, 0; 29)$ & 8 & $(8, 3)$ & 4 & 1 & YES & YES & YES & -- & 1020\\
$(g; 1, 0, 0; 7)$ & 7 & $(12, 5)$ & 5 & 1 & YES & YES & YES & -- & 1021\\
$(g; 2, 0, 0; 37)$ & 8 & $(7, 3)$ & 4 & 1 & YES & YES & YES & -- & 1022\\
$(g; 2, 0, 1; 10)$ & 9 & $(10, 3)$ & 5 & 10 & YES & NO & YES & -- & 1023\\
$(g; 2, 0, 2; 63)$ & 10 & $(3, 1)$ & 2 & 3 & YES & YES & YES & -- & 1024\\
$(h; 0, 3, 0; 12)$ & 8 & $(7, 2)$ & 4 & 1 & YES & YES & YES & -- & 1025\\
$(j; 0, 0, 0; 8)$ & 5 & $(76, 29)$ & 9 & 4 & YES & YES & YES & -- & 1026\\
$(j; 0, 1, 0; 10)$ & 6 & $(58, 17)$ & 9 & 2 & YES & YES & YES & -- & 1027\\
$(j; 0, 1, 0; 10)$ & 6 & $(61, 17)$ & 9 & 1 & YES & YES & YES & -- & 1028\\
$(j; 0, 3, 0; 14)$ & 8 & $(25, 7)$ & 7 & 1 & YES & YES & YES & -- & 1029
\end{longtable}
\subsection{2 chains, $K^2 = 4$}
\begin{longtable}{|c|c|c|c|c|c|c|c|c|c|}
\hline
\multicolumn{10}{|c|}{2 chains, $K^2 = 4$}\\
\hline
$(n,a)$ & Length & $(n,a)$ & Length & GCD & Nef & $\mathbb Q$-ef & Obstruction 0 & WH & Index\\
\hline
\endfirsthead

\hline
$(n,a)$ & Length & $(n,a)$ & Length & GCD & Nef & $\mathbb Q$-ef & Obstruction 0 & WH & Index\\
\hline
\endhead
\hline
\endfoot

$(82, 31)$ & 10 & $(18, 5)$ & 6 & 2 & YES & YES & YES & -- & 1030\\
$(92, 27)$ & 11 & $(9, 2)$ & 5 & 1 & YES & YES & YES & NO & 1031\\
$(103, 24)$ & 11 & $(26, 11)$ & 7 & 1 & YES & YES & YES & NO & 1032\\
$(121, 46)$ & 10 & $(82, 31)$ & 10 & 1 & YES & YES & YES & NO & 1033\\
$(129, 50)$ & 10 & $(12, 5)$ & 5 & 3 & YES & YES & YES & -- & 1034\\
$(129, 50)$ & 10 & $(12, 5)$ & 5 & 3 & YES & YES & YES & NO & 1035\\
$(215, 83)$ & 12 & $(13, 4)$ & 6 & 1 & YES & YES & YES & -- & 1036\\
$(312, 119)$ & 13 & $(9, 4)$ & 5 & 3 & YES & YES & YES & NO & 1037\\
$(337, 141)$ & 13 & $(26, 11)$ & 7 & 1 & YES & YES & YES & NO & 1038\\
$(434, 179)$ & 14 & $(7, 3)$ & 4 & 7 & YES & YES & YES & NO & 1039\\
$(478, 201)$ & 14 & $(2, 1)$ & 1 & 2 & YES & YES & YES & -- & 1040\\
$(527, 201)$ & 14 & $(3, 1)$ & 2 & 1 & YES & YES & YES & NO & 1041\\
$(553, 167)$ & 15 & $(3, 1)$ & 2 & 1 & YES & YES & YES & -- & 1042\\
$(574, 155)$ & 14 & $(311, 84)$ & 13 & 1 & YES & NO & YES & NO & 1043\\
$(581, 222)$ & 14 & $(7, 1)$ & 6 & 7 & YES & YES & YES & NO & 1044\\
$(595, 111)$ & 16 & $(49, 9)$ & 10 & 7 & YES & YES & YES & NO & 1045\\
$(772, 279)$ & 15 & $(4, 1)$ & 3 & 4 & YES & NO & YES & NO & 1046\\
$(835, 323)$ & 15 & $(3, 1)$ & 2 & 1 & YES & NO & YES & -- & 1047\\
$(843, 326)$ & 15 & $(3, 1)$ & 2 & 3 & YES & NO & YES & -- & 1048\\
$(870, 269)$ & 16 & $(42, 13)$ & 9 & 6 & YES & YES & YES & NO & 1049\\
$(907, 335)$ & 15 & $(5, 1)$ & 4 & 1 & YES & NO & YES & -- & 1050\\
$(923, 259)$ & 16 & $(11, 3)$ & 5 & 1 & YES & YES & YES & NO & 1051\\
$(945, 388)$ & 15 & $(945, 388)$ & 15 & 945 & YES & NO & YES & NO & 1052\\
$(955, 268)$ & 16 & $(2, 1)$ & 1 & 1 & YES & YES & YES & NO & 1053\\
$(1057, 321)$ & 16 & $(3, 1)$ & 2 & 1 & YES & NO & YES & -- & 1054\\
$(1058, 409)$ & 15 & $(2, 1)$ & 1 & 2 & YES & NO & YES & -- & 1055\\
$(1081, 327)$ & 16 & $(5, 1)$ & 4 & 1 & YES & YES & YES & -- & 1056\\
$(1081, 327)$ & 16 & $(400, 121)$ & 14 & 1 & YES & YES & YES & NO & 1057\\
$(1168, 183)$ & 18 & $(2, 1)$ & 1 & 2 & YES & YES & YES & -- & 1058\\
$(1212, 217)$ & 18 & $(1212, 217)$ & 18 & 1212 & YES & NO & YES & NO & 1059\\
$(1223, 285)$ & 17 & $(17, 4)$ & 7 & 1 & YES & YES & YES & NO & 1060\\
$(1237, 345)$ & 16 & $(7, 2)$ & 4 & 1 & YES & NO & YES & NO & 1061\\
$(1783, 331)$ & 18 & $(27, 5)$ & 8 & 1 & YES & NO & YES & NO & 1062\\
$(a; 0, 0, 0; 3)$ & 4 & $(265, 62)$ & 14 & 1 & YES & NO & YES & -- & 1063\\
$(b; 0, 0, 0; 14)$ & 5 & $(51, 14)$ & 9 & 1 & YES & YES & YES & -- & 1064\\
$(h; 0, 0, 0; 6)$ & 5 & $(65, 18)$ & 9 & 1 & YES & NO & YES & -- & 1065
\end{longtable}


%%%%%%%%%%%%%%%%%%%%%%%%%%%%%%%%%%%%%%%%%%%
\section{$I_2^* + 2I_2$}
Input:
\lstinputlisting[language=config]{../Tests/2s22.txt}
Result:
%\usepackage{longtable}
\subsection{2 chains, $K^2 = 1$}
\begin{longtable}{|c|c|c|c|c|c|c|c|c|c|}
\hline
\multicolumn{10}{|c|}{2 chains, $K^2 = 1$}\\
\hline
$(n,a)$ & Length & $(n,a)$ & Length & GCD & Nef & $\mathbb Q$-ef & Obstruction 0 & WH & Index\\
\hline
\endfirsthead

\hline
$(n,a)$ & Length & $(n,a)$ & Length & GCD & Nef & $\mathbb Q$-ef & Obstruction 0 & WH & Index\\
\hline
\endhead
\hline
\endfoot

$(8, 3)$ & 4 & $(5, 2)$ & 3 & 1 & YES & YES & YES & -- & 1\\
$(a; 0, 0, 0; 3)$ & 4 & $(4, 1)$ & 3 & 1 & YES & YES & YES & -- & 2
\end{longtable}
\subsection{2 chains, $K^2 = 2$}
\begin{longtable}{|c|c|c|c|c|c|c|c|c|c|}
\hline
\multicolumn{10}{|c|}{2 chains, $K^2 = 2$}\\
\hline
$(n,a)$ & Length & $(n,a)$ & Length & GCD & Nef & $\mathbb Q$-ef & Obstruction 0 & WH & Index\\
\hline
\endfirsthead

\hline
$(n,a)$ & Length & $(n,a)$ & Length & GCD & Nef & $\mathbb Q$-ef & Obstruction 0 & WH & Index\\
\hline
\endhead
\hline
\endfoot

$(16, 5)$ & 7 & $(11, 3)$ & 5 & 1 & YES & YES & YES & -- & 3\\
$(27, 8)$ & 7 & $(4, 1)$ & 3 & 1 & YES & YES & YES & -- & 4\\
$(27, 8)$ & 7 & $(4, 1)$ & 3 & 1 & YES & YES & YES & NO & 5\\
$(27, 8)$ & 7 & $(5, 1)$ & 4 & 1 & YES & YES & YES & NO & 6\\
$(27, 8)$ & 7 & $(6, 1)$ & 5 & 3 & YES & YES & YES & -- & 7\\
$(27, 8)$ & 7 & $(6, 1)$ & 5 & 3 & YES & YES & YES & NO & 8\\
$(27, 8)$ & 7 & $(6, 1)$ & 5 & 3 & YES & YES & YES & NO & 9\\
$(37, 8)$ & 8 & $(3, 1)$ & 2 & 1 & YES & YES & YES & -- & 10\\
$(37, 8)$ & 8 & $(3, 1)$ & 2 & 1 & YES & YES & YES & NO & 11\\
$(39, 14)$ & 8 & $(3, 1)$ & 2 & 3 & YES & YES & YES & -- & 12\\
$(39, 14)$ & 8 & $(25, 9)$ & 7 & 1 & YES & YES & YES & NO & 13\\
$(42, 13)$ & 9 & $(3, 1)$ & 2 & 3 & YES & YES & YES & -- & 14\\
$(42, 13)$ & 9 & $(16, 5)$ & 7 & 2 & YES & YES & YES & 16 & 15\\
$(45, 14)$ & 9 & $(13, 4)$ & 6 & 1 & YES & YES & YES & 15 & 16\\
$(49, 18)$ & 8 & $(3, 1)$ & 2 & 1 & YES & YES & YES & -- & 17\\
$(d; 0, 0, 0; 5)$ & 5 & $(9, 2)$ & 5 & 1 & YES & YES & YES & -- & 18
\end{longtable}
\subsection{2 chains, $K^2 = 3$}
\begin{longtable}{|c|c|c|c|c|c|c|c|c|c|}
\hline
\multicolumn{10}{|c|}{2 chains, $K^2 = 3$}\\
\hline
$(n,a)$ & Length & $(n,a)$ & Length & GCD & Nef & $\mathbb Q$-ef & Obstruction 0 & WH & Index\\
\hline
\endfirsthead

\hline
$(n,a)$ & Length & $(n,a)$ & Length & GCD & Nef & $\mathbb Q$-ef & Obstruction 0 & WH & Index\\
\hline
\endhead
\hline
\endfoot

$(47, 20)$ & 10 & $(11, 3)$ & 5 & 1 & YES & YES & YES & -- & 19\\
$(61, 16)$ & 10 & $(12, 5)$ & 5 & 1 & YES & YES & YES & -- & 20\\
$(61, 16)$ & 10 & $(23, 7)$ & 7 & 1 & YES & YES & YES & NO & 21\\
$(82, 37)$ & 10 & $(4, 1)$ & 3 & 2 & YES & YES & YES & -- & 22\\
$(82, 37)$ & 10 & $(4, 1)$ & 3 & 2 & YES & YES & YES & NO & 23\\
$(85, 18)$ & 10 & $(5, 2)$ & 3 & 5 & YES & YES & YES & -- & 24\\
$(85, 18)$ & 10 & $(5, 2)$ & 3 & 5 & YES & YES & YES & NO & 25\\
$(121, 36)$ & 11 & $(23, 7)$ & 7 & 1 & YES & YES & YES & NO & 26\\
$(127, 54)$ & 12 & $(3, 1)$ & 2 & 1 & YES & YES & YES & -- & 27\\
$(127, 27)$ & 11 & $(4, 1)$ & 3 & 1 & YES & YES & YES & -- & 28\\
$(127, 27)$ & 11 & $(4, 1)$ & 3 & 1 & YES & YES & YES & NO & 29\\
$(177, 74)$ & 12 & $(12, 5)$ & 5 & 3 & YES & YES & YES & NO & 30\\
$(181, 54)$ & 13 & $(3, 1)$ & 2 & 1 & YES & YES & YES & NO & 31\\
$(a; 1, 0, 0; 13)$ & 5 & $(33, 14)$ & 8 & 1 & YES & YES & YES & -- & 32\\
$(c; 0, 2, 0; 7)$ & 6 & $(62, 13)$ & 10 & 1 & YES & YES & YES & -- & 33
\end{longtable}


%%%%%%%%%%%%%%%%%%%%%%%%%%%%%%%%%%%%%%%%%%%
\section{$I_1^* + I_4 + I_1$}
Input:
\lstinputlisting[language=config]{../Tests/1s41.txt}
Result:
%\usepackage{longtable}
\subsection{1 chain, $K^2 = 2$}
\begin{longtable}{|c|c|c|c|c|c|}
\hline
\multicolumn{6}{|c|}{1 chain, $K^2 = 2$}\\
\hline
$(n,a)$ & Length & Nef & $\mathbb Q$-ef & Obstruction 0 & Index\\
\hline
\endfirsthead

\hline
$(n,a)$ & Length & Nef & $\mathbb Q$-ef & Obstruction 0 & Index\\
\hline
\endhead
\hline
\endfoot

$(49, 19)$ & 8 & YES & YES & YES & 1\\
$(50, 19)$ & 8 & YES & YES & YES & 2\\
$(58, 17)$ & 9 & YES & YES & YES & 3\\
$(62, 23)$ & 9 & YES & YES & YES & 4\\
$(65, 24)$ & 9 & YES & YES & YES & 5\\
$(66, 25)$ & 9 & YES & YES & YES & 6\\
$(68, 25)$ & 9 & YES & YES & YES & 7\\
$(79, 30)$ & 9 & YES & YES & YES & 8\\
$(81, 31)$ & 9 & YES & YES & YES & 9\\
$(89, 26)$ & 10 & YES & YES & YES & 10\\
$(119, 26)$ & 11 & YES & YES & YES & 11\\
$(131, 30)$ & 11 & YES & YES & YES & 12\\
$(a; 2, 1, 1; 37)$ & 8 & YES & YES & YES & 13\\
$(g; 0, 1, 1; 33)$ & 8 & YES & YES & YES & 14
\end{longtable}
\subsection{1 chain, $K^2 = 3$}
\begin{longtable}{|c|c|c|c|c|c|}
\hline
\multicolumn{6}{|c|}{1 chain, $K^2 = 3$}\\
\hline
$(n,a)$ & Length & Nef & $\mathbb Q$-ef & Obstruction 0 & Index\\
\hline
\endfirsthead

\hline
$(n,a)$ & Length & Nef & $\mathbb Q$-ef & Obstruction 0 & Index\\
\hline
\endhead
\hline
\endfoot

$(157, 99)$ & 11 & YES & YES & YES & 15\\
$(166, 105)$ & 11 & YES & YES & YES & 16\\
$(193, 81)$ & 11 & YES & YES & YES & 17\\
$(219, 154)$ & 12 & YES & YES & YES & 18\\
$(225, 98)$ & 12 & YES & YES & YES & 19
\end{longtable}
\subsection{1 chain, $K^2 = 4$}
\begin{longtable}{|c|c|c|c|c|c|}
\hline
\multicolumn{6}{|c|}{1 chain, $K^2 = 4$}\\
\hline
$(n,a)$ & Length & Nef & $\mathbb Q$-ef & Obstruction 0 & Index\\
\hline
\endfirsthead

\hline
$(n,a)$ & Length & Nef & $\mathbb Q$-ef & Obstruction 0 & Index\\
\hline
\endhead
\hline
\endfoot

$(541, 236)$ & 14 & YES & YES & YES & 20\\
$(551, 240)$ & 14 & YES & YES & YES & 21\\
$(613, 269)$ & 14 & YES & YES & YES & 22\\
$(640, 243)$ & 14 & YES & YES & YES & 23\\
$(658, 255)$ & 14 & YES & YES & YES & 24\\
$(781, 324)$ & 15 & YES & YES & YES & 25\\
$(822, 341)$ & 15 & YES & YES & YES & 26
\end{longtable}
\subsection{2 chains, $K^2 = 2$}
\begin{longtable}{|c|c|c|c|c|c|c|c|c|c|}
\hline
\multicolumn{10}{|c|}{2 chains, $K^2 = 2$}\\
\hline
$(n,a)$ & Length & $(n,a)$ & Length & GCD & Nef & $\mathbb Q$-ef & Obstruction 0 & WH & Index\\
\hline
\endfirsthead

\hline
$(n,a)$ & Length & $(n,a)$ & Length & GCD & Nef & $\mathbb Q$-ef & Obstruction 0 & WH & Index\\
\hline
\endhead
\hline
\endfoot

$(13, 5)$ & 5 & $(10, 3)$ & 5 & 1 & YES & YES & YES & NO & 27\\
$(13, 5)$ & 5 & $(10, 3)$ & 5 & 1 & YES & YES & YES & NO & 28\\
$(13, 5)$ & 5 & $(13, 4)$ & 6 & 13 & YES & YES & YES & NO & 29\\
$(13, 5)$ & 5 & $(13, 4)$ & 6 & 13 & YES & YES & YES & NO & 30\\
$(16, 7)$ & 6 & $(13, 5)$ & 5 & 1 & YES & YES & YES & NO & 31\\
$(17, 7)$ & 6 & $(10, 3)$ & 5 & 1 & YES & YES & YES & NO & 32\\
$(17, 7)$ & 6 & $(10, 3)$ & 5 & 1 & YES & YES & YES & NO & 33\\
$(21, 8)$ & 6 & $(8, 3)$ & 4 & 1 & YES & YES & YES & NO & 34\\
$(21, 8)$ & 6 & $(10, 3)$ & 5 & 1 & YES & YES & YES & NO & 35\\
$(23, 10)$ & 7 & $(10, 3)$ & 5 & 1 & YES & YES & YES & NO & 36\\
$(23, 10)$ & 7 & $(10, 3)$ & 5 & 1 & YES & YES & YES & NO & 37\\
$(23, 10)$ & 7 & $(13, 3)$ & 6 & 1 & YES & YES & YES & NO & 38\\
$(23, 5)$ & 7 & $(16, 5)$ & 7 & 1 & YES & YES & YES & NO & 39\\
$(24, 7)$ & 7 & $(5, 2)$ & 3 & 1 & YES & YES & YES & NO & 40\\
$(24, 7)$ & 7 & $(5, 2)$ & 3 & 1 & YES & YES & YES & NO & 41\\
$(24, 7)$ & 7 & $(17, 5)$ & 6 & 1 & YES & YES & YES & NO & 42\\
$(26, 11)$ & 7 & $(10, 3)$ & 5 & 2 & YES & YES & YES & NO & 43\\
$(26, 11)$ & 7 & $(13, 3)$ & 6 & 13 & YES & YES & YES & NO & 44\\
$(27, 10)$ & 7 & $(5, 2)$ & 3 & 1 & YES & YES & YES & NO & 45\\
$(27, 10)$ & 7 & $(5, 2)$ & 3 & 1 & YES & YES & YES & NO & 46\\
$(27, 10)$ & 7 & $(7, 2)$ & 4 & 1 & YES & YES & YES & NO & 47\\
$(28, 11)$ & 8 & $(11, 2)$ & 6 & 1 & YES & YES & YES & NO & 48\\
$(29, 11)$ & 7 & $(4, 1)$ & 3 & 1 & YES & YES & YES & NO & 49\\
$(29, 8)$ & 7 & $(5, 2)$ & 3 & 1 & YES & YES & YES & NO & 50\\
$(29, 8)$ & 7 & $(5, 2)$ & 3 & 1 & YES & YES & YES & NO & 51\\
$(29, 12)$ & 7 & $(5, 2)$ & 3 & 1 & YES & YES & YES & NO & 52\\
$(29, 11)$ & 7 & $(7, 2)$ & 4 & 1 & YES & YES & YES & NO & 53\\
$(29, 11)$ & 7 & $(7, 2)$ & 4 & 1 & YES & YES & YES & NO & 54\\
$(29, 11)$ & 7 & $(7, 3)$ & 4 & 1 & YES & YES & YES & NO & 55\\
$(29, 12)$ & 7 & $(8, 3)$ & 4 & 1 & YES & YES & YES & NO & 56\\
$(29, 8)$ & 7 & $(9, 4)$ & 5 & 1 & YES & YES & YES & NO & 57\\
$(29, 8)$ & 7 & $(9, 4)$ & 5 & 1 & YES & YES & YES & NO & 58\\
$(29, 8)$ & 7 & $(9, 4)$ & 5 & 1 & YES & YES & YES & NO & 59\\
$(29, 11)$ & 7 & $(9, 2)$ & 5 & 1 & YES & YES & YES & NO & 60\\
$(29, 8)$ & 7 & $(10, 3)$ & 5 & 1 & YES & YES & YES & NO & 61\\
$(29, 8)$ & 7 & $(11, 4)$ & 5 & 1 & YES & YES & YES & NO & 62\\
$(29, 8)$ & 7 & $(11, 4)$ & 5 & 1 & YES & YES & YES & NO & 63\\
$(29, 8)$ & 7 & $(16, 5)$ & 7 & 1 & YES & YES & YES & NO & 64\\
$(29, 8)$ & 7 & $(24, 7)$ & 7 & 1 & YES & YES & YES & NO & 65\\
$(31, 12)$ & 7 & $(4, 1)$ & 3 & 1 & YES & YES & YES & NO & 66\\
$(31, 12)$ & 7 & $(4, 1)$ & 3 & 1 & YES & YES & YES & NO & 67\\
$(31, 12)$ & 7 & $(4, 1)$ & 3 & 1 & YES & YES & YES & NO & 68\\
$(31, 9)$ & 8 & $(5, 2)$ & 3 & 1 & YES & YES & YES & NO & 69\\
$(31, 12)$ & 7 & $(7, 3)$ & 4 & 1 & YES & YES & YES & NO & 70\\
$(31, 12)$ & 7 & $(28, 11)$ & 8 & 1 & YES & YES & YES & 141 & 71\\
$(34, 13)$ & 7 & $(4, 1)$ & 3 & 2 & YES & YES & YES & NO & 72\\
$(34, 13)$ & 7 & $(5, 2)$ & 3 & 1 & YES & YES & YES & NO & 73\\
$(34, 13)$ & 7 & $(7, 3)$ & 4 & 1 & YES & YES & YES & NO & 74\\
$(34, 13)$ & 7 & $(7, 3)$ & 4 & 1 & YES & YES & YES & NO & 75\\
$(34, 15)$ & 8 & $(7, 2)$ & 4 & 1 & YES & YES & YES & NO & 76\\
$(34, 13)$ & 7 & $(9, 2)$ & 5 & 1 & YES & YES & YES & NO & 77\\
$(34, 13)$ & 7 & $(9, 4)$ & 5 & 1 & YES & YES & YES & NO & 78\\
$(34, 15)$ & 8 & $(23, 10)$ & 7 & 1 & YES & YES & YES & 163 & 79\\
$(35, 8)$ & 8 & $(9, 4)$ & 5 & 1 & YES & YES & YES & NO & 80\\
$(35, 13)$ & 8 & $(27, 10)$ & 7 & 1 & YES & YES & YES & NO & 81\\
$(37, 11)$ & 8 & $(10, 3)$ & 5 & 1 & YES & YES & YES & NO & 82\\
$(38, 11)$ & 9 & $(5, 2)$ & 3 & 1 & YES & YES & YES & NO & 83\\
$(38, 11)$ & 9 & $(9, 2)$ & 5 & 1 & YES & YES & YES & NO & 84\\
$(38, 11)$ & 9 & $(9, 2)$ & 5 & 1 & YES & YES & YES & NO & 85\\
$(38, 11)$ & 9 & $(11, 2)$ & 6 & 1 & YES & YES & YES & NO & 86\\
$(38, 11)$ & 9 & $(11, 2)$ & 6 & 1 & YES & YES & YES & NO & 87\\
$(39, 14)$ & 8 & $(5, 2)$ & 3 & 1 & YES & YES & YES & NO & 88\\
$(39, 17)$ & 8 & $(5, 2)$ & 3 & 1 & YES & YES & YES & NO & 89\\
$(39, 17)$ & 8 & $(8, 3)$ & 4 & 1 & YES & YES & YES & NO & 90\\
$(39, 11)$ & 9 & $(11, 2)$ & 6 & 1 & YES & YES & YES & NO & 91\\
$(39, 14)$ & 8 & $(19, 7)$ & 6 & 1 & YES & YES & YES & NO & 92\\
$(39, 11)$ & 9 & $(29, 8)$ & 7 & 1 & YES & YES & YES & NO & 93\\
$(41, 12)$ & 8 & $(2, 1)$ & 1 & 1 & YES & YES & YES & NO & 94\\
$(41, 12)$ & 8 & $(2, 1)$ & 1 & 1 & YES & YES & YES & NO & 95\\
$(41, 12)$ & 8 & $(4, 1)$ & 3 & 1 & YES & YES & YES & NO & 96\\
$(41, 12)$ & 8 & $(4, 1)$ & 3 & 1 & YES & YES & YES & NO & 97\\
$(41, 12)$ & 8 & $(4, 1)$ & 3 & 1 & YES & YES & YES & NO & 98\\
$(41, 12)$ & 8 & $(5, 2)$ & 3 & 1 & YES & YES & YES & NO & 99\\
$(41, 17)$ & 8 & $(5, 2)$ & 3 & 1 & YES & YES & YES & NO & 100\\
$(41, 17)$ & 8 & $(5, 2)$ & 3 & 1 & YES & YES & YES & NO & 101\\
$(41, 12)$ & 8 & $(7, 3)$ & 4 & 1 & YES & YES & YES & NO & 102\\
$(41, 15)$ & 8 & $(7, 2)$ & 4 & 1 & YES & YES & YES & NO & 103\\
$(41, 17)$ & 8 & $(7, 2)$ & 4 & 1 & YES & YES & YES & NO & 104\\
$(41, 17)$ & 8 & $(7, 2)$ & 4 & 1 & YES & YES & YES & NO & 105\\
$(41, 12)$ & 8 & $(17, 5)$ & 6 & 1 & YES & YES & YES & NO & 106\\
$(41, 12)$ & 8 & $(38, 11)$ & 9 & 1 & YES & YES & YES & 186 & 107\\
$(42, 13)$ & 9 & $(7, 2)$ & 4 & 7 & YES & YES & YES & NO & 108\\
$(43, 18)$ & 8 & $(5, 2)$ & 3 & 1 & YES & YES & YES & NO & 109\\
$(43, 18)$ & 8 & $(7, 2)$ & 4 & 1 & YES & YES & YES & NO & 110\\
$(43, 18)$ & 8 & $(7, 2)$ & 4 & 1 & YES & YES & YES & NO & 111\\
$(43, 18)$ & 8 & $(26, 11)$ & 7 & 1 & YES & YES & YES & 202 & 112\\
$(45, 17)$ & 9 & $(4, 1)$ & 3 & 1 & YES & YES & YES & NO & 113\\
$(45, 17)$ & 9 & $(4, 1)$ & 3 & 1 & YES & YES & YES & NO & 114\\
$(45, 17)$ & 9 & $(5, 1)$ & 4 & 5 & YES & YES & YES & NO & 115\\
$(45, 19)$ & 8 & $(5, 2)$ & 3 & 5 & YES & YES & YES & NO & 116\\
$(45, 19)$ & 8 & $(8, 3)$ & 4 & 1 & YES & YES & YES & NO & 117\\
$(45, 14)$ & 9 & $(11, 3)$ & 5 & 1 & YES & YES & YES & NO & 118\\
$(45, 14)$ & 9 & $(23, 7)$ & 7 & 1 & YES & YES & YES & NO & 119\\
$(45, 17)$ & 9 & $(29, 11)$ & 7 & 1 & YES & YES & YES & 183 & 120\\
$(45, 17)$ & 9 & $(37, 14)$ & 8 & 1 & YES & YES & YES & NO & 121\\
$(47, 18)$ & 8 & $(7, 3)$ & 4 & 1 & YES & YES & YES & NO & 122\\
$(47, 13)$ & 8 & $(8, 3)$ & 4 & 1 & YES & YES & YES & NO & 123\\
$(47, 18)$ & 8 & $(34, 13)$ & 7 & 1 & YES & YES & YES & NO & 124\\
$(49, 11)$ & 10 & $(5, 2)$ & 3 & 1 & YES & YES & YES & NO & 125\\
$(49, 18)$ & 8 & $(7, 2)$ & 4 & 7 & YES & YES & YES & NO & 126\\
$(49, 11)$ & 10 & $(11, 2)$ & 6 & 1 & YES & YES & YES & NO & 127\\
$(49, 18)$ & 8 & $(14, 5)$ & 6 & 7 & YES & YES & YES & NO & 128\\
$(49, 11)$ & 10 & $(35, 8)$ & 8 & 7 & YES & YES & YES & NO & 129\\
$(50, 19)$ & 8 & $(2, 1)$ & 1 & 2 & YES & YES & YES & NO & 130\\
$(50, 19)$ & 8 & $(3, 1)$ & 2 & 1 & YES & YES & YES & NO & 131\\
$(50, 19)$ & 8 & $(3, 1)$ & 2 & 1 & YES & YES & YES & NO & 132\\
$(50, 19)$ & 8 & $(4, 1)$ & 3 & 2 & YES & YES & YES & NO & 133\\
$(50, 21)$ & 8 & $(5, 2)$ & 3 & 5 & YES & YES & YES & NO & 134\\
$(50, 19)$ & 8 & $(7, 2)$ & 4 & 1 & YES & YES & YES & NO & 135\\
$(50, 21)$ & 8 & $(9, 4)$ & 5 & 1 & YES & YES & YES & NO & 136\\
$(50, 19)$ & 8 & $(29, 11)$ & 7 & 1 & YES & YES & YES & NO & 137\\
$(51, 20)$ & 9 & $(3, 1)$ & 2 & 3 & YES & YES & YES & NO & 138\\
$(51, 20)$ & 9 & $(4, 1)$ & 3 & 1 & YES & YES & YES & NO & 139\\
$(51, 20)$ & 9 & $(8, 3)$ & 4 & 1 & YES & YES & YES & NO & 140\\
$(51, 20)$ & 9 & $(13, 5)$ & 5 & 1 & YES & YES & YES & 71 & 141\\
$(53, 19)$ & 9 & $(3, 1)$ & 2 & 1 & YES & YES & YES & NO & 142\\
$(53, 19)$ & 9 & $(4, 1)$ & 3 & 1 & YES & YES & YES & NO & 143\\
$(53, 14)$ & 9 & $(5, 2)$ & 3 & 1 & YES & YES & YES & NO & 144\\
$(53, 14)$ & 9 & $(10, 3)$ & 5 & 1 & YES & YES & YES & NO & 145\\
$(55, 24)$ & 9 & $(4, 1)$ & 3 & 1 & YES & YES & YES & NO & 146\\
$(55, 13)$ & 10 & $(5, 2)$ & 3 & 5 & YES & YES & YES & NO & 147\\
$(55, 21)$ & 8 & $(7, 3)$ & 4 & 1 & YES & YES & YES & NO & 148\\
$(55, 16)$ & 9 & $(9, 2)$ & 5 & 1 & YES & YES & YES & NO & 149\\
$(55, 16)$ & 9 & $(11, 3)$ & 5 & 11 & YES & YES & YES & NO & 150\\
$(55, 21)$ & 8 & $(11, 4)$ & 5 & 11 & YES & YES & YES & NO & 151\\
$(55, 24)$ & 9 & $(23, 10)$ & 7 & 1 & YES & YES & YES & 172 & 152\\
$(55, 16)$ & 9 & $(38, 11)$ & 9 & 1 & YES & YES & YES & NO & 153\\
$(55, 24)$ & 9 & $(39, 17)$ & 8 & 1 & YES & YES & YES & NO & 154\\
$(56, 17)$ & 9 & $(5, 2)$ & 3 & 1 & YES & YES & YES & NO & 155\\
$(57, 13)$ & 9 & $(7, 3)$ & 4 & 1 & YES & YES & YES & NO & 156\\
$(57, 16)$ & 9 & $(10, 3)$ & 5 & 1 & YES & YES & YES & NO & 157\\
$(58, 17)$ & 9 & $(5, 2)$ & 3 & 1 & YES & YES & YES & NO & 158\\
$(59, 26)$ & 9 & $(4, 1)$ & 3 & 1 & YES & YES & YES & NO & 159\\
$(59, 26)$ & 9 & $(4, 1)$ & 3 & 1 & YES & YES & YES & NO & 160\\
$(59, 18)$ & 9 & $(5, 2)$ & 3 & 1 & YES & YES & YES & NO & 161\\
$(59, 18)$ & 9 & $(5, 2)$ & 3 & 1 & YES & YES & YES & NO & 162\\
$(59, 26)$ & 9 & $(7, 3)$ & 4 & 1 & YES & YES & YES & 79 & 163\\
$(59, 18)$ & 9 & $(16, 5)$ & 7 & 1 & YES & YES & YES & NO & 164\\
$(60, 11)$ & 11 & $(5, 2)$ & 3 & 5 & YES & YES & YES & NO & 165\\
$(61, 22)$ & 9 & $(5, 2)$ & 3 & 1 & YES & YES & YES & NO & 166\\
$(61, 22)$ & 9 & $(8, 3)$ & 4 & 1 & YES & YES & YES & NO & 167\\
$(62, 27)$ & 9 & $(3, 1)$ & 2 & 1 & YES & YES & YES & NO & 168\\
$(62, 27)$ & 9 & $(3, 1)$ & 2 & 1 & YES & YES & YES & NO & 169\\
$(62, 27)$ & 9 & $(3, 1)$ & 2 & 1 & YES & YES & YES & NO & 170\\
$(62, 27)$ & 9 & $(9, 4)$ & 5 & 1 & YES & YES & YES & 182 & 171\\
$(62, 27)$ & 9 & $(16, 7)$ & 6 & 2 & YES & YES & YES & 152 & 172\\
$(62, 27)$ & 9 & $(39, 17)$ & 8 & 1 & YES & YES & YES & NO & 173\\
$(64, 23)$ & 9 & $(3, 1)$ & 2 & 1 & YES & YES & YES & NO & 174\\
$(64, 27)$ & 9 & $(3, 1)$ & 2 & 1 & YES & YES & YES & NO & 175\\
$(64, 27)$ & 9 & $(5, 2)$ & 3 & 1 & YES & YES & YES & NO & 176\\
$(64, 23)$ & 9 & $(11, 4)$ & 5 & 1 & YES & YES & YES & NO & 177\\
$(65, 19)$ & 9 & $(7, 2)$ & 4 & 1 & YES & YES & YES & NO & 178\\
$(66, 29)$ & 9 & $(3, 1)$ & 2 & 3 & YES & YES & YES & NO & 179\\
$(66, 29)$ & 9 & $(4, 1)$ & 3 & 2 & YES & YES & YES & NO & 180\\
$(66, 25)$ & 9 & $(5, 1)$ & 4 & 1 & YES & YES & YES & NO & 181\\
$(66, 29)$ & 9 & $(7, 3)$ & 4 & 1 & YES & YES & YES & 171 & 182\\
$(66, 25)$ & 9 & $(8, 3)$ & 4 & 2 & YES & YES & YES & 120 & 183\\
$(69, 20)$ & 10 & $(4, 1)$ & 3 & 1 & YES & YES & YES & NO & 184\\
$(69, 20)$ & 10 & $(4, 1)$ & 3 & 1 & YES & YES & YES & NO & 185\\
$(69, 20)$ & 10 & $(17, 5)$ & 6 & 1 & YES & YES & YES & 107 & 186\\
$(70, 29)$ & 9 & $(2, 1)$ & 1 & 2 & YES & YES & YES & NO & 187\\
$(70, 29)$ & 9 & $(2, 1)$ & 1 & 2 & YES & YES & YES & NO & 188\\
$(70, 29)$ & 9 & $(3, 1)$ & 2 & 1 & YES & YES & YES & NO & 189\\
$(70, 29)$ & 9 & $(4, 1)$ & 3 & 2 & YES & YES & YES & NO & 190\\
$(70, 29)$ & 9 & $(7, 3)$ & 4 & 7 & YES & YES & YES & NO & 191\\
$(70, 29)$ & 9 & $(29, 12)$ & 7 & 1 & YES & YES & YES & NO & 192\\
$(70, 29)$ & 9 & $(41, 17)$ & 8 & 1 & YES & YES & YES & NO & 193\\
$(71, 26)$ & 9 & $(19, 7)$ & 6 & 1 & YES & YES & YES & NO & 194\\
$(72, 19)$ & 10 & $(2, 1)$ & 1 & 2 & YES & YES & YES & NO & 195\\
$(72, 19)$ & 10 & $(3, 1)$ & 2 & 3 & YES & YES & YES & NO & 196\\
$(72, 19)$ & 10 & $(5, 1)$ & 4 & 1 & YES & YES & YES & NO & 197\\
$(72, 19)$ & 10 & $(15, 4)$ & 6 & 3 & YES & YES & YES & NO & 198\\
$(74, 31)$ & 9 & $(3, 1)$ & 2 & 1 & YES & YES & YES & NO & 199\\
$(74, 31)$ & 9 & $(3, 1)$ & 2 & 1 & YES & YES & YES & NO & 200\\
$(74, 17)$ & 11 & $(5, 1)$ & 4 & 1 & YES & YES & YES & NO & 201\\
$(74, 31)$ & 9 & $(7, 3)$ & 4 & 1 & YES & YES & YES & 112 & 202\\
$(74, 17)$ & 11 & $(9, 2)$ & 5 & 1 & YES & YES & YES & NO & 203\\
$(74, 31)$ & 9 & $(74, 31)$ & 9 & 74 & YES & YES & YES & NO & 204\\
$(75, 29)$ & 9 & $(2, 1)$ & 1 & 1 & YES & YES & YES & NO & 205\\
$(75, 22)$ & 10 & $(17, 5)$ & 6 & 1 & YES & YES & YES & NO & 206\\
$(75, 22)$ & 10 & $(58, 17)$ & 9 & 1 & YES & YES & YES & NO & 207\\
$(79, 17)$ & 11 & $(3, 1)$ & 2 & 1 & YES & YES & YES & NO & 208\\
$(79, 30)$ & 9 & $(3, 1)$ & 2 & 1 & YES & YES & YES & NO & 209\\
$(79, 30)$ & 9 & $(3, 1)$ & 2 & 1 & YES & YES & YES & NO & 210\\
$(79, 17)$ & 11 & $(4, 1)$ & 3 & 1 & YES & YES & YES & NO & 211\\
$(79, 30)$ & 9 & $(4, 1)$ & 3 & 1 & YES & YES & YES & NO & 212\\
$(79, 30)$ & 9 & $(13, 5)$ & 5 & 1 & YES & YES & YES & 222 & 213\\
$(79, 30)$ & 9 & $(79, 30)$ & 9 & 79 & YES & YES & YES & NO & 214\\
$(80, 31)$ & 9 & $(8, 3)$ & 4 & 8 & YES & YES & YES & NO & 215\\
$(81, 31)$ & 9 & $(2, 1)$ & 1 & 1 & YES & YES & YES & NO & 216\\
$(82, 25)$ & 10 & $(3, 1)$ & 2 & 1 & YES & YES & YES & NO & 217\\
$(89, 27)$ & 10 & $(2, 1)$ & 1 & 1 & YES & YES & YES & NO & 218\\
$(89, 20)$ & 11 & $(3, 1)$ & 2 & 1 & YES & YES & YES & NO & 219\\
$(89, 26)$ & 10 & $(3, 1)$ & 2 & 1 & YES & YES & YES & NO & 220\\
$(89, 27)$ & 10 & $(4, 1)$ & 3 & 1 & YES & YES & YES & NO & 221\\
$(89, 34)$ & 9 & $(8, 3)$ & 4 & 1 & YES & YES & YES & 213 & 222\\
$(89, 26)$ & 10 & $(10, 3)$ & 5 & 1 & YES & YES & YES & NO & 223\\
$(89, 20)$ & 11 & $(49, 11)$ & 10 & 1 & YES & YES & YES & NO & 224\\
$(89, 27)$ & 10 & $(56, 17)$ & 9 & 1 & YES & YES & YES & NO & 225\\
$(91, 27)$ & 10 & $(17, 5)$ & 6 & 1 & YES & YES & YES & NO & 226\\
$(96, 17)$ & 12 & $(3, 1)$ & 2 & 3 & YES & YES & YES & NO & 227\\
$(96, 17)$ & 12 & $(11, 2)$ & 6 & 1 & YES & YES & YES & NO & 228\\
$(97, 26)$ & 10 & $(3, 1)$ & 2 & 1 & YES & YES & YES & NO & 229\\
$(97, 22)$ & 11 & $(22, 5)$ & 7 & 1 & YES & YES & YES & NO & 230\\
$(98, 29)$ & 10 & $(3, 1)$ & 2 & 1 & YES & YES & YES & NO & 231\\
$(99, 29)$ & 10 & $(2, 1)$ & 1 & 1 & YES & YES & YES & NO & 232\\
$(101, 30)$ & 10 & $(2, 1)$ & 1 & 1 & YES & YES & YES & NO & 233\\
$(101, 30)$ & 10 & $(4, 1)$ & 3 & 1 & YES & YES & YES & NO & 234\\
$(101, 30)$ & 10 & $(7, 2)$ & 4 & 1 & YES & YES & YES & NO & 235\\
$(101, 23)$ & 11 & $(22, 5)$ & 7 & 1 & YES & YES & YES & NO & 236\\
$(101, 22)$ & 11 & $(23, 5)$ & 7 & 1 & YES & YES & YES & NO & 237\\
$(101, 23)$ & 11 & $(57, 13)$ & 9 & 1 & YES & YES & YES & 244 & 238\\
$(105, 31)$ & 10 & $(61, 18)$ & 9 & 1 & YES & YES & YES & NO & 239\\
$(105, 31)$ & 10 & $(105, 31)$ & 10 & 105 & YES & YES & YES & NO & 240\\
$(106, 31)$ & 10 & $(5, 1)$ & 4 & 1 & YES & YES & YES & NO & 241\\
$(106, 31)$ & 10 & $(65, 19)$ & 9 & 1 & YES & YES & YES & NO & 242\\
$(112, 31)$ & 10 & $(47, 13)$ & 8 & 1 & YES & YES & YES & NO & 243\\
$(136, 31)$ & 11 & $(22, 5)$ & 7 & 2 & YES & YES & YES & 238 & 244\\
$(a; 1, 0, 0; 13)$ & 5 & $(10, 3)$ & 5 & 1 & YES & YES & YES & NO & 245\\
$(a; 1, 1, 1; 4)$ & 7 & $(4, 1)$ & 3 & 4 & YES & YES & YES & NO & 246\\
$(a; 2, 1, 0; 5)$ & 7 & $(10, 3)$ & 5 & 5 & YES & YES & YES & NO & 247\\
$(a; 2, 1, 1; 37)$ & 8 & $(3, 1)$ & 2 & 1 & YES & YES & YES & NO & 248\\
$(b; 0, 0, 1; 4)$ & 6 & $(4, 1)$ & 3 & 4 & YES & YES & YES & NO & 249\\
$(b; 0, 0, 1; 4)$ & 6 & $(7, 3)$ & 4 & 1 & YES & YES & YES & NO & 250\\
$(b; 0, 0, 2; 26)$ & 7 & $(5, 2)$ & 3 & 1 & YES & YES & YES & NO & 251\\
$(b; 0, 0, 3; 32)$ & 8 & $(3, 1)$ & 2 & 1 & YES & YES & YES & NO & 252\\
$(b; 0, 1, 2; 5)$ & 8 & $(4, 1)$ & 3 & 1 & YES & YES & YES & NO & 253\\
$(b; 0, 3, 0; 29)$ & 8 & $(3, 1)$ & 2 & 1 & YES & YES & YES & NO & 254\\
$(b; 0, 3, 0; 29)$ & 8 & $(4, 1)$ & 3 & 1 & YES & YES & YES & NO & 255\\
$(b; 1, 0, 1; 29)$ & 7 & $(5, 2)$ & 3 & 1 & YES & YES & YES & NO & 256\\
$(b; 1, 0, 1; 29)$ & 7 & $(10, 3)$ & 5 & 1 & YES & YES & YES & NO & 257\\
$(b; 1, 0, 2; 19)$ & 8 & $(3, 1)$ & 2 & 1 & YES & YES & YES & NO & 258\\
$(b; 1, 1, 0; 27)$ & 7 & $(5, 2)$ & 3 & 1 & YES & YES & YES & NO & 259\\
$(b; 1, 1, 1; 39)$ & 8 & $(3, 1)$ & 2 & 3 & YES & YES & YES & NO & 260\\
$(b; 2, 0, 0; 26)$ & 7 & $(13, 3)$ & 6 & 13 & YES & YES & YES & NO & 261\\
$(b; 2, 0, 1; 38)$ & 8 & $(3, 1)$ & 2 & 1 & YES & YES & YES & NO & 262\\
$(b; 2, 1, 0; 7)$ & 8 & $(2, 1)$ & 1 & 1 & YES & YES & YES & NO & 263\\
$(b; 2, 1, 0; 7)$ & 8 & $(4, 1)$ & 3 & 1 & YES & YES & YES & NO & 264\\
$(c; 0, 0, 0; 4)$ & 4 & $(23, 10)$ & 7 & 1 & YES & YES & YES & NO & 265\\
$(c; 0, 0, 0; 4)$ & 4 & $(25, 9)$ & 7 & 1 & YES & YES & YES & NO & 266\\
$(c; 0, 0, 0; 4)$ & 4 & $(29, 11)$ & 7 & 1 & YES & YES & YES & NO & 267\\
$(c; 0, 0, 0; 4)$ & 4 & $(33, 10)$ & 8 & 1 & YES & YES & YES & NO & 268\\
$(c; 0, 0, 0; 4)$ & 4 & $(40, 11)$ & 8 & 4 & YES & YES & YES & NO & 269\\
$(c; 0, 1, 0; 11)$ & 5 & $(17, 7)$ & 6 & 1 & YES & YES & YES & NO & 270\\
$(c; 0, 1, 0; 11)$ & 5 & $(21, 8)$ & 6 & 1 & YES & YES & YES & NO & 271\\
$(c; 0, 3, 0; 17)$ & 7 & $(13, 3)$ & 6 & 1 & YES & YES & YES & NO & 272\\
$(c; 0, 3, 0; 17)$ & 7 & $(14, 3)$ & 6 & 1 & YES & YES & YES & NO & 273\\
$(c; 0, 3, 1; 23)$ & 8 & $(5, 2)$ & 3 & 1 & YES & YES & YES & NO & 274\\
$(d; 0, 0, 0; 5)$ & 5 & $(11, 4)$ & 5 & 1 & YES & YES & YES & NO & 275\\
$(d; 0, 0, 0; 5)$ & 5 & $(13, 5)$ & 5 & 1 & YES & YES & YES & NO & 276\\
$(d; 0, 0, 0; 5)$ & 5 & $(21, 8)$ & 6 & 1 & YES & YES & YES & NO & 277\\
$(d; 0, 0, 0; 5)$ & 5 & $(27, 8)$ & 7 & 1 & YES & YES & YES & NO & 278\\
$(d; 0, 0, 1; 14)$ & 6 & $(8, 3)$ & 4 & 2 & YES & YES & YES & NO & 279\\
$(d; 0, 0, 1; 14)$ & 6 & $(10, 3)$ & 5 & 2 & YES & YES & YES & NO & 280\\
$(d; 0, 0, 2; 9)$ & 7 & $(10, 3)$ & 5 & 1 & YES & YES & YES & NO & 281\\
$(d; 0, 2, 0; 7)$ & 7 & $(10, 3)$ & 5 & 1 & YES & YES & YES & NO & 282\\
$(d; 0, 2, 0; 7)$ & 7 & $(13, 3)$ & 6 & 1 & YES & YES & YES & NO & 283\\
$(d; 0, 2, 0; 7)$ & 7 & $(14, 3)$ & 6 & 7 & YES & YES & YES & NO & 284\\
$(d; 0, 2, 1; 20)$ & 8 & $(5, 2)$ & 3 & 5 & YES & YES & YES & NO & 285\\
$(e; 1, 0, 0; 18)$ & 6 & $(4, 1)$ & 3 & 2 & YES & YES & YES & NO & 286\\
$(e; 1, 0, 0; 18)$ & 6 & $(7, 3)$ & 4 & 1 & YES & YES & YES & NO & 287\\
$(e; 1, 0, 0; 18)$ & 6 & $(9, 4)$ & 5 & 9 & YES & YES & YES & NO & 288\\
$(e; 1, 2, 0; 28)$ & 8 & $(3, 1)$ & 2 & 1 & YES & YES & YES & NO & 289\\
$(e; 2, 0, 0; 24)$ & 7 & $(5, 2)$ & 3 & 1 & YES & YES & YES & NO & 290\\
$(f; 0, 0, 0; 6)$ & 4 & $(29, 11)$ & 7 & 1 & YES & YES & YES & NO & 291\\
$(f; 0, 0, 0; 6)$ & 4 & $(41, 12)$ & 8 & 1 & YES & YES & YES & NO & 292\\
$(g; 0, 0, 2; 11)$ & 8 & $(2, 1)$ & 1 & 1 & YES & YES & YES & NO & 293\\
$(g; 0, 2, 0; 29)$ & 8 & $(2, 1)$ & 1 & 1 & YES & YES & YES & NO & 294\\
$(g; 1, 0, 1; 38)$ & 8 & $(2, 1)$ & 1 & 2 & YES & YES & YES & NO & 295\\
$(g; 1, 0, 1; 38)$ & 8 & $(3, 1)$ & 2 & 1 & YES & YES & YES & NO & 296\\
$(i; 0, 0, 0; 9)$ & 5 & $(10, 3)$ & 5 & 1 & YES & YES & YES & NO & 297\\
$(i; 0, 0, 0; 9)$ & 5 & $(17, 5)$ & 6 & 1 & YES & YES & YES & NO & 298\\
$(i; 0, 2, 0; 15)$ & 7 & $(5, 2)$ & 3 & 5 & YES & YES & YES & NO & 299\\
$(j; 0, 1, 0; 10)$ & 6 & $(13, 5)$ & 5 & 1 & YES & YES & YES & NO & 300\\
$(j; 0, 1, 0; 10)$ & 6 & $(17, 5)$ & 6 & 1 & YES & YES & YES & NO & 301
\end{longtable}
\subsection{2 chains, $K^2 = 3$}
\begin{longtable}{|c|c|c|c|c|c|c|c|c|c|}
\hline
\multicolumn{10}{|c|}{2 chains, $K^2 = 3$}\\
\hline
$(n,a)$ & Length & $(n,a)$ & Length & GCD & Nef & $\mathbb Q$-ef & Obstruction 0 & WH & Index\\
\hline
\endfirsthead

\hline
$(n,a)$ & Length & $(n,a)$ & Length & GCD & Nef & $\mathbb Q$-ef & Obstruction 0 & WH & Index\\
\hline
\endhead
\hline
\endfoot

$(21, 4)$ & 8 & $(18, 7)$ & 6 & 3 & YES & YES & YES & NO & 302\\
$(21, 4)$ & 8 & $(18, 7)$ & 6 & 3 & YES & YES & YES & NO & 303\\
$(21, 4)$ & 8 & $(18, 7)$ & 6 & 3 & YES & YES & YES & NO & 304\\
$(22, 5)$ & 7 & $(18, 7)$ & 6 & 2 & YES & YES & YES & NO & 305\\
$(22, 5)$ & 7 & $(18, 7)$ & 6 & 2 & YES & YES & YES & NO & 306\\
$(22, 7)$ & 9 & $(21, 8)$ & 6 & 1 & YES & YES & YES & NO & 307\\
$(24, 7)$ & 7 & $(21, 4)$ & 8 & 3 & YES & YES & YES & NO & 308\\
$(24, 7)$ & 7 & $(21, 4)$ & 8 & 3 & YES & YES & YES & NO & 309\\
$(24, 7)$ & 7 & $(21, 4)$ & 8 & 3 & YES & YES & YES & NO & 310\\
$(27, 8)$ & 7 & $(25, 11)$ & 7 & 1 & YES & YES & YES & NO & 311\\
$(29, 11)$ & 7 & $(13, 3)$ & 6 & 1 & YES & YES & YES & NO & 312\\
$(29, 11)$ & 7 & $(16, 3)$ & 7 & 1 & YES & YES & YES & NO & 313\\
$(29, 11)$ & 7 & $(16, 3)$ & 7 & 1 & YES & YES & YES & NO & 314\\
$(29, 11)$ & 7 & $(16, 3)$ & 7 & 1 & YES & YES & YES & NO & 315\\
$(29, 13)$ & 8 & $(19, 7)$ & 6 & 1 & YES & YES & YES & NO & 316\\
$(29, 8)$ & 7 & $(20, 7)$ & 8 & 1 & YES & YES & YES & NO & 317\\
$(29, 11)$ & 7 & $(23, 10)$ & 7 & 1 & YES & YES & YES & NO & 318\\
$(29, 13)$ & 8 & $(24, 7)$ & 7 & 1 & YES & YES & YES & NO & 319\\
$(29, 13)$ & 8 & $(29, 11)$ & 7 & 29 & YES & YES & YES & 492 & 320\\
$(30, 13)$ & 8 & $(18, 7)$ & 6 & 6 & YES & YES & YES & NO & 321\\
$(31, 9)$ & 8 & $(16, 3)$ & 7 & 1 & YES & YES & YES & NO & 322\\
$(31, 9)$ & 8 & $(16, 3)$ & 7 & 1 & YES & YES & YES & NO & 323\\
$(31, 12)$ & 7 & $(18, 5)$ & 6 & 1 & YES & YES & YES & NO & 324\\
$(31, 7)$ & 8 & $(20, 7)$ & 8 & 1 & YES & YES & YES & NO & 325\\
$(31, 7)$ & 8 & $(22, 7)$ & 9 & 1 & YES & YES & YES & NO & 326\\
$(31, 7)$ & 8 & $(24, 11)$ & 8 & 1 & YES & YES & YES & NO & 327\\
$(31, 11)$ & 8 & $(25, 7)$ & 7 & 1 & YES & YES & YES & NO & 328\\
$(31, 11)$ & 8 & $(27, 8)$ & 7 & 1 & YES & YES & YES & NO & 329\\
$(32, 7)$ & 8 & $(13, 5)$ & 5 & 1 & YES & YES & YES & NO & 330\\
$(32, 7)$ & 8 & $(13, 5)$ & 5 & 1 & YES & YES & YES & NO & 331\\
$(32, 7)$ & 8 & $(24, 11)$ & 8 & 8 & YES & YES & YES & NO & 332\\
$(33, 13)$ & 9 & $(32, 5)$ & 9 & 1 & YES & YES & YES & NO & 333\\
$(34, 13)$ & 7 & $(14, 3)$ & 6 & 2 & YES & YES & YES & NO & 334\\
$(34, 13)$ & 7 & $(14, 3)$ & 6 & 2 & YES & YES & YES & NO & 335\\
$(35, 16)$ & 9 & $(17, 5)$ & 6 & 1 & YES & YES & YES & NO & 336\\
$(35, 16)$ & 9 & $(18, 5)$ & 6 & 1 & YES & YES & YES & NO & 337\\
$(36, 13)$ & 8 & $(32, 7)$ & 8 & 4 & YES & YES & YES & NO & 338\\
$(37, 8)$ & 8 & $(13, 5)$ & 5 & 1 & YES & YES & YES & NO & 339\\
$(37, 8)$ & 8 & $(13, 5)$ & 5 & 1 & YES & YES & YES & NO & 340\\
$(37, 8)$ & 8 & $(17, 5)$ & 6 & 1 & YES & YES & YES & NO & 341\\
$(37, 8)$ & 8 & $(17, 5)$ & 6 & 1 & YES & YES & YES & NO & 342\\
$(37, 17)$ & 9 & $(17, 5)$ & 6 & 1 & YES & YES & YES & NO & 343\\
$(37, 17)$ & 9 & $(18, 5)$ & 6 & 1 & YES & YES & YES & NO & 344\\
$(37, 8)$ & 8 & $(20, 7)$ & 8 & 1 & YES & YES & YES & NO & 345\\
$(37, 10)$ & 8 & $(20, 7)$ & 8 & 1 & YES & YES & YES & NO & 346\\
$(37, 8)$ & 8 & $(22, 7)$ & 9 & 1 & YES & YES & YES & NO & 347\\
$(37, 8)$ & 8 & $(28, 11)$ & 8 & 1 & YES & YES & YES & NO & 348\\
$(38, 7)$ & 9 & $(21, 5)$ & 8 & 1 & YES & YES & YES & NO & 349\\
$(38, 7)$ & 9 & $(22, 7)$ & 9 & 2 & YES & YES & YES & NO & 350\\
$(38, 11)$ & 9 & $(37, 8)$ & 8 & 1 & YES & YES & YES & NO & 351\\
$(39, 14)$ & 8 & $(25, 7)$ & 7 & 1 & YES & YES & YES & NO & 352\\
$(39, 7)$ & 9 & $(27, 7)$ & 9 & 3 & YES & YES & YES & NO & 353\\
$(39, 7)$ & 9 & $(29, 7)$ & 10 & 1 & YES & YES & YES & NO & 354\\
$(39, 14)$ & 8 & $(31, 7)$ & 8 & 1 & YES & YES & YES & NO & 355\\
$(40, 11)$ & 8 & $(10, 3)$ & 5 & 10 & YES & YES & YES & NO & 356\\
$(41, 18)$ & 8 & $(18, 7)$ & 6 & 1 & YES & YES & YES & NO & 357\\
$(41, 17)$ & 8 & $(24, 11)$ & 8 & 1 & YES & YES & YES & NO & 358\\
$(41, 18)$ & 8 & $(25, 7)$ & 7 & 1 & YES & YES & YES & NO & 359\\
$(41, 17)$ & 8 & $(29, 13)$ & 8 & 1 & YES & YES & YES & NO & 360\\
$(42, 11)$ & 9 & $(18, 7)$ & 6 & 6 & YES & YES & YES & NO & 361\\
$(42, 19)$ & 9 & $(22, 5)$ & 7 & 2 & YES & YES & YES & NO & 362\\
$(42, 19)$ & 9 & $(30, 13)$ & 8 & 6 & YES & YES & YES & NO & 363\\
$(42, 11)$ & 9 & $(37, 11)$ & 8 & 1 & YES & YES & YES & NO & 364\\
$(43, 12)$ & 8 & $(17, 6)$ & 7 & 1 & YES & YES & YES & NO & 365\\
$(43, 18)$ & 8 & $(17, 7)$ & 6 & 1 & YES & YES & YES & NO & 366\\
$(43, 8)$ & 9 & $(20, 7)$ & 8 & 1 & YES & YES & YES & NO & 367\\
$(43, 19)$ & 9 & $(23, 5)$ & 7 & 1 & YES & YES & YES & NO & 368\\
$(43, 18)$ & 8 & $(24, 11)$ & 8 & 1 & YES & YES & YES & NO & 369\\
$(43, 8)$ & 9 & $(28, 11)$ & 8 & 1 & YES & YES & YES & NO & 370\\
$(43, 8)$ & 9 & $(38, 11)$ & 9 & 1 & YES & YES & YES & NO & 371\\
$(43, 13)$ & 9 & $(39, 7)$ & 9 & 1 & YES & YES & YES & NO & 372\\
$(44, 13)$ & 8 & $(14, 3)$ & 6 & 2 & YES & YES & YES & NO & 373\\
$(44, 13)$ & 8 & $(14, 3)$ & 6 & 2 & YES & YES & YES & NO & 374\\
$(44, 13)$ & 8 & $(23, 10)$ & 7 & 1 & YES & YES & YES & NO & 375\\
$(45, 16)$ & 9 & $(17, 4)$ & 7 & 1 & YES & YES & YES & NO & 376\\
$(45, 8)$ & 9 & $(20, 7)$ & 8 & 5 & YES & YES & YES & NO & 377\\
$(45, 8)$ & 9 & $(20, 7)$ & 8 & 5 & YES & YES & YES & NO & 378\\
$(45, 17)$ & 9 & $(22, 5)$ & 7 & 1 & YES & YES & YES & NO & 379\\
$(45, 17)$ & 9 & $(27, 5)$ & 8 & 9 & YES & YES & YES & NO & 380\\
$(45, 8)$ & 9 & $(29, 7)$ & 10 & 1 & YES & YES & YES & NO & 381\\
$(45, 16)$ & 9 & $(31, 12)$ & 7 & 1 & YES & YES & YES & NO & 382\\
$(46, 13)$ & 10 & $(13, 5)$ & 5 & 1 & YES & YES & YES & NO & 383\\
$(46, 7)$ & 10 & $(20, 7)$ & 8 & 2 & YES & YES & YES & NO & 384\\
$(46, 17)$ & 8 & $(24, 7)$ & 7 & 2 & YES & YES & YES & NO & 385\\
$(46, 7)$ & 10 & $(36, 7)$ & 11 & 2 & YES & YES & YES & NO & 386\\
$(46, 7)$ & 10 & $(38, 11)$ & 9 & 2 & YES & YES & YES & NO & 387\\
$(47, 18)$ & 8 & $(9, 2)$ & 5 & 1 & YES & YES & YES & NO & 388\\
$(47, 10)$ & 9 & $(10, 3)$ & 5 & 1 & YES & YES & YES & NO & 389\\
$(47, 10)$ & 9 & $(10, 3)$ & 5 & 1 & YES & YES & YES & NO & 390\\
$(47, 18)$ & 8 & $(11, 5)$ & 6 & 1 & YES & YES & YES & NO & 391\\
$(47, 13)$ & 8 & $(13, 6)$ & 7 & 1 & YES & YES & YES & NO & 392\\
$(47, 13)$ & 8 & $(13, 6)$ & 7 & 1 & YES & YES & YES & NO & 393\\
$(47, 13)$ & 8 & $(17, 6)$ & 7 & 1 & YES & YES & YES & NO & 394\\
$(47, 10)$ & 9 & $(22, 5)$ & 7 & 1 & YES & YES & YES & NO & 395\\
$(47, 13)$ & 8 & $(22, 7)$ & 9 & 1 & YES & YES & YES & NO & 396\\
$(47, 18)$ & 8 & $(31, 11)$ & 8 & 1 & YES & YES & YES & NO & 397\\
$(47, 13)$ & 8 & $(38, 11)$ & 9 & 1 & YES & YES & YES & NO & 398\\
$(48, 17)$ & 9 & $(13, 5)$ & 5 & 1 & YES & YES & YES & NO & 399\\
$(49, 9)$ & 10 & $(10, 3)$ & 5 & 1 & YES & YES & YES & NO & 400\\
$(49, 9)$ & 10 & $(10, 3)$ & 5 & 1 & YES & YES & YES & NO & 401\\
$(49, 11)$ & 10 & $(10, 3)$ & 5 & 1 & YES & YES & YES & NO & 402\\
$(49, 19)$ & 8 & $(11, 3)$ & 5 & 1 & YES & YES & YES & NO & 403\\
$(49, 19)$ & 8 & $(11, 3)$ & 5 & 1 & YES & YES & YES & NO & 404\\
$(49, 19)$ & 8 & $(17, 7)$ & 6 & 1 & YES & YES & YES & NO & 405\\
$(51, 11)$ & 9 & $(13, 6)$ & 7 & 1 & YES & YES & YES & NO & 406\\
$(51, 11)$ & 9 & $(17, 6)$ & 7 & 17 & YES & YES & YES & NO & 407\\
$(51, 20)$ & 9 & $(17, 4)$ & 7 & 17 & YES & YES & YES & NO & 408\\
$(51, 11)$ & 9 & $(27, 10)$ & 7 & 3 & YES & YES & YES & NO & 409\\
$(51, 11)$ & 9 & $(29, 7)$ & 10 & 1 & YES & YES & YES & NO & 410\\
$(51, 11)$ & 9 & $(49, 11)$ & 10 & 1 & YES & YES & YES & NO & 411\\
$(52, 15)$ & 11 & $(16, 3)$ & 7 & 4 & YES & YES & YES & NO & 412\\
$(52, 15)$ & 11 & $(17, 3)$ & 7 & 1 & YES & YES & YES & NO & 413\\
$(52, 15)$ & 11 & $(17, 3)$ & 7 & 1 & YES & YES & YES & NO & 414\\
$(52, 15)$ & 11 & $(19, 3)$ & 8 & 1 & YES & YES & YES & NO & 415\\
$(52, 15)$ & 11 & $(20, 3)$ & 8 & 4 & YES & YES & YES & NO & 416\\
$(52, 7)$ & 11 & $(27, 7)$ & 9 & 1 & YES & YES & YES & NO & 417\\
$(52, 7)$ & 11 & $(36, 7)$ & 11 & 4 & YES & YES & YES & NO & 418\\
$(53, 14)$ & 9 & $(8, 3)$ & 4 & 1 & YES & YES & YES & NO & 419\\
$(53, 14)$ & 9 & $(8, 3)$ & 4 & 1 & YES & YES & YES & NO & 420\\
$(53, 20)$ & 10 & $(11, 3)$ & 5 & 1 & YES & YES & YES & NO & 421\\
$(53, 15)$ & 11 & $(17, 3)$ & 7 & 1 & YES & YES & YES & NO & 422\\
$(53, 15)$ & 11 & $(47, 13)$ & 8 & 1 & YES & YES & YES & NO & 423\\
$(55, 21)$ & 8 & $(11, 5)$ & 6 & 11 & YES & YES & YES & NO & 424\\
$(55, 21)$ & 8 & $(13, 6)$ & 7 & 1 & YES & YES & YES & NO & 425\\
$(55, 21)$ & 8 & $(20, 7)$ & 8 & 5 & YES & YES & YES & NO & 426\\
$(55, 21)$ & 8 & $(23, 10)$ & 7 & 1 & YES & YES & YES & NO & 427\\
$(55, 13)$ & 10 & $(51, 11)$ & 9 & 1 & YES & YES & YES & 652 & 428\\
$(55, 21)$ & 8 & $(53, 20)$ & 10 & 1 & YES & YES & YES & NO & 429\\
$(57, 22)$ & 9 & $(6, 1)$ & 5 & 3 & YES & YES & YES & NO & 430\\
$(57, 22)$ & 9 & $(6, 1)$ & 5 & 3 & YES & YES & YES & NO & 431\\
$(57, 22)$ & 9 & $(6, 1)$ & 5 & 3 & YES & YES & YES & NO & 432\\
$(58, 17)$ & 9 & $(10, 3)$ & 5 & 2 & YES & YES & YES & NO & 433\\
$(58, 17)$ & 9 & $(10, 3)$ & 5 & 2 & YES & YES & YES & NO & 434\\
$(58, 17)$ & 9 & $(11, 5)$ & 6 & 1 & YES & YES & YES & NO & 435\\
$(58, 17)$ & 9 & $(11, 5)$ & 6 & 1 & YES & YES & YES & NO & 436\\
$(58, 17)$ & 9 & $(22, 7)$ & 9 & 2 & YES & YES & YES & NO & 437\\
$(59, 13)$ & 11 & $(13, 5)$ & 5 & 1 & YES & YES & YES & NO & 438\\
$(59, 16)$ & 10 & $(32, 5)$ & 9 & 1 & YES & YES & YES & NO & 439\\
$(60, 13)$ & 9 & $(10, 3)$ & 5 & 10 & YES & YES & YES & NO & 440\\
$(60, 13)$ & 9 & $(10, 3)$ & 5 & 10 & YES & YES & YES & NO & 441\\
$(60, 13)$ & 9 & $(13, 6)$ & 7 & 1 & YES & YES & YES & NO & 442\\
$(60, 13)$ & 9 & $(23, 10)$ & 7 & 1 & YES & YES & YES & NO & 443\\
$(60, 13)$ & 9 & $(29, 7)$ & 10 & 1 & YES & YES & YES & NO & 444\\
$(60, 13)$ & 9 & $(55, 13)$ & 10 & 5 & YES & YES & YES & NO & 445\\
$(61, 17)$ & 9 & $(7, 3)$ & 4 & 1 & YES & YES & YES & NO & 446\\
$(61, 17)$ & 9 & $(10, 3)$ & 5 & 1 & YES & YES & YES & NO & 447\\
$(61, 17)$ & 9 & $(11, 5)$ & 6 & 1 & YES & YES & YES & NO & 448\\
$(61, 17)$ & 9 & $(17, 7)$ & 6 & 1 & YES & YES & YES & NO & 449\\
$(61, 17)$ & 9 & $(29, 8)$ & 7 & 1 & YES & NO & YES & NO & 450\\
$(61, 17)$ & 9 & $(31, 9)$ & 8 & 1 & YES & YES & YES & NO & 451\\
$(61, 18)$ & 9 & $(42, 13)$ & 9 & 1 & YES & YES & YES & NO & 452\\
$(62, 11)$ & 10 & $(13, 6)$ & 7 & 1 & YES & YES & YES & NO & 453\\
$(62, 19)$ & 10 & $(13, 5)$ & 5 & 1 & YES & YES & YES & NO & 454\\
$(62, 11)$ & 10 & $(17, 6)$ & 7 & 1 & YES & YES & YES & NO & 455\\
$(62, 11)$ & 10 & $(36, 7)$ & 11 & 2 & YES & YES & YES & NO & 456\\
$(62, 11)$ & 10 & $(60, 11)$ & 11 & 2 & YES & YES & YES & NO & 457\\
$(63, 23)$ & 10 & $(14, 3)$ & 6 & 7 & YES & YES & YES & NO & 458\\
$(63, 23)$ & 10 & $(17, 3)$ & 7 & 1 & YES & YES & YES & NO & 459\\
$(63, 17)$ & 9 & $(19, 7)$ & 6 & 1 & YES & YES & YES & NO & 460\\
$(63, 17)$ & 9 & $(34, 9)$ & 8 & 1 & YES & YES & YES & NO & 461\\
$(64, 19)$ & 9 & $(16, 5)$ & 7 & 16 & YES & YES & YES & NO & 462\\
$(64, 17)$ & 10 & $(22, 5)$ & 7 & 2 & YES & YES & YES & NO & 463\\
$(65, 18)$ & 9 & $(9, 2)$ & 5 & 1 & YES & YES & YES & NO & 464\\
$(65, 18)$ & 9 & $(9, 2)$ & 5 & 1 & YES & YES & YES & NO & 465\\
$(65, 24)$ & 9 & $(9, 4)$ & 5 & 1 & YES & YES & YES & NO & 466\\
$(65, 18)$ & 9 & $(11, 5)$ & 6 & 1 & YES & YES & YES & NO & 467\\
$(65, 23)$ & 10 & $(11, 3)$ & 5 & 1 & YES & YES & YES & NO & 468\\
$(65, 24)$ & 9 & $(11, 5)$ & 6 & 1 & YES & YES & YES & NO & 469\\
$(65, 12)$ & 10 & $(13, 6)$ & 7 & 13 & YES & YES & YES & NO & 470\\
$(65, 24)$ & 9 & $(13, 4)$ & 6 & 13 & YES & YES & YES & NO & 471\\
$(65, 24)$ & 9 & $(20, 7)$ & 8 & 5 & YES & YES & YES & NO & 472\\
$(65, 23)$ & 10 & $(21, 8)$ & 6 & 1 & YES & YES & YES & NO & 473\\
$(65, 18)$ & 9 & $(25, 7)$ & 7 & 5 & YES & NO & YES & NO & 474\\
$(65, 18)$ & 9 & $(31, 9)$ & 8 & 1 & YES & YES & YES & NO & 475\\
$(67, 26)$ & 9 & $(6, 1)$ & 5 & 1 & YES & YES & YES & NO & 476\\
$(67, 26)$ & 9 & $(6, 1)$ & 5 & 1 & YES & YES & YES & NO & 477\\
$(67, 26)$ & 9 & $(6, 1)$ & 5 & 1 & YES & YES & YES & NO & 478\\
$(69, 29)$ & 9 & $(7, 2)$ & 4 & 1 & YES & YES & YES & NO & 479\\
$(69, 31)$ & 10 & $(7, 3)$ & 4 & 1 & YES & YES & YES & NO & 480\\
$(69, 29)$ & 9 & $(9, 4)$ & 5 & 3 & YES & YES & YES & NO & 481\\
$(69, 29)$ & 9 & $(12, 5)$ & 5 & 3 & YES & YES & YES & NO & 482\\
$(69, 20)$ & 10 & $(15, 4)$ & 6 & 3 & YES & YES & YES & NO & 483\\
$(69, 31)$ & 10 & $(17, 7)$ & 6 & 1 & YES & YES & YES & NO & 484\\
$(69, 29)$ & 9 & $(20, 9)$ & 7 & 1 & YES & YES & YES & NO & 485\\
$(69, 16)$ & 11 & $(32, 5)$ & 9 & 1 & YES & YES & YES & NO & 486\\
$(70, 29)$ & 9 & $(18, 7)$ & 6 & 2 & YES & YES & YES & NO & 487\\
$(70, 27)$ & 10 & $(20, 3)$ & 8 & 10 & YES & YES & YES & NO & 488\\
$(71, 27)$ & 9 & $(5, 1)$ & 4 & 1 & YES & YES & YES & NO & 489\\
$(71, 32)$ & 10 & $(10, 3)$ & 5 & 1 & YES & YES & YES & NO & 490\\
$(71, 25)$ & 11 & $(11, 3)$ & 5 & 1 & YES & YES & YES & NO & 491\\
$(71, 27)$ & 9 & $(11, 5)$ & 6 & 1 & YES & YES & YES & 320 & 492\\
$(71, 27)$ & 9 & $(12, 5)$ & 5 & 1 & YES & YES & YES & NO & 493\\
$(71, 31)$ & 10 & $(13, 3)$ & 6 & 1 & YES & YES & YES & NO & 494\\
$(71, 27)$ & 9 & $(20, 7)$ & 8 & 1 & YES & YES & YES & NO & 495\\
$(71, 25)$ & 11 & $(21, 8)$ & 6 & 1 & YES & YES & YES & NO & 496\\
$(71, 32)$ & 10 & $(23, 10)$ & 7 & 1 & YES & YES & YES & NO & 497\\
$(71, 27)$ & 9 & $(35, 13)$ & 8 & 1 & YES & YES & YES & NO & 498\\
$(73, 20)$ & 11 & $(8, 3)$ & 4 & 1 & YES & YES & YES & NO & 499\\
$(73, 17)$ & 10 & $(14, 5)$ & 6 & 1 & YES & YES & YES & NO & 500\\
$(73, 17)$ & 10 & $(14, 5)$ & 6 & 1 & YES & YES & YES & NO & 501\\
$(73, 19)$ & 11 & $(17, 5)$ & 6 & 1 & YES & YES & YES & NO & 502\\
$(73, 20)$ & 11 & $(43, 12)$ & 8 & 1 & YES & YES & YES & NO & 503\\
$(74, 31)$ & 9 & $(13, 6)$ & 7 & 1 & YES & YES & YES & NO & 504\\
$(74, 29)$ & 10 & $(29, 11)$ & 7 & 1 & YES & YES & YES & NO & 505\\
$(74, 29)$ & 10 & $(49, 19)$ & 8 & 1 & YES & YES & YES & NO & 506\\
$(75, 31)$ & 9 & $(13, 5)$ & 5 & 1 & YES & YES & YES & NO & 507\\
$(75, 22)$ & 10 & $(14, 5)$ & 6 & 1 & YES & YES & YES & NO & 508\\
$(75, 22)$ & 10 & $(23, 4)$ & 8 & 1 & YES & YES & YES & NO & 509\\
$(75, 17)$ & 10 & $(51, 11)$ & 9 & 3 & YES & YES & YES & NO & 510\\
$(76, 29)$ & 9 & $(7, 2)$ & 4 & 1 & YES & YES & YES & NO & 511\\
$(76, 29)$ & 9 & $(11, 5)$ & 6 & 1 & YES & YES & YES & NO & 512\\
$(76, 21)$ & 9 & $(14, 5)$ & 6 & 2 & YES & YES & YES & NO & 513\\
$(77, 12)$ & 11 & $(13, 6)$ & 7 & 1 & YES & YES & YES & NO & 514\\
$(77, 16)$ & 11 & $(13, 5)$ & 5 & 1 & YES & YES & YES & NO & 515\\
$(78, 17)$ & 10 & $(19, 5)$ & 7 & 1 & YES & YES & YES & NO & 516\\
$(78, 17)$ & 10 & $(40, 9)$ & 9 & 2 & YES & YES & YES & NO & 517\\
$(79, 29)$ & 9 & $(7, 2)$ & 4 & 1 & YES & YES & YES & NO & 518\\
$(79, 30)$ & 9 & $(16, 7)$ & 6 & 1 & YES & YES & YES & NO & 519\\
$(79, 18)$ & 10 & $(21, 5)$ & 8 & 1 & YES & YES & YES & NO & 520\\
$(79, 22)$ & 10 & $(23, 4)$ & 8 & 1 & YES & YES & YES & NO & 521\\
$(79, 23)$ & 10 & $(25, 7)$ & 7 & 1 & YES & YES & YES & NO & 522\\
$(79, 22)$ & 10 & $(31, 9)$ & 8 & 1 & YES & YES & YES & NO & 523\\
$(79, 23)$ & 10 & $(52, 15)$ & 11 & 1 & YES & YES & YES & NO & 524\\
$(79, 22)$ & 10 & $(65, 18)$ & 9 & 1 & YES & YES & YES & NO & 525\\
$(80, 31)$ & 9 & $(33, 13)$ & 9 & 1 & YES & YES & YES & NO & 526\\
$(81, 31)$ & 9 & $(17, 6)$ & 7 & 1 & YES & YES & YES & NO & 527\\
$(82, 31)$ & 10 & $(8, 3)$ & 4 & 2 & YES & YES & YES & NO & 528\\
$(82, 23)$ & 10 & $(24, 7)$ & 7 & 2 & YES & YES & YES & NO & 529\\
$(82, 31)$ & 10 & $(31, 12)$ & 7 & 1 & YES & YES & YES & NO & 530\\
$(83, 24)$ & 11 & $(44, 13)$ & 8 & 1 & YES & YES & YES & NO & 531\\
$(83, 22)$ & 10 & $(53, 14)$ & 9 & 1 & YES & YES & YES & NO & 532\\
$(83, 23)$ & 10 & $(61, 17)$ & 9 & 1 & YES & YES & YES & NO & 533\\
$(84, 37)$ & 10 & $(8, 3)$ & 4 & 4 & YES & YES & YES & NO & 534\\
$(84, 37)$ & 10 & $(17, 7)$ & 6 & 1 & YES & YES & YES & NO & 535\\
$(84, 37)$ & 10 & $(39, 17)$ & 8 & 3 & YES & YES & YES & NO & 536\\
$(85, 37)$ & 10 & $(13, 6)$ & 7 & 1 & YES & YES & YES & NO & 537\\
$(86, 25)$ & 10 & $(25, 7)$ & 7 & 1 & YES & YES & YES & NO & 538\\
$(86, 25)$ & 10 & $(52, 15)$ & 11 & 2 & YES & YES & YES & 701 & 539\\
$(89, 24)$ & 10 & $(5, 2)$ & 3 & 1 & YES & YES & YES & NO & 540\\
$(89, 24)$ & 10 & $(5, 2)$ & 3 & 1 & YES & YES & YES & NO & 541\\
$(89, 34)$ & 9 & $(5, 2)$ & 3 & 1 & YES & YES & YES & NO & 542\\
$(89, 35)$ & 11 & $(7, 2)$ & 4 & 1 & YES & YES & YES & NO & 543\\
$(89, 35)$ & 11 & $(11, 4)$ & 5 & 1 & YES & YES & YES & NO & 544\\
$(89, 24)$ & 10 & $(12, 5)$ & 5 & 1 & YES & YES & YES & NO & 545\\
$(89, 25)$ & 10 & $(24, 7)$ & 7 & 1 & YES & YES & YES & NO & 546\\
$(89, 34)$ & 9 & $(28, 11)$ & 8 & 1 & YES & YES & YES & NO & 547\\
$(89, 34)$ & 9 & $(45, 17)$ & 9 & 1 & YES & YES & YES & NO & 548\\
$(89, 27)$ & 10 & $(62, 19)$ & 10 & 1 & YES & YES & YES & NO & 549\\
$(91, 40)$ & 10 & $(10, 3)$ & 5 & 1 & YES & YES & YES & NO & 550\\
$(91, 27)$ & 10 & $(11, 4)$ & 5 & 1 & YES & YES & YES & NO & 551\\
$(92, 19)$ & 12 & $(10, 3)$ & 5 & 2 & YES & YES & YES & NO & 552\\
$(92, 27)$ & 11 & $(13, 3)$ & 6 & 1 & YES & YES & YES & NO & 553\\
$(92, 17)$ & 11 & $(36, 7)$ & 11 & 4 & YES & YES & YES & NO & 554\\
$(93, 26)$ & 10 & $(5, 2)$ & 3 & 1 & YES & YES & YES & NO & 555\\
$(93, 20)$ & 12 & $(7, 3)$ & 4 & 1 & YES & YES & YES & NO & 556\\
$(93, 20)$ & 12 & $(8, 3)$ & 4 & 1 & YES & YES & YES & NO & 557\\
$(93, 20)$ & 12 & $(10, 3)$ & 5 & 1 & YES & YES & YES & NO & 558\\
$(93, 41)$ & 10 & $(13, 5)$ & 5 & 1 & YES & YES & YES & NO & 559\\
$(93, 20)$ & 12 & $(17, 4)$ & 7 & 1 & YES & YES & YES & NO & 560\\
$(93, 41)$ & 10 & $(17, 7)$ & 6 & 1 & YES & YES & YES & NO & 561\\
$(93, 26)$ & 10 & $(61, 17)$ & 9 & 1 & YES & YES & YES & 759 & 562\\
$(94, 41)$ & 10 & $(8, 3)$ & 4 & 2 & YES & YES & YES & NO & 563\\
$(95, 28)$ & 11 & $(9, 2)$ & 5 & 1 & YES & YES & YES & NO & 564\\
$(95, 28)$ & 11 & $(13, 2)$ & 7 & 1 & YES & YES & YES & NO & 565\\
$(95, 17)$ & 11 & $(21, 5)$ & 8 & 1 & YES & YES & YES & NO & 566\\
$(96, 25)$ & 12 & $(8, 3)$ & 4 & 8 & YES & YES & YES & NO & 567\\
$(96, 25)$ & 12 & $(17, 5)$ & 6 & 1 & YES & YES & YES & NO & 568\\
$(97, 35)$ & 10 & $(5, 1)$ & 4 & 1 & YES & YES & YES & NO & 569\\
$(97, 27)$ & 11 & $(11, 3)$ & 5 & 1 & YES & YES & YES & NO & 570\\
$(97, 35)$ & 10 & $(20, 7)$ & 8 & 1 & YES & YES & YES & NO & 571\\
$(97, 27)$ & 11 & $(47, 13)$ & 8 & 1 & YES & YES & YES & NO & 572\\
$(98, 27)$ & 10 & $(5, 2)$ & 3 & 1 & YES & YES & YES & NO & 573\\
$(98, 43)$ & 10 & $(13, 6)$ & 7 & 1 & YES & YES & YES & NO & 574\\
$(98, 23)$ & 12 & $(17, 3)$ & 7 & 1 & YES & YES & YES & NO & 575\\
$(98, 23)$ & 12 & $(73, 17)$ & 10 & 1 & YES & YES & YES & NO & 576\\
$(99, 29)$ & 10 & $(7, 2)$ & 4 & 1 & YES & YES & YES & NO & 577\\
$(99, 29)$ & 10 & $(7, 2)$ & 4 & 1 & YES & YES & YES & NO & 578\\
$(99, 41)$ & 10 & $(7, 3)$ & 4 & 1 & YES & YES & YES & NO & 579\\
$(100, 29)$ & 11 & $(18, 5)$ & 6 & 2 & YES & YES & YES & NO & 580\\
$(100, 29)$ & 11 & $(52, 15)$ & 11 & 4 & YES & YES & YES & NO & 581\\
$(101, 44)$ & 10 & $(10, 3)$ & 5 & 1 & YES & YES & YES & NO & 582\\
$(101, 30)$ & 10 & $(11, 3)$ & 5 & 1 & YES & YES & YES & NO & 583\\
$(101, 28)$ & 11 & $(43, 12)$ & 8 & 1 & YES & YES & YES & NO & 584\\
$(101, 37)$ & 10 & $(63, 23)$ & 10 & 1 & YES & YES & YES & NO & 585\\
$(103, 29)$ & 11 & $(17, 5)$ & 6 & 1 & YES & YES & YES & NO & 586\\
$(103, 37)$ & 10 & $(20, 7)$ & 8 & 1 & YES & YES & YES & NO & 587\\
$(104, 29)$ & 10 & $(11, 4)$ & 5 & 1 & YES & YES & YES & NO & 588\\
$(104, 31)$ & 11 & $(14, 3)$ & 6 & 2 & YES & YES & YES & NO & 589\\
$(104, 31)$ & 11 & $(16, 3)$ & 7 & 8 & YES & YES & YES & NO & 590\\
$(104, 31)$ & 11 & $(41, 12)$ & 8 & 1 & YES & YES & YES & NO & 591\\
$(104, 31)$ & 11 & $(61, 18)$ & 9 & 1 & YES & YES & YES & NO & 592\\
$(105, 44)$ & 10 & $(7, 3)$ & 4 & 7 & YES & YES & YES & NO & 593\\
$(105, 31)$ & 10 & $(13, 5)$ & 5 & 1 & YES & YES & YES & NO & 594\\
$(106, 31)$ & 10 & $(5, 2)$ & 3 & 1 & YES & YES & YES & NO & 595\\
$(106, 31)$ & 10 & $(58, 17)$ & 9 & 2 & YES & YES & YES & NO & 596\\
$(107, 38)$ & 11 & $(5, 2)$ & 3 & 1 & YES & YES & YES & NO & 597\\
$(107, 31)$ & 11 & $(7, 2)$ & 4 & 1 & YES & YES & YES & NO & 598\\
$(107, 48)$ & 11 & $(7, 2)$ & 4 & 1 & YES & YES & YES & NO & 599\\
$(107, 48)$ & 11 & $(13, 6)$ & 7 & 1 & YES & YES & YES & NO & 600\\
$(107, 31)$ & 11 & $(18, 5)$ & 6 & 1 & YES & YES & YES & NO & 601\\
$(107, 31)$ & 11 & $(27, 8)$ & 7 & 1 & YES & YES & YES & NO & 602\\
$(107, 38)$ & 11 & $(48, 17)$ & 9 & 1 & YES & YES & YES & NO & 603\\
$(107, 31)$ & 11 & $(52, 15)$ & 11 & 1 & YES & YES & YES & NO & 604\\
$(107, 29)$ & 11 & $(89, 24)$ & 10 & 1 & YES & YES & YES & 738 & 605\\
$(108, 41)$ & 10 & $(9, 4)$ & 5 & 9 & YES & YES & YES & NO & 606\\
$(109, 30)$ & 10 & $(5, 2)$ & 3 & 1 & YES & YES & YES & NO & 607\\
$(109, 30)$ & 10 & $(5, 2)$ & 3 & 1 & YES & YES & YES & NO & 608\\
$(109, 30)$ & 10 & $(10, 3)$ & 5 & 1 & YES & YES & YES & NO & 609\\
$(109, 40)$ & 10 & $(17, 6)$ & 7 & 1 & YES & YES & YES & NO & 610\\
$(110, 31)$ & 11 & $(17, 5)$ & 6 & 1 & YES & YES & YES & NO & 611\\
$(110, 39)$ & 11 & $(20, 7)$ & 8 & 10 & YES & YES & YES & 658 & 612\\
$(110, 39)$ & 11 & $(45, 16)$ & 9 & 5 & YES & YES & YES & NO & 613\\
$(111, 49)$ & 11 & $(7, 2)$ & 4 & 1 & YES & YES & YES & NO & 614\\
$(111, 46)$ & 10 & $(8, 3)$ & 4 & 1 & YES & YES & YES & NO & 615\\
$(111, 46)$ & 10 & $(16, 7)$ & 6 & 1 & YES & YES & YES & NO & 616\\
$(111, 43)$ & 10 & $(28, 11)$ & 8 & 1 & YES & YES & YES & NO & 617\\
$(112, 47)$ & 10 & $(8, 3)$ & 4 & 8 & YES & YES & YES & NO & 618\\
$(112, 47)$ & 10 & $(81, 34)$ & 9 & 1 & YES & YES & YES & NO & 619\\
$(113, 33)$ & 11 & $(7, 3)$ & 4 & 1 & YES & YES & YES & NO & 620\\
$(113, 24)$ & 11 & $(9, 4)$ & 5 & 1 & YES & YES & YES & NO & 621\\
$(115, 32)$ & 12 & $(4, 1)$ & 3 & 1 & YES & YES & YES & NO & 622\\
$(115, 32)$ & 12 & $(4, 1)$ & 3 & 1 & YES & YES & YES & NO & 623\\
$(115, 44)$ & 10 & $(9, 4)$ & 5 & 1 & YES & YES & YES & NO & 624\\
$(117, 31)$ & 11 & $(10, 3)$ & 5 & 1 & YES & YES & YES & NO & 625\\
$(117, 31)$ & 11 & $(17, 5)$ & 6 & 1 & YES & YES & YES & 942 & 626\\
$(117, 31)$ & 11 & $(25, 7)$ & 7 & 1 & YES & YES & YES & NO & 627\\
$(119, 50)$ & 10 & $(3, 1)$ & 2 & 1 & YES & YES & YES & NO & 628\\
$(119, 45)$ & 11 & $(5, 2)$ & 3 & 1 & YES & YES & YES & NO & 629\\
$(119, 46)$ & 10 & $(7, 3)$ & 4 & 7 & YES & YES & YES & NO & 630\\
$(119, 27)$ & 12 & $(8, 3)$ & 4 & 1 & YES & YES & YES & NO & 631\\
$(119, 44)$ & 10 & $(25, 9)$ & 7 & 1 & YES & YES & YES & NO & 632\\
$(119, 46)$ & 10 & $(28, 11)$ & 8 & 7 & YES & YES & YES & NO & 633\\
$(121, 35)$ & 12 & $(4, 1)$ & 3 & 1 & YES & YES & YES & NO & 634\\
$(121, 38)$ & 12 & $(5, 2)$ & 3 & 1 & YES & YES & YES & NO & 635\\
$(121, 43)$ & 11 & $(5, 2)$ & 3 & 1 & YES & YES & YES & NO & 636\\
$(121, 43)$ & 11 & $(7, 2)$ & 4 & 1 & YES & YES & YES & NO & 637\\
$(121, 25)$ & 13 & $(8, 3)$ & 4 & 1 & YES & YES & YES & NO & 638\\
$(121, 32)$ & 11 & $(9, 2)$ & 5 & 1 & YES & YES & YES & NO & 639\\
$(121, 25)$ & 13 & $(10, 3)$ & 5 & 1 & YES & YES & YES & NO & 640\\
$(121, 43)$ & 11 & $(20, 7)$ & 8 & 1 & YES & YES & YES & 828 & 641\\
$(121, 32)$ & 11 & $(27, 7)$ & 9 & 1 & YES & YES & YES & NO & 642\\
$(121, 43)$ & 11 & $(48, 17)$ & 9 & 1 & YES & YES & YES & NO & 643\\
$(122, 37)$ & 11 & $(8, 3)$ & 4 & 2 & YES & YES & YES & NO & 644\\
$(123, 28)$ & 12 & $(7, 2)$ & 4 & 1 & YES & YES & YES & NO & 645\\
$(123, 47)$ & 10 & $(89, 34)$ & 9 & 1 & YES & YES & YES & NO & 646\\
$(124, 27)$ & 12 & $(5, 1)$ & 4 & 1 & YES & YES & YES & NO & 647\\
$(124, 27)$ & 12 & $(8, 3)$ & 4 & 4 & YES & YES & YES & NO & 648\\
$(124, 39)$ & 12 & $(22, 7)$ & 9 & 2 & YES & YES & YES & 697 & 649\\
$(124, 27)$ & 12 & $(60, 13)$ & 9 & 4 & YES & YES & YES & NO & 650\\
$(124, 23)$ & 12 & $(92, 17)$ & 11 & 4 & YES & YES & YES & NO & 651\\
$(125, 27)$ & 11 & $(21, 5)$ & 8 & 1 & YES & YES & YES & 428 & 652\\
$(125, 27)$ & 11 & $(31, 7)$ & 8 & 1 & YES & YES & YES & NO & 653\\
$(127, 57)$ & 11 & $(13, 6)$ & 7 & 1 & YES & YES & YES & NO & 654\\
$(128, 45)$ & 12 & $(3, 1)$ & 2 & 1 & YES & YES & YES & NO & 655\\
$(128, 45)$ & 12 & $(5, 2)$ & 3 & 1 & YES & YES & YES & NO & 656\\
$(128, 45)$ & 12 & $(6, 1)$ & 5 & 2 & YES & YES & YES & NO & 657\\
$(128, 45)$ & 12 & $(14, 5)$ & 6 & 2 & YES & YES & YES & 612 & 658\\
$(128, 47)$ & 10 & $(79, 29)$ & 9 & 1 & YES & YES & YES & NO & 659\\
$(129, 28)$ & 12 & $(13, 2)$ & 7 & 1 & YES & YES & YES & NO & 660\\
$(129, 49)$ & 10 & $(45, 17)$ & 9 & 3 & YES & YES & YES & NO & 661\\
$(129, 28)$ & 12 & $(55, 12)$ & 9 & 1 & YES & YES & YES & NO & 662\\
$(129, 59)$ & 12 & $(59, 27)$ & 10 & 1 & YES & YES & YES & 743 & 663\\
$(131, 46)$ & 12 & $(3, 1)$ & 2 & 1 & YES & YES & YES & NO & 664\\
$(131, 46)$ & 12 & $(3, 1)$ & 2 & 1 & YES & YES & YES & NO & 665\\
$(131, 50)$ & 10 & $(3, 1)$ & 2 & 1 & YES & YES & YES & NO & 666\\
$(131, 46)$ & 12 & $(5, 2)$ & 3 & 1 & YES & YES & YES & NO & 667\\
$(131, 47)$ & 11 & $(5, 2)$ & 3 & 1 & YES & YES & YES & NO & 668\\
$(131, 24)$ & 13 & $(10, 3)$ & 5 & 1 & YES & YES & YES & NO & 669\\
$(131, 24)$ & 13 & $(11, 3)$ & 5 & 1 & YES & YES & YES & NO & 670\\
$(131, 47)$ & 11 & $(36, 13)$ & 8 & 1 & YES & YES & YES & NO & 671\\
$(131, 47)$ & 11 & $(67, 24)$ & 10 & 1 & YES & YES & YES & 886 & 672\\
$(133, 37)$ & 13 & $(4, 1)$ & 3 & 1 & YES & YES & YES & NO & 673\\
$(133, 37)$ & 13 & $(6, 1)$ & 5 & 1 & YES & YES & YES & NO & 674\\
$(133, 37)$ & 13 & $(43, 12)$ & 8 & 1 & YES & YES & YES & NO & 675\\
$(133, 39)$ & 11 & $(44, 13)$ & 8 & 1 & YES & YES & YES & NO & 676\\
$(133, 37)$ & 13 & $(79, 22)$ & 10 & 1 & YES & YES & YES & NO & 677\\
$(135, 62)$ & 12 & $(5, 2)$ & 3 & 5 & YES & YES & YES & NO & 678\\
$(135, 62)$ & 12 & $(6, 1)$ & 5 & 3 & YES & YES & YES & NO & 679\\
$(135, 41)$ & 11 & $(7, 3)$ & 4 & 1 & YES & YES & YES & NO & 680\\
$(135, 62)$ & 12 & $(9, 4)$ & 5 & 9 & YES & YES & YES & NO & 681\\
$(135, 62)$ & 12 & $(61, 28)$ & 10 & 1 & YES & YES & YES & 754 & 682\\
$(136, 59)$ & 11 & $(5, 2)$ & 3 & 1 & YES & YES & YES & NO & 683\\
$(139, 61)$ & 11 & $(5, 2)$ & 3 & 1 & YES & YES & YES & NO & 684\\
$(139, 61)$ & 11 & $(12, 5)$ & 5 & 1 & YES & YES & YES & NO & 685\\
$(140, 37)$ & 11 & $(5, 2)$ & 3 & 5 & YES & YES & YES & NO & 686\\
$(140, 37)$ & 11 & $(5, 2)$ & 3 & 5 & YES & YES & YES & NO & 687\\
$(140, 37)$ & 11 & $(9, 2)$ & 5 & 1 & YES & YES & YES & NO & 688\\
$(140, 37)$ & 11 & $(10, 3)$ & 5 & 10 & YES & YES & YES & NO & 689\\
$(140, 37)$ & 11 & $(26, 7)$ & 7 & 2 & YES & YES & YES & NO & 690\\
$(140, 37)$ & 11 & $(121, 32)$ & 11 & 1 & YES & YES & YES & NO & 691\\
$(142, 41)$ & 13 & $(3, 1)$ & 2 & 1 & YES & YES & YES & NO & 692\\
$(142, 45)$ & 13 & $(3, 1)$ & 2 & 1 & YES & YES & YES & NO & 693\\
$(142, 41)$ & 13 & $(4, 1)$ & 3 & 2 & YES & YES & YES & NO & 694\\
$(142, 41)$ & 13 & $(5, 1)$ & 4 & 1 & YES & YES & YES & NO & 695\\
$(142, 41)$ & 13 & $(6, 1)$ & 5 & 2 & YES & YES & YES & NO & 696\\
$(142, 45)$ & 13 & $(16, 5)$ & 7 & 2 & YES & YES & YES & 649 & 697\\
$(142, 41)$ & 13 & $(17, 5)$ & 6 & 1 & YES & YES & YES & NO & 698\\
$(142, 51)$ & 11 & $(17, 6)$ & 7 & 1 & YES & YES & YES & NO & 699\\
$(142, 51)$ & 11 & $(19, 7)$ & 6 & 1 & YES & YES & YES & NO & 700\\
$(142, 41)$ & 13 & $(24, 7)$ & 7 & 2 & YES & YES & YES & 539 & 701\\
$(142, 41)$ & 13 & $(31, 9)$ & 8 & 1 & YES & YES & YES & NO & 702\\
$(142, 65)$ & 12 & $(35, 16)$ & 9 & 1 & YES & YES & YES & NO & 703\\
$(142, 41)$ & 13 & $(38, 11)$ & 9 & 2 & YES & YES & YES & NO & 704\\
$(142, 65)$ & 12 & $(59, 27)$ & 10 & 1 & YES & YES & YES & NO & 705\\
$(142, 65)$ & 12 & $(142, 65)$ & 12 & 142 & YES & YES & YES & NO & 706\\
$(143, 54)$ & 12 & $(3, 1)$ & 2 & 1 & YES & YES & YES & NO & 707\\
$(143, 63)$ & 11 & $(43, 19)$ & 9 & 1 & YES & YES & YES & NO & 708\\
$(143, 54)$ & 12 & $(143, 54)$ & 12 & 143 & YES & YES & YES & NO & 709\\
$(144, 31)$ & 12 & $(8, 3)$ & 4 & 8 & YES & YES & YES & NO & 710\\
$(144, 31)$ & 12 & $(10, 3)$ & 5 & 2 & YES & YES & YES & NO & 711\\
$(145, 44)$ & 11 & $(3, 1)$ & 2 & 1 & YES & YES & YES & NO & 712\\
$(145, 51)$ & 12 & $(3, 1)$ & 2 & 1 & YES & YES & YES & NO & 713\\
$(145, 41)$ & 13 & $(5, 1)$ & 4 & 5 & YES & YES & YES & NO & 714\\
$(145, 52)$ & 11 & $(5, 2)$ & 3 & 5 & YES & YES & YES & NO & 715\\
$(145, 41)$ & 13 & $(11, 3)$ & 5 & 1 & YES & YES & YES & NO & 716\\
$(145, 42)$ & 12 & $(11, 3)$ & 5 & 1 & YES & YES & YES & NO & 717\\
$(145, 51)$ & 12 & $(20, 7)$ & 8 & 5 & YES & YES & YES & 734 & 718\\
$(145, 52)$ & 11 & $(67, 24)$ & 10 & 1 & YES & YES & YES & NO & 719\\
$(146, 41)$ & 11 & $(4, 1)$ & 3 & 2 & YES & YES & YES & NO & 720\\
$(146, 33)$ & 12 & $(23, 5)$ & 7 & 1 & YES & YES & YES & NO & 721\\
$(146, 67)$ & 12 & $(37, 17)$ & 9 & 1 & YES & YES & YES & NO & 722\\
$(146, 67)$ & 12 & $(61, 28)$ & 10 & 1 & YES & YES & YES & NO & 723\\
$(146, 67)$ & 12 & $(146, 67)$ & 12 & 146 & YES & YES & YES & NO & 724\\
$(147, 32)$ & 13 & $(5, 2)$ & 3 & 1 & YES & YES & YES & NO & 725\\
$(147, 53)$ & 11 & $(7, 3)$ & 4 & 7 & YES & YES & YES & NO & 726\\
$(147, 41)$ & 11 & $(17, 5)$ & 6 & 1 & YES & YES & YES & NO & 727\\
$(147, 53)$ & 11 & $(19, 7)$ & 6 & 1 & YES & YES & YES & NO & 728\\
$(149, 65)$ & 11 & $(8, 3)$ & 4 & 1 & YES & YES & YES & NO & 729\\
$(149, 40)$ & 11 & $(26, 7)$ & 7 & 1 & YES & YES & YES & NO & 730\\
$(149, 65)$ & 11 & $(71, 31)$ & 10 & 1 & YES & YES & YES & NO & 731\\
$(151, 53)$ & 12 & $(3, 1)$ & 2 & 1 & YES & YES & YES & NO & 732\\
$(151, 27)$ & 13 & $(10, 3)$ & 5 & 1 & YES & YES & YES & NO & 733\\
$(151, 53)$ & 12 & $(17, 6)$ & 7 & 1 & YES & YES & YES & 718 & 734\\
$(151, 28)$ & 13 & $(65, 12)$ & 10 & 1 & YES & YES & YES & NO & 735\\
$(152, 67)$ & 11 & $(12, 5)$ & 5 & 4 & YES & YES & YES & NO & 736\\
$(152, 67)$ & 11 & $(23, 10)$ & 7 & 1 & YES & YES & YES & NO & 737\\
$(152, 41)$ & 11 & $(59, 16)$ & 10 & 1 & YES & YES & YES & 605 & 738\\
$(153, 59)$ & 12 & $(5, 1)$ & 4 & 1 & YES & YES & YES & NO & 739\\
$(153, 56)$ & 11 & $(7, 2)$ & 4 & 1 & YES & YES & YES & NO & 740\\
$(153, 70)$ & 12 & $(13, 6)$ & 7 & 1 & YES & YES & YES & NO & 741\\
$(153, 56)$ & 11 & $(27, 10)$ & 7 & 9 & YES & YES & YES & NO & 742\\
$(153, 70)$ & 12 & $(35, 16)$ & 9 & 1 & YES & YES & YES & 663 & 743\\
$(154, 47)$ & 11 & $(2, 1)$ & 1 & 2 & YES & YES & YES & NO & 744\\
$(155, 48)$ & 12 & $(5, 2)$ & 3 & 5 & YES & YES & YES & NO & 745\\
$(156, 35)$ & 13 & $(14, 3)$ & 6 & 2 & YES & YES & YES & NO & 746\\
$(157, 69)$ & 11 & $(3, 1)$ & 2 & 1 & YES & YES & YES & NO & 747\\
$(157, 69)$ & 11 & $(5, 2)$ & 3 & 1 & YES & YES & YES & NO & 748\\
$(157, 69)$ & 11 & $(41, 18)$ & 8 & 1 & YES & YES & YES & NO & 749\\
$(157, 69)$ & 11 & $(157, 69)$ & 11 & 157 & YES & YES & YES & NO & 750\\
$(158, 57)$ & 11 & $(7, 2)$ & 4 & 1 & YES & YES & YES & NO & 751\\
$(159, 61)$ & 12 & $(5, 1)$ & 4 & 1 & YES & YES & YES & NO & 752\\
$(159, 46)$ & 13 & $(11, 3)$ & 5 & 1 & YES & YES & YES & NO & 753\\
$(159, 73)$ & 12 & $(37, 17)$ & 9 & 1 & YES & YES & YES & 682 & 754\\
$(161, 57)$ & 12 & $(2, 1)$ & 1 & 1 & YES & YES & YES & NO & 755\\
$(161, 57)$ & 12 & $(3, 1)$ & 2 & 1 & YES & YES & YES & NO & 756\\
$(161, 45)$ & 11 & $(4, 1)$ & 3 & 1 & YES & YES & YES & NO & 757\\
$(161, 57)$ & 12 & $(11, 4)$ & 5 & 1 & YES & YES & YES & NO & 758\\
$(161, 45)$ & 11 & $(18, 5)$ & 6 & 1 & YES & YES & YES & 562 & 759\\
$(163, 71)$ & 11 & $(3, 1)$ & 2 & 1 & YES & YES & YES & NO & 760\\
$(165, 58)$ & 13 & $(3, 1)$ & 2 & 3 & YES & YES & YES & NO & 761\\
$(165, 61)$ & 11 & $(3, 1)$ & 2 & 3 & YES & YES & YES & NO & 762\\
$(165, 61)$ & 11 & $(3, 1)$ & 2 & 3 & YES & YES & YES & NO & 763\\
$(165, 58)$ & 13 & $(14, 5)$ & 6 & 1 & YES & YES & YES & NO & 764\\
$(166, 59)$ & 12 & $(3, 1)$ & 2 & 1 & YES & YES & YES & NO & 765\\
$(166, 59)$ & 12 & $(4, 1)$ & 3 & 2 & YES & YES & YES & NO & 766\\
$(167, 46)$ & 11 & $(2, 1)$ & 1 & 1 & YES & YES & YES & NO & 767\\
$(167, 46)$ & 11 & $(2, 1)$ & 1 & 1 & YES & YES & YES & NO & 768\\
$(167, 75)$ & 12 & $(3, 1)$ & 2 & 1 & YES & YES & YES & NO & 769\\
$(167, 46)$ & 11 & $(7, 2)$ & 4 & 1 & YES & YES & YES & NO & 770\\
$(167, 75)$ & 12 & $(9, 4)$ & 5 & 1 & YES & YES & YES & NO & 771\\
$(167, 75)$ & 12 & $(29, 13)$ & 8 & 1 & YES & YES & YES & NO & 772\\
$(167, 46)$ & 11 & $(69, 19)$ & 9 & 1 & YES & YES & YES & NO & 773\\
$(172, 75)$ & 12 & $(133, 58)$ & 11 & 1 & YES & YES & YES & NO & 774\\
$(172, 75)$ & 12 & $(172, 75)$ & 12 & 172 & YES & YES & YES & NO & 775\\
$(173, 76)$ & 11 & $(2, 1)$ & 1 & 1 & YES & YES & YES & NO & 776\\
$(173, 78)$ & 12 & $(2, 1)$ & 1 & 1 & YES & YES & YES & NO & 777\\
$(173, 45)$ & 13 & $(3, 1)$ & 2 & 1 & YES & YES & YES & NO & 778\\
$(173, 76)$ & 11 & $(3, 1)$ & 2 & 1 & YES & YES & YES & NO & 779\\
$(173, 78)$ & 12 & $(3, 1)$ & 2 & 1 & YES & YES & YES & NO & 780\\
$(173, 51)$ & 12 & $(4, 1)$ & 3 & 1 & YES & YES & YES & NO & 781\\
$(173, 76)$ & 11 & $(4, 1)$ & 3 & 1 & YES & YES & YES & NO & 782\\
$(173, 66)$ & 11 & $(7, 2)$ & 4 & 1 & YES & YES & YES & NO & 783\\
$(173, 76)$ & 11 & $(25, 11)$ & 7 & 1 & YES & YES & YES & NO & 784\\
$(173, 45)$ & 13 & $(27, 7)$ & 9 & 1 & YES & YES & YES & NO & 785\\
$(173, 32)$ & 14 & $(43, 8)$ & 9 & 1 & YES & YES & YES & NO & 786\\
$(173, 76)$ & 11 & $(66, 29)$ & 9 & 1 & YES & YES & YES & NO & 787\\
$(173, 78)$ & 12 & $(71, 32)$ & 10 & 1 & YES & YES & YES & 826 & 788\\
$(173, 76)$ & 11 & $(173, 76)$ & 11 & 173 & YES & YES & YES & NO & 789\\
$(173, 78)$ & 12 & $(173, 78)$ & 12 & 173 & YES & YES & YES & NO & 790\\
$(175, 62)$ & 12 & $(3, 1)$ & 2 & 1 & YES & YES & YES & NO & 791\\
$(175, 48)$ & 12 & $(5, 2)$ & 3 & 5 & YES & YES & YES & NO & 792\\
$(175, 48)$ & 12 & $(5, 2)$ & 3 & 5 & YES & YES & YES & NO & 793\\
$(175, 62)$ & 12 & $(5, 1)$ & 4 & 5 & YES & YES & YES & NO & 794\\
$(175, 62)$ & 12 & $(5, 2)$ & 3 & 5 & YES & YES & YES & NO & 795\\
$(175, 48)$ & 12 & $(10, 3)$ & 5 & 5 & YES & YES & YES & NO & 796\\
$(175, 62)$ & 12 & $(14, 5)$ & 6 & 7 & YES & YES & YES & NO & 797\\
$(176, 79)$ & 12 & $(2, 1)$ & 1 & 2 & YES & YES & YES & NO & 798\\
$(177, 40)$ & 13 & $(5, 2)$ & 3 & 1 & YES & YES & YES & NO & 799\\
$(177, 40)$ & 13 & $(5, 2)$ & 3 & 1 & YES & YES & YES & NO & 800\\
$(177, 40)$ & 13 & $(14, 3)$ & 6 & 1 & YES & YES & YES & NO & 801\\
$(179, 52)$ & 13 & $(3, 1)$ & 2 & 1 & YES & YES & YES & NO & 802\\
$(179, 63)$ & 13 & $(3, 1)$ & 2 & 1 & YES & YES & YES & NO & 803\\
$(179, 28)$ & 14 & $(7, 2)$ & 4 & 1 & YES & YES & YES & NO & 804\\
$(179, 28)$ & 14 & $(9, 2)$ & 5 & 1 & YES & YES & YES & NO & 805\\
$(179, 78)$ & 12 & $(9, 4)$ & 5 & 1 & YES & YES & YES & NO & 806\\
$(179, 32)$ & 14 & $(45, 8)$ & 9 & 1 & YES & YES & YES & NO & 807\\
$(179, 28)$ & 14 & $(77, 12)$ & 11 & 1 & YES & YES & YES & NO & 808\\
$(179, 78)$ & 12 & $(179, 78)$ & 12 & 179 & YES & YES & YES & NO & 809\\
$(181, 51)$ & 12 & $(4, 1)$ & 3 & 1 & YES & YES & YES & NO & 810\\
$(183, 53)$ & 13 & $(3, 1)$ & 2 & 3 & YES & YES & YES & NO & 811\\
$(183, 65)$ & 12 & $(3, 1)$ & 2 & 3 & YES & YES & YES & NO & 812\\
$(183, 40)$ & 13 & $(5, 2)$ & 3 & 1 & YES & YES & YES & NO & 813\\
$(183, 53)$ & 13 & $(183, 53)$ & 13 & 183 & YES & YES & YES & NO & 814\\
$(188, 79)$ & 11 & $(2, 1)$ & 1 & 2 & YES & YES & YES & NO & 815\\
$(188, 73)$ & 12 & $(6, 1)$ & 5 & 2 & YES & YES & YES & NO & 816\\
$(188, 41)$ & 12 & $(24, 5)$ & 8 & 4 & YES & YES & YES & NO & 817\\
$(188, 41)$ & 12 & $(60, 13)$ & 9 & 4 & YES & YES & YES & NO & 818\\
$(188, 73)$ & 12 & $(85, 33)$ & 10 & 1 & YES & YES & YES & NO & 819\\
$(189, 82)$ & 12 & $(3, 1)$ & 2 & 3 & YES & YES & YES & NO & 820\\
$(191, 40)$ & 13 & $(5, 2)$ & 3 & 1 & YES & YES & YES & NO & 821\\
$(193, 81)$ & 11 & $(2, 1)$ & 1 & 1 & YES & YES & YES & NO & 822\\
$(193, 87)$ & 12 & $(3, 1)$ & 2 & 1 & YES & YES & YES & NO & 823\\
$(193, 51)$ & 12 & $(4, 1)$ & 3 & 1 & YES & YES & YES & NO & 824\\
$(193, 80)$ & 12 & $(7, 3)$ & 4 & 1 & YES & YES & YES & NO & 825\\
$(193, 87)$ & 12 & $(51, 23)$ & 9 & 1 & YES & YES & YES & 788 & 826\\
$(193, 51)$ & 12 & $(140, 37)$ & 11 & 1 & YES & YES & YES & NO & 827\\
$(197, 70)$ & 12 & $(3, 1)$ & 2 & 1 & YES & YES & YES & 641 & 828\\
$(197, 76)$ & 12 & $(5, 1)$ & 4 & 1 & YES & YES & YES & NO & 829\\
$(197, 76)$ & 12 & $(70, 27)$ & 10 & 1 & YES & YES & YES & NO & 830\\
$(200, 37)$ & 15 & $(2, 1)$ & 1 & 2 & YES & YES & YES & NO & 831\\
$(200, 37)$ & 15 & $(2, 1)$ & 1 & 2 & YES & YES & YES & NO & 832\\
$(200, 53)$ & 12 & $(5, 2)$ & 3 & 5 & YES & YES & YES & NO & 833\\
$(200, 37)$ & 15 & $(6, 1)$ & 5 & 2 & YES & YES & YES & NO & 834\\
$(200, 37)$ & 15 & $(16, 3)$ & 7 & 8 & YES & YES & YES & NO & 835\\
$(201, 76)$ & 12 & $(82, 31)$ & 10 & 1 & YES & YES & YES & NO & 836\\
$(202, 89)$ & 12 & $(2, 1)$ & 1 & 2 & YES & YES & YES & NO & 837\\
$(202, 89)$ & 12 & $(3, 1)$ & 2 & 1 & YES & YES & YES & NO & 838\\
$(202, 89)$ & 12 & $(4, 1)$ & 3 & 2 & YES & YES & YES & NO & 839\\
$(202, 89)$ & 12 & $(7, 3)$ & 4 & 1 & YES & YES & YES & NO & 840\\
$(202, 89)$ & 12 & $(9, 4)$ & 5 & 1 & YES & YES & YES & NO & 841\\
$(203, 86)$ & 12 & $(3, 1)$ & 2 & 1 & YES & YES & YES & NO & 842\\
$(203, 86)$ & 12 & $(203, 86)$ & 12 & 203 & YES & YES & YES & NO & 843\\
$(204, 89)$ & 12 & $(2, 1)$ & 1 & 2 & YES & YES & YES & NO & 844\\
$(205, 38)$ & 15 & $(2, 1)$ & 1 & 1 & YES & YES & YES & NO & 845\\
$(205, 92)$ & 12 & $(2, 1)$ & 1 & 1 & YES & YES & YES & NO & 846\\
$(205, 38)$ & 15 & $(6, 1)$ & 5 & 1 & YES & YES & YES & NO & 847\\
$(206, 73)$ & 12 & $(2, 1)$ & 1 & 2 & YES & YES & YES & NO & 848\\
$(206, 79)$ & 12 & $(2, 1)$ & 1 & 2 & YES & YES & YES & NO & 849\\
$(206, 73)$ & 12 & $(3, 1)$ & 2 & 1 & YES & YES & YES & NO & 850\\
$(206, 79)$ & 12 & $(5, 1)$ & 4 & 1 & YES & YES & YES & NO & 851\\
$(206, 73)$ & 12 & $(14, 5)$ & 6 & 2 & YES & YES & YES & NO & 852\\
$(206, 73)$ & 12 & $(31, 11)$ & 8 & 1 & YES & YES & YES & NO & 853\\
$(206, 79)$ & 12 & $(73, 28)$ & 10 & 1 & YES & YES & YES & NO & 854\\
$(207, 37)$ & 15 & $(5, 1)$ & 4 & 1 & YES & YES & YES & NO & 855\\
$(207, 37)$ & 15 & $(17, 3)$ & 7 & 1 & YES & YES & YES & NO & 856\\
$(208, 61)$ & 12 & $(92, 27)$ & 11 & 4 & YES & YES & YES & NO & 857\\
$(209, 91)$ & 12 & $(2, 1)$ & 1 & 1 & YES & YES & YES & NO & 858\\
$(209, 79)$ & 12 & $(3, 1)$ & 2 & 1 & YES & YES & YES & NO & 859\\
$(209, 56)$ & 12 & $(5, 2)$ & 3 & 1 & YES & YES & YES & NO & 860\\
$(209, 79)$ & 12 & $(82, 31)$ & 10 & 1 & YES & YES & YES & NO & 861\\
$(211, 93)$ & 12 & $(7, 3)$ & 4 & 1 & YES & YES & YES & NO & 862\\
$(212, 97)$ & 13 & $(2, 1)$ & 1 & 2 & NO & YES & YES & NO & 863\\
$(212, 81)$ & 11 & $(7, 2)$ & 4 & 1 & YES & YES & YES & NO & 864\\
$(212, 93)$ & 12 & $(212, 93)$ & 12 & 212 & YES & YES & YES & NO & 865\\
$(213, 38)$ & 15 & $(5, 1)$ & 4 & 1 & YES & YES & YES & NO & 866\\
$(213, 62)$ & 12 & $(7, 2)$ & 4 & 1 & YES & YES & YES & NO & 867\\
$(213, 59)$ & 12 & $(25, 7)$ & 7 & 1 & YES & YES & YES & NO & 868\\
$(213, 62)$ & 12 & $(38, 11)$ & 9 & 1 & YES & YES & YES & 968 & 869\\
$(215, 57)$ & 12 & $(5, 2)$ & 3 & 5 & YES & YES & YES & NO & 870\\
$(215, 82)$ & 12 & $(5, 1)$ & 4 & 5 & YES & YES & YES & NO & 871\\
$(215, 83)$ & 12 & $(13, 5)$ & 5 & 1 & YES & YES & YES & NO & 872\\
$(215, 57)$ & 12 & $(64, 17)$ & 10 & 1 & YES & YES & YES & NO & 873\\
$(217, 78)$ & 12 & $(14, 5)$ & 6 & 7 & YES & YES & YES & NO & 874\\
$(218, 45)$ & 14 & $(4, 1)$ & 3 & 2 & YES & YES & YES & NO & 875\\
$(218, 57)$ & 13 & $(19, 5)$ & 7 & 1 & YES & YES & YES & NO & 876\\
$(219, 79)$ & 12 & $(3, 1)$ & 2 & 3 & YES & YES & YES & NO & 877\\
$(219, 79)$ & 12 & $(3, 1)$ & 2 & 3 & YES & YES & YES & NO & 878\\
$(219, 95)$ & 12 & $(16, 7)$ & 6 & 1 & YES & YES & YES & 932 & 879\\
$(219, 95)$ & 12 & $(136, 59)$ & 11 & 1 & YES & YES & YES & NO & 880\\
$(221, 82)$ & 12 & $(19, 7)$ & 6 & 1 & YES & YES & YES & NO & 881\\
$(223, 58)$ & 14 & $(2, 1)$ & 1 & 1 & YES & YES & YES & NO & 882\\
$(223, 80)$ & 12 & $(2, 1)$ & 1 & 1 & YES & YES & YES & NO & 883\\
$(223, 98)$ & 12 & $(3, 1)$ & 2 & 1 & YES & YES & YES & NO & 884\\
$(223, 80)$ & 12 & $(11, 4)$ & 5 & 1 & YES & YES & YES & NO & 885\\
$(223, 80)$ & 12 & $(14, 5)$ & 6 & 1 & YES & YES & YES & 672 & 886\\
$(223, 80)$ & 12 & $(92, 33)$ & 10 & 1 & YES & YES & YES & NO & 887\\
$(225, 98)$ & 12 & $(2, 1)$ & 1 & 1 & YES & YES & YES & NO & 888\\
$(225, 98)$ & 12 & $(23, 10)$ & 7 & 1 & YES & YES & YES & NO & 889\\
$(227, 83)$ & 12 & $(3, 1)$ & 2 & 1 & YES & YES & YES & NO & 890\\
$(227, 100)$ & 12 & $(7, 3)$ & 4 & 1 & YES & YES & YES & NO & 891\\
$(229, 94)$ & 12 & $(2, 1)$ & 1 & 1 & YES & YES & YES & NO & 892\\
$(231, 83)$ & 12 & $(3, 1)$ & 2 & 3 & YES & YES & YES & NO & 893\\
$(231, 83)$ & 12 & $(39, 14)$ & 8 & 3 & YES & YES & YES & NO & 894\\
$(232, 101)$ & 12 & $(3, 1)$ & 2 & 1 & YES & YES & YES & NO & 895\\
$(232, 101)$ & 12 & $(39, 17)$ & 8 & 1 & YES & YES & YES & 933 & 896\\
$(233, 89)$ & 11 & $(5, 2)$ & 3 & 1 & YES & NO & YES & NO & 897\\
$(233, 89)$ & 11 & $(76, 29)$ & 9 & 1 & YES & NO & YES & NO & 898\\
$(236, 55)$ & 13 & $(5, 2)$ & 3 & 1 & YES & YES & YES & NO & 899\\
$(237, 37)$ & 16 & $(19, 3)$ & 8 & 1 & YES & YES & YES & NO & 900\\
$(238, 69)$ & 13 & $(4, 1)$ & 3 & 2 & YES & YES & YES & NO & 901\\
$(242, 75)$ & 13 & $(2, 1)$ & 1 & 2 & YES & YES & YES & NO & 902\\
$(242, 63)$ & 14 & $(3, 1)$ & 2 & 1 & YES & YES & YES & NO & 903\\
$(242, 71)$ & 13 & $(58, 17)$ & 9 & 2 & YES & YES & YES & NO & 904\\
$(243, 38)$ & 16 & $(2, 1)$ & 1 & 1 & YES & YES & YES & NO & 905\\
$(243, 65)$ & 13 & $(3, 1)$ & 2 & 3 & YES & YES & YES & NO & 906\\
$(243, 89)$ & 12 & $(3, 1)$ & 2 & 3 & YES & YES & YES & NO & 907\\
$(243, 65)$ & 13 & $(4, 1)$ & 3 & 1 & YES & YES & YES & NO & 908\\
$(243, 94)$ & 12 & $(5, 1)$ & 4 & 1 & YES & YES & YES & NO & 909\\
$(243, 38)$ & 16 & $(7, 1)$ & 6 & 1 & YES & YES & YES & NO & 910\\
$(244, 37)$ & 16 & $(2, 1)$ & 1 & 2 & YES & YES & YES & NO & 911\\
$(244, 37)$ & 16 & $(2, 1)$ & 1 & 2 & YES & YES & YES & NO & 912\\
$(244, 37)$ & 16 & $(6, 1)$ & 5 & 2 & YES & YES & YES & NO & 913\\
$(244, 37)$ & 16 & $(20, 3)$ & 8 & 4 & YES & YES & YES & NO & 914\\
$(245, 71)$ & 13 & $(2, 1)$ & 1 & 1 & YES & YES & YES & NO & 915\\
$(245, 69)$ & 13 & $(3, 1)$ & 2 & 1 & YES & YES & YES & NO & 916\\
$(245, 71)$ & 13 & $(4, 1)$ & 3 & 1 & YES & YES & YES & NO & 917\\
$(245, 71)$ & 13 & $(38, 11)$ & 9 & 1 & YES & YES & YES & NO & 918\\
$(251, 104)$ & 12 & $(2, 1)$ & 1 & 1 & YES & YES & YES & NO & 919\\
$(251, 38)$ & 16 & $(6, 1)$ & 5 & 1 & YES & YES & YES & NO & 920\\
$(251, 74)$ & 13 & $(61, 18)$ & 9 & 1 & YES & YES & YES & NO & 921\\
$(251, 74)$ & 13 & $(251, 74)$ & 13 & 251 & YES & YES & YES & NO & 922\\
$(252, 71)$ & 13 & $(2, 1)$ & 1 & 2 & YES & YES & YES & NO & 923\\
$(252, 71)$ & 13 & $(3, 1)$ & 2 & 3 & YES & YES & YES & NO & 924\\
$(253, 106)$ & 12 & $(2, 1)$ & 1 & 1 & YES & YES & YES & NO & 925\\
$(255, 71)$ & 13 & $(3, 1)$ & 2 & 3 & YES & YES & YES & NO & 926\\
$(255, 71)$ & 13 & $(61, 17)$ & 9 & 1 & YES & YES & YES & NO & 927\\
$(263, 78)$ & 13 & $(3, 1)$ & 2 & 1 & YES & YES & YES & NO & 928\\
$(263, 78)$ & 13 & $(37, 11)$ & 8 & 1 & YES & YES & YES & NO & 929\\
$(264, 71)$ & 13 & $(3, 1)$ & 2 & 3 & YES & YES & YES & NO & 930\\
$(264, 115)$ & 12 & $(3, 1)$ & 2 & 3 & YES & YES & YES & NO & 931\\
$(264, 115)$ & 12 & $(7, 3)$ & 4 & 1 & YES & YES & YES & 879 & 932\\
$(264, 115)$ & 12 & $(23, 10)$ & 7 & 1 & YES & YES & YES & 896 & 933\\
$(265, 98)$ & 12 & $(3, 1)$ & 2 & 1 & YES & YES & YES & NO & 934\\
$(265, 98)$ & 12 & $(19, 7)$ & 6 & 1 & YES & YES & YES & NO & 935\\
$(268, 111)$ & 12 & $(2, 1)$ & 1 & 2 & YES & YES & YES & NO & 936\\
$(270, 103)$ & 12 & $(55, 21)$ & 8 & 5 & YES & YES & YES & 966 & 937\\
$(270, 103)$ & 12 & $(173, 66)$ & 11 & 1 & YES & YES & YES & NO & 938\\
$(273, 76)$ & 13 & $(5, 1)$ & 4 & 1 & YES & YES & YES & NO & 939\\
$(274, 43)$ & 15 & $(20, 3)$ & 8 & 2 & YES & YES & YES & NO & 940\\
$(275, 73)$ & 13 & $(34, 9)$ & 8 & 1 & YES & YES & YES & NO & 941\\
$(283, 75)$ & 13 & $(3, 1)$ & 2 & 1 & YES & YES & YES & 626 & 942\\
$(287, 51)$ & 15 & $(3, 1)$ & 2 & 1 & YES & YES & YES & NO & 943\\
$(287, 106)$ & 12 & $(5, 2)$ & 3 & 1 & YES & YES & YES & NO & 944\\
$(287, 65)$ & 14 & $(13, 3)$ & 6 & 1 & YES & YES & YES & NO & 945\\
$(291, 85)$ & 13 & $(7, 2)$ & 4 & 1 & YES & YES & YES & NO & 946\\
$(298, 53)$ & 15 & $(2, 1)$ & 1 & 2 & YES & YES & YES & NO & 947\\
$(298, 53)$ & 15 & $(3, 1)$ & 2 & 1 & YES & YES & YES & NO & 948\\
$(298, 79)$ & 13 & $(3, 1)$ & 2 & 1 & YES & YES & YES & NO & 949\\
$(298, 53)$ & 15 & $(11, 2)$ & 6 & 1 & YES & YES & YES & NO & 950\\
$(298, 45)$ & 15 & $(46, 7)$ & 10 & 2 & YES & YES & YES & NO & 951\\
$(301, 88)$ & 13 & $(65, 19)$ & 9 & 1 & YES & YES & YES & NO & 952\\
$(313, 71)$ & 14 & $(3, 1)$ & 2 & 1 & YES & YES & YES & NO & 953\\
$(313, 71)$ & 14 & $(5, 1)$ & 4 & 1 & YES & YES & YES & NO & 954\\
$(314, 83)$ & 13 & $(3, 1)$ & 2 & 1 & YES & YES & YES & NO & 955\\
$(314, 83)$ & 13 & $(34, 9)$ & 8 & 2 & YES & YES & YES & NO & 956\\
$(317, 121)$ & 12 & $(2, 1)$ & 1 & 1 & NO & YES & YES & NO & 957\\
$(317, 131)$ & 12 & $(2, 1)$ & 1 & 1 & YES & YES & YES & NO & 958\\
$(325, 74)$ & 14 & $(5, 1)$ & 4 & 5 & YES & YES & YES & NO & 959\\
$(325, 74)$ & 14 & $(101, 23)$ & 11 & 1 & YES & YES & YES & NO & 960\\
$(328, 97)$ & 13 & $(2, 1)$ & 1 & 2 & YES & YES & YES & NO & 961\\
$(335, 76)$ & 14 & $(2, 1)$ & 1 & 1 & YES & YES & YES & NO & 962\\
$(337, 147)$ & 13 & $(2, 1)$ & 1 & 1 & NO & YES & YES & NO & 963\\
$(338, 129)$ & 12 & $(3, 1)$ & 2 & 1 & YES & YES & YES & NO & 964\\
$(338, 131)$ & 12 & $(8, 3)$ & 4 & 2 & YES & YES & YES & 974 & 965\\
$(338, 129)$ & 12 & $(21, 8)$ & 6 & 1 & YES & YES & YES & 937 & 966\\
$(347, 79)$ & 14 & $(2, 1)$ & 1 & 1 & YES & YES & YES & NO & 967\\
$(347, 101)$ & 13 & $(7, 2)$ & 4 & 1 & YES & YES & YES & 869 & 968\\
$(347, 79)$ & 14 & $(123, 28)$ & 12 & 1 & YES & YES & YES & NO & 969\\
$(348, 103)$ & 13 & $(2, 1)$ & 1 & 2 & YES & YES & YES & NO & 970\\
$(356, 105)$ & 13 & $(10, 3)$ & 5 & 2 & YES & YES & YES & NO & 971\\
$(358, 75)$ & 14 & $(2, 1)$ & 1 & 2 & YES & YES & YES & NO & 972\\
$(362, 107)$ & 13 & $(2, 1)$ & 1 & 2 & YES & YES & YES & NO & 973\\
$(377, 144)$ & 12 & $(5, 2)$ & 3 & 1 & YES & YES & YES & 965 & 974\\
$(398, 111)$ & 13 & $(3, 1)$ & 2 & 1 & YES & NO & YES & NO & 975\\
$(413, 121)$ & 13 & $(2, 1)$ & 1 & 1 & YES & NO & YES & NO & 976\\
$(416, 115)$ & 13 & $(2, 1)$ & 1 & 2 & YES & NO & YES & NO & 977\\
$(416, 115)$ & 13 & $(29, 8)$ & 7 & 1 & YES & NO & YES & NO & 978\\
$(434, 121)$ & 13 & $(2, 1)$ & 1 & 2 & YES & NO & YES & NO & 979\\
$(a; 0, 0, 0; 3)$ & 4 & $(37, 8)$ & 8 & 1 & YES & YES & YES & NO & 980\\
$(a; 0, 0, 0; 3)$ & 4 & $(81, 34)$ & 9 & 3 & YES & YES & YES & NO & 981\\
$(a; 1, 0, 0; 13)$ & 5 & $(43, 18)$ & 8 & 1 & YES & YES & YES & NO & 982\\
$(a; 1, 0, 0; 13)$ & 5 & $(50, 21)$ & 8 & 1 & YES & YES & YES & NO & 983\\
$(a; 1, 1, 0; 19)$ & 6 & $(26, 11)$ & 7 & 1 & YES & YES & YES & NO & 984\\
$(a; 1, 1, 0; 19)$ & 6 & $(30, 13)$ & 8 & 1 & YES & YES & YES & NO & 985\\
$(a; 2, 0, 0; 17)$ & 6 & $(34, 13)$ & 7 & 17 & YES & YES & YES & NO & 986\\
$(a; 2, 0, 0; 17)$ & 6 & $(44, 13)$ & 8 & 1 & YES & YES & YES & NO & 987\\
$(a; 2, 1, 1; 37)$ & 8 & $(11, 4)$ & 5 & 1 & YES & YES & YES & NO & 988\\
$(a; 3, 0, 0; 7)$ & 7 & $(24, 7)$ & 7 & 1 & YES & YES & YES & NO & 989\\
$(a; 3, 0, 0; 7)$ & 7 & $(25, 7)$ & 7 & 1 & YES & YES & YES & NO & 990\\
$(a; 3, 0, 0; 7)$ & 7 & $(27, 8)$ & 7 & 1 & YES & YES & YES & NO & 991\\
$(a; 3, 0, 0; 7)$ & 7 & $(29, 8)$ & 7 & 1 & YES & YES & YES & NO & 992\\
$(a; 3, 0, 0; 7)$ & 7 & $(31, 7)$ & 8 & 1 & YES & YES & YES & NO & 993\\
$(a; 3, 1, 1; 46)$ & 9 & $(8, 3)$ & 4 & 2 & YES & YES & YES & NO & 994\\
$(a; 3, 1, 1; 46)$ & 9 & $(10, 3)$ & 5 & 2 & YES & YES & YES & NO & 995\\
$(a; 4, 1, 1; 55)$ & 10 & $(5, 2)$ & 3 & 5 & YES & YES & YES & NO & 996\\
$(a; 5, 1, 1; 64)$ & 11 & $(3, 1)$ & 2 & 1 & YES & YES & YES & NO & 997\\
$(a; 5, 1, 1; 64)$ & 11 & $(6, 1)$ & 5 & 2 & YES & YES & YES & NO & 998\\
$(b; 0, 0, 0; 14)$ & 5 & $(29, 13)$ & 8 & 1 & YES & YES & YES & NO & 999\\
$(b; 0, 0, 0; 14)$ & 5 & $(34, 15)$ & 8 & 2 & YES & YES & YES & NO & 1000\\
$(b; 0, 0, 0; 14)$ & 5 & $(35, 16)$ & 9 & 7 & YES & YES & YES & NO & 1001\\
$(b; 0, 0, 0; 14)$ & 5 & $(38, 17)$ & 9 & 2 & YES & YES & YES & NO & 1002\\
$(b; 0, 0, 0; 14)$ & 5 & $(46, 19)$ & 8 & 2 & YES & YES & YES & NO & 1003\\
$(b; 0, 0, 1; 4)$ & 6 & $(27, 10)$ & 7 & 1 & YES & YES & YES & NO & 1004\\
$(b; 0, 1, 0; 19)$ & 6 & $(23, 10)$ & 7 & 1 & YES & YES & YES & NO & 1005\\
$(b; 0, 1, 1; 27)$ & 7 & $(17, 7)$ & 6 & 1 & YES & YES & YES & NO & 1006\\
$(b; 0, 1, 3; 43)$ & 9 & $(10, 3)$ & 5 & 1 & YES & YES & YES & NO & 1007\\
$(b; 1, 0, 1; 29)$ & 7 & $(21, 8)$ & 6 & 1 & YES & YES & YES & NO & 1008\\
$(b; 1, 0, 3; 47)$ & 9 & $(8, 3)$ & 4 & 1 & YES & YES & YES & NO & 1009\\
$(b; 1, 1, 0; 27)$ & 7 & $(17, 7)$ & 6 & 1 & YES & YES & YES & NO & 1010\\
$(b; 1, 1, 1; 39)$ & 8 & $(9, 4)$ & 5 & 3 & YES & YES & YES & NO & 1011\\
$(b; 1, 3, 0; 41)$ & 9 & $(13, 3)$ & 6 & 1 & YES & YES & YES & NO & 1012\\
$(b; 1, 3, 0; 41)$ & 9 & $(14, 3)$ & 6 & 1 & YES & YES & YES & NO & 1013\\
$(b; 2, 2, 0; 44)$ & 9 & $(13, 3)$ & 6 & 1 & YES & YES & YES & NO & 1014\\
$(c; 0, 0, 0; 4)$ & 4 & $(54, 19)$ & 10 & 2 & YES & YES & YES & NO & 1015\\
$(c; 0, 0, 0; 4)$ & 4 & $(69, 31)$ & 10 & 1 & YES & YES & YES & NO & 1016\\
$(c; 0, 0, 0; 4)$ & 4 & $(89, 33)$ & 10 & 1 & YES & YES & YES & NO & 1017\\
$(c; 0, 0, 0; 4)$ & 4 & $(119, 44)$ & 10 & 1 & YES & YES & YES & NO & 1018\\
$(c; 0, 1, 0; 11)$ & 5 & $(45, 16)$ & 9 & 1 & YES & YES & YES & NO & 1019\\
$(c; 0, 1, 0; 11)$ & 5 & $(55, 24)$ & 9 & 11 & YES & YES & YES & NO & 1020\\
$(c; 0, 1, 0; 11)$ & 5 & $(89, 24)$ & 10 & 1 & YES & YES & YES & NO & 1021\\
$(c; 0, 1, 1; 5)$ & 6 & $(61, 17)$ & 9 & 1 & YES & YES & YES & NO & 1022\\
$(c; 0, 3, 1; 23)$ & 8 & $(25, 7)$ & 7 & 1 & YES & YES & YES & NO & 1023\\
$(c; 0, 3, 1; 23)$ & 8 & $(29, 8)$ & 7 & 1 & YES & YES & YES & NO & 1024\\
$(d; 0, 0, 0; 5)$ & 5 & $(45, 14)$ & 9 & 5 & YES & YES & YES & NO & 1025\\
$(e; 0, 2, 0; 6)$ & 7 & $(24, 7)$ & 7 & 6 & YES & YES & YES & NO & 1026\\
$(e; 1, 3, 0; 33)$ & 9 & $(10, 3)$ & 5 & 1 & YES & YES & YES & NO & 1027\\
$(e; 1, 3, 0; 33)$ & 9 & $(11, 3)$ & 5 & 11 & YES & YES & YES & NO & 1028\\
$(f; 0, 0, 0; 6)$ & 4 & $(104, 43)$ & 10 & 2 & YES & YES & YES & NO & 1029\\
$(f; 0, 0, 0; 6)$ & 4 & $(113, 33)$ & 11 & 1 & YES & YES & YES & NO & 1030\\
$(f; 0, 1, 0; 7)$ & 5 & $(75, 17)$ & 10 & 1 & YES & YES & YES & NO & 1031\\
$(g; 0, 0, 0; 19)$ & 6 & $(31, 13)$ & 7 & 1 & YES & YES & YES & NO & 1032\\
$(g; 0, 0, 2; 11)$ & 8 & $(7, 2)$ & 4 & 1 & YES & YES & YES & NO & 1033\\
$(g; 0, 0, 3; 40)$ & 9 & $(7, 2)$ & 4 & 1 & YES & YES & YES & NO & 1034\\
$(g; 0, 0, 3; 40)$ & 9 & $(9, 2)$ & 5 & 1 & YES & YES & YES & NO & 1035\\
$(g; 0, 1, 0; 24)$ & 7 & $(19, 8)$ & 6 & 1 & YES & YES & YES & NO & 1036\\
$(g; 1, 2, 0; 11)$ & 9 & $(4, 1)$ & 3 & 1 & YES & YES & YES & NO & 1037\\
$(g; 1, 3, 0; 13)$ & 10 & $(4, 1)$ & 3 & 1 & YES & YES & YES & NO & 1038\\
$(h; 0, 0, 0; 6)$ & 5 & $(29, 13)$ & 8 & 1 & YES & YES & YES & NO & 1039\\
$(h; 0, 1, 0; 8)$ & 6 & $(31, 13)$ & 7 & 1 & YES & YES & YES & NO & 1040\\
$(i; 0, 0, 0; 9)$ & 5 & $(47, 18)$ & 8 & 1 & YES & YES & YES & NO & 1041\\
$(i; 0, 0, 0; 9)$ & 5 & $(58, 17)$ & 9 & 1 & YES & YES & YES & NO & 1042\\
$(i; 0, 0, 0; 9)$ & 5 & $(75, 17)$ & 10 & 3 & YES & YES & YES & NO & 1043\\
$(i; 0, 0, 0; 9)$ & 5 & $(79, 18)$ & 10 & 1 & YES & YES & YES & NO & 1044\\
$(i; 0, 0, 0; 9)$ & 5 & $(83, 18)$ & 10 & 1 & YES & YES & YES & NO & 1045\\
$(i; 0, 2, 0; 15)$ & 7 & $(31, 7)$ & 8 & 1 & YES & YES & YES & NO & 1046\\
$(i; 0, 2, 0; 15)$ & 7 & $(35, 8)$ & 8 & 5 & YES & YES & YES & NO & 1047\\
$(j; 0, 0, 0; 8)$ & 5 & $(76, 29)$ & 9 & 4 & YES & YES & YES & NO & 1048\\
$(j; 0, 1, 0; 10)$ & 6 & $(44, 17)$ & 8 & 2 & YES & YES & YES & NO & 1049\\
$(j; 0, 1, 0; 10)$ & 6 & $(47, 18)$ & 8 & 1 & YES & YES & YES & NO & 1050\\
$(j; 0, 2, 0; 12)$ & 7 & $(37, 11)$ & 8 & 1 & YES & YES & YES & NO & 1051\\
$(j; 0, 3, 0; 14)$ & 8 & $(21, 8)$ & 6 & 7 & YES & YES & YES & NO & 1052
\end{longtable}
\subsection{2 chains, $K^2 = 4$}
\begin{longtable}{|c|c|c|c|c|c|c|c|c|c|}
\hline
\multicolumn{10}{|c|}{2 chains, $K^2 = 4$}\\
\hline
$(n,a)$ & Length & $(n,a)$ & Length & GCD & Nef & $\mathbb Q$-ef & Obstruction 0 & WH & Index\\
\hline
\endfirsthead

\hline
$(n,a)$ & Length & $(n,a)$ & Length & GCD & Nef & $\mathbb Q$-ef & Obstruction 0 & WH & Index\\
\hline
\endhead
\hline
\endfoot

$(47, 14)$ & 9 & $(34, 13)$ & 7 & 1 & YES & YES & YES & NO & 1053\\
$(53, 22)$ & 9 & $(47, 13)$ & 8 & 1 & YES & YES & YES & NO & 1054\\
$(57, 17)$ & 10 & $(34, 13)$ & 7 & 1 & YES & YES & YES & NO & 1055\\
$(59, 23)$ & 9 & $(49, 18)$ & 8 & 1 & YES & YES & YES & NO & 1056\\
$(63, 26)$ & 9 & $(23, 9)$ & 7 & 1 & YES & YES & YES & NO & 1057\\
$(71, 27)$ & 9 & $(17, 6)$ & 7 & 1 & YES & YES & YES & NO & 1058\\
$(79, 18)$ & 10 & $(41, 17)$ & 8 & 1 & YES & YES & YES & NO & 1059\\
$(87, 16)$ & 11 & $(53, 22)$ & 9 & 1 & YES & YES & YES & NO & 1060\\
$(87, 16)$ & 11 & $(53, 22)$ & 9 & 1 & YES & YES & YES & NO & 1061\\
$(89, 34)$ & 9 & $(40, 11)$ & 8 & 1 & YES & YES & YES & NO & 1062\\
$(95, 36)$ & 10 & $(17, 7)$ & 6 & 1 & YES & YES & YES & NO & 1063\\
$(102, 31)$ & 11 & $(23, 10)$ & 7 & 1 & YES & YES & YES & NO & 1064\\
$(106, 41)$ & 10 & $(30, 13)$ & 8 & 2 & YES & YES & YES & NO & 1065\\
$(141, 26)$ & 12 & $(22, 9)$ & 7 & 1 & YES & YES & YES & NO & 1066\\
$(151, 56)$ & 11 & $(11, 4)$ & 5 & 1 & YES & YES & YES & NO & 1067\\
$(154, 47)$ & 11 & $(40, 11)$ & 8 & 2 & YES & YES & YES & NO & 1068\\
$(167, 60)$ & 11 & $(13, 3)$ & 6 & 1 & YES & YES & YES & NO & 1069\\
$(179, 68)$ & 11 & $(10, 3)$ & 5 & 1 & YES & YES & YES & NO & 1070\\
$(179, 68)$ & 11 & $(11, 3)$ & 5 & 1 & YES & YES & YES & NO & 1071\\
$(193, 60)$ & 12 & $(13, 3)$ & 6 & 1 & YES & YES & YES & NO & 1072\\
$(193, 60)$ & 12 & $(13, 3)$ & 6 & 1 & YES & YES & YES & NO & 1073\\
$(222, 91)$ & 12 & $(41, 17)$ & 8 & 1 & YES & YES & YES & NO & 1074\\
$(231, 97)$ & 12 & $(8, 3)$ & 4 & 1 & YES & YES & YES & NO & 1075\\
$(239, 99)$ & 12 & $(11, 4)$ & 5 & 1 & YES & YES & YES & NO & 1076\\
$(245, 88)$ & 12 & $(167, 60)$ & 11 & 1 & YES & YES & YES & NO & 1077\\
$(253, 93)$ & 12 & $(8, 3)$ & 4 & 1 & YES & YES & YES & NO & 1078\\
$(257, 71)$ & 12 & $(8, 3)$ & 4 & 1 & YES & NO & YES & NO & 1079\\
$(275, 102)$ & 13 & $(25, 9)$ & 7 & 25 & YES & YES & YES & NO & 1080\\
$(277, 116)$ & 12 & $(10, 3)$ & 5 & 1 & YES & YES & YES & NO & 1081\\
$(283, 88)$ & 13 & $(193, 60)$ & 12 & 1 & YES & YES & YES & NO & 1082\\
$(283, 104)$ & 12 & $(253, 93)$ & 12 & 1 & YES & YES & YES & NO & 1083\\
$(287, 111)$ & 12 & $(8, 3)$ & 4 & 1 & YES & NO & YES & NO & 1084\\
$(287, 111)$ & 12 & $(13, 3)$ & 6 & 1 & YES & NO & YES & NO & 1085\\
$(287, 109)$ & 12 & $(34, 13)$ & 7 & 1 & YES & NO & YES & NO & 1086\\
$(289, 112)$ & 12 & $(5, 2)$ & 3 & 1 & YES & NO & YES & NO & 1087\\
$(292, 111)$ & 12 & $(5, 2)$ & 3 & 1 & YES & YES & YES & NO & 1088\\
$(292, 111)$ & 12 & $(8, 3)$ & 4 & 4 & YES & NO & YES & NO & 1089\\
$(313, 86)$ & 13 & $(76, 21)$ & 9 & 1 & YES & YES & YES & NO & 1090\\
$(317, 131)$ & 12 & $(53, 22)$ & 9 & 1 & YES & YES & YES & NO & 1091\\
$(321, 118)$ & 13 & $(283, 104)$ & 12 & 1 & YES & YES & YES & 1146 & 1092\\
$(360, 101)$ & 13 & $(7, 3)$ & 4 & 1 & YES & YES & YES & NO & 1093\\
$(366, 139)$ & 13 & $(34, 13)$ & 7 & 2 & YES & YES & YES & NO & 1094\\
$(368, 107)$ & 13 & $(13, 3)$ & 6 & 1 & YES & YES & YES & NO & 1095\\
$(369, 143)$ & 13 & $(4, 1)$ & 3 & 1 & YES & YES & YES & NO & 1096\\
$(369, 143)$ & 13 & $(289, 112)$ & 12 & 1 & YES & YES & YES & NO & 1097\\
$(373, 154)$ & 13 & $(41, 17)$ & 8 & 1 & YES & YES & YES & NO & 1098\\
$(380, 157)$ & 13 & $(8, 3)$ & 4 & 4 & YES & YES & YES & NO & 1099\\
$(380, 157)$ & 13 & $(53, 22)$ & 9 & 1 & YES & YES & YES & NO & 1100\\
$(381, 107)$ & 13 & $(7, 3)$ & 4 & 1 & YES & YES & YES & NO & 1101\\
$(393, 152)$ & 13 & $(256, 99)$ & 12 & 1 & YES & NO & YES & 1132 & 1102\\
$(403, 153)$ & 13 & $(7, 2)$ & 4 & 1 & YES & YES & YES & NO & 1103\\
$(403, 153)$ & 13 & $(71, 27)$ & 9 & 1 & YES & YES & YES & NO & 1104\\
$(413, 157)$ & 13 & $(5, 2)$ & 3 & 1 & YES & YES & YES & NO & 1105\\
$(431, 160)$ & 14 & $(2, 1)$ & 1 & 1 & YES & YES & YES & NO & 1106\\
$(431, 128)$ & 13 & $(5, 2)$ & 3 & 1 & YES & YES & YES & NO & 1107\\
$(433, 179)$ & 13 & $(10, 3)$ & 5 & 1 & YES & YES & YES & NO & 1108\\
$(437, 169)$ & 13 & $(5, 2)$ & 3 & 1 & YES & NO & YES & NO & 1109\\
$(437, 166)$ & 13 & $(7, 3)$ & 4 & 1 & YES & YES & YES & NO & 1110\\
$(437, 169)$ & 13 & $(287, 111)$ & 12 & 1 & YES & NO & YES & NO & 1111\\
$(443, 182)$ & 14 & $(3, 1)$ & 2 & 1 & YES & YES & YES & NO & 1112\\
$(467, 87)$ & 15 & $(23, 4)$ & 8 & 1 & YES & YES & YES & NO & 1113\\
$(497, 204)$ & 14 & $(22, 9)$ & 7 & 1 & YES & YES & YES & NO & 1114\\
$(505, 192)$ & 13 & $(21, 8)$ & 6 & 1 & YES & NO & YES & NO & 1115\\
$(511, 194)$ & 14 & $(4, 1)$ & 3 & 1 & YES & YES & YES & NO & 1116\\
$(515, 191)$ & 14 & $(27, 10)$ & 7 & 1 & YES & YES & YES & NO & 1117\\
$(523, 204)$ & 14 & $(23, 9)$ & 7 & 1 & YES & YES & YES & NO & 1118\\
$(536, 209)$ & 14 & $(3, 1)$ & 2 & 1 & YES & YES & YES & NO & 1119\\
$(542, 201)$ & 14 & $(11, 4)$ & 5 & 1 & YES & YES & YES & NO & 1120\\
$(571, 223)$ & 14 & $(18, 7)$ & 6 & 1 & YES & YES & YES & NO & 1121\\
$(571, 223)$ & 14 & $(105, 41)$ & 10 & 1 & YES & YES & YES & NO & 1122\\
$(577, 225)$ & 14 & $(3, 1)$ & 2 & 1 & YES & YES & YES & NO & 1123\\
$(602, 255)$ & 14 & $(5, 1)$ & 4 & 1 & YES & YES & YES & NO & 1124\\
$(613, 181)$ & 14 & $(7, 3)$ & 4 & 1 & YES & YES & YES & NO & 1125\\
$(621, 227)$ & 14 & $(5, 1)$ & 4 & 1 & YES & YES & YES & NO & 1126\\
$(623, 116)$ & 16 & $(5, 2)$ & 3 & 1 & YES & YES & YES & NO & 1127\\
$(629, 147)$ & 15 & $(7, 3)$ & 4 & 1 & YES & YES & YES & NO & 1128\\
$(641, 177)$ & 15 & $(5, 1)$ & 4 & 1 & YES & YES & YES & NO & 1129\\
$(641, 177)$ & 15 & $(105, 29)$ & 10 & 1 & YES & YES & YES & NO & 1130\\
$(665, 258)$ & 14 & $(8, 3)$ & 4 & 1 & YES & YES & YES & NO & 1131\\
$(680, 263)$ & 14 & $(75, 29)$ & 9 & 5 & YES & NO & YES & 1102 & 1132\\
$(680, 263)$ & 14 & $(287, 111)$ & 12 & 1 & YES & NO & YES & NO & 1133\\
$(693, 268)$ & 14 & $(13, 5)$ & 5 & 1 & YES & NO & YES & NO & 1134\\
$(703, 274)$ & 15 & $(449, 175)$ & 14 & 1 & YES & YES & YES & NO & 1135\\
$(705, 268)$ & 14 & $(5, 2)$ & 3 & 5 & YES & NO & YES & NO & 1136\\
$(708, 215)$ & 15 & $(3, 1)$ & 2 & 3 & YES & YES & YES & NO & 1137\\
$(724, 219)$ & 15 & $(400, 121)$ & 14 & 4 & YES & YES & YES & NO & 1138\\
$(752, 287)$ & 14 & $(2, 1)$ & 1 & 2 & YES & YES & YES & NO & 1139\\
$(752, 309)$ & 15 & $(4, 1)$ & 3 & 4 & YES & YES & YES & NO & 1140\\
$(752, 309)$ & 15 & $(56, 23)$ & 9 & 8 & YES & YES & YES & NO & 1141\\
$(793, 242)$ & 15 & $(7, 2)$ & 4 & 1 & YES & YES & YES & NO & 1142\\
$(807, 235)$ & 16 & $(5, 1)$ & 4 & 1 & YES & YES & YES & NO & 1143\\
$(807, 235)$ & 16 & $(625, 182)$ & 15 & 1 & YES & YES & YES & NO & 1144\\
$(822, 227)$ & 15 & $(2, 1)$ & 1 & 2 & YES & YES & YES & NO & 1145\\
$(827, 304)$ & 15 & $(49, 18)$ & 8 & 1 & YES & YES & YES & 1092 & 1146\\
$(853, 313)$ & 15 & $(2, 1)$ & 1 & 1 & YES & YES & YES & NO & 1147\\
$(853, 313)$ & 15 & $(19, 7)$ & 6 & 1 & YES & YES & YES & NO & 1148\\
$(875, 363)$ & 15 & $(5, 2)$ & 3 & 5 & YES & YES & YES & NO & 1149\\
$(917, 337)$ & 15 & $(5, 2)$ & 3 & 1 & YES & YES & YES & NO & 1150\\
$(935, 283)$ & 16 & $(337, 102)$ & 14 & 1 & YES & YES & YES & NO & 1151\\
$(997, 184)$ & 17 & $(38, 7)$ & 9 & 1 & YES & YES & YES & NO & 1152\\
$(a; 1, 0, 0; 13)$ & 5 & $(131, 55)$ & 10 & 1 & YES & YES & YES & NO & 1153\\
$(a; 2, 1, 0; 5)$ & 7 & $(102, 31)$ & 11 & 1 & YES & YES & YES & NO & 1154\\
$(a; 2, 1, 1; 37)$ & 8 & $(31, 13)$ & 7 & 1 & YES & YES & YES & NO & 1155\\
$(d; 0, 1, 1; 17)$ & 7 & $(127, 29)$ & 11 & 1 & YES & YES & YES & NO & 1156\\
$(e; 1, 0, 0; 18)$ & 6 & $(95, 39)$ & 10 & 1 & YES & YES & YES & NO & 1157\\
$(e; 1, 1, 0; 23)$ & 7 & $(89, 27)$ & 10 & 1 & YES & YES & YES & NO & 1158\\
$(f; 0, 3, 0; 9)$ & 7 & $(71, 27)$ & 9 & 1 & YES & YES & YES & NO & 1159\\
$(g; 0, 0, 1; 26)$ & 7 & $(61, 17)$ & 9 & 1 & YES & YES & YES & NO & 1160\\
$(g; 0, 1, 0; 24)$ & 7 & $(61, 17)$ & 9 & 1 & YES & YES & YES & NO & 1161\\
$(g; 2, 1, 0; 48)$ & 9 & $(18, 7)$ & 6 & 6 & YES & YES & YES & NO & 1162\\
$(i; 0, 0, 0; 9)$ & 5 & $(207, 56)$ & 12 & 9 & YES & YES & YES & NO & 1163\\
$(j; 0, 0, 0; 8)$ & 5 & $(155, 64)$ & 11 & 1 & YES & YES & YES & NO & 1164\\
$(j; 0, 0, 0; 8)$ & 5 & $(167, 69)$ & 11 & 1 & YES & YES & YES & NO & 1165
\end{longtable}


%%%%%%%%%%%%%%%%%%%%%%%%%%%%%%%%%%%%%%%%%%%
\section{$2I_0^*$}
Input:
\lstinputlisting[language=config]{../Tests/0s0s.txt}
Result:
\input{../summary/0s0s.tex}


%%%%%%%%%%%%%%%%%%%%%%%%%%%%%%%%%%%%%%%%%%%
\section{Extra: $3IV$}

Dual Hesse configuration

$R_{16}$ is a section through $S_1$ and $S_6$. Same for $R_{24}$ and $R_{35}$. These three sections are concurrent.

$Q_1$, $Q_2$ and $Q_3$ are conics through 5 special points. Each pair of them share 4 of those points.


Input:
\lstinputlisting[language=config]{../Tests/3IV.txt}
Result:
%\usepackage{longtable}
\subsection{2 chains, $K^2 = 5$}
\begin{longtable}{|c|c|c|c|c|c|c|c|c|c|}
\hline
\multicolumn{10}{|c|}{2 chains, $K^2 = 5$}\\
\hline
$(n,a)$ & Length & $(n,a)$ & Length & GCD & Nef & $\mathbb Q$-ef & Obstruction 0 & WH & Index\\
\hline
\endfirsthead

\hline
$(n,a)$ & Length & $(n,a)$ & Length & GCD & Nef & $\mathbb Q$-ef & Obstruction 0 & WH & Index\\
\hline
\endhead
\hline
\endfoot

$(1376, 379)$ & 16 & $(7, 3)$ & 4 & 1 & YES & YES & NO(3) & -- & 1\\
$(1702, 705)$ & 16 & $(886, 367)$ & 15 & 2 & YES & YES & NO(3) & NO & 2
\end{longtable}



%%%%%%%%%%%%%%%%%%%%%%%%%%%%%%%%%%%%%%%%%%%
\section{Extra: $I_6 + 2I_2 + 2I_1$}

Input:
\lstinputlisting[language=config]{../Tests/62211.txt}
Result:
%\usepackage{longtable}
\subsection{1 chain, $K^2 = 1$}
\begin{longtable}{|c|c|c|c|c|c|c|c|}
\hline
\multicolumn{8}{|c|}{1 chain, $K^2 = 1$}\\
\hline
$(n,a)$ & Len & Nef & $\mathbb Q$-ef & Obs 0 & $\overline c_1^2 / \overline c_2$ & $(P,K)$ & Index\\
\hline
\endfirsthead

\hline
$(n,a)$ & Len & Nef & $\mathbb Q$-ef & Obs 0 & $\overline c_1^2 / \overline c_2$ & $(P,K)$ & Index\\
\hline
\endhead
\hline
\endfoot

$(16,7)$ & 6 & YES & YES & NO(2) & $0.40$ & $(5,-1)$ & 1\\
$(17,5)$ & 6 & YES & YES & YES & $0.60$ & $(1,1)$ & 2\\
$(19,8)$ & 6 & YES & YES & YES & $0.60$ & $(3,0)$ & 3\\
$(21,5)$ & 8 & YES & YES & YES & $0.40$ & $(1,1)$ & 4\\
$(21,8)$ & 6 & YES & YES & YES & $0.70$ & $(1,1)$ & 5\\
$(24,5)$ & 8 & YES & YES & YES & $0.50$ & $(1,1)$ & 6\\
$(24,7)$ & 7 & YES & YES & YES & $0.60$ & $(1,1)$ & 7\\
$(25,7)$ & 7 & YES & YES & YES & $0.40$ & $(1,1)$ & 8\\
$(30,7)$ & 8 & YES & YES & NO(2) & $0.20$ & $(5,-1)$ & 9\\
$(31,7)$ & 8 & YES & YES & YES & $0.50$ & $(1,1)$ & 10\\
$(32,7)$ & 8 & YES & YES & YES & $0.70$ & $(1,1)$ & 11\\
$(35,8)$ & 8 & YES & YES & YES & $0.70$ & $(1,1)$ & 12\\
$(b;0,0,0;14)$ & 5 & YES & YES & YES & $0.60$ & $(1,1)$ & 13\\
$(c;0,1,1;5)$ & 6 & YES & YES & YES & $0.50$ & $(1,1)$ & 14\\
$(d;0,0,0;5)$ & 5 & YES & YES & YES & $0.50$ & $(1,1)$ & 15\\
$(d;0,0,1;14)$ & 6 & YES & YES & YES & $0.60$ & $(1,1)$ & 16\\
$(e;0,0,0;4)$ & 5 & YES & YES & YES & $0.70$ & $(1,1)$ & 17\\
$(j;0,1,0;10)$ & 6 & YES & YES & NO(2) & $0.20$ & $(5,-1)$ & 18
\end{longtable}
\subsection{1 chain, $K^2 = 2$}
\begin{longtable}{|c|c|c|c|c|c|c|c|}
\hline
\multicolumn{8}{|c|}{1 chain, $K^2 = 2$}\\
\hline
$(n,a)$ & Len & Nef & $\mathbb Q$-ef & Obs 0 & $\overline c_1^2 / \overline c_2$ & $(P,K)$ & Index\\
\hline
\endfirsthead

\hline
$(n,a)$ & Len & Nef & $\mathbb Q$-ef & Obs 0 & $\overline c_1^2 / \overline c_2$ & $(P,K)$ & Index\\
\hline
\endhead
\hline
\endfoot

$(27,5)$ & 8 & YES & YES & NO(2) & $1.00$ & $(1,2)$ & 19\\
$(28,5)$ & 8 & YES & YES & YES & $0.78$ & $(1,2)$ & 20\\
$(34,9)$ & 8 & YES & YES & YES & $0.89$ & $(1,2)$ & 21\\
$(37,17)$ & 9 & YES & YES & YES & $0.89$ & $(1,2)$ & 22\\
$(37,8)$ & 8 & YES & YES & YES & $1.00$ & $(1,2)$ & 23\\
$(39,17)$ & 8 & YES & YES & YES & $0.89$ & $(1,2)$ & 24\\
$(39,16)$ & 8 & YES & YES & YES & $0.89$ & $(1,2)$ & 25\\
$(39,14)$ & 8 & YES & YES & NO(2) & $1.09$ & $(1,2)$ & 26\\
$(41,13)$ & 10 & YES & YES & YES & $0.89$ & $(1,2)$ & 27\\
$(41,16)$ & 8 & YES & YES & YES & $0.89$ & $(1,2)$ & 28\\
$(41,11)$ & 8 & YES & YES & YES & $1.00$ & $(1,2)$ & 29\\
$(42,13)$ & 9 & YES & YES & YES & $1.00$ & $(1,2)$ & 30\\
$(43,19)$ & 9 & YES & YES & NO(2) & $1.00$ & $(3,1)$ & 31\\
$(44,17)$ & 8 & YES & YES & YES & $1.00$ & $(1,2)$ & 32\\
$(45,19)$ & 8 & YES & YES & NO(2) & $1.00$ & $(1,2)$ & 33\\
$(49,20)$ & 9 & YES & YES & YES & $0.89$ & $(1,2)$ & 34\\
$(49,22)$ & 9 & YES & YES & NO(2) & $1.00$ & $(1,2)$ & 35\\
$(49,19)$ & 8 & YES & YES & YES & $1.00$ & $(1,2)$ & 36\\
$(49,18)$ & 8 & YES & YES & YES & $0.89$ & $(1,2)$ & 37\\
$(49,15)$ & 9 & YES & YES & YES & $0.89$ & $(3,1)$ & 38\\
$(49,13)$ & 9 & YES & YES & YES & $0.89$ & $(1,2)$ & 39\\
$(50,19)$ & 8 & YES & YES & YES & $0.78$ & $(3,1)$ & 40\\
$(51,20)$ & 9 & YES & YES & YES & $0.78$ & $(3,1)$ & 41\\
$(51,23)$ & 9 & YES & YES & YES & $1.00$ & $(1,2)$ & 42\\
$(51,16)$ & 10 & YES & YES & YES & $1.00$ & $(1,2)$ & 43\\
$(52,19)$ & 9 & YES & YES & NO(2) & $0.90$ & $(3,1)$ & 44\\
$(53,19)$ & 9 & YES & YES & YES & $0.78$ & $(3,1)$ & 45\\
$(54,19)$ & 10 & YES & YES & YES & $0.89$ & $(1,2)$ & 46\\
$(57,25)$ & 9 & YES & YES & YES & $1.00$ & $(1,2)$ & 47\\
$(58,17)$ & 9 & YES & YES & YES & $0.89$ & $(1,2)$ & 48\\
$(59,21)$ & 10 & YES & YES & YES & $0.78$ & $(3,1)$ & 49\\
$(59,23)$ & 9 & YES & YES & YES & $0.78$ & $(1,2)$ & 50\\
$(59,26)$ & 9 & YES & YES & YES & $0.78$ & $(3,1)$ & 51\\
$(59,27)$ & 10 & YES & YES & YES & $1.00$ & $(1,2)$ & 52\\
$(59,18)$ & 9 & YES & YES & YES & $1.00$ & $(1,2)$ & 53\\
$(62,19)$ & 10 & YES & YES & YES & $1.10$ & $(1,2)$ & 54\\
$(62,23)$ & 9 & YES & YES & YES & $0.89$ & $(1,2)$ & 55\\
$(62,27)$ & 9 & YES & YES & YES & $1.11$ & $(1,2)$ & 56\\
$(63,26)$ & 9 & YES & YES & YES & $1.00$ & $(1,2)$ & 57\\
$(64,23)$ & 9 & YES & YES & YES & $0.89$ & $(1,2)$ & 58\\
$(64,27)$ & 9 & YES & YES & YES & $0.78$ & $(3,1)$ & 59\\
$(65,17)$ & 10 & YES & YES & YES & $0.89$ & $(1,2)$ & 60\\
$(65,24)$ & 9 & YES & YES & NO(2) & $0.80$ & $(3,1)$ & 61\\
$(66,29)$ & 9 & YES & YES & YES & $1.11$ & $(1,2)$ & 62\\
$(69,19)$ & 9 & YES & YES & YES & $0.90$ & $(1,2)$ & 63\\
$(69,20)$ & 10 & YES & YES & YES & $1.11$ & $(1,2)$ & 64\\
$(71,15)$ & 10 & YES & YES & YES & $0.78$ & $(1,2)$ & 65\\
$(71,20)$ & 10 & YES & YES & YES & $1.11$ & $(1,2)$ & 66\\
$(71,22)$ & 10 & YES & YES & YES & $1.00$ & $(1,2)$ & 67\\
$(71,26)$ & 9 & YES & YES & YES & $1.00$ & $(1,2)$ & 68\\
$(71,31)$ & 10 & YES & YES & YES & $0.89$ & $(3,1)$ & 69\\
$(71,32)$ & 10 & YES & YES & YES & $1.00$ & $(1,2)$ & 70\\
$(73,27)$ & 9 & YES & YES & YES & $1.00$ & $(1,2)$ & 71\\
$(76,13)$ & 12 & YES & YES & NO(2) & $1.00$ & $(1,2)$ & 72\\
$(76,27)$ & 10 & YES & YES & YES & $0.89$ & $(1,2)$ & 73\\
$(78,35)$ & 10 & YES & YES & YES & $1.00$ & $(1,2)$ & 74\\
$(79,17)$ & 11 & YES & YES & YES & $0.78$ & $(3,1)$ & 75\\
$(79,24)$ & 10 & YES & YES & YES & $0.90$ & $(1,2)$ & 76\\
$(79,28)$ & 10 & YES & YES & YES & $1.00$ & $(1,2)$ & 77\\
$(79,29)$ & 9 & YES & YES & YES & $1.00$ & $(1,2)$ & 78\\
$(81,19)$ & 11 & YES & YES & YES & $0.90$ & $(1,2)$ & 79\\
$(82,17)$ & 11 & YES & YES & YES & $0.89$ & $(1,2)$ & 80\\
$(82,19)$ & 12 & YES & YES & YES & $0.89$ & $(3,1)$ & 81\\
$(82,37)$ & 10 & YES & YES & YES & $1.00$ & $(1,2)$ & 82\\
$(83,22)$ & 10 & YES & YES & YES & $0.78$ & $(3,1)$ & 83\\
$(83,24)$ & 11 & YES & YES & YES & $1.00$ & $(1,2)$ & 84\\
$(85,24)$ & 11 & YES & YES & YES & $1.11$ & $(1,2)$ & 85\\
$(88,21)$ & 12 & YES & YES & YES & $0.89$ & $(3,1)$ & 86\\
$(89,17)$ & 12 & YES & YES & YES & $0.78$ & $(1,2)$ & 87\\
$(89,20)$ & 11 & YES & YES & YES & $1.11$ & $(1,2)$ & 88\\
$(89,27)$ & 10 & YES & YES & YES & $1.11$ & $(1,2)$ & 89\\
$(90,19)$ & 11 & YES & YES & YES & $1.00$ & $(1,2)$ & 90\\
$(91,17)$ & 12 & YES & YES & YES & $0.67$ & $(3,1)$ & 91\\
$(92,19)$ & 12 & YES & YES & YES & $0.89$ & $(1,2)$ & 92\\
$(94,41)$ & 10 & YES & YES & YES & $1.00$ & $(1,2)$ & 93\\
$(97,26)$ & 10 & YES & YES & YES & $0.67$ & $(3,1)$ & 94\\
$(99,17)$ & 12 & YES & YES & NO(2) & $1.00$ & $(1,2)$ & 95\\
$(101,44)$ & 10 & YES & YES & YES & $0.89$ & $(1,2)$ & 96\\
$(103,27)$ & 11 & YES & YES & YES & $1.11$ & $(1,2)$ & 97\\
$(103,37)$ & 10 & YES & YES & YES & $1.00$ & $(1,2)$ & 98\\
$(107,24)$ & 12 & YES & YES & YES & $1.11$ & $(1,2)$ & 99\\
$(111,26)$ & 11 & YES & YES & YES & $1.00$ & $(1,2)$ & 100\\
$(113,24)$ & 11 & YES & YES & YES & $0.90$ & $(1,2)$ & 101\\
$(113,35)$ & 11 & YES & YES & YES & $0.89$ & $(1,2)$ & 102\\
$(120,19)$ & 14 & YES & YES & YES & $0.78$ & $(3,1)$ & 103\\
$(131,24)$ & 13 & YES & YES & YES & $1.00$ & $(1,2)$ & 104\\
$(a;3,0,1;31)$ & 8 & YES & YES & YES & $0.89$ & $(1,2)$ & 105\\
$(b;0,0,2;26)$ & 7 & YES & YES & YES & $0.80$ & $(1,2)$ & 106\\
$(b;0,0,3;32)$ & 8 & YES & YES & YES & $1.00$ & $(1,2)$ & 107\\
$(b;0,1,1;27)$ & 7 & YES & YES & NO(3) & $0.67$ & $(1,2)$ & 108\\
$(b;1,1,0;27)$ & 7 & YES & YES & NO(2) & $1.00$ & $(1,2)$ & 109\\
$(b;3,0,0;16)$ & 8 & YES & YES & YES & $0.89$ & $(1,2)$ & 110\\
$(c;0,2,2;6)$ & 8 & YES & YES & YES & $0.78$ & $(1,2)$ & 111\\
$(c;0,3,2;29)$ & 9 & YES & YES & YES & $0.78$ & $(1,2)$ & 112\\
$(d;0,1,2;11)$ & 8 & YES & YES & YES & $0.78$ & $(1,2)$ & 113\\
$(d;0,2,2;13)$ & 9 & YES & YES & YES & $0.78$ & $(1,2)$ & 114\\
$(e;3,0,0;10)$ & 8 & YES & YES & YES & $1.00$ & $(1,2)$ & 115\\
$(h;0,1,0;8)$ & 6 & YES & YES & NO(2) & $0.82$ & $(1,2)$ & 116
\end{longtable}
\subsection{1 chain, $K^2 = 3$}
\begin{longtable}{|c|c|c|c|c|c|c|c|}
\hline
\multicolumn{8}{|c|}{1 chain, $K^2 = 3$}\\
\hline
$(n,a)$ & Len & Nef & $\mathbb Q$-ef & Obs 0 & $\overline c_1^2 / \overline c_2$ & $(P,K)$ & Index\\
\hline
\endfirsthead

\hline
$(n,a)$ & Len & Nef & $\mathbb Q$-ef & Obs 0 & $\overline c_1^2 / \overline c_2$ & $(P,K)$ & Index\\
\hline
\endhead
\hline
\endfoot

$(75,17)$ & 10 & YES & YES & NO(2) & $1.11$ & $(3,2)$ & 117\\
$(89,25)$ & 10 & YES & YES & YES & $1.25$ & $(1,3)$ & 118\\
$(89,24)$ & 10 & YES & YES & NO(2) & $1.11$ & $(3,2)$ & 119\\
$(97,26)$ & 10 & YES & YES & YES & $1.12$ & $(1,3)$ & 120\\
$(98,27)$ & 10 & YES & YES & NO(2) & $1.11$ & $(3,2)$ & 121\\
$(100,31)$ & 11 & YES & YES & NO(2) & $1.40$ & $(1,3)$ & 122\\
$(101,39)$ & 10 & YES & YES & NO(2) & $1.40$ & $(1,3)$ & 123\\
$(103,29)$ & 11 & YES & YES & YES & $1.25$ & $(1,3)$ & 124\\
$(107,41)$ & 10 & YES & YES & NO(2) & $1.40$ & $(1,3)$ & 125\\
$(113,51)$ & 11 & YES & YES & YES & $1.25$ & $(1,3)$ & 126\\
$(113,48)$ & 11 & YES & YES & YES & $1.38$ & $(1,3)$ & 127\\
$(115,52)$ & 11 & YES & YES & YES & $1.25$ & $(1,3)$ & 128\\
$(119,45)$ & 11 & YES & YES & NO(2) & $1.40$ & $(1,3)$ & 129\\
$(125,49)$ & 11 & YES & YES & YES & $1.25$ & $(1,3)$ & 130\\
$(125,53)$ & 11 & YES & YES & NO(2) & $1.40$ & $(1,3)$ & 131\\
$(127,54)$ & 12 & YES & YES & YES & $1.38$ & $(1,3)$ & 132\\
$(128,37)$ & 12 & YES & YES & YES & $1.25$ & $(1,3)$ & 133\\
$(128,49)$ & 10 & YES & YES & YES & $1.33$ & $(1,3)$ & 134\\
$(129,53)$ & 11 & YES & YES & NO(2) & $1.40$ & $(1,3)$ & 135\\
$(131,50)$ & 10 & YES & YES & YES & $1.25$ & $(1,3)$ & 136\\
$(131,36)$ & 11 & YES & YES & YES & $1.25$ & $(1,3)$ & 137\\
$(134,41)$ & 11 & YES & YES & NO(2) & $1.30$ & $(1,3)$ & 138\\
$(137,52)$ & 11 & YES & YES & NO(2) & $1.22$ & $(3,2)$ & 139\\
$(139,61)$ & 11 & YES & YES & YES & $1.12$ & $(3,2)$ & 140\\
$(140,53)$ & 11 & YES & YES & NO(2) & $1.40$ & $(1,3)$ & 141\\
$(144,59)$ & 11 & YES & YES & NO(2) & $1.30$ & $(1,3)$ & 142\\
$(144,61)$ & 11 & YES & YES & YES & $1.25$ & $(1,3)$ & 143\\
$(145,42)$ & 12 & YES & YES & NO(2) & $1.22$ & $(3,2)$ & 144\\
$(148,65)$ & 11 & YES & YES & YES & $1.38$ & $(1,3)$ & 145\\
$(151,53)$ & 12 & YES & YES & YES & $1.25$ & $(1,3)$ & 146\\
$(152,67)$ & 11 & YES & YES & YES & $1.25$ & $(1,3)$ & 147\\
$(153,40)$ & 12 & YES & YES & YES & $1.25$ & $(1,3)$ & 148\\
$(153,41)$ & 11 & YES & YES & NO(2) & $1.30$ & $(1,3)$ & 149\\
$(153,43)$ & 12 & YES & YES & NO(2) & $1.40$ & $(1,3)$ & 150\\
$(153,70)$ & 12 & YES & YES & YES & $1.38$ & $(1,3)$ & 151\\
$(154,65)$ & 11 & YES & YES & YES & $1.12$ & $(3,2)$ & 152\\
$(155,57)$ & 11 & YES & YES & NO(2) & $1.40$ & $(1,3)$ & 153\\
$(157,42)$ & 12 & YES & YES & YES & $1.12$ & $(1,3)$ & 154\\
$(159,62)$ & 11 & YES & YES & YES & $1.25$ & $(1,3)$ & 155\\
$(159,73)$ & 12 & YES & YES & YES & $1.38$ & $(1,3)$ & 156\\
$(159,59)$ & 11 & YES & YES & YES & $1.38$ & $(1,3)$ & 157\\
$(161,66)$ & 11 & YES & YES & NO(2) & $1.30$ & $(1,3)$ & 158\\
$(164,51)$ & 12 & YES & YES & YES & $1.38$ & $(1,3)$ & 159\\
$(164,71)$ & 12 & YES & YES & YES & $1.38$ & $(1,3)$ & 160\\
$(169,64)$ & 11 & YES & YES & NO(2) & $1.30$ & $(1,3)$ & 161\\
$(171,65)$ & 11 & YES & YES & NO(2) & $1.11$ & $(3,2)$ & 162\\
$(171,71)$ & 12 & YES & YES & YES & $1.25$ & $(3,2)$ & 163\\
$(173,76)$ & 11 & YES & YES & YES & $1.12$ & $(1,3)$ & 164\\
$(175,62)$ & 12 & YES & YES & YES & $1.38$ & $(1,3)$ & 165\\
$(176,51)$ & 12 & YES & YES & YES & $1.25$ & $(1,3)$ & 166\\
$(177,46)$ & 13 & YES & YES & YES & $1.38$ & $(1,3)$ & 167\\
$(178,47)$ & 12 & YES & YES & NO(2) & $1.11$ & $(3,2)$ & 168\\
$(178,69)$ & 11 & YES & YES & YES & $1.33$ & $(1,3)$ & 169\\
$(179,42)$ & 13 & YES & YES & YES & $1.25$ & $(1,3)$ & 170\\
$(179,76)$ & 12 & YES & YES & YES & $1.38$ & $(1,3)$ & 171\\
$(181,65)$ & 12 & YES & YES & YES & $1.38$ & $(1,3)$ & 172\\
$(181,79)$ & 12 & YES & YES & YES & $1.44$ & $(1,3)$ & 173\\
$(182,53)$ & 12 & YES & YES & NO(2) & $1.40$ & $(1,3)$ & 174\\
$(182,79)$ & 12 & YES & YES & YES & $1.33$ & $(1,3)$ & 175\\
$(183,82)$ & 14 & YES & YES & YES & $1.50$ & $(3,2)$ & 176\\
$(186,79)$ & 12 & YES & YES & YES & $1.25$ & $(1,3)$ & 177\\
$(188,85)$ & 12 & YES & YES & YES & $1.25$ & $(1,3)$ & 178\\
$(189,40)$ & 12 & YES & YES & NO(2) & $1.30$ & $(1,3)$ & 179\\
$(192,73)$ & 11 & YES & YES & YES & $1.33$ & $(1,3)$ & 180\\
$(193,74)$ & 12 & YES & YES & YES & $1.33$ & $(1,3)$ & 181\\
$(194,71)$ & 12 & YES & YES & YES & $1.38$ & $(1,3)$ & 182\\
$(199,78)$ & 12 & YES & YES & YES & $1.12$ & $(3,2)$ & 183\\
$(201,76)$ & 12 & YES & YES & YES & $1.38$ & $(1,3)$ & 184\\
$(201,82)$ & 12 & YES & YES & YES & $1.38$ & $(1,3)$ & 185\\
$(202,73)$ & 12 & YES & YES & YES & $1.33$ & $(1,3)$ & 186\\
$(203,59)$ & 12 & YES & YES & YES & $1.12$ & $(1,3)$ & 187\\
$(206,73)$ & 12 & YES & YES & YES & $1.38$ & $(1,3)$ & 188\\
$(207,80)$ & 12 & YES & YES & YES & $1.38$ & $(1,3)$ & 189\\
$(208,55)$ & 12 & YES & YES & YES & $1.25$ & $(1,3)$ & 190\\
$(211,89)$ & 12 & YES & YES & YES & $1.56$ & $(1,3)$ & 191\\
$(211,93)$ & 12 & YES & YES & YES & $1.12$ & $(3,2)$ & 192\\
$(212,93)$ & 12 & YES & YES & YES & $1.25$ & $(1,3)$ & 193\\
$(213,89)$ & 13 & YES & YES & YES & $1.50$ & $(1,3)$ & 194\\
$(215,97)$ & 12 & YES & YES & YES & $1.25$ & $(1,3)$ & 195\\
$(219,83)$ & 12 & YES & YES & YES & $1.44$ & $(1,3)$ & 196\\
$(223,82)$ & 12 & YES & YES & YES & $1.25$ & $(1,3)$ & 197\\
$(223,92)$ & 12 & YES & YES & YES & $1.50$ & $(1,3)$ & 198\\
$(227,93)$ & 12 & YES & YES & YES & $1.38$ & $(1,3)$ & 199\\
$(227,94)$ & 12 & YES & YES & YES & $1.44$ & $(1,3)$ & 200\\
$(228,61)$ & 13 & YES & YES & YES & $1.38$ & $(1,3)$ & 201\\
$(229,97)$ & 12 & YES & YES & YES & $1.25$ & $(1,3)$ & 202\\
$(230,67)$ & 13 & YES & YES & YES & $1.33$ & $(1,3)$ & 203\\
$(233,61)$ & 14 & YES & YES & YES & $1.38$ & $(1,3)$ & 204\\
$(233,73)$ & 14 & YES & YES & YES & $1.50$ & $(1,3)$ & 205\\
$(234,89)$ & 12 & YES & YES & YES & $1.56$ & $(1,3)$ & 206\\
$(237,100)$ & 12 & YES & YES & YES & $1.38$ & $(1,3)$ & 207\\
$(239,99)$ & 12 & YES & YES & YES & $1.67$ & $(1,3)$ & 208\\
$(240,89)$ & 12 & YES & YES & YES & $1.44$ & $(1,3)$ & 209\\
$(241,92)$ & 12 & YES & YES & YES & $1.50$ & $(1,3)$ & 210\\
$(243,38)$ & 16 & YES & YES & YES & $1.38$ & $(1,3)$ & 211\\
$(249,73)$ & 13 & YES & YES & YES & $1.56$ & $(1,3)$ & 212\\
$(249,95)$ & 12 & YES & YES & YES & $1.38$ & $(1,3)$ & 213\\
$(251,74)$ & 13 & YES & YES & YES & $1.38$ & $(1,3)$ & 214\\
$(252,47)$ & 14 & YES & YES & NO(2) & $1.30$ & $(1,3)$ & 215\\
$(253,57)$ & 13 & YES & YES & YES & $1.22$ & $(1,3)$ & 216\\
$(255,61)$ & 15 & YES & YES & YES & $1.62$ & $(1,3)$ & 217\\
$(259,59)$ & 13 & YES & YES & YES & $1.22$ & $(1,3)$ & 218\\
$(261,76)$ & 13 & YES & YES & YES & $1.25$ & $(1,3)$ & 219\\
$(261,100)$ & 12 & YES & YES & YES & $1.25$ & $(1,3)$ & 220\\
$(261,107)$ & 12 & YES & YES & YES & $1.25$ & $(1,3)$ & 221\\
$(262,73)$ & 13 & YES & YES & YES & $1.56$ & $(1,3)$ & 222\\
$(263,78)$ & 13 & YES & YES & YES & $1.38$ & $(1,3)$ & 223\\
$(269,78)$ & 13 & YES & YES & YES & $1.25$ & $(1,3)$ & 224\\
$(272,103)$ & 12 & YES & YES & YES & $1.25$ & $(1,3)$ & 225\\
$(274,115)$ & 12 & YES & YES & YES & $1.25$ & $(1,3)$ & 226\\
$(278,75)$ & 13 & YES & YES & YES & $1.56$ & $(1,3)$ & 227\\
$(282,59)$ & 15 & YES & YES & YES & $1.50$ & $(1,3)$ & 228\\
$(283,64)$ & 13 & YES & YES & YES & $1.22$ & $(1,3)$ & 229\\
$(289,112)$ & 12 & YES & YES & YES & $1.50$ & $(1,3)$ & 230\\
$(291,85)$ & 13 & YES & YES & YES & $1.50$ & $(1,3)$ & 231\\
$(292,79)$ & 13 & YES & YES & YES & $1.56$ & $(1,3)$ & 232\\
$(293,123)$ & 12 & YES & YES & YES & $1.25$ & $(1,3)$ & 233\\
$(295,87)$ & 13 & YES & YES & YES & $1.38$ & $(1,3)$ & 234\\
$(301,115)$ & 12 & YES & YES & YES & $1.56$ & $(1,3)$ & 235\\
$(306,73)$ & 15 & YES & YES & YES & $1.50$ & $(1,3)$ & 236\\
$(313,86)$ & 13 & YES & YES & YES & $1.38$ & $(1,3)$ & 237\\
$(322,73)$ & 14 & YES & YES & YES & $1.56$ & $(1,3)$ & 238\\
$(327,100)$ & 13 & YES & YES & YES & $1.25$ & $(1,3)$ & 239\\
$(329,61)$ & 15 & YES & YES & YES & $1.50$ & $(1,3)$ & 240\\
$(367,59)$ & 17 & YES & YES & YES & $1.50$ & $(1,3)$ & 241\\
$(372,89)$ & 15 & YES & YES & YES & $1.50$ & $(1,3)$ & 242\\
$(432,77)$ & 15 & YES & YES & YES & $1.25$ & $(1,3)$ & 243\\
$(452,79)$ & 15 & YES & YES & YES & $1.50$ & $(1,3)$ & 244\\
$(b;3,0,4;46)$ & 12 & YES & YES & YES & $1.38$ & $(1,3)$ & 245\\
$(g;2,0,1;10)$ & 9 & YES & YES & NO(2) & $1.20$ & $(1,3)$ & 246
\end{longtable}
\subsection{1 chain, $K^2 = 4$}
\begin{longtable}{|c|c|c|c|c|c|c|c|}
\hline
\multicolumn{8}{|c|}{1 chain, $K^2 = 4$}\\
\hline
$(n,a)$ & Len & Nef & $\mathbb Q$-ef & Obs 0 & $\overline c_1^2 / \overline c_2$ & $(P,K)$ & Index\\
\hline
\endfirsthead

\hline
$(n,a)$ & Len & Nef & $\mathbb Q$-ef & Obs 0 & $\overline c_1^2 / \overline c_2$ & $(P,K)$ & Index\\
\hline
\endhead
\hline
\endfoot

$(373,154)$ & 13 & YES & YES & NO(3) & $1.50$ & $(3,3)$ & 247\\
$(457,192)$ & 13 & YES & YES & YES & $1.62$ & $(1,4)$ & 248\\
$(473,196)$ & 13 & YES & YES & YES & $1.62$ & $(1,4)$ & 249\\
$(611,256)$ & 14 & YES & YES & NO(3) & $1.71$ & $(1,4)$ & 250\\
$(747,169)$ & 15 & YES & YES & NO(3) & $1.62$ & $(1,4)$ & 251\\
$(923,259)$ & 16 & YES & YES & YES & $1.86$ & $(1,4)$ & 252
\end{longtable}
\subsection{2 chains, $K^2 = 1$}
\begin{longtable}{|c|c|c|c|c|c|c|c|c|c|c|c|}
\hline
\multicolumn{12}{|c|}{2 chains, $K^2 = 1$}\\
\hline
$(n,a)$ & Len & $(n,a)$ & Len & GCD & Nef & $\mathbb Q$-ef & Obs 0 & $\overline c_1^2 / \overline c_2$ & $(P,K)$ & WH & Index\\
\hline
\endfirsthead

\hline
$(n,a)$ & Len & $(n,a)$ & Len & GCD & Nef & $\mathbb Q$-ef & Obs 0 & $\overline c_1^2 / \overline c_2$ & $(P,K)$ & WH & Index\\
\hline
\endhead
\hline
\endfoot

$(7,2)$ & 4 & $(7,2)$ & 4 & 7 & YES & YES & YES & $0.56$ & $(2,1)$ & NO & 253\\
$(7,2)$ & 4 & $(7,2)$ & 4 & 7 & YES & YES & YES & $0.56$ & $(2,1)$ & -- & 254\\
$(7,3)$ & 4 & $(7,2)$ & 4 & 7 & YES & YES & YES & $0.67$ & $(2,1)$ & NO & 255\\
$(7,3)$ & 4 & $(7,2)$ & 4 & 7 & YES & YES & YES & $0.67$ & $(2,1)$ & -- & 256\\
$(7,3)$ & 4 & $(7,2)$ & 4 & 7 & YES & YES & YES & $0.56$ & $(4,0)$ & NO & 257\\
$(7,3)$ & 4 & $(7,3)$ & 4 & 7 & YES & YES & YES & $0.44$ & $(2,1)$ & NO & 258\\
$(8,3)$ & 4 & $(4,1)$ & 3 & 4 & YES & YES & YES & $0.56$ & $(2,1)$ & -- & 259\\
$(8,3)$ & 4 & $(4,1)$ & 3 & 4 & YES & YES & YES & $0.67$ & $(2,1)$ & NO & 260\\
$(8,3)$ & 4 & $(5,1)$ & 4 & 1 & YES & YES & YES & $0.56$ & $(2,1)$ & NO & 261\\
$(8,3)$ & 4 & $(5,1)$ & 4 & 1 & YES & YES & YES & $0.56$ & $(2,1)$ & -- & 262\\
$(8,3)$ & 4 & $(5,1)$ & 4 & 1 & YES & YES & YES & $0.67$ & $(2,1)$ & NO & 263\\
$(8,3)$ & 4 & $(5,2)$ & 3 & 1 & YES & YES & YES & $0.44$ & $(4,0)$ & -- & 264\\
$(8,3)$ & 4 & $(7,2)$ & 4 & 1 & YES & YES & YES & $0.78$ & $(2,1)$ & -- & 265\\
$(8,3)$ & 4 & $(7,2)$ & 4 & 1 & YES & YES & YES & $0.67$ & $(2,1)$ & NO & 266\\
$(8,3)$ & 4 & $(7,2)$ & 4 & 1 & YES & YES & YES & $0.67$ & $(2,1)$ & NO & 267\\
$(8,3)$ & 4 & $(7,3)$ & 4 & 1 & YES & YES & YES & $0.67$ & $(2,1)$ & -- & 268\\
$(8,3)$ & 4 & $(7,3)$ & 4 & 1 & YES & YES & YES & $0.44$ & $(2,1)$ & NO & 269\\
$(8,3)$ & 4 & $(7,3)$ & 4 & 1 & YES & YES & YES & $0.56$ & $(2,1)$ & NO & 270\\
$(9,2)$ & 5 & $(4,1)$ & 3 & 1 & YES & YES & NO(2) & $0.73$ & $(2,1)$ & -- & 271\\
$(9,2)$ & 5 & $(4,1)$ & 3 & 1 & YES & YES & NO(2) & $0.82$ & $(2,1)$ & NO & 272\\
$(9,4)$ & 5 & $(4,1)$ & 3 & 1 & YES & YES & YES & $0.56$ & $(2,1)$ & NO & 273\\
$(9,2)$ & 5 & $(5,1)$ & 4 & 1 & YES & YES & NO(2) & $0.82$ & $(2,1)$ & NO & 274\\
$(9,2)$ & 5 & $(5,1)$ & 4 & 1 & YES & YES & NO(2) & $0.82$ & $(2,1)$ & NO & 275\\
$(9,2)$ & 5 & $(5,1)$ & 4 & 1 & YES & YES & NO(2) & $0.82$ & $(2,1)$ & -- & 276\\
$(9,2)$ & 5 & $(5,2)$ & 3 & 1 & YES & YES & YES & $0.67$ & $(2,1)$ & NO & 277\\
$(9,2)$ & 5 & $(5,2)$ & 3 & 1 & YES & YES & YES & $0.67$ & $(2,1)$ & -- & 278\\
$(9,2)$ & 5 & $(5,2)$ & 3 & 1 & YES & YES & YES & $0.67$ & $(2,1)$ & NO & 279\\
$(9,4)$ & 5 & $(5,2)$ & 3 & 1 & YES & YES & YES & $0.56$ & $(2,1)$ & -- & 280\\
$(9,4)$ & 5 & $(5,2)$ & 3 & 1 & YES & YES & NO(2) & $0.70$ & $(4,0)$ & NO & 281\\
$(9,2)$ & 5 & $(7,2)$ & 4 & 1 & YES & YES & YES & $0.56$ & $(2,1)$ & NO & 282\\
$(9,2)$ & 5 & $(7,2)$ & 4 & 1 & YES & YES & YES & $0.56$ & $(2,1)$ & -- & 283\\
$(9,2)$ & 5 & $(7,2)$ & 4 & 1 & YES & YES & YES & $0.56$ & $(2,1)$ & NO & 284\\
$(9,2)$ & 5 & $(7,3)$ & 4 & 1 & YES & YES & YES & $0.56$ & $(2,1)$ & NO & 285\\
$(9,2)$ & 5 & $(7,3)$ & 4 & 1 & YES & YES & YES & $0.56$ & $(2,1)$ & -- & 286\\
$(9,4)$ & 5 & $(7,2)$ & 4 & 1 & YES & YES & YES & $0.70$ & $(2,1)$ & -- & 287\\
$(9,4)$ & 5 & $(7,2)$ & 4 & 1 & YES & YES & YES & $0.90$ & $(2,1)$ & NO & 288\\
$(9,4)$ & 5 & $(7,2)$ & 4 & 1 & YES & YES & YES & $0.67$ & $(2,1)$ & NO & 289\\
$(9,4)$ & 5 & $(8,3)$ & 4 & 1 & YES & YES & YES & $0.67$ & $(2,1)$ & -- & 290\\
$(9,4)$ & 5 & $(8,3)$ & 4 & 1 & YES & YES & YES & $0.67$ & $(2,1)$ & 401 & 291\\
$(9,4)$ & 5 & $(9,2)$ & 5 & 9 & YES & YES & YES & $0.80$ & $(2,1)$ & NO & 292\\
$(9,4)$ & 5 & $(9,2)$ & 5 & 9 & YES & YES & YES & $0.80$ & $(2,1)$ & -- & 293\\
$(9,4)$ & 5 & $(9,2)$ & 5 & 9 & YES & YES & YES & $0.80$ & $(2,1)$ & NO & 294\\
$(10,3)$ & 5 & $(4,1)$ & 3 & 2 & YES & YES & YES & $0.67$ & $(2,1)$ & NO & 295\\
$(10,3)$ & 5 & $(5,1)$ & 4 & 5 & YES & YES & YES & $0.67$ & $(2,1)$ & NO & 296\\
$(10,3)$ & 5 & $(5,1)$ & 4 & 5 & YES & YES & YES & $0.67$ & $(2,1)$ & -- & 297\\
$(10,3)$ & 5 & $(5,1)$ & 4 & 5 & YES & YES & YES & $0.67$ & $(2,1)$ & NO & 298\\
$(10,3)$ & 5 & $(5,2)$ & 3 & 5 & YES & YES & YES & $0.78$ & $(2,1)$ & -- & 299\\
$(10,3)$ & 5 & $(5,2)$ & 3 & 5 & YES & YES & YES & $0.78$ & $(2,1)$ & NO & 300\\
$(10,3)$ & 5 & $(7,2)$ & 4 & 1 & YES & YES & YES & $0.67$ & $(2,1)$ & -- & 301\\
$(10,3)$ & 5 & $(7,2)$ & 4 & 1 & YES & YES & YES & $0.78$ & $(2,1)$ & NO & 302\\
$(10,3)$ & 5 & $(7,3)$ & 4 & 1 & YES & YES & YES & $0.67$ & $(2,1)$ & -- & 303\\
$(10,3)$ & 5 & $(8,3)$ & 4 & 2 & YES & YES & YES & $0.80$ & $(2,1)$ & NO & 304\\
$(11,2)$ & 6 & $(2,1)$ & 1 & 1 & YES & YES & YES & $0.67$ & $(2,1)$ & NO & 305\\
$(11,3)$ & 5 & $(3,1)$ & 2 & 1 & YES & YES & YES & $0.67$ & $(2,1)$ & NO & 306\\
$(11,3)$ & 5 & $(4,1)$ & 3 & 1 & YES & YES & YES & $0.56$ & $(2,1)$ & -- & 307\\
$(11,3)$ & 5 & $(4,1)$ & 3 & 1 & YES & YES & YES & $0.67$ & $(2,1)$ & NO & 308\\
$(11,4)$ & 5 & $(4,1)$ & 3 & 1 & YES & YES & YES & $0.56$ & $(4,0)$ & NO & 309\\
$(11,4)$ & 5 & $(4,1)$ & 3 & 1 & YES & YES & YES & $0.56$ & $(4,0)$ & -- & 310\\
$(11,5)$ & 6 & $(4,1)$ & 3 & 1 & YES & YES & NO(2) & $0.70$ & $(4,0)$ & NO & 311\\
$(11,5)$ & 6 & $(4,1)$ & 3 & 1 & YES & YES & NO(2) & $0.70$ & $(4,0)$ & NO & 312\\
$(11,5)$ & 6 & $(4,1)$ & 3 & 1 & YES & YES & NO(2) & $0.70$ & $(4,0)$ & -- & 313\\
$(11,2)$ & 6 & $(5,1)$ & 4 & 1 & YES & YES & YES & $0.56$ & $(2,1)$ & NO & 314\\
$(11,2)$ & 6 & $(5,1)$ & 4 & 1 & YES & YES & YES & $0.56$ & $(2,1)$ & NO & 315\\
$(11,2)$ & 6 & $(5,1)$ & 4 & 1 & YES & YES & YES & $0.56$ & $(2,1)$ & -- & 316\\
$(11,3)$ & 5 & $(5,1)$ & 4 & 1 & YES & YES & YES & $0.67$ & $(2,1)$ & NO & 317\\
$(11,3)$ & 5 & $(5,1)$ & 4 & 1 & YES & YES & YES & $0.67$ & $(2,1)$ & -- & 318\\
$(11,3)$ & 5 & $(5,1)$ & 4 & 1 & YES & YES & YES & $0.67$ & $(2,1)$ & NO & 319\\
$(11,3)$ & 5 & $(5,2)$ & 3 & 1 & YES & YES & YES & $0.78$ & $(2,1)$ & NO & 320\\
$(11,3)$ & 5 & $(5,2)$ & 3 & 1 & YES & YES & YES & $0.67$ & $(2,1)$ & -- & 321\\
$(11,3)$ & 5 & $(5,2)$ & 3 & 1 & YES & YES & NO(2) & $0.50$ & $(4,0)$ & NO & 322\\
$(11,5)$ & 6 & $(5,1)$ & 4 & 1 & YES & YES & YES & $0.67$ & $(2,1)$ & NO & 323\\
$(11,5)$ & 6 & $(5,1)$ & 4 & 1 & YES & YES & YES & $0.67$ & $(2,1)$ & NO & 324\\
$(11,5)$ & 6 & $(5,2)$ & 3 & 1 & YES & YES & NO(2) & $0.70$ & $(4,0)$ & NO & 325\\
$(11,5)$ & 6 & $(5,2)$ & 3 & 1 & YES & YES & NO(2) & $0.70$ & $(4,0)$ & -- & 326\\
$(11,5)$ & 6 & $(6,1)$ & 5 & 1 & YES & YES & NO(2) & $0.70$ & $(4,0)$ & NO & 327\\
$(11,5)$ & 6 & $(6,1)$ & 5 & 1 & YES & YES & NO(2) & $0.70$ & $(4,0)$ & NO & 328\\
$(11,3)$ & 5 & $(7,2)$ & 4 & 1 & YES & YES & YES & $0.56$ & $(2,1)$ & NO & 329\\
$(11,3)$ & 5 & $(7,2)$ & 4 & 1 & YES & YES & YES & $0.56$ & $(2,1)$ & -- & 330\\
$(11,4)$ & 5 & $(7,2)$ & 4 & 1 & YES & YES & YES & $0.56$ & $(2,1)$ & NO & 331\\
$(11,5)$ & 6 & $(7,3)$ & 4 & 1 & YES & YES & YES & $0.67$ & $(2,1)$ & 396 & 332\\
$(11,4)$ & 5 & $(8,3)$ & 4 & 1 & YES & YES & YES & $0.44$ & $(4,0)$ & NO & 333\\
$(11,2)$ & 6 & $(9,4)$ & 5 & 1 & YES & YES & YES & $0.56$ & $(2,1)$ & NO & 334\\
$(11,5)$ & 6 & $(9,4)$ & 5 & 1 & YES & YES & NO(2) & $0.70$ & $(4,0)$ & NO & 335\\
$(11,3)$ & 5 & $(10,3)$ & 5 & 1 & YES & YES & YES & $0.67$ & $(2,1)$ & NO & 336\\
$(11,3)$ & 5 & $(10,3)$ & 5 & 1 & YES & YES & YES & $0.67$ & $(2,1)$ & -- & 337\\
$(11,3)$ & 5 & $(11,3)$ & 5 & 11 & YES & YES & YES & $0.67$ & $(2,1)$ & NO & 338\\
$(12,5)$ & 5 & $(4,1)$ & 3 & 4 & YES & YES & YES & $0.44$ & $(4,0)$ & NO & 339\\
$(12,5)$ & 5 & $(4,1)$ & 3 & 4 & YES & YES & YES & $0.56$ & $(4,0)$ & NO & 340\\
$(12,5)$ & 5 & $(4,1)$ & 3 & 4 & YES & YES & YES & $0.56$ & $(4,0)$ & -- & 341\\
$(12,5)$ & 5 & $(5,2)$ & 3 & 1 & YES & YES & YES & $0.80$ & $(2,1)$ & NO & 342\\
$(12,5)$ & 5 & $(5,2)$ & 3 & 1 & YES & YES & YES & $0.80$ & $(2,1)$ & -- & 343\\
$(12,5)$ & 5 & $(7,2)$ & 4 & 1 & YES & YES & YES & $0.56$ & $(2,1)$ & NO & 344\\
$(12,5)$ & 5 & $(7,2)$ & 4 & 1 & YES & YES & YES & $0.44$ & $(4,0)$ & -- & 345\\
$(12,5)$ & 5 & $(7,2)$ & 4 & 1 & YES & YES & YES & $0.44$ & $(4,0)$ & NO & 346\\
$(12,5)$ & 5 & $(8,3)$ & 4 & 4 & YES & YES & YES & $0.56$ & $(2,1)$ & NO & 347\\
$(12,5)$ & 5 & $(9,2)$ & 5 & 3 & YES & YES & YES & $0.70$ & $(2,1)$ & NO & 348\\
$(12,5)$ & 5 & $(9,2)$ & 5 & 3 & YES & YES & YES & $0.56$ & $(2,1)$ & NO & 349\\
$(12,5)$ & 5 & $(9,4)$ & 5 & 3 & YES & YES & YES & $0.80$ & $(2,1)$ & NO & 350\\
$(13,3)$ & 6 & $(3,1)$ & 2 & 1 & YES & YES & YES & $0.67$ & $(2,1)$ & NO & 351\\
$(13,5)$ & 5 & $(3,1)$ & 2 & 1 & YES & YES & YES & $0.78$ & $(2,1)$ & -- & 352\\
$(13,5)$ & 5 & $(3,1)$ & 2 & 1 & YES & YES & YES & $0.89$ & $(2,1)$ & NO & 353\\
$(13,4)$ & 6 & $(4,1)$ & 3 & 1 & YES & YES & YES & $0.56$ & $(4,0)$ & NO & 354\\
$(13,4)$ & 6 & $(4,1)$ & 3 & 1 & YES & YES & YES & $0.56$ & $(4,0)$ & -- & 355\\
$(13,5)$ & 5 & $(4,1)$ & 3 & 1 & YES & YES & YES & $0.67$ & $(2,1)$ & NO & 356\\
$(13,5)$ & 5 & $(4,1)$ & 3 & 1 & YES & YES & YES & $0.67$ & $(2,1)$ & -- & 357\\
$(13,5)$ & 5 & $(4,1)$ & 3 & 1 & YES & YES & YES & $0.78$ & $(2,1)$ & NO & 358\\
$(13,3)$ & 6 & $(5,1)$ & 4 & 1 & YES & YES & YES & $0.67$ & $(2,1)$ & NO & 359\\
$(13,3)$ & 6 & $(5,1)$ & 4 & 1 & YES & YES & YES & $0.67$ & $(2,1)$ & -- & 360\\
$(13,3)$ & 6 & $(5,1)$ & 4 & 1 & YES & YES & YES & $0.67$ & $(2,1)$ & 380 & 361\\
$(13,4)$ & 6 & $(5,2)$ & 3 & 1 & YES & YES & YES & $0.44$ & $(2,1)$ & NO & 362\\
$(13,5)$ & 5 & $(5,2)$ & 3 & 1 & YES & YES & YES & $0.78$ & $(2,1)$ & -- & 363\\
$(13,5)$ & 5 & $(5,2)$ & 3 & 1 & YES & YES & NO(2) & $0.60$ & $(4,0)$ & NO & 364\\
$(13,3)$ & 6 & $(7,2)$ & 4 & 1 & YES & YES & YES & $0.56$ & $(2,1)$ & NO & 365\\
$(13,3)$ & 6 & $(7,2)$ & 4 & 1 & YES & YES & YES & $0.56$ & $(2,1)$ & -- & 366\\
$(13,3)$ & 6 & $(7,3)$ & 4 & 1 & YES & YES & YES & $0.70$ & $(2,1)$ & NO & 367\\
$(13,4)$ & 6 & $(7,2)$ & 4 & 1 & YES & YES & YES & $0.56$ & $(4,0)$ & 429 & 368\\
$(13,4)$ & 6 & $(7,2)$ & 4 & 1 & YES & YES & YES & $0.56$ & $(4,0)$ & -- & 369\\
$(13,5)$ & 5 & $(7,3)$ & 4 & 1 & YES & YES & YES & $0.44$ & $(2,1)$ & NO & 370\\
$(13,5)$ & 5 & $(7,3)$ & 4 & 1 & YES & YES & YES & $0.67$ & $(2,1)$ & -- & 371\\
$(13,5)$ & 5 & $(8,3)$ & 4 & 1 & YES & YES & YES & $0.78$ & $(2,1)$ & NO & 372\\
$(13,3)$ & 6 & $(11,3)$ & 5 & 1 & YES & YES & NO(2) & $0.50$ & $(4,0)$ & NO & 373\\
$(13,4)$ & 6 & $(11,2)$ & 6 & 1 & YES & YES & YES & $0.56$ & $(2,1)$ & NO & 374\\
$(13,3)$ & 6 & $(13,3)$ & 6 & 13 & YES & YES & YES & $0.67$ & $(2,1)$ & NO & 375\\
$(13,5)$ & 5 & $(13,5)$ & 5 & 13 & YES & YES & YES & $0.67$ & $(2,1)$ & NO & 376\\
$(14,3)$ & 6 & $(2,1)$ & 1 & 2 & YES & YES & YES & $0.78$ & $(2,1)$ & NO & 377\\
$(14,5)$ & 6 & $(3,1)$ & 2 & 1 & YES & YES & YES & $0.70$ & $(2,1)$ & -- & 378\\
$(14,5)$ & 6 & $(3,1)$ & 2 & 1 & YES & YES & YES & $0.80$ & $(2,1)$ & NO & 379\\
$(14,3)$ & 6 & $(4,1)$ & 3 & 2 & YES & YES & YES & $0.67$ & $(2,1)$ & 361 & 380\\
$(14,3)$ & 6 & $(4,1)$ & 3 & 2 & YES & YES & YES & $0.67$ & $(2,1)$ & NO & 381\\
$(14,3)$ & 6 & $(4,1)$ & 3 & 2 & YES & YES & YES & $0.67$ & $(2,1)$ & -- & 382\\
$(14,5)$ & 6 & $(4,1)$ & 3 & 2 & YES & YES & YES & $0.56$ & $(2,1)$ & -- & 383\\
$(14,5)$ & 6 & $(4,1)$ & 3 & 2 & YES & YES & YES & $0.70$ & $(2,1)$ & NO & 384\\
$(14,3)$ & 6 & $(5,1)$ & 4 & 1 & NO & YES & NO(2) & $0.82$ & $(2,1)$ & -- & 385\\
$(14,5)$ & 6 & $(5,1)$ & 4 & 1 & YES & YES & YES & $0.56$ & $(2,1)$ & NO & 386\\
$(14,5)$ & 6 & $(5,2)$ & 3 & 1 & YES & YES & YES & $0.80$ & $(2,1)$ & NO & 387\\
$(14,5)$ & 6 & $(6,1)$ & 5 & 2 & YES & YES & YES & $0.80$ & $(2,1)$ & NO & 388\\
$(14,5)$ & 6 & $(8,3)$ & 4 & 2 & YES & YES & YES & $0.56$ & $(2,1)$ & 460 & 389\\
$(14,5)$ & 6 & $(11,4)$ & 5 & 1 & YES & YES & YES & $0.56$ & $(2,1)$ & NO & 390\\
$(14,5)$ & 6 & $(14,5)$ & 6 & 14 & YES & YES & YES & $0.70$ & $(2,1)$ & NO & 391\\
$(15,4)$ & 6 & $(7,2)$ & 4 & 1 & YES & YES & YES & $0.56$ & $(2,1)$ & -- & 392\\
$(15,4)$ & 6 & $(9,2)$ & 5 & 3 & YES & YES & YES & $0.70$ & $(2,1)$ & NO & 393\\
$(15,4)$ & 6 & $(11,2)$ & 6 & 1 & YES & YES & YES & $0.56$ & $(2,1)$ & NO & 394\\
$(16,7)$ & 6 & $(2,1)$ & 1 & 2 & YES & YES & YES & $0.67$ & $(2,1)$ & -- & 395\\
$(16,7)$ & 6 & $(2,1)$ & 1 & 2 & YES & YES & YES & $0.67$ & $(2,1)$ & 332 & 396\\
$(16,5)$ & 7 & $(3,1)$ & 2 & 1 & YES & YES & YES & $0.56$ & $(2,1)$ & NO & 397\\
$(16,5)$ & 7 & $(3,1)$ & 2 & 1 & YES & YES & YES & $0.56$ & $(2,1)$ & -- & 398\\
$(16,7)$ & 6 & $(3,1)$ & 2 & 1 & YES & YES & YES & $0.70$ & $(2,1)$ & NO & 399\\
$(16,7)$ & 6 & $(3,1)$ & 2 & 1 & YES & YES & YES & $0.70$ & $(2,1)$ & -- & 400\\
$(16,7)$ & 6 & $(3,1)$ & 2 & 1 & YES & YES & YES & $0.67$ & $(2,1)$ & 291 & 401\\
$(16,5)$ & 7 & $(4,1)$ & 3 & 4 & YES & YES & YES & $0.60$ & $(2,1)$ & NO & 402\\
$(16,7)$ & 6 & $(4,1)$ & 3 & 4 & YES & YES & YES & $0.80$ & $(2,1)$ & NO & 403\\
$(16,7)$ & 6 & $(4,1)$ & 3 & 4 & YES & YES & YES & $0.80$ & $(2,1)$ & -- & 404\\
$(16,7)$ & 6 & $(4,1)$ & 3 & 4 & YES & YES & YES & $0.67$ & $(2,1)$ & NO & 405\\
$(16,3)$ & 7 & $(5,1)$ & 4 & 1 & NO & YES & YES & $0.56$ & $(2,1)$ & NO & 406\\
$(16,3)$ & 7 & $(5,1)$ & 4 & 1 & NO & YES & YES & $0.56$ & $(2,1)$ & -- & 407\\
$(16,5)$ & 7 & $(5,1)$ & 4 & 1 & YES & YES & YES & $0.60$ & $(2,1)$ & NO & 408\\
$(16,7)$ & 6 & $(5,1)$ & 4 & 1 & YES & YES & NO(2) & $0.33$ & $(6,-1)$ & NO & 409\\
$(16,7)$ & 6 & $(5,1)$ & 4 & 1 & YES & YES & NO(2) & $0.33$ & $(6,-1)$ & -- & 410\\
$(16,7)$ & 6 & $(5,1)$ & 4 & 1 & YES & YES & YES & $0.56$ & $(2,1)$ & NO & 411\\
$(16,7)$ & 6 & $(5,2)$ & 3 & 1 & YES & YES & YES & $0.80$ & $(2,1)$ & NO & 412\\
$(16,7)$ & 6 & $(5,2)$ & 3 & 1 & YES & YES & YES & $0.80$ & $(2,1)$ & -- & 413\\
$(16,5)$ & 7 & $(6,1)$ & 5 & 2 & YES & YES & YES & $0.67$ & $(2,1)$ & NO & 414\\
$(16,5)$ & 7 & $(7,1)$ & 6 & 1 & YES & YES & NO(2) & $0.50$ & $(4,0)$ & NO & 415\\
$(16,5)$ & 7 & $(7,2)$ & 4 & 1 & YES & YES & YES & $0.60$ & $(2,1)$ & NO & 416\\
$(16,7)$ & 6 & $(7,3)$ & 4 & 1 & YES & YES & YES & $0.67$ & $(2,1)$ & NO & 417\\
$(16,7)$ & 6 & $(9,4)$ & 5 & 1 & YES & YES & YES & $0.70$ & $(2,1)$ & NO & 418\\
$(16,5)$ & 7 & $(10,3)$ & 5 & 2 & YES & YES & YES & $0.67$ & $(2,1)$ & 491 & 419\\
$(16,5)$ & 7 & $(13,4)$ & 6 & 1 & YES & YES & NO(2) & $0.50$ & $(4,0)$ & NO & 420\\
$(16,5)$ & 7 & $(16,5)$ & 7 & 16 & YES & YES & YES & $0.56$ & $(2,1)$ & NO & 421\\
$(16,7)$ & 6 & $(16,7)$ & 6 & 16 & YES & YES & YES & $0.80$ & $(2,1)$ & NO & 422\\
$(17,5)$ & 6 & $(2,1)$ & 1 & 1 & YES & YES & YES & $0.67$ & $(2,1)$ & -- & 423\\
$(17,5)$ & 6 & $(2,1)$ & 1 & 1 & YES & YES & YES & $0.78$ & $(2,1)$ & NO & 424\\
$(17,7)$ & 6 & $(2,1)$ & 1 & 1 & YES & YES & YES & $0.70$ & $(2,1)$ & -- & 425\\
$(17,7)$ & 6 & $(2,1)$ & 1 & 1 & YES & YES & YES & $0.80$ & $(2,1)$ & NO & 426\\
$(17,5)$ & 6 & $(3,1)$ & 2 & 1 & YES & YES & YES & $0.67$ & $(2,1)$ & -- & 427\\
$(17,5)$ & 6 & $(3,1)$ & 2 & 1 & YES & YES & YES & $0.78$ & $(2,1)$ & NO & 428\\
$(17,5)$ & 6 & $(3,1)$ & 2 & 1 & YES & YES & YES & $0.56$ & $(4,0)$ & 368 & 429\\
$(17,7)$ & 6 & $(3,1)$ & 2 & 1 & YES & YES & YES & $0.67$ & $(2,1)$ & -- & 430\\
$(17,7)$ & 6 & $(3,1)$ & 2 & 1 & YES & YES & YES & $0.70$ & $(2,1)$ & NO & 431\\
$(17,5)$ & 6 & $(4,1)$ & 3 & 1 & YES & YES & YES & $0.56$ & $(2,1)$ & NO & 432\\
$(17,5)$ & 6 & $(4,1)$ & 3 & 1 & YES & YES & YES & $0.56$ & $(2,1)$ & -- & 433\\
$(17,7)$ & 6 & $(4,1)$ & 3 & 1 & YES & YES & YES & $0.56$ & $(2,1)$ & NO & 434\\
$(17,7)$ & 6 & $(4,1)$ & 3 & 1 & YES & YES & YES & $0.70$ & $(2,1)$ & -- & 435\\
$(17,5)$ & 6 & $(5,2)$ & 3 & 1 & YES & YES & YES & $0.56$ & $(2,1)$ & -- & 436\\
$(17,5)$ & 6 & $(7,2)$ & 4 & 1 & YES & YES & YES & $0.67$ & $(2,1)$ & NO & 437\\
$(17,5)$ & 6 & $(7,2)$ & 4 & 1 & YES & YES & YES & $0.67$ & $(2,1)$ & -- & 438\\
$(17,7)$ & 6 & $(7,3)$ & 4 & 1 & YES & YES & YES & $0.67$ & $(2,1)$ & 469 & 439\\
$(17,7)$ & 6 & $(12,5)$ & 5 & 1 & YES & YES & YES & $0.70$ & $(2,1)$ & NO & 440\\
$(17,5)$ & 6 & $(13,4)$ & 6 & 1 & YES & YES & YES & $0.56$ & $(2,1)$ & NO & 441\\
$(17,5)$ & 6 & $(17,5)$ & 6 & 17 & YES & YES & YES & $0.67$ & $(2,1)$ & NO & 442\\
$(17,7)$ & 6 & $(17,7)$ & 6 & 17 & YES & YES & YES & $0.70$ & $(2,1)$ & NO & 443\\
$(18,5)$ & 6 & $(2,1)$ & 1 & 2 & YES & YES & YES & $0.50$ & $(2,1)$ & -- & 444\\
$(18,5)$ & 6 & $(2,1)$ & 1 & 2 & YES & YES & YES & $0.60$ & $(2,1)$ & NO & 445\\
$(18,7)$ & 6 & $(2,1)$ & 1 & 2 & YES & YES & YES & $0.70$ & $(2,1)$ & NO & 446\\
$(18,7)$ & 6 & $(2,1)$ & 1 & 2 & YES & YES & YES & $0.70$ & $(2,1)$ & -- & 447\\
$(18,5)$ & 6 & $(3,1)$ & 2 & 3 & YES & YES & NO(3) & $0.33$ & $(2,1)$ & -- & 448\\
$(18,5)$ & 6 & $(3,1)$ & 2 & 3 & YES & YES & YES & $0.70$ & $(2,1)$ & NO & 449\\
$(18,5)$ & 6 & $(7,2)$ & 4 & 1 & YES & YES & NO(2) & $0.73$ & $(2,1)$ & NO & 450\\
$(18,5)$ & 6 & $(9,2)$ & 5 & 9 & YES & YES & YES & $0.44$ & $(2,1)$ & NO & 451\\
$(18,5)$ & 6 & $(15,4)$ & 6 & 3 & YES & YES & YES & $0.44$ & $(2,1)$ & NO & 452\\
$(19,5)$ & 7 & $(2,1)$ & 1 & 1 & YES & YES & YES & $0.70$ & $(2,1)$ & -- & 453\\
$(19,5)$ & 7 & $(2,1)$ & 1 & 1 & YES & YES & YES & $0.80$ & $(2,1)$ & NO & 454\\
$(19,8)$ & 6 & $(2,1)$ & 1 & 1 & YES & YES & YES & $0.70$ & $(2,1)$ & NO & 455\\
$(19,8)$ & 6 & $(2,1)$ & 1 & 1 & YES & YES & YES & $0.70$ & $(2,1)$ & -- & 456\\
$(19,5)$ & 7 & $(3,1)$ & 2 & 1 & YES & YES & YES & $0.67$ & $(2,1)$ & NO & 457\\
$(19,5)$ & 7 & $(3,1)$ & 2 & 1 & YES & YES & YES & $0.67$ & $(2,1)$ & -- & 458\\
$(19,5)$ & 7 & $(3,1)$ & 2 & 1 & YES & YES & YES & $0.70$ & $(2,1)$ & NO & 459\\
$(19,7)$ & 6 & $(3,1)$ & 2 & 1 & YES & YES & YES & $0.56$ & $(2,1)$ & 389 & 460\\
$(19,8)$ & 6 & $(3,1)$ & 2 & 1 & YES & YES & YES & $0.67$ & $(2,1)$ & -- & 461\\
$(19,8)$ & 6 & $(3,1)$ & 2 & 1 & YES & YES & YES & $0.67$ & $(2,1)$ & NO & 462\\
$(19,8)$ & 6 & $(3,1)$ & 2 & 1 & YES & YES & YES & $0.67$ & $(2,1)$ & 482 & 463\\
$(19,5)$ & 7 & $(4,1)$ & 3 & 1 & YES & YES & NO(2) & $0.60$ & $(4,0)$ & NO & 464\\
$(19,8)$ & 6 & $(4,1)$ & 3 & 1 & YES & YES & YES & $0.70$ & $(2,1)$ & -- & 465\\
$(19,5)$ & 7 & $(5,1)$ & 4 & 1 & YES & YES & YES & $0.70$ & $(2,1)$ & NO & 466\\
$(19,7)$ & 6 & $(5,1)$ & 4 & 1 & YES & YES & YES & $0.44$ & $(2,1)$ & NO & 467\\
$(19,7)$ & 6 & $(5,2)$ & 3 & 1 & YES & YES & YES & $0.70$ & $(2,1)$ & NO & 468\\
$(19,8)$ & 6 & $(5,2)$ & 3 & 1 & YES & YES & YES & $0.67$ & $(2,1)$ & 439 & 469\\
$(19,5)$ & 7 & $(6,1)$ & 5 & 1 & YES & YES & YES & $0.70$ & $(2,1)$ & NO & 470\\
$(19,5)$ & 7 & $(7,1)$ & 6 & 1 & YES & YES & NO(2) & $0.50$ & $(4,0)$ & NO & 471\\
$(19,5)$ & 7 & $(7,2)$ & 4 & 1 & YES & YES & YES & $0.56$ & $(2,1)$ & NO & 472\\
$(19,4)$ & 7 & $(11,2)$ & 6 & 1 & YES & YES & YES & $0.44$ & $(2,1)$ & NO & 473\\
$(19,5)$ & 7 & $(11,3)$ & 5 & 1 & YES & YES & NO(2) & $0.50$ & $(4,0)$ & 513 & 474\\
$(19,8)$ & 6 & $(12,5)$ & 5 & 1 & YES & YES & YES & $0.56$ & $(2,1)$ & NO & 475\\
$(19,5)$ & 7 & $(15,4)$ & 6 & 1 & YES & YES & YES & $0.70$ & $(2,1)$ & NO & 476\\
$(19,5)$ & 7 & $(19,5)$ & 7 & 19 & YES & YES & YES & $0.70$ & $(2,1)$ & NO & 477\\
$(19,7)$ & 6 & $(19,7)$ & 6 & 19 & YES & YES & YES & $0.44$ & $(2,1)$ & NO & 478\\
$(19,8)$ & 6 & $(19,8)$ & 6 & 19 & YES & YES & YES & $0.70$ & $(2,1)$ & NO & 479\\
$(20,9)$ & 7 & $(2,1)$ & 1 & 2 & NO & YES & YES & $0.67$ & $(2,1)$ & -- & 480\\
$(21,5)$ & 8 & $(2,1)$ & 1 & 1 & YES & YES & YES & $0.44$ & $(2,1)$ & NO & 481\\
$(21,8)$ & 6 & $(2,1)$ & 1 & 1 & YES & YES & YES & $0.67$ & $(2,1)$ & 463 & 482\\
$(21,5)$ & 8 & $(3,1)$ & 2 & 3 & YES & YES & YES & $0.44$ & $(2,1)$ & NO & 483\\
$(21,8)$ & 6 & $(3,1)$ & 2 & 3 & YES & YES & YES & $0.56$ & $(2,1)$ & -- & 484\\
$(21,5)$ & 8 & $(4,1)$ & 3 & 1 & YES & YES & YES & $0.44$ & $(2,1)$ & NO & 485\\
$(21,5)$ & 8 & $(5,1)$ & 4 & 1 & YES & YES & YES & $0.44$ & $(2,1)$ & NO & 486\\
$(21,5)$ & 8 & $(6,1)$ & 5 & 3 & YES & YES & YES & $0.44$ & $(2,1)$ & NO & 487\\
$(21,5)$ & 8 & $(9,2)$ & 5 & 3 & YES & YES & YES & $0.44$ & $(2,1)$ & NO & 488\\
$(22,5)$ & 7 & $(4,1)$ & 3 & 2 & NO & YES & YES & $0.67$ & $(2,1)$ & -- & 489\\
$(23,10)$ & 7 & $(2,1)$ & 1 & 1 & NO & YES & YES & $0.80$ & $(2,1)$ & -- & 490\\
$(23,7)$ & 7 & $(3,1)$ & 2 & 1 & YES & YES & YES & $0.67$ & $(2,1)$ & 419 & 491\\
$(23,7)$ & 7 & $(3,1)$ & 2 & 1 & NO & YES & YES & $0.56$ & $(4,0)$ & -- & 492\\
$(23,7)$ & 7 & $(5,1)$ & 4 & 1 & YES & YES & YES & $0.56$ & $(2,1)$ & NO & 493\\
$(23,7)$ & 7 & $(23,7)$ & 7 & 23 & YES & YES & YES & $0.56$ & $(2,1)$ & NO & 494\\
$(24,5)$ & 8 & $(2,1)$ & 1 & 2 & YES & YES & YES & $0.56$ & $(2,1)$ & NO & 495\\
$(24,5)$ & 8 & $(2,1)$ & 1 & 2 & YES & YES & YES & $0.56$ & $(2,1)$ & -- & 496\\
$(24,7)$ & 7 & $(2,1)$ & 1 & 2 & YES & YES & YES & $0.56$ & $(2,1)$ & -- & 497\\
$(24,5)$ & 8 & $(3,1)$ & 2 & 3 & YES & YES & YES & $0.56$ & $(2,1)$ & NO & 498\\
$(24,5)$ & 8 & $(3,1)$ & 2 & 3 & YES & YES & YES & $0.80$ & $(2,1)$ & -- & 499\\
$(24,5)$ & 8 & $(3,1)$ & 2 & 3 & YES & YES & YES & $0.80$ & $(2,1)$ & NO & 500\\
$(24,5)$ & 8 & $(4,1)$ & 3 & 4 & YES & YES & YES & $0.56$ & $(2,1)$ & NO & 501\\
$(24,5)$ & 8 & $(5,1)$ & 4 & 1 & YES & YES & YES & $0.56$ & $(2,1)$ & NO & 502\\
$(24,5)$ & 8 & $(6,1)$ & 5 & 6 & YES & YES & YES & $0.56$ & $(2,1)$ & NO & 503\\
$(24,5)$ & 8 & $(7,1)$ & 6 & 1 & YES & YES & YES & $0.44$ & $(2,1)$ & NO & 504\\
$(24,5)$ & 8 & $(19,4)$ & 7 & 1 & YES & YES & YES & $0.44$ & $(2,1)$ & NO & 505\\
$(24,5)$ & 8 & $(24,5)$ & 8 & 24 & YES & YES & YES & $0.56$ & $(2,1)$ & NO & 506\\
$(25,9)$ & 7 & $(2,1)$ & 1 & 1 & NO & YES & YES & $0.67$ & $(2,1)$ & -- & 507\\
$(25,11)$ & 7 & $(2,1)$ & 1 & 1 & NO & YES & YES & $0.80$ & $(2,1)$ & -- & 508\\
$(26,7)$ & 7 & $(2,1)$ & 1 & 2 & YES & YES & YES & $0.60$ & $(2,1)$ & -- & 509\\
$(26,7)$ & 7 & $(2,1)$ & 1 & 2 & YES & YES & YES & $0.70$ & $(2,1)$ & NO & 510\\
$(26,7)$ & 7 & $(3,1)$ & 2 & 1 & YES & YES & YES & $0.56$ & $(2,1)$ & NO & 511\\
$(26,7)$ & 7 & $(3,1)$ & 2 & 1 & YES & YES & YES & $0.56$ & $(2,1)$ & -- & 512\\
$(26,7)$ & 7 & $(4,1)$ & 3 & 2 & YES & YES & NO(2) & $0.50$ & $(4,0)$ & 474 & 513\\
$(26,7)$ & 7 & $(15,4)$ & 6 & 1 & YES & YES & YES & $0.56$ & $(2,1)$ & NO & 514\\
$(26,7)$ & 7 & $(26,7)$ & 7 & 26 & YES & YES & YES & $0.60$ & $(2,1)$ & NO & 515\\
$(30,11)$ & 7 & $(2,1)$ & 1 & 2 & NO & YES & YES & $0.78$ & $(2,1)$ & -- & 516\\
$(30,7)$ & 8 & $(3,1)$ & 2 & 3 & YES & YES & YES & $0.70$ & $(2,1)$ & NO & 517\\
$(30,7)$ & 8 & $(3,1)$ & 2 & 3 & YES & YES & YES & $0.70$ & $(2,1)$ & -- & 518\\
$(30,7)$ & 8 & $(9,2)$ & 5 & 3 & YES & YES & YES & $0.70$ & $(2,1)$ & NO & 519\\
$(31,12)$ & 7 & $(2,1)$ & 1 & 1 & NO & YES & YES & $0.78$ & $(2,1)$ & -- & 520\\
$(a;0,0,0;3)$ & 4 & $(7,3)$ & 4 & 1 & YES & YES & YES & $0.56$ & $(2,1)$ & -- & 521\\
$(a;1,0,0;13)$ & 5 & $(2,1)$ & 1 & 1 & YES & YES & YES & $0.70$ & $(2,1)$ & -- & 522\\
$(a;1,0,0;13)$ & 5 & $(3,1)$ & 2 & 1 & YES & YES & YES & $0.67$ & $(2,1)$ & -- & 523\\
$(a;1,0,0;13)$ & 5 & $(5,2)$ & 3 & 1 & YES & YES & YES & $0.67$ & $(2,1)$ & -- & 524\\
$(a;1,0,0;13)$ & 5 & $(7,2)$ & 4 & 1 & YES & YES & YES & $0.56$ & $(2,1)$ & -- & 525\\
$(a;1,1,0;19)$ & 6 & $(4,1)$ & 3 & 1 & YES & YES & YES & $0.56$ & $(2,1)$ & -- & 526\\
$(a;2,0,0;17)$ & 6 & $(3,1)$ & 2 & 1 & YES & YES & YES & $0.67$ & $(2,1)$ & -- & 527\\
$(a;2,0,0;17)$ & 6 & $(5,1)$ & 4 & 1 & YES & YES & YES & $0.56$ & $(2,1)$ & -- & 528\\
$(b;0,0,0;14)$ & 5 & $(2,1)$ & 1 & 2 & YES & YES & YES & $0.70$ & $(2,1)$ & -- & 529\\
$(b;0,0,0;14)$ & 5 & $(3,1)$ & 2 & 1 & YES & YES & YES & $0.56$ & $(2,1)$ & -- & 530\\
$(c;0,0,0;4)$ & 4 & $(5,2)$ & 3 & 1 & YES & YES & NO(2) & $0.50$ & $(4,0)$ & -- & 531\\
$(c;0,0,0;4)$ & 4 & $(7,3)$ & 4 & 1 & YES & YES & YES & $0.56$ & $(2,1)$ & -- & 532\\
$(c;0,1,0;11)$ & 5 & $(2,1)$ & 1 & 1 & YES & YES & NO(2) & $0.82$ & $(2,1)$ & -- & 533\\
$(c;0,1,0;11)$ & 5 & $(3,1)$ & 2 & 1 & YES & YES & YES & $0.78$ & $(2,1)$ & -- & 534\\
$(c;0,1,1;5)$ & 6 & $(2,1)$ & 1 & 1 & YES & YES & NO(2) & $0.22$ & $(6,-1)$ & -- & 535\\
$(c;0,2,0;7)$ & 6 & $(3,1)$ & 2 & 1 & YES & YES & YES & $0.67$ & $(2,1)$ & -- & 536\\
$(d;0,0,0;5)$ & 5 & $(2,1)$ & 1 & 1 & YES & YES & YES & $0.56$ & $(2,1)$ & -- & 537\\
$(d;0,0,0;5)$ & 5 & $(4,1)$ & 3 & 1 & YES & YES & YES & $0.56$ & $(2,1)$ & -- & 538\\
$(d;0,1,0;6)$ & 6 & $(3,1)$ & 2 & 3 & YES & YES & YES & $0.44$ & $(2,1)$ & -- & 539\\
$(f;0,0,0;6)$ & 4 & $(5,2)$ & 3 & 1 & YES & YES & YES & $0.56$ & $(4,0)$ & -- & 540\\
$(f;0,0,0;6)$ & 4 & $(7,2)$ & 4 & 1 & YES & YES & YES & $0.56$ & $(4,0)$ & -- & 541\\
$(f;0,0,0;6)$ & 4 & $(7,3)$ & 4 & 1 & YES & YES & YES & $0.56$ & $(2,1)$ & -- & 542\\
$(f;0,0,0;6)$ & 4 & $(8,3)$ & 4 & 2 & YES & YES & YES & $0.44$ & $(2,1)$ & -- & 543\\
$(f;0,0,0;6)$ & 4 & $(9,2)$ & 5 & 3 & YES & YES & NO(2) & $0.33$ & $(6,-1)$ & -- & 544\\
$(f;0,0,0;6)$ & 4 & $(10,3)$ & 5 & 2 & YES & YES & YES & $0.56$ & $(2,1)$ & -- & 545\\
$(f;0,1,0;7)$ & 5 & $(4,1)$ & 3 & 1 & YES & YES & NO(2) & $0.33$ & $(6,-1)$ & -- & 546\\
$(i;0,0,0;9)$ & 5 & $(2,1)$ & 1 & 1 & YES & YES & YES & $0.80$ & $(2,1)$ & -- & 547\\
$(j;0,1,0;10)$ & 6 & $(3,1)$ & 2 & 1 & YES & YES & YES & $0.70$ & $(2,1)$ & -- & 548\\
$(j;0,1,0;10)$ & 6 & $(4,1)$ & 3 & 2 & YES & YES & YES & $0.70$ & $(2,1)$ & -- & 549
\end{longtable}
\subsection{2 chains, $K^2 = 2$}
\begin{longtable}{|c|c|c|c|c|c|c|c|c|c|c|c|}
\hline
\multicolumn{12}{|c|}{2 chains, $K^2 = 2$}\\
\hline
$(n,a)$ & Len & $(n,a)$ & Len & GCD & Nef & $\mathbb Q$-ef & Obs 0 & $\overline c_1^2 / \overline c_2$ & $(P,K)$ & WH & Index\\
\hline
\endfirsthead

\hline
$(n,a)$ & Len & $(n,a)$ & Len & GCD & Nef & $\mathbb Q$-ef & Obs 0 & $\overline c_1^2 / \overline c_2$ & $(P,K)$ & WH & Index\\
\hline
\endhead
\hline
\endfoot

$(11,2)$ & 6 & $(7,3)$ & 4 & 1 & YES & YES & YES & $0.88$ & $(2,2)$ & -- & 550\\
$(11,4)$ & 5 & $(10,3)$ & 5 & 1 & YES & YES & NO(2) & $0.75$ & $(6,0)$ & NO & 551\\
$(11,5)$ & 6 & $(11,3)$ & 5 & 11 & YES & YES & YES & $1.00$ & $(2,2)$ & -- & 552\\
$(11,5)$ & 6 & $(11,3)$ & 5 & 11 & YES & YES & YES & $1.00$ & $(2,2)$ & NO & 553\\
$(11,5)$ & 6 & $(11,5)$ & 6 & 11 & YES & YES & YES & $1.00$ & $(2,2)$ & -- & 554\\
$(12,5)$ & 5 & $(9,2)$ & 5 & 3 & YES & YES & YES & $1.00$ & $(2,2)$ & -- & 555\\
$(12,5)$ & 5 & $(9,4)$ & 5 & 3 & YES & YES & YES & $0.88$ & $(2,2)$ & -- & 556\\
$(12,5)$ & 5 & $(11,5)$ & 6 & 1 & YES & YES & YES & $0.88$ & $(2,2)$ & -- & 557\\
$(13,4)$ & 6 & $(6,1)$ & 5 & 1 & YES & YES & YES & $1.12$ & $(2,2)$ & NO & 558\\
$(13,5)$ & 5 & $(7,2)$ & 4 & 1 & YES & YES & NO(2) & $0.89$ & $(4,1)$ & NO & 559\\
$(13,5)$ & 5 & $(7,2)$ & 4 & 1 & YES & YES & NO(2) & $0.89$ & $(4,1)$ & -- & 560\\
$(13,4)$ & 6 & $(8,3)$ & 4 & 1 & YES & YES & NO(2) & $1.00$ & $(4,1)$ & NO & 561\\
$(13,5)$ & 5 & $(8,3)$ & 4 & 1 & YES & YES & NO(2) & $0.89$ & $(4,1)$ & -- & 562\\
$(13,4)$ & 6 & $(9,4)$ & 5 & 1 & YES & YES & YES & $1.12$ & $(2,2)$ & NO & 563\\
$(13,5)$ & 5 & $(9,4)$ & 5 & 1 & YES & YES & YES & $0.88$ & $(2,2)$ & -- & 564\\
$(13,6)$ & 7 & $(9,4)$ & 5 & 1 & YES & YES & YES & $1.12$ & $(2,2)$ & -- & 565\\
$(13,5)$ & 5 & $(10,3)$ & 5 & 1 & YES & YES & YES & $1.00$ & $(2,2)$ & NO & 566\\
$(13,3)$ & 6 & $(11,4)$ & 5 & 1 & YES & YES & YES & $0.88$ & $(4,1)$ & NO & 567\\
$(13,3)$ & 6 & $(11,4)$ & 5 & 1 & YES & YES & YES & $0.88$ & $(4,1)$ & -- & 568\\
$(13,3)$ & 6 & $(11,4)$ & 5 & 1 & YES & YES & NO(2) & $0.75$ & $(6,0)$ & 1014 & 569\\
$(13,4)$ & 6 & $(11,2)$ & 6 & 1 & YES & YES & YES & $1.12$ & $(2,2)$ & NO & 570\\
$(13,4)$ & 6 & $(11,2)$ & 6 & 1 & YES & YES & YES & $1.12$ & $(2,2)$ & -- & 571\\
$(13,4)$ & 6 & $(11,4)$ & 5 & 1 & YES & YES & NO(2) & $1.00$ & $(4,1)$ & NO & 572\\
$(13,4)$ & 6 & $(11,4)$ & 5 & 1 & YES & YES & NO(2) & $1.00$ & $(4,1)$ & -- & 573\\
$(13,4)$ & 6 & $(11,5)$ & 6 & 1 & YES & YES & NO(2) & $0.88$ & $(6,0)$ & NO & 574\\
$(13,4)$ & 6 & $(11,5)$ & 6 & 1 & YES & YES & NO(2) & $0.88$ & $(6,0)$ & -- & 575\\
$(13,4)$ & 6 & $(11,5)$ & 6 & 1 & YES & YES & YES & $1.12$ & $(2,2)$ & NO & 576\\
$(13,5)$ & 5 & $(11,4)$ & 5 & 1 & YES & YES & YES & $0.88$ & $(4,1)$ & -- & 577\\
$(13,5)$ & 5 & $(11,5)$ & 6 & 1 & YES & YES & YES & $1.00$ & $(2,2)$ & NO & 578\\
$(13,5)$ & 5 & $(11,5)$ & 6 & 1 & YES & YES & YES & $0.88$ & $(2,2)$ & -- & 579\\
$(13,6)$ & 7 & $(11,2)$ & 6 & 1 & YES & YES & YES & $0.88$ & $(2,2)$ & NO & 580\\
$(13,6)$ & 7 & $(11,3)$ & 5 & 1 & YES & YES & YES & $1.00$ & $(2,2)$ & NO & 581\\
$(13,6)$ & 7 & $(11,3)$ & 5 & 1 & YES & YES & YES & $1.00$ & $(2,2)$ & -- & 582\\
$(13,4)$ & 6 & $(12,5)$ & 5 & 1 & YES & YES & NO(2) & $1.00$ & $(4,1)$ & NO & 583\\
$(13,4)$ & 6 & $(12,5)$ & 5 & 1 & YES & YES & NO(2) & $1.00$ & $(4,1)$ & -- & 584\\
$(13,5)$ & 5 & $(13,3)$ & 6 & 13 & YES & YES & YES & $1.22$ & $(2,2)$ & NO & 585\\
$(13,5)$ & 5 & $(13,3)$ & 6 & 13 & YES & YES & YES & $1.22$ & $(2,2)$ & -- & 586\\
$(13,5)$ & 5 & $(13,3)$ & 6 & 13 & YES & YES & YES & $1.22$ & $(2,2)$ & NO & 587\\
$(13,6)$ & 7 & $(13,4)$ & 6 & 13 & YES & YES & YES & $1.12$ & $(2,2)$ & NO & 588\\
$(14,3)$ & 6 & $(5,2)$ & 3 & 1 & YES & YES & YES & $0.75$ & $(2,2)$ & -- & 589\\
$(14,3)$ & 6 & $(7,2)$ & 4 & 7 & YES & YES & YES & $1.11$ & $(2,2)$ & -- & 590\\
$(14,3)$ & 6 & $(9,4)$ & 5 & 1 & YES & YES & YES & $1.00$ & $(2,2)$ & NO & 591\\
$(14,3)$ & 6 & $(9,4)$ & 5 & 1 & YES & YES & YES & $1.00$ & $(2,2)$ & -- & 592\\
$(14,3)$ & 6 & $(9,4)$ & 5 & 1 & YES & YES & YES & $1.00$ & $(2,2)$ & NO & 593\\
$(14,3)$ & 6 & $(10,3)$ & 5 & 2 & YES & YES & YES & $1.00$ & $(2,2)$ & NO & 594\\
$(14,3)$ & 6 & $(10,3)$ & 5 & 2 & YES & YES & YES & $1.00$ & $(2,2)$ & -- & 595\\
$(14,3)$ & 6 & $(10,3)$ & 5 & 2 & YES & YES & YES & $0.88$ & $(2,2)$ & NO & 596\\
$(14,5)$ & 6 & $(10,3)$ & 5 & 2 & YES & YES & YES & $1.12$ & $(2,2)$ & -- & 597\\
$(14,5)$ & 6 & $(10,3)$ & 5 & 2 & YES & YES & NO(2) & $0.75$ & $(6,0)$ & NO & 598\\
$(14,3)$ & 6 & $(11,3)$ & 5 & 1 & YES & YES & YES & $1.00$ & $(2,2)$ & NO & 599\\
$(14,3)$ & 6 & $(11,3)$ & 5 & 1 & YES & YES & YES & $1.00$ & $(2,2)$ & -- & 600\\
$(14,5)$ & 6 & $(11,2)$ & 6 & 1 & YES & YES & YES & $1.12$ & $(2,2)$ & NO & 601\\
$(14,5)$ & 6 & $(11,2)$ & 6 & 1 & YES & YES & YES & $1.12$ & $(2,2)$ & -- & 602\\
$(14,5)$ & 6 & $(11,2)$ & 6 & 1 & YES & YES & YES & $1.12$ & $(2,2)$ & NO & 603\\
$(14,5)$ & 6 & $(11,3)$ & 5 & 1 & YES & YES & YES & $0.88$ & $(4,1)$ & NO & 604\\
$(14,5)$ & 6 & $(11,3)$ & 5 & 1 & YES & YES & YES & $0.88$ & $(4,1)$ & -- & 605\\
$(14,5)$ & 6 & $(11,5)$ & 6 & 1 & YES & YES & YES & $1.00$ & $(2,2)$ & NO & 606\\
$(14,5)$ & 6 & $(11,5)$ & 6 & 1 & YES & YES & YES & $1.12$ & $(2,2)$ & -- & 607\\
$(14,3)$ & 6 & $(13,4)$ & 6 & 1 & YES & YES & YES & $0.88$ & $(4,1)$ & NO & 608\\
$(14,3)$ & 6 & $(13,4)$ & 6 & 1 & YES & YES & YES & $0.88$ & $(4,1)$ & -- & 609\\
$(14,3)$ & 6 & $(13,6)$ & 7 & 1 & YES & YES & YES & $1.00$ & $(2,2)$ & NO & 610\\
$(14,5)$ & 6 & $(13,3)$ & 6 & 1 & YES & YES & YES & $1.00$ & $(4,1)$ & -- & 611\\
$(14,5)$ & 6 & $(13,6)$ & 7 & 1 & YES & YES & YES & $1.12$ & $(2,2)$ & NO & 612\\
$(15,4)$ & 6 & $(10,3)$ & 5 & 5 & YES & YES & YES & $0.88$ & $(4,1)$ & NO & 613\\
$(15,4)$ & 6 & $(10,3)$ & 5 & 5 & YES & YES & YES & $0.88$ & $(4,1)$ & -- & 614\\
$(15,4)$ & 6 & $(11,2)$ & 6 & 1 & YES & YES & YES & $1.12$ & $(2,2)$ & NO & 615\\
$(15,4)$ & 6 & $(11,2)$ & 6 & 1 & YES & YES & YES & $1.12$ & $(2,2)$ & -- & 616\\
$(15,4)$ & 6 & $(13,3)$ & 6 & 1 & YES & YES & YES & $0.88$ & $(2,2)$ & NO & 617\\
$(15,4)$ & 6 & $(13,3)$ & 6 & 1 & YES & YES & YES & $0.88$ & $(2,2)$ & -- & 618\\
$(16,5)$ & 7 & $(7,3)$ & 4 & 1 & YES & YES & NO(2) & $1.11$ & $(4,1)$ & NO & 619\\
$(16,7)$ & 6 & $(7,2)$ & 4 & 1 & YES & YES & YES & $1.00$ & $(2,2)$ & -- & 620\\
$(16,7)$ & 6 & $(7,3)$ & 4 & 1 & YES & YES & YES & $0.88$ & $(2,2)$ & -- & 621\\
$(16,7)$ & 6 & $(8,3)$ & 4 & 8 & YES & YES & YES & $0.88$ & $(4,1)$ & -- & 622\\
$(16,5)$ & 7 & $(9,4)$ & 5 & 1 & YES & YES & NO(2) & $1.11$ & $(4,1)$ & NO & 623\\
$(16,5)$ & 7 & $(9,4)$ & 5 & 1 & YES & YES & NO(2) & $1.11$ & $(4,1)$ & -- & 624\\
$(16,5)$ & 7 & $(9,4)$ & 5 & 1 & YES & YES & NO(2) & $1.11$ & $(4,1)$ & NO & 625\\
$(16,7)$ & 6 & $(10,3)$ & 5 & 2 & YES & YES & YES & $0.88$ & $(4,1)$ & NO & 626\\
$(16,7)$ & 6 & $(10,3)$ & 5 & 2 & YES & YES & YES & $0.88$ & $(4,1)$ & -- & 627\\
$(16,3)$ & 7 & $(11,5)$ & 6 & 1 & YES & YES & YES & $0.88$ & $(2,2)$ & -- & 628\\
$(16,5)$ & 7 & $(11,3)$ & 5 & 1 & YES & YES & YES & $1.12$ & $(2,2)$ & NO & 629\\
$(16,5)$ & 7 & $(11,3)$ & 5 & 1 & YES & YES & YES & $1.12$ & $(2,2)$ & -- & 630\\
$(16,7)$ & 6 & $(11,4)$ & 5 & 1 & YES & YES & YES & $0.88$ & $(2,2)$ & -- & 631\\
$(16,7)$ & 6 & $(11,4)$ & 5 & 1 & YES & YES & NO(2) & $1.10$ & $(2,2)$ & 725 & 632\\
$(16,7)$ & 6 & $(11,5)$ & 6 & 1 & YES & YES & YES & $1.00$ & $(2,2)$ & -- & 633\\
$(16,5)$ & 7 & $(12,5)$ & 5 & 4 & YES & YES & YES & $0.88$ & $(2,2)$ & -- & 634\\
$(16,5)$ & 7 & $(12,5)$ & 5 & 4 & YES & YES & YES & $0.88$ & $(2,2)$ & NO & 635\\
$(16,5)$ & 7 & $(12,5)$ & 5 & 4 & YES & YES & NO(2) & $1.00$ & $(4,1)$ & NO & 636\\
$(16,7)$ & 6 & $(12,5)$ & 5 & 4 & YES & YES & YES & $0.88$ & $(2,2)$ & -- & 637\\
$(16,7)$ & 6 & $(13,4)$ & 6 & 1 & YES & YES & YES & $0.88$ & $(2,2)$ & -- & 638\\
$(16,7)$ & 6 & $(13,5)$ & 5 & 1 & YES & YES & YES & $1.00$ & $(2,2)$ & NO & 639\\
$(16,7)$ & 6 & $(13,5)$ & 5 & 1 & YES & YES & YES & $1.00$ & $(2,2)$ & -- & 640\\
$(16,7)$ & 6 & $(13,6)$ & 7 & 1 & YES & YES & YES & $0.88$ & $(2,2)$ & 905 & 641\\
$(16,3)$ & 7 & $(14,5)$ & 6 & 2 & YES & YES & YES & $0.88$ & $(2,2)$ & -- & 642\\
$(16,7)$ & 6 & $(14,5)$ & 6 & 2 & YES & YES & YES & $1.00$ & $(2,2)$ & -- & 643\\
$(16,7)$ & 6 & $(14,5)$ & 6 & 2 & YES & YES & YES & $1.12$ & $(2,2)$ & NO & 644\\
$(16,7)$ & 6 & $(15,4)$ & 6 & 1 & YES & YES & YES & $1.12$ & $(2,2)$ & -- & 645\\
$(16,7)$ & 6 & $(15,4)$ & 6 & 1 & YES & YES & YES & $1.11$ & $(2,2)$ & NO & 646\\
$(17,3)$ & 7 & $(7,2)$ & 4 & 1 & YES & YES & YES & $0.88$ & $(2,2)$ & -- & 647\\
$(17,3)$ & 7 & $(7,2)$ & 4 & 1 & YES & YES & YES & $0.88$ & $(2,2)$ & 1058 & 648\\
$(17,3)$ & 7 & $(7,2)$ & 4 & 1 & YES & YES & YES & $1.00$ & $(2,2)$ & NO & 649\\
$(17,5)$ & 6 & $(7,2)$ & 4 & 1 & YES & YES & NO(2) & $0.89$ & $(4,1)$ & NO & 650\\
$(17,5)$ & 6 & $(7,2)$ & 4 & 1 & YES & YES & NO(2) & $0.89$ & $(4,1)$ & -- & 651\\
$(17,7)$ & 6 & $(7,2)$ & 4 & 1 & YES & YES & YES & $0.88$ & $(4,1)$ & -- & 652\\
$(17,7)$ & 6 & $(7,3)$ & 4 & 1 & YES & YES & NO(2) & $1.20$ & $(2,2)$ & -- & 653\\
$(17,4)$ & 7 & $(8,3)$ & 4 & 1 & YES & YES & NO(2) & $1.00$ & $(4,1)$ & NO & 654\\
$(17,4)$ & 7 & $(8,3)$ & 4 & 1 & YES & YES & NO(2) & $1.00$ & $(4,1)$ & NO & 655\\
$(17,4)$ & 7 & $(8,3)$ & 4 & 1 & YES & YES & NO(2) & $1.00$ & $(4,1)$ & -- & 656\\
$(17,5)$ & 6 & $(8,3)$ & 4 & 1 & YES & YES & NO(2) & $0.89$ & $(4,1)$ & NO & 657\\
$(17,5)$ & 6 & $(8,3)$ & 4 & 1 & YES & YES & NO(2) & $0.89$ & $(4,1)$ & -- & 658\\
$(17,5)$ & 6 & $(9,4)$ & 5 & 1 & YES & YES & YES & $1.00$ & $(2,2)$ & -- & 659\\
$(17,6)$ & 7 & $(9,4)$ & 5 & 1 & YES & YES & YES & $1.12$ & $(2,2)$ & NO & 660\\
$(17,6)$ & 7 & $(9,4)$ & 5 & 1 & YES & YES & YES & $1.12$ & $(2,2)$ & -- & 661\\
$(17,7)$ & 6 & $(9,2)$ & 5 & 1 & YES & YES & YES & $0.88$ & $(2,2)$ & -- & 662\\
$(17,7)$ & 6 & $(10,3)$ & 5 & 1 & YES & YES & YES & $1.22$ & $(2,2)$ & NO & 663\\
$(17,7)$ & 6 & $(10,3)$ & 5 & 1 & YES & YES & YES & $1.00$ & $(2,2)$ & NO & 664\\
$(17,7)$ & 6 & $(10,3)$ & 5 & 1 & YES & YES & YES & $1.00$ & $(2,2)$ & -- & 665\\
$(17,5)$ & 6 & $(11,4)$ & 5 & 1 & YES & YES & YES & $1.11$ & $(2,2)$ & NO & 666\\
$(17,5)$ & 6 & $(11,4)$ & 5 & 1 & YES & YES & YES & $1.11$ & $(2,2)$ & -- & 667\\
$(17,5)$ & 6 & $(11,5)$ & 6 & 1 & YES & YES & YES & $0.88$ & $(4,1)$ & -- & 668\\
$(17,6)$ & 7 & $(11,5)$ & 6 & 1 & YES & YES & YES & $1.12$ & $(2,2)$ & NO & 669\\
$(17,6)$ & 7 & $(13,3)$ & 6 & 1 & YES & YES & YES & $0.88$ & $(4,1)$ & NO & 670\\
$(17,6)$ & 7 & $(13,4)$ & 6 & 1 & YES & YES & YES & $1.12$ & $(2,2)$ & NO & 671\\
$(17,7)$ & 6 & $(13,4)$ & 6 & 1 & YES & YES & NO(2) & $1.00$ & $(4,1)$ & NO & 672\\
$(17,3)$ & 7 & $(14,5)$ & 6 & 1 & YES & YES & NO(2) & $1.20$ & $(2,2)$ & NO & 673\\
$(17,3)$ & 7 & $(14,5)$ & 6 & 1 & YES & YES & NO(2) & $1.20$ & $(2,2)$ & -- & 674\\
$(17,4)$ & 7 & $(14,5)$ & 6 & 1 & YES & YES & YES & $1.22$ & $(2,2)$ & NO & 675\\
$(17,5)$ & 6 & $(14,5)$ & 6 & 1 & YES & YES & YES & $1.12$ & $(2,2)$ & -- & 676\\
$(17,6)$ & 7 & $(14,3)$ & 6 & 1 & YES & YES & YES & $0.88$ & $(4,1)$ & NO & 677\\
$(17,7)$ & 6 & $(14,3)$ & 6 & 1 & YES & YES & NO(2) & $0.89$ & $(4,1)$ & NO & 678\\
$(17,7)$ & 6 & $(14,3)$ & 6 & 1 & YES & YES & NO(2) & $1.10$ & $(2,2)$ & -- & 679\\
$(17,5)$ & 6 & $(16,7)$ & 6 & 1 & YES & YES & YES & $1.00$ & $(2,2)$ & -- & 680\\
$(17,7)$ & 6 & $(16,7)$ & 6 & 1 & YES & YES & YES & $0.88$ & $(4,1)$ & NO & 681\\
$(17,7)$ & 6 & $(17,5)$ & 6 & 17 & YES & YES & YES & $1.22$ & $(2,2)$ & -- & 682\\
$(17,7)$ & 6 & $(17,5)$ & 6 & 17 & YES & YES & YES & $1.00$ & $(2,2)$ & NO & 683\\
$(18,7)$ & 6 & $(5,1)$ & 4 & 1 & YES & YES & YES & $1.00$ & $(2,2)$ & NO & 684\\
$(18,7)$ & 6 & $(7,2)$ & 4 & 1 & YES & YES & YES & $1.00$ & $(2,2)$ & NO & 685\\
$(18,7)$ & 6 & $(7,2)$ & 4 & 1 & YES & YES & YES & $1.00$ & $(2,2)$ & -- & 686\\
$(18,7)$ & 6 & $(7,3)$ & 4 & 1 & YES & YES & YES & $1.00$ & $(2,2)$ & -- & 687\\
$(18,5)$ & 6 & $(8,3)$ & 4 & 2 & YES & YES & NO(2) & $0.78$ & $(4,1)$ & -- & 688\\
$(18,5)$ & 6 & $(8,3)$ & 4 & 2 & YES & YES & NO(2) & $0.89$ & $(4,1)$ & NO & 689\\
$(18,5)$ & 6 & $(9,4)$ & 5 & 9 & YES & YES & YES & $1.00$ & $(2,2)$ & NO & 690\\
$(18,5)$ & 6 & $(9,4)$ & 5 & 9 & YES & YES & YES & $1.00$ & $(2,2)$ & -- & 691\\
$(18,5)$ & 6 & $(9,4)$ & 5 & 9 & YES & YES & YES & $1.00$ & $(2,2)$ & NO & 692\\
$(18,7)$ & 6 & $(9,2)$ & 5 & 9 & YES & YES & YES & $0.88$ & $(4,1)$ & NO & 693\\
$(18,7)$ & 6 & $(9,2)$ & 5 & 9 & YES & YES & YES & $0.88$ & $(4,1)$ & -- & 694\\
$(18,7)$ & 6 & $(9,4)$ & 5 & 9 & YES & YES & YES & $1.00$ & $(4,1)$ & NO & 695\\
$(18,7)$ & 6 & $(9,4)$ & 5 & 9 & YES & YES & YES & $1.00$ & $(4,1)$ & -- & 696\\
$(18,7)$ & 6 & $(10,3)$ & 5 & 2 & YES & YES & YES & $1.11$ & $(2,2)$ & -- & 697\\
$(18,5)$ & 6 & $(11,5)$ & 6 & 1 & YES & YES & YES & $0.88$ & $(4,1)$ & NO & 698\\
$(18,5)$ & 6 & $(11,5)$ & 6 & 1 & YES & YES & NO(2) & $1.10$ & $(2,2)$ & -- & 699\\
$(18,7)$ & 6 & $(11,5)$ & 6 & 1 & YES & YES & NO(2) & $1.20$ & $(2,2)$ & -- & 700\\
$(18,7)$ & 6 & $(12,5)$ & 5 & 6 & YES & YES & YES & $1.00$ & $(2,2)$ & -- & 701\\
$(18,5)$ & 6 & $(14,5)$ & 6 & 2 & YES & YES & YES & $1.12$ & $(2,2)$ & -- & 702\\
$(18,5)$ & 6 & $(14,5)$ & 6 & 2 & YES & YES & YES & $0.88$ & $(4,1)$ & NO & 703\\
$(18,7)$ & 6 & $(14,5)$ & 6 & 2 & YES & YES & NO(2) & $0.75$ & $(6,0)$ & NO & 704\\
$(18,7)$ & 6 & $(16,7)$ & 6 & 2 & YES & YES & YES & $0.88$ & $(2,2)$ & NO & 705\\
$(18,5)$ & 6 & $(17,7)$ & 6 & 1 & YES & YES & YES & $1.22$ & $(2,2)$ & -- & 706\\
$(18,7)$ & 6 & $(17,5)$ & 6 & 1 & YES & YES & YES & $1.22$ & $(2,2)$ & -- & 707\\
$(18,7)$ & 6 & $(18,5)$ & 6 & 18 & YES & YES & YES & $1.22$ & $(2,2)$ & -- & 708\\
$(19,4)$ & 7 & $(3,1)$ & 2 & 1 & YES & YES & NO(2) & $1.00$ & $(4,1)$ & NO & 709\\
$(19,8)$ & 6 & $(5,1)$ & 4 & 1 & YES & YES & YES & $1.00$ & $(2,2)$ & -- & 710\\
$(19,8)$ & 6 & $(6,1)$ & 5 & 1 & YES & YES & YES & $1.00$ & $(2,2)$ & NO & 711\\
$(19,8)$ & 6 & $(6,1)$ & 5 & 1 & YES & YES & YES & $1.00$ & $(2,2)$ & -- & 712\\
$(19,8)$ & 6 & $(6,1)$ & 5 & 1 & YES & YES & YES & $1.12$ & $(2,2)$ & NO & 713\\
$(19,5)$ & 7 & $(7,2)$ & 4 & 1 & YES & YES & YES & $1.00$ & $(2,2)$ & NO & 714\\
$(19,5)$ & 7 & $(7,2)$ & 4 & 1 & YES & YES & YES & $1.00$ & $(2,2)$ & -- & 715\\
$(19,6)$ & 8 & $(7,3)$ & 4 & 1 & YES & YES & YES & $1.00$ & $(2,2)$ & NO & 716\\
$(19,6)$ & 8 & $(7,3)$ & 4 & 1 & YES & YES & YES & $1.00$ & $(2,2)$ & -- & 717\\
$(19,7)$ & 6 & $(7,3)$ & 4 & 1 & YES & YES & YES & $0.88$ & $(2,2)$ & -- & 718\\
$(19,7)$ & 6 & $(7,3)$ & 4 & 1 & YES & YES & NO(2) & $1.10$ & $(2,2)$ & NO & 719\\
$(19,8)$ & 6 & $(7,3)$ & 4 & 1 & YES & YES & YES & $0.88$ & $(2,2)$ & -- & 720\\
$(19,7)$ & 6 & $(8,3)$ & 4 & 1 & YES & YES & YES & $0.88$ & $(4,1)$ & -- & 721\\
$(19,6)$ & 8 & $(9,4)$ & 5 & 1 & YES & YES & YES & $1.00$ & $(2,2)$ & NO & 722\\
$(19,7)$ & 6 & $(9,4)$ & 5 & 1 & YES & YES & YES & $0.88$ & $(2,2)$ & NO & 723\\
$(19,7)$ & 6 & $(9,4)$ & 5 & 1 & YES & YES & YES & $0.88$ & $(2,2)$ & -- & 724\\
$(19,7)$ & 6 & $(9,4)$ & 5 & 1 & YES & YES & NO(2) & $1.10$ & $(2,2)$ & 632 & 725\\
$(19,8)$ & 6 & $(9,2)$ & 5 & 1 & YES & YES & YES & $1.00$ & $(2,2)$ & 1179 & 726\\
$(19,8)$ & 6 & $(9,2)$ & 5 & 1 & YES & YES & YES & $1.00$ & $(2,2)$ & -- & 727\\
$(19,8)$ & 6 & $(9,2)$ & 5 & 1 & YES & YES & YES & $1.12$ & $(2,2)$ & NO & 728\\
$(19,8)$ & 6 & $(9,4)$ & 5 & 1 & YES & YES & YES & $0.88$ & $(2,2)$ & NO & 729\\
$(19,8)$ & 6 & $(9,4)$ & 5 & 1 & YES & YES & YES & $0.88$ & $(2,2)$ & -- & 730\\
$(19,7)$ & 6 & $(10,3)$ & 5 & 1 & YES & YES & NO(2) & $1.10$ & $(2,2)$ & NO & 731\\
$(19,7)$ & 6 & $(10,3)$ & 5 & 1 & YES & YES & NO(2) & $1.10$ & $(2,2)$ & -- & 732\\
$(19,4)$ & 7 & $(11,4)$ & 5 & 1 & YES & YES & NO(2) & $1.00$ & $(4,1)$ & -- & 733\\
$(19,6)$ & 8 & $(11,4)$ & 5 & 1 & YES & YES & YES & $1.00$ & $(2,2)$ & NO & 734\\
$(19,7)$ & 6 & $(11,3)$ & 5 & 1 & YES & YES & YES & $0.88$ & $(4,1)$ & -- & 735\\
$(19,8)$ & 6 & $(11,4)$ & 5 & 1 & YES & YES & NO(2) & $1.10$ & $(2,2)$ & NO & 736\\
$(19,7)$ & 6 & $(12,5)$ & 5 & 1 & YES & YES & YES & $1.12$ & $(2,2)$ & -- & 737\\
$(19,7)$ & 6 & $(12,5)$ & 5 & 1 & YES & YES & YES & $1.22$ & $(2,2)$ & NO & 738\\
$(19,8)$ & 6 & $(12,5)$ & 5 & 1 & YES & YES & YES & $0.88$ & $(2,2)$ & -- & 739\\
$(19,8)$ & 6 & $(12,5)$ & 5 & 1 & YES & YES & YES & $1.12$ & $(2,2)$ & NO & 740\\
$(19,3)$ & 8 & $(13,6)$ & 7 & 1 & YES & YES & YES & $1.00$ & $(2,2)$ & NO & 741\\
$(19,6)$ & 8 & $(13,3)$ & 6 & 1 & YES & YES & YES & $1.25$ & $(2,2)$ & -- & 742\\
$(19,6)$ & 8 & $(13,3)$ & 6 & 1 & YES & YES & YES & $1.38$ & $(2,2)$ & NO & 743\\
$(19,7)$ & 6 & $(13,4)$ & 6 & 1 & YES & YES & YES & $1.12$ & $(2,2)$ & NO & 744\\
$(19,7)$ & 6 & $(13,4)$ & 6 & 1 & YES & YES & YES & $1.12$ & $(2,2)$ & -- & 745\\
$(19,8)$ & 6 & $(13,5)$ & 5 & 1 & YES & YES & YES & $0.88$ & $(2,2)$ & -- & 746\\
$(19,8)$ & 6 & $(13,6)$ & 7 & 1 & YES & YES & YES & $1.00$ & $(2,2)$ & 1342 & 747\\
$(19,6)$ & 8 & $(14,3)$ & 6 & 1 & YES & YES & YES & $1.12$ & $(2,2)$ & -- & 748\\
$(19,6)$ & 8 & $(14,3)$ & 6 & 1 & YES & YES & YES & $1.25$ & $(2,2)$ & NO & 749\\
$(19,6)$ & 8 & $(14,3)$ & 6 & 1 & YES & YES & NO(2) & $1.10$ & $(2,2)$ & NO & 750\\
$(19,7)$ & 6 & $(14,5)$ & 6 & 1 & YES & YES & YES & $0.88$ & $(2,2)$ & NO & 751\\
$(19,7)$ & 6 & $(15,4)$ & 6 & 1 & YES & YES & YES & $1.11$ & $(2,2)$ & NO & 752\\
$(19,6)$ & 8 & $(16,3)$ & 7 & 1 & YES & YES & YES & $1.00$ & $(2,2)$ & -- & 753\\
$(19,6)$ & 8 & $(16,3)$ & 7 & 1 & YES & YES & YES & $1.12$ & $(2,2)$ & NO & 754\\
$(19,3)$ & 8 & $(17,6)$ & 7 & 1 & YES & YES & YES & $0.88$ & $(4,1)$ & NO & 755\\
$(19,4)$ & 7 & $(17,4)$ & 7 & 1 & YES & YES & NO(2) & $1.00$ & $(2,2)$ & -- & 756\\
$(19,4)$ & 7 & $(17,7)$ & 6 & 1 & YES & YES & YES & $1.12$ & $(2,2)$ & -- & 757\\
$(19,4)$ & 7 & $(17,7)$ & 6 & 1 & YES & YES & YES & $1.12$ & $(2,2)$ & NO & 758\\
$(19,7)$ & 6 & $(17,4)$ & 7 & 1 & YES & YES & YES & $1.11$ & $(2,2)$ & NO & 759\\
$(19,5)$ & 7 & $(18,5)$ & 6 & 1 & YES & YES & YES & $1.00$ & $(2,2)$ & 1298 & 760\\
$(19,6)$ & 8 & $(18,5)$ & 6 & 1 & YES & YES & NO(2) & $1.10$ & $(2,2)$ & NO & 761\\
$(19,7)$ & 6 & $(18,5)$ & 6 & 1 & YES & YES & YES & $1.22$ & $(2,2)$ & -- & 762\\
$(19,7)$ & 6 & $(18,7)$ & 6 & 1 & YES & YES & YES & $0.88$ & $(4,1)$ & NO & 763\\
$(20,9)$ & 7 & $(5,2)$ & 3 & 5 & YES & YES & NO(2) & $1.20$ & $(2,2)$ & NO & 764\\
$(20,9)$ & 7 & $(5,2)$ & 3 & 5 & YES & YES & NO(2) & $1.20$ & $(2,2)$ & -- & 765\\
$(20,9)$ & 7 & $(7,2)$ & 4 & 1 & YES & YES & YES & $0.88$ & $(2,2)$ & -- & 766\\
$(20,9)$ & 7 & $(8,3)$ & 4 & 4 & YES & YES & YES & $0.88$ & $(4,1)$ & -- & 767\\
$(20,9)$ & 7 & $(9,4)$ & 5 & 1 & YES & YES & YES & $1.12$ & $(2,2)$ & -- & 768\\
$(20,9)$ & 7 & $(10,3)$ & 5 & 10 & YES & YES & YES & $0.88$ & $(4,1)$ & -- & 769\\
$(20,9)$ & 7 & $(10,3)$ & 5 & 10 & YES & YES & YES & $0.88$ & $(2,2)$ & NO & 770\\
$(20,7)$ & 8 & $(11,2)$ & 6 & 1 & YES & YES & YES & $0.88$ & $(2,2)$ & -- & 771\\
$(20,9)$ & 7 & $(11,3)$ & 5 & 1 & YES & YES & YES & $1.00$ & $(2,2)$ & -- & 772\\
$(20,9)$ & 7 & $(11,3)$ & 5 & 1 & YES & YES & YES & $0.88$ & $(4,1)$ & NO & 773\\
$(20,9)$ & 7 & $(11,4)$ & 5 & 1 & YES & YES & YES & $0.88$ & $(2,2)$ & NO & 774\\
$(20,9)$ & 7 & $(12,5)$ & 5 & 4 & YES & YES & YES & $1.00$ & $(2,2)$ & NO & 775\\
$(20,7)$ & 8 & $(13,3)$ & 6 & 1 & YES & YES & YES & $1.12$ & $(2,2)$ & -- & 776\\
$(20,9)$ & 7 & $(13,3)$ & 6 & 1 & YES & YES & NO(2) & $1.10$ & $(2,2)$ & NO & 777\\
$(20,9)$ & 7 & $(13,6)$ & 7 & 1 & YES & YES & YES & $0.88$ & $(2,2)$ & NO & 778\\
$(20,7)$ & 8 & $(14,3)$ & 6 & 2 & YES & YES & YES & $1.00$ & $(2,2)$ & -- & 779\\
$(20,9)$ & 7 & $(14,5)$ & 6 & 2 & YES & YES & YES & $1.12$ & $(2,2)$ & NO & 780\\
$(20,3)$ & 8 & $(17,6)$ & 7 & 1 & YES & YES & NO(2) & $0.75$ & $(6,0)$ & NO & 781\\
$(20,9)$ & 7 & $(17,7)$ & 6 & 1 & YES & YES & YES & $1.00$ & $(2,2)$ & 1126 & 782\\
$(20,7)$ & 8 & $(19,3)$ & 8 & 1 & YES & YES & YES & $1.12$ & $(2,2)$ & -- & 783\\
$(20,9)$ & 7 & $(19,8)$ & 6 & 1 & YES & YES & YES & $1.00$ & $(2,2)$ & NO & 784\\
$(20,7)$ & 8 & $(20,3)$ & 8 & 20 & YES & YES & YES & $1.25$ & $(2,2)$ & -- & 785\\
$(21,4)$ & 8 & $(5,1)$ & 4 & 1 & YES & YES & YES & $0.88$ & $(2,2)$ & -- & 786\\
$(21,4)$ & 8 & $(5,1)$ & 4 & 1 & YES & YES & YES & $1.00$ & $(2,2)$ & NO & 787\\
$(21,8)$ & 6 & $(5,1)$ & 4 & 1 & YES & YES & YES & $1.12$ & $(2,2)$ & NO & 788\\
$(21,8)$ & 6 & $(6,1)$ & 5 & 3 & YES & YES & YES & $1.12$ & $(2,2)$ & NO & 789\\
$(21,8)$ & 6 & $(6,1)$ & 5 & 3 & YES & YES & YES & $1.12$ & $(2,2)$ & NO & 790\\
$(21,8)$ & 6 & $(6,1)$ & 5 & 3 & YES & YES & YES & $1.12$ & $(2,2)$ & -- & 791\\
$(21,5)$ & 8 & $(7,3)$ & 4 & 7 & YES & YES & YES & $1.12$ & $(4,1)$ & -- & 792\\
$(21,8)$ & 6 & $(7,2)$ & 4 & 7 & YES & YES & YES & $1.00$ & $(2,2)$ & -- & 793\\
$(21,8)$ & 6 & $(7,3)$ & 4 & 7 & YES & YES & YES & $1.12$ & $(2,2)$ & -- & 794\\
$(21,8)$ & 6 & $(7,3)$ & 4 & 7 & YES & YES & NO(2) & $1.10$ & $(2,2)$ & NO & 795\\
$(21,8)$ & 6 & $(9,4)$ & 5 & 3 & YES & YES & YES & $1.12$ & $(2,2)$ & NO & 796\\
$(21,8)$ & 6 & $(9,4)$ & 5 & 3 & YES & YES & YES & $1.12$ & $(2,2)$ & -- & 797\\
$(21,8)$ & 6 & $(9,4)$ & 5 & 3 & YES & YES & NO(2) & $1.10$ & $(2,2)$ & NO & 798\\
$(21,8)$ & 6 & $(10,3)$ & 5 & 1 & YES & YES & YES & $1.12$ & $(2,2)$ & NO & 799\\
$(21,8)$ & 6 & $(10,3)$ & 5 & 1 & YES & YES & YES & $1.12$ & $(2,2)$ & -- & 800\\
$(21,5)$ & 8 & $(11,2)$ & 6 & 1 & YES & YES & YES & $1.12$ & $(2,2)$ & NO & 801\\
$(21,5)$ & 8 & $(11,2)$ & 6 & 1 & YES & YES & YES & $1.12$ & $(2,2)$ & -- & 802\\
$(21,5)$ & 8 & $(11,2)$ & 6 & 1 & YES & YES & YES & $1.12$ & $(2,2)$ & NO & 803\\
$(21,8)$ & 6 & $(11,5)$ & 6 & 1 & YES & YES & YES & $0.88$ & $(4,1)$ & NO & 804\\
$(21,8)$ & 6 & $(12,5)$ & 5 & 3 & YES & YES & YES & $1.22$ & $(2,2)$ & -- & 805\\
$(21,8)$ & 6 & $(12,5)$ & 5 & 3 & YES & YES & YES & $1.00$ & $(2,2)$ & NO & 806\\
$(21,4)$ & 8 & $(13,3)$ & 6 & 1 & YES & YES & YES & $1.12$ & $(2,2)$ & NO & 807\\
$(21,4)$ & 8 & $(13,3)$ & 6 & 1 & YES & YES & YES & $1.12$ & $(2,2)$ & -- & 808\\
$(21,8)$ & 6 & $(13,5)$ & 5 & 1 & YES & YES & YES & $1.33$ & $(2,2)$ & -- & 809\\
$(21,8)$ & 6 & $(14,5)$ & 6 & 7 & YES & YES & YES & $0.88$ & $(4,1)$ & NO & 810\\
$(21,5)$ & 8 & $(16,7)$ & 6 & 1 & YES & YES & YES & $1.12$ & $(2,2)$ & NO & 811\\
$(21,5)$ & 8 & $(16,7)$ & 6 & 1 & YES & YES & YES & $1.12$ & $(2,2)$ & -- & 812\\
$(21,8)$ & 6 & $(16,7)$ & 6 & 1 & YES & YES & YES & $1.00$ & $(2,2)$ & NO & 813\\
$(21,8)$ & 6 & $(18,7)$ & 6 & 3 & YES & YES & NO(2) & $1.10$ & $(2,2)$ & NO & 814\\
$(21,5)$ & 8 & $(21,4)$ & 8 & 21 & YES & YES & NO(2) & $0.89$ & $(4,1)$ & NO & 815\\
$(22,9)$ & 7 & $(5,1)$ & 4 & 1 & YES & YES & YES & $1.00$ & $(2,2)$ & -- & 816\\
$(22,9)$ & 7 & $(5,2)$ & 3 & 1 & YES & YES & NO(2) & $1.00$ & $(4,1)$ & -- & 817\\
$(22,7)$ & 9 & $(6,1)$ & 5 & 2 & YES & YES & NO(2) & $1.30$ & $(2,2)$ & NO & 818\\
$(22,7)$ & 9 & $(6,1)$ & 5 & 2 & YES & YES & NO(2) & $1.30$ & $(2,2)$ & -- & 819\\
$(22,7)$ & 9 & $(7,2)$ & 4 & 1 & YES & YES & YES & $1.00$ & $(2,2)$ & -- & 820\\
$(22,9)$ & 7 & $(7,2)$ & 4 & 1 & YES & YES & NO(2) & $1.00$ & $(4,1)$ & NO & 821\\
$(22,9)$ & 7 & $(7,2)$ & 4 & 1 & YES & YES & NO(2) & $1.00$ & $(4,1)$ & -- & 822\\
$(22,5)$ & 7 & $(9,4)$ & 5 & 1 & YES & YES & YES & $1.12$ & $(2,2)$ & NO & 823\\
$(22,5)$ & 7 & $(9,4)$ & 5 & 1 & YES & YES & YES & $1.12$ & $(2,2)$ & -- & 824\\
$(22,9)$ & 7 & $(9,4)$ & 5 & 1 & YES & YES & YES & $1.00$ & $(2,2)$ & -- & 825\\
$(22,5)$ & 7 & $(11,4)$ & 5 & 11 & YES & YES & YES & $0.88$ & $(4,1)$ & -- & 826\\
$(22,5)$ & 7 & $(11,4)$ & 5 & 11 & YES & YES & NO(2) & $0.89$ & $(4,1)$ & NO & 827\\
$(22,5)$ & 7 & $(11,5)$ & 6 & 11 & YES & YES & YES & $1.00$ & $(2,2)$ & NO & 828\\
$(22,7)$ & 9 & $(11,2)$ & 6 & 11 & YES & YES & YES & $1.12$ & $(2,2)$ & NO & 829\\
$(22,7)$ & 9 & $(11,2)$ & 6 & 11 & YES & YES & YES & $1.00$ & $(2,2)$ & -- & 830\\
$(22,9)$ & 7 & $(11,5)$ & 6 & 11 & YES & YES & YES & $1.00$ & $(2,2)$ & NO & 831\\
$(22,7)$ & 9 & $(13,4)$ & 6 & 1 & YES & YES & NO(2) & $1.30$ & $(2,2)$ & NO & 832\\
$(22,5)$ & 7 & $(14,5)$ & 6 & 2 & YES & YES & YES & $0.88$ & $(2,2)$ & NO & 833\\
$(22,9)$ & 7 & $(16,3)$ & 7 & 2 & YES & YES & YES & $0.88$ & $(2,2)$ & NO & 834\\
$(22,9)$ & 7 & $(16,3)$ & 7 & 2 & YES & YES & YES & $0.88$ & $(2,2)$ & -- & 835\\
$(22,9)$ & 7 & $(16,7)$ & 6 & 2 & YES & YES & YES & $1.00$ & $(2,2)$ & 1965 & 836\\
$(22,5)$ & 7 & $(17,7)$ & 6 & 1 & YES & YES & YES & $1.22$ & $(2,2)$ & -- & 837\\
$(22,7)$ & 9 & $(17,5)$ & 6 & 1 & YES & YES & YES & $1.12$ & $(2,2)$ & NO & 838\\
$(22,9)$ & 7 & $(17,3)$ & 7 & 1 & YES & YES & YES & $1.00$ & $(2,2)$ & NO & 839\\
$(22,9)$ & 7 & $(17,3)$ & 7 & 1 & YES & YES & YES & $1.00$ & $(2,2)$ & -- & 840\\
$(22,9)$ & 7 & $(18,7)$ & 6 & 2 & YES & YES & YES & $1.00$ & $(2,2)$ & NO & 841\\
$(22,3)$ & 9 & $(19,6)$ & 8 & 1 & YES & YES & YES & $1.00$ & $(2,2)$ & NO & 842\\
$(23,5)$ & 7 & $(3,1)$ & 2 & 1 & YES & YES & YES & $1.11$ & $(2,2)$ & NO & 843\\
$(23,9)$ & 7 & $(4,1)$ & 3 & 1 & YES & YES & NO(2) & $1.00$ & $(4,1)$ & NO & 844\\
$(23,9)$ & 7 & $(5,1)$ & 4 & 1 & YES & YES & YES & $0.88$ & $(4,1)$ & NO & 845\\
$(23,9)$ & 7 & $(5,1)$ & 4 & 1 & YES & YES & NO(2) & $1.00$ & $(4,1)$ & -- & 846\\
$(23,9)$ & 7 & $(5,1)$ & 4 & 1 & YES & YES & NO(2) & $1.00$ & $(4,1)$ & NO & 847\\
$(23,9)$ & 7 & $(5,2)$ & 3 & 1 & YES & YES & NO(2) & $1.00$ & $(4,1)$ & NO & 848\\
$(23,10)$ & 7 & $(6,1)$ & 5 & 1 & YES & YES & YES & $1.12$ & $(2,2)$ & NO & 849\\
$(23,7)$ & 7 & $(7,3)$ & 4 & 1 & YES & YES & YES & $1.22$ & $(2,2)$ & NO & 850\\
$(23,7)$ & 7 & $(7,3)$ & 4 & 1 & YES & YES & YES & $1.22$ & $(2,2)$ & -- & 851\\
$(23,9)$ & 7 & $(7,2)$ & 4 & 1 & YES & YES & YES & $1.22$ & $(2,2)$ & NO & 852\\
$(23,9)$ & 7 & $(7,3)$ & 4 & 1 & YES & YES & YES & $1.22$ & $(2,2)$ & -- & 853\\
$(23,9)$ & 7 & $(7,3)$ & 4 & 1 & YES & YES & YES & $1.22$ & $(2,2)$ & NO & 854\\
$(23,10)$ & 7 & $(7,2)$ & 4 & 1 & YES & YES & YES & $1.12$ & $(2,2)$ & NO & 855\\
$(23,10)$ & 7 & $(7,2)$ & 4 & 1 & YES & YES & YES & $1.12$ & $(2,2)$ & -- & 856\\
$(23,7)$ & 7 & $(8,3)$ & 4 & 1 & YES & YES & YES & $1.22$ & $(2,2)$ & NO & 857\\
$(23,7)$ & 7 & $(8,3)$ & 4 & 1 & YES & YES & YES & $1.22$ & $(2,2)$ & -- & 858\\
$(23,9)$ & 7 & $(8,3)$ & 4 & 1 & YES & YES & YES & $1.11$ & $(2,2)$ & NO & 859\\
$(23,9)$ & 7 & $(8,3)$ & 4 & 1 & YES & YES & YES & $1.11$ & $(2,2)$ & -- & 860\\
$(23,5)$ & 7 & $(9,4)$ & 5 & 1 & YES & YES & YES & $0.88$ & $(4,1)$ & NO & 861\\
$(23,5)$ & 7 & $(9,4)$ & 5 & 1 & YES & YES & YES & $0.88$ & $(4,1)$ & -- & 862\\
$(23,6)$ & 8 & $(9,4)$ & 5 & 1 & YES & YES & YES & $1.00$ & $(2,2)$ & NO & 863\\
$(23,7)$ & 7 & $(9,4)$ & 5 & 1 & YES & YES & YES & $1.00$ & $(2,2)$ & -- & 864\\
$(23,7)$ & 7 & $(9,4)$ & 5 & 1 & YES & YES & YES & $0.88$ & $(2,2)$ & NO & 865\\
$(23,9)$ & 7 & $(9,2)$ & 5 & 1 & YES & YES & YES & $1.11$ & $(2,2)$ & NO & 866\\
$(23,9)$ & 7 & $(9,2)$ & 5 & 1 & YES & YES & YES & $1.11$ & $(2,2)$ & -- & 867\\
$(23,10)$ & 7 & $(9,2)$ & 5 & 1 & YES & YES & NO(2) & $1.00$ & $(2,2)$ & -- & 868\\
$(23,5)$ & 7 & $(10,3)$ & 5 & 1 & YES & YES & YES & $1.00$ & $(2,2)$ & NO & 869\\
$(23,5)$ & 7 & $(10,3)$ & 5 & 1 & YES & YES & YES & $1.00$ & $(2,2)$ & -- & 870\\
$(23,6)$ & 8 & $(10,3)$ & 5 & 1 & YES & YES & YES & $1.25$ & $(2,2)$ & -- & 871\\
$(23,6)$ & 8 & $(10,3)$ & 5 & 1 & YES & YES & YES & $1.38$ & $(2,2)$ & NO & 872\\
$(23,10)$ & 7 & $(10,3)$ & 5 & 1 & YES & YES & YES & $1.12$ & $(2,2)$ & -- & 873\\
$(23,4)$ & 8 & $(11,5)$ & 6 & 1 & YES & YES & YES & $1.00$ & $(2,2)$ & -- & 874\\
$(23,5)$ & 7 & $(11,4)$ & 5 & 1 & YES & YES & NO(2) & $0.89$ & $(4,1)$ & NO & 875\\
$(23,9)$ & 7 & $(11,3)$ & 5 & 1 & YES & YES & YES & $0.88$ & $(2,2)$ & -- & 876\\
$(23,9)$ & 7 & $(11,4)$ & 5 & 1 & YES & YES & NO(2) & $1.00$ & $(4,1)$ & NO & 877\\
$(23,7)$ & 7 & $(12,5)$ & 5 & 1 & YES & YES & YES & $0.88$ & $(2,2)$ & NO & 878\\
$(23,7)$ & 7 & $(13,5)$ & 5 & 1 & YES & YES & YES & $1.11$ & $(2,2)$ & -- & 879\\
$(23,7)$ & 7 & $(13,5)$ & 5 & 1 & YES & YES & YES & $1.12$ & $(2,2)$ & NO & 880\\
$(23,9)$ & 7 & $(13,3)$ & 6 & 1 & YES & YES & YES & $1.00$ & $(2,2)$ & -- & 881\\
$(23,4)$ & 8 & $(14,5)$ & 6 & 1 & YES & YES & NO(2) & $1.00$ & $(4,1)$ & NO & 882\\
$(23,4)$ & 8 & $(14,5)$ & 6 & 1 & YES & YES & NO(2) & $1.20$ & $(2,2)$ & -- & 883\\
$(23,6)$ & 8 & $(14,3)$ & 6 & 1 & YES & YES & YES & $0.88$ & $(4,1)$ & -- & 884\\
$(23,10)$ & 7 & $(14,3)$ & 6 & 1 & YES & YES & YES & $0.88$ & $(2,2)$ & NO & 885\\
$(23,10)$ & 7 & $(14,3)$ & 6 & 1 & YES & YES & YES & $1.11$ & $(2,2)$ & -- & 886\\
$(23,5)$ & 7 & $(15,4)$ & 6 & 1 & YES & YES & YES & $0.88$ & $(2,2)$ & -- & 887\\
$(23,4)$ & 8 & $(16,5)$ & 7 & 1 & YES & YES & YES & $1.00$ & $(2,2)$ & -- & 888\\
$(23,6)$ & 8 & $(16,3)$ & 7 & 1 & YES & YES & YES & $0.75$ & $(4,1)$ & -- & 889\\
$(23,6)$ & 8 & $(16,3)$ & 7 & 1 & YES & YES & YES & $0.88$ & $(4,1)$ & NO & 890\\
$(23,4)$ & 8 & $(19,5)$ & 7 & 1 & YES & YES & YES & $1.00$ & $(2,2)$ & -- & 891\\
$(23,6)$ & 8 & $(20,3)$ & 8 & 1 & YES & YES & YES & $0.88$ & $(4,1)$ & NO & 892\\
$(23,4)$ & 8 & $(21,5)$ & 8 & 1 & YES & YES & NO(2) & $0.89$ & $(4,1)$ & NO & 893\\
$(24,7)$ & 7 & $(3,1)$ & 2 & 3 & YES & YES & NO(2) & $0.89$ & $(4,1)$ & -- & 894\\
$(24,11)$ & 8 & $(3,1)$ & 2 & 3 & YES & YES & YES & $1.00$ & $(2,2)$ & NO & 895\\
$(24,11)$ & 8 & $(4,1)$ & 3 & 4 & YES & YES & YES & $0.88$ & $(2,2)$ & -- & 896\\
$(24,11)$ & 8 & $(5,1)$ & 4 & 1 & YES & YES & YES & $0.88$ & $(2,2)$ & -- & 897\\
$(24,11)$ & 8 & $(5,2)$ & 3 & 1 & YES & YES & NO(2) & $1.20$ & $(2,2)$ & NO & 898\\
$(24,7)$ & 7 & $(6,1)$ & 5 & 6 & YES & YES & YES & $1.12$ & $(2,2)$ & NO & 899\\
$(24,7)$ & 7 & $(6,1)$ & 5 & 6 & YES & YES & YES & $1.12$ & $(2,2)$ & NO & 900\\
$(24,7)$ & 7 & $(6,1)$ & 5 & 6 & YES & YES & YES & $1.12$ & $(2,2)$ & -- & 901\\
$(24,5)$ & 8 & $(7,2)$ & 4 & 1 & YES & YES & YES & $1.00$ & $(2,2)$ & NO & 902\\
$(24,5)$ & 8 & $(7,2)$ & 4 & 1 & YES & YES & YES & $1.00$ & $(2,2)$ & -- & 903\\
$(24,11)$ & 8 & $(7,2)$ & 4 & 1 & YES & YES & YES & $0.88$ & $(2,2)$ & -- & 904\\
$(24,11)$ & 8 & $(7,3)$ & 4 & 1 & YES & YES & YES & $0.88$ & $(2,2)$ & 641 & 905\\
$(24,5)$ & 8 & $(9,4)$ & 5 & 3 & YES & YES & NO(2) & $1.00$ & $(4,1)$ & -- & 906\\
$(24,7)$ & 7 & $(9,2)$ & 5 & 3 & YES & YES & YES & $1.00$ & $(2,2)$ & NO & 907\\
$(24,7)$ & 7 & $(9,2)$ & 5 & 3 & YES & YES & YES & $1.00$ & $(2,2)$ & -- & 908\\
$(24,7)$ & 7 & $(9,4)$ & 5 & 3 & YES & YES & YES & $1.12$ & $(2,2)$ & NO & 909\\
$(24,7)$ & 7 & $(9,4)$ & 5 & 3 & YES & YES & YES & $1.12$ & $(2,2)$ & -- & 910\\
$(24,11)$ & 8 & $(9,2)$ & 5 & 3 & YES & YES & YES & $0.88$ & $(2,2)$ & NO & 911\\
$(24,7)$ & 7 & $(10,3)$ & 5 & 2 & YES & YES & NO(2) & $0.89$ & $(4,1)$ & 1022 & 912\\
$(24,5)$ & 8 & $(11,4)$ & 5 & 1 & YES & YES & NO(2) & $1.00$ & $(4,1)$ & -- & 913\\
$(24,5)$ & 8 & $(11,5)$ & 6 & 1 & YES & YES & YES & $1.25$ & $(2,2)$ & NO & 914\\
$(24,5)$ & 8 & $(11,5)$ & 6 & 1 & YES & YES & YES & $1.25$ & $(2,2)$ & -- & 915\\
$(24,11)$ & 8 & $(12,5)$ & 5 & 12 & YES & YES & YES & $0.88$ & $(2,2)$ & NO & 916\\
$(24,5)$ & 8 & $(13,4)$ & 6 & 1 & YES & YES & YES & $1.11$ & $(2,2)$ & -- & 917\\
$(24,7)$ & 7 & $(13,4)$ & 6 & 1 & YES & YES & YES & $1.22$ & $(2,2)$ & -- & 918\\
$(24,5)$ & 8 & $(15,4)$ & 6 & 3 & YES & YES & YES & $1.00$ & $(2,2)$ & -- & 919\\
$(24,7)$ & 7 & $(15,4)$ & 6 & 3 & YES & YES & YES & $1.22$ & $(2,2)$ & -- & 920\\
$(24,11)$ & 8 & $(16,7)$ & 6 & 8 & YES & YES & YES & $0.88$ & $(2,2)$ & NO & 921\\
$(24,7)$ & 7 & $(19,6)$ & 8 & 1 & YES & YES & YES & $1.12$ & $(2,2)$ & NO & 922\\
$(24,5)$ & 8 & $(21,5)$ & 8 & 3 & YES & YES & NO(2) & $0.89$ & $(4,1)$ & NO & 923\\
$(25,7)$ & 7 & $(3,1)$ & 2 & 1 & YES & YES & NO(2) & $0.89$ & $(4,1)$ & -- & 924\\
$(25,9)$ & 7 & $(3,1)$ & 2 & 1 & YES & YES & YES & $1.00$ & $(2,2)$ & -- & 925\\
$(25,9)$ & 7 & $(4,1)$ & 3 & 1 & YES & YES & YES & $1.00$ & $(2,2)$ & NO & 926\\
$(25,9)$ & 7 & $(4,1)$ & 3 & 1 & YES & YES & YES & $1.00$ & $(2,2)$ & -- & 927\\
$(25,7)$ & 7 & $(5,2)$ & 3 & 5 & YES & YES & YES & $1.11$ & $(2,2)$ & NO & 928\\
$(25,7)$ & 7 & $(5,2)$ & 3 & 5 & YES & YES & YES & $1.11$ & $(2,2)$ & -- & 929\\
$(25,9)$ & 7 & $(5,2)$ & 3 & 5 & YES & YES & YES & $0.88$ & $(4,1)$ & -- & 930\\
$(25,9)$ & 7 & $(5,2)$ & 3 & 5 & YES & YES & YES & $1.00$ & $(2,2)$ & NO & 931\\
$(25,11)$ & 7 & $(5,1)$ & 4 & 5 & YES & YES & YES & $1.00$ & $(2,2)$ & NO & 932\\
$(25,11)$ & 7 & $(5,2)$ & 3 & 5 & YES & YES & YES & $1.12$ & $(2,2)$ & NO & 933\\
$(25,11)$ & 7 & $(5,2)$ & 3 & 5 & YES & YES & YES & $1.12$ & $(2,2)$ & -- & 934\\
$(25,9)$ & 7 & $(6,1)$ & 5 & 1 & YES & YES & YES & $1.12$ & $(2,2)$ & NO & 935\\
$(25,6)$ & 9 & $(7,3)$ & 4 & 1 & YES & YES & YES & $0.88$ & $(2,2)$ & NO & 936\\
$(25,9)$ & 7 & $(7,2)$ & 4 & 1 & YES & YES & YES & $1.12$ & $(2,2)$ & -- & 937\\
$(25,9)$ & 7 & $(7,2)$ & 4 & 1 & YES & YES & YES & $0.88$ & $(2,2)$ & NO & 938\\
$(25,9)$ & 7 & $(7,3)$ & 4 & 1 & YES & YES & NO(2) & $1.10$ & $(2,2)$ & NO & 939\\
$(25,11)$ & 7 & $(7,2)$ & 4 & 1 & YES & YES & YES & $0.88$ & $(2,2)$ & -- & 940\\
$(25,11)$ & 7 & $(7,2)$ & 4 & 1 & YES & YES & NO(2) & $1.00$ & $(2,2)$ & NO & 941\\
$(25,11)$ & 7 & $(7,3)$ & 4 & 1 & YES & YES & YES & $1.00$ & $(2,2)$ & -- & 942\\
$(25,11)$ & 7 & $(7,3)$ & 4 & 1 & YES & YES & YES & $1.00$ & $(2,2)$ & NO & 943\\
$(25,11)$ & 7 & $(8,3)$ & 4 & 1 & YES & YES & YES & $1.00$ & $(2,2)$ & NO & 944\\
$(25,11)$ & 7 & $(8,3)$ & 4 & 1 & YES & YES & YES & $1.00$ & $(2,2)$ & -- & 945\\
$(25,11)$ & 7 & $(8,3)$ & 4 & 1 & YES & YES & YES & $1.00$ & $(2,2)$ & 1531 & 946\\
$(25,9)$ & 7 & $(9,4)$ & 5 & 1 & YES & YES & YES & $1.12$ & $(2,2)$ & NO & 947\\
$(25,9)$ & 7 & $(9,4)$ & 5 & 1 & YES & YES & YES & $1.12$ & $(2,2)$ & -- & 948\\
$(25,11)$ & 7 & $(9,2)$ & 5 & 1 & YES & YES & NO(2) & $1.00$ & $(2,2)$ & NO & 949\\
$(25,11)$ & 7 & $(9,4)$ & 5 & 1 & YES & YES & YES & $1.00$ & $(2,2)$ & -- & 950\\
$(25,9)$ & 7 & $(10,3)$ & 5 & 5 & YES & YES & YES & $1.12$ & $(2,2)$ & -- & 951\\
$(25,9)$ & 7 & $(10,3)$ & 5 & 5 & YES & YES & YES & $0.88$ & $(2,2)$ & NO & 952\\
$(25,11)$ & 7 & $(10,3)$ & 5 & 5 & YES & YES & YES & $1.12$ & $(2,2)$ & -- & 953\\
$(25,7)$ & 7 & $(11,3)$ & 5 & 1 & YES & YES & NO(2) & $0.89$ & $(4,1)$ & 1092 & 954\\
$(25,9)$ & 7 & $(11,3)$ & 5 & 1 & YES & YES & YES & $0.75$ & $(4,1)$ & NO & 955\\
$(25,11)$ & 7 & $(11,3)$ & 5 & 1 & YES & YES & YES & $1.00$ & $(2,2)$ & NO & 956\\
$(25,7)$ & 7 & $(13,4)$ & 6 & 1 & YES & YES & YES & $1.33$ & $(2,2)$ & -- & 957\\
$(25,9)$ & 7 & $(13,3)$ & 6 & 1 & YES & YES & YES & $0.75$ & $(4,1)$ & NO & 958\\
$(25,9)$ & 7 & $(13,5)$ & 5 & 1 & YES & YES & YES & $1.00$ & $(2,2)$ & NO & 959\\
$(25,11)$ & 7 & $(13,3)$ & 6 & 1 & YES & YES & YES & $1.00$ & $(2,2)$ & NO & 960\\
$(25,11)$ & 7 & $(13,3)$ & 6 & 1 & YES & YES & YES & $1.11$ & $(2,2)$ & -- & 961\\
$(25,11)$ & 7 & $(13,5)$ & 5 & 1 & YES & YES & YES & $1.00$ & $(2,2)$ & NO & 962\\
$(25,7)$ & 7 & $(15,4)$ & 6 & 5 & YES & YES & YES & $1.22$ & $(2,2)$ & -- & 963\\
$(25,7)$ & 7 & $(15,4)$ & 6 & 5 & YES & YES & YES & $0.88$ & $(4,1)$ & 1629 & 964\\
$(25,11)$ & 7 & $(17,7)$ & 6 & 1 & YES & YES & YES & $1.00$ & $(2,2)$ & NO & 965\\
$(25,6)$ & 9 & $(19,3)$ & 8 & 1 & YES & YES & YES & $1.00$ & $(2,2)$ & NO & 966\\
$(25,6)$ & 9 & $(20,3)$ & 8 & 5 & YES & YES & YES & $0.88$ & $(4,1)$ & NO & 967\\
$(25,9)$ & 7 & $(20,7)$ & 8 & 5 & YES & YES & YES & $0.88$ & $(2,2)$ & 1422 & 968\\
$(25,6)$ & 9 & $(21,4)$ & 8 & 1 & YES & YES & YES & $1.12$ & $(2,2)$ & NO & 969\\
$(25,9)$ & 7 & $(21,8)$ & 6 & 1 & YES & YES & YES & $1.00$ & $(2,2)$ & NO & 970\\
$(25,11)$ & 7 & $(23,10)$ & 7 & 1 & YES & YES & NO(2) & $1.00$ & $(2,2)$ & NO & 971\\
$(25,6)$ & 9 & $(24,5)$ & 8 & 1 & YES & YES & YES & $1.12$ & $(2,2)$ & NO & 972\\
$(26,7)$ & 7 & $(3,1)$ & 2 & 1 & YES & YES & YES & $1.00$ & $(2,2)$ & NO & 973\\
$(26,7)$ & 7 & $(3,1)$ & 2 & 1 & YES & YES & YES & $1.00$ & $(2,2)$ & -- & 974\\
$(26,11)$ & 7 & $(4,1)$ & 3 & 2 & YES & YES & NO(2) & $1.10$ & $(2,2)$ & -- & 975\\
$(26,5)$ & 9 & $(5,2)$ & 3 & 1 & YES & YES & YES & $1.25$ & $(2,2)$ & NO & 976\\
$(26,7)$ & 7 & $(5,1)$ & 4 & 1 & YES & YES & YES & $1.00$ & $(2,2)$ & NO & 977\\
$(26,7)$ & 7 & $(5,1)$ & 4 & 1 & YES & YES & YES & $1.00$ & $(2,2)$ & -- & 978\\
$(26,7)$ & 7 & $(5,1)$ & 4 & 1 & YES & YES & YES & $1.00$ & $(2,2)$ & NO & 979\\
$(26,11)$ & 7 & $(5,1)$ & 4 & 1 & YES & YES & YES & $1.00$ & $(4,1)$ & NO & 980\\
$(26,11)$ & 7 & $(5,1)$ & 4 & 1 & YES & YES & YES & $1.00$ & $(4,1)$ & -- & 981\\
$(26,11)$ & 7 & $(5,2)$ & 3 & 1 & YES & YES & NO(2) & $0.89$ & $(4,1)$ & -- & 982\\
$(26,9)$ & 10 & $(6,1)$ & 5 & 2 & YES & YES & YES & $1.12$ & $(2,2)$ & -- & 983\\
$(26,7)$ & 7 & $(7,2)$ & 4 & 1 & YES & YES & YES & $1.00$ & $(2,2)$ & NO & 984\\
$(26,7)$ & 7 & $(7,2)$ & 4 & 1 & YES & YES & YES & $1.00$ & $(2,2)$ & -- & 985\\
$(26,9)$ & 10 & $(7,1)$ & 6 & 1 & YES & YES & YES & $1.12$ & $(2,2)$ & NO & 986\\
$(26,11)$ & 7 & $(7,2)$ & 4 & 1 & YES & YES & YES & $1.11$ & $(2,2)$ & NO & 987\\
$(26,11)$ & 7 & $(7,2)$ & 4 & 1 & YES & YES & YES & $1.11$ & $(2,2)$ & -- & 988\\
$(26,11)$ & 7 & $(7,3)$ & 4 & 1 & YES & YES & YES & $1.00$ & $(2,2)$ & -- & 989\\
$(26,11)$ & 7 & $(7,3)$ & 4 & 1 & YES & YES & YES & $1.00$ & $(4,1)$ & NO & 990\\
$(26,11)$ & 7 & $(8,3)$ & 4 & 2 & YES & YES & YES & $1.00$ & $(2,2)$ & NO & 991\\
$(26,11)$ & 7 & $(8,3)$ & 4 & 2 & YES & YES & YES & $1.00$ & $(2,2)$ & -- & 992\\
$(26,11)$ & 7 & $(9,2)$ & 5 & 1 & YES & YES & YES & $0.88$ & $(2,2)$ & NO & 993\\
$(26,11)$ & 7 & $(10,3)$ & 5 & 2 & YES & YES & YES & $1.22$ & $(2,2)$ & -- & 994\\
$(26,7)$ & 7 & $(11,3)$ & 5 & 1 & YES & YES & YES & $0.88$ & $(2,2)$ & -- & 995\\
$(26,11)$ & 7 & $(11,3)$ & 5 & 1 & YES & YES & YES & $1.22$ & $(2,2)$ & -- & 996\\
$(26,11)$ & 7 & $(11,3)$ & 5 & 1 & YES & YES & YES & $1.00$ & $(2,2)$ & NO & 997\\
$(26,7)$ & 7 & $(13,5)$ & 5 & 13 & YES & YES & YES & $1.11$ & $(2,2)$ & -- & 998\\
$(26,11)$ & 7 & $(13,3)$ & 6 & 13 & YES & YES & YES & $1.22$ & $(2,2)$ & NO & 999\\
$(26,11)$ & 7 & $(13,3)$ & 6 & 13 & YES & YES & YES & $1.22$ & $(2,2)$ & -- & 1000\\
$(26,9)$ & 10 & $(14,5)$ & 6 & 2 & YES & YES & YES & $1.12$ & $(2,2)$ & NO & 1001\\
$(26,11)$ & 7 & $(16,3)$ & 7 & 2 & YES & YES & YES & $0.88$ & $(2,2)$ & NO & 1002\\
$(26,9)$ & 10 & $(17,6)$ & 7 & 1 & YES & YES & YES & $1.12$ & $(2,2)$ & NO & 1003\\
$(26,11)$ & 7 & $(19,8)$ & 6 & 1 & YES & YES & NO(2) & $1.10$ & $(2,2)$ & NO & 1004\\
$(27,5)$ & 8 & $(2,1)$ & 1 & 1 & YES & YES & YES & $0.88$ & $(2,2)$ & -- & 1005\\
$(27,5)$ & 8 & $(3,1)$ & 2 & 3 & YES & YES & YES & $1.00$ & $(2,2)$ & NO & 1006\\
$(27,5)$ & 8 & $(3,1)$ & 2 & 3 & YES & YES & YES & $1.00$ & $(2,2)$ & -- & 1007\\
$(27,8)$ & 7 & $(3,1)$ & 2 & 3 & YES & YES & YES & $1.00$ & $(2,2)$ & -- & 1008\\
$(27,10)$ & 7 & $(3,1)$ & 2 & 3 & YES & YES & NO(2) & $0.89$ & $(4,1)$ & -- & 1009\\
$(27,8)$ & 7 & $(4,1)$ & 3 & 1 & YES & YES & YES & $1.00$ & $(2,2)$ & -- & 1010\\
$(27,11)$ & 8 & $(4,1)$ & 3 & 1 & YES & YES & NO(2) & $1.00$ & $(4,1)$ & -- & 1011\\
$(27,8)$ & 7 & $(5,2)$ & 3 & 1 & YES & YES & YES & $1.00$ & $(2,2)$ & NO & 1012\\
$(27,8)$ & 7 & $(5,2)$ & 3 & 1 & YES & YES & YES & $1.00$ & $(2,2)$ & -- & 1013\\
$(27,10)$ & 7 & $(5,1)$ & 4 & 1 & YES & YES & NO(2) & $0.75$ & $(6,0)$ & 569 & 1014\\
$(27,10)$ & 7 & $(5,2)$ & 3 & 1 & YES & YES & NO(2) & $0.89$ & $(4,1)$ & NO & 1015\\
$(27,10)$ & 7 & $(5,2)$ & 3 & 1 & YES & YES & NO(2) & $0.89$ & $(4,1)$ & -- & 1016\\
$(27,11)$ & 8 & $(5,1)$ & 4 & 1 & YES & YES & YES & $1.00$ & $(2,2)$ & -- & 1017\\
$(27,11)$ & 8 & $(5,2)$ & 3 & 1 & YES & YES & YES & $1.22$ & $(2,2)$ & -- & 1018\\
$(27,5)$ & 8 & $(6,1)$ & 5 & 3 & YES & YES & YES & $1.00$ & $(2,2)$ & NO & 1019\\
$(27,5)$ & 8 & $(6,1)$ & 5 & 3 & YES & YES & YES & $1.00$ & $(2,2)$ & -- & 1020\\
$(27,5)$ & 8 & $(6,1)$ & 5 & 3 & YES & YES & YES & $1.00$ & $(2,2)$ & NO & 1021\\
$(27,8)$ & 7 & $(7,2)$ & 4 & 1 & YES & YES & NO(2) & $0.89$ & $(4,1)$ & 912 & 1022\\
$(27,10)$ & 7 & $(7,2)$ & 4 & 1 & YES & YES & YES & $1.22$ & $(2,2)$ & NO & 1023\\
$(27,10)$ & 7 & $(7,2)$ & 4 & 1 & YES & YES & YES & $1.22$ & $(2,2)$ & -- & 1024\\
$(27,10)$ & 7 & $(7,3)$ & 4 & 1 & YES & YES & NO(2) & $1.20$ & $(2,2)$ & NO & 1025\\
$(27,10)$ & 7 & $(7,3)$ & 4 & 1 & YES & YES & NO(2) & $1.20$ & $(2,2)$ & -- & 1026\\
$(27,11)$ & 8 & $(7,2)$ & 4 & 1 & YES & YES & YES & $1.00$ & $(2,2)$ & -- & 1027\\
$(27,11)$ & 8 & $(7,2)$ & 4 & 1 & YES & YES & NO(2) & $1.00$ & $(4,1)$ & NO & 1028\\
$(27,5)$ & 8 & $(9,4)$ & 5 & 9 & YES & YES & YES & $0.88$ & $(2,2)$ & -- & 1029\\
$(27,8)$ & 7 & $(9,4)$ & 5 & 9 & YES & YES & YES & $1.00$ & $(2,2)$ & -- & 1030\\
$(27,8)$ & 7 & $(9,4)$ & 5 & 9 & YES & YES & YES & $1.00$ & $(2,2)$ & NO & 1031\\
$(27,10)$ & 7 & $(9,2)$ & 5 & 9 & YES & YES & NO(2) & $1.10$ & $(2,2)$ & -- & 1032\\
$(27,11)$ & 8 & $(9,2)$ & 5 & 9 & YES & YES & YES & $1.00$ & $(2,2)$ & -- & 1033\\
$(27,11)$ & 8 & $(9,2)$ & 5 & 9 & YES & YES & YES & $1.00$ & $(2,2)$ & NO & 1034\\
$(27,11)$ & 8 & $(9,4)$ & 5 & 9 & YES & YES & NO(2) & $1.00$ & $(4,1)$ & NO & 1035\\
$(27,10)$ & 7 & $(10,3)$ & 5 & 1 & YES & YES & YES & $1.00$ & $(2,2)$ & -- & 1036\\
$(27,5)$ & 8 & $(11,5)$ & 6 & 1 & YES & YES & YES & $0.88$ & $(2,2)$ & -- & 1037\\
$(27,5)$ & 8 & $(11,5)$ & 6 & 1 & YES & YES & YES & $1.00$ & $(2,2)$ & NO & 1038\\
$(27,7)$ & 9 & $(11,2)$ & 6 & 1 & YES & YES & YES & $1.12$ & $(2,2)$ & NO & 1039\\
$(27,8)$ & 7 & $(11,3)$ & 5 & 1 & YES & YES & YES & $1.00$ & $(2,2)$ & NO & 1040\\
$(27,10)$ & 7 & $(11,4)$ & 5 & 1 & YES & YES & NO(2) & $0.89$ & $(4,1)$ & 1156 & 1041\\
$(27,11)$ & 8 & $(11,2)$ & 6 & 1 & YES & YES & YES & $1.00$ & $(2,2)$ & -- & 1042\\
$(27,10)$ & 7 & $(13,4)$ & 6 & 1 & YES & YES & YES & $1.25$ & $(2,2)$ & NO & 1043\\
$(27,5)$ & 8 & $(14,5)$ & 6 & 1 & YES & YES & YES & $0.88$ & $(2,2)$ & -- & 1044\\
$(27,5)$ & 8 & $(14,5)$ & 6 & 1 & YES & YES & YES & $1.00$ & $(2,2)$ & NO & 1045\\
$(27,7)$ & 9 & $(15,2)$ & 8 & 3 & YES & YES & YES & $1.00$ & $(2,2)$ & NO & 1046\\
$(27,8)$ & 7 & $(17,5)$ & 6 & 1 & YES & YES & YES & $1.00$ & $(2,2)$ & NO & 1047\\
$(27,10)$ & 7 & $(17,6)$ & 7 & 1 & YES & YES & NO(2) & $0.75$ & $(6,0)$ & NO & 1048\\
$(27,11)$ & 8 & $(17,7)$ & 6 & 1 & YES & YES & YES & $1.00$ & $(2,2)$ & 1476 & 1049\\
$(27,11)$ & 8 & $(19,8)$ & 6 & 1 & YES & YES & YES & $1.00$ & $(2,2)$ & NO & 1050\\
$(27,10)$ & 7 & $(20,7)$ & 8 & 1 & YES & YES & YES & $1.25$ & $(2,2)$ & NO & 1051\\
$(27,5)$ & 8 & $(21,5)$ & 8 & 3 & YES & YES & YES & $1.00$ & $(2,2)$ & 2260 & 1052\\
$(27,5)$ & 8 & $(24,5)$ & 8 & 3 & YES & YES & YES & $1.00$ & $(2,2)$ & NO & 1053\\
$(27,10)$ & 7 & $(25,9)$ & 7 & 1 & YES & YES & NO(2) & $1.10$ & $(2,2)$ & NO & 1054\\
$(27,8)$ & 7 & $(27,8)$ & 7 & 27 & YES & YES & YES & $1.00$ & $(2,2)$ & NO & 1055\\
$(28,5)$ & 8 & $(2,1)$ & 1 & 2 & YES & YES & YES & $0.88$ & $(2,2)$ & NO & 1056\\
$(28,5)$ & 8 & $(2,1)$ & 1 & 2 & YES & YES & YES & $0.88$ & $(2,2)$ & -- & 1057\\
$(28,5)$ & 8 & $(3,1)$ & 2 & 1 & YES & YES & YES & $0.88$ & $(2,2)$ & 648 & 1058\\
$(28,5)$ & 8 & $(3,1)$ & 2 & 1 & YES & YES & YES & $0.88$ & $(2,2)$ & NO & 1059\\
$(28,5)$ & 8 & $(3,1)$ & 2 & 1 & YES & YES & YES & $0.88$ & $(2,2)$ & -- & 1060\\
$(28,11)$ & 8 & $(5,1)$ & 4 & 1 & YES & YES & YES & $1.00$ & $(2,2)$ & -- & 1061\\
$(28,11)$ & 8 & $(5,2)$ & 3 & 1 & YES & YES & YES & $1.11$ & $(2,2)$ & NO & 1062\\
$(28,11)$ & 8 & $(5,2)$ & 3 & 1 & YES & YES & YES & $1.11$ & $(2,2)$ & -- & 1063\\
$(28,11)$ & 8 & $(7,2)$ & 4 & 7 & YES & YES & YES & $1.22$ & $(2,2)$ & -- & 1064\\
$(28,11)$ & 8 & $(9,2)$ & 5 & 1 & YES & YES & YES & $1.00$ & $(2,2)$ & -- & 1065\\
$(28,11)$ & 8 & $(9,2)$ & 5 & 1 & YES & YES & YES & $1.12$ & $(2,2)$ & NO & 1066\\
$(28,5)$ & 8 & $(11,5)$ & 6 & 1 & YES & YES & YES & $0.88$ & $(4,1)$ & NO & 1067\\
$(28,5)$ & 8 & $(11,5)$ & 6 & 1 & YES & YES & NO(2) & $1.10$ & $(2,2)$ & -- & 1068\\
$(28,11)$ & 8 & $(11,2)$ & 6 & 1 & YES & YES & YES & $1.11$ & $(2,2)$ & -- & 1069\\
$(28,5)$ & 8 & $(14,5)$ & 6 & 14 & YES & YES & YES & $0.88$ & $(2,2)$ & NO & 1070\\
$(28,5)$ & 8 & $(14,5)$ & 6 & 14 & YES & YES & YES & $0.88$ & $(2,2)$ & -- & 1071\\
$(28,5)$ & 8 & $(17,3)$ & 7 & 1 & YES & YES & YES & $0.88$ & $(2,2)$ & NO & 1072\\
$(28,11)$ & 8 & $(18,7)$ & 6 & 2 & YES & YES & YES & $1.00$ & $(2,2)$ & 1554 & 1073\\
$(28,5)$ & 8 & $(21,5)$ & 8 & 7 & YES & YES & YES & $1.00$ & $(2,2)$ & NO & 1074\\
$(28,11)$ & 8 & $(21,8)$ & 6 & 7 & YES & YES & YES & $1.12$ & $(2,2)$ & NO & 1075\\
$(29,8)$ & 7 & $(3,1)$ & 2 & 1 & YES & YES & YES & $1.00$ & $(2,2)$ & -- & 1076\\
$(29,11)$ & 7 & $(3,1)$ & 2 & 1 & YES & YES & NO(2) & $0.89$ & $(4,1)$ & -- & 1077\\
$(29,8)$ & 7 & $(4,1)$ & 3 & 1 & YES & YES & YES & $1.00$ & $(2,2)$ & -- & 1078\\
$(29,8)$ & 7 & $(4,1)$ & 3 & 1 & YES & YES & YES & $1.11$ & $(2,2)$ & NO & 1079\\
$(29,11)$ & 7 & $(4,1)$ & 3 & 1 & YES & YES & YES & $1.00$ & $(4,1)$ & -- & 1080\\
$(29,11)$ & 7 & $(4,1)$ & 3 & 1 & YES & YES & YES & $1.11$ & $(2,2)$ & NO & 1081\\
$(29,8)$ & 7 & $(5,2)$ & 3 & 1 & YES & YES & YES & $1.12$ & $(2,2)$ & -- & 1082\\
$(29,8)$ & 7 & $(5,2)$ & 3 & 1 & YES & YES & YES & $1.25$ & $(2,2)$ & NO & 1083\\
$(29,9)$ & 8 & $(5,2)$ & 3 & 1 & YES & YES & YES & $1.00$ & $(2,2)$ & NO & 1084\\
$(29,9)$ & 8 & $(5,2)$ & 3 & 1 & YES & YES & YES & $1.00$ & $(2,2)$ & -- & 1085\\
$(29,11)$ & 7 & $(5,1)$ & 4 & 1 & YES & YES & YES & $1.11$ & $(2,2)$ & NO & 1086\\
$(29,11)$ & 7 & $(5,1)$ & 4 & 1 & YES & YES & YES & $1.11$ & $(2,2)$ & -- & 1087\\
$(29,11)$ & 7 & $(5,1)$ & 4 & 1 & YES & YES & YES & $1.11$ & $(2,2)$ & NO & 1088\\
$(29,11)$ & 7 & $(5,2)$ & 3 & 1 & YES & YES & YES & $1.11$ & $(2,2)$ & -- & 1089\\
$(29,12)$ & 7 & $(5,2)$ & 3 & 1 & YES & YES & YES & $1.11$ & $(2,2)$ & -- & 1090\\
$(29,13)$ & 8 & $(5,2)$ & 3 & 1 & YES & YES & YES & $1.00$ & $(2,2)$ & -- & 1091\\
$(29,8)$ & 7 & $(7,2)$ & 4 & 1 & YES & YES & NO(2) & $0.89$ & $(4,1)$ & 954 & 1092\\
$(29,8)$ & 7 & $(7,3)$ & 4 & 1 & YES & YES & YES & $0.88$ & $(2,2)$ & -- & 1093\\
$(29,9)$ & 8 & $(7,2)$ & 4 & 1 & YES & YES & YES & $1.25$ & $(2,2)$ & NO & 1094\\
$(29,9)$ & 8 & $(7,2)$ & 4 & 1 & YES & YES & YES & $1.25$ & $(2,2)$ & -- & 1095\\
$(29,9)$ & 8 & $(7,3)$ & 4 & 1 & YES & YES & YES & $1.00$ & $(2,2)$ & -- & 1096\\
$(29,9)$ & 8 & $(7,3)$ & 4 & 1 & YES & YES & YES & $1.00$ & $(2,2)$ & NO & 1097\\
$(29,9)$ & 8 & $(7,3)$ & 4 & 1 & YES & YES & YES & $1.00$ & $(2,2)$ & NO & 1098\\
$(29,11)$ & 7 & $(7,2)$ & 4 & 1 & YES & YES & YES & $1.11$ & $(2,2)$ & NO & 1099\\
$(29,11)$ & 7 & $(7,2)$ & 4 & 1 & YES & YES & YES & $1.11$ & $(2,2)$ & -- & 1100\\
$(29,11)$ & 7 & $(7,3)$ & 4 & 1 & YES & YES & NO(2) & $0.89$ & $(4,1)$ & 1526 & 1101\\
$(29,12)$ & 7 & $(7,2)$ & 4 & 1 & YES & YES & YES & $0.75$ & $(4,1)$ & -- & 1102\\
$(29,12)$ & 7 & $(7,2)$ & 4 & 1 & YES & YES & YES & $1.11$ & $(2,2)$ & NO & 1103\\
$(29,12)$ & 7 & $(7,3)$ & 4 & 1 & YES & YES & YES & $1.00$ & $(2,2)$ & -- & 1104\\
$(29,13)$ & 8 & $(7,2)$ & 4 & 1 & YES & YES & YES & $1.12$ & $(2,2)$ & NO & 1105\\
$(29,13)$ & 8 & $(7,2)$ & 4 & 1 & YES & YES & YES & $1.12$ & $(2,2)$ & -- & 1106\\
$(29,13)$ & 8 & $(7,3)$ & 4 & 1 & YES & YES & YES & $1.00$ & $(2,2)$ & -- & 1107\\
$(29,9)$ & 8 & $(8,3)$ & 4 & 1 & YES & YES & YES & $0.88$ & $(2,2)$ & NO & 1108\\
$(29,13)$ & 8 & $(8,3)$ & 4 & 1 & YES & YES & YES & $1.12$ & $(2,2)$ & -- & 1109\\
$(29,13)$ & 8 & $(8,3)$ & 4 & 1 & YES & YES & YES & $1.25$ & $(2,2)$ & NO & 1110\\
$(29,8)$ & 7 & $(9,2)$ & 5 & 1 & YES & YES & YES & $1.12$ & $(2,2)$ & NO & 1111\\
$(29,8)$ & 7 & $(9,2)$ & 5 & 1 & YES & YES & YES & $1.12$ & $(2,2)$ & -- & 1112\\
$(29,8)$ & 7 & $(9,2)$ & 5 & 1 & YES & YES & YES & $1.25$ & $(2,2)$ & NO & 1113\\
$(29,12)$ & 7 & $(9,4)$ & 5 & 1 & YES & YES & YES & $1.12$ & $(2,2)$ & NO & 1114\\
$(29,13)$ & 8 & $(9,2)$ & 5 & 1 & YES & YES & YES & $0.88$ & $(4,1)$ & NO & 1115\\
$(29,13)$ & 8 & $(9,2)$ & 5 & 1 & YES & YES & YES & $0.88$ & $(4,1)$ & NO & 1116\\
$(29,13)$ & 8 & $(9,2)$ & 5 & 1 & YES & YES & YES & $0.88$ & $(4,1)$ & -- & 1117\\
$(29,8)$ & 7 & $(10,3)$ & 5 & 1 & YES & YES & YES & $1.00$ & $(2,2)$ & NO & 1118\\
$(29,11)$ & 7 & $(10,3)$ & 5 & 1 & YES & YES & YES & $1.12$ & $(2,2)$ & NO & 1119\\
$(29,12)$ & 7 & $(10,3)$ & 5 & 1 & YES & YES & YES & $1.22$ & $(2,2)$ & -- & 1120\\
$(29,12)$ & 7 & $(10,3)$ & 5 & 1 & YES & YES & YES & $1.00$ & $(2,2)$ & NO & 1121\\
$(29,12)$ & 7 & $(10,3)$ & 5 & 1 & YES & YES & YES & $1.11$ & $(2,2)$ & NO & 1122\\
$(29,8)$ & 7 & $(11,4)$ & 5 & 1 & YES & YES & YES & $1.22$ & $(2,2)$ & NO & 1123\\
$(29,8)$ & 7 & $(11,4)$ & 5 & 1 & YES & YES & YES & $1.22$ & $(2,2)$ & -- & 1124\\
$(29,11)$ & 7 & $(11,3)$ & 5 & 1 & YES & YES & YES & $1.33$ & $(2,2)$ & -- & 1125\\
$(29,12)$ & 7 & $(11,5)$ & 6 & 1 & YES & YES & YES & $1.00$ & $(2,2)$ & 782 & 1126\\
$(29,13)$ & 8 & $(12,5)$ & 5 & 1 & YES & YES & YES & $1.00$ & $(2,2)$ & NO & 1127\\
$(29,11)$ & 7 & $(13,3)$ & 6 & 1 & YES & YES & YES & $1.22$ & $(2,2)$ & NO & 1128\\
$(29,11)$ & 7 & $(13,5)$ & 5 & 1 & YES & YES & NO(2) & $0.89$ & $(4,1)$ & 1308 & 1129\\
$(29,13)$ & 8 & $(13,6)$ & 7 & 1 & YES & YES & YES & $1.00$ & $(2,2)$ & 1785 & 1130\\
$(29,9)$ & 8 & $(15,4)$ & 6 & 1 & YES & YES & YES & $1.00$ & $(2,2)$ & NO & 1131\\
$(29,8)$ & 7 & $(18,5)$ & 6 & 1 & YES & YES & YES & $1.00$ & $(2,2)$ & NO & 1132\\
$(29,12)$ & 7 & $(18,7)$ & 6 & 1 & YES & YES & YES & $1.00$ & $(2,2)$ & NO & 1133\\
$(29,11)$ & 7 & $(21,8)$ & 6 & 1 & YES & YES & NO(2) & $1.10$ & $(2,2)$ & NO & 1134\\
$(29,9)$ & 8 & $(22,7)$ & 9 & 1 & YES & YES & YES & $1.00$ & $(2,2)$ & 1578 & 1135\\
$(29,12)$ & 7 & $(22,9)$ & 7 & 1 & YES & YES & NO(2) & $1.10$ & $(2,2)$ & NO & 1136\\
$(29,8)$ & 7 & $(25,7)$ & 7 & 1 & YES & YES & YES & $1.12$ & $(2,2)$ & NO & 1137\\
$(29,13)$ & 8 & $(25,11)$ & 7 & 1 & YES & YES & YES & $1.00$ & $(2,2)$ & NO & 1138\\
$(29,12)$ & 7 & $(26,11)$ & 7 & 1 & YES & YES & YES & $1.00$ & $(2,2)$ & NO & 1139\\
$(29,12)$ & 7 & $(27,11)$ & 8 & 1 & YES & YES & YES & $1.00$ & $(2,2)$ & 1826 & 1140\\
$(29,8)$ & 7 & $(29,8)$ & 7 & 29 & YES & YES & YES & $1.00$ & $(2,2)$ & NO & 1141\\
$(30,11)$ & 7 & $(4,1)$ & 3 & 2 & YES & YES & YES & $1.00$ & $(2,2)$ & -- & 1142\\
$(30,11)$ & 7 & $(4,1)$ & 3 & 2 & YES & YES & NO(2) & $1.00$ & $(2,2)$ & NO & 1143\\
$(30,13)$ & 8 & $(4,1)$ & 3 & 2 & YES & YES & NO(2) & $1.00$ & $(2,2)$ & -- & 1144\\
$(30,11)$ & 7 & $(5,1)$ & 4 & 5 & YES & YES & YES & $0.88$ & $(4,1)$ & NO & 1145\\
$(30,11)$ & 7 & $(5,2)$ & 3 & 5 & YES & YES & YES & $1.11$ & $(2,2)$ & NO & 1146\\
$(30,11)$ & 7 & $(5,2)$ & 3 & 5 & YES & YES & YES & $1.11$ & $(2,2)$ & -- & 1147\\
$(30,13)$ & 8 & $(5,1)$ & 4 & 5 & YES & YES & YES & $1.00$ & $(4,1)$ & NO & 1148\\
$(30,13)$ & 8 & $(5,1)$ & 4 & 5 & YES & YES & YES & $1.00$ & $(4,1)$ & -- & 1149\\
$(30,13)$ & 8 & $(5,1)$ & 4 & 5 & YES & YES & YES & $1.00$ & $(4,1)$ & NO & 1150\\
$(30,13)$ & 8 & $(5,2)$ & 3 & 5 & YES & YES & YES & $1.11$ & $(2,2)$ & -- & 1151\\
$(30,7)$ & 8 & $(7,2)$ & 4 & 1 & YES & YES & YES & $1.12$ & $(2,2)$ & -- & 1152\\
$(30,7)$ & 8 & $(7,2)$ & 4 & 1 & YES & YES & YES & $1.25$ & $(2,2)$ & NO & 1153\\
$(30,11)$ & 7 & $(7,3)$ & 4 & 1 & YES & YES & YES & $1.25$ & $(2,2)$ & -- & 1154\\
$(30,11)$ & 7 & $(7,3)$ & 4 & 1 & YES & YES & YES & $1.00$ & $(2,2)$ & NO & 1155\\
$(30,11)$ & 7 & $(8,3)$ & 4 & 2 & YES & YES & NO(2) & $0.89$ & $(4,1)$ & 1041 & 1156\\
$(30,13)$ & 8 & $(8,3)$ & 4 & 2 & YES & YES & YES & $1.11$ & $(2,2)$ & NO & 1157\\
$(30,11)$ & 7 & $(9,4)$ & 5 & 3 & YES & YES & YES & $1.00$ & $(2,2)$ & NO & 1158\\
$(30,11)$ & 7 & $(10,3)$ & 5 & 10 & YES & YES & YES & $1.00$ & $(2,2)$ & -- & 1159\\
$(30,11)$ & 7 & $(10,3)$ & 5 & 10 & YES & YES & YES & $1.25$ & $(2,2)$ & NO & 1160\\
$(30,13)$ & 8 & $(11,2)$ & 6 & 1 & YES & YES & YES & $1.00$ & $(2,2)$ & -- & 1161\\
$(30,13)$ & 8 & $(11,2)$ & 6 & 1 & YES & YES & YES & $1.11$ & $(2,2)$ & NO & 1162\\
$(30,11)$ & 7 & $(12,5)$ & 5 & 6 & YES & YES & YES & $1.12$ & $(2,2)$ & NO & 1163\\
$(30,7)$ & 8 & $(13,4)$ & 6 & 1 & YES & YES & YES & $1.00$ & $(2,2)$ & NO & 1164\\
$(30,7)$ & 8 & $(15,4)$ & 6 & 15 & YES & YES & YES & $1.00$ & $(2,2)$ & NO & 1165\\
$(30,11)$ & 7 & $(17,6)$ & 7 & 1 & YES & YES & YES & $0.88$ & $(4,1)$ & NO & 1166\\
$(30,11)$ & 7 & $(20,7)$ & 8 & 10 & YES & YES & YES & $1.12$ & $(2,2)$ & 1947 & 1167\\
$(30,11)$ & 7 & $(27,10)$ & 7 & 3 & YES & YES & NO(2) & $1.10$ & $(2,2)$ & NO & 1168\\
$(30,11)$ & 7 & $(30,11)$ & 7 & 30 & YES & YES & YES & $1.00$ & $(2,2)$ & NO & 1169\\
$(31,12)$ & 7 & $(3,1)$ & 2 & 1 & YES & YES & NO(2) & $0.78$ & $(4,1)$ & -- & 1170\\
$(31,14)$ & 8 & $(3,1)$ & 2 & 1 & YES & YES & YES & $1.12$ & $(2,2)$ & NO & 1171\\
$(31,14)$ & 8 & $(3,1)$ & 2 & 1 & YES & YES & YES & $1.00$ & $(2,2)$ & -- & 1172\\
$(31,14)$ & 8 & $(4,1)$ & 3 & 1 & YES & YES & YES & $1.00$ & $(2,2)$ & -- & 1173\\
$(31,11)$ & 8 & $(5,1)$ & 4 & 1 & YES & YES & NO(2) & $1.10$ & $(2,2)$ & -- & 1174\\
$(31,11)$ & 8 & $(5,2)$ & 3 & 1 & YES & YES & YES & $0.88$ & $(4,1)$ & -- & 1175\\
$(31,11)$ & 8 & $(5,2)$ & 3 & 1 & YES & YES & NO(2) & $1.20$ & $(2,2)$ & NO & 1176\\
$(31,12)$ & 7 & $(5,2)$ & 3 & 1 & YES & YES & YES & $1.11$ & $(2,2)$ & NO & 1177\\
$(31,12)$ & 7 & $(5,2)$ & 3 & 1 & YES & YES & YES & $1.11$ & $(2,2)$ & -- & 1178\\
$(31,13)$ & 7 & $(5,1)$ & 4 & 1 & YES & YES & YES & $1.00$ & $(2,2)$ & 726 & 1179\\
$(31,13)$ & 7 & $(5,1)$ & 4 & 1 & YES & YES & YES & $1.00$ & $(2,2)$ & -- & 1180\\
$(31,13)$ & 7 & $(5,1)$ & 4 & 1 & YES & YES & YES & $1.12$ & $(2,2)$ & NO & 1181\\
$(31,13)$ & 7 & $(5,2)$ & 3 & 1 & YES & YES & YES & $1.12$ & $(2,2)$ & -- & 1182\\
$(31,14)$ & 8 & $(5,2)$ & 3 & 1 & YES & YES & YES & $1.12$ & $(2,2)$ & -- & 1183\\
$(31,14)$ & 8 & $(5,2)$ & 3 & 1 & YES & YES & YES & $1.12$ & $(2,2)$ & NO & 1184\\
$(31,14)$ & 8 & $(5,2)$ & 3 & 1 & YES & YES & YES & $1.12$ & $(2,2)$ & NO & 1185\\
$(31,11)$ & 8 & $(7,2)$ & 4 & 1 & YES & YES & YES & $0.88$ & $(2,2)$ & -- & 1186\\
$(31,11)$ & 8 & $(7,3)$ & 4 & 1 & YES & YES & YES & $1.12$ & $(2,2)$ & -- & 1187\\
$(31,11)$ & 8 & $(7,3)$ & 4 & 1 & YES & YES & YES & $1.33$ & $(2,2)$ & NO & 1188\\
$(31,11)$ & 8 & $(7,3)$ & 4 & 1 & YES & YES & YES & $1.00$ & $(2,2)$ & NO & 1189\\
$(31,12)$ & 7 & $(7,2)$ & 4 & 1 & YES & YES & YES & $1.11$ & $(2,2)$ & NO & 1190\\
$(31,12)$ & 7 & $(7,2)$ & 4 & 1 & YES & YES & YES & $1.11$ & $(2,2)$ & -- & 1191\\
$(31,12)$ & 7 & $(7,3)$ & 4 & 1 & YES & YES & YES & $1.11$ & $(2,2)$ & NO & 1192\\
$(31,12)$ & 7 & $(7,3)$ & 4 & 1 & YES & YES & YES & $1.11$ & $(2,2)$ & -- & 1193\\
$(31,13)$ & 7 & $(7,3)$ & 4 & 1 & YES & YES & YES & $1.12$ & $(2,2)$ & NO & 1194\\
$(31,13)$ & 7 & $(7,3)$ & 4 & 1 & YES & YES & YES & $1.12$ & $(2,2)$ & -- & 1195\\
$(31,14)$ & 8 & $(7,2)$ & 4 & 1 & YES & YES & YES & $0.88$ & $(2,2)$ & -- & 1196\\
$(31,14)$ & 8 & $(7,2)$ & 4 & 1 & YES & YES & YES & $1.00$ & $(2,2)$ & NO & 1197\\
$(31,12)$ & 7 & $(8,3)$ & 4 & 1 & YES & YES & NO(2) & $0.89$ & $(4,1)$ & NO & 1198\\
$(31,13)$ & 7 & $(8,3)$ & 4 & 1 & YES & YES & YES & $1.33$ & $(2,2)$ & -- & 1199\\
$(31,13)$ & 7 & $(8,3)$ & 4 & 1 & YES & YES & YES & $1.12$ & $(2,2)$ & NO & 1200\\
$(31,14)$ & 8 & $(8,3)$ & 4 & 1 & YES & YES & YES & $1.12$ & $(2,2)$ & NO & 1201\\
$(31,7)$ & 8 & $(9,4)$ & 5 & 1 & YES & YES & YES & $1.00$ & $(2,2)$ & -- & 1202\\
$(31,11)$ & 8 & $(9,2)$ & 5 & 1 & YES & YES & YES & $1.12$ & $(2,2)$ & NO & 1203\\
$(31,11)$ & 8 & $(9,2)$ & 5 & 1 & YES & YES & YES & $1.12$ & $(2,2)$ & -- & 1204\\
$(31,12)$ & 7 & $(9,2)$ & 5 & 1 & YES & YES & YES & $1.11$ & $(2,2)$ & -- & 1205\\
$(31,12)$ & 7 & $(9,2)$ & 5 & 1 & YES & YES & YES & $1.11$ & $(2,2)$ & NO & 1206\\
$(31,13)$ & 7 & $(9,4)$ & 5 & 1 & YES & YES & YES & $1.12$ & $(2,2)$ & NO & 1207\\
$(31,14)$ & 8 & $(9,2)$ & 5 & 1 & YES & YES & YES & $1.00$ & $(2,2)$ & NO & 1208\\
$(31,14)$ & 8 & $(9,2)$ & 5 & 1 & YES & YES & YES & $1.00$ & $(2,2)$ & -- & 1209\\
$(31,12)$ & 7 & $(10,3)$ & 5 & 1 & YES & YES & YES & $1.11$ & $(2,2)$ & -- & 1210\\
$(31,12)$ & 7 & $(11,4)$ & 5 & 1 & YES & YES & YES & $1.11$ & $(2,2)$ & NO & 1211\\
$(31,14)$ & 8 & $(12,5)$ & 5 & 1 & YES & YES & YES & $1.00$ & $(2,2)$ & 1889 & 1212\\
$(31,11)$ & 8 & $(13,5)$ & 5 & 1 & YES & YES & YES & $1.00$ & $(2,2)$ & NO & 1213\\
$(31,9)$ & 8 & $(14,3)$ & 6 & 1 & YES & YES & YES & $1.11$ & $(2,2)$ & -- & 1214\\
$(31,14)$ & 8 & $(16,7)$ & 6 & 1 & YES & YES & YES & $1.00$ & $(2,2)$ & NO & 1215\\
$(31,6)$ & 10 & $(19,3)$ & 8 & 1 & YES & YES & YES & $0.88$ & $(2,2)$ & NO & 1216\\
$(31,11)$ & 8 & $(19,7)$ & 6 & 1 & YES & YES & YES & $0.88$ & $(4,1)$ & NO & 1217\\
$(31,13)$ & 7 & $(19,8)$ & 6 & 1 & YES & YES & NO(2) & $1.00$ & $(2,2)$ & NO & 1218\\
$(31,6)$ & 10 & $(20,3)$ & 8 & 1 & YES & YES & NO(2) & $1.10$ & $(2,2)$ & NO & 1219\\
$(31,6)$ & 10 & $(23,4)$ & 8 & 1 & YES & YES & YES & $0.88$ & $(2,2)$ & NO & 1220\\
$(31,12)$ & 7 & $(23,9)$ & 7 & 1 & YES & YES & YES & $1.11$ & $(2,2)$ & NO & 1221\\
$(31,7)$ & 8 & $(24,5)$ & 8 & 1 & YES & YES & NO(2) & $0.89$ & $(4,1)$ & NO & 1222\\
$(31,7)$ & 8 & $(25,6)$ & 9 & 1 & YES & YES & YES & $1.00$ & $(2,2)$ & NO & 1223\\
$(31,11)$ & 8 & $(25,9)$ & 7 & 1 & YES & YES & YES & $0.75$ & $(4,1)$ & NO & 1224\\
$(31,13)$ & 7 & $(26,11)$ & 7 & 1 & YES & YES & YES & $0.88$ & $(2,2)$ & NO & 1225\\
$(31,6)$ & 10 & $(28,5)$ & 8 & 1 & YES & YES & NO(2) & $1.10$ & $(2,2)$ & NO & 1226\\
$(31,12)$ & 7 & $(28,11)$ & 8 & 1 & YES & YES & YES & $1.11$ & $(2,2)$ & 1897 & 1227\\
$(31,14)$ & 8 & $(29,13)$ & 8 & 1 & YES & YES & YES & $1.12$ & $(2,2)$ & NO & 1228\\
$(32,7)$ & 8 & $(3,1)$ & 2 & 1 & YES & YES & YES & $1.00$ & $(2,2)$ & NO & 1229\\
$(32,7)$ & 8 & $(3,1)$ & 2 & 1 & YES & YES & YES & $1.00$ & $(2,2)$ & -- & 1230\\
$(32,13)$ & 9 & $(5,1)$ & 4 & 1 & YES & YES & YES & $1.00$ & $(2,2)$ & -- & 1231\\
$(32,13)$ & 9 & $(6,1)$ & 5 & 2 & YES & YES & YES & $1.00$ & $(2,2)$ & NO & 1232\\
$(32,7)$ & 8 & $(11,4)$ & 5 & 1 & YES & YES & YES & $0.88$ & $(2,2)$ & -- & 1233\\
$(32,7)$ & 8 & $(14,3)$ & 6 & 2 & YES & YES & YES & $1.00$ & $(2,2)$ & 1414 & 1234\\
$(32,13)$ & 9 & $(17,7)$ & 6 & 1 & YES & YES & YES & $1.00$ & $(2,2)$ & NO & 1235\\
$(32,7)$ & 8 & $(21,5)$ & 8 & 1 & YES & YES & NO(2) & $0.89$ & $(4,1)$ & NO & 1236\\
$(32,13)$ & 9 & $(22,9)$ & 7 & 2 & YES & YES & YES & $1.00$ & $(2,2)$ & 1812 & 1237\\
$(33,7)$ & 8 & $(3,1)$ & 2 & 3 & YES & YES & YES & $0.88$ & $(2,2)$ & NO & 1238\\
$(33,7)$ & 8 & $(3,1)$ & 2 & 3 & YES & YES & YES & $0.88$ & $(2,2)$ & -- & 1239\\
$(33,7)$ & 8 & $(3,1)$ & 2 & 3 & YES & YES & YES & $1.00$ & $(2,2)$ & NO & 1240\\
$(33,10)$ & 8 & $(4,1)$ & 3 & 1 & YES & YES & NO(2) & $0.89$ & $(4,1)$ & NO & 1241\\
$(33,10)$ & 8 & $(4,1)$ & 3 & 1 & YES & YES & NO(2) & $0.89$ & $(4,1)$ & -- & 1242\\
$(33,10)$ & 8 & $(4,1)$ & 3 & 1 & YES & YES & NO(2) & $0.89$ & $(4,1)$ & NO & 1243\\
$(33,14)$ & 8 & $(4,1)$ & 3 & 1 & YES & YES & YES & $1.11$ & $(2,2)$ & NO & 1244\\
$(33,10)$ & 8 & $(5,2)$ & 3 & 1 & YES & YES & YES & $1.22$ & $(2,2)$ & NO & 1245\\
$(33,13)$ & 9 & $(5,1)$ & 4 & 1 & YES & YES & YES & $0.88$ & $(4,1)$ & -- & 1246\\
$(33,14)$ & 8 & $(5,1)$ & 4 & 1 & YES & YES & NO(2) & $1.10$ & $(2,2)$ & NO & 1247\\
$(33,14)$ & 8 & $(5,2)$ & 3 & 1 & YES & YES & YES & $1.11$ & $(2,2)$ & -- & 1248\\
$(33,14)$ & 8 & $(5,2)$ & 3 & 1 & YES & YES & YES & $1.11$ & $(2,2)$ & NO & 1249\\
$(33,13)$ & 9 & $(6,1)$ & 5 & 3 & YES & YES & YES & $1.00$ & $(4,1)$ & NO & 1250\\
$(33,13)$ & 9 & $(6,1)$ & 5 & 3 & YES & YES & YES & $1.00$ & $(4,1)$ & NO & 1251\\
$(33,10)$ & 8 & $(7,3)$ & 4 & 1 & YES & YES & YES & $1.00$ & $(2,2)$ & NO & 1252\\
$(33,10)$ & 8 & $(7,3)$ & 4 & 1 & YES & YES & YES & $1.00$ & $(2,2)$ & -- & 1253\\
$(33,10)$ & 8 & $(7,3)$ & 4 & 1 & YES & YES & YES & $1.11$ & $(2,2)$ & NO & 1254\\
$(33,14)$ & 8 & $(8,3)$ & 4 & 1 & YES & YES & YES & $1.11$ & $(2,2)$ & NO & 1255\\
$(33,7)$ & 8 & $(9,2)$ & 5 & 3 & YES & YES & YES & $0.88$ & $(2,2)$ & NO & 1256\\
$(33,7)$ & 8 & $(9,4)$ & 5 & 3 & YES & YES & YES & $1.00$ & $(2,2)$ & -- & 1257\\
$(33,7)$ & 8 & $(9,4)$ & 5 & 3 & YES & YES & YES & $1.00$ & $(2,2)$ & NO & 1258\\
$(33,10)$ & 8 & $(9,4)$ & 5 & 3 & YES & YES & YES & $1.00$ & $(2,2)$ & NO & 1259\\
$(33,7)$ & 8 & $(11,4)$ & 5 & 11 & YES & YES & YES & $1.00$ & $(2,2)$ & -- & 1260\\
$(33,7)$ & 8 & $(11,4)$ & 5 & 11 & YES & YES & YES & $1.00$ & $(2,2)$ & NO & 1261\\
$(33,10)$ & 8 & $(11,3)$ & 5 & 11 & YES & YES & YES & $1.11$ & $(2,2)$ & -- & 1262\\
$(33,10)$ & 8 & $(11,4)$ & 5 & 11 & YES & YES & YES & $1.00$ & $(2,2)$ & NO & 1263\\
$(33,7)$ & 8 & $(13,4)$ & 6 & 1 & YES & YES & YES & $0.88$ & $(2,2)$ & -- & 1264\\
$(33,7)$ & 8 & $(13,4)$ & 6 & 1 & YES & YES & YES & $1.00$ & $(2,2)$ & NO & 1265\\
$(33,7)$ & 8 & $(15,4)$ & 6 & 3 & YES & YES & YES & $0.88$ & $(2,2)$ & -- & 1266\\
$(33,7)$ & 8 & $(15,4)$ & 6 & 3 & YES & YES & YES & $1.00$ & $(2,2)$ & NO & 1267\\
$(33,13)$ & 9 & $(18,7)$ & 6 & 3 & YES & YES & YES & $1.00$ & $(4,1)$ & NO & 1268\\
$(33,14)$ & 8 & $(19,8)$ & 6 & 1 & YES & YES & NO(2) & $1.10$ & $(2,2)$ & 1704 & 1269\\
$(33,7)$ & 8 & $(21,5)$ & 8 & 3 & YES & YES & YES & $1.00$ & $(2,2)$ & NO & 1270\\
$(33,13)$ & 9 & $(23,9)$ & 7 & 1 & YES & YES & YES & $0.88$ & $(4,1)$ & 1886 & 1271\\
$(33,7)$ & 8 & $(31,7)$ & 8 & 1 & YES & YES & YES & $1.00$ & $(2,2)$ & NO & 1272\\
$(34,9)$ & 8 & $(2,1)$ & 1 & 2 & YES & YES & YES & $1.00$ & $(2,2)$ & -- & 1273\\
$(34,9)$ & 8 & $(3,1)$ & 2 & 1 & YES & YES & YES & $1.00$ & $(2,2)$ & NO & 1274\\
$(34,9)$ & 8 & $(3,1)$ & 2 & 1 & YES & YES & YES & $1.00$ & $(2,2)$ & -- & 1275\\
$(34,13)$ & 7 & $(3,1)$ & 2 & 1 & YES & YES & YES & $1.00$ & $(2,2)$ & -- & 1276\\
$(34,15)$ & 8 & $(3,1)$ & 2 & 1 & YES & YES & NO(2) & $1.20$ & $(2,2)$ & NO & 1277\\
$(34,15)$ & 8 & $(3,1)$ & 2 & 1 & YES & YES & NO(2) & $1.20$ & $(2,2)$ & -- & 1278\\
$(34,7)$ & 10 & $(4,1)$ & 3 & 2 & YES & YES & YES & $0.75$ & $(2,2)$ & -- & 1279\\
$(34,9)$ & 8 & $(4,1)$ & 3 & 2 & YES & YES & YES & $0.88$ & $(2,2)$ & NO & 1280\\
$(34,9)$ & 8 & $(4,1)$ & 3 & 2 & YES & YES & YES & $0.88$ & $(2,2)$ & -- & 1281\\
$(34,13)$ & 7 & $(4,1)$ & 3 & 2 & YES & YES & YES & $1.12$ & $(2,2)$ & -- & 1282\\
$(34,13)$ & 7 & $(4,1)$ & 3 & 2 & YES & YES & YES & $1.25$ & $(2,2)$ & NO & 1283\\
$(34,15)$ & 8 & $(4,1)$ & 3 & 2 & YES & YES & YES & $1.00$ & $(4,1)$ & NO & 1284\\
$(34,15)$ & 8 & $(4,1)$ & 3 & 2 & YES & YES & YES & $1.00$ & $(4,1)$ & -- & 1285\\
$(34,9)$ & 8 & $(5,1)$ & 4 & 1 & YES & YES & YES & $1.00$ & $(2,2)$ & -- & 1286\\
$(34,9)$ & 8 & $(5,1)$ & 4 & 1 & YES & YES & YES & $1.12$ & $(2,2)$ & NO & 1287\\
$(34,9)$ & 8 & $(5,2)$ & 3 & 1 & YES & YES & NO(2) & $1.20$ & $(2,2)$ & NO & 1288\\
$(34,9)$ & 8 & $(5,2)$ & 3 & 1 & YES & YES & NO(2) & $1.20$ & $(2,2)$ & -- & 1289\\
$(34,13)$ & 7 & $(5,1)$ & 4 & 1 & YES & YES & YES & $1.12$ & $(2,2)$ & NO & 1290\\
$(34,13)$ & 7 & $(5,1)$ & 4 & 1 & YES & YES & YES & $1.12$ & $(2,2)$ & -- & 1291\\
$(34,13)$ & 7 & $(5,2)$ & 3 & 1 & YES & YES & YES & $1.12$ & $(2,2)$ & -- & 1292\\
$(34,15)$ & 8 & $(5,1)$ & 4 & 1 & YES & YES & YES & $0.88$ & $(2,2)$ & NO & 1293\\
$(34,15)$ & 8 & $(5,1)$ & 4 & 1 & YES & YES & YES & $0.88$ & $(2,2)$ & -- & 1294\\
$(34,15)$ & 8 & $(5,2)$ & 3 & 1 & YES & YES & YES & $0.88$ & $(2,2)$ & -- & 1295\\
$(34,15)$ & 8 & $(5,2)$ & 3 & 1 & YES & YES & YES & $1.00$ & $(4,1)$ & NO & 1296\\
$(34,9)$ & 8 & $(7,2)$ & 4 & 1 & YES & YES & YES & $1.12$ & $(2,2)$ & -- & 1297\\
$(34,9)$ & 8 & $(7,2)$ & 4 & 1 & YES & YES & YES & $1.00$ & $(2,2)$ & 760 & 1298\\
$(34,9)$ & 8 & $(7,3)$ & 4 & 1 & YES & YES & YES & $1.00$ & $(2,2)$ & -- & 1299\\
$(34,9)$ & 8 & $(7,3)$ & 4 & 1 & YES & YES & NO(2) & $1.20$ & $(2,2)$ & NO & 1300\\
$(34,13)$ & 7 & $(7,3)$ & 4 & 1 & YES & YES & YES & $1.22$ & $(2,2)$ & -- & 1301\\
$(34,13)$ & 7 & $(7,3)$ & 4 & 1 & YES & YES & YES & $1.25$ & $(2,2)$ & NO & 1302\\
$(34,13)$ & 7 & $(7,3)$ & 4 & 1 & YES & YES & YES & $1.12$ & $(2,2)$ & NO & 1303\\
$(34,9)$ & 8 & $(8,3)$ & 4 & 2 & YES & YES & YES & $1.00$ & $(2,2)$ & -- & 1304\\
$(34,9)$ & 8 & $(8,3)$ & 4 & 2 & YES & YES & YES & $0.88$ & $(2,2)$ & NO & 1305\\
$(34,13)$ & 7 & $(8,3)$ & 4 & 2 & YES & YES & YES & $1.22$ & $(2,2)$ & -- & 1306\\
$(34,13)$ & 7 & $(8,3)$ & 4 & 2 & YES & YES & YES & $1.22$ & $(2,2)$ & NO & 1307\\
$(34,13)$ & 7 & $(8,3)$ & 4 & 2 & YES & YES & NO(2) & $0.89$ & $(4,1)$ & 1129 & 1308\\
$(34,15)$ & 8 & $(8,3)$ & 4 & 2 & YES & YES & YES & $1.00$ & $(2,2)$ & NO & 1309\\
$(34,13)$ & 7 & $(9,4)$ & 5 & 1 & YES & YES & YES & $1.12$ & $(2,2)$ & NO & 1310\\
$(34,15)$ & 8 & $(9,2)$ & 5 & 1 & YES & YES & YES & $0.88$ & $(4,1)$ & NO & 1311\\
$(34,15)$ & 8 & $(9,2)$ & 5 & 1 & YES & YES & YES & $0.88$ & $(4,1)$ & -- & 1312\\
$(34,9)$ & 8 & $(11,3)$ & 5 & 1 & YES & YES & YES & $0.88$ & $(2,2)$ & NO & 1313\\
$(34,13)$ & 7 & $(11,4)$ & 5 & 1 & YES & YES & YES & $1.12$ & $(2,2)$ & NO & 1314\\
$(34,15)$ & 8 & $(12,5)$ & 5 & 2 & YES & YES & YES & $1.00$ & $(2,2)$ & 2102 & 1315\\
$(34,9)$ & 8 & $(13,3)$ & 6 & 1 & YES & YES & YES & $1.00$ & $(2,2)$ & -- & 1316\\
$(34,9)$ & 8 & $(13,4)$ & 6 & 1 & YES & YES & YES & $1.00$ & $(2,2)$ & NO & 1317\\
$(34,7)$ & 10 & $(14,3)$ & 6 & 2 & YES & YES & YES & $0.88$ & $(2,2)$ & NO & 1318\\
$(34,9)$ & 8 & $(14,3)$ & 6 & 2 & YES & YES & YES & $0.88$ & $(2,2)$ & -- & 1319\\
$(34,13)$ & 7 & $(14,5)$ & 6 & 2 & YES & YES & YES & $1.12$ & $(2,2)$ & NO & 1320\\
$(34,9)$ & 8 & $(15,4)$ & 6 & 1 & YES & YES & YES & $1.00$ & $(2,2)$ & NO & 1321\\
$(34,15)$ & 8 & $(16,7)$ & 6 & 2 & YES & YES & NO(2) & $1.10$ & $(2,2)$ & 1572 & 1322\\
$(34,9)$ & 8 & $(17,5)$ & 6 & 17 & YES & YES & YES & $1.12$ & $(2,2)$ & NO & 1323\\
$(34,15)$ & 8 & $(20,9)$ & 7 & 2 & YES & YES & YES & $1.12$ & $(2,2)$ & NO & 1324\\
$(34,15)$ & 8 & $(23,10)$ & 7 & 1 & YES & YES & YES & $1.11$ & $(2,2)$ & 2022 & 1325\\
$(34,15)$ & 8 & $(25,11)$ & 7 & 1 & YES & YES & YES & $0.88$ & $(2,2)$ & NO & 1326\\
$(34,9)$ & 8 & $(27,7)$ & 9 & 1 & YES & YES & YES & $1.00$ & $(2,2)$ & 1859 & 1327\\
$(34,9)$ & 8 & $(34,9)$ & 8 & 34 & YES & YES & YES & $1.00$ & $(2,2)$ & NO & 1328\\
$(34,13)$ & 7 & $(34,13)$ & 7 & 34 & YES & YES & YES & $1.00$ & $(2,2)$ & NO & 1329\\
$(34,15)$ & 8 & $(34,15)$ & 8 & 34 & YES & YES & NO(2) & $1.20$ & $(2,2)$ & NO & 1330\\
$(35,16)$ & 9 & $(3,1)$ & 2 & 1 & YES & YES & YES & $1.00$ & $(2,2)$ & NO & 1331\\
$(35,16)$ & 9 & $(3,1)$ & 2 & 1 & YES & YES & YES & $1.00$ & $(2,2)$ & -- & 1332\\
$(35,11)$ & 9 & $(4,1)$ & 3 & 1 & YES & YES & YES & $1.12$ & $(2,2)$ & NO & 1333\\
$(35,11)$ & 9 & $(4,1)$ & 3 & 1 & YES & YES & YES & $1.12$ & $(2,2)$ & -- & 1334\\
$(35,13)$ & 8 & $(4,1)$ & 3 & 1 & YES & YES & NO(2) & $1.00$ & $(4,1)$ & NO & 1335\\
$(35,13)$ & 8 & $(4,1)$ & 3 & 1 & YES & YES & NO(2) & $1.00$ & $(4,1)$ & -- & 1336\\
$(35,11)$ & 9 & $(5,2)$ & 3 & 5 & YES & YES & YES & $1.25$ & $(2,2)$ & NO & 1337\\
$(35,11)$ & 9 & $(5,2)$ & 3 & 5 & YES & YES & YES & $1.25$ & $(2,2)$ & -- & 1338\\
$(35,13)$ & 8 & $(5,2)$ & 3 & 5 & YES & YES & YES & $1.22$ & $(2,2)$ & -- & 1339\\
$(35,13)$ & 8 & $(5,2)$ & 3 & 5 & YES & YES & NO(2) & $1.00$ & $(4,1)$ & NO & 1340\\
$(35,16)$ & 9 & $(5,2)$ & 3 & 5 & YES & YES & YES & $1.25$ & $(2,2)$ & -- & 1341\\
$(35,16)$ & 9 & $(5,2)$ & 3 & 5 & YES & YES & YES & $1.00$ & $(2,2)$ & 747 & 1342\\
$(35,11)$ & 9 & $(6,1)$ & 5 & 1 & YES & YES & YES & $1.00$ & $(2,2)$ & -- & 1343\\
$(35,11)$ & 9 & $(6,1)$ & 5 & 1 & YES & YES & YES & $1.12$ & $(2,2)$ & NO & 1344\\
$(35,8)$ & 8 & $(7,3)$ & 4 & 7 & YES & YES & YES & $1.25$ & $(2,2)$ & NO & 1345\\
$(35,8)$ & 8 & $(7,3)$ & 4 & 7 & YES & YES & YES & $1.25$ & $(2,2)$ & -- & 1346\\
$(35,11)$ & 9 & $(7,2)$ & 4 & 7 & YES & YES & YES & $1.00$ & $(2,2)$ & -- & 1347\\
$(35,13)$ & 8 & $(7,2)$ & 4 & 7 & YES & YES & YES & $1.25$ & $(2,2)$ & -- & 1348\\
$(35,16)$ & 9 & $(7,2)$ & 4 & 7 & YES & YES & YES & $1.12$ & $(2,2)$ & NO & 1349\\
$(35,16)$ & 9 & $(7,2)$ & 4 & 7 & YES & YES & YES & $1.25$ & $(2,2)$ & -- & 1350\\
$(35,16)$ & 9 & $(8,3)$ & 4 & 1 & YES & YES & YES & $1.12$ & $(2,2)$ & NO & 1351\\
$(35,8)$ & 8 & $(9,4)$ & 5 & 1 & YES & YES & YES & $1.00$ & $(2,2)$ & -- & 1352\\
$(35,13)$ & 8 & $(9,2)$ & 5 & 1 & YES & YES & YES & $1.11$ & $(2,2)$ & -- & 1353\\
$(35,16)$ & 9 & $(13,6)$ & 7 & 1 & YES & YES & YES & $1.00$ & $(2,2)$ & 1424 & 1354\\
$(35,13)$ & 8 & $(14,5)$ & 6 & 7 & YES & YES & NO(2) & $1.00$ & $(4,1)$ & NO & 1355\\
$(35,11)$ & 9 & $(17,5)$ & 6 & 1 & YES & YES & YES & $1.00$ & $(2,2)$ & NO & 1356\\
$(35,11)$ & 9 & $(23,7)$ & 7 & 1 & YES & YES & YES & $1.00$ & $(2,2)$ & NO & 1357\\
$(35,8)$ & 8 & $(25,6)$ & 9 & 5 & YES & YES & YES & $0.88$ & $(4,1)$ & 2182 & 1358\\
$(35,13)$ & 8 & $(30,11)$ & 7 & 5 & YES & YES & YES & $1.00$ & $(2,2)$ & 2076 & 1359\\
$(36,13)$ & 8 & $(3,1)$ & 2 & 3 & YES & YES & YES & $0.88$ & $(4,1)$ & NO & 1360\\
$(36,13)$ & 8 & $(3,1)$ & 2 & 3 & YES & YES & YES & $0.88$ & $(4,1)$ & -- & 1361\\
$(36,11)$ & 8 & $(5,2)$ & 3 & 1 & YES & YES & NO(2) & $1.10$ & $(2,2)$ & -- & 1362\\
$(36,13)$ & 8 & $(5,2)$ & 3 & 1 & YES & YES & YES & $0.88$ & $(2,2)$ & -- & 1363\\
$(36,13)$ & 8 & $(5,2)$ & 3 & 1 & YES & YES & NO(2) & $0.89$ & $(4,1)$ & NO & 1364\\
$(36,7)$ & 11 & $(7,2)$ & 4 & 1 & YES & YES & YES & $0.88$ & $(2,2)$ & NO & 1365\\
$(36,11)$ & 8 & $(7,3)$ & 4 & 1 & YES & YES & YES & $0.88$ & $(2,2)$ & NO & 1366\\
$(36,11)$ & 8 & $(7,3)$ & 4 & 1 & YES & YES & YES & $0.88$ & $(2,2)$ & -- & 1367\\
$(36,13)$ & 8 & $(7,3)$ & 4 & 1 & YES & YES & YES & $1.11$ & $(2,2)$ & NO & 1368\\
$(36,11)$ & 8 & $(8,3)$ & 4 & 4 & YES & YES & YES & $0.88$ & $(2,2)$ & NO & 1369\\
$(36,13)$ & 8 & $(9,4)$ & 5 & 9 & YES & YES & YES & $1.00$ & $(2,2)$ & NO & 1370\\
$(36,7)$ & 11 & $(15,2)$ & 8 & 3 & YES & YES & YES & $0.88$ & $(2,2)$ & NO & 1371\\
$(36,11)$ & 8 & $(19,6)$ & 8 & 1 & YES & YES & YES & $1.00$ & $(2,2)$ & NO & 1372\\
$(36,13)$ & 8 & $(19,7)$ & 6 & 1 & YES & YES & YES & $1.11$ & $(2,2)$ & NO & 1373\\
$(36,7)$ & 11 & $(27,5)$ & 8 & 9 & YES & YES & YES & $0.88$ & $(2,2)$ & NO & 1374\\
$(36,11)$ & 8 & $(29,9)$ & 8 & 1 & YES & YES & YES & $1.00$ & $(2,2)$ & NO & 1375\\
$(36,13)$ & 8 & $(36,13)$ & 8 & 36 & YES & YES & YES & $0.88$ & $(4,1)$ & NO & 1376\\
$(37,8)$ & 8 & $(2,1)$ & 1 & 1 & YES & YES & YES & $1.00$ & $(2,2)$ & NO & 1377\\
$(37,8)$ & 8 & $(2,1)$ & 1 & 1 & YES & YES & YES & $1.00$ & $(2,2)$ & -- & 1378\\
$(37,17)$ & 9 & $(2,1)$ & 1 & 1 & YES & YES & YES & $1.00$ & $(2,2)$ & -- & 1379\\
$(37,8)$ & 8 & $(3,1)$ & 2 & 1 & YES & YES & YES & $0.88$ & $(2,2)$ & NO & 1380\\
$(37,8)$ & 8 & $(3,1)$ & 2 & 1 & YES & YES & YES & $0.88$ & $(2,2)$ & -- & 1381\\
$(37,10)$ & 8 & $(3,1)$ & 2 & 1 & YES & YES & NO(2) & $0.89$ & $(4,1)$ & NO & 1382\\
$(37,10)$ & 8 & $(3,1)$ & 2 & 1 & YES & YES & NO(2) & $0.89$ & $(4,1)$ & -- & 1383\\
$(37,17)$ & 9 & $(3,1)$ & 2 & 1 & YES & YES & YES & $1.00$ & $(2,2)$ & -- & 1384\\
$(37,17)$ & 9 & $(3,1)$ & 2 & 1 & YES & YES & YES & $1.00$ & $(2,2)$ & NO & 1385\\
$(37,13)$ & 9 & $(4,1)$ & 3 & 1 & YES & YES & YES & $1.00$ & $(2,2)$ & NO & 1386\\
$(37,14)$ & 8 & $(4,1)$ & 3 & 1 & YES & YES & YES & $1.11$ & $(2,2)$ & NO & 1387\\
$(37,14)$ & 8 & $(4,1)$ & 3 & 1 & YES & YES & YES & $1.11$ & $(2,2)$ & -- & 1388\\
$(37,17)$ & 9 & $(4,1)$ & 3 & 1 & YES & YES & YES & $1.00$ & $(2,2)$ & NO & 1389\\
$(37,17)$ & 9 & $(4,1)$ & 3 & 1 & YES & YES & YES & $1.00$ & $(2,2)$ & -- & 1390\\
$(37,17)$ & 9 & $(4,1)$ & 3 & 1 & YES & YES & YES & $1.00$ & $(2,2)$ & NO & 1391\\
$(37,8)$ & 8 & $(5,1)$ & 4 & 1 & YES & YES & YES & $1.00$ & $(2,2)$ & NO & 1392\\
$(37,8)$ & 8 & $(5,1)$ & 4 & 1 & YES & YES & YES & $1.00$ & $(2,2)$ & -- & 1393\\
$(37,13)$ & 9 & $(5,2)$ & 3 & 1 & YES & YES & YES & $1.12$ & $(2,2)$ & -- & 1394\\
$(37,14)$ & 8 & $(5,2)$ & 3 & 1 & YES & YES & YES & $1.11$ & $(2,2)$ & -- & 1395\\
$(37,16)$ & 9 & $(5,2)$ & 3 & 1 & YES & YES & YES & $1.25$ & $(2,2)$ & NO & 1396\\
$(37,17)$ & 9 & $(5,1)$ & 4 & 1 & YES & YES & YES & $1.00$ & $(2,2)$ & NO & 1397\\
$(37,17)$ & 9 & $(5,1)$ & 4 & 1 & YES & YES & YES & $1.00$ & $(2,2)$ & -- & 1398\\
$(37,17)$ & 9 & $(5,2)$ & 3 & 1 & YES & YES & YES & $1.00$ & $(2,2)$ & NO & 1399\\
$(37,16)$ & 9 & $(6,1)$ & 5 & 1 & YES & YES & NO(2) & $0.89$ & $(4,1)$ & -- & 1400\\
$(37,17)$ & 9 & $(6,1)$ & 5 & 1 & YES & YES & YES & $1.00$ & $(2,2)$ & NO & 1401\\
$(37,8)$ & 8 & $(7,3)$ & 4 & 1 & YES & YES & YES & $1.00$ & $(2,2)$ & -- & 1402\\
$(37,10)$ & 8 & $(7,3)$ & 4 & 1 & YES & YES & YES & $1.11$ & $(2,2)$ & NO & 1403\\
$(37,10)$ & 8 & $(7,3)$ & 4 & 1 & YES & YES & YES & $0.88$ & $(2,2)$ & -- & 1404\\
$(37,11)$ & 8 & $(7,3)$ & 4 & 1 & YES & YES & YES & $0.88$ & $(2,2)$ & NO & 1405\\
$(37,14)$ & 8 & $(7,2)$ & 4 & 1 & YES & YES & YES & $1.12$ & $(2,2)$ & NO & 1406\\
$(37,14)$ & 8 & $(7,3)$ & 4 & 1 & YES & YES & YES & $1.11$ & $(2,2)$ & NO & 1407\\
$(37,16)$ & 9 & $(7,1)$ & 6 & 1 & YES & YES & NO(2) & $1.00$ & $(4,1)$ & NO & 1408\\
$(37,16)$ & 9 & $(7,1)$ & 6 & 1 & YES & YES & NO(2) & $1.00$ & $(4,1)$ & NO & 1409\\
$(37,17)$ & 9 & $(7,3)$ & 4 & 1 & YES & YES & YES & $1.00$ & $(2,2)$ & NO & 1410\\
$(37,10)$ & 8 & $(8,3)$ & 4 & 1 & YES & YES & YES & $1.12$ & $(2,2)$ & -- & 1411\\
$(37,10)$ & 8 & $(8,3)$ & 4 & 1 & YES & YES & YES & $1.11$ & $(2,2)$ & NO & 1412\\
$(37,11)$ & 8 & $(8,3)$ & 4 & 1 & YES & YES & YES & $1.00$ & $(2,2)$ & NO & 1413\\
$(37,8)$ & 8 & $(9,2)$ & 5 & 1 & YES & YES & YES & $1.00$ & $(2,2)$ & 1234 & 1414\\
$(37,8)$ & 8 & $(9,2)$ & 5 & 1 & YES & YES & YES & $1.00$ & $(2,2)$ & -- & 1415\\
$(37,8)$ & 8 & $(9,4)$ & 5 & 1 & YES & YES & YES & $1.00$ & $(2,2)$ & -- & 1416\\
$(37,14)$ & 8 & $(9,2)$ & 5 & 1 & YES & YES & YES & $1.22$ & $(2,2)$ & -- & 1417\\
$(37,14)$ & 8 & $(9,2)$ & 5 & 1 & YES & YES & YES & $1.33$ & $(2,2)$ & NO & 1418\\
$(37,17)$ & 9 & $(9,4)$ & 5 & 1 & YES & YES & YES & $1.12$ & $(2,2)$ & 1869 & 1419\\
$(37,8)$ & 8 & $(11,4)$ & 5 & 1 & YES & YES & YES & $1.00$ & $(2,2)$ & -- & 1420\\
$(37,10)$ & 8 & $(11,3)$ & 5 & 1 & YES & YES & YES & $1.11$ & $(2,2)$ & -- & 1421\\
$(37,13)$ & 9 & $(11,4)$ & 5 & 1 & YES & YES & YES & $0.88$ & $(2,2)$ & 968 & 1422\\
$(37,16)$ & 9 & $(11,5)$ & 6 & 1 & YES & YES & YES & $1.25$ & $(2,2)$ & NO & 1423\\
$(37,17)$ & 9 & $(11,5)$ & 6 & 1 & YES & YES & YES & $1.00$ & $(2,2)$ & 1354 & 1424\\
$(37,8)$ & 8 & $(14,3)$ & 6 & 1 & YES & YES & YES & $1.00$ & $(2,2)$ & NO & 1425\\
$(37,10)$ & 8 & $(14,3)$ & 6 & 1 & YES & YES & YES & $1.00$ & $(2,2)$ & NO & 1426\\
$(37,10)$ & 8 & $(15,4)$ & 6 & 1 & YES & YES & YES & $0.88$ & $(2,2)$ & 1574 & 1427\\
$(37,8)$ & 8 & $(21,5)$ & 8 & 1 & YES & YES & YES & $1.00$ & $(2,2)$ & NO & 1428\\
$(37,17)$ & 9 & $(24,11)$ & 8 & 1 & YES & YES & YES & $1.00$ & $(2,2)$ & NO & 1429\\
$(37,14)$ & 8 & $(29,11)$ & 7 & 1 & YES & YES & YES & $1.11$ & $(2,2)$ & NO & 1430\\
$(37,16)$ & 9 & $(30,13)$ & 8 & 1 & YES & YES & NO(2) & $0.89$ & $(4,1)$ & NO & 1431\\
$(37,8)$ & 8 & $(31,7)$ & 8 & 1 & YES & YES & YES & $1.00$ & $(2,2)$ & NO & 1432\\
$(37,10)$ & 8 & $(34,9)$ & 8 & 1 & YES & YES & YES & $1.00$ & $(2,2)$ & NO & 1433\\
$(37,14)$ & 8 & $(34,13)$ & 7 & 1 & YES & YES & YES & $1.11$ & $(2,2)$ & 2169 & 1434\\
$(37,16)$ & 9 & $(37,16)$ & 9 & 37 & YES & YES & NO(2) & $1.00$ & $(4,1)$ & NO & 1435\\
$(37,17)$ & 9 & $(37,17)$ & 9 & 37 & YES & YES & YES & $1.00$ & $(2,2)$ & NO & 1436\\
$(38,9)$ & 9 & $(3,1)$ & 2 & 1 & YES & YES & YES & $1.12$ & $(2,2)$ & NO & 1437\\
$(38,9)$ & 9 & $(3,1)$ & 2 & 1 & YES & YES & YES & $1.12$ & $(2,2)$ & -- & 1438\\
$(38,9)$ & 9 & $(4,1)$ & 3 & 2 & YES & YES & YES & $1.00$ & $(2,2)$ & NO & 1439\\
$(38,9)$ & 9 & $(4,1)$ & 3 & 2 & YES & YES & YES & $1.00$ & $(2,2)$ & -- & 1440\\
$(38,11)$ & 9 & $(4,1)$ & 3 & 2 & YES & YES & YES & $1.12$ & $(2,2)$ & NO & 1441\\
$(38,17)$ & 9 & $(5,1)$ & 4 & 1 & YES & YES & YES & $0.88$ & $(2,2)$ & -- & 1442\\
$(38,17)$ & 9 & $(6,1)$ & 5 & 2 & YES & YES & YES & $1.00$ & $(2,2)$ & NO & 1443\\
$(38,17)$ & 9 & $(6,1)$ & 5 & 2 & YES & YES & YES & $1.00$ & $(2,2)$ & NO & 1444\\
$(38,11)$ & 9 & $(9,2)$ & 5 & 1 & YES & YES & YES & $1.12$ & $(2,2)$ & NO & 1445\\
$(38,11)$ & 9 & $(9,2)$ & 5 & 1 & YES & YES & YES & $1.12$ & $(2,2)$ & -- & 1446\\
$(38,17)$ & 9 & $(9,2)$ & 5 & 1 & YES & YES & YES & $1.00$ & $(2,2)$ & -- & 1447\\
$(38,9)$ & 9 & $(11,3)$ & 5 & 1 & YES & YES & YES & $0.88$ & $(2,2)$ & -- & 1448\\
$(38,17)$ & 9 & $(16,7)$ & 6 & 2 & YES & YES & YES & $1.12$ & $(2,2)$ & NO & 1449\\
$(38,17)$ & 9 & $(29,13)$ & 8 & 1 & YES & YES & YES & $0.88$ & $(2,2)$ & NO & 1450\\
$(38,17)$ & 9 & $(38,17)$ & 9 & 38 & YES & YES & YES & $1.00$ & $(2,2)$ & NO & 1451\\
$(39,14)$ & 8 & $(2,1)$ & 1 & 1 & YES & YES & YES & $1.00$ & $(2,2)$ & NO & 1452\\
$(39,16)$ & 8 & $(2,1)$ & 1 & 1 & YES & YES & NO(2) & $1.20$ & $(2,2)$ & -- & 1453\\
$(39,16)$ & 8 & $(2,1)$ & 1 & 1 & YES & YES & YES & $1.00$ & $(2,2)$ & NO & 1454\\
$(39,17)$ & 8 & $(2,1)$ & 1 & 1 & YES & YES & NO(2) & $1.10$ & $(2,2)$ & -- & 1455\\
$(39,14)$ & 8 & $(3,1)$ & 2 & 3 & YES & YES & YES & $1.00$ & $(2,2)$ & -- & 1456\\
$(39,14)$ & 8 & $(3,1)$ & 2 & 3 & YES & YES & YES & $1.00$ & $(2,2)$ & NO & 1457\\
$(39,16)$ & 8 & $(3,1)$ & 2 & 3 & YES & YES & YES & $1.00$ & $(2,2)$ & 1459 & 1458\\
$(39,16)$ & 8 & $(3,1)$ & 2 & 3 & YES & YES & YES & $1.00$ & $(2,2)$ & 1458 & 1459\\
$(39,16)$ & 8 & $(3,1)$ & 2 & 3 & YES & YES & YES & $1.00$ & $(2,2)$ & -- & 1460\\
$(39,16)$ & 8 & $(3,1)$ & 2 & 3 & YES & YES & YES & $1.00$ & $(2,2)$ & NO & 1461\\
$(39,17)$ & 8 & $(3,1)$ & 2 & 3 & YES & YES & YES & $1.00$ & $(2,2)$ & -- & 1462\\
$(39,17)$ & 8 & $(3,1)$ & 2 & 3 & YES & YES & NO(2) & $1.10$ & $(2,2)$ & NO & 1463\\
$(39,11)$ & 9 & $(4,1)$ & 3 & 1 & YES & YES & YES & $1.00$ & $(2,2)$ & NO & 1464\\
$(39,14)$ & 8 & $(4,1)$ & 3 & 1 & YES & YES & YES & $1.00$ & $(2,2)$ & NO & 1465\\
$(39,14)$ & 8 & $(4,1)$ & 3 & 1 & YES & YES & YES & $1.00$ & $(2,2)$ & -- & 1466\\
$(39,17)$ & 8 & $(4,1)$ & 3 & 1 & YES & YES & YES & $1.00$ & $(2,2)$ & NO & 1467\\
$(39,17)$ & 8 & $(4,1)$ & 3 & 1 & YES & YES & YES & $1.00$ & $(2,2)$ & -- & 1468\\
$(39,11)$ & 9 & $(5,1)$ & 4 & 1 & YES & YES & YES & $1.12$ & $(2,2)$ & NO & 1469\\
$(39,11)$ & 9 & $(5,1)$ & 4 & 1 & YES & YES & YES & $1.12$ & $(2,2)$ & -- & 1470\\
$(39,14)$ & 8 & $(5,1)$ & 4 & 1 & YES & YES & YES & $0.88$ & $(2,2)$ & -- & 1471\\
$(39,14)$ & 8 & $(5,1)$ & 4 & 1 & YES & YES & NO(2) & $1.10$ & $(2,2)$ & NO & 1472\\
$(39,14)$ & 8 & $(5,2)$ & 3 & 1 & YES & YES & YES & $1.00$ & $(2,2)$ & -- & 1473\\
$(39,14)$ & 8 & $(5,2)$ & 3 & 1 & YES & YES & YES & $1.00$ & $(2,2)$ & NO & 1474\\
$(39,16)$ & 8 & $(5,1)$ & 4 & 1 & YES & YES & YES & $0.88$ & $(2,2)$ & -- & 1475\\
$(39,16)$ & 8 & $(5,2)$ & 3 & 1 & YES & YES & YES & $1.00$ & $(2,2)$ & 1049 & 1476\\
$(39,17)$ & 8 & $(5,2)$ & 3 & 1 & YES & YES & YES & $1.00$ & $(2,2)$ & -- & 1477\\
$(39,17)$ & 8 & $(5,2)$ & 3 & 1 & YES & YES & YES & $1.00$ & $(2,2)$ & NO & 1478\\
$(39,14)$ & 8 & $(7,2)$ & 4 & 1 & YES & YES & YES & $1.00$ & $(2,2)$ & NO & 1479\\
$(39,16)$ & 8 & $(7,2)$ & 4 & 1 & YES & YES & YES & $1.00$ & $(2,2)$ & NO & 1480\\
$(39,14)$ & 8 & $(8,3)$ & 4 & 1 & YES & YES & YES & $1.00$ & $(2,2)$ & 1787 & 1481\\
$(39,14)$ & 8 & $(9,2)$ & 5 & 3 & YES & YES & YES & $0.88$ & $(2,2)$ & NO & 1482\\
$(39,16)$ & 8 & $(9,2)$ & 5 & 3 & YES & YES & YES & $0.88$ & $(2,2)$ & NO & 1483\\
$(39,16)$ & 8 & $(9,4)$ & 5 & 3 & YES & YES & YES & $1.00$ & $(2,2)$ & NO & 1484\\
$(39,17)$ & 8 & $(9,4)$ & 5 & 3 & YES & YES & NO(2) & $1.10$ & $(2,2)$ & NO & 1485\\
$(39,17)$ & 8 & $(11,5)$ & 6 & 1 & YES & YES & YES & $1.00$ & $(2,2)$ & NO & 1486\\
$(39,14)$ & 8 & $(13,5)$ & 5 & 13 & YES & YES & YES & $1.00$ & $(2,2)$ & 2113 & 1487\\
$(39,14)$ & 8 & $(14,5)$ & 6 & 1 & YES & YES & NO(2) & $1.20$ & $(2,2)$ & NO & 1488\\
$(39,16)$ & 8 & $(17,7)$ & 6 & 1 & YES & YES & NO(2) & $1.10$ & $(2,2)$ & NO & 1489\\
$(39,16)$ & 8 & $(19,8)$ & 6 & 1 & YES & YES & YES & $0.88$ & $(2,2)$ & NO & 1490\\
$(39,17)$ & 8 & $(23,10)$ & 7 & 1 & YES & YES & YES & $1.00$ & $(2,2)$ & NO & 1491\\
$(39,11)$ & 9 & $(25,7)$ & 7 & 1 & YES & YES & YES & $1.12$ & $(2,2)$ & 1986 & 1492\\
$(39,14)$ & 8 & $(25,9)$ & 7 & 1 & YES & YES & YES & $1.00$ & $(2,2)$ & NO & 1493\\
$(39,17)$ & 8 & $(25,11)$ & 7 & 1 & YES & YES & YES & $1.00$ & $(2,2)$ & NO & 1494\\
$(39,16)$ & 8 & $(27,11)$ & 8 & 3 & YES & YES & YES & $1.00$ & $(2,2)$ & NO & 1495\\
$(39,14)$ & 8 & $(39,14)$ & 8 & 39 & YES & YES & YES & $1.00$ & $(2,2)$ & NO & 1496\\
$(39,17)$ & 8 & $(39,17)$ & 8 & 39 & YES & YES & YES & $1.00$ & $(2,2)$ & NO & 1497\\
$(40,17)$ & 9 & $(4,1)$ & 3 & 4 & YES & YES & YES & $1.11$ & $(2,2)$ & -- & 1498\\
$(40,17)$ & 9 & $(4,1)$ & 3 & 4 & YES & YES & YES & $1.25$ & $(2,2)$ & NO & 1499\\
$(40,17)$ & 9 & $(5,1)$ & 4 & 5 & YES & YES & YES & $1.00$ & $(2,2)$ & -- & 1500\\
$(40,17)$ & 9 & $(5,1)$ & 4 & 5 & YES & YES & YES & $1.12$ & $(2,2)$ & NO & 1501\\
$(40,17)$ & 9 & $(5,1)$ & 4 & 5 & YES & YES & YES & $1.22$ & $(2,2)$ & NO & 1502\\
$(40,17)$ & 9 & $(6,1)$ & 5 & 2 & YES & YES & YES & $1.12$ & $(2,2)$ & NO & 1503\\
$(40,17)$ & 9 & $(6,1)$ & 5 & 2 & YES & YES & YES & $1.12$ & $(2,2)$ & NO & 1504\\
$(40,17)$ & 9 & $(19,8)$ & 6 & 1 & YES & YES & YES & $1.12$ & $(2,2)$ & NO & 1505\\
$(40,17)$ & 9 & $(26,11)$ & 7 & 2 & YES & YES & YES & $1.00$ & $(2,2)$ & 2021 & 1506\\
$(40,9)$ & 9 & $(32,7)$ & 8 & 8 & YES & YES & YES & $1.12$ & $(2,2)$ & NO & 1507\\
$(40,17)$ & 9 & $(33,14)$ & 8 & 1 & YES & YES & YES & $1.11$ & $(2,2)$ & NO & 1508\\
$(41,11)$ & 8 & $(2,1)$ & 1 & 1 & YES & YES & YES & $1.00$ & $(4,1)$ & NO & 1509\\
$(41,11)$ & 8 & $(2,1)$ & 1 & 1 & YES & YES & YES & $1.00$ & $(2,2)$ & -- & 1510\\
$(41,13)$ & 10 & $(2,1)$ & 1 & 1 & YES & YES & YES & $1.00$ & $(2,2)$ & -- & 1511\\
$(41,15)$ & 8 & $(2,1)$ & 1 & 1 & YES & YES & NO(2) & $1.00$ & $(4,1)$ & NO & 1512\\
$(41,16)$ & 8 & $(2,1)$ & 1 & 1 & YES & YES & YES & $1.00$ & $(2,2)$ & -- & 1513\\
$(41,16)$ & 8 & $(2,1)$ & 1 & 1 & YES & YES & YES & $1.00$ & $(2,2)$ & NO & 1514\\
$(41,17)$ & 8 & $(2,1)$ & 1 & 1 & YES & YES & NO(2) & $0.89$ & $(4,1)$ & -- & 1515\\
$(41,18)$ & 8 & $(2,1)$ & 1 & 1 & YES & YES & YES & $0.88$ & $(2,2)$ & -- & 1516\\
$(41,19)$ & 10 & $(2,1)$ & 1 & 1 & YES & YES & YES & $1.00$ & $(2,2)$ & -- & 1517\\
$(41,19)$ & 10 & $(2,1)$ & 1 & 1 & YES & YES & YES & $1.00$ & $(2,2)$ & NO & 1518\\
$(41,11)$ & 8 & $(3,1)$ & 2 & 1 & YES & YES & YES & $1.00$ & $(2,2)$ & NO & 1519\\
$(41,11)$ & 8 & $(3,1)$ & 2 & 1 & YES & YES & YES & $1.00$ & $(2,2)$ & -- & 1520\\
$(41,13)$ & 10 & $(3,1)$ & 2 & 1 & YES & YES & YES & $1.00$ & $(2,2)$ & -- & 1521\\
$(41,15)$ & 8 & $(3,1)$ & 2 & 1 & YES & YES & YES & $1.00$ & $(4,1)$ & -- & 1522\\
$(41,16)$ & 8 & $(3,1)$ & 2 & 1 & YES & YES & NO(2) & $1.20$ & $(2,2)$ & NO & 1523\\
$(41,16)$ & 8 & $(3,1)$ & 2 & 1 & YES & YES & NO(2) & $1.20$ & $(2,2)$ & -- & 1524\\
$(41,16)$ & 8 & $(3,1)$ & 2 & 1 & YES & YES & YES & $1.00$ & $(2,2)$ & NO & 1525\\
$(41,17)$ & 8 & $(3,1)$ & 2 & 1 & YES & YES & NO(2) & $0.89$ & $(4,1)$ & 1101 & 1526\\
$(41,17)$ & 8 & $(3,1)$ & 2 & 1 & YES & YES & NO(2) & $0.89$ & $(4,1)$ & -- & 1527\\
$(41,17)$ & 8 & $(3,1)$ & 2 & 1 & YES & YES & NO(2) & $0.89$ & $(4,1)$ & NO & 1528\\
$(41,18)$ & 8 & $(3,1)$ & 2 & 1 & YES & YES & YES & $1.00$ & $(2,2)$ & NO & 1529\\
$(41,18)$ & 8 & $(3,1)$ & 2 & 1 & YES & YES & YES & $1.00$ & $(2,2)$ & -- & 1530\\
$(41,18)$ & 8 & $(3,1)$ & 2 & 1 & YES & YES & YES & $1.00$ & $(2,2)$ & 946 & 1531\\
$(41,19)$ & 10 & $(3,1)$ & 2 & 1 & YES & YES & YES & $1.00$ & $(2,2)$ & NO & 1532\\
$(41,11)$ & 8 & $(4,1)$ & 3 & 1 & YES & YES & YES & $1.12$ & $(2,2)$ & NO & 1533\\
$(41,11)$ & 8 & $(4,1)$ & 3 & 1 & YES & YES & YES & $1.12$ & $(2,2)$ & -- & 1534\\
$(41,12)$ & 8 & $(4,1)$ & 3 & 1 & YES & YES & YES & $1.11$ & $(2,2)$ & NO & 1535\\
$(41,12)$ & 8 & $(4,1)$ & 3 & 1 & YES & YES & YES & $1.11$ & $(2,2)$ & -- & 1536\\
$(41,13)$ & 10 & $(4,1)$ & 3 & 1 & YES & YES & YES & $1.00$ & $(2,2)$ & NO & 1537\\
$(41,15)$ & 8 & $(4,1)$ & 3 & 1 & YES & YES & YES & $1.22$ & $(2,2)$ & NO & 1538\\
$(41,15)$ & 8 & $(4,1)$ & 3 & 1 & YES & YES & NO(2) & $1.00$ & $(2,2)$ & -- & 1539\\
$(41,16)$ & 8 & $(4,1)$ & 3 & 1 & YES & YES & YES & $1.11$ & $(2,2)$ & NO & 1540\\
$(41,16)$ & 8 & $(4,1)$ & 3 & 1 & YES & YES & YES & $1.11$ & $(2,2)$ & -- & 1541\\
$(41,17)$ & 8 & $(4,1)$ & 3 & 1 & YES & YES & YES & $1.00$ & $(2,2)$ & -- & 1542\\
$(41,11)$ & 8 & $(5,1)$ & 4 & 1 & YES & YES & YES & $1.25$ & $(2,2)$ & NO & 1543\\
$(41,11)$ & 8 & $(5,1)$ & 4 & 1 & YES & YES & YES & $1.25$ & $(2,2)$ & -- & 1544\\
$(41,11)$ & 8 & $(5,1)$ & 4 & 1 & YES & YES & YES & $1.38$ & $(2,2)$ & NO & 1545\\
$(41,11)$ & 8 & $(5,2)$ & 3 & 1 & YES & YES & YES & $1.00$ & $(2,2)$ & -- & 1546\\
$(41,11)$ & 8 & $(5,2)$ & 3 & 1 & YES & YES & YES & $1.00$ & $(2,2)$ & NO & 1547\\
$(41,12)$ & 8 & $(5,2)$ & 3 & 1 & YES & YES & YES & $1.11$ & $(2,2)$ & -- & 1548\\
$(41,13)$ & 10 & $(5,1)$ & 4 & 1 & YES & YES & YES & $1.00$ & $(2,2)$ & NO & 1549\\
$(41,16)$ & 8 & $(5,1)$ & 4 & 1 & YES & YES & NO(2) & $1.10$ & $(2,2)$ & NO & 1550\\
$(41,16)$ & 8 & $(5,1)$ & 4 & 1 & YES & YES & NO(2) & $1.10$ & $(2,2)$ & -- & 1551\\
$(41,16)$ & 8 & $(5,2)$ & 3 & 1 & YES & YES & YES & $1.11$ & $(2,2)$ & -- & 1552\\
$(41,16)$ & 8 & $(5,2)$ & 3 & 1 & YES & YES & YES & $1.22$ & $(2,2)$ & NO & 1553\\
$(41,16)$ & 8 & $(5,2)$ & 3 & 1 & YES & YES & YES & $1.00$ & $(2,2)$ & 1073 & 1554\\
$(41,17)$ & 8 & $(5,2)$ & 3 & 1 & YES & YES & NO(2) & $0.89$ & $(4,1)$ & -- & 1555\\
$(41,17)$ & 8 & $(5,2)$ & 3 & 1 & YES & YES & NO(2) & $0.89$ & $(4,1)$ & NO & 1556\\
$(41,18)$ & 8 & $(5,2)$ & 3 & 1 & YES & YES & YES & $1.00$ & $(2,2)$ & -- & 1557\\
$(41,18)$ & 8 & $(5,2)$ & 3 & 1 & YES & YES & YES & $1.12$ & $(2,2)$ & NO & 1558\\
$(41,13)$ & 10 & $(6,1)$ & 5 & 1 & YES & YES & YES & $1.00$ & $(2,2)$ & NO & 1559\\
$(41,11)$ & 8 & $(7,3)$ & 4 & 1 & YES & YES & YES & $1.12$ & $(2,2)$ & NO & 1560\\
$(41,12)$ & 8 & $(7,3)$ & 4 & 1 & YES & YES & YES & $1.33$ & $(2,2)$ & -- & 1561\\
$(41,13)$ & 10 & $(7,1)$ & 6 & 1 & YES & YES & YES & $1.00$ & $(2,2)$ & NO & 1562\\
$(41,13)$ & 10 & $(7,2)$ & 4 & 1 & YES & YES & YES & $1.00$ & $(2,2)$ & NO & 1563\\
$(41,15)$ & 8 & $(7,2)$ & 4 & 1 & YES & YES & YES & $1.12$ & $(2,2)$ & -- & 1564\\
$(41,15)$ & 8 & $(7,2)$ & 4 & 1 & YES & YES & NO(2) & $0.89$ & $(4,1)$ & NO & 1565\\
$(41,15)$ & 8 & $(7,3)$ & 4 & 1 & YES & YES & YES & $1.00$ & $(2,2)$ & NO & 1566\\
$(41,16)$ & 8 & $(7,3)$ & 4 & 1 & YES & YES & NO(2) & $1.20$ & $(2,2)$ & NO & 1567\\
$(41,17)$ & 8 & $(7,2)$ & 4 & 1 & YES & YES & YES & $1.11$ & $(2,2)$ & -- & 1568\\
$(41,18)$ & 8 & $(7,2)$ & 4 & 1 & YES & YES & YES & $1.00$ & $(2,2)$ & -- & 1569\\
$(41,11)$ & 8 & $(8,3)$ & 4 & 1 & YES & YES & YES & $1.12$ & $(2,2)$ & NO & 1570\\
$(41,16)$ & 8 & $(9,2)$ & 5 & 1 & YES & YES & YES & $1.00$ & $(2,2)$ & NO & 1571\\
$(41,18)$ & 8 & $(9,4)$ & 5 & 1 & YES & YES & NO(2) & $1.10$ & $(2,2)$ & 1322 & 1572\\
$(41,13)$ & 10 & $(10,3)$ & 5 & 1 & YES & YES & YES & $1.00$ & $(2,2)$ & NO & 1573\\
$(41,11)$ & 8 & $(11,3)$ & 5 & 1 & YES & YES & YES & $0.88$ & $(2,2)$ & 1427 & 1574\\
$(41,15)$ & 8 & $(11,2)$ & 6 & 1 & YES & YES & YES & $1.00$ & $(2,2)$ & NO & 1575\\
$(41,18)$ & 8 & $(11,5)$ & 6 & 1 & YES & YES & YES & $0.88$ & $(4,1)$ & NO & 1576\\
$(41,18)$ & 8 & $(12,5)$ & 5 & 1 & YES & YES & YES & $1.00$ & $(2,2)$ & NO & 1577\\
$(41,13)$ & 10 & $(13,4)$ & 6 & 1 & YES & YES & YES & $1.00$ & $(2,2)$ & 1135 & 1578\\
$(41,9)$ & 9 & $(16,3)$ & 7 & 1 & YES & YES & YES & $0.88$ & $(2,2)$ & NO & 1579\\
$(41,9)$ & 9 & $(17,3)$ & 7 & 1 & YES & YES & YES & $0.88$ & $(2,2)$ & NO & 1580\\
$(41,16)$ & 8 & $(18,7)$ & 6 & 1 & YES & YES & YES & $0.88$ & $(2,2)$ & NO & 1581\\
$(41,13)$ & 10 & $(19,6)$ & 8 & 1 & YES & YES & YES & $1.00$ & $(2,2)$ & NO & 1582\\
$(41,15)$ & 8 & $(19,7)$ & 6 & 1 & YES & YES & YES & $0.88$ & $(4,1)$ & 1824 & 1583\\
$(41,16)$ & 8 & $(21,8)$ & 6 & 1 & YES & YES & YES & $1.11$ & $(2,2)$ & NO & 1584\\
$(41,11)$ & 8 & $(23,6)$ & 8 & 1 & YES & YES & YES & $0.75$ & $(4,1)$ & NO & 1585\\
$(41,9)$ & 9 & $(24,5)$ & 8 & 1 & YES & YES & YES & $0.88$ & $(2,2)$ & NO & 1586\\
$(41,18)$ & 8 & $(25,11)$ & 7 & 1 & YES & YES & NO(2) & $1.00$ & $(2,2)$ & NO & 1587\\
$(41,11)$ & 8 & $(26,7)$ & 7 & 1 & YES & YES & YES & $0.88$ & $(2,2)$ & NO & 1588\\
$(41,16)$ & 8 & $(28,11)$ & 8 & 1 & YES & YES & YES & $0.88$ & $(2,2)$ & NO & 1589\\
$(41,17)$ & 8 & $(29,12)$ & 7 & 1 & YES & YES & YES & $1.00$ & $(2,2)$ & NO & 1590\\
$(41,15)$ & 8 & $(30,11)$ & 7 & 1 & YES & YES & NO(2) & $1.10$ & $(2,2)$ & NO & 1591\\
$(41,9)$ & 9 & $(33,7)$ & 8 & 1 & YES & YES & YES & $0.88$ & $(2,2)$ & 2485 & 1592\\
$(41,11)$ & 8 & $(34,9)$ & 8 & 1 & YES & YES & YES & $0.88$ & $(2,2)$ & NO & 1593\\
$(41,11)$ & 8 & $(41,11)$ & 8 & 41 & YES & YES & YES & $1.12$ & $(2,2)$ & NO & 1594\\
$(41,13)$ & 10 & $(41,13)$ & 10 & 41 & YES & YES & YES & $1.00$ & $(2,2)$ & NO & 1595\\
$(41,17)$ & 8 & $(41,17)$ & 8 & 41 & YES & YES & YES & $1.00$ & $(2,2)$ & NO & 1596\\
$(42,13)$ & 9 & $(2,1)$ & 1 & 2 & YES & YES & YES & $1.00$ & $(4,1)$ & NO & 1597\\
$(42,19)$ & 9 & $(2,1)$ & 1 & 2 & YES & YES & YES & $1.00$ & $(4,1)$ & -- & 1598\\
$(42,13)$ & 9 & $(3,1)$ & 2 & 3 & YES & YES & YES & $1.12$ & $(2,2)$ & NO & 1599\\
$(42,13)$ & 9 & $(3,1)$ & 2 & 3 & YES & YES & YES & $1.12$ & $(2,2)$ & -- & 1600\\
$(42,19)$ & 9 & $(3,1)$ & 2 & 3 & YES & YES & YES & $1.00$ & $(4,1)$ & NO & 1601\\
$(42,19)$ & 9 & $(3,1)$ & 2 & 3 & YES & YES & YES & $1.00$ & $(4,1)$ & -- & 1602\\
$(42,13)$ & 9 & $(4,1)$ & 3 & 2 & YES & YES & YES & $1.12$ & $(2,2)$ & NO & 1603\\
$(42,13)$ & 9 & $(4,1)$ & 3 & 2 & YES & YES & YES & $1.12$ & $(2,2)$ & -- & 1604\\
$(42,19)$ & 9 & $(4,1)$ & 3 & 2 & YES & YES & YES & $1.00$ & $(4,1)$ & NO & 1605\\
$(42,19)$ & 9 & $(4,1)$ & 3 & 2 & YES & YES & YES & $1.00$ & $(4,1)$ & -- & 1606\\
$(42,13)$ & 9 & $(5,2)$ & 3 & 1 & YES & YES & YES & $1.00$ & $(2,2)$ & -- & 1607\\
$(42,19)$ & 9 & $(5,2)$ & 3 & 1 & YES & YES & YES & $1.00$ & $(4,1)$ & NO & 1608\\
$(42,19)$ & 9 & $(6,1)$ & 5 & 6 & YES & YES & YES & $1.00$ & $(2,2)$ & NO & 1609\\
$(42,19)$ & 9 & $(6,1)$ & 5 & 6 & YES & YES & YES & $1.00$ & $(2,2)$ & NO & 1610\\
$(42,11)$ & 9 & $(7,2)$ & 4 & 7 & YES & YES & YES & $0.88$ & $(2,2)$ & -- & 1611\\
$(42,13)$ & 9 & $(7,2)$ & 4 & 7 & YES & YES & YES & $1.12$ & $(2,2)$ & NO & 1612\\
$(42,19)$ & 9 & $(7,3)$ & 4 & 7 & YES & YES & YES & $1.00$ & $(4,1)$ & NO & 1613\\
$(42,19)$ & 9 & $(9,4)$ & 5 & 3 & YES & YES & YES & $1.00$ & $(4,1)$ & NO & 1614\\
$(42,13)$ & 9 & $(10,3)$ & 5 & 2 & YES & YES & YES & $1.12$ & $(2,2)$ & NO & 1615\\
$(42,19)$ & 9 & $(11,5)$ & 6 & 1 & YES & YES & YES & $1.12$ & $(2,2)$ & NO & 1616\\
$(42,13)$ & 9 & $(19,6)$ & 8 & 1 & YES & YES & YES & $1.12$ & $(2,2)$ & 2213 & 1617\\
$(42,19)$ & 9 & $(20,9)$ & 7 & 2 & YES & YES & YES & $0.88$ & $(4,1)$ & 1895 & 1618\\
$(42,13)$ & 9 & $(29,9)$ & 8 & 1 & YES & YES & YES & $1.12$ & $(2,2)$ & NO & 1619\\
$(42,13)$ & 9 & $(42,13)$ & 9 & 42 & YES & YES & YES & $1.12$ & $(2,2)$ & NO & 1620\\
$(42,19)$ & 9 & $(42,19)$ & 9 & 42 & YES & YES & YES & $1.00$ & $(2,2)$ & NO & 1621\\
$(43,19)$ & 9 & $(2,1)$ & 1 & 1 & YES & YES & NO(2) & $1.11$ & $(4,1)$ & -- & 1622\\
$(43,12)$ & 8 & $(3,1)$ & 2 & 1 & YES & YES & YES & $1.00$ & $(2,2)$ & NO & 1623\\
$(43,12)$ & 8 & $(3,1)$ & 2 & 1 & YES & YES & YES & $1.00$ & $(2,2)$ & -- & 1624\\
$(43,19)$ & 9 & $(3,1)$ & 2 & 1 & YES & YES & YES & $1.22$ & $(2,2)$ & NO & 1625\\
$(43,19)$ & 9 & $(3,1)$ & 2 & 1 & YES & YES & YES & $1.22$ & $(2,2)$ & -- & 1626\\
$(43,12)$ & 8 & $(4,1)$ & 3 & 1 & YES & YES & YES & $0.75$ & $(4,1)$ & NO & 1627\\
$(43,12)$ & 8 & $(4,1)$ & 3 & 1 & YES & YES & YES & $0.75$ & $(4,1)$ & -- & 1628\\
$(43,12)$ & 8 & $(4,1)$ & 3 & 1 & YES & YES & YES & $0.88$ & $(4,1)$ & 964 & 1629\\
$(43,16)$ & 9 & $(4,1)$ & 3 & 1 & YES & YES & NO(2) & $1.00$ & $(4,1)$ & -- & 1630\\
$(43,19)$ & 9 & $(4,1)$ & 3 & 1 & YES & YES & YES & $1.12$ & $(2,2)$ & NO & 1631\\
$(43,19)$ & 9 & $(4,1)$ & 3 & 1 & YES & YES & YES & $1.11$ & $(2,2)$ & -- & 1632\\
$(43,12)$ & 8 & $(5,2)$ & 3 & 1 & YES & YES & YES & $0.88$ & $(4,1)$ & -- & 1633\\
$(43,16)$ & 9 & $(5,1)$ & 4 & 1 & YES & YES & NO(2) & $1.20$ & $(2,2)$ & -- & 1634\\
$(43,18)$ & 8 & $(5,2)$ & 3 & 1 & YES & YES & YES & $1.33$ & $(2,2)$ & -- & 1635\\
$(43,19)$ & 9 & $(5,1)$ & 4 & 1 & YES & YES & YES & $1.00$ & $(2,2)$ & -- & 1636\\
$(43,13)$ & 9 & $(6,1)$ & 5 & 1 & YES & YES & YES & $1.11$ & $(2,2)$ & NO & 1637\\
$(43,13)$ & 9 & $(6,1)$ & 5 & 1 & YES & YES & YES & $1.11$ & $(2,2)$ & -- & 1638\\
$(43,16)$ & 9 & $(6,1)$ & 5 & 1 & YES & YES & NO(2) & $1.00$ & $(4,1)$ & -- & 1639\\
$(43,19)$ & 9 & $(6,1)$ & 5 & 1 & YES & YES & YES & $1.12$ & $(2,2)$ & NO & 1640\\
$(43,19)$ & 9 & $(6,1)$ & 5 & 1 & YES & YES & YES & $1.12$ & $(2,2)$ & NO & 1641\\
$(43,12)$ & 8 & $(7,3)$ & 4 & 1 & YES & YES & YES & $1.33$ & $(2,2)$ & -- & 1642\\
$(43,16)$ & 9 & $(7,1)$ & 6 & 1 & YES & YES & YES & $1.11$ & $(2,2)$ & NO & 1643\\
$(43,18)$ & 8 & $(7,2)$ & 4 & 1 & YES & YES & YES & $1.33$ & $(2,2)$ & NO & 1644\\
$(43,19)$ & 9 & $(7,1)$ & 6 & 1 & YES & YES & NO(2) & $1.00$ & $(4,1)$ & NO & 1645\\
$(43,19)$ & 9 & $(7,1)$ & 6 & 1 & YES & YES & NO(2) & $1.00$ & $(4,1)$ & NO & 1646\\
$(43,19)$ & 9 & $(7,3)$ & 4 & 1 & YES & YES & YES & $1.22$ & $(2,2)$ & NO & 1647\\
$(43,19)$ & 9 & $(9,4)$ & 5 & 1 & YES & YES & NO(2) & $1.11$ & $(4,1)$ & NO & 1648\\
$(43,13)$ & 9 & $(11,2)$ & 6 & 1 & YES & YES & YES & $0.88$ & $(2,2)$ & -- & 1649\\
$(43,12)$ & 8 & $(13,4)$ & 6 & 1 & YES & YES & YES & $1.00$ & $(2,2)$ & NO & 1650\\
$(43,10)$ & 9 & $(14,3)$ & 6 & 1 & YES & YES & YES & $1.00$ & $(2,2)$ & -- & 1651\\
$(43,19)$ & 9 & $(16,7)$ & 6 & 1 & YES & YES & YES & $1.12$ & $(2,2)$ & NO & 1652\\
$(43,12)$ & 8 & $(17,5)$ & 6 & 1 & YES & YES & YES & $1.12$ & $(2,2)$ & NO & 1653\\
$(43,10)$ & 9 & $(19,4)$ & 7 & 1 & YES & YES & YES & $1.00$ & $(2,2)$ & NO & 1654\\
$(43,16)$ & 9 & $(19,7)$ & 6 & 1 & YES & YES & NO(2) & $1.00$ & $(4,1)$ & NO & 1655\\
$(43,19)$ & 9 & $(25,11)$ & 7 & 1 & YES & YES & YES & $1.00$ & $(2,2)$ & 2024 & 1656\\
$(43,18)$ & 8 & $(26,11)$ & 7 & 1 & YES & YES & YES & $1.22$ & $(2,2)$ & 2284 & 1657\\
$(43,16)$ & 9 & $(27,10)$ & 7 & 1 & YES & YES & NO(2) & $1.20$ & $(2,2)$ & 2073 & 1658\\
$(43,19)$ & 9 & $(34,15)$ & 8 & 1 & YES & YES & YES & $1.12$ & $(2,2)$ & NO & 1659\\
$(43,10)$ & 9 & $(35,8)$ & 8 & 1 & YES & YES & YES & $1.00$ & $(2,2)$ & NO & 1660\\
$(43,16)$ & 9 & $(35,13)$ & 8 & 1 & YES & YES & YES & $1.11$ & $(2,2)$ & NO & 1661\\
$(43,19)$ & 9 & $(43,19)$ & 9 & 43 & YES & YES & YES & $1.22$ & $(2,2)$ & NO & 1662\\
$(44,17)$ & 8 & $(2,1)$ & 1 & 2 & YES & YES & NO(2) & $0.89$ & $(4,1)$ & -- & 1663\\
$(44,17)$ & 8 & $(3,1)$ & 2 & 1 & YES & YES & NO(2) & $0.89$ & $(4,1)$ & NO & 1664\\
$(44,17)$ & 8 & $(3,1)$ & 2 & 1 & YES & YES & NO(2) & $0.89$ & $(4,1)$ & -- & 1665\\
$(44,17)$ & 8 & $(4,1)$ & 3 & 4 & YES & YES & YES & $1.11$ & $(2,2)$ & -- & 1666\\
$(44,19)$ & 10 & $(4,1)$ & 3 & 4 & YES & YES & YES & $1.25$ & $(2,2)$ & -- & 1667\\
$(44,19)$ & 10 & $(4,1)$ & 3 & 4 & YES & YES & YES & $1.25$ & $(2,2)$ & NO & 1668\\
$(44,17)$ & 8 & $(5,2)$ & 3 & 1 & YES & YES & NO(2) & $0.89$ & $(4,1)$ & NO & 1669\\
$(44,19)$ & 10 & $(5,1)$ & 4 & 1 & YES & YES & YES & $1.12$ & $(2,2)$ & NO & 1670\\
$(44,19)$ & 10 & $(5,1)$ & 4 & 1 & YES & YES & YES & $1.33$ & $(2,2)$ & NO & 1671\\
$(44,19)$ & 10 & $(5,1)$ & 4 & 1 & YES & YES & YES & $1.33$ & $(2,2)$ & -- & 1672\\
$(44,19)$ & 10 & $(6,1)$ & 5 & 2 & YES & YES & YES & $1.12$ & $(2,2)$ & -- & 1673\\
$(44,19)$ & 10 & $(6,1)$ & 5 & 2 & YES & YES & YES & $1.33$ & $(2,2)$ & NO & 1674\\
$(44,17)$ & 8 & $(7,2)$ & 4 & 1 & YES & YES & YES & $1.33$ & $(2,2)$ & -- & 1675\\
$(44,17)$ & 8 & $(7,3)$ & 4 & 1 & YES & YES & YES & $1.11$ & $(2,2)$ & NO & 1676\\
$(44,19)$ & 10 & $(7,1)$ & 6 & 1 & YES & YES & YES & $1.25$ & $(2,2)$ & NO & 1677\\
$(44,19)$ & 10 & $(7,1)$ & 6 & 1 & YES & YES & YES & $1.25$ & $(2,2)$ & NO & 1678\\
$(44,19)$ & 10 & $(16,7)$ & 6 & 4 & YES & YES & YES & $1.25$ & $(2,2)$ & NO & 1679\\
$(44,19)$ & 10 & $(23,10)$ & 7 & 1 & YES & YES & YES & $1.12$ & $(2,2)$ & NO & 1680\\
$(44,19)$ & 10 & $(30,13)$ & 8 & 2 & YES & YES & YES & $1.12$ & $(2,2)$ & 2181 & 1681\\
$(44,17)$ & 8 & $(31,12)$ & 7 & 1 & YES & YES & YES & $1.11$ & $(2,2)$ & NO & 1682\\
$(44,19)$ & 10 & $(37,16)$ & 9 & 1 & YES & YES & YES & $1.25$ & $(2,2)$ & NO & 1683\\
$(45,19)$ & 8 & $(2,1)$ & 1 & 1 & YES & YES & NO(2) & $1.10$ & $(2,2)$ & -- & 1684\\
$(45,19)$ & 8 & $(2,1)$ & 1 & 1 & YES & YES & NO(2) & $1.10$ & $(2,2)$ & NO & 1685\\
$(45,13)$ & 10 & $(4,1)$ & 3 & 1 & YES & YES & YES & $1.12$ & $(2,2)$ & -- & 1686\\
$(45,17)$ & 9 & $(4,1)$ & 3 & 1 & YES & YES & YES & $1.25$ & $(2,2)$ & -- & 1687\\
$(45,19)$ & 8 & $(4,1)$ & 3 & 1 & YES & YES & YES & $0.75$ & $(2,2)$ & -- & 1688\\
$(45,19)$ & 8 & $(4,1)$ & 3 & 1 & YES & YES & YES & $0.88$ & $(2,2)$ & NO & 1689\\
$(45,14)$ & 9 & $(5,2)$ & 3 & 5 & YES & YES & YES & $1.38$ & $(2,2)$ & NO & 1690\\
$(45,14)$ & 9 & $(5,2)$ & 3 & 5 & YES & YES & YES & $1.38$ & $(2,2)$ & -- & 1691\\
$(45,16)$ & 9 & $(5,1)$ & 4 & 5 & YES & YES & YES & $0.88$ & $(4,1)$ & -- & 1692\\
$(45,17)$ & 9 & $(5,1)$ & 4 & 5 & YES & YES & YES & $1.00$ & $(2,2)$ & -- & 1693\\
$(45,17)$ & 9 & $(5,1)$ & 4 & 5 & YES & YES & YES & $1.12$ & $(2,2)$ & NO & 1694\\
$(45,19)$ & 8 & $(5,1)$ & 4 & 5 & YES & YES & NO(2) & $1.00$ & $(2,2)$ & NO & 1695\\
$(45,19)$ & 8 & $(5,2)$ & 3 & 5 & YES & YES & YES & $1.11$ & $(2,2)$ & -- & 1696\\
$(45,8)$ & 9 & $(6,1)$ & 5 & 3 & NO & YES & YES & $1.00$ & $(2,2)$ & -- & 1697\\
$(45,13)$ & 10 & $(6,1)$ & 5 & 3 & YES & YES & YES & $1.12$ & $(2,2)$ & NO & 1698\\
$(45,17)$ & 9 & $(6,1)$ & 5 & 3 & YES & YES & YES & $1.12$ & $(2,2)$ & NO & 1699\\
$(45,17)$ & 9 & $(6,1)$ & 5 & 3 & YES & YES & YES & $1.12$ & $(2,2)$ & NO & 1700\\
$(45,16)$ & 9 & $(7,2)$ & 4 & 1 & YES & YES & YES & $1.00$ & $(2,2)$ & NO & 1701\\
$(45,16)$ & 9 & $(7,3)$ & 4 & 1 & YES & YES & YES & $1.00$ & $(2,2)$ & NO & 1702\\
$(45,19)$ & 8 & $(7,2)$ & 4 & 1 & YES & YES & YES & $1.11$ & $(2,2)$ & NO & 1703\\
$(45,19)$ & 8 & $(7,3)$ & 4 & 1 & YES & YES & NO(2) & $1.10$ & $(2,2)$ & 1269 & 1704\\
$(45,19)$ & 8 & $(9,4)$ & 5 & 9 & YES & YES & YES & $1.00$ & $(2,2)$ & NO & 1705\\
$(45,16)$ & 9 & $(17,6)$ & 7 & 1 & YES & YES & YES & $0.88$ & $(4,1)$ & 1773 & 1706\\
$(45,14)$ & 9 & $(19,6)$ & 8 & 1 & YES & YES & YES & $1.25$ & $(2,2)$ & NO & 1707\\
$(45,16)$ & 9 & $(20,7)$ & 8 & 5 & YES & YES & YES & $1.00$ & $(2,2)$ & 2305 & 1708\\
$(45,17)$ & 9 & $(21,8)$ & 6 & 3 & YES & YES & YES & $1.25$ & $(2,2)$ & NO & 1709\\
$(45,13)$ & 10 & $(24,7)$ & 7 & 3 & YES & YES & YES & $1.12$ & $(2,2)$ & NO & 1710\\
$(45,19)$ & 8 & $(26,11)$ & 7 & 1 & YES & YES & YES & $0.88$ & $(2,2)$ & NO & 1711\\
$(45,17)$ & 9 & $(29,11)$ & 7 & 1 & YES & YES & YES & $1.00$ & $(2,2)$ & 2167 & 1712\\
$(45,19)$ & 8 & $(31,13)$ & 7 & 1 & YES & YES & YES & $1.22$ & $(2,2)$ & NO & 1713\\
$(45,17)$ & 9 & $(37,14)$ & 8 & 1 & YES & YES & YES & $1.12$ & $(2,2)$ & NO & 1714\\
$(45,13)$ & 10 & $(38,11)$ & 9 & 1 & YES & YES & YES & $1.12$ & $(2,2)$ & NO & 1715\\
$(46,19)$ & 8 & $(2,1)$ & 1 & 2 & YES & YES & YES & $1.11$ & $(2,2)$ & -- & 1716\\
$(46,17)$ & 8 & $(3,1)$ & 2 & 1 & YES & YES & YES & $1.00$ & $(2,2)$ & -- & 1717\\
$(46,19)$ & 8 & $(3,1)$ & 2 & 1 & YES & YES & YES & $1.11$ & $(2,2)$ & NO & 1718\\
$(46,19)$ & 8 & $(3,1)$ & 2 & 1 & YES & YES & YES & $1.11$ & $(2,2)$ & -- & 1719\\
$(46,19)$ & 8 & $(3,1)$ & 2 & 1 & YES & YES & YES & $1.11$ & $(2,2)$ & NO & 1720\\
$(46,13)$ & 10 & $(4,1)$ & 3 & 2 & YES & YES & YES & $1.00$ & $(2,2)$ & -- & 1721\\
$(46,17)$ & 8 & $(4,1)$ & 3 & 2 & YES & YES & YES & $1.11$ & $(2,2)$ & -- & 1722\\
$(46,17)$ & 8 & $(4,1)$ & 3 & 2 & YES & YES & YES & $1.11$ & $(2,2)$ & NO & 1723\\
$(46,17)$ & 8 & $(5,2)$ & 3 & 1 & YES & YES & YES & $1.00$ & $(2,2)$ & -- & 1724\\
$(46,17)$ & 8 & $(5,2)$ & 3 & 1 & YES & YES & YES & $1.11$ & $(2,2)$ & NO & 1725\\
$(46,19)$ & 8 & $(5,2)$ & 3 & 1 & YES & YES & YES & $1.22$ & $(2,2)$ & -- & 1726\\
$(46,19)$ & 8 & $(5,2)$ & 3 & 1 & YES & YES & YES & $1.11$ & $(2,2)$ & NO & 1727\\
$(46,21)$ & 10 & $(6,1)$ & 5 & 2 & YES & YES & YES & $1.00$ & $(2,2)$ & -- & 1728\\
$(46,11)$ & 10 & $(7,3)$ & 4 & 1 & YES & YES & YES & $1.12$ & $(2,2)$ & -- & 1729\\
$(46,17)$ & 8 & $(7,2)$ & 4 & 1 & YES & YES & YES & $1.00$ & $(2,2)$ & -- & 1730\\
$(46,17)$ & 8 & $(7,2)$ & 4 & 1 & YES & YES & YES & $1.12$ & $(2,2)$ & NO & 1731\\
$(46,19)$ & 8 & $(7,2)$ & 4 & 1 & YES & YES & YES & $1.00$ & $(2,2)$ & NO & 1732\\
$(46,17)$ & 8 & $(9,2)$ & 5 & 1 & YES & YES & YES & $1.11$ & $(2,2)$ & -- & 1733\\
$(46,19)$ & 8 & $(9,4)$ & 5 & 1 & YES & YES & YES & $1.00$ & $(2,2)$ & 2287 & 1734\\
$(46,17)$ & 8 & $(11,4)$ & 5 & 1 & YES & YES & YES & $1.12$ & $(2,2)$ & NO & 1735\\
$(46,21)$ & 10 & $(11,5)$ & 6 & 1 & YES & YES & YES & $1.12$ & $(2,2)$ & NO & 1736\\
$(46,17)$ & 8 & $(14,5)$ & 6 & 2 & YES & YES & YES & $0.88$ & $(2,2)$ & NO & 1737\\
$(46,17)$ & 8 & $(27,10)$ & 7 & 1 & YES & YES & NO(2) & $1.10$ & $(2,2)$ & NO & 1738\\
$(46,17)$ & 8 & $(30,11)$ & 7 & 2 & YES & YES & YES & $0.88$ & $(2,2)$ & NO & 1739\\
$(46,17)$ & 8 & $(35,13)$ & 8 & 1 & YES & YES & YES & $1.11$ & $(2,2)$ & NO & 1740\\
$(46,13)$ & 10 & $(39,11)$ & 9 & 1 & YES & YES & YES & $1.00$ & $(2,2)$ & NO & 1741\\
$(46,19)$ & 8 & $(41,17)$ & 8 & 1 & YES & YES & YES & $1.11$ & $(2,2)$ & NO & 1742\\
$(46,21)$ & 10 & $(46,21)$ & 10 & 46 & YES & YES & YES & $1.00$ & $(2,2)$ & NO & 1743\\
$(47,22)$ & 11 & $(2,1)$ & 1 & 1 & NO & YES & YES & $1.00$ & $(2,2)$ & -- & 1744\\
$(47,11)$ & 9 & $(3,1)$ & 2 & 1 & YES & YES & YES & $1.12$ & $(2,2)$ & NO & 1745\\
$(47,11)$ & 9 & $(3,1)$ & 2 & 1 & YES & YES & YES & $1.12$ & $(2,2)$ & -- & 1746\\
$(47,11)$ & 9 & $(3,1)$ & 2 & 1 & YES & YES & YES & $1.38$ & $(2,2)$ & NO & 1747\\
$(47,11)$ & 9 & $(4,1)$ & 3 & 1 & YES & YES & YES & $1.12$ & $(2,2)$ & NO & 1748\\
$(47,11)$ & 9 & $(4,1)$ & 3 & 1 & YES & YES & YES & $1.12$ & $(2,2)$ & -- & 1749\\
$(47,18)$ & 8 & $(5,2)$ & 3 & 1 & YES & YES & YES & $1.22$ & $(2,2)$ & NO & 1750\\
$(47,18)$ & 8 & $(5,2)$ & 3 & 1 & YES & YES & YES & $1.44$ & $(2,2)$ & -- & 1751\\
$(47,17)$ & 9 & $(6,1)$ & 5 & 1 & YES & YES & YES & $1.11$ & $(2,2)$ & -- & 1752\\
$(47,17)$ & 9 & $(6,1)$ & 5 & 1 & YES & YES & YES & $0.88$ & $(2,2)$ & NO & 1753\\
$(47,18)$ & 8 & $(7,2)$ & 4 & 1 & YES & YES & YES & $1.33$ & $(2,2)$ & NO & 1754\\
$(47,14)$ & 9 & $(9,2)$ & 5 & 1 & YES & YES & YES & $1.33$ & $(2,2)$ & NO & 1755\\
$(47,14)$ & 9 & $(9,2)$ & 5 & 1 & YES & YES & YES & $1.33$ & $(2,2)$ & -- & 1756\\
$(47,14)$ & 9 & $(24,7)$ & 7 & 1 & YES & YES & YES & $1.33$ & $(2,2)$ & NO & 1757\\
$(47,11)$ & 9 & $(30,7)$ & 8 & 1 & YES & YES & YES & $1.12$ & $(2,2)$ & NO & 1758\\
$(47,17)$ & 9 & $(36,13)$ & 8 & 1 & YES & YES & YES & $0.88$ & $(2,2)$ & NO & 1759\\
$(47,11)$ & 9 & $(47,11)$ & 9 & 47 & YES & YES & YES & $1.25$ & $(2,2)$ & NO & 1760\\
$(48,17)$ & 9 & $(2,1)$ & 1 & 2 & YES & YES & YES & $0.88$ & $(4,1)$ & -- & 1761\\
$(48,17)$ & 9 & $(2,1)$ & 1 & 2 & YES & YES & YES & $1.00$ & $(2,2)$ & NO & 1762\\
$(48,17)$ & 9 & $(3,1)$ & 2 & 3 & YES & YES & YES & $1.00$ & $(2,2)$ & -- & 1763\\
$(48,17)$ & 9 & $(3,1)$ & 2 & 3 & YES & YES & YES & $1.25$ & $(2,2)$ & NO & 1764\\
$(48,17)$ & 9 & $(3,1)$ & 2 & 3 & YES & YES & YES & $1.00$ & $(2,2)$ & NO & 1765\\
$(48,7)$ & 12 & $(4,1)$ & 3 & 4 & YES & YES & YES & $0.88$ & $(2,2)$ & NO & 1766\\
$(48,17)$ & 9 & $(4,1)$ & 3 & 4 & YES & YES & YES & $1.00$ & $(2,2)$ & -- & 1767\\
$(48,17)$ & 9 & $(5,1)$ & 4 & 1 & YES & YES & YES & $1.12$ & $(2,2)$ & -- & 1768\\
$(48,17)$ & 9 & $(5,1)$ & 4 & 1 & YES & YES & YES & $1.12$ & $(2,2)$ & NO & 1769\\
$(48,17)$ & 9 & $(5,2)$ & 3 & 1 & YES & YES & YES & $1.12$ & $(2,2)$ & NO & 1770\\
$(48,17)$ & 9 & $(6,1)$ & 5 & 6 & YES & YES & YES & $0.88$ & $(2,2)$ & NO & 1771\\
$(48,17)$ & 9 & $(11,4)$ & 5 & 1 & YES & YES & YES & $1.00$ & $(2,2)$ & 2105 & 1772\\
$(48,17)$ & 9 & $(14,5)$ & 6 & 2 & YES & YES & YES & $0.88$ & $(4,1)$ & 1706 & 1773\\
$(48,17)$ & 9 & $(17,6)$ & 7 & 1 & YES & YES & YES & $0.88$ & $(4,1)$ & NO & 1774\\
$(48,17)$ & 9 & $(20,7)$ & 8 & 4 & YES & YES & YES & $1.12$ & $(2,2)$ & NO & 1775\\
$(48,17)$ & 9 & $(31,11)$ & 8 & 1 & YES & YES & YES & $1.00$ & $(2,2)$ & NO & 1776\\
$(48,17)$ & 9 & $(48,17)$ & 9 & 48 & YES & YES & YES & $1.00$ & $(2,2)$ & NO & 1777\\
$(49,15)$ & 9 & $(2,1)$ & 1 & 1 & YES & YES & NO(2) & $1.20$ & $(2,2)$ & NO & 1778\\
$(49,18)$ & 8 & $(2,1)$ & 1 & 1 & YES & YES & YES & $1.00$ & $(2,2)$ & -- & 1779\\
$(49,19)$ & 8 & $(2,1)$ & 1 & 1 & YES & YES & NO(2) & $1.10$ & $(2,2)$ & -- & 1780\\
$(49,19)$ & 8 & $(2,1)$ & 1 & 1 & YES & YES & NO(2) & $1.10$ & $(2,2)$ & NO & 1781\\
$(49,20)$ & 9 & $(2,1)$ & 1 & 1 & YES & YES & YES & $1.00$ & $(2,2)$ & -- & 1782\\
$(49,20)$ & 9 & $(2,1)$ & 1 & 1 & YES & YES & NO(2) & $1.00$ & $(4,1)$ & NO & 1783\\
$(49,22)$ & 9 & $(2,1)$ & 1 & 1 & YES & YES & YES & $0.88$ & $(2,2)$ & -- & 1784\\
$(49,22)$ & 9 & $(2,1)$ & 1 & 1 & YES & YES & YES & $1.00$ & $(2,2)$ & 1130 & 1785\\
$(49,15)$ & 9 & $(3,1)$ & 2 & 1 & YES & YES & NO(2) & $1.20$ & $(2,2)$ & NO & 1786\\
$(49,18)$ & 8 & $(3,1)$ & 2 & 1 & YES & YES & YES & $1.00$ & $(2,2)$ & 1481 & 1787\\
$(49,18)$ & 8 & $(3,1)$ & 2 & 1 & YES & YES & YES & $1.00$ & $(2,2)$ & -- & 1788\\
$(49,18)$ & 8 & $(3,1)$ & 2 & 1 & YES & YES & YES & $1.00$ & $(2,2)$ & NO & 1789\\
$(49,19)$ & 8 & $(3,1)$ & 2 & 1 & YES & YES & YES & $1.11$ & $(2,2)$ & NO & 1790\\
$(49,19)$ & 8 & $(3,1)$ & 2 & 1 & YES & YES & YES & $1.11$ & $(2,2)$ & -- & 1791\\
$(49,20)$ & 9 & $(3,1)$ & 2 & 1 & YES & YES & YES & $1.00$ & $(2,2)$ & -- & 1792\\
$(49,20)$ & 9 & $(3,1)$ & 2 & 1 & YES & YES & NO(2) & $1.00$ & $(4,1)$ & NO & 1793\\
$(49,22)$ & 9 & $(3,1)$ & 2 & 1 & YES & YES & YES & $1.12$ & $(2,2)$ & NO & 1794\\
$(49,22)$ & 9 & $(3,1)$ & 2 & 1 & YES & YES & YES & $1.12$ & $(2,2)$ & -- & 1795\\
$(49,22)$ & 9 & $(3,1)$ & 2 & 1 & YES & YES & YES & $0.88$ & $(4,1)$ & NO & 1796\\
$(49,13)$ & 9 & $(4,1)$ & 3 & 1 & YES & YES & YES & $0.88$ & $(4,1)$ & NO & 1797\\
$(49,13)$ & 9 & $(4,1)$ & 3 & 1 & YES & YES & YES & $0.88$ & $(4,1)$ & -- & 1798\\
$(49,18)$ & 8 & $(4,1)$ & 3 & 1 & YES & YES & YES & $0.88$ & $(2,2)$ & -- & 1799\\
$(49,19)$ & 8 & $(4,1)$ & 3 & 1 & YES & YES & YES & $1.00$ & $(2,2)$ & -- & 1800\\
$(49,20)$ & 9 & $(4,1)$ & 3 & 1 & YES & YES & YES & $1.00$ & $(2,2)$ & NO & 1801\\
$(49,22)$ & 9 & $(4,1)$ & 3 & 1 & YES & YES & YES & $1.00$ & $(2,2)$ & -- & 1802\\
$(49,22)$ & 9 & $(4,1)$ & 3 & 1 & YES & YES & YES & $1.00$ & $(2,2)$ & NO & 1803\\
$(49,15)$ & 9 & $(5,1)$ & 4 & 1 & YES & YES & YES & $0.88$ & $(4,1)$ & NO & 1804\\
$(49,15)$ & 9 & $(5,1)$ & 4 & 1 & YES & YES & YES & $0.88$ & $(4,1)$ & -- & 1805\\
$(49,15)$ & 9 & $(5,2)$ & 3 & 1 & YES & YES & YES & $1.00$ & $(2,2)$ & -- & 1806\\
$(49,18)$ & 8 & $(5,2)$ & 3 & 1 & YES & YES & YES & $1.33$ & $(2,2)$ & -- & 1807\\
$(49,19)$ & 8 & $(5,2)$ & 3 & 1 & YES & YES & YES & $1.00$ & $(2,2)$ & -- & 1808\\
$(49,19)$ & 8 & $(5,2)$ & 3 & 1 & YES & YES & NO(2) & $1.10$ & $(2,2)$ & NO & 1809\\
$(49,20)$ & 9 & $(5,1)$ & 4 & 1 & YES & YES & YES & $1.00$ & $(2,2)$ & NO & 1810\\
$(49,20)$ & 9 & $(5,1)$ & 4 & 1 & YES & YES & YES & $1.00$ & $(2,2)$ & -- & 1811\\
$(49,20)$ & 9 & $(5,2)$ & 3 & 1 & YES & YES & YES & $1.00$ & $(2,2)$ & 1237 & 1812\\
$(49,22)$ & 9 & $(5,2)$ & 3 & 1 & YES & YES & YES & $0.88$ & $(2,2)$ & NO & 1813\\
$(49,15)$ & 9 & $(6,1)$ & 5 & 1 & YES & YES & YES & $1.11$ & $(2,2)$ & NO & 1814\\
$(49,15)$ & 9 & $(6,1)$ & 5 & 1 & YES & YES & YES & $1.11$ & $(2,2)$ & -- & 1815\\
$(49,15)$ & 9 & $(6,1)$ & 5 & 1 & YES & YES & NO(2) & $1.10$ & $(2,2)$ & NO & 1816\\
$(49,18)$ & 8 & $(7,2)$ & 4 & 7 & YES & YES & YES & $1.00$ & $(2,2)$ & NO & 1817\\
$(49,18)$ & 8 & $(7,3)$ & 4 & 7 & YES & YES & YES & $0.88$ & $(2,2)$ & NO & 1818\\
$(49,19)$ & 8 & $(7,3)$ & 4 & 7 & YES & YES & YES & $1.12$ & $(2,2)$ & NO & 1819\\
$(49,20)$ & 9 & $(7,3)$ & 4 & 7 & YES & YES & YES & $1.00$ & $(2,2)$ & NO & 1820\\
$(49,22)$ & 9 & $(7,3)$ & 4 & 7 & YES & YES & YES & $1.00$ & $(2,2)$ & NO & 1821\\
$(49,9)$ & 10 & $(9,2)$ & 5 & 1 & YES & YES & NO(2) & $1.10$ & $(2,2)$ & 2050 & 1822\\
$(49,13)$ & 9 & $(9,2)$ & 5 & 1 & YES & YES & YES & $0.88$ & $(4,1)$ & NO & 1823\\
$(49,18)$ & 8 & $(11,4)$ & 5 & 1 & YES & YES & YES & $0.88$ & $(4,1)$ & 1583 & 1824\\
$(49,22)$ & 9 & $(11,5)$ & 6 & 1 & YES & YES & YES & $1.00$ & $(2,2)$ & NO & 1825\\
$(49,20)$ & 9 & $(12,5)$ & 5 & 1 & YES & YES & YES & $1.00$ & $(2,2)$ & 1140 & 1826\\
$(49,15)$ & 9 & $(13,4)$ & 6 & 1 & YES & YES & YES & $1.22$ & $(2,2)$ & NO & 1827\\
$(49,18)$ & 8 & $(13,5)$ & 5 & 1 & YES & YES & YES & $1.33$ & $(2,2)$ & NO & 1828\\
$(49,18)$ & 8 & $(14,5)$ & 6 & 7 & YES & YES & YES & $1.12$ & $(2,2)$ & NO & 1829\\
$(49,20)$ & 9 & $(17,7)$ & 6 & 1 & YES & YES & YES & $1.00$ & $(2,2)$ & NO & 1830\\
$(49,22)$ & 9 & $(20,9)$ & 7 & 1 & YES & YES & YES & $0.88$ & $(4,1)$ & NO & 1831\\
$(49,19)$ & 8 & $(21,8)$ & 6 & 7 & YES & YES & YES & $1.22$ & $(2,2)$ & NO & 1832\\
$(49,11)$ & 10 & $(22,5)$ & 7 & 1 & YES & YES & NO(2) & $1.00$ & $(2,2)$ & NO & 1833\\
$(49,20)$ & 9 & $(22,9)$ & 7 & 1 & YES & YES & YES & $1.00$ & $(2,2)$ & NO & 1834\\
$(49,9)$ & 10 & $(23,4)$ & 8 & 1 & YES & YES & NO(2) & $1.10$ & $(2,2)$ & NO & 1835\\
$(49,13)$ & 9 & $(23,6)$ & 8 & 1 & YES & YES & YES & $0.88$ & $(4,1)$ & 2374 & 1836\\
$(49,19)$ & 8 & $(23,9)$ & 7 & 1 & YES & YES & YES & $1.00$ & $(2,2)$ & NO & 1837\\
$(49,18)$ & 8 & $(27,10)$ & 7 & 1 & YES & YES & YES & $1.00$ & $(2,2)$ & NO & 1838\\
$(49,20)$ & 9 & $(27,11)$ & 8 & 1 & YES & YES & YES & $1.00$ & $(2,2)$ & NO & 1839\\
$(49,9)$ & 10 & $(28,5)$ & 8 & 7 & YES & YES & NO(2) & $0.62$ & $(6,0)$ & NO & 1840\\
$(49,22)$ & 9 & $(29,13)$ & 8 & 1 & YES & YES & YES & $1.00$ & $(2,2)$ & NO & 1841\\
$(49,18)$ & 8 & $(30,11)$ & 7 & 1 & YES & YES & YES & $1.00$ & $(2,2)$ & NO & 1842\\
$(49,15)$ & 9 & $(36,11)$ & 8 & 1 & YES & YES & NO(2) & $1.10$ & $(2,2)$ & NO & 1843\\
$(49,19)$ & 8 & $(44,17)$ & 8 & 1 & YES & YES & YES & $1.22$ & $(2,2)$ & NO & 1844\\
$(49,18)$ & 8 & $(49,18)$ & 8 & 49 & YES & YES & YES & $0.88$ & $(2,2)$ & NO & 1845\\
$(49,20)$ & 9 & $(49,20)$ & 9 & 49 & YES & YES & YES & $1.00$ & $(2,2)$ & NO & 1846\\
$(49,22)$ & 9 & $(49,22)$ & 9 & 49 & YES & YES & YES & $1.12$ & $(2,2)$ & NO & 1847\\
$(50,19)$ & 8 & $(2,1)$ & 1 & 2 & YES & YES & YES & $1.22$ & $(2,2)$ & -- & 1848\\
$(50,21)$ & 8 & $(2,1)$ & 1 & 2 & YES & YES & YES & $1.00$ & $(2,2)$ & -- & 1849\\
$(50,13)$ & 10 & $(4,1)$ & 3 & 2 & YES & YES & YES & $0.88$ & $(2,2)$ & -- & 1850\\
$(50,13)$ & 10 & $(5,1)$ & 4 & 5 & YES & YES & YES & $1.00$ & $(2,2)$ & NO & 1851\\
$(50,19)$ & 8 & $(5,2)$ & 3 & 5 & YES & YES & YES & $1.22$ & $(2,2)$ & -- & 1852\\
$(50,11)$ & 10 & $(7,2)$ & 4 & 1 & YES & YES & YES & $1.11$ & $(2,2)$ & NO & 1853\\
$(50,19)$ & 8 & $(7,2)$ & 4 & 1 & YES & YES & YES & $1.11$ & $(2,2)$ & NO & 1854\\
$(50,21)$ & 8 & $(7,3)$ & 4 & 1 & YES & YES & YES & $1.00$ & $(2,2)$ & NO & 1855\\
$(50,21)$ & 8 & $(8,3)$ & 4 & 2 & YES & YES & YES & $1.22$ & $(2,2)$ & NO & 1856\\
$(50,19)$ & 8 & $(9,2)$ & 5 & 1 & YES & YES & YES & $1.22$ & $(2,2)$ & NO & 1857\\
$(50,13)$ & 10 & $(11,3)$ & 5 & 1 & YES & YES & YES & $0.88$ & $(2,2)$ & NO & 1858\\
$(50,13)$ & 10 & $(15,4)$ & 6 & 5 & YES & YES & YES & $1.00$ & $(2,2)$ & 1327 & 1859\\
$(50,21)$ & 8 & $(17,7)$ & 6 & 1 & YES & YES & YES & $1.22$ & $(2,2)$ & NO & 1860\\
$(50,19)$ & 8 & $(21,8)$ & 6 & 1 & YES & YES & YES & $1.11$ & $(2,2)$ & NO & 1861\\
$(50,21)$ & 8 & $(26,11)$ & 7 & 2 & YES & YES & YES & $1.22$ & $(2,2)$ & NO & 1862\\
$(50,13)$ & 10 & $(27,7)$ & 9 & 1 & YES & YES & YES & $1.12$ & $(2,2)$ & NO & 1863\\
$(50,19)$ & 8 & $(37,14)$ & 8 & 1 & YES & YES & YES & $1.11$ & $(2,2)$ & NO & 1864\\
$(50,21)$ & 8 & $(43,18)$ & 8 & 1 & YES & YES & YES & $1.22$ & $(2,2)$ & NO & 1865\\
$(51,20)$ & 9 & $(2,1)$ & 1 & 1 & YES & YES & YES & $1.11$ & $(2,2)$ & -- & 1866\\
$(51,20)$ & 9 & $(2,1)$ & 1 & 1 & YES & YES & YES & $1.22$ & $(2,2)$ & NO & 1867\\
$(51,23)$ & 9 & $(2,1)$ & 1 & 1 & YES & YES & YES & $1.12$ & $(2,2)$ & -- & 1868\\
$(51,23)$ & 9 & $(2,1)$ & 1 & 1 & YES & YES & YES & $1.12$ & $(2,2)$ & 1419 & 1869\\
$(51,16)$ & 10 & $(3,1)$ & 2 & 3 & YES & YES & YES & $1.12$ & $(2,2)$ & -- & 1870\\
$(51,16)$ & 10 & $(3,1)$ & 2 & 3 & YES & YES & YES & $1.25$ & $(2,2)$ & NO & 1871\\
$(51,20)$ & 9 & $(3,1)$ & 2 & 3 & YES & YES & YES & $1.12$ & $(2,2)$ & -- & 1872\\
$(51,20)$ & 9 & $(3,1)$ & 2 & 3 & YES & YES & YES & $1.22$ & $(2,2)$ & NO & 1873\\
$(51,23)$ & 9 & $(3,1)$ & 2 & 3 & YES & YES & YES & $1.00$ & $(2,2)$ & -- & 1874\\
$(51,23)$ & 9 & $(3,1)$ & 2 & 3 & YES & YES & YES & $1.12$ & $(2,2)$ & NO & 1875\\
$(51,23)$ & 9 & $(3,1)$ & 2 & 3 & YES & YES & YES & $1.12$ & $(2,2)$ & NO & 1876\\
$(51,14)$ & 9 & $(4,1)$ & 3 & 1 & YES & YES & YES & $0.75$ & $(2,2)$ & -- & 1877\\
$(51,16)$ & 10 & $(4,1)$ & 3 & 1 & YES & YES & YES & $1.12$ & $(2,2)$ & -- & 1878\\
$(51,16)$ & 10 & $(4,1)$ & 3 & 1 & YES & YES & YES & $1.25$ & $(2,2)$ & NO & 1879\\
$(51,20)$ & 9 & $(4,1)$ & 3 & 1 & YES & YES & YES & $1.22$ & $(2,2)$ & NO & 1880\\
$(51,23)$ & 9 & $(4,1)$ & 3 & 1 & YES & YES & YES & $1.00$ & $(2,2)$ & -- & 1881\\
$(51,14)$ & 9 & $(5,2)$ & 3 & 1 & YES & YES & YES & $1.22$ & $(2,2)$ & NO & 1882\\
$(51,16)$ & 10 & $(5,2)$ & 3 & 1 & YES & YES & YES & $1.25$ & $(2,2)$ & NO & 1883\\
$(51,20)$ & 9 & $(5,1)$ & 4 & 1 & YES & YES & YES & $0.75$ & $(4,1)$ & -- & 1884\\
$(51,20)$ & 9 & $(5,1)$ & 4 & 1 & YES & YES & YES & $1.11$ & $(2,2)$ & NO & 1885\\
$(51,20)$ & 9 & $(5,2)$ & 3 & 1 & YES & YES & YES & $0.88$ & $(4,1)$ & 1271 & 1886\\
$(51,23)$ & 9 & $(5,1)$ & 4 & 1 & YES & YES & YES & $1.12$ & $(2,2)$ & NO & 1887\\
$(51,23)$ & 9 & $(5,1)$ & 4 & 1 & YES & YES & YES & $1.12$ & $(2,2)$ & -- & 1888\\
$(51,23)$ & 9 & $(5,2)$ & 3 & 1 & YES & YES & YES & $1.00$ & $(2,2)$ & 1212 & 1889\\
$(51,20)$ & 9 & $(7,3)$ & 4 & 1 & YES & YES & YES & $1.12$ & $(2,2)$ & NO & 1890\\
$(51,23)$ & 9 & $(7,3)$ & 4 & 1 & YES & YES & YES & $1.00$ & $(2,2)$ & NO & 1891\\
$(51,20)$ & 9 & $(8,3)$ & 4 & 1 & YES & YES & YES & $1.11$ & $(2,2)$ & NO & 1892\\
$(51,23)$ & 9 & $(9,4)$ & 5 & 3 & YES & YES & YES & $1.12$ & $(2,2)$ & NO & 1893\\
$(51,16)$ & 10 & $(10,3)$ & 5 & 1 & YES & YES & YES & $1.12$ & $(2,2)$ & NO & 1894\\
$(51,23)$ & 9 & $(11,5)$ & 6 & 1 & YES & YES & YES & $0.88$ & $(4,1)$ & 1618 & 1895\\
$(51,16)$ & 10 & $(13,4)$ & 6 & 1 & YES & YES & YES & $1.12$ & $(2,2)$ & NO & 1896\\
$(51,20)$ & 9 & $(13,5)$ & 5 & 1 & YES & YES & YES & $1.11$ & $(2,2)$ & 1227 & 1897\\
$(51,11)$ & 9 & $(16,3)$ & 7 & 1 & YES & YES & YES & $0.88$ & $(2,2)$ & NO & 1898\\
$(51,20)$ & 9 & $(18,7)$ & 6 & 3 & YES & YES & YES & $1.00$ & $(2,2)$ & NO & 1899\\
$(51,20)$ & 9 & $(23,9)$ & 7 & 1 & YES & YES & YES & $1.11$ & $(2,2)$ & NO & 1900\\
$(51,11)$ & 9 & $(24,5)$ & 8 & 3 & YES & YES & YES & $0.88$ & $(2,2)$ & NO & 1901\\
$(51,14)$ & 9 & $(25,7)$ & 7 & 1 & YES & YES & YES & $1.22$ & $(2,2)$ & NO & 1902\\
$(51,14)$ & 9 & $(29,8)$ & 7 & 1 & YES & YES & YES & $0.88$ & $(2,2)$ & 2189 & 1903\\
$(51,11)$ & 9 & $(33,7)$ & 8 & 3 & YES & YES & YES & $0.88$ & $(2,2)$ & NO & 1904\\
$(51,20)$ & 9 & $(51,20)$ & 9 & 51 & YES & YES & YES & $1.11$ & $(2,2)$ & NO & 1905\\
$(51,23)$ & 9 & $(51,23)$ & 9 & 51 & YES & YES & YES & $1.00$ & $(2,2)$ & NO & 1906\\
$(52,19)$ & 9 & $(2,1)$ & 1 & 2 & YES & YES & YES & $1.11$ & $(2,2)$ & -- & 1907\\
$(52,19)$ & 9 & $(2,1)$ & 1 & 2 & YES & YES & NO(2) & $1.00$ & $(4,1)$ & NO & 1908\\
$(52,19)$ & 9 & $(3,1)$ & 2 & 1 & YES & YES & YES & $1.12$ & $(2,2)$ & -- & 1909\\
$(52,19)$ & 9 & $(3,1)$ & 2 & 1 & YES & YES & YES & $1.22$ & $(2,2)$ & NO & 1910\\
$(52,19)$ & 9 & $(4,1)$ & 3 & 4 & YES & YES & YES & $1.12$ & $(2,2)$ & NO & 1911\\
$(52,19)$ & 9 & $(4,1)$ & 3 & 4 & YES & YES & YES & $1.11$ & $(2,2)$ & -- & 1912\\
$(52,19)$ & 9 & $(5,1)$ & 4 & 1 & YES & YES & YES & $1.11$ & $(2,2)$ & NO & 1913\\
$(52,19)$ & 9 & $(6,1)$ & 5 & 2 & YES & YES & YES & $1.00$ & $(2,2)$ & NO & 1914\\
$(52,19)$ & 9 & $(7,1)$ & 6 & 1 & YES & YES & YES & $1.11$ & $(2,2)$ & NO & 1915\\
$(52,19)$ & 9 & $(8,3)$ & 4 & 4 & YES & YES & YES & $1.11$ & $(2,2)$ & NO & 1916\\
$(52,19)$ & 9 & $(11,4)$ & 5 & 1 & YES & YES & YES & $1.22$ & $(2,2)$ & NO & 1917\\
$(52,19)$ & 9 & $(19,7)$ & 6 & 1 & YES & YES & YES & $1.00$ & $(2,2)$ & NO & 1918\\
$(52,19)$ & 9 & $(30,11)$ & 7 & 2 & YES & YES & YES & $1.00$ & $(2,2)$ & 2238 & 1919\\
$(52,19)$ & 9 & $(41,15)$ & 8 & 1 & YES & YES & YES & $1.11$ & $(2,2)$ & NO & 1920\\
$(52,19)$ & 9 & $(52,19)$ & 9 & 52 & YES & YES & YES & $1.12$ & $(2,2)$ & NO & 1921\\
$(53,19)$ & 9 & $(2,1)$ & 1 & 1 & YES & YES & NO(2) & $1.00$ & $(4,1)$ & NO & 1922\\
$(53,23)$ & 9 & $(2,1)$ & 1 & 1 & NO & YES & YES & $1.00$ & $(2,2)$ & -- & 1923\\
$(53,19)$ & 9 & $(3,1)$ & 2 & 1 & YES & YES & YES & $1.00$ & $(2,2)$ & -- & 1924\\
$(53,19)$ & 9 & $(3,1)$ & 2 & 1 & YES & YES & YES & $0.88$ & $(4,1)$ & NO & 1925\\
$(53,19)$ & 9 & $(3,1)$ & 2 & 1 & YES & YES & YES & $1.22$ & $(2,2)$ & NO & 1926\\
$(53,19)$ & 9 & $(4,1)$ & 3 & 1 & YES & YES & YES & $1.12$ & $(2,2)$ & -- & 1927\\
$(53,23)$ & 9 & $(4,1)$ & 3 & 1 & YES & YES & YES & $0.88$ & $(2,2)$ & NO & 1928\\
$(53,23)$ & 9 & $(4,1)$ & 3 & 1 & YES & YES & YES & $1.11$ & $(2,2)$ & -- & 1929\\
$(53,16)$ & 10 & $(5,1)$ & 4 & 1 & YES & YES & YES & $0.88$ & $(2,2)$ & -- & 1930\\
$(53,19)$ & 9 & $(5,1)$ & 4 & 1 & YES & YES & YES & $0.88$ & $(2,2)$ & -- & 1931\\
$(53,19)$ & 9 & $(5,1)$ & 4 & 1 & YES & YES & YES & $1.00$ & $(2,2)$ & NO & 1932\\
$(53,19)$ & 9 & $(5,2)$ & 3 & 1 & YES & YES & YES & $1.25$ & $(2,2)$ & NO & 1933\\
$(53,23)$ & 9 & $(5,2)$ & 3 & 1 & YES & YES & YES & $1.00$ & $(2,2)$ & NO & 1934\\
$(53,19)$ & 9 & $(6,1)$ & 5 & 1 & YES & YES & YES & $1.11$ & $(2,2)$ & -- & 1935\\
$(53,19)$ & 9 & $(6,1)$ & 5 & 1 & YES & YES & YES & $0.88$ & $(2,2)$ & NO & 1936\\
$(53,16)$ & 10 & $(7,1)$ & 6 & 1 & YES & YES & YES & $1.11$ & $(2,2)$ & NO & 1937\\
$(53,19)$ & 9 & $(8,3)$ & 4 & 1 & YES & YES & YES & $1.00$ & $(2,2)$ & NO & 1938\\
$(53,19)$ & 9 & $(14,5)$ & 6 & 1 & YES & YES & YES & $1.22$ & $(2,2)$ & NO & 1939\\
$(53,19)$ & 9 & $(25,9)$ & 7 & 1 & YES & YES & YES & $0.88$ & $(2,2)$ & 2119 & 1940\\
$(53,23)$ & 9 & $(30,13)$ & 8 & 1 & YES & YES & YES & $1.00$ & $(2,2)$ & NO & 1941\\
$(53,19)$ & 9 & $(53,19)$ & 9 & 53 & YES & YES & YES & $1.00$ & $(2,2)$ & NO & 1942\\
$(54,19)$ & 10 & $(2,1)$ & 1 & 2 & YES & YES & YES & $1.00$ & $(2,2)$ & NO & 1943\\
$(54,17)$ & 10 & $(4,1)$ & 3 & 2 & YES & YES & YES & $1.00$ & $(2,2)$ & -- & 1944\\
$(54,19)$ & 10 & $(4,1)$ & 3 & 2 & YES & YES & YES & $1.00$ & $(2,2)$ & -- & 1945\\
$(54,19)$ & 10 & $(5,1)$ & 4 & 1 & YES & YES & YES & $1.12$ & $(2,2)$ & NO & 1946\\
$(54,19)$ & 10 & $(8,3)$ & 4 & 2 & YES & YES & YES & $1.12$ & $(2,2)$ & 1167 & 1947\\
$(54,17)$ & 10 & $(10,3)$ & 5 & 2 & YES & YES & YES & $1.12$ & $(2,2)$ & NO & 1948\\
$(54,19)$ & 10 & $(11,4)$ & 5 & 1 & YES & YES & YES & $1.12$ & $(2,2)$ & NO & 1949\\
$(54,19)$ & 10 & $(37,13)$ & 9 & 1 & YES & YES & YES & $1.00$ & $(2,2)$ & NO & 1950\\
$(55,9)$ & 14 & $(2,1)$ & 1 & 1 & YES & YES & YES & $1.00$ & $(2,2)$ & NO & 1951\\
$(55,24)$ & 9 & $(3,1)$ & 2 & 1 & YES & YES & YES & $0.88$ & $(4,1)$ & -- & 1952\\
$(55,17)$ & 10 & $(4,1)$ & 3 & 1 & YES & YES & YES & $0.88$ & $(2,2)$ & -- & 1953\\
$(55,17)$ & 10 & $(6,1)$ & 5 & 1 & YES & YES & YES & $1.00$ & $(2,2)$ & NO & 1954\\
$(55,9)$ & 14 & $(7,1)$ & 6 & 1 & YES & YES & YES & $1.00$ & $(2,2)$ & NO & 1955\\
$(55,21)$ & 8 & $(7,3)$ & 4 & 1 & YES & YES & YES & $1.22$ & $(2,2)$ & NO & 1956\\
$(55,16)$ & 9 & $(9,2)$ & 5 & 1 & YES & YES & YES & $1.00$ & $(2,2)$ & NO & 1957\\
$(55,24)$ & 9 & $(9,4)$ & 5 & 1 & YES & YES & YES & $1.12$ & $(2,2)$ & NO & 1958\\
$(55,21)$ & 8 & $(11,4)$ & 5 & 11 & YES & YES & YES & $1.22$ & $(2,2)$ & NO & 1959\\
$(55,24)$ & 9 & $(16,7)$ & 6 & 1 & YES & YES & YES & $1.11$ & $(2,2)$ & NO & 1960\\
$(55,21)$ & 8 & $(18,7)$ & 6 & 1 & YES & YES & YES & $1.22$ & $(2,2)$ & NO & 1961\\
$(55,21)$ & 8 & $(29,11)$ & 7 & 1 & YES & YES & YES & $1.22$ & $(2,2)$ & NO & 1962\\
$(55,16)$ & 9 & $(38,11)$ & 9 & 1 & YES & YES & YES & $1.00$ & $(2,2)$ & NO & 1963\\
$(56,17)$ & 9 & $(2,1)$ & 1 & 2 & YES & YES & YES & $1.11$ & $(2,2)$ & NO & 1964\\
$(56,23)$ & 9 & $(2,1)$ & 1 & 2 & YES & YES & YES & $1.00$ & $(2,2)$ & 836 & 1965\\
$(56,15)$ & 9 & $(3,1)$ & 2 & 1 & YES & YES & YES & $0.88$ & $(2,2)$ & -- & 1966\\
$(56,23)$ & 9 & $(3,1)$ & 2 & 1 & YES & YES & YES & $1.11$ & $(2,2)$ & NO & 1967\\
$(56,13)$ & 10 & $(4,1)$ & 3 & 4 & YES & YES & YES & $1.00$ & $(4,1)$ & NO & 1968\\
$(56,13)$ & 10 & $(4,1)$ & 3 & 4 & YES & YES & YES & $1.00$ & $(4,1)$ & -- & 1969\\
$(56,15)$ & 9 & $(5,2)$ & 3 & 1 & YES & YES & YES & $1.11$ & $(2,2)$ & -- & 1970\\
$(56,23)$ & 9 & $(5,2)$ & 3 & 1 & YES & YES & YES & $1.00$ & $(2,2)$ & NO & 1971\\
$(56,15)$ & 9 & $(7,2)$ & 4 & 7 & YES & YES & YES & $1.11$ & $(2,2)$ & -- & 1972\\
$(56,15)$ & 9 & $(10,3)$ & 5 & 2 & YES & YES & YES & $1.11$ & $(2,2)$ & NO & 1973\\
$(56,13)$ & 10 & $(13,3)$ & 6 & 1 & YES & YES & YES & $1.00$ & $(4,1)$ & NO & 1974\\
$(56,15)$ & 9 & $(18,5)$ & 6 & 2 & YES & YES & YES & $1.11$ & $(2,2)$ & 2450 & 1975\\
$(56,17)$ & 9 & $(23,7)$ & 7 & 1 & YES & YES & YES & $1.11$ & $(2,2)$ & NO & 1976\\
$(56,15)$ & 9 & $(26,7)$ & 7 & 2 & YES & YES & YES & $1.00$ & $(2,2)$ & 2183 & 1977\\
$(57,25)$ & 9 & $(2,1)$ & 1 & 1 & YES & YES & YES & $1.12$ & $(2,2)$ & NO & 1978\\
$(57,25)$ & 9 & $(2,1)$ & 1 & 1 & YES & YES & YES & $1.00$ & $(2,2)$ & -- & 1979\\
$(57,25)$ & 9 & $(3,1)$ & 2 & 3 & YES & YES & YES & $1.00$ & $(2,2)$ & -- & 1980\\
$(57,25)$ & 9 & $(3,1)$ & 2 & 3 & YES & YES & YES & $1.00$ & $(2,2)$ & NO & 1981\\
$(57,25)$ & 9 & $(4,1)$ & 3 & 1 & YES & YES & YES & $1.00$ & $(2,2)$ & NO & 1982\\
$(57,25)$ & 9 & $(4,1)$ & 3 & 1 & YES & YES & YES & $1.00$ & $(2,2)$ & -- & 1983\\
$(57,16)$ & 9 & $(5,1)$ & 4 & 1 & YES & YES & YES & $1.00$ & $(2,2)$ & NO & 1984\\
$(57,16)$ & 9 & $(5,1)$ & 4 & 1 & YES & YES & YES & $1.00$ & $(2,2)$ & -- & 1985\\
$(57,16)$ & 9 & $(7,2)$ & 4 & 1 & YES & YES & YES & $1.12$ & $(2,2)$ & 1492 & 1986\\
$(57,25)$ & 9 & $(9,4)$ & 5 & 3 & YES & YES & YES & $1.12$ & $(2,2)$ & NO & 1987\\
$(57,25)$ & 9 & $(25,11)$ & 7 & 1 & YES & YES & YES & $1.00$ & $(2,2)$ & 2170 & 1988\\
$(57,25)$ & 9 & $(41,18)$ & 8 & 1 & YES & YES & YES & $1.00$ & $(2,2)$ & NO & 1989\\
$(58,17)$ & 9 & $(3,1)$ & 2 & 1 & NO & YES & NO(2) & $0.89$ & $(4,1)$ & -- & 1990\\
$(58,13)$ & 11 & $(4,1)$ & 3 & 2 & YES & YES & YES & $1.12$ & $(2,2)$ & NO & 1991\\
$(58,13)$ & 11 & $(5,2)$ & 3 & 1 & YES & YES & YES & $1.12$ & $(2,2)$ & -- & 1992\\
$(58,17)$ & 9 & $(13,4)$ & 6 & 1 & YES & YES & YES & $1.22$ & $(2,2)$ & NO & 1993\\
$(58,13)$ & 11 & $(14,3)$ & 6 & 2 & YES & YES & YES & $1.12$ & $(2,2)$ & NO & 1994\\
$(58,17)$ & 9 & $(27,8)$ & 7 & 1 & YES & YES & YES & $1.33$ & $(2,2)$ & NO & 1995\\
$(58,17)$ & 9 & $(31,9)$ & 8 & 1 & YES & YES & YES & $1.22$ & $(2,2)$ & 2459 & 1996\\
$(59,18)$ & 9 & $(2,1)$ & 1 & 1 & YES & YES & YES & $1.00$ & $(2,2)$ & NO & 1997\\
$(59,23)$ & 9 & $(2,1)$ & 1 & 1 & YES & YES & YES & $1.00$ & $(2,2)$ & NO & 1998\\
$(59,25)$ & 9 & $(2,1)$ & 1 & 1 & YES & YES & YES & $1.00$ & $(2,2)$ & -- & 1999\\
$(59,25)$ & 9 & $(2,1)$ & 1 & 1 & YES & YES & YES & $1.00$ & $(2,2)$ & NO & 2000\\
$(59,26)$ & 9 & $(2,1)$ & 1 & 1 & YES & YES & YES & $0.88$ & $(2,2)$ & -- & 2001\\
$(59,26)$ & 9 & $(2,1)$ & 1 & 1 & YES & YES & YES & $0.88$ & $(4,1)$ & NO & 2002\\
$(59,27)$ & 10 & $(2,1)$ & 1 & 1 & YES & YES & YES & $1.12$ & $(2,2)$ & -- & 2003\\
$(59,13)$ & 11 & $(3,1)$ & 2 & 1 & YES & YES & YES & $1.00$ & $(2,2)$ & NO & 2004\\
$(59,13)$ & 11 & $(3,1)$ & 2 & 1 & YES & YES & YES & $1.00$ & $(2,2)$ & -- & 2005\\
$(59,23)$ & 9 & $(3,1)$ & 2 & 1 & YES & YES & YES & $0.88$ & $(2,2)$ & -- & 2006\\
$(59,26)$ & 9 & $(3,1)$ & 2 & 1 & YES & YES & YES & $1.11$ & $(2,2)$ & NO & 2007\\
$(59,27)$ & 10 & $(3,1)$ & 2 & 1 & YES & YES & YES & $1.12$ & $(2,2)$ & NO & 2008\\
$(59,27)$ & 10 & $(3,1)$ & 2 & 1 & YES & YES & YES & $1.12$ & $(2,2)$ & -- & 2009\\
$(59,13)$ & 11 & $(4,1)$ & 3 & 1 & YES & YES & YES & $1.25$ & $(2,2)$ & NO & 2010\\
$(59,23)$ & 9 & $(4,1)$ & 3 & 1 & YES & YES & YES & $1.00$ & $(2,2)$ & NO & 2011\\
$(59,23)$ & 9 & $(5,2)$ & 3 & 1 & YES & YES & YES & $1.00$ & $(2,2)$ & NO & 2012\\
$(59,25)$ & 9 & $(5,1)$ & 4 & 1 & YES & YES & YES & $0.88$ & $(2,2)$ & -- & 2013\\
$(59,25)$ & 9 & $(5,1)$ & 4 & 1 & YES & YES & YES & $1.00$ & $(2,2)$ & NO & 2014\\
$(59,25)$ & 9 & $(5,1)$ & 4 & 1 & YES & YES & YES & $1.11$ & $(2,2)$ & NO & 2015\\
$(59,25)$ & 9 & $(5,2)$ & 3 & 1 & YES & YES & YES & $1.00$ & $(2,2)$ & NO & 2016\\
$(59,26)$ & 9 & $(5,1)$ & 4 & 1 & YES & YES & YES & $0.88$ & $(2,2)$ & -- & 2017\\
$(59,13)$ & 11 & $(6,1)$ & 5 & 1 & YES & YES & YES & $1.12$ & $(2,2)$ & NO & 2018\\
$(59,14)$ & 10 & $(7,2)$ & 4 & 1 & YES & YES & YES & $0.75$ & $(4,1)$ & NO & 2019\\
$(59,16)$ & 10 & $(7,1)$ & 6 & 1 & YES & YES & YES & $0.88$ & $(2,2)$ & NO & 2020\\
$(59,25)$ & 9 & $(7,3)$ & 4 & 1 & YES & YES & YES & $1.00$ & $(2,2)$ & 1506 & 2021\\
$(59,26)$ & 9 & $(7,3)$ & 4 & 1 & YES & YES & YES & $1.11$ & $(2,2)$ & 1325 & 2022\\
$(59,23)$ & 9 & $(8,3)$ & 4 & 1 & YES & YES & YES & $1.12$ & $(2,2)$ & NO & 2023\\
$(59,26)$ & 9 & $(9,4)$ & 5 & 1 & YES & YES & YES & $1.00$ & $(2,2)$ & 1656 & 2024\\
$(59,13)$ & 11 & $(13,3)$ & 6 & 1 & YES & YES & YES & $1.00$ & $(2,2)$ & NO & 2025\\
$(59,23)$ & 9 & $(13,5)$ & 5 & 1 & YES & YES & YES & $1.12$ & $(2,2)$ & NO & 2026\\
$(59,13)$ & 11 & $(14,3)$ & 6 & 1 & YES & YES & YES & $1.12$ & $(2,2)$ & NO & 2027\\
$(59,23)$ & 9 & $(23,9)$ & 7 & 1 & YES & YES & YES & $0.88$ & $(2,2)$ & 2120 & 2028\\
$(59,27)$ & 10 & $(24,11)$ & 8 & 1 & YES & YES & YES & $1.25$ & $(2,2)$ & NO & 2029\\
$(59,26)$ & 9 & $(25,11)$ & 7 & 1 & YES & YES & YES & $1.00$ & $(2,2)$ & NO & 2030\\
$(59,25)$ & 9 & $(26,11)$ & 7 & 1 & YES & YES & YES & $0.88$ & $(2,2)$ & NO & 2031\\
$(59,16)$ & 10 & $(37,10)$ & 8 & 1 & YES & YES & YES & $0.88$ & $(2,2)$ & 2386 & 2032\\
$(59,23)$ & 9 & $(59,23)$ & 9 & 59 & YES & YES & YES & $1.00$ & $(2,2)$ & NO & 2033\\
$(59,27)$ & 10 & $(59,27)$ & 10 & 59 & YES & YES & YES & $1.12$ & $(2,2)$ & NO & 2034\\
$(60,23)$ & 9 & $(4,1)$ & 3 & 4 & YES & YES & YES & $1.11$ & $(2,2)$ & -- & 2035\\
$(60,23)$ & 9 & $(5,1)$ & 4 & 5 & YES & YES & YES & $1.22$ & $(2,2)$ & NO & 2036\\
$(60,23)$ & 9 & $(34,13)$ & 7 & 2 & YES & YES & YES & $1.11$ & $(2,2)$ & 2359 & 2037\\
$(60,23)$ & 9 & $(47,18)$ & 8 & 1 & YES & YES & YES & $1.22$ & $(2,2)$ & NO & 2038\\
$(61,25)$ & 9 & $(2,1)$ & 1 & 1 & YES & YES & YES & $1.00$ & $(2,2)$ & NO & 2039\\
$(61,25)$ & 9 & $(2,1)$ & 1 & 1 & NO & YES & YES & $1.00$ & $(2,2)$ & -- & 2040\\
$(61,28)$ & 10 & $(2,1)$ & 1 & 1 & NO & YES & YES & $1.12$ & $(2,2)$ & -- & 2041\\
$(61,25)$ & 9 & $(3,1)$ & 2 & 1 & YES & YES & YES & $1.00$ & $(2,2)$ & NO & 2042\\
$(61,25)$ & 9 & $(3,1)$ & 2 & 1 & YES & YES & YES & $1.00$ & $(2,2)$ & -- & 2043\\
$(61,13)$ & 10 & $(5,1)$ & 4 & 1 & YES & YES & NO(2) & $1.10$ & $(2,2)$ & NO & 2044\\
$(61,18)$ & 9 & $(5,2)$ & 3 & 1 & YES & YES & YES & $1.22$ & $(2,2)$ & -- & 2045\\
$(61,18)$ & 9 & $(5,2)$ & 3 & 1 & YES & YES & YES & $1.22$ & $(2,2)$ & NO & 2046\\
$(61,25)$ & 9 & $(5,1)$ & 4 & 1 & YES & YES & YES & $0.88$ & $(2,2)$ & NO & 2047\\
$(61,25)$ & 9 & $(5,1)$ & 4 & 1 & YES & YES & YES & $0.88$ & $(2,2)$ & -- & 2048\\
$(61,25)$ & 9 & $(5,2)$ & 3 & 1 & YES & YES & YES & $1.00$ & $(2,2)$ & NO & 2049\\
$(61,13)$ & 10 & $(6,1)$ & 5 & 1 & YES & YES & NO(2) & $1.10$ & $(2,2)$ & 1822 & 2050\\
$(61,19)$ & 10 & $(6,1)$ & 5 & 1 & YES & YES & YES & $1.00$ & $(2,2)$ & NO & 2051\\
$(61,25)$ & 9 & $(12,5)$ & 5 & 1 & YES & YES & YES & $1.00$ & $(2,2)$ & 2293 & 2052\\
$(61,17)$ & 9 & $(15,4)$ & 6 & 1 & YES & YES & YES & $1.11$ & $(2,2)$ & NO & 2053\\
$(61,19)$ & 10 & $(16,5)$ & 7 & 1 & YES & YES & YES & $1.12$ & $(2,2)$ & NO & 2054\\
$(61,13)$ & 10 & $(19,4)$ & 7 & 1 & YES & YES & NO(2) & $1.10$ & $(2,2)$ & NO & 2055\\
$(61,25)$ & 9 & $(22,9)$ & 7 & 1 & YES & YES & YES & $1.00$ & $(2,2)$ & NO & 2056\\
$(61,18)$ & 9 & $(24,7)$ & 7 & 1 & YES & YES & YES & $1.22$ & $(2,2)$ & NO & 2057\\
$(61,17)$ & 9 & $(29,8)$ & 7 & 1 & YES & YES & YES & $1.11$ & $(2,2)$ & NO & 2058\\
$(61,25)$ & 9 & $(39,16)$ & 8 & 1 & YES & YES & YES & $0.88$ & $(2,2)$ & NO & 2059\\
$(61,19)$ & 10 & $(45,14)$ & 9 & 1 & YES & YES & YES & $1.00$ & $(2,2)$ & NO & 2060\\
$(62,9)$ & 14 & $(2,1)$ & 1 & 2 & YES & YES & YES & $1.12$ & $(2,2)$ & NO & 2061\\
$(62,19)$ & 10 & $(2,1)$ & 1 & 2 & YES & YES & YES & $1.12$ & $(2,2)$ & -- & 2062\\
$(62,19)$ & 10 & $(3,1)$ & 2 & 1 & YES & YES & YES & $1.12$ & $(2,2)$ & NO & 2063\\
$(62,23)$ & 9 & $(3,1)$ & 2 & 1 & YES & YES & YES & $1.12$ & $(2,2)$ & -- & 2064\\
$(62,23)$ & 9 & $(3,1)$ & 2 & 1 & YES & YES & YES & $1.22$ & $(2,2)$ & NO & 2065\\
$(62,23)$ & 9 & $(3,1)$ & 2 & 1 & YES & YES & NO(2) & $1.20$ & $(2,2)$ & NO & 2066\\
$(62,27)$ & 9 & $(3,1)$ & 2 & 1 & YES & YES & YES & $1.00$ & $(2,2)$ & -- & 2067\\
$(62,19)$ & 10 & $(4,1)$ & 3 & 2 & YES & YES & YES & $1.00$ & $(2,2)$ & NO & 2068\\
$(62,23)$ & 9 & $(4,1)$ & 3 & 2 & YES & YES & YES & $1.00$ & $(2,2)$ & -- & 2069\\
$(62,23)$ & 9 & $(5,1)$ & 4 & 1 & YES & YES & NO(2) & $1.10$ & $(2,2)$ & -- & 2070\\
$(62,9)$ & 14 & $(6,1)$ & 5 & 2 & YES & YES & YES & $1.00$ & $(2,2)$ & NO & 2071\\
$(62,19)$ & 10 & $(7,1)$ & 6 & 1 & YES & YES & YES & $1.11$ & $(2,2)$ & NO & 2072\\
$(62,23)$ & 9 & $(8,3)$ & 4 & 2 & YES & YES & NO(2) & $1.20$ & $(2,2)$ & 1658 & 2073\\
$(62,27)$ & 9 & $(9,4)$ & 5 & 1 & YES & YES & YES & $1.12$ & $(2,2)$ & 2166 & 2074\\
$(62,19)$ & 10 & $(10,3)$ & 5 & 2 & YES & YES & YES & $1.00$ & $(2,2)$ & NO & 2075\\
$(62,23)$ & 9 & $(11,4)$ & 5 & 1 & YES & YES & YES & $1.00$ & $(2,2)$ & 1359 & 2076\\
$(62,19)$ & 10 & $(13,4)$ & 6 & 1 & YES & YES & YES & $1.22$ & $(2,2)$ & NO & 2077\\
$(62,23)$ & 9 & $(19,7)$ & 6 & 1 & YES & YES & YES & $1.00$ & $(2,2)$ & NO & 2078\\
$(62,23)$ & 9 & $(35,13)$ & 8 & 1 & YES & YES & YES & $1.00$ & $(2,2)$ & NO & 2079\\
$(62,27)$ & 9 & $(39,17)$ & 8 & 1 & YES & YES & YES & $1.00$ & $(2,2)$ & NO & 2080\\
$(62,19)$ & 10 & $(49,15)$ & 9 & 1 & YES & YES & YES & $1.11$ & $(2,2)$ & NO & 2081\\
$(62,23)$ & 9 & $(62,23)$ & 9 & 62 & YES & YES & YES & $1.00$ & $(2,2)$ & NO & 2082\\
$(63,26)$ & 9 & $(2,1)$ & 1 & 1 & YES & YES & YES & $1.12$ & $(2,2)$ & NO & 2083\\
$(63,26)$ & 9 & $(2,1)$ & 1 & 1 & NO & YES & NO(2) & $1.00$ & $(4,1)$ & -- & 2084\\
$(63,13)$ & 11 & $(3,1)$ & 2 & 3 & YES & YES & YES & $0.88$ & $(2,2)$ & -- & 2085\\
$(63,26)$ & 9 & $(3,1)$ & 2 & 3 & YES & YES & YES & $1.00$ & $(2,2)$ & -- & 2086\\
$(63,26)$ & 9 & $(4,1)$ & 3 & 1 & YES & YES & YES & $1.12$ & $(2,2)$ & -- & 2087\\
$(63,17)$ & 9 & $(5,2)$ & 3 & 1 & YES & YES & YES & $1.22$ & $(2,2)$ & -- & 2088\\
$(63,26)$ & 9 & $(5,2)$ & 3 & 1 & YES & YES & YES & $1.12$ & $(2,2)$ & NO & 2089\\
$(63,13)$ & 11 & $(6,1)$ & 5 & 3 & YES & YES & YES & $1.00$ & $(2,2)$ & NO & 2090\\
$(63,26)$ & 9 & $(6,1)$ & 5 & 3 & YES & YES & YES & $1.00$ & $(2,2)$ & NO & 2091\\
$(63,26)$ & 9 & $(6,1)$ & 5 & 3 & YES & YES & YES & $1.00$ & $(2,2)$ & -- & 2092\\
$(63,13)$ & 11 & $(9,2)$ & 5 & 9 & YES & YES & YES & $0.88$ & $(2,2)$ & NO & 2093\\
$(63,17)$ & 9 & $(10,3)$ & 5 & 1 & YES & YES & YES & $1.22$ & $(2,2)$ & NO & 2094\\
$(63,26)$ & 9 & $(17,7)$ & 6 & 1 & YES & YES & YES & $1.12$ & $(2,2)$ & NO & 2095\\
$(63,13)$ & 11 & $(24,5)$ & 8 & 3 & YES & YES & YES & $1.00$ & $(2,2)$ & NO & 2096\\
$(64,23)$ & 9 & $(2,1)$ & 1 & 2 & YES & YES & YES & $0.75$ & $(4,1)$ & -- & 2097\\
$(64,23)$ & 9 & $(2,1)$ & 1 & 2 & YES & YES & YES & $1.00$ & $(2,2)$ & NO & 2098\\
$(64,25)$ & 9 & $(2,1)$ & 1 & 2 & YES & YES & YES & $1.00$ & $(2,2)$ & -- & 2099\\
$(64,25)$ & 9 & $(2,1)$ & 1 & 2 & YES & YES & YES & $1.00$ & $(2,2)$ & NO & 2100\\
$(64,27)$ & 9 & $(2,1)$ & 1 & 2 & YES & YES & YES & $1.00$ & $(2,2)$ & -- & 2101\\
$(64,27)$ & 9 & $(2,1)$ & 1 & 2 & YES & YES & YES & $1.00$ & $(2,2)$ & 1315 & 2102\\
$(64,19)$ & 9 & $(3,1)$ & 2 & 1 & NO & YES & YES & $1.00$ & $(2,2)$ & -- & 2103\\
$(64,23)$ & 9 & $(3,1)$ & 2 & 1 & YES & YES & YES & $0.88$ & $(2,2)$ & -- & 2104\\
$(64,23)$ & 9 & $(3,1)$ & 2 & 1 & YES & YES & YES & $1.00$ & $(2,2)$ & 1772 & 2105\\
$(64,27)$ & 9 & $(3,1)$ & 2 & 1 & YES & YES & YES & $1.12$ & $(2,2)$ & NO & 2106\\
$(64,17)$ & 10 & $(4,1)$ & 3 & 4 & YES & YES & YES & $1.00$ & $(2,2)$ & -- & 2107\\
$(64,27)$ & 9 & $(4,1)$ & 3 & 4 & YES & YES & YES & $1.22$ & $(2,2)$ & NO & 2108\\
$(64,17)$ & 10 & $(5,1)$ & 4 & 1 & YES & YES & YES & $1.00$ & $(2,2)$ & -- & 2109\\
$(64,19)$ & 9 & $(5,2)$ & 3 & 1 & YES & YES & YES & $1.11$ & $(2,2)$ & NO & 2110\\
$(64,23)$ & 9 & $(5,1)$ & 4 & 1 & YES & YES & YES & $0.88$ & $(2,2)$ & -- & 2111\\
$(64,23)$ & 9 & $(5,1)$ & 4 & 1 & YES & YES & YES & $1.00$ & $(2,2)$ & NO & 2112\\
$(64,23)$ & 9 & $(5,2)$ & 3 & 1 & YES & YES & YES & $1.00$ & $(2,2)$ & 1487 & 2113\\
$(64,25)$ & 9 & $(5,2)$ & 3 & 1 & YES & YES & YES & $1.00$ & $(2,2)$ & NO & 2114\\
$(64,27)$ & 9 & $(5,2)$ & 3 & 1 & YES & YES & YES & $1.00$ & $(2,2)$ & NO & 2115\\
$(64,17)$ & 10 & $(6,1)$ & 5 & 2 & YES & YES & YES & $1.00$ & $(2,2)$ & NO & 2116\\
$(64,27)$ & 9 & $(7,3)$ & 4 & 1 & YES & YES & YES & $1.00$ & $(2,2)$ & NO & 2117\\
$(64,19)$ & 9 & $(9,2)$ & 5 & 1 & YES & YES & YES & $1.22$ & $(2,2)$ & NO & 2118\\
$(64,23)$ & 9 & $(14,5)$ & 6 & 2 & YES & YES & YES & $0.88$ & $(2,2)$ & 1940 & 2119\\
$(64,25)$ & 9 & $(18,7)$ & 6 & 2 & YES & YES & YES & $0.88$ & $(2,2)$ & 2028 & 2120\\
$(64,27)$ & 9 & $(26,11)$ & 7 & 2 & YES & YES & YES & $0.88$ & $(2,2)$ & 2244 & 2121\\
$(64,17)$ & 10 & $(34,9)$ & 8 & 2 & YES & YES & YES & $1.00$ & $(2,2)$ & 2376 & 2122\\
$(64,23)$ & 9 & $(39,14)$ & 8 & 1 & YES & YES & YES & $0.88$ & $(2,2)$ & NO & 2123\\
$(64,27)$ & 9 & $(45,19)$ & 8 & 1 & YES & YES & YES & $1.22$ & $(2,2)$ & NO & 2124\\
$(64,19)$ & 9 & $(47,14)$ & 9 & 1 & YES & YES & YES & $1.22$ & $(2,2)$ & NO & 2125\\
$(64,17)$ & 10 & $(49,13)$ & 9 & 1 & YES & YES & YES & $1.00$ & $(2,2)$ & NO & 2126\\
$(65,17)$ & 10 & $(2,1)$ & 1 & 1 & YES & YES & YES & $1.00$ & $(2,2)$ & -- & 2127\\
$(65,17)$ & 10 & $(2,1)$ & 1 & 1 & YES & YES & YES & $1.12$ & $(2,2)$ & NO & 2128\\
$(65,24)$ & 9 & $(2,1)$ & 1 & 1 & YES & YES & YES & $1.11$ & $(2,2)$ & NO & 2129\\
$(65,24)$ & 9 & $(2,1)$ & 1 & 1 & YES & YES & YES & $1.00$ & $(2,2)$ & -- & 2130\\
$(65,17)$ & 10 & $(3,1)$ & 2 & 1 & YES & YES & YES & $1.00$ & $(2,2)$ & -- & 2131\\
$(65,24)$ & 9 & $(3,1)$ & 2 & 1 & YES & YES & YES & $1.00$ & $(2,2)$ & -- & 2132\\
$(65,24)$ & 9 & $(3,1)$ & 2 & 1 & YES & YES & NO(2) & $0.89$ & $(4,1)$ & NO & 2133\\
$(65,17)$ & 10 & $(4,1)$ & 3 & 1 & YES & YES & YES & $1.12$ & $(2,2)$ & -- & 2134\\
$(65,17)$ & 10 & $(4,1)$ & 3 & 1 & YES & YES & YES & $1.00$ & $(2,2)$ & NO & 2135\\
$(65,24)$ & 9 & $(4,1)$ & 3 & 1 & YES & YES & YES & $1.11$ & $(2,2)$ & -- & 2136\\
$(65,14)$ & 10 & $(5,2)$ & 3 & 5 & YES & YES & YES & $1.22$ & $(2,2)$ & NO & 2137\\
$(65,14)$ & 10 & $(5,2)$ & 3 & 5 & YES & YES & YES & $1.22$ & $(2,2)$ & NO & 2138\\
$(65,18)$ & 9 & $(5,2)$ & 3 & 5 & YES & YES & YES & $1.22$ & $(2,2)$ & NO & 2139\\
$(65,18)$ & 9 & $(5,2)$ & 3 & 5 & YES & YES & YES & $1.22$ & $(2,2)$ & NO & 2140\\
$(65,18)$ & 9 & $(5,2)$ & 3 & 5 & YES & YES & YES & $1.33$ & $(2,2)$ & -- & 2141\\
$(65,19)$ & 9 & $(5,2)$ & 3 & 5 & YES & YES & YES & $1.22$ & $(2,2)$ & -- & 2142\\
$(65,19)$ & 9 & $(5,2)$ & 3 & 5 & YES & YES & YES & $1.22$ & $(2,2)$ & NO & 2143\\
$(65,24)$ & 9 & $(8,3)$ & 4 & 1 & YES & YES & YES & $1.11$ & $(2,2)$ & NO & 2144\\
$(65,17)$ & 10 & $(11,3)$ & 5 & 1 & YES & YES & YES & $0.88$ & $(2,2)$ & NO & 2145\\
$(65,18)$ & 9 & $(15,4)$ & 6 & 5 & YES & YES & YES & $1.22$ & $(2,2)$ & 2406 & 2146\\
$(65,14)$ & 10 & $(23,5)$ & 7 & 1 & YES & YES & YES & $0.88$ & $(2,2)$ & NO & 2147\\
$(65,17)$ & 10 & $(23,6)$ & 8 & 1 & YES & YES & YES & $0.88$ & $(4,1)$ & NO & 2148\\
$(65,18)$ & 9 & $(25,7)$ & 7 & 5 & YES & YES & YES & $1.22$ & $(2,2)$ & NO & 2149\\
$(65,24)$ & 9 & $(27,10)$ & 7 & 1 & YES & YES & YES & $1.00$ & $(2,2)$ & 2275 & 2150\\
$(65,17)$ & 10 & $(42,11)$ & 9 & 1 & YES & YES & YES & $1.00$ & $(2,2)$ & NO & 2151\\
$(65,24)$ & 9 & $(46,17)$ & 8 & 1 & YES & YES & YES & $1.11$ & $(2,2)$ & NO & 2152\\
$(65,19)$ & 9 & $(58,17)$ & 9 & 1 & YES & YES & YES & $1.11$ & $(2,2)$ & NO & 2153\\
$(65,17)$ & 10 & $(65,17)$ & 10 & 65 & YES & YES & YES & $1.25$ & $(2,2)$ & NO & 2154\\
$(66,25)$ & 9 & $(2,1)$ & 1 & 2 & YES & YES & YES & $1.00$ & $(2,2)$ & NO & 2155\\
$(66,29)$ & 9 & $(2,1)$ & 1 & 2 & YES & YES & YES & $1.00$ & $(2,2)$ & -- & 2156\\
$(66,29)$ & 9 & $(2,1)$ & 1 & 2 & YES & YES & YES & $1.12$ & $(2,2)$ & NO & 2157\\
$(66,25)$ & 9 & $(3,1)$ & 2 & 3 & YES & YES & YES & $1.22$ & $(2,2)$ & -- & 2158\\
$(66,25)$ & 9 & $(3,1)$ & 2 & 3 & YES & YES & YES & $1.00$ & $(2,2)$ & NO & 2159\\
$(66,29)$ & 9 & $(3,1)$ & 2 & 3 & YES & YES & YES & $1.00$ & $(2,2)$ & -- & 2160\\
$(66,25)$ & 9 & $(4,1)$ & 3 & 2 & YES & YES & YES & $1.00$ & $(2,2)$ & -- & 2161\\
$(66,25)$ & 9 & $(5,1)$ & 4 & 1 & YES & YES & YES & $1.00$ & $(2,2)$ & NO & 2162\\
$(66,25)$ & 9 & $(5,1)$ & 4 & 1 & YES & YES & YES & $1.00$ & $(2,2)$ & -- & 2163\\
$(66,29)$ & 9 & $(5,1)$ & 4 & 1 & YES & YES & YES & $1.12$ & $(2,2)$ & NO & 2164\\
$(66,29)$ & 9 & $(5,1)$ & 4 & 1 & YES & YES & YES & $1.12$ & $(2,2)$ & -- & 2165\\
$(66,29)$ & 9 & $(7,3)$ & 4 & 1 & YES & YES & YES & $1.12$ & $(2,2)$ & 2074 & 2166\\
$(66,25)$ & 9 & $(8,3)$ & 4 & 2 & YES & YES & YES & $1.00$ & $(2,2)$ & 1712 & 2167\\
$(66,29)$ & 9 & $(9,4)$ & 5 & 3 & YES & YES & YES & $1.00$ & $(2,2)$ & NO & 2168\\
$(66,25)$ & 9 & $(13,5)$ & 5 & 1 & YES & YES & YES & $1.11$ & $(2,2)$ & 1434 & 2169\\
$(66,29)$ & 9 & $(16,7)$ & 6 & 2 & YES & YES & YES & $1.00$ & $(2,2)$ & 1988 & 2170\\
$(66,25)$ & 9 & $(21,8)$ & 6 & 3 & YES & YES & YES & $1.11$ & $(2,2)$ & NO & 2171\\
$(66,29)$ & 9 & $(41,18)$ & 8 & 1 & YES & YES & YES & $0.88$ & $(2,2)$ & NO & 2172\\
$(66,25)$ & 9 & $(66,25)$ & 9 & 66 & YES & YES & YES & $1.00$ & $(2,2)$ & NO & 2173\\
$(67,29)$ & 10 & $(2,1)$ & 1 & 1 & YES & YES & YES & $1.12$ & $(2,2)$ & NO & 2174\\
$(67,18)$ & 9 & $(3,1)$ & 2 & 1 & NO & YES & YES & $0.88$ & $(2,2)$ & -- & 2175\\
$(67,29)$ & 10 & $(5,1)$ & 4 & 1 & YES & YES & YES & $1.00$ & $(2,2)$ & NO & 2176\\
$(67,29)$ & 10 & $(5,1)$ & 4 & 1 & YES & YES & YES & $1.22$ & $(2,2)$ & -- & 2177\\
$(67,29)$ & 10 & $(5,1)$ & 4 & 1 & YES & YES & YES & $1.22$ & $(2,2)$ & NO & 2178\\
$(67,29)$ & 10 & $(6,1)$ & 5 & 1 & YES & YES & YES & $1.00$ & $(2,2)$ & -- & 2179\\
$(67,29)$ & 10 & $(6,1)$ & 5 & 1 & YES & YES & YES & $1.22$ & $(2,2)$ & NO & 2180\\
$(67,29)$ & 10 & $(7,3)$ & 4 & 1 & YES & YES & YES & $1.12$ & $(2,2)$ & 1681 & 2181\\
$(67,16)$ & 11 & $(9,2)$ & 5 & 1 & YES & YES & YES & $0.88$ & $(4,1)$ & 1358 & 2182\\
$(67,18)$ & 9 & $(15,4)$ & 6 & 1 & YES & YES & YES & $1.00$ & $(2,2)$ & 1977 & 2183\\
$(67,29)$ & 10 & $(30,13)$ & 8 & 1 & YES & YES & YES & $1.00$ & $(2,2)$ & NO & 2184\\
$(68,19)$ & 9 & $(10,3)$ & 5 & 2 & YES & YES & YES & $1.22$ & $(2,2)$ & NO & 2185\\
$(68,19)$ & 9 & $(61,17)$ & 9 & 1 & YES & YES & YES & $1.22$ & $(2,2)$ & NO & 2186\\
$(69,19)$ & 9 & $(3,1)$ & 2 & 3 & NO & YES & YES & $1.00$ & $(2,2)$ & -- & 2187\\
$(69,19)$ & 9 & $(4,1)$ & 3 & 1 & YES & YES & YES & $0.88$ & $(2,2)$ & NO & 2188\\
$(69,19)$ & 9 & $(11,3)$ & 5 & 1 & YES & YES & YES & $0.88$ & $(2,2)$ & 1903 & 2189\\
$(69,13)$ & 11 & $(17,3)$ & 7 & 1 & YES & YES & YES & $0.88$ & $(2,2)$ & NO & 2190\\
$(69,13)$ & 11 & $(27,5)$ & 8 & 3 & YES & YES & YES & $0.88$ & $(2,2)$ & NO & 2191\\
$(70,29)$ & 9 & $(3,1)$ & 2 & 1 & YES & YES & YES & $1.22$ & $(2,2)$ & -- & 2192\\
$(70,29)$ & 9 & $(4,1)$ & 3 & 2 & YES & YES & YES & $1.00$ & $(2,2)$ & -- & 2193\\
$(70,29)$ & 9 & $(7,3)$ & 4 & 7 & YES & YES & YES & $1.22$ & $(2,2)$ & NO & 2194\\
$(70,29)$ & 9 & $(17,7)$ & 6 & 1 & YES & YES & YES & $1.22$ & $(2,2)$ & NO & 2195\\
$(70,29)$ & 9 & $(41,17)$ & 8 & 1 & YES & YES & YES & $1.11$ & $(2,2)$ & NO & 2196\\
$(70,29)$ & 9 & $(70,29)$ & 9 & 70 & YES & YES & YES & $1.11$ & $(2,2)$ & NO & 2197\\
$(71,13)$ & 12 & $(2,1)$ & 1 & 1 & YES & YES & YES & $0.88$ & $(4,1)$ & NO & 2198\\
$(71,13)$ & 12 & $(2,1)$ & 1 & 1 & YES & YES & YES & $0.88$ & $(2,2)$ & -- & 2199\\
$(71,15)$ & 10 & $(2,1)$ & 1 & 1 & YES & YES & NO(2) & $1.10$ & $(2,2)$ & NO & 2200\\
$(71,19)$ & 10 & $(2,1)$ & 1 & 1 & YES & YES & YES & $1.25$ & $(2,2)$ & NO & 2201\\
$(71,19)$ & 10 & $(2,1)$ & 1 & 1 & YES & YES & YES & $1.00$ & $(2,2)$ & -- & 2202\\
$(71,22)$ & 10 & $(2,1)$ & 1 & 1 & YES & YES & YES & $0.88$ & $(2,2)$ & NO & 2203\\
$(71,26)$ & 9 & $(2,1)$ & 1 & 1 & YES & YES & YES & $1.00$ & $(2,2)$ & NO & 2204\\
$(71,30)$ & 9 & $(2,1)$ & 1 & 1 & YES & YES & YES & $1.00$ & $(2,2)$ & -- & 2205\\
$(71,30)$ & 9 & $(2,1)$ & 1 & 1 & YES & YES & YES & $1.00$ & $(2,2)$ & NO & 2206\\
$(71,32)$ & 10 & $(2,1)$ & 1 & 1 & NO & YES & YES & $1.12$ & $(2,2)$ & -- & 2207\\
$(71,15)$ & 10 & $(3,1)$ & 2 & 1 & YES & YES & YES & $0.88$ & $(2,2)$ & NO & 2208\\
$(71,15)$ & 10 & $(3,1)$ & 2 & 1 & YES & YES & YES & $0.88$ & $(2,2)$ & -- & 2209\\
$(71,17)$ & 11 & $(3,1)$ & 2 & 1 & YES & YES & YES & $0.88$ & $(4,1)$ & NO & 2210\\
$(71,17)$ & 11 & $(3,1)$ & 2 & 1 & YES & YES & YES & $0.88$ & $(4,1)$ & -- & 2211\\
$(71,20)$ & 10 & $(3,1)$ & 2 & 1 & NO & YES & YES & $1.12$ & $(2,2)$ & -- & 2212\\
$(71,22)$ & 10 & $(3,1)$ & 2 & 1 & YES & YES & YES & $1.12$ & $(2,2)$ & 1617 & 2213\\
$(71,22)$ & 10 & $(3,1)$ & 2 & 1 & YES & YES & YES & $1.12$ & $(2,2)$ & -- & 2214\\
$(71,26)$ & 9 & $(3,1)$ & 2 & 1 & YES & YES & YES & $1.11$ & $(2,2)$ & -- & 2215\\
$(71,30)$ & 9 & $(3,1)$ & 2 & 1 & YES & YES & YES & $1.22$ & $(2,2)$ & -- & 2216\\
$(71,30)$ & 9 & $(3,1)$ & 2 & 1 & YES & YES & YES & $1.22$ & $(2,2)$ & NO & 2217\\
$(71,11)$ & 12 & $(4,1)$ & 3 & 1 & YES & YES & NO(2) & $1.10$ & $(2,2)$ & NO & 2218\\
$(71,13)$ & 12 & $(4,1)$ & 3 & 1 & YES & YES & YES & $1.00$ & $(2,2)$ & NO & 2219\\
$(71,15)$ & 10 & $(4,1)$ & 3 & 1 & YES & YES & NO(2) & $1.00$ & $(2,2)$ & -- & 2220\\
$(71,15)$ & 10 & $(4,1)$ & 3 & 1 & YES & YES & NO(2) & $1.00$ & $(2,2)$ & NO & 2221\\
$(71,19)$ & 10 & $(4,1)$ & 3 & 1 & YES & YES & YES & $1.00$ & $(2,2)$ & NO & 2222\\
$(71,19)$ & 10 & $(4,1)$ & 3 & 1 & YES & YES & YES & $1.00$ & $(2,2)$ & -- & 2223\\
$(71,26)$ & 9 & $(4,1)$ & 3 & 1 & YES & YES & YES & $1.00$ & $(2,2)$ & -- & 2224\\
$(71,30)$ & 9 & $(4,1)$ & 3 & 1 & YES & YES & YES & $1.22$ & $(2,2)$ & NO & 2225\\
$(71,30)$ & 9 & $(4,1)$ & 3 & 1 & YES & YES & YES & $1.22$ & $(2,2)$ & -- & 2226\\
$(71,15)$ & 10 & $(5,1)$ & 4 & 1 & YES & YES & YES & $0.88$ & $(2,2)$ & NO & 2227\\
$(71,26)$ & 9 & $(5,1)$ & 4 & 1 & YES & YES & YES & $1.11$ & $(2,2)$ & NO & 2228\\
$(71,26)$ & 9 & $(5,2)$ & 3 & 1 & YES & YES & YES & $1.11$ & $(2,2)$ & NO & 2229\\
$(71,30)$ & 9 & $(5,2)$ & 3 & 1 & YES & YES & YES & $1.22$ & $(2,2)$ & NO & 2230\\
$(71,13)$ & 12 & $(6,1)$ & 5 & 1 & YES & YES & YES & $0.88$ & $(4,1)$ & NO & 2231\\
$(71,19)$ & 10 & $(7,1)$ & 6 & 1 & YES & YES & YES & $0.88$ & $(2,2)$ & NO & 2232\\
$(71,30)$ & 9 & $(7,3)$ & 4 & 1 & YES & YES & YES & $1.00$ & $(2,2)$ & NO & 2233\\
$(71,15)$ & 10 & $(9,2)$ & 5 & 1 & YES & YES & YES & $0.88$ & $(2,2)$ & NO & 2234\\
$(71,22)$ & 10 & $(10,3)$ & 5 & 1 & YES & YES & YES & $1.00$ & $(2,2)$ & NO & 2235\\
$(71,13)$ & 12 & $(11,2)$ & 6 & 1 & YES & YES & YES & $1.00$ & $(2,2)$ & NO & 2236\\
$(71,19)$ & 10 & $(11,3)$ & 5 & 1 & YES & YES & YES & $0.88$ & $(2,2)$ & NO & 2237\\
$(71,26)$ & 9 & $(11,4)$ & 5 & 1 & YES & YES & YES & $1.00$ & $(2,2)$ & 1919 & 2238\\
$(71,15)$ & 10 & $(14,3)$ & 6 & 1 & YES & YES & NO(2) & $1.10$ & $(2,2)$ & NO & 2239\\
$(71,19)$ & 10 & $(15,4)$ & 6 & 1 & YES & YES & YES & $1.00$ & $(2,2)$ & NO & 2240\\
$(71,13)$ & 12 & $(16,3)$ & 7 & 1 & YES & YES & YES & $1.00$ & $(2,2)$ & NO & 2241\\
$(71,31)$ & 10 & $(16,7)$ & 6 & 1 & YES & YES & YES & $1.12$ & $(2,2)$ & NO & 2242\\
$(71,26)$ & 9 & $(19,7)$ & 6 & 1 & YES & YES & YES & $1.22$ & $(2,2)$ & NO & 2243\\
$(71,30)$ & 9 & $(19,8)$ & 6 & 1 & YES & YES & YES & $0.88$ & $(2,2)$ & 2121 & 2244\\
$(71,19)$ & 10 & $(41,11)$ & 8 & 1 & YES & YES & YES & $0.88$ & $(2,2)$ & 2448 & 2245\\
$(71,22)$ & 10 & $(42,13)$ & 9 & 1 & YES & YES & YES & $1.00$ & $(2,2)$ & NO & 2246\\
$(71,30)$ & 9 & $(45,19)$ & 8 & 1 & YES & YES & YES & $1.22$ & $(2,2)$ & NO & 2247\\
$(71,19)$ & 10 & $(56,15)$ & 9 & 1 & YES & YES & YES & $1.00$ & $(2,2)$ & NO & 2248\\
$(71,13)$ & 12 & $(60,11)$ & 11 & 1 & YES & YES & YES & $1.00$ & $(2,2)$ & NO & 2249\\
$(71,31)$ & 10 & $(71,31)$ & 10 & 71 & YES & YES & YES & $1.00$ & $(2,2)$ & NO & 2250\\
$(72,13)$ & 12 & $(2,1)$ & 1 & 2 & YES & YES & YES & $0.75$ & $(4,1)$ & NO & 2251\\
$(72,19)$ & 10 & $(2,1)$ & 1 & 2 & YES & YES & YES & $1.00$ & $(2,2)$ & NO & 2252\\
$(72,19)$ & 10 & $(2,1)$ & 1 & 2 & YES & YES & YES & $1.11$ & $(2,2)$ & -- & 2253\\
$(72,13)$ & 12 & $(3,1)$ & 2 & 3 & YES & YES & YES & $0.88$ & $(2,2)$ & NO & 2254\\
$(72,17)$ & 11 & $(3,1)$ & 2 & 3 & YES & YES & YES & $0.88$ & $(2,2)$ & -- & 2255\\
$(72,19)$ & 10 & $(3,1)$ & 2 & 3 & YES & YES & YES & $1.00$ & $(2,2)$ & NO & 2256\\
$(72,13)$ & 12 & $(4,1)$ & 3 & 4 & YES & YES & YES & $0.88$ & $(2,2)$ & NO & 2257\\
$(72,17)$ & 11 & $(4,1)$ & 3 & 4 & NO & YES & YES & $1.00$ & $(4,1)$ & -- & 2258\\
$(72,13)$ & 12 & $(5,1)$ & 4 & 1 & YES & YES & YES & $0.75$ & $(4,1)$ & NO & 2259\\
$(72,17)$ & 11 & $(6,1)$ & 5 & 6 & YES & YES & YES & $1.00$ & $(2,2)$ & 1052 & 2260\\
$(72,19)$ & 10 & $(6,1)$ & 5 & 6 & YES & YES & YES & $1.00$ & $(2,2)$ & NO & 2261\\
$(72,13)$ & 12 & $(11,2)$ & 6 & 1 & YES & YES & YES & $0.88$ & $(2,2)$ & NO & 2262\\
$(72,19)$ & 10 & $(19,5)$ & 7 & 1 & YES & YES & YES & $1.11$ & $(2,2)$ & NO & 2263\\
$(73,27)$ & 9 & $(2,1)$ & 1 & 1 & YES & YES & YES & $1.00$ & $(2,2)$ & NO & 2264\\
$(73,31)$ & 10 & $(2,1)$ & 1 & 1 & NO & YES & YES & $1.22$ & $(2,2)$ & -- & 2265\\
$(73,27)$ & 9 & $(3,1)$ & 2 & 1 & YES & YES & YES & $1.22$ & $(2,2)$ & NO & 2266\\
$(73,27)$ & 9 & $(3,1)$ & 2 & 1 & YES & YES & YES & $1.22$ & $(2,2)$ & -- & 2267\\
$(73,27)$ & 9 & $(3,1)$ & 2 & 1 & YES & YES & YES & $1.00$ & $(2,2)$ & NO & 2268\\
$(73,27)$ & 9 & $(4,1)$ & 3 & 1 & YES & YES & YES & $1.22$ & $(2,2)$ & NO & 2269\\
$(73,27)$ & 9 & $(4,1)$ & 3 & 1 & YES & YES & YES & $1.00$ & $(2,2)$ & -- & 2270\\
$(73,27)$ & 9 & $(4,1)$ & 3 & 1 & YES & YES & YES & $1.00$ & $(2,2)$ & NO & 2271\\
$(73,27)$ & 9 & $(5,2)$ & 3 & 1 & YES & YES & YES & $1.22$ & $(2,2)$ & NO & 2272\\
$(73,27)$ & 9 & $(8,3)$ & 4 & 1 & YES & YES & YES & $1.12$ & $(2,2)$ & NO & 2273\\
$(73,27)$ & 9 & $(11,4)$ & 5 & 1 & YES & YES & YES & $1.22$ & $(2,2)$ & 2331 & 2274\\
$(73,27)$ & 9 & $(19,7)$ & 6 & 1 & YES & YES & YES & $1.00$ & $(2,2)$ & 2150 & 2275\\
$(73,27)$ & 9 & $(73,27)$ & 9 & 73 & YES & YES & YES & $1.22$ & $(2,2)$ & NO & 2276\\
$(74,17)$ & 11 & $(2,1)$ & 1 & 2 & YES & YES & YES & $1.00$ & $(2,2)$ & -- & 2277\\
$(74,29)$ & 10 & $(2,1)$ & 1 & 2 & NO & YES & YES & $1.00$ & $(4,1)$ & -- & 2278\\
$(74,31)$ & 9 & $(2,1)$ & 1 & 2 & YES & YES & YES & $1.22$ & $(2,2)$ & -- & 2279\\
$(74,31)$ & 9 & $(2,1)$ & 1 & 2 & YES & YES & YES & $1.12$ & $(2,2)$ & NO & 2280\\
$(74,31)$ & 9 & $(3,1)$ & 2 & 1 & YES & YES & YES & $1.22$ & $(2,2)$ & NO & 2281\\
$(74,17)$ & 11 & $(4,1)$ & 3 & 2 & YES & YES & YES & $1.00$ & $(2,2)$ & NO & 2282\\
$(74,17)$ & 11 & $(5,1)$ & 4 & 1 & YES & YES & YES & $1.00$ & $(2,2)$ & NO & 2283\\
$(74,31)$ & 9 & $(7,3)$ & 4 & 1 & YES & YES & YES & $1.22$ & $(2,2)$ & 1657 & 2284\\
$(74,17)$ & 11 & $(9,2)$ & 5 & 1 & YES & YES & YES & $1.00$ & $(2,2)$ & NO & 2285\\
$(74,31)$ & 9 & $(74,31)$ & 9 & 74 & YES & YES & YES & $1.22$ & $(2,2)$ & NO & 2286\\
$(75,31)$ & 9 & $(2,1)$ & 1 & 1 & YES & YES & YES & $1.00$ & $(2,2)$ & 1734 & 2287\\
$(75,22)$ & 10 & $(3,1)$ & 2 & 3 & YES & YES & YES & $1.33$ & $(2,2)$ & -- & 2288\\
$(75,29)$ & 9 & $(3,1)$ & 2 & 3 & YES & YES & YES & $1.22$ & $(2,2)$ & -- & 2289\\
$(75,31)$ & 9 & $(3,1)$ & 2 & 3 & YES & YES & YES & $1.22$ & $(2,2)$ & -- & 2290\\
$(75,31)$ & 9 & $(3,1)$ & 2 & 3 & YES & YES & YES & $1.00$ & $(2,2)$ & NO & 2291\\
$(75,22)$ & 10 & $(5,1)$ & 4 & 5 & YES & YES & YES & $1.22$ & $(2,2)$ & NO & 2292\\
$(75,31)$ & 9 & $(5,2)$ & 3 & 5 & YES & YES & YES & $1.00$ & $(2,2)$ & 2052 & 2293\\
$(75,22)$ & 10 & $(6,1)$ & 5 & 3 & YES & YES & YES & $1.11$ & $(2,2)$ & NO & 2294\\
$(75,31)$ & 9 & $(12,5)$ & 5 & 3 & YES & YES & YES & $1.22$ & $(2,2)$ & NO & 2295\\
$(75,29)$ & 9 & $(18,7)$ & 6 & 3 & YES & YES & YES & $1.33$ & $(2,2)$ & NO & 2296\\
$(75,22)$ & 10 & $(24,7)$ & 7 & 3 & YES & YES & YES & $1.22$ & $(2,2)$ & NO & 2297\\
$(75,22)$ & 10 & $(41,12)$ & 8 & 1 & YES & YES & YES & $1.22$ & $(2,2)$ & 2460 & 2298\\
$(75,22)$ & 10 & $(75,22)$ & 10 & 75 & YES & YES & YES & $1.33$ & $(2,2)$ & NO & 2299\\
$(76,13)$ & 12 & $(2,1)$ & 1 & 2 & YES & YES & YES & $0.88$ & $(2,2)$ & NO & 2300\\
$(76,13)$ & 12 & $(2,1)$ & 1 & 2 & YES & YES & YES & $0.88$ & $(2,2)$ & -- & 2301\\
$(76,27)$ & 10 & $(2,1)$ & 1 & 2 & YES & YES & YES & $1.00$ & $(2,2)$ & NO & 2302\\
$(76,13)$ & 12 & $(3,1)$ & 2 & 1 & YES & YES & YES & $0.88$ & $(2,2)$ & NO & 2303\\
$(76,13)$ & 12 & $(3,1)$ & 2 & 1 & YES & YES & YES & $0.88$ & $(2,2)$ & -- & 2304\\
$(76,27)$ & 10 & $(3,1)$ & 2 & 1 & YES & YES & YES & $1.00$ & $(2,2)$ & 1708 & 2305\\
$(76,13)$ & 12 & $(4,1)$ & 3 & 4 & YES & YES & YES & $0.88$ & $(2,2)$ & NO & 2306\\
$(76,13)$ & 12 & $(76,13)$ & 12 & 76 & YES & YES & YES & $0.88$ & $(2,2)$ & NO & 2307\\
$(77,34)$ & 10 & $(2,1)$ & 1 & 1 & NO & YES & YES & $1.22$ & $(2,2)$ & -- & 2308\\
$(77,16)$ & 11 & $(3,1)$ & 2 & 1 & YES & YES & YES & $0.88$ & $(4,1)$ & NO & 2309\\
$(77,16)$ & 11 & $(3,1)$ & 2 & 1 & YES & YES & YES & $0.88$ & $(4,1)$ & -- & 2310\\
$(79,14)$ & 11 & $(2,1)$ & 1 & 1 & YES & YES & YES & $1.00$ & $(2,2)$ & NO & 2311\\
$(79,17)$ & 11 & $(2,1)$ & 1 & 1 & YES & YES & YES & $1.00$ & $(2,2)$ & NO & 2312\\
$(79,17)$ & 11 & $(2,1)$ & 1 & 1 & YES & YES & YES & $1.12$ & $(2,2)$ & -- & 2313\\
$(79,24)$ & 10 & $(2,1)$ & 1 & 1 & YES & YES & YES & $1.00$ & $(2,2)$ & NO & 2314\\
$(79,29)$ & 9 & $(2,1)$ & 1 & 1 & YES & YES & YES & $1.22$ & $(2,2)$ & -- & 2315\\
$(79,29)$ & 9 & $(2,1)$ & 1 & 1 & YES & YES & YES & $1.00$ & $(2,2)$ & NO & 2316\\
$(79,30)$ & 9 & $(2,1)$ & 1 & 1 & NO & YES & YES & $1.00$ & $(4,1)$ & -- & 2317\\
$(79,14)$ & 11 & $(3,1)$ & 2 & 1 & YES & YES & YES & $1.00$ & $(2,2)$ & NO & 2318\\
$(79,14)$ & 11 & $(3,1)$ & 2 & 1 & YES & YES & YES & $1.00$ & $(2,2)$ & -- & 2319\\
$(79,22)$ & 10 & $(3,1)$ & 2 & 1 & YES & YES & YES & $1.22$ & $(2,2)$ & -- & 2320\\
$(79,24)$ & 10 & $(3,1)$ & 2 & 1 & YES & YES & YES & $1.22$ & $(2,2)$ & -- & 2321\\
$(79,29)$ & 9 & $(3,1)$ & 2 & 1 & YES & YES & YES & $1.00$ & $(2,2)$ & -- & 2322\\
$(79,30)$ & 9 & $(3,1)$ & 2 & 1 & YES & YES & YES & $1.22$ & $(2,2)$ & NO & 2323\\
$(79,14)$ & 11 & $(4,1)$ & 3 & 1 & YES & YES & YES & $0.88$ & $(2,2)$ & NO & 2324\\
$(79,17)$ & 11 & $(4,1)$ & 3 & 1 & YES & YES & YES & $1.00$ & $(2,2)$ & NO & 2325\\
$(79,17)$ & 11 & $(5,1)$ & 4 & 1 & YES & YES & YES & $1.00$ & $(2,2)$ & NO & 2326\\
$(79,18)$ & 10 & $(5,2)$ & 3 & 1 & YES & YES & YES & $1.22$ & $(2,2)$ & NO & 2327\\
$(79,18)$ & 10 & $(5,2)$ & 3 & 1 & YES & YES & YES & $1.22$ & $(2,2)$ & -- & 2328\\
$(79,22)$ & 10 & $(5,1)$ & 4 & 1 & YES & YES & YES & $1.22$ & $(2,2)$ & NO & 2329\\
$(79,22)$ & 10 & $(6,1)$ & 5 & 1 & YES & YES & YES & $1.22$ & $(2,2)$ & NO & 2330\\
$(79,29)$ & 9 & $(8,3)$ & 4 & 1 & YES & YES & YES & $1.22$ & $(2,2)$ & 2274 & 2331\\
$(79,17)$ & 11 & $(9,2)$ & 5 & 1 & YES & YES & YES & $1.12$ & $(2,2)$ & NO & 2332\\
$(79,29)$ & 9 & $(11,4)$ & 5 & 1 & YES & YES & YES & $1.00$ & $(2,2)$ & NO & 2333\\
$(79,17)$ & 11 & $(14,3)$ & 6 & 1 & YES & YES & YES & $0.88$ & $(4,1)$ & NO & 2334\\
$(79,14)$ & 11 & $(23,4)$ & 8 & 1 & YES & YES & YES & $1.00$ & $(2,2)$ & NO & 2335\\
$(79,22)$ & 10 & $(25,7)$ & 7 & 1 & YES & YES & YES & $1.22$ & $(2,2)$ & NO & 2336\\
$(79,29)$ & 9 & $(30,11)$ & 7 & 1 & YES & YES & YES & $1.22$ & $(2,2)$ & NO & 2337\\
$(79,24)$ & 10 & $(33,10)$ & 8 & 1 & YES & YES & YES & $1.11$ & $(2,2)$ & 2410 & 2338\\
$(79,22)$ & 10 & $(43,12)$ & 8 & 1 & YES & YES & YES & $1.22$ & $(2,2)$ & 2483 & 2339\\
$(79,30)$ & 9 & $(50,19)$ & 8 & 1 & YES & YES & YES & $1.22$ & $(2,2)$ & NO & 2340\\
$(79,17)$ & 11 & $(51,11)$ & 9 & 1 & YES & YES & YES & $0.88$ & $(2,2)$ & 2502 & 2341\\
$(79,24)$ & 10 & $(79,24)$ & 10 & 79 & YES & YES & YES & $1.22$ & $(2,2)$ & NO & 2342\\
$(80,19)$ & 11 & $(2,1)$ & 1 & 2 & YES & YES & YES & $0.88$ & $(2,2)$ & NO & 2343\\
$(80,19)$ & 11 & $(2,1)$ & 1 & 2 & YES & YES & YES & $1.00$ & $(2,2)$ & -- & 2344\\
$(80,31)$ & 9 & $(3,1)$ & 2 & 1 & YES & YES & YES & $1.22$ & $(2,2)$ & -- & 2345\\
$(80,19)$ & 11 & $(6,1)$ & 5 & 2 & YES & YES & YES & $1.00$ & $(2,2)$ & NO & 2346\\
$(80,31)$ & 9 & $(13,5)$ & 5 & 1 & YES & YES & YES & $1.22$ & $(2,2)$ & NO & 2347\\
$(80,19)$ & 11 & $(21,5)$ & 8 & 1 & YES & YES & YES & $1.00$ & $(2,2)$ & NO & 2348\\
$(80,31)$ & 9 & $(31,12)$ & 7 & 1 & YES & YES & YES & $1.22$ & $(2,2)$ & NO & 2349\\
$(80,19)$ & 11 & $(38,9)$ & 9 & 2 & YES & YES & YES & $1.00$ & $(2,2)$ & 2449 & 2350\\
$(81,19)$ & 11 & $(2,1)$ & 1 & 1 & YES & YES & YES & $1.00$ & $(2,2)$ & NO & 2351\\
$(81,19)$ & 11 & $(2,1)$ & 1 & 1 & YES & YES & YES & $1.22$ & $(2,2)$ & -- & 2352\\
$(81,31)$ & 9 & $(2,1)$ & 1 & 1 & YES & YES & YES & $1.11$ & $(2,2)$ & NO & 2353\\
$(81,31)$ & 9 & $(3,1)$ & 2 & 3 & YES & YES & YES & $1.22$ & $(2,2)$ & NO & 2354\\
$(81,31)$ & 9 & $(3,1)$ & 2 & 3 & YES & YES & YES & $1.22$ & $(2,2)$ & -- & 2355\\
$(81,19)$ & 11 & $(5,1)$ & 4 & 1 & YES & YES & YES & $1.11$ & $(2,2)$ & NO & 2356\\
$(81,31)$ & 9 & $(5,1)$ & 4 & 1 & YES & YES & YES & $1.11$ & $(2,2)$ & NO & 2357\\
$(81,19)$ & 11 & $(13,3)$ & 6 & 1 & YES & YES & YES & $1.11$ & $(2,2)$ & NO & 2358\\
$(81,31)$ & 9 & $(13,5)$ & 5 & 1 & YES & YES & YES & $1.11$ & $(2,2)$ & 2037 & 2359\\
$(81,31)$ & 9 & $(34,13)$ & 7 & 1 & YES & YES & YES & $1.11$ & $(2,2)$ & NO & 2360\\
$(81,31)$ & 9 & $(81,31)$ & 9 & 81 & YES & YES & YES & $1.22$ & $(2,2)$ & NO & 2361\\
$(82,37)$ & 10 & $(2,1)$ & 1 & 2 & NO & YES & YES & $1.12$ & $(2,2)$ & -- & 2362\\
$(82,17)$ & 11 & $(3,1)$ & 2 & 1 & YES & YES & YES & $1.00$ & $(2,2)$ & NO & 2363\\
$(82,17)$ & 11 & $(3,1)$ & 2 & 1 & YES & YES & YES & $1.00$ & $(2,2)$ & -- & 2364\\
$(82,17)$ & 11 & $(4,1)$ & 3 & 2 & YES & YES & YES & $1.00$ & $(2,2)$ & -- & 2365\\
$(82,19)$ & 12 & $(4,1)$ & 3 & 2 & YES & YES & YES & $1.12$ & $(2,2)$ & NO & 2366\\
$(82,23)$ & 10 & $(11,3)$ & 5 & 1 & YES & YES & YES & $1.11$ & $(2,2)$ & NO & 2367\\
$(82,23)$ & 10 & $(18,5)$ & 6 & 2 & YES & YES & YES & $1.11$ & $(2,2)$ & NO & 2368\\
$(82,19)$ & 12 & $(43,10)$ & 9 & 1 & YES & YES & YES & $1.00$ & $(2,2)$ & NO & 2369\\
$(82,17)$ & 11 & $(53,11)$ & 10 & 1 & YES & YES & YES & $1.00$ & $(2,2)$ & NO & 2370\\
$(82,17)$ & 11 & $(82,17)$ & 11 & 82 & YES & YES & YES & $1.00$ & $(2,2)$ & NO & 2371\\
$(83,22)$ & 10 & $(3,1)$ & 2 & 1 & YES & YES & YES & $1.12$ & $(2,2)$ & NO & 2372\\
$(83,22)$ & 10 & $(4,1)$ & 3 & 1 & YES & YES & YES & $0.75$ & $(4,1)$ & -- & 2373\\
$(83,22)$ & 10 & $(4,1)$ & 3 & 1 & YES & YES & YES & $0.88$ & $(4,1)$ & 1836 & 2374\\
$(83,22)$ & 10 & $(5,1)$ & 4 & 1 & YES & YES & YES & $0.88$ & $(2,2)$ & -- & 2375\\
$(83,22)$ & 10 & $(15,4)$ & 6 & 1 & YES & YES & YES & $1.00$ & $(2,2)$ & 2122 & 2376\\
$(83,22)$ & 10 & $(34,9)$ & 8 & 1 & YES & YES & YES & $0.88$ & $(2,2)$ & NO & 2377\\
$(84,37)$ & 10 & $(2,1)$ & 1 & 2 & NO & YES & YES & $0.88$ & $(4,1)$ & -- & 2378\\
$(84,13)$ & 13 & $(4,1)$ & 3 & 4 & YES & YES & YES & $1.25$ & $(2,2)$ & -- & 2379\\
$(84,13)$ & 13 & $(4,1)$ & 3 & 4 & YES & YES & YES & $1.38$ & $(2,2)$ & NO & 2380\\
$(84,13)$ & 13 & $(20,3)$ & 8 & 4 & YES & YES & YES & $1.12$ & $(2,2)$ & NO & 2381\\
$(85,36)$ & 10 & $(2,1)$ & 1 & 1 & NO & YES & YES & $1.00$ & $(2,2)$ & -- & 2382\\
$(85,38)$ & 11 & $(2,1)$ & 1 & 1 & NO & YES & YES & $1.33$ & $(2,2)$ & -- & 2383\\
$(85,23)$ & 10 & $(4,1)$ & 3 & 1 & YES & YES & YES & $0.88$ & $(2,2)$ & NO & 2384\\
$(85,16)$ & 12 & $(7,1)$ & 6 & 1 & YES & YES & YES & $0.88$ & $(2,2)$ & NO & 2385\\
$(85,23)$ & 10 & $(11,3)$ & 5 & 1 & YES & YES & YES & $0.88$ & $(2,2)$ & 2032 & 2386\\
$(85,23)$ & 10 & $(37,10)$ & 8 & 1 & YES & YES & YES & $1.00$ & $(2,2)$ & NO & 2387\\
$(86,27)$ & 11 & $(3,1)$ & 2 & 1 & NO & YES & YES & $1.12$ & $(2,2)$ & -- & 2388\\
$(88,21)$ & 12 & $(21,5)$ & 8 & 1 & YES & YES & YES & $1.12$ & $(2,2)$ & NO & 2389\\
$(88,21)$ & 12 & $(46,11)$ & 10 & 2 & YES & YES & YES & $1.00$ & $(2,2)$ & 2498 & 2390\\
$(89,17)$ & 12 & $(2,1)$ & 1 & 1 & YES & YES & YES & $0.88$ & $(2,2)$ & -- & 2391\\
$(89,17)$ & 12 & $(2,1)$ & 1 & 1 & YES & YES & YES & $0.88$ & $(2,2)$ & NO & 2392\\
$(89,26)$ & 10 & $(2,1)$ & 1 & 1 & YES & YES & YES & $1.22$ & $(2,2)$ & -- & 2393\\
$(89,27)$ & 10 & $(2,1)$ & 1 & 1 & YES & YES & YES & $1.11$ & $(2,2)$ & NO & 2394\\
$(89,17)$ & 12 & $(3,1)$ & 2 & 1 & YES & YES & YES & $0.88$ & $(2,2)$ & NO & 2395\\
$(89,17)$ & 12 & $(3,1)$ & 2 & 1 & YES & YES & YES & $0.88$ & $(2,2)$ & -- & 2396\\
$(89,17)$ & 12 & $(3,1)$ & 2 & 1 & YES & YES & YES & $1.22$ & $(2,2)$ & NO & 2397\\
$(89,24)$ & 10 & $(3,1)$ & 2 & 1 & YES & YES & YES & $1.11$ & $(2,2)$ & -- & 2398\\
$(89,25)$ & 10 & $(3,1)$ & 2 & 1 & NO & YES & YES & $1.12$ & $(2,2)$ & -- & 2399\\
$(89,17)$ & 12 & $(4,1)$ & 3 & 1 & YES & YES & YES & $0.88$ & $(2,2)$ & NO & 2400\\
$(89,27)$ & 10 & $(4,1)$ & 3 & 1 & YES & YES & YES & $1.11$ & $(2,2)$ & NO & 2401\\
$(89,17)$ & 12 & $(5,1)$ & 4 & 1 & YES & YES & YES & $0.88$ & $(2,2)$ & NO & 2402\\
$(89,19)$ & 12 & $(5,1)$ & 4 & 1 & YES & YES & YES & $1.00$ & $(2,2)$ & NO & 2403\\
$(89,17)$ & 12 & $(6,1)$ & 5 & 1 & YES & YES & NO(2) & $1.10$ & $(2,2)$ & NO & 2404\\
$(89,26)$ & 10 & $(6,1)$ & 5 & 1 & YES & YES & YES & $1.11$ & $(2,2)$ & NO & 2405\\
$(89,24)$ & 10 & $(7,2)$ & 4 & 1 & YES & YES & YES & $1.22$ & $(2,2)$ & 2146 & 2406\\
$(89,17)$ & 12 & $(11,2)$ & 6 & 1 & YES & YES & YES & $0.88$ & $(2,2)$ & NO & 2407\\
$(89,17)$ & 12 & $(16,3)$ & 7 & 1 & YES & YES & YES & $0.88$ & $(2,2)$ & NO & 2408\\
$(89,26)$ & 10 & $(17,5)$ & 6 & 1 & YES & YES & YES & $1.22$ & $(2,2)$ & NO & 2409\\
$(89,27)$ & 10 & $(23,7)$ & 7 & 1 & YES & YES & YES & $1.11$ & $(2,2)$ & 2338 & 2410\\
$(89,26)$ & 10 & $(24,7)$ & 7 & 1 & YES & YES & YES & $1.22$ & $(2,2)$ & NO & 2411\\
$(89,24)$ & 10 & $(37,10)$ & 8 & 1 & YES & YES & YES & $1.11$ & $(2,2)$ & 2468 & 2412\\
$(89,24)$ & 10 & $(89,24)$ & 10 & 89 & YES & YES & YES & $1.22$ & $(2,2)$ & NO & 2413\\
$(89,27)$ & 10 & $(89,27)$ & 10 & 89 & YES & YES & YES & $1.22$ & $(2,2)$ & NO & 2414\\
$(90,19)$ & 11 & $(2,1)$ & 1 & 2 & YES & YES & YES & $1.11$ & $(2,2)$ & NO & 2415\\
$(90,19)$ & 11 & $(4,1)$ & 3 & 2 & YES & YES & YES & $1.00$ & $(2,2)$ & NO & 2416\\
$(91,17)$ & 12 & $(2,1)$ & 1 & 1 & YES & YES & YES & $0.88$ & $(2,2)$ & -- & 2417\\
$(91,17)$ & 12 & $(2,1)$ & 1 & 1 & YES & YES & YES & $1.00$ & $(2,2)$ & NO & 2418\\
$(91,19)$ & 11 & $(3,1)$ & 2 & 1 & YES & YES & YES & $0.88$ & $(4,1)$ & NO & 2419\\
$(91,19)$ & 11 & $(3,1)$ & 2 & 1 & YES & YES & YES & $0.88$ & $(4,1)$ & -- & 2420\\
$(91,27)$ & 10 & $(3,1)$ & 2 & 1 & YES & YES & YES & $1.33$ & $(2,2)$ & -- & 2421\\
$(91,17)$ & 12 & $(5,1)$ & 4 & 1 & YES & YES & YES & $0.88$ & $(2,2)$ & NO & 2422\\
$(91,17)$ & 12 & $(6,1)$ & 5 & 1 & YES & YES & YES & $1.11$ & $(2,2)$ & NO & 2423\\
$(91,27)$ & 10 & $(7,2)$ & 4 & 7 & YES & YES & YES & $1.33$ & $(2,2)$ & NO & 2424\\
$(91,17)$ & 12 & $(11,2)$ & 6 & 1 & YES & YES & YES & $1.11$ & $(2,2)$ & NO & 2425\\
$(91,17)$ & 12 & $(16,3)$ & 7 & 1 & YES & YES & YES & $0.75$ & $(4,1)$ & NO & 2426\\
$(92,19)$ & 12 & $(2,1)$ & 1 & 2 & YES & YES & YES & $1.00$ & $(2,2)$ & -- & 2427\\
$(92,19)$ & 12 & $(2,1)$ & 1 & 2 & YES & YES & YES & $1.12$ & $(2,2)$ & NO & 2428\\
$(92,19)$ & 12 & $(4,1)$ & 3 & 4 & YES & YES & YES & $1.12$ & $(2,2)$ & NO & 2429\\
$(93,26)$ & 10 & $(2,1)$ & 1 & 1 & YES & YES & YES & $1.33$ & $(2,2)$ & NO & 2430\\
$(93,26)$ & 10 & $(2,1)$ & 1 & 1 & YES & YES & YES & $1.33$ & $(2,2)$ & -- & 2431\\
$(93,26)$ & 10 & $(6,1)$ & 5 & 3 & YES & YES & YES & $1.22$ & $(2,2)$ & NO & 2432\\
$(93,25)$ & 10 & $(15,4)$ & 6 & 3 & YES & YES & YES & $1.11$ & $(2,2)$ & NO & 2433\\
$(93,26)$ & 10 & $(18,5)$ & 6 & 3 & YES & YES & YES & $1.33$ & $(2,2)$ & NO & 2434\\
$(93,26)$ & 10 & $(25,7)$ & 7 & 1 & YES & YES & YES & $1.22$ & $(2,2)$ & NO & 2435\\
$(96,17)$ & 12 & $(2,1)$ & 1 & 2 & YES & YES & YES & $0.88$ & $(2,2)$ & NO & 2436\\
$(96,17)$ & 12 & $(5,1)$ & 4 & 1 & YES & YES & YES & $0.88$ & $(2,2)$ & NO & 2437\\
$(96,17)$ & 12 & $(6,1)$ & 5 & 6 & YES & YES & YES & $1.00$ & $(2,2)$ & NO & 2438\\
$(96,17)$ & 12 & $(79,14)$ & 11 & 1 & YES & YES & YES & $1.00$ & $(2,2)$ & NO & 2439\\
$(97,23)$ & 11 & $(2,1)$ & 1 & 1 & YES & YES & YES & $0.88$ & $(2,2)$ & NO & 2440\\
$(97,26)$ & 10 & $(2,1)$ & 1 & 1 & YES & YES & YES & $1.00$ & $(2,2)$ & -- & 2441\\
$(97,26)$ & 10 & $(2,1)$ & 1 & 1 & YES & YES & YES & $1.12$ & $(2,2)$ & NO & 2442\\
$(97,22)$ & 11 & $(3,1)$ & 2 & 1 & YES & YES & YES & $1.22$ & $(2,2)$ & NO & 2443\\
$(97,26)$ & 10 & $(3,1)$ & 2 & 1 & YES & YES & YES & $1.00$ & $(2,2)$ & NO & 2444\\
$(97,26)$ & 10 & $(4,1)$ & 3 & 1 & YES & YES & YES & $0.75$ & $(4,1)$ & NO & 2445\\
$(97,22)$ & 11 & $(6,1)$ & 5 & 1 & YES & YES & YES & $1.11$ & $(2,2)$ & NO & 2446\\
$(97,22)$ & 11 & $(13,3)$ & 6 & 1 & YES & YES & YES & $1.11$ & $(2,2)$ & NO & 2447\\
$(97,26)$ & 10 & $(15,4)$ & 6 & 1 & YES & YES & YES & $0.88$ & $(2,2)$ & 2245 & 2448\\
$(97,23)$ & 11 & $(21,5)$ & 8 & 1 & YES & YES & YES & $1.00$ & $(2,2)$ & 2350 & 2449\\
$(98,27)$ & 10 & $(4,1)$ & 3 & 2 & YES & YES & YES & $1.11$ & $(2,2)$ & 1975 & 2450\\
$(98,27)$ & 10 & $(11,3)$ & 5 & 1 & YES & YES & YES & $1.11$ & $(2,2)$ & NO & 2451\\
$(99,17)$ & 12 & $(2,1)$ & 1 & 1 & YES & YES & YES & $0.88$ & $(2,2)$ & NO & 2452\\
$(99,17)$ & 12 & $(2,1)$ & 1 & 1 & YES & YES & YES & $0.88$ & $(2,2)$ & -- & 2453\\
$(99,29)$ & 10 & $(2,1)$ & 1 & 1 & YES & YES & YES & $1.22$ & $(2,2)$ & -- & 2454\\
$(99,17)$ & 12 & $(3,1)$ & 2 & 3 & YES & YES & YES & $0.88$ & $(2,2)$ & NO & 2455\\
$(99,17)$ & 12 & $(3,1)$ & 2 & 3 & YES & YES & YES & $0.88$ & $(2,2)$ & -- & 2456\\
$(99,29)$ & 10 & $(3,1)$ & 2 & 3 & YES & YES & YES & $1.11$ & $(2,2)$ & NO & 2457\\
$(99,29)$ & 10 & $(5,1)$ & 4 & 1 & YES & YES & YES & $1.22$ & $(2,2)$ & NO & 2458\\
$(99,29)$ & 10 & $(7,2)$ & 4 & 1 & YES & YES & YES & $1.22$ & $(2,2)$ & 1996 & 2459\\
$(99,29)$ & 10 & $(17,5)$ & 6 & 1 & YES & YES & YES & $1.22$ & $(2,2)$ & 2298 & 2460\\
$(99,17)$ & 12 & $(99,17)$ & 12 & 99 & YES & YES & YES & $0.88$ & $(2,2)$ & NO & 2461\\
$(100,19)$ & 12 & $(2,1)$ & 1 & 2 & YES & YES & YES & $0.88$ & $(2,2)$ & NO & 2462\\
$(100,27)$ & 10 & $(2,1)$ & 1 & 2 & YES & YES & YES & $1.22$ & $(2,2)$ & NO & 2463\\
$(100,39)$ & 10 & $(2,1)$ & 1 & 2 & NO & YES & YES & $1.11$ & $(2,2)$ & -- & 2464\\
$(100,27)$ & 10 & $(3,1)$ & 2 & 1 & YES & YES & YES & $1.22$ & $(2,2)$ & NO & 2465\\
$(100,19)$ & 12 & $(5,1)$ & 4 & 5 & YES & YES & YES & $0.88$ & $(2,2)$ & NO & 2466\\
$(100,19)$ & 12 & $(21,4)$ & 8 & 1 & YES & YES & YES & $0.88$ & $(2,2)$ & NO & 2467\\
$(100,27)$ & 10 & $(26,7)$ & 7 & 2 & YES & YES & YES & $1.11$ & $(2,2)$ & 2412 & 2468\\
$(101,16)$ & 13 & $(2,1)$ & 1 & 1 & YES & YES & YES & $1.00$ & $(2,2)$ & NO & 2469\\
$(101,23)$ & 11 & $(2,1)$ & 1 & 1 & YES & YES & YES & $1.22$ & $(2,2)$ & NO & 2470\\
$(101,16)$ & 13 & $(3,1)$ & 2 & 1 & YES & YES & YES & $1.00$ & $(2,2)$ & NO & 2471\\
$(101,16)$ & 13 & $(3,1)$ & 2 & 1 & YES & YES & YES & $1.00$ & $(2,2)$ & -- & 2472\\
$(101,22)$ & 11 & $(3,1)$ & 2 & 1 & YES & YES & YES & $1.22$ & $(2,2)$ & NO & 2473\\
$(101,16)$ & 13 & $(5,1)$ & 4 & 1 & YES & YES & YES & $1.00$ & $(2,2)$ & NO & 2474\\
$(101,23)$ & 11 & $(5,1)$ & 4 & 1 & YES & YES & YES & $1.22$ & $(2,2)$ & NO & 2475\\
$(101,16)$ & 13 & $(6,1)$ & 5 & 1 & YES & YES & NO(2) & $0.89$ & $(4,1)$ & NO & 2476\\
$(101,22)$ & 11 & $(6,1)$ & 5 & 1 & YES & YES & YES & $1.00$ & $(2,2)$ & NO & 2477\\
$(101,16)$ & 13 & $(7,1)$ & 6 & 1 & YES & YES & NO(2) & $0.89$ & $(4,1)$ & NO & 2478\\
$(101,23)$ & 11 & $(9,2)$ & 5 & 1 & YES & YES & YES & $1.22$ & $(2,2)$ & NO & 2479\\
$(101,22)$ & 11 & $(14,3)$ & 6 & 1 & YES & YES & YES & $1.00$ & $(2,2)$ & NO & 2480\\
$(104,29)$ & 10 & $(4,1)$ & 3 & 4 & YES & YES & YES & $1.22$ & $(2,2)$ & NO & 2481\\
$(104,29)$ & 10 & $(5,1)$ & 4 & 1 & YES & YES & YES & $1.22$ & $(2,2)$ & NO & 2482\\
$(104,29)$ & 10 & $(18,5)$ & 6 & 2 & YES & YES & YES & $1.22$ & $(2,2)$ & 2339 & 2483\\
$(104,29)$ & 10 & $(61,17)$ & 9 & 1 & YES & YES & YES & $1.22$ & $(2,2)$ & NO & 2484\\
$(105,23)$ & 11 & $(5,1)$ & 4 & 5 & YES & YES & YES & $0.88$ & $(2,2)$ & 1592 & 2485\\
$(105,23)$ & 11 & $(32,7)$ & 8 & 1 & YES & YES & YES & $0.88$ & $(2,2)$ & NO & 2486\\
$(106,23)$ & 11 & $(2,1)$ & 1 & 2 & YES & YES & YES & $1.11$ & $(2,2)$ & -- & 2487\\
$(106,23)$ & 11 & $(2,1)$ & 1 & 2 & YES & YES & YES & $1.22$ & $(2,2)$ & NO & 2488\\
$(106,23)$ & 11 & $(3,1)$ & 2 & 1 & YES & YES & YES & $1.33$ & $(2,2)$ & -- & 2489\\
$(106,23)$ & 11 & $(4,1)$ & 3 & 2 & YES & YES & YES & $1.11$ & $(2,2)$ & NO & 2490\\
$(106,23)$ & 11 & $(9,2)$ & 5 & 1 & YES & YES & YES & $1.33$ & $(2,2)$ & NO & 2491\\
$(110,23)$ & 11 & $(3,1)$ & 2 & 1 & YES & YES & YES & $1.00$ & $(2,2)$ & NO & 2492\\
$(110,23)$ & 11 & $(3,1)$ & 2 & 1 & YES & YES & YES & $1.00$ & $(2,2)$ & -- & 2493\\
$(110,23)$ & 11 & $(67,14)$ & 10 & 1 & YES & YES & YES & $1.00$ & $(2,2)$ & NO & 2494\\
$(111,26)$ & 11 & $(2,1)$ & 1 & 1 & YES & YES & YES & $1.00$ & $(2,2)$ & NO & 2495\\
$(113,24)$ & 11 & $(4,1)$ & 3 & 1 & YES & YES & YES & $1.00$ & $(2,2)$ & -- & 2496\\
$(113,24)$ & 11 & $(14,3)$ & 6 & 1 & YES & YES & YES & $1.00$ & $(2,2)$ & NO & 2497\\
$(113,27)$ & 12 & $(21,5)$ & 8 & 1 & YES & YES & YES & $1.00$ & $(2,2)$ & 2390 & 2498\\
$(113,24)$ & 11 & $(33,7)$ & 8 & 1 & YES & YES & YES & $1.11$ & $(2,2)$ & NO & 2499\\
$(116,25)$ & 11 & $(2,1)$ & 1 & 2 & YES & YES & YES & $0.88$ & $(2,2)$ & NO & 2500\\
$(116,25)$ & 11 & $(5,1)$ & 4 & 1 & YES & YES & YES & $0.88$ & $(2,2)$ & NO & 2501\\
$(116,25)$ & 11 & $(14,3)$ & 6 & 2 & YES & YES & YES & $0.88$ & $(2,2)$ & 2341 & 2502\\
$(120,19)$ & 14 & $(6,1)$ & 5 & 6 & YES & YES & YES & $1.00$ & $(2,2)$ & NO & 2503\\
$(123,22)$ & 12 & $(3,1)$ & 2 & 3 & YES & YES & YES & $0.88$ & $(2,2)$ & -- & 2504\\
$(123,22)$ & 12 & $(6,1)$ & 5 & 3 & YES & YES & YES & $0.88$ & $(2,2)$ & NO & 2505\\
$(123,22)$ & 12 & $(11,2)$ & 6 & 1 & YES & YES & YES & $1.00$ & $(2,2)$ & NO & 2506\\
$(123,22)$ & 12 & $(28,5)$ & 8 & 1 & YES & YES & YES & $0.88$ & $(2,2)$ & NO & 2507\\
$(a;1,0,0;13)$ & 5 & $(9,4)$ & 5 & 1 & YES & YES & YES & $0.88$ & $(2,2)$ & -- & 2508\\
$(a;1,0,0;13)$ & 5 & $(10,3)$ & 5 & 1 & YES & YES & YES & $1.22$ & $(2,2)$ & -- & 2509\\
$(a;1,0,0;13)$ & 5 & $(16,5)$ & 7 & 1 & YES & YES & YES & $1.22$ & $(2,2)$ & -- & 2510\\
$(a;1,1,0;19)$ & 6 & $(7,3)$ & 4 & 1 & YES & YES & YES & $1.00$ & $(2,2)$ & -- & 2511\\
$(a;2,0,0;17)$ & 6 & $(6,1)$ & 5 & 1 & YES & YES & YES & $1.12$ & $(2,2)$ & -- & 2512\\
$(a;2,0,0;17)$ & 6 & $(7,3)$ & 4 & 1 & YES & YES & YES & $0.88$ & $(2,2)$ & -- & 2513\\
$(a;2,0,0;17)$ & 6 & $(9,4)$ & 5 & 1 & YES & YES & YES & $1.12$ & $(2,2)$ & -- & 2514\\
$(a;2,0,0;17)$ & 6 & $(10,3)$ & 5 & 1 & YES & YES & YES & $0.88$ & $(2,2)$ & -- & 2515\\
$(a;2,0,0;17)$ & 6 & $(11,5)$ & 6 & 1 & YES & YES & YES & $1.12$ & $(2,2)$ & -- & 2516\\
$(a;2,0,1;25)$ & 7 & $(5,2)$ & 3 & 5 & YES & YES & YES & $1.00$ & $(2,2)$ & -- & 2517\\
$(a;2,0,1;25)$ & 7 & $(7,2)$ & 4 & 1 & YES & YES & YES & $0.88$ & $(2,2)$ & -- & 2518\\
$(a;2,0,1;25)$ & 7 & $(13,3)$ & 6 & 1 & YES & YES & YES & $1.11$ & $(2,2)$ & -- & 2519\\
$(a;2,1,0;5)$ & 7 & $(3,1)$ & 2 & 1 & YES & YES & YES & $1.12$ & $(2,2)$ & -- & 2520\\
$(a;2,1,1;37)$ & 8 & $(2,1)$ & 1 & 1 & YES & YES & YES & $1.12$ & $(2,2)$ & -- & 2521\\
$(a;2,1,1;37)$ & 8 & $(3,1)$ & 2 & 1 & YES & YES & YES & $1.12$ & $(2,2)$ & -- & 2522\\
$(a;2,2,0;33)$ & 8 & $(3,1)$ & 2 & 3 & YES & YES & YES & $1.00$ & $(2,2)$ & -- & 2523\\
$(a;3,0,0;7)$ & 7 & $(7,2)$ & 4 & 7 & YES & YES & YES & $0.88$ & $(2,2)$ & -- & 2524\\
$(a;3,0,1;31)$ & 8 & $(2,1)$ & 1 & 1 & YES & YES & YES & $1.00$ & $(2,2)$ & -- & 2525\\
$(a;3,0,1;31)$ & 8 & $(3,1)$ & 2 & 1 & YES & YES & YES & $1.00$ & $(2,2)$ & -- & 2526\\
$(a;3,0,1;31)$ & 8 & $(4,1)$ & 3 & 1 & YES & YES & YES & $1.00$ & $(2,2)$ & -- & 2527\\
$(a;3,0,1;31)$ & 8 & $(5,1)$ & 4 & 1 & YES & YES & YES & $1.00$ & $(2,2)$ & -- & 2528\\
$(a;3,0,1;31)$ & 8 & $(11,2)$ & 6 & 1 & YES & YES & YES & $1.11$ & $(2,2)$ & -- & 2529\\
$(a;3,1,0;31)$ & 8 & $(2,1)$ & 1 & 1 & YES & YES & YES & $0.88$ & $(4,1)$ & -- & 2530\\
$(a;3,1,0;31)$ & 8 & $(3,1)$ & 2 & 1 & YES & YES & YES & $0.88$ & $(4,1)$ & -- & 2531\\
$(a;3,1,0;31)$ & 8 & $(4,1)$ & 3 & 1 & YES & YES & YES & $1.00$ & $(2,2)$ & -- & 2532\\
$(a;4,0,0;25)$ & 8 & $(3,1)$ & 2 & 1 & YES & YES & YES & $1.00$ & $(2,2)$ & -- & 2533\\
$(a;4,0,0;25)$ & 8 & $(5,1)$ & 4 & 5 & YES & YES & YES & $0.88$ & $(2,2)$ & -- & 2534\\
$(a;4,0,0;25)$ & 8 & $(5,2)$ & 3 & 5 & YES & YES & YES & $1.25$ & $(2,2)$ & -- & 2535\\
$(a;4,0,2;49)$ & 10 & $(7,1)$ & 6 & 7 & YES & YES & YES & $1.12$ & $(2,2)$ & -- & 2536\\
$(b;0,0,0;14)$ & 5 & $(9,4)$ & 5 & 1 & YES & YES & YES & $0.88$ & $(2,2)$ & -- & 2537\\
$(b;0,0,2;26)$ & 7 & $(5,2)$ & 3 & 1 & YES & YES & YES & $1.22$ & $(2,2)$ & -- & 2538\\
$(b;0,0,3;32)$ & 8 & $(2,1)$ & 1 & 2 & YES & YES & YES & $0.88$ & $(2,2)$ & -- & 2539\\
$(b;0,0,3;32)$ & 8 & $(5,1)$ & 4 & 1 & YES & YES & YES & $0.88$ & $(2,2)$ & -- & 2540\\
$(b;0,1,0;19)$ & 6 & $(7,3)$ & 4 & 1 & YES & YES & YES & $1.00$ & $(2,2)$ & -- & 2541\\
$(b;0,1,1;27)$ & 7 & $(5,2)$ & 3 & 1 & YES & YES & YES & $1.11$ & $(2,2)$ & -- & 2542\\
$(b;0,1,1;27)$ & 7 & $(7,2)$ & 4 & 1 & YES & YES & YES & $1.11$ & $(2,2)$ & -- & 2543\\
$(b;0,1,2;5)$ & 8 & $(3,1)$ & 2 & 1 & YES & YES & YES & $1.00$ & $(2,2)$ & -- & 2544\\
$(b;0,2,0;8)$ & 7 & $(3,1)$ & 2 & 1 & YES & YES & NO(2) & $0.78$ & $(4,1)$ & -- & 2545\\
$(b;0,2,0;8)$ & 7 & $(4,1)$ & 3 & 4 & YES & YES & YES & $1.11$ & $(2,2)$ & -- & 2546\\
$(b;0,2,0;8)$ & 7 & $(5,2)$ & 3 & 1 & YES & YES & YES & $1.11$ & $(2,2)$ & -- & 2547\\
$(b;0,2,0;8)$ & 7 & $(7,2)$ & 4 & 1 & YES & YES & YES & $1.11$ & $(2,2)$ & -- & 2548\\
$(b;0,2,1;34)$ & 8 & $(3,1)$ & 2 & 1 & YES & YES & YES & $1.11$ & $(2,2)$ & -- & 2549\\
$(b;0,2,1;34)$ & 8 & $(4,1)$ & 3 & 2 & YES & YES & YES & $1.00$ & $(2,2)$ & -- & 2550\\
$(b;0,3,0;29)$ & 8 & $(2,1)$ & 1 & 1 & YES & YES & YES & $1.11$ & $(2,2)$ & -- & 2551\\
$(b;0,3,0;29)$ & 8 & $(3,1)$ & 2 & 1 & YES & YES & YES & $1.00$ & $(2,2)$ & -- & 2552\\
$(b;0,3,0;29)$ & 8 & $(5,1)$ & 4 & 1 & YES & YES & YES & $0.88$ & $(2,2)$ & -- & 2553\\
$(b;0,3,0;29)$ & 8 & $(6,1)$ & 5 & 1 & YES & YES & YES & $1.11$ & $(2,2)$ & -- & 2554\\
$(b;1,0,0;5)$ & 6 & $(7,3)$ & 4 & 1 & YES & YES & YES & $0.88$ & $(2,2)$ & -- & 2555\\
$(b;1,0,1;29)$ & 7 & $(5,2)$ & 3 & 1 & YES & YES & YES & $1.22$ & $(2,2)$ & -- & 2556\\
$(b;1,0,2;19)$ & 8 & $(2,1)$ & 1 & 1 & YES & YES & YES & $1.22$ & $(2,2)$ & -- & 2557\\
$(b;1,0,2;19)$ & 8 & $(3,1)$ & 2 & 1 & YES & YES & YES & $1.22$ & $(2,2)$ & -- & 2558\\
$(b;1,1,0;27)$ & 7 & $(2,1)$ & 1 & 1 & YES & YES & NO(2) & $1.10$ & $(2,2)$ & -- & 2559\\
$(b;1,1,0;27)$ & 7 & $(7,2)$ & 4 & 1 & YES & YES & YES & $1.22$ & $(2,2)$ & -- & 2560\\
$(b;1,1,1;39)$ & 8 & $(2,1)$ & 1 & 1 & YES & YES & YES & $1.33$ & $(2,2)$ & -- & 2561\\
$(b;1,1,1;39)$ & 8 & $(3,1)$ & 2 & 3 & YES & YES & YES & $1.33$ & $(2,2)$ & -- & 2562\\
$(b;1,2,0;17)$ & 8 & $(2,1)$ & 1 & 1 & YES & YES & YES & $1.00$ & $(2,2)$ & -- & 2563\\
$(b;1,2,0;17)$ & 8 & $(5,1)$ & 4 & 1 & YES & YES & YES & $0.88$ & $(2,2)$ & -- & 2564\\
$(b;2,0,0;26)$ & 7 & $(2,1)$ & 1 & 2 & YES & YES & YES & $0.88$ & $(2,2)$ & -- & 2565\\
$(b;2,0,0;26)$ & 7 & $(3,1)$ & 2 & 1 & YES & YES & NO(2) & $1.10$ & $(2,2)$ & -- & 2566\\
$(b;2,0,0;26)$ & 7 & $(5,2)$ & 3 & 1 & YES & YES & YES & $1.12$ & $(2,2)$ & -- & 2567\\
$(b;2,2,0;44)$ & 9 & $(6,1)$ & 5 & 2 & YES & YES & YES & $1.00$ & $(2,2)$ & -- & 2568\\
$(b;3,0,0;16)$ & 8 & $(2,1)$ & 1 & 2 & YES & YES & YES & $1.00$ & $(2,2)$ & -- & 2569\\
$(b;3,0,0;16)$ & 8 & $(3,1)$ & 2 & 1 & YES & YES & YES & $1.00$ & $(2,2)$ & -- & 2570\\
$(b;3,0,0;16)$ & 8 & $(4,1)$ & 3 & 4 & YES & YES & YES & $1.00$ & $(2,2)$ & -- & 2571\\
$(b;3,0,0;16)$ & 8 & $(5,1)$ & 4 & 1 & YES & YES & YES & $0.88$ & $(2,2)$ & -- & 2572\\
$(c;0,0,0;4)$ & 4 & $(11,3)$ & 5 & 1 & YES & YES & YES & $1.00$ & $(2,2)$ & -- & 2573\\
$(c;0,0,0;4)$ & 4 & $(16,7)$ & 6 & 4 & YES & YES & YES & $0.88$ & $(2,2)$ & -- & 2574\\
$(c;0,0,0;4)$ & 4 & $(17,5)$ & 6 & 1 & YES & YES & NO(2) & $0.90$ & $(2,2)$ & -- & 2575\\
$(c;0,0,0;4)$ & 4 & $(19,7)$ & 6 & 1 & YES & YES & YES & $1.12$ & $(2,2)$ & -- & 2576\\
$(c;0,0,0;4)$ & 4 & $(20,9)$ & 7 & 4 & YES & YES & YES & $1.12$ & $(2,2)$ & -- & 2577\\
$(c;0,0,0;4)$ & 4 & $(25,9)$ & 7 & 1 & YES & YES & YES & $1.00$ & $(2,2)$ & -- & 2578\\
$(c;0,0,0;4)$ & 4 & $(25,11)$ & 7 & 1 & YES & YES & YES & $1.12$ & $(2,2)$ & -- & 2579\\
$(c;0,0,0;4)$ & 4 & $(27,10)$ & 7 & 1 & YES & YES & YES & $1.44$ & $(2,2)$ & -- & 2580\\
$(c;0,1,0;11)$ & 5 & $(11,4)$ & 5 & 11 & YES & YES & YES & $1.11$ & $(2,2)$ & -- & 2581\\
$(c;0,1,0;11)$ & 5 & $(13,5)$ & 5 & 1 & YES & YES & YES & $1.12$ & $(2,2)$ & -- & 2582\\
$(c;0,1,0;11)$ & 5 & $(17,7)$ & 6 & 1 & YES & YES & YES & $1.33$ & $(2,2)$ & -- & 2583\\
$(c;0,1,0;11)$ & 5 & $(19,7)$ & 6 & 1 & YES & YES & YES & $1.33$ & $(2,2)$ & -- & 2584\\
$(c;0,1,0;11)$ & 5 & $(26,7)$ & 7 & 1 & YES & YES & YES & $1.12$ & $(2,2)$ & -- & 2585\\
$(c;0,1,1;5)$ & 6 & $(13,4)$ & 6 & 1 & YES & YES & NO(2) & $0.89$ & $(4,1)$ & -- & 2586\\
$(c;0,1,1;5)$ & 6 & $(17,5)$ & 6 & 1 & YES & YES & YES & $1.00$ & $(2,2)$ & -- & 2587\\
$(c;0,1,1;5)$ & 6 & $(18,5)$ & 6 & 1 & YES & YES & YES & $1.00$ & $(2,2)$ & -- & 2588\\
$(c;0,2,0;7)$ & 6 & $(5,1)$ & 4 & 1 & YES & YES & YES & $1.00$ & $(2,2)$ & -- & 2589\\
$(c;0,2,0;7)$ & 6 & $(5,2)$ & 3 & 1 & YES & YES & NO(2) & $1.10$ & $(2,2)$ & -- & 2590\\
$(c;0,2,0;7)$ & 6 & $(6,1)$ & 5 & 1 & YES & YES & YES & $1.00$ & $(2,2)$ & -- & 2591\\
$(c;0,2,0;7)$ & 6 & $(7,3)$ & 4 & 7 & YES & YES & NO(2) & $1.10$ & $(2,2)$ & -- & 2592\\
$(c;0,2,0;7)$ & 6 & $(9,2)$ & 5 & 1 & YES & YES & YES & $1.00$ & $(2,2)$ & -- & 2593\\
$(c;0,2,0;7)$ & 6 & $(9,4)$ & 5 & 1 & YES & YES & YES & $0.88$ & $(2,2)$ & -- & 2594\\
$(c;0,2,0;7)$ & 6 & $(11,4)$ & 5 & 1 & YES & YES & YES & $0.88$ & $(2,2)$ & -- & 2595\\
$(c;0,2,0;7)$ & 6 & $(12,5)$ & 5 & 1 & YES & YES & YES & $0.88$ & $(2,2)$ & -- & 2596\\
$(c;0,2,0;7)$ & 6 & $(13,5)$ & 5 & 1 & YES & YES & YES & $1.00$ & $(2,2)$ & -- & 2597\\
$(c;0,2,0;7)$ & 6 & $(14,3)$ & 6 & 7 & YES & YES & YES & $1.11$ & $(2,2)$ & -- & 2598\\
$(c;0,2,0;7)$ & 6 & $(19,4)$ & 7 & 1 & YES & YES & YES & $1.11$ & $(2,2)$ & -- & 2599\\
$(c;0,2,0;7)$ & 6 & $(23,5)$ & 7 & 1 & YES & YES & YES & $0.88$ & $(2,2)$ & -- & 2600\\
$(c;0,2,1;19)$ & 7 & $(5,2)$ & 3 & 1 & YES & YES & YES & $1.00$ & $(2,2)$ & -- & 2601\\
$(c;0,2,1;19)$ & 7 & $(7,2)$ & 4 & 1 & YES & YES & YES & $0.75$ & $(4,1)$ & -- & 2602\\
$(c;0,2,1;19)$ & 7 & $(9,2)$ & 5 & 1 & YES & YES & NO(2) & $1.00$ & $(2,2)$ & -- & 2603\\
$(c;0,2,1;19)$ & 7 & $(10,3)$ & 5 & 1 & YES & YES & YES & $1.22$ & $(2,2)$ & -- & 2604\\
$(c;0,2,1;19)$ & 7 & $(11,3)$ & 5 & 1 & YES & YES & YES & $0.75$ & $(4,1)$ & -- & 2605\\
$(c;0,2,1;19)$ & 7 & $(14,3)$ & 6 & 1 & YES & YES & YES & $1.00$ & $(2,2)$ & -- & 2606\\
$(c;0,2,2;6)$ & 8 & $(2,1)$ & 1 & 2 & YES & YES & NO(2) & $1.10$ & $(2,2)$ & -- & 2607\\
$(c;0,2,2;6)$ & 8 & $(3,1)$ & 2 & 3 & YES & YES & YES & $0.88$ & $(2,2)$ & -- & 2608\\
$(c;0,2,2;6)$ & 8 & $(5,1)$ & 4 & 1 & YES & YES & YES & $0.88$ & $(2,2)$ & -- & 2609\\
$(c;0,2,2;6)$ & 8 & $(7,2)$ & 4 & 1 & YES & YES & YES & $0.88$ & $(2,2)$ & -- & 2610\\
$(c;0,2,2;6)$ & 8 & $(9,2)$ & 5 & 3 & YES & YES & YES & $0.88$ & $(2,2)$ & -- & 2611\\
$(c;0,3,0;17)$ & 7 & $(7,3)$ & 4 & 1 & YES & YES & YES & $1.00$ & $(2,2)$ & -- & 2612\\
$(c;0,3,0;17)$ & 7 & $(8,3)$ & 4 & 1 & YES & YES & YES & $0.88$ & $(2,2)$ & -- & 2613\\
$(c;0,3,0;17)$ & 7 & $(10,3)$ & 5 & 1 & YES & YES & YES & $1.00$ & $(2,2)$ & -- & 2614\\
$(c;0,3,0;17)$ & 7 & $(17,3)$ & 7 & 17 & YES & YES & YES & $1.00$ & $(2,2)$ & -- & 2615\\
$(c;0,3,1;23)$ & 8 & $(3,1)$ & 2 & 1 & YES & YES & YES & $0.88$ & $(4,1)$ & -- & 2616\\
$(c;0,3,1;23)$ & 8 & $(4,1)$ & 3 & 1 & YES & YES & YES & $0.88$ & $(4,1)$ & -- & 2617\\
$(c;0,3,2;29)$ & 9 & $(2,1)$ & 1 & 1 & YES & YES & YES & $0.88$ & $(2,2)$ & -- & 2618\\
$(c;0,3,2;29)$ & 9 & $(3,1)$ & 2 & 1 & YES & YES & YES & $0.88$ & $(2,2)$ & -- & 2619\\
$(c;0,3,2;29)$ & 9 & $(4,1)$ & 3 & 1 & YES & YES & YES & $0.88$ & $(2,2)$ & -- & 2620\\
$(c;0,3,2;29)$ & 9 & $(5,1)$ & 4 & 1 & YES & YES & YES & $0.88$ & $(2,2)$ & -- & 2621\\
$(c;0,3,2;29)$ & 9 & $(6,1)$ & 5 & 1 & YES & YES & NO(2) & $0.89$ & $(4,1)$ & -- & 2622\\
$(c;0,4,0;10)$ & 8 & $(4,1)$ & 3 & 2 & YES & YES & YES & $1.00$ & $(2,2)$ & -- & 2623\\
$(c;0,4,0;10)$ & 8 & $(5,1)$ & 4 & 5 & YES & YES & YES & $1.00$ & $(2,2)$ & -- & 2624\\
$(c;0,4,0;10)$ & 8 & $(7,2)$ & 4 & 1 & YES & YES & YES & $1.00$ & $(2,2)$ & -- & 2625\\
$(c;0,4,0;10)$ & 8 & $(9,2)$ & 5 & 1 & YES & YES & YES & $1.00$ & $(2,2)$ & -- & 2626\\
$(c;0,4,0;10)$ & 8 & $(11,2)$ & 6 & 1 & YES & YES & YES & $1.12$ & $(2,2)$ & -- & 2627\\
$(c;0,4,1;9)$ & 9 & $(4,1)$ & 3 & 1 & YES & YES & YES & $0.88$ & $(4,1)$ & -- & 2628\\
$(d;0,0,0;5)$ & 5 & $(8,3)$ & 4 & 1 & YES & YES & YES & $0.88$ & $(2,2)$ & -- & 2629\\
$(d;0,0,0;5)$ & 5 & $(13,5)$ & 5 & 1 & YES & YES & YES & $1.12$ & $(2,2)$ & -- & 2630\\
$(d;0,0,0;5)$ & 5 & $(17,7)$ & 6 & 1 & YES & YES & YES & $0.88$ & $(2,2)$ & -- & 2631\\
$(d;0,0,1;14)$ & 6 & $(9,4)$ & 5 & 1 & YES & YES & YES & $1.00$ & $(2,2)$ & -- & 2632\\
$(d;0,0,2;9)$ & 7 & $(5,2)$ & 3 & 1 & YES & YES & YES & $1.11$ & $(2,2)$ & -- & 2633\\
$(d;0,0,2;9)$ & 7 & $(7,3)$ & 4 & 1 & YES & YES & YES & $0.88$ & $(2,2)$ & -- & 2634\\
$(d;0,0,2;9)$ & 7 & $(9,2)$ & 5 & 9 & YES & YES & YES & $1.00$ & $(2,2)$ & -- & 2635\\
$(d;0,0,2;9)$ & 7 & $(10,3)$ & 5 & 1 & YES & YES & YES & $1.11$ & $(2,2)$ & -- & 2636\\
$(d;0,0,3;22)$ & 8 & $(7,2)$ & 4 & 1 & YES & YES & YES & $0.75$ & $(4,1)$ & -- & 2637\\
$(d;0,0,4;13)$ & 9 & $(3,1)$ & 2 & 1 & YES & YES & YES & $0.88$ & $(4,1)$ & -- & 2638\\
$(d;0,1,0;6)$ & 6 & $(5,1)$ & 4 & 1 & YES & YES & YES & $1.00$ & $(2,2)$ & -- & 2639\\
$(d;0,1,0;6)$ & 6 & $(5,2)$ & 3 & 1 & YES & YES & NO(2) & $1.10$ & $(2,2)$ & -- & 2640\\
$(d;0,1,0;6)$ & 6 & $(6,1)$ & 5 & 6 & YES & YES & YES & $1.00$ & $(2,2)$ & -- & 2641\\
$(d;0,1,0;6)$ & 6 & $(7,2)$ & 4 & 1 & YES & YES & YES & $1.00$ & $(2,2)$ & -- & 2642\\
$(d;0,1,0;6)$ & 6 & $(7,3)$ & 4 & 1 & YES & YES & YES & $0.88$ & $(2,2)$ & -- & 2643\\
$(d;0,1,0;6)$ & 6 & $(9,2)$ & 5 & 3 & YES & YES & NO(2) & $1.10$ & $(2,2)$ & -- & 2644\\
$(d;0,1,0;6)$ & 6 & $(9,4)$ & 5 & 3 & YES & YES & YES & $1.00$ & $(2,2)$ & -- & 2645\\
$(d;0,1,0;6)$ & 6 & $(13,5)$ & 5 & 1 & YES & YES & YES & $1.00$ & $(2,2)$ & -- & 2646\\
$(d;0,1,0;6)$ & 6 & $(19,4)$ & 7 & 1 & YES & YES & YES & $1.00$ & $(2,2)$ & -- & 2647\\
$(d;0,1,1;17)$ & 7 & $(7,2)$ & 4 & 1 & YES & YES & YES & $0.75$ & $(4,1)$ & -- & 2648\\
$(d;0,1,1;17)$ & 7 & $(7,3)$ & 4 & 1 & YES & YES & YES & $1.00$ & $(2,2)$ & -- & 2649\\
$(d;0,1,1;17)$ & 7 & $(10,3)$ & 5 & 1 & YES & YES & YES & $1.11$ & $(2,2)$ & -- & 2650\\
$(d;0,1,1;17)$ & 7 & $(14,3)$ & 6 & 1 & YES & YES & YES & $1.00$ & $(2,2)$ & -- & 2651\\
$(d;0,1,2;11)$ & 8 & $(2,1)$ & 1 & 1 & YES & YES & YES & $0.88$ & $(2,2)$ & -- & 2652\\
$(d;0,1,2;11)$ & 8 & $(4,1)$ & 3 & 1 & YES & YES & YES & $1.11$ & $(2,2)$ & -- & 2653\\
$(d;0,1,2;11)$ & 8 & $(5,1)$ & 4 & 1 & YES & YES & YES & $0.88$ & $(2,2)$ & -- & 2654\\
$(d;0,1,2;11)$ & 8 & $(5,2)$ & 3 & 1 & YES & YES & YES & $1.11$ & $(2,2)$ & -- & 2655\\
$(d;0,1,3;27)$ & 9 & $(2,1)$ & 1 & 1 & YES & YES & YES & $1.00$ & $(2,2)$ & -- & 2656\\
$(d;0,1,3;27)$ & 9 & $(3,1)$ & 2 & 3 & YES & YES & YES & $1.00$ & $(2,2)$ & -- & 2657\\
$(d;0,1,3;27)$ & 9 & $(5,1)$ & 4 & 1 & YES & YES & YES & $1.00$ & $(2,2)$ & -- & 2658\\
$(d;0,2,2;13)$ & 9 & $(2,1)$ & 1 & 1 & YES & YES & YES & $0.88$ & $(2,2)$ & -- & 2659\\
$(d;0,2,2;13)$ & 9 & $(3,1)$ & 2 & 1 & YES & YES & YES & $0.88$ & $(2,2)$ & -- & 2660\\
$(d;0,2,2;13)$ & 9 & $(4,1)$ & 3 & 1 & YES & YES & YES & $0.88$ & $(2,2)$ & -- & 2661\\
$(d;0,2,2;13)$ & 9 & $(5,1)$ & 4 & 1 & YES & YES & YES & $0.88$ & $(2,2)$ & -- & 2662\\
$(d;0,3,0;8)$ & 8 & $(4,1)$ & 3 & 4 & YES & YES & YES & $1.12$ & $(2,2)$ & -- & 2663\\
$(d;0,3,0;8)$ & 8 & $(5,1)$ & 4 & 1 & YES & YES & YES & $1.00$ & $(2,2)$ & -- & 2664\\
$(d;0,3,0;8)$ & 8 & $(7,2)$ & 4 & 1 & YES & YES & YES & $1.12$ & $(2,2)$ & -- & 2665\\
$(d;0,3,0;8)$ & 8 & $(11,2)$ & 6 & 1 & YES & YES & YES & $1.12$ & $(2,2)$ & -- & 2666\\
$(d;0,3,1;23)$ & 9 & $(3,1)$ & 2 & 1 & YES & YES & YES & $0.88$ & $(4,1)$ & -- & 2667\\
$(e;0,1,0;5)$ & 6 & $(8,3)$ & 4 & 1 & YES & YES & YES & $0.88$ & $(2,2)$ & -- & 2668\\
$(e;0,1,0;5)$ & 6 & $(10,3)$ & 5 & 5 & YES & YES & YES & $0.88$ & $(2,2)$ & -- & 2669\\
$(e;0,2,0;6)$ & 7 & $(3,1)$ & 2 & 3 & YES & YES & NO(2) & $1.10$ & $(2,2)$ & -- & 2670\\
$(e;0,2,0;6)$ & 7 & $(4,1)$ & 3 & 2 & YES & YES & YES & $1.11$ & $(2,2)$ & -- & 2671\\
$(e;0,2,0;6)$ & 7 & $(5,2)$ & 3 & 1 & YES & YES & YES & $1.00$ & $(2,2)$ & -- & 2672\\
$(e;0,2,0;6)$ & 7 & $(7,2)$ & 4 & 1 & YES & YES & YES & $1.00$ & $(2,2)$ & -- & 2673\\
$(e;0,2,0;6)$ & 7 & $(9,2)$ & 5 & 3 & YES & YES & YES & $1.11$ & $(2,2)$ & -- & 2674\\
$(e;0,3,0;7)$ & 8 & $(2,1)$ & 1 & 1 & YES & YES & NO(2) & $1.00$ & $(4,1)$ & -- & 2675\\
$(e;0,3,0;7)$ & 8 & $(3,1)$ & 2 & 1 & YES & YES & YES & $1.00$ & $(2,2)$ & -- & 2676\\
$(e;0,3,0;7)$ & 8 & $(5,1)$ & 4 & 1 & YES & YES & YES & $0.88$ & $(2,2)$ & -- & 2677\\
$(e;0,3,0;7)$ & 8 & $(6,1)$ & 5 & 1 & YES & YES & NO(2) & $0.89$ & $(4,1)$ & -- & 2678\\
$(e;1,2,0;28)$ & 8 & $(2,1)$ & 1 & 2 & YES & YES & YES & $0.88$ & $(2,2)$ & -- & 2679\\
$(e;1,2,0;28)$ & 8 & $(3,1)$ & 2 & 1 & YES & YES & YES & $1.11$ & $(2,2)$ & -- & 2680\\
$(e;2,0,0;24)$ & 7 & $(2,1)$ & 1 & 2 & YES & YES & NO(2) & $1.00$ & $(2,2)$ & -- & 2681\\
$(e;2,0,0;24)$ & 7 & $(5,2)$ & 3 & 1 & YES & YES & YES & $1.22$ & $(2,2)$ & -- & 2682\\
$(e;3,0,0;10)$ & 8 & $(2,1)$ & 1 & 2 & YES & YES & YES & $0.88$ & $(2,2)$ & -- & 2683\\
$(e;3,0,0;10)$ & 8 & $(3,1)$ & 2 & 1 & YES & YES & YES & $1.12$ & $(2,2)$ & -- & 2684\\
$(f;0,0,0;6)$ & 4 & $(12,5)$ & 5 & 6 & YES & YES & YES & $0.88$ & $(2,2)$ & -- & 2685\\
$(f;0,0,0;6)$ & 4 & $(16,5)$ & 7 & 2 & YES & YES & YES & $0.88$ & $(2,2)$ & -- & 2686\\
$(f;0,0,0;6)$ & 4 & $(17,7)$ & 6 & 1 & YES & YES & NO(2) & $1.10$ & $(2,2)$ & -- & 2687\\
$(f;0,0,0;6)$ & 4 & $(18,7)$ & 6 & 6 & YES & YES & YES & $0.88$ & $(2,2)$ & -- & 2688\\
$(f;0,0,0;6)$ & 4 & $(19,6)$ & 8 & 1 & YES & YES & YES & $0.88$ & $(2,2)$ & -- & 2689\\
$(f;0,0,0;6)$ & 4 & $(19,7)$ & 6 & 1 & YES & YES & YES & $0.88$ & $(2,2)$ & -- & 2690\\
$(f;0,0,0;6)$ & 4 & $(19,8)$ & 6 & 1 & YES & YES & YES & $0.88$ & $(2,2)$ & -- & 2691\\
$(f;0,0,0;6)$ & 4 & $(21,8)$ & 6 & 3 & YES & YES & YES & $1.00$ & $(2,2)$ & -- & 2692\\
$(f;0,0,0;6)$ & 4 & $(23,7)$ & 7 & 1 & YES & YES & YES & $1.11$ & $(2,2)$ & -- & 2693\\
$(f;0,0,0;6)$ & 4 & $(23,9)$ & 7 & 1 & YES & YES & YES & $1.11$ & $(2,2)$ & -- & 2694\\
$(f;0,0,0;6)$ & 4 & $(25,11)$ & 7 & 1 & YES & YES & YES & $0.88$ & $(2,2)$ & -- & 2695\\
$(f;0,0,0;6)$ & 4 & $(26,11)$ & 7 & 2 & YES & YES & YES & $1.00$ & $(2,2)$ & -- & 2696\\
$(f;0,0,0;6)$ & 4 & $(27,10)$ & 7 & 3 & YES & YES & NO(2) & $1.10$ & $(2,2)$ & -- & 2697\\
$(f;0,0,0;6)$ & 4 & $(29,9)$ & 8 & 1 & YES & YES & YES & $1.11$ & $(2,2)$ & -- & 2698\\
$(f;0,0,0;6)$ & 4 & $(30,11)$ & 7 & 6 & YES & YES & YES & $1.12$ & $(2,2)$ & -- & 2699\\
$(f;0,0,0;6)$ & 4 & $(31,12)$ & 7 & 1 & YES & YES & YES & $1.11$ & $(2,2)$ & -- & 2700\\
$(f;0,0,0;6)$ & 4 & $(34,13)$ & 7 & 2 & YES & YES & YES & $1.22$ & $(2,2)$ & -- & 2701\\
$(f;0,0,0;6)$ & 4 & $(37,8)$ & 8 & 1 & YES & YES & YES & $1.00$ & $(2,2)$ & -- & 2702\\
$(f;0,0,0;6)$ & 4 & $(41,12)$ & 8 & 1 & YES & YES & YES & $1.33$ & $(2,2)$ & -- & 2703\\
$(f;0,0,0;6)$ & 4 & $(43,12)$ & 8 & 1 & YES & YES & YES & $1.33$ & $(2,2)$ & -- & 2704\\
$(f;0,1,0;7)$ & 5 & $(12,5)$ & 5 & 1 & YES & YES & YES & $0.88$ & $(2,2)$ & -- & 2705\\
$(f;0,1,0;7)$ & 5 & $(13,5)$ & 5 & 1 & YES & YES & YES & $0.88$ & $(4,1)$ & -- & 2706\\
$(f;0,1,0;7)$ & 5 & $(17,5)$ & 6 & 1 & YES & YES & YES & $0.88$ & $(4,1)$ & -- & 2707\\
$(f;0,1,0;7)$ & 5 & $(18,5)$ & 6 & 1 & YES & YES & YES & $1.00$ & $(2,2)$ & -- & 2708\\
$(f;0,1,0;7)$ & 5 & $(22,5)$ & 7 & 1 & YES & YES & YES & $1.12$ & $(2,2)$ & -- & 2709\\
$(f;0,1,0;7)$ & 5 & $(23,5)$ & 7 & 1 & YES & YES & YES & $0.88$ & $(4,1)$ & -- & 2710\\
$(f;0,1,0;7)$ & 5 & $(28,5)$ & 8 & 7 & YES & YES & NO(2) & $1.10$ & $(2,2)$ & -- & 2711\\
$(f;0,2,0;8)$ & 6 & $(10,3)$ & 5 & 2 & YES & YES & YES & $1.12$ & $(2,2)$ & -- & 2712\\
$(f;0,2,0;8)$ & 6 & $(17,5)$ & 6 & 1 & YES & YES & YES & $1.12$ & $(2,2)$ & -- & 2713\\
$(f;0,2,0;8)$ & 6 & $(18,5)$ & 6 & 2 & YES & YES & YES & $1.12$ & $(2,2)$ & -- & 2714\\
$(f;0,2,0;8)$ & 6 & $(22,5)$ & 7 & 2 & YES & YES & YES & $1.12$ & $(2,2)$ & -- & 2715\\
$(f;0,2,0;8)$ & 6 & $(23,5)$ & 7 & 1 & YES & YES & YES & $1.12$ & $(2,2)$ & -- & 2716\\
$(g;0,0,1;26)$ & 7 & $(13,3)$ & 6 & 13 & YES & YES & YES & $1.22$ & $(2,2)$ & -- & 2717\\
$(g;0,1,0;24)$ & 7 & $(2,1)$ & 1 & 2 & YES & YES & YES & $0.88$ & $(2,2)$ & -- & 2718\\
$(i;0,0,0;9)$ & 5 & $(8,3)$ & 4 & 1 & YES & YES & YES & $0.88$ & $(2,2)$ & -- & 2719\\
$(i;0,0,0;9)$ & 5 & $(9,4)$ & 5 & 9 & YES & YES & YES & $0.88$ & $(2,2)$ & -- & 2720\\
$(i;0,0,0;9)$ & 5 & $(10,3)$ & 5 & 1 & YES & YES & NO(2) & $0.89$ & $(4,1)$ & -- & 2721\\
$(i;0,0,0;9)$ & 5 & $(12,5)$ & 5 & 3 & YES & YES & YES & $0.88$ & $(2,2)$ & -- & 2722\\
$(i;0,0,0;9)$ & 5 & $(13,5)$ & 5 & 1 & YES & YES & YES & $1.00$ & $(2,2)$ & -- & 2723\\
$(i;0,0,0;9)$ & 5 & $(19,4)$ & 7 & 1 & YES & YES & NO(2) & $1.00$ & $(4,1)$ & -- & 2724\\
$(i;0,0,0;9)$ & 5 & $(22,5)$ & 7 & 1 & YES & YES & YES & $1.00$ & $(2,2)$ & -- & 2725\\
$(i;0,0,0;9)$ & 5 & $(23,5)$ & 7 & 1 & YES & YES & YES & $1.00$ & $(2,2)$ & -- & 2726\\
$(i;0,1,0;12)$ & 6 & $(5,1)$ & 4 & 1 & YES & YES & YES & $0.88$ & $(4,1)$ & -- & 2727\\
$(i;0,1,0;12)$ & 6 & $(5,2)$ & 3 & 1 & YES & YES & NO(2) & $1.10$ & $(2,2)$ & -- & 2728\\
$(i;0,1,0;12)$ & 6 & $(7,2)$ & 4 & 1 & YES & YES & YES & $1.00$ & $(2,2)$ & -- & 2729\\
$(i;0,1,0;12)$ & 6 & $(7,3)$ & 4 & 1 & YES & YES & YES & $1.00$ & $(2,2)$ & -- & 2730\\
$(i;0,1,0;12)$ & 6 & $(8,3)$ & 4 & 4 & YES & YES & YES & $0.88$ & $(4,1)$ & -- & 2731\\
$(i;0,1,0;12)$ & 6 & $(9,2)$ & 5 & 3 & YES & YES & NO(2) & $1.10$ & $(2,2)$ & -- & 2732\\
$(i;0,1,0;12)$ & 6 & $(13,3)$ & 6 & 1 & YES & YES & YES & $0.88$ & $(2,2)$ & -- & 2733\\
$(i;0,1,0;12)$ & 6 & $(14,3)$ & 6 & 2 & YES & YES & YES & $1.00$ & $(2,2)$ & -- & 2734\\
$(i;0,2,0;15)$ & 7 & $(5,2)$ & 3 & 5 & YES & YES & YES & $0.88$ & $(2,2)$ & -- & 2735\\
$(i;0,3,0;18)$ & 8 & $(2,1)$ & 1 & 2 & YES & YES & YES & $0.75$ & $(4,1)$ & -- & 2736\\
$(i;0,3,0;18)$ & 8 & $(3,1)$ & 2 & 3 & YES & YES & YES & $0.75$ & $(4,1)$ & -- & 2737\\
$(i;0,3,0;18)$ & 8 & $(4,1)$ & 3 & 2 & YES & YES & YES & $0.75$ & $(4,1)$ & -- & 2738\\
$(i;0,3,0;18)$ & 8 & $(5,1)$ & 4 & 1 & YES & YES & NO(2) & $0.62$ & $(6,0)$ & -- & 2739\\
$(j;0,0,0;8)$ & 5 & $(9,4)$ & 5 & 1 & YES & YES & YES & $0.88$ & $(2,2)$ & -- & 2740\\
$(j;0,0,0;8)$ & 5 & $(11,4)$ & 5 & 1 & YES & YES & NO(2) & $1.10$ & $(2,2)$ & -- & 2741\\
$(j;0,0,0;8)$ & 5 & $(11,5)$ & 6 & 1 & YES & YES & YES & $0.88$ & $(2,2)$ & -- & 2742\\
$(j;0,0,0;8)$ & 5 & $(12,5)$ & 5 & 4 & YES & YES & YES & $0.88$ & $(2,2)$ & -- & 2743\\
$(j;0,0,0;8)$ & 5 & $(16,7)$ & 6 & 8 & YES & YES & YES & $0.88$ & $(2,2)$ & -- & 2744\\
$(j;0,0,0;8)$ & 5 & $(17,7)$ & 6 & 1 & YES & YES & YES & $1.12$ & $(2,2)$ & -- & 2745\\
$(j;0,0,0;8)$ & 5 & $(19,8)$ & 6 & 1 & YES & YES & YES & $0.88$ & $(2,2)$ & -- & 2746\\
$(j;0,0,0;8)$ & 5 & $(21,8)$ & 6 & 1 & YES & YES & YES & $1.22$ & $(2,2)$ & -- & 2747\\
$(j;0,1,0;10)$ & 6 & $(7,3)$ & 4 & 1 & YES & YES & YES & $0.88$ & $(2,2)$ & -- & 2748\\
$(j;0,1,0;10)$ & 6 & $(8,3)$ & 4 & 2 & YES & YES & NO(2) & $1.10$ & $(2,2)$ & -- & 2749\\
$(j;0,1,0;10)$ & 6 & $(10,3)$ & 5 & 10 & YES & YES & YES & $1.00$ & $(2,2)$ & -- & 2750\\
$(j;0,1,0;10)$ & 6 & $(11,3)$ & 5 & 1 & YES & YES & YES & $0.88$ & $(4,1)$ & -- & 2751\\
$(j;0,1,0;10)$ & 6 & $(11,4)$ & 5 & 1 & YES & YES & YES & $0.88$ & $(2,2)$ & -- & 2752\\
$(j;0,1,0;10)$ & 6 & $(11,5)$ & 6 & 1 & YES & YES & YES & $1.12$ & $(2,2)$ & -- & 2753\\
$(j;0,1,0;10)$ & 6 & $(12,5)$ & 5 & 2 & YES & YES & YES & $0.88$ & $(2,2)$ & -- & 2754\\
$(j;0,1,0;10)$ & 6 & $(13,5)$ & 5 & 1 & YES & YES & YES & $1.00$ & $(2,2)$ & -- & 2755\\
$(j;0,2,0;12)$ & 7 & $(7,2)$ & 4 & 1 & YES & YES & YES & $0.88$ & $(2,2)$ & -- & 2756
\end{longtable}
\subsection{2 chains, $K^2 = 3$}
\begin{longtable}{|c|c|c|c|c|c|c|c|c|c|c|c|}
\hline
\multicolumn{12}{|c|}{2 chains, $K^2 = 3$}\\
\hline
$(n,a)$ & Len & $(n,a)$ & Len & GCD & Nef & $\mathbb Q$-ef & Obs 0 & $\overline c_1^2 / \overline c_2$ & $(P,K)$ & WH & Index\\
\hline
\endfirsthead

\hline
$(n,a)$ & Len & $(n,a)$ & Len & GCD & Nef & $\mathbb Q$-ef & Obs 0 & $\overline c_1^2 / \overline c_2$ & $(P,K)$ & WH & Index\\
\hline
\endhead
\hline
\endfoot

$(19,8)$ & 6 & $(18,5)$ & 6 & 1 & YES & YES & YES & $1.29$ & $(2,3)$ & -- & 2757\\
$(20,9)$ & 7 & $(19,7)$ & 6 & 1 & YES & YES & YES & $1.29$ & $(2,3)$ & -- & 2758\\
$(21,8)$ & 6 & $(20,9)$ & 7 & 1 & YES & YES & YES & $1.29$ & $(2,3)$ & -- & 2759\\
$(22,5)$ & 7 & $(15,4)$ & 6 & 1 & YES & YES & NO(2) & $1.25$ & $(4,2)$ & NO & 2760\\
$(22,5)$ & 7 & $(15,4)$ & 6 & 1 & YES & YES & NO(2) & $1.25$ & $(4,2)$ & -- & 2761\\
$(22,9)$ & 7 & $(19,6)$ & 8 & 1 & YES & YES & YES & $1.43$ & $(2,3)$ & -- & 2762\\
$(23,10)$ & 7 & $(11,4)$ & 5 & 1 & YES & YES & NO(2) & $1.14$ & $(6,1)$ & -- & 2763\\
$(23,9)$ & 7 & $(13,4)$ & 6 & 1 & YES & YES & YES & $1.57$ & $(2,3)$ & -- & 2764\\
$(23,9)$ & 7 & $(16,5)$ & 7 & 1 & YES & YES & NO(2) & $1.50$ & $(4,2)$ & -- & 2765\\
$(23,9)$ & 7 & $(16,5)$ & 7 & 1 & YES & YES & YES & $1.57$ & $(2,3)$ & NO & 2766\\
$(23,10)$ & 7 & $(17,7)$ & 6 & 1 & YES & YES & NO(2) & $1.56$ & $(2,3)$ & -- & 2767\\
$(23,7)$ & 7 & $(22,9)$ & 7 & 1 & YES & YES & YES & $1.57$ & $(2,3)$ & NO & 2768\\
$(23,7)$ & 7 & $(22,9)$ & 7 & 1 & YES & YES & YES & $1.57$ & $(2,3)$ & -- & 2769\\
$(23,9)$ & 7 & $(23,5)$ & 7 & 23 & YES & YES & YES & $1.29$ & $(4,2)$ & NO & 2770\\
$(23,9)$ & 7 & $(23,5)$ & 7 & 23 & YES & YES & YES & $1.29$ & $(4,2)$ & -- & 2771\\
$(23,9)$ & 7 & $(23,5)$ & 7 & 23 & YES & YES & NO(2) & $1.25$ & $(4,2)$ & NO & 2772\\
$(23,9)$ & 7 & $(23,7)$ & 7 & 23 & YES & YES & YES & $1.57$ & $(2,3)$ & -- & 2773\\
$(23,9)$ & 7 & $(23,7)$ & 7 & 23 & YES & YES & YES & $1.57$ & $(2,3)$ & NO & 2774\\
$(23,9)$ & 7 & $(23,9)$ & 7 & 23 & YES & YES & YES & $1.57$ & $(2,3)$ & -- & 2775\\
$(24,7)$ & 7 & $(11,4)$ & 5 & 1 & YES & YES & NO(2) & $1.14$ & $(6,1)$ & NO & 2776\\
$(24,7)$ & 7 & $(17,7)$ & 6 & 1 & YES & YES & YES & $1.62$ & $(2,3)$ & NO & 2777\\
$(24,7)$ & 7 & $(19,6)$ & 8 & 1 & YES & YES & YES & $1.62$ & $(2,3)$ & NO & 2778\\
$(24,7)$ & 7 & $(19,6)$ & 8 & 1 & YES & YES & YES & $1.62$ & $(2,3)$ & -- & 2779\\
$(24,5)$ & 8 & $(23,9)$ & 7 & 1 & YES & YES & NO(2) & $1.38$ & $(4,2)$ & -- & 2780\\
$(25,7)$ & 7 & $(16,7)$ & 6 & 1 & YES & YES & YES & $1.43$ & $(2,3)$ & NO & 2781\\
$(25,7)$ & 7 & $(16,7)$ & 6 & 1 & YES & YES & YES & $1.43$ & $(2,3)$ & -- & 2782\\
$(26,7)$ & 7 & $(13,4)$ & 6 & 13 & YES & YES & YES & $1.57$ & $(2,3)$ & NO & 2783\\
$(26,11)$ & 7 & $(13,4)$ & 6 & 13 & YES & YES & NO(2) & $1.56$ & $(2,3)$ & -- & 2784\\
$(26,11)$ & 7 & $(13,4)$ & 6 & 13 & YES & YES & NO(2) & $1.56$ & $(2,3)$ & NO & 2785\\
$(26,11)$ & 7 & $(23,9)$ & 7 & 1 & YES & YES & YES & $1.57$ & $(2,3)$ & -- & 2786\\
$(27,11)$ & 8 & $(12,5)$ & 5 & 3 & YES & YES & NO(2) & $1.38$ & $(4,2)$ & -- & 2787\\
$(27,11)$ & 8 & $(13,3)$ & 6 & 1 & YES & YES & NO(2) & $1.50$ & $(4,2)$ & NO & 2788\\
$(27,11)$ & 8 & $(13,3)$ & 6 & 1 & YES & YES & NO(2) & $1.50$ & $(4,2)$ & -- & 2789\\
$(27,11)$ & 8 & $(13,3)$ & 6 & 1 & YES & YES & NO(2) & $1.50$ & $(4,2)$ & NO & 2790\\
$(27,11)$ & 8 & $(13,4)$ & 6 & 1 & YES & YES & NO(2) & $1.67$ & $(2,3)$ & -- & 2791\\
$(27,11)$ & 8 & $(13,5)$ & 5 & 1 & YES & YES & NO(2) & $1.38$ & $(4,2)$ & -- & 2792\\
$(27,10)$ & 7 & $(16,7)$ & 6 & 1 & YES & YES & YES & $1.62$ & $(2,3)$ & -- & 2793\\
$(27,8)$ & 7 & $(17,6)$ & 7 & 1 & YES & YES & YES & $1.43$ & $(2,3)$ & NO & 2794\\
$(27,8)$ & 7 & $(17,6)$ & 7 & 1 & YES & YES & YES & $1.43$ & $(2,3)$ & -- & 2795\\
$(27,10)$ & 7 & $(17,7)$ & 6 & 1 & YES & YES & YES & $1.62$ & $(2,3)$ & NO & 2796\\
$(27,11)$ & 8 & $(22,5)$ & 7 & 1 & YES & YES & YES & $1.50$ & $(2,3)$ & -- & 2797\\
$(27,5)$ & 8 & $(23,9)$ & 7 & 1 & YES & YES & NO(2) & $1.44$ & $(2,3)$ & -- & 2798\\
$(27,10)$ & 7 & $(23,9)$ & 7 & 1 & YES & YES & YES & $1.43$ & $(2,3)$ & -- & 2799\\
$(27,11)$ & 8 & $(27,8)$ & 7 & 27 & YES & YES & YES & $1.43$ & $(2,3)$ & -- & 2800\\
$(27,11)$ & 8 & $(27,8)$ & 7 & 27 & YES & YES & YES & $1.43$ & $(2,3)$ & NO & 2801\\
$(28,11)$ & 8 & $(10,3)$ & 5 & 2 & YES & YES & NO(2) & $1.50$ & $(4,2)$ & -- & 2802\\
$(28,11)$ & 8 & $(17,7)$ & 6 & 1 & YES & YES & YES & $1.29$ & $(4,2)$ & -- & 2803\\
$(28,11)$ & 8 & $(18,7)$ & 6 & 2 & YES & YES & YES & $1.62$ & $(2,3)$ & -- & 2804\\
$(28,5)$ & 8 & $(23,9)$ & 7 & 1 & YES & YES & YES & $1.29$ & $(2,3)$ & -- & 2805\\
$(28,11)$ & 8 & $(23,4)$ & 8 & 1 & YES & YES & NO(2) & $1.38$ & $(4,2)$ & NO & 2806\\
$(28,11)$ & 8 & $(23,4)$ & 8 & 1 & YES & YES & NO(2) & $1.56$ & $(2,3)$ & -- & 2807\\
$(28,11)$ & 8 & $(23,7)$ & 7 & 1 & YES & YES & YES & $1.57$ & $(2,3)$ & -- & 2808\\
$(28,5)$ & 8 & $(27,11)$ & 8 & 1 & YES & YES & NO(2) & $1.44$ & $(2,3)$ & -- & 2809\\
$(28,11)$ & 8 & $(28,5)$ & 8 & 28 & YES & YES & NO(2) & $1.44$ & $(2,3)$ & -- & 2810\\
$(29,13)$ & 8 & $(11,4)$ & 5 & 1 & YES & YES & YES & $1.29$ & $(4,2)$ & -- & 2811\\
$(29,13)$ & 8 & $(13,5)$ & 5 & 1 & YES & YES & NO(2) & $1.00$ & $(6,1)$ & -- & 2812\\
$(29,8)$ & 7 & $(17,6)$ & 7 & 1 & YES & YES & YES & $1.43$ & $(2,3)$ & NO & 2813\\
$(29,8)$ & 7 & $(17,6)$ & 7 & 1 & YES & YES & YES & $1.43$ & $(2,3)$ & -- & 2814\\
$(29,12)$ & 7 & $(19,8)$ & 6 & 1 & YES & YES & YES & $1.29$ & $(2,3)$ & -- & 2815\\
$(29,9)$ & 8 & $(23,9)$ & 7 & 1 & YES & YES & YES & $1.62$ & $(2,3)$ & -- & 2816\\
$(29,8)$ & 7 & $(28,11)$ & 8 & 1 & YES & YES & YES & $1.43$ & $(2,3)$ & NO & 2817\\
$(30,13)$ & 8 & $(12,5)$ & 5 & 6 & YES & YES & NO(2) & $1.38$ & $(4,2)$ & -- & 2818\\
$(30,13)$ & 8 & $(22,5)$ & 7 & 2 & YES & YES & YES & $1.38$ & $(2,3)$ & -- & 2819\\
$(30,13)$ & 8 & $(23,5)$ & 7 & 1 & YES & YES & YES & $1.50$ & $(2,3)$ & -- & 2820\\
$(30,7)$ & 8 & $(28,11)$ & 8 & 2 & YES & YES & YES & $1.62$ & $(2,3)$ & NO & 2821\\
$(31,14)$ & 8 & $(10,3)$ & 5 & 1 & YES & YES & YES & $1.29$ & $(2,3)$ & -- & 2822\\
$(31,9)$ & 8 & $(12,5)$ & 5 & 1 & YES & YES & NO(2) & $1.38$ & $(4,2)$ & NO & 2823\\
$(31,9)$ & 8 & $(12,5)$ & 5 & 1 & YES & YES & NO(2) & $1.38$ & $(4,2)$ & -- & 2824\\
$(31,9)$ & 8 & $(16,7)$ & 6 & 1 & YES & YES & NO(2) & $1.38$ & $(4,2)$ & NO & 2825\\
$(31,9)$ & 8 & $(16,7)$ & 6 & 1 & YES & YES & NO(2) & $1.38$ & $(4,2)$ & -- & 2826\\
$(31,13)$ & 7 & $(17,7)$ & 6 & 1 & YES & YES & YES & $1.43$ & $(2,3)$ & -- & 2827\\
$(31,14)$ & 8 & $(17,5)$ & 6 & 1 & YES & YES & YES & $1.29$ & $(2,3)$ & -- & 2828\\
$(31,7)$ & 8 & $(18,7)$ & 6 & 1 & YES & YES & NO(2) & $1.25$ & $(4,2)$ & -- & 2829\\
$(31,9)$ & 8 & $(20,9)$ & 7 & 1 & YES & YES & YES & $1.57$ & $(2,3)$ & -- & 2830\\
$(31,12)$ & 7 & $(23,9)$ & 7 & 1 & YES & YES & YES & $1.43$ & $(2,3)$ & -- & 2831\\
$(31,7)$ & 8 & $(27,11)$ & 8 & 1 & YES & YES & YES & $1.43$ & $(2,3)$ & NO & 2832\\
$(32,7)$ & 8 & $(8,3)$ & 4 & 8 & YES & YES & NO(3) & $1.29$ & $(2,3)$ & -- & 2833\\
$(32,9)$ & 8 & $(21,8)$ & 6 & 1 & YES & YES & YES & $1.29$ & $(2,3)$ & NO & 2834\\
$(32,9)$ & 8 & $(29,11)$ & 7 & 1 & YES & YES & YES & $1.57$ & $(4,2)$ & -- & 2835\\
$(33,13)$ & 9 & $(10,3)$ & 5 & 1 & YES & YES & NO(2) & $1.50$ & $(4,2)$ & -- & 2836\\
$(33,14)$ & 8 & $(13,4)$ & 6 & 1 & YES & YES & YES & $1.57$ & $(2,3)$ & -- & 2837\\
$(33,10)$ & 8 & $(16,5)$ & 7 & 1 & YES & YES & YES & $1.62$ & $(2,3)$ & NO & 2838\\
$(33,10)$ & 8 & $(16,5)$ & 7 & 1 & YES & YES & YES & $1.62$ & $(2,3)$ & -- & 2839\\
$(33,14)$ & 8 & $(17,5)$ & 6 & 1 & YES & YES & YES & $1.57$ & $(2,3)$ & NO & 2840\\
$(33,14)$ & 8 & $(17,5)$ & 6 & 1 & YES & YES & YES & $1.57$ & $(2,3)$ & -- & 2841\\
$(33,13)$ & 9 & $(18,5)$ & 6 & 3 & YES & YES & YES & $1.43$ & $(2,3)$ & -- & 2842\\
$(33,14)$ & 8 & $(21,8)$ & 6 & 3 & YES & YES & YES & $1.43$ & $(2,3)$ & -- & 2843\\
$(33,13)$ & 9 & $(23,5)$ & 7 & 1 & YES & YES & YES & $1.57$ & $(2,3)$ & -- & 2844\\
$(33,13)$ & 9 & $(23,5)$ & 7 & 1 & YES & YES & YES & $1.57$ & $(2,3)$ & NO & 2845\\
$(33,14)$ & 8 & $(23,7)$ & 7 & 1 & YES & YES & YES & $1.57$ & $(2,3)$ & -- & 2846\\
$(33,14)$ & 8 & $(27,8)$ & 7 & 3 & YES & YES & YES & $1.57$ & $(2,3)$ & -- & 2847\\
$(33,14)$ & 8 & $(28,5)$ & 8 & 1 & YES & YES & NO(2) & $1.44$ & $(2,3)$ & -- & 2848\\
$(33,7)$ & 8 & $(29,11)$ & 7 & 1 & YES & YES & YES & $1.50$ & $(2,3)$ & -- & 2849\\
$(33,13)$ & 9 & $(29,11)$ & 7 & 1 & YES & YES & NO(2) & $1.50$ & $(4,2)$ & NO & 2850\\
$(34,9)$ & 8 & $(10,3)$ & 5 & 2 & YES & YES & YES & $1.29$ & $(2,3)$ & NO & 2851\\
$(34,9)$ & 8 & $(10,3)$ & 5 & 2 & YES & YES & YES & $1.29$ & $(2,3)$ & -- & 2852\\
$(34,15)$ & 8 & $(13,4)$ & 6 & 1 & YES & YES & YES & $1.75$ & $(2,3)$ & -- & 2853\\
$(34,15)$ & 8 & $(18,7)$ & 6 & 2 & YES & YES & YES & $1.43$ & $(2,3)$ & -- & 2854\\
$(34,13)$ & 7 & $(19,8)$ & 6 & 1 & YES & YES & YES & $1.43$ & $(2,3)$ & -- & 2855\\
$(34,15)$ & 8 & $(19,8)$ & 6 & 1 & YES & YES & YES & $1.43$ & $(2,3)$ & -- & 2856\\
$(34,13)$ & 7 & $(23,10)$ & 7 & 1 & YES & YES & YES & $1.43$ & $(2,3)$ & NO & 2857\\
$(34,15)$ & 8 & $(24,5)$ & 8 & 2 & YES & YES & YES & $1.71$ & $(2,3)$ & -- & 2858\\
$(34,13)$ & 7 & $(33,10)$ & 8 & 1 & YES & YES & YES & $1.86$ & $(2,3)$ & -- & 2859\\
$(35,13)$ & 8 & $(8,3)$ & 4 & 1 & YES & YES & NO(2) & $1.38$ & $(4,2)$ & NO & 2860\\
$(35,8)$ & 8 & $(17,4)$ & 7 & 1 & YES & YES & YES & $1.43$ & $(2,3)$ & NO & 2861\\
$(35,8)$ & 8 & $(17,4)$ & 7 & 1 & YES & YES & YES & $1.43$ & $(2,3)$ & -- & 2862\\
$(35,8)$ & 8 & $(25,11)$ & 7 & 5 & YES & YES & YES & $1.38$ & $(2,3)$ & -- & 2863\\
$(36,13)$ & 8 & $(17,7)$ & 6 & 1 & YES & YES & YES & $1.62$ & $(2,3)$ & NO & 2864\\
$(36,13)$ & 8 & $(17,7)$ & 6 & 1 & YES & YES & YES & $1.62$ & $(2,3)$ & -- & 2865\\
$(36,13)$ & 8 & $(23,5)$ & 7 & 1 & YES & YES & YES & $1.29$ & $(2,3)$ & -- & 2866\\
$(37,8)$ & 8 & $(5,2)$ & 3 & 1 & YES & YES & NO(3) & $1.33$ & $(2,3)$ & NO & 2867\\
$(37,8)$ & 8 & $(5,2)$ & 3 & 1 & YES & YES & NO(3) & $1.33$ & $(2,3)$ & -- & 2868\\
$(37,8)$ & 8 & $(13,3)$ & 6 & 1 & YES & YES & NO(3) & $1.33$ & $(2,3)$ & NO & 2869\\
$(37,14)$ & 8 & $(16,3)$ & 7 & 1 & YES & YES & NO(2) & $1.44$ & $(2,3)$ & -- & 2870\\
$(37,8)$ & 8 & $(17,4)$ & 7 & 1 & YES & YES & YES & $1.43$ & $(2,3)$ & NO & 2871\\
$(37,8)$ & 8 & $(17,4)$ & 7 & 1 & YES & YES & YES & $1.43$ & $(2,3)$ & -- & 2872\\
$(37,14)$ & 8 & $(17,3)$ & 7 & 1 & YES & YES & NO(2) & $1.56$ & $(2,3)$ & NO & 2873\\
$(37,14)$ & 8 & $(17,3)$ & 7 & 1 & YES & YES & NO(2) & $1.56$ & $(2,3)$ & -- & 2874\\
$(37,16)$ & 9 & $(18,5)$ & 6 & 1 & YES & YES & YES & $1.62$ & $(2,3)$ & NO & 2875\\
$(37,10)$ & 8 & $(19,8)$ & 6 & 1 & YES & YES & YES & $1.50$ & $(2,3)$ & -- & 2876\\
$(37,11)$ & 8 & $(19,8)$ & 6 & 1 & YES & YES & YES & $1.29$ & $(2,3)$ & -- & 2877\\
$(37,11)$ & 8 & $(23,9)$ & 7 & 1 & YES & YES & YES & $1.57$ & $(2,3)$ & NO & 2878\\
$(37,14)$ & 8 & $(23,4)$ & 8 & 1 & YES & YES & NO(2) & $1.38$ & $(4,2)$ & NO & 2879\\
$(37,14)$ & 8 & $(23,4)$ & 8 & 1 & YES & YES & NO(2) & $1.56$ & $(2,3)$ & -- & 2880\\
$(37,14)$ & 8 & $(23,7)$ & 7 & 1 & YES & YES & YES & $1.57$ & $(2,3)$ & -- & 2881\\
$(37,8)$ & 8 & $(26,11)$ & 7 & 1 & YES & YES & YES & $1.29$ & $(2,3)$ & -- & 2882\\
$(37,14)$ & 8 & $(27,5)$ & 8 & 1 & YES & YES & YES & $1.38$ & $(2,3)$ & NO & 2883\\
$(37,14)$ & 8 & $(27,5)$ & 8 & 1 & YES & YES & YES & $1.50$ & $(2,3)$ & -- & 2884\\
$(37,14)$ & 8 & $(28,5)$ & 8 & 1 & YES & YES & NO(2) & $1.44$ & $(2,3)$ & -- & 2885\\
$(37,11)$ & 8 & $(32,9)$ & 8 & 1 & YES & YES & YES & $1.86$ & $(2,3)$ & -- & 2886\\
$(37,11)$ & 8 & $(34,13)$ & 7 & 1 & YES & YES & YES & $1.71$ & $(2,3)$ & -- & 2887\\
$(38,11)$ & 9 & $(11,3)$ & 5 & 1 & YES & YES & YES & $1.43$ & $(2,3)$ & -- & 2888\\
$(38,11)$ & 9 & $(11,3)$ & 5 & 1 & YES & YES & YES & $1.29$ & $(4,2)$ & NO & 2889\\
$(38,17)$ & 9 & $(13,4)$ & 6 & 1 & YES & YES & YES & $1.43$ & $(2,3)$ & -- & 2890\\
$(38,9)$ & 9 & $(17,7)$ & 6 & 1 & YES & YES & NO(2) & $1.56$ & $(2,3)$ & NO & 2891\\
$(38,9)$ & 9 & $(18,7)$ & 6 & 2 & YES & YES & NO(2) & $1.56$ & $(2,3)$ & NO & 2892\\
$(38,7)$ & 9 & $(22,9)$ & 7 & 2 & YES & YES & NO(2) & $1.44$ & $(2,3)$ & -- & 2893\\
$(38,7)$ & 9 & $(23,9)$ & 7 & 1 & YES & YES & NO(3) & $1.29$ & $(2,3)$ & -- & 2894\\
$(38,7)$ & 9 & $(23,9)$ & 7 & 1 & YES & YES & NO(2) & $1.25$ & $(4,2)$ & NO & 2895\\
$(38,11)$ & 9 & $(23,7)$ & 7 & 1 & YES & YES & YES & $1.43$ & $(2,3)$ & -- & 2896\\
$(39,14)$ & 8 & $(7,3)$ & 4 & 1 & YES & YES & YES & $1.29$ & $(2,3)$ & -- & 2897\\
$(39,16)$ & 8 & $(7,3)$ & 4 & 1 & YES & YES & NO(2) & $1.44$ & $(2,3)$ & -- & 2898\\
$(39,14)$ & 8 & $(8,3)$ & 4 & 1 & YES & YES & YES & $1.29$ & $(4,2)$ & -- & 2899\\
$(39,11)$ & 9 & $(10,3)$ & 5 & 1 & YES & YES & YES & $1.57$ & $(2,3)$ & -- & 2900\\
$(39,11)$ & 9 & $(11,3)$ & 5 & 1 & YES & YES & YES & $1.43$ & $(2,3)$ & -- & 2901\\
$(39,11)$ & 9 & $(11,4)$ & 5 & 1 & YES & YES & YES & $1.29$ & $(4,2)$ & NO & 2902\\
$(39,11)$ & 9 & $(11,4)$ & 5 & 1 & YES & YES & YES & $1.29$ & $(4,2)$ & -- & 2903\\
$(39,14)$ & 8 & $(12,5)$ & 5 & 3 & YES & YES & YES & $1.29$ & $(2,3)$ & -- & 2904\\
$(39,14)$ & 8 & $(13,3)$ & 6 & 13 & YES & YES & NO(2) & $1.00$ & $(6,1)$ & NO & 2905\\
$(39,14)$ & 8 & $(13,3)$ & 6 & 13 & YES & YES & NO(2) & $1.00$ & $(6,1)$ & -- & 2906\\
$(39,14)$ & 8 & $(27,10)$ & 7 & 3 & YES & YES & YES & $1.29$ & $(4,2)$ & NO & 2907\\
$(39,16)$ & 8 & $(29,13)$ & 8 & 1 & YES & YES & YES & $1.57$ & $(2,3)$ & NO & 2908\\
$(40,11)$ & 8 & $(7,3)$ & 4 & 1 & YES & YES & NO(2) & $1.56$ & $(2,3)$ & NO & 2909\\
$(40,11)$ & 8 & $(9,4)$ & 5 & 1 & YES & YES & YES & $1.57$ & $(2,3)$ & NO & 2910\\
$(40,17)$ & 9 & $(10,3)$ & 5 & 10 & YES & YES & NO(2) & $1.50$ & $(4,2)$ & NO & 2911\\
$(40,17)$ & 9 & $(10,3)$ & 5 & 10 & YES & YES & NO(2) & $1.50$ & $(4,2)$ & -- & 2912\\
$(40,17)$ & 9 & $(13,3)$ & 6 & 1 & YES & YES & NO(2) & $1.56$ & $(2,3)$ & NO & 2913\\
$(40,17)$ & 9 & $(13,4)$ & 6 & 1 & YES & YES & NO(2) & $1.50$ & $(4,2)$ & -- & 2914\\
$(40,17)$ & 9 & $(13,5)$ & 5 & 1 & YES & YES & YES & $1.57$ & $(2,3)$ & -- & 2915\\
$(40,7)$ & 9 & $(17,6)$ & 7 & 1 & YES & YES & YES & $1.29$ & $(2,3)$ & -- & 2916\\
$(40,9)$ & 9 & $(18,7)$ & 6 & 2 & YES & YES & NO(2) & $1.25$ & $(4,2)$ & -- & 2917\\
$(40,9)$ & 9 & $(23,10)$ & 7 & 1 & YES & YES & YES & $1.57$ & $(2,3)$ & NO & 2918\\
$(40,9)$ & 9 & $(23,10)$ & 7 & 1 & YES & YES & YES & $1.57$ & $(2,3)$ & -- & 2919\\
$(40,11)$ & 8 & $(32,9)$ & 8 & 8 & YES & YES & NO(2) & $1.56$ & $(2,3)$ & NO & 2920\\
$(40,9)$ & 9 & $(38,7)$ & 9 & 2 & YES & YES & NO(2) & $1.25$ & $(4,2)$ & NO & 2921\\
$(40,11)$ & 8 & $(39,11)$ & 9 & 1 & YES & YES & YES & $1.57$ & $(2,3)$ & NO & 2922\\
$(40,17)$ & 9 & $(39,16)$ & 8 & 1 & YES & YES & NO(2) & $1.38$ & $(4,2)$ & NO & 2923\\
$(41,18)$ & 8 & $(8,3)$ & 4 & 1 & YES & YES & YES & $1.14$ & $(4,2)$ & -- & 2924\\
$(41,17)$ & 8 & $(11,3)$ & 5 & 1 & YES & YES & NO(2) & $1.25$ & $(4,2)$ & -- & 2925\\
$(41,15)$ & 8 & $(13,4)$ & 6 & 1 & YES & YES & YES & $1.62$ & $(2,3)$ & NO & 2926\\
$(41,15)$ & 8 & $(13,4)$ & 6 & 1 & YES & YES & YES & $1.62$ & $(2,3)$ & -- & 2927\\
$(41,18)$ & 8 & $(13,5)$ & 5 & 1 & YES & YES & YES & $1.14$ & $(4,2)$ & -- & 2928\\
$(41,16)$ & 8 & $(16,5)$ & 7 & 1 & YES & YES & YES & $1.57$ & $(2,3)$ & -- & 2929\\
$(41,17)$ & 8 & $(16,7)$ & 6 & 1 & YES & YES & YES & $1.43$ & $(2,3)$ & -- & 2930\\
$(41,16)$ & 8 & $(17,5)$ & 6 & 1 & YES & YES & YES & $1.43$ & $(2,3)$ & -- & 2931\\
$(41,15)$ & 8 & $(21,5)$ & 8 & 1 & YES & YES & YES & $1.57$ & $(2,3)$ & -- & 2932\\
$(41,15)$ & 8 & $(22,5)$ & 7 & 1 & YES & YES & YES & $1.43$ & $(2,3)$ & -- & 2933\\
$(41,15)$ & 8 & $(23,7)$ & 7 & 1 & YES & YES & YES & $1.57$ & $(2,3)$ & -- & 2934\\
$(41,17)$ & 8 & $(23,10)$ & 7 & 1 & YES & YES & NO(2) & $1.56$ & $(2,3)$ & NO & 2935\\
$(41,17)$ & 8 & $(24,5)$ & 8 & 1 & YES & YES & YES & $1.57$ & $(2,3)$ & -- & 2936\\
$(41,12)$ & 8 & $(26,7)$ & 7 & 1 & YES & YES & YES & $1.71$ & $(2,3)$ & -- & 2937\\
$(41,15)$ & 8 & $(28,11)$ & 8 & 1 & YES & YES & YES & $1.62$ & $(2,3)$ & NO & 2938\\
$(41,9)$ & 9 & $(29,11)$ & 7 & 1 & YES & YES & YES & $1.71$ & $(2,3)$ & -- & 2939\\
$(41,12)$ & 8 & $(35,11)$ & 9 & 1 & YES & YES & YES & $1.43$ & $(2,3)$ & NO & 2940\\
$(41,11)$ & 8 & $(37,10)$ & 8 & 1 & YES & YES & NO(2) & $1.44$ & $(2,3)$ & NO & 2941\\
$(41,16)$ & 8 & $(41,15)$ & 8 & 41 & YES & YES & YES & $1.43$ & $(2,3)$ & NO & 2942\\
$(42,13)$ & 9 & $(11,3)$ & 5 & 1 & YES & YES & NO(2) & $1.12$ & $(4,2)$ & -- & 2943\\
$(42,13)$ & 9 & $(18,7)$ & 6 & 6 & YES & YES & YES & $1.62$ & $(2,3)$ & -- & 2944\\
$(43,16)$ & 9 & $(7,3)$ & 4 & 1 & YES & YES & NO(2) & $1.50$ & $(4,2)$ & -- & 2945\\
$(43,13)$ & 9 & $(8,3)$ & 4 & 1 & YES & YES & NO(2) & $1.38$ & $(4,2)$ & NO & 2946\\
$(43,13)$ & 9 & $(8,3)$ & 4 & 1 & YES & YES & NO(2) & $1.38$ & $(4,2)$ & -- & 2947\\
$(43,12)$ & 8 & $(9,4)$ & 5 & 1 & YES & YES & YES & $1.57$ & $(2,3)$ & NO & 2948\\
$(43,18)$ & 8 & $(10,3)$ & 5 & 1 & YES & YES & NO(2) & $1.56$ & $(2,3)$ & NO & 2949\\
$(43,18)$ & 8 & $(10,3)$ & 5 & 1 & YES & YES & NO(2) & $1.56$ & $(2,3)$ & -- & 2950\\
$(43,16)$ & 9 & $(14,3)$ & 6 & 1 & YES & YES & YES & $1.50$ & $(2,3)$ & NO & 2951\\
$(43,16)$ & 9 & $(14,3)$ & 6 & 1 & YES & YES & YES & $1.50$ & $(2,3)$ & -- & 2952\\
$(43,18)$ & 8 & $(16,7)$ & 6 & 1 & YES & YES & YES & $1.57$ & $(2,3)$ & -- & 2953\\
$(43,10)$ & 9 & $(17,7)$ & 6 & 1 & YES & YES & YES & $1.50$ & $(2,3)$ & NO & 2954\\
$(43,9)$ & 9 & $(18,7)$ & 6 & 1 & YES & YES & NO(2) & $1.56$ & $(2,3)$ & NO & 2955\\
$(43,19)$ & 9 & $(18,7)$ & 6 & 1 & YES & YES & YES & $1.43$ & $(2,3)$ & NO & 2956\\
$(43,19)$ & 9 & $(22,9)$ & 7 & 1 & YES & YES & YES & $1.43$ & $(2,3)$ & NO & 2957\\
$(43,10)$ & 9 & $(23,9)$ & 7 & 1 & YES & YES & YES & $1.43$ & $(2,3)$ & NO & 2958\\
$(43,9)$ & 9 & $(24,7)$ & 7 & 1 & YES & YES & NO(2) & $1.56$ & $(2,3)$ & NO & 2959\\
$(43,18)$ & 8 & $(40,17)$ & 9 & 1 & YES & YES & NO(2) & $1.56$ & $(2,3)$ & NO & 2960\\
$(44,17)$ & 8 & $(9,2)$ & 5 & 1 & YES & YES & NO(2) & $1.44$ & $(2,3)$ & -- & 2961\\
$(44,13)$ & 8 & $(11,5)$ & 6 & 11 & YES & YES & YES & $1.57$ & $(2,3)$ & NO & 2962\\
$(44,13)$ & 8 & $(11,5)$ & 6 & 11 & YES & YES & YES & $1.57$ & $(2,3)$ & -- & 2963\\
$(44,17)$ & 8 & $(11,4)$ & 5 & 11 & YES & YES & NO(2) & $1.56$ & $(2,3)$ & NO & 2964\\
$(44,17)$ & 8 & $(11,4)$ & 5 & 11 & YES & YES & NO(2) & $1.56$ & $(2,3)$ & -- & 2965\\
$(44,17)$ & 8 & $(13,4)$ & 6 & 1 & YES & YES & YES & $1.62$ & $(2,3)$ & -- & 2966\\
$(44,13)$ & 8 & $(18,7)$ & 6 & 2 & YES & YES & YES & $1.62$ & $(2,3)$ & -- & 2967\\
$(44,17)$ & 8 & $(24,5)$ & 8 & 4 & YES & YES & YES & $1.57$ & $(2,3)$ & -- & 2968\\
$(44,13)$ & 8 & $(25,7)$ & 7 & 1 & YES & YES & YES & $1.43$ & $(2,3)$ & NO & 2969\\
$(44,17)$ & 8 & $(25,7)$ & 7 & 1 & YES & YES & YES & $1.62$ & $(2,3)$ & -- & 2970\\
$(44,17)$ & 8 & $(31,7)$ & 8 & 1 & YES & YES & YES & $1.57$ & $(2,3)$ & NO & 2971\\
$(44,17)$ & 8 & $(31,12)$ & 7 & 1 & YES & YES & YES & $1.57$ & $(2,3)$ & -- & 2972\\
$(45,13)$ & 10 & $(10,3)$ & 5 & 5 & YES & YES & YES & $1.62$ & $(2,3)$ & -- & 2973\\
$(45,19)$ & 8 & $(10,3)$ & 5 & 5 & YES & YES & NO(2) & $1.44$ & $(2,3)$ & -- & 2974\\
$(45,17)$ & 9 & $(13,3)$ & 6 & 1 & YES & YES & NO(2) & $1.56$ & $(2,3)$ & NO & 2975\\
$(45,17)$ & 9 & $(17,5)$ & 6 & 1 & YES & YES & YES & $1.71$ & $(2,3)$ & -- & 2976\\
$(45,17)$ & 9 & $(18,5)$ & 6 & 9 & YES & YES & YES & $1.43$ & $(2,3)$ & NO & 2977\\
$(45,19)$ & 8 & $(18,7)$ & 6 & 9 & YES & YES & YES & $1.57$ & $(2,3)$ & -- & 2978\\
$(45,13)$ & 10 & $(22,5)$ & 7 & 1 & YES & YES & YES & $1.57$ & $(2,3)$ & NO & 2979\\
$(45,8)$ & 9 & $(23,9)$ & 7 & 1 & YES & YES & YES & $1.43$ & $(2,3)$ & NO & 2980\\
$(45,17)$ & 9 & $(23,9)$ & 7 & 1 & YES & YES & NO(2) & $1.50$ & $(4,2)$ & NO & 2981\\
$(45,17)$ & 9 & $(41,16)$ & 8 & 1 & YES & YES & YES & $1.57$ & $(2,3)$ & NO & 2982\\
$(45,19)$ & 8 & $(43,18)$ & 8 & 1 & YES & YES & NO(2) & $1.44$ & $(2,3)$ & NO & 2983\\
$(46,19)$ & 8 & $(7,3)$ & 4 & 1 & YES & YES & NO(2) & $1.44$ & $(2,3)$ & -- & 2984\\
$(46,13)$ & 10 & $(8,3)$ & 4 & 2 & YES & YES & NO(2) & $1.38$ & $(4,2)$ & NO & 2985\\
$(46,19)$ & 8 & $(13,5)$ & 5 & 1 & YES & YES & YES & $1.29$ & $(4,2)$ & -- & 2986\\
$(46,17)$ & 8 & $(18,7)$ & 6 & 2 & YES & YES & YES & $1.43$ & $(2,3)$ & -- & 2987\\
$(46,19)$ & 8 & $(20,9)$ & 7 & 2 & YES & YES & YES & $1.43$ & $(2,3)$ & 3308 & 2988\\
$(46,19)$ & 8 & $(24,7)$ & 7 & 2 & YES & YES & YES & $1.57$ & $(2,3)$ & -- & 2989\\
$(47,17)$ & 9 & $(8,3)$ & 4 & 1 & YES & YES & NO(2) & $1.38$ & $(4,2)$ & NO & 2990\\
$(47,17)$ & 9 & $(8,3)$ & 4 & 1 & YES & YES & NO(2) & $1.38$ & $(4,2)$ & -- & 2991\\
$(47,20)$ & 10 & $(14,3)$ & 6 & 1 & YES & YES & YES & $1.57$ & $(2,3)$ & -- & 2992\\
$(47,17)$ & 9 & $(16,3)$ & 7 & 1 & YES & YES & YES & $1.14$ & $(4,2)$ & -- & 2993\\
$(47,18)$ & 8 & $(16,7)$ & 6 & 1 & YES & YES & YES & $1.43$ & $(2,3)$ & NO & 2994\\
$(47,18)$ & 8 & $(17,7)$ & 6 & 1 & YES & YES & YES & $1.43$ & $(2,3)$ & NO & 2995\\
$(47,20)$ & 10 & $(19,3)$ & 8 & 1 & YES & YES & YES & $1.57$ & $(2,3)$ & -- & 2996\\
$(47,18)$ & 8 & $(22,5)$ & 7 & 1 & YES & YES & YES & $1.38$ & $(2,3)$ & -- & 2997\\
$(47,14)$ & 9 & $(24,7)$ & 7 & 1 & YES & YES & YES & $1.57$ & $(2,3)$ & -- & 2998\\
$(47,14)$ & 9 & $(25,7)$ & 7 & 1 & YES & YES & YES & $1.86$ & $(2,3)$ & -- & 2999\\
$(47,14)$ & 9 & $(38,7)$ & 9 & 1 & YES & YES & YES & $1.57$ & $(2,3)$ & NO & 3000\\
$(47,10)$ & 9 & $(40,9)$ & 9 & 1 & YES & YES & NO(2) & $1.25$ & $(4,2)$ & NO & 3001\\
$(48,17)$ & 9 & $(7,3)$ & 4 & 1 & YES & YES & YES & $1.29$ & $(2,3)$ & -- & 3002\\
$(48,13)$ & 9 & $(27,5)$ & 8 & 3 & YES & YES & YES & $1.50$ & $(2,3)$ & NO & 3003\\
$(48,13)$ & 9 & $(27,5)$ & 8 & 3 & YES & YES & YES & $1.50$ & $(2,3)$ & -- & 3004\\
$(49,18)$ & 8 & $(7,3)$ & 4 & 7 & YES & YES & YES & $1.43$ & $(2,3)$ & -- & 3005\\
$(49,22)$ & 9 & $(8,3)$ & 4 & 1 & YES & YES & YES & $1.29$ & $(2,3)$ & -- & 3006\\
$(49,19)$ & 8 & $(9,4)$ & 5 & 1 & YES & YES & YES & $1.43$ & $(2,3)$ & -- & 3007\\
$(49,15)$ & 9 & $(10,3)$ & 5 & 1 & YES & YES & NO(2) & $1.44$ & $(2,3)$ & -- & 3008\\
$(49,11)$ & 10 & $(11,4)$ & 5 & 1 & YES & YES & NO(2) & $1.25$ & $(4,2)$ & -- & 3009\\
$(49,19)$ & 8 & $(12,5)$ & 5 & 1 & YES & YES & YES & $1.43$ & $(2,3)$ & -- & 3010\\
$(49,20)$ & 9 & $(13,3)$ & 6 & 1 & YES & YES & YES & $1.50$ & $(2,3)$ & NO & 3011\\
$(49,20)$ & 9 & $(13,3)$ & 6 & 1 & YES & YES & YES & $1.50$ & $(2,3)$ & -- & 3012\\
$(49,20)$ & 9 & $(13,3)$ & 6 & 1 & YES & YES & NO(2) & $1.44$ & $(2,3)$ & NO & 3013\\
$(49,18)$ & 8 & $(16,5)$ & 7 & 1 & YES & YES & YES & $1.43$ & $(2,3)$ & -- & 3014\\
$(49,13)$ & 9 & $(17,7)$ & 6 & 1 & YES & YES & YES & $1.43$ & $(2,3)$ & -- & 3015\\
$(49,19)$ & 8 & $(17,5)$ & 6 & 1 & YES & YES & YES & $1.50$ & $(2,3)$ & -- & 3016\\
$(49,19)$ & 8 & $(17,6)$ & 7 & 1 & YES & YES & YES & $1.43$ & $(2,3)$ & NO & 3017\\
$(49,20)$ & 9 & $(17,4)$ & 7 & 1 & YES & YES & YES & $1.43$ & $(2,3)$ & NO & 3018\\
$(49,19)$ & 8 & $(18,7)$ & 6 & 1 & YES & YES & YES & $1.57$ & $(2,3)$ & -- & 3019\\
$(49,13)$ & 9 & $(19,8)$ & 6 & 1 & YES & YES & YES & $1.43$ & $(2,3)$ & -- & 3020\\
$(49,11)$ & 10 & $(23,4)$ & 8 & 1 & YES & YES & NO(2) & $1.25$ & $(4,2)$ & NO & 3021\\
$(49,19)$ & 8 & $(24,7)$ & 7 & 1 & YES & YES & YES & $1.86$ & $(2,3)$ & NO & 3022\\
$(49,19)$ & 8 & $(24,7)$ & 7 & 1 & YES & YES & YES & $1.86$ & $(2,3)$ & -- & 3023\\
$(49,18)$ & 8 & $(28,11)$ & 8 & 7 & YES & YES & YES & $1.43$ & $(2,3)$ & NO & 3024\\
$(49,18)$ & 8 & $(35,13)$ & 8 & 7 & YES & YES & NO(2) & $1.44$ & $(2,3)$ & NO & 3025\\
$(49,9)$ & 10 & $(39,11)$ & 9 & 1 & YES & YES & YES & $1.71$ & $(2,3)$ & NO & 3026\\
$(49,20)$ & 9 & $(41,17)$ & 8 & 1 & YES & YES & YES & $1.50$ & $(2,3)$ & 3253 & 3027\\
$(49,19)$ & 8 & $(45,17)$ & 9 & 1 & YES & YES & YES & $1.43$ & $(2,3)$ & NO & 3028\\
$(50,19)$ & 8 & $(10,3)$ & 5 & 10 & YES & YES & YES & $1.29$ & $(2,3)$ & -- & 3029\\
$(50,21)$ & 8 & $(11,5)$ & 6 & 1 & YES & YES & YES & $1.57$ & $(2,3)$ & -- & 3030\\
$(50,19)$ & 8 & $(17,5)$ & 6 & 1 & YES & YES & YES & $1.62$ & $(2,3)$ & -- & 3031\\
$(50,19)$ & 8 & $(47,18)$ & 8 & 1 & YES & YES & NO(2) & $1.44$ & $(2,3)$ & NO & 3032\\
$(51,14)$ & 9 & $(8,3)$ & 4 & 1 & YES & YES & YES & $1.57$ & $(2,3)$ & -- & 3033\\
$(51,20)$ & 9 & $(8,3)$ & 4 & 1 & YES & YES & YES & $1.43$ & $(2,3)$ & -- & 3034\\
$(51,23)$ & 9 & $(8,3)$ & 4 & 1 & YES & YES & YES & $1.29$ & $(2,3)$ & -- & 3035\\
$(51,20)$ & 9 & $(13,3)$ & 6 & 1 & YES & YES & NO(2) & $1.44$ & $(2,3)$ & NO & 3036\\
$(51,20)$ & 9 & $(13,4)$ & 6 & 1 & YES & YES & YES & $1.62$ & $(2,3)$ & -- & 3037\\
$(51,20)$ & 9 & $(17,4)$ & 7 & 17 & YES & YES & YES & $1.57$ & $(2,3)$ & NO & 3038\\
$(51,23)$ & 9 & $(19,8)$ & 6 & 1 & YES & YES & YES & $1.29$ & $(2,3)$ & NO & 3039\\
$(51,11)$ & 9 & $(23,9)$ & 7 & 1 & YES & YES & YES & $1.57$ & $(2,3)$ & NO & 3040\\
$(51,20)$ & 9 & $(29,11)$ & 7 & 1 & YES & YES & NO(2) & $1.44$ & $(2,3)$ & NO & 3041\\
$(51,11)$ & 9 & $(31,9)$ & 8 & 1 & YES & YES & YES & $1.57$ & $(2,3)$ & NO & 3042\\
$(51,20)$ & 9 & $(37,14)$ & 8 & 1 & YES & YES & YES & $1.57$ & $(2,3)$ & NO & 3043\\
$(52,19)$ & 9 & $(10,3)$ & 5 & 2 & YES & YES & NO(2) & $1.56$ & $(2,3)$ & -- & 3044\\
$(52,19)$ & 9 & $(13,4)$ & 6 & 13 & YES & YES & NO(2) & $1.56$ & $(2,3)$ & -- & 3045\\
$(52,19)$ & 9 & $(23,9)$ & 7 & 1 & YES & YES & NO(2) & $1.56$ & $(2,3)$ & NO & 3046\\
$(52,19)$ & 9 & $(49,18)$ & 8 & 1 & YES & YES & YES & $1.43$ & $(2,3)$ & 3466 & 3047\\
$(53,12)$ & 9 & $(4,1)$ & 3 & 1 & YES & YES & NO(2) & $1.33$ & $(2,3)$ & -- & 3048\\
$(53,12)$ & 9 & $(4,1)$ & 3 & 1 & YES & YES & NO(2) & $1.44$ & $(2,3)$ & NO & 3049\\
$(53,19)$ & 9 & $(8,3)$ & 4 & 1 & YES & YES & YES & $1.43$ & $(2,3)$ & -- & 3050\\
$(53,14)$ & 9 & $(10,3)$ & 5 & 1 & YES & YES & YES & $1.43$ & $(2,3)$ & -- & 3051\\
$(53,22)$ & 9 & $(10,3)$ & 5 & 1 & YES & YES & NO(2) & $1.56$ & $(2,3)$ & -- & 3052\\
$(53,19)$ & 9 & $(12,5)$ & 5 & 1 & YES & YES & NO(2) & $1.38$ & $(4,2)$ & NO & 3053\\
$(53,12)$ & 9 & $(13,4)$ & 6 & 1 & YES & YES & NO(2) & $1.33$ & $(2,3)$ & -- & 3054\\
$(53,14)$ & 9 & $(13,5)$ & 5 & 1 & YES & YES & YES & $1.43$ & $(2,3)$ & NO & 3055\\
$(53,19)$ & 9 & $(13,4)$ & 6 & 1 & YES & YES & YES & $1.57$ & $(2,3)$ & -- & 3056\\
$(53,23)$ & 9 & $(13,4)$ & 6 & 1 & YES & YES & YES & $1.43$ & $(2,3)$ & -- & 3057\\
$(53,19)$ & 9 & $(15,4)$ & 6 & 1 & YES & YES & YES & $1.43$ & $(2,3)$ & NO & 3058\\
$(53,19)$ & 9 & $(16,3)$ & 7 & 1 & YES & YES & YES & $1.14$ & $(4,2)$ & NO & 3059\\
$(53,22)$ & 9 & $(16,3)$ & 7 & 1 & YES & YES & YES & $1.29$ & $(2,3)$ & -- & 3060\\
$(53,19)$ & 9 & $(17,5)$ & 6 & 1 & YES & YES & YES & $1.71$ & $(2,3)$ & -- & 3061\\
$(53,10)$ & 10 & $(23,6)$ & 8 & 1 & YES & YES & YES & $1.57$ & $(2,3)$ & NO & 3062\\
$(53,19)$ & 9 & $(23,9)$ & 7 & 1 & YES & YES & YES & $1.57$ & $(2,3)$ & NO & 3063\\
$(53,14)$ & 9 & $(24,7)$ & 7 & 1 & YES & YES & YES & $1.50$ & $(2,3)$ & NO & 3064\\
$(53,16)$ & 10 & $(29,8)$ & 7 & 1 & YES & YES & YES & $1.43$ & $(2,3)$ & NO & 3065\\
$(53,12)$ & 9 & $(31,7)$ & 8 & 1 & YES & YES & NO(2) & $1.44$ & $(2,3)$ & NO & 3066\\
$(53,19)$ & 9 & $(34,13)$ & 7 & 1 & YES & YES & YES & $1.71$ & $(2,3)$ & NO & 3067\\
$(53,23)$ & 9 & $(34,15)$ & 8 & 1 & YES & YES & NO(2) & $1.44$ & $(2,3)$ & 3126 & 3068\\
$(53,19)$ & 9 & $(47,17)$ & 9 & 1 & YES & YES & NO(2) & $1.56$ & $(2,3)$ & NO & 3069\\
$(55,21)$ & 8 & $(7,3)$ & 4 & 1 & YES & YES & NO(2) & $1.44$ & $(2,3)$ & -- & 3070\\
$(55,24)$ & 9 & $(7,3)$ & 4 & 1 & YES & YES & NO(2) & $1.56$ & $(2,3)$ & -- & 3071\\
$(55,23)$ & 9 & $(8,3)$ & 4 & 1 & YES & YES & YES & $1.62$ & $(2,3)$ & -- & 3072\\
$(55,21)$ & 8 & $(11,5)$ & 6 & 11 & YES & YES & YES & $1.57$ & $(2,3)$ & NO & 3073\\
$(55,21)$ & 8 & $(12,5)$ & 5 & 1 & YES & YES & YES & $1.43$ & $(2,3)$ & -- & 3074\\
$(55,23)$ & 9 & $(12,5)$ & 5 & 1 & YES & YES & YES & $1.57$ & $(2,3)$ & -- & 3075\\
$(55,16)$ & 9 & $(13,3)$ & 6 & 1 & YES & YES & YES & $1.50$ & $(2,3)$ & NO & 3076\\
$(55,16)$ & 9 & $(13,3)$ & 6 & 1 & YES & YES & YES & $1.50$ & $(2,3)$ & -- & 3077\\
$(55,17)$ & 10 & $(14,3)$ & 6 & 1 & YES & YES & NO(2) & $1.44$ & $(2,3)$ & NO & 3078\\
$(55,16)$ & 9 & $(16,7)$ & 6 & 1 & YES & YES & YES & $1.50$ & $(2,3)$ & -- & 3079\\
$(55,13)$ & 10 & $(17,7)$ & 6 & 1 & YES & YES & YES & $1.71$ & $(2,3)$ & -- & 3080\\
$(55,21)$ & 8 & $(17,6)$ & 7 & 1 & YES & YES & YES & $1.43$ & $(2,3)$ & NO & 3081\\
$(55,23)$ & 9 & $(20,9)$ & 7 & 5 & YES & YES & YES & $1.57$ & $(2,3)$ & NO & 3082\\
$(55,16)$ & 9 & $(23,7)$ & 7 & 1 & YES & YES & YES & $1.62$ & $(2,3)$ & NO & 3083\\
$(55,21)$ & 8 & $(23,9)$ & 7 & 1 & YES & YES & NO(2) & $1.25$ & $(4,2)$ & NO & 3084\\
$(55,21)$ & 8 & $(25,7)$ & 7 & 5 & YES & YES & YES & $1.57$ & $(2,3)$ & NO & 3085\\
$(55,16)$ & 9 & $(27,8)$ & 7 & 1 & YES & YES & YES & $1.50$ & $(2,3)$ & NO & 3086\\
$(55,21)$ & 8 & $(33,13)$ & 9 & 11 & YES & YES & YES & $1.57$ & $(2,3)$ & NO & 3087\\
$(55,12)$ & 9 & $(37,8)$ & 8 & 1 & YES & YES & NO(3) & $1.29$ & $(2,3)$ & NO & 3088\\
$(55,21)$ & 8 & $(37,14)$ & 8 & 1 & YES & YES & NO(2) & $1.25$ & $(4,2)$ & NO & 3089\\
$(55,21)$ & 8 & $(41,16)$ & 8 & 1 & YES & YES & YES & $1.43$ & $(2,3)$ & NO & 3090\\
$(55,24)$ & 9 & $(53,23)$ & 9 & 1 & YES & YES & NO(2) & $1.44$ & $(2,3)$ & NO & 3091\\
$(56,15)$ & 9 & $(13,5)$ & 5 & 1 & YES & YES & NO(2) & $1.38$ & $(4,2)$ & NO & 3092\\
$(56,17)$ & 9 & $(18,7)$ & 6 & 2 & YES & YES & YES & $1.43$ & $(2,3)$ & -- & 3093\\
$(56,23)$ & 9 & $(49,20)$ & 9 & 7 & YES & YES & NO(2) & $1.44$ & $(2,3)$ & NO & 3094\\
$(57,13)$ & 9 & $(4,1)$ & 3 & 1 & YES & YES & NO(2) & $1.33$ & $(2,3)$ & -- & 3095\\
$(57,16)$ & 9 & $(7,2)$ & 4 & 1 & YES & YES & YES & $1.29$ & $(2,3)$ & -- & 3096\\
$(57,16)$ & 9 & $(7,3)$ & 4 & 1 & YES & YES & NO(2) & $1.56$ & $(2,3)$ & NO & 3097\\
$(57,16)$ & 9 & $(7,3)$ & 4 & 1 & YES & YES & NO(2) & $1.56$ & $(2,3)$ & -- & 3098\\
$(57,17)$ & 10 & $(9,4)$ & 5 & 3 & YES & YES & YES & $1.57$ & $(2,3)$ & NO & 3099\\
$(57,17)$ & 10 & $(9,4)$ & 5 & 3 & YES & YES & YES & $1.57$ & $(2,3)$ & -- & 3100\\
$(57,13)$ & 9 & $(11,4)$ & 5 & 1 & YES & YES & YES & $1.38$ & $(2,3)$ & NO & 3101\\
$(57,13)$ & 9 & $(11,4)$ & 5 & 1 & YES & YES & YES & $1.38$ & $(2,3)$ & -- & 3102\\
$(57,16)$ & 9 & $(11,4)$ & 5 & 1 & YES & YES & NO(2) & $1.14$ & $(6,1)$ & NO & 3103\\
$(57,16)$ & 9 & $(13,5)$ & 5 & 1 & YES & YES & NO(2) & $1.56$ & $(2,3)$ & NO & 3104\\
$(57,17)$ & 10 & $(13,5)$ & 5 & 1 & YES & YES & YES & $1.57$ & $(2,3)$ & NO & 3105\\
$(57,13)$ & 9 & $(34,9)$ & 8 & 1 & YES & YES & YES & $1.71$ & $(2,3)$ & -- & 3106\\
$(58,17)$ & 9 & $(7,3)$ & 4 & 1 & YES & YES & NO(2) & $1.25$ & $(4,2)$ & NO & 3107\\
$(58,17)$ & 9 & $(7,3)$ & 4 & 1 & YES & YES & NO(2) & $1.25$ & $(4,2)$ & -- & 3108\\
$(58,17)$ & 9 & $(9,4)$ & 5 & 1 & YES & YES & NO(2) & $1.25$ & $(4,2)$ & NO & 3109\\
$(58,21)$ & 10 & $(11,3)$ & 5 & 1 & YES & YES & YES & $1.62$ & $(2,3)$ & -- & 3110\\
$(58,9)$ & 11 & $(17,6)$ & 7 & 1 & YES & YES & YES & $1.43$ & $(2,3)$ & -- & 3111\\
$(58,17)$ & 9 & $(19,8)$ & 6 & 1 & YES & YES & YES & $1.86$ & $(2,3)$ & -- & 3112\\
$(58,17)$ & 9 & $(43,13)$ & 9 & 1 & YES & YES & YES & $1.43$ & $(2,3)$ & NO & 3113\\
$(59,18)$ & 9 & $(7,2)$ & 4 & 1 & YES & YES & YES & $1.43$ & $(2,3)$ & NO & 3114\\
$(59,18)$ & 9 & $(7,2)$ & 4 & 1 & YES & YES & YES & $1.43$ & $(2,3)$ & -- & 3115\\
$(59,26)$ & 9 & $(10,3)$ & 5 & 1 & YES & YES & NO(2) & $1.44$ & $(2,3)$ & NO & 3116\\
$(59,13)$ & 11 & $(12,5)$ & 5 & 1 & YES & YES & YES & $1.43$ & $(2,3)$ & NO & 3117\\
$(59,23)$ & 9 & $(12,5)$ & 5 & 1 & YES & YES & YES & $1.57$ & $(2,3)$ & -- & 3118\\
$(59,18)$ & 9 & $(13,5)$ & 5 & 1 & YES & YES & YES & $1.43$ & $(2,3)$ & NO & 3119\\
$(59,25)$ & 9 & $(13,3)$ & 6 & 1 & YES & YES & NO(2) & $1.44$ & $(2,3)$ & NO & 3120\\
$(59,26)$ & 9 & $(15,4)$ & 6 & 1 & YES & YES & YES & $1.43$ & $(2,3)$ & NO & 3121\\
$(59,9)$ & 11 & $(17,6)$ & 7 & 1 & YES & YES & YES & $1.43$ & $(2,3)$ & -- & 3122\\
$(59,23)$ & 9 & $(17,5)$ & 6 & 1 & YES & YES & YES & $1.57$ & $(2,3)$ & -- & 3123\\
$(59,26)$ & 9 & $(17,4)$ & 7 & 1 & YES & YES & YES & $1.43$ & $(2,3)$ & NO & 3124\\
$(59,13)$ & 11 & $(18,5)$ & 6 & 1 & YES & YES & YES & $1.43$ & $(2,3)$ & NO & 3125\\
$(59,26)$ & 9 & $(30,13)$ & 8 & 1 & YES & YES & NO(2) & $1.44$ & $(2,3)$ & 3068 & 3126\\
$(59,18)$ & 9 & $(35,11)$ & 9 & 1 & YES & YES & YES & $1.43$ & $(2,3)$ & NO & 3127\\
$(59,25)$ & 9 & $(43,18)$ & 8 & 1 & YES & YES & NO(2) & $1.44$ & $(2,3)$ & 3302 & 3128\\
$(59,23)$ & 9 & $(51,20)$ & 9 & 1 & YES & YES & NO(2) & $1.00$ & $(6,1)$ & NO & 3129\\
$(59,23)$ & 9 & $(55,21)$ & 8 & 1 & YES & YES & YES & $1.57$ & $(2,3)$ & NO & 3130\\
$(60,23)$ & 9 & $(7,3)$ & 4 & 1 & YES & YES & NO(2) & $1.38$ & $(4,2)$ & -- & 3131\\
$(60,11)$ & 11 & $(8,3)$ & 4 & 4 & YES & YES & NO(2) & $1.00$ & $(6,1)$ & -- & 3132\\
$(60,11)$ & 11 & $(12,5)$ & 5 & 12 & YES & YES & YES & $1.14$ & $(4,2)$ & -- & 3133\\
$(60,23)$ & 9 & $(18,5)$ & 6 & 6 & YES & YES & YES & $1.71$ & $(2,3)$ & -- & 3134\\
$(60,23)$ & 9 & $(26,5)$ & 9 & 2 & YES & YES & YES & $1.57$ & $(2,3)$ & NO & 3135\\
$(60,13)$ & 9 & $(31,9)$ & 8 & 1 & YES & YES & YES & $1.43$ & $(2,3)$ & NO & 3136\\
$(60,13)$ & 9 & $(37,11)$ & 8 & 1 & YES & YES & YES & $1.43$ & $(2,3)$ & NO & 3137\\
$(61,17)$ & 9 & $(7,3)$ & 4 & 1 & YES & YES & NO(2) & $1.25$ & $(4,2)$ & NO & 3138\\
$(61,17)$ & 9 & $(7,3)$ & 4 & 1 & YES & YES & NO(2) & $1.25$ & $(4,2)$ & -- & 3139\\
$(61,17)$ & 9 & $(8,3)$ & 4 & 1 & YES & YES & NO(2) & $1.25$ & $(4,2)$ & NO & 3140\\
$(61,17)$ & 9 & $(8,3)$ & 4 & 1 & YES & YES & NO(2) & $1.25$ & $(4,2)$ & -- & 3141\\
$(61,25)$ & 9 & $(9,4)$ & 5 & 1 & YES & YES & YES & $1.43$ & $(2,3)$ & -- & 3142\\
$(61,17)$ & 9 & $(11,4)$ & 5 & 1 & YES & YES & NO(2) & $1.25$ & $(4,2)$ & NO & 3143\\
$(61,18)$ & 9 & $(11,5)$ & 6 & 1 & YES & YES & YES & $1.57$ & $(2,3)$ & -- & 3144\\
$(61,16)$ & 10 & $(13,5)$ & 5 & 1 & YES & YES & YES & $1.43$ & $(2,3)$ & NO & 3145\\
$(61,17)$ & 9 & $(13,5)$ & 5 & 1 & YES & YES & NO(2) & $1.25$ & $(4,2)$ & NO & 3146\\
$(61,25)$ & 9 & $(13,3)$ & 6 & 1 & YES & YES & NO(2) & $1.44$ & $(2,3)$ & NO & 3147\\
$(61,25)$ & 9 & $(13,4)$ & 6 & 1 & YES & YES & YES & $1.57$ & $(2,3)$ & -- & 3148\\
$(61,25)$ & 9 & $(13,4)$ & 6 & 1 & YES & YES & YES & $1.43$ & $(2,3)$ & NO & 3149\\
$(61,25)$ & 9 & $(17,4)$ & 7 & 1 & YES & YES & YES & $1.43$ & $(2,3)$ & NO & 3150\\
$(61,17)$ & 9 & $(18,7)$ & 6 & 1 & YES & YES & YES & $1.62$ & $(2,3)$ & -- & 3151\\
$(61,18)$ & 9 & $(18,7)$ & 6 & 1 & YES & YES & YES & $1.86$ & $(2,3)$ & NO & 3152\\
$(61,18)$ & 9 & $(18,7)$ & 6 & 1 & YES & YES & YES & $1.86$ & $(2,3)$ & -- & 3153\\
$(61,14)$ & 10 & $(23,4)$ & 8 & 1 & YES & YES & NO(2) & $1.25$ & $(4,2)$ & NO & 3154\\
$(61,25)$ & 9 & $(23,10)$ & 7 & 1 & YES & YES & YES & $1.43$ & $(2,3)$ & NO & 3155\\
$(61,14)$ & 10 & $(27,5)$ & 8 & 1 & YES & YES & YES & $1.38$ & $(2,3)$ & NO & 3156\\
$(61,22)$ & 9 & $(39,14)$ & 8 & 1 & YES & YES & NO(2) & $1.00$ & $(6,1)$ & NO & 3157\\
$(61,17)$ & 9 & $(40,9)$ & 9 & 1 & YES & YES & YES & $1.57$ & $(2,3)$ & NO & 3158\\
$(61,22)$ & 9 & $(53,19)$ & 9 & 1 & YES & YES & YES & $1.43$ & $(2,3)$ & NO & 3159\\
$(62,23)$ & 9 & $(10,3)$ & 5 & 2 & YES & YES & NO(2) & $1.44$ & $(2,3)$ & -- & 3160\\
$(62,17)$ & 10 & $(12,5)$ & 5 & 2 & YES & YES & YES & $1.43$ & $(2,3)$ & -- & 3161\\
$(62,23)$ & 9 & $(13,3)$ & 6 & 1 & YES & YES & YES & $1.50$ & $(2,3)$ & -- & 3162\\
$(62,11)$ & 10 & $(14,5)$ & 6 & 2 & YES & YES & NO(2) & $1.44$ & $(2,3)$ & -- & 3163\\
$(62,13)$ & 10 & $(18,7)$ & 6 & 2 & YES & YES & YES & $1.43$ & $(2,3)$ & -- & 3164\\
$(62,11)$ & 10 & $(23,9)$ & 7 & 1 & YES & YES & YES & $1.57$ & $(2,3)$ & NO & 3165\\
$(62,17)$ & 10 & $(23,7)$ & 7 & 1 & YES & YES & YES & $1.43$ & $(2,3)$ & NO & 3166\\
$(62,11)$ & 10 & $(40,9)$ & 9 & 2 & YES & YES & YES & $1.57$ & $(2,3)$ & NO & 3167\\
$(63,17)$ & 9 & $(3,1)$ & 2 & 3 & YES & YES & NO(2) & $1.12$ & $(4,2)$ & -- & 3168\\
$(63,17)$ & 9 & $(4,1)$ & 3 & 1 & YES & YES & NO(2) & $1.33$ & $(2,3)$ & -- & 3169\\
$(63,26)$ & 9 & $(4,1)$ & 3 & 1 & YES & YES & NO(2) & $1.38$ & $(4,2)$ & NO & 3170\\
$(63,26)$ & 9 & $(4,1)$ & 3 & 1 & YES & YES & NO(2) & $1.38$ & $(4,2)$ & -- & 3171\\
$(63,26)$ & 9 & $(4,1)$ & 3 & 1 & YES & YES & NO(2) & $1.38$ & $(4,2)$ & NO & 3172\\
$(63,23)$ & 10 & $(10,3)$ & 5 & 1 & YES & YES & NO(2) & $1.38$ & $(4,2)$ & -- & 3173\\
$(63,17)$ & 9 & $(15,4)$ & 6 & 3 & YES & YES & NO(2) & $1.25$ & $(4,2)$ & NO & 3174\\
$(63,23)$ & 10 & $(18,7)$ & 6 & 9 & YES & YES & NO(2) & $1.38$ & $(4,2)$ & NO & 3175\\
$(63,26)$ & 9 & $(19,5)$ & 7 & 1 & YES & YES & YES & $1.57$ & $(2,3)$ & -- & 3176\\
$(63,26)$ & 9 & $(23,5)$ & 7 & 1 & YES & YES & YES & $1.71$ & $(2,3)$ & NO & 3177\\
$(63,26)$ & 9 & $(23,5)$ & 7 & 1 & YES & YES & YES & $1.71$ & $(2,3)$ & -- & 3178\\
$(64,23)$ & 9 & $(5,2)$ & 3 & 1 & YES & YES & YES & $1.29$ & $(4,2)$ & -- & 3179\\
$(64,23)$ & 9 & $(7,3)$ & 4 & 1 & YES & YES & NO(2) & $1.44$ & $(2,3)$ & -- & 3180\\
$(64,23)$ & 9 & $(8,3)$ & 4 & 8 & YES & YES & YES & $1.43$ & $(2,3)$ & -- & 3181\\
$(64,17)$ & 10 & $(13,3)$ & 6 & 1 & YES & YES & YES & $1.38$ & $(2,3)$ & -- & 3182\\
$(64,25)$ & 9 & $(51,20)$ & 9 & 1 & YES & YES & NO(2) & $1.56$ & $(2,3)$ & NO & 3183\\
$(64,23)$ & 9 & $(53,19)$ & 9 & 1 & YES & YES & YES & $1.29$ & $(4,2)$ & NO & 3184\\
$(64,23)$ & 9 & $(61,22)$ & 9 & 1 & YES & YES & NO(2) & $1.44$ & $(2,3)$ & NO & 3185\\
$(65,24)$ & 9 & $(7,3)$ & 4 & 1 & YES & YES & NO(2) & $1.25$ & $(4,2)$ & -- & 3186\\
$(65,27)$ & 10 & $(7,3)$ & 4 & 1 & YES & YES & NO(2) & $1.50$ & $(4,2)$ & -- & 3187\\
$(65,27)$ & 10 & $(8,3)$ & 4 & 1 & YES & YES & YES & $1.50$ & $(2,3)$ & -- & 3188\\
$(65,19)$ & 9 & $(9,4)$ & 5 & 1 & YES & YES & YES & $1.50$ & $(2,3)$ & NO & 3189\\
$(65,24)$ & 9 & $(9,4)$ & 5 & 1 & YES & YES & NO(2) & $1.25$ & $(4,2)$ & NO & 3190\\
$(65,27)$ & 10 & $(11,3)$ & 5 & 1 & YES & YES & YES & $1.29$ & $(4,2)$ & -- & 3191\\
$(65,19)$ & 9 & $(12,5)$ & 5 & 1 & YES & YES & YES & $1.43$ & $(2,3)$ & -- & 3192\\
$(65,24)$ & 9 & $(17,6)$ & 7 & 1 & YES & YES & YES & $1.29$ & $(4,2)$ & NO & 3193\\
$(65,24)$ & 9 & $(19,4)$ & 7 & 1 & YES & YES & YES & $1.57$ & $(2,3)$ & -- & 3194\\
$(65,27)$ & 10 & $(39,16)$ & 8 & 13 & YES & YES & NO(2) & $1.38$ & $(4,2)$ & NO & 3195\\
$(66,25)$ & 9 & $(7,3)$ & 4 & 1 & YES & YES & YES & $1.62$ & $(2,3)$ & NO & 3196\\
$(66,25)$ & 9 & $(7,3)$ & 4 & 1 & YES & YES & YES & $1.62$ & $(2,3)$ & -- & 3197\\
$(66,25)$ & 9 & $(10,3)$ & 5 & 2 & YES & YES & YES & $1.29$ & $(2,3)$ & NO & 3198\\
$(66,25)$ & 9 & $(13,3)$ & 6 & 1 & YES & YES & NO(2) & $1.44$ & $(2,3)$ & NO & 3199\\
$(66,25)$ & 9 & $(13,4)$ & 6 & 1 & YES & YES & YES & $1.57$ & $(2,3)$ & NO & 3200\\
$(66,29)$ & 9 & $(21,8)$ & 6 & 3 & YES & YES & YES & $1.43$ & $(2,3)$ & NO & 3201\\
$(66,25)$ & 9 & $(28,11)$ & 8 & 2 & YES & YES & YES & $1.57$ & $(2,3)$ & NO & 3202\\
$(66,25)$ & 9 & $(34,13)$ & 7 & 2 & YES & YES & YES & $1.29$ & $(2,3)$ & NO & 3203\\
$(66,29)$ & 9 & $(43,19)$ & 9 & 1 & YES & YES & YES & $1.14$ & $(4,2)$ & NO & 3204\\
$(67,18)$ & 9 & $(3,1)$ & 2 & 1 & YES & YES & YES & $1.38$ & $(2,3)$ & -- & 3205\\
$(67,26)$ & 9 & $(3,1)$ & 2 & 1 & YES & YES & NO(2) & $1.38$ & $(4,2)$ & NO & 3206\\
$(67,26)$ & 9 & $(4,1)$ & 3 & 1 & YES & YES & NO(2) & $1.38$ & $(4,2)$ & NO & 3207\\
$(67,26)$ & 9 & $(4,1)$ & 3 & 1 & YES & YES & NO(2) & $1.38$ & $(4,2)$ & -- & 3208\\
$(67,26)$ & 9 & $(4,1)$ & 3 & 1 & YES & YES & NO(2) & $1.38$ & $(4,2)$ & NO & 3209\\
$(67,18)$ & 9 & $(5,2)$ & 3 & 1 & YES & YES & YES & $1.43$ & $(2,3)$ & NO & 3210\\
$(67,18)$ & 9 & $(5,2)$ & 3 & 1 & YES & YES & YES & $1.43$ & $(2,3)$ & -- & 3211\\
$(67,26)$ & 9 & $(5,2)$ & 3 & 1 & YES & YES & NO(2) & $1.38$ & $(4,2)$ & -- & 3212\\
$(67,18)$ & 9 & $(9,2)$ & 5 & 1 & YES & YES & YES & $1.43$ & $(2,3)$ & -- & 3213\\
$(67,18)$ & 9 & $(9,2)$ & 5 & 1 & YES & YES & YES & $1.43$ & $(2,3)$ & NO & 3214\\
$(67,18)$ & 9 & $(10,3)$ & 5 & 1 & YES & YES & YES & $1.29$ & $(2,3)$ & NO & 3215\\
$(67,28)$ & 10 & $(11,3)$ & 5 & 1 & YES & YES & YES & $1.71$ & $(2,3)$ & -- & 3216\\
$(67,26)$ & 9 & $(12,5)$ & 5 & 1 & YES & YES & NO(2) & $1.38$ & $(4,2)$ & NO & 3217\\
$(67,18)$ & 9 & $(13,5)$ & 5 & 1 & YES & YES & YES & $1.43$ & $(2,3)$ & -- & 3218\\
$(67,18)$ & 9 & $(13,5)$ & 5 & 1 & YES & YES & YES & $1.43$ & $(2,3)$ & NO & 3219\\
$(67,20)$ & 11 & $(13,2)$ & 7 & 1 & YES & YES & YES & $1.43$ & $(2,3)$ & NO & 3220\\
$(67,28)$ & 10 & $(14,3)$ & 6 & 1 & YES & YES & YES & $1.71$ & $(2,3)$ & NO & 3221\\
$(67,18)$ & 9 & $(18,7)$ & 6 & 1 & YES & YES & YES & $1.57$ & $(2,3)$ & -- & 3222\\
$(67,26)$ & 9 & $(18,5)$ & 6 & 1 & YES & YES & YES & $1.57$ & $(2,3)$ & -- & 3223\\
$(67,10)$ & 11 & $(23,6)$ & 8 & 1 & YES & YES & YES & $1.57$ & $(2,3)$ & NO & 3224\\
$(67,10)$ & 11 & $(25,6)$ & 9 & 1 & YES & YES & YES & $1.57$ & $(2,3)$ & NO & 3225\\
$(67,26)$ & 9 & $(28,11)$ & 8 & 1 & YES & YES & NO(2) & $1.38$ & $(4,2)$ & NO & 3226\\
$(67,18)$ & 9 & $(37,11)$ & 8 & 1 & YES & YES & YES & $1.57$ & $(2,3)$ & NO & 3227\\
$(67,18)$ & 9 & $(41,11)$ & 8 & 1 & YES & YES & YES & $1.38$ & $(2,3)$ & NO & 3228\\
$(67,18)$ & 9 & $(56,15)$ & 9 & 1 & YES & YES & YES & $1.43$ & $(2,3)$ & NO & 3229\\
$(67,20)$ & 11 & $(64,19)$ & 9 & 1 & YES & YES & YES & $1.43$ & $(2,3)$ & NO & 3230\\
$(68,25)$ & 9 & $(7,3)$ & 4 & 1 & YES & YES & NO(2) & $1.56$ & $(2,3)$ & NO & 3231\\
$(68,25)$ & 9 & $(7,3)$ & 4 & 1 & YES & YES & NO(2) & $1.56$ & $(2,3)$ & -- & 3232\\
$(68,19)$ & 9 & $(8,3)$ & 4 & 4 & YES & YES & YES & $1.50$ & $(2,3)$ & -- & 3233\\
$(68,25)$ & 9 & $(9,2)$ & 5 & 1 & YES & YES & NO(2) & $1.44$ & $(2,3)$ & -- & 3234\\
$(68,25)$ & 9 & $(9,4)$ & 5 & 1 & YES & YES & YES & $1.57$ & $(2,3)$ & -- & 3235\\
$(68,21)$ & 11 & $(11,5)$ & 6 & 1 & YES & YES & YES & $1.71$ & $(2,3)$ & NO & 3236\\
$(68,25)$ & 9 & $(19,4)$ & 7 & 1 & YES & YES & YES & $1.71$ & $(2,3)$ & -- & 3237\\
$(68,25)$ & 9 & $(52,19)$ & 9 & 4 & YES & YES & NO(2) & $1.56$ & $(2,3)$ & NO & 3238\\
$(69,29)$ & 9 & $(5,2)$ & 3 & 1 & YES & YES & NO(2) & $1.25$ & $(4,2)$ & -- & 3239\\
$(69,25)$ & 11 & $(12,5)$ & 5 & 3 & YES & YES & YES & $1.71$ & $(2,3)$ & -- & 3240\\
$(69,19)$ & 9 & $(17,7)$ & 6 & 1 & YES & YES & YES & $1.57$ & $(2,3)$ & -- & 3241\\
$(69,29)$ & 9 & $(17,7)$ & 6 & 1 & YES & YES & NO(2) & $1.25$ & $(4,2)$ & 3523 & 3242\\
$(69,19)$ & 9 & $(18,7)$ & 6 & 3 & YES & YES & YES & $1.43$ & $(2,3)$ & -- & 3243\\
$(69,29)$ & 9 & $(18,5)$ & 6 & 3 & YES & YES & YES & $1.71$ & $(2,3)$ & -- & 3244\\
$(69,29)$ & 9 & $(19,5)$ & 7 & 1 & YES & YES & YES & $1.71$ & $(2,3)$ & NO & 3245\\
$(69,16)$ & 11 & $(25,4)$ & 9 & 1 & YES & YES & YES & $1.43$ & $(2,3)$ & -- & 3246\\
$(69,19)$ & 9 & $(27,8)$ & 7 & 3 & YES & YES & YES & $1.71$ & $(2,3)$ & -- & 3247\\
$(69,20)$ & 10 & $(41,12)$ & 8 & 1 & YES & YES & NO(2) & $1.44$ & $(2,3)$ & NO & 3248\\
$(69,20)$ & 10 & $(47,14)$ & 9 & 1 & YES & YES & YES & $1.43$ & $(2,3)$ & NO & 3249\\
$(70,29)$ & 9 & $(9,4)$ & 5 & 1 & YES & YES & YES & $1.57$ & $(2,3)$ & -- & 3250\\
$(70,29)$ & 9 & $(11,4)$ & 5 & 1 & YES & YES & YES & $1.57$ & $(2,3)$ & -- & 3251\\
$(70,29)$ & 9 & $(12,5)$ & 5 & 2 & YES & YES & YES & $1.57$ & $(2,3)$ & -- & 3252\\
$(70,29)$ & 9 & $(27,11)$ & 8 & 1 & YES & YES & YES & $1.50$ & $(2,3)$ & 3027 & 3253\\
$(70,29)$ & 9 & $(63,26)$ & 9 & 7 & YES & YES & NO(3) & $1.29$ & $(2,3)$ & NO & 3254\\
$(71,30)$ & 9 & $(4,1)$ & 3 & 1 & YES & YES & NO(2) & $1.33$ & $(2,3)$ & -- & 3255\\
$(71,16)$ & 10 & $(7,2)$ & 4 & 1 & YES & YES & YES & $1.57$ & $(2,3)$ & NO & 3256\\
$(71,16)$ & 10 & $(7,2)$ & 4 & 1 & YES & YES & YES & $1.57$ & $(2,3)$ & -- & 3257\\
$(71,26)$ & 9 & $(7,2)$ & 4 & 1 & YES & YES & NO(2) & $1.44$ & $(2,3)$ & -- & 3258\\
$(71,27)$ & 9 & $(7,3)$ & 4 & 1 & YES & YES & NO(2) & $1.56$ & $(2,3)$ & -- & 3259\\
$(71,30)$ & 9 & $(7,3)$ & 4 & 1 & YES & YES & YES & $1.43$ & $(2,3)$ & -- & 3260\\
$(71,32)$ & 10 & $(7,2)$ & 4 & 1 & YES & YES & YES & $1.29$ & $(2,3)$ & -- & 3261\\
$(71,22)$ & 10 & $(8,3)$ & 4 & 1 & YES & YES & YES & $1.29$ & $(2,3)$ & -- & 3262\\
$(71,16)$ & 10 & $(9,4)$ & 5 & 1 & YES & YES & YES & $1.62$ & $(2,3)$ & -- & 3263\\
$(71,26)$ & 9 & $(10,3)$ & 5 & 1 & YES & YES & NO(3) & $1.29$ & $(2,3)$ & -- & 3264\\
$(71,32)$ & 10 & $(12,5)$ & 5 & 1 & YES & YES & YES & $1.43$ & $(2,3)$ & NO & 3265\\
$(71,30)$ & 9 & $(13,4)$ & 6 & 1 & YES & YES & YES & $1.43$ & $(2,3)$ & NO & 3266\\
$(71,26)$ & 9 & $(15,4)$ & 6 & 1 & YES & YES & YES & $1.43$ & $(2,3)$ & NO & 3267\\
$(71,27)$ & 9 & $(22,5)$ & 7 & 1 & YES & YES & YES & $1.71$ & $(2,3)$ & -- & 3268\\
$(71,27)$ & 9 & $(22,5)$ & 7 & 1 & YES & YES & YES & $1.86$ & $(2,3)$ & NO & 3269\\
$(71,30)$ & 9 & $(22,5)$ & 7 & 1 & YES & YES & YES & $1.57$ & $(2,3)$ & NO & 3270\\
$(71,30)$ & 9 & $(27,11)$ & 8 & 1 & YES & YES & YES & $1.43$ & $(2,3)$ & NO & 3271\\
$(71,16)$ & 10 & $(35,8)$ & 8 & 1 & YES & YES & YES & $1.57$ & $(2,3)$ & NO & 3272\\
$(71,29)$ & 10 & $(41,17)$ & 8 & 1 & YES & YES & YES & $1.43$ & $(2,3)$ & NO & 3273\\
$(71,27)$ & 9 & $(45,17)$ & 9 & 1 & YES & YES & NO(2) & $1.56$ & $(2,3)$ & NO & 3274\\
$(71,30)$ & 9 & $(47,20)$ & 10 & 1 & YES & YES & YES & $1.57$ & $(2,3)$ & 3757 & 3275\\
$(71,16)$ & 10 & $(61,14)$ & 10 & 1 & YES & YES & YES & $1.38$ & $(2,3)$ & NO & 3276\\
$(71,27)$ & 9 & $(66,25)$ & 9 & 1 & YES & YES & NO(2) & $1.44$ & $(2,3)$ & NO & 3277\\
$(71,26)$ & 9 & $(68,25)$ & 9 & 1 & YES & YES & NO(2) & $1.44$ & $(2,3)$ & NO & 3278\\
$(71,30)$ & 9 & $(71,30)$ & 9 & 71 & YES & YES & NO(2) & $1.44$ & $(2,3)$ & NO & 3279\\
$(72,19)$ & 10 & $(8,3)$ & 4 & 8 & YES & YES & YES & $1.29$ & $(2,3)$ & -- & 3280\\
$(73,30)$ & 10 & $(5,1)$ & 4 & 1 & YES & YES & NO(2) & $1.44$ & $(2,3)$ & -- & 3281\\
$(73,27)$ & 9 & $(7,3)$ & 4 & 1 & YES & YES & YES & $1.50$ & $(2,3)$ & -- & 3282\\
$(73,31)$ & 10 & $(7,2)$ & 4 & 1 & YES & YES & YES & $1.29$ & $(2,3)$ & -- & 3283\\
$(73,33)$ & 10 & $(7,2)$ & 4 & 1 & YES & YES & YES & $1.43$ & $(2,3)$ & -- & 3284\\
$(73,28)$ & 10 & $(8,3)$ & 4 & 1 & YES & YES & YES & $1.57$ & $(2,3)$ & -- & 3285\\
$(73,26)$ & 11 & $(9,4)$ & 5 & 1 & YES & YES & YES & $1.71$ & $(2,3)$ & -- & 3286\\
$(73,26)$ & 11 & $(9,4)$ & 5 & 1 & YES & YES & YES & $1.88$ & $(2,3)$ & NO & 3287\\
$(73,27)$ & 9 & $(9,4)$ & 5 & 1 & YES & YES & YES & $1.50$ & $(2,3)$ & NO & 3288\\
$(73,27)$ & 9 & $(10,3)$ & 5 & 1 & YES & YES & NO(2) & $1.44$ & $(2,3)$ & -- & 3289\\
$(73,13)$ & 10 & $(11,5)$ & 6 & 1 & YES & YES & YES & $1.43$ & $(2,3)$ & NO & 3290\\
$(73,13)$ & 10 & $(11,5)$ & 6 & 1 & YES & YES & YES & $1.43$ & $(2,3)$ & -- & 3291\\
$(73,26)$ & 11 & $(11,5)$ & 6 & 1 & YES & YES & YES & $1.71$ & $(2,3)$ & NO & 3292\\
$(73,30)$ & 10 & $(16,7)$ & 6 & 1 & YES & YES & NO(2) & $1.38$ & $(4,2)$ & NO & 3293\\
$(73,27)$ & 9 & $(43,16)$ & 9 & 1 & YES & YES & YES & $1.50$ & $(2,3)$ & NO & 3294\\
$(74,31)$ & 9 & $(10,3)$ & 5 & 2 & YES & YES & NO(2) & $1.44$ & $(2,3)$ & NO & 3295\\
$(74,23)$ & 10 & $(11,3)$ & 5 & 1 & YES & YES & NO(2) & $1.44$ & $(2,3)$ & NO & 3296\\
$(74,17)$ & 11 & $(13,4)$ & 6 & 1 & YES & YES & YES & $1.43$ & $(2,3)$ & NO & 3297\\
$(74,31)$ & 9 & $(13,3)$ & 6 & 1 & YES & YES & YES & $1.62$ & $(2,3)$ & NO & 3298\\
$(74,31)$ & 9 & $(13,3)$ & 6 & 1 & YES & YES & YES & $1.62$ & $(2,3)$ & -- & 3299\\
$(74,29)$ & 10 & $(21,8)$ & 6 & 1 & YES & YES & YES & $1.57$ & $(2,3)$ & NO & 3300\\
$(74,23)$ & 10 & $(23,7)$ & 7 & 1 & YES & YES & NO(2) & $1.44$ & $(2,3)$ & NO & 3301\\
$(74,31)$ & 9 & $(33,14)$ & 8 & 1 & YES & YES & NO(2) & $1.44$ & $(2,3)$ & 3128 & 3302\\
$(75,31)$ & 9 & $(4,1)$ & 3 & 1 & YES & YES & YES & $1.50$ & $(2,3)$ & NO & 3303\\
$(75,31)$ & 9 & $(4,1)$ & 3 & 1 & YES & YES & YES & $1.50$ & $(2,3)$ & -- & 3304\\
$(75,31)$ & 9 & $(4,1)$ & 3 & 1 & YES & YES & YES & $1.50$ & $(2,3)$ & NO & 3305\\
$(75,31)$ & 9 & $(9,4)$ & 5 & 3 & YES & YES & YES & $1.71$ & $(2,3)$ & -- & 3306\\
$(75,17)$ & 10 & $(11,4)$ & 5 & 1 & YES & YES & NO(2) & $1.25$ & $(4,2)$ & NO & 3307\\
$(75,31)$ & 9 & $(11,5)$ & 6 & 1 & YES & YES & YES & $1.43$ & $(2,3)$ & 2988 & 3308\\
$(75,17)$ & 10 & $(13,4)$ & 6 & 1 & YES & YES & NO(2) & $1.25$ & $(4,2)$ & NO & 3309\\
$(75,29)$ & 9 & $(13,4)$ & 6 & 1 & YES & YES & YES & $1.57$ & $(2,3)$ & -- & 3310\\
$(75,31)$ & 9 & $(18,5)$ & 6 & 3 & YES & YES & YES & $1.57$ & $(2,3)$ & NO & 3311\\
$(75,31)$ & 9 & $(18,5)$ & 6 & 3 & YES & YES & YES & $1.71$ & $(2,3)$ & -- & 3312\\
$(75,31)$ & 9 & $(27,11)$ & 8 & 3 & YES & YES & YES & $1.29$ & $(4,2)$ & NO & 3313\\
$(75,17)$ & 10 & $(43,10)$ & 9 & 1 & YES & YES & NO(2) & $1.25$ & $(4,2)$ & NO & 3314\\
$(75,31)$ & 9 & $(53,22)$ & 9 & 1 & YES & YES & NO(2) & $1.44$ & $(2,3)$ & NO & 3315\\
$(75,17)$ & 10 & $(61,14)$ & 10 & 1 & YES & YES & NO(2) & $1.25$ & $(4,2)$ & NO & 3316\\
$(75,31)$ & 9 & $(65,27)$ & 10 & 5 & YES & YES & YES & $1.71$ & $(2,3)$ & 4091 & 3317\\
$(76,31)$ & 10 & $(5,2)$ & 3 & 1 & YES & YES & YES & $1.43$ & $(2,3)$ & -- & 3318\\
$(76,29)$ & 9 & $(9,4)$ & 5 & 1 & YES & YES & YES & $1.43$ & $(2,3)$ & -- & 3319\\
$(76,21)$ & 9 & $(17,7)$ & 6 & 1 & YES & YES & YES & $1.57$ & $(2,3)$ & -- & 3320\\
$(76,29)$ & 9 & $(17,5)$ & 6 & 1 & YES & YES & YES & $1.86$ & $(2,3)$ & -- & 3321\\
$(76,21)$ & 9 & $(18,7)$ & 6 & 2 & YES & YES & YES & $1.57$ & $(2,3)$ & NO & 3322\\
$(76,21)$ & 9 & $(18,7)$ & 6 & 2 & YES & YES & YES & $1.57$ & $(2,3)$ & -- & 3323\\
$(76,29)$ & 9 & $(19,5)$ & 7 & 19 & YES & YES & YES & $1.57$ & $(2,3)$ & NO & 3324\\
$(76,29)$ & 9 & $(26,5)$ & 9 & 2 & YES & YES & YES & $1.57$ & $(2,3)$ & NO & 3325\\
$(76,21)$ & 9 & $(37,11)$ & 8 & 1 & YES & YES & YES & $1.57$ & $(2,3)$ & NO & 3326\\
$(77,34)$ & 10 & $(5,2)$ & 3 & 1 & YES & YES & YES & $1.29$ & $(4,2)$ & -- & 3327\\
$(77,34)$ & 10 & $(7,2)$ & 4 & 7 & YES & YES & YES & $1.43$ & $(2,3)$ & -- & 3328\\
$(77,16)$ & 11 & $(15,4)$ & 6 & 1 & YES & YES & YES & $1.57$ & $(2,3)$ & -- & 3329\\
$(77,16)$ & 11 & $(51,11)$ & 9 & 1 & YES & YES & YES & $1.57$ & $(2,3)$ & NO & 3330\\
$(77,34)$ & 10 & $(59,26)$ & 9 & 1 & YES & YES & YES & $1.14$ & $(4,2)$ & NO & 3331\\
$(78,23)$ & 10 & $(7,3)$ & 4 & 1 & YES & YES & YES & $1.57$ & $(2,3)$ & NO & 3332\\
$(78,23)$ & 10 & $(7,3)$ & 4 & 1 & YES & YES & YES & $1.57$ & $(2,3)$ & -- & 3333\\
$(78,23)$ & 10 & $(9,4)$ & 5 & 3 & YES & YES & YES & $1.43$ & $(2,3)$ & -- & 3334\\
$(78,17)$ & 10 & $(43,10)$ & 9 & 1 & YES & YES & YES & $1.43$ & $(2,3)$ & NO & 3335\\
$(78,29)$ & 10 & $(62,23)$ & 9 & 2 & YES & YES & YES & $1.62$ & $(2,3)$ & NO & 3336\\
$(79,30)$ & 9 & $(4,1)$ & 3 & 1 & YES & YES & NO(2) & $1.33$ & $(2,3)$ & -- & 3337\\
$(79,30)$ & 9 & $(5,2)$ & 3 & 1 & YES & YES & YES & $1.43$ & $(2,3)$ & -- & 3338\\
$(79,24)$ & 10 & $(7,3)$ & 4 & 1 & YES & YES & NO(2) & $1.25$ & $(4,2)$ & -- & 3339\\
$(79,28)$ & 10 & $(7,2)$ & 4 & 1 & YES & YES & YES & $1.43$ & $(2,3)$ & -- & 3340\\
$(79,29)$ & 9 & $(7,3)$ & 4 & 1 & YES & YES & YES & $1.57$ & $(2,3)$ & -- & 3341\\
$(79,30)$ & 9 & $(7,2)$ & 4 & 1 & YES & YES & YES & $1.29$ & $(2,3)$ & -- & 3342\\
$(79,30)$ & 9 & $(10,3)$ & 5 & 1 & YES & YES & NO(2) & $1.44$ & $(2,3)$ & NO & 3343\\
$(79,17)$ & 11 & $(13,4)$ & 6 & 1 & YES & YES & YES & $1.43$ & $(2,3)$ & NO & 3344\\
$(79,23)$ & 10 & $(13,3)$ & 6 & 1 & YES & YES & YES & $1.29$ & $(4,2)$ & -- & 3345\\
$(79,23)$ & 10 & $(13,5)$ & 5 & 1 & YES & YES & YES & $1.43$ & $(2,3)$ & NO & 3346\\
$(79,28)$ & 10 & $(13,5)$ & 5 & 1 & YES & YES & YES & $1.43$ & $(2,3)$ & NO & 3347\\
$(79,17)$ & 11 & $(17,3)$ & 7 & 1 & YES & YES & YES & $1.14$ & $(4,2)$ & NO & 3348\\
$(79,30)$ & 9 & $(18,5)$ & 6 & 1 & YES & YES & YES & $1.57$ & $(2,3)$ & -- & 3349\\
$(79,30)$ & 9 & $(18,7)$ & 6 & 1 & YES & YES & YES & $1.43$ & $(2,3)$ & 3782 & 3350\\
$(79,30)$ & 9 & $(23,5)$ & 7 & 1 & YES & YES & YES & $1.71$ & $(2,3)$ & -- & 3351\\
$(79,30)$ & 9 & $(28,11)$ & 8 & 1 & YES & YES & YES & $1.43$ & $(2,3)$ & NO & 3352\\
$(79,30)$ & 9 & $(34,13)$ & 7 & 1 & YES & YES & YES & $1.29$ & $(2,3)$ & NO & 3353\\
$(79,30)$ & 9 & $(45,17)$ & 9 & 1 & YES & YES & NO(2) & $1.44$ & $(2,3)$ & NO & 3354\\
$(79,29)$ & 9 & $(68,25)$ & 9 & 1 & YES & YES & NO(2) & $1.44$ & $(2,3)$ & NO & 3355\\
$(79,30)$ & 9 & $(71,27)$ & 9 & 1 & YES & YES & NO(2) & $1.44$ & $(2,3)$ & NO & 3356\\
$(79,30)$ & 9 & $(79,30)$ & 9 & 79 & YES & YES & NO(2) & $1.44$ & $(2,3)$ & NO & 3357\\
$(80,31)$ & 9 & $(8,3)$ & 4 & 8 & YES & YES & YES & $1.43$ & $(2,3)$ & -- & 3358\\
$(80,31)$ & 9 & $(9,4)$ & 5 & 1 & YES & YES & YES & $1.57$ & $(2,3)$ & -- & 3359\\
$(80,31)$ & 9 & $(10,3)$ & 5 & 10 & YES & YES & YES & $1.50$ & $(2,3)$ & -- & 3360\\
$(80,33)$ & 10 & $(10,3)$ & 5 & 10 & YES & YES & YES & $1.57$ & $(2,3)$ & NO & 3361\\
$(80,33)$ & 10 & $(10,3)$ & 5 & 10 & YES & YES & YES & $1.57$ & $(2,3)$ & -- & 3362\\
$(80,31)$ & 9 & $(12,5)$ & 5 & 4 & YES & YES & YES & $1.71$ & $(2,3)$ & -- & 3363\\
$(80,31)$ & 9 & $(15,4)$ & 6 & 5 & YES & YES & YES & $1.71$ & $(2,3)$ & NO & 3364\\
$(80,31)$ & 9 & $(15,4)$ & 6 & 5 & YES & YES & YES & $1.71$ & $(2,3)$ & -- & 3365\\
$(80,19)$ & 11 & $(17,4)$ & 7 & 1 & YES & YES & YES & $1.43$ & $(2,3)$ & -- & 3366\\
$(80,31)$ & 9 & $(18,5)$ & 6 & 2 & YES & YES & YES & $1.71$ & $(2,3)$ & NO & 3367\\
$(80,33)$ & 10 & $(26,11)$ & 7 & 2 & YES & YES & YES & $1.57$ & $(2,3)$ & NO & 3368\\
$(80,31)$ & 9 & $(59,23)$ & 9 & 1 & YES & YES & YES & $1.57$ & $(2,3)$ & NO & 3369\\
$(80,33)$ & 10 & $(70,29)$ & 9 & 10 & YES & YES & YES & $1.57$ & $(2,3)$ & 4277 & 3370\\
$(81,25)$ & 12 & $(7,3)$ & 4 & 1 & YES & YES & YES & $1.71$ & $(4,2)$ & NO & 3371\\
$(81,25)$ & 12 & $(7,3)$ & 4 & 1 & YES & YES & YES & $1.71$ & $(4,2)$ & -- & 3372\\
$(81,19)$ & 11 & $(13,4)$ & 6 & 1 & YES & YES & YES & $1.43$ & $(2,3)$ & NO & 3373\\
$(81,19)$ & 11 & $(25,4)$ & 9 & 1 & YES & YES & YES & $1.43$ & $(2,3)$ & NO & 3374\\
$(82,31)$ & 10 & $(7,3)$ & 4 & 1 & YES & YES & YES & $1.57$ & $(2,3)$ & -- & 3375\\
$(82,17)$ & 11 & $(11,3)$ & 5 & 1 & YES & YES & YES & $1.43$ & $(2,3)$ & -- & 3376\\
$(82,23)$ & 10 & $(17,5)$ & 6 & 1 & YES & YES & YES & $1.57$ & $(2,3)$ & -- & 3377\\
$(82,31)$ & 10 & $(18,7)$ & 6 & 2 & YES & YES & NO(2) & $1.56$ & $(2,3)$ & NO & 3378\\
$(82,29)$ & 11 & $(19,7)$ & 6 & 1 & YES & YES & YES & $1.57$ & $(2,3)$ & NO & 3379\\
$(82,23)$ & 10 & $(23,7)$ & 7 & 1 & YES & YES & YES & $1.43$ & $(2,3)$ & NO & 3380\\
$(82,31)$ & 10 & $(23,9)$ & 7 & 1 & YES & YES & YES & $1.62$ & $(2,3)$ & NO & 3381\\
$(82,23)$ & 10 & $(69,19)$ & 9 & 1 & YES & YES & YES & $1.57$ & $(2,3)$ & NO & 3382\\
$(82,23)$ & 10 & $(76,21)$ & 9 & 2 & YES & YES & YES & $1.57$ & $(2,3)$ & NO & 3383\\
$(83,22)$ & 10 & $(3,1)$ & 2 & 1 & YES & YES & YES & $1.29$ & $(2,3)$ & NO & 3384\\
$(83,22)$ & 10 & $(3,1)$ & 2 & 1 & YES & YES & YES & $1.29$ & $(2,3)$ & -- & 3385\\
$(83,34)$ & 10 & $(5,1)$ & 4 & 1 & YES & YES & NO(2) & $1.44$ & $(2,3)$ & -- & 3386\\
$(83,36)$ & 10 & $(5,2)$ & 3 & 1 & YES & YES & NO(2) & $1.56$ & $(2,3)$ & -- & 3387\\
$(83,22)$ & 10 & $(7,3)$ & 4 & 1 & YES & YES & NO(2) & $1.44$ & $(2,3)$ & NO & 3388\\
$(83,24)$ & 11 & $(7,2)$ & 4 & 1 & YES & YES & YES & $1.29$ & $(2,3)$ & -- & 3389\\
$(83,34)$ & 10 & $(8,3)$ & 4 & 1 & YES & YES & NO(2) & $1.38$ & $(4,2)$ & NO & 3390\\
$(83,13)$ & 11 & $(11,5)$ & 6 & 1 & YES & YES & YES & $1.57$ & $(2,3)$ & NO & 3391\\
$(83,13)$ & 11 & $(11,5)$ & 6 & 1 & YES & YES & YES & $1.57$ & $(2,3)$ & -- & 3392\\
$(83,22)$ & 10 & $(72,19)$ & 10 & 1 & YES & YES & NO(3) & $1.29$ & $(2,3)$ & NO & 3393\\
$(83,34)$ & 10 & $(83,34)$ & 10 & 83 & YES & YES & NO(2) & $1.56$ & $(2,3)$ & NO & 3394\\
$(84,25)$ & 10 & $(17,4)$ & 7 & 1 & YES & YES & YES & $1.43$ & $(2,3)$ & NO & 3395\\
$(84,25)$ & 10 & $(38,11)$ & 9 & 2 & YES & YES & YES & $1.43$ & $(2,3)$ & NO & 3396\\
$(85,36)$ & 10 & $(7,3)$ & 4 & 1 & YES & YES & YES & $1.57$ & $(2,3)$ & -- & 3397\\
$(85,16)$ & 12 & $(8,3)$ & 4 & 1 & YES & YES & NO(2) & $1.44$ & $(2,3)$ & -- & 3398\\
$(85,16)$ & 12 & $(28,5)$ & 8 & 1 & YES & YES & NO(2) & $1.44$ & $(2,3)$ & NO & 3399\\
$(85,23)$ & 10 & $(37,10)$ & 8 & 1 & YES & YES & NO(2) & $1.44$ & $(2,3)$ & NO & 3400\\
$(85,23)$ & 10 & $(56,15)$ & 9 & 1 & YES & YES & YES & $1.50$ & $(2,3)$ & 3503 & 3401\\
$(86,35)$ & 11 & $(5,2)$ & 3 & 1 & YES & YES & YES & $1.43$ & $(2,3)$ & -- & 3402\\
$(86,25)$ & 10 & $(9,4)$ & 5 & 1 & YES & YES & YES & $1.43$ & $(2,3)$ & NO & 3403\\
$(86,35)$ & 11 & $(9,2)$ & 5 & 1 & YES & YES & YES & $1.57$ & $(2,3)$ & NO & 3404\\
$(86,35)$ & 11 & $(9,2)$ & 5 & 1 & YES & YES & YES & $1.57$ & $(2,3)$ & -- & 3405\\
$(86,25)$ & 10 & $(13,3)$ & 6 & 1 & YES & YES & NO(2) & $1.44$ & $(2,3)$ & NO & 3406\\
$(86,25)$ & 10 & $(17,4)$ & 7 & 1 & YES & YES & YES & $1.43$ & $(2,3)$ & NO & 3407\\
$(86,35)$ & 11 & $(39,16)$ & 8 & 1 & YES & YES & YES & $1.57$ & $(2,3)$ & NO & 3408\\
$(86,35)$ & 11 & $(49,20)$ & 9 & 1 & YES & YES & YES & $1.43$ & $(2,3)$ & NO & 3409\\
$(87,23)$ & 10 & $(3,1)$ & 2 & 3 & YES & YES & YES & $1.29$ & $(2,3)$ & NO & 3410\\
$(87,23)$ & 10 & $(3,1)$ & 2 & 3 & YES & YES & YES & $1.29$ & $(2,3)$ & -- & 3411\\
$(87,23)$ & 10 & $(3,1)$ & 2 & 3 & YES & YES & YES & $1.29$ & $(2,3)$ & NO & 3412\\
$(87,31)$ & 12 & $(7,3)$ & 4 & 1 & YES & YES & YES & $1.71$ & $(2,3)$ & -- & 3413\\
$(87,31)$ & 12 & $(7,3)$ & 4 & 1 & YES & YES & YES & $1.88$ & $(2,3)$ & NO & 3414\\
$(87,34)$ & 10 & $(7,3)$ & 4 & 1 & YES & YES & NO(2) & $1.56$ & $(2,3)$ & NO & 3415\\
$(87,31)$ & 12 & $(9,4)$ & 5 & 3 & YES & YES & YES & $1.71$ & $(2,3)$ & NO & 3416\\
$(87,23)$ & 10 & $(10,3)$ & 5 & 1 & YES & YES & NO(2) & $1.44$ & $(2,3)$ & NO & 3417\\
$(87,23)$ & 10 & $(26,7)$ & 7 & 1 & YES & YES & NO(2) & $1.44$ & $(2,3)$ & NO & 3418\\
$(87,34)$ & 10 & $(28,11)$ & 8 & 1 & YES & YES & NO(2) & $1.56$ & $(2,3)$ & NO & 3419\\
$(87,32)$ & 10 & $(41,15)$ & 8 & 1 & YES & YES & NO(2) & $1.38$ & $(4,2)$ & NO & 3420\\
$(88,23)$ & 11 & $(7,2)$ & 4 & 1 & YES & YES & YES & $1.43$ & $(2,3)$ & -- & 3421\\
$(89,25)$ & 10 & $(3,1)$ & 2 & 1 & YES & YES & YES & $1.43$ & $(2,3)$ & -- & 3422\\
$(89,35)$ & 11 & $(3,1)$ & 2 & 1 & YES & YES & YES & $1.43$ & $(2,3)$ & -- & 3423\\
$(89,25)$ & 10 & $(4,1)$ & 3 & 1 & YES & YES & YES & $1.43$ & $(2,3)$ & -- & 3424\\
$(89,34)$ & 9 & $(5,2)$ & 3 & 1 & YES & YES & YES & $1.38$ & $(2,3)$ & -- & 3425\\
$(89,24)$ & 10 & $(7,3)$ & 4 & 1 & YES & YES & NO(2) & $1.25$ & $(4,2)$ & -- & 3426\\
$(89,24)$ & 10 & $(13,4)$ & 6 & 1 & YES & YES & NO(2) & $1.25$ & $(4,2)$ & NO & 3427\\
$(89,25)$ & 10 & $(18,5)$ & 6 & 1 & YES & YES & YES & $1.43$ & $(2,3)$ & 3625 & 3428\\
$(89,34)$ & 9 & $(29,11)$ & 7 & 1 & YES & YES & YES & $1.38$ & $(2,3)$ & NO & 3429\\
$(89,33)$ & 10 & $(35,13)$ & 8 & 1 & YES & YES & NO(2) & $1.44$ & $(2,3)$ & 3501 & 3430\\
$(89,17)$ & 12 & $(38,7)$ & 9 & 1 & YES & YES & NO(2) & $1.44$ & $(2,3)$ & NO & 3431\\
$(89,25)$ & 10 & $(57,16)$ & 9 & 1 & YES & YES & YES & $1.43$ & $(2,3)$ & NO & 3432\\
$(89,25)$ & 10 & $(89,25)$ & 10 & 89 & YES & YES & YES & $1.43$ & $(2,3)$ & NO & 3433\\
$(89,26)$ & 10 & $(89,26)$ & 10 & 89 & YES & YES & YES & $1.50$ & $(2,3)$ & NO & 3434\\
$(91,25)$ & 10 & $(2,1)$ & 1 & 1 & YES & YES & NO(2) & $1.56$ & $(2,3)$ & NO & 3435\\
$(91,19)$ & 11 & $(5,2)$ & 3 & 1 & YES & YES & NO(3) & $1.14$ & $(2,3)$ & -- & 3436\\
$(91,25)$ & 10 & $(5,1)$ & 4 & 1 & YES & YES & NO(2) & $1.44$ & $(2,3)$ & NO & 3437\\
$(91,25)$ & 10 & $(5,1)$ & 4 & 1 & YES & YES & NO(2) & $1.44$ & $(2,3)$ & -- & 3438\\
$(91,40)$ & 10 & $(5,2)$ & 3 & 1 & YES & YES & YES & $1.43$ & $(2,3)$ & -- & 3439\\
$(91,25)$ & 10 & $(7,2)$ & 4 & 7 & YES & YES & NO(2) & $1.56$ & $(2,3)$ & NO & 3440\\
$(91,27)$ & 10 & $(10,3)$ & 5 & 1 & YES & YES & YES & $1.50$ & $(2,3)$ & -- & 3441\\
$(91,27)$ & 10 & $(12,5)$ & 5 & 1 & YES & YES & YES & $1.57$ & $(2,3)$ & -- & 3442\\
$(91,27)$ & 10 & $(12,5)$ & 5 & 1 & YES & YES & YES & $1.71$ & $(2,3)$ & NO & 3443\\
$(91,27)$ & 10 & $(13,5)$ & 5 & 13 & YES & YES & YES & $1.71$ & $(2,3)$ & -- & 3444\\
$(91,27)$ & 10 & $(13,5)$ & 5 & 13 & YES & YES & YES & $1.86$ & $(2,3)$ & NO & 3445\\
$(91,17)$ & 12 & $(20,3)$ & 8 & 1 & YES & YES & NO(3) & $1.29$ & $(2,3)$ & NO & 3446\\
$(91,17)$ & 12 & $(23,4)$ & 8 & 1 & YES & YES & NO(2) & $1.44$ & $(2,3)$ & NO & 3447\\
$(91,17)$ & 12 & $(28,5)$ & 8 & 7 & YES & YES & NO(2) & $1.44$ & $(2,3)$ & NO & 3448\\
$(91,25)$ & 10 & $(40,11)$ & 8 & 1 & YES & YES & NO(2) & $1.44$ & $(2,3)$ & NO & 3449\\
$(91,17)$ & 12 & $(65,12)$ & 10 & 13 & YES & YES & NO(3) & $1.29$ & $(2,3)$ & NO & 3450\\
$(92,35)$ & 10 & $(4,1)$ & 3 & 4 & YES & YES & NO(2) & $1.38$ & $(4,2)$ & -- & 3451\\
$(92,33)$ & 10 & $(5,2)$ & 3 & 1 & YES & YES & YES & $1.43$ & $(2,3)$ & -- & 3452\\
$(92,39)$ & 10 & $(5,2)$ & 3 & 1 & YES & YES & NO(2) & $1.56$ & $(2,3)$ & -- & 3453\\
$(92,21)$ & 10 & $(7,3)$ & 4 & 1 & YES & YES & YES & $1.43$ & $(2,3)$ & -- & 3454\\
$(92,33)$ & 10 & $(7,2)$ & 4 & 1 & YES & YES & NO(3) & $1.29$ & $(2,3)$ & -- & 3455\\
$(92,27)$ & 11 & $(8,3)$ & 4 & 4 & YES & YES & YES & $1.50$ & $(2,3)$ & NO & 3456\\
$(92,35)$ & 10 & $(9,2)$ & 5 & 1 & YES & YES & YES & $1.50$ & $(2,3)$ & -- & 3457\\
$(92,39)$ & 10 & $(9,2)$ & 5 & 1 & YES & YES & YES & $1.50$ & $(2,3)$ & NO & 3458\\
$(92,27)$ & 11 & $(23,7)$ & 7 & 23 & YES & YES & YES & $1.57$ & $(2,3)$ & NO & 3459\\
$(92,33)$ & 10 & $(36,13)$ & 8 & 4 & YES & YES & YES & $1.29$ & $(2,3)$ & NO & 3460\\
$(93,34)$ & 10 & $(3,1)$ & 2 & 3 & YES & YES & YES & $1.43$ & $(2,3)$ & -- & 3461\\
$(93,41)$ & 10 & $(5,2)$ & 3 & 1 & YES & YES & YES & $1.43$ & $(2,3)$ & -- & 3462\\
$(93,20)$ & 12 & $(13,2)$ & 7 & 1 & YES & YES & YES & $1.29$ & $(2,3)$ & NO & 3463\\
$(93,34)$ & 10 & $(13,5)$ & 5 & 1 & YES & YES & NO(2) & $1.56$ & $(2,3)$ & NO & 3464\\
$(93,25)$ & 10 & $(14,3)$ & 6 & 1 & YES & YES & YES & $1.43$ & $(2,3)$ & -- & 3465\\
$(93,34)$ & 10 & $(19,7)$ & 6 & 1 & YES & YES & YES & $1.43$ & $(2,3)$ & 3047 & 3466\\
$(93,41)$ & 10 & $(41,18)$ & 8 & 1 & YES & YES & YES & $1.14$ & $(4,2)$ & NO & 3467\\
$(93,25)$ & 10 & $(67,18)$ & 9 & 1 & YES & YES & YES & $1.43$ & $(2,3)$ & NO & 3468\\
$(93,38)$ & 11 & $(93,38)$ & 11 & 93 & YES & YES & NO(2) & $1.56$ & $(2,3)$ & NO & 3469\\
$(94,39)$ & 10 & $(3,1)$ & 2 & 1 & YES & YES & NO(2) & $1.56$ & $(2,3)$ & NO & 3470\\
$(94,39)$ & 10 & $(3,1)$ & 2 & 1 & YES & YES & NO(2) & $1.56$ & $(2,3)$ & -- & 3471\\
$(94,39)$ & 10 & $(5,1)$ & 4 & 1 & YES & YES & NO(2) & $1.44$ & $(2,3)$ & -- & 3472\\
$(94,39)$ & 10 & $(5,2)$ & 3 & 1 & YES & YES & NO(2) & $1.56$ & $(2,3)$ & -- & 3473\\
$(94,39)$ & 10 & $(7,2)$ & 4 & 1 & YES & YES & YES & $1.29$ & $(2,3)$ & -- & 3474\\
$(94,39)$ & 10 & $(7,3)$ & 4 & 1 & YES & YES & NO(2) & $1.56$ & $(2,3)$ & NO & 3475\\
$(94,41)$ & 10 & $(9,2)$ & 5 & 1 & YES & YES & YES & $1.38$ & $(2,3)$ & NO & 3476\\
$(94,39)$ & 10 & $(46,19)$ & 8 & 2 & YES & YES & YES & $1.43$ & $(2,3)$ & NO & 3477\\
$(95,39)$ & 10 & $(2,1)$ & 1 & 1 & YES & YES & NO(2) & $1.44$ & $(2,3)$ & -- & 3478\\
$(95,39)$ & 10 & $(3,1)$ & 2 & 1 & YES & YES & NO(2) & $1.44$ & $(2,3)$ & NO & 3479\\
$(95,39)$ & 10 & $(3,1)$ & 2 & 1 & YES & YES & NO(2) & $1.44$ & $(2,3)$ & -- & 3480\\
$(95,39)$ & 10 & $(3,1)$ & 2 & 1 & YES & YES & NO(2) & $1.44$ & $(2,3)$ & NO & 3481\\
$(95,28)$ & 11 & $(7,3)$ & 4 & 1 & YES & YES & YES & $1.57$ & $(2,3)$ & -- & 3482\\
$(95,28)$ & 11 & $(13,2)$ & 7 & 1 & YES & YES & YES & $1.43$ & $(2,3)$ & NO & 3483\\
$(95,36)$ & 10 & $(18,7)$ & 6 & 1 & YES & YES & YES & $1.43$ & $(2,3)$ & NO & 3484\\
$(95,36)$ & 10 & $(23,9)$ & 7 & 1 & YES & YES & YES & $1.43$ & $(2,3)$ & NO & 3485\\
$(95,36)$ & 10 & $(37,14)$ & 8 & 1 & YES & YES & YES & $1.29$ & $(2,3)$ & 3564 & 3486\\
$(95,39)$ & 10 & $(61,25)$ & 9 & 1 & YES & YES & NO(2) & $1.44$ & $(2,3)$ & NO & 3487\\
$(96,23)$ & 12 & $(7,2)$ & 4 & 1 & YES & YES & YES & $1.43$ & $(2,3)$ & -- & 3488\\
$(96,17)$ & 12 & $(8,3)$ & 4 & 8 & YES & YES & YES & $1.29$ & $(2,3)$ & -- & 3489\\
$(96,17)$ & 12 & $(8,3)$ & 4 & 8 & YES & YES & YES & $1.29$ & $(2,3)$ & NO & 3490\\
$(97,36)$ & 10 & $(2,1)$ & 1 & 1 & YES & YES & NO(2) & $1.56$ & $(2,3)$ & NO & 3491\\
$(97,26)$ & 10 & $(3,1)$ & 2 & 1 & YES & YES & YES & $1.29$ & $(2,3)$ & -- & 3492\\
$(97,36)$ & 10 & $(5,2)$ & 3 & 1 & YES & YES & YES & $1.43$ & $(2,3)$ & NO & 3493\\
$(97,37)$ & 10 & $(7,2)$ & 4 & 1 & YES & YES & YES & $1.57$ & $(2,3)$ & -- & 3494\\
$(97,37)$ & 10 & $(9,2)$ & 5 & 1 & YES & YES & YES & $1.57$ & $(2,3)$ & NO & 3495\\
$(97,26)$ & 10 & $(10,3)$ & 5 & 1 & YES & YES & YES & $1.29$ & $(2,3)$ & -- & 3496\\
$(97,26)$ & 10 & $(11,3)$ & 5 & 1 & YES & YES & YES & $1.50$ & $(2,3)$ & -- & 3497\\
$(97,27)$ & 11 & $(13,5)$ & 5 & 1 & YES & YES & YES & $1.71$ & $(2,3)$ & -- & 3498\\
$(97,36)$ & 10 & $(13,5)$ & 5 & 1 & YES & YES & YES & $1.43$ & $(2,3)$ & NO & 3499\\
$(97,18)$ & 11 & $(19,5)$ & 7 & 1 & YES & YES & YES & $1.57$ & $(2,3)$ & -- & 3500\\
$(97,36)$ & 10 & $(27,10)$ & 7 & 1 & YES & YES & NO(2) & $1.44$ & $(2,3)$ & 3430 & 3501\\
$(97,38)$ & 11 & $(31,12)$ & 7 & 1 & YES & YES & YES & $1.43$ & $(2,3)$ & NO & 3502\\
$(97,26)$ & 10 & $(48,13)$ & 9 & 1 & YES & YES & YES & $1.50$ & $(2,3)$ & 3401 & 3503\\
$(97,26)$ & 10 & $(56,15)$ & 9 & 1 & YES & YES & YES & $1.29$ & $(2,3)$ & NO & 3504\\
$(97,18)$ & 11 & $(69,13)$ & 11 & 1 & YES & YES & YES & $1.57$ & $(2,3)$ & NO & 3505\\
$(97,40)$ & 11 & $(80,33)$ & 10 & 1 & YES & YES & YES & $1.50$ & $(2,3)$ & NO & 3506\\
$(97,27)$ & 11 & $(89,25)$ & 10 & 1 & YES & YES & YES & $1.71$ & $(2,3)$ & NO & 3507\\
$(97,26)$ & 10 & $(97,26)$ & 10 & 97 & YES & YES & YES & $1.29$ & $(2,3)$ & NO & 3508\\
$(97,38)$ & 11 & $(97,38)$ & 11 & 97 & YES & YES & YES & $1.43$ & $(2,3)$ & NO & 3509\\
$(98,27)$ & 10 & $(2,1)$ & 1 & 2 & YES & YES & NO(2) & $1.25$ & $(4,2)$ & -- & 3510\\
$(98,41)$ & 10 & $(2,1)$ & 1 & 2 & YES & YES & NO(2) & $1.56$ & $(2,3)$ & -- & 3511\\
$(98,41)$ & 10 & $(3,1)$ & 2 & 1 & YES & YES & NO(2) & $1.56$ & $(2,3)$ & NO & 3512\\
$(98,41)$ & 10 & $(3,1)$ & 2 & 1 & YES & YES & NO(2) & $1.56$ & $(2,3)$ & -- & 3513\\
$(98,41)$ & 10 & $(5,2)$ & 3 & 1 & YES & YES & NO(2) & $1.56$ & $(2,3)$ & -- & 3514\\
$(98,41)$ & 10 & $(7,2)$ & 4 & 7 & YES & YES & YES & $1.50$ & $(2,3)$ & NO & 3515\\
$(98,41)$ & 10 & $(9,2)$ & 5 & 1 & YES & YES & YES & $1.50$ & $(2,3)$ & NO & 3516\\
$(98,27)$ & 10 & $(10,3)$ & 5 & 2 & YES & YES & YES & $1.50$ & $(2,3)$ & -- & 3517\\
$(98,43)$ & 10 & $(25,11)$ & 7 & 1 & YES & YES & YES & $1.43$ & $(2,3)$ & NO & 3518\\
$(98,41)$ & 10 & $(26,11)$ & 7 & 2 & YES & YES & YES & $1.50$ & $(2,3)$ & NO & 3519\\
$(98,41)$ & 10 & $(43,18)$ & 8 & 1 & YES & YES & NO(2) & $1.44$ & $(2,3)$ & NO & 3520\\
$(98,41)$ & 10 & $(50,21)$ & 8 & 2 & YES & YES & YES & $1.50$ & $(2,3)$ & NO & 3521\\
$(99,41)$ & 10 & $(5,2)$ & 3 & 1 & YES & YES & NO(2) & $1.25$ & $(4,2)$ & -- & 3522\\
$(99,41)$ & 10 & $(7,3)$ & 4 & 1 & YES & YES & NO(2) & $1.25$ & $(4,2)$ & 3242 & 3523\\
$(99,41)$ & 10 & $(27,11)$ & 8 & 9 & YES & YES & YES & $1.43$ & $(2,3)$ & NO & 3524\\
$(100,29)$ & 11 & $(2,1)$ & 1 & 2 & YES & YES & NO(2) & $1.14$ & $(6,1)$ & NO & 3525\\
$(100,41)$ & 10 & $(2,1)$ & 1 & 2 & YES & YES & NO(2) & $1.44$ & $(2,3)$ & -- & 3526\\
$(100,29)$ & 11 & $(3,1)$ & 2 & 1 & YES & YES & YES & $1.29$ & $(4,2)$ & NO & 3527\\
$(100,29)$ & 11 & $(3,1)$ & 2 & 1 & YES & YES & YES & $1.29$ & $(4,2)$ & -- & 3528\\
$(100,29)$ & 11 & $(5,2)$ & 3 & 5 & YES & YES & YES & $1.43$ & $(4,2)$ & -- & 3529\\
$(100,31)$ & 11 & $(5,2)$ & 3 & 5 & YES & YES & NO(2) & $1.44$ & $(2,3)$ & -- & 3530\\
$(100,37)$ & 10 & $(5,2)$ & 3 & 5 & YES & YES & NO(2) & $1.38$ & $(4,2)$ & -- & 3531\\
$(100,39)$ & 10 & $(5,2)$ & 3 & 5 & YES & YES & NO(2) & $1.44$ & $(2,3)$ & -- & 3532\\
$(100,41)$ & 10 & $(5,2)$ & 3 & 5 & YES & YES & NO(2) & $1.44$ & $(2,3)$ & -- & 3533\\
$(100,29)$ & 11 & $(8,3)$ & 4 & 4 & YES & YES & YES & $1.43$ & $(2,3)$ & -- & 3534\\
$(100,37)$ & 10 & $(13,5)$ & 5 & 1 & YES & YES & NO(2) & $1.38$ & $(4,2)$ & NO & 3535\\
$(100,29)$ & 11 & $(23,7)$ & 7 & 1 & YES & YES & YES & $1.43$ & $(2,3)$ & NO & 3536\\
$(100,39)$ & 10 & $(23,9)$ & 7 & 1 & YES & YES & NO(2) & $1.44$ & $(2,3)$ & NO & 3537\\
$(100,41)$ & 10 & $(27,11)$ & 8 & 1 & YES & YES & NO(2) & $1.44$ & $(2,3)$ & NO & 3538\\
$(100,19)$ & 12 & $(28,5)$ & 8 & 4 & YES & YES & NO(2) & $1.44$ & $(2,3)$ & NO & 3539\\
$(100,19)$ & 12 & $(38,7)$ & 9 & 2 & YES & YES & NO(2) & $1.44$ & $(2,3)$ & NO & 3540\\
$(100,29)$ & 11 & $(100,29)$ & 11 & 100 & YES & YES & YES & $1.29$ & $(4,2)$ & NO & 3541\\
$(101,39)$ & 10 & $(3,1)$ & 2 & 1 & YES & YES & NO(2) & $1.56$ & $(2,3)$ & -- & 3542\\
$(101,28)$ & 11 & $(7,3)$ & 4 & 1 & YES & YES & YES & $1.57$ & $(2,3)$ & NO & 3543\\
$(101,30)$ & 10 & $(7,3)$ & 4 & 1 & YES & YES & YES & $1.43$ & $(2,3)$ & -- & 3544\\
$(101,28)$ & 11 & $(8,3)$ & 4 & 1 & YES & YES & YES & $1.57$ & $(2,3)$ & NO & 3545\\
$(101,28)$ & 11 & $(13,4)$ & 6 & 1 & YES & YES & YES & $1.57$ & $(2,3)$ & NO & 3546\\
$(101,37)$ & 10 & $(13,5)$ & 5 & 1 & YES & YES & NO(2) & $1.25$ & $(4,2)$ & NO & 3547\\
$(101,44)$ & 10 & $(25,11)$ & 7 & 1 & YES & YES & YES & $1.38$ & $(2,3)$ & NO & 3548\\
$(101,16)$ & 13 & $(58,9)$ & 11 & 1 & YES & YES & YES & $1.43$ & $(2,3)$ & NO & 3549\\
$(101,30)$ & 10 & $(78,23)$ & 10 & 1 & YES & YES & YES & $1.57$ & $(2,3)$ & NO & 3550\\
$(102,31)$ & 11 & $(29,9)$ & 8 & 1 & YES & YES & YES & $1.50$ & $(2,3)$ & NO & 3551\\
$(103,39)$ & 10 & $(2,1)$ & 1 & 1 & YES & YES & YES & $1.43$ & $(2,3)$ & -- & 3552\\
$(103,29)$ & 11 & $(3,1)$ & 2 & 1 & YES & YES & YES & $1.43$ & $(2,3)$ & -- & 3553\\
$(103,39)$ & 10 & $(5,1)$ & 4 & 1 & YES & YES & NO(2) & $1.44$ & $(2,3)$ & -- & 3554\\
$(103,27)$ & 11 & $(7,2)$ & 4 & 1 & YES & YES & YES & $1.43$ & $(2,3)$ & -- & 3555\\
$(103,39)$ & 10 & $(7,2)$ & 4 & 1 & YES & YES & YES & $1.57$ & $(2,3)$ & -- & 3556\\
$(103,39)$ & 10 & $(7,2)$ & 4 & 1 & YES & YES & YES & $1.43$ & $(2,3)$ & NO & 3557\\
$(103,30)$ & 11 & $(8,3)$ & 4 & 1 & YES & YES & NO(2) & $1.38$ & $(4,2)$ & NO & 3558\\
$(103,37)$ & 10 & $(11,3)$ & 5 & 1 & YES & YES & YES & $1.43$ & $(2,3)$ & -- & 3559\\
$(103,29)$ & 11 & $(18,5)$ & 6 & 1 & YES & YES & YES & $1.43$ & $(2,3)$ & NO & 3560\\
$(103,37)$ & 10 & $(21,8)$ & 6 & 1 & YES & YES & YES & $1.43$ & $(2,3)$ & NO & 3561\\
$(103,29)$ & 11 & $(25,7)$ & 7 & 1 & YES & YES & YES & $1.43$ & $(2,3)$ & NO & 3562\\
$(103,27)$ & 11 & $(26,7)$ & 7 & 1 & YES & YES & YES & $1.43$ & $(2,3)$ & NO & 3563\\
$(103,39)$ & 10 & $(29,11)$ & 7 & 1 & YES & YES & YES & $1.29$ & $(2,3)$ & 3486 & 3564\\
$(103,39)$ & 10 & $(34,13)$ & 7 & 1 & YES & YES & YES & $1.29$ & $(2,3)$ & 4069 & 3565\\
$(103,39)$ & 10 & $(37,14)$ & 8 & 1 & YES & YES & NO(2) & $1.56$ & $(2,3)$ & NO & 3566\\
$(103,30)$ & 11 & $(69,20)$ & 10 & 1 & YES & YES & YES & $1.43$ & $(2,3)$ & 4435 & 3567\\
$(103,29)$ & 11 & $(71,20)$ & 10 & 1 & YES & YES & YES & $1.43$ & $(2,3)$ & NO & 3568\\
$(103,29)$ & 11 & $(103,29)$ & 11 & 103 & YES & YES & YES & $1.43$ & $(2,3)$ & NO & 3569\\
$(104,31)$ & 11 & $(2,1)$ & 1 & 2 & YES & YES & YES & $1.25$ & $(2,3)$ & -- & 3570\\
$(104,47)$ & 11 & $(3,1)$ & 2 & 1 & YES & YES & YES & $1.43$ & $(2,3)$ & -- & 3571\\
$(104,31)$ & 11 & $(67,20)$ & 11 & 1 & YES & YES & YES & $1.43$ & $(2,3)$ & NO & 3572\\
$(105,44)$ & 10 & $(3,1)$ & 2 & 3 & YES & YES & YES & $1.50$ & $(2,3)$ & NO & 3573\\
$(105,44)$ & 10 & $(3,1)$ & 2 & 3 & YES & YES & YES & $1.50$ & $(2,3)$ & -- & 3574\\
$(105,44)$ & 10 & $(3,1)$ & 2 & 3 & YES & YES & YES & $1.50$ & $(2,3)$ & NO & 3575\\
$(105,44)$ & 10 & $(7,3)$ & 4 & 7 & YES & YES & NO(2) & $1.25$ & $(4,2)$ & NO & 3576\\
$(105,31)$ & 10 & $(12,5)$ & 5 & 3 & YES & YES & YES & $1.71$ & $(2,3)$ & -- & 3577\\
$(105,31)$ & 10 & $(17,5)$ & 6 & 1 & YES & YES & YES & $1.57$ & $(2,3)$ & -- & 3578\\
$(105,31)$ & 10 & $(18,5)$ & 6 & 3 & YES & YES & YES & $1.71$ & $(2,3)$ & -- & 3579\\
$(105,32)$ & 11 & $(27,8)$ & 7 & 3 & YES & YES & YES & $1.57$ & $(2,3)$ & NO & 3580\\
$(105,31)$ & 10 & $(91,27)$ & 10 & 7 & YES & YES & YES & $1.71$ & $(2,3)$ & NO & 3581\\
$(106,31)$ & 10 & $(3,1)$ & 2 & 1 & YES & YES & YES & $1.43$ & $(2,3)$ & NO & 3582\\
$(106,31)$ & 10 & $(3,1)$ & 2 & 1 & YES & YES & YES & $1.43$ & $(2,3)$ & -- & 3583\\
$(106,41)$ & 10 & $(3,1)$ & 2 & 1 & YES & YES & NO(2) & $1.25$ & $(4,2)$ & NO & 3584\\
$(106,41)$ & 10 & $(3,1)$ & 2 & 1 & YES & YES & NO(2) & $1.25$ & $(4,2)$ & -- & 3585\\
$(106,45)$ & 11 & $(3,1)$ & 2 & 1 & YES & YES & NO(2) & $1.56$ & $(2,3)$ & NO & 3586\\
$(106,45)$ & 11 & $(3,1)$ & 2 & 1 & YES & YES & NO(2) & $1.56$ & $(2,3)$ & -- & 3587\\
$(106,45)$ & 11 & $(5,2)$ & 3 & 1 & YES & YES & YES & $1.57$ & $(2,3)$ & -- & 3588\\
$(106,31)$ & 10 & $(7,3)$ & 4 & 1 & YES & YES & YES & $1.29$ & $(2,3)$ & -- & 3589\\
$(106,31)$ & 10 & $(7,3)$ & 4 & 1 & YES & YES & YES & $1.43$ & $(2,3)$ & NO & 3590\\
$(106,23)$ & 11 & $(17,3)$ & 7 & 1 & YES & YES & YES & $1.50$ & $(2,3)$ & NO & 3591\\
$(106,41)$ & 10 & $(18,7)$ & 6 & 2 & YES & YES & NO(2) & $1.25$ & $(4,2)$ & NO & 3592\\
$(106,41)$ & 10 & $(23,9)$ & 7 & 1 & YES & YES & NO(2) & $1.25$ & $(4,2)$ & NO & 3593\\
$(106,45)$ & 11 & $(47,20)$ & 10 & 1 & YES & YES & YES & $1.57$ & $(2,3)$ & 4150 & 3594\\
$(107,41)$ & 10 & $(2,1)$ & 1 & 1 & YES & YES & NO(2) & $1.56$ & $(2,3)$ & -- & 3595\\
$(107,47)$ & 10 & $(4,1)$ & 3 & 1 & YES & YES & YES & $1.43$ & $(2,3)$ & -- & 3596\\
$(107,29)$ & 11 & $(7,2)$ & 4 & 1 & YES & YES & YES & $1.29$ & $(2,3)$ & -- & 3597\\
$(107,31)$ & 11 & $(7,2)$ & 4 & 1 & YES & YES & YES & $1.57$ & $(2,3)$ & -- & 3598\\
$(107,44)$ & 12 & $(7,3)$ & 4 & 1 & YES & YES & YES & $1.71$ & $(2,3)$ & -- & 3599\\
$(107,41)$ & 10 & $(8,3)$ & 4 & 1 & YES & YES & YES & $1.57$ & $(2,3)$ & -- & 3600\\
$(107,41)$ & 10 & $(10,3)$ & 5 & 1 & YES & YES & YES & $1.71$ & $(2,3)$ & -- & 3601\\
$(107,44)$ & 12 & $(13,5)$ & 5 & 1 & YES & YES & YES & $1.71$ & $(2,3)$ & 3650 & 3602\\
$(107,41)$ & 10 & $(18,7)$ & 6 & 1 & YES & YES & NO(2) & $1.56$ & $(2,3)$ & NO & 3603\\
$(108,41)$ & 10 & $(4,1)$ & 3 & 4 & YES & YES & NO(2) & $1.44$ & $(2,3)$ & -- & 3604\\
$(108,41)$ & 10 & $(5,1)$ & 4 & 1 & YES & YES & YES & $1.29$ & $(2,3)$ & NO & 3605\\
$(108,41)$ & 10 & $(5,1)$ & 4 & 1 & YES & YES & YES & $1.29$ & $(2,3)$ & -- & 3606\\
$(108,29)$ & 10 & $(7,3)$ & 4 & 1 & YES & YES & YES & $1.50$ & $(2,3)$ & NO & 3607\\
$(108,29)$ & 10 & $(7,3)$ & 4 & 1 & YES & YES & YES & $1.50$ & $(2,3)$ & -- & 3608\\
$(108,41)$ & 10 & $(7,2)$ & 4 & 1 & YES & YES & YES & $1.50$ & $(2,3)$ & NO & 3609\\
$(108,41)$ & 10 & $(10,3)$ & 5 & 2 & YES & YES & YES & $1.43$ & $(2,3)$ & -- & 3610\\
$(108,41)$ & 10 & $(11,2)$ & 6 & 1 & YES & YES & YES & $1.50$ & $(2,3)$ & NO & 3611\\
$(108,29)$ & 10 & $(63,17)$ & 9 & 9 & YES & YES & YES & $1.50$ & $(2,3)$ & NO & 3612\\
$(108,41)$ & 10 & $(79,30)$ & 9 & 1 & YES & YES & NO(2) & $1.44$ & $(2,3)$ & NO & 3613\\
$(108,41)$ & 10 & $(108,41)$ & 10 & 108 & YES & YES & YES & $1.43$ & $(2,3)$ & NO & 3614\\
$(109,24)$ & 12 & $(4,1)$ & 3 & 1 & YES & YES & NO(3) & $1.14$ & $(2,3)$ & -- & 3615\\
$(109,40)$ & 10 & $(4,1)$ & 3 & 1 & YES & YES & NO(2) & $1.44$ & $(2,3)$ & -- & 3616\\
$(109,48)$ & 11 & $(4,1)$ & 3 & 1 & YES & YES & NO(2) & $1.25$ & $(4,2)$ & -- & 3617\\
$(109,30)$ & 10 & $(7,3)$ & 4 & 1 & YES & YES & YES & $1.43$ & $(2,3)$ & -- & 3618\\
$(109,48)$ & 11 & $(7,3)$ & 4 & 1 & YES & YES & NO(2) & $1.38$ & $(4,2)$ & NO & 3619\\
$(109,40)$ & 10 & $(79,29)$ & 9 & 1 & YES & YES & NO(2) & $1.44$ & $(2,3)$ & NO & 3620\\
$(110,43)$ & 11 & $(6,1)$ & 5 & 2 & YES & YES & YES & $1.29$ & $(2,3)$ & NO & 3621\\
$(110,43)$ & 11 & $(110,43)$ & 11 & 110 & YES & YES & YES & $1.43$ & $(2,3)$ & NO & 3622\\
$(111,46)$ & 10 & $(4,1)$ & 3 & 1 & YES & YES & NO(2) & $1.33$ & $(2,3)$ & -- & 3623\\
$(111,34)$ & 11 & $(5,2)$ & 3 & 1 & YES & YES & YES & $1.29$ & $(4,2)$ & NO & 3624\\
$(111,31)$ & 10 & $(7,2)$ & 4 & 1 & YES & YES & YES & $1.43$ & $(2,3)$ & 3428 & 3625\\
$(111,31)$ & 10 & $(7,3)$ & 4 & 1 & YES & YES & YES & $1.43$ & $(2,3)$ & NO & 3626\\
$(111,31)$ & 10 & $(8,3)$ & 4 & 1 & YES & YES & YES & $1.43$ & $(2,3)$ & NO & 3627\\
$(111,46)$ & 10 & $(10,3)$ & 5 & 1 & YES & YES & YES & $1.71$ & $(2,3)$ & -- & 3628\\
$(111,31)$ & 10 & $(12,5)$ & 5 & 3 & YES & YES & YES & $1.57$ & $(2,3)$ & NO & 3629\\
$(111,46)$ & 10 & $(111,46)$ & 10 & 111 & YES & YES & NO(2) & $1.44$ & $(2,3)$ & NO & 3630\\
$(112,41)$ & 10 & $(7,2)$ & 4 & 7 & YES & YES & YES & $1.29$ & $(2,3)$ & -- & 3631\\
$(112,31)$ & 10 & $(9,4)$ & 5 & 1 & YES & YES & YES & $1.57$ & $(2,3)$ & -- & 3632\\
$(112,41)$ & 10 & $(10,3)$ & 5 & 2 & YES & YES & YES & $1.57$ & $(2,3)$ & -- & 3633\\
$(112,41)$ & 10 & $(18,7)$ & 6 & 2 & YES & YES & YES & $1.57$ & $(2,3)$ & NO & 3634\\
$(112,41)$ & 10 & $(27,10)$ & 7 & 1 & YES & YES & YES & $1.43$ & $(2,3)$ & 4182 & 3635\\
$(113,35)$ & 11 & $(3,1)$ & 2 & 1 & YES & YES & YES & $1.43$ & $(2,3)$ & NO & 3636\\
$(113,35)$ & 11 & $(3,1)$ & 2 & 1 & YES & YES & YES & $1.43$ & $(2,3)$ & -- & 3637\\
$(113,51)$ & 11 & $(3,1)$ & 2 & 1 & YES & YES & YES & $1.57$ & $(2,3)$ & NO & 3638\\
$(113,51)$ & 11 & $(3,1)$ & 2 & 1 & YES & YES & YES & $1.57$ & $(2,3)$ & -- & 3639\\
$(113,31)$ & 11 & $(4,1)$ & 3 & 1 & YES & YES & NO(3) & $1.14$ & $(2,3)$ & -- & 3640\\
$(113,48)$ & 11 & $(4,1)$ & 3 & 1 & YES & YES & YES & $1.43$ & $(2,3)$ & -- & 3641\\
$(113,51)$ & 11 & $(4,1)$ & 3 & 1 & YES & YES & YES & $1.43$ & $(2,3)$ & NO & 3642\\
$(113,51)$ & 11 & $(4,1)$ & 3 & 1 & YES & YES & YES & $1.43$ & $(2,3)$ & -- & 3643\\
$(113,42)$ & 11 & $(5,2)$ & 3 & 1 & YES & YES & YES & $1.43$ & $(4,2)$ & NO & 3644\\
$(113,51)$ & 11 & $(5,2)$ & 3 & 1 & YES & YES & YES & $1.43$ & $(2,3)$ & NO & 3645\\
$(113,31)$ & 11 & $(7,3)$ & 4 & 1 & YES & YES & YES & $1.50$ & $(2,3)$ & -- & 3646\\
$(113,33)$ & 11 & $(7,2)$ & 4 & 1 & YES & YES & YES & $1.57$ & $(2,3)$ & -- & 3647\\
$(113,24)$ & 11 & $(9,4)$ & 5 & 1 & YES & YES & YES & $1.57$ & $(2,3)$ & NO & 3648\\
$(113,44)$ & 12 & $(9,4)$ & 5 & 1 & YES & YES & YES & $1.71$ & $(2,3)$ & NO & 3649\\
$(113,44)$ & 12 & $(12,5)$ & 5 & 1 & YES & YES & YES & $1.71$ & $(2,3)$ & 3602 & 3650\\
$(113,31)$ & 11 & $(13,4)$ & 6 & 1 & YES & YES & YES & $1.50$ & $(2,3)$ & NO & 3651\\
$(113,42)$ & 11 & $(13,5)$ & 5 & 1 & YES & YES & YES & $1.43$ & $(2,3)$ & NO & 3652\\
$(113,48)$ & 11 & $(19,8)$ & 6 & 1 & YES & YES & YES & $1.43$ & $(2,3)$ & NO & 3653\\
$(113,24)$ & 11 & $(51,11)$ & 9 & 1 & YES & YES & YES & $1.57$ & $(2,3)$ & NO & 3654\\
$(113,24)$ & 11 & $(62,13)$ & 10 & 1 & YES & YES & YES & $1.43$ & $(2,3)$ & NO & 3655\\
$(113,51)$ & 11 & $(82,37)$ & 10 & 1 & YES & YES & YES & $1.43$ & $(2,3)$ & NO & 3656\\
$(113,33)$ & 11 & $(99,29)$ & 10 & 1 & YES & YES & YES & $1.57$ & $(2,3)$ & 4513 & 3657\\
$(113,33)$ & 11 & $(106,31)$ & 10 & 1 & YES & YES & YES & $1.29$ & $(2,3)$ & 4258 & 3658\\
$(115,42)$ & 11 & $(3,1)$ & 2 & 1 & YES & YES & NO(2) & $1.56$ & $(2,3)$ & -- & 3659\\
$(115,52)$ & 11 & $(3,1)$ & 2 & 1 & YES & YES & YES & $1.43$ & $(2,3)$ & -- & 3660\\
$(115,52)$ & 11 & $(4,1)$ & 3 & 1 & YES & YES & YES & $1.43$ & $(2,3)$ & -- & 3661\\
$(115,52)$ & 11 & $(5,2)$ & 3 & 5 & YES & YES & YES & $1.43$ & $(2,3)$ & NO & 3662\\
$(115,24)$ & 12 & $(7,2)$ & 4 & 1 & YES & YES & YES & $1.57$ & $(2,3)$ & NO & 3663\\
$(115,24)$ & 12 & $(7,2)$ & 4 & 1 & YES & YES & YES & $1.57$ & $(2,3)$ & -- & 3664\\
$(115,44)$ & 10 & $(7,3)$ & 4 & 1 & YES & YES & YES & $1.71$ & $(2,3)$ & -- & 3665\\
$(115,52)$ & 11 & $(7,3)$ & 4 & 1 & YES & YES & YES & $1.43$ & $(2,3)$ & NO & 3666\\
$(115,44)$ & 10 & $(8,3)$ & 4 & 1 & YES & YES & YES & $1.57$ & $(2,3)$ & -- & 3667\\
$(115,44)$ & 10 & $(8,3)$ & 4 & 1 & YES & YES & NO(2) & $1.25$ & $(4,2)$ & NO & 3668\\
$(115,52)$ & 11 & $(9,4)$ & 5 & 1 & YES & YES & YES & $1.43$ & $(2,3)$ & NO & 3669\\
$(115,42)$ & 11 & $(41,15)$ & 8 & 1 & YES & YES & YES & $1.29$ & $(2,3)$ & NO & 3670\\
$(116,49)$ & 10 & $(5,2)$ & 3 & 1 & YES & YES & YES & $1.57$ & $(2,3)$ & -- & 3671\\
$(116,49)$ & 10 & $(9,2)$ & 5 & 1 & YES & YES & YES & $1.43$ & $(2,3)$ & NO & 3672\\
$(116,49)$ & 10 & $(9,2)$ & 5 & 1 & YES & YES & YES & $1.43$ & $(2,3)$ & -- & 3673\\
$(116,45)$ & 10 & $(11,3)$ & 5 & 1 & YES & YES & YES & $1.71$ & $(2,3)$ & -- & 3674\\
$(117,43)$ & 10 & $(2,1)$ & 1 & 1 & YES & YES & YES & $1.43$ & $(2,3)$ & NO & 3675\\
$(117,34)$ & 11 & $(4,1)$ & 3 & 1 & YES & YES & NO(2) & $1.44$ & $(2,3)$ & -- & 3676\\
$(117,31)$ & 11 & $(7,2)$ & 4 & 1 & YES & YES & YES & $1.43$ & $(2,3)$ & NO & 3677\\
$(117,34)$ & 11 & $(9,2)$ & 5 & 9 & YES & YES & NO(2) & $1.25$ & $(4,2)$ & NO & 3678\\
$(117,34)$ & 11 & $(13,4)$ & 6 & 13 & YES & YES & NO(2) & $1.25$ & $(4,2)$ & NO & 3679\\
$(117,31)$ & 11 & $(19,5)$ & 7 & 1 & YES & YES & NO(2) & $1.25$ & $(4,2)$ & NO & 3680\\
$(117,46)$ & 12 & $(23,9)$ & 7 & 1 & YES & YES & YES & $1.43$ & $(2,3)$ & NO & 3681\\
$(117,34)$ & 11 & $(86,25)$ & 10 & 1 & YES & YES & NO(2) & $1.44$ & $(2,3)$ & NO & 3682\\
$(118,49)$ & 11 & $(3,1)$ & 2 & 1 & YES & YES & NO(2) & $1.56$ & $(2,3)$ & -- & 3683\\
$(118,49)$ & 11 & $(5,1)$ & 4 & 1 & YES & YES & YES & $1.29$ & $(2,3)$ & -- & 3684\\
$(118,33)$ & 11 & $(11,2)$ & 6 & 1 & YES & YES & YES & $1.43$ & $(2,3)$ & NO & 3685\\
$(118,35)$ & 11 & $(16,3)$ & 7 & 2 & YES & YES & YES & $1.57$ & $(2,3)$ & -- & 3686\\
$(118,35)$ & 11 & $(16,3)$ & 7 & 2 & YES & YES & YES & $1.71$ & $(2,3)$ & NO & 3687\\
$(118,49)$ & 11 & $(17,7)$ & 6 & 1 & YES & YES & NO(2) & $1.56$ & $(2,3)$ & NO & 3688\\
$(118,27)$ & 11 & $(25,6)$ & 9 & 1 & YES & YES & YES & $1.57$ & $(2,3)$ & NO & 3689\\
$(118,49)$ & 11 & $(41,17)$ & 8 & 1 & YES & YES & NO(2) & $1.44$ & $(2,3)$ & NO & 3690\\
$(119,45)$ & 11 & $(2,1)$ & 1 & 1 & YES & YES & YES & $1.29$ & $(4,2)$ & -- & 3691\\
$(119,45)$ & 11 & $(2,1)$ & 1 & 1 & YES & YES & YES & $1.43$ & $(2,3)$ & NO & 3692\\
$(119,45)$ & 11 & $(3,1)$ & 2 & 1 & YES & YES & YES & $1.43$ & $(2,3)$ & NO & 3693\\
$(119,45)$ & 11 & $(3,1)$ & 2 & 1 & YES & YES & YES & $1.43$ & $(2,3)$ & -- & 3694\\
$(119,45)$ & 11 & $(4,1)$ & 3 & 1 & YES & YES & YES & $1.29$ & $(4,2)$ & -- & 3695\\
$(119,44)$ & 10 & $(5,2)$ & 3 & 1 & YES & YES & YES & $1.43$ & $(2,3)$ & NO & 3696\\
$(119,45)$ & 11 & $(5,2)$ & 3 & 1 & YES & YES & NO(2) & $1.14$ & $(6,1)$ & NO & 3697\\
$(119,44)$ & 10 & $(7,3)$ & 4 & 7 & YES & YES & YES & $1.43$ & $(2,3)$ & NO & 3698\\
$(119,50)$ & 10 & $(7,2)$ & 4 & 7 & YES & YES & YES & $1.50$ & $(2,3)$ & -- & 3699\\
$(119,46)$ & 10 & $(8,3)$ & 4 & 1 & YES & YES & YES & $1.50$ & $(2,3)$ & NO & 3700\\
$(119,46)$ & 10 & $(10,3)$ & 5 & 1 & YES & YES & YES & $1.57$ & $(2,3)$ & -- & 3701\\
$(119,44)$ & 10 & $(11,4)$ & 5 & 1 & YES & YES & YES & $1.38$ & $(2,3)$ & NO & 3702\\
$(119,44)$ & 10 & $(13,5)$ & 5 & 1 & YES & YES & YES & $1.43$ & $(2,3)$ & NO & 3703\\
$(119,45)$ & 11 & $(18,7)$ & 6 & 1 & YES & YES & YES & $1.43$ & $(2,3)$ & NO & 3704\\
$(119,45)$ & 11 & $(21,8)$ & 6 & 7 & YES & YES & YES & $1.29$ & $(4,2)$ & NO & 3705\\
$(119,43)$ & 11 & $(47,17)$ & 9 & 1 & YES & YES & NO(2) & $1.44$ & $(2,3)$ & 3792 & 3706\\
$(120,43)$ & 11 & $(2,1)$ & 1 & 2 & YES & YES & YES & $1.29$ & $(4,2)$ & -- & 3707\\
$(120,43)$ & 11 & $(3,1)$ & 2 & 3 & YES & YES & YES & $1.29$ & $(2,3)$ & -- & 3708\\
$(120,43)$ & 11 & $(4,1)$ & 3 & 4 & YES & YES & YES & $1.29$ & $(4,2)$ & -- & 3709\\
$(120,43)$ & 11 & $(11,4)$ & 5 & 1 & YES & YES & NO(2) & $1.14$ & $(6,1)$ & NO & 3710\\
$(120,43)$ & 11 & $(39,14)$ & 8 & 3 & YES & YES & YES & $1.29$ & $(4,2)$ & NO & 3711\\
$(120,43)$ & 11 & $(53,19)$ & 9 & 1 & YES & YES & YES & $1.14$ & $(4,2)$ & NO & 3712\\
$(120,43)$ & 11 & $(120,43)$ & 11 & 120 & YES & YES & YES & $1.29$ & $(2,3)$ & NO & 3713\\
$(120,47)$ & 12 & $(120,47)$ & 12 & 120 & YES & YES & YES & $1.57$ & $(2,3)$ & NO & 3714\\
$(121,32)$ & 11 & $(2,1)$ & 1 & 1 & YES & YES & NO(2) & $1.25$ & $(4,2)$ & -- & 3715\\
$(121,46)$ & 10 & $(3,1)$ & 2 & 1 & YES & YES & YES & $1.29$ & $(2,3)$ & -- & 3716\\
$(121,35)$ & 12 & $(4,1)$ & 3 & 1 & YES & YES & YES & $1.43$ & $(2,3)$ & -- & 3717\\
$(121,32)$ & 11 & $(5,2)$ & 3 & 1 & YES & YES & YES & $1.38$ & $(2,3)$ & -- & 3718\\
$(121,32)$ & 11 & $(7,2)$ & 4 & 1 & YES & YES & YES & $1.57$ & $(2,3)$ & -- & 3719\\
$(121,50)$ & 10 & $(8,3)$ & 4 & 1 & YES & YES & YES & $1.71$ & $(2,3)$ & -- & 3720\\
$(121,46)$ & 10 & $(9,2)$ & 5 & 1 & YES & YES & YES & $1.38$ & $(2,3)$ & -- & 3721\\
$(121,32)$ & 11 & $(11,3)$ & 5 & 11 & YES & YES & YES & $1.38$ & $(2,3)$ & NO & 3722\\
$(121,34)$ & 11 & $(13,4)$ & 6 & 1 & YES & YES & YES & $1.62$ & $(2,3)$ & NO & 3723\\
$(121,36)$ & 11 & $(61,18)$ & 9 & 1 & YES & YES & YES & $1.57$ & $(2,3)$ & NO & 3724\\
$(121,46)$ & 10 & $(92,35)$ & 10 & 1 & YES & YES & YES & $1.50$ & $(2,3)$ & NO & 3725\\
$(122,51)$ & 11 & $(2,1)$ & 1 & 2 & YES & YES & NO(2) & $1.56$ & $(2,3)$ & -- & 3726\\
$(122,51)$ & 11 & $(3,1)$ & 2 & 1 & YES & YES & NO(2) & $1.56$ & $(2,3)$ & NO & 3727\\
$(122,55)$ & 11 & $(5,2)$ & 3 & 1 & YES & YES & YES & $1.43$ & $(2,3)$ & NO & 3728\\
$(122,33)$ & 11 & $(7,2)$ & 4 & 1 & YES & YES & YES & $1.57$ & $(2,3)$ & NO & 3729\\
$(122,55)$ & 11 & $(7,3)$ & 4 & 1 & YES & YES & YES & $1.43$ & $(2,3)$ & NO & 3730\\
$(122,51)$ & 11 & $(43,18)$ & 8 & 1 & YES & YES & NO(2) & $1.44$ & $(2,3)$ & NO & 3731\\
$(123,47)$ & 10 & $(7,3)$ & 4 & 1 & YES & YES & YES & $1.43$ & $(2,3)$ & NO & 3732\\
$(123,47)$ & 10 & $(8,3)$ & 4 & 1 & YES & YES & YES & $1.86$ & $(2,3)$ & -- & 3733\\
$(123,47)$ & 10 & $(8,3)$ & 4 & 1 & YES & YES & YES & $1.86$ & $(2,3)$ & NO & 3734\\
$(123,47)$ & 10 & $(23,9)$ & 7 & 1 & YES & YES & YES & $1.57$ & $(2,3)$ & NO & 3735\\
$(123,22)$ & 12 & $(62,11)$ & 10 & 1 & YES & YES & NO(2) & $1.44$ & $(2,3)$ & NO & 3736\\
$(124,23)$ & 12 & $(23,4)$ & 8 & 1 & YES & YES & NO(2) & $1.44$ & $(2,3)$ & NO & 3737\\
$(124,23)$ & 12 & $(28,5)$ & 8 & 4 & YES & YES & NO(2) & $1.44$ & $(2,3)$ & NO & 3738\\
$(125,49)$ & 11 & $(2,1)$ & 1 & 1 & YES & YES & YES & $1.29$ & $(2,3)$ & -- & 3739\\
$(125,49)$ & 11 & $(3,1)$ & 2 & 1 & YES & YES & YES & $1.29$ & $(2,3)$ & -- & 3740\\
$(125,53)$ & 11 & $(3,1)$ & 2 & 1 & YES & YES & NO(2) & $1.56$ & $(2,3)$ & -- & 3741\\
$(125,53)$ & 11 & $(4,1)$ & 3 & 1 & YES & YES & YES & $1.50$ & $(2,3)$ & -- & 3742\\
$(125,49)$ & 11 & $(5,1)$ & 4 & 5 & YES & YES & YES & $1.29$ & $(2,3)$ & -- & 3743\\
$(125,49)$ & 11 & $(8,3)$ & 4 & 1 & YES & YES & YES & $1.29$ & $(2,3)$ & NO & 3744\\
$(125,49)$ & 11 & $(33,13)$ & 9 & 1 & YES & YES & YES & $1.57$ & $(2,3)$ & NO & 3745\\
$(125,49)$ & 11 & $(51,20)$ & 9 & 1 & YES & YES & YES & $1.43$ & $(2,3)$ & NO & 3746\\
$(125,53)$ & 11 & $(92,39)$ & 10 & 1 & YES & YES & YES & $1.50$ & $(2,3)$ & NO & 3747\\
$(125,49)$ & 11 & $(125,49)$ & 11 & 125 & YES & YES & YES & $1.29$ & $(2,3)$ & NO & 3748\\
$(126,55)$ & 11 & $(2,1)$ & 1 & 2 & YES & YES & YES & $1.29$ & $(4,2)$ & -- & 3749\\
$(126,55)$ & 11 & $(3,1)$ & 2 & 3 & YES & YES & YES & $1.50$ & $(2,3)$ & NO & 3750\\
$(126,55)$ & 11 & $(3,1)$ & 2 & 3 & YES & YES & YES & $1.50$ & $(2,3)$ & -- & 3751\\
$(126,55)$ & 11 & $(126,55)$ & 11 & 126 & YES & YES & YES & $1.29$ & $(4,2)$ & NO & 3752\\
$(127,48)$ & 11 & $(2,1)$ & 1 & 1 & YES & YES & YES & $1.43$ & $(2,3)$ & -- & 3753\\
$(127,48)$ & 11 & $(3,1)$ & 2 & 1 & YES & YES & YES & $1.43$ & $(2,3)$ & NO & 3754\\
$(127,54)$ & 12 & $(5,1)$ & 4 & 1 & YES & YES & YES & $1.57$ & $(2,3)$ & NO & 3755\\
$(127,54)$ & 12 & $(6,1)$ & 5 & 1 & YES & YES & YES & $1.57$ & $(2,3)$ & NO & 3756\\
$(127,54)$ & 12 & $(19,8)$ & 6 & 1 & YES & YES & YES & $1.57$ & $(2,3)$ & 3275 & 3757\\
$(127,54)$ & 12 & $(26,11)$ & 7 & 1 & YES & YES & YES & $1.57$ & $(2,3)$ & NO & 3758\\
$(127,48)$ & 11 & $(29,11)$ & 7 & 1 & YES & YES & YES & $1.29$ & $(2,3)$ & 4067 & 3759\\
$(127,35)$ & 11 & $(39,11)$ & 9 & 1 & YES & YES & YES & $1.71$ & $(2,3)$ & NO & 3760\\
$(127,57)$ & 11 & $(78,35)$ & 10 & 1 & YES & YES & YES & $1.29$ & $(2,3)$ & NO & 3761\\
$(127,54)$ & 12 & $(87,37)$ & 11 & 1 & YES & YES & YES & $1.57$ & $(2,3)$ & NO & 3762\\
$(127,54)$ & 12 & $(127,54)$ & 12 & 127 & YES & YES & YES & $1.57$ & $(2,3)$ & NO & 3763\\
$(128,37)$ & 12 & $(4,1)$ & 3 & 4 & YES & YES & YES & $1.43$ & $(2,3)$ & -- & 3764\\
$(128,27)$ & 13 & $(5,2)$ & 3 & 1 & YES & YES & YES & $1.57$ & $(2,3)$ & -- & 3765\\
$(128,47)$ & 10 & $(5,2)$ & 3 & 1 & YES & YES & YES & $1.38$ & $(2,3)$ & -- & 3766\\
$(128,53)$ & 11 & $(5,2)$ & 3 & 1 & YES & YES & YES & $1.57$ & $(2,3)$ & -- & 3767\\
$(128,37)$ & 12 & $(17,5)$ & 6 & 1 & YES & YES & YES & $1.43$ & $(2,3)$ & NO & 3768\\
$(128,47)$ & 10 & $(18,7)$ & 6 & 2 & YES & YES & YES & $1.57$ & $(2,3)$ & NO & 3769\\
$(128,27)$ & 13 & $(23,5)$ & 7 & 1 & YES & YES & YES & $1.57$ & $(2,3)$ & NO & 3770\\
$(128,37)$ & 12 & $(31,9)$ & 8 & 1 & YES & YES & YES & $1.43$ & $(2,3)$ & 4125 & 3771\\
$(128,47)$ & 10 & $(41,15)$ & 8 & 1 & YES & YES & YES & $1.38$ & $(2,3)$ & NO & 3772\\
$(128,37)$ & 12 & $(83,24)$ & 11 & 1 & YES & YES & YES & $1.43$ & $(2,3)$ & NO & 3773\\
$(129,40)$ & 12 & $(2,1)$ & 1 & 1 & YES & YES & NO(2) & $1.56$ & $(2,3)$ & -- & 3774\\
$(129,40)$ & 12 & $(2,1)$ & 1 & 1 & YES & YES & NO(2) & $1.56$ & $(2,3)$ & NO & 3775\\
$(129,49)$ & 10 & $(2,1)$ & 1 & 1 & YES & YES & YES & $1.29$ & $(2,3)$ & -- & 3776\\
$(129,50)$ & 10 & $(2,1)$ & 1 & 1 & YES & YES & YES & $1.50$ & $(2,3)$ & -- & 3777\\
$(129,50)$ & 10 & $(3,1)$ & 2 & 3 & YES & YES & YES & $1.43$ & $(2,3)$ & NO & 3778\\
$(129,50)$ & 10 & $(3,1)$ & 2 & 3 & YES & YES & YES & $1.43$ & $(2,3)$ & -- & 3779\\
$(129,56)$ & 11 & $(3,1)$ & 2 & 3 & YES & YES & NO(2) & $1.44$ & $(2,3)$ & NO & 3780\\
$(129,40)$ & 12 & $(4,1)$ & 3 & 1 & YES & YES & NO(2) & $1.44$ & $(2,3)$ & NO & 3781\\
$(129,49)$ & 10 & $(5,2)$ & 3 & 1 & YES & YES & YES & $1.43$ & $(2,3)$ & 3350 & 3782\\
$(129,56)$ & 11 & $(5,2)$ & 3 & 1 & YES & YES & NO(2) & $1.44$ & $(2,3)$ & NO & 3783\\
$(129,49)$ & 10 & $(7,3)$ & 4 & 1 & YES & YES & YES & $1.43$ & $(2,3)$ & NO & 3784\\
$(129,28)$ & 12 & $(13,2)$ & 7 & 1 & YES & YES & YES & $1.43$ & $(2,3)$ & NO & 3785\\
$(129,49)$ & 10 & $(18,7)$ & 6 & 3 & YES & YES & YES & $1.62$ & $(2,3)$ & NO & 3786\\
$(129,50)$ & 10 & $(23,9)$ & 7 & 1 & YES & YES & YES & $1.43$ & $(2,3)$ & NO & 3787\\
$(129,49)$ & 10 & $(50,19)$ & 8 & 1 & YES & YES & YES & $1.29$ & $(2,3)$ & NO & 3788\\
$(129,49)$ & 10 & $(108,41)$ & 10 & 3 & YES & YES & YES & $1.50$ & $(2,3)$ & NO & 3789\\
$(130,47)$ & 11 & $(2,1)$ & 1 & 2 & YES & YES & NO(2) & $1.56$ & $(2,3)$ & -- & 3790\\
$(130,47)$ & 11 & $(5,2)$ & 3 & 5 & YES & YES & NO(2) & $1.56$ & $(2,3)$ & NO & 3791\\
$(130,47)$ & 11 & $(36,13)$ & 8 & 2 & YES & YES & NO(2) & $1.44$ & $(2,3)$ & 3706 & 3792\\
$(131,36)$ & 11 & $(2,1)$ & 1 & 1 & YES & YES & YES & $1.43$ & $(2,3)$ & -- & 3793\\
$(131,36)$ & 11 & $(2,1)$ & 1 & 1 & YES & YES & NO(2) & $1.25$ & $(4,2)$ & NO & 3794\\
$(131,40)$ & 11 & $(2,1)$ & 1 & 1 & YES & YES & NO(2) & $1.44$ & $(2,3)$ & NO & 3795\\
$(131,40)$ & 11 & $(2,1)$ & 1 & 1 & YES & YES & NO(2) & $1.44$ & $(2,3)$ & -- & 3796\\
$(131,50)$ & 10 & $(2,1)$ & 1 & 1 & YES & YES & YES & $1.43$ & $(2,3)$ & NO & 3797\\
$(131,40)$ & 11 & $(3,1)$ & 2 & 1 & YES & YES & NO(2) & $1.44$ & $(2,3)$ & NO & 3798\\
$(131,39)$ & 11 & $(5,2)$ & 3 & 1 & YES & YES & YES & $1.38$ & $(2,3)$ & NO & 3799\\
$(131,47)$ & 11 & $(5,1)$ & 4 & 1 & YES & YES & YES & $1.29$ & $(2,3)$ & -- & 3800\\
$(131,24)$ & 13 & $(7,2)$ & 4 & 1 & YES & YES & YES & $1.29$ & $(2,3)$ & -- & 3801\\
$(131,50)$ & 10 & $(8,3)$ & 4 & 1 & YES & YES & YES & $1.62$ & $(2,3)$ & NO & 3802\\
$(131,25)$ & 14 & $(9,4)$ & 5 & 1 & YES & YES & YES & $1.71$ & $(2,3)$ & -- & 3803\\
$(131,36)$ & 11 & $(11,3)$ & 5 & 1 & YES & YES & YES & $1.57$ & $(2,3)$ & -- & 3804\\
$(131,47)$ & 11 & $(11,4)$ & 5 & 1 & YES & YES & NO(2) & $1.25$ & $(4,2)$ & NO & 3805\\
$(133,58)$ & 11 & $(2,1)$ & 1 & 1 & YES & YES & NO(2) & $1.25$ & $(4,2)$ & -- & 3806\\
$(133,60)$ & 11 & $(5,1)$ & 4 & 1 & YES & YES & YES & $1.29$ & $(2,3)$ & -- & 3807\\
$(133,58)$ & 11 & $(7,3)$ & 4 & 7 & YES & YES & NO(2) & $1.25$ & $(4,2)$ & NO & 3808\\
$(133,60)$ & 11 & $(7,3)$ & 4 & 7 & YES & YES & YES & $1.43$ & $(2,3)$ & NO & 3809\\
$(133,51)$ & 11 & $(9,2)$ & 5 & 1 & YES & YES & YES & $1.71$ & $(2,3)$ & NO & 3810\\
$(133,60)$ & 11 & $(82,37)$ & 10 & 1 & YES & YES & YES & $1.29$ & $(2,3)$ & NO & 3811\\
$(133,30)$ & 12 & $(89,20)$ & 11 & 1 & YES & YES & YES & $1.50$ & $(2,3)$ & 4592 & 3812\\
$(133,39)$ & 11 & $(89,26)$ & 10 & 1 & YES & YES & YES & $1.57$ & $(2,3)$ & 3967 & 3813\\
$(133,60)$ & 11 & $(133,60)$ & 11 & 133 & YES & YES & YES & $1.29$ & $(2,3)$ & NO & 3814\\
$(134,59)$ & 12 & $(6,1)$ & 5 & 2 & YES & YES & YES & $1.43$ & $(2,3)$ & -- & 3815\\
$(134,55)$ & 11 & $(7,3)$ & 4 & 1 & YES & YES & NO(2) & $1.25$ & $(4,2)$ & NO & 3816\\
$(134,59)$ & 12 & $(7,1)$ & 6 & 1 & YES & YES & YES & $1.57$ & $(2,3)$ & NO & 3817\\
$(134,59)$ & 12 & $(7,1)$ & 6 & 1 & YES & YES & YES & $1.57$ & $(2,3)$ & NO & 3818\\
$(134,49)$ & 11 & $(52,19)$ & 9 & 2 & YES & YES & YES & $1.29$ & $(2,3)$ & 3893 & 3819\\
$(135,41)$ & 11 & $(3,1)$ & 2 & 3 & YES & YES & NO(2) & $1.33$ & $(2,3)$ & -- & 3820\\
$(135,41)$ & 11 & $(13,4)$ & 6 & 1 & YES & YES & NO(2) & $1.33$ & $(2,3)$ & NO & 3821\\
$(135,56)$ & 11 & $(17,7)$ & 6 & 1 & YES & YES & YES & $1.50$ & $(2,3)$ & NO & 3822\\
$(135,56)$ & 11 & $(22,9)$ & 7 & 1 & YES & YES & YES & $1.43$ & $(2,3)$ & NO & 3823\\
$(135,56)$ & 11 & $(53,22)$ & 9 & 1 & YES & YES & NO(2) & $1.44$ & $(2,3)$ & 3900 & 3824\\
$(136,59)$ & 11 & $(2,1)$ & 1 & 2 & YES & YES & NO(2) & $1.44$ & $(2,3)$ & -- & 3825\\
$(136,59)$ & 11 & $(5,2)$ & 3 & 1 & YES & YES & NO(2) & $1.44$ & $(2,3)$ & NO & 3826\\
$(136,31)$ & 11 & $(11,3)$ & 5 & 1 & YES & YES & YES & $1.43$ & $(2,3)$ & -- & 3827\\
$(136,57)$ & 11 & $(69,29)$ & 9 & 1 & YES & YES & YES & $1.71$ & $(2,3)$ & NO & 3828\\
$(137,62)$ & 12 & $(3,1)$ & 2 & 1 & YES & YES & YES & $1.57$ & $(2,3)$ & -- & 3829\\
$(137,37)$ & 11 & $(4,1)$ & 3 & 1 & YES & YES & YES & $1.43$ & $(2,3)$ & -- & 3830\\
$(137,53)$ & 11 & $(4,1)$ & 3 & 1 & YES & YES & YES & $1.57$ & $(2,3)$ & -- & 3831\\
$(137,62)$ & 12 & $(4,1)$ & 3 & 1 & YES & YES & YES & $1.57$ & $(2,3)$ & NO & 3832\\
$(137,53)$ & 11 & $(5,2)$ & 3 & 1 & YES & YES & YES & $1.57$ & $(2,3)$ & -- & 3833\\
$(137,62)$ & 12 & $(5,2)$ & 3 & 1 & YES & YES & YES & $1.57$ & $(2,3)$ & NO & 3834\\
$(137,62)$ & 12 & $(7,3)$ & 4 & 1 & YES & YES & YES & $1.57$ & $(2,3)$ & NO & 3835\\
$(137,62)$ & 12 & $(20,9)$ & 7 & 1 & YES & YES & YES & $1.57$ & $(2,3)$ & NO & 3836\\
$(137,52)$ & 11 & $(79,30)$ & 9 & 1 & YES & YES & YES & $1.50$ & $(2,3)$ & 4197 & 3837\\
$(137,32)$ & 12 & $(81,19)$ & 11 & 1 & YES & YES & YES & $1.43$ & $(2,3)$ & NO & 3838\\
$(137,52)$ & 11 & $(108,41)$ & 10 & 1 & YES & YES & YES & $1.50$ & $(2,3)$ & NO & 3839\\
$(138,41)$ & 11 & $(2,1)$ & 1 & 2 & YES & YES & NO(2) & $1.44$ & $(2,3)$ & -- & 3840\\
$(138,31)$ & 12 & $(4,1)$ & 3 & 2 & YES & YES & NO(2) & $1.33$ & $(2,3)$ & -- & 3841\\
$(138,41)$ & 11 & $(16,3)$ & 7 & 2 & YES & YES & YES & $1.71$ & $(2,3)$ & NO & 3842\\
$(138,41)$ & 11 & $(44,13)$ & 8 & 2 & YES & YES & YES & $1.50$ & $(2,3)$ & 4373 & 3843\\
$(138,41)$ & 11 & $(61,18)$ & 9 & 1 & YES & YES & YES & $1.86$ & $(2,3)$ & NO & 3844\\
$(138,37)$ & 11 & $(138,37)$ & 11 & 138 & YES & YES & NO(2) & $1.44$ & $(2,3)$ & NO & 3845\\
$(139,61)$ & 11 & $(2,1)$ & 1 & 1 & YES & YES & YES & $1.43$ & $(2,3)$ & NO & 3846\\
$(139,51)$ & 11 & $(4,1)$ & 3 & 1 & YES & YES & YES & $1.50$ & $(2,3)$ & NO & 3847\\
$(139,51)$ & 11 & $(5,2)$ & 3 & 1 & YES & YES & YES & $1.71$ & $(2,3)$ & -- & 3848\\
$(139,51)$ & 11 & $(6,1)$ & 5 & 1 & YES & YES & YES & $1.50$ & $(2,3)$ & NO & 3849\\
$(139,41)$ & 11 & $(8,3)$ & 4 & 1 & YES & YES & YES & $1.57$ & $(2,3)$ & -- & 3850\\
$(139,51)$ & 11 & $(68,25)$ & 9 & 1 & YES & YES & YES & $1.71$ & $(2,3)$ & NO & 3851\\
$(139,51)$ & 11 & $(79,29)$ & 9 & 1 & YES & YES & YES & $1.50$ & $(2,3)$ & 4206 & 3852\\
$(139,51)$ & 11 & $(139,51)$ & 11 & 139 & YES & YES & YES & $1.50$ & $(2,3)$ & NO & 3853\\
$(139,59)$ & 12 & $(139,59)$ & 12 & 139 & YES & YES & YES & $1.43$ & $(2,3)$ & NO & 3854\\
$(140,37)$ & 11 & $(5,2)$ & 3 & 5 & YES & YES & YES & $1.43$ & $(2,3)$ & -- & 3855\\
$(140,53)$ & 11 & $(5,1)$ & 4 & 5 & YES & YES & NO(3) & $1.29$ & $(2,3)$ & -- & 3856\\
$(140,53)$ & 11 & $(5,2)$ & 3 & 5 & YES & YES & YES & $1.57$ & $(2,3)$ & -- & 3857\\
$(140,41)$ & 11 & $(8,3)$ & 4 & 4 & YES & YES & YES & $1.50$ & $(2,3)$ & -- & 3858\\
$(140,53)$ & 11 & $(9,2)$ & 5 & 1 & YES & YES & YES & $1.71$ & $(2,3)$ & NO & 3859\\
$(140,37)$ & 11 & $(10,3)$ & 5 & 10 & YES & YES & YES & $1.43$ & $(2,3)$ & NO & 3860\\
$(140,53)$ & 11 & $(18,7)$ & 6 & 2 & YES & YES & YES & $1.57$ & $(2,3)$ & NO & 3861\\
$(140,53)$ & 11 & $(50,19)$ & 8 & 10 & YES & YES & YES & $1.71$ & $(2,3)$ & 4448 & 3862\\
$(141,55)$ & 11 & $(2,1)$ & 1 & 1 & YES & YES & NO(2) & $1.44$ & $(2,3)$ & -- & 3863\\
$(141,41)$ & 11 & $(3,1)$ & 2 & 3 & NO & YES & YES & $1.29$ & $(2,3)$ & -- & 3864\\
$(141,55)$ & 11 & $(3,1)$ & 2 & 3 & YES & YES & NO(3) & $1.29$ & $(2,3)$ & -- & 3865\\
$(141,59)$ & 11 & $(3,1)$ & 2 & 3 & YES & YES & YES & $1.43$ & $(2,3)$ & NO & 3866\\
$(141,59)$ & 11 & $(3,1)$ & 2 & 3 & YES & YES & YES & $1.43$ & $(2,3)$ & -- & 3867\\
$(141,41)$ & 11 & $(7,3)$ & 4 & 1 & YES & YES & YES & $1.43$ & $(2,3)$ & NO & 3868\\
$(141,41)$ & 11 & $(8,3)$ & 4 & 1 & YES & YES & YES & $1.43$ & $(2,3)$ & NO & 3869\\
$(141,55)$ & 11 & $(23,9)$ & 7 & 1 & YES & YES & NO(3) & $1.29$ & $(2,3)$ & NO & 3870\\
$(141,55)$ & 11 & $(41,16)$ & 8 & 1 & YES & YES & NO(2) & $1.44$ & $(2,3)$ & NO & 3871\\
$(142,39)$ & 11 & $(9,2)$ & 5 & 1 & YES & YES & YES & $1.38$ & $(2,3)$ & NO & 3872\\
$(143,54)$ & 12 & $(2,1)$ & 1 & 1 & YES & YES & YES & $1.43$ & $(2,3)$ & -- & 3873\\
$(143,59)$ & 11 & $(2,1)$ & 1 & 1 & YES & YES & NO(2) & $1.56$ & $(2,3)$ & -- & 3874\\
$(143,54)$ & 12 & $(3,1)$ & 2 & 1 & YES & YES & YES & $1.43$ & $(2,3)$ & NO & 3875\\
$(143,59)$ & 11 & $(63,26)$ & 9 & 1 & YES & YES & NO(2) & $1.44$ & $(2,3)$ & NO & 3876\\
$(144,61)$ & 11 & $(2,1)$ & 1 & 2 & YES & YES & YES & $1.29$ & $(4,2)$ & NO & 3877\\
$(144,61)$ & 11 & $(3,1)$ & 2 & 3 & YES & YES & YES & $1.29$ & $(2,3)$ & NO & 3878\\
$(144,59)$ & 11 & $(4,1)$ & 3 & 4 & YES & YES & NO(2) & $1.44$ & $(2,3)$ & NO & 3879\\
$(144,61)$ & 11 & $(5,2)$ & 3 & 1 & YES & YES & NO(2) & $1.44$ & $(2,3)$ & NO & 3880\\
$(144,23)$ & 14 & $(7,2)$ & 4 & 1 & YES & YES & YES & $1.29$ & $(2,3)$ & NO & 3881\\
$(144,55)$ & 10 & $(7,3)$ & 4 & 1 & YES & YES & YES & $1.43$ & $(2,3)$ & -- & 3882\\
$(144,55)$ & 10 & $(9,4)$ & 5 & 9 & YES & YES & YES & $1.71$ & $(2,3)$ & NO & 3883\\
$(144,59)$ & 11 & $(12,5)$ & 5 & 12 & YES & YES & YES & $1.50$ & $(2,3)$ & NO & 3884\\
$(144,55)$ & 10 & $(13,3)$ & 6 & 1 & YES & YES & YES & $1.71$ & $(2,3)$ & NO & 3885\\
$(144,61)$ & 11 & $(19,8)$ & 6 & 1 & YES & YES & YES & $1.43$ & $(2,3)$ & NO & 3886\\
$(144,61)$ & 11 & $(85,36)$ & 10 & 1 & YES & YES & YES & $1.29$ & $(2,3)$ & NO & 3887\\
$(144,59)$ & 11 & $(144,59)$ & 11 & 144 & YES & YES & YES & $1.50$ & $(2,3)$ & NO & 3888\\
$(145,52)$ & 11 & $(2,1)$ & 1 & 1 & YES & YES & YES & $1.29$ & $(2,3)$ & -- & 3889\\
$(145,53)$ & 11 & $(2,1)$ & 1 & 1 & YES & YES & YES & $1.43$ & $(2,3)$ & -- & 3890\\
$(145,33)$ & 13 & $(8,1)$ & 7 & 1 & YES & YES & YES & $1.43$ & $(2,3)$ & NO & 3891\\
$(145,44)$ & 11 & $(18,5)$ & 6 & 1 & YES & YES & YES & $1.71$ & $(2,3)$ & NO & 3892\\
$(145,53)$ & 11 & $(41,15)$ & 8 & 1 & YES & YES & YES & $1.29$ & $(2,3)$ & 3819 & 3893\\
$(145,34)$ & 12 & $(77,18)$ & 10 & 1 & YES & YES & YES & $1.43$ & $(2,3)$ & NO & 3894\\
$(145,64)$ & 12 & $(145,64)$ & 12 & 145 & YES & YES & YES & $1.43$ & $(2,3)$ & NO & 3895\\
$(147,61)$ & 11 & $(2,1)$ & 1 & 1 & YES & YES & NO(2) & $1.56$ & $(2,3)$ & NO & 3896\\
$(147,61)$ & 11 & $(2,1)$ & 1 & 1 & YES & YES & NO(2) & $1.56$ & $(2,3)$ & -- & 3897\\
$(147,61)$ & 11 & $(3,1)$ & 2 & 3 & YES & YES & NO(2) & $1.56$ & $(2,3)$ & NO & 3898\\
$(147,61)$ & 11 & $(5,1)$ & 4 & 1 & YES & YES & NO(2) & $1.44$ & $(2,3)$ & -- & 3899\\
$(147,61)$ & 11 & $(41,17)$ & 8 & 1 & YES & YES & NO(2) & $1.44$ & $(2,3)$ & 3824 & 3900\\
$(147,61)$ & 11 & $(53,22)$ & 9 & 1 & YES & YES & NO(2) & $1.56$ & $(2,3)$ & NO & 3901\\
$(147,62)$ & 11 & $(83,35)$ & 10 & 1 & YES & YES & YES & $1.50$ & $(2,3)$ & NO & 3902\\
$(148,43)$ & 12 & $(2,1)$ & 1 & 2 & YES & YES & NO(2) & $1.56$ & $(2,3)$ & NO & 3903\\
$(148,43)$ & 12 & $(4,1)$ & 3 & 4 & YES & YES & NO(2) & $1.56$ & $(2,3)$ & NO & 3904\\
$(148,43)$ & 12 & $(7,1)$ & 6 & 1 & YES & YES & NO(2) & $1.44$ & $(2,3)$ & NO & 3905\\
$(148,43)$ & 12 & $(31,9)$ & 8 & 1 & YES & YES & NO(2) & $1.56$ & $(2,3)$ & NO & 3906\\
$(148,43)$ & 12 & $(86,25)$ & 10 & 2 & YES & YES & NO(2) & $1.44$ & $(2,3)$ & 4268 & 3907\\
$(148,65)$ & 11 & $(148,65)$ & 11 & 148 & YES & YES & YES & $1.57$ & $(2,3)$ & NO & 3908\\
$(149,55)$ & 11 & $(2,1)$ & 1 & 1 & YES & YES & NO(2) & $1.56$ & $(2,3)$ & -- & 3909\\
$(149,65)$ & 11 & $(3,1)$ & 2 & 1 & YES & YES & YES & $1.38$ & $(2,3)$ & -- & 3910\\
$(149,65)$ & 11 & $(4,1)$ & 3 & 1 & YES & YES & YES & $1.38$ & $(2,3)$ & -- & 3911\\
$(149,41)$ & 11 & $(7,3)$ & 4 & 1 & YES & YES & YES & $1.71$ & $(2,3)$ & -- & 3912\\
$(149,65)$ & 11 & $(7,3)$ & 4 & 1 & YES & YES & YES & $1.50$ & $(2,3)$ & NO & 3913\\
$(149,44)$ & 11 & $(9,2)$ & 5 & 1 & YES & YES & YES & $1.57$ & $(2,3)$ & -- & 3914\\
$(149,40)$ & 11 & $(14,3)$ & 6 & 1 & YES & YES & YES & $1.71$ & $(2,3)$ & NO & 3915\\
$(149,40)$ & 11 & $(17,5)$ & 6 & 1 & YES & YES & YES & $1.71$ & $(2,3)$ & NO & 3916\\
$(149,41)$ & 11 & $(32,9)$ & 8 & 1 & YES & YES & YES & $1.71$ & $(2,3)$ & NO & 3917\\
$(149,44)$ & 11 & $(47,14)$ & 9 & 1 & YES & YES & YES & $1.57$ & $(2,3)$ & NO & 3918\\
$(149,44)$ & 11 & $(64,19)$ & 9 & 1 & YES & YES & YES & $1.57$ & $(2,3)$ & NO & 3919\\
$(149,55)$ & 11 & $(65,24)$ & 9 & 1 & YES & YES & NO(2) & $1.44$ & $(2,3)$ & NO & 3920\\
$(149,40)$ & 11 & $(93,25)$ & 10 & 1 & YES & YES & YES & $1.43$ & $(2,3)$ & 4425 & 3921\\
$(149,65)$ & 11 & $(94,41)$ & 10 & 1 & YES & YES & YES & $1.38$ & $(2,3)$ & NO & 3922\\
$(150,59)$ & 12 & $(3,1)$ & 2 & 3 & YES & YES & YES & $1.43$ & $(2,3)$ & -- & 3923\\
$(150,59)$ & 12 & $(150,59)$ & 12 & 150 & YES & YES & YES & $1.43$ & $(2,3)$ & NO & 3924\\
$(151,56)$ & 11 & $(3,1)$ & 2 & 1 & YES & YES & NO(2) & $1.44$ & $(2,3)$ & -- & 3925\\
$(151,59)$ & 11 & $(3,1)$ & 2 & 1 & YES & YES & YES & $1.50$ & $(2,3)$ & -- & 3926\\
$(151,62)$ & 11 & $(4,1)$ & 3 & 1 & YES & YES & YES & $1.50$ & $(2,3)$ & NO & 3927\\
$(151,56)$ & 11 & $(5,2)$ & 3 & 1 & YES & YES & YES & $1.50$ & $(2,3)$ & NO & 3928\\
$(151,34)$ & 12 & $(35,8)$ & 8 & 1 & YES & YES & YES & $1.43$ & $(2,3)$ & NO & 3929\\
$(151,59)$ & 11 & $(41,16)$ & 8 & 1 & YES & YES & YES & $1.50$ & $(2,3)$ & NO & 3930\\
$(152,67)$ & 11 & $(2,1)$ & 1 & 2 & YES & YES & YES & $1.43$ & $(2,3)$ & -- & 3931\\
$(152,67)$ & 11 & $(2,1)$ & 1 & 2 & YES & YES & YES & $1.43$ & $(2,3)$ & NO & 3932\\
$(152,55)$ & 12 & $(3,1)$ & 2 & 1 & YES & YES & YES & $1.62$ & $(2,3)$ & NO & 3933\\
$(152,55)$ & 12 & $(3,1)$ & 2 & 1 & YES & YES & YES & $1.62$ & $(2,3)$ & -- & 3934\\
$(152,59)$ & 11 & $(3,1)$ & 2 & 1 & YES & YES & YES & $1.50$ & $(2,3)$ & -- & 3935\\
$(152,55)$ & 12 & $(5,2)$ & 3 & 1 & YES & YES & YES & $1.43$ & $(4,2)$ & NO & 3936\\
$(152,59)$ & 11 & $(7,2)$ & 4 & 1 & YES & YES & YES & $1.71$ & $(2,3)$ & -- & 3937\\
$(152,33)$ & 13 & $(8,1)$ & 7 & 8 & YES & YES & YES & $1.29$ & $(2,3)$ & NO & 3938\\
$(152,47)$ & 12 & $(9,2)$ & 5 & 1 & YES & YES & YES & $1.50$ & $(2,3)$ & NO & 3939\\
$(152,55)$ & 12 & $(36,13)$ & 8 & 4 & YES & YES & YES & $1.62$ & $(2,3)$ & NO & 3940\\
$(152,59)$ & 11 & $(80,31)$ & 9 & 8 & YES & YES & YES & $1.71$ & $(2,3)$ & NO & 3941\\
$(152,47)$ & 12 & $(139,43)$ & 12 & 1 & YES & YES & YES & $1.50$ & $(2,3)$ & NO & 3942\\
$(152,55)$ & 12 & $(152,55)$ & 12 & 152 & YES & YES & YES & $1.62$ & $(2,3)$ & NO & 3943\\
$(152,59)$ & 11 & $(152,59)$ & 11 & 152 & YES & YES & YES & $1.50$ & $(2,3)$ & NO & 3944\\
$(153,43)$ & 12 & $(2,1)$ & 1 & 1 & YES & YES & NO(2) & $1.56$ & $(2,3)$ & -- & 3945\\
$(153,43)$ & 12 & $(2,1)$ & 1 & 1 & YES & YES & NO(2) & $1.56$ & $(2,3)$ & NO & 3946\\
$(153,43)$ & 12 & $(3,1)$ & 2 & 3 & YES & YES & NO(2) & $1.56$ & $(2,3)$ & NO & 3947\\
$(153,41)$ & 11 & $(5,2)$ & 3 & 1 & YES & YES & YES & $1.43$ & $(2,3)$ & -- & 3948\\
$(153,56)$ & 11 & $(5,1)$ & 4 & 1 & YES & YES & YES & $1.50$ & $(2,3)$ & NO & 3949\\
$(153,43)$ & 12 & $(7,1)$ & 6 & 1 & YES & YES & NO(2) & $1.44$ & $(2,3)$ & NO & 3950\\
$(153,41)$ & 11 & $(10,3)$ & 5 & 1 & YES & YES & YES & $1.29$ & $(2,3)$ & NO & 3951\\
$(153,43)$ & 12 & $(18,5)$ & 6 & 9 & YES & YES & YES & $1.50$ & $(2,3)$ & NO & 3952\\
$(153,43)$ & 12 & $(25,7)$ & 7 & 1 & YES & YES & NO(2) & $1.44$ & $(2,3)$ & NO & 3953\\
$(153,43)$ & 12 & $(32,9)$ & 8 & 1 & YES & YES & NO(2) & $1.56$ & $(2,3)$ & NO & 3954\\
$(153,43)$ & 12 & $(57,16)$ & 9 & 3 & YES & YES & YES & $1.50$ & $(2,3)$ & NO & 3955\\
$(153,41)$ & 11 & $(67,18)$ & 9 & 1 & YES & YES & YES & $1.29$ & $(2,3)$ & NO & 3956\\
$(153,40)$ & 12 & $(88,23)$ & 11 & 1 & YES & YES & YES & $1.43$ & $(2,3)$ & NO & 3957\\
$(153,56)$ & 11 & $(153,56)$ & 11 & 153 & YES & YES & YES & $1.50$ & $(2,3)$ & NO & 3958\\
$(154,45)$ & 11 & $(2,1)$ & 1 & 2 & YES & YES & NO(2) & $1.44$ & $(2,3)$ & NO & 3959\\
$(154,65)$ & 11 & $(3,1)$ & 2 & 1 & YES & YES & YES & $1.57$ & $(2,3)$ & NO & 3960\\
$(154,65)$ & 11 & $(3,1)$ & 2 & 1 & YES & YES & YES & $1.57$ & $(2,3)$ & -- & 3961\\
$(154,65)$ & 11 & $(5,2)$ & 3 & 1 & YES & YES & YES & $1.43$ & $(2,3)$ & -- & 3962\\
$(154,45)$ & 11 & $(7,2)$ & 4 & 7 & YES & YES & NO(2) & $1.44$ & $(2,3)$ & NO & 3963\\
$(154,45)$ & 11 & $(7,3)$ & 4 & 7 & YES & YES & YES & $1.57$ & $(2,3)$ & -- & 3964\\
$(154,65)$ & 11 & $(9,4)$ & 5 & 1 & YES & YES & YES & $1.57$ & $(2,3)$ & NO & 3965\\
$(154,45)$ & 11 & $(11,3)$ & 5 & 11 & YES & YES & YES & $1.71$ & $(2,3)$ & -- & 3966\\
$(154,45)$ & 11 & $(75,22)$ & 10 & 1 & YES & YES & YES & $1.57$ & $(2,3)$ & 3813 & 3967\\
$(154,65)$ & 11 & $(154,65)$ & 11 & 154 & YES & YES & YES & $1.29$ & $(2,3)$ & NO & 3968\\
$(155,68)$ & 11 & $(3,1)$ & 2 & 1 & YES & YES & YES & $1.57$ & $(2,3)$ & -- & 3969\\
$(155,57)$ & 11 & $(4,1)$ & 3 & 1 & YES & YES & YES & $1.50$ & $(2,3)$ & -- & 3970\\
$(155,47)$ & 12 & $(5,2)$ & 3 & 5 & YES & YES & YES & $1.71$ & $(2,3)$ & NO & 3971\\
$(155,47)$ & 12 & $(5,2)$ & 3 & 5 & YES & YES & YES & $1.71$ & $(2,3)$ & -- & 3972\\
$(155,64)$ & 11 & $(5,2)$ & 3 & 5 & YES & YES & YES & $1.29$ & $(4,2)$ & -- & 3973\\
$(155,68)$ & 11 & $(5,2)$ & 3 & 5 & YES & YES & YES & $1.57$ & $(2,3)$ & NO & 3974\\
$(155,46)$ & 11 & $(7,3)$ & 4 & 1 & YES & YES & YES & $1.71$ & $(2,3)$ & -- & 3975\\
$(155,46)$ & 11 & $(8,3)$ & 4 & 1 & YES & YES & YES & $1.71$ & $(2,3)$ & -- & 3976\\
$(155,47)$ & 12 & $(11,3)$ & 5 & 1 & YES & YES & YES & $1.71$ & $(2,3)$ & NO & 3977\\
$(155,46)$ & 11 & $(84,25)$ & 10 & 1 & YES & YES & YES & $1.71$ & $(2,3)$ & NO & 3978\\
$(155,68)$ & 11 & $(98,43)$ & 10 & 1 & YES & YES & YES & $1.57$ & $(2,3)$ & NO & 3979\\
$(155,57)$ & 11 & $(155,57)$ & 11 & 155 & YES & YES & YES & $1.50$ & $(2,3)$ & NO & 3980\\
$(155,68)$ & 11 & $(155,68)$ & 11 & 155 & YES & YES & YES & $1.57$ & $(2,3)$ & NO & 3981\\
$(157,58)$ & 11 & $(2,1)$ & 1 & 1 & YES & YES & NO(2) & $1.25$ & $(4,2)$ & -- & 3982\\
$(157,42)$ & 12 & $(3,1)$ & 2 & 1 & YES & YES & YES & $1.29$ & $(2,3)$ & -- & 3983\\
$(157,58)$ & 11 & $(3,1)$ & 2 & 1 & YES & YES & YES & $1.43$ & $(2,3)$ & -- & 3984\\
$(157,36)$ & 12 & $(4,1)$ & 3 & 1 & YES & YES & NO(2) & $1.33$ & $(2,3)$ & -- & 3985\\
$(157,58)$ & 11 & $(5,2)$ & 3 & 1 & YES & YES & YES & $1.43$ & $(2,3)$ & NO & 3986\\
$(157,36)$ & 12 & $(7,2)$ & 4 & 1 & YES & YES & YES & $1.50$ & $(2,3)$ & NO & 3987\\
$(157,66)$ & 11 & $(69,29)$ & 9 & 1 & YES & YES & YES & $1.50$ & $(2,3)$ & NO & 3988\\
$(157,42)$ & 12 & $(86,23)$ & 11 & 1 & YES & YES & YES & $1.29$ & $(2,3)$ & NO & 3989\\
$(157,42)$ & 12 & $(157,42)$ & 12 & 157 & YES & YES & YES & $1.29$ & $(2,3)$ & NO & 3990\\
$(158,61)$ & 11 & $(49,19)$ & 8 & 1 & YES & YES & YES & $1.57$ & $(2,3)$ & 4431 & 3991\\
$(159,59)$ & 11 & $(3,1)$ & 2 & 3 & YES & YES & NO(2) & $1.44$ & $(2,3)$ & -- & 3992\\
$(159,62)$ & 11 & $(3,1)$ & 2 & 3 & YES & YES & YES & $1.14$ & $(4,2)$ & -- & 3993\\
$(159,47)$ & 11 & $(7,3)$ & 4 & 1 & YES & YES & YES & $1.86$ & $(2,3)$ & -- & 3994\\
$(159,47)$ & 11 & $(7,3)$ & 4 & 1 & YES & YES & YES & $1.86$ & $(2,3)$ & NO & 3995\\
$(159,59)$ & 11 & $(7,3)$ & 4 & 1 & YES & YES & YES & $1.43$ & $(2,3)$ & NO & 3996\\
$(159,62)$ & 11 & $(7,2)$ & 4 & 1 & YES & YES & YES & $1.71$ & $(2,3)$ & -- & 3997\\
$(159,47)$ & 11 & $(18,5)$ & 6 & 3 & YES & YES & YES & $1.86$ & $(2,3)$ & NO & 3998\\
$(159,47)$ & 11 & $(47,14)$ & 9 & 1 & YES & YES & YES & $1.86$ & $(2,3)$ & NO & 3999\\
$(160,67)$ & 11 & $(2,1)$ & 1 & 2 & YES & YES & YES & $1.57$ & $(2,3)$ & -- & 4000\\
$(160,49)$ & 12 & $(3,1)$ & 2 & 1 & YES & YES & YES & $1.50$ & $(2,3)$ & -- & 4001\\
$(160,43)$ & 11 & $(7,2)$ & 4 & 1 & YES & YES & YES & $1.29$ & $(2,3)$ & -- & 4002\\
$(160,49)$ & 12 & $(7,2)$ & 4 & 1 & YES & YES & YES & $1.50$ & $(2,3)$ & NO & 4003\\
$(161,61)$ & 11 & $(2,1)$ & 1 & 1 & YES & YES & YES & $1.62$ & $(2,3)$ & NO & 4004\\
$(161,66)$ & 11 & $(2,1)$ & 1 & 1 & YES & YES & YES & $1.43$ & $(2,3)$ & -- & 4005\\
$(161,37)$ & 13 & $(3,1)$ & 2 & 1 & YES & YES & YES & $1.29$ & $(2,3)$ & -- & 4006\\
$(161,66)$ & 11 & $(3,1)$ & 2 & 1 & YES & YES & YES & $1.50$ & $(2,3)$ & NO & 4007\\
$(161,66)$ & 11 & $(3,1)$ & 2 & 1 & YES & YES & YES & $1.50$ & $(2,3)$ & -- & 4008\\
$(161,71)$ & 12 & $(3,1)$ & 2 & 1 & YES & YES & YES & $1.57$ & $(2,3)$ & NO & 4009\\
$(161,71)$ & 12 & $(7,1)$ & 6 & 7 & YES & YES & YES & $1.57$ & $(2,3)$ & NO & 4010\\
$(161,71)$ & 12 & $(7,1)$ & 6 & 7 & YES & YES & YES & $1.57$ & $(2,3)$ & NO & 4011\\
$(161,66)$ & 11 & $(12,5)$ & 5 & 1 & YES & YES & YES & $1.38$ & $(2,3)$ & 4242 & 4012\\
$(161,61)$ & 11 & $(18,7)$ & 6 & 1 & YES & YES & YES & $1.43$ & $(2,3)$ & NO & 4013\\
$(161,71)$ & 12 & $(25,11)$ & 7 & 1 & YES & YES & YES & $1.57$ & $(2,3)$ & NO & 4014\\
$(161,71)$ & 12 & $(34,15)$ & 8 & 1 & YES & YES & YES & $1.71$ & $(2,3)$ & NO & 4015\\
$(161,66)$ & 11 & $(61,25)$ & 9 & 1 & YES & YES & NO(2) & $1.44$ & $(2,3)$ & NO & 4016\\
$(161,45)$ & 11 & $(104,29)$ & 10 & 1 & YES & YES & YES & $2.00$ & $(2,3)$ & NO & 4017\\
$(161,68)$ & 11 & $(116,49)$ & 10 & 1 & YES & YES & YES & $1.57$ & $(2,3)$ & NO & 4018\\
$(161,37)$ & 13 & $(161,37)$ & 13 & 161 & YES & YES & YES & $1.43$ & $(2,3)$ & NO & 4019\\
$(161,66)$ & 11 & $(161,66)$ & 11 & 161 & YES & YES & YES & $1.50$ & $(2,3)$ & NO & 4020\\
$(161,68)$ & 11 & $(161,68)$ & 11 & 161 & YES & YES & YES & $1.43$ & $(2,3)$ & NO & 4021\\
$(161,71)$ & 12 & $(161,71)$ & 12 & 161 & YES & YES & YES & $1.57$ & $(2,3)$ & NO & 4022\\
$(162,71)$ & 12 & $(2,1)$ & 1 & 2 & YES & YES & YES & $1.57$ & $(2,3)$ & -- & 4023\\
$(162,49)$ & 12 & $(5,2)$ & 3 & 1 & YES & YES & YES & $1.57$ & $(2,3)$ & -- & 4024\\
$(162,43)$ & 12 & $(23,6)$ & 8 & 1 & YES & YES & YES & $1.57$ & $(2,3)$ & NO & 4025\\
$(162,49)$ & 12 & $(56,17)$ & 9 & 2 & YES & YES & YES & $1.43$ & $(2,3)$ & 4540 & 4026\\
$(162,43)$ & 12 & $(64,17)$ & 10 & 2 & YES & YES & NO(2) & $1.44$ & $(2,3)$ & 4133 & 4027\\
$(163,43)$ & 12 & $(2,1)$ & 1 & 1 & YES & YES & YES & $1.29$ & $(2,3)$ & NO & 4028\\
$(163,62)$ & 11 & $(2,1)$ & 1 & 1 & YES & YES & NO(2) & $1.56$ & $(2,3)$ & -- & 4029\\
$(163,43)$ & 12 & $(3,1)$ & 2 & 1 & YES & YES & YES & $1.43$ & $(2,3)$ & NO & 4030\\
$(163,62)$ & 11 & $(3,1)$ & 2 & 1 & YES & YES & NO(2) & $1.56$ & $(2,3)$ & NO & 4031\\
$(163,62)$ & 11 & $(5,2)$ & 3 & 1 & YES & YES & NO(2) & $1.56$ & $(2,3)$ & NO & 4032\\
$(163,63)$ & 11 & $(5,2)$ & 3 & 1 & YES & YES & YES & $1.43$ & $(2,3)$ & -- & 4033\\
$(163,71)$ & 11 & $(5,2)$ & 3 & 1 & YES & YES & YES & $1.57$ & $(2,3)$ & -- & 4034\\
$(163,63)$ & 11 & $(7,2)$ & 4 & 1 & YES & YES & YES & $1.57$ & $(2,3)$ & -- & 4035\\
$(163,63)$ & 11 & $(8,3)$ & 4 & 1 & YES & YES & YES & $1.57$ & $(2,3)$ & -- & 4036\\
$(163,44)$ & 11 & $(17,5)$ & 6 & 1 & YES & YES & YES & $1.71$ & $(2,3)$ & NO & 4037\\
$(163,44)$ & 11 & $(34,9)$ & 8 & 1 & YES & YES & YES & $1.71$ & $(2,3)$ & NO & 4038\\
$(163,62)$ & 11 & $(71,27)$ & 9 & 1 & YES & YES & NO(2) & $1.44$ & $(2,3)$ & NO & 4039\\
$(163,38)$ & 13 & $(163,38)$ & 13 & 163 & YES & YES & YES & $1.50$ & $(2,3)$ & NO & 4040\\
$(164,51)$ & 12 & $(3,1)$ & 2 & 1 & YES & YES & YES & $1.29$ & $(2,3)$ & -- & 4041\\
$(165,49)$ & 11 & $(7,3)$ & 4 & 1 & YES & YES & YES & $1.71$ & $(2,3)$ & -- & 4042\\
$(165,49)$ & 11 & $(7,3)$ & 4 & 1 & YES & YES & YES & $1.86$ & $(2,3)$ & NO & 4043\\
$(165,49)$ & 11 & $(15,4)$ & 6 & 15 & YES & YES & YES & $1.71$ & $(2,3)$ & NO & 4044\\
$(165,49)$ & 11 & $(155,46)$ & 11 & 5 & YES & YES & YES & $1.71$ & $(2,3)$ & NO & 4045\\
$(166,61)$ & 11 & $(2,1)$ & 1 & 2 & YES & YES & NO(2) & $1.25$ & $(4,2)$ & -- & 4046\\
$(166,61)$ & 11 & $(11,4)$ & 5 & 1 & YES & YES & NO(2) & $1.25$ & $(4,2)$ & NO & 4047\\
$(167,69)$ & 11 & $(2,1)$ & 1 & 1 & YES & YES & NO(2) & $1.25$ & $(4,2)$ & -- & 4048\\
$(167,64)$ & 11 & $(3,1)$ & 2 & 1 & YES & YES & YES & $1.43$ & $(2,3)$ & -- & 4049\\
$(167,69)$ & 11 & $(3,1)$ & 2 & 1 & YES & YES & YES & $1.57$ & $(2,3)$ & NO & 4050\\
$(167,69)$ & 11 & $(3,1)$ & 2 & 1 & YES & YES & YES & $1.57$ & $(2,3)$ & -- & 4051\\
$(167,51)$ & 12 & $(4,1)$ & 3 & 1 & YES & YES & YES & $1.50$ & $(2,3)$ & NO & 4052\\
$(167,69)$ & 11 & $(5,2)$ & 3 & 1 & YES & YES & NO(2) & $1.25$ & $(4,2)$ & NO & 4053\\
$(167,64)$ & 11 & $(34,13)$ & 7 & 1 & YES & YES & YES & $1.43$ & $(2,3)$ & 4306 & 4054\\
$(167,69)$ & 11 & $(167,69)$ & 11 & 167 & YES & YES & YES & $1.43$ & $(2,3)$ & NO & 4055\\
$(168,65)$ & 12 & $(4,1)$ & 3 & 4 & YES & YES & YES & $1.57$ & $(2,3)$ & -- & 4056\\
$(168,71)$ & 11 & $(7,2)$ & 4 & 7 & YES & YES & YES & $1.71$ & $(2,3)$ & -- & 4057\\
$(168,65)$ & 12 & $(137,53)$ & 11 & 1 & YES & YES & YES & $1.57$ & $(2,3)$ & NO & 4058\\
$(169,38)$ & 13 & $(2,1)$ & 1 & 1 & YES & YES & NO(2) & $1.25$ & $(4,2)$ & -- & 4059\\
$(169,64)$ & 11 & $(2,1)$ & 1 & 1 & YES & YES & YES & $1.29$ & $(2,3)$ & -- & 4060\\
$(169,64)$ & 11 & $(3,1)$ & 2 & 1 & YES & YES & YES & $1.43$ & $(2,3)$ & -- & 4061\\
$(169,64)$ & 11 & $(3,1)$ & 2 & 1 & YES & YES & YES & $1.43$ & $(2,3)$ & NO & 4062\\
$(169,71)$ & 11 & $(3,1)$ & 2 & 1 & YES & YES & YES & $1.57$ & $(2,3)$ & NO & 4063\\
$(169,71)$ & 11 & $(3,1)$ & 2 & 1 & YES & YES & YES & $1.57$ & $(2,3)$ & -- & 4064\\
$(169,38)$ & 13 & $(5,2)$ & 3 & 1 & YES & YES & YES & $1.57$ & $(2,3)$ & -- & 4065\\
$(169,64)$ & 11 & $(5,2)$ & 3 & 1 & YES & YES & NO(2) & $1.44$ & $(2,3)$ & NO & 4066\\
$(169,64)$ & 11 & $(8,3)$ & 4 & 1 & YES & YES & YES & $1.29$ & $(2,3)$ & 3759 & 4067\\
$(169,38)$ & 13 & $(13,3)$ & 6 & 13 & YES & YES & YES & $1.38$ & $(2,3)$ & NO & 4068\\
$(169,64)$ & 11 & $(13,5)$ & 5 & 13 & YES & YES & YES & $1.29$ & $(2,3)$ & 3565 & 4069\\
$(169,32)$ & 13 & $(17,3)$ & 7 & 1 & YES & YES & YES & $1.50$ & $(2,3)$ & NO & 4070\\
$(169,71)$ & 11 & $(19,8)$ & 6 & 1 & YES & YES & YES & $1.50$ & $(2,3)$ & NO & 4071\\
$(169,38)$ & 13 & $(22,5)$ & 7 & 1 & YES & YES & YES & $1.38$ & $(2,3)$ & NO & 4072\\
$(169,32)$ & 13 & $(27,5)$ & 8 & 1 & YES & YES & YES & $1.50$ & $(2,3)$ & NO & 4073\\
$(170,47)$ & 11 & $(8,3)$ & 4 & 2 & YES & YES & YES & $1.71$ & $(2,3)$ & -- & 4074\\
$(170,47)$ & 11 & $(8,3)$ & 4 & 2 & YES & YES & YES & $1.71$ & $(2,3)$ & NO & 4075\\
$(170,47)$ & 11 & $(13,4)$ & 6 & 1 & YES & YES & YES & $1.71$ & $(2,3)$ & NO & 4076\\
$(170,47)$ & 11 & $(17,5)$ & 6 & 17 & YES & YES & YES & $1.86$ & $(2,3)$ & NO & 4077\\
$(170,47)$ & 11 & $(32,9)$ & 8 & 2 & YES & YES & YES & $1.71$ & $(2,3)$ & NO & 4078\\
$(171,65)$ & 11 & $(2,1)$ & 1 & 1 & YES & YES & YES & $1.43$ & $(2,3)$ & -- & 4079\\
$(171,71)$ & 12 & $(2,1)$ & 1 & 1 & YES & YES & YES & $1.29$ & $(4,2)$ & -- & 4080\\
$(171,71)$ & 12 & $(2,1)$ & 1 & 1 & YES & YES & YES & $1.43$ & $(4,2)$ & NO & 4081\\
$(171,65)$ & 11 & $(3,1)$ & 2 & 3 & YES & YES & NO(2) & $1.25$ & $(4,2)$ & NO & 4082\\
$(171,71)$ & 12 & $(3,1)$ & 2 & 3 & YES & YES & YES & $1.57$ & $(2,3)$ & -- & 4083\\
$(171,65)$ & 11 & $(5,2)$ & 3 & 1 & YES & YES & YES & $1.43$ & $(4,2)$ & -- & 4084\\
$(171,65)$ & 11 & $(5,2)$ & 3 & 1 & YES & YES & YES & $1.43$ & $(2,3)$ & NO & 4085\\
$(171,71)$ & 12 & $(5,2)$ & 3 & 1 & YES & YES & YES & $1.43$ & $(2,3)$ & NO & 4086\\
$(171,50)$ & 11 & $(7,3)$ & 4 & 1 & YES & YES & YES & $1.71$ & $(2,3)$ & -- & 4087\\
$(171,50)$ & 11 & $(7,3)$ & 4 & 1 & YES & YES & YES & $1.86$ & $(2,3)$ & NO & 4088\\
$(171,71)$ & 12 & $(7,3)$ & 4 & 1 & YES & YES & YES & $1.43$ & $(2,3)$ & NO & 4089\\
$(171,37)$ & 12 & $(10,3)$ & 5 & 1 & YES & YES & YES & $1.71$ & $(2,3)$ & NO & 4090\\
$(171,71)$ & 12 & $(17,7)$ & 6 & 1 & YES & YES & YES & $1.71$ & $(2,3)$ & 3317 & 4091\\
$(171,50)$ & 11 & $(37,11)$ & 8 & 1 & YES & YES & YES & $1.43$ & $(2,3)$ & NO & 4092\\
$(171,50)$ & 11 & $(99,29)$ & 10 & 9 & YES & YES & YES & $1.71$ & $(2,3)$ & NO & 4093\\
$(171,50)$ & 11 & $(154,45)$ & 11 & 1 & YES & YES & YES & $1.71$ & $(2,3)$ & NO & 4094\\
$(172,63)$ & 11 & $(3,1)$ & 2 & 1 & YES & YES & YES & $1.29$ & $(2,3)$ & -- & 4095\\
$(172,71)$ & 11 & $(5,2)$ & 3 & 1 & YES & YES & YES & $1.43$ & $(2,3)$ & -- & 4096\\
$(172,71)$ & 11 & $(7,2)$ & 4 & 1 & YES & YES & YES & $1.57$ & $(2,3)$ & -- & 4097\\
$(172,71)$ & 11 & $(7,3)$ & 4 & 1 & YES & YES & YES & $1.43$ & $(2,3)$ & NO & 4098\\
$(172,75)$ & 12 & $(23,10)$ & 7 & 1 & YES & YES & YES & $1.57$ & $(2,3)$ & NO & 4099\\
$(172,63)$ & 11 & $(41,15)$ & 8 & 1 & YES & YES & YES & $1.43$ & $(2,3)$ & NO & 4100\\
$(172,71)$ & 11 & $(41,17)$ & 8 & 1 & YES & YES & YES & $1.43$ & $(2,3)$ & 4473 & 4101\\
$(172,71)$ & 11 & $(75,31)$ & 9 & 1 & YES & YES & YES & $1.57$ & $(2,3)$ & NO & 4102\\
$(173,66)$ & 11 & $(2,1)$ & 1 & 1 & NO & YES & NO(2) & $1.56$ & $(2,3)$ & -- & 4103\\
$(173,76)$ & 11 & $(2,1)$ & 1 & 1 & YES & YES & YES & $1.29$ & $(2,3)$ & NO & 4104\\
$(173,66)$ & 11 & $(3,1)$ & 2 & 1 & YES & YES & YES & $1.57$ & $(2,3)$ & -- & 4105\\
$(173,73)$ & 11 & $(3,1)$ & 2 & 1 & YES & YES & YES & $1.29$ & $(2,3)$ & -- & 4106\\
$(173,76)$ & 11 & $(3,1)$ & 2 & 1 & YES & YES & YES & $1.43$ & $(2,3)$ & -- & 4107\\
$(173,38)$ & 13 & $(4,1)$ & 3 & 1 & YES & YES & YES & $1.50$ & $(2,3)$ & -- & 4108\\
$(173,38)$ & 13 & $(4,1)$ & 3 & 1 & YES & YES & NO(2) & $1.25$ & $(4,2)$ & NO & 4109\\
$(173,51)$ & 12 & $(4,1)$ & 3 & 1 & YES & YES & YES & $1.57$ & $(2,3)$ & -- & 4110\\
$(173,73)$ & 11 & $(7,3)$ & 4 & 1 & YES & YES & YES & $1.43$ & $(2,3)$ & NO & 4111\\
$(173,51)$ & 12 & $(9,2)$ & 5 & 1 & YES & YES & YES & $1.71$ & $(2,3)$ & NO & 4112\\
$(173,38)$ & 13 & $(23,5)$ & 7 & 1 & YES & YES & YES & $1.50$ & $(2,3)$ & NO & 4113\\
$(173,76)$ & 11 & $(25,11)$ & 7 & 1 & YES & YES & YES & $1.43$ & $(2,3)$ & NO & 4114\\
$(173,73)$ & 11 & $(26,11)$ & 7 & 1 & YES & YES & YES & $1.29$ & $(2,3)$ & 4196 & 4115\\
$(173,51)$ & 12 & $(95,28)$ & 11 & 1 & YES & YES & YES & $1.57$ & $(2,3)$ & NO & 4116\\
$(173,51)$ & 12 & $(105,31)$ & 10 & 1 & YES & YES & YES & $1.71$ & $(2,3)$ & NO & 4117\\
$(173,76)$ & 11 & $(173,76)$ & 11 & 173 & YES & YES & YES & $1.43$ & $(2,3)$ & NO & 4118\\
$(175,52)$ & 12 & $(3,1)$ & 2 & 1 & YES & YES & YES & $1.71$ & $(2,3)$ & NO & 4119\\
$(175,52)$ & 12 & $(3,1)$ & 2 & 1 & YES & YES & YES & $1.71$ & $(2,3)$ & -- & 4120\\
$(175,67)$ & 11 & $(3,1)$ & 2 & 1 & YES & YES & YES & $1.57$ & $(2,3)$ & -- & 4121\\
$(175,52)$ & 12 & $(101,30)$ & 10 & 1 & YES & YES & YES & $1.43$ & $(2,3)$ & 4371 & 4122\\
$(175,52)$ & 12 & $(175,52)$ & 12 & 175 & YES & YES & YES & $1.29$ & $(2,3)$ & NO & 4123\\
$(176,51)$ & 12 & $(3,1)$ & 2 & 1 & YES & YES & YES & $1.43$ & $(2,3)$ & -- & 4124\\
$(176,51)$ & 12 & $(7,2)$ & 4 & 1 & YES & YES & YES & $1.43$ & $(2,3)$ & 3771 & 4125\\
$(177,47)$ & 12 & $(2,1)$ & 1 & 1 & YES & YES & NO(2) & $1.56$ & $(2,3)$ & NO & 4126\\
$(177,73)$ & 12 & $(2,1)$ & 1 & 1 & YES & YES & YES & $1.71$ & $(2,3)$ & NO & 4127\\
$(177,74)$ & 12 & $(2,1)$ & 1 & 1 & YES & YES & YES & $1.43$ & $(2,3)$ & -- & 4128\\
$(177,74)$ & 12 & $(3,1)$ & 2 & 3 & YES & YES & YES & $1.57$ & $(2,3)$ & -- & 4129\\
$(177,74)$ & 12 & $(3,1)$ & 2 & 3 & YES & YES & YES & $1.43$ & $(2,3)$ & NO & 4130\\
$(177,73)$ & 12 & $(4,1)$ & 3 & 1 & YES & YES & YES & $1.57$ & $(2,3)$ & -- & 4131\\
$(177,73)$ & 12 & $(12,5)$ & 5 & 3 & YES & YES & YES & $1.57$ & $(2,3)$ & NO & 4132\\
$(177,47)$ & 12 & $(49,13)$ & 9 & 1 & YES & YES & NO(2) & $1.44$ & $(2,3)$ & 4027 & 4133\\
$(177,49)$ & 11 & $(69,19)$ & 9 & 3 & YES & YES & YES & $1.71$ & $(2,3)$ & NO & 4134\\
$(178,53)$ & 12 & $(2,1)$ & 1 & 2 & YES & YES & YES & $1.57$ & $(2,3)$ & NO & 4135\\
$(178,53)$ & 12 & $(3,1)$ & 2 & 1 & YES & YES & YES & $1.29$ & $(2,3)$ & -- & 4136\\
$(178,69)$ & 11 & $(5,1)$ & 4 & 1 & YES & YES & YES & $1.50$ & $(2,3)$ & NO & 4137\\
$(178,53)$ & 12 & $(6,1)$ & 5 & 2 & YES & YES & YES & $1.43$ & $(2,3)$ & NO & 4138\\
$(178,33)$ & 14 & $(8,1)$ & 7 & 2 & YES & YES & YES & $1.29$ & $(2,3)$ & NO & 4139\\
$(178,49)$ & 11 & $(8,3)$ & 4 & 2 & YES & YES & YES & $1.71$ & $(2,3)$ & -- & 4140\\
$(178,49)$ & 11 & $(10,3)$ & 5 & 2 & YES & YES & YES & $1.71$ & $(2,3)$ & -- & 4141\\
$(178,69)$ & 11 & $(23,9)$ & 7 & 1 & YES & YES & YES & $1.57$ & $(2,3)$ & NO & 4142\\
$(178,49)$ & 11 & $(43,12)$ & 8 & 1 & YES & YES & YES & $1.71$ & $(2,3)$ & NO & 4143\\
$(178,49)$ & 11 & $(65,18)$ & 9 & 1 & YES & YES & YES & $1.71$ & $(2,3)$ & NO & 4144\\
$(179,74)$ & 11 & $(2,1)$ & 1 & 1 & NO & YES & YES & $1.43$ & $(2,3)$ & -- & 4145\\
$(179,74)$ & 11 & $(3,1)$ & 2 & 1 & YES & YES & YES & $1.71$ & $(2,3)$ & -- & 4146\\
$(179,68)$ & 11 & $(5,2)$ & 3 & 1 & YES & YES & YES & $1.71$ & $(2,3)$ & -- & 4147\\
$(179,74)$ & 11 & $(5,2)$ & 3 & 1 & YES & YES & YES & $1.57$ & $(2,3)$ & -- & 4148\\
$(179,74)$ & 11 & $(5,2)$ & 3 & 1 & YES & YES & YES & $1.57$ & $(2,3)$ & NO & 4149\\
$(179,76)$ & 12 & $(7,3)$ & 4 & 1 & YES & YES & YES & $1.57$ & $(2,3)$ & 3594 & 4150\\
$(179,41)$ & 12 & $(57,13)$ & 9 & 1 & YES & YES & YES & $1.38$ & $(2,3)$ & 4560 & 4151\\
$(179,42)$ & 13 & $(98,23)$ & 12 & 1 & YES & YES & YES & $1.43$ & $(2,3)$ & NO & 4152\\
$(179,50)$ & 11 & $(104,29)$ & 10 & 1 & YES & YES & YES & $2.00$ & $(2,3)$ & NO & 4153\\
$(179,74)$ & 11 & $(121,50)$ & 10 & 1 & YES & YES & YES & $1.71$ & $(2,3)$ & NO & 4154\\
$(179,74)$ & 11 & $(179,74)$ & 11 & 179 & YES & YES & YES & $1.71$ & $(2,3)$ & NO & 4155\\
$(181,65)$ & 12 & $(2,1)$ & 1 & 1 & YES & YES & YES & $1.57$ & $(2,3)$ & -- & 4156\\
$(181,75)$ & 11 & $(3,1)$ & 2 & 1 & YES & YES & YES & $1.57$ & $(2,3)$ & -- & 4157\\
$(181,70)$ & 11 & $(4,1)$ & 3 & 1 & YES & YES & YES & $1.38$ & $(2,3)$ & -- & 4158\\
$(181,70)$ & 11 & $(5,2)$ & 3 & 1 & YES & YES & YES & $1.62$ & $(2,3)$ & -- & 4159\\
$(181,75)$ & 11 & $(7,3)$ & 4 & 1 & YES & YES & YES & $1.29$ & $(2,3)$ & NO & 4160\\
$(181,41)$ & 12 & $(8,3)$ & 4 & 1 & YES & YES & YES & $1.50$ & $(2,3)$ & NO & 4161\\
$(181,50)$ & 11 & $(8,3)$ & 4 & 1 & YES & YES & YES & $1.57$ & $(2,3)$ & -- & 4162\\
$(181,70)$ & 11 & $(8,3)$ & 4 & 1 & YES & YES & YES & $1.57$ & $(2,3)$ & -- & 4163\\
$(181,76)$ & 11 & $(8,3)$ & 4 & 1 & YES & YES & YES & $1.86$ & $(2,3)$ & NO & 4164\\
$(181,50)$ & 11 & $(10,3)$ & 5 & 1 & YES & YES & YES & $1.57$ & $(2,3)$ & -- & 4165\\
$(181,65)$ & 12 & $(25,9)$ & 7 & 1 & YES & YES & YES & $1.43$ & $(2,3)$ & NO & 4166\\
$(181,76)$ & 11 & $(43,18)$ & 8 & 1 & YES & YES & YES & $1.86$ & $(2,3)$ & 4413 & 4167\\
$(181,70)$ & 11 & $(106,41)$ & 10 & 1 & YES & YES & YES & $1.38$ & $(2,3)$ & NO & 4168\\
$(181,50)$ & 11 & $(112,31)$ & 10 & 1 & YES & YES & YES & $1.43$ & $(2,3)$ & NO & 4169\\
$(181,70)$ & 11 & $(119,46)$ & 10 & 1 & YES & YES & YES & $1.50$ & $(2,3)$ & NO & 4170\\
$(181,70)$ & 11 & $(163,63)$ & 11 & 1 & YES & YES & YES & $1.57$ & $(2,3)$ & NO & 4171\\
$(181,50)$ & 11 & $(170,47)$ & 11 & 1 & YES & YES & YES & $1.71$ & $(2,3)$ & NO & 4172\\
$(181,65)$ & 12 & $(181,65)$ & 12 & 181 & YES & YES & YES & $1.57$ & $(2,3)$ & NO & 4173\\
$(182,53)$ & 12 & $(5,2)$ & 3 & 1 & YES & YES & YES & $1.57$ & $(2,3)$ & -- & 4174\\
$(182,79)$ & 12 & $(30,13)$ & 8 & 2 & YES & YES & YES & $1.43$ & $(2,3)$ & NO & 4175\\
$(182,53)$ & 12 & $(134,39)$ & 11 & 2 & YES & YES & YES & $1.43$ & $(2,3)$ & NO & 4176\\
$(183,67)$ & 11 & $(2,1)$ & 1 & 1 & YES & YES & YES & $1.43$ & $(2,3)$ & NO & 4177\\
$(183,67)$ & 11 & $(2,1)$ & 1 & 1 & YES & YES & YES & $1.43$ & $(2,3)$ & -- & 4178\\
$(183,71)$ & 11 & $(2,1)$ & 1 & 1 & YES & YES & YES & $1.50$ & $(2,3)$ & -- & 4179\\
$(183,67)$ & 11 & $(3,1)$ & 2 & 3 & YES & YES & YES & $1.43$ & $(2,3)$ & -- & 4180\\
$(183,82)$ & 14 & $(5,1)$ & 4 & 1 & YES & YES & YES & $1.71$ & $(2,3)$ & -- & 4181\\
$(183,67)$ & 11 & $(8,3)$ & 4 & 1 & YES & YES & YES & $1.43$ & $(2,3)$ & 3635 & 4182\\
$(183,71)$ & 11 & $(13,5)$ & 5 & 1 & YES & YES & YES & $1.38$ & $(2,3)$ & NO & 4183\\
$(183,49)$ & 12 & $(183,49)$ & 12 & 183 & YES & YES & YES & $1.29$ & $(2,3)$ & NO & 4184\\
$(183,67)$ & 11 & $(183,67)$ & 11 & 183 & YES & YES & YES & $1.57$ & $(2,3)$ & NO & 4185\\
$(184,71)$ & 12 & $(2,1)$ & 1 & 2 & NO & YES & NO(2) & $1.14$ & $(6,1)$ & -- & 4186\\
$(184,83)$ & 12 & $(11,5)$ & 6 & 1 & YES & YES & YES & $1.57$ & $(2,3)$ & NO & 4187\\
$(185,49)$ & 13 & $(2,1)$ & 1 & 1 & YES & YES & YES & $1.62$ & $(2,3)$ & NO & 4188\\
$(185,33)$ & 14 & $(8,1)$ & 7 & 1 & YES & YES & YES & $1.43$ & $(2,3)$ & NO & 4189\\
$(186,71)$ & 11 & $(5,2)$ & 3 & 1 & YES & YES & YES & $1.86$ & $(2,3)$ & -- & 4190\\
$(187,79)$ & 11 & $(2,1)$ & 1 & 1 & YES & YES & YES & $1.29$ & $(4,2)$ & -- & 4191\\
$(187,71)$ & 11 & $(3,1)$ & 2 & 1 & YES & YES & YES & $1.50$ & $(2,3)$ & NO & 4192\\
$(187,71)$ & 11 & $(5,1)$ & 4 & 1 & YES & YES & YES & $1.38$ & $(2,3)$ & NO & 4193\\
$(187,79)$ & 11 & $(5,1)$ & 4 & 1 & YES & YES & YES & $1.29$ & $(2,3)$ & NO & 4194\\
$(187,79)$ & 11 & $(5,2)$ & 3 & 1 & YES & YES & YES & $1.29$ & $(2,3)$ & NO & 4195\\
$(187,79)$ & 11 & $(19,8)$ & 6 & 1 & YES & YES & YES & $1.29$ & $(2,3)$ & 4115 & 4196\\
$(187,71)$ & 11 & $(29,11)$ & 7 & 1 & YES & YES & YES & $1.50$ & $(2,3)$ & 3837 & 4197\\
$(187,79)$ & 11 & $(116,49)$ & 10 & 1 & YES & YES & YES & $1.43$ & $(2,3)$ & NO & 4198\\
$(187,71)$ & 11 & $(187,71)$ & 11 & 187 & YES & YES & YES & $1.38$ & $(2,3)$ & NO & 4199\\
$(187,79)$ & 11 & $(187,79)$ & 11 & 187 & YES & YES & YES & $1.29$ & $(2,3)$ & NO & 4200\\
$(188,79)$ & 11 & $(2,1)$ & 1 & 2 & YES & YES & YES & $1.43$ & $(2,3)$ & NO & 4201\\
$(188,39)$ & 13 & $(3,1)$ & 2 & 1 & YES & YES & YES & $1.43$ & $(2,3)$ & NO & 4202\\
$(188,39)$ & 13 & $(3,1)$ & 2 & 1 & YES & YES & YES & $1.43$ & $(2,3)$ & -- & 4203\\
$(188,79)$ & 11 & $(3,1)$ & 2 & 1 & YES & YES & YES & $1.57$ & $(2,3)$ & -- & 4204\\
$(188,79)$ & 11 & $(7,3)$ & 4 & 1 & YES & YES & YES & $1.43$ & $(2,3)$ & NO & 4205\\
$(188,69)$ & 11 & $(30,11)$ & 7 & 2 & YES & YES & YES & $1.50$ & $(2,3)$ & 3852 & 4206\\
$(188,39)$ & 13 & $(82,17)$ & 11 & 2 & YES & YES & YES & $1.43$ & $(2,3)$ & 4322 & 4207\\
$(189,55)$ & 12 & $(2,1)$ & 1 & 1 & YES & YES & NO(2) & $1.25$ & $(4,2)$ & NO & 4208\\
$(189,55)$ & 12 & $(3,1)$ & 2 & 3 & YES & YES & NO(2) & $1.25$ & $(4,2)$ & NO & 4209\\
$(189,40)$ & 12 & $(4,1)$ & 3 & 1 & YES & YES & NO(2) & $1.33$ & $(2,3)$ & -- & 4210\\
$(189,52)$ & 12 & $(5,2)$ & 3 & 1 & YES & YES & YES & $1.71$ & $(2,3)$ & -- & 4211\\
$(189,52)$ & 12 & $(7,2)$ & 4 & 7 & YES & YES & YES & $1.86$ & $(2,3)$ & -- & 4212\\
$(189,40)$ & 12 & $(14,3)$ & 6 & 7 & YES & YES & NO(2) & $1.44$ & $(2,3)$ & NO & 4213\\
$(189,52)$ & 12 & $(25,7)$ & 7 & 1 & YES & YES & YES & $1.71$ & $(2,3)$ & NO & 4214\\
$(189,52)$ & 12 & $(47,13)$ & 8 & 1 & YES & YES & YES & $1.86$ & $(2,3)$ & NO & 4215\\
$(190,41)$ & 12 & $(8,3)$ & 4 & 2 & YES & YES & YES & $1.86$ & $(2,3)$ & -- & 4216\\
$(190,43)$ & 12 & $(9,2)$ & 5 & 1 & YES & YES & NO(2) & $1.33$ & $(2,3)$ & NO & 4217\\
$(191,80)$ & 11 & $(2,1)$ & 1 & 1 & YES & YES & YES & $1.43$ & $(2,3)$ & -- & 4218\\
$(191,80)$ & 11 & $(2,1)$ & 1 & 1 & YES & YES & YES & $1.43$ & $(2,3)$ & NO & 4219\\
$(191,80)$ & 11 & $(3,1)$ & 2 & 1 & YES & YES & YES & $1.43$ & $(2,3)$ & -- & 4220\\
$(191,80)$ & 11 & $(4,1)$ & 3 & 1 & YES & YES & YES & $1.43$ & $(2,3)$ & -- & 4221\\
$(191,74)$ & 11 & $(5,2)$ & 3 & 1 & YES & YES & YES & $1.71$ & $(2,3)$ & -- & 4222\\
$(191,74)$ & 11 & $(129,50)$ & 10 & 1 & YES & YES & YES & $1.71$ & $(2,3)$ & NO & 4223\\
$(191,80)$ & 11 & $(191,80)$ & 11 & 191 & YES & YES & YES & $1.43$ & $(2,3)$ & NO & 4224\\
$(192,73)$ & 11 & $(121,46)$ & 10 & 1 & YES & YES & YES & $1.50$ & $(2,3)$ & NO & 4225\\
$(193,81)$ & 11 & $(2,1)$ & 1 & 1 & NO & YES & YES & $1.43$ & $(2,3)$ & -- & 4226\\
$(193,74)$ & 12 & $(3,1)$ & 2 & 1 & YES & YES & YES & $1.57$ & $(2,3)$ & -- & 4227\\
$(193,81)$ & 11 & $(4,1)$ & 3 & 1 & YES & YES & YES & $1.38$ & $(2,3)$ & -- & 4228\\
$(193,53)$ & 12 & $(6,1)$ & 5 & 1 & YES & YES & YES & $1.38$ & $(2,3)$ & NO & 4229\\
$(193,44)$ & 12 & $(40,9)$ & 9 & 1 & YES & YES & YES & $1.57$ & $(2,3)$ & NO & 4230\\
$(193,81)$ & 11 & $(81,34)$ & 9 & 1 & YES & YES & YES & $1.50$ & $(2,3)$ & NO & 4231\\
$(193,53)$ & 12 & $(142,39)$ & 11 & 1 & YES & YES & YES & $1.38$ & $(2,3)$ & NO & 4232\\
$(194,71)$ & 12 & $(2,1)$ & 1 & 2 & YES & YES & YES & $1.57$ & $(2,3)$ & -- & 4233\\
$(194,71)$ & 12 & $(3,1)$ & 2 & 1 & YES & YES & YES & $1.57$ & $(2,3)$ & NO & 4234\\
$(194,71)$ & 12 & $(7,1)$ & 6 & 1 & YES & YES & YES & $1.43$ & $(2,3)$ & NO & 4235\\
$(194,71)$ & 12 & $(30,11)$ & 7 & 2 & YES & YES & YES & $1.43$ & $(2,3)$ & NO & 4236\\
$(194,71)$ & 12 & $(41,15)$ & 8 & 1 & YES & YES & YES & $1.57$ & $(2,3)$ & NO & 4237\\
$(194,71)$ & 12 & $(71,26)$ & 9 & 1 & YES & YES & YES & $1.43$ & $(2,3)$ & NO & 4238\\
$(194,75)$ & 11 & $(163,63)$ & 11 & 1 & YES & YES & YES & $1.43$ & $(2,3)$ & NO & 4239\\
$(196,81)$ & 11 & $(2,1)$ & 1 & 2 & YES & YES & YES & $1.57$ & $(2,3)$ & -- & 4240\\
$(196,81)$ & 11 & $(3,1)$ & 2 & 1 & YES & YES & YES & $1.57$ & $(2,3)$ & -- & 4241\\
$(196,81)$ & 11 & $(5,2)$ & 3 & 1 & YES & YES & YES & $1.38$ & $(2,3)$ & 4012 & 4242\\
$(196,81)$ & 11 & $(104,43)$ & 10 & 4 & YES & YES & YES & $1.71$ & $(2,3)$ & NO & 4243\\
$(197,43)$ & 12 & $(7,2)$ & 4 & 1 & YES & YES & YES & $1.50$ & $(2,3)$ & NO & 4244\\
$(199,89)$ & 14 & $(9,4)$ & 5 & 1 & YES & YES & YES & $1.71$ & $(2,3)$ & NO & 4245\\
$(199,89)$ & 14 & $(38,17)$ & 9 & 1 & YES & YES & YES & $1.71$ & $(2,3)$ & NO & 4246\\
$(201,76)$ & 12 & $(2,1)$ & 1 & 1 & YES & YES & YES & $1.43$ & $(2,3)$ & -- & 4247\\
$(201,76)$ & 12 & $(3,1)$ & 2 & 3 & YES & YES & YES & $1.43$ & $(2,3)$ & NO & 4248\\
$(201,76)$ & 12 & $(5,2)$ & 3 & 1 & YES & YES & YES & $1.43$ & $(2,3)$ & NO & 4249\\
$(201,37)$ & 14 & $(6,1)$ & 5 & 3 & YES & YES & NO(2) & $1.44$ & $(2,3)$ & NO & 4250\\
$(201,37)$ & 14 & $(7,1)$ & 6 & 1 & YES & YES & NO(2) & $1.44$ & $(2,3)$ & NO & 4251\\
$(201,47)$ & 12 & $(10,3)$ & 5 & 1 & YES & YES & YES & $1.71$ & $(2,3)$ & NO & 4252\\
$(202,61)$ & 13 & $(2,1)$ & 1 & 2 & YES & YES & YES & $1.57$ & $(2,3)$ & NO & 4253\\
$(202,59)$ & 12 & $(3,1)$ & 2 & 1 & YES & YES & YES & $1.57$ & $(2,3)$ & -- & 4254\\
$(202,59)$ & 12 & $(4,1)$ & 3 & 2 & YES & YES & YES & $1.43$ & $(2,3)$ & -- & 4255\\
$(202,61)$ & 13 & $(6,1)$ & 5 & 2 & YES & YES & YES & $1.43$ & $(2,3)$ & NO & 4256\\
$(202,61)$ & 13 & $(6,1)$ & 5 & 2 & YES & YES & YES & $1.43$ & $(2,3)$ & -- & 4257\\
$(202,59)$ & 12 & $(41,12)$ & 8 & 1 & YES & YES & YES & $1.29$ & $(2,3)$ & 3658 & 4258\\
$(202,89)$ & 12 & $(59,26)$ & 9 & 1 & YES & YES & YES & $1.43$ & $(2,3)$ & NO & 4259\\
$(202,59)$ & 12 & $(65,19)$ & 9 & 1 & YES & YES & YES & $1.43$ & $(2,3)$ & NO & 4260\\
$(203,59)$ & 12 & $(2,1)$ & 1 & 1 & YES & YES & NO(3) & $1.29$ & $(2,3)$ & NO & 4261\\
$(203,59)$ & 12 & $(3,1)$ & 2 & 1 & YES & YES & YES & $1.29$ & $(2,3)$ & -- & 4262\\
$(203,59)$ & 12 & $(4,1)$ & 3 & 1 & YES & YES & NO(2) & $1.44$ & $(2,3)$ & NO & 4263\\
$(203,60)$ & 12 & $(5,2)$ & 3 & 1 & YES & YES & YES & $2.00$ & $(2,3)$ & -- & 4264\\
$(203,59)$ & 12 & $(7,2)$ & 4 & 7 & YES & YES & NO(3) & $1.29$ & $(2,3)$ & NO & 4265\\
$(203,60)$ & 12 & $(11,3)$ & 5 & 1 & YES & YES & YES & $1.86$ & $(2,3)$ & NO & 4266\\
$(203,59)$ & 12 & $(17,5)$ & 6 & 1 & YES & YES & YES & $1.29$ & $(2,3)$ & NO & 4267\\
$(203,59)$ & 12 & $(31,9)$ & 8 & 1 & YES & YES & NO(2) & $1.44$ & $(2,3)$ & 3907 & 4268\\
$(203,60)$ & 12 & $(37,11)$ & 8 & 1 & YES & YES & YES & $2.00$ & $(2,3)$ & NO & 4269\\
$(205,47)$ & 13 & $(6,1)$ & 5 & 1 & YES & YES & YES & $1.38$ & $(2,3)$ & NO & 4270\\
$(205,47)$ & 13 & $(205,47)$ & 13 & 205 & YES & YES & YES & $1.38$ & $(2,3)$ & NO & 4271\\
$(206,37)$ & 14 & $(2,1)$ & 1 & 2 & YES & YES & NO(2) & $1.44$ & $(2,3)$ & -- & 4272\\
$(206,85)$ & 12 & $(3,1)$ & 2 & 1 & YES & YES & YES & $1.43$ & $(2,3)$ & NO & 4273\\
$(206,37)$ & 14 & $(7,1)$ & 6 & 1 & YES & YES & NO(2) & $1.44$ & $(2,3)$ & NO & 4274\\
$(206,47)$ & 12 & $(7,3)$ & 4 & 1 & YES & YES & YES & $1.71$ & $(2,3)$ & -- & 4275\\
$(206,47)$ & 12 & $(8,3)$ & 4 & 2 & YES & YES & YES & $1.86$ & $(2,3)$ & -- & 4276\\
$(206,85)$ & 12 & $(12,5)$ & 5 & 2 & YES & YES & YES & $1.57$ & $(2,3)$ & 3370 & 4277\\
$(207,55)$ & 13 & $(2,1)$ & 1 & 1 & YES & YES & YES & $1.43$ & $(4,2)$ & NO & 4278\\
$(207,55)$ & 13 & $(2,1)$ & 1 & 1 & YES & YES & YES & $1.62$ & $(2,3)$ & -- & 4279\\
$(207,64)$ & 13 & $(2,1)$ & 1 & 1 & YES & YES & YES & $1.62$ & $(2,3)$ & NO & 4280\\
$(207,80)$ & 12 & $(2,1)$ & 1 & 1 & YES & YES & YES & $1.57$ & $(2,3)$ & NO & 4281\\
$(207,80)$ & 12 & $(2,1)$ & 1 & 1 & YES & YES & YES & $1.57$ & $(2,3)$ & -- & 4282\\
$(207,55)$ & 13 & $(3,1)$ & 2 & 3 & YES & YES & YES & $1.50$ & $(2,3)$ & NO & 4283\\
$(207,80)$ & 12 & $(3,1)$ & 2 & 3 & YES & YES & YES & $1.57$ & $(2,3)$ & -- & 4284\\
$(207,64)$ & 13 & $(4,1)$ & 3 & 1 & YES & YES & YES & $1.62$ & $(2,3)$ & NO & 4285\\
$(207,80)$ & 12 & $(4,1)$ & 3 & 1 & YES & YES & YES & $1.43$ & $(2,3)$ & -- & 4286\\
$(207,79)$ & 11 & $(5,1)$ & 4 & 1 & YES & YES & YES & $1.38$ & $(2,3)$ & NO & 4287\\
$(207,80)$ & 12 & $(7,1)$ & 6 & 1 & YES & YES & YES & $1.43$ & $(2,3)$ & NO & 4288\\
$(207,64)$ & 13 & $(42,13)$ & 9 & 3 & YES & YES & YES & $1.62$ & $(2,3)$ & NO & 4289\\
$(207,55)$ & 13 & $(49,13)$ & 9 & 1 & YES & YES & YES & $1.57$ & $(2,3)$ & NO & 4290\\
$(207,80)$ & 12 & $(75,29)$ & 9 & 3 & YES & YES & YES & $1.57$ & $(2,3)$ & NO & 4291\\
$(207,80)$ & 12 & $(119,46)$ & 10 & 1 & YES & YES & YES & $1.57$ & $(2,3)$ & 4479 & 4292\\
$(207,55)$ & 13 & $(207,55)$ & 13 & 207 & YES & YES & YES & $1.57$ & $(2,3)$ & NO & 4293\\
$(208,79)$ & 11 & $(3,1)$ & 2 & 1 & YES & YES & YES & $1.50$ & $(2,3)$ & -- & 4294\\
$(208,75)$ & 12 & $(5,2)$ & 3 & 1 & YES & YES & YES & $1.43$ & $(2,3)$ & NO & 4295\\
$(208,79)$ & 11 & $(7,2)$ & 4 & 1 & YES & YES & YES & $1.43$ & $(2,3)$ & -- & 4296\\
$(208,37)$ & 13 & $(23,4)$ & 8 & 1 & YES & YES & YES & $1.38$ & $(2,3)$ & NO & 4297\\
$(208,79)$ & 11 & $(71,27)$ & 9 & 1 & YES & YES & YES & $1.86$ & $(2,3)$ & NO & 4298\\
$(208,55)$ & 12 & $(121,32)$ & 11 & 1 & YES & YES & YES & $1.57$ & $(2,3)$ & NO & 4299\\
$(208,79)$ & 11 & $(208,79)$ & 11 & 208 & YES & YES & YES & $1.43$ & $(2,3)$ & NO & 4300\\
$(209,56)$ & 12 & $(2,1)$ & 1 & 1 & YES & YES & YES & $1.50$ & $(2,3)$ & NO & 4301\\
$(209,80)$ & 11 & $(2,1)$ & 1 & 1 & YES & YES & YES & $1.43$ & $(2,3)$ & NO & 4302\\
$(209,80)$ & 11 & $(2,1)$ & 1 & 1 & YES & YES & YES & $1.57$ & $(2,3)$ & -- & 4303\\
$(209,49)$ & 13 & $(3,1)$ & 2 & 1 & YES & YES & YES & $1.38$ & $(2,3)$ & -- & 4304\\
$(209,49)$ & 13 & $(9,2)$ & 5 & 1 & YES & YES & YES & $1.38$ & $(2,3)$ & NO & 4305\\
$(209,80)$ & 11 & $(13,5)$ & 5 & 1 & YES & YES & YES & $1.43$ & $(2,3)$ & 4054 & 4306\\
$(211,93)$ & 12 & $(59,26)$ & 9 & 1 & YES & YES & YES & $1.43$ & $(2,3)$ & NO & 4307\\
$(212,45)$ & 14 & $(3,1)$ & 2 & 1 & YES & YES & YES & $1.57$ & $(2,3)$ & -- & 4308\\
$(212,81)$ & 11 & $(5,2)$ & 3 & 1 & YES & YES & YES & $1.57$ & $(2,3)$ & -- & 4309\\
$(212,81)$ & 11 & $(18,7)$ & 6 & 2 & YES & YES & YES & $1.43$ & $(2,3)$ & NO & 4310\\
$(212,81)$ & 11 & $(144,55)$ & 10 & 4 & YES & YES & YES & $1.71$ & $(2,3)$ & NO & 4311\\
$(212,93)$ & 12 & $(212,93)$ & 12 & 212 & YES & YES & YES & $1.43$ & $(2,3)$ & NO & 4312\\
$(213,65)$ & 12 & $(4,1)$ & 3 & 1 & YES & YES & YES & $1.38$ & $(2,3)$ & -- & 4313\\
$(213,62)$ & 12 & $(5,2)$ & 3 & 1 & YES & YES & YES & $1.57$ & $(2,3)$ & NO & 4314\\
$(213,62)$ & 12 & $(5,2)$ & 3 & 1 & YES & YES & YES & $1.57$ & $(2,3)$ & -- & 4315\\
$(213,77)$ & 12 & $(8,3)$ & 4 & 1 & YES & YES & YES & $1.43$ & $(2,3)$ & NO & 4316\\
$(213,62)$ & 12 & $(103,30)$ & 11 & 1 & YES & YES & YES & $1.57$ & $(2,3)$ & NO & 4317\\
$(217,52)$ & 14 & $(3,1)$ & 2 & 1 & YES & YES & YES & $1.57$ & $(2,3)$ & -- & 4318\\
$(217,60)$ & 12 & $(5,2)$ & 3 & 1 & YES & YES & YES & $1.71$ & $(2,3)$ & NO & 4319\\
$(217,60)$ & 12 & $(10,3)$ & 5 & 1 & YES & YES & YES & $1.57$ & $(2,3)$ & NO & 4320\\
$(217,60)$ & 12 & $(25,7)$ & 7 & 1 & YES & YES & YES & $1.71$ & $(2,3)$ & NO & 4321\\
$(217,45)$ & 13 & $(53,11)$ & 10 & 1 & YES & YES & YES & $1.43$ & $(2,3)$ & 4207 & 4322\\
$(218,47)$ & 13 & $(6,1)$ & 5 & 2 & YES & YES & YES & $1.43$ & $(2,3)$ & NO & 4323\\
$(218,57)$ & 13 & $(23,6)$ & 8 & 1 & YES & YES & YES & $1.57$ & $(2,3)$ & NO & 4324\\
$(218,51)$ & 13 & $(43,10)$ & 9 & 1 & YES & YES & YES & $1.43$ & $(2,3)$ & NO & 4325\\
$(219,65)$ & 12 & $(5,2)$ & 3 & 1 & YES & YES & YES & $1.71$ & $(2,3)$ & NO & 4326\\
$(219,64)$ & 12 & $(10,3)$ & 5 & 1 & YES & YES & YES & $1.50$ & $(2,3)$ & NO & 4327\\
$(219,67)$ & 12 & $(134,41)$ & 11 & 1 & YES & YES & YES & $1.43$ & $(2,3)$ & NO & 4328\\
$(221,84)$ & 12 & $(29,11)$ & 7 & 1 & YES & YES & YES & $1.57$ & $(4,2)$ & NO & 4329\\
$(221,84)$ & 12 & $(221,84)$ & 12 & 221 & YES & YES & YES & $1.43$ & $(4,2)$ & NO & 4330\\
$(222,91)$ & 12 & $(2,1)$ & 1 & 2 & YES & YES & YES & $1.57$ & $(2,3)$ & NO & 4331\\
$(222,91)$ & 12 & $(61,25)$ & 9 & 1 & YES & YES & YES & $1.57$ & $(2,3)$ & NO & 4332\\
$(223,82)$ & 12 & $(2,1)$ & 1 & 1 & YES & YES & YES & $1.43$ & $(2,3)$ & NO & 4333\\
$(223,82)$ & 12 & $(19,7)$ & 6 & 1 & YES & YES & YES & $1.43$ & $(2,3)$ & NO & 4334\\
$(227,52)$ & 13 & $(2,1)$ & 1 & 1 & YES & YES & YES & $1.38$ & $(2,3)$ & -- & 4335\\
$(227,88)$ & 12 & $(3,1)$ & 2 & 1 & YES & YES & YES & $1.71$ & $(2,3)$ & NO & 4336\\
$(227,88)$ & 12 & $(3,1)$ & 2 & 1 & YES & YES & YES & $1.71$ & $(2,3)$ & -- & 4337\\
$(227,67)$ & 12 & $(4,1)$ & 3 & 1 & YES & YES & YES & $1.38$ & $(2,3)$ & -- & 4338\\
$(227,66)$ & 12 & $(11,3)$ & 5 & 1 & YES & YES & YES & $1.57$ & $(2,3)$ & NO & 4339\\
$(227,88)$ & 12 & $(31,12)$ & 7 & 1 & YES & YES & YES & $1.57$ & $(4,2)$ & NO & 4340\\
$(227,67)$ & 12 & $(44,13)$ & 8 & 1 & YES & YES & YES & $1.50$ & $(2,3)$ & NO & 4341\\
$(227,67)$ & 12 & $(61,18)$ & 9 & 1 & YES & YES & YES & $1.57$ & $(2,3)$ & NO & 4342\\
$(228,61)$ & 13 & $(2,1)$ & 1 & 2 & YES & YES & YES & $1.43$ & $(2,3)$ & NO & 4343\\
$(228,61)$ & 13 & $(15,4)$ & 6 & 3 & YES & YES & YES & $1.57$ & $(2,3)$ & NO & 4344\\
$(229,100)$ & 13 & $(2,1)$ & 1 & 1 & NO & YES & YES & $1.62$ & $(2,3)$ & -- & 4345\\
$(229,68)$ & 12 & $(3,1)$ & 2 & 1 & YES & YES & YES & $1.43$ & $(2,3)$ & -- & 4346\\
$(229,87)$ & 12 & $(29,11)$ & 7 & 1 & YES & YES & YES & $1.71$ & $(2,3)$ & NO & 4347\\
$(230,67)$ & 13 & $(103,30)$ & 11 & 1 & YES & YES & YES & $1.43$ & $(2,3)$ & NO & 4348\\
$(230,97)$ & 12 & $(230,97)$ & 12 & 230 & YES & YES & YES & $1.71$ & $(2,3)$ & NO & 4349\\
$(231,53)$ & 13 & $(2,1)$ & 1 & 1 & YES & YES & YES & $1.43$ & $(2,3)$ & -- & 4350\\
$(231,67)$ & 13 & $(2,1)$ & 1 & 1 & YES & YES & YES & $1.43$ & $(2,3)$ & -- & 4351\\
$(231,53)$ & 13 & $(3,1)$ & 2 & 3 & YES & YES & YES & $1.43$ & $(2,3)$ & -- & 4352\\
$(231,61)$ & 13 & $(3,1)$ & 2 & 3 & YES & YES & YES & $1.43$ & $(2,3)$ & -- & 4353\\
$(231,83)$ & 12 & $(3,1)$ & 2 & 3 & YES & YES & YES & $1.71$ & $(2,3)$ & -- & 4354\\
$(231,53)$ & 13 & $(4,1)$ & 3 & 1 & YES & YES & YES & $1.43$ & $(2,3)$ & NO & 4355\\
$(231,61)$ & 13 & $(34,9)$ & 8 & 1 & YES & YES & YES & $1.43$ & $(2,3)$ & NO & 4356\\
$(231,67)$ & 13 & $(100,29)$ & 11 & 1 & YES & YES & YES & $1.43$ & $(2,3)$ & NO & 4357\\
$(232,101)$ & 12 & $(5,2)$ & 3 & 1 & YES & YES & YES & $1.43$ & $(2,3)$ & NO & 4358\\
$(234,43)$ & 14 & $(6,1)$ & 5 & 6 & YES & YES & YES & $1.29$ & $(2,3)$ & NO & 4359\\
$(235,63)$ & 12 & $(3,1)$ & 2 & 1 & YES & YES & YES & $1.43$ & $(2,3)$ & -- & 4360\\
$(235,97)$ & 12 & $(3,1)$ & 2 & 1 & YES & YES & YES & $1.57$ & $(2,3)$ & -- & 4361\\
$(235,63)$ & 12 & $(26,7)$ & 7 & 1 & YES & YES & YES & $1.57$ & $(2,3)$ & NO & 4362\\
$(237,100)$ & 12 & $(5,2)$ & 3 & 1 & YES & YES & YES & $1.57$ & $(2,3)$ & NO & 4363\\
$(237,100)$ & 12 & $(6,1)$ & 5 & 3 & YES & YES & YES & $1.57$ & $(2,3)$ & NO & 4364\\
$(237,100)$ & 12 & $(64,27)$ & 9 & 1 & YES & YES & YES & $1.71$ & $(2,3)$ & NO & 4365\\
$(239,56)$ & 13 & $(2,1)$ & 1 & 1 & YES & YES & YES & $1.38$ & $(2,3)$ & NO & 4366\\
$(239,99)$ & 12 & $(2,1)$ & 1 & 1 & YES & YES & YES & $1.43$ & $(2,3)$ & -- & 4367\\
$(239,100)$ & 12 & $(2,1)$ & 1 & 1 & NO & YES & YES & $1.62$ & $(2,3)$ & -- & 4368\\
$(239,71)$ & 12 & $(10,3)$ & 5 & 1 & YES & YES & YES & $1.43$ & $(2,3)$ & NO & 4369\\
$(239,101)$ & 12 & $(12,5)$ & 5 & 1 & YES & YES & YES & $1.57$ & $(2,3)$ & NO & 4370\\
$(239,71)$ & 12 & $(37,11)$ & 8 & 1 & YES & YES & YES & $1.43$ & $(2,3)$ & 4122 & 4371\\
$(240,71)$ & 12 & $(2,1)$ & 1 & 2 & YES & YES & YES & $1.50$ & $(2,3)$ & -- & 4372\\
$(240,71)$ & 12 & $(10,3)$ & 5 & 10 & YES & YES & YES & $1.50$ & $(2,3)$ & 3843 & 4373\\
$(241,52)$ & 13 & $(2,1)$ & 1 & 1 & YES & YES & YES & $1.50$ & $(2,3)$ & -- & 4374\\
$(241,105)$ & 12 & $(3,1)$ & 2 & 1 & YES & YES & YES & $1.57$ & $(2,3)$ & -- & 4375\\
$(241,89)$ & 12 & $(5,2)$ & 3 & 1 & YES & YES & YES & $1.57$ & $(2,3)$ & NO & 4376\\
$(241,105)$ & 12 & $(5,2)$ & 3 & 1 & YES & YES & YES & $1.57$ & $(2,3)$ & NO & 4377\\
$(241,89)$ & 12 & $(6,1)$ & 5 & 1 & YES & YES & YES & $1.43$ & $(2,3)$ & -- & 4378\\
$(241,89)$ & 12 & $(19,7)$ & 6 & 1 & YES & YES & YES & $1.57$ & $(2,3)$ & NO & 4379\\
$(241,89)$ & 12 & $(65,24)$ & 9 & 1 & YES & YES & YES & $1.57$ & $(2,3)$ & NO & 4380\\
$(241,92)$ & 12 & $(241,92)$ & 12 & 241 & YES & YES & YES & $1.71$ & $(2,3)$ & NO & 4381\\
$(242,65)$ & 12 & $(2,1)$ & 1 & 2 & YES & YES & YES & $1.57$ & $(2,3)$ & -- & 4382\\
$(243,71)$ & 12 & $(3,1)$ & 2 & 3 & YES & YES & YES & $1.57$ & $(2,3)$ & -- & 4383\\
$(243,89)$ & 12 & $(3,1)$ & 2 & 3 & YES & YES & YES & $1.71$ & $(2,3)$ & -- & 4384\\
$(244,55)$ & 13 & $(53,12)$ & 9 & 1 & YES & YES & YES & $1.43$ & $(2,3)$ & NO & 4385\\
$(244,71)$ & 13 & $(86,25)$ & 10 & 2 & YES & YES & YES & $1.71$ & $(2,3)$ & NO & 4386\\
$(245,93)$ & 12 & $(245,93)$ & 12 & 245 & YES & YES & YES & $1.57$ & $(2,3)$ & NO & 4387\\
$(246,53)$ & 13 & $(2,1)$ & 1 & 2 & YES & YES & YES & $1.43$ & $(2,3)$ & NO & 4388\\
$(246,65)$ & 13 & $(2,1)$ & 1 & 2 & YES & YES & YES & $1.43$ & $(2,3)$ & -- & 4389\\
$(246,73)$ & 12 & $(2,1)$ & 1 & 2 & YES & YES & YES & $1.50$ & $(2,3)$ & NO & 4390\\
$(246,91)$ & 12 & $(2,1)$ & 1 & 2 & YES & YES & YES & $1.43$ & $(2,3)$ & -- & 4391\\
$(246,91)$ & 12 & $(27,10)$ & 7 & 3 & YES & YES & YES & $1.43$ & $(2,3)$ & NO & 4392\\
$(246,103)$ & 13 & $(31,13)$ & 7 & 1 & YES & YES & YES & $1.71$ & $(2,3)$ & NO & 4393\\
$(246,73)$ & 12 & $(91,27)$ & 10 & 1 & YES & YES & YES & $1.50$ & $(2,3)$ & NO & 4394\\
$(247,102)$ & 13 & $(3,1)$ & 2 & 1 & YES & YES & YES & $1.71$ & $(2,3)$ & -- & 4395\\
$(248,91)$ & 12 & $(19,7)$ & 6 & 1 & YES & YES & YES & $1.71$ & $(2,3)$ & NO & 4396\\
$(248,95)$ & 13 & $(21,8)$ & 6 & 1 & YES & YES & YES & $1.71$ & $(2,3)$ & NO & 4397\\
$(249,95)$ & 12 & $(2,1)$ & 1 & 1 & YES & YES & YES & $1.43$ & $(4,2)$ & -- & 4398\\
$(249,95)$ & 12 & $(249,95)$ & 12 & 249 & YES & YES & YES & $1.57$ & $(2,3)$ & NO & 4399\\
$(250,67)$ & 12 & $(2,1)$ & 1 & 2 & YES & YES & YES & $1.43$ & $(2,3)$ & NO & 4400\\
$(250,67)$ & 12 & $(3,1)$ & 2 & 1 & YES & YES & YES & $1.29$ & $(2,3)$ & -- & 4401\\
$(250,67)$ & 12 & $(3,1)$ & 2 & 1 & YES & YES & YES & $1.43$ & $(2,3)$ & NO & 4402\\
$(250,67)$ & 12 & $(97,26)$ & 10 & 1 & YES & YES & YES & $1.29$ & $(2,3)$ & NO & 4403\\
$(251,54)$ & 14 & $(5,1)$ & 4 & 1 & YES & YES & YES & $1.57$ & $(2,3)$ & NO & 4404\\
$(251,104)$ & 12 & $(29,12)$ & 7 & 1 & YES & YES & YES & $1.57$ & $(2,3)$ & NO & 4405\\
$(251,74)$ & 13 & $(251,74)$ & 13 & 251 & YES & YES & YES & $1.57$ & $(2,3)$ & NO & 4406\\
$(252,47)$ & 14 & $(3,1)$ & 2 & 3 & YES & YES & YES & $1.50$ & $(2,3)$ & NO & 4407\\
$(252,47)$ & 14 & $(3,1)$ & 2 & 3 & YES & YES & YES & $1.50$ & $(2,3)$ & -- & 4408\\
$(252,47)$ & 14 & $(11,2)$ & 6 & 1 & YES & YES & NO(2) & $1.44$ & $(2,3)$ & NO & 4409\\
$(252,47)$ & 14 & $(27,5)$ & 8 & 9 & YES & YES & YES & $1.50$ & $(2,3)$ & NO & 4410\\
$(253,74)$ & 12 & $(2,1)$ & 1 & 1 & YES & YES & YES & $1.57$ & $(2,3)$ & -- & 4411\\
$(253,106)$ & 12 & $(5,2)$ & 3 & 1 & YES & YES & YES & $1.29$ & $(4,2)$ & NO & 4412\\
$(253,106)$ & 12 & $(19,8)$ & 6 & 1 & YES & YES & YES & $1.86$ & $(2,3)$ & 4167 & 4413\\
$(254,105)$ & 12 & $(17,7)$ & 6 & 1 & YES & YES & YES & $1.43$ & $(4,2)$ & NO & 4414\\
$(254,97)$ & 12 & $(76,29)$ & 9 & 2 & YES & YES & YES & $1.57$ & $(2,3)$ & NO & 4415\\
$(256,97)$ & 12 & $(4,1)$ & 3 & 4 & YES & YES & YES & $1.57$ & $(2,3)$ & NO & 4416\\
$(256,99)$ & 12 & $(5,2)$ & 3 & 1 & YES & YES & YES & $1.71$ & $(2,3)$ & -- & 4417\\
$(256,99)$ & 12 & $(8,3)$ & 4 & 8 & YES & YES & YES & $1.62$ & $(2,3)$ & NO & 4418\\
$(256,99)$ & 12 & $(44,17)$ & 8 & 4 & YES & YES & YES & $1.62$ & $(2,3)$ & NO & 4419\\
$(257,69)$ & 12 & $(2,1)$ & 1 & 1 & YES & YES & YES & $1.43$ & $(2,3)$ & NO & 4420\\
$(257,69)$ & 12 & $(3,1)$ & 2 & 1 & YES & YES & YES & $1.29$ & $(2,3)$ & NO & 4421\\
$(257,69)$ & 12 & $(4,1)$ & 3 & 1 & YES & YES & YES & $1.57$ & $(2,3)$ & -- & 4422\\
$(257,69)$ & 12 & $(11,3)$ & 5 & 1 & YES & YES & YES & $1.29$ & $(2,3)$ & NO & 4423\\
$(257,78)$ & 13 & $(23,7)$ & 7 & 1 & YES & YES & YES & $1.57$ & $(2,3)$ & NO & 4424\\
$(257,69)$ & 12 & $(26,7)$ & 7 & 1 & YES & YES & YES & $1.43$ & $(2,3)$ & 3921 & 4425\\
$(258,55)$ & 14 & $(6,1)$ & 5 & 6 & YES & YES & YES & $1.43$ & $(2,3)$ & NO & 4426\\
$(258,71)$ & 12 & $(11,3)$ & 5 & 1 & YES & YES & YES & $1.50$ & $(2,3)$ & NO & 4427\\
$(258,55)$ & 14 & $(258,55)$ & 14 & 258 & YES & YES & YES & $1.43$ & $(2,3)$ & NO & 4428\\
$(259,100)$ & 12 & $(3,1)$ & 2 & 1 & YES & YES & YES & $1.57$ & $(2,3)$ & -- & 4429\\
$(259,100)$ & 12 & $(3,1)$ & 2 & 1 & YES & YES & YES & $1.57$ & $(2,3)$ & NO & 4430\\
$(259,100)$ & 12 & $(18,7)$ & 6 & 1 & YES & YES & YES & $1.57$ & $(2,3)$ & 3991 & 4431\\
$(259,59)$ & 13 & $(79,18)$ & 10 & 1 & YES & YES & YES & $1.38$ & $(2,3)$ & NO & 4432\\
$(261,100)$ & 12 & $(2,1)$ & 1 & 1 & YES & YES & YES & $1.43$ & $(2,3)$ & -- & 4433\\
$(261,76)$ & 13 & $(4,1)$ & 3 & 1 & YES & YES & YES & $1.43$ & $(2,3)$ & -- & 4434\\
$(261,76)$ & 13 & $(7,2)$ & 4 & 1 & YES & YES & YES & $1.43$ & $(2,3)$ & 3567 & 4435\\
$(261,76)$ & 13 & $(24,7)$ & 7 & 3 & YES & YES & YES & $1.43$ & $(2,3)$ & NO & 4436\\
$(261,100)$ & 12 & $(261,100)$ & 12 & 261 & YES & YES & YES & $1.43$ & $(2,3)$ & NO & 4437\\
$(263,61)$ & 14 & $(2,1)$ & 1 & 1 & YES & YES & YES & $1.43$ & $(2,3)$ & -- & 4438\\
$(263,78)$ & 13 & $(5,1)$ & 4 & 1 & YES & YES & YES & $1.57$ & $(2,3)$ & -- & 4439\\
$(263,61)$ & 14 & $(17,4)$ & 7 & 1 & YES & YES & YES & $1.43$ & $(2,3)$ & NO & 4440\\
$(263,109)$ & 12 & $(17,7)$ & 6 & 1 & YES & YES & YES & $1.57$ & $(2,3)$ & NO & 4441\\
$(263,61)$ & 14 & $(69,16)$ & 11 & 1 & YES & YES & YES & $1.43$ & $(2,3)$ & NO & 4442\\
$(263,78)$ & 13 & $(91,27)$ & 10 & 1 & YES & YES & YES & $1.43$ & $(2,3)$ & NO & 4443\\
$(265,73)$ & 12 & $(2,1)$ & 1 & 1 & YES & YES & YES & $1.50$ & $(2,3)$ & NO & 4444\\
$(265,112)$ & 12 & $(4,1)$ & 3 & 1 & YES & YES & YES & $1.57$ & $(2,3)$ & NO & 4445\\
$(265,112)$ & 12 & $(5,2)$ & 3 & 5 & YES & YES & YES & $1.43$ & $(2,3)$ & NO & 4446\\
$(265,73)$ & 12 & $(98,27)$ & 10 & 1 & YES & YES & YES & $1.50$ & $(2,3)$ & NO & 4447\\
$(266,101)$ & 12 & $(8,3)$ & 4 & 2 & YES & YES & YES & $1.71$ & $(2,3)$ & 3862 & 4448\\
$(268,111)$ & 12 & $(41,17)$ & 8 & 1 & YES & YES & YES & $1.43$ & $(2,3)$ & 4517 & 4449\\
$(269,113)$ & 13 & $(3,1)$ & 2 & 1 & YES & YES & YES & $1.86$ & $(2,3)$ & NO & 4450\\
$(269,113)$ & 13 & $(3,1)$ & 2 & 1 & YES & YES & YES & $1.86$ & $(2,3)$ & -- & 4451\\
$(269,113)$ & 13 & $(12,5)$ & 5 & 1 & YES & YES & YES & $1.71$ & $(2,3)$ & NO & 4452\\
$(269,104)$ & 12 & $(13,5)$ & 5 & 1 & YES & YES & YES & $1.43$ & $(4,2)$ & NO & 4453\\
$(269,104)$ & 12 & $(18,7)$ & 6 & 1 & YES & YES & YES & $1.71$ & $(2,3)$ & NO & 4454\\
$(269,78)$ & 13 & $(24,7)$ & 7 & 1 & YES & YES & YES & $1.43$ & $(2,3)$ & NO & 4455\\
$(269,104)$ & 12 & $(31,12)$ & 7 & 1 & YES & YES & YES & $1.62$ & $(2,3)$ & NO & 4456\\
$(269,103)$ & 13 & $(34,13)$ & 7 & 1 & YES & YES & YES & $1.71$ & $(2,3)$ & NO & 4457\\
$(269,104)$ & 12 & $(44,17)$ & 8 & 1 & YES & YES & YES & $1.57$ & $(2,3)$ & NO & 4458\\
$(271,112)$ & 12 & $(3,1)$ & 2 & 1 & YES & YES & YES & $1.57$ & $(2,3)$ & -- & 4459\\
$(271,112)$ & 12 & $(12,5)$ & 5 & 1 & YES & YES & YES & $1.43$ & $(2,3)$ & NO & 4460\\
$(272,65)$ & 14 & $(3,1)$ & 2 & 1 & YES & YES & YES & $1.57$ & $(2,3)$ & -- & 4461\\
$(273,80)$ & 13 & $(3,1)$ & 2 & 3 & YES & YES & YES & $1.57$ & $(2,3)$ & NO & 4462\\
$(274,115)$ & 12 & $(2,1)$ & 1 & 2 & YES & YES & YES & $1.43$ & $(2,3)$ & -- & 4463\\
$(274,115)$ & 12 & $(3,1)$ & 2 & 1 & YES & YES & YES & $1.71$ & $(2,3)$ & -- & 4464\\
$(275,101)$ & 13 & $(30,11)$ & 7 & 5 & YES & YES & YES & $1.71$ & $(2,3)$ & NO & 4465\\
$(277,106)$ & 12 & $(3,1)$ & 2 & 1 & YES & YES & YES & $1.71$ & $(2,3)$ & NO & 4466\\
$(277,106)$ & 12 & $(8,3)$ & 4 & 1 & YES & YES & YES & $1.50$ & $(2,3)$ & NO & 4467\\
$(278,63)$ & 13 & $(31,7)$ & 8 & 1 & YES & YES & YES & $1.38$ & $(2,3)$ & NO & 4468\\
$(280,107)$ & 12 & $(123,47)$ & 10 & 1 & YES & YES & YES & $1.86$ & $(2,3)$ & NO & 4469\\
$(281,116)$ & 12 & $(2,1)$ & 1 & 1 & NO & YES & YES & $1.43$ & $(2,3)$ & -- & 4470\\
$(281,116)$ & 12 & $(7,3)$ & 4 & 1 & YES & YES & YES & $1.57$ & $(2,3)$ & NO & 4471\\
$(281,109)$ & 12 & $(8,3)$ & 4 & 1 & YES & YES & YES & $1.57$ & $(2,3)$ & NO & 4472\\
$(281,116)$ & 12 & $(12,5)$ & 5 & 1 & YES & YES & YES & $1.43$ & $(2,3)$ & 4101 & 4473\\
$(281,109)$ & 12 & $(31,12)$ & 7 & 1 & YES & YES & YES & $1.57$ & $(2,3)$ & NO & 4474\\
$(281,109)$ & 12 & $(67,26)$ & 9 & 1 & YES & YES & YES & $1.57$ & $(2,3)$ & NO & 4475\\
$(281,109)$ & 12 & $(281,109)$ & 12 & 281 & YES & YES & YES & $1.57$ & $(2,3)$ & NO & 4476\\
$(282,109)$ & 12 & $(5,1)$ & 4 & 1 & YES & YES & YES & $1.57$ & $(2,3)$ & NO & 4477\\
$(282,109)$ & 12 & $(13,5)$ & 5 & 1 & YES & YES & YES & $1.57$ & $(2,3)$ & NO & 4478\\
$(282,109)$ & 12 & $(44,17)$ & 8 & 2 & YES & YES & YES & $1.57$ & $(2,3)$ & 4292 & 4479\\
$(282,109)$ & 12 & $(282,109)$ & 12 & 282 & YES & YES & YES & $1.43$ & $(2,3)$ & NO & 4480\\
$(283,75)$ & 13 & $(2,1)$ & 1 & 1 & YES & YES & YES & $1.43$ & $(2,3)$ & NO & 4481\\
$(283,84)$ & 13 & $(3,1)$ & 2 & 1 & YES & YES & YES & $1.71$ & $(2,3)$ & -- & 4482\\
$(283,84)$ & 13 & $(3,1)$ & 2 & 1 & YES & YES & YES & $2.00$ & $(2,3)$ & NO & 4483\\
$(283,64)$ & 13 & $(4,1)$ & 3 & 1 & YES & YES & YES & $1.38$ & $(2,3)$ & -- & 4484\\
$(283,84)$ & 13 & $(37,11)$ & 8 & 1 & YES & YES & YES & $1.71$ & $(2,3)$ & NO & 4485\\
$(283,64)$ & 13 & $(40,9)$ & 9 & 1 & YES & YES & YES & $1.43$ & $(2,3)$ & NO & 4486\\
$(283,108)$ & 12 & $(55,21)$ & 8 & 1 & YES & YES & YES & $2.00$ & $(2,3)$ & NO & 4487\\
$(283,64)$ & 13 & $(84,19)$ & 10 & 1 & YES & YES & YES & $1.38$ & $(2,3)$ & NO & 4488\\
$(283,83)$ & 13 & $(283,83)$ & 13 & 283 & YES & YES & YES & $1.57$ & $(2,3)$ & NO & 4489\\
$(284,65)$ & 13 & $(3,1)$ & 2 & 1 & YES & YES & YES & $1.50$ & $(2,3)$ & -- & 4490\\
$(284,83)$ & 13 & $(41,12)$ & 8 & 1 & YES & YES & YES & $1.57$ & $(2,3)$ & NO & 4491\\
$(286,61)$ & 14 & $(2,1)$ & 1 & 2 & YES & YES & YES & $1.43$ & $(2,3)$ & NO & 4492\\
$(286,67)$ & 13 & $(3,1)$ & 2 & 1 & YES & YES & YES & $1.43$ & $(2,3)$ & NO & 4493\\
$(286,67)$ & 13 & $(3,1)$ & 2 & 1 & YES & YES & YES & $1.43$ & $(2,3)$ & -- & 4494\\
$(286,61)$ & 14 & $(5,1)$ & 4 & 1 & YES & YES & YES & $1.43$ & $(2,3)$ & NO & 4495\\
$(286,105)$ & 12 & $(8,3)$ & 4 & 2 & YES & YES & YES & $1.86$ & $(2,3)$ & NO & 4496\\
$(287,111)$ & 12 & $(4,1)$ & 3 & 1 & YES & YES & YES & $1.43$ & $(2,3)$ & -- & 4497\\
$(287,111)$ & 12 & $(4,1)$ & 3 & 1 & YES & YES & YES & $1.57$ & $(2,3)$ & NO & 4498\\
$(287,80)$ & 13 & $(25,7)$ & 7 & 1 & YES & YES & YES & $1.86$ & $(2,3)$ & NO & 4499\\
$(287,111)$ & 12 & $(287,111)$ & 12 & 287 & YES & YES & YES & $1.43$ & $(2,3)$ & NO & 4500\\
$(288,121)$ & 12 & $(7,3)$ & 4 & 1 & YES & YES & YES & $1.71$ & $(2,3)$ & NO & 4501\\
$(289,84)$ & 13 & $(3,1)$ & 2 & 1 & YES & YES & YES & $1.57$ & $(2,3)$ & -- & 4502\\
$(289,112)$ & 12 & $(3,1)$ & 2 & 1 & YES & YES & YES & $1.71$ & $(2,3)$ & -- & 4503\\
$(289,112)$ & 12 & $(3,1)$ & 2 & 1 & YES & YES & YES & $1.86$ & $(2,3)$ & NO & 4504\\
$(289,112)$ & 12 & $(13,5)$ & 5 & 1 & YES & YES & YES & $1.71$ & $(2,3)$ & NO & 4505\\
$(289,112)$ & 12 & $(18,7)$ & 6 & 1 & YES & YES & YES & $1.71$ & $(2,3)$ & NO & 4506\\
$(289,112)$ & 12 & $(49,19)$ & 8 & 1 & YES & YES & YES & $1.86$ & $(2,3)$ & NO & 4507\\
$(289,86)$ & 13 & $(121,36)$ & 11 & 1 & YES & YES & YES & $1.57$ & $(2,3)$ & 4585 & 4508\\
$(289,112)$ & 12 & $(129,50)$ & 10 & 1 & YES & YES & YES & $1.71$ & $(2,3)$ & 4597 & 4509\\
$(289,86)$ & 13 & $(289,86)$ & 13 & 289 & YES & YES & YES & $1.71$ & $(2,3)$ & NO & 4510\\
$(291,85)$ & 13 & $(3,1)$ & 2 & 3 & YES & YES & YES & $1.71$ & $(2,3)$ & NO & 4511\\
$(291,85)$ & 13 & $(4,1)$ & 3 & 1 & YES & YES & YES & $1.57$ & $(2,3)$ & -- & 4512\\
$(291,85)$ & 13 & $(17,5)$ & 6 & 1 & YES & YES & YES & $1.57$ & $(2,3)$ & 3657 & 4513\\
$(292,121)$ & 12 & $(3,1)$ & 2 & 1 & YES & YES & YES & $1.71$ & $(2,3)$ & -- & 4514\\
$(292,121)$ & 12 & $(7,3)$ & 4 & 1 & YES & YES & YES & $1.57$ & $(2,3)$ & NO & 4515\\
$(292,121)$ & 12 & $(17,7)$ & 6 & 1 & YES & YES & YES & $1.57$ & $(2,3)$ & 4579 & 4516\\
$(292,121)$ & 12 & $(29,12)$ & 7 & 1 & YES & YES & YES & $1.43$ & $(2,3)$ & 4449 & 4517\\
$(292,121)$ & 12 & $(41,17)$ & 8 & 1 & YES & YES & YES & $1.43$ & $(2,3)$ & NO & 4518\\
$(292,111)$ & 12 & $(71,27)$ & 9 & 1 & YES & YES & YES & $1.71$ & $(2,3)$ & NO & 4519\\
$(293,52)$ & 14 & $(2,1)$ & 1 & 1 & YES & YES & YES & $1.43$ & $(2,3)$ & NO & 4520\\
$(293,123)$ & 12 & $(2,1)$ & 1 & 1 & YES & YES & YES & $1.43$ & $(2,3)$ & -- & 4521\\
$(293,89)$ & 13 & $(6,1)$ & 5 & 1 & YES & YES & YES & $1.43$ & $(2,3)$ & NO & 4522\\
$(293,123)$ & 12 & $(7,3)$ & 4 & 1 & YES & YES & YES & $1.57$ & $(2,3)$ & NO & 4523\\
$(293,89)$ & 13 & $(23,7)$ & 7 & 1 & YES & YES & YES & $1.57$ & $(2,3)$ & NO & 4524\\
$(293,52)$ & 14 & $(45,8)$ & 9 & 1 & YES & YES & YES & $1.43$ & $(2,3)$ & NO & 4525\\
$(293,89)$ & 13 & $(214,65)$ & 12 & 1 & YES & YES & YES & $1.43$ & $(2,3)$ & NO & 4526\\
$(294,67)$ & 13 & $(3,1)$ & 2 & 3 & YES & YES & YES & $1.43$ & $(2,3)$ & NO & 4527\\
$(294,67)$ & 13 & $(3,1)$ & 2 & 3 & YES & YES & YES & $1.43$ & $(2,3)$ & -- & 4528\\
$(294,67)$ & 13 & $(136,31)$ & 11 & 2 & YES & YES & YES & $1.43$ & $(2,3)$ & 4607 & 4529\\
$(295,112)$ & 12 & $(3,1)$ & 2 & 1 & YES & YES & YES & $1.43$ & $(2,3)$ & -- & 4530\\
$(295,87)$ & 13 & $(61,18)$ & 9 & 1 & YES & YES & YES & $1.57$ & $(2,3)$ & NO & 4531\\
$(296,113)$ & 13 & $(2,1)$ & 1 & 2 & YES & YES & YES & $1.71$ & $(2,3)$ & -- & 4532\\
$(296,113)$ & 13 & $(3,1)$ & 2 & 1 & YES & YES & YES & $1.71$ & $(2,3)$ & NO & 4533\\
$(296,113)$ & 13 & $(13,5)$ & 5 & 1 & YES & YES & YES & $1.71$ & $(2,3)$ & NO & 4534\\
$(296,113)$ & 13 & $(21,8)$ & 6 & 1 & YES & YES & YES & $1.71$ & $(2,3)$ & NO & 4535\\
$(298,79)$ & 13 & $(19,5)$ & 7 & 1 & YES & YES & YES & $1.43$ & $(2,3)$ & NO & 4536\\
$(298,83)$ & 13 & $(140,39)$ & 11 & 2 & YES & YES & YES & $1.71$ & $(2,3)$ & 4613 & 4537\\
$(299,116)$ & 12 & $(8,3)$ & 4 & 1 & YES & YES & YES & $1.57$ & $(2,3)$ & NO & 4538\\
$(300,89)$ & 13 & $(3,1)$ & 2 & 3 & YES & YES & YES & $1.57$ & $(2,3)$ & -- & 4539\\
$(300,91)$ & 13 & $(10,3)$ & 5 & 10 & YES & YES & YES & $1.43$ & $(2,3)$ & 4026 & 4540\\
$(300,89)$ & 13 & $(17,5)$ & 6 & 1 & YES & YES & YES & $1.57$ & $(2,3)$ & NO & 4541\\
$(300,91)$ & 13 & $(33,10)$ & 8 & 3 & YES & YES & YES & $1.43$ & $(2,3)$ & NO & 4542\\
$(300,89)$ & 13 & $(118,35)$ & 11 & 2 & YES & YES & YES & $1.57$ & $(2,3)$ & 4586 & 4543\\
$(300,89)$ & 13 & $(300,89)$ & 13 & 300 & YES & YES & YES & $1.57$ & $(2,3)$ & NO & 4544\\
$(301,115)$ & 12 & $(2,1)$ & 1 & 1 & YES & YES & YES & $1.86$ & $(2,3)$ & -- & 4545\\
$(301,88)$ & 13 & $(4,1)$ & 3 & 1 & YES & YES & YES & $1.71$ & $(2,3)$ & NO & 4546\\
$(301,115)$ & 12 & $(5,2)$ & 3 & 1 & YES & YES & YES & $1.86$ & $(2,3)$ & NO & 4547\\
$(301,88)$ & 13 & $(24,7)$ & 7 & 1 & YES & YES & YES & $1.57$ & $(2,3)$ & NO & 4548\\
$(301,115)$ & 12 & $(34,13)$ & 7 & 1 & YES & YES & YES & $1.86$ & $(2,3)$ & NO & 4549\\
$(304,85)$ & 13 & $(3,1)$ & 2 & 1 & YES & YES & YES & $1.71$ & $(2,3)$ & NO & 4550\\
$(304,85)$ & 13 & $(3,1)$ & 2 & 1 & YES & YES & YES & $1.71$ & $(2,3)$ & -- & 4551\\
$(305,69)$ & 13 & $(2,1)$ & 1 & 1 & YES & YES & YES & $1.38$ & $(2,3)$ & NO & 4552\\
$(307,70)$ & 14 & $(3,1)$ & 2 & 1 & YES & YES & YES & $1.71$ & $(2,3)$ & -- & 4553\\
$(307,129)$ & 12 & $(7,3)$ & 4 & 1 & YES & YES & YES & $1.57$ & $(2,3)$ & NO & 4554\\
$(307,119)$ & 12 & $(8,3)$ & 4 & 1 & YES & YES & YES & $1.57$ & $(2,3)$ & NO & 4555\\
$(307,69)$ & 14 & $(40,9)$ & 9 & 1 & YES & YES & YES & $1.43$ & $(2,3)$ & NO & 4556\\
$(308,117)$ & 12 & $(3,1)$ & 2 & 1 & YES & YES & YES & $1.57$ & $(2,3)$ & -- & 4557\\
$(308,129)$ & 12 & $(7,3)$ & 4 & 7 & YES & YES & YES & $1.71$ & $(2,3)$ & NO & 4558\\
$(311,71)$ & 13 & $(2,1)$ & 1 & 1 & YES & YES & YES & $1.38$ & $(2,3)$ & -- & 4559\\
$(311,71)$ & 13 & $(13,3)$ & 6 & 1 & YES & YES & YES & $1.38$ & $(2,3)$ & 4151 & 4560\\
$(313,119)$ & 12 & $(2,1)$ & 1 & 1 & YES & YES & YES & $1.57$ & $(2,3)$ & -- & 4561\\
$(313,121)$ & 12 & $(2,1)$ & 1 & 1 & YES & YES & YES & $1.57$ & $(2,3)$ & NO & 4562\\
$(313,93)$ & 13 & $(5,1)$ & 4 & 1 & YES & YES & YES & $1.71$ & $(2,3)$ & NO & 4563\\
$(313,93)$ & 13 & $(7,2)$ & 4 & 1 & YES & YES & YES & $1.57$ & $(2,3)$ & NO & 4564\\
$(313,121)$ & 12 & $(8,3)$ & 4 & 1 & YES & YES & YES & $1.71$ & $(2,3)$ & NO & 4565\\
$(313,121)$ & 12 & $(13,5)$ & 5 & 1 & YES & YES & YES & $1.57$ & $(2,3)$ & NO & 4566\\
$(313,121)$ & 12 & $(18,7)$ & 6 & 1 & YES & YES & YES & $1.57$ & $(2,3)$ & 4596 & 4567\\
$(313,86)$ & 13 & $(131,36)$ & 11 & 1 & YES & YES & YES & $1.57$ & $(2,3)$ & 4609 & 4568\\
$(314,83)$ & 13 & $(2,1)$ & 1 & 2 & YES & YES & YES & $1.43$ & $(2,3)$ & -- & 4569\\
$(314,83)$ & 13 & $(3,1)$ & 2 & 1 & YES & YES & YES & $1.43$ & $(2,3)$ & -- & 4570\\
$(315,88)$ & 13 & $(3,1)$ & 2 & 3 & YES & YES & YES & $1.71$ & $(2,3)$ & NO & 4571\\
$(315,68)$ & 13 & $(5,1)$ & 4 & 5 & YES & YES & YES & $1.29$ & $(2,3)$ & NO & 4572\\
$(315,68)$ & 13 & $(14,3)$ & 6 & 7 & YES & YES & YES & $1.29$ & $(2,3)$ & NO & 4573\\
$(315,88)$ & 13 & $(18,5)$ & 6 & 9 & YES & YES & YES & $1.71$ & $(2,3)$ & NO & 4574\\
$(315,88)$ & 13 & $(25,7)$ & 7 & 5 & YES & YES & YES & $1.86$ & $(2,3)$ & NO & 4575\\
$(317,131)$ & 12 & $(2,1)$ & 1 & 1 & YES & YES & YES & $1.43$ & $(2,3)$ & -- & 4576\\
$(317,131)$ & 12 & $(2,1)$ & 1 & 1 & YES & YES & YES & $1.43$ & $(2,3)$ & NO & 4577\\
$(317,121)$ & 12 & $(8,3)$ & 4 & 1 & YES & YES & YES & $1.71$ & $(2,3)$ & NO & 4578\\
$(317,131)$ & 12 & $(12,5)$ & 5 & 1 & YES & YES & YES & $1.57$ & $(2,3)$ & 4516 & 4579\\
$(322,123)$ & 12 & $(3,1)$ & 2 & 1 & YES & YES & YES & $1.57$ & $(2,3)$ & -- & 4580\\
$(322,123)$ & 12 & $(5,2)$ & 3 & 1 & YES & YES & YES & $1.71$ & $(2,3)$ & NO & 4581\\
$(322,123)$ & 12 & $(8,3)$ & 4 & 2 & YES & YES & YES & $1.71$ & $(2,3)$ & NO & 4582\\
$(322,123)$ & 12 & $(34,13)$ & 7 & 2 & YES & YES & YES & $1.71$ & $(2,3)$ & NO & 4583\\
$(325,124)$ & 13 & $(76,29)$ & 9 & 1 & YES & YES & YES & $1.57$ & $(2,3)$ & NO & 4584\\
$(326,97)$ & 13 & $(84,25)$ & 10 & 2 & YES & YES & YES & $1.57$ & $(2,3)$ & 4508 & 4585\\
$(327,97)$ & 13 & $(91,27)$ & 10 & 1 & YES & YES & YES & $1.57$ & $(2,3)$ & 4543 & 4586\\
$(334,93)$ & 14 & $(61,17)$ & 9 & 1 & YES & YES & YES & $1.57$ & $(2,3)$ & NO & 4587\\
$(337,94)$ & 13 & $(3,1)$ & 2 & 1 & YES & YES & YES & $1.57$ & $(2,3)$ & -- & 4588\\
$(337,128)$ & 12 & $(3,1)$ & 2 & 1 & YES & YES & YES & $1.43$ & $(2,3)$ & -- & 4589\\
$(337,128)$ & 12 & $(5,1)$ & 4 & 1 & YES & YES & YES & $1.57$ & $(2,3)$ & -- & 4590\\
$(337,129)$ & 12 & $(8,3)$ & 4 & 1 & YES & YES & YES & $1.71$ & $(2,3)$ & NO & 4591\\
$(337,76)$ & 14 & $(9,2)$ & 5 & 1 & YES & YES & YES & $1.50$ & $(2,3)$ & 3812 & 4592\\
$(337,128)$ & 12 & $(21,8)$ & 6 & 1 & YES & YES & YES & $1.57$ & $(2,3)$ & NO & 4593\\
$(337,76)$ & 14 & $(31,7)$ & 8 & 1 & YES & YES & YES & $1.50$ & $(2,3)$ & NO & 4594\\
$(338,129)$ & 12 & $(8,3)$ & 4 & 2 & YES & YES & YES & $1.57$ & $(2,3)$ & NO & 4595\\
$(338,131)$ & 12 & $(13,5)$ & 5 & 13 & YES & YES & YES & $1.57$ & $(2,3)$ & 4567 & 4596\\
$(338,131)$ & 12 & $(80,31)$ & 9 & 2 & YES & YES & YES & $1.71$ & $(2,3)$ & 4509 & 4597\\
$(340,101)$ & 13 & $(2,1)$ & 1 & 2 & YES & YES & YES & $1.43$ & $(2,3)$ & -- & 4598\\
$(341,100)$ & 13 & $(2,1)$ & 1 & 1 & YES & YES & YES & $1.71$ & $(2,3)$ & -- & 4599\\
$(343,131)$ & 12 & $(8,3)$ & 4 & 1 & YES & YES & YES & $1.71$ & $(2,3)$ & NO & 4600\\
$(344,95)$ & 13 & $(3,1)$ & 2 & 1 & YES & YES & YES & $1.71$ & $(2,3)$ & -- & 4601\\
$(344,95)$ & 13 & $(3,1)$ & 2 & 1 & YES & YES & YES & $1.71$ & $(2,3)$ & NO & 4602\\
$(344,95)$ & 13 & $(11,3)$ & 5 & 1 & YES & YES & YES & $1.71$ & $(2,3)$ & NO & 4603\\
$(347,92)$ & 13 & $(2,1)$ & 1 & 1 & YES & YES & YES & $1.43$ & $(2,3)$ & NO & 4604\\
$(347,92)$ & 13 & $(49,13)$ & 9 & 1 & YES & YES & YES & $1.57$ & $(2,3)$ & NO & 4605\\
$(347,97)$ & 13 & $(68,19)$ & 9 & 1 & YES & YES & YES & $1.57$ & $(2,3)$ & NO & 4606\\
$(351,80)$ & 13 & $(79,18)$ & 10 & 1 & YES & YES & YES & $1.43$ & $(2,3)$ & 4529 & 4607\\
$(353,75)$ & 14 & $(6,1)$ & 5 & 1 & YES & YES & YES & $1.43$ & $(2,3)$ & NO & 4608\\
$(353,97)$ & 13 & $(91,25)$ & 10 & 1 & YES & YES & YES & $1.57$ & $(2,3)$ & 4568 & 4609\\
$(359,100)$ & 13 & $(2,1)$ & 1 & 1 & YES & YES & YES & $1.57$ & $(2,3)$ & -- & 4610\\
$(359,100)$ & 13 & $(2,1)$ & 1 & 1 & YES & YES & YES & $1.57$ & $(2,3)$ & NO & 4611\\
$(359,100)$ & 13 & $(3,1)$ & 2 & 1 & YES & YES & YES & $1.43$ & $(2,3)$ & NO & 4612\\
$(359,100)$ & 13 & $(79,22)$ & 10 & 1 & YES & YES & YES & $1.71$ & $(2,3)$ & 4537 & 4613\\
$(360,101)$ & 13 & $(3,1)$ & 2 & 3 & YES & YES & YES & $1.71$ & $(2,3)$ & NO & 4614\\
$(366,79)$ & 13 & $(4,1)$ & 3 & 2 & YES & YES & YES & $1.43$ & $(2,3)$ & NO & 4615\\
$(366,79)$ & 13 & $(4,1)$ & 3 & 2 & YES & YES & YES & $1.57$ & $(2,3)$ & -- & 4616\\
$(366,79)$ & 13 & $(9,2)$ & 5 & 3 & YES & YES & YES & $1.43$ & $(2,3)$ & NO & 4617\\
$(366,97)$ & 14 & $(49,13)$ & 9 & 1 & YES & YES & YES & $1.71$ & $(2,3)$ & NO & 4618\\
$(367,101)$ & 13 & $(2,1)$ & 1 & 1 & YES & YES & YES & $1.43$ & $(2,3)$ & NO & 4619\\
$(370,59)$ & 16 & $(6,1)$ & 5 & 2 & YES & YES & YES & $1.43$ & $(2,3)$ & NO & 4620\\
$(370,59)$ & 16 & $(7,1)$ & 6 & 1 & YES & YES & YES & $1.43$ & $(2,3)$ & NO & 4621\\
$(374,81)$ & 13 & $(9,2)$ & 5 & 1 & YES & YES & YES & $1.57$ & $(2,3)$ & NO & 4622\\
$(375,71)$ & 15 & $(16,3)$ & 7 & 1 & YES & YES & YES & $1.50$ & $(2,3)$ & NO & 4623\\
$(376,111)$ & 13 & $(166,49)$ & 11 & 2 & YES & YES & YES & $1.71$ & $(2,3)$ & 4646 & 4624\\
$(377,70)$ & 15 & $(3,1)$ & 2 & 1 & YES & YES & YES & $1.57$ & $(2,3)$ & NO & 4625\\
$(377,70)$ & 15 & $(97,18)$ & 11 & 1 & YES & YES & YES & $1.57$ & $(2,3)$ & NO & 4626\\
$(381,101)$ & 14 & $(49,13)$ & 9 & 1 & YES & YES & YES & $1.71$ & $(2,3)$ & NO & 4627\\
$(385,82)$ & 14 & $(2,1)$ & 1 & 1 & YES & YES & YES & $1.43$ & $(2,3)$ & NO & 4628\\
$(393,152)$ & 13 & $(2,1)$ & 1 & 1 & YES & YES & YES & $1.57$ & $(2,3)$ & -- & 4629\\
$(393,152)$ & 13 & $(75,29)$ & 9 & 3 & YES & YES & YES & $1.71$ & $(2,3)$ & NO & 4630\\
$(397,154)$ & 13 & $(13,5)$ & 5 & 1 & YES & YES & YES & $1.71$ & $(2,3)$ & NO & 4631\\
$(401,155)$ & 13 & $(31,12)$ & 7 & 1 & YES & YES & YES & $1.57$ & $(2,3)$ & NO & 4632\\
$(407,112)$ & 13 & $(2,1)$ & 1 & 1 & YES & YES & YES & $1.57$ & $(2,3)$ & NO & 4633\\
$(407,112)$ & 13 & $(7,2)$ & 4 & 1 & YES & YES & YES & $1.57$ & $(2,3)$ & NO & 4634\\
$(407,119)$ & 13 & $(10,3)$ & 5 & 1 & YES & YES & YES & $1.57$ & $(2,3)$ & NO & 4635\\
$(413,157)$ & 13 & $(5,2)$ & 3 & 1 & YES & YES & YES & $1.57$ & $(2,3)$ & NO & 4636\\
$(417,65)$ & 16 & $(2,1)$ & 1 & 1 & YES & YES & YES & $1.43$ & $(2,3)$ & -- & 4637\\
$(423,97)$ & 14 & $(48,11)$ & 9 & 3 & YES & YES & YES & $1.43$ & $(2,3)$ & NO & 4638\\
$(427,100)$ & 14 & $(47,11)$ & 9 & 1 & YES & YES & YES & $1.43$ & $(2,3)$ & NO & 4639\\
$(427,100)$ & 14 & $(158,37)$ & 12 & 1 & YES & YES & YES & $1.43$ & $(2,3)$ & NO & 4640\\
$(433,128)$ & 13 & $(159,47)$ & 11 & 1 & YES & YES & YES & $1.57$ & $(2,3)$ & NO & 4641\\
$(437,100)$ & 14 & $(2,1)$ & 1 & 1 & YES & YES & YES & $1.71$ & $(2,3)$ & NO & 4642\\
$(437,169)$ & 13 & $(2,1)$ & 1 & 1 & YES & YES & YES & $1.57$ & $(2,3)$ & -- & 4643\\
$(437,129)$ & 13 & $(7,2)$ & 4 & 1 & YES & YES & YES & $1.43$ & $(2,3)$ & NO & 4644\\
$(437,100)$ & 14 & $(83,19)$ & 10 & 1 & YES & YES & YES & $1.57$ & $(2,3)$ & NO & 4645\\
$(437,129)$ & 13 & $(105,31)$ & 10 & 1 & YES & YES & YES & $1.71$ & $(2,3)$ & 4624 & 4646\\
$(437,169)$ & 13 & $(181,70)$ & 11 & 1 & YES & YES & YES & $1.71$ & $(2,3)$ & NO & 4647\\
$(441,101)$ & 14 & $(2,1)$ & 1 & 1 & YES & YES & YES & $1.43$ & $(2,3)$ & NO & 4648\\
$(441,101)$ & 14 & $(35,8)$ & 8 & 7 & YES & YES & YES & $1.57$ & $(2,3)$ & NO & 4649\\
$(489,106)$ & 14 & $(2,1)$ & 1 & 1 & YES & YES & YES & $1.57$ & $(2,3)$ & -- & 4650\\
$(489,106)$ & 14 & $(14,3)$ & 6 & 1 & YES & YES & YES & $1.43$ & $(2,3)$ & NO & 4651\\
$(a;0,0,0;3)$ & 4 & $(37,14)$ & 8 & 1 & YES & YES & YES & $1.62$ & $(2,3)$ & -- & 4652\\
$(a;0,0,0;3)$ & 4 & $(106,31)$ & 10 & 1 & YES & YES & YES & $2.00$ & $(2,3)$ & -- & 4653\\
$(a;1,0,0;13)$ & 5 & $(24,7)$ & 7 & 1 & YES & YES & YES & $1.62$ & $(2,3)$ & -- & 4654\\
$(a;1,0,0;13)$ & 5 & $(27,10)$ & 7 & 1 & YES & YES & YES & $1.62$ & $(2,3)$ & -- & 4655\\
$(a;1,0,0;13)$ & 5 & $(36,13)$ & 8 & 1 & YES & YES & YES & $1.62$ & $(2,3)$ & -- & 4656\\
$(a;1,0,0;13)$ & 5 & $(41,17)$ & 8 & 1 & YES & YES & YES & $1.43$ & $(2,3)$ & -- & 4657\\
$(a;1,0,0;13)$ & 5 & $(49,13)$ & 9 & 1 & YES & YES & YES & $1.43$ & $(2,3)$ & -- & 4658\\
$(a;1,0,0;13)$ & 5 & $(55,21)$ & 8 & 1 & YES & YES & YES & $1.57$ & $(2,3)$ & -- & 4659\\
$(a;1,1,0;19)$ & 6 & $(31,13)$ & 7 & 1 & YES & YES & YES & $1.43$ & $(2,3)$ & -- & 4660\\
$(a;2,0,0;17)$ & 6 & $(31,9)$ & 8 & 1 & YES & YES & YES & $1.62$ & $(2,3)$ & -- & 4661\\
$(a;2,0,1;25)$ & 7 & $(13,4)$ & 6 & 1 & YES & YES & YES & $1.57$ & $(2,3)$ & -- & 4662\\
$(a;2,0,1;25)$ & 7 & $(13,5)$ & 5 & 1 & YES & YES & YES & $1.62$ & $(2,3)$ & -- & 4663\\
$(a;2,0,1;25)$ & 7 & $(18,7)$ & 6 & 1 & YES & YES & YES & $1.43$ & $(2,3)$ & -- & 4664\\
$(a;2,0,1;25)$ & 7 & $(19,8)$ & 6 & 1 & YES & YES & YES & $1.43$ & $(2,3)$ & -- & 4665\\
$(a;2,1,0;5)$ & 7 & $(11,4)$ & 5 & 1 & YES & YES & YES & $1.62$ & $(2,3)$ & -- & 4666\\
$(a;2,1,0;5)$ & 7 & $(17,7)$ & 6 & 1 & YES & YES & YES & $1.43$ & $(2,3)$ & -- & 4667\\
$(a;2,1,0;5)$ & 7 & $(19,8)$ & 6 & 1 & YES & YES & YES & $1.43$ & $(2,3)$ & -- & 4668\\
$(a;2,1,0;5)$ & 7 & $(24,7)$ & 7 & 1 & YES & YES & YES & $1.43$ & $(2,3)$ & -- & 4669\\
$(a;3,1,0;31)$ & 8 & $(7,3)$ & 4 & 1 & YES & YES & NO(2) & $1.00$ & $(6,1)$ & -- & 4670\\
$(a;3,1,0;31)$ & 8 & $(8,3)$ & 4 & 1 & YES & YES & YES & $1.14$ & $(4,2)$ & -- & 4671\\
$(a;4,0,0;25)$ & 8 & $(8,3)$ & 4 & 1 & YES & YES & YES & $1.29$ & $(2,3)$ & -- & 4672\\
$(a;4,0,0;25)$ & 8 & $(13,4)$ & 6 & 1 & YES & YES & YES & $1.43$ & $(2,3)$ & -- & 4673\\
$(b;0,0,0;14)$ & 5 & $(29,12)$ & 7 & 1 & YES & YES & YES & $1.29$ & $(4,2)$ & -- & 4674\\
$(b;0,0,1;4)$ & 6 & $(18,7)$ & 6 & 2 & YES & YES & YES & $1.43$ & $(2,3)$ & -- & 4675\\
$(b;0,0,1;4)$ & 6 & $(26,11)$ & 7 & 2 & YES & YES & YES & $1.43$ & $(2,3)$ & -- & 4676\\
$(b;0,0,1;4)$ & 6 & $(31,9)$ & 8 & 1 & YES & YES & YES & $1.71$ & $(2,3)$ & -- & 4677\\
$(b;0,0,2;26)$ & 7 & $(23,7)$ & 7 & 1 & YES & YES & YES & $1.71$ & $(2,3)$ & -- & 4678\\
$(b;0,0,3;32)$ & 8 & $(8,3)$ & 4 & 8 & YES & YES & YES & $1.29$ & $(2,3)$ & -- & 4679\\
$(b;0,1,0;19)$ & 6 & $(44,13)$ & 8 & 1 & YES & YES & YES & $1.57$ & $(2,3)$ & -- & 4680\\
$(b;0,1,0;19)$ & 6 & $(47,13)$ & 8 & 1 & YES & YES & YES & $1.57$ & $(2,3)$ & -- & 4681\\
$(b;0,1,1;27)$ & 7 & $(18,7)$ & 6 & 9 & YES & YES & YES & $1.57$ & $(2,3)$ & -- & 4682\\
$(b;0,2,0;8)$ & 7 & $(13,4)$ & 6 & 1 & YES & YES & YES & $1.57$ & $(2,3)$ & -- & 4683\\
$(b;0,2,0;8)$ & 7 & $(18,7)$ & 6 & 2 & YES & YES & YES & $1.43$ & $(2,3)$ & -- & 4684\\
$(b;0,2,0;8)$ & 7 & $(19,4)$ & 7 & 1 & YES & YES & NO(2) & $1.38$ & $(4,2)$ & -- & 4685\\
$(b;0,2,1;34)$ & 8 & $(7,3)$ & 4 & 1 & YES & YES & NO(2) & $1.44$ & $(2,3)$ & -- & 4686\\
$(b;0,2,1;34)$ & 8 & $(17,5)$ & 6 & 17 & YES & YES & YES & $1.57$ & $(2,3)$ & -- & 4687\\
$(b;0,2,1;34)$ & 8 & $(18,5)$ & 6 & 2 & YES & YES & YES & $1.57$ & $(2,3)$ & -- & 4688\\
$(b;0,2,3;6)$ & 10 & $(2,1)$ & 1 & 2 & YES & YES & NO(2) & $1.44$ & $(2,3)$ & -- & 4689\\
$(b;0,3,0;29)$ & 8 & $(11,3)$ & 5 & 1 & YES & YES & YES & $1.29$ & $(4,2)$ & -- & 4690\\
$(b;0,3,0;29)$ & 8 & $(13,3)$ & 6 & 1 & YES & YES & YES & $1.29$ & $(4,2)$ & -- & 4691\\
$(b;0,3,0;29)$ & 8 & $(13,4)$ & 6 & 1 & YES & YES & YES & $1.57$ & $(2,3)$ & -- & 4692\\
$(b;0,3,0;29)$ & 8 & $(16,3)$ & 7 & 1 & YES & YES & NO(2) & $1.00$ & $(6,1)$ & -- & 4693\\
$(b;0,3,0;29)$ & 8 & $(17,3)$ & 7 & 1 & YES & YES & NO(2) & $1.56$ & $(2,3)$ & -- & 4694\\
$(b;0,3,1;41)$ & 9 & $(3,1)$ & 2 & 1 & YES & YES & NO(2) & $1.25$ & $(4,2)$ & -- & 4695\\
$(b;0,3,1;41)$ & 9 & $(5,2)$ & 3 & 1 & YES & YES & NO(2) & $1.25$ & $(4,2)$ & -- & 4696\\
$(b;0,3,2;53)$ & 10 & $(4,1)$ & 3 & 1 & YES & YES & YES & $1.50$ & $(2,3)$ & -- & 4697\\
$(b;1,0,0;5)$ & 6 & $(18,7)$ & 6 & 1 & YES & YES & YES & $1.14$ & $(4,2)$ & -- & 4698\\
$(b;1,0,1;29)$ & 7 & $(17,7)$ & 6 & 1 & YES & YES & YES & $1.57$ & $(2,3)$ & -- & 4699\\
$(b;1,1,1;39)$ & 8 & $(12,5)$ & 5 & 3 & YES & YES & YES & $1.71$ & $(2,3)$ & -- & 4700\\
$(b;1,1,1;39)$ & 8 & $(17,5)$ & 6 & 1 & YES & YES & YES & $1.71$ & $(2,3)$ & -- & 4701\\
$(b;1,2,1;7)$ & 9 & $(10,3)$ & 5 & 1 & YES & YES & YES & $1.57$ & $(2,3)$ & -- & 4702\\
$(b;1,3,0;41)$ & 9 & $(7,2)$ & 4 & 1 & YES & YES & YES & $1.29$ & $(2,3)$ & -- & 4703\\
$(b;2,0,0;26)$ & 7 & $(8,3)$ & 4 & 2 & YES & YES & YES & $1.14$ & $(4,2)$ & -- & 4704\\
$(b;2,2,0;44)$ & 9 & $(7,2)$ & 4 & 1 & YES & YES & YES & $1.43$ & $(2,3)$ & -- & 4705\\
$(b;2,2,1;64)$ & 10 & $(3,1)$ & 2 & 1 & YES & YES & YES & $1.50$ & $(2,3)$ & -- & 4706\\
$(b;3,0,1;47)$ & 9 & $(7,2)$ & 4 & 1 & YES & YES & YES & $1.43$ & $(2,3)$ & -- & 4707\\
$(b;3,1,0;43)$ & 9 & $(7,2)$ & 4 & 1 & YES & YES & YES & $1.43$ & $(2,3)$ & -- & 4708\\
$(b;4,0,1;56)$ & 10 & $(5,1)$ & 4 & 1 & YES & YES & YES & $1.43$ & $(2,3)$ & -- & 4709\\
$(c;0,0,0;4)$ & 4 & $(36,11)$ & 8 & 4 & YES & YES & NO(3) & $1.22$ & $(2,3)$ & -- & 4710\\
$(c;0,0,0;4)$ & 4 & $(59,23)$ & 9 & 1 & YES & YES & YES & $1.71$ & $(2,3)$ & -- & 4711\\
$(c;0,0,0;4)$ & 4 & $(62,27)$ & 9 & 2 & YES & YES & YES & $1.57$ & $(2,3)$ & -- & 4712\\
$(c;0,0,0;4)$ & 4 & $(64,23)$ & 9 & 4 & YES & YES & YES & $1.43$ & $(2,3)$ & -- & 4713\\
$(c;0,0,0;4)$ & 4 & $(71,30)$ & 9 & 1 & YES & YES & YES & $1.57$ & $(2,3)$ & -- & 4714\\
$(c;0,0,0;4)$ & 4 & $(82,25)$ & 10 & 2 & YES & YES & YES & $1.57$ & $(2,3)$ & -- & 4715\\
$(c;0,0,0;4)$ & 4 & $(100,37)$ & 10 & 4 & YES & YES & YES & $1.86$ & $(2,3)$ & -- & 4716\\
$(c;0,0,0;4)$ & 4 & $(107,47)$ & 10 & 1 & YES & YES & YES & $1.57$ & $(2,3)$ & -- & 4717\\
$(c;0,0,0;4)$ & 4 & $(108,41)$ & 10 & 4 & YES & YES & YES & $1.57$ & $(2,3)$ & -- & 4718\\
$(c;0,0,0;4)$ & 4 & $(111,43)$ & 10 & 1 & YES & YES & YES & $1.57$ & $(2,3)$ & -- & 4719\\
$(c;0,0,0;4)$ & 4 & $(116,45)$ & 10 & 4 & YES & YES & YES & $1.71$ & $(2,3)$ & -- & 4720\\
$(c;0,1,0;11)$ & 5 & $(28,11)$ & 8 & 1 & YES & YES & YES & $1.38$ & $(2,3)$ & -- & 4721\\
$(c;0,1,0;11)$ & 5 & $(30,11)$ & 7 & 1 & YES & YES & YES & $1.50$ & $(2,3)$ & -- & 4722\\
$(c;0,1,0;11)$ & 5 & $(39,11)$ & 9 & 1 & YES & YES & YES & $1.38$ & $(2,3)$ & -- & 4723\\
$(c;0,1,0;11)$ & 5 & $(41,16)$ & 8 & 1 & YES & YES & YES & $1.50$ & $(2,3)$ & -- & 4724\\
$(c;0,1,0;11)$ & 5 & $(64,27)$ & 9 & 1 & YES & YES & YES & $1.57$ & $(2,3)$ & -- & 4725\\
$(c;0,1,0;11)$ & 5 & $(79,30)$ & 9 & 1 & YES & YES & YES & $1.43$ & $(2,3)$ & -- & 4726\\
$(c;0,1,0;11)$ & 5 & $(81,31)$ & 9 & 1 & YES & YES & YES & $1.57$ & $(2,3)$ & -- & 4727\\
$(c;0,1,0;11)$ & 5 & $(101,30)$ & 10 & 1 & YES & YES & YES & $1.43$ & $(2,3)$ & -- & 4728\\
$(c;0,1,0;11)$ & 5 & $(105,31)$ & 10 & 1 & YES & YES & YES & $1.43$ & $(2,3)$ & -- & 4729\\
$(c;0,1,1;5)$ & 6 & $(18,7)$ & 6 & 1 & YES & YES & NO(2) & $1.25$ & $(4,2)$ & -- & 4730\\
$(c;0,1,1;5)$ & 6 & $(24,7)$ & 7 & 1 & YES & YES & NO(2) & $1.25$ & $(4,2)$ & -- & 4731\\
$(c;0,2,0;7)$ & 6 & $(33,10)$ & 8 & 1 & YES & YES & YES & $1.50$ & $(2,3)$ & -- & 4732\\
$(c;0,2,0;7)$ & 6 & $(34,13)$ & 7 & 1 & YES & YES & YES & $1.57$ & $(2,3)$ & -- & 4733\\
$(c;0,2,0;7)$ & 6 & $(34,15)$ & 8 & 1 & YES & YES & YES & $1.43$ & $(2,3)$ & -- & 4734\\
$(c;0,2,0;7)$ & 6 & $(37,10)$ & 8 & 1 & YES & YES & YES & $1.50$ & $(2,3)$ & -- & 4735\\
$(c;0,2,0;7)$ & 6 & $(42,13)$ & 9 & 7 & YES & YES & YES & $1.71$ & $(2,3)$ & -- & 4736\\
$(c;0,2,0;7)$ & 6 & $(49,13)$ & 9 & 7 & YES & YES & YES & $1.71$ & $(2,3)$ & -- & 4737\\
$(c;0,2,0;7)$ & 6 & $(55,13)$ & 10 & 1 & YES & YES & YES & $1.43$ & $(2,3)$ & -- & 4738\\
$(c;0,2,0;7)$ & 6 & $(57,13)$ & 9 & 1 & YES & YES & YES & $1.43$ & $(2,3)$ & -- & 4739\\
$(c;0,2,1;19)$ & 7 & $(27,8)$ & 7 & 1 & YES & YES & YES & $1.57$ & $(2,3)$ & -- & 4740\\
$(c;0,2,1;19)$ & 7 & $(31,7)$ & 8 & 1 & YES & YES & NO(2) & $1.25$ & $(4,2)$ & -- & 4741\\
$(c;0,2,1;19)$ & 7 & $(37,10)$ & 8 & 1 & YES & YES & YES & $1.43$ & $(2,3)$ & -- & 4742\\
$(c;0,2,2;6)$ & 8 & $(7,3)$ & 4 & 1 & YES & YES & NO(2) & $1.44$ & $(2,3)$ & -- & 4743\\
$(c;0,2,2;6)$ & 8 & $(32,7)$ & 8 & 2 & YES & YES & YES & $1.57$ & $(2,3)$ & -- & 4744\\
$(c;0,3,0;17)$ & 7 & $(23,9)$ & 7 & 1 & YES & YES & YES & $1.43$ & $(2,3)$ & -- & 4745\\
$(c;0,3,0;17)$ & 7 & $(27,8)$ & 7 & 1 & YES & YES & YES & $1.43$ & $(2,3)$ & -- & 4746\\
$(c;0,3,0;17)$ & 7 & $(37,8)$ & 8 & 1 & YES & YES & YES & $1.29$ & $(2,3)$ & -- & 4747\\
$(c;0,3,1;23)$ & 8 & $(24,5)$ & 8 & 1 & YES & YES & YES & $1.57$ & $(2,3)$ & -- & 4748\\
$(c;0,3,2;29)$ & 9 & $(13,3)$ & 6 & 1 & YES & YES & NO(2) & $1.44$ & $(2,3)$ & -- & 4749\\
$(c;0,3,2;29)$ & 9 & $(17,4)$ & 7 & 1 & YES & YES & YES & $1.43$ & $(2,3)$ & -- & 4750\\
$(c;0,4,0;10)$ & 8 & $(17,5)$ & 6 & 1 & YES & YES & YES & $1.43$ & $(2,3)$ & -- & 4751\\
$(c;0,4,0;10)$ & 8 & $(22,5)$ & 7 & 2 & YES & YES & YES & $1.43$ & $(2,3)$ & -- & 4752\\
$(c;0,4,0;10)$ & 8 & $(28,5)$ & 8 & 2 & YES & YES & YES & $1.43$ & $(2,3)$ & -- & 4753\\
$(d;0,0,0;5)$ & 5 & $(18,7)$ & 6 & 1 & YES & YES & NO(2) & $1.25$ & $(4,2)$ & -- & 4754\\
$(d;0,0,0;5)$ & 5 & $(25,11)$ & 7 & 5 & YES & YES & YES & $1.43$ & $(2,3)$ & -- & 4755\\
$(d;0,0,0;5)$ & 5 & $(41,18)$ & 8 & 1 & YES & YES & YES & $1.43$ & $(2,3)$ & -- & 4756\\
$(d;0,0,0;5)$ & 5 & $(45,14)$ & 9 & 5 & YES & YES & YES & $1.43$ & $(2,3)$ & -- & 4757\\
$(d;0,0,0;5)$ & 5 & $(46,19)$ & 8 & 1 & YES & YES & YES & $1.71$ & $(2,3)$ & -- & 4758\\
$(d;0,0,0;5)$ & 5 & $(47,14)$ & 9 & 1 & YES & YES & YES & $1.43$ & $(2,3)$ & -- & 4759\\
$(d;0,0,1;14)$ & 6 & $(18,7)$ & 6 & 2 & YES & YES & NO(2) & $1.25$ & $(4,2)$ & -- & 4760\\
$(d;0,0,1;14)$ & 6 & $(19,5)$ & 7 & 1 & YES & YES & YES & $1.43$ & $(2,3)$ & -- & 4761\\
$(d;0,0,1;14)$ & 6 & $(23,9)$ & 7 & 1 & YES & YES & NO(2) & $1.25$ & $(4,2)$ & -- & 4762\\
$(d;0,0,1;14)$ & 6 & $(31,9)$ & 8 & 1 & YES & YES & NO(2) & $1.25$ & $(4,2)$ & -- & 4763\\
$(d;0,0,1;14)$ & 6 & $(33,10)$ & 8 & 1 & YES & YES & YES & $1.43$ & $(2,3)$ & -- & 4764\\
$(d;0,0,1;14)$ & 6 & $(56,17)$ & 9 & 14 & YES & YES & YES & $1.43$ & $(2,3)$ & -- & 4765\\
$(d;0,0,2;9)$ & 7 & $(12,5)$ & 5 & 3 & YES & YES & NO(2) & $1.44$ & $(2,3)$ & -- & 4766\\
$(d;0,0,2;9)$ & 7 & $(13,5)$ & 5 & 1 & YES & YES & NO(2) & $1.44$ & $(2,3)$ & -- & 4767\\
$(d;0,0,2;9)$ & 7 & $(17,7)$ & 6 & 1 & YES & YES & NO(2) & $1.44$ & $(2,3)$ & -- & 4768\\
$(d;0,0,2;9)$ & 7 & $(18,7)$ & 6 & 9 & YES & YES & NO(3) & $1.29$ & $(2,3)$ & -- & 4769\\
$(d;0,0,2;9)$ & 7 & $(23,9)$ & 7 & 1 & YES & YES & YES & $1.43$ & $(2,3)$ & -- & 4770\\
$(d;0,0,2;9)$ & 7 & $(24,7)$ & 7 & 3 & YES & YES & NO(3) & $1.29$ & $(2,3)$ & -- & 4771\\
$(d;0,0,2;9)$ & 7 & $(31,12)$ & 7 & 1 & YES & YES & YES & $1.43$ & $(2,3)$ & -- & 4772\\
$(d;0,0,4;13)$ & 9 & $(7,2)$ & 4 & 1 & YES & YES & YES & $1.43$ & $(2,3)$ & -- & 4773\\
$(d;0,1,0;6)$ & 6 & $(33,14)$ & 8 & 3 & YES & YES & YES & $1.43$ & $(2,3)$ & -- & 4774\\
$(d;0,1,0;6)$ & 6 & $(42,13)$ & 9 & 6 & YES & YES & YES & $1.57$ & $(2,3)$ & -- & 4775\\
$(d;0,1,0;6)$ & 6 & $(49,13)$ & 9 & 1 & YES & YES & YES & $1.57$ & $(2,3)$ & -- & 4776\\
$(d;0,1,1;17)$ & 7 & $(12,5)$ & 5 & 1 & YES & YES & NO(2) & $1.44$ & $(2,3)$ & -- & 4777\\
$(d;0,1,1;17)$ & 7 & $(13,5)$ & 5 & 1 & YES & YES & NO(2) & $1.44$ & $(2,3)$ & -- & 4778\\
$(d;0,1,1;17)$ & 7 & $(23,9)$ & 7 & 1 & YES & YES & YES & $1.43$ & $(2,3)$ & -- & 4779\\
$(d;0,1,1;17)$ & 7 & $(41,9)$ & 9 & 1 & YES & YES & YES & $1.43$ & $(2,3)$ & -- & 4780\\
$(d;0,1,2;11)$ & 8 & $(18,5)$ & 6 & 1 & YES & YES & YES & $1.50$ & $(2,3)$ & -- & 4781\\
$(d;0,2,0;7)$ & 7 & $(17,5)$ & 6 & 1 & YES & YES & YES & $1.43$ & $(2,3)$ & -- & 4782\\
$(d;0,2,0;7)$ & 7 & $(23,5)$ & 7 & 1 & YES & YES & YES & $1.43$ & $(2,3)$ & -- & 4783\\
$(d;0,2,1;20)$ & 8 & $(24,5)$ & 8 & 4 & YES & YES & YES & $1.57$ & $(2,3)$ & -- & 4784\\
$(d;0,2,2;13)$ & 9 & $(7,2)$ & 4 & 1 & YES & YES & YES & $1.29$ & $(2,3)$ & -- & 4785\\
$(e;0,0,0;4)$ & 5 & $(36,13)$ & 8 & 4 & YES & YES & YES & $1.57$ & $(2,3)$ & -- & 4786\\
$(e;0,0,0;4)$ & 5 & $(37,11)$ & 8 & 1 & YES & YES & YES & $1.43$ & $(2,3)$ & -- & 4787\\
$(e;0,0,0;4)$ & 5 & $(40,11)$ & 8 & 4 & YES & YES & YES & $1.57$ & $(2,3)$ & -- & 4788\\
$(e;0,1,0;5)$ & 6 & $(22,9)$ & 7 & 1 & YES & YES & YES & $1.43$ & $(2,3)$ & -- & 4789\\
$(e;0,2,0;6)$ & 7 & $(18,7)$ & 6 & 6 & YES & YES & YES & $1.43$ & $(2,3)$ & -- & 4790\\
$(e;0,3,0;7)$ & 8 & $(13,4)$ & 6 & 1 & YES & YES & YES & $1.57$ & $(2,3)$ & -- & 4791\\
$(e;0,3,0;7)$ & 8 & $(19,4)$ & 7 & 1 & YES & YES & YES & $1.57$ & $(2,3)$ & -- & 4792\\
$(e;1,1,0;23)$ & 7 & $(17,7)$ & 6 & 1 & YES & YES & YES & $1.43$ & $(4,2)$ & -- & 4793\\
$(e;1,2,0;28)$ & 8 & $(7,3)$ & 4 & 7 & YES & YES & NO(2) & $1.44$ & $(2,3)$ & -- & 4794\\
$(e;2,0,0;24)$ & 7 & $(18,7)$ & 6 & 6 & YES & YES & YES & $1.29$ & $(4,2)$ & -- & 4795\\
$(e;3,0,0;10)$ & 8 & $(19,4)$ & 7 & 1 & YES & YES & YES & $1.57$ & $(2,3)$ & -- & 4796\\
$(e;3,2,0;16)$ & 10 & $(2,1)$ & 1 & 2 & YES & YES & NO(2) & $1.44$ & $(2,3)$ & -- & 4797\\
$(e;3,2,0;16)$ & 10 & $(3,1)$ & 2 & 1 & YES & YES & YES & $1.29$ & $(2,3)$ & -- & 4798\\
$(f;0,0,0;6)$ & 4 & $(40,11)$ & 8 & 2 & YES & YES & NO(2) & $1.44$ & $(2,3)$ & -- & 4799\\
$(f;0,0,0;6)$ & 4 & $(43,16)$ & 9 & 1 & YES & YES & NO(2) & $1.38$ & $(4,2)$ & -- & 4800\\
$(f;0,0,0;6)$ & 4 & $(47,17)$ & 9 & 1 & YES & YES & NO(2) & $1.38$ & $(4,2)$ & -- & 4801\\
$(f;0,0,0;6)$ & 4 & $(51,20)$ & 9 & 3 & YES & YES & YES & $1.29$ & $(2,3)$ & -- & 4802\\
$(f;0,0,0;6)$ & 4 & $(53,19)$ & 9 & 1 & YES & YES & YES & $1.29$ & $(2,3)$ & -- & 4803\\
$(f;0,0,0;6)$ & 4 & $(53,22)$ & 9 & 1 & YES & YES & NO(2) & $1.38$ & $(4,2)$ & -- & 4804\\
$(f;0,0,0;6)$ & 4 & $(55,23)$ & 9 & 1 & YES & YES & YES & $1.50$ & $(2,3)$ & -- & 4805\\
$(f;0,0,0;6)$ & 4 & $(55,24)$ & 9 & 1 & YES & YES & NO(2) & $1.44$ & $(2,3)$ & -- & 4806\\
$(f;0,0,0;6)$ & 4 & $(57,13)$ & 9 & 3 & YES & YES & YES & $1.57$ & $(2,3)$ & -- & 4807\\
$(f;0,0,0;6)$ & 4 & $(58,17)$ & 9 & 2 & YES & YES & YES & $1.50$ & $(2,3)$ & -- & 4808\\
$(f;0,0,0;6)$ & 4 & $(62,17)$ & 10 & 2 & YES & YES & YES & $1.71$ & $(2,3)$ & -- & 4809\\
$(f;0,0,0;6)$ & 4 & $(65,19)$ & 9 & 1 & YES & YES & YES & $1.50$ & $(2,3)$ & -- & 4810\\
$(f;0,0,0;6)$ & 4 & $(65,24)$ & 9 & 1 & YES & YES & YES & $1.50$ & $(2,3)$ & -- & 4811\\
$(f;0,0,0;6)$ & 4 & $(65,27)$ & 10 & 1 & YES & YES & NO(2) & $1.38$ & $(4,2)$ & -- & 4812\\
$(f;0,0,0;6)$ & 4 & $(68,25)$ & 9 & 2 & YES & YES & NO(2) & $1.44$ & $(2,3)$ & -- & 4813\\
$(f;0,0,0;6)$ & 4 & $(71,27)$ & 9 & 1 & YES & YES & NO(2) & $1.44$ & $(2,3)$ & -- & 4814\\
$(f;0,0,0;6)$ & 4 & $(73,28)$ & 10 & 1 & YES & YES & YES & $1.57$ & $(2,3)$ & -- & 4815\\
$(f;0,0,0;6)$ & 4 & $(87,31)$ & 12 & 3 & YES & YES & YES & $1.71$ & $(2,3)$ & -- & 4816\\
$(f;0,0,0;6)$ & 4 & $(115,44)$ & 10 & 1 & YES & YES & YES & $1.57$ & $(2,3)$ & -- & 4817\\
$(f;0,0,0;6)$ & 4 & $(167,46)$ & 11 & 1 & YES & YES & YES & $1.71$ & $(2,3)$ & -- & 4818\\
$(f;0,1,0;7)$ & 5 & $(50,21)$ & 8 & 1 & YES & YES & YES & $1.57$ & $(2,3)$ & -- & 4819\\
$(g;0,0,0;19)$ & 6 & $(18,7)$ & 6 & 1 & YES & YES & YES & $1.57$ & $(2,3)$ & -- & 4820\\
$(g;0,0,0;19)$ & 6 & $(23,9)$ & 7 & 1 & YES & YES & YES & $1.57$ & $(2,3)$ & -- & 4821\\
$(g;0,0,1;26)$ & 7 & $(18,7)$ & 6 & 2 & YES & YES & YES & $1.71$ & $(2,3)$ & -- & 4822\\
$(g;0,0,2;11)$ & 8 & $(5,2)$ & 3 & 1 & YES & YES & NO(2) & $1.44$ & $(2,3)$ & -- & 4823\\
$(g;0,0,2;11)$ & 8 & $(21,5)$ & 8 & 1 & YES & YES & YES & $1.71$ & $(2,3)$ & -- & 4824\\
$(g;0,1,2;14)$ & 9 & $(7,2)$ & 4 & 7 & YES & YES & YES & $1.29$ & $(2,3)$ & -- & 4825\\
$(g;0,2,0;29)$ & 8 & $(5,2)$ & 3 & 1 & YES & YES & NO(2) & $1.44$ & $(2,3)$ & -- & 4826\\
$(g;0,2,0;29)$ & 8 & $(9,2)$ & 5 & 1 & YES & YES & NO(2) & $1.44$ & $(2,3)$ & -- & 4827\\
$(g;0,2,1;40)$ & 9 & $(7,2)$ & 4 & 1 & YES & YES & YES & $1.29$ & $(2,3)$ & -- & 4828\\
$(g;0,2,2;17)$ & 10 & $(2,1)$ & 1 & 1 & YES & YES & YES & $1.29$ & $(2,3)$ & -- & 4829\\
$(g;1,0,2;24)$ & 9 & $(5,2)$ & 3 & 1 & YES & YES & YES & $1.38$ & $(2,3)$ & -- & 4830\\
$(g;1,1,0;9)$ & 8 & $(12,5)$ & 5 & 3 & YES & YES & YES & $1.57$ & $(2,3)$ & -- & 4831\\
$(g;1,1,1;49)$ & 9 & $(2,1)$ & 1 & 1 & YES & YES & NO(2) & $1.33$ & $(2,3)$ & -- & 4832\\
$(g;1,1,1;49)$ & 9 & $(7,2)$ & 4 & 7 & YES & YES & YES & $1.38$ & $(2,3)$ & -- & 4833\\
$(g;1,1,2;31)$ & 10 & $(2,1)$ & 1 & 1 & YES & YES & YES & $1.29$ & $(2,3)$ & -- & 4834\\
$(g;1,1,2;31)$ & 10 & $(3,1)$ & 2 & 1 & YES & YES & YES & $1.43$ & $(2,3)$ & -- & 4835\\
$(g;1,1,2;31)$ & 10 & $(4,1)$ & 3 & 1 & YES & YES & YES & $1.29$ & $(2,3)$ & -- & 4836\\
$(g;2,1,0;48)$ & 9 & $(10,3)$ & 5 & 2 & YES & YES & YES & $1.71$ & $(2,3)$ & -- & 4837\\
$(g;2,1,1;13)$ & 10 & $(2,1)$ & 1 & 1 & YES & YES & YES & $1.38$ & $(2,3)$ & -- & 4838\\
$(h;0,0,0;6)$ & 5 & $(12,5)$ & 5 & 6 & YES & YES & NO(2) & $1.22$ & $(2,3)$ & -- & 4839\\
$(h;0,0,0;6)$ & 5 & $(44,17)$ & 8 & 2 & YES & YES & YES & $1.86$ & $(2,3)$ & -- & 4840\\
$(h;0,1,0;8)$ & 6 & $(18,7)$ & 6 & 2 & YES & YES & YES & $1.57$ & $(2,3)$ & -- & 4841\\
$(h;0,1,0;8)$ & 6 & $(23,9)$ & 7 & 1 & YES & YES & YES & $1.57$ & $(2,3)$ & -- & 4842\\
$(h;0,2,0;10)$ & 7 & $(18,7)$ & 6 & 2 & YES & YES & YES & $1.57$ & $(2,3)$ & -- & 4843\\
$(h;0,2,0;10)$ & 7 & $(24,7)$ & 7 & 2 & YES & YES & YES & $1.57$ & $(2,3)$ & -- & 4844\\
$(h;0,3,0;12)$ & 8 & $(2,1)$ & 1 & 2 & YES & YES & NO(3) & $1.22$ & $(2,3)$ & -- & 4845\\
$(h;0,3,0;12)$ & 8 & $(3,1)$ & 2 & 3 & YES & YES & NO(3) & $1.22$ & $(2,3)$ & -- & 4846\\
$(h;0,3,0;12)$ & 8 & $(4,1)$ & 3 & 4 & YES & YES & NO(3) & $1.22$ & $(2,3)$ & -- & 4847\\
$(i;0,0,0;9)$ & 5 & $(32,9)$ & 8 & 1 & YES & YES & YES & $1.29$ & $(2,3)$ & -- & 4848\\
$(i;0,0,0;9)$ & 5 & $(34,15)$ & 8 & 1 & YES & YES & YES & $1.57$ & $(2,3)$ & -- & 4849\\
$(i;0,0,0;9)$ & 5 & $(37,11)$ & 8 & 1 & YES & YES & YES & $1.29$ & $(2,3)$ & -- & 4850\\
$(i;0,0,0;9)$ & 5 & $(49,13)$ & 9 & 1 & YES & YES & YES & $1.43$ & $(2,3)$ & -- & 4851\\
$(i;0,0,0;9)$ & 5 & $(63,17)$ & 9 & 9 & YES & YES & YES & $1.71$ & $(2,3)$ & -- & 4852\\
$(i;0,0,0;9)$ & 5 & $(64,15)$ & 10 & 1 & YES & YES & YES & $1.57$ & $(2,3)$ & -- & 4853\\
$(i;0,1,0;12)$ & 6 & $(12,5)$ & 5 & 12 & YES & YES & YES & $1.29$ & $(2,3)$ & -- & 4854\\
$(i;0,2,0;15)$ & 7 & $(12,5)$ & 5 & 3 & YES & YES & YES & $1.29$ & $(2,3)$ & -- & 4855\\
$(i;0,2,0;15)$ & 7 & $(13,5)$ & 5 & 1 & YES & YES & YES & $1.29$ & $(2,3)$ & -- & 4856\\
$(i;0,3,0;18)$ & 8 & $(7,3)$ & 4 & 1 & YES & YES & YES & $1.29$ & $(2,3)$ & -- & 4857\\
$(j;0,0,0;8)$ & 5 & $(27,11)$ & 8 & 1 & YES & YES & NO(2) & $1.38$ & $(4,2)$ & -- & 4858\\
$(j;0,0,0;8)$ & 5 & $(28,11)$ & 8 & 4 & YES & YES & NO(2) & $1.38$ & $(4,2)$ & -- & 4859\\
$(j;0,0,0;8)$ & 5 & $(33,14)$ & 8 & 1 & YES & YES & YES & $1.57$ & $(2,3)$ & -- & 4860\\
$(j;0,0,0;8)$ & 5 & $(34,13)$ & 7 & 2 & YES & YES & YES & $1.29$ & $(2,3)$ & -- & 4861\\
$(j;0,0,0;8)$ & 5 & $(41,17)$ & 8 & 1 & YES & YES & NO(2) & $1.44$ & $(2,3)$ & -- & 4862\\
$(j;0,0,0;8)$ & 5 & $(43,18)$ & 8 & 1 & YES & YES & NO(2) & $1.44$ & $(2,3)$ & -- & 4863\\
$(j;0,0,0;8)$ & 5 & $(43,19)$ & 9 & 1 & YES & YES & YES & $1.57$ & $(2,3)$ & -- & 4864\\
$(j;0,0,0;8)$ & 5 & $(45,17)$ & 9 & 1 & YES & YES & YES & $1.57$ & $(2,3)$ & -- & 4865\\
$(j;0,0,0;8)$ & 5 & $(49,19)$ & 8 & 1 & YES & YES & YES & $1.38$ & $(2,3)$ & -- & 4866\\
$(j;0,0,0;8)$ & 5 & $(56,25)$ & 11 & 8 & YES & YES & YES & $1.71$ & $(2,3)$ & -- & 4867\\
$(j;0,0,0;8)$ & 5 & $(76,29)$ & 9 & 4 & YES & YES & YES & $1.71$ & $(2,3)$ & -- & 4868\\
$(j;0,0,0;8)$ & 5 & $(92,27)$ & 11 & 4 & YES & YES & YES & $1.71$ & $(2,3)$ & -- & 4869\\
$(j;0,1,0;10)$ & 6 & $(29,11)$ & 7 & 1 & YES & YES & NO(2) & $1.44$ & $(2,3)$ & -- & 4870\\
$(j;0,2,0;12)$ & 7 & $(24,7)$ & 7 & 12 & YES & YES & YES & $1.43$ & $(2,3)$ & -- & 4871
\end{longtable}
\subsection{2 chains, $K^2 = 4$}
\begin{longtable}{|c|c|c|c|c|c|c|c|c|c|c|c|}
\hline
\multicolumn{12}{|c|}{2 chains, $K^2 = 4$}\\
\hline
$(n,a)$ & Len & $(n,a)$ & Len & GCD & Nef & $\mathbb Q$-ef & Obs 0 & $\overline c_1^2 / \overline c_2$ & $(P,K)$ & WH & Index\\
\hline
\endfirsthead

\hline
$(n,a)$ & Len & $(n,a)$ & Len & GCD & Nef & $\mathbb Q$-ef & Obs 0 & $\overline c_1^2 / \overline c_2$ & $(P,K)$ & WH & Index\\
\hline
\endhead
\hline
\endfoot

$(43,12)$ & 8 & $(41,18)$ & 8 & 1 & YES & YES & NO(3) & $1.83$ & $(2,4)$ & -- & 4872\\
$(47,13)$ & 8 & $(41,18)$ & 8 & 1 & YES & YES & NO(3) & $1.83$ & $(2,4)$ & -- & 4873\\
$(66,29)$ & 9 & $(29,8)$ & 7 & 1 & YES & YES & NO(3) & $1.83$ & $(2,4)$ & -- & 4874\\
$(108,41)$ & 10 & $(60,23)$ & 9 & 12 & YES & YES & NO(2) & $2.00$ & $(2,4)$ & NO & 4875\\
$(117,31)$ & 11 & $(3,1)$ & 2 & 3 & YES & YES & NO(2) & $1.71$ & $(4,3)$ & -- & 4876\\
$(119,37)$ & 11 & $(29,8)$ & 7 & 1 & YES & YES & NO(3) & $1.83$ & $(2,4)$ & NO & 4877\\
$(140,37)$ & 11 & $(13,5)$ & 5 & 1 & YES & YES & NO(3) & $1.83$ & $(2,4)$ & -- & 4878\\
$(140,37)$ & 11 & $(63,17)$ & 9 & 7 & YES & YES & NO(3) & $1.83$ & $(2,4)$ & NO & 4879\\
$(155,46)$ & 11 & $(131,39)$ & 11 & 1 & YES & YES & NO(3) & $1.83$ & $(2,4)$ & NO & 4880\\
$(161,68)$ & 11 & $(10,3)$ & 5 & 1 & YES & YES & NO(3) & $1.83$ & $(2,4)$ & -- & 4881\\
$(181,51)$ & 12 & $(47,13)$ & 8 & 1 & YES & YES & NO(3) & $1.83$ & $(2,4)$ & NO & 4882\\
$(239,105)$ & 12 & $(5,2)$ & 3 & 1 & YES & YES & NO(3) & $1.83$ & $(2,4)$ & -- & 4883\\
$(239,105)$ & 12 & $(148,65)$ & 11 & 1 & YES & YES & NO(3) & $1.83$ & $(2,4)$ & 4897 & 4884\\
$(257,69)$ & 12 & $(3,1)$ & 2 & 1 & YES & YES & NO(3) & $1.62$ & $(2,4)$ & -- & 4885\\
$(258,109)$ & 12 & $(5,2)$ & 3 & 1 & YES & YES & NO(3) & $1.83$ & $(2,4)$ & -- & 4886\\
$(258,109)$ & 12 & $(161,68)$ & 11 & 1 & YES & YES & NO(3) & $1.83$ & $(2,4)$ & 4900 & 4887\\
$(264,109)$ & 12 & $(4,1)$ & 3 & 4 & YES & YES & NO(3) & $1.75$ & $(2,4)$ & -- & 4888\\
$(269,75)$ & 12 & $(12,5)$ & 5 & 1 & YES & YES & NO(3) & $2.00$ & $(2,4)$ & -- & 4889\\
$(287,61)$ & 13 & $(5,2)$ & 3 & 1 & YES & YES & NO(3) & $1.75$ & $(2,4)$ & -- & 4890\\
$(287,61)$ & 13 & $(13,3)$ & 6 & 1 & YES & YES & NO(3) & $1.75$ & $(2,4)$ & NO & 4891\\
$(312,119)$ & 13 & $(8,3)$ & 4 & 8 & YES & YES & NO(2) & $2.00$ & $(2,4)$ & NO & 4892\\
$(317,121)$ & 12 & $(97,37)$ & 10 & 1 & YES & YES & NO(3) & $1.83$ & $(2,4)$ & NO & 4893\\
$(329,87)$ & 13 & $(7,2)$ & 4 & 7 & YES & YES & NO(3) & $1.83$ & $(2,4)$ & -- & 4894\\
$(402,157)$ & 13 & $(8,3)$ & 4 & 2 & YES & YES & NO(3) & $1.83$ & $(2,4)$ & NO & 4895\\
$(403,177)$ & 13 & $(66,29)$ & 9 & 1 & YES & YES & NO(3) & $1.83$ & $(2,4)$ & 4901 & 4896\\
$(412,181)$ & 13 & $(41,18)$ & 8 & 1 & YES & YES & NO(3) & $1.83$ & $(2,4)$ & 4884 & 4897\\
$(439,136)$ & 14 & $(3,1)$ & 2 & 1 & YES & YES & NO(3) & $1.83$ & $(2,4)$ & -- & 4898\\
$(445,188)$ & 13 & $(3,1)$ & 2 & 1 & YES & YES & NO(3) & $1.83$ & $(2,4)$ & NO & 4899\\
$(445,188)$ & 13 & $(45,19)$ & 8 & 5 & YES & YES & NO(3) & $1.83$ & $(2,4)$ & 4887 & 4900\\
$(453,199)$ & 13 & $(41,18)$ & 8 & 1 & YES & YES & NO(3) & $1.83$ & $(2,4)$ & 4896 & 4901\\
$(495,112)$ & 14 & $(2,1)$ & 1 & 1 & YES & YES & NO(3) & $1.75$ & $(2,4)$ & -- & 4902\\
$(512,151)$ & 14 & $(5,2)$ & 3 & 1 & YES & YES & NO(3) & $2.00$ & $(2,4)$ & -- & 4903\\
$(580,177)$ & 14 & $(2,1)$ & 1 & 2 & YES & YES & NO(3) & $1.83$ & $(2,4)$ & NO & 4904\\
$(580,177)$ & 14 & $(3,1)$ & 2 & 1 & YES & YES & NO(3) & $1.83$ & $(2,4)$ & NO & 4905\\
$(580,177)$ & 14 & $(13,4)$ & 6 & 1 & YES & YES & NO(3) & $1.83$ & $(2,4)$ & NO & 4906\\
$(607,237)$ & 14 & $(64,25)$ & 9 & 1 & YES & YES & NO(3) & $2.00$ & $(2,4)$ & NO & 4907\\
$(611,256)$ & 14 & $(4,1)$ & 3 & 1 & YES & YES & NO(3) & $2.00$ & $(2,4)$ & -- & 4908\\
$(680,287)$ & 14 & $(12,5)$ & 5 & 4 & YES & YES & NO(3) & $2.00$ & $(2,4)$ & NO & 4909\\
$(747,169)$ & 15 & $(75,17)$ & 10 & 3 & YES & YES & NO(3) & $2.00$ & $(2,4)$ & NO & 4910\\
$(773,236)$ & 15 & $(3,1)$ & 2 & 1 & YES & YES & NO(3) & $2.00$ & $(2,4)$ & -- & 4911\\
$(788,241)$ & 15 & $(788,241)$ & 15 & 788 & YES & YES & NO(3) & $2.00$ & $(2,4)$ & NO & 4912\\
$(1048,237)$ & 16 & $(199,45)$ & 12 & 1 & YES & YES & NO(3) & $2.00$ & $(2,4)$ & NO & 4913\\
$(1078,193)$ & 17 & $(95,17)$ & 11 & 1 & YES & YES & NO(3) & $2.00$ & $(2,4)$ & NO & 4914\\
$(c;0,0,0;4)$ & 4 & $(41,18)$ & 8 & 1 & YES & YES & NO(3) & $1.62$ & $(2,4)$ & -- & 4915\\
$(d;0,0,2;9)$ & 7 & $(35,13)$ & 8 & 1 & YES & YES & NO(2) & $1.86$ & $(4,3)$ & -- & 4916\\
$(g;0,0,3;40)$ & 9 & $(17,5)$ & 6 & 1 & YES & YES & NO(3) & $1.83$ & $(2,4)$ & -- & 4917\\
$(g;0,0,3;40)$ & 9 & $(18,5)$ & 6 & 2 & YES & YES & NO(3) & $1.83$ & $(2,4)$ & -- & 4918\\
$(g;3,0,0;23)$ & 9 & $(22,5)$ & 7 & 1 & YES & YES & NO(3) & $1.83$ & $(2,4)$ & -- & 4919
\end{longtable}



%%%%%%%%%%%%%%%%%%%%%%%%%%%%%%%%%%%%%%%%%%%
\section{Extra: \textbf{Chilean}}

Input:
\lstinputlisting[language=config]{../Tests/Chilean.txt}
Result:
%\usepackage{longtable}
\subsection{2 chains, $K^2 = 2$}
\begin{longtable}{|c|c|c|c|c|c|c|c|c|c|c|c|}
\hline
\multicolumn{12}{|c|}{2 chains, $K^2 = 2$}\\
\hline
$(n,a)$ & Len & $(n,a)$ & Len & GCD & Nef & $\mathbb Q$-ef & Obs 0 & $\overline c_1^2 / \overline c_2$ & $(P,K)$ & WH & Index\\
\hline
\endfirsthead

\hline
$(n,a)$ & Len & $(n,a)$ & Len & GCD & Nef & $\mathbb Q$-ef & Obs 0 & $\overline c_1^2 / \overline c_2$ & $(P,K)$ & WH & Index\\
\hline
\endhead
\hline
\endfoot

$(34,13)$ & 7 & $(4,1)$ & 3 & 2 & YES & YES & YES & $0.62$ & $(2,2)$ & NO & 1\\
$(34,13)$ & 7 & $(34,13)$ & 7 & 34 & YES & YES & YES & $0.62$ & $(2,2)$ & NO & 2\\
$(41,12)$ & 8 & $(10,3)$ & 5 & 1 & YES & YES & YES & $0.62$ & $(2,2)$ & NO & 3\\
$(44,13)$ & 8 & $(2,1)$ & 1 & 2 & YES & YES & YES & $0.62$ & $(2,2)$ & -- & 4\\
$(44,13)$ & 8 & $(7,2)$ & 4 & 1 & YES & YES & YES & $0.62$ & $(2,2)$ & NO & 5\\
$(47,13)$ & 8 & $(2,1)$ & 1 & 1 & YES & YES & YES & $0.62$ & $(2,2)$ & NO & 6\\
$(47,13)$ & 8 & $(3,1)$ & 2 & 1 & YES & YES & YES & $0.62$ & $(2,2)$ & NO & 7\\
$(g;0,0,0;19)$ & 6 & $(2,1)$ & 1 & 1 & YES & YES & YES & $0.62$ & $(2,2)$ & -- & 8\\
$(g;0,0,0;19)$ & 6 & $(3,1)$ & 2 & 1 & YES & YES & YES & $0.62$ & $(2,2)$ & -- & 9
\end{longtable}



%%%%%%%%%%%%%%%%%%%%%%%%%%%%%%%%%%%%%%%%%%%
\section{Extra: $I_4 + 4I_2$}

Input:
\lstinputlisting[language=config]{../Tests/42222.txt}
Result:
%\usepackage{longtable}
\subsection{1 chain, $K^2 = 1$}
\begin{longtable}{|c|c|c|c|c|c|c|c|}
\hline
\multicolumn{8}{|c|}{1 chain, $K^2 = 1$}\\
\hline
$(n,a)$ & Len & Nef & $\mathbb Q$-ef & Obs 0 & $\overline c_1^2 / \overline c_2$ & $(P,K)$ & Index\\
\hline
\endfirsthead

\hline
$(n,a)$ & Len & Nef & $\mathbb Q$-ef & Obs 0 & $\overline c_1^2 / \overline c_2$ & $(P,K)$ & Index\\
\hline
\endhead
\hline
\endfoot

$(21,8)$ & 6 & YES & YES & YES & $0.70$ & $(1,1)$ & 1\\
$(24,7)$ & 7 & YES & YES & YES & $0.60$ & $(1,1)$ & 2\\
$(25,7)$ & 7 & YES & YES & YES & $0.40$ & $(1,1)$ & 3\\
$(27,8)$ & 7 & YES & YES & YES & $0.50$ & $(1,1)$ & 4\\
$(29,8)$ & 7 & YES & YES & YES & $0.50$ & $(1,1)$ & 5\\
$(31,7)$ & 8 & YES & YES & YES & $0.50$ & $(1,1)$ & 6\\
$(32,7)$ & 8 & YES & YES & YES & $0.70$ & $(1,1)$ & 7\\
$(b;0,0,0;14)$ & 5 & YES & YES & YES & $0.60$ & $(1,1)$ & 8\\
$(c;0,1,1;5)$ & 6 & YES & YES & YES & $0.50$ & $(1,1)$ & 9\\
$(d;0,0,1;14)$ & 6 & YES & YES & YES & $0.60$ & $(1,1)$ & 10\\
$(e;0,0,0;4)$ & 5 & YES & YES & YES & $0.70$ & $(1,1)$ & 11\\
$(h;0,0,0;6)$ & 5 & YES & YES & YES & $0.60$ & $(1,1)$ & 12
\end{longtable}
\subsection{1 chain, $K^2 = 2$}
\begin{longtable}{|c|c|c|c|c|c|c|c|}
\hline
\multicolumn{8}{|c|}{1 chain, $K^2 = 2$}\\
\hline
$(n,a)$ & Len & Nef & $\mathbb Q$-ef & Obs 0 & $\overline c_1^2 / \overline c_2$ & $(P,K)$ & Index\\
\hline
\endfirsthead

\hline
$(n,a)$ & Len & Nef & $\mathbb Q$-ef & Obs 0 & $\overline c_1^2 / \overline c_2$ & $(P,K)$ & Index\\
\hline
\endhead
\hline
\endfoot

$(43,12)$ & 8 & YES & YES & YES & $1.10$ & $(1,2)$ & 13\\
$(45,19)$ & 8 & YES & YES & YES & $1.00$ & $(1,2)$ & 14\\
$(46,19)$ & 8 & YES & YES & YES & $0.89$ & $(1,2)$ & 15\\
$(49,18)$ & 8 & YES & YES & YES & $1.00$ & $(1,2)$ & 16\\
$(50,19)$ & 8 & YES & YES & YES & $1.00$ & $(1,2)$ & 17\\
$(55,21)$ & 8 & YES & YES & YES & $0.89$ & $(1,2)$ & 18\\
$(57,25)$ & 9 & YES & YES & YES & $1.00$ & $(1,2)$ & 19\\
$(57,13)$ & 9 & YES & YES & YES & $1.00$ & $(1,2)$ & 20\\
$(58,17)$ & 9 & YES & YES & YES & $1.00$ & $(1,2)$ & 21\\
$(59,18)$ & 9 & YES & YES & YES & $1.00$ & $(1,2)$ & 22\\
$(59,23)$ & 9 & YES & YES & YES & $1.00$ & $(1,2)$ & 23\\
$(61,18)$ & 9 & YES & YES & YES & $0.89$ & $(1,2)$ & 24\\
$(63,26)$ & 9 & YES & YES & YES & $1.00$ & $(1,2)$ & 25\\
$(64,23)$ & 9 & YES & YES & YES & $0.89$ & $(1,2)$ & 26\\
$(64,25)$ & 9 & YES & YES & YES & $1.10$ & $(1,2)$ & 27\\
$(64,27)$ & 9 & YES & YES & YES & $1.00$ & $(1,2)$ & 28\\
$(64,19)$ & 9 & YES & YES & YES & $1.00$ & $(1,2)$ & 29\\
$(65,19)$ & 9 & YES & YES & YES & $1.00$ & $(1,2)$ & 30\\
$(66,29)$ & 9 & YES & YES & YES & $1.00$ & $(1,2)$ & 31\\
$(66,25)$ & 9 & YES & YES & YES & $1.11$ & $(1,2)$ & 32\\
$(67,26)$ & 9 & YES & YES & YES & $1.00$ & $(1,2)$ & 33\\
$(68,19)$ & 9 & YES & YES & YES & $0.89$ & $(1,2)$ & 34\\
$(68,25)$ & 9 & YES & YES & YES & $1.11$ & $(1,2)$ & 35\\
$(70,29)$ & 9 & YES & YES & YES & $1.00$ & $(1,2)$ & 36\\
$(71,21)$ & 9 & YES & YES & YES & $0.89$ & $(1,2)$ & 37\\
$(71,22)$ & 10 & YES & YES & YES & $1.00$ & $(1,2)$ & 38\\
$(71,27)$ & 9 & YES & YES & YES & $0.89$ & $(1,2)$ & 39\\
$(71,29)$ & 10 & YES & YES & YES & $1.00$ & $(1,2)$ & 40\\
$(71,30)$ & 9 & YES & YES & YES & $1.00$ & $(1,2)$ & 41\\
$(71,31)$ & 10 & YES & YES & YES & $1.11$ & $(1,2)$ & 42\\
$(71,26)$ & 9 & YES & YES & YES & $1.10$ & $(1,2)$ & 43\\
$(74,29)$ & 10 & YES & YES & YES & $1.00$ & $(1,2)$ & 44\\
$(74,31)$ & 9 & YES & YES & YES & $0.89$ & $(1,2)$ & 45\\
$(75,22)$ & 10 & YES & YES & YES & $1.11$ & $(1,2)$ & 46\\
$(75,29)$ & 9 & YES & YES & YES & $1.11$ & $(1,2)$ & 47\\
$(75,31)$ & 9 & YES & YES & YES & $1.00$ & $(1,2)$ & 48\\
$(76,31)$ & 10 & YES & YES & YES & $1.00$ & $(1,2)$ & 49\\
$(76,33)$ & 10 & YES & YES & YES & $1.00$ & $(1,2)$ & 50\\
$(78,29)$ & 10 & YES & YES & YES & $1.11$ & $(1,2)$ & 51\\
$(79,22)$ & 10 & YES & YES & YES & $1.00$ & $(1,2)$ & 52\\
$(79,23)$ & 10 & YES & YES & YES & $0.89$ & $(1,2)$ & 53\\
$(79,24)$ & 10 & YES & YES & YES & $1.10$ & $(1,2)$ & 54\\
$(79,29)$ & 9 & YES & YES & YES & $1.00$ & $(1,2)$ & 55\\
$(79,30)$ & 9 & YES & YES & YES & $0.89$ & $(1,2)$ & 56\\
$(80,31)$ & 9 & YES & YES & YES & $1.00$ & $(1,2)$ & 57\\
$(82,23)$ & 10 & YES & YES & YES & $0.89$ & $(1,2)$ & 58\\
$(82,25)$ & 10 & YES & YES & YES & $1.00$ & $(1,2)$ & 59\\
$(83,36)$ & 10 & YES & YES & YES & $1.00$ & $(1,2)$ & 60\\
$(84,31)$ & 10 & YES & YES & YES & $1.11$ & $(1,2)$ & 61\\
$(85,26)$ & 10 & YES & YES & YES & $1.00$ & $(1,2)$ & 62\\
$(86,31)$ & 10 & YES & YES & YES & $1.00$ & $(1,2)$ & 63\\
$(89,24)$ & 10 & YES & YES & YES & $0.90$ & $(1,2)$ & 64\\
$(89,26)$ & 10 & YES & YES & YES & $1.00$ & $(1,2)$ & 65\\
$(89,27)$ & 10 & YES & YES & YES & $0.89$ & $(1,2)$ & 66\\
$(89,32)$ & 10 & YES & YES & YES & $0.89$ & $(1,2)$ & 67\\
$(91,27)$ & 10 & YES & YES & YES & $0.89$ & $(1,2)$ & 68\\
$(93,25)$ & 10 & YES & YES & YES & $0.78$ & $(1,2)$ & 69\\
$(93,26)$ & 10 & YES & YES & YES & $1.00$ & $(1,2)$ & 70\\
$(94,39)$ & 10 & YES & YES & YES & $1.11$ & $(1,2)$ & 71\\
$(94,41)$ & 10 & YES & YES & YES & $1.00$ & $(1,2)$ & 72\\
$(95,39)$ & 10 & YES & YES & YES & $1.00$ & $(1,2)$ & 73\\
$(96,29)$ & 11 & YES & YES & YES & $1.00$ & $(1,2)$ & 74\\
$(97,30)$ & 11 & YES & YES & YES & $1.00$ & $(1,2)$ & 75\\
$(98,41)$ & 10 & YES & YES & YES & $1.11$ & $(1,2)$ & 76\\
$(99,29)$ & 10 & YES & YES & YES & $0.89$ & $(1,2)$ & 77\\
$(99,41)$ & 10 & YES & YES & YES & $0.89$ & $(1,2)$ & 78\\
$(100,27)$ & 10 & YES & YES & YES & $0.78$ & $(1,2)$ & 79\\
$(101,23)$ & 11 & YES & YES & YES & $1.00$ & $(1,2)$ & 80\\
$(106,23)$ & 11 & YES & YES & YES & $1.11$ & $(1,2)$ & 81\\
$(109,25)$ & 11 & YES & YES & YES & $1.00$ & $(1,2)$ & 82\\
$(111,41)$ & 10 & YES & YES & YES & $1.00$ & $(1,2)$ & 83\\
$(115,26)$ & 11 & YES & YES & YES & $0.89$ & $(1,2)$ & 84\\
$(125,29)$ & 12 & YES & YES & YES & $1.00$ & $(1,2)$ & 85\\
$(133,31)$ & 12 & YES & YES & YES & $0.89$ & $(1,2)$ & 86\\
$(146,27)$ & 13 & YES & YES & YES & $1.00$ & $(1,2)$ & 87\\
$(146,31)$ & 12 & YES & YES & YES & $0.89$ & $(1,2)$ & 88\\
$(155,36)$ & 12 & YES & YES & YES & $0.89$ & $(1,2)$ & 89\\
$(176,41)$ & 12 & YES & YES & YES & $1.00$ & $(1,2)$ & 90\\
$(a;2,1,1;37)$ & 8 & YES & YES & YES & $1.00$ & $(1,2)$ & 91\\
$(a;3,1,1;46)$ & 9 & YES & YES & YES & $0.89$ & $(1,2)$ & 92\\
$(b;0,0,3;32)$ & 8 & YES & YES & YES & $1.00$ & $(1,2)$ & 93\\
$(b;0,2,1;34)$ & 8 & YES & YES & YES & $0.89$ & $(1,2)$ & 94\\
$(b;1,0,1;29)$ & 7 & YES & YES & YES & $0.89$ & $(1,2)$ & 95\\
$(b;1,0,2;19)$ & 8 & YES & YES & YES & $0.89$ & $(1,2)$ & 96\\
$(b;1,1,1;39)$ & 8 & YES & YES & YES & $0.89$ & $(1,2)$ & 97\\
$(b;2,0,1;38)$ & 8 & YES & YES & YES & $1.00$ & $(1,2)$ & 98\\
$(e;1,2,0;28)$ & 8 & YES & YES & YES & $0.89$ & $(1,2)$ & 99\\
$(e;3,0,0;10)$ & 8 & YES & YES & YES & $1.00$ & $(1,2)$ & 100\\
$(g;0,0,1;26)$ & 7 & YES & YES & YES & $0.89$ & $(1,2)$ & 101\\
$(g;1,0,0;7)$ & 7 & YES & YES & YES & $0.89$ & $(1,2)$ & 102
\end{longtable}
\subsection{1 chain, $K^2 = 3$}
\begin{longtable}{|c|c|c|c|c|c|c|c|}
\hline
\multicolumn{8}{|c|}{1 chain, $K^2 = 3$}\\
\hline
$(n,a)$ & Len & Nef & $\mathbb Q$-ef & Obs 0 & $\overline c_1^2 / \overline c_2$ & $(P,K)$ & Index\\
\hline
\endfirsthead

\hline
$(n,a)$ & Len & Nef & $\mathbb Q$-ef & Obs 0 & $\overline c_1^2 / \overline c_2$ & $(P,K)$ & Index\\
\hline
\endhead
\hline
\endfoot

$(100,27)$ & 10 & YES & YES & NO(2) & $1.33$ & $(1,3)$ & 103\\
$(118,27)$ & 11 & YES & YES & YES & $1.38$ & $(1,3)$ & 104\\
$(122,33)$ & 11 & YES & YES & YES & $1.38$ & $(1,3)$ & 105\\
$(125,49)$ & 11 & YES & YES & YES & $1.25$ & $(1,3)$ & 106\\
$(125,27)$ & 11 & YES & YES & YES & $1.38$ & $(1,3)$ & 107\\
$(128,49)$ & 10 & YES & YES & YES & $1.38$ & $(1,3)$ & 108\\
$(129,49)$ & 10 & YES & YES & YES & $1.25$ & $(1,3)$ & 109\\
$(131,50)$ & 10 & YES & YES & YES & $1.38$ & $(1,3)$ & 110\\
$(131,36)$ & 11 & YES & YES & YES & $1.38$ & $(1,3)$ & 111\\
$(131,30)$ & 11 & YES & YES & YES & $1.38$ & $(1,3)$ & 112\\
$(133,36)$ & 11 & YES & YES & YES & $1.38$ & $(1,3)$ & 113\\
$(138,49)$ & 12 & YES & YES & YES & $1.38$ & $(1,3)$ & 114\\
$(139,30)$ & 11 & YES & YES & YES & $1.38$ & $(1,3)$ & 115\\
$(140,41)$ & 11 & YES & YES & YES & $1.38$ & $(1,3)$ & 116\\
$(140,39)$ & 11 & YES & YES & YES & $1.50$ & $(1,3)$ & 117\\
$(141,59)$ & 11 & YES & YES & YES & $1.38$ & $(1,3)$ & 118\\
$(147,41)$ & 11 & YES & YES & YES & $1.38$ & $(1,3)$ & 119\\
$(153,64)$ & 11 & YES & YES & YES & $1.50$ & $(1,3)$ & 120\\
$(154,65)$ & 11 & YES & YES & YES & $1.38$ & $(1,3)$ & 121\\
$(157,66)$ & 11 & YES & YES & YES & $1.38$ & $(1,3)$ & 122\\
$(157,58)$ & 11 & YES & YES & YES & $1.38$ & $(1,3)$ & 123\\
$(157,46)$ & 11 & YES & YES & YES & $1.38$ & $(1,3)$ & 124\\
$(159,59)$ & 11 & YES & YES & YES & $1.38$ & $(1,3)$ & 125\\
$(163,44)$ & 11 & YES & YES & YES & $1.38$ & $(1,3)$ & 126\\
$(165,49)$ & 11 & YES & YES & YES & $1.38$ & $(1,3)$ & 127\\
$(169,61)$ & 12 & YES & YES & NO(2) & $1.50$ & $(1,3)$ & 128\\
$(169,66)$ & 11 & YES & YES & YES & $1.25$ & $(1,3)$ & 129\\
$(169,71)$ & 11 & YES & YES & YES & $1.25$ & $(1,3)$ & 130\\
$(169,64)$ & 11 & YES & YES & YES & $1.25$ & $(1,3)$ & 131\\
$(171,71)$ & 12 & YES & YES & NO(2) & $1.60$ & $(1,3)$ & 132\\
$(171,65)$ & 11 & YES & YES & YES & $1.38$ & $(1,3)$ & 133\\
$(171,40)$ & 12 & YES & YES & YES & $1.50$ & $(1,3)$ & 134\\
$(173,66)$ & 11 & YES & YES & YES & $1.50$ & $(1,3)$ & 135\\
$(176,65)$ & 11 & YES & YES & YES & $1.38$ & $(1,3)$ & 136\\
$(178,63)$ & 12 & YES & YES & YES & $1.38$ & $(1,3)$ & 137\\
$(178,69)$ & 11 & YES & YES & YES & $1.38$ & $(1,3)$ & 138\\
$(179,68)$ & 11 & YES & YES & YES & $1.25$ & $(1,3)$ & 139\\
$(179,74)$ & 11 & YES & YES & YES & $1.38$ & $(1,3)$ & 140\\
$(179,75)$ & 11 & YES & YES & NO(2) & $1.22$ & $(1,3)$ & 141\\
$(181,79)$ & 12 & YES & YES & NO(2) & $1.50$ & $(1,3)$ & 142\\
$(181,75)$ & 11 & YES & YES & YES & $1.25$ & $(1,3)$ & 143\\
$(183,67)$ & 11 & YES & YES & YES & $1.25$ & $(1,3)$ & 144\\
$(185,76)$ & 12 & YES & YES & YES & $1.38$ & $(1,3)$ & 145\\
$(187,71)$ & 11 & YES & YES & YES & $1.25$ & $(1,3)$ & 146\\
$(187,79)$ & 11 & YES & YES & YES & $1.25$ & $(1,3)$ & 147\\
$(188,69)$ & 11 & YES & YES & YES & $1.25$ & $(1,3)$ & 148\\
$(188,79)$ & 11 & YES & YES & YES & $1.25$ & $(1,3)$ & 149\\
$(191,78)$ & 12 & YES & YES & NO(2) & $1.56$ & $(1,3)$ & 150\\
$(191,80)$ & 11 & YES & YES & YES & $1.12$ & $(1,3)$ & 151\\
$(193,81)$ & 11 & YES & YES & YES & $1.38$ & $(1,3)$ & 152\\
$(194,71)$ & 12 & YES & YES & YES & $1.50$ & $(1,3)$ & 153\\
$(196,81)$ & 11 & YES & YES & YES & $1.12$ & $(1,3)$ & 154\\
$(200,53)$ & 12 & YES & YES & YES & $1.38$ & $(1,3)$ & 155\\
$(200,59)$ & 12 & YES & YES & YES & $1.50$ & $(1,3)$ & 156\\
$(201,76)$ & 12 & YES & YES & YES & $1.38$ & $(1,3)$ & 157\\
$(201,77)$ & 12 & YES & YES & YES & $1.50$ & $(1,3)$ & 158\\
$(202,83)$ & 12 & YES & YES & NO(2) & $1.25$ & $(3,2)$ & 159\\
$(203,59)$ & 12 & YES & YES & YES & $1.38$ & $(1,3)$ & 160\\
$(205,89)$ & 12 & YES & YES & NO(2) & $1.56$ & $(1,3)$ & 161\\
$(205,61)$ & 12 & YES & YES & YES & $1.38$ & $(1,3)$ & 162\\
$(206,85)$ & 12 & YES & YES & YES & $1.38$ & $(1,3)$ & 163\\
$(208,79)$ & 11 & YES & YES & YES & $1.25$ & $(1,3)$ & 164\\
$(209,80)$ & 11 & YES & YES & YES & $1.25$ & $(1,3)$ & 165\\
$(209,81)$ & 11 & YES & YES & YES & $1.25$ & $(1,3)$ & 166\\
$(211,89)$ & 12 & YES & YES & YES & $1.38$ & $(1,3)$ & 167\\
$(211,93)$ & 12 & YES & YES & YES & $1.38$ & $(1,3)$ & 168\\
$(211,64)$ & 12 & YES & YES & YES & $1.50$ & $(1,3)$ & 169\\
$(212,45)$ & 14 & YES & YES & YES & $1.38$ & $(1,3)$ & 170\\
$(213,77)$ & 12 & YES & YES & NO(2) & $1.44$ & $(1,3)$ & 171\\
$(215,47)$ & 14 & YES & YES & YES & $1.50$ & $(1,3)$ & 172\\
$(218,79)$ & 14 & YES & YES & YES & $1.62$ & $(1,3)$ & 173\\
$(219,79)$ & 12 & YES & YES & NO(2) & $1.56$ & $(1,3)$ & 174\\
$(219,80)$ & 13 & YES & YES & YES & $1.50$ & $(1,3)$ & 175\\
$(221,84)$ & 12 & YES & YES & YES & $1.50$ & $(1,3)$ & 176\\
$(222,91)$ & 12 & YES & YES & YES & $1.38$ & $(1,3)$ & 177\\
$(223,68)$ & 12 & YES & YES & YES & $1.25$ & $(1,3)$ & 178\\
$(223,82)$ & 12 & YES & YES & NO(2) & $1.12$ & $(3,2)$ & 179\\
$(223,92)$ & 12 & YES & YES & YES & $1.38$ & $(1,3)$ & 180\\
$(225,98)$ & 12 & YES & YES & YES & $1.50$ & $(1,3)$ & 181\\
$(226,69)$ & 12 & YES & YES & YES & $1.25$ & $(1,3)$ & 182\\
$(226,95)$ & 12 & YES & YES & YES & $1.38$ & $(1,3)$ & 183\\
$(227,86)$ & 12 & YES & YES & YES & $1.38$ & $(1,3)$ & 184\\
$(227,93)$ & 12 & YES & YES & YES & $1.50$ & $(1,3)$ & 185\\
$(227,100)$ & 12 & YES & YES & YES & $1.38$ & $(1,3)$ & 186\\
$(229,68)$ & 12 & YES & YES & YES & $1.25$ & $(1,3)$ & 187\\
$(230,97)$ & 12 & YES & YES & NO(2) & $1.50$ & $(3,2)$ & 188\\
$(231,83)$ & 12 & YES & YES & YES & $1.50$ & $(1,3)$ & 189\\
$(232,97)$ & 13 & YES & YES & YES & $1.38$ & $(1,3)$ & 190\\
$(233,84)$ & 12 & YES & YES & YES & $1.38$ & $(1,3)$ & 191\\
$(233,89)$ & 11 & YES & YES & YES & $1.12$ & $(1,3)$ & 192\\
$(235,89)$ & 12 & YES & YES & YES & $1.25$ & $(1,3)$ & 193\\
$(237,98)$ & 12 & YES & YES & NO(2) & $1.38$ & $(3,2)$ & 194\\
$(237,104)$ & 12 & YES & YES & YES & $1.50$ & $(1,3)$ & 195\\
$(238,69)$ & 13 & YES & YES & YES & $1.38$ & $(1,3)$ & 196\\
$(239,71)$ & 12 & YES & YES & YES & $1.25$ & $(1,3)$ & 197\\
$(239,98)$ & 12 & YES & YES & YES & $1.50$ & $(1,3)$ & 198\\
$(239,99)$ & 12 & YES & YES & YES & $1.25$ & $(1,3)$ & 199\\
$(239,100)$ & 12 & YES & YES & YES & $1.25$ & $(1,3)$ & 200\\
$(239,101)$ & 12 & YES & YES & YES & $1.38$ & $(1,3)$ & 201\\
$(241,100)$ & 12 & YES & YES & YES & $1.38$ & $(1,3)$ & 202\\
$(241,101)$ & 12 & YES & YES & YES & $1.50$ & $(1,3)$ & 203\\
$(241,89)$ & 12 & YES & YES & NO(2) & $1.56$ & $(1,3)$ & 204\\
$(243,94)$ & 12 & YES & YES & NO(2) & $1.44$ & $(1,3)$ & 205\\
$(245,69)$ & 13 & YES & YES & YES & $1.50$ & $(1,3)$ & 206\\
$(245,88)$ & 12 & YES & YES & YES & $1.38$ & $(1,3)$ & 207\\
$(246,95)$ & 12 & YES & YES & NO(2) & $1.44$ & $(1,3)$ & 208\\
$(246,101)$ & 12 & YES & YES & NO(2) & $1.38$ & $(3,2)$ & 209\\
$(246,91)$ & 12 & YES & YES & NO(2) & $1.44$ & $(1,3)$ & 210\\
$(248,91)$ & 12 & YES & YES & YES & $1.50$ & $(1,3)$ & 211\\
$(249,95)$ & 12 & YES & YES & NO(2) & $1.12$ & $(3,2)$ & 212\\
$(250,73)$ & 13 & YES & YES & YES & $1.50$ & $(1,3)$ & 213\\
$(251,76)$ & 13 & YES & YES & NO(2) & $1.25$ & $(3,2)$ & 214\\
$(251,104)$ & 12 & YES & YES & YES & $1.38$ & $(1,3)$ & 215\\
$(251,105)$ & 12 & YES & YES & YES & $1.25$ & $(1,3)$ & 216\\
$(251,109)$ & 13 & YES & YES & NO(2) & $1.56$ & $(1,3)$ & 217\\
$(255,107)$ & 12 & YES & YES & YES & $1.50$ & $(1,3)$ & 218\\
$(257,76)$ & 12 & YES & YES & YES & $1.25$ & $(1,3)$ & 219\\
$(260,79)$ & 13 & YES & YES & YES & $1.25$ & $(1,3)$ & 220\\
$(261,100)$ & 12 & YES & YES & YES & $1.25$ & $(1,3)$ & 221\\
$(261,97)$ & 13 & YES & YES & YES & $1.50$ & $(1,3)$ & 222\\
$(263,60)$ & 13 & YES & YES & YES & $1.25$ & $(1,3)$ & 223\\
$(263,109)$ & 12 & YES & YES & YES & $1.38$ & $(1,3)$ & 224\\
$(264,101)$ & 12 & YES & YES & YES & $1.50$ & $(1,3)$ & 225\\
$(265,71)$ & 13 & YES & YES & YES & $1.50$ & $(1,3)$ & 226\\
$(265,97)$ & 12 & YES & YES & YES & $1.38$ & $(1,3)$ & 227\\
$(266,79)$ & 12 & YES & YES & YES & $1.12$ & $(1,3)$ & 228\\
$(266,101)$ & 12 & YES & YES & YES & $1.38$ & $(1,3)$ & 229\\
$(268,83)$ & 13 & YES & YES & YES & $1.38$ & $(1,3)$ & 230\\
$(268,99)$ & 12 & YES & YES & YES & $1.38$ & $(1,3)$ & 231\\
$(269,82)$ & 13 & YES & YES & YES & $1.25$ & $(1,3)$ & 232\\
$(269,104)$ & 12 & YES & YES & YES & $1.25$ & $(1,3)$ & 233\\
$(271,82)$ & 13 & YES & YES & YES & $1.25$ & $(1,3)$ & 234\\
$(272,65)$ & 14 & YES & YES & YES & $1.38$ & $(1,3)$ & 235\\
$(277,60)$ & 13 & YES & YES & YES & $1.12$ & $(1,3)$ & 236\\
$(277,81)$ & 12 & YES & YES & YES & $1.38$ & $(1,3)$ & 237\\
$(277,106)$ & 12 & YES & YES & YES & $1.38$ & $(1,3)$ & 238\\
$(278,77)$ & 13 & YES & YES & YES & $1.50$ & $(1,3)$ & 239\\
$(278,85)$ & 13 & YES & YES & YES & $1.25$ & $(1,3)$ & 240\\
$(278,121)$ & 13 & YES & YES & YES & $1.38$ & $(1,3)$ & 241\\
$(279,83)$ & 13 & YES & YES & NO(2) & $1.38$ & $(3,2)$ & 242\\
$(279,121)$ & 13 & YES & YES & YES & $1.38$ & $(1,3)$ & 243\\
$(281,85)$ & 13 & YES & YES & NO(2) & $1.25$ & $(3,2)$ & 244\\
$(281,109)$ & 12 & YES & YES & YES & $1.25$ & $(1,3)$ & 245\\
$(283,75)$ & 13 & YES & YES & YES & $1.25$ & $(1,3)$ & 246\\
$(285,83)$ & 13 & YES & YES & NO(2) & $1.25$ & $(3,2)$ & 247\\
$(285,103)$ & 13 & YES & YES & NO(2) & $1.44$ & $(1,3)$ & 248\\
$(287,79)$ & 12 & YES & YES & YES & $1.25$ & $(1,3)$ & 249\\
$(287,111)$ & 12 & YES & YES & YES & $1.38$ & $(1,3)$ & 250\\
$(290,81)$ & 12 & YES & YES & YES & $1.12$ & $(1,3)$ & 251\\
$(295,112)$ & 12 & YES & YES & YES & $1.38$ & $(1,3)$ & 252\\
$(296,83)$ & 13 & YES & YES & NO(2) & $1.12$ & $(3,2)$ & 253\\
$(300,91)$ & 13 & YES & YES & YES & $1.25$ & $(1,3)$ & 254\\
$(302,47)$ & 16 & YES & YES & YES & $1.38$ & $(1,3)$ & 255\\
$(305,116)$ & 13 & YES & YES & YES & $1.62$ & $(1,3)$ & 256\\
$(307,69)$ & 14 & YES & YES & YES & $1.38$ & $(1,3)$ & 257\\
$(307,90)$ & 14 & YES & YES & YES & $1.50$ & $(1,3)$ & 258\\
$(307,119)$ & 12 & YES & YES & YES & $1.38$ & $(1,3)$ & 259\\
$(309,67)$ & 13 & YES & YES & YES & $1.25$ & $(1,3)$ & 260\\
$(313,71)$ & 14 & YES & YES & YES & $1.50$ & $(1,3)$ & 261\\
$(313,119)$ & 12 & YES & YES & YES & $1.50$ & $(1,3)$ & 262\\
$(313,121)$ & 12 & YES & YES & YES & $1.50$ & $(1,3)$ & 263\\
$(315,68)$ & 13 & YES & YES & YES & $1.25$ & $(1,3)$ & 264\\
$(317,84)$ & 13 & YES & YES & YES & $1.25$ & $(1,3)$ & 265\\
$(317,93)$ & 14 & YES & YES & YES & $1.50$ & $(1,3)$ & 266\\
$(317,121)$ & 12 & YES & YES & YES & $1.25$ & $(1,3)$ & 267\\
$(319,86)$ & 13 & YES & YES & YES & $1.38$ & $(1,3)$ & 268\\
$(320,139)$ & 14 & YES & YES & YES & $1.75$ & $(1,3)$ & 269\\
$(323,89)$ & 13 & YES & YES & YES & $1.25$ & $(1,3)$ & 270\\
$(324,95)$ & 13 & YES & YES & YES & $1.25$ & $(1,3)$ & 271\\
$(326,135)$ & 13 & YES & YES & YES & $1.50$ & $(1,3)$ & 272\\
$(329,89)$ & 13 & YES & YES & YES & $1.38$ & $(1,3)$ & 273\\
$(331,96)$ & 14 & YES & YES & YES & $1.62$ & $(1,3)$ & 274\\
$(332,97)$ & 13 & YES & YES & YES & $1.38$ & $(1,3)$ & 275\\
$(334,129)$ & 13 & YES & YES & YES & $1.50$ & $(1,3)$ & 276\\
$(335,78)$ & 14 & YES & YES & NO(2) & $1.38$ & $(3,2)$ & 277\\
$(338,77)$ & 14 & YES & YES & NO(2) & $1.38$ & $(3,2)$ & 278\\
$(343,81)$ & 15 & YES & YES & YES & $1.50$ & $(1,3)$ & 279\\
$(346,125)$ & 14 & YES & YES & YES & $1.62$ & $(1,3)$ & 280\\
$(347,78)$ & 14 & YES & YES & YES & $1.38$ & $(1,3)$ & 281\\
$(347,97)$ & 13 & YES & YES & YES & $1.38$ & $(1,3)$ & 282\\
$(349,143)$ & 13 & YES & YES & YES & $1.62$ & $(1,3)$ & 283\\
$(350,107)$ & 14 & YES & YES & YES & $1.62$ & $(1,3)$ & 284\\
$(351,76)$ & 13 & YES & YES & YES & $1.12$ & $(1,3)$ & 285\\
$(353,75)$ & 14 & YES & YES & YES & $1.25$ & $(1,3)$ & 286\\
$(359,105)$ & 13 & YES & YES & YES & $1.38$ & $(1,3)$ & 287\\
$(362,131)$ & 14 & YES & YES & YES & $1.62$ & $(1,3)$ & 288\\
$(366,97)$ & 14 & YES & YES & YES & $1.38$ & $(1,3)$ & 289\\
$(367,78)$ & 14 & YES & YES & NO(2) & $1.38$ & $(3,2)$ & 290\\
$(367,87)$ & 14 & YES & YES & YES & $1.25$ & $(1,3)$ & 291\\
$(377,79)$ & 14 & YES & YES & YES & $1.50$ & $(1,3)$ & 292\\
$(381,89)$ & 14 & YES & YES & YES & $1.25$ & $(1,3)$ & 293\\
$(385,82)$ & 14 & YES & YES & YES & $1.25$ & $(1,3)$ & 294\\
$(386,81)$ & 15 & YES & YES & YES & $1.38$ & $(1,3)$ & 295\\
$(389,91)$ & 14 & YES & YES & YES & $1.25$ & $(1,3)$ & 296\\
$(393,116)$ & 13 & YES & YES & YES & $1.38$ & $(1,3)$ & 297\\
$(397,90)$ & 15 & YES & YES & YES & $1.38$ & $(1,3)$ & 298\\
$(397,116)$ & 13 & YES & YES & YES & $1.38$ & $(1,3)$ & 299\\
$(399,86)$ & 14 & YES & YES & YES & $1.38$ & $(1,3)$ & 300\\
$(406,73)$ & 15 & YES & YES & YES & $1.25$ & $(1,3)$ & 301\\
$(417,79)$ & 15 & YES & YES & NO(2) & $1.25$ & $(3,2)$ & 302\\
$(419,89)$ & 14 & YES & YES & YES & $1.25$ & $(1,3)$ & 303\\
$(423,97)$ & 14 & YES & YES & YES & $1.25$ & $(1,3)$ & 304\\
$(427,100)$ & 14 & YES & YES & YES & $1.25$ & $(1,3)$ & 305\\
$(436,103)$ & 15 & YES & YES & YES & $1.38$ & $(1,3)$ & 306\\
$(470,111)$ & 15 & YES & YES & YES & $1.38$ & $(1,3)$ & 307\\
$(491,116)$ & 15 & YES & YES & YES & $1.50$ & $(1,3)$ & 308
\end{longtable}
\subsection{1 chain, $K^2 = 4$}
\begin{longtable}{|c|c|c|c|c|c|c|c|}
\hline
\multicolumn{8}{|c|}{1 chain, $K^2 = 4$}\\
\hline
$(n,a)$ & Len & Nef & $\mathbb Q$-ef & Obs 0 & $\overline c_1^2 / \overline c_2$ & $(P,K)$ & Index\\
\hline
\endfirsthead

\hline
$(n,a)$ & Len & Nef & $\mathbb Q$-ef & Obs 0 & $\overline c_1^2 / \overline c_2$ & $(P,K)$ & Index\\
\hline
\endhead
\hline
\endfoot

$(382,141)$ & 13 & YES & YES & NO(2) & $1.75$ & $(1,4)$ & 309\\
$(419,173)$ & 13 & YES & YES & YES & $1.71$ & $(1,4)$ & 310\\
$(443,169)$ & 13 & YES & YES & YES & $1.71$ & $(1,4)$ & 311\\
$(446,165)$ & 13 & YES & YES & NO(2) & $1.75$ & $(1,4)$ & 312\\
$(448,171)$ & 13 & YES & YES & YES & $1.86$ & $(1,4)$ & 313\\
$(484,189)$ & 14 & YES & YES & YES & $1.86$ & $(1,4)$ & 314\\
$(497,195)$ & 14 & YES & YES & YES & $1.86$ & $(1,4)$ & 315\\
$(499,135)$ & 14 & YES & YES & YES & $1.86$ & $(1,4)$ & 316\\
$(509,210)$ & 14 & YES & YES & YES & $1.86$ & $(1,4)$ & 317\\
$(520,197)$ & 14 & YES & YES & NO(2) & $1.88$ & $(1,4)$ & 318\\
$(533,220)$ & 14 & YES & YES & NO(2) & $1.88$ & $(1,4)$ & 319\\
$(548,209)$ & 14 & YES & YES & YES & $1.86$ & $(1,4)$ & 320\\
$(549,224)$ & 15 & YES & YES & YES & $1.86$ & $(1,4)$ & 321\\
$(585,262)$ & 15 & YES & YES & YES & $1.86$ & $(1,4)$ & 322\\
$(635,248)$ & 14 & YES & YES & YES & $1.86$ & $(1,4)$ & 323\\
$(671,281)$ & 14 & YES & YES & YES & $1.86$ & $(1,4)$ & 324\\
$(685,283)$ & 15 & YES & YES & YES & $2.00$ & $(1,4)$ & 325\\
$(699,287)$ & 15 & YES & YES & YES & $2.00$ & $(1,4)$ & 326\\
$(705,206)$ & 15 & YES & YES & YES & $1.86$ & $(1,4)$ & 327\\
$(714,299)$ & 14 & YES & YES & YES & $1.86$ & $(1,4)$ & 328\\
$(737,206)$ & 15 & YES & YES & YES & $1.86$ & $(1,4)$ & 329\\
$(747,202)$ & 16 & YES & YES & YES & $2.00$ & $(1,4)$ & 330\\
$(772,317)$ & 15 & YES & YES & YES & $2.00$ & $(1,4)$ & 331\\
$(781,299)$ & 14 & YES & YES & YES & $1.86$ & $(1,4)$ & 332\\
$(781,305)$ & 15 & YES & YES & YES & $2.00$ & $(1,4)$ & 333\\
$(782,297)$ & 14 & YES & YES & YES & $1.71$ & $(1,4)$ & 334\\
$(795,308)$ & 14 & YES & YES & YES & $1.86$ & $(1,4)$ & 335\\
$(801,190)$ & 16 & YES & YES & YES & $1.86$ & $(1,4)$ & 336\\
$(827,217)$ & 16 & YES & YES & YES & $1.86$ & $(1,4)$ & 337\\
$(851,202)$ & 17 & YES & YES & YES & $1.86$ & $(1,4)$ & 338\\
$(883,258)$ & 15 & YES & YES & YES & $1.86$ & $(1,4)$ & 339\\
$(921,269)$ & 16 & YES & YES & YES & $1.86$ & $(1,4)$ & 340\\
$(923,258)$ & 15 & YES & YES & YES & $1.86$ & $(1,4)$ & 341\\
$(930,389)$ & 15 & YES & YES & YES & $1.86$ & $(1,4)$ & 342\\
$(947,277)$ & 15 & YES & YES & YES & $1.86$ & $(1,4)$ & 343\\
$(948,277)$ & 15 & YES & YES & YES & $1.86$ & $(1,4)$ & 344\\
$(991,277)$ & 15 & YES & YES & YES & $1.86$ & $(1,4)$ & 345\\
$(1013,299)$ & 15 & YES & YES & YES & $1.71$ & $(1,4)$ & 346\\
$(1015,389)$ & 15 & YES & YES & YES & $1.86$ & $(1,4)$ & 347\\
$(1177,276)$ & 17 & YES & YES & YES & $1.86$ & $(1,4)$ & 348\\
$(1181,207)$ & 18 & YES & YES & YES & $1.86$ & $(1,4)$ & 349\\
$(1635,358)$ & 18 & YES & YES & YES & $2.00$ & $(1,4)$ & 350
\end{longtable}
\subsection{2 chains, $K^2 = 1$}
\begin{longtable}{|c|c|c|c|c|c|c|c|c|c|c|c|}
\hline
\multicolumn{12}{|c|}{2 chains, $K^2 = 1$}\\
\hline
$(n,a)$ & Len & $(n,a)$ & Len & GCD & Nef & $\mathbb Q$-ef & Obs 0 & $\overline c_1^2 / \overline c_2$ & $(P,K)$ & WH & Index\\
\hline
\endfirsthead

\hline
$(n,a)$ & Len & $(n,a)$ & Len & GCD & Nef & $\mathbb Q$-ef & Obs 0 & $\overline c_1^2 / \overline c_2$ & $(P,K)$ & WH & Index\\
\hline
\endhead
\hline
\endfoot

$(7,3)$ & 4 & $(7,3)$ & 4 & 7 & YES & YES & YES & $0.78$ & $(2,1)$ & -- & 351\\
$(8,3)$ & 4 & $(5,2)$ & 3 & 1 & YES & YES & YES & $0.56$ & $(2,1)$ & -- & 352\\
$(8,3)$ & 4 & $(7,2)$ & 4 & 1 & YES & YES & YES & $0.56$ & $(2,1)$ & -- & 353\\
$(8,3)$ & 4 & $(7,2)$ & 4 & 1 & YES & YES & YES & $0.67$ & $(2,1)$ & NO & 354\\
$(8,3)$ & 4 & $(7,3)$ & 4 & 1 & YES & YES & YES & $0.67$ & $(2,1)$ & -- & 355\\
$(8,3)$ & 4 & $(8,3)$ & 4 & 8 & YES & YES & YES & $0.44$ & $(2,1)$ & -- & 356\\
$(8,3)$ & 4 & $(8,3)$ & 4 & 8 & YES & YES & YES & $0.67$ & $(2,1)$ & NO & 357\\
$(10,3)$ & 5 & $(5,2)$ & 3 & 5 & YES & YES & YES & $0.78$ & $(2,1)$ & -- & 358\\
$(10,3)$ & 5 & $(5,2)$ & 3 & 5 & YES & YES & YES & $0.78$ & $(2,1)$ & NO & 359\\
$(10,3)$ & 5 & $(7,2)$ & 4 & 1 & YES & YES & YES & $0.67$ & $(2,1)$ & -- & 360\\
$(10,3)$ & 5 & $(7,2)$ & 4 & 1 & YES & YES & YES & $0.78$ & $(2,1)$ & NO & 361\\
$(10,3)$ & 5 & $(7,3)$ & 4 & 1 & YES & YES & YES & $0.78$ & $(2,1)$ & NO & 362\\
$(10,3)$ & 5 & $(7,3)$ & 4 & 1 & YES & YES & YES & $0.78$ & $(2,1)$ & -- & 363\\
$(10,3)$ & 5 & $(8,3)$ & 4 & 2 & YES & YES & YES & $0.67$ & $(2,1)$ & -- & 364\\
$(10,3)$ & 5 & $(10,3)$ & 5 & 10 & YES & YES & YES & $0.44$ & $(2,1)$ & -- & 365\\
$(11,4)$ & 5 & $(5,2)$ & 3 & 1 & YES & YES & YES & $0.56$ & $(2,1)$ & -- & 366\\
$(11,4)$ & 5 & $(5,2)$ & 3 & 1 & YES & YES & YES & $0.67$ & $(2,1)$ & NO & 367\\
$(11,3)$ & 5 & $(7,3)$ & 4 & 1 & YES & YES & YES & $0.78$ & $(2,1)$ & NO & 368\\
$(11,3)$ & 5 & $(7,3)$ & 4 & 1 & YES & YES & YES & $0.78$ & $(2,1)$ & -- & 369\\
$(11,4)$ & 5 & $(7,2)$ & 4 & 1 & YES & YES & YES & $0.56$ & $(2,1)$ & -- & 370\\
$(11,4)$ & 5 & $(7,3)$ & 4 & 1 & YES & YES & YES & $0.67$ & $(2,1)$ & 433 & 371\\
$(11,4)$ & 5 & $(7,3)$ & 4 & 1 & YES & YES & YES & $0.78$ & $(2,1)$ & -- & 372\\
$(11,4)$ & 5 & $(9,2)$ & 5 & 1 & YES & YES & YES & $0.67$ & $(2,1)$ & -- & 373\\
$(11,4)$ & 5 & $(9,2)$ & 5 & 1 & YES & YES & YES & $0.78$ & $(2,1)$ & NO & 374\\
$(12,5)$ & 5 & $(5,2)$ & 3 & 1 & YES & YES & YES & $0.56$ & $(2,1)$ & -- & 375\\
$(12,5)$ & 5 & $(7,2)$ & 4 & 1 & YES & YES & YES & $0.67$ & $(2,1)$ & NO & 376\\
$(12,5)$ & 5 & $(7,2)$ & 4 & 1 & YES & YES & YES & $0.67$ & $(2,1)$ & -- & 377\\
$(12,5)$ & 5 & $(7,2)$ & 4 & 1 & YES & YES & YES & $0.67$ & $(2,1)$ & NO & 378\\
$(12,5)$ & 5 & $(8,3)$ & 4 & 4 & YES & YES & YES & $0.56$ & $(2,1)$ & NO & 379\\
$(12,5)$ & 5 & $(8,3)$ & 4 & 4 & YES & YES & YES & $0.67$ & $(2,1)$ & -- & 380\\
$(12,5)$ & 5 & $(10,3)$ & 5 & 2 & YES & YES & YES & $0.78$ & $(2,1)$ & -- & 381\\
$(13,5)$ & 5 & $(3,1)$ & 2 & 1 & YES & YES & YES & $0.78$ & $(2,1)$ & -- & 382\\
$(13,5)$ & 5 & $(3,1)$ & 2 & 1 & YES & YES & YES & $0.89$ & $(2,1)$ & NO & 383\\
$(13,5)$ & 5 & $(4,1)$ & 3 & 1 & YES & YES & YES & $0.67$ & $(2,1)$ & NO & 384\\
$(13,5)$ & 5 & $(4,1)$ & 3 & 1 & YES & YES & YES & $0.67$ & $(2,1)$ & -- & 385\\
$(13,5)$ & 5 & $(4,1)$ & 3 & 1 & YES & YES & YES & $0.78$ & $(2,1)$ & NO & 386\\
$(13,5)$ & 5 & $(5,2)$ & 3 & 1 & YES & YES & YES & $0.78$ & $(2,1)$ & -- & 387\\
$(13,5)$ & 5 & $(5,2)$ & 3 & 1 & YES & YES & YES & $0.78$ & $(2,1)$ & NO & 388\\
$(13,4)$ & 6 & $(7,2)$ & 4 & 1 & YES & YES & YES & $0.56$ & $(2,1)$ & -- & 389\\
$(13,5)$ & 5 & $(7,2)$ & 4 & 1 & YES & YES & YES & $0.67$ & $(2,1)$ & NO & 390\\
$(13,5)$ & 5 & $(7,2)$ & 4 & 1 & YES & YES & YES & $0.56$ & $(2,1)$ & -- & 391\\
$(13,5)$ & 5 & $(7,3)$ & 4 & 1 & YES & YES & YES & $0.78$ & $(2,1)$ & -- & 392\\
$(13,5)$ & 5 & $(7,3)$ & 4 & 1 & YES & YES & YES & $0.56$ & $(2,1)$ & NO & 393\\
$(13,5)$ & 5 & $(8,3)$ & 4 & 1 & YES & YES & YES & $0.78$ & $(2,1)$ & NO & 394\\
$(13,4)$ & 6 & $(9,2)$ & 5 & 1 & YES & YES & YES & $0.67$ & $(2,1)$ & NO & 395\\
$(13,4)$ & 6 & $(9,2)$ & 5 & 1 & YES & YES & YES & $0.67$ & $(2,1)$ & -- & 396\\
$(13,5)$ & 5 & $(9,2)$ & 5 & 1 & YES & YES & YES & $0.67$ & $(2,1)$ & NO & 397\\
$(13,4)$ & 6 & $(11,3)$ & 5 & 1 & YES & YES & YES & $0.67$ & $(2,1)$ & 472 & 398\\
$(13,5)$ & 5 & $(11,4)$ & 5 & 1 & YES & YES & YES & $0.67$ & $(2,1)$ & NO & 399\\
$(13,5)$ & 5 & $(13,5)$ & 5 & 13 & YES & YES & YES & $0.67$ & $(2,1)$ & NO & 400\\
$(15,4)$ & 6 & $(5,2)$ & 3 & 5 & YES & YES & YES & $0.56$ & $(2,1)$ & NO & 401\\
$(15,4)$ & 6 & $(5,2)$ & 3 & 5 & YES & YES & YES & $0.67$ & $(2,1)$ & NO & 402\\
$(17,5)$ & 6 & $(2,1)$ & 1 & 1 & YES & YES & YES & $0.67$ & $(2,1)$ & -- & 403\\
$(17,5)$ & 6 & $(2,1)$ & 1 & 1 & YES & YES & YES & $0.78$ & $(2,1)$ & NO & 404\\
$(17,5)$ & 6 & $(3,1)$ & 2 & 1 & YES & YES & YES & $0.67$ & $(2,1)$ & -- & 405\\
$(17,5)$ & 6 & $(3,1)$ & 2 & 1 & YES & YES & YES & $0.78$ & $(2,1)$ & NO & 406\\
$(17,7)$ & 6 & $(3,1)$ & 2 & 1 & YES & YES & YES & $0.78$ & $(2,1)$ & -- & 407\\
$(17,7)$ & 6 & $(3,1)$ & 2 & 1 & YES & YES & YES & $0.78$ & $(2,1)$ & NO & 408\\
$(17,5)$ & 6 & $(4,1)$ & 3 & 1 & YES & YES & YES & $0.67$ & $(2,1)$ & NO & 409\\
$(17,5)$ & 6 & $(4,1)$ & 3 & 1 & YES & YES & YES & $0.67$ & $(2,1)$ & -- & 410\\
$(17,7)$ & 6 & $(4,1)$ & 3 & 1 & YES & YES & YES & $0.67$ & $(2,1)$ & NO & 411\\
$(17,7)$ & 6 & $(4,1)$ & 3 & 1 & YES & YES & YES & $0.78$ & $(2,1)$ & -- & 412\\
$(17,5)$ & 6 & $(5,2)$ & 3 & 1 & YES & YES & YES & $0.44$ & $(2,1)$ & NO & 413\\
$(17,5)$ & 6 & $(5,2)$ & 3 & 1 & YES & YES & YES & $0.67$ & $(2,1)$ & -- & 414\\
$(17,7)$ & 6 & $(5,1)$ & 4 & 1 & YES & YES & YES & $0.89$ & $(2,1)$ & NO & 415\\
$(17,7)$ & 6 & $(5,1)$ & 4 & 1 & YES & YES & YES & $0.89$ & $(2,1)$ & NO & 416\\
$(17,5)$ & 6 & $(7,2)$ & 4 & 1 & YES & YES & YES & $0.67$ & $(2,1)$ & NO & 417\\
$(17,5)$ & 6 & $(7,2)$ & 4 & 1 & YES & YES & YES & $0.67$ & $(2,1)$ & -- & 418\\
$(17,7)$ & 6 & $(7,3)$ & 4 & 1 & YES & YES & YES & $0.78$ & $(2,1)$ & 449 & 419\\
$(17,5)$ & 6 & $(9,2)$ & 5 & 1 & YES & YES & YES & $0.56$ & $(2,1)$ & NO & 420\\
$(17,5)$ & 6 & $(10,3)$ & 5 & 1 & YES & YES & YES & $0.67$ & $(2,1)$ & NO & 421\\
$(17,7)$ & 6 & $(12,5)$ & 5 & 1 & YES & YES & YES & $0.67$ & $(2,1)$ & NO & 422\\
$(17,5)$ & 6 & $(13,4)$ & 6 & 1 & YES & YES & YES & $0.67$ & $(2,1)$ & NO & 423\\
$(17,5)$ & 6 & $(17,5)$ & 6 & 17 & YES & YES & YES & $0.67$ & $(2,1)$ & NO & 424\\
$(17,7)$ & 6 & $(17,7)$ & 6 & 17 & YES & YES & YES & $0.89$ & $(2,1)$ & NO & 425\\
$(18,7)$ & 6 & $(4,1)$ & 3 & 2 & YES & YES & YES & $0.67$ & $(2,1)$ & 478 & 426\\
$(18,5)$ & 6 & $(5,2)$ & 3 & 1 & YES & YES & YES & $0.67$ & $(2,1)$ & NO & 427\\
$(18,7)$ & 6 & $(5,1)$ & 4 & 1 & YES & YES & YES & $0.78$ & $(2,1)$ & NO & 428\\
$(18,7)$ & 6 & $(8,3)$ & 4 & 2 & YES & YES & YES & $0.67$ & $(2,1)$ & 464 & 429\\
$(18,5)$ & 6 & $(10,3)$ & 5 & 2 & YES & YES & YES & $0.67$ & $(2,1)$ & NO & 430\\
$(18,7)$ & 6 & $(13,5)$ & 5 & 1 & YES & YES & YES & $0.78$ & $(2,1)$ & NO & 431\\
$(19,7)$ & 6 & $(2,1)$ & 1 & 1 & YES & YES & YES & $0.56$ & $(2,1)$ & -- & 432\\
$(19,7)$ & 6 & $(2,1)$ & 1 & 1 & YES & YES & YES & $0.67$ & $(2,1)$ & 371 & 433\\
$(19,8)$ & 6 & $(2,1)$ & 1 & 1 & YES & YES & YES & $0.78$ & $(2,1)$ & -- & 434\\
$(19,7)$ & 6 & $(3,1)$ & 2 & 1 & YES & YES & YES & $0.67$ & $(2,1)$ & -- & 435\\
$(19,7)$ & 6 & $(3,1)$ & 2 & 1 & YES & YES & YES & $0.89$ & $(2,1)$ & NO & 436\\
$(19,8)$ & 6 & $(3,1)$ & 2 & 1 & YES & YES & YES & $0.67$ & $(2,1)$ & NO & 437\\
$(19,8)$ & 6 & $(3,1)$ & 2 & 1 & YES & YES & YES & $0.78$ & $(2,1)$ & 458 & 438\\
$(19,8)$ & 6 & $(3,1)$ & 2 & 1 & YES & YES & YES & $0.78$ & $(2,1)$ & -- & 439\\
$(19,7)$ & 6 & $(4,1)$ & 3 & 1 & YES & YES & YES & $0.56$ & $(2,1)$ & NO & 440\\
$(19,7)$ & 6 & $(4,1)$ & 3 & 1 & YES & YES & YES & $0.67$ & $(2,1)$ & -- & 441\\
$(19,7)$ & 6 & $(4,1)$ & 3 & 1 & YES & YES & YES & $0.67$ & $(2,1)$ & NO & 442\\
$(19,8)$ & 6 & $(4,1)$ & 3 & 1 & YES & YES & YES & $0.78$ & $(2,1)$ & NO & 443\\
$(19,8)$ & 6 & $(4,1)$ & 3 & 1 & YES & YES & YES & $0.78$ & $(2,1)$ & -- & 444\\
$(19,8)$ & 6 & $(4,1)$ & 3 & 1 & YES & YES & YES & $0.67$ & $(2,1)$ & NO & 445\\
$(19,7)$ & 6 & $(5,2)$ & 3 & 1 & YES & YES & YES & $0.67$ & $(2,1)$ & NO & 446\\
$(19,8)$ & 6 & $(5,1)$ & 4 & 1 & YES & YES & YES & $0.78$ & $(2,1)$ & NO & 447\\
$(19,8)$ & 6 & $(5,1)$ & 4 & 1 & YES & YES & YES & $0.78$ & $(2,1)$ & NO & 448\\
$(19,8)$ & 6 & $(5,2)$ & 3 & 1 & YES & YES & YES & $0.78$ & $(2,1)$ & 419 & 449\\
$(19,8)$ & 6 & $(5,2)$ & 3 & 1 & YES & YES & YES & $0.78$ & $(2,1)$ & -- & 450\\
$(19,8)$ & 6 & $(7,3)$ & 4 & 1 & YES & YES & YES & $0.78$ & $(2,1)$ & NO & 451\\
$(19,7)$ & 6 & $(8,3)$ & 4 & 1 & YES & YES & YES & $0.56$ & $(2,1)$ & NO & 452\\
$(19,7)$ & 6 & $(11,4)$ & 5 & 1 & YES & YES & YES & $0.56$ & $(2,1)$ & NO & 453\\
$(19,8)$ & 6 & $(12,5)$ & 5 & 1 & YES & YES & YES & $0.78$ & $(2,1)$ & NO & 454\\
$(19,7)$ & 6 & $(19,7)$ & 6 & 19 & YES & YES & YES & $0.67$ & $(2,1)$ & NO & 455\\
$(19,8)$ & 6 & $(19,8)$ & 6 & 19 & YES & YES & YES & $0.78$ & $(2,1)$ & NO & 456\\
$(21,8)$ & 6 & $(2,1)$ & 1 & 1 & YES & YES & YES & $0.67$ & $(2,1)$ & -- & 457\\
$(21,8)$ & 6 & $(2,1)$ & 1 & 1 & YES & YES & YES & $0.78$ & $(2,1)$ & 438 & 458\\
$(21,8)$ & 6 & $(3,1)$ & 2 & 3 & YES & YES & YES & $0.67$ & $(2,1)$ & -- & 459\\
$(21,8)$ & 6 & $(3,1)$ & 2 & 3 & YES & YES & YES & $0.78$ & $(2,1)$ & NO & 460\\
$(21,8)$ & 6 & $(3,1)$ & 2 & 3 & YES & YES & YES & $0.67$ & $(2,1)$ & NO & 461\\
$(21,8)$ & 6 & $(4,1)$ & 3 & 1 & YES & YES & YES & $0.67$ & $(2,1)$ & NO & 462\\
$(21,8)$ & 6 & $(5,1)$ & 4 & 1 & YES & YES & YES & $0.67$ & $(2,1)$ & NO & 463\\
$(21,8)$ & 6 & $(5,2)$ & 3 & 1 & YES & YES & YES & $0.67$ & $(2,1)$ & 429 & 464\\
$(21,8)$ & 6 & $(8,3)$ & 4 & 1 & YES & YES & YES & $0.67$ & $(2,1)$ & NO & 465\\
$(21,8)$ & 6 & $(13,5)$ & 5 & 1 & YES & YES & YES & $0.67$ & $(2,1)$ & NO & 466\\
$(21,8)$ & 6 & $(21,8)$ & 6 & 21 & YES & YES & YES & $0.56$ & $(2,1)$ & NO & 467\\
$(23,7)$ & 7 & $(2,1)$ & 1 & 1 & YES & YES & YES & $0.67$ & $(2,1)$ & -- & 468\\
$(23,7)$ & 7 & $(2,1)$ & 1 & 1 & YES & YES & YES & $0.78$ & $(2,1)$ & NO & 469\\
$(23,7)$ & 7 & $(3,1)$ & 2 & 1 & YES & YES & YES & $0.56$ & $(2,1)$ & NO & 470\\
$(23,7)$ & 7 & $(3,1)$ & 2 & 1 & YES & YES & YES & $0.56$ & $(2,1)$ & -- & 471\\
$(23,7)$ & 7 & $(4,1)$ & 3 & 1 & YES & YES & YES & $0.67$ & $(2,1)$ & 398 & 472\\
$(23,7)$ & 7 & $(4,1)$ & 3 & 1 & YES & YES & YES & $0.67$ & $(2,1)$ & -- & 473\\
$(23,7)$ & 7 & $(7,2)$ & 4 & 1 & YES & YES & YES & $0.67$ & $(2,1)$ & NO & 474\\
$(23,7)$ & 7 & $(10,3)$ & 5 & 1 & YES & YES & YES & $0.67$ & $(2,1)$ & NO & 475\\
$(23,7)$ & 7 & $(13,4)$ & 6 & 1 & YES & YES & YES & $0.56$ & $(2,1)$ & NO & 476\\
$(24,7)$ & 7 & $(2,1)$ & 1 & 2 & YES & YES & YES & $0.56$ & $(2,1)$ & -- & 477\\
$(24,7)$ & 7 & $(2,1)$ & 1 & 2 & YES & YES & YES & $0.67$ & $(2,1)$ & 426 & 478\\
$(24,7)$ & 7 & $(3,1)$ & 2 & 3 & YES & YES & YES & $0.56$ & $(2,1)$ & -- & 479\\
$(24,7)$ & 7 & $(3,1)$ & 2 & 3 & YES & YES & YES & $0.67$ & $(2,1)$ & NO & 480\\
$(24,7)$ & 7 & $(4,1)$ & 3 & 4 & YES & YES & YES & $0.56$ & $(2,1)$ & NO & 481\\
$(24,7)$ & 7 & $(7,2)$ & 4 & 1 & YES & YES & YES & $0.67$ & $(2,1)$ & NO & 482\\
$(24,7)$ & 7 & $(10,3)$ & 5 & 2 & YES & YES & YES & $0.56$ & $(2,1)$ & 493 & 483\\
$(24,7)$ & 7 & $(17,5)$ & 6 & 1 & YES & YES & YES & $0.56$ & $(2,1)$ & NO & 484\\
$(24,7)$ & 7 & $(24,7)$ & 7 & 24 & YES & YES & YES & $0.56$ & $(2,1)$ & NO & 485\\
$(25,7)$ & 7 & $(2,1)$ & 1 & 1 & YES & YES & YES & $0.44$ & $(2,1)$ & NO & 486\\
$(25,7)$ & 7 & $(4,1)$ & 3 & 1 & YES & YES & YES & $0.56$ & $(2,1)$ & NO & 487\\
$(25,7)$ & 7 & $(7,2)$ & 4 & 1 & YES & YES & YES & $0.44$ & $(2,1)$ & NO & 488\\
$(27,8)$ & 7 & $(2,1)$ & 1 & 1 & YES & YES & YES & $0.67$ & $(2,1)$ & -- & 489\\
$(27,8)$ & 7 & $(3,1)$ & 2 & 3 & YES & YES & YES & $0.44$ & $(2,1)$ & -- & 490\\
$(27,8)$ & 7 & $(3,1)$ & 2 & 3 & YES & YES & YES & $0.56$ & $(2,1)$ & NO & 491\\
$(27,8)$ & 7 & $(4,1)$ & 3 & 1 & YES & YES & YES & $0.56$ & $(2,1)$ & 496 & 492\\
$(27,8)$ & 7 & $(7,2)$ & 4 & 1 & YES & YES & YES & $0.56$ & $(2,1)$ & 483 & 493\\
$(29,8)$ & 7 & $(2,1)$ & 1 & 1 & YES & YES & YES & $0.67$ & $(2,1)$ & NO & 494\\
$(29,12)$ & 7 & $(2,1)$ & 1 & 1 & NO & YES & YES & $0.78$ & $(2,1)$ & -- & 495\\
$(29,8)$ & 7 & $(3,1)$ & 2 & 1 & YES & YES & YES & $0.56$ & $(2,1)$ & 492 & 496\\
$(29,8)$ & 7 & $(4,1)$ & 3 & 1 & YES & YES & YES & $0.56$ & $(2,1)$ & NO & 497\\
$(29,8)$ & 7 & $(29,8)$ & 7 & 29 & YES & YES & YES & $0.44$ & $(2,1)$ & NO & 498\\
$(31,7)$ & 8 & $(2,1)$ & 1 & 1 & YES & YES & YES & $0.67$ & $(2,1)$ & NO & 499\\
$(31,12)$ & 7 & $(2,1)$ & 1 & 1 & NO & YES & YES & $0.78$ & $(2,1)$ & -- & 500\\
$(31,7)$ & 8 & $(4,1)$ & 3 & 1 & YES & YES & YES & $0.67$ & $(2,1)$ & NO & 501\\
$(31,7)$ & 8 & $(9,2)$ & 5 & 1 & YES & YES & YES & $0.56$ & $(2,1)$ & NO & 502\\
$(32,7)$ & 8 & $(3,1)$ & 2 & 1 & YES & YES & YES & $0.67$ & $(2,1)$ & -- & 503\\
$(32,7)$ & 8 & $(9,2)$ & 5 & 1 & YES & YES & YES & $0.78$ & $(2,1)$ & NO & 504\\
$(35,8)$ & 8 & $(2,1)$ & 1 & 1 & YES & YES & YES & $0.78$ & $(2,1)$ & NO & 505\\
$(35,8)$ & 8 & $(4,1)$ & 3 & 1 & YES & YES & YES & $0.78$ & $(2,1)$ & NO & 506\\
$(a;0,0,0;3)$ & 4 & $(4,1)$ & 3 & 1 & YES & YES & YES & $0.67$ & $(2,1)$ & -- & 507\\
$(a;0,0,0;3)$ & 4 & $(5,2)$ & 3 & 1 & YES & YES & YES & $0.56$ & $(2,1)$ & -- & 508\\
$(a;0,0,0;3)$ & 4 & $(7,3)$ & 4 & 1 & YES & YES & YES & $0.67$ & $(2,1)$ & -- & 509\\
$(a;1,0,0;13)$ & 5 & $(3,1)$ & 2 & 1 & YES & YES & YES & $0.67$ & $(2,1)$ & -- & 510\\
$(a;1,0,0;13)$ & 5 & $(4,1)$ & 3 & 1 & YES & YES & YES & $0.67$ & $(2,1)$ & -- & 511\\
$(a;1,0,0;13)$ & 5 & $(5,2)$ & 3 & 1 & YES & YES & YES & $0.78$ & $(2,1)$ & -- & 512\\
$(a;1,0,0;13)$ & 5 & $(7,2)$ & 4 & 1 & YES & YES & YES & $0.56$ & $(2,1)$ & -- & 513\\
$(a;1,1,0;19)$ & 6 & $(4,1)$ & 3 & 1 & YES & YES & YES & $0.67$ & $(2,1)$ & -- & 514\\
$(b;0,0,0;14)$ & 5 & $(2,1)$ & 1 & 2 & YES & YES & YES & $0.67$ & $(2,1)$ & -- & 515\\
$(b;0,0,0;14)$ & 5 & $(3,1)$ & 2 & 1 & YES & YES & YES & $0.56$ & $(2,1)$ & -- & 516\\
$(b;0,0,0;14)$ & 5 & $(5,2)$ & 3 & 1 & YES & YES & YES & $0.67$ & $(2,1)$ & -- & 517\\
$(b;0,0,0;14)$ & 5 & $(7,2)$ & 4 & 7 & YES & YES & YES & $0.56$ & $(2,1)$ & -- & 518\\
$(b;0,0,1;4)$ & 6 & $(4,1)$ & 3 & 4 & YES & YES & YES & $0.56$ & $(2,1)$ & -- & 519\\
$(b;0,1,0;19)$ & 6 & $(5,1)$ & 4 & 1 & YES & YES & YES & $0.78$ & $(2,1)$ & -- & 520\\
$(b;1,0,0;5)$ & 6 & $(4,1)$ & 3 & 1 & YES & YES & YES & $0.67$ & $(2,1)$ & -- & 521\\
$(c;0,0,0;4)$ & 4 & $(7,3)$ & 4 & 1 & YES & YES & YES & $0.67$ & $(2,1)$ & -- & 522\\
$(c;0,0,0;4)$ & 4 & $(8,3)$ & 4 & 4 & YES & YES & YES & $0.56$ & $(2,1)$ & -- & 523\\
$(c;0,1,0;11)$ & 5 & $(3,1)$ & 2 & 1 & YES & YES & YES & $0.78$ & $(2,1)$ & -- & 524\\
$(c;0,1,0;11)$ & 5 & $(4,1)$ & 3 & 1 & YES & YES & YES & $0.67$ & $(2,1)$ & -- & 525\\
$(c;0,1,0;11)$ & 5 & $(5,2)$ & 3 & 1 & YES & YES & YES & $0.44$ & $(2,1)$ & -- & 526\\
$(c;0,1,1;5)$ & 6 & $(2,1)$ & 1 & 1 & YES & YES & YES & $0.67$ & $(2,1)$ & -- & 527\\
$(c;0,1,1;5)$ & 6 & $(4,1)$ & 3 & 1 & YES & YES & YES & $0.56$ & $(2,1)$ & -- & 528\\
$(c;0,2,0;7)$ & 6 & $(2,1)$ & 1 & 1 & YES & YES & YES & $0.67$ & $(2,1)$ & -- & 529\\
$(c;0,2,0;7)$ & 6 & $(3,1)$ & 2 & 1 & YES & YES & YES & $0.67$ & $(2,1)$ & -- & 530\\
$(c;0,2,0;7)$ & 6 & $(4,1)$ & 3 & 1 & YES & YES & YES & $0.67$ & $(2,1)$ & -- & 531\\
$(c;0,2,0;7)$ & 6 & $(5,1)$ & 4 & 1 & YES & YES & YES & $0.67$ & $(2,1)$ & -- & 532\\
$(d;0,0,0;5)$ & 5 & $(2,1)$ & 1 & 1 & YES & YES & YES & $0.56$ & $(2,1)$ & -- & 533\\
$(d;0,0,0;5)$ & 5 & $(3,1)$ & 2 & 1 & YES & YES & YES & $0.56$ & $(2,1)$ & -- & 534\\
$(d;0,0,0;5)$ & 5 & $(4,1)$ & 3 & 1 & YES & YES & YES & $0.56$ & $(2,1)$ & -- & 535\\
$(d;0,0,1;14)$ & 6 & $(2,1)$ & 1 & 2 & YES & YES & YES & $0.78$ & $(2,1)$ & -- & 536\\
$(d;0,0,1;14)$ & 6 & $(3,1)$ & 2 & 1 & YES & YES & YES & $0.67$ & $(2,1)$ & -- & 537\\
$(d;0,0,1;14)$ & 6 & $(4,1)$ & 3 & 2 & YES & YES & YES & $0.67$ & $(2,1)$ & -- & 538\\
$(d;0,1,0;6)$ & 6 & $(2,1)$ & 1 & 2 & YES & YES & YES & $0.56$ & $(2,1)$ & -- & 539\\
$(d;0,1,0;6)$ & 6 & $(3,1)$ & 2 & 3 & YES & YES & YES & $0.56$ & $(2,1)$ & -- & 540\\
$(d;0,1,0;6)$ & 6 & $(4,1)$ & 3 & 2 & YES & YES & YES & $0.56$ & $(2,1)$ & -- & 541\\
$(d;0,1,0;6)$ & 6 & $(5,1)$ & 4 & 1 & YES & YES & YES & $0.56$ & $(2,1)$ & -- & 542\\
$(e;0,0,0;4)$ & 5 & $(2,1)$ & 1 & 2 & YES & YES & YES & $0.78$ & $(2,1)$ & -- & 543\\
$(e;0,0,0;4)$ & 5 & $(3,1)$ & 2 & 1 & YES & YES & YES & $0.67$ & $(2,1)$ & -- & 544\\
$(f;0,0,0;6)$ & 4 & $(7,3)$ & 4 & 1 & YES & YES & YES & $0.67$ & $(2,1)$ & -- & 545\\
$(f;0,0,0;6)$ & 4 & $(8,3)$ & 4 & 2 & YES & YES & YES & $0.56$ & $(2,1)$ & -- & 546\\
$(f;0,0,0;6)$ & 4 & $(10,3)$ & 5 & 2 & YES & YES & YES & $0.67$ & $(2,1)$ & -- & 547\\
$(f;0,0,0;6)$ & 4 & $(11,3)$ & 5 & 1 & YES & YES & YES & $0.67$ & $(2,1)$ & -- & 548\\
$(h;0,0,0;6)$ & 5 & $(2,1)$ & 1 & 2 & YES & YES & YES & $0.67$ & $(2,1)$ & -- & 549\\
$(h;0,0,0;6)$ & 5 & $(3,1)$ & 2 & 3 & YES & YES & YES & $0.56$ & $(2,1)$ & -- & 550\\
$(i;0,0,0;9)$ & 5 & $(3,1)$ & 2 & 3 & YES & YES & YES & $0.78$ & $(2,1)$ & -- & 551\\
$(i;0,0,0;9)$ & 5 & $(4,1)$ & 3 & 1 & YES & YES & YES & $0.67$ & $(2,1)$ & -- & 552\\
$(j;0,0,0;8)$ & 5 & $(5,2)$ & 3 & 1 & YES & YES & YES & $0.67$ & $(2,1)$ & -- & 553\\
$(j;0,0,0;8)$ & 5 & $(7,2)$ & 4 & 1 & YES & YES & YES & $0.56$ & $(2,1)$ & -- & 554
\end{longtable}
\subsection{2 chains, $K^2 = 2$}
\begin{longtable}{|c|c|c|c|c|c|c|c|c|c|c|c|}
\hline
\multicolumn{12}{|c|}{2 chains, $K^2 = 2$}\\
\hline
$(n,a)$ & Len & $(n,a)$ & Len & GCD & Nef & $\mathbb Q$-ef & Obs 0 & $\overline c_1^2 / \overline c_2$ & $(P,K)$ & WH & Index\\
\hline
\endfirsthead

\hline
$(n,a)$ & Len & $(n,a)$ & Len & GCD & Nef & $\mathbb Q$-ef & Obs 0 & $\overline c_1^2 / \overline c_2$ & $(P,K)$ & WH & Index\\
\hline
\endhead
\hline
\endfoot

$(11,3)$ & 5 & $(10,3)$ & 5 & 1 & YES & YES & YES & $1.11$ & $(2,2)$ & -- & 555\\
$(11,3)$ & 5 & $(10,3)$ & 5 & 1 & YES & YES & YES & $1.22$ & $(2,2)$ & NO & 556\\
$(12,5)$ & 5 & $(10,3)$ & 5 & 2 & YES & YES & NO(2) & $1.11$ & $(2,2)$ & NO & 557\\
$(12,5)$ & 5 & $(10,3)$ & 5 & 2 & YES & YES & YES & $1.12$ & $(2,2)$ & -- & 558\\
$(12,5)$ & 5 & $(11,3)$ & 5 & 1 & YES & YES & NO(2) & $1.11$ & $(2,2)$ & NO & 559\\
$(12,5)$ & 5 & $(11,3)$ & 5 & 1 & YES & YES & NO(2) & $1.22$ & $(2,2)$ & -- & 560\\
$(13,5)$ & 5 & $(9,2)$ & 5 & 1 & YES & YES & YES & $1.12$ & $(2,2)$ & -- & 561\\
$(13,3)$ & 6 & $(10,3)$ & 5 & 1 & YES & YES & YES & $1.00$ & $(2,2)$ & NO & 562\\
$(13,3)$ & 6 & $(10,3)$ & 5 & 1 & YES & YES & YES & $1.00$ & $(2,2)$ & -- & 563\\
$(13,5)$ & 5 & $(10,3)$ & 5 & 1 & YES & YES & YES & $1.12$ & $(2,2)$ & -- & 564\\
$(13,5)$ & 5 & $(11,3)$ & 5 & 1 & YES & YES & NO(2) & $1.33$ & $(2,2)$ & -- & 565\\
$(13,5)$ & 5 & $(11,3)$ & 5 & 1 & YES & YES & YES & $1.00$ & $(2,2)$ & NO & 566\\
$(13,5)$ & 5 & $(11,4)$ & 5 & 1 & YES & YES & YES & $1.00$ & $(2,2)$ & -- & 567\\
$(13,3)$ & 6 & $(12,5)$ & 5 & 1 & YES & YES & YES & $1.12$ & $(2,2)$ & NO & 568\\
$(13,3)$ & 6 & $(12,5)$ & 5 & 1 & YES & YES & YES & $1.12$ & $(2,2)$ & -- & 569\\
$(13,5)$ & 5 & $(12,5)$ & 5 & 1 & YES & YES & NO(2) & $1.33$ & $(2,2)$ & -- & 570\\
$(13,4)$ & 6 & $(13,3)$ & 6 & 13 & YES & YES & YES & $1.00$ & $(2,2)$ & NO & 571\\
$(13,4)$ & 6 & $(13,3)$ & 6 & 13 & YES & YES & YES & $1.00$ & $(2,2)$ & -- & 572\\
$(13,5)$ & 5 & $(13,3)$ & 6 & 13 & YES & YES & YES & $1.12$ & $(2,2)$ & NO & 573\\
$(13,5)$ & 5 & $(13,3)$ & 6 & 13 & YES & YES & YES & $1.12$ & $(2,2)$ & -- & 574\\
$(13,5)$ & 5 & $(13,4)$ & 6 & 13 & YES & YES & YES & $1.25$ & $(2,2)$ & NO & 575\\
$(13,5)$ & 5 & $(13,4)$ & 6 & 13 & YES & YES & YES & $1.25$ & $(2,2)$ & -- & 576\\
$(13,5)$ & 5 & $(13,5)$ & 5 & 13 & YES & YES & YES & $1.11$ & $(2,2)$ & NO & 577\\
$(13,5)$ & 5 & $(13,5)$ & 5 & 13 & YES & YES & YES & $1.11$ & $(2,2)$ & -- & 578\\
$(14,3)$ & 6 & $(10,3)$ & 5 & 2 & YES & YES & YES & $1.00$ & $(2,2)$ & NO & 579\\
$(14,3)$ & 6 & $(10,3)$ & 5 & 2 & YES & YES & YES & $1.00$ & $(2,2)$ & -- & 580\\
$(14,5)$ & 6 & $(12,5)$ & 5 & 2 & YES & YES & YES & $1.00$ & $(2,2)$ & -- & 581\\
$(14,5)$ & 6 & $(13,3)$ & 6 & 1 & YES & YES & YES & $1.12$ & $(2,2)$ & -- & 582\\
$(14,5)$ & 6 & $(13,3)$ & 6 & 1 & YES & YES & YES & $1.25$ & $(2,2)$ & NO & 583\\
$(14,5)$ & 6 & $(13,4)$ & 6 & 1 & YES & YES & YES & $1.12$ & $(2,2)$ & -- & 584\\
$(14,5)$ & 6 & $(13,5)$ & 5 & 1 & YES & YES & YES & $1.00$ & $(2,2)$ & -- & 585\\
$(14,5)$ & 6 & $(13,5)$ & 5 & 1 & YES & YES & YES & $1.00$ & $(2,2)$ & NO & 586\\
$(14,5)$ & 6 & $(14,5)$ & 6 & 14 & YES & YES & YES & $1.00$ & $(2,2)$ & -- & 587\\
$(15,4)$ & 6 & $(8,3)$ & 4 & 1 & YES & YES & YES & $1.00$ & $(2,2)$ & -- & 588\\
$(15,4)$ & 6 & $(10,3)$ & 5 & 5 & YES & YES & YES & $1.12$ & $(2,2)$ & NO & 589\\
$(15,4)$ & 6 & $(10,3)$ & 5 & 5 & YES & YES & YES & $1.12$ & $(2,2)$ & -- & 590\\
$(15,4)$ & 6 & $(11,3)$ & 5 & 1 & YES & YES & YES & $1.00$ & $(2,2)$ & NO & 591\\
$(15,4)$ & 6 & $(11,3)$ & 5 & 1 & YES & YES & YES & $1.00$ & $(2,2)$ & -- & 592\\
$(15,4)$ & 6 & $(13,3)$ & 6 & 1 & YES & YES & YES & $1.00$ & $(2,2)$ & NO & 593\\
$(15,4)$ & 6 & $(13,3)$ & 6 & 1 & YES & YES & YES & $1.00$ & $(2,2)$ & -- & 594\\
$(16,7)$ & 6 & $(11,4)$ & 5 & 1 & YES & YES & YES & $0.88$ & $(2,2)$ & -- & 595\\
$(16,7)$ & 6 & $(12,5)$ & 5 & 4 & YES & YES & YES & $0.88$ & $(2,2)$ & -- & 596\\
$(16,7)$ & 6 & $(13,3)$ & 6 & 1 & YES & YES & YES & $1.25$ & $(2,2)$ & -- & 597\\
$(16,7)$ & 6 & $(13,3)$ & 6 & 1 & YES & YES & YES & $1.25$ & $(2,2)$ & NO & 598\\
$(16,7)$ & 6 & $(13,4)$ & 6 & 1 & YES & YES & YES & $1.00$ & $(2,2)$ & -- & 599\\
$(16,7)$ & 6 & $(13,5)$ & 5 & 1 & YES & YES & YES & $1.00$ & $(2,2)$ & -- & 600\\
$(16,7)$ & 6 & $(13,5)$ & 5 & 1 & YES & YES & YES & $1.00$ & $(2,2)$ & NO & 601\\
$(16,7)$ & 6 & $(14,5)$ & 6 & 2 & YES & YES & YES & $1.00$ & $(2,2)$ & -- & 602\\
$(16,5)$ & 7 & $(15,4)$ & 6 & 1 & YES & YES & YES & $1.12$ & $(2,2)$ & -- & 603\\
$(16,5)$ & 7 & $(15,4)$ & 6 & 1 & YES & YES & YES & $1.25$ & $(2,2)$ & NO & 604\\
$(16,7)$ & 6 & $(16,5)$ & 7 & 16 & YES & YES & YES & $1.12$ & $(2,2)$ & -- & 605\\
$(16,7)$ & 6 & $(16,7)$ & 6 & 16 & YES & YES & YES & $1.00$ & $(2,2)$ & -- & 606\\
$(17,7)$ & 6 & $(8,3)$ & 4 & 1 & YES & YES & YES & $1.00$ & $(2,2)$ & -- & 607\\
$(17,5)$ & 6 & $(9,2)$ & 5 & 1 & YES & YES & YES & $1.12$ & $(2,2)$ & NO & 608\\
$(17,5)$ & 6 & $(9,2)$ & 5 & 1 & YES & YES & YES & $1.00$ & $(2,2)$ & -- & 609\\
$(17,7)$ & 6 & $(9,2)$ & 5 & 1 & YES & YES & YES & $1.12$ & $(2,2)$ & NO & 610\\
$(17,7)$ & 6 & $(9,2)$ & 5 & 1 & YES & YES & YES & $1.12$ & $(2,2)$ & -- & 611\\
$(17,7)$ & 6 & $(9,2)$ & 5 & 1 & YES & YES & YES & $1.12$ & $(2,2)$ & NO & 612\\
$(17,7)$ & 6 & $(10,3)$ & 5 & 1 & YES & YES & YES & $1.12$ & $(2,2)$ & NO & 613\\
$(17,7)$ & 6 & $(10,3)$ & 5 & 1 & YES & YES & YES & $1.12$ & $(2,2)$ & -- & 614\\
$(17,5)$ & 6 & $(11,4)$ & 5 & 1 & YES & YES & YES & $1.12$ & $(2,2)$ & -- & 615\\
$(17,6)$ & 7 & $(11,4)$ & 5 & 1 & YES & YES & YES & $1.00$ & $(2,2)$ & -- & 616\\
$(17,7)$ & 6 & $(11,4)$ & 5 & 1 & YES & YES & YES & $1.12$ & $(2,2)$ & -- & 617\\
$(17,7)$ & 6 & $(11,5)$ & 6 & 1 & YES & YES & YES & $1.12$ & $(2,2)$ & -- & 618\\
$(17,5)$ & 6 & $(12,5)$ & 5 & 1 & YES & YES & YES & $1.00$ & $(2,2)$ & NO & 619\\
$(17,5)$ & 6 & $(12,5)$ & 5 & 1 & YES & YES & YES & $1.00$ & $(2,2)$ & -- & 620\\
$(17,6)$ & 7 & $(12,5)$ & 5 & 1 & YES & YES & YES & $1.12$ & $(2,2)$ & -- & 621\\
$(17,7)$ & 6 & $(12,5)$ & 5 & 1 & YES & YES & YES & $1.00$ & $(2,2)$ & -- & 622\\
$(17,5)$ & 6 & $(13,4)$ & 6 & 1 & YES & YES & YES & $1.12$ & $(2,2)$ & -- & 623\\
$(17,5)$ & 6 & $(13,4)$ & 6 & 1 & YES & YES & YES & $1.12$ & $(2,2)$ & NO & 624\\
$(17,5)$ & 6 & $(13,5)$ & 5 & 1 & YES & YES & YES & $1.12$ & $(2,2)$ & -- & 625\\
$(17,5)$ & 6 & $(13,5)$ & 5 & 1 & YES & YES & YES & $1.00$ & $(2,2)$ & NO & 626\\
$(17,6)$ & 7 & $(13,5)$ & 5 & 1 & YES & YES & YES & $1.12$ & $(2,2)$ & -- & 627\\
$(17,7)$ & 6 & $(13,4)$ & 6 & 1 & YES & YES & YES & $1.38$ & $(2,2)$ & -- & 628\\
$(17,7)$ & 6 & $(13,5)$ & 5 & 1 & YES & YES & YES & $1.12$ & $(2,2)$ & -- & 629\\
$(17,6)$ & 7 & $(14,3)$ & 6 & 1 & YES & YES & YES & $1.00$ & $(2,2)$ & NO & 630\\
$(17,7)$ & 6 & $(14,5)$ & 6 & 1 & YES & YES & YES & $1.12$ & $(2,2)$ & -- & 631\\
$(17,6)$ & 7 & $(15,4)$ & 6 & 1 & YES & YES & YES & $1.00$ & $(2,2)$ & -- & 632\\
$(17,5)$ & 6 & $(16,7)$ & 6 & 1 & YES & YES & YES & $1.12$ & $(2,2)$ & NO & 633\\
$(17,5)$ & 6 & $(16,7)$ & 6 & 1 & YES & YES & YES & $1.12$ & $(2,2)$ & -- & 634\\
$(17,7)$ & 6 & $(16,7)$ & 6 & 1 & YES & YES & YES & $1.25$ & $(2,2)$ & -- & 635\\
$(17,5)$ & 6 & $(17,5)$ & 6 & 17 & YES & YES & NO(3) & $0.75$ & $(2,2)$ & -- & 636\\
$(17,7)$ & 6 & $(17,5)$ & 6 & 17 & YES & YES & YES & $1.00$ & $(2,2)$ & -- & 637\\
$(17,7)$ & 6 & $(17,6)$ & 7 & 17 & YES & YES & YES & $1.12$ & $(2,2)$ & 2112 & 638\\
$(18,5)$ & 6 & $(5,2)$ & 3 & 1 & YES & YES & YES & $0.75$ & $(2,2)$ & -- & 639\\
$(18,7)$ & 6 & $(8,3)$ & 4 & 2 & YES & YES & YES & $0.88$ & $(2,2)$ & -- & 640\\
$(18,7)$ & 6 & $(9,2)$ & 5 & 9 & YES & YES & YES & $1.12$ & $(2,2)$ & NO & 641\\
$(18,7)$ & 6 & $(9,2)$ & 5 & 9 & YES & YES & YES & $1.12$ & $(2,2)$ & NO & 642\\
$(18,7)$ & 6 & $(9,2)$ & 5 & 9 & YES & YES & YES & $1.12$ & $(2,2)$ & -- & 643\\
$(18,7)$ & 6 & $(10,3)$ & 5 & 2 & YES & YES & YES & $1.22$ & $(2,2)$ & -- & 644\\
$(18,7)$ & 6 & $(10,3)$ & 5 & 2 & YES & YES & YES & $1.12$ & $(2,2)$ & NO & 645\\
$(18,5)$ & 6 & $(11,4)$ & 5 & 1 & YES & YES & YES & $1.00$ & $(2,2)$ & -- & 646\\
$(18,5)$ & 6 & $(11,4)$ & 5 & 1 & YES & YES & YES & $1.12$ & $(2,2)$ & NO & 647\\
$(18,7)$ & 6 & $(11,3)$ & 5 & 1 & YES & YES & YES & $1.12$ & $(2,2)$ & NO & 648\\
$(18,7)$ & 6 & $(11,3)$ & 5 & 1 & YES & YES & YES & $1.12$ & $(2,2)$ & -- & 649\\
$(18,7)$ & 6 & $(11,4)$ & 5 & 1 & YES & YES & YES & $1.12$ & $(2,2)$ & -- & 650\\
$(18,7)$ & 6 & $(12,5)$ & 5 & 6 & YES & YES & YES & $1.00$ & $(2,2)$ & -- & 651\\
$(18,7)$ & 6 & $(12,5)$ & 5 & 6 & YES & YES & YES & $1.12$ & $(2,2)$ & NO & 652\\
$(18,5)$ & 6 & $(13,4)$ & 6 & 1 & YES & YES & YES & $1.00$ & $(2,2)$ & -- & 653\\
$(18,5)$ & 6 & $(13,4)$ & 6 & 1 & YES & YES & YES & $1.00$ & $(2,2)$ & NO & 654\\
$(18,7)$ & 6 & $(13,4)$ & 6 & 1 & YES & YES & YES & $1.25$ & $(2,2)$ & -- & 655\\
$(18,7)$ & 6 & $(13,5)$ & 5 & 1 & YES & YES & YES & $1.12$ & $(2,2)$ & -- & 656\\
$(18,7)$ & 6 & $(13,5)$ & 5 & 1 & YES & YES & YES & $1.00$ & $(2,2)$ & NO & 657\\
$(18,5)$ & 6 & $(14,5)$ & 6 & 2 & YES & YES & YES & $1.12$ & $(2,2)$ & -- & 658\\
$(18,5)$ & 6 & $(14,5)$ & 6 & 2 & YES & YES & YES & $1.00$ & $(2,2)$ & NO & 659\\
$(18,7)$ & 6 & $(14,3)$ & 6 & 2 & YES & YES & YES & $1.12$ & $(2,2)$ & -- & 660\\
$(18,7)$ & 6 & $(14,3)$ & 6 & 2 & YES & YES & YES & $1.25$ & $(2,2)$ & NO & 661\\
$(18,7)$ & 6 & $(15,4)$ & 6 & 3 & YES & YES & YES & $1.25$ & $(2,2)$ & -- & 662\\
$(18,5)$ & 6 & $(16,7)$ & 6 & 2 & YES & YES & YES & $1.12$ & $(2,2)$ & -- & 663\\
$(18,7)$ & 6 & $(16,7)$ & 6 & 2 & YES & YES & YES & $0.88$ & $(2,2)$ & NO & 664\\
$(18,5)$ & 6 & $(17,7)$ & 6 & 1 & YES & YES & YES & $1.00$ & $(2,2)$ & -- & 665\\
$(18,5)$ & 6 & $(17,7)$ & 6 & 1 & YES & YES & YES & $1.12$ & $(2,2)$ & NO & 666\\
$(18,7)$ & 6 & $(17,4)$ & 7 & 1 & YES & YES & YES & $1.12$ & $(2,2)$ & -- & 667\\
$(18,7)$ & 6 & $(17,5)$ & 6 & 1 & YES & YES & YES & $1.00$ & $(2,2)$ & -- & 668\\
$(18,5)$ & 6 & $(18,5)$ & 6 & 18 & YES & YES & YES & $0.75$ & $(2,2)$ & -- & 669\\
$(18,7)$ & 6 & $(18,5)$ & 6 & 18 & YES & YES & YES & $1.12$ & $(2,2)$ & -- & 670\\
$(18,7)$ & 6 & $(18,5)$ & 6 & 18 & YES & YES & YES & $1.12$ & $(2,2)$ & NO & 671\\
$(19,7)$ & 6 & $(5,2)$ & 3 & 1 & YES & YES & YES & $0.75$ & $(2,2)$ & -- & 672\\
$(19,8)$ & 6 & $(7,2)$ & 4 & 1 & YES & YES & NO(2) & $1.00$ & $(2,2)$ & -- & 673\\
$(19,8)$ & 6 & $(7,2)$ & 4 & 1 & YES & YES & NO(2) & $1.11$ & $(2,2)$ & NO & 674\\
$(19,8)$ & 6 & $(8,3)$ & 4 & 1 & YES & YES & YES & $1.11$ & $(2,2)$ & -- & 675\\
$(19,8)$ & 6 & $(9,2)$ & 5 & 1 & YES & YES & YES & $1.00$ & $(2,2)$ & 1138 & 676\\
$(19,8)$ & 6 & $(9,2)$ & 5 & 1 & YES & YES & YES & $1.00$ & $(2,2)$ & -- & 677\\
$(19,8)$ & 6 & $(9,2)$ & 5 & 1 & YES & YES & YES & $1.12$ & $(2,2)$ & NO & 678\\
$(19,8)$ & 6 & $(9,4)$ & 5 & 1 & YES & YES & YES & $1.00$ & $(2,2)$ & -- & 679\\
$(19,7)$ & 6 & $(10,3)$ & 5 & 1 & YES & YES & YES & $1.12$ & $(2,2)$ & -- & 680\\
$(19,7)$ & 6 & $(10,3)$ & 5 & 1 & YES & YES & YES & $1.12$ & $(2,2)$ & NO & 681\\
$(19,8)$ & 6 & $(10,3)$ & 5 & 1 & YES & YES & YES & $1.12$ & $(2,2)$ & -- & 682\\
$(19,8)$ & 6 & $(10,3)$ & 5 & 1 & YES & YES & YES & $1.12$ & $(2,2)$ & NO & 683\\
$(19,7)$ & 6 & $(11,3)$ & 5 & 1 & YES & YES & YES & $1.12$ & $(2,2)$ & -- & 684\\
$(19,7)$ & 6 & $(11,3)$ & 5 & 1 & YES & YES & YES & $1.25$ & $(2,2)$ & NO & 685\\
$(19,7)$ & 6 & $(11,4)$ & 5 & 1 & YES & YES & YES & $1.00$ & $(2,2)$ & -- & 686\\
$(19,8)$ & 6 & $(11,4)$ & 5 & 1 & YES & YES & YES & $1.12$ & $(2,2)$ & -- & 687\\
$(19,5)$ & 7 & $(12,5)$ & 5 & 1 & YES & YES & YES & $0.88$ & $(2,2)$ & -- & 688\\
$(19,7)$ & 6 & $(12,5)$ & 5 & 1 & YES & YES & YES & $1.00$ & $(2,2)$ & -- & 689\\
$(19,7)$ & 6 & $(12,5)$ & 5 & 1 & YES & YES & YES & $1.12$ & $(2,2)$ & NO & 690\\
$(19,8)$ & 6 & $(12,5)$ & 5 & 1 & YES & YES & YES & $0.88$ & $(2,2)$ & -- & 691\\
$(19,8)$ & 6 & $(12,5)$ & 5 & 1 & YES & YES & YES & $1.12$ & $(2,2)$ & NO & 692\\
$(19,7)$ & 6 & $(13,4)$ & 6 & 1 & YES & YES & YES & $1.12$ & $(2,2)$ & -- & 693\\
$(19,7)$ & 6 & $(13,5)$ & 5 & 1 & YES & YES & YES & $1.00$ & $(2,2)$ & -- & 694\\
$(19,7)$ & 6 & $(13,5)$ & 5 & 1 & YES & YES & YES & $1.00$ & $(2,2)$ & NO & 695\\
$(19,8)$ & 6 & $(13,3)$ & 6 & 1 & YES & YES & YES & $1.00$ & $(2,2)$ & NO & 696\\
$(19,8)$ & 6 & $(13,3)$ & 6 & 1 & YES & YES & YES & $1.00$ & $(2,2)$ & -- & 697\\
$(19,8)$ & 6 & $(13,5)$ & 5 & 1 & YES & YES & YES & $0.88$ & $(2,2)$ & -- & 698\\
$(19,5)$ & 7 & $(14,5)$ & 6 & 1 & YES & YES & YES & $1.00$ & $(2,2)$ & -- & 699\\
$(19,6)$ & 8 & $(14,3)$ & 6 & 1 & YES & YES & YES & $1.12$ & $(2,2)$ & -- & 700\\
$(19,6)$ & 8 & $(14,3)$ & 6 & 1 & YES & YES & YES & $1.12$ & $(2,2)$ & NO & 701\\
$(19,7)$ & 6 & $(14,5)$ & 6 & 1 & YES & YES & YES & $1.12$ & $(2,2)$ & -- & 702\\
$(19,8)$ & 6 & $(14,5)$ & 6 & 1 & YES & YES & YES & $1.00$ & $(2,2)$ & NO & 703\\
$(19,7)$ & 6 & $(15,4)$ & 6 & 1 & YES & YES & YES & $1.12$ & $(2,2)$ & -- & 704\\
$(19,7)$ & 6 & $(15,4)$ & 6 & 1 & YES & YES & YES & $1.25$ & $(2,2)$ & NO & 705\\
$(19,5)$ & 7 & $(16,5)$ & 7 & 1 & YES & YES & YES & $1.12$ & $(2,2)$ & -- & 706\\
$(19,5)$ & 7 & $(16,7)$ & 6 & 1 & YES & YES & YES & $1.00$ & $(2,2)$ & -- & 707\\
$(19,6)$ & 8 & $(16,3)$ & 7 & 1 & YES & YES & YES & $1.00$ & $(2,2)$ & -- & 708\\
$(19,6)$ & 8 & $(16,3)$ & 7 & 1 & YES & YES & YES & $1.12$ & $(2,2)$ & NO & 709\\
$(19,7)$ & 6 & $(16,7)$ & 6 & 1 & YES & YES & YES & $1.25$ & $(2,2)$ & -- & 710\\
$(19,7)$ & 6 & $(16,7)$ & 6 & 1 & YES & YES & YES & $1.12$ & $(2,2)$ & NO & 711\\
$(19,8)$ & 6 & $(16,7)$ & 6 & 1 & YES & YES & YES & $1.12$ & $(2,2)$ & -- & 712\\
$(19,4)$ & 7 & $(17,6)$ & 7 & 1 & YES & YES & YES & $1.25$ & $(2,2)$ & NO & 713\\
$(19,6)$ & 8 & $(17,3)$ & 7 & 1 & YES & YES & YES & $1.12$ & $(2,2)$ & NO & 714\\
$(19,7)$ & 6 & $(17,4)$ & 7 & 1 & YES & YES & YES & $1.12$ & $(2,2)$ & -- & 715\\
$(19,7)$ & 6 & $(17,5)$ & 6 & 1 & YES & YES & NO(2) & $1.12$ & $(4,1)$ & -- & 716\\
$(19,7)$ & 6 & $(17,5)$ & 6 & 1 & YES & YES & YES & $1.12$ & $(2,2)$ & NO & 717\\
$(19,4)$ & 7 & $(18,7)$ & 6 & 1 & YES & YES & YES & $1.12$ & $(2,2)$ & -- & 718\\
$(19,6)$ & 8 & $(18,5)$ & 6 & 1 & YES & YES & YES & $1.12$ & $(2,2)$ & NO & 719\\
$(19,8)$ & 6 & $(18,5)$ & 6 & 1 & YES & YES & YES & $1.12$ & $(2,2)$ & NO & 720\\
$(19,8)$ & 6 & $(18,7)$ & 6 & 1 & YES & YES & YES & $1.00$ & $(2,2)$ & NO & 721\\
$(19,5)$ & 7 & $(19,5)$ & 7 & 19 & YES & YES & YES & $1.00$ & $(2,2)$ & -- & 722\\
$(19,6)$ & 8 & $(19,4)$ & 7 & 19 & YES & YES & YES & $1.12$ & $(2,2)$ & NO & 723\\
$(19,8)$ & 6 & $(19,5)$ & 7 & 19 & YES & YES & YES & $1.12$ & $(2,2)$ & NO & 724\\
$(20,9)$ & 7 & $(11,3)$ & 5 & 1 & YES & YES & YES & $1.00$ & $(2,2)$ & -- & 725\\
$(20,7)$ & 8 & $(13,3)$ & 6 & 1 & YES & YES & YES & $1.12$ & $(2,2)$ & -- & 726\\
$(20,7)$ & 8 & $(14,3)$ & 6 & 2 & YES & YES & YES & $1.00$ & $(2,2)$ & -- & 727\\
$(20,9)$ & 7 & $(17,7)$ & 6 & 1 & YES & YES & YES & $1.00$ & $(2,2)$ & 1066 & 728\\
$(20,7)$ & 8 & $(19,3)$ & 8 & 1 & YES & YES & YES & $1.12$ & $(2,2)$ & -- & 729\\
$(20,9)$ & 7 & $(19,8)$ & 6 & 1 & YES & YES & YES & $1.00$ & $(2,2)$ & NO & 730\\
$(20,7)$ & 8 & $(20,3)$ & 8 & 20 & YES & YES & YES & $1.25$ & $(2,2)$ & -- & 731\\
$(21,8)$ & 6 & $(7,2)$ & 4 & 7 & YES & YES & YES & $1.00$ & $(2,2)$ & -- & 732\\
$(21,8)$ & 6 & $(7,3)$ & 4 & 7 & YES & YES & NO(2) & $1.33$ & $(2,2)$ & -- & 733\\
$(21,5)$ & 8 & $(8,3)$ & 4 & 1 & YES & YES & YES & $1.25$ & $(2,2)$ & NO & 734\\
$(21,5)$ & 8 & $(8,3)$ & 4 & 1 & YES & YES & YES & $1.25$ & $(2,2)$ & -- & 735\\
$(21,8)$ & 6 & $(8,3)$ & 4 & 1 & YES & YES & NO(2) & $1.22$ & $(2,2)$ & NO & 736\\
$(21,8)$ & 6 & $(8,3)$ & 4 & 1 & YES & YES & NO(2) & $1.22$ & $(2,2)$ & -- & 737\\
$(21,8)$ & 6 & $(9,2)$ & 5 & 3 & YES & YES & YES & $1.00$ & $(2,2)$ & NO & 738\\
$(21,8)$ & 6 & $(9,2)$ & 5 & 3 & YES & YES & YES & $1.00$ & $(2,2)$ & -- & 739\\
$(21,8)$ & 6 & $(9,4)$ & 5 & 3 & YES & YES & YES & $1.00$ & $(2,2)$ & -- & 740\\
$(21,8)$ & 6 & $(9,4)$ & 5 & 3 & YES & YES & YES & $1.12$ & $(2,2)$ & NO & 741\\
$(21,8)$ & 6 & $(10,3)$ & 5 & 1 & YES & YES & YES & $1.12$ & $(2,2)$ & -- & 742\\
$(21,8)$ & 6 & $(10,3)$ & 5 & 1 & YES & YES & YES & $1.12$ & $(2,2)$ & NO & 743\\
$(21,8)$ & 6 & $(10,3)$ & 5 & 1 & YES & YES & YES & $1.12$ & $(2,2)$ & NO & 744\\
$(21,8)$ & 6 & $(11,4)$ & 5 & 1 & YES & YES & YES & $1.00$ & $(2,2)$ & -- & 745\\
$(21,8)$ & 6 & $(11,5)$ & 6 & 1 & YES & YES & YES & $1.12$ & $(2,2)$ & -- & 746\\
$(21,8)$ & 6 & $(12,5)$ & 5 & 3 & YES & YES & YES & $1.00$ & $(2,2)$ & -- & 747\\
$(21,8)$ & 6 & $(12,5)$ & 5 & 3 & YES & YES & YES & $1.00$ & $(2,2)$ & NO & 748\\
$(21,5)$ & 8 & $(13,3)$ & 6 & 1 & YES & YES & YES & $1.25$ & $(2,2)$ & -- & 749\\
$(21,5)$ & 8 & $(13,3)$ & 6 & 1 & YES & YES & YES & $1.38$ & $(2,2)$ & NO & 750\\
$(21,5)$ & 8 & $(13,4)$ & 6 & 1 & YES & YES & YES & $1.00$ & $(2,2)$ & -- & 751\\
$(21,8)$ & 6 & $(13,3)$ & 6 & 1 & YES & YES & YES & $1.00$ & $(2,2)$ & -- & 752\\
$(21,8)$ & 6 & $(13,3)$ & 6 & 1 & YES & YES & YES & $1.12$ & $(2,2)$ & NO & 753\\
$(21,8)$ & 6 & $(13,4)$ & 6 & 1 & YES & YES & YES & $1.12$ & $(2,2)$ & -- & 754\\
$(21,8)$ & 6 & $(13,5)$ & 5 & 1 & YES & YES & YES & $1.00$ & $(2,2)$ & -- & 755\\
$(21,5)$ & 8 & $(14,3)$ & 6 & 7 & YES & YES & YES & $1.12$ & $(2,2)$ & -- & 756\\
$(21,5)$ & 8 & $(14,3)$ & 6 & 7 & YES & YES & YES & $1.25$ & $(2,2)$ & NO & 757\\
$(21,5)$ & 8 & $(14,5)$ & 6 & 7 & YES & YES & YES & $1.25$ & $(2,2)$ & -- & 758\\
$(21,5)$ & 8 & $(15,4)$ & 6 & 3 & YES & YES & YES & $1.00$ & $(2,2)$ & -- & 759\\
$(21,8)$ & 6 & $(16,7)$ & 6 & 1 & YES & YES & YES & $1.00$ & $(2,2)$ & NO & 760\\
$(21,8)$ & 6 & $(16,7)$ & 6 & 1 & YES & YES & YES & $1.00$ & $(2,2)$ & -- & 761\\
$(21,5)$ & 8 & $(17,5)$ & 6 & 1 & YES & YES & YES & $1.12$ & $(2,2)$ & -- & 762\\
$(21,8)$ & 6 & $(17,4)$ & 7 & 1 & YES & YES & YES & $1.12$ & $(2,2)$ & -- & 763\\
$(21,8)$ & 6 & $(18,5)$ & 6 & 3 & YES & YES & YES & $1.00$ & $(2,2)$ & NO & 764\\
$(21,8)$ & 6 & $(18,5)$ & 6 & 3 & YES & YES & YES & $1.00$ & $(2,2)$ & -- & 765\\
$(21,5)$ & 8 & $(19,5)$ & 7 & 1 & YES & YES & YES & $1.25$ & $(2,2)$ & -- & 766\\
$(21,5)$ & 8 & $(19,7)$ & 6 & 1 & YES & YES & YES & $1.12$ & $(2,2)$ & -- & 767\\
$(21,8)$ & 6 & $(19,8)$ & 6 & 1 & YES & YES & YES & $1.00$ & $(2,2)$ & NO & 768\\
$(21,5)$ & 8 & $(21,5)$ & 8 & 21 & YES & YES & YES & $1.25$ & $(2,2)$ & -- & 769\\
$(22,5)$ & 7 & $(5,2)$ & 3 & 1 & YES & YES & YES & $1.00$ & $(2,2)$ & NO & 770\\
$(22,5)$ & 7 & $(5,2)$ & 3 & 1 & YES & YES & YES & $1.00$ & $(2,2)$ & -- & 771\\
$(22,9)$ & 7 & $(8,3)$ & 4 & 2 & YES & YES & YES & $1.00$ & $(2,2)$ & -- & 772\\
$(22,9)$ & 7 & $(8,3)$ & 4 & 2 & YES & YES & YES & $1.12$ & $(2,2)$ & NO & 773\\
$(22,5)$ & 7 & $(10,3)$ & 5 & 2 & YES & YES & YES & $1.00$ & $(2,2)$ & NO & 774\\
$(22,5)$ & 7 & $(10,3)$ & 5 & 2 & YES & YES & YES & $1.00$ & $(2,2)$ & -- & 775\\
$(22,9)$ & 7 & $(10,3)$ & 5 & 2 & YES & YES & YES & $1.12$ & $(2,2)$ & -- & 776\\
$(22,5)$ & 7 & $(11,4)$ & 5 & 11 & YES & YES & YES & $1.12$ & $(2,2)$ & NO & 777\\
$(22,5)$ & 7 & $(11,4)$ & 5 & 11 & YES & YES & YES & $1.12$ & $(2,2)$ & -- & 778\\
$(22,5)$ & 7 & $(11,5)$ & 6 & 11 & YES & YES & YES & $1.00$ & $(2,2)$ & NO & 779\\
$(22,9)$ & 7 & $(11,3)$ & 5 & 11 & YES & YES & YES & $1.00$ & $(2,2)$ & -- & 780\\
$(22,5)$ & 7 & $(13,4)$ & 6 & 1 & YES & YES & YES & $1.12$ & $(2,2)$ & NO & 781\\
$(22,5)$ & 7 & $(13,4)$ & 6 & 1 & YES & YES & YES & $1.12$ & $(2,2)$ & -- & 782\\
$(22,9)$ & 7 & $(13,3)$ & 6 & 1 & YES & YES & YES & $1.12$ & $(2,2)$ & NO & 783\\
$(22,5)$ & 7 & $(14,5)$ & 6 & 2 & YES & YES & YES & $1.12$ & $(2,2)$ & NO & 784\\
$(22,9)$ & 7 & $(14,3)$ & 6 & 2 & YES & YES & YES & $1.00$ & $(2,2)$ & NO & 785\\
$(22,9)$ & 7 & $(16,3)$ & 7 & 2 & YES & YES & YES & $1.00$ & $(2,2)$ & -- & 786\\
$(22,9)$ & 7 & $(16,3)$ & 7 & 2 & YES & YES & YES & $1.00$ & $(2,2)$ & NO & 787\\
$(22,9)$ & 7 & $(16,7)$ & 6 & 2 & YES & YES & YES & $1.00$ & $(2,2)$ & 2021 & 788\\
$(22,5)$ & 7 & $(17,7)$ & 6 & 1 & YES & YES & YES & $1.00$ & $(2,2)$ & -- & 789\\
$(22,5)$ & 7 & $(19,5)$ & 7 & 1 & YES & YES & YES & $1.00$ & $(2,2)$ & -- & 790\\
$(22,5)$ & 7 & $(19,6)$ & 8 & 1 & YES & YES & YES & $1.12$ & $(2,2)$ & NO & 791\\
$(22,5)$ & 7 & $(19,8)$ & 6 & 1 & YES & YES & YES & $0.88$ & $(2,2)$ & -- & 792\\
$(22,5)$ & 7 & $(19,8)$ & 6 & 1 & YES & YES & YES & $1.00$ & $(2,2)$ & NO & 793\\
$(22,9)$ & 7 & $(21,8)$ & 6 & 1 & YES & YES & YES & $1.12$ & $(2,2)$ & NO & 794\\
$(22,9)$ & 7 & $(22,5)$ & 7 & 22 & YES & YES & YES & $1.12$ & $(2,2)$ & -- & 795\\
$(23,5)$ & 7 & $(5,2)$ & 3 & 1 & YES & YES & YES & $1.11$ & $(2,2)$ & NO & 796\\
$(23,5)$ & 7 & $(5,2)$ & 3 & 1 & YES & YES & YES & $1.11$ & $(2,2)$ & -- & 797\\
$(23,5)$ & 7 & $(5,2)$ & 3 & 1 & YES & YES & YES & $1.12$ & $(2,2)$ & NO & 798\\
$(23,5)$ & 7 & $(7,2)$ & 4 & 1 & YES & YES & YES & $1.00$ & $(2,2)$ & NO & 799\\
$(23,5)$ & 7 & $(7,2)$ & 4 & 1 & YES & YES & YES & $1.00$ & $(2,2)$ & -- & 800\\
$(23,5)$ & 7 & $(7,2)$ & 4 & 1 & YES & YES & YES & $1.00$ & $(2,2)$ & NO & 801\\
$(23,7)$ & 7 & $(7,2)$ & 4 & 1 & YES & YES & NO(2) & $1.22$ & $(2,2)$ & NO & 802\\
$(23,7)$ & 7 & $(7,2)$ & 4 & 1 & YES & YES & NO(2) & $1.22$ & $(2,2)$ & -- & 803\\
$(23,9)$ & 7 & $(7,2)$ & 4 & 1 & YES & YES & YES & $1.11$ & $(2,2)$ & -- & 804\\
$(23,9)$ & 7 & $(7,2)$ & 4 & 1 & YES & YES & YES & $1.12$ & $(2,2)$ & NO & 805\\
$(23,10)$ & 7 & $(7,2)$ & 4 & 1 & YES & YES & YES & $1.12$ & $(2,2)$ & NO & 806\\
$(23,10)$ & 7 & $(7,2)$ & 4 & 1 & YES & YES & YES & $1.12$ & $(2,2)$ & -- & 807\\
$(23,7)$ & 7 & $(8,3)$ & 4 & 1 & YES & YES & YES & $1.11$ & $(2,2)$ & NO & 808\\
$(23,7)$ & 7 & $(8,3)$ & 4 & 1 & YES & YES & YES & $1.11$ & $(2,2)$ & -- & 809\\
$(23,9)$ & 7 & $(8,3)$ & 4 & 1 & YES & YES & YES & $0.88$ & $(2,2)$ & -- & 810\\
$(23,9)$ & 7 & $(8,3)$ & 4 & 1 & YES & YES & YES & $1.00$ & $(2,2)$ & NO & 811\\
$(23,10)$ & 7 & $(8,3)$ & 4 & 1 & YES & YES & YES & $1.12$ & $(2,2)$ & -- & 812\\
$(23,7)$ & 7 & $(9,4)$ & 5 & 1 & YES & YES & YES & $1.12$ & $(2,2)$ & -- & 813\\
$(23,10)$ & 7 & $(9,2)$ & 5 & 1 & YES & YES & YES & $1.12$ & $(2,2)$ & -- & 814\\
$(23,10)$ & 7 & $(9,2)$ & 5 & 1 & YES & YES & YES & $1.38$ & $(2,2)$ & NO & 815\\
$(23,10)$ & 7 & $(9,4)$ & 5 & 1 & YES & YES & YES & $1.12$ & $(2,2)$ & -- & 816\\
$(23,7)$ & 7 & $(10,3)$ & 5 & 1 & YES & YES & YES & $1.00$ & $(2,2)$ & -- & 817\\
$(23,7)$ & 7 & $(10,3)$ & 5 & 1 & YES & YES & YES & $1.12$ & $(2,2)$ & NO & 818\\
$(23,9)$ & 7 & $(10,3)$ & 5 & 1 & YES & YES & YES & $1.25$ & $(2,2)$ & -- & 819\\
$(23,10)$ & 7 & $(10,3)$ & 5 & 1 & YES & YES & YES & $1.12$ & $(2,2)$ & -- & 820\\
$(23,5)$ & 7 & $(11,5)$ & 6 & 1 & YES & YES & YES & $1.00$ & $(2,2)$ & NO & 821\\
$(23,6)$ & 8 & $(11,4)$ & 5 & 1 & YES & YES & YES & $1.25$ & $(2,2)$ & -- & 822\\
$(23,7)$ & 7 & $(11,3)$ & 5 & 1 & YES & YES & NO(2) & $1.22$ & $(2,2)$ & NO & 823\\
$(23,7)$ & 7 & $(11,3)$ & 5 & 1 & YES & YES & YES & $1.00$ & $(2,2)$ & -- & 824\\
$(23,7)$ & 7 & $(11,3)$ & 5 & 1 & YES & YES & YES & $1.12$ & $(2,2)$ & NO & 825\\
$(23,7)$ & 7 & $(11,4)$ & 5 & 1 & YES & YES & YES & $1.12$ & $(2,2)$ & -- & 826\\
$(23,9)$ & 7 & $(11,2)$ & 6 & 1 & YES & YES & YES & $1.12$ & $(2,2)$ & NO & 827\\
$(23,9)$ & 7 & $(11,2)$ & 6 & 1 & YES & YES & YES & $1.12$ & $(2,2)$ & -- & 828\\
$(23,9)$ & 7 & $(11,3)$ & 5 & 1 & YES & YES & YES & $1.00$ & $(2,2)$ & -- & 829\\
$(23,9)$ & 7 & $(11,3)$ & 5 & 1 & YES & YES & YES & $1.00$ & $(2,2)$ & NO & 830\\
$(23,10)$ & 7 & $(11,4)$ & 5 & 1 & YES & YES & YES & $1.12$ & $(2,2)$ & -- & 831\\
$(23,6)$ & 8 & $(12,5)$ & 5 & 1 & YES & YES & YES & $1.12$ & $(2,2)$ & NO & 832\\
$(23,6)$ & 8 & $(12,5)$ & 5 & 1 & YES & YES & YES & $1.12$ & $(2,2)$ & -- & 833\\
$(23,7)$ & 7 & $(12,5)$ & 5 & 1 & YES & YES & YES & $1.12$ & $(2,2)$ & -- & 834\\
$(23,10)$ & 7 & $(12,5)$ & 5 & 1 & YES & YES & YES & $1.12$ & $(2,2)$ & -- & 835\\
$(23,6)$ & 8 & $(13,5)$ & 5 & 1 & YES & YES & YES & $1.12$ & $(2,2)$ & NO & 836\\
$(23,7)$ & 7 & $(13,4)$ & 6 & 1 & YES & YES & YES & $1.25$ & $(2,2)$ & -- & 837\\
$(23,7)$ & 7 & $(13,5)$ & 5 & 1 & YES & YES & YES & $1.25$ & $(2,2)$ & -- & 838\\
$(23,7)$ & 7 & $(13,5)$ & 5 & 1 & YES & YES & YES & $1.12$ & $(2,2)$ & NO & 839\\
$(23,9)$ & 7 & $(13,3)$ & 6 & 1 & YES & YES & YES & $1.12$ & $(2,2)$ & NO & 840\\
$(23,9)$ & 7 & $(13,3)$ & 6 & 1 & YES & YES & YES & $1.25$ & $(2,2)$ & -- & 841\\
$(23,10)$ & 7 & $(13,3)$ & 6 & 1 & YES & YES & YES & $1.00$ & $(2,2)$ & NO & 842\\
$(23,10)$ & 7 & $(13,4)$ & 6 & 1 & YES & YES & YES & $1.25$ & $(2,2)$ & NO & 843\\
$(23,10)$ & 7 & $(13,4)$ & 6 & 1 & YES & YES & YES & $1.25$ & $(2,2)$ & -- & 844\\
$(23,5)$ & 7 & $(14,5)$ & 6 & 1 & YES & YES & YES & $1.12$ & $(2,2)$ & NO & 845\\
$(23,7)$ & 7 & $(14,5)$ & 6 & 1 & YES & YES & YES & $1.25$ & $(2,2)$ & -- & 846\\
$(23,9)$ & 7 & $(14,3)$ & 6 & 1 & YES & YES & YES & $1.00$ & $(2,2)$ & NO & 847\\
$(23,9)$ & 7 & $(14,3)$ & 6 & 1 & YES & YES & YES & $1.25$ & $(2,2)$ & -- & 848\\
$(23,10)$ & 7 & $(14,5)$ & 6 & 1 & YES & YES & YES & $1.12$ & $(2,2)$ & NO & 849\\
$(23,7)$ & 7 & $(15,4)$ & 6 & 1 & YES & YES & YES & $1.00$ & $(2,2)$ & -- & 850\\
$(23,10)$ & 7 & $(15,4)$ & 6 & 1 & YES & YES & YES & $1.25$ & $(2,2)$ & 1365 & 851\\
$(23,10)$ & 7 & $(15,4)$ & 6 & 1 & YES & YES & YES & $1.25$ & $(2,2)$ & -- & 852\\
$(23,7)$ & 7 & $(16,7)$ & 6 & 1 & YES & YES & YES & $1.12$ & $(2,2)$ & -- & 853\\
$(23,9)$ & 7 & $(16,3)$ & 7 & 1 & YES & YES & YES & $1.12$ & $(2,2)$ & -- & 854\\
$(23,9)$ & 7 & $(16,3)$ & 7 & 1 & YES & YES & YES & $1.12$ & $(2,2)$ & NO & 855\\
$(23,9)$ & 7 & $(16,3)$ & 7 & 1 & YES & YES & YES & $1.25$ & $(2,2)$ & NO & 856\\
$(23,9)$ & 7 & $(16,7)$ & 6 & 1 & YES & YES & YES & $1.00$ & $(2,2)$ & NO & 857\\
$(23,5)$ & 7 & $(17,7)$ & 6 & 1 & YES & YES & YES & $1.00$ & $(2,2)$ & -- & 858\\
$(23,5)$ & 7 & $(17,7)$ & 6 & 1 & YES & YES & YES & $1.12$ & $(2,2)$ & NO & 859\\
$(23,6)$ & 8 & $(17,3)$ & 7 & 1 & YES & YES & YES & $1.00$ & $(2,2)$ & NO & 860\\
$(23,7)$ & 7 & $(17,4)$ & 7 & 1 & YES & YES & YES & $1.12$ & $(2,2)$ & -- & 861\\
$(23,7)$ & 7 & $(17,5)$ & 6 & 1 & YES & YES & YES & $1.12$ & $(2,2)$ & -- & 862\\
$(23,5)$ & 7 & $(18,7)$ & 6 & 1 & YES & YES & YES & $1.00$ & $(2,2)$ & -- & 863\\
$(23,5)$ & 7 & $(19,5)$ & 7 & 1 & YES & YES & YES & $1.00$ & $(2,2)$ & NO & 864\\
$(23,5)$ & 7 & $(19,5)$ & 7 & 1 & YES & YES & YES & $0.88$ & $(2,2)$ & -- & 865\\
$(23,6)$ & 8 & $(19,3)$ & 8 & 1 & YES & YES & YES & $1.00$ & $(2,2)$ & NO & 866\\
$(23,7)$ & 7 & $(21,5)$ & 8 & 1 & YES & YES & YES & $1.12$ & $(2,2)$ & -- & 867\\
$(24,7)$ & 7 & $(4,1)$ & 3 & 4 & YES & YES & YES & $1.12$ & $(2,2)$ & NO & 868\\
$(24,7)$ & 7 & $(8,3)$ & 4 & 8 & YES & YES & YES & $1.12$ & $(2,2)$ & NO & 869\\
$(24,11)$ & 8 & $(8,3)$ & 4 & 8 & YES & YES & YES & $1.38$ & $(2,2)$ & -- & 870\\
$(24,7)$ & 7 & $(9,2)$ & 5 & 3 & YES & YES & YES & $1.12$ & $(2,2)$ & NO & 871\\
$(24,7)$ & 7 & $(9,2)$ & 5 & 3 & YES & YES & YES & $1.12$ & $(2,2)$ & -- & 872\\
$(24,7)$ & 7 & $(9,4)$ & 5 & 3 & YES & YES & YES & $1.12$ & $(2,2)$ & NO & 873\\
$(24,7)$ & 7 & $(9,4)$ & 5 & 3 & YES & YES & YES & $1.12$ & $(2,2)$ & -- & 874\\
$(24,7)$ & 7 & $(11,4)$ & 5 & 1 & YES & YES & YES & $1.12$ & $(2,2)$ & -- & 875\\
$(24,7)$ & 7 & $(11,4)$ & 5 & 1 & YES & YES & YES & $1.12$ & $(2,2)$ & NO & 876\\
$(24,7)$ & 7 & $(11,5)$ & 6 & 1 & YES & YES & YES & $1.12$ & $(2,2)$ & NO & 877\\
$(24,7)$ & 7 & $(12,5)$ & 5 & 12 & YES & YES & YES & $1.12$ & $(2,2)$ & -- & 878\\
$(24,7)$ & 7 & $(12,5)$ & 5 & 12 & YES & YES & YES & $1.12$ & $(2,2)$ & NO & 879\\
$(24,5)$ & 8 & $(14,5)$ & 6 & 2 & YES & YES & YES & $1.25$ & $(2,2)$ & -- & 880\\
$(24,7)$ & 7 & $(14,3)$ & 6 & 2 & YES & YES & YES & $1.12$ & $(2,2)$ & -- & 881\\
$(24,7)$ & 7 & $(14,3)$ & 6 & 2 & YES & YES & YES & $1.25$ & $(2,2)$ & NO & 882\\
$(24,7)$ & 7 & $(16,7)$ & 6 & 8 & YES & YES & YES & $1.25$ & $(2,2)$ & NO & 883\\
$(24,5)$ & 8 & $(19,5)$ & 7 & 1 & YES & YES & YES & $1.25$ & $(2,2)$ & -- & 884\\
$(24,7)$ & 7 & $(19,4)$ & 7 & 1 & YES & YES & YES & $1.00$ & $(2,2)$ & -- & 885\\
$(24,7)$ & 7 & $(19,6)$ & 8 & 1 & YES & YES & YES & $1.12$ & $(2,2)$ & NO & 886\\
$(24,5)$ & 8 & $(21,5)$ & 8 & 3 & YES & YES & YES & $1.12$ & $(2,2)$ & -- & 887\\
$(24,5)$ & 8 & $(24,5)$ & 8 & 24 & YES & YES & YES & $1.00$ & $(2,2)$ & -- & 888\\
$(25,7)$ & 7 & $(4,1)$ & 3 & 1 & YES & YES & YES & $1.22$ & $(2,2)$ & NO & 889\\
$(25,9)$ & 7 & $(5,2)$ & 3 & 5 & YES & YES & YES & $0.75$ & $(2,2)$ & -- & 890\\
$(25,7)$ & 7 & $(7,3)$ & 4 & 1 & YES & YES & YES & $1.12$ & $(2,2)$ & NO & 891\\
$(25,7)$ & 7 & $(7,3)$ & 4 & 1 & YES & YES & YES & $1.12$ & $(2,2)$ & -- & 892\\
$(25,11)$ & 7 & $(7,2)$ & 4 & 1 & YES & YES & YES & $1.00$ & $(2,2)$ & -- & 893\\
$(25,7)$ & 7 & $(8,3)$ & 4 & 1 & YES & YES & NO(3) & $0.75$ & $(2,2)$ & -- & 894\\
$(25,11)$ & 7 & $(8,3)$ & 4 & 1 & YES & YES & YES & $1.12$ & $(2,2)$ & -- & 895\\
$(25,11)$ & 7 & $(8,3)$ & 4 & 1 & YES & YES & YES & $1.00$ & $(2,2)$ & NO & 896\\
$(25,7)$ & 7 & $(9,4)$ & 5 & 1 & YES & YES & YES & $1.12$ & $(2,2)$ & NO & 897\\
$(25,7)$ & 7 & $(9,4)$ & 5 & 1 & YES & YES & YES & $1.12$ & $(2,2)$ & -- & 898\\
$(25,9)$ & 7 & $(9,2)$ & 5 & 1 & YES & YES & YES & $1.12$ & $(2,2)$ & NO & 899\\
$(25,9)$ & 7 & $(9,4)$ & 5 & 1 & YES & YES & YES & $1.12$ & $(2,2)$ & -- & 900\\
$(25,11)$ & 7 & $(10,3)$ & 5 & 5 & YES & YES & YES & $1.25$ & $(2,2)$ & NO & 901\\
$(25,11)$ & 7 & $(10,3)$ & 5 & 5 & YES & YES & YES & $1.25$ & $(2,2)$ & -- & 902\\
$(25,9)$ & 7 & $(11,3)$ & 5 & 1 & YES & YES & YES & $0.88$ & $(2,2)$ & -- & 903\\
$(25,9)$ & 7 & $(11,4)$ & 5 & 1 & YES & YES & YES & $1.12$ & $(2,2)$ & -- & 904\\
$(25,11)$ & 7 & $(11,3)$ & 5 & 1 & YES & YES & YES & $1.00$ & $(2,2)$ & NO & 905\\
$(25,9)$ & 7 & $(12,5)$ & 5 & 1 & YES & YES & YES & $1.38$ & $(2,2)$ & -- & 906\\
$(25,9)$ & 7 & $(12,5)$ & 5 & 1 & YES & YES & YES & $1.12$ & $(2,2)$ & NO & 907\\
$(25,7)$ & 7 & $(13,4)$ & 6 & 1 & YES & YES & YES & $1.12$ & $(2,2)$ & -- & 908\\
$(25,9)$ & 7 & $(13,5)$ & 5 & 1 & YES & YES & YES & $1.12$ & $(2,2)$ & -- & 909\\
$(25,11)$ & 7 & $(13,3)$ & 6 & 1 & YES & YES & YES & $1.00$ & $(2,2)$ & NO & 910\\
$(25,11)$ & 7 & $(13,5)$ & 5 & 1 & YES & YES & YES & $1.00$ & $(2,2)$ & NO & 911\\
$(25,11)$ & 7 & $(13,5)$ & 5 & 1 & YES & YES & YES & $1.12$ & $(2,2)$ & -- & 912\\
$(25,7)$ & 7 & $(14,5)$ & 6 & 1 & YES & YES & YES & $1.12$ & $(2,2)$ & NO & 913\\
$(25,11)$ & 7 & $(14,5)$ & 6 & 1 & YES & YES & YES & $1.12$ & $(2,2)$ & NO & 914\\
$(25,7)$ & 7 & $(15,4)$ & 6 & 5 & YES & YES & YES & $1.12$ & $(2,2)$ & 1570 & 915\\
$(25,11)$ & 7 & $(15,4)$ & 6 & 5 & YES & YES & YES & $1.00$ & $(2,2)$ & -- & 916\\
$(25,4)$ & 9 & $(17,6)$ & 7 & 1 & YES & YES & YES & $1.12$ & $(2,2)$ & -- & 917\\
$(25,7)$ & 7 & $(17,4)$ & 7 & 1 & YES & YES & YES & $1.12$ & $(2,2)$ & -- & 918\\
$(25,11)$ & 7 & $(17,4)$ & 7 & 1 & YES & YES & YES & $1.25$ & $(2,2)$ & NO & 919\\
$(25,11)$ & 7 & $(17,4)$ & 7 & 1 & YES & YES & YES & $1.25$ & $(2,2)$ & -- & 920\\
$(25,11)$ & 7 & $(17,7)$ & 6 & 1 & YES & YES & YES & $1.00$ & $(2,2)$ & NO & 921\\
$(25,6)$ & 9 & $(19,3)$ & 8 & 1 & YES & YES & YES & $1.00$ & $(2,2)$ & NO & 922\\
$(25,7)$ & 7 & $(19,4)$ & 7 & 1 & YES & YES & YES & $1.12$ & $(2,2)$ & -- & 923\\
$(25,7)$ & 7 & $(19,6)$ & 8 & 1 & YES & YES & YES & $1.12$ & $(2,2)$ & 2438 & 924\\
$(25,9)$ & 7 & $(19,8)$ & 6 & 1 & YES & YES & YES & $1.38$ & $(2,2)$ & NO & 925\\
$(25,11)$ & 7 & $(21,5)$ & 8 & 1 & YES & YES & YES & $1.25$ & $(2,2)$ & -- & 926\\
$(26,11)$ & 7 & $(4,1)$ & 3 & 2 & YES & YES & YES & $1.12$ & $(2,2)$ & NO & 927\\
$(26,7)$ & 7 & $(5,2)$ & 3 & 1 & YES & YES & YES & $1.00$ & $(2,2)$ & -- & 928\\
$(26,11)$ & 7 & $(5,2)$ & 3 & 1 & YES & YES & YES & $1.12$ & $(2,2)$ & -- & 929\\
$(26,7)$ & 7 & $(7,2)$ & 4 & 1 & YES & YES & YES & $1.00$ & $(2,2)$ & NO & 930\\
$(26,7)$ & 7 & $(7,2)$ & 4 & 1 & YES & YES & YES & $1.00$ & $(2,2)$ & -- & 931\\
$(26,11)$ & 7 & $(7,2)$ & 4 & 1 & YES & YES & YES & $1.00$ & $(2,2)$ & -- & 932\\
$(26,11)$ & 7 & $(7,2)$ & 4 & 1 & YES & YES & YES & $1.12$ & $(2,2)$ & NO & 933\\
$(26,11)$ & 7 & $(7,2)$ & 4 & 1 & YES & YES & YES & $1.00$ & $(2,2)$ & NO & 934\\
$(26,11)$ & 7 & $(7,3)$ & 4 & 1 & YES & YES & YES & $1.12$ & $(2,2)$ & -- & 935\\
$(26,11)$ & 7 & $(8,3)$ & 4 & 2 & YES & YES & YES & $1.00$ & $(2,2)$ & -- & 936\\
$(26,11)$ & 7 & $(8,3)$ & 4 & 2 & YES & YES & YES & $1.00$ & $(2,2)$ & NO & 937\\
$(26,11)$ & 7 & $(9,4)$ & 5 & 1 & YES & YES & YES & $1.12$ & $(2,2)$ & -- & 938\\
$(26,11)$ & 7 & $(10,3)$ & 5 & 2 & YES & YES & YES & $1.12$ & $(2,2)$ & -- & 939\\
$(26,11)$ & 7 & $(10,3)$ & 5 & 2 & YES & YES & YES & $1.25$ & $(2,2)$ & NO & 940\\
$(26,11)$ & 7 & $(10,3)$ & 5 & 2 & YES & YES & YES & $1.12$ & $(2,2)$ & NO & 941\\
$(26,7)$ & 7 & $(11,4)$ & 5 & 1 & YES & YES & YES & $1.12$ & $(2,2)$ & -- & 942\\
$(26,11)$ & 7 & $(11,3)$ & 5 & 1 & YES & YES & YES & $1.12$ & $(2,2)$ & NO & 943\\
$(26,11)$ & 7 & $(11,3)$ & 5 & 1 & YES & YES & YES & $1.00$ & $(2,2)$ & NO & 944\\
$(26,11)$ & 7 & $(11,3)$ & 5 & 1 & YES & YES & YES & $1.00$ & $(2,2)$ & -- & 945\\
$(26,11)$ & 7 & $(11,4)$ & 5 & 1 & YES & YES & YES & $1.12$ & $(2,2)$ & -- & 946\\
$(26,7)$ & 7 & $(12,5)$ & 5 & 2 & YES & YES & YES & $1.00$ & $(2,2)$ & -- & 947\\
$(26,7)$ & 7 & $(12,5)$ & 5 & 2 & YES & YES & YES & $1.25$ & $(2,2)$ & NO & 948\\
$(26,7)$ & 7 & $(12,5)$ & 5 & 2 & YES & YES & YES & $1.25$ & $(2,2)$ & NO & 949\\
$(26,7)$ & 7 & $(13,4)$ & 6 & 13 & YES & YES & YES & $0.88$ & $(2,2)$ & -- & 950\\
$(26,7)$ & 7 & $(13,5)$ & 5 & 13 & YES & YES & YES & $1.12$ & $(2,2)$ & NO & 951\\
$(26,11)$ & 7 & $(13,3)$ & 6 & 13 & YES & YES & YES & $1.12$ & $(2,2)$ & NO & 952\\
$(26,11)$ & 7 & $(13,3)$ & 6 & 13 & YES & YES & YES & $1.12$ & $(2,2)$ & -- & 953\\
$(26,11)$ & 7 & $(13,3)$ & 6 & 13 & YES & YES & YES & $1.12$ & $(2,2)$ & NO & 954\\
$(26,11)$ & 7 & $(13,4)$ & 6 & 13 & YES & YES & YES & $1.12$ & $(2,2)$ & -- & 955\\
$(26,11)$ & 7 & $(13,5)$ & 5 & 13 & YES & YES & YES & $1.12$ & $(2,2)$ & NO & 956\\
$(26,11)$ & 7 & $(14,3)$ & 6 & 2 & YES & YES & YES & $1.00$ & $(2,2)$ & NO & 957\\
$(26,11)$ & 7 & $(14,5)$ & 6 & 2 & YES & YES & YES & $1.12$ & $(2,2)$ & NO & 958\\
$(26,7)$ & 7 & $(15,4)$ & 6 & 1 & YES & YES & YES & $0.88$ & $(2,2)$ & -- & 959\\
$(26,11)$ & 7 & $(16,3)$ & 7 & 2 & YES & YES & YES & $1.00$ & $(2,2)$ & NO & 960\\
$(26,7)$ & 7 & $(17,5)$ & 6 & 1 & YES & YES & YES & $0.88$ & $(2,2)$ & -- & 961\\
$(26,11)$ & 7 & $(17,7)$ & 6 & 1 & YES & YES & YES & $1.12$ & $(2,2)$ & 1655 & 962\\
$(26,7)$ & 7 & $(18,5)$ & 6 & 2 & YES & YES & YES & $1.00$ & $(2,2)$ & NO & 963\\
$(26,7)$ & 7 & $(19,6)$ & 8 & 1 & YES & YES & YES & $1.12$ & $(2,2)$ & NO & 964\\
$(26,7)$ & 7 & $(21,5)$ & 8 & 1 & YES & YES & YES & $1.12$ & $(2,2)$ & -- & 965\\
$(26,11)$ & 7 & $(22,9)$ & 7 & 2 & YES & YES & YES & $1.12$ & $(2,2)$ & NO & 966\\
$(26,5)$ & 9 & $(24,5)$ & 8 & 2 & YES & YES & YES & $1.25$ & $(2,2)$ & -- & 967\\
$(26,11)$ & 7 & $(25,11)$ & 7 & 1 & YES & YES & YES & $1.00$ & $(2,2)$ & NO & 968\\
$(27,8)$ & 7 & $(7,2)$ & 4 & 1 & YES & YES & YES & $1.00$ & $(2,2)$ & -- & 969\\
$(27,8)$ & 7 & $(7,3)$ & 4 & 1 & YES & YES & YES & $1.12$ & $(2,2)$ & NO & 970\\
$(27,8)$ & 7 & $(7,3)$ & 4 & 1 & YES & YES & YES & $1.12$ & $(2,2)$ & -- & 971\\
$(27,10)$ & 7 & $(7,3)$ & 4 & 1 & YES & YES & YES & $1.12$ & $(2,2)$ & -- & 972\\
$(27,8)$ & 7 & $(8,3)$ & 4 & 1 & YES & YES & YES & $1.12$ & $(2,2)$ & NO & 973\\
$(27,8)$ & 7 & $(8,3)$ & 4 & 1 & YES & YES & YES & $1.12$ & $(2,2)$ & -- & 974\\
$(27,10)$ & 7 & $(8,3)$ & 4 & 1 & YES & YES & NO(2) & $1.22$ & $(2,2)$ & NO & 975\\
$(27,10)$ & 7 & $(8,3)$ & 4 & 1 & YES & YES & NO(2) & $1.22$ & $(2,2)$ & -- & 976\\
$(27,8)$ & 7 & $(9,2)$ & 5 & 9 & YES & YES & YES & $1.00$ & $(2,2)$ & NO & 977\\
$(27,8)$ & 7 & $(9,2)$ & 5 & 9 & YES & YES & YES & $1.00$ & $(2,2)$ & -- & 978\\
$(27,8)$ & 7 & $(9,4)$ & 5 & 9 & YES & YES & YES & $1.12$ & $(2,2)$ & -- & 979\\
$(27,8)$ & 7 & $(9,4)$ & 5 & 9 & YES & YES & YES & $1.12$ & $(2,2)$ & NO & 980\\
$(27,10)$ & 7 & $(9,4)$ & 5 & 9 & YES & YES & YES & $1.12$ & $(2,2)$ & -- & 981\\
$(27,8)$ & 7 & $(10,3)$ & 5 & 1 & YES & YES & YES & $1.00$ & $(2,2)$ & -- & 982\\
$(27,10)$ & 7 & $(10,3)$ & 5 & 1 & YES & YES & YES & $1.25$ & $(2,2)$ & -- & 983\\
$(27,5)$ & 8 & $(11,5)$ & 6 & 1 & YES & YES & YES & $1.00$ & $(2,2)$ & -- & 984\\
$(27,5)$ & 8 & $(11,5)$ & 6 & 1 & YES & YES & YES & $1.12$ & $(2,2)$ & NO & 985\\
$(27,8)$ & 7 & $(11,3)$ & 5 & 1 & YES & YES & YES & $1.00$ & $(2,2)$ & NO & 986\\
$(27,8)$ & 7 & $(11,3)$ & 5 & 1 & YES & YES & YES & $1.00$ & $(2,2)$ & -- & 987\\
$(27,8)$ & 7 & $(11,4)$ & 5 & 1 & YES & YES & YES & $1.00$ & $(2,2)$ & -- & 988\\
$(27,8)$ & 7 & $(11,4)$ & 5 & 1 & YES & YES & YES & $1.12$ & $(2,2)$ & NO & 989\\
$(27,8)$ & 7 & $(11,4)$ & 5 & 1 & YES & YES & YES & $1.12$ & $(2,2)$ & NO & 990\\
$(27,10)$ & 7 & $(11,3)$ & 5 & 1 & YES & YES & YES & $1.12$ & $(2,2)$ & -- & 991\\
$(27,10)$ & 7 & $(11,5)$ & 6 & 1 & YES & YES & YES & $1.12$ & $(2,2)$ & NO & 992\\
$(27,11)$ & 8 & $(11,3)$ & 5 & 1 & YES & YES & YES & $1.00$ & $(2,2)$ & -- & 993\\
$(27,8)$ & 7 & $(12,5)$ & 5 & 3 & YES & YES & YES & $1.12$ & $(2,2)$ & -- & 994\\
$(27,8)$ & 7 & $(12,5)$ & 5 & 3 & YES & YES & YES & $1.12$ & $(2,2)$ & NO & 995\\
$(27,10)$ & 7 & $(12,5)$ & 5 & 3 & YES & YES & YES & $1.25$ & $(2,2)$ & NO & 996\\
$(27,10)$ & 7 & $(12,5)$ & 5 & 3 & YES & YES & YES & $1.25$ & $(2,2)$ & -- & 997\\
$(27,8)$ & 7 & $(13,3)$ & 6 & 1 & YES & YES & YES & $1.00$ & $(2,2)$ & -- & 998\\
$(27,8)$ & 7 & $(13,3)$ & 6 & 1 & YES & YES & YES & $1.12$ & $(2,2)$ & NO & 999\\
$(27,8)$ & 7 & $(13,4)$ & 6 & 1 & YES & YES & YES & $1.12$ & $(2,2)$ & -- & 1000\\
$(27,10)$ & 7 & $(13,3)$ & 6 & 1 & YES & YES & YES & $1.25$ & $(2,2)$ & NO & 1001\\
$(27,10)$ & 7 & $(13,4)$ & 6 & 1 & YES & YES & YES & $1.25$ & $(2,2)$ & NO & 1002\\
$(27,11)$ & 8 & $(13,3)$ & 6 & 1 & YES & YES & YES & $1.25$ & $(2,2)$ & -- & 1003\\
$(27,5)$ & 8 & $(14,5)$ & 6 & 1 & YES & YES & YES & $1.00$ & $(2,2)$ & -- & 1004\\
$(27,5)$ & 8 & $(14,5)$ & 6 & 1 & YES & YES & YES & $1.12$ & $(2,2)$ & NO & 1005\\
$(27,10)$ & 7 & $(14,3)$ & 6 & 1 & YES & YES & YES & $1.12$ & $(2,2)$ & NO & 1006\\
$(27,11)$ & 8 & $(14,3)$ & 6 & 1 & YES & YES & YES & $1.12$ & $(2,2)$ & -- & 1007\\
$(27,8)$ & 7 & $(15,4)$ & 6 & 3 & YES & YES & YES & $1.00$ & $(2,2)$ & -- & 1008\\
$(27,10)$ & 7 & $(15,4)$ & 6 & 3 & YES & YES & YES & $1.25$ & $(2,2)$ & NO & 1009\\
$(27,8)$ & 7 & $(16,7)$ & 6 & 1 & YES & YES & YES & $1.12$ & $(2,2)$ & NO & 1010\\
$(27,11)$ & 8 & $(16,3)$ & 7 & 1 & YES & YES & YES & $1.12$ & $(2,2)$ & -- & 1011\\
$(27,11)$ & 8 & $(16,3)$ & 7 & 1 & YES & YES & YES & $1.25$ & $(2,2)$ & NO & 1012\\
$(27,10)$ & 7 & $(18,7)$ & 6 & 9 & YES & YES & YES & $1.25$ & $(2,2)$ & NO & 1013\\
$(27,8)$ & 7 & $(19,6)$ & 8 & 1 & YES & YES & YES & $1.12$ & $(2,2)$ & 1909 & 1014\\
$(27,10)$ & 7 & $(20,7)$ & 8 & 1 & YES & YES & YES & $1.25$ & $(2,2)$ & NO & 1015\\
$(28,11)$ & 8 & $(7,2)$ & 4 & 7 & YES & YES & YES & $1.00$ & $(2,2)$ & -- & 1016\\
$(28,11)$ & 8 & $(9,2)$ & 5 & 1 & YES & YES & YES & $1.12$ & $(2,2)$ & NO & 1017\\
$(28,11)$ & 8 & $(9,2)$ & 5 & 1 & YES & YES & YES & $1.12$ & $(2,2)$ & -- & 1018\\
$(28,11)$ & 8 & $(9,2)$ & 5 & 1 & YES & YES & YES & $1.12$ & $(2,2)$ & NO & 1019\\
$(28,11)$ & 8 & $(21,8)$ & 6 & 7 & YES & YES & YES & $1.12$ & $(2,2)$ & NO & 1020\\
$(29,11)$ & 7 & $(4,1)$ & 3 & 1 & YES & YES & YES & $1.12$ & $(2,2)$ & NO & 1021\\
$(29,12)$ & 7 & $(4,1)$ & 3 & 1 & YES & YES & YES & $1.12$ & $(2,2)$ & NO & 1022\\
$(29,12)$ & 7 & $(4,1)$ & 3 & 1 & YES & YES & YES & $1.12$ & $(2,2)$ & -- & 1023\\
$(29,12)$ & 7 & $(4,1)$ & 3 & 1 & YES & YES & YES & $1.12$ & $(2,2)$ & NO & 1024\\
$(29,8)$ & 7 & $(5,2)$ & 3 & 1 & YES & YES & YES & $1.00$ & $(2,2)$ & -- & 1025\\
$(29,8)$ & 7 & $(5,2)$ & 3 & 1 & YES & YES & YES & $1.25$ & $(2,2)$ & NO & 1026\\
$(29,8)$ & 7 & $(5,2)$ & 3 & 1 & YES & YES & YES & $1.12$ & $(2,2)$ & NO & 1027\\
$(29,9)$ & 8 & $(5,2)$ & 3 & 1 & YES & YES & YES & $1.25$ & $(2,2)$ & -- & 1028\\
$(29,11)$ & 7 & $(5,1)$ & 4 & 1 & YES & YES & YES & $1.25$ & $(2,2)$ & NO & 1029\\
$(29,11)$ & 7 & $(5,1)$ & 4 & 1 & YES & YES & YES & $1.25$ & $(2,2)$ & NO & 1030\\
$(29,11)$ & 7 & $(5,1)$ & 4 & 1 & YES & YES & YES & $1.25$ & $(2,2)$ & -- & 1031\\
$(29,11)$ & 7 & $(5,2)$ & 3 & 1 & YES & YES & NO(2) & $1.33$ & $(2,2)$ & -- & 1032\\
$(29,11)$ & 7 & $(5,2)$ & 3 & 1 & YES & YES & YES & $1.12$ & $(2,2)$ & NO & 1033\\
$(29,12)$ & 7 & $(5,2)$ & 3 & 1 & YES & YES & YES & $1.12$ & $(2,2)$ & NO & 1034\\
$(29,12)$ & 7 & $(5,2)$ & 3 & 1 & YES & YES & YES & $1.12$ & $(2,2)$ & -- & 1035\\
$(29,11)$ & 7 & $(7,2)$ & 4 & 1 & YES & YES & YES & $1.12$ & $(2,2)$ & NO & 1036\\
$(29,11)$ & 7 & $(7,2)$ & 4 & 1 & YES & YES & YES & $1.12$ & $(2,2)$ & -- & 1037\\
$(29,11)$ & 7 & $(7,2)$ & 4 & 1 & YES & YES & YES & $1.00$ & $(2,2)$ & NO & 1038\\
$(29,11)$ & 7 & $(7,3)$ & 4 & 1 & YES & YES & YES & $1.25$ & $(2,2)$ & -- & 1039\\
$(29,12)$ & 7 & $(7,2)$ & 4 & 1 & YES & YES & YES & $1.00$ & $(2,2)$ & -- & 1040\\
$(29,12)$ & 7 & $(7,2)$ & 4 & 1 & YES & YES & YES & $1.12$ & $(2,2)$ & NO & 1041\\
$(29,12)$ & 7 & $(7,3)$ & 4 & 1 & YES & YES & YES & $1.00$ & $(2,2)$ & -- & 1042\\
$(29,13)$ & 8 & $(7,2)$ & 4 & 1 & YES & YES & YES & $1.00$ & $(2,2)$ & -- & 1043\\
$(29,13)$ & 8 & $(7,2)$ & 4 & 1 & YES & YES & YES & $1.12$ & $(2,2)$ & NO & 1044\\
$(29,11)$ & 7 & $(8,3)$ & 4 & 1 & YES & YES & YES & $1.25$ & $(2,2)$ & -- & 1045\\
$(29,12)$ & 7 & $(8,3)$ & 4 & 1 & YES & YES & YES & $0.88$ & $(2,2)$ & -- & 1046\\
$(29,12)$ & 7 & $(8,3)$ & 4 & 1 & YES & YES & YES & $1.12$ & $(2,2)$ & NO & 1047\\
$(29,13)$ & 8 & $(8,3)$ & 4 & 1 & YES & YES & YES & $1.12$ & $(2,2)$ & -- & 1048\\
$(29,13)$ & 8 & $(8,3)$ & 4 & 1 & YES & YES & YES & $1.25$ & $(2,2)$ & NO & 1049\\
$(29,8)$ & 7 & $(9,2)$ & 5 & 1 & YES & YES & YES & $1.12$ & $(2,2)$ & NO & 1050\\
$(29,8)$ & 7 & $(9,2)$ & 5 & 1 & YES & YES & YES & $1.12$ & $(2,2)$ & -- & 1051\\
$(29,8)$ & 7 & $(9,2)$ & 5 & 1 & YES & YES & YES & $1.25$ & $(2,2)$ & NO & 1052\\
$(29,9)$ & 8 & $(9,4)$ & 5 & 1 & YES & YES & YES & $1.12$ & $(2,2)$ & -- & 1053\\
$(29,11)$ & 7 & $(9,2)$ & 5 & 1 & YES & YES & YES & $1.12$ & $(2,2)$ & -- & 1054\\
$(29,11)$ & 7 & $(9,2)$ & 5 & 1 & YES & YES & YES & $1.25$ & $(2,2)$ & NO & 1055\\
$(29,11)$ & 7 & $(9,4)$ & 5 & 1 & YES & YES & YES & $1.25$ & $(2,2)$ & -- & 1056\\
$(29,12)$ & 7 & $(9,4)$ & 5 & 1 & YES & YES & YES & $1.12$ & $(2,2)$ & -- & 1057\\
$(29,8)$ & 7 & $(10,3)$ & 5 & 1 & YES & YES & YES & $1.12$ & $(2,2)$ & NO & 1058\\
$(29,8)$ & 7 & $(10,3)$ & 5 & 1 & YES & YES & YES & $1.12$ & $(2,2)$ & -- & 1059\\
$(29,11)$ & 7 & $(10,3)$ & 5 & 1 & YES & YES & YES & $1.12$ & $(2,2)$ & NO & 1060\\
$(29,12)$ & 7 & $(10,3)$ & 5 & 1 & YES & YES & YES & $1.12$ & $(2,2)$ & -- & 1061\\
$(29,12)$ & 7 & $(10,3)$ & 5 & 1 & YES & YES & YES & $1.12$ & $(2,2)$ & NO & 1062\\
$(29,11)$ & 7 & $(11,3)$ & 5 & 1 & YES & YES & YES & $1.00$ & $(2,2)$ & NO & 1063\\
$(29,12)$ & 7 & $(11,3)$ & 5 & 1 & YES & YES & YES & $1.12$ & $(2,2)$ & -- & 1064\\
$(29,12)$ & 7 & $(11,3)$ & 5 & 1 & YES & YES & YES & $1.00$ & $(2,2)$ & NO & 1065\\
$(29,12)$ & 7 & $(11,5)$ & 6 & 1 & YES & YES & YES & $1.00$ & $(2,2)$ & 728 & 1066\\
$(29,8)$ & 7 & $(12,5)$ & 5 & 1 & YES & YES & YES & $1.12$ & $(2,2)$ & NO & 1067\\
$(29,8)$ & 7 & $(12,5)$ & 5 & 1 & YES & YES & YES & $1.12$ & $(2,2)$ & -- & 1068\\
$(29,11)$ & 7 & $(12,5)$ & 5 & 1 & YES & YES & YES & $1.12$ & $(2,2)$ & NO & 1069\\
$(29,13)$ & 8 & $(12,5)$ & 5 & 1 & YES & YES & YES & $1.00$ & $(2,2)$ & NO & 1070\\
$(29,8)$ & 7 & $(13,5)$ & 5 & 1 & YES & YES & YES & $1.12$ & $(2,2)$ & NO & 1071\\
$(29,11)$ & 7 & $(13,3)$ & 6 & 1 & YES & YES & YES & $1.25$ & $(2,2)$ & NO & 1072\\
$(29,12)$ & 7 & $(13,3)$ & 6 & 1 & YES & YES & YES & $1.00$ & $(2,2)$ & NO & 1073\\
$(29,8)$ & 7 & $(15,4)$ & 6 & 1 & YES & YES & YES & $1.00$ & $(2,2)$ & NO & 1074\\
$(29,11)$ & 7 & $(16,3)$ & 7 & 1 & YES & YES & YES & $1.00$ & $(2,2)$ & NO & 1075\\
$(29,12)$ & 7 & $(16,7)$ & 6 & 1 & YES & YES & YES & $1.00$ & $(2,2)$ & NO & 1076\\
$(29,11)$ & 7 & $(17,6)$ & 7 & 1 & YES & YES & YES & $1.12$ & $(2,2)$ & NO & 1077\\
$(29,11)$ & 7 & $(18,7)$ & 6 & 1 & YES & YES & YES & $1.25$ & $(2,2)$ & 1871 & 1078\\
$(29,12)$ & 7 & $(19,8)$ & 6 & 1 & YES & YES & YES & $1.00$ & $(2,2)$ & NO & 1079\\
$(29,12)$ & 7 & $(21,8)$ & 6 & 1 & YES & YES & YES & $1.00$ & $(2,2)$ & NO & 1080\\
$(29,8)$ & 7 & $(23,7)$ & 7 & 1 & YES & YES & YES & $1.12$ & $(2,2)$ & NO & 1081\\
$(29,11)$ & 7 & $(23,9)$ & 7 & 1 & YES & YES & YES & $1.12$ & $(2,2)$ & NO & 1082\\
$(29,8)$ & 7 & $(25,7)$ & 7 & 1 & YES & YES & YES & $1.12$ & $(2,2)$ & NO & 1083\\
$(29,12)$ & 7 & $(26,11)$ & 7 & 1 & YES & YES & YES & $1.00$ & $(2,2)$ & NO & 1084\\
$(29,11)$ & 7 & $(27,10)$ & 7 & 1 & YES & YES & YES & $1.12$ & $(2,2)$ & NO & 1085\\
$(30,11)$ & 7 & $(5,2)$ & 3 & 5 & YES & YES & YES & $0.88$ & $(2,2)$ & -- & 1086\\
$(30,11)$ & 7 & $(7,2)$ & 4 & 1 & YES & YES & YES & $1.00$ & $(2,2)$ & -- & 1087\\
$(30,11)$ & 7 & $(7,2)$ & 4 & 1 & YES & YES & YES & $1.12$ & $(2,2)$ & NO & 1088\\
$(30,11)$ & 7 & $(7,3)$ & 4 & 1 & YES & YES & YES & $1.12$ & $(2,2)$ & -- & 1089\\
$(30,11)$ & 7 & $(7,3)$ & 4 & 1 & YES & YES & YES & $1.25$ & $(2,2)$ & NO & 1090\\
$(30,13)$ & 8 & $(7,2)$ & 4 & 1 & YES & YES & YES & $1.38$ & $(2,2)$ & -- & 1091\\
$(30,13)$ & 8 & $(7,2)$ & 4 & 1 & YES & YES & YES & $1.38$ & $(2,2)$ & NO & 1092\\
$(30,11)$ & 7 & $(8,3)$ & 4 & 2 & YES & YES & YES & $0.88$ & $(2,2)$ & -- & 1093\\
$(30,11)$ & 7 & $(8,3)$ & 4 & 2 & YES & YES & YES & $1.12$ & $(2,2)$ & NO & 1094\\
$(30,13)$ & 8 & $(8,3)$ & 4 & 2 & YES & YES & YES & $1.25$ & $(2,2)$ & -- & 1095\\
$(30,11)$ & 7 & $(9,4)$ & 5 & 3 & YES & YES & YES & $1.00$ & $(2,2)$ & NO & 1096\\
$(30,11)$ & 7 & $(9,4)$ & 5 & 3 & YES & YES & YES & $1.12$ & $(2,2)$ & -- & 1097\\
$(30,11)$ & 7 & $(10,3)$ & 5 & 10 & YES & YES & NO(2) & $1.00$ & $(4,1)$ & -- & 1098\\
$(30,11)$ & 7 & $(10,3)$ & 5 & 10 & YES & YES & YES & $1.12$ & $(2,2)$ & NO & 1099\\
$(30,13)$ & 8 & $(10,3)$ & 5 & 10 & YES & YES & YES & $1.12$ & $(2,2)$ & -- & 1100\\
$(30,7)$ & 8 & $(11,4)$ & 5 & 1 & YES & YES & YES & $1.12$ & $(2,2)$ & NO & 1101\\
$(30,11)$ & 7 & $(11,3)$ & 5 & 1 & YES & YES & NO(2) & $0.88$ & $(4,1)$ & -- & 1102\\
$(30,11)$ & 7 & $(11,3)$ & 5 & 1 & YES & YES & YES & $1.00$ & $(2,2)$ & NO & 1103\\
$(30,13)$ & 8 & $(11,3)$ & 5 & 1 & YES & YES & YES & $1.00$ & $(2,2)$ & -- & 1104\\
$(30,11)$ & 7 & $(12,5)$ & 5 & 6 & YES & YES & YES & $1.12$ & $(2,2)$ & NO & 1105\\
$(30,11)$ & 7 & $(12,5)$ & 5 & 6 & YES & YES & YES & $1.12$ & $(2,2)$ & -- & 1106\\
$(30,7)$ & 8 & $(13,3)$ & 6 & 1 & YES & YES & YES & $1.12$ & $(2,2)$ & -- & 1107\\
$(30,7)$ & 8 & $(13,3)$ & 6 & 1 & YES & YES & YES & $1.25$ & $(2,2)$ & NO & 1108\\
$(30,7)$ & 8 & $(13,4)$ & 6 & 1 & YES & YES & YES & $1.00$ & $(2,2)$ & -- & 1109\\
$(30,11)$ & 7 & $(13,3)$ & 6 & 1 & YES & YES & YES & $1.00$ & $(2,2)$ & NO & 1110\\
$(30,11)$ & 7 & $(13,5)$ & 5 & 1 & YES & YES & YES & $1.00$ & $(2,2)$ & NO & 1111\\
$(30,13)$ & 8 & $(13,3)$ & 6 & 1 & YES & YES & YES & $1.25$ & $(2,2)$ & NO & 1112\\
$(30,13)$ & 8 & $(13,3)$ & 6 & 1 & YES & YES & YES & $1.25$ & $(2,2)$ & -- & 1113\\
$(30,13)$ & 8 & $(13,3)$ & 6 & 1 & YES & YES & YES & $1.25$ & $(2,2)$ & NO & 1114\\
$(30,7)$ & 8 & $(14,5)$ & 6 & 2 & YES & YES & YES & $1.12$ & $(2,2)$ & -- & 1115\\
$(30,13)$ & 8 & $(14,3)$ & 6 & 2 & YES & YES & YES & $1.12$ & $(2,2)$ & NO & 1116\\
$(30,13)$ & 8 & $(14,3)$ & 6 & 2 & YES & YES & YES & $1.12$ & $(2,2)$ & -- & 1117\\
$(30,13)$ & 8 & $(14,3)$ & 6 & 2 & YES & YES & YES & $1.12$ & $(2,2)$ & NO & 1118\\
$(30,7)$ & 8 & $(15,4)$ & 6 & 15 & YES & YES & YES & $1.00$ & $(2,2)$ & -- & 1119\\
$(30,7)$ & 8 & $(16,5)$ & 7 & 2 & YES & YES & YES & $1.12$ & $(2,2)$ & NO & 1120\\
$(30,7)$ & 8 & $(16,7)$ & 6 & 2 & YES & YES & YES & $1.12$ & $(2,2)$ & NO & 1121\\
$(30,7)$ & 8 & $(16,7)$ & 6 & 2 & YES & YES & YES & $1.12$ & $(2,2)$ & -- & 1122\\
$(30,11)$ & 7 & $(16,7)$ & 6 & 2 & YES & YES & YES & $1.00$ & $(2,2)$ & 1821 & 1123\\
$(30,7)$ & 8 & $(17,5)$ & 6 & 1 & YES & YES & YES & $1.00$ & $(2,2)$ & -- & 1124\\
$(30,7)$ & 8 & $(18,5)$ & 6 & 6 & YES & YES & YES & $1.00$ & $(2,2)$ & -- & 1125\\
$(30,11)$ & 7 & $(18,7)$ & 6 & 6 & YES & YES & YES & $1.00$ & $(2,2)$ & NO & 1126\\
$(30,7)$ & 8 & $(19,7)$ & 6 & 1 & YES & YES & YES & $1.12$ & $(2,2)$ & -- & 1127\\
$(30,11)$ & 7 & $(20,7)$ & 8 & 10 & YES & YES & YES & $1.12$ & $(2,2)$ & 1975 & 1128\\
$(30,7)$ & 8 & $(21,5)$ & 8 & 3 & YES & YES & YES & $1.12$ & $(2,2)$ & -- & 1129\\
$(30,11)$ & 7 & $(21,8)$ & 6 & 3 & YES & YES & YES & $1.12$ & $(2,2)$ & NO & 1130\\
$(30,7)$ & 8 & $(30,7)$ & 8 & 30 & YES & YES & YES & $1.00$ & $(2,2)$ & -- & 1131\\
$(31,7)$ & 8 & $(3,1)$ & 2 & 1 & YES & YES & YES & $1.12$ & $(2,2)$ & NO & 1132\\
$(31,9)$ & 8 & $(4,1)$ & 3 & 1 & YES & YES & YES & $1.12$ & $(2,2)$ & NO & 1133\\
$(31,9)$ & 8 & $(4,1)$ & 3 & 1 & YES & YES & YES & $1.12$ & $(2,2)$ & -- & 1134\\
$(31,13)$ & 7 & $(4,1)$ & 3 & 1 & YES & YES & YES & $1.12$ & $(2,2)$ & -- & 1135\\
$(31,12)$ & 7 & $(5,2)$ & 3 & 1 & YES & YES & YES & $1.12$ & $(2,2)$ & -- & 1136\\
$(31,12)$ & 7 & $(5,2)$ & 3 & 1 & YES & YES & YES & $1.12$ & $(2,2)$ & NO & 1137\\
$(31,13)$ & 7 & $(5,1)$ & 4 & 1 & YES & YES & YES & $1.00$ & $(2,2)$ & 676 & 1138\\
$(31,13)$ & 7 & $(5,1)$ & 4 & 1 & YES & YES & YES & $1.00$ & $(2,2)$ & -- & 1139\\
$(31,13)$ & 7 & $(5,1)$ & 4 & 1 & YES & YES & YES & $1.12$ & $(2,2)$ & NO & 1140\\
$(31,13)$ & 7 & $(5,2)$ & 3 & 1 & YES & YES & NO(2) & $1.11$ & $(2,2)$ & -- & 1141\\
$(31,14)$ & 8 & $(5,2)$ & 3 & 1 & YES & YES & YES & $1.25$ & $(2,2)$ & -- & 1142\\
$(31,7)$ & 8 & $(7,3)$ & 4 & 1 & YES & YES & YES & $1.25$ & $(2,2)$ & -- & 1143\\
$(31,9)$ & 8 & $(7,2)$ & 4 & 1 & YES & YES & YES & $1.00$ & $(2,2)$ & NO & 1144\\
$(31,9)$ & 8 & $(7,2)$ & 4 & 1 & YES & YES & YES & $1.00$ & $(2,2)$ & -- & 1145\\
$(31,9)$ & 8 & $(7,3)$ & 4 & 1 & YES & YES & YES & $1.25$ & $(2,2)$ & -- & 1146\\
$(31,11)$ & 8 & $(7,2)$ & 4 & 1 & YES & YES & YES & $0.88$ & $(2,2)$ & -- & 1147\\
$(31,12)$ & 7 & $(7,2)$ & 4 & 1 & YES & YES & YES & $1.00$ & $(2,2)$ & NO & 1148\\
$(31,12)$ & 7 & $(7,2)$ & 4 & 1 & YES & YES & YES & $1.00$ & $(2,2)$ & -- & 1149\\
$(31,12)$ & 7 & $(7,3)$ & 4 & 1 & YES & YES & YES & $1.00$ & $(2,2)$ & -- & 1150\\
$(31,13)$ & 7 & $(7,2)$ & 4 & 1 & YES & YES & YES & $1.12$ & $(2,2)$ & NO & 1151\\
$(31,13)$ & 7 & $(7,2)$ & 4 & 1 & YES & YES & YES & $1.12$ & $(2,2)$ & -- & 1152\\
$(31,13)$ & 7 & $(7,3)$ & 4 & 1 & YES & YES & YES & $1.00$ & $(2,2)$ & -- & 1153\\
$(31,14)$ & 8 & $(7,2)$ & 4 & 1 & YES & YES & YES & $1.12$ & $(2,2)$ & NO & 1154\\
$(31,14)$ & 8 & $(7,2)$ & 4 & 1 & YES & YES & YES & $1.00$ & $(2,2)$ & -- & 1155\\
$(31,9)$ & 8 & $(8,3)$ & 4 & 1 & YES & YES & YES & $1.00$ & $(2,2)$ & NO & 1156\\
$(31,9)$ & 8 & $(8,3)$ & 4 & 1 & YES & YES & YES & $1.00$ & $(2,2)$ & -- & 1157\\
$(31,9)$ & 8 & $(8,3)$ & 4 & 1 & YES & YES & YES & $1.00$ & $(2,2)$ & NO & 1158\\
$(31,12)$ & 7 & $(8,3)$ & 4 & 1 & YES & YES & YES & $1.12$ & $(2,2)$ & -- & 1159\\
$(31,12)$ & 7 & $(8,3)$ & 4 & 1 & YES & YES & YES & $1.12$ & $(2,2)$ & NO & 1160\\
$(31,13)$ & 7 & $(8,3)$ & 4 & 1 & YES & YES & NO(2) & $1.22$ & $(2,2)$ & NO & 1161\\
$(31,13)$ & 7 & $(8,3)$ & 4 & 1 & YES & YES & YES & $1.12$ & $(2,2)$ & -- & 1162\\
$(31,7)$ & 8 & $(9,4)$ & 5 & 1 & YES & YES & YES & $1.00$ & $(2,2)$ & -- & 1163\\
$(31,9)$ & 8 & $(9,2)$ & 5 & 1 & YES & YES & YES & $1.12$ & $(2,2)$ & NO & 1164\\
$(31,9)$ & 8 & $(9,2)$ & 5 & 1 & YES & YES & YES & $1.12$ & $(2,2)$ & -- & 1165\\
$(31,11)$ & 8 & $(9,2)$ & 5 & 1 & YES & YES & YES & $1.12$ & $(2,2)$ & NO & 1166\\
$(31,11)$ & 8 & $(9,2)$ & 5 & 1 & YES & YES & YES & $1.12$ & $(2,2)$ & -- & 1167\\
$(31,13)$ & 7 & $(9,4)$ & 5 & 1 & YES & YES & YES & $1.12$ & $(2,2)$ & -- & 1168\\
$(31,9)$ & 8 & $(10,3)$ & 5 & 1 & YES & YES & YES & $1.12$ & $(2,2)$ & -- & 1169\\
$(31,12)$ & 7 & $(10,3)$ & 5 & 1 & YES & YES & YES & $1.00$ & $(2,2)$ & -- & 1170\\
$(31,7)$ & 8 & $(11,4)$ & 5 & 1 & YES & YES & YES & $1.00$ & $(2,2)$ & -- & 1171\\
$(31,9)$ & 8 & $(11,2)$ & 6 & 1 & YES & YES & YES & $1.12$ & $(2,2)$ & NO & 1172\\
$(31,9)$ & 8 & $(11,2)$ & 6 & 1 & YES & YES & YES & $1.12$ & $(2,2)$ & -- & 1173\\
$(31,12)$ & 7 & $(11,3)$ & 5 & 1 & YES & YES & YES & $1.12$ & $(2,2)$ & NO & 1174\\
$(31,13)$ & 7 & $(11,4)$ & 5 & 1 & YES & YES & YES & $1.12$ & $(2,2)$ & NO & 1175\\
$(31,13)$ & 7 & $(11,5)$ & 6 & 1 & YES & YES & YES & $1.00$ & $(2,2)$ & NO & 1176\\
$(31,14)$ & 8 & $(12,5)$ & 5 & 1 & YES & YES & YES & $1.12$ & $(2,2)$ & 1907 & 1177\\
$(31,7)$ & 8 & $(13,4)$ & 6 & 1 & YES & YES & YES & $1.00$ & $(2,2)$ & -- & 1178\\
$(31,7)$ & 8 & $(13,5)$ & 5 & 1 & YES & YES & YES & $1.12$ & $(2,2)$ & NO & 1179\\
$(31,9)$ & 8 & $(13,3)$ & 6 & 1 & YES & YES & YES & $1.25$ & $(2,2)$ & NO & 1180\\
$(31,11)$ & 8 & $(13,5)$ & 5 & 1 & YES & YES & YES & $1.00$ & $(2,2)$ & NO & 1181\\
$(31,12)$ & 7 & $(13,3)$ & 6 & 1 & YES & YES & YES & $1.12$ & $(2,2)$ & NO & 1182\\
$(31,12)$ & 7 & $(13,3)$ & 6 & 1 & YES & YES & YES & $1.12$ & $(2,2)$ & -- & 1183\\
$(31,13)$ & 7 & $(13,3)$ & 6 & 1 & YES & YES & YES & $1.00$ & $(2,2)$ & -- & 1184\\
$(31,13)$ & 7 & $(13,4)$ & 6 & 1 & YES & YES & YES & $1.12$ & $(2,2)$ & -- & 1185\\
$(31,13)$ & 7 & $(13,5)$ & 5 & 1 & YES & YES & YES & $1.00$ & $(2,2)$ & NO & 1186\\
$(31,9)$ & 8 & $(14,3)$ & 6 & 1 & YES & YES & YES & $1.00$ & $(2,2)$ & NO & 1187\\
$(31,13)$ & 7 & $(14,5)$ & 6 & 1 & YES & YES & YES & $1.12$ & $(2,2)$ & NO & 1188\\
$(31,9)$ & 8 & $(16,3)$ & 7 & 1 & YES & YES & YES & $1.00$ & $(2,2)$ & -- & 1189\\
$(31,12)$ & 7 & $(16,7)$ & 6 & 1 & YES & YES & YES & $1.12$ & $(2,2)$ & NO & 1190\\
$(31,14)$ & 8 & $(16,7)$ & 6 & 1 & YES & YES & YES & $1.00$ & $(2,2)$ & NO & 1191\\
$(31,7)$ & 8 & $(19,4)$ & 7 & 1 & YES & YES & YES & $1.00$ & $(2,2)$ & -- & 1192\\
$(31,11)$ & 8 & $(19,7)$ & 6 & 1 & YES & YES & YES & $1.12$ & $(2,2)$ & NO & 1193\\
$(31,12)$ & 7 & $(19,7)$ & 6 & 1 & YES & YES & YES & $1.12$ & $(2,2)$ & NO & 1194\\
$(31,12)$ & 7 & $(19,8)$ & 6 & 1 & YES & YES & YES & $1.12$ & $(2,2)$ & NO & 1195\\
$(31,7)$ & 8 & $(21,5)$ & 8 & 1 & YES & YES & YES & $1.25$ & $(2,2)$ & NO & 1196\\
$(31,12)$ & 7 & $(21,8)$ & 6 & 1 & YES & YES & YES & $1.00$ & $(2,2)$ & NO & 1197\\
$(31,13)$ & 7 & $(22,9)$ & 7 & 1 & YES & YES & YES & $1.00$ & $(2,2)$ & NO & 1198\\
$(31,9)$ & 8 & $(23,7)$ & 7 & 1 & YES & YES & YES & $1.00$ & $(2,2)$ & 2787 & 1199\\
$(31,9)$ & 8 & $(24,7)$ & 7 & 1 & YES & YES & YES & $1.12$ & $(2,2)$ & NO & 1200\\
$(31,7)$ & 8 & $(25,6)$ & 9 & 1 & YES & YES & YES & $1.00$ & $(2,2)$ & NO & 1201\\
$(31,12)$ & 7 & $(29,11)$ & 7 & 1 & YES & YES & YES & $1.12$ & $(2,2)$ & NO & 1202\\
$(31,13)$ & 7 & $(29,12)$ & 7 & 1 & YES & YES & YES & $1.12$ & $(2,2)$ & NO & 1203\\
$(31,14)$ & 8 & $(29,13)$ & 8 & 1 & YES & YES & YES & $1.12$ & $(2,2)$ & NO & 1204\\
$(32,7)$ & 8 & $(3,1)$ & 2 & 1 & YES & YES & YES & $1.00$ & $(2,2)$ & NO & 1205\\
$(32,7)$ & 8 & $(3,1)$ & 2 & 1 & YES & YES & YES & $1.00$ & $(2,2)$ & -- & 1206\\
$(32,9)$ & 8 & $(8,3)$ & 4 & 8 & YES & YES & YES & $0.88$ & $(2,2)$ & NO & 1207\\
$(32,7)$ & 8 & $(9,4)$ & 5 & 1 & YES & YES & YES & $1.00$ & $(2,2)$ & -- & 1208\\
$(32,9)$ & 8 & $(10,3)$ & 5 & 2 & YES & YES & YES & $1.12$ & $(2,2)$ & -- & 1209\\
$(32,7)$ & 8 & $(11,4)$ & 5 & 1 & YES & YES & YES & $1.00$ & $(2,2)$ & -- & 1210\\
$(32,9)$ & 8 & $(13,4)$ & 6 & 1 & YES & YES & YES & $1.00$ & $(2,2)$ & NO & 1211\\
$(32,7)$ & 8 & $(14,3)$ & 6 & 2 & YES & YES & YES & $1.00$ & $(2,2)$ & 1363 & 1212\\
$(32,9)$ & 8 & $(14,3)$ & 6 & 2 & YES & YES & YES & $1.12$ & $(2,2)$ & NO & 1213\\
$(32,9)$ & 8 & $(26,7)$ & 7 & 2 & YES & YES & YES & $1.12$ & $(2,2)$ & 2861 & 1214\\
$(33,10)$ & 8 & $(5,1)$ & 4 & 1 & YES & YES & YES & $1.12$ & $(2,2)$ & NO & 1215\\
$(33,10)$ & 8 & $(5,1)$ & 4 & 1 & YES & YES & YES & $1.12$ & $(2,2)$ & -- & 1216\\
$(33,10)$ & 8 & $(5,1)$ & 4 & 1 & YES & YES & YES & $1.12$ & $(2,2)$ & NO & 1217\\
$(33,10)$ & 8 & $(5,2)$ & 3 & 1 & YES & YES & YES & $1.22$ & $(2,2)$ & -- & 1218\\
$(33,10)$ & 8 & $(5,2)$ & 3 & 1 & YES & YES & YES & $1.12$ & $(2,2)$ & NO & 1219\\
$(33,14)$ & 8 & $(5,2)$ & 3 & 1 & YES & YES & YES & $1.25$ & $(2,2)$ & -- & 1220\\
$(33,10)$ & 8 & $(7,2)$ & 4 & 1 & YES & YES & YES & $1.12$ & $(2,2)$ & -- & 1221\\
$(33,10)$ & 8 & $(7,2)$ & 4 & 1 & YES & YES & YES & $1.12$ & $(2,2)$ & NO & 1222\\
$(33,10)$ & 8 & $(7,3)$ & 4 & 1 & YES & YES & YES & $1.12$ & $(2,2)$ & -- & 1223\\
$(33,10)$ & 8 & $(7,3)$ & 4 & 1 & YES & YES & YES & $1.12$ & $(2,2)$ & NO & 1224\\
$(33,14)$ & 8 & $(7,2)$ & 4 & 1 & YES & YES & YES & $1.00$ & $(2,2)$ & -- & 1225\\
$(33,10)$ & 8 & $(8,3)$ & 4 & 1 & YES & YES & YES & $0.88$ & $(2,2)$ & -- & 1226\\
$(33,10)$ & 8 & $(9,2)$ & 5 & 3 & YES & YES & YES & $1.00$ & $(2,2)$ & -- & 1227\\
$(33,10)$ & 8 & $(9,2)$ & 5 & 3 & YES & YES & YES & $1.12$ & $(2,2)$ & NO & 1228\\
$(33,10)$ & 8 & $(9,4)$ & 5 & 3 & YES & YES & YES & $1.00$ & $(2,2)$ & NO & 1229\\
$(33,10)$ & 8 & $(9,4)$ & 5 & 3 & YES & YES & YES & $1.12$ & $(2,2)$ & -- & 1230\\
$(33,14)$ & 8 & $(9,2)$ & 5 & 3 & YES & YES & YES & $1.00$ & $(2,2)$ & NO & 1231\\
$(33,14)$ & 8 & $(9,2)$ & 5 & 3 & YES & YES & YES & $1.12$ & $(2,2)$ & NO & 1232\\
$(33,14)$ & 8 & $(9,2)$ & 5 & 3 & YES & YES & YES & $1.12$ & $(2,2)$ & -- & 1233\\
$(33,10)$ & 8 & $(10,3)$ & 5 & 1 & YES & YES & YES & $0.88$ & $(2,2)$ & -- & 1234\\
$(33,10)$ & 8 & $(11,4)$ & 5 & 11 & YES & YES & YES & $1.00$ & $(2,2)$ & NO & 1235\\
$(33,14)$ & 8 & $(11,2)$ & 6 & 11 & YES & YES & YES & $1.00$ & $(2,2)$ & NO & 1236\\
$(33,7)$ & 8 & $(14,5)$ & 6 & 1 & YES & YES & YES & $1.12$ & $(2,2)$ & -- & 1237\\
$(33,10)$ & 8 & $(14,3)$ & 6 & 1 & YES & YES & YES & $1.00$ & $(2,2)$ & NO & 1238\\
$(33,10)$ & 8 & $(14,3)$ & 6 & 1 & YES & YES & YES & $1.00$ & $(2,2)$ & -- & 1239\\
$(33,10)$ & 8 & $(18,5)$ & 6 & 3 & YES & YES & YES & $1.00$ & $(2,2)$ & NO & 1240\\
$(33,10)$ & 8 & $(24,7)$ & 7 & 3 & YES & YES & YES & $1.12$ & $(2,2)$ & NO & 1241\\
$(33,10)$ & 8 & $(27,8)$ & 7 & 3 & YES & YES & YES & $1.12$ & $(2,2)$ & NO & 1242\\
$(33,14)$ & 8 & $(31,13)$ & 7 & 1 & YES & YES & YES & $1.00$ & $(2,2)$ & 2159 & 1243\\
$(34,13)$ & 7 & $(4,1)$ & 3 & 2 & YES & YES & YES & $1.12$ & $(2,2)$ & -- & 1244\\
$(34,13)$ & 7 & $(4,1)$ & 3 & 2 & YES & YES & YES & $1.25$ & $(2,2)$ & NO & 1245\\
$(34,13)$ & 7 & $(5,1)$ & 4 & 1 & YES & YES & YES & $1.12$ & $(2,2)$ & NO & 1246\\
$(34,13)$ & 7 & $(5,1)$ & 4 & 1 & YES & YES & YES & $1.12$ & $(2,2)$ & -- & 1247\\
$(34,13)$ & 7 & $(5,2)$ & 3 & 1 & YES & YES & YES & $1.12$ & $(2,2)$ & NO & 1248\\
$(34,13)$ & 7 & $(5,2)$ & 3 & 1 & YES & YES & YES & $1.12$ & $(2,2)$ & -- & 1249\\
$(34,13)$ & 7 & $(7,2)$ & 4 & 1 & YES & YES & YES & $1.25$ & $(2,2)$ & NO & 1250\\
$(34,13)$ & 7 & $(7,2)$ & 4 & 1 & YES & YES & YES & $1.25$ & $(2,2)$ & -- & 1251\\
$(34,13)$ & 7 & $(7,3)$ & 4 & 1 & YES & YES & YES & $1.12$ & $(2,2)$ & -- & 1252\\
$(34,13)$ & 7 & $(7,3)$ & 4 & 1 & YES & YES & NO(2) & $1.22$ & $(2,2)$ & NO & 1253\\
$(34,15)$ & 8 & $(7,2)$ & 4 & 1 & YES & YES & YES & $1.00$ & $(2,2)$ & -- & 1254\\
$(34,13)$ & 7 & $(8,3)$ & 4 & 2 & YES & YES & YES & $1.12$ & $(2,2)$ & -- & 1255\\
$(34,15)$ & 8 & $(8,3)$ & 4 & 2 & YES & YES & YES & $1.12$ & $(2,2)$ & NO & 1256\\
$(34,9)$ & 8 & $(9,4)$ & 5 & 1 & YES & YES & YES & $1.12$ & $(2,2)$ & NO & 1257\\
$(34,9)$ & 8 & $(9,4)$ & 5 & 1 & YES & YES & YES & $1.12$ & $(2,2)$ & -- & 1258\\
$(34,13)$ & 7 & $(9,2)$ & 5 & 1 & YES & YES & YES & $1.12$ & $(2,2)$ & NO & 1259\\
$(34,13)$ & 7 & $(9,2)$ & 5 & 1 & YES & YES & YES & $1.12$ & $(2,2)$ & -- & 1260\\
$(34,13)$ & 7 & $(9,4)$ & 5 & 1 & YES & YES & YES & $1.12$ & $(2,2)$ & -- & 1261\\
$(34,15)$ & 8 & $(10,3)$ & 5 & 2 & YES & YES & YES & $1.12$ & $(2,2)$ & NO & 1262\\
$(34,9)$ & 8 & $(11,4)$ & 5 & 1 & YES & YES & YES & $1.12$ & $(2,2)$ & -- & 1263\\
$(34,13)$ & 7 & $(11,3)$ & 5 & 1 & YES & YES & YES & $1.00$ & $(2,2)$ & NO & 1264\\
$(34,13)$ & 7 & $(11,4)$ & 5 & 1 & YES & YES & YES & $1.00$ & $(2,2)$ & NO & 1265\\
$(34,15)$ & 8 & $(11,4)$ & 5 & 1 & YES & YES & YES & $1.12$ & $(2,2)$ & NO & 1266\\
$(34,13)$ & 7 & $(12,5)$ & 5 & 2 & YES & YES & YES & $1.25$ & $(2,2)$ & NO & 1267\\
$(34,15)$ & 8 & $(12,5)$ & 5 & 2 & YES & YES & YES & $1.12$ & $(2,2)$ & 2295 & 1268\\
$(34,13)$ & 7 & $(13,3)$ & 6 & 1 & YES & YES & YES & $1.12$ & $(2,2)$ & NO & 1269\\
$(34,13)$ & 7 & $(14,5)$ & 6 & 2 & YES & YES & YES & $1.00$ & $(2,2)$ & NO & 1270\\
$(34,9)$ & 8 & $(16,5)$ & 7 & 2 & YES & YES & YES & $1.12$ & $(2,2)$ & NO & 1271\\
$(34,9)$ & 8 & $(17,4)$ & 7 & 17 & YES & YES & YES & $1.00$ & $(2,2)$ & -- & 1272\\
$(34,13)$ & 7 & $(18,7)$ & 6 & 2 & YES & YES & YES & $1.12$ & $(2,2)$ & NO & 1273\\
$(34,13)$ & 7 & $(19,7)$ & 6 & 1 & YES & YES & YES & $1.12$ & $(2,2)$ & NO & 1274\\
$(34,13)$ & 7 & $(23,9)$ & 7 & 1 & YES & YES & YES & $1.00$ & $(2,2)$ & NO & 1275\\
$(34,9)$ & 8 & $(24,7)$ & 7 & 2 & YES & YES & YES & $1.12$ & $(2,2)$ & NO & 1276\\
$(34,13)$ & 7 & $(29,11)$ & 7 & 1 & YES & YES & YES & $1.12$ & $(2,2)$ & NO & 1277\\
$(34,13)$ & 7 & $(31,12)$ & 7 & 1 & YES & YES & YES & $1.12$ & $(2,2)$ & NO & 1278\\
$(35,8)$ & 8 & $(3,1)$ & 2 & 1 & YES & YES & YES & $1.00$ & $(2,2)$ & NO & 1279\\
$(35,8)$ & 8 & $(3,1)$ & 2 & 1 & YES & YES & YES & $1.25$ & $(2,2)$ & -- & 1280\\
$(35,8)$ & 8 & $(4,1)$ & 3 & 1 & YES & YES & YES & $1.12$ & $(2,2)$ & -- & 1281\\
$(35,8)$ & 8 & $(4,1)$ & 3 & 1 & YES & YES & YES & $1.25$ & $(2,2)$ & NO & 1282\\
$(35,8)$ & 8 & $(5,1)$ & 4 & 5 & YES & YES & YES & $1.25$ & $(2,2)$ & 1345 & 1283\\
$(35,8)$ & 8 & $(5,1)$ & 4 & 5 & YES & YES & YES & $1.25$ & $(2,2)$ & -- & 1284\\
$(35,8)$ & 8 & $(5,1)$ & 4 & 5 & YES & YES & YES & $1.25$ & $(2,2)$ & NO & 1285\\
$(35,16)$ & 9 & $(5,2)$ & 3 & 5 & YES & YES & YES & $1.25$ & $(2,2)$ & -- & 1286\\
$(35,8)$ & 8 & $(7,2)$ & 4 & 7 & YES & YES & YES & $1.25$ & $(2,2)$ & NO & 1287\\
$(35,8)$ & 8 & $(7,2)$ & 4 & 7 & YES & YES & YES & $1.25$ & $(2,2)$ & -- & 1288\\
$(35,8)$ & 8 & $(7,2)$ & 4 & 7 & YES & YES & YES & $1.25$ & $(2,2)$ & NO & 1289\\
$(35,13)$ & 8 & $(7,2)$ & 4 & 7 & YES & YES & YES & $1.25$ & $(2,2)$ & NO & 1290\\
$(35,13)$ & 8 & $(7,2)$ & 4 & 7 & YES & YES & YES & $1.25$ & $(2,2)$ & -- & 1291\\
$(35,16)$ & 9 & $(7,2)$ & 4 & 7 & YES & YES & YES & $1.12$ & $(2,2)$ & NO & 1292\\
$(35,16)$ & 9 & $(7,2)$ & 4 & 7 & YES & YES & YES & $1.25$ & $(2,2)$ & -- & 1293\\
$(35,16)$ & 9 & $(8,3)$ & 4 & 1 & YES & YES & YES & $1.12$ & $(2,2)$ & NO & 1294\\
$(35,8)$ & 8 & $(11,4)$ & 5 & 1 & YES & YES & YES & $1.12$ & $(2,2)$ & -- & 1295\\
$(35,8)$ & 8 & $(17,4)$ & 7 & 1 & YES & YES & YES & $1.00$ & $(2,2)$ & -- & 1296\\
$(35,8)$ & 8 & $(21,5)$ & 8 & 7 & YES & YES & YES & $1.12$ & $(2,2)$ & NO & 1297\\
$(35,13)$ & 8 & $(21,8)$ & 6 & 7 & YES & YES & YES & $1.12$ & $(2,2)$ & NO & 1298\\
$(35,8)$ & 8 & $(22,5)$ & 7 & 1 & YES & YES & YES & $1.00$ & $(2,2)$ & NO & 1299\\
$(35,8)$ & 8 & $(35,8)$ & 8 & 35 & YES & YES & YES & $1.12$ & $(2,2)$ & NO & 1300\\
$(36,11)$ & 8 & $(3,1)$ & 2 & 3 & YES & YES & NO(2) & $1.22$ & $(2,2)$ & NO & 1301\\
$(36,11)$ & 8 & $(3,1)$ & 2 & 3 & YES & YES & NO(2) & $1.22$ & $(2,2)$ & -- & 1302\\
$(36,11)$ & 8 & $(4,1)$ & 3 & 4 & YES & YES & NO(2) & $1.11$ & $(2,2)$ & NO & 1303\\
$(36,11)$ & 8 & $(4,1)$ & 3 & 4 & YES & YES & NO(2) & $1.11$ & $(2,2)$ & -- & 1304\\
$(36,11)$ & 8 & $(5,1)$ & 4 & 1 & YES & YES & YES & $1.12$ & $(2,2)$ & NO & 1305\\
$(36,11)$ & 8 & $(5,1)$ & 4 & 1 & YES & YES & YES & $1.12$ & $(2,2)$ & -- & 1306\\
$(36,11)$ & 8 & $(5,1)$ & 4 & 1 & YES & YES & YES & $1.12$ & $(2,2)$ & NO & 1307\\
$(36,11)$ & 8 & $(5,2)$ & 3 & 1 & YES & YES & YES & $0.88$ & $(2,2)$ & -- & 1308\\
$(36,11)$ & 8 & $(5,2)$ & 3 & 1 & YES & YES & YES & $1.25$ & $(2,2)$ & NO & 1309\\
$(36,13)$ & 8 & $(5,2)$ & 3 & 1 & YES & YES & YES & $1.12$ & $(2,2)$ & -- & 1310\\
$(36,11)$ & 8 & $(7,3)$ & 4 & 1 & YES & YES & YES & $1.12$ & $(2,2)$ & -- & 1311\\
$(36,11)$ & 8 & $(7,3)$ & 4 & 1 & YES & YES & YES & $1.12$ & $(2,2)$ & NO & 1312\\
$(36,13)$ & 8 & $(7,3)$ & 4 & 1 & YES & YES & YES & $1.12$ & $(2,2)$ & -- & 1313\\
$(36,13)$ & 8 & $(7,3)$ & 4 & 1 & YES & YES & YES & $1.12$ & $(2,2)$ & NO & 1314\\
$(36,13)$ & 8 & $(7,3)$ & 4 & 1 & YES & YES & YES & $1.12$ & $(2,2)$ & NO & 1315\\
$(36,11)$ & 8 & $(8,3)$ & 4 & 4 & YES & YES & YES & $1.25$ & $(2,2)$ & NO & 1316\\
$(36,11)$ & 8 & $(8,3)$ & 4 & 4 & YES & YES & YES & $1.25$ & $(2,2)$ & -- & 1317\\
$(36,11)$ & 8 & $(8,3)$ & 4 & 4 & YES & YES & YES & $0.88$ & $(2,2)$ & NO & 1318\\
$(36,13)$ & 8 & $(8,3)$ & 4 & 4 & YES & YES & YES & $1.12$ & $(2,2)$ & -- & 1319\\
$(36,13)$ & 8 & $(8,3)$ & 4 & 4 & YES & YES & YES & $1.25$ & $(2,2)$ & NO & 1320\\
$(36,11)$ & 8 & $(9,2)$ & 5 & 9 & YES & YES & YES & $1.00$ & $(2,2)$ & -- & 1321\\
$(36,11)$ & 8 & $(9,2)$ & 5 & 9 & YES & YES & YES & $1.12$ & $(2,2)$ & NO & 1322\\
$(36,11)$ & 8 & $(9,4)$ & 5 & 9 & YES & YES & YES & $1.12$ & $(2,2)$ & NO & 1323\\
$(36,11)$ & 8 & $(9,4)$ & 5 & 9 & YES & YES & YES & $1.12$ & $(2,2)$ & -- & 1324\\
$(36,13)$ & 8 & $(9,4)$ & 5 & 9 & YES & YES & YES & $1.12$ & $(2,2)$ & NO & 1325\\
$(36,13)$ & 8 & $(12,5)$ & 5 & 12 & YES & YES & YES & $1.12$ & $(2,2)$ & NO & 1326\\
$(36,13)$ & 8 & $(13,5)$ & 5 & 1 & YES & YES & YES & $0.88$ & $(2,2)$ & NO & 1327\\
$(36,11)$ & 8 & $(14,3)$ & 6 & 2 & YES & YES & YES & $1.00$ & $(2,2)$ & -- & 1328\\
$(36,11)$ & 8 & $(17,5)$ & 6 & 1 & YES & YES & YES & $1.00$ & $(2,2)$ & NO & 1329\\
$(36,11)$ & 8 & $(19,6)$ & 8 & 1 & YES & YES & YES & $1.00$ & $(2,2)$ & NO & 1330\\
$(36,13)$ & 8 & $(19,7)$ & 6 & 1 & YES & YES & YES & $1.12$ & $(2,2)$ & NO & 1331\\
$(36,11)$ & 8 & $(23,7)$ & 7 & 1 & YES & YES & NO(2) & $1.11$ & $(2,2)$ & NO & 1332\\
$(36,13)$ & 8 & $(27,10)$ & 7 & 9 & YES & YES & YES & $1.25$ & $(2,2)$ & NO & 1333\\
$(36,13)$ & 8 & $(30,11)$ & 7 & 6 & YES & YES & YES & $1.12$ & $(2,2)$ & NO & 1334\\
$(36,11)$ & 8 & $(33,10)$ & 8 & 3 & YES & YES & YES & $1.00$ & $(2,2)$ & NO & 1335\\
$(36,11)$ & 8 & $(36,11)$ & 8 & 36 & YES & YES & NO(2) & $1.22$ & $(2,2)$ & NO & 1336\\
$(37,8)$ & 8 & $(2,1)$ & 1 & 1 & YES & YES & YES & $1.00$ & $(2,2)$ & -- & 1337\\
$(37,8)$ & 8 & $(2,1)$ & 1 & 1 & YES & YES & YES & $1.12$ & $(2,2)$ & NO & 1338\\
$(37,8)$ & 8 & $(3,1)$ & 2 & 1 & YES & YES & YES & $1.00$ & $(2,2)$ & -- & 1339\\
$(37,8)$ & 8 & $(3,1)$ & 2 & 1 & YES & YES & YES & $1.12$ & $(2,2)$ & NO & 1340\\
$(37,10)$ & 8 & $(3,1)$ & 2 & 1 & YES & YES & YES & $1.00$ & $(2,2)$ & NO & 1341\\
$(37,10)$ & 8 & $(3,1)$ & 2 & 1 & YES & YES & YES & $1.00$ & $(2,2)$ & -- & 1342\\
$(37,8)$ & 8 & $(4,1)$ & 3 & 1 & YES & YES & YES & $1.00$ & $(2,2)$ & -- & 1343\\
$(37,8)$ & 8 & $(4,1)$ & 3 & 1 & YES & YES & YES & $1.12$ & $(2,2)$ & NO & 1344\\
$(37,8)$ & 8 & $(4,1)$ & 3 & 1 & YES & YES & YES & $1.25$ & $(2,2)$ & 1283 & 1345\\
$(37,14)$ & 8 & $(4,1)$ & 3 & 1 & YES & YES & YES & $1.25$ & $(2,2)$ & -- & 1346\\
$(37,14)$ & 8 & $(4,1)$ & 3 & 1 & YES & YES & YES & $1.38$ & $(2,2)$ & NO & 1347\\
$(37,11)$ & 8 & $(5,2)$ & 3 & 1 & YES & YES & YES & $1.25$ & $(2,2)$ & -- & 1348\\
$(37,11)$ & 8 & $(5,2)$ & 3 & 1 & YES & YES & YES & $1.25$ & $(2,2)$ & NO & 1349\\
$(37,13)$ & 9 & $(5,2)$ & 3 & 1 & YES & YES & YES & $1.12$ & $(2,2)$ & -- & 1350\\
$(37,14)$ & 8 & $(5,1)$ & 4 & 1 & YES & YES & NO(2) & $1.22$ & $(2,2)$ & NO & 1351\\
$(37,14)$ & 8 & $(5,1)$ & 4 & 1 & YES & YES & NO(2) & $1.22$ & $(2,2)$ & -- & 1352\\
$(37,14)$ & 8 & $(5,2)$ & 3 & 1 & YES & YES & YES & $1.25$ & $(2,2)$ & NO & 1353\\
$(37,14)$ & 8 & $(5,2)$ & 3 & 1 & YES & YES & YES & $1.25$ & $(2,2)$ & -- & 1354\\
$(37,16)$ & 9 & $(5,2)$ & 3 & 1 & YES & YES & YES & $1.25$ & $(2,2)$ & -- & 1355\\
$(37,10)$ & 8 & $(7,3)$ & 4 & 1 & YES & YES & YES & $1.00$ & $(2,2)$ & -- & 1356\\
$(37,11)$ & 8 & $(7,3)$ & 4 & 1 & YES & YES & YES & $1.12$ & $(2,2)$ & -- & 1357\\
$(37,14)$ & 8 & $(7,2)$ & 4 & 1 & YES & YES & YES & $1.00$ & $(2,2)$ & -- & 1358\\
$(37,14)$ & 8 & $(7,2)$ & 4 & 1 & YES & YES & YES & $1.25$ & $(2,2)$ & NO & 1359\\
$(37,16)$ & 9 & $(7,2)$ & 4 & 1 & YES & YES & YES & $1.12$ & $(2,2)$ & NO & 1360\\
$(37,10)$ & 8 & $(8,3)$ & 4 & 1 & YES & YES & YES & $1.00$ & $(2,2)$ & -- & 1361\\
$(37,11)$ & 8 & $(8,3)$ & 4 & 1 & YES & YES & YES & $0.88$ & $(2,2)$ & NO & 1362\\
$(37,8)$ & 8 & $(9,2)$ & 5 & 1 & YES & YES & YES & $1.00$ & $(2,2)$ & 1212 & 1363\\
$(37,8)$ & 8 & $(9,2)$ & 5 & 1 & YES & YES & YES & $1.00$ & $(2,2)$ & -- & 1364\\
$(37,10)$ & 8 & $(9,4)$ & 5 & 1 & YES & YES & YES & $1.25$ & $(2,2)$ & 851 & 1365\\
$(37,10)$ & 8 & $(9,4)$ & 5 & 1 & YES & YES & YES & $1.38$ & $(2,2)$ & -- & 1366\\
$(37,14)$ & 8 & $(9,2)$ & 5 & 1 & YES & YES & YES & $1.00$ & $(2,2)$ & -- & 1367\\
$(37,14)$ & 8 & $(9,2)$ & 5 & 1 & YES & YES & YES & $1.00$ & $(2,2)$ & NO & 1368\\
$(37,16)$ & 9 & $(9,2)$ & 5 & 1 & YES & YES & YES & $1.12$ & $(2,2)$ & -- & 1369\\
$(37,16)$ & 9 & $(9,2)$ & 5 & 1 & YES & YES & YES & $1.25$ & $(2,2)$ & NO & 1370\\
$(37,10)$ & 8 & $(10,3)$ & 5 & 1 & YES & YES & YES & $0.88$ & $(2,2)$ & -- & 1371\\
$(37,8)$ & 8 & $(11,4)$ & 5 & 1 & YES & YES & YES & $1.00$ & $(2,2)$ & -- & 1372\\
$(37,10)$ & 8 & $(11,4)$ & 5 & 1 & YES & YES & YES & $1.25$ & $(2,2)$ & NO & 1373\\
$(37,14)$ & 8 & $(11,2)$ & 6 & 1 & YES & YES & YES & $1.12$ & $(2,2)$ & -- & 1374\\
$(37,14)$ & 8 & $(11,2)$ & 6 & 1 & YES & YES & YES & $1.12$ & $(2,2)$ & NO & 1375\\
$(37,16)$ & 9 & $(11,2)$ & 6 & 1 & YES & YES & YES & $1.12$ & $(2,2)$ & NO & 1376\\
$(37,16)$ & 9 & $(11,2)$ & 6 & 1 & YES & YES & YES & $1.25$ & $(2,2)$ & -- & 1377\\
$(37,16)$ & 9 & $(11,2)$ & 6 & 1 & YES & YES & YES & $1.25$ & $(2,2)$ & NO & 1378\\
$(37,8)$ & 8 & $(13,4)$ & 6 & 1 & YES & YES & YES & $1.00$ & $(2,2)$ & -- & 1379\\
$(37,11)$ & 8 & $(13,3)$ & 6 & 1 & YES & YES & YES & $1.25$ & $(2,2)$ & NO & 1380\\
$(37,8)$ & 8 & $(15,4)$ & 6 & 1 & YES & YES & YES & $0.88$ & $(2,2)$ & -- & 1381\\
$(37,10)$ & 8 & $(15,4)$ & 6 & 1 & YES & YES & YES & $1.00$ & $(2,2)$ & 1528 & 1382\\
$(37,11)$ & 8 & $(15,4)$ & 6 & 1 & YES & YES & YES & $1.12$ & $(2,2)$ & NO & 1383\\
$(37,11)$ & 8 & $(18,5)$ & 6 & 1 & YES & YES & YES & $1.12$ & $(2,2)$ & NO & 1384\\
$(37,14)$ & 8 & $(18,7)$ & 6 & 1 & YES & YES & YES & $1.25$ & $(2,2)$ & NO & 1385\\
$(37,14)$ & 8 & $(21,8)$ & 6 & 1 & YES & YES & YES & $1.12$ & $(2,2)$ & 1877 & 1386\\
$(37,8)$ & 8 & $(23,5)$ & 7 & 1 & YES & YES & YES & $1.00$ & $(2,2)$ & NO & 1387\\
$(37,11)$ & 8 & $(23,7)$ & 7 & 1 & YES & YES & YES & $1.00$ & $(2,2)$ & NO & 1388\\
$(37,14)$ & 8 & $(29,11)$ & 7 & 1 & YES & YES & NO(2) & $1.22$ & $(2,2)$ & NO & 1389\\
$(37,11)$ & 8 & $(31,9)$ & 8 & 1 & YES & YES & YES & $1.25$ & $(2,2)$ & NO & 1390\\
$(37,14)$ & 8 & $(34,13)$ & 7 & 1 & YES & YES & YES & $1.12$ & $(2,2)$ & 2420 & 1391\\
$(37,8)$ & 8 & $(37,8)$ & 8 & 37 & YES & YES & YES & $1.00$ & $(2,2)$ & NO & 1392\\
$(38,11)$ & 9 & $(4,1)$ & 3 & 2 & YES & YES & YES & $1.12$ & $(2,2)$ & NO & 1393\\
$(38,9)$ & 9 & $(5,2)$ & 3 & 1 & YES & YES & YES & $1.38$ & $(2,2)$ & NO & 1394\\
$(38,9)$ & 9 & $(5,2)$ & 3 & 1 & YES & YES & YES & $1.25$ & $(2,2)$ & -- & 1395\\
$(38,11)$ & 9 & $(5,2)$ & 3 & 1 & YES & YES & YES & $1.12$ & $(2,2)$ & -- & 1396\\
$(38,11)$ & 9 & $(7,3)$ & 4 & 1 & YES & YES & YES & $1.12$ & $(2,2)$ & -- & 1397\\
$(38,9)$ & 9 & $(9,4)$ & 5 & 1 & YES & YES & YES & $1.25$ & $(2,2)$ & -- & 1398\\
$(38,9)$ & 9 & $(9,4)$ & 5 & 1 & YES & YES & YES & $1.25$ & $(2,2)$ & NO & 1399\\
$(38,11)$ & 9 & $(9,2)$ & 5 & 1 & YES & YES & YES & $1.00$ & $(2,2)$ & -- & 1400\\
$(38,11)$ & 9 & $(9,2)$ & 5 & 1 & YES & YES & YES & $1.12$ & $(2,2)$ & NO & 1401\\
$(38,9)$ & 9 & $(11,4)$ & 5 & 1 & YES & YES & YES & $1.25$ & $(2,2)$ & -- & 1402\\
$(38,9)$ & 9 & $(12,5)$ & 5 & 2 & YES & YES & YES & $1.25$ & $(2,2)$ & NO & 1403\\
$(38,9)$ & 9 & $(12,5)$ & 5 & 2 & YES & YES & YES & $1.25$ & $(2,2)$ & -- & 1404\\
$(38,11)$ & 9 & $(15,4)$ & 6 & 1 & YES & YES & YES & $1.12$ & $(2,2)$ & NO & 1405\\
$(38,9)$ & 9 & $(30,7)$ & 8 & 2 & YES & YES & YES & $1.12$ & $(2,2)$ & NO & 1406\\
$(39,11)$ & 9 & $(4,1)$ & 3 & 1 & YES & YES & YES & $1.00$ & $(2,2)$ & NO & 1407\\
$(39,11)$ & 9 & $(5,2)$ & 3 & 1 & YES & YES & YES & $1.00$ & $(2,2)$ & -- & 1408\\
$(39,16)$ & 8 & $(5,2)$ & 3 & 1 & YES & YES & YES & $1.12$ & $(2,2)$ & -- & 1409\\
$(39,17)$ & 8 & $(5,1)$ & 4 & 1 & YES & YES & YES & $1.12$ & $(2,2)$ & NO & 1410\\
$(39,17)$ & 8 & $(5,1)$ & 4 & 1 & YES & YES & YES & $1.12$ & $(2,2)$ & -- & 1411\\
$(39,17)$ & 8 & $(5,1)$ & 4 & 1 & YES & YES & YES & $1.38$ & $(2,2)$ & NO & 1412\\
$(39,17)$ & 8 & $(5,2)$ & 3 & 1 & YES & YES & YES & $1.00$ & $(2,2)$ & -- & 1413\\
$(39,17)$ & 8 & $(5,2)$ & 3 & 1 & YES & YES & YES & $1.12$ & $(2,2)$ & NO & 1414\\
$(39,14)$ & 8 & $(7,2)$ & 4 & 1 & YES & YES & YES & $1.00$ & $(2,2)$ & NO & 1415\\
$(39,14)$ & 8 & $(7,2)$ & 4 & 1 & YES & YES & YES & $1.00$ & $(2,2)$ & -- & 1416\\
$(39,14)$ & 8 & $(7,3)$ & 4 & 1 & YES & YES & YES & $1.12$ & $(2,2)$ & -- & 1417\\
$(39,14)$ & 8 & $(7,3)$ & 4 & 1 & YES & YES & YES & $1.25$ & $(2,2)$ & NO & 1418\\
$(39,14)$ & 8 & $(7,3)$ & 4 & 1 & YES & YES & YES & $1.12$ & $(2,2)$ & NO & 1419\\
$(39,16)$ & 8 & $(7,2)$ & 4 & 1 & YES & YES & YES & $0.88$ & $(2,2)$ & -- & 1420\\
$(39,16)$ & 8 & $(7,2)$ & 4 & 1 & YES & YES & YES & $1.12$ & $(2,2)$ & NO & 1421\\
$(39,17)$ & 8 & $(7,2)$ & 4 & 1 & YES & YES & YES & $1.25$ & $(2,2)$ & -- & 1422\\
$(39,17)$ & 8 & $(7,3)$ & 4 & 1 & YES & YES & YES & $1.12$ & $(2,2)$ & -- & 1423\\
$(39,16)$ & 8 & $(8,3)$ & 4 & 1 & YES & YES & YES & $1.25$ & $(2,2)$ & NO & 1424\\
$(39,16)$ & 8 & $(8,3)$ & 4 & 1 & YES & YES & YES & $1.25$ & $(2,2)$ & -- & 1425\\
$(39,17)$ & 8 & $(8,3)$ & 4 & 1 & YES & YES & YES & $1.12$ & $(2,2)$ & -- & 1426\\
$(39,17)$ & 8 & $(8,3)$ & 4 & 1 & YES & YES & YES & $1.00$ & $(2,2)$ & NO & 1427\\
$(39,14)$ & 8 & $(9,2)$ & 5 & 3 & YES & YES & YES & $1.00$ & $(2,2)$ & -- & 1428\\
$(39,14)$ & 8 & $(9,4)$ & 5 & 3 & YES & YES & YES & $1.12$ & $(2,2)$ & NO & 1429\\
$(39,16)$ & 8 & $(9,2)$ & 5 & 3 & YES & YES & YES & $1.00$ & $(2,2)$ & NO & 1430\\
$(39,16)$ & 8 & $(9,4)$ & 5 & 3 & YES & YES & YES & $1.12$ & $(2,2)$ & NO & 1431\\
$(39,17)$ & 8 & $(11,5)$ & 6 & 1 & YES & YES & YES & $1.12$ & $(2,2)$ & NO & 1432\\
$(39,14)$ & 8 & $(12,5)$ & 5 & 3 & YES & YES & YES & $1.12$ & $(2,2)$ & NO & 1433\\
$(39,14)$ & 8 & $(13,3)$ & 6 & 13 & YES & YES & YES & $1.12$ & $(2,2)$ & -- & 1434\\
$(39,14)$ & 8 & $(13,5)$ & 5 & 13 & YES & YES & YES & $1.00$ & $(2,2)$ & 2320 & 1435\\
$(39,16)$ & 8 & $(13,3)$ & 6 & 13 & YES & YES & YES & $1.25$ & $(2,2)$ & -- & 1436\\
$(39,16)$ & 8 & $(13,3)$ & 6 & 13 & YES & YES & YES & $1.25$ & $(2,2)$ & NO & 1437\\
$(39,17)$ & 8 & $(13,3)$ & 6 & 13 & YES & YES & YES & $1.25$ & $(2,2)$ & NO & 1438\\
$(39,17)$ & 8 & $(13,3)$ & 6 & 13 & YES & YES & YES & $1.25$ & $(2,2)$ & -- & 1439\\
$(39,16)$ & 8 & $(16,7)$ & 6 & 1 & YES & YES & YES & $1.25$ & $(2,2)$ & NO & 1440\\
$(39,16)$ & 8 & $(19,8)$ & 6 & 1 & YES & YES & YES & $1.12$ & $(2,2)$ & NO & 1441\\
$(39,17)$ & 8 & $(19,8)$ & 6 & 1 & YES & YES & YES & $1.12$ & $(2,2)$ & NO & 1442\\
$(39,17)$ & 8 & $(25,11)$ & 7 & 1 & YES & YES & YES & $1.00$ & $(2,2)$ & NO & 1443\\
$(39,16)$ & 8 & $(29,12)$ & 7 & 1 & YES & YES & YES & $1.00$ & $(2,2)$ & NO & 1444\\
$(39,17)$ & 8 & $(37,16)$ & 9 & 1 & YES & YES & YES & $1.12$ & $(2,2)$ & 2451 & 1445\\
$(40,11)$ & 8 & $(5,2)$ & 3 & 5 & YES & YES & YES & $0.75$ & $(2,2)$ & -- & 1446\\
$(40,9)$ & 9 & $(7,3)$ & 4 & 1 & YES & YES & YES & $1.25$ & $(2,2)$ & NO & 1447\\
$(40,11)$ & 8 & $(7,3)$ & 4 & 1 & YES & YES & YES & $1.00$ & $(2,2)$ & NO & 1448\\
$(40,11)$ & 8 & $(7,3)$ & 4 & 1 & YES & YES & YES & $1.12$ & $(2,2)$ & -- & 1449\\
$(40,9)$ & 9 & $(8,3)$ & 4 & 8 & YES & YES & YES & $0.88$ & $(2,2)$ & NO & 1450\\
$(40,9)$ & 9 & $(10,3)$ & 5 & 10 & YES & YES & YES & $1.12$ & $(2,2)$ & NO & 1451\\
$(40,9)$ & 9 & $(10,3)$ & 5 & 10 & YES & YES & YES & $1.12$ & $(2,2)$ & -- & 1452\\
$(40,9)$ & 9 & $(11,3)$ & 5 & 1 & YES & YES & YES & $1.12$ & $(2,2)$ & NO & 1453\\
$(40,9)$ & 9 & $(16,3)$ & 7 & 8 & YES & YES & YES & $1.25$ & $(2,2)$ & NO & 1454\\
$(40,9)$ & 9 & $(17,3)$ & 7 & 1 & YES & YES & YES & $1.00$ & $(2,2)$ & NO & 1455\\
$(40,9)$ & 9 & $(24,5)$ & 8 & 8 & YES & YES & YES & $1.12$ & $(2,2)$ & NO & 1456\\
$(40,11)$ & 8 & $(25,7)$ & 7 & 5 & YES & YES & YES & $0.88$ & $(2,2)$ & 2503 & 1457\\
$(40,9)$ & 9 & $(32,7)$ & 8 & 8 & YES & YES & YES & $1.12$ & $(2,2)$ & NO & 1458\\
$(41,11)$ & 8 & $(2,1)$ & 1 & 1 & YES & YES & YES & $1.00$ & $(2,2)$ & -- & 1459\\
$(41,11)$ & 8 & $(2,1)$ & 1 & 1 & YES & YES & YES & $1.12$ & $(2,2)$ & NO & 1460\\
$(41,11)$ & 8 & $(3,1)$ & 2 & 1 & YES & YES & YES & $1.00$ & $(2,2)$ & NO & 1461\\
$(41,11)$ & 8 & $(3,1)$ & 2 & 1 & YES & YES & YES & $1.00$ & $(2,2)$ & -- & 1462\\
$(41,11)$ & 8 & $(3,1)$ & 2 & 1 & YES & YES & YES & $1.25$ & $(2,2)$ & NO & 1463\\
$(41,15)$ & 8 & $(3,1)$ & 2 & 1 & YES & YES & YES & $1.25$ & $(2,2)$ & -- & 1464\\
$(41,15)$ & 8 & $(3,1)$ & 2 & 1 & YES & YES & YES & $1.12$ & $(2,2)$ & NO & 1465\\
$(41,16)$ & 8 & $(3,1)$ & 2 & 1 & YES & YES & YES & $1.11$ & $(2,2)$ & -- & 1466\\
$(41,16)$ & 8 & $(3,1)$ & 2 & 1 & YES & YES & YES & $1.12$ & $(2,2)$ & NO & 1467\\
$(41,17)$ & 8 & $(3,1)$ & 2 & 1 & YES & YES & YES & $1.12$ & $(2,2)$ & -- & 1468\\
$(41,17)$ & 8 & $(3,1)$ & 2 & 1 & YES & YES & YES & $1.12$ & $(2,2)$ & NO & 1469\\
$(41,11)$ & 8 & $(4,1)$ & 3 & 1 & YES & YES & NO(2) & $1.22$ & $(2,2)$ & NO & 1470\\
$(41,11)$ & 8 & $(4,1)$ & 3 & 1 & YES & YES & NO(2) & $1.22$ & $(2,2)$ & -- & 1471\\
$(41,12)$ & 8 & $(4,1)$ & 3 & 1 & YES & YES & YES & $1.12$ & $(2,2)$ & NO & 1472\\
$(41,12)$ & 8 & $(4,1)$ & 3 & 1 & YES & YES & YES & $1.12$ & $(2,2)$ & NO & 1473\\
$(41,12)$ & 8 & $(4,1)$ & 3 & 1 & YES & YES & YES & $1.12$ & $(2,2)$ & -- & 1474\\
$(41,15)$ & 8 & $(4,1)$ & 3 & 1 & YES & YES & YES & $1.12$ & $(2,2)$ & NO & 1475\\
$(41,15)$ & 8 & $(4,1)$ & 3 & 1 & YES & YES & YES & $1.25$ & $(2,2)$ & -- & 1476\\
$(41,16)$ & 8 & $(4,1)$ & 3 & 1 & YES & YES & YES & $1.00$ & $(2,2)$ & -- & 1477\\
$(41,16)$ & 8 & $(4,1)$ & 3 & 1 & YES & YES & YES & $1.12$ & $(2,2)$ & NO & 1478\\
$(41,17)$ & 8 & $(4,1)$ & 3 & 1 & YES & YES & YES & $1.12$ & $(2,2)$ & NO & 1479\\
$(41,17)$ & 8 & $(4,1)$ & 3 & 1 & YES & YES & YES & $1.12$ & $(2,2)$ & -- & 1480\\
$(41,17)$ & 8 & $(4,1)$ & 3 & 1 & YES & YES & YES & $1.12$ & $(2,2)$ & NO & 1481\\
$(41,11)$ & 8 & $(5,2)$ & 3 & 1 & YES & YES & YES & $0.88$ & $(2,2)$ & -- & 1482\\
$(41,12)$ & 8 & $(5,2)$ & 3 & 1 & YES & YES & YES & $1.12$ & $(2,2)$ & NO & 1483\\
$(41,12)$ & 8 & $(5,2)$ & 3 & 1 & YES & YES & YES & $1.25$ & $(2,2)$ & -- & 1484\\
$(41,15)$ & 8 & $(5,1)$ & 4 & 1 & YES & YES & YES & $1.12$ & $(2,2)$ & NO & 1485\\
$(41,15)$ & 8 & $(5,1)$ & 4 & 1 & YES & YES & YES & $1.12$ & $(2,2)$ & -- & 1486\\
$(41,15)$ & 8 & $(5,2)$ & 3 & 1 & YES & YES & YES & $1.12$ & $(2,2)$ & NO & 1487\\
$(41,15)$ & 8 & $(5,2)$ & 3 & 1 & YES & YES & YES & $1.12$ & $(2,2)$ & -- & 1488\\
$(41,16)$ & 8 & $(5,2)$ & 3 & 1 & YES & YES & YES & $1.12$ & $(2,2)$ & -- & 1489\\
$(41,16)$ & 8 & $(5,2)$ & 3 & 1 & YES & YES & YES & $1.12$ & $(2,2)$ & NO & 1490\\
$(41,17)$ & 8 & $(5,2)$ & 3 & 1 & YES & YES & YES & $1.00$ & $(2,2)$ & -- & 1491\\
$(41,17)$ & 8 & $(5,2)$ & 3 & 1 & YES & YES & YES & $1.12$ & $(2,2)$ & NO & 1492\\
$(41,18)$ & 8 & $(5,2)$ & 3 & 1 & YES & YES & YES & $1.12$ & $(2,2)$ & -- & 1493\\
$(41,18)$ & 8 & $(5,2)$ & 3 & 1 & YES & YES & YES & $1.12$ & $(2,2)$ & NO & 1494\\
$(41,11)$ & 8 & $(7,3)$ & 4 & 1 & YES & YES & YES & $1.00$ & $(2,2)$ & -- & 1495\\
$(41,12)$ & 8 & $(7,3)$ & 4 & 1 & YES & YES & YES & $1.12$ & $(2,2)$ & -- & 1496\\
$(41,12)$ & 8 & $(7,3)$ & 4 & 1 & YES & YES & YES & $1.12$ & $(2,2)$ & NO & 1497\\
$(41,15)$ & 8 & $(7,2)$ & 4 & 1 & YES & YES & YES & $1.12$ & $(2,2)$ & NO & 1498\\
$(41,15)$ & 8 & $(7,2)$ & 4 & 1 & YES & YES & YES & $1.12$ & $(2,2)$ & -- & 1499\\
$(41,15)$ & 8 & $(7,3)$ & 4 & 1 & YES & YES & YES & $1.25$ & $(2,2)$ & NO & 1500\\
$(41,15)$ & 8 & $(7,3)$ & 4 & 1 & YES & YES & YES & $1.25$ & $(2,2)$ & -- & 1501\\
$(41,15)$ & 8 & $(7,3)$ & 4 & 1 & YES & YES & YES & $1.12$ & $(2,2)$ & NO & 1502\\
$(41,16)$ & 8 & $(7,2)$ & 4 & 1 & YES & YES & YES & $0.88$ & $(2,2)$ & -- & 1503\\
$(41,16)$ & 8 & $(7,2)$ & 4 & 1 & YES & YES & YES & $1.12$ & $(2,2)$ & 2705 & 1504\\
$(41,16)$ & 8 & $(7,2)$ & 4 & 1 & YES & YES & YES & $1.00$ & $(2,2)$ & NO & 1505\\
$(41,17)$ & 8 & $(7,3)$ & 4 & 1 & YES & YES & YES & $1.25$ & $(2,2)$ & -- & 1506\\
$(41,18)$ & 8 & $(7,2)$ & 4 & 1 & YES & YES & YES & $1.00$ & $(2,2)$ & NO & 1507\\
$(41,11)$ & 8 & $(8,3)$ & 4 & 1 & YES & YES & YES & $1.00$ & $(2,2)$ & -- & 1508\\
$(41,12)$ & 8 & $(8,3)$ & 4 & 1 & YES & YES & YES & $1.12$ & $(2,2)$ & NO & 1509\\
$(41,12)$ & 8 & $(8,3)$ & 4 & 1 & YES & YES & YES & $1.12$ & $(2,2)$ & -- & 1510\\
$(41,17)$ & 8 & $(8,3)$ & 4 & 1 & YES & YES & NO(2) & $1.22$ & $(2,2)$ & NO & 1511\\
$(41,18)$ & 8 & $(8,3)$ & 4 & 1 & YES & YES & YES & $1.00$ & $(2,2)$ & NO & 1512\\
$(41,11)$ & 8 & $(9,4)$ & 5 & 1 & YES & YES & YES & $1.25$ & $(2,2)$ & NO & 1513\\
$(41,12)$ & 8 & $(9,4)$ & 5 & 1 & YES & YES & YES & $1.25$ & $(2,2)$ & NO & 1514\\
$(41,15)$ & 8 & $(9,2)$ & 5 & 1 & YES & YES & YES & $1.00$ & $(2,2)$ & -- & 1515\\
$(41,15)$ & 8 & $(9,2)$ & 5 & 1 & YES & YES & YES & $1.00$ & $(2,2)$ & NO & 1516\\
$(41,15)$ & 8 & $(9,4)$ & 5 & 1 & YES & YES & YES & $1.12$ & $(2,2)$ & NO & 1517\\
$(41,16)$ & 8 & $(9,2)$ & 5 & 1 & YES & YES & YES & $1.12$ & $(2,2)$ & NO & 1518\\
$(41,16)$ & 8 & $(9,2)$ & 5 & 1 & YES & YES & YES & $1.00$ & $(2,2)$ & -- & 1519\\
$(41,17)$ & 8 & $(9,2)$ & 5 & 1 & YES & YES & YES & $1.00$ & $(2,2)$ & NO & 1520\\
$(41,17)$ & 8 & $(9,2)$ & 5 & 1 & YES & YES & YES & $1.00$ & $(2,2)$ & -- & 1521\\
$(41,17)$ & 8 & $(9,2)$ & 5 & 1 & YES & YES & YES & $1.12$ & $(2,2)$ & NO & 1522\\
$(41,17)$ & 8 & $(9,4)$ & 5 & 1 & YES & YES & YES & $1.00$ & $(2,2)$ & NO & 1523\\
$(41,11)$ & 8 & $(10,3)$ & 5 & 1 & YES & YES & YES & $1.00$ & $(2,2)$ & -- & 1524\\
$(41,12)$ & 8 & $(10,3)$ & 5 & 1 & YES & YES & YES & $1.00$ & $(2,2)$ & -- & 1525\\
$(41,15)$ & 8 & $(10,3)$ & 5 & 1 & YES & YES & YES & $1.12$ & $(2,2)$ & NO & 1526\\
$(41,11)$ & 8 & $(11,3)$ & 5 & 1 & YES & YES & YES & $1.00$ & $(2,2)$ & -- & 1527\\
$(41,11)$ & 8 & $(11,3)$ & 5 & 1 & YES & YES & YES & $1.00$ & $(2,2)$ & 1382 & 1528\\
$(41,11)$ & 8 & $(11,4)$ & 5 & 1 & YES & YES & YES & $1.25$ & $(2,2)$ & NO & 1529\\
$(41,16)$ & 8 & $(11,4)$ & 5 & 1 & YES & YES & YES & $1.12$ & $(2,2)$ & NO & 1530\\
$(41,12)$ & 8 & $(13,3)$ & 6 & 1 & YES & YES & YES & $1.12$ & $(2,2)$ & NO & 1531\\
$(41,12)$ & 8 & $(13,4)$ & 6 & 1 & YES & YES & YES & $1.12$ & $(2,2)$ & NO & 1532\\
$(41,15)$ & 8 & $(13,5)$ & 5 & 1 & YES & YES & YES & $1.12$ & $(2,2)$ & 2548 & 1533\\
$(41,16)$ & 8 & $(14,5)$ & 6 & 1 & YES & YES & YES & $1.25$ & $(2,2)$ & NO & 1534\\
$(41,12)$ & 8 & $(15,4)$ & 6 & 1 & YES & YES & YES & $1.12$ & $(2,2)$ & NO & 1535\\
$(41,17)$ & 8 & $(17,7)$ & 6 & 1 & YES & YES & YES & $1.12$ & $(2,2)$ & 1704 & 1536\\
$(41,15)$ & 8 & $(19,7)$ & 6 & 1 & YES & YES & YES & $1.12$ & $(2,2)$ & 1828 & 1537\\
$(41,16)$ & 8 & $(21,8)$ & 6 & 1 & YES & YES & YES & $1.00$ & $(2,2)$ & NO & 1538\\
$(41,17)$ & 8 & $(22,9)$ & 7 & 1 & YES & YES & YES & $1.12$ & $(2,2)$ & 2519 & 1539\\
$(41,12)$ & 8 & $(23,7)$ & 7 & 1 & YES & YES & YES & $1.12$ & $(2,2)$ & NO & 1540\\
$(41,16)$ & 8 & $(23,9)$ & 7 & 1 & YES & YES & YES & $1.22$ & $(2,2)$ & NO & 1541\\
$(41,11)$ & 8 & $(26,7)$ & 7 & 1 & YES & YES & YES & $0.88$ & $(2,2)$ & NO & 1542\\
$(41,12)$ & 8 & $(27,8)$ & 7 & 1 & YES & YES & YES & $1.12$ & $(2,2)$ & NO & 1543\\
$(41,15)$ & 8 & $(27,10)$ & 7 & 1 & YES & YES & YES & $1.25$ & $(2,2)$ & 2582 & 1544\\
$(41,16)$ & 8 & $(28,11)$ & 8 & 1 & YES & YES & YES & $1.12$ & $(2,2)$ & NO & 1545\\
$(41,12)$ & 8 & $(31,9)$ & 8 & 1 & YES & YES & YES & $1.12$ & $(2,2)$ & NO & 1546\\
$(41,16)$ & 8 & $(31,12)$ & 7 & 1 & YES & YES & YES & $1.12$ & $(2,2)$ & NO & 1547\\
$(41,18)$ & 8 & $(34,15)$ & 8 & 1 & YES & YES & YES & $1.12$ & $(2,2)$ & NO & 1548\\
$(41,17)$ & 8 & $(39,16)$ & 8 & 1 & YES & YES & YES & $1.25$ & $(2,2)$ & NO & 1549\\
$(41,11)$ & 8 & $(41,11)$ & 8 & 41 & YES & YES & YES & $1.12$ & $(2,2)$ & NO & 1550\\
$(41,15)$ & 8 & $(41,15)$ & 8 & 41 & YES & YES & YES & $1.25$ & $(2,2)$ & NO & 1551\\
$(42,13)$ & 9 & $(4,1)$ & 3 & 2 & YES & YES & YES & $1.12$ & $(2,2)$ & NO & 1552\\
$(42,13)$ & 9 & $(4,1)$ & 3 & 2 & YES & YES & YES & $1.12$ & $(2,2)$ & -- & 1553\\
$(42,11)$ & 9 & $(5,2)$ & 3 & 1 & YES & YES & YES & $1.00$ & $(2,2)$ & -- & 1554\\
$(42,13)$ & 9 & $(5,2)$ & 3 & 1 & YES & YES & YES & $1.12$ & $(2,2)$ & -- & 1555\\
$(42,11)$ & 9 & $(7,2)$ & 4 & 7 & YES & YES & YES & $0.88$ & $(2,2)$ & -- & 1556\\
$(42,13)$ & 9 & $(7,2)$ & 4 & 7 & YES & YES & YES & $1.12$ & $(2,2)$ & NO & 1557\\
$(42,13)$ & 9 & $(10,3)$ & 5 & 2 & YES & YES & YES & $1.12$ & $(2,2)$ & NO & 1558\\
$(42,13)$ & 9 & $(19,6)$ & 8 & 1 & YES & YES & YES & $1.12$ & $(2,2)$ & 2545 & 1559\\
$(42,13)$ & 9 & $(23,7)$ & 7 & 1 & YES & YES & YES & $1.12$ & $(2,2)$ & NO & 1560\\
$(42,13)$ & 9 & $(29,9)$ & 8 & 1 & YES & YES & YES & $1.12$ & $(2,2)$ & NO & 1561\\
$(42,11)$ & 9 & $(34,9)$ & 8 & 2 & YES & YES & YES & $1.00$ & $(2,2)$ & NO & 1562\\
$(42,13)$ & 9 & $(36,11)$ & 8 & 6 & YES & YES & YES & $1.12$ & $(2,2)$ & NO & 1563\\
$(42,13)$ & 9 & $(42,13)$ & 9 & 42 & YES & YES & YES & $1.12$ & $(2,2)$ & NO & 1564\\
$(43,12)$ & 8 & $(2,1)$ & 1 & 1 & YES & YES & YES & $1.12$ & $(2,2)$ & NO & 1565\\
$(43,10)$ & 9 & $(4,1)$ & 3 & 1 & YES & YES & YES & $1.25$ & $(2,2)$ & -- & 1566\\
$(43,10)$ & 9 & $(4,1)$ & 3 & 1 & YES & YES & YES & $1.38$ & $(2,2)$ & NO & 1567\\
$(43,12)$ & 8 & $(4,1)$ & 3 & 1 & YES & YES & YES & $1.00$ & $(2,2)$ & NO & 1568\\
$(43,12)$ & 8 & $(4,1)$ & 3 & 1 & YES & YES & YES & $1.00$ & $(2,2)$ & -- & 1569\\
$(43,12)$ & 8 & $(4,1)$ & 3 & 1 & YES & YES & YES & $1.12$ & $(2,2)$ & 915 & 1570\\
$(43,16)$ & 9 & $(4,1)$ & 3 & 1 & YES & YES & YES & $1.25$ & $(2,2)$ & -- & 1571\\
$(43,16)$ & 9 & $(4,1)$ & 3 & 1 & YES & YES & YES & $1.38$ & $(2,2)$ & NO & 1572\\
$(43,18)$ & 8 & $(4,1)$ & 3 & 1 & YES & YES & YES & $1.12$ & $(2,2)$ & -- & 1573\\
$(43,18)$ & 8 & $(4,1)$ & 3 & 1 & YES & YES & YES & $1.12$ & $(2,2)$ & NO & 1574\\
$(43,12)$ & 8 & $(5,2)$ & 3 & 1 & YES & YES & YES & $1.12$ & $(2,2)$ & NO & 1575\\
$(43,12)$ & 8 & $(5,2)$ & 3 & 1 & YES & YES & YES & $1.12$ & $(2,2)$ & -- & 1576\\
$(43,13)$ & 9 & $(5,2)$ & 3 & 1 & YES & YES & YES & $1.12$ & $(2,2)$ & -- & 1577\\
$(43,13)$ & 9 & $(5,2)$ & 3 & 1 & YES & YES & YES & $1.25$ & $(2,2)$ & NO & 1578\\
$(43,18)$ & 8 & $(5,2)$ & 3 & 1 & YES & YES & NO(2) & $1.00$ & $(4,1)$ & -- & 1579\\
$(43,10)$ & 9 & $(7,2)$ & 4 & 1 & YES & YES & YES & $1.00$ & $(2,2)$ & -- & 1580\\
$(43,10)$ & 9 & $(7,2)$ & 4 & 1 & YES & YES & YES & $1.12$ & $(2,2)$ & NO & 1581\\
$(43,10)$ & 9 & $(7,3)$ & 4 & 1 & YES & YES & YES & $1.12$ & $(2,2)$ & -- & 1582\\
$(43,10)$ & 9 & $(7,3)$ & 4 & 1 & YES & YES & YES & $1.12$ & $(2,2)$ & NO & 1583\\
$(43,12)$ & 8 & $(7,3)$ & 4 & 1 & YES & YES & YES & $1.12$ & $(2,2)$ & NO & 1584\\
$(43,13)$ & 9 & $(7,2)$ & 4 & 1 & YES & YES & YES & $1.00$ & $(2,2)$ & -- & 1585\\
$(43,16)$ & 9 & $(7,2)$ & 4 & 1 & YES & YES & YES & $1.25$ & $(2,2)$ & 1945 & 1586\\
$(43,18)$ & 8 & $(7,2)$ & 4 & 1 & YES & YES & YES & $1.00$ & $(2,2)$ & -- & 1587\\
$(43,10)$ & 9 & $(8,3)$ & 4 & 1 & YES & YES & YES & $1.12$ & $(2,2)$ & NO & 1588\\
$(43,12)$ & 8 & $(8,3)$ & 4 & 1 & YES & YES & YES & $1.12$ & $(2,2)$ & NO & 1589\\
$(43,12)$ & 8 & $(8,3)$ & 4 & 1 & YES & YES & YES & $1.12$ & $(2,2)$ & -- & 1590\\
$(43,18)$ & 8 & $(8,3)$ & 4 & 1 & YES & YES & YES & $1.12$ & $(2,2)$ & NO & 1591\\
$(43,18)$ & 8 & $(9,2)$ & 5 & 1 & YES & YES & YES & $1.12$ & $(2,2)$ & NO & 1592\\
$(43,18)$ & 8 & $(9,2)$ & 5 & 1 & YES & YES & YES & $1.12$ & $(2,2)$ & -- & 1593\\
$(43,18)$ & 8 & $(9,2)$ & 5 & 1 & YES & YES & YES & $1.12$ & $(2,2)$ & NO & 1594\\
$(43,18)$ & 8 & $(9,4)$ & 5 & 1 & YES & YES & YES & $1.00$ & $(2,2)$ & 2001 & 1595\\
$(43,13)$ & 9 & $(11,2)$ & 6 & 1 & YES & YES & YES & $1.00$ & $(2,2)$ & NO & 1596\\
$(43,13)$ & 9 & $(11,3)$ & 5 & 1 & YES & YES & YES & $1.12$ & $(2,2)$ & NO & 1597\\
$(43,16)$ & 9 & $(11,2)$ & 6 & 1 & YES & YES & YES & $1.25$ & $(2,2)$ & NO & 1598\\
$(43,16)$ & 9 & $(11,2)$ & 6 & 1 & YES & YES & YES & $1.25$ & $(2,2)$ & NO & 1599\\
$(43,16)$ & 9 & $(11,2)$ & 6 & 1 & YES & YES & YES & $1.25$ & $(2,2)$ & -- & 1600\\
$(43,12)$ & 8 & $(13,3)$ & 6 & 1 & YES & YES & YES & $1.00$ & $(2,2)$ & NO & 1601\\
$(43,16)$ & 9 & $(14,5)$ & 6 & 1 & YES & YES & YES & $1.25$ & $(2,2)$ & NO & 1602\\
$(43,13)$ & 9 & $(17,5)$ & 6 & 1 & YES & YES & YES & $1.12$ & $(2,2)$ & NO & 1603\\
$(43,10)$ & 9 & $(21,5)$ & 8 & 1 & YES & YES & YES & $1.12$ & $(2,2)$ & 2633 & 1604\\
$(43,10)$ & 9 & $(23,5)$ & 7 & 1 & YES & YES & YES & $1.00$ & $(2,2)$ & NO & 1605\\
$(43,18)$ & 8 & $(26,11)$ & 7 & 1 & YES & YES & YES & $1.12$ & $(2,2)$ & 2658 & 1606\\
$(43,18)$ & 8 & $(29,12)$ & 7 & 1 & YES & YES & YES & $1.12$ & $(2,2)$ & NO & 1607\\
$(43,10)$ & 9 & $(31,7)$ & 8 & 1 & YES & YES & YES & $1.00$ & $(2,2)$ & NO & 1608\\
$(43,13)$ & 9 & $(36,11)$ & 8 & 1 & YES & YES & YES & $1.00$ & $(2,2)$ & 2729 & 1609\\
$(44,13)$ & 8 & $(3,1)$ & 2 & 1 & YES & YES & YES & $1.00$ & $(2,2)$ & NO & 1610\\
$(44,13)$ & 8 & $(3,1)$ & 2 & 1 & YES & YES & YES & $1.00$ & $(2,2)$ & -- & 1611\\
$(44,17)$ & 8 & $(3,1)$ & 2 & 1 & YES & YES & YES & $1.12$ & $(2,2)$ & -- & 1612\\
$(44,17)$ & 8 & $(3,1)$ & 2 & 1 & YES & YES & YES & $1.12$ & $(2,2)$ & NO & 1613\\
$(44,13)$ & 8 & $(4,1)$ & 3 & 4 & YES & YES & YES & $1.00$ & $(2,2)$ & NO & 1614\\
$(44,13)$ & 8 & $(4,1)$ & 3 & 4 & YES & YES & YES & $1.00$ & $(2,2)$ & -- & 1615\\
$(44,17)$ & 8 & $(4,1)$ & 3 & 4 & YES & YES & YES & $1.25$ & $(2,2)$ & NO & 1616\\
$(44,17)$ & 8 & $(4,1)$ & 3 & 4 & YES & YES & YES & $1.25$ & $(2,2)$ & NO & 1617\\
$(44,13)$ & 8 & $(5,2)$ & 3 & 1 & YES & YES & YES & $1.12$ & $(2,2)$ & NO & 1618\\
$(44,13)$ & 8 & $(5,2)$ & 3 & 1 & YES & YES & YES & $1.12$ & $(2,2)$ & NO & 1619\\
$(44,13)$ & 8 & $(5,2)$ & 3 & 1 & YES & YES & YES & $1.12$ & $(2,2)$ & -- & 1620\\
$(44,17)$ & 8 & $(5,1)$ & 4 & 1 & YES & YES & YES & $1.25$ & $(2,2)$ & NO & 1621\\
$(44,17)$ & 8 & $(5,1)$ & 4 & 1 & YES & YES & YES & $1.25$ & $(2,2)$ & -- & 1622\\
$(44,17)$ & 8 & $(5,2)$ & 3 & 1 & YES & YES & YES & $1.00$ & $(2,2)$ & -- & 1623\\
$(44,17)$ & 8 & $(5,2)$ & 3 & 1 & YES & YES & YES & $1.25$ & $(2,2)$ & NO & 1624\\
$(44,13)$ & 8 & $(7,3)$ & 4 & 1 & YES & YES & YES & $1.12$ & $(2,2)$ & -- & 1625\\
$(44,17)$ & 8 & $(7,2)$ & 4 & 1 & YES & YES & YES & $1.00$ & $(2,2)$ & -- & 1626\\
$(44,17)$ & 8 & $(7,3)$ & 4 & 1 & YES & YES & NO(2) & $1.22$ & $(2,2)$ & NO & 1627\\
$(44,17)$ & 8 & $(9,4)$ & 5 & 1 & YES & YES & YES & $1.25$ & $(2,2)$ & NO & 1628\\
$(44,17)$ & 8 & $(11,4)$ & 5 & 11 & YES & YES & YES & $1.12$ & $(2,2)$ & NO & 1629\\
$(44,17)$ & 8 & $(12,5)$ & 5 & 4 & YES & YES & YES & $1.12$ & $(2,2)$ & NO & 1630\\
$(44,13)$ & 8 & $(13,3)$ & 6 & 1 & YES & YES & YES & $1.12$ & $(2,2)$ & NO & 1631\\
$(44,13)$ & 8 & $(13,4)$ & 6 & 1 & YES & YES & YES & $1.12$ & $(2,2)$ & NO & 1632\\
$(44,13)$ & 8 & $(15,4)$ & 6 & 1 & YES & YES & YES & $1.00$ & $(2,2)$ & NO & 1633\\
$(44,13)$ & 8 & $(18,5)$ & 6 & 2 & YES & YES & YES & $1.00$ & $(2,2)$ & NO & 1634\\
$(44,17)$ & 8 & $(18,7)$ & 6 & 2 & YES & YES & YES & $1.12$ & $(2,2)$ & 1833 & 1635\\
$(44,17)$ & 8 & $(21,8)$ & 6 & 1 & YES & YES & YES & $1.12$ & $(2,2)$ & NO & 1636\\
$(44,17)$ & 8 & $(23,9)$ & 7 & 1 & YES & YES & YES & $1.25$ & $(2,2)$ & 2683 & 1637\\
$(44,13)$ & 8 & $(27,8)$ & 7 & 1 & YES & YES & YES & $1.00$ & $(2,2)$ & NO & 1638\\
$(44,13)$ & 8 & $(31,9)$ & 8 & 1 & YES & YES & YES & $1.00$ & $(2,2)$ & NO & 1639\\
$(44,13)$ & 8 & $(41,12)$ & 8 & 1 & YES & YES & YES & $1.00$ & $(2,2)$ & NO & 1640\\
$(44,13)$ & 8 & $(44,13)$ & 8 & 44 & YES & YES & YES & $1.00$ & $(2,2)$ & NO & 1641\\
$(44,17)$ & 8 & $(44,17)$ & 8 & 44 & YES & YES & YES & $1.25$ & $(2,2)$ & NO & 1642\\
$(45,19)$ & 8 & $(2,1)$ & 1 & 1 & YES & YES & YES & $1.00$ & $(2,2)$ & -- & 1643\\
$(45,19)$ & 8 & $(3,1)$ & 2 & 3 & YES & YES & YES & $1.00$ & $(2,2)$ & -- & 1644\\
$(45,19)$ & 8 & $(3,1)$ & 2 & 3 & YES & YES & YES & $1.12$ & $(2,2)$ & NO & 1645\\
$(45,19)$ & 8 & $(3,1)$ & 2 & 3 & YES & YES & YES & $1.12$ & $(2,2)$ & NO & 1646\\
$(45,13)$ & 10 & $(4,1)$ & 3 & 1 & YES & YES & YES & $1.12$ & $(2,2)$ & -- & 1647\\
$(45,19)$ & 8 & $(4,1)$ & 3 & 1 & YES & YES & YES & $1.00$ & $(2,2)$ & NO & 1648\\
$(45,19)$ & 8 & $(4,1)$ & 3 & 1 & YES & YES & YES & $1.00$ & $(2,2)$ & -- & 1649\\
$(45,19)$ & 8 & $(4,1)$ & 3 & 1 & YES & YES & YES & $1.38$ & $(2,2)$ & NO & 1650\\
$(45,14)$ & 9 & $(5,2)$ & 3 & 5 & YES & YES & YES & $1.25$ & $(2,2)$ & NO & 1651\\
$(45,16)$ & 9 & $(5,2)$ & 3 & 5 & YES & YES & YES & $1.12$ & $(2,2)$ & -- & 1652\\
$(45,17)$ & 9 & $(5,1)$ & 4 & 5 & YES & YES & YES & $1.12$ & $(2,2)$ & -- & 1653\\
$(45,19)$ & 8 & $(5,2)$ & 3 & 5 & YES & YES & YES & $1.12$ & $(2,2)$ & -- & 1654\\
$(45,19)$ & 8 & $(5,2)$ & 3 & 5 & YES & YES & YES & $1.12$ & $(2,2)$ & 962 & 1655\\
$(45,13)$ & 10 & $(6,1)$ & 5 & 3 & YES & YES & YES & $1.12$ & $(2,2)$ & NO & 1656\\
$(45,17)$ & 9 & $(6,1)$ & 5 & 3 & YES & YES & YES & $1.12$ & $(2,2)$ & NO & 1657\\
$(45,16)$ & 9 & $(7,2)$ & 4 & 1 & YES & YES & YES & $1.00$ & $(2,2)$ & NO & 1658\\
$(45,19)$ & 8 & $(7,2)$ & 4 & 1 & YES & YES & YES & $1.00$ & $(2,2)$ & NO & 1659\\
$(45,19)$ & 8 & $(7,2)$ & 4 & 1 & YES & YES & YES & $1.00$ & $(2,2)$ & -- & 1660\\
$(45,19)$ & 8 & $(7,2)$ & 4 & 1 & YES & YES & YES & $1.12$ & $(2,2)$ & NO & 1661\\
$(45,19)$ & 8 & $(9,2)$ & 5 & 9 & YES & YES & YES & $1.00$ & $(2,2)$ & -- & 1662\\
$(45,19)$ & 8 & $(9,2)$ & 5 & 9 & YES & YES & YES & $1.00$ & $(2,2)$ & NO & 1663\\
$(45,19)$ & 8 & $(9,2)$ & 5 & 9 & YES & YES & YES & $1.00$ & $(2,2)$ & NO & 1664\\
$(45,14)$ & 9 & $(13,3)$ & 6 & 1 & YES & YES & YES & $1.00$ & $(2,2)$ & NO & 1665\\
$(45,14)$ & 9 & $(15,4)$ & 6 & 15 & YES & YES & YES & $1.00$ & $(2,2)$ & NO & 1666\\
$(45,19)$ & 8 & $(16,7)$ & 6 & 1 & YES & YES & YES & $1.25$ & $(2,2)$ & NO & 1667\\
$(45,16)$ & 9 & $(17,6)$ & 7 & 1 & YES & YES & YES & $1.00$ & $(2,2)$ & 1762 & 1668\\
$(45,19)$ & 8 & $(17,7)$ & 6 & 1 & YES & YES & YES & $1.00$ & $(2,2)$ & NO & 1669\\
$(45,19)$ & 8 & $(19,8)$ & 6 & 1 & YES & YES & YES & $1.00$ & $(2,2)$ & NO & 1670\\
$(45,16)$ & 9 & $(20,7)$ & 8 & 5 & YES & YES & YES & $1.00$ & $(2,2)$ & 2711 & 1671\\
$(45,17)$ & 9 & $(21,8)$ & 6 & 3 & YES & YES & YES & $1.12$ & $(2,2)$ & NO & 1672\\
$(45,13)$ & 10 & $(24,7)$ & 7 & 3 & YES & YES & YES & $1.12$ & $(2,2)$ & NO & 1673\\
$(45,19)$ & 8 & $(26,11)$ & 7 & 1 & YES & YES & YES & $1.00$ & $(2,2)$ & NO & 1674\\
$(45,17)$ & 9 & $(29,11)$ & 7 & 1 & YES & YES & YES & $1.12$ & $(2,2)$ & 2418 & 1675\\
$(45,19)$ & 8 & $(31,13)$ & 7 & 1 & YES & YES & YES & $1.00$ & $(2,2)$ & NO & 1676\\
$(45,19)$ & 8 & $(33,14)$ & 8 & 3 & YES & YES & YES & $0.88$ & $(2,2)$ & NO & 1677\\
$(45,13)$ & 10 & $(38,11)$ & 9 & 1 & YES & YES & YES & $1.12$ & $(2,2)$ & NO & 1678\\
$(45,19)$ & 8 & $(45,19)$ & 8 & 45 & YES & YES & YES & $1.00$ & $(2,2)$ & NO & 1679\\
$(46,17)$ & 8 & $(3,1)$ & 2 & 1 & YES & YES & NO(2) & $1.10$ & $(2,2)$ & -- & 1680\\
$(46,17)$ & 8 & $(3,1)$ & 2 & 1 & YES & YES & YES & $1.12$ & $(2,2)$ & NO & 1681\\
$(46,19)$ & 8 & $(3,1)$ & 2 & 1 & YES & YES & YES & $1.12$ & $(2,2)$ & NO & 1682\\
$(46,19)$ & 8 & $(3,1)$ & 2 & 1 & YES & YES & YES & $1.12$ & $(2,2)$ & -- & 1683\\
$(46,13)$ & 10 & $(4,1)$ & 3 & 2 & YES & YES & YES & $1.00$ & $(2,2)$ & -- & 1684\\
$(46,19)$ & 8 & $(4,1)$ & 3 & 2 & YES & YES & YES & $1.25$ & $(2,2)$ & -- & 1685\\
$(46,19)$ & 8 & $(4,1)$ & 3 & 2 & YES & YES & YES & $1.25$ & $(2,2)$ & NO & 1686\\
$(46,17)$ & 8 & $(5,2)$ & 3 & 1 & YES & YES & NO(2) & $1.11$ & $(2,2)$ & NO & 1687\\
$(46,17)$ & 8 & $(5,2)$ & 3 & 1 & YES & YES & NO(2) & $1.11$ & $(2,2)$ & -- & 1688\\
$(46,17)$ & 8 & $(5,2)$ & 3 & 1 & YES & YES & YES & $1.12$ & $(2,2)$ & NO & 1689\\
$(46,19)$ & 8 & $(5,2)$ & 3 & 1 & YES & YES & YES & $1.00$ & $(2,2)$ & -- & 1690\\
$(46,17)$ & 8 & $(7,2)$ & 4 & 1 & YES & YES & YES & $1.00$ & $(2,2)$ & -- & 1691\\
$(46,17)$ & 8 & $(7,3)$ & 4 & 1 & YES & YES & YES & $1.25$ & $(2,2)$ & NO & 1692\\
$(46,17)$ & 8 & $(7,3)$ & 4 & 1 & YES & YES & YES & $1.25$ & $(2,2)$ & -- & 1693\\
$(46,17)$ & 8 & $(7,3)$ & 4 & 1 & YES & YES & YES & $1.12$ & $(2,2)$ & NO & 1694\\
$(46,19)$ & 8 & $(7,2)$ & 4 & 1 & YES & YES & YES & $1.00$ & $(2,2)$ & -- & 1695\\
$(46,19)$ & 8 & $(7,2)$ & 4 & 1 & YES & YES & YES & $1.25$ & $(2,2)$ & NO & 1696\\
$(46,19)$ & 8 & $(7,2)$ & 4 & 1 & YES & YES & YES & $1.12$ & $(2,2)$ & NO & 1697\\
$(46,19)$ & 8 & $(8,3)$ & 4 & 2 & YES & YES & YES & $1.25$ & $(2,2)$ & NO & 1698\\
$(46,17)$ & 8 & $(9,4)$ & 5 & 1 & YES & YES & YES & $1.00$ & $(2,2)$ & NO & 1699\\
$(46,19)$ & 8 & $(9,2)$ & 5 & 1 & YES & YES & YES & $1.00$ & $(2,2)$ & NO & 1700\\
$(46,19)$ & 8 & $(9,2)$ & 5 & 1 & YES & YES & YES & $1.00$ & $(2,2)$ & -- & 1701\\
$(46,19)$ & 8 & $(9,2)$ & 5 & 1 & YES & YES & YES & $1.12$ & $(2,2)$ & NO & 1702\\
$(46,17)$ & 8 & $(11,4)$ & 5 & 1 & YES & YES & YES & $1.12$ & $(2,2)$ & NO & 1703\\
$(46,19)$ & 8 & $(12,5)$ & 5 & 2 & YES & YES & YES & $1.12$ & $(2,2)$ & 1536 & 1704\\
$(46,17)$ & 8 & $(13,5)$ & 5 & 1 & YES & YES & YES & $1.12$ & $(2,2)$ & NO & 1705\\
$(46,19)$ & 8 & $(13,5)$ & 5 & 1 & YES & YES & YES & $1.12$ & $(2,2)$ & NO & 1706\\
$(46,19)$ & 8 & $(19,8)$ & 6 & 1 & YES & YES & YES & $0.88$ & $(2,2)$ & NO & 1707\\
$(46,17)$ & 8 & $(21,8)$ & 6 & 1 & YES & YES & YES & $1.00$ & $(2,2)$ & NO & 1708\\
$(46,19)$ & 8 & $(22,9)$ & 7 & 2 & YES & YES & YES & $1.00$ & $(2,2)$ & NO & 1709\\
$(46,19)$ & 8 & $(27,11)$ & 8 & 1 & YES & YES & YES & $1.12$ & $(2,2)$ & NO & 1710\\
$(46,19)$ & 8 & $(29,12)$ & 7 & 1 & YES & YES & YES & $1.12$ & $(2,2)$ & NO & 1711\\
$(46,17)$ & 8 & $(30,11)$ & 7 & 2 & YES & YES & YES & $1.00$ & $(2,2)$ & NO & 1712\\
$(46,13)$ & 10 & $(39,11)$ & 9 & 1 & YES & YES & YES & $1.00$ & $(2,2)$ & NO & 1713\\
$(46,17)$ & 8 & $(43,16)$ & 9 & 1 & YES & YES & YES & $1.25$ & $(2,2)$ & 2764 & 1714\\
$(46,19)$ & 8 & $(46,19)$ & 8 & 46 & YES & YES & YES & $1.12$ & $(2,2)$ & NO & 1715\\
$(47,11)$ & 9 & $(3,1)$ & 2 & 1 & YES & YES & YES & $1.12$ & $(2,2)$ & NO & 1716\\
$(47,11)$ & 9 & $(3,1)$ & 2 & 1 & YES & YES & YES & $1.12$ & $(2,2)$ & -- & 1717\\
$(47,14)$ & 9 & $(4,1)$ & 3 & 1 & YES & YES & YES & $1.12$ & $(2,2)$ & NO & 1718\\
$(47,14)$ & 9 & $(4,1)$ & 3 & 1 & YES & YES & YES & $1.12$ & $(2,2)$ & -- & 1719\\
$(47,17)$ & 9 & $(4,1)$ & 3 & 1 & YES & YES & YES & $1.25$ & $(2,2)$ & -- & 1720\\
$(47,17)$ & 9 & $(4,1)$ & 3 & 1 & YES & YES & YES & $1.38$ & $(2,2)$ & NO & 1721\\
$(47,18)$ & 8 & $(4,1)$ & 3 & 1 & YES & YES & YES & $1.12$ & $(2,2)$ & NO & 1722\\
$(47,18)$ & 8 & $(4,1)$ & 3 & 1 & YES & YES & YES & $1.12$ & $(2,2)$ & -- & 1723\\
$(47,11)$ & 9 & $(5,2)$ & 3 & 1 & YES & YES & YES & $1.25$ & $(2,2)$ & NO & 1724\\
$(47,11)$ & 9 & $(5,2)$ & 3 & 1 & YES & YES & YES & $1.25$ & $(2,2)$ & -- & 1725\\
$(47,14)$ & 9 & $(5,1)$ & 4 & 1 & YES & YES & YES & $1.25$ & $(2,2)$ & NO & 1726\\
$(47,14)$ & 9 & $(5,1)$ & 4 & 1 & YES & YES & YES & $1.25$ & $(2,2)$ & -- & 1727\\
$(47,14)$ & 9 & $(5,2)$ & 3 & 1 & YES & YES & YES & $1.00$ & $(2,2)$ & -- & 1728\\
$(47,14)$ & 9 & $(5,2)$ & 3 & 1 & YES & YES & YES & $1.25$ & $(2,2)$ & NO & 1729\\
$(47,17)$ & 9 & $(5,2)$ & 3 & 1 & YES & YES & YES & $1.25$ & $(2,2)$ & -- & 1730\\
$(47,18)$ & 8 & $(5,2)$ & 3 & 1 & YES & YES & YES & $1.25$ & $(2,2)$ & -- & 1731\\
$(47,18)$ & 8 & $(7,3)$ & 4 & 1 & YES & YES & YES & $1.12$ & $(2,2)$ & NO & 1732\\
$(47,14)$ & 9 & $(9,2)$ & 5 & 1 & YES & YES & YES & $0.88$ & $(2,2)$ & -- & 1733\\
$(47,14)$ & 9 & $(9,2)$ & 5 & 1 & YES & YES & YES & $1.00$ & $(2,2)$ & NO & 1734\\
$(47,17)$ & 9 & $(9,2)$ & 5 & 1 & YES & YES & YES & $1.12$ & $(2,2)$ & -- & 1735\\
$(47,11)$ & 9 & $(11,3)$ & 5 & 1 & YES & YES & YES & $1.00$ & $(2,2)$ & -- & 1736\\
$(47,14)$ & 9 & $(11,2)$ & 6 & 1 & YES & YES & YES & $1.00$ & $(2,2)$ & NO & 1737\\
$(47,14)$ & 9 & $(11,3)$ & 5 & 1 & YES & YES & YES & $1.00$ & $(2,2)$ & NO & 1738\\
$(47,18)$ & 8 & $(11,4)$ & 5 & 1 & YES & YES & YES & $1.12$ & $(2,2)$ & 2375 & 1739\\
$(47,14)$ & 9 & $(13,4)$ & 6 & 1 & YES & YES & YES & $1.25$ & $(2,2)$ & NO & 1740\\
$(47,14)$ & 9 & $(16,5)$ & 7 & 1 & YES & YES & YES & $1.00$ & $(2,2)$ & NO & 1741\\
$(47,18)$ & 8 & $(18,7)$ & 6 & 1 & YES & YES & YES & $1.25$ & $(2,2)$ & NO & 1742\\
$(47,17)$ & 9 & $(19,7)$ & 6 & 1 & YES & YES & YES & $1.00$ & $(2,2)$ & NO & 1743\\
$(47,11)$ & 9 & $(21,5)$ & 8 & 1 & YES & YES & YES & $1.25$ & $(2,2)$ & NO & 1744\\
$(47,14)$ & 9 & $(24,7)$ & 7 & 1 & YES & YES & YES & $1.00$ & $(2,2)$ & NO & 1745\\
$(47,14)$ & 9 & $(27,8)$ & 7 & 1 & YES & YES & YES & $1.12$ & $(2,2)$ & 2332 & 1746\\
$(47,18)$ & 8 & $(34,13)$ & 7 & 1 & YES & YES & YES & $1.12$ & $(2,2)$ & NO & 1747\\
$(47,14)$ & 9 & $(37,11)$ & 8 & 1 & YES & YES & YES & $1.38$ & $(2,2)$ & NO & 1748\\
$(47,14)$ & 9 & $(44,13)$ & 8 & 1 & YES & YES & YES & $1.00$ & $(2,2)$ & 2913 & 1749\\
$(47,17)$ & 9 & $(47,17)$ & 9 & 47 & YES & YES & YES & $0.88$ & $(2,2)$ & NO & 1750\\
$(48,17)$ & 9 & $(2,1)$ & 1 & 2 & YES & YES & YES & $1.00$ & $(2,2)$ & -- & 1751\\
$(48,17)$ & 9 & $(3,1)$ & 2 & 3 & YES & YES & YES & $1.00$ & $(2,2)$ & -- & 1752\\
$(48,17)$ & 9 & $(3,1)$ & 2 & 3 & YES & YES & YES & $1.25$ & $(2,2)$ & NO & 1753\\
$(48,17)$ & 9 & $(4,1)$ & 3 & 4 & YES & YES & YES & $1.00$ & $(2,2)$ & -- & 1754\\
$(48,17)$ & 9 & $(5,1)$ & 4 & 1 & YES & YES & YES & $1.12$ & $(2,2)$ & NO & 1755\\
$(48,17)$ & 9 & $(5,1)$ & 4 & 1 & YES & YES & YES & $1.12$ & $(2,2)$ & -- & 1756\\
$(48,17)$ & 9 & $(5,2)$ & 3 & 1 & YES & YES & YES & $1.12$ & $(2,2)$ & NO & 1757\\
$(48,11)$ & 9 & $(7,3)$ & 4 & 1 & YES & YES & YES & $1.25$ & $(2,2)$ & -- & 1758\\
$(48,13)$ & 9 & $(11,2)$ & 6 & 1 & YES & YES & YES & $0.88$ & $(2,2)$ & NO & 1759\\
$(48,17)$ & 9 & $(11,4)$ & 5 & 1 & YES & YES & YES & $1.00$ & $(2,2)$ & 2301 & 1760\\
$(48,13)$ & 9 & $(14,3)$ & 6 & 2 & YES & YES & YES & $1.12$ & $(2,2)$ & NO & 1761\\
$(48,17)$ & 9 & $(14,5)$ & 6 & 2 & YES & YES & YES & $1.00$ & $(2,2)$ & 1668 & 1762\\
$(48,11)$ & 9 & $(19,4)$ & 7 & 1 & YES & YES & YES & $1.12$ & $(2,2)$ & NO & 1763\\
$(48,17)$ & 9 & $(20,7)$ & 8 & 4 & YES & YES & YES & $1.12$ & $(2,2)$ & NO & 1764\\
$(48,11)$ & 9 & $(23,5)$ & 7 & 1 & YES & YES & YES & $1.12$ & $(2,2)$ & NO & 1765\\
$(48,17)$ & 9 & $(31,11)$ & 8 & 1 & YES & YES & YES & $1.00$ & $(2,2)$ & NO & 1766\\
$(48,13)$ & 9 & $(34,9)$ & 8 & 2 & YES & YES & YES & $1.12$ & $(2,2)$ & 3192 & 1767\\
$(48,13)$ & 9 & $(41,11)$ & 8 & 1 & YES & YES & YES & $0.88$ & $(2,2)$ & 2937 & 1768\\
$(48,17)$ & 9 & $(48,17)$ & 9 & 48 & YES & YES & YES & $1.00$ & $(2,2)$ & NO & 1769\\
$(49,15)$ & 9 & $(2,1)$ & 1 & 1 & YES & YES & YES & $1.33$ & $(2,2)$ & -- & 1770\\
$(49,18)$ & 8 & $(2,1)$ & 1 & 1 & YES & YES & YES & $1.12$ & $(2,2)$ & -- & 1771\\
$(49,19)$ & 8 & $(2,1)$ & 1 & 1 & YES & YES & YES & $1.12$ & $(2,2)$ & -- & 1772\\
$(49,22)$ & 9 & $(2,1)$ & 1 & 1 & YES & YES & YES & $1.00$ & $(2,2)$ & -- & 1773\\
$(49,15)$ & 9 & $(3,1)$ & 2 & 1 & YES & YES & YES & $1.25$ & $(2,2)$ & NO & 1774\\
$(49,15)$ & 9 & $(3,1)$ & 2 & 1 & YES & YES & YES & $1.12$ & $(2,2)$ & -- & 1775\\
$(49,18)$ & 8 & $(3,1)$ & 2 & 1 & YES & YES & YES & $1.00$ & $(2,2)$ & NO & 1776\\
$(49,18)$ & 8 & $(3,1)$ & 2 & 1 & YES & YES & YES & $1.00$ & $(2,2)$ & -- & 1777\\
$(49,19)$ & 8 & $(3,1)$ & 2 & 1 & YES & YES & YES & $1.22$ & $(2,2)$ & -- & 1778\\
$(49,19)$ & 8 & $(3,1)$ & 2 & 1 & YES & YES & YES & $1.12$ & $(2,2)$ & NO & 1779\\
$(49,22)$ & 9 & $(3,1)$ & 2 & 1 & YES & YES & YES & $1.00$ & $(2,2)$ & -- & 1780\\
$(49,22)$ & 9 & $(3,1)$ & 2 & 1 & YES & YES & YES & $1.12$ & $(2,2)$ & NO & 1781\\
$(49,15)$ & 9 & $(4,1)$ & 3 & 1 & YES & YES & YES & $1.00$ & $(2,2)$ & NO & 1782\\
$(49,15)$ & 9 & $(4,1)$ & 3 & 1 & YES & YES & YES & $1.00$ & $(2,2)$ & -- & 1783\\
$(49,18)$ & 8 & $(4,1)$ & 3 & 1 & YES & YES & YES & $1.00$ & $(2,2)$ & NO & 1784\\
$(49,18)$ & 8 & $(4,1)$ & 3 & 1 & YES & YES & YES & $1.00$ & $(2,2)$ & NO & 1785\\
$(49,18)$ & 8 & $(4,1)$ & 3 & 1 & YES & YES & YES & $1.00$ & $(2,2)$ & -- & 1786\\
$(49,19)$ & 8 & $(4,1)$ & 3 & 1 & YES & YES & YES & $1.12$ & $(2,2)$ & NO & 1787\\
$(49,19)$ & 8 & $(4,1)$ & 3 & 1 & YES & YES & YES & $1.12$ & $(2,2)$ & -- & 1788\\
$(49,19)$ & 8 & $(4,1)$ & 3 & 1 & YES & YES & YES & $1.00$ & $(2,2)$ & NO & 1789\\
$(49,22)$ & 9 & $(4,1)$ & 3 & 1 & YES & YES & YES & $1.00$ & $(2,2)$ & -- & 1790\\
$(49,22)$ & 9 & $(4,1)$ & 3 & 1 & YES & YES & YES & $1.00$ & $(2,2)$ & NO & 1791\\
$(49,11)$ & 10 & $(5,2)$ & 3 & 1 & YES & YES & YES & $1.00$ & $(2,2)$ & NO & 1792\\
$(49,15)$ & 9 & $(5,1)$ & 4 & 1 & YES & YES & YES & $1.12$ & $(2,2)$ & NO & 1793\\
$(49,15)$ & 9 & $(5,1)$ & 4 & 1 & YES & YES & YES & $1.12$ & $(2,2)$ & -- & 1794\\
$(49,15)$ & 9 & $(5,2)$ & 3 & 1 & YES & YES & YES & $1.12$ & $(2,2)$ & -- & 1795\\
$(49,18)$ & 8 & $(5,1)$ & 4 & 1 & YES & YES & YES & $1.00$ & $(2,2)$ & NO & 1796\\
$(49,18)$ & 8 & $(5,1)$ & 4 & 1 & YES & YES & YES & $1.00$ & $(2,2)$ & -- & 1797\\
$(49,18)$ & 8 & $(5,2)$ & 3 & 1 & YES & YES & YES & $1.00$ & $(2,2)$ & -- & 1798\\
$(49,18)$ & 8 & $(5,2)$ & 3 & 1 & YES & YES & YES & $1.12$ & $(2,2)$ & NO & 1799\\
$(49,19)$ & 8 & $(5,1)$ & 4 & 1 & YES & YES & YES & $1.00$ & $(2,2)$ & NO & 1800\\
$(49,19)$ & 8 & $(5,1)$ & 4 & 1 & YES & YES & YES & $1.00$ & $(2,2)$ & -- & 1801\\
$(49,19)$ & 8 & $(5,2)$ & 3 & 1 & YES & YES & YES & $0.88$ & $(2,2)$ & -- & 1802\\
$(49,19)$ & 8 & $(5,2)$ & 3 & 1 & YES & YES & YES & $1.12$ & $(2,2)$ & NO & 1803\\
$(49,20)$ & 9 & $(5,2)$ & 3 & 1 & YES & YES & YES & $1.12$ & $(2,2)$ & -- & 1804\\
$(49,20)$ & 9 & $(5,2)$ & 3 & 1 & YES & YES & YES & $1.25$ & $(2,2)$ & NO & 1805\\
$(49,22)$ & 9 & $(5,2)$ & 3 & 1 & YES & YES & YES & $1.12$ & $(2,2)$ & NO & 1806\\
$(49,22)$ & 9 & $(5,2)$ & 3 & 1 & YES & YES & YES & $1.12$ & $(2,2)$ & -- & 1807\\
$(49,22)$ & 9 & $(5,2)$ & 3 & 1 & YES & YES & YES & $1.00$ & $(2,2)$ & NO & 1808\\
$(49,18)$ & 8 & $(7,2)$ & 4 & 7 & YES & YES & YES & $1.00$ & $(2,2)$ & -- & 1809\\
$(49,18)$ & 8 & $(7,2)$ & 4 & 7 & YES & YES & NO(2) & $1.12$ & $(4,1)$ & NO & 1810\\
$(49,18)$ & 8 & $(7,2)$ & 4 & 7 & YES & YES & YES & $1.00$ & $(2,2)$ & NO & 1811\\
$(49,18)$ & 8 & $(7,3)$ & 4 & 7 & YES & YES & YES & $0.88$ & $(2,2)$ & NO & 1812\\
$(49,18)$ & 8 & $(7,3)$ & 4 & 7 & YES & YES & YES & $1.12$ & $(2,2)$ & -- & 1813\\
$(49,19)$ & 8 & $(7,2)$ & 4 & 7 & YES & YES & YES & $1.25$ & $(2,2)$ & NO & 1814\\
$(49,22)$ & 9 & $(7,3)$ & 4 & 7 & YES & YES & YES & $1.00$ & $(2,2)$ & NO & 1815\\
$(49,18)$ & 8 & $(8,3)$ & 4 & 1 & YES & YES & YES & $1.12$ & $(2,2)$ & NO & 1816\\
$(49,19)$ & 8 & $(8,3)$ & 4 & 1 & YES & YES & YES & $1.00$ & $(2,2)$ & 1996 & 1817\\
$(49,22)$ & 9 & $(8,3)$ & 4 & 1 & YES & YES & YES & $1.00$ & $(2,2)$ & NO & 1818\\
$(49,15)$ & 9 & $(9,2)$ & 5 & 1 & YES & YES & YES & $1.00$ & $(2,2)$ & NO & 1819\\
$(49,18)$ & 8 & $(9,2)$ & 5 & 1 & YES & YES & YES & $0.88$ & $(2,2)$ & NO & 1820\\
$(49,18)$ & 8 & $(9,4)$ & 5 & 1 & YES & YES & YES & $1.00$ & $(2,2)$ & 1123 & 1821\\
$(49,19)$ & 8 & $(9,4)$ & 5 & 1 & YES & YES & YES & $1.25$ & $(2,2)$ & NO & 1822\\
$(49,20)$ & 9 & $(9,4)$ & 5 & 1 & YES & YES & YES & $1.25$ & $(2,2)$ & NO & 1823\\
$(49,15)$ & 9 & $(10,3)$ & 5 & 1 & YES & YES & YES & $1.22$ & $(2,2)$ & NO & 1824\\
$(49,11)$ & 10 & $(11,3)$ & 5 & 1 & YES & YES & YES & $1.00$ & $(2,2)$ & 2800 & 1825\\
$(49,15)$ & 9 & $(11,2)$ & 6 & 1 & YES & YES & YES & $1.00$ & $(2,2)$ & NO & 1826\\
$(49,15)$ & 9 & $(11,3)$ & 5 & 1 & YES & YES & YES & $1.00$ & $(2,2)$ & NO & 1827\\
$(49,18)$ & 8 & $(11,4)$ & 5 & 1 & YES & YES & YES & $1.12$ & $(2,2)$ & 1537 & 1828\\
$(49,19)$ & 8 & $(11,4)$ & 5 & 1 & YES & YES & YES & $1.12$ & $(2,2)$ & 2832 & 1829\\
$(49,20)$ & 9 & $(11,2)$ & 6 & 1 & YES & YES & YES & $1.12$ & $(2,2)$ & NO & 1830\\
$(49,20)$ & 9 & $(11,2)$ & 6 & 1 & YES & YES & YES & $1.12$ & $(2,2)$ & NO & 1831\\
$(49,18)$ & 8 & $(13,5)$ & 5 & 1 & YES & YES & YES & $0.88$ & $(2,2)$ & NO & 1832\\
$(49,19)$ & 8 & $(13,5)$ & 5 & 1 & YES & YES & YES & $1.12$ & $(2,2)$ & 1635 & 1833\\
$(49,18)$ & 8 & $(14,5)$ & 6 & 7 & YES & YES & YES & $1.12$ & $(2,2)$ & NO & 1834\\
$(49,15)$ & 9 & $(17,5)$ & 6 & 1 & YES & YES & YES & $1.00$ & $(2,2)$ & 2999 & 1835\\
$(49,19)$ & 8 & $(18,7)$ & 6 & 1 & YES & YES & YES & $1.12$ & $(2,2)$ & NO & 1836\\
$(49,18)$ & 8 & $(19,7)$ & 6 & 1 & YES & YES & YES & $1.00$ & $(2,2)$ & NO & 1837\\
$(49,11)$ & 10 & $(21,5)$ & 8 & 7 & YES & YES & YES & $1.25$ & $(2,2)$ & NO & 1838\\
$(49,19)$ & 8 & $(21,8)$ & 6 & 7 & YES & YES & YES & $0.88$ & $(2,2)$ & NO & 1839\\
$(49,15)$ & 9 & $(23,7)$ & 7 & 1 & YES & YES & YES & $1.12$ & $(2,2)$ & 2161 & 1840\\
$(49,18)$ & 8 & $(27,10)$ & 7 & 1 & YES & YES & YES & $1.12$ & $(2,2)$ & NO & 1841\\
$(49,15)$ & 9 & $(29,9)$ & 8 & 1 & YES & YES & YES & $1.00$ & $(2,2)$ & NO & 1842\\
$(49,22)$ & 9 & $(29,13)$ & 8 & 1 & YES & YES & YES & $1.00$ & $(2,2)$ & NO & 1843\\
$(49,18)$ & 8 & $(30,11)$ & 7 & 1 & YES & YES & YES & $1.00$ & $(2,2)$ & NO & 1844\\
$(49,19)$ & 8 & $(31,12)$ & 7 & 1 & YES & YES & YES & $1.00$ & $(2,2)$ & NO & 1845\\
$(49,15)$ & 9 & $(33,10)$ & 8 & 1 & YES & YES & YES & $1.00$ & $(2,2)$ & 2935 & 1846\\
$(49,15)$ & 9 & $(36,11)$ & 8 & 1 & YES & YES & YES & $1.12$ & $(2,2)$ & NO & 1847\\
$(49,20)$ & 9 & $(39,16)$ & 8 & 1 & YES & YES & YES & $1.12$ & $(2,2)$ & NO & 1848\\
$(49,18)$ & 8 & $(41,15)$ & 8 & 1 & YES & YES & YES & $1.00$ & $(2,2)$ & NO & 1849\\
$(49,19)$ & 8 & $(44,17)$ & 8 & 1 & YES & YES & YES & $1.00$ & $(2,2)$ & NO & 1850\\
$(49,15)$ & 9 & $(49,15)$ & 9 & 49 & YES & YES & YES & $1.12$ & $(2,2)$ & NO & 1851\\
$(49,18)$ & 8 & $(49,18)$ & 8 & 49 & YES & YES & YES & $0.88$ & $(2,2)$ & NO & 1852\\
$(49,19)$ & 8 & $(49,19)$ & 8 & 49 & YES & YES & YES & $1.12$ & $(2,2)$ & NO & 1853\\
$(49,22)$ & 9 & $(49,22)$ & 9 & 49 & YES & YES & YES & $1.00$ & $(2,2)$ & NO & 1854\\
$(50,19)$ & 8 & $(2,1)$ & 1 & 2 & YES & YES & NO(2) & $1.33$ & $(2,2)$ & -- & 1855\\
$(50,19)$ & 8 & $(2,1)$ & 1 & 2 & YES & YES & YES & $1.12$ & $(2,2)$ & NO & 1856\\
$(50,21)$ & 8 & $(2,1)$ & 1 & 2 & YES & YES & YES & $0.75$ & $(2,2)$ & -- & 1857\\
$(50,19)$ & 8 & $(3,1)$ & 2 & 1 & YES & YES & YES & $1.00$ & $(2,2)$ & NO & 1858\\
$(50,19)$ & 8 & $(3,1)$ & 2 & 1 & YES & YES & YES & $1.00$ & $(2,2)$ & -- & 1859\\
$(50,19)$ & 8 & $(3,1)$ & 2 & 1 & YES & YES & YES & $1.12$ & $(2,2)$ & NO & 1860\\
$(50,19)$ & 8 & $(4,1)$ & 3 & 2 & YES & YES & YES & $1.00$ & $(2,2)$ & NO & 1861\\
$(50,19)$ & 8 & $(4,1)$ & 3 & 2 & YES & YES & YES & $1.00$ & $(2,2)$ & -- & 1862\\
$(50,19)$ & 8 & $(4,1)$ & 3 & 2 & YES & YES & YES & $1.25$ & $(2,2)$ & NO & 1863\\
$(50,21)$ & 8 & $(4,1)$ & 3 & 2 & YES & YES & YES & $1.00$ & $(2,2)$ & NO & 1864\\
$(50,21)$ & 8 & $(4,1)$ & 3 & 2 & YES & YES & YES & $1.00$ & $(2,2)$ & -- & 1865\\
$(50,11)$ & 10 & $(5,2)$ & 3 & 5 & YES & YES & YES & $0.88$ & $(2,2)$ & NO & 1866\\
$(50,19)$ & 8 & $(5,1)$ & 4 & 5 & YES & YES & YES & $1.00$ & $(2,2)$ & NO & 1867\\
$(50,19)$ & 8 & $(5,1)$ & 4 & 5 & YES & YES & YES & $1.00$ & $(2,2)$ & -- & 1868\\
$(50,19)$ & 8 & $(5,2)$ & 3 & 5 & YES & YES & YES & $1.12$ & $(2,2)$ & -- & 1869\\
$(50,19)$ & 8 & $(5,2)$ & 3 & 5 & YES & YES & YES & $1.12$ & $(2,2)$ & NO & 1870\\
$(50,19)$ & 8 & $(5,2)$ & 3 & 5 & YES & YES & YES & $1.25$ & $(2,2)$ & 1078 & 1871\\
$(50,21)$ & 8 & $(5,2)$ & 3 & 5 & YES & YES & YES & $1.00$ & $(2,2)$ & -- & 1872\\
$(50,21)$ & 8 & $(5,2)$ & 3 & 5 & YES & YES & YES & $1.12$ & $(2,2)$ & NO & 1873\\
$(50,11)$ & 10 & $(7,2)$ & 4 & 1 & YES & YES & YES & $0.88$ & $(2,2)$ & NO & 1874\\
$(50,19)$ & 8 & $(7,2)$ & 4 & 1 & YES & YES & YES & $1.00$ & $(2,2)$ & NO & 1875\\
$(50,19)$ & 8 & $(7,3)$ & 4 & 1 & YES & YES & YES & $1.00$ & $(2,2)$ & NO & 1876\\
$(50,19)$ & 8 & $(8,3)$ & 4 & 2 & YES & YES & YES & $1.12$ & $(2,2)$ & 1386 & 1877\\
$(50,19)$ & 8 & $(9,2)$ & 5 & 1 & YES & YES & YES & $0.88$ & $(2,2)$ & NO & 1878\\
$(50,21)$ & 8 & $(10,3)$ & 5 & 10 & YES & YES & YES & $1.12$ & $(2,2)$ & -- & 1879\\
$(50,19)$ & 8 & $(11,4)$ & 5 & 1 & YES & YES & YES & $1.12$ & $(2,2)$ & NO & 1880\\
$(50,11)$ & 10 & $(16,3)$ & 7 & 2 & YES & YES & YES & $1.25$ & $(2,2)$ & 3151 & 1881\\
$(50,21)$ & 8 & $(17,7)$ & 6 & 1 & YES & YES & YES & $1.00$ & $(2,2)$ & NO & 1882\\
$(50,19)$ & 8 & $(18,7)$ & 6 & 2 & YES & YES & YES & $1.12$ & $(2,2)$ & NO & 1883\\
$(50,19)$ & 8 & $(21,8)$ & 6 & 1 & YES & YES & YES & $1.00$ & $(2,2)$ & NO & 1884\\
$(50,11)$ & 10 & $(24,5)$ & 8 & 2 & YES & YES & YES & $1.25$ & $(2,2)$ & NO & 1885\\
$(50,21)$ & 8 & $(26,11)$ & 7 & 2 & YES & YES & YES & $1.00$ & $(2,2)$ & NO & 1886\\
$(50,19)$ & 8 & $(29,11)$ & 7 & 1 & YES & YES & YES & $1.00$ & $(2,2)$ & NO & 1887\\
$(50,19)$ & 8 & $(34,13)$ & 7 & 2 & YES & YES & YES & $1.00$ & $(2,2)$ & NO & 1888\\
$(50,19)$ & 8 & $(37,14)$ & 8 & 1 & YES & YES & YES & $1.00$ & $(2,2)$ & NO & 1889\\
$(50,21)$ & 8 & $(43,18)$ & 8 & 1 & YES & YES & YES & $1.00$ & $(2,2)$ & NO & 1890\\
$(50,19)$ & 8 & $(50,19)$ & 8 & 50 & YES & YES & YES & $1.00$ & $(2,2)$ & NO & 1891\\
$(51,23)$ & 9 & $(2,1)$ & 1 & 1 & YES & YES & YES & $1.12$ & $(2,2)$ & -- & 1892\\
$(51,20)$ & 9 & $(3,1)$ & 2 & 3 & YES & YES & YES & $1.00$ & $(2,2)$ & -- & 1893\\
$(51,23)$ & 9 & $(3,1)$ & 2 & 3 & YES & YES & YES & $1.00$ & $(2,2)$ & -- & 1894\\
$(51,23)$ & 9 & $(3,1)$ & 2 & 3 & YES & YES & YES & $1.12$ & $(2,2)$ & NO & 1895\\
$(51,14)$ & 9 & $(4,1)$ & 3 & 1 & YES & YES & YES & $1.12$ & $(2,2)$ & -- & 1896\\
$(51,14)$ & 9 & $(4,1)$ & 3 & 1 & YES & YES & YES & $1.12$ & $(2,2)$ & NO & 1897\\
$(51,23)$ & 9 & $(4,1)$ & 3 & 1 & YES & YES & YES & $1.12$ & $(2,2)$ & NO & 1898\\
$(51,23)$ & 9 & $(4,1)$ & 3 & 1 & YES & YES & YES & $1.12$ & $(2,2)$ & -- & 1899\\
$(51,11)$ & 9 & $(5,2)$ & 3 & 1 & YES & YES & YES & $0.88$ & $(2,2)$ & NO & 1900\\
$(51,11)$ & 9 & $(5,2)$ & 3 & 1 & YES & YES & YES & $0.88$ & $(2,2)$ & -- & 1901\\
$(51,14)$ & 9 & $(5,2)$ & 3 & 1 & YES & YES & YES & $1.12$ & $(2,2)$ & -- & 1902\\
$(51,14)$ & 9 & $(5,2)$ & 3 & 1 & YES & YES & YES & $1.25$ & $(2,2)$ & NO & 1903\\
$(51,14)$ & 9 & $(5,2)$ & 3 & 1 & YES & YES & YES & $1.25$ & $(2,2)$ & NO & 1904\\
$(51,16)$ & 10 & $(5,2)$ & 3 & 1 & YES & YES & YES & $1.12$ & $(2,2)$ & -- & 1905\\
$(51,16)$ & 10 & $(5,2)$ & 3 & 1 & YES & YES & YES & $1.25$ & $(2,2)$ & NO & 1906\\
$(51,23)$ & 9 & $(5,2)$ & 3 & 1 & YES & YES & YES & $1.12$ & $(2,2)$ & 1177 & 1907\\
$(51,11)$ & 9 & $(7,2)$ & 4 & 1 & YES & YES & YES & $1.00$ & $(2,2)$ & NO & 1908\\
$(51,16)$ & 10 & $(7,2)$ & 4 & 1 & YES & YES & YES & $1.12$ & $(2,2)$ & 1014 & 1909\\
$(51,14)$ & 9 & $(9,2)$ & 5 & 3 & YES & YES & YES & $1.00$ & $(2,2)$ & -- & 1910\\
$(51,14)$ & 9 & $(9,2)$ & 5 & 3 & YES & YES & YES & $1.12$ & $(2,2)$ & NO & 1911\\
$(51,23)$ & 9 & $(9,4)$ & 5 & 3 & YES & YES & YES & $1.12$ & $(2,2)$ & NO & 1912\\
$(51,14)$ & 9 & $(10,3)$ & 5 & 1 & YES & YES & YES & $1.12$ & $(2,2)$ & NO & 1913\\
$(51,14)$ & 9 & $(11,2)$ & 6 & 1 & YES & YES & YES & $1.12$ & $(2,2)$ & NO & 1914\\
$(51,20)$ & 9 & $(18,7)$ & 6 & 3 & YES & YES & YES & $1.12$ & $(2,2)$ & NO & 1915\\
$(51,20)$ & 9 & $(23,9)$ & 7 & 1 & YES & YES & YES & $1.00$ & $(2,2)$ & NO & 1916\\
$(51,14)$ & 9 & $(25,7)$ & 7 & 1 & YES & YES & YES & $1.12$ & $(2,2)$ & NO & 1917\\
$(51,23)$ & 9 & $(31,14)$ & 8 & 1 & YES & YES & YES & $1.00$ & $(2,2)$ & NO & 1918\\
$(51,20)$ & 9 & $(51,20)$ & 9 & 51 & YES & YES & YES & $1.00$ & $(2,2)$ & NO & 1919\\
$(51,23)$ & 9 & $(51,23)$ & 9 & 51 & YES & YES & YES & $1.00$ & $(2,2)$ & NO & 1920\\
$(52,19)$ & 9 & $(3,1)$ & 2 & 1 & YES & YES & YES & $1.25$ & $(2,2)$ & -- & 1921\\
$(52,19)$ & 9 & $(4,1)$ & 3 & 4 & YES & YES & YES & $1.25$ & $(2,2)$ & NO & 1922\\
$(52,19)$ & 9 & $(5,1)$ & 4 & 1 & YES & YES & YES & $1.12$ & $(2,2)$ & -- & 1923\\
$(52,19)$ & 9 & $(6,1)$ & 5 & 2 & YES & YES & YES & $1.12$ & $(2,2)$ & NO & 1924\\
$(52,19)$ & 9 & $(6,1)$ & 5 & 2 & YES & YES & YES & $1.12$ & $(2,2)$ & NO & 1925\\
$(52,19)$ & 9 & $(14,5)$ & 6 & 2 & YES & YES & YES & $1.25$ & $(2,2)$ & NO & 1926\\
$(52,19)$ & 9 & $(19,7)$ & 6 & 1 & YES & YES & YES & $1.12$ & $(2,2)$ & NO & 1927\\
$(52,19)$ & 9 & $(30,11)$ & 7 & 2 & YES & YES & YES & $1.12$ & $(2,2)$ & 2586 & 1928\\
$(52,19)$ & 9 & $(52,19)$ & 9 & 52 & YES & YES & YES & $1.25$ & $(2,2)$ & NO & 1929\\
$(53,19)$ & 9 & $(2,1)$ & 1 & 1 & YES & YES & YES & $1.25$ & $(2,2)$ & NO & 1930\\
$(53,16)$ & 10 & $(3,1)$ & 2 & 1 & YES & YES & YES & $1.25$ & $(2,2)$ & -- & 1931\\
$(53,19)$ & 9 & $(3,1)$ & 2 & 1 & YES & YES & YES & $1.00$ & $(2,2)$ & -- & 1932\\
$(53,23)$ & 9 & $(3,1)$ & 2 & 1 & YES & YES & YES & $0.88$ & $(2,2)$ & -- & 1933\\
$(53,16)$ & 10 & $(4,1)$ & 3 & 1 & YES & YES & YES & $1.25$ & $(2,2)$ & -- & 1934\\
$(53,16)$ & 10 & $(4,1)$ & 3 & 1 & YES & YES & YES & $1.38$ & $(2,2)$ & NO & 1935\\
$(53,19)$ & 9 & $(4,1)$ & 3 & 1 & YES & YES & YES & $1.12$ & $(2,2)$ & -- & 1936\\
$(53,22)$ & 9 & $(4,1)$ & 3 & 1 & YES & YES & YES & $1.12$ & $(2,2)$ & NO & 1937\\
$(53,22)$ & 9 & $(4,1)$ & 3 & 1 & YES & YES & YES & $1.12$ & $(2,2)$ & -- & 1938\\
$(53,23)$ & 9 & $(4,1)$ & 3 & 1 & YES & YES & YES & $0.88$ & $(2,2)$ & NO & 1939\\
$(53,23)$ & 9 & $(4,1)$ & 3 & 1 & YES & YES & YES & $1.25$ & $(2,2)$ & -- & 1940\\
$(53,23)$ & 9 & $(4,1)$ & 3 & 1 & YES & YES & YES & $1.25$ & $(2,2)$ & NO & 1941\\
$(53,11)$ & 10 & $(5,2)$ & 3 & 1 & YES & YES & YES & $1.00$ & $(2,2)$ & NO & 1942\\
$(53,14)$ & 9 & $(5,2)$ & 3 & 1 & YES & YES & YES & $1.00$ & $(2,2)$ & -- & 1943\\
$(53,16)$ & 10 & $(5,1)$ & 4 & 1 & YES & YES & YES & $1.12$ & $(2,2)$ & NO & 1944\\
$(53,16)$ & 10 & $(5,2)$ & 3 & 1 & YES & YES & YES & $1.25$ & $(2,2)$ & 1586 & 1945\\
$(53,16)$ & 10 & $(5,2)$ & 3 & 1 & YES & YES & YES & $1.25$ & $(2,2)$ & -- & 1946\\
$(53,19)$ & 9 & $(5,1)$ & 4 & 1 & YES & YES & YES & $1.12$ & $(2,2)$ & NO & 1947\\
$(53,19)$ & 9 & $(5,1)$ & 4 & 1 & YES & YES & YES & $1.12$ & $(2,2)$ & -- & 1948\\
$(53,19)$ & 9 & $(5,2)$ & 3 & 1 & YES & YES & YES & $1.12$ & $(2,2)$ & NO & 1949\\
$(53,22)$ & 9 & $(5,1)$ & 4 & 1 & YES & YES & YES & $1.25$ & $(2,2)$ & -- & 1950\\
$(53,23)$ & 9 & $(5,2)$ & 3 & 1 & YES & YES & YES & $1.12$ & $(2,2)$ & -- & 1951\\
$(53,23)$ & 9 & $(7,2)$ & 4 & 1 & YES & YES & YES & $1.25$ & $(2,2)$ & NO & 1952\\
$(53,19)$ & 9 & $(8,3)$ & 4 & 1 & YES & YES & YES & $1.12$ & $(2,2)$ & NO & 1953\\
$(53,23)$ & 9 & $(8,3)$ & 4 & 1 & YES & YES & YES & $1.38$ & $(2,2)$ & NO & 1954\\
$(53,23)$ & 9 & $(9,2)$ & 5 & 1 & YES & YES & YES & $1.00$ & $(2,2)$ & -- & 1955\\
$(53,23)$ & 9 & $(9,2)$ & 5 & 1 & YES & YES & YES & $1.25$ & $(2,2)$ & NO & 1956\\
$(53,23)$ & 9 & $(9,2)$ & 5 & 1 & YES & YES & YES & $1.25$ & $(2,2)$ & NO & 1957\\
$(53,16)$ & 10 & $(11,2)$ & 6 & 1 & YES & YES & YES & $1.12$ & $(2,2)$ & -- & 1958\\
$(53,19)$ & 9 & $(11,4)$ & 5 & 1 & YES & YES & YES & $1.12$ & $(2,2)$ & NO & 1959\\
$(53,16)$ & 10 & $(13,4)$ & 6 & 1 & YES & YES & YES & $1.12$ & $(2,2)$ & NO & 1960\\
$(53,14)$ & 9 & $(23,6)$ & 8 & 1 & YES & YES & YES & $1.00$ & $(2,2)$ & NO & 1961\\
$(53,19)$ & 9 & $(25,9)$ & 7 & 1 & YES & YES & YES & $1.00$ & $(2,2)$ & 2336 & 1962\\
$(53,22)$ & 9 & $(29,12)$ & 7 & 1 & YES & YES & YES & $1.25$ & $(2,2)$ & 2521 & 1963\\
$(53,23)$ & 9 & $(30,13)$ & 8 & 1 & YES & YES & YES & $1.00$ & $(2,2)$ & NO & 1964\\
$(53,16)$ & 10 & $(33,10)$ & 8 & 1 & YES & YES & YES & $1.00$ & $(2,2)$ & 2728 & 1965\\
$(53,23)$ & 9 & $(37,16)$ & 9 & 1 & YES & YES & YES & $1.12$ & $(2,2)$ & NO & 1966\\
$(53,23)$ & 9 & $(39,17)$ & 8 & 1 & YES & YES & YES & $1.00$ & $(2,2)$ & NO & 1967\\
$(53,22)$ & 9 & $(41,17)$ & 8 & 1 & YES & YES & YES & $1.00$ & $(2,2)$ & NO & 1968\\
$(53,16)$ & 10 & $(43,13)$ & 9 & 1 & YES & YES & YES & $1.25$ & $(2,2)$ & NO & 1969\\
$(53,22)$ & 9 & $(53,22)$ & 9 & 53 & YES & YES & YES & $1.12$ & $(2,2)$ & NO & 1970\\
$(53,23)$ & 9 & $(53,23)$ & 9 & 53 & YES & YES & YES & $1.00$ & $(2,2)$ & NO & 1971\\
$(54,19)$ & 10 & $(2,1)$ & 1 & 2 & YES & YES & YES & $1.00$ & $(2,2)$ & -- & 1972\\
$(54,19)$ & 10 & $(4,1)$ & 3 & 2 & YES & YES & YES & $1.00$ & $(2,2)$ & -- & 1973\\
$(54,19)$ & 10 & $(5,1)$ & 4 & 1 & YES & YES & YES & $1.12$ & $(2,2)$ & NO & 1974\\
$(54,19)$ & 10 & $(8,3)$ & 4 & 2 & YES & YES & YES & $1.12$ & $(2,2)$ & 1128 & 1975\\
$(54,19)$ & 10 & $(11,4)$ & 5 & 1 & YES & YES & YES & $1.12$ & $(2,2)$ & NO & 1976\\
$(54,19)$ & 10 & $(37,13)$ & 9 & 1 & YES & YES & YES & $1.00$ & $(2,2)$ & NO & 1977\\
$(55,17)$ & 10 & $(2,1)$ & 1 & 1 & YES & YES & YES & $1.12$ & $(2,2)$ & -- & 1978\\
$(55,17)$ & 10 & $(2,1)$ & 1 & 1 & YES & YES & YES & $1.25$ & $(2,2)$ & NO & 1979\\
$(55,21)$ & 8 & $(2,1)$ & 1 & 1 & YES & YES & YES & $1.00$ & $(2,2)$ & -- & 1980\\
$(55,17)$ & 10 & $(3,1)$ & 2 & 1 & YES & YES & YES & $1.25$ & $(2,2)$ & -- & 1981\\
$(55,17)$ & 10 & $(3,1)$ & 2 & 1 & YES & YES & YES & $1.38$ & $(2,2)$ & NO & 1982\\
$(55,24)$ & 9 & $(3,1)$ & 2 & 1 & YES & YES & YES & $1.12$ & $(2,2)$ & NO & 1983\\
$(55,24)$ & 9 & $(3,1)$ & 2 & 1 & YES & YES & YES & $1.12$ & $(2,2)$ & -- & 1984\\
$(55,24)$ & 9 & $(3,1)$ & 2 & 1 & YES & YES & YES & $1.12$ & $(2,2)$ & NO & 1985\\
$(55,17)$ & 10 & $(4,1)$ & 3 & 1 & YES & YES & YES & $1.12$ & $(2,2)$ & -- & 1986\\
$(55,23)$ & 9 & $(4,1)$ & 3 & 1 & YES & YES & YES & $1.12$ & $(2,2)$ & -- & 1987\\
$(55,23)$ & 9 & $(4,1)$ & 3 & 1 & YES & YES & YES & $1.25$ & $(2,2)$ & NO & 1988\\
$(55,16)$ & 9 & $(5,2)$ & 3 & 5 & YES & YES & YES & $1.12$ & $(2,2)$ & -- & 1989\\
$(55,16)$ & 9 & $(5,2)$ & 3 & 5 & YES & YES & YES & $1.12$ & $(2,2)$ & 2245 & 1990\\
$(55,16)$ & 9 & $(5,2)$ & 3 & 5 & YES & YES & YES & $1.25$ & $(2,2)$ & NO & 1991\\
$(55,17)$ & 10 & $(5,1)$ & 4 & 5 & YES & YES & YES & $1.12$ & $(2,2)$ & -- & 1992\\
$(55,17)$ & 10 & $(5,1)$ & 4 & 5 & YES & YES & YES & $1.25$ & $(2,2)$ & NO & 1993\\
$(55,21)$ & 8 & $(5,2)$ & 3 & 5 & YES & YES & YES & $1.00$ & $(2,2)$ & NO & 1994\\
$(55,21)$ & 8 & $(5,2)$ & 3 & 5 & YES & YES & YES & $1.00$ & $(2,2)$ & -- & 1995\\
$(55,21)$ & 8 & $(5,2)$ & 3 & 5 & YES & YES & YES & $1.00$ & $(2,2)$ & 1817 & 1996\\
$(55,23)$ & 9 & $(5,1)$ & 4 & 5 & YES & YES & YES & $1.00$ & $(2,2)$ & NO & 1997\\
$(55,23)$ & 9 & $(5,1)$ & 4 & 5 & YES & YES & NO(2) & $1.25$ & $(4,1)$ & -- & 1998\\
$(55,23)$ & 9 & $(5,1)$ & 4 & 5 & YES & YES & YES & $1.25$ & $(2,2)$ & NO & 1999\\
$(55,24)$ & 9 & $(5,2)$ & 3 & 5 & YES & YES & YES & $1.25$ & $(2,2)$ & -- & 2000\\
$(55,24)$ & 9 & $(5,2)$ & 3 & 5 & YES & YES & YES & $1.00$ & $(2,2)$ & 1595 & 2001\\
$(55,21)$ & 8 & $(7,2)$ & 4 & 1 & YES & YES & YES & $1.12$ & $(2,2)$ & NO & 2002\\
$(55,16)$ & 9 & $(9,2)$ & 5 & 1 & YES & YES & YES & $1.00$ & $(2,2)$ & NO & 2003\\
$(55,17)$ & 10 & $(9,2)$ & 5 & 1 & YES & YES & YES & $1.00$ & $(2,2)$ & -- & 2004\\
$(55,24)$ & 9 & $(9,4)$ & 5 & 1 & YES & YES & YES & $1.12$ & $(2,2)$ & NO & 2005\\
$(55,16)$ & 9 & $(11,3)$ & 5 & 11 & YES & YES & YES & $1.00$ & $(2,2)$ & NO & 2006\\
$(55,21)$ & 8 & $(11,4)$ & 5 & 11 & YES & YES & YES & $1.12$ & $(2,2)$ & NO & 2007\\
$(55,17)$ & 10 & $(16,5)$ & 7 & 1 & YES & YES & YES & $1.12$ & $(2,2)$ & NO & 2008\\
$(55,21)$ & 8 & $(18,7)$ & 6 & 1 & YES & YES & YES & $1.00$ & $(2,2)$ & NO & 2009\\
$(55,17)$ & 10 & $(23,7)$ & 7 & 1 & YES & YES & YES & $1.12$ & $(2,2)$ & NO & 2010\\
$(55,17)$ & 10 & $(29,9)$ & 8 & 1 & YES & YES & YES & $1.12$ & $(2,2)$ & 2589 & 2011\\
$(55,21)$ & 8 & $(29,11)$ & 7 & 1 & YES & YES & YES & $1.00$ & $(2,2)$ & NO & 2012\\
$(55,24)$ & 9 & $(30,13)$ & 8 & 5 & YES & YES & YES & $1.25$ & $(2,2)$ & 3035 & 2013\\
$(55,23)$ & 9 & $(31,13)$ & 7 & 1 & YES & YES & YES & $1.00$ & $(2,2)$ & 2660 & 2014\\
$(55,16)$ & 9 & $(38,11)$ & 9 & 1 & YES & YES & YES & $1.00$ & $(2,2)$ & NO & 2015\\
$(55,17)$ & 10 & $(42,13)$ & 9 & 1 & YES & YES & YES & $1.12$ & $(2,2)$ & NO & 2016\\
$(55,23)$ & 9 & $(43,18)$ & 8 & 1 & YES & YES & YES & $1.12$ & $(2,2)$ & NO & 2017\\
$(55,16)$ & 9 & $(55,16)$ & 9 & 55 & YES & YES & YES & $1.12$ & $(2,2)$ & NO & 2018\\
$(55,17)$ & 10 & $(55,17)$ & 10 & 55 & YES & YES & YES & $1.12$ & $(2,2)$ & NO & 2019\\
$(56,23)$ & 9 & $(2,1)$ & 1 & 2 & YES & YES & YES & $1.00$ & $(2,2)$ & -- & 2020\\
$(56,23)$ & 9 & $(2,1)$ & 1 & 2 & YES & YES & YES & $1.00$ & $(2,2)$ & 788 & 2021\\
$(56,17)$ & 9 & $(3,1)$ & 2 & 1 & YES & YES & YES & $1.00$ & $(2,2)$ & -- & 2022\\
$(56,17)$ & 9 & $(3,1)$ & 2 & 1 & YES & YES & YES & $1.12$ & $(2,2)$ & NO & 2023\\
$(56,23)$ & 9 & $(3,1)$ & 2 & 1 & YES & YES & YES & $1.00$ & $(2,2)$ & -- & 2024\\
$(56,23)$ & 9 & $(3,1)$ & 2 & 1 & YES & YES & YES & $1.12$ & $(2,2)$ & NO & 2025\\
$(56,15)$ & 9 & $(4,1)$ & 3 & 4 & YES & YES & YES & $1.12$ & $(2,2)$ & NO & 2026\\
$(56,15)$ & 9 & $(4,1)$ & 3 & 4 & YES & YES & YES & $1.12$ & $(2,2)$ & -- & 2027\\
$(56,23)$ & 9 & $(4,1)$ & 3 & 4 & YES & YES & YES & $1.00$ & $(2,2)$ & NO & 2028\\
$(56,23)$ & 9 & $(4,1)$ & 3 & 4 & YES & YES & YES & $1.00$ & $(2,2)$ & -- & 2029\\
$(56,13)$ & 10 & $(5,2)$ & 3 & 1 & YES & YES & NO(2) & $1.11$ & $(2,2)$ & NO & 2030\\
$(56,13)$ & 10 & $(5,2)$ & 3 & 1 & YES & YES & NO(2) & $1.11$ & $(2,2)$ & -- & 2031\\
$(56,15)$ & 9 & $(5,2)$ & 3 & 1 & YES & YES & YES & $1.12$ & $(2,2)$ & -- & 2032\\
$(56,17)$ & 9 & $(5,2)$ & 3 & 1 & YES & YES & YES & $1.12$ & $(2,2)$ & -- & 2033\\
$(56,17)$ & 9 & $(5,2)$ & 3 & 1 & YES & YES & YES & $1.25$ & $(2,2)$ & NO & 2034\\
$(56,17)$ & 9 & $(5,2)$ & 3 & 1 & YES & YES & YES & $1.12$ & $(2,2)$ & NO & 2035\\
$(56,23)$ & 9 & $(5,1)$ & 4 & 1 & YES & YES & YES & $1.00$ & $(2,2)$ & NO & 2036\\
$(56,23)$ & 9 & $(5,1)$ & 4 & 1 & YES & YES & YES & $1.25$ & $(2,2)$ & -- & 2037\\
$(56,23)$ & 9 & $(5,2)$ & 3 & 1 & YES & YES & YES & $1.25$ & $(2,2)$ & NO & 2038\\
$(56,23)$ & 9 & $(5,2)$ & 3 & 1 & YES & YES & YES & $1.25$ & $(2,2)$ & -- & 2039\\
$(56,15)$ & 9 & $(7,3)$ & 4 & 7 & YES & YES & YES & $1.12$ & $(2,2)$ & NO & 2040\\
$(56,17)$ & 9 & $(7,3)$ & 4 & 7 & YES & YES & YES & $1.25$ & $(2,2)$ & NO & 2041\\
$(56,23)$ & 9 & $(7,2)$ & 4 & 7 & YES & YES & YES & $1.12$ & $(2,2)$ & NO & 2042\\
$(56,23)$ & 9 & $(7,3)$ & 4 & 7 & YES & YES & YES & $1.00$ & $(2,2)$ & NO & 2043\\
$(56,23)$ & 9 & $(8,3)$ & 4 & 8 & YES & YES & YES & $1.12$ & $(2,2)$ & NO & 2044\\
$(56,13)$ & 10 & $(10,3)$ & 5 & 2 & YES & YES & YES & $1.00$ & $(2,2)$ & NO & 2045\\
$(56,15)$ & 9 & $(10,3)$ & 5 & 2 & YES & YES & YES & $1.00$ & $(2,2)$ & NO & 2046\\
$(56,15)$ & 9 & $(11,2)$ & 6 & 1 & YES & YES & YES & $0.88$ & $(2,2)$ & NO & 2047\\
$(56,17)$ & 9 & $(11,3)$ & 5 & 1 & YES & YES & YES & $0.88$ & $(2,2)$ & NO & 2048\\
$(56,23)$ & 9 & $(12,5)$ & 5 & 4 & YES & YES & YES & $1.00$ & $(2,2)$ & NO & 2049\\
$(56,17)$ & 9 & $(13,4)$ & 6 & 1 & YES & YES & YES & $1.00$ & $(2,2)$ & NO & 2050\\
$(56,17)$ & 9 & $(17,5)$ & 6 & 1 & YES & YES & YES & $0.88$ & $(2,2)$ & NO & 2051\\
$(56,23)$ & 9 & $(22,9)$ & 7 & 2 & YES & YES & YES & $1.00$ & $(2,2)$ & 2225 & 2052\\
$(56,17)$ & 9 & $(27,8)$ & 7 & 1 & YES & YES & YES & $1.00$ & $(2,2)$ & NO & 2053\\
$(56,23)$ & 9 & $(27,11)$ & 8 & 1 & YES & YES & YES & $1.12$ & $(2,2)$ & 3046 & 2054\\
$(56,13)$ & 10 & $(31,7)$ & 8 & 1 & YES & YES & YES & $1.12$ & $(2,2)$ & NO & 2055\\
$(56,15)$ & 9 & $(34,9)$ & 8 & 2 & YES & YES & YES & $1.12$ & $(2,2)$ & NO & 2056\\
$(56,13)$ & 10 & $(38,9)$ & 9 & 2 & YES & YES & YES & $1.25$ & $(2,2)$ & 3225 & 2057\\
$(56,23)$ & 9 & $(39,16)$ & 8 & 1 & YES & YES & YES & $1.12$ & $(2,2)$ & NO & 2058\\
$(56,17)$ & 9 & $(53,16)$ & 10 & 1 & YES & YES & YES & $1.12$ & $(2,2)$ & 3058 & 2059\\
$(56,17)$ & 9 & $(56,17)$ & 9 & 56 & YES & YES & YES & $1.25$ & $(2,2)$ & NO & 2060\\
$(56,23)$ & 9 & $(56,23)$ & 9 & 56 & YES & YES & YES & $1.12$ & $(2,2)$ & NO & 2061\\
$(57,13)$ & 9 & $(2,1)$ & 1 & 1 & YES & YES & YES & $1.00$ & $(2,2)$ & -- & 2062\\
$(57,25)$ & 9 & $(2,1)$ & 1 & 1 & YES & YES & YES & $1.12$ & $(2,2)$ & NO & 2063\\
$(57,25)$ & 9 & $(2,1)$ & 1 & 1 & YES & YES & YES & $1.12$ & $(2,2)$ & -- & 2064\\
$(57,22)$ & 9 & $(3,1)$ & 2 & 3 & YES & YES & YES & $1.12$ & $(2,2)$ & -- & 2065\\
$(57,22)$ & 9 & $(3,1)$ & 2 & 3 & YES & YES & YES & $1.25$ & $(2,2)$ & NO & 2066\\
$(57,25)$ & 9 & $(3,1)$ & 2 & 3 & YES & YES & YES & $1.00$ & $(2,2)$ & NO & 2067\\
$(57,25)$ & 9 & $(3,1)$ & 2 & 3 & YES & YES & YES & $1.00$ & $(2,2)$ & -- & 2068\\
$(57,13)$ & 9 & $(4,1)$ & 3 & 1 & NO & YES & YES & $1.25$ & $(2,2)$ & -- & 2069\\
$(57,16)$ & 9 & $(4,1)$ & 3 & 1 & YES & YES & YES & $1.12$ & $(2,2)$ & NO & 2070\\
$(57,16)$ & 9 & $(4,1)$ & 3 & 1 & YES & YES & YES & $1.12$ & $(2,2)$ & -- & 2071\\
$(57,22)$ & 9 & $(4,1)$ & 3 & 1 & YES & YES & YES & $1.12$ & $(2,2)$ & NO & 2072\\
$(57,22)$ & 9 & $(4,1)$ & 3 & 1 & YES & YES & YES & $1.38$ & $(2,2)$ & NO & 2073\\
$(57,22)$ & 9 & $(4,1)$ & 3 & 1 & YES & YES & YES & $1.38$ & $(2,2)$ & -- & 2074\\
$(57,16)$ & 9 & $(5,2)$ & 3 & 1 & YES & YES & YES & $1.12$ & $(2,2)$ & NO & 2075\\
$(57,22)$ & 9 & $(5,1)$ & 4 & 1 & YES & YES & YES & $1.25$ & $(2,2)$ & NO & 2076\\
$(57,22)$ & 9 & $(5,1)$ & 4 & 1 & YES & YES & YES & $1.25$ & $(2,2)$ & -- & 2077\\
$(57,22)$ & 9 & $(6,1)$ & 5 & 3 & YES & YES & YES & $1.12$ & $(2,2)$ & NO & 2078\\
$(57,25)$ & 9 & $(9,4)$ & 5 & 3 & YES & YES & YES & $1.12$ & $(2,2)$ & NO & 2079\\
$(57,16)$ & 9 & $(15,4)$ & 6 & 3 & YES & YES & YES & $1.12$ & $(2,2)$ & NO & 2080\\
$(57,22)$ & 9 & $(18,7)$ & 6 & 3 & YES & YES & YES & $1.12$ & $(2,2)$ & NO & 2081\\
$(57,25)$ & 9 & $(25,11)$ & 7 & 1 & YES & YES & YES & $1.00$ & $(2,2)$ & 2421 & 2082\\
$(57,16)$ & 9 & $(29,8)$ & 7 & 1 & YES & YES & YES & $1.12$ & $(2,2)$ & NO & 2083\\
$(57,22)$ & 9 & $(31,12)$ & 7 & 1 & YES & YES & YES & $1.12$ & $(2,2)$ & 2692 & 2084\\
$(57,16)$ & 9 & $(43,12)$ & 8 & 1 & YES & YES & YES & $1.12$ & $(2,2)$ & NO & 2085\\
$(57,22)$ & 9 & $(44,17)$ & 8 & 1 & YES & YES & YES & $1.25$ & $(2,2)$ & NO & 2086\\
$(57,22)$ & 9 & $(57,22)$ & 9 & 57 & YES & YES & YES & $1.12$ & $(2,2)$ & NO & 2087\\
$(58,17)$ & 9 & $(2,1)$ & 1 & 2 & YES & YES & YES & $1.25$ & $(2,2)$ & -- & 2088\\
$(58,17)$ & 9 & $(2,1)$ & 1 & 2 & YES & YES & YES & $1.12$ & $(2,2)$ & NO & 2089\\
$(58,13)$ & 11 & $(3,1)$ & 2 & 1 & YES & YES & YES & $1.12$ & $(2,2)$ & NO & 2090\\
$(58,13)$ & 11 & $(3,1)$ & 2 & 1 & YES & YES & YES & $1.12$ & $(2,2)$ & -- & 2091\\
$(58,17)$ & 9 & $(3,1)$ & 2 & 1 & YES & YES & YES & $1.12$ & $(2,2)$ & NO & 2092\\
$(58,17)$ & 9 & $(3,1)$ & 2 & 1 & YES & YES & YES & $1.12$ & $(2,2)$ & -- & 2093\\
$(58,17)$ & 9 & $(4,1)$ & 3 & 2 & YES & YES & YES & $1.00$ & $(2,2)$ & -- & 2094\\
$(58,21)$ & 10 & $(4,1)$ & 3 & 2 & YES & YES & YES & $1.25$ & $(2,2)$ & -- & 2095\\
$(58,21)$ & 10 & $(4,1)$ & 3 & 2 & YES & YES & YES & $1.25$ & $(2,2)$ & NO & 2096\\
$(58,13)$ & 11 & $(5,2)$ & 3 & 1 & YES & YES & YES & $1.12$ & $(2,2)$ & -- & 2097\\
$(58,17)$ & 9 & $(5,2)$ & 3 & 1 & YES & YES & YES & $1.00$ & $(2,2)$ & -- & 2098\\
$(58,13)$ & 11 & $(6,1)$ & 5 & 2 & YES & YES & YES & $1.12$ & $(2,2)$ & NO & 2099\\
$(58,21)$ & 10 & $(6,1)$ & 5 & 2 & YES & YES & YES & $1.38$ & $(2,2)$ & -- & 2100\\
$(58,21)$ & 10 & $(6,1)$ & 5 & 2 & YES & YES & YES & $1.38$ & $(2,2)$ & NO & 2101\\
$(58,13)$ & 11 & $(14,3)$ & 6 & 2 & YES & YES & YES & $1.12$ & $(2,2)$ & NO & 2102\\
$(58,17)$ & 9 & $(24,7)$ & 7 & 2 & YES & YES & YES & $1.00$ & $(2,2)$ & 2392 & 2103\\
$(58,21)$ & 10 & $(25,9)$ & 7 & 1 & YES & YES & YES & $1.25$ & $(2,2)$ & NO & 2104\\
$(58,17)$ & 9 & $(27,8)$ & 7 & 1 & YES & YES & YES & $1.00$ & $(2,2)$ & NO & 2105\\
$(58,17)$ & 9 & $(41,12)$ & 8 & 1 & YES & YES & YES & $1.00$ & $(2,2)$ & NO & 2106\\
$(58,21)$ & 10 & $(47,17)$ & 9 & 1 & YES & YES & YES & $1.25$ & $(2,2)$ & NO & 2107\\
$(58,17)$ & 9 & $(58,17)$ & 9 & 58 & YES & YES & YES & $1.00$ & $(2,2)$ & NO & 2108\\
$(59,18)$ & 9 & $(2,1)$ & 1 & 1 & YES & YES & YES & $0.88$ & $(2,2)$ & -- & 2109\\
$(59,18)$ & 9 & $(2,1)$ & 1 & 1 & YES & YES & YES & $1.12$ & $(2,2)$ & NO & 2110\\
$(59,21)$ & 10 & $(2,1)$ & 1 & 1 & YES & YES & YES & $1.00$ & $(2,2)$ & -- & 2111\\
$(59,21)$ & 10 & $(2,1)$ & 1 & 1 & YES & YES & YES & $1.12$ & $(2,2)$ & 638 & 2112\\
$(59,23)$ & 9 & $(2,1)$ & 1 & 1 & YES & YES & YES & $1.00$ & $(2,2)$ & -- & 2113\\
$(59,23)$ & 9 & $(2,1)$ & 1 & 1 & YES & YES & YES & $1.12$ & $(2,2)$ & NO & 2114\\
$(59,25)$ & 9 & $(2,1)$ & 1 & 1 & YES & YES & YES & $1.00$ & $(2,2)$ & -- & 2115\\
$(59,26)$ & 9 & $(2,1)$ & 1 & 1 & YES & YES & YES & $1.00$ & $(2,2)$ & -- & 2116\\
$(59,27)$ & 10 & $(2,1)$ & 1 & 1 & YES & YES & YES & $1.12$ & $(2,2)$ & -- & 2117\\
$(59,13)$ & 11 & $(3,1)$ & 2 & 1 & YES & YES & YES & $1.00$ & $(2,2)$ & NO & 2118\\
$(59,13)$ & 11 & $(3,1)$ & 2 & 1 & YES & YES & YES & $1.00$ & $(2,2)$ & -- & 2119\\
$(59,21)$ & 10 & $(3,1)$ & 2 & 1 & YES & YES & YES & $1.12$ & $(2,2)$ & -- & 2120\\
$(59,23)$ & 9 & $(3,1)$ & 2 & 1 & YES & YES & YES & $1.00$ & $(2,2)$ & -- & 2121\\
$(59,23)$ & 9 & $(3,1)$ & 2 & 1 & YES & YES & YES & $1.00$ & $(2,2)$ & NO & 2122\\
$(59,24)$ & 10 & $(3,1)$ & 2 & 1 & YES & YES & YES & $1.25$ & $(2,2)$ & NO & 2123\\
$(59,24)$ & 10 & $(3,1)$ & 2 & 1 & YES & YES & YES & $1.25$ & $(2,2)$ & -- & 2124\\
$(59,25)$ & 9 & $(3,1)$ & 2 & 1 & YES & YES & YES & $1.00$ & $(2,2)$ & -- & 2125\\
$(59,25)$ & 9 & $(3,1)$ & 2 & 1 & YES & YES & YES & $1.25$ & $(2,2)$ & NO & 2126\\
$(59,25)$ & 9 & $(3,1)$ & 2 & 1 & YES & YES & YES & $1.25$ & $(2,2)$ & NO & 2127\\
$(59,26)$ & 9 & $(3,1)$ & 2 & 1 & YES & YES & YES & $1.00$ & $(2,2)$ & -- & 2128\\
$(59,27)$ & 10 & $(3,1)$ & 2 & 1 & YES & YES & YES & $1.12$ & $(2,2)$ & NO & 2129\\
$(59,27)$ & 10 & $(3,1)$ & 2 & 1 & YES & YES & YES & $1.12$ & $(2,2)$ & -- & 2130\\
$(59,21)$ & 10 & $(4,1)$ & 3 & 1 & YES & YES & YES & $1.00$ & $(2,2)$ & NO & 2131\\
$(59,23)$ & 9 & $(4,1)$ & 3 & 1 & YES & YES & YES & $1.00$ & $(2,2)$ & NO & 2132\\
$(59,23)$ & 9 & $(4,1)$ & 3 & 1 & YES & YES & YES & $1.00$ & $(2,2)$ & -- & 2133\\
$(59,23)$ & 9 & $(4,1)$ & 3 & 1 & YES & YES & YES & $1.00$ & $(2,2)$ & NO & 2134\\
$(59,25)$ & 9 & $(4,1)$ & 3 & 1 & YES & YES & YES & $0.88$ & $(2,2)$ & -- & 2135\\
$(59,25)$ & 9 & $(4,1)$ & 3 & 1 & YES & YES & YES & $1.25$ & $(2,2)$ & NO & 2136\\
$(59,27)$ & 10 & $(4,1)$ & 3 & 1 & YES & YES & YES & $1.12$ & $(2,2)$ & NO & 2137\\
$(59,27)$ & 10 & $(4,1)$ & 3 & 1 & YES & YES & YES & $1.12$ & $(2,2)$ & -- & 2138\\
$(59,16)$ & 10 & $(5,1)$ & 4 & 1 & YES & YES & YES & $1.00$ & $(2,2)$ & NO & 2139\\
$(59,18)$ & 9 & $(5,1)$ & 4 & 1 & YES & YES & YES & $1.00$ & $(2,2)$ & NO & 2140\\
$(59,18)$ & 9 & $(5,1)$ & 4 & 1 & YES & YES & YES & $1.00$ & $(2,2)$ & -- & 2141\\
$(59,18)$ & 9 & $(5,2)$ & 3 & 1 & YES & YES & YES & $1.00$ & $(2,2)$ & -- & 2142\\
$(59,18)$ & 9 & $(5,2)$ & 3 & 1 & YES & YES & YES & $1.12$ & $(2,2)$ & NO & 2143\\
$(59,21)$ & 10 & $(5,2)$ & 3 & 1 & YES & YES & YES & $1.12$ & $(2,2)$ & NO & 2144\\
$(59,23)$ & 9 & $(5,1)$ & 4 & 1 & YES & YES & YES & $1.00$ & $(2,2)$ & NO & 2145\\
$(59,23)$ & 9 & $(5,2)$ & 3 & 1 & YES & YES & YES & $1.12$ & $(2,2)$ & NO & 2146\\
$(59,25)$ & 9 & $(5,1)$ & 4 & 1 & YES & YES & YES & $1.12$ & $(2,2)$ & NO & 2147\\
$(59,25)$ & 9 & $(5,1)$ & 4 & 1 & YES & YES & YES & $1.12$ & $(2,2)$ & -- & 2148\\
$(59,25)$ & 9 & $(5,2)$ & 3 & 1 & YES & YES & YES & $1.12$ & $(2,2)$ & NO & 2149\\
$(59,13)$ & 11 & $(6,1)$ & 5 & 1 & YES & YES & YES & $1.12$ & $(2,2)$ & NO & 2150\\
$(59,14)$ & 10 & $(7,3)$ & 4 & 1 & YES & YES & YES & $1.25$ & $(2,2)$ & -- & 2151\\
$(59,27)$ & 10 & $(7,3)$ & 4 & 1 & YES & YES & YES & $1.12$ & $(2,2)$ & NO & 2152\\
$(59,23)$ & 9 & $(8,3)$ & 4 & 1 & YES & YES & YES & $1.00$ & $(2,2)$ & NO & 2153\\
$(59,16)$ & 10 & $(9,2)$ & 5 & 1 & YES & YES & YES & $1.25$ & $(2,2)$ & 2502 & 2154\\
$(59,16)$ & 10 & $(9,2)$ & 5 & 1 & YES & YES & YES & $1.12$ & $(2,2)$ & -- & 2155\\
$(59,18)$ & 9 & $(9,2)$ & 5 & 1 & YES & YES & YES & $0.88$ & $(2,2)$ & NO & 2156\\
$(59,25)$ & 9 & $(9,4)$ & 5 & 1 & YES & YES & YES & $1.25$ & $(2,2)$ & NO & 2157\\
$(59,18)$ & 9 & $(10,3)$ & 5 & 1 & YES & YES & YES & $1.00$ & $(2,2)$ & NO & 2158\\
$(59,25)$ & 9 & $(12,5)$ & 5 & 1 & YES & YES & YES & $1.00$ & $(2,2)$ & 1243 & 2159\\
$(59,13)$ & 11 & $(13,3)$ & 6 & 1 & YES & YES & YES & $1.00$ & $(2,2)$ & NO & 2160\\
$(59,18)$ & 9 & $(13,4)$ & 6 & 1 & YES & YES & YES & $1.12$ & $(2,2)$ & 1840 & 2161\\
$(59,23)$ & 9 & $(13,5)$ & 5 & 1 & YES & YES & YES & $1.00$ & $(2,2)$ & NO & 2162\\
$(59,13)$ & 11 & $(14,3)$ & 6 & 1 & YES & YES & YES & $1.12$ & $(2,2)$ & NO & 2163\\
$(59,16)$ & 10 & $(15,4)$ & 6 & 1 & YES & YES & YES & $1.00$ & $(2,2)$ & NO & 2164\\
$(59,21)$ & 10 & $(17,6)$ & 7 & 1 & YES & YES & YES & $1.00$ & $(2,2)$ & NO & 2165\\
$(59,16)$ & 10 & $(19,5)$ & 7 & 1 & YES & YES & YES & $1.25$ & $(2,2)$ & NO & 2166\\
$(59,25)$ & 9 & $(19,8)$ & 6 & 1 & YES & YES & YES & $1.12$ & $(2,2)$ & NO & 2167\\
$(59,24)$ & 10 & $(22,9)$ & 7 & 1 & YES & YES & YES & $1.25$ & $(2,2)$ & NO & 2168\\
$(59,18)$ & 9 & $(23,7)$ & 7 & 1 & YES & YES & YES & $1.00$ & $(2,2)$ & NO & 2169\\
$(59,23)$ & 9 & $(23,9)$ & 7 & 1 & YES & YES & YES & $1.12$ & $(2,2)$ & 2337 & 2170\\
$(59,27)$ & 10 & $(24,11)$ & 8 & 1 & YES & YES & YES & $1.25$ & $(2,2)$ & NO & 2171\\
$(59,26)$ & 9 & $(25,11)$ & 7 & 1 & YES & YES & YES & $1.00$ & $(2,2)$ & NO & 2172\\
$(59,25)$ & 9 & $(26,11)$ & 7 & 1 & YES & YES & YES & $1.00$ & $(2,2)$ & NO & 2173\\
$(59,24)$ & 10 & $(27,11)$ & 8 & 1 & YES & YES & YES & $1.25$ & $(2,2)$ & NO & 2174\\
$(59,18)$ & 9 & $(33,10)$ & 8 & 1 & YES & YES & YES & $1.00$ & $(2,2)$ & NO & 2175\\
$(59,25)$ & 9 & $(33,14)$ & 8 & 1 & YES & YES & YES & $1.00$ & $(2,2)$ & NO & 2176\\
$(59,23)$ & 9 & $(41,16)$ & 8 & 1 & YES & YES & YES & $1.00$ & $(2,2)$ & NO & 2177\\
$(59,18)$ & 9 & $(49,15)$ & 9 & 1 & YES & YES & YES & $1.00$ & $(2,2)$ & NO & 2178\\
$(59,23)$ & 9 & $(59,23)$ & 9 & 59 & YES & YES & YES & $1.00$ & $(2,2)$ & NO & 2179\\
$(59,25)$ & 9 & $(59,25)$ & 9 & 59 & YES & YES & YES & $0.88$ & $(2,2)$ & NO & 2180\\
$(59,26)$ & 9 & $(59,26)$ & 9 & 59 & YES & YES & YES & $1.00$ & $(2,2)$ & NO & 2181\\
$(59,27)$ & 10 & $(59,27)$ & 10 & 59 & YES & YES & YES & $1.12$ & $(2,2)$ & NO & 2182\\
$(60,23)$ & 9 & $(4,1)$ & 3 & 4 & YES & YES & YES & $1.12$ & $(2,2)$ & -- & 2183\\
$(60,13)$ & 9 & $(5,1)$ & 4 & 5 & NO & YES & YES & $1.25$ & $(2,2)$ & -- & 2184\\
$(60,23)$ & 9 & $(5,1)$ & 4 & 5 & YES & YES & YES & $1.12$ & $(2,2)$ & NO & 2185\\
$(60,11)$ & 11 & $(14,3)$ & 6 & 2 & YES & YES & YES & $1.25$ & $(2,2)$ & 3094 & 2186\\
$(60,23)$ & 9 & $(34,13)$ & 7 & 2 & YES & YES & YES & $1.12$ & $(2,2)$ & 2851 & 2187\\
$(60,23)$ & 9 & $(47,18)$ & 8 & 1 & YES & YES & YES & $1.12$ & $(2,2)$ & NO & 2188\\
$(61,18)$ & 9 & $(2,1)$ & 1 & 1 & YES & YES & YES & $1.00$ & $(2,2)$ & NO & 2189\\
$(61,25)$ & 9 & $(2,1)$ & 1 & 1 & YES & YES & YES & $1.00$ & $(2,2)$ & -- & 2190\\
$(61,25)$ & 9 & $(2,1)$ & 1 & 1 & YES & YES & YES & $1.00$ & $(2,2)$ & NO & 2191\\
$(61,14)$ & 10 & $(3,1)$ & 2 & 1 & YES & YES & YES & $1.12$ & $(2,2)$ & NO & 2192\\
$(61,22)$ & 9 & $(3,1)$ & 2 & 1 & YES & YES & YES & $1.12$ & $(2,2)$ & -- & 2193\\
$(61,24)$ & 10 & $(3,1)$ & 2 & 1 & YES & YES & YES & $1.25$ & $(2,2)$ & NO & 2194\\
$(61,25)$ & 9 & $(3,1)$ & 2 & 1 & YES & YES & YES & $1.00$ & $(2,2)$ & -- & 2195\\
$(61,25)$ & 9 & $(3,1)$ & 2 & 1 & YES & YES & YES & $1.25$ & $(2,2)$ & NO & 2196\\
$(61,25)$ & 9 & $(3,1)$ & 2 & 1 & YES & YES & YES & $1.25$ & $(2,2)$ & NO & 2197\\
$(61,22)$ & 9 & $(4,1)$ & 3 & 1 & YES & YES & YES & $1.00$ & $(2,2)$ & NO & 2198\\
$(61,22)$ & 9 & $(4,1)$ & 3 & 1 & YES & YES & YES & $1.12$ & $(2,2)$ & -- & 2199\\
$(61,25)$ & 9 & $(4,1)$ & 3 & 1 & YES & YES & YES & $1.00$ & $(2,2)$ & -- & 2200\\
$(61,25)$ & 9 & $(4,1)$ & 3 & 1 & YES & YES & YES & $1.12$ & $(2,2)$ & NO & 2201\\
$(61,14)$ & 10 & $(5,2)$ & 3 & 1 & YES & YES & YES & $1.12$ & $(2,2)$ & NO & 2202\\
$(61,18)$ & 9 & $(5,2)$ & 3 & 1 & YES & YES & YES & $1.00$ & $(2,2)$ & -- & 2203\\
$(61,18)$ & 9 & $(5,2)$ & 3 & 1 & YES & YES & YES & $1.12$ & $(2,2)$ & NO & 2204\\
$(61,18)$ & 9 & $(5,2)$ & 3 & 1 & YES & YES & NO(2) & $1.00$ & $(4,1)$ & NO & 2205\\
$(61,22)$ & 9 & $(5,2)$ & 3 & 1 & YES & YES & YES & $1.00$ & $(2,2)$ & -- & 2206\\
$(61,22)$ & 9 & $(5,2)$ & 3 & 1 & YES & YES & YES & $1.12$ & $(2,2)$ & NO & 2207\\
$(61,22)$ & 9 & $(5,2)$ & 3 & 1 & YES & YES & YES & $1.12$ & $(2,2)$ & NO & 2208\\
$(61,25)$ & 9 & $(5,1)$ & 4 & 1 & YES & YES & YES & $1.00$ & $(2,2)$ & -- & 2209\\
$(61,25)$ & 9 & $(5,1)$ & 4 & 1 & YES & YES & YES & $1.00$ & $(2,2)$ & NO & 2210\\
$(61,25)$ & 9 & $(5,2)$ & 3 & 1 & YES & YES & YES & $1.25$ & $(2,2)$ & NO & 2211\\
$(61,25)$ & 9 & $(5,2)$ & 3 & 1 & YES & YES & YES & $1.25$ & $(2,2)$ & -- & 2212\\
$(61,14)$ & 10 & $(7,2)$ & 4 & 1 & YES & YES & YES & $0.88$ & $(2,2)$ & NO & 2213\\
$(61,22)$ & 9 & $(7,3)$ & 4 & 1 & YES & YES & YES & $1.00$ & $(2,2)$ & NO & 2214\\
$(61,24)$ & 10 & $(7,1)$ & 6 & 1 & YES & YES & YES & $1.12$ & $(2,2)$ & NO & 2215\\
$(61,25)$ & 9 & $(7,3)$ & 4 & 1 & YES & YES & YES & $1.12$ & $(2,2)$ & NO & 2216\\
$(61,22)$ & 9 & $(8,3)$ & 4 & 1 & YES & YES & YES & $1.12$ & $(2,2)$ & NO & 2217\\
$(61,17)$ & 9 & $(9,2)$ & 5 & 1 & YES & YES & YES & $1.12$ & $(2,2)$ & NO & 2218\\
$(61,22)$ & 9 & $(9,2)$ & 5 & 1 & YES & YES & YES & $1.12$ & $(2,2)$ & -- & 2219\\
$(61,14)$ & 10 & $(11,2)$ & 6 & 1 & YES & YES & YES & $1.00$ & $(2,2)$ & NO & 2220\\
$(61,18)$ & 9 & $(11,2)$ & 6 & 1 & YES & YES & YES & $1.00$ & $(2,2)$ & NO & 2221\\
$(61,25)$ & 9 & $(12,5)$ & 5 & 1 & YES & YES & YES & $1.12$ & $(2,2)$ & 2685 & 2222\\
$(61,18)$ & 9 & $(13,4)$ & 6 & 1 & YES & YES & YES & $1.00$ & $(2,2)$ & 2808 & 2223\\
$(61,22)$ & 9 & $(14,5)$ & 6 & 1 & YES & YES & YES & $1.12$ & $(2,2)$ & NO & 2224\\
$(61,25)$ & 9 & $(17,7)$ & 6 & 1 & YES & YES & YES & $1.00$ & $(2,2)$ & 2052 & 2225\\
$(61,22)$ & 9 & $(19,7)$ & 6 & 1 & YES & YES & YES & $1.12$ & $(2,2)$ & NO & 2226\\
$(61,14)$ & 10 & $(21,5)$ & 8 & 1 & YES & YES & YES & $1.25$ & $(2,2)$ & NO & 2227\\
$(61,25)$ & 9 & $(22,9)$ & 7 & 1 & YES & YES & YES & $1.12$ & $(2,2)$ & NO & 2228\\
$(61,16)$ & 10 & $(23,6)$ & 8 & 1 & YES & YES & YES & $1.00$ & $(2,2)$ & 2393 & 2229\\
$(61,24)$ & 10 & $(23,9)$ & 7 & 1 & YES & YES & YES & $1.12$ & $(2,2)$ & NO & 2230\\
$(61,18)$ & 9 & $(24,7)$ & 7 & 1 & YES & YES & YES & $1.00$ & $(2,2)$ & NO & 2231\\
$(61,25)$ & 9 & $(27,11)$ & 8 & 1 & YES & YES & YES & $1.25$ & $(2,2)$ & NO & 2232\\
$(61,14)$ & 10 & $(31,7)$ & 8 & 1 & YES & YES & YES & $1.00$ & $(2,2)$ & NO & 2233\\
$(61,25)$ & 9 & $(39,16)$ & 8 & 1 & YES & YES & YES & $1.00$ & $(2,2)$ & NO & 2234\\
$(61,22)$ & 9 & $(47,17)$ & 9 & 1 & YES & YES & YES & $1.12$ & $(2,2)$ & NO & 2235\\
$(61,22)$ & 9 & $(61,22)$ & 9 & 61 & YES & YES & YES & $0.88$ & $(2,2)$ & NO & 2236\\
$(61,25)$ & 9 & $(61,25)$ & 9 & 61 & YES & YES & YES & $1.00$ & $(2,2)$ & NO & 2237\\
$(62,19)$ & 10 & $(3,1)$ & 2 & 1 & YES & YES & YES & $1.12$ & $(2,2)$ & -- & 2238\\
$(62,23)$ & 9 & $(3,1)$ & 2 & 1 & YES & YES & YES & $1.00$ & $(2,2)$ & -- & 2239\\
$(62,27)$ & 9 & $(3,1)$ & 2 & 1 & YES & YES & YES & $1.12$ & $(2,2)$ & -- & 2240\\
$(62,19)$ & 10 & $(4,1)$ & 3 & 2 & YES & YES & YES & $1.00$ & $(2,2)$ & -- & 2241\\
$(62,19)$ & 10 & $(4,1)$ & 3 & 2 & YES & YES & YES & $1.25$ & $(2,2)$ & NO & 2242\\
$(62,23)$ & 9 & $(4,1)$ & 3 & 2 & YES & YES & YES & $1.12$ & $(2,2)$ & NO & 2243\\
$(62,23)$ & 9 & $(4,1)$ & 3 & 2 & YES & YES & YES & $1.12$ & $(2,2)$ & -- & 2244\\
$(62,23)$ & 9 & $(4,1)$ & 3 & 2 & YES & YES & YES & $1.12$ & $(2,2)$ & 1990 & 2245\\
$(62,27)$ & 9 & $(4,1)$ & 3 & 2 & YES & YES & YES & $1.00$ & $(2,2)$ & NO & 2246\\
$(62,19)$ & 10 & $(5,1)$ & 4 & 1 & YES & YES & YES & $1.12$ & $(2,2)$ & NO & 2247\\
$(62,23)$ & 9 & $(5,2)$ & 3 & 1 & YES & YES & YES & $1.25$ & $(2,2)$ & NO & 2248\\
$(62,27)$ & 9 & $(5,2)$ & 3 & 1 & YES & YES & YES & $1.25$ & $(2,2)$ & -- & 2249\\
$(62,19)$ & 10 & $(6,1)$ & 5 & 2 & YES & YES & YES & $1.12$ & $(2,2)$ & NO & 2250\\
$(62,23)$ & 9 & $(7,2)$ & 4 & 1 & YES & YES & YES & $1.25$ & $(2,2)$ & NO & 2251\\
$(62,19)$ & 10 & $(10,3)$ & 5 & 2 & YES & YES & YES & $1.25$ & $(2,2)$ & NO & 2252\\
$(62,23)$ & 9 & $(14,5)$ & 6 & 2 & YES & YES & YES & $1.12$ & $(2,2)$ & NO & 2253\\
$(62,19)$ & 10 & $(23,7)$ & 7 & 1 & YES & YES & YES & $1.12$ & $(2,2)$ & NO & 2254\\
$(62,27)$ & 9 & $(30,13)$ & 8 & 2 & YES & YES & YES & $1.12$ & $(2,2)$ & NO & 2255\\
$(62,23)$ & 9 & $(35,13)$ & 8 & 1 & YES & YES & YES & $1.00$ & $(2,2)$ & NO & 2256\\
$(62,19)$ & 10 & $(36,11)$ & 8 & 2 & YES & YES & YES & $1.12$ & $(2,2)$ & 2936 & 2257\\
$(62,27)$ & 9 & $(39,17)$ & 8 & 1 & YES & YES & YES & $1.00$ & $(2,2)$ & NO & 2258\\
$(62,19)$ & 10 & $(49,15)$ & 9 & 1 & YES & YES & YES & $1.00$ & $(2,2)$ & NO & 2259\\
$(62,19)$ & 10 & $(62,19)$ & 10 & 62 & YES & YES & YES & $1.25$ & $(2,2)$ & NO & 2260\\
$(62,23)$ & 9 & $(62,23)$ & 9 & 62 & YES & YES & YES & $1.12$ & $(2,2)$ & NO & 2261\\
$(62,27)$ & 9 & $(62,27)$ & 9 & 62 & YES & YES & YES & $1.12$ & $(2,2)$ & NO & 2262\\
$(63,26)$ & 9 & $(2,1)$ & 1 & 1 & YES & YES & YES & $1.12$ & $(2,2)$ & -- & 2263\\
$(63,17)$ & 9 & $(3,1)$ & 2 & 3 & YES & YES & YES & $0.88$ & $(2,2)$ & NO & 2264\\
$(63,17)$ & 9 & $(3,1)$ & 2 & 3 & YES & YES & YES & $0.88$ & $(2,2)$ & -- & 2265\\
$(63,26)$ & 9 & $(3,1)$ & 2 & 3 & YES & YES & YES & $1.12$ & $(2,2)$ & NO & 2266\\
$(63,26)$ & 9 & $(3,1)$ & 2 & 3 & YES & YES & YES & $1.12$ & $(2,2)$ & -- & 2267\\
$(63,26)$ & 9 & $(3,1)$ & 2 & 3 & YES & YES & YES & $1.12$ & $(2,2)$ & NO & 2268\\
$(63,17)$ & 9 & $(4,1)$ & 3 & 1 & YES & YES & YES & $1.12$ & $(2,2)$ & NO & 2269\\
$(63,26)$ & 9 & $(4,1)$ & 3 & 1 & YES & YES & YES & $1.12$ & $(2,2)$ & -- & 2270\\
$(63,26)$ & 9 & $(4,1)$ & 3 & 1 & YES & YES & YES & $1.12$ & $(2,2)$ & NO & 2271\\
$(63,26)$ & 9 & $(4,1)$ & 3 & 1 & YES & YES & YES & $1.25$ & $(2,2)$ & NO & 2272\\
$(63,13)$ & 11 & $(5,2)$ & 3 & 1 & YES & YES & YES & $1.12$ & $(2,2)$ & -- & 2273\\
$(63,13)$ & 11 & $(5,2)$ & 3 & 1 & YES & YES & YES & $1.25$ & $(2,2)$ & NO & 2274\\
$(63,26)$ & 9 & $(5,1)$ & 4 & 1 & YES & YES & YES & $1.00$ & $(2,2)$ & NO & 2275\\
$(63,26)$ & 9 & $(5,1)$ & 4 & 1 & YES & YES & YES & $1.12$ & $(2,2)$ & -- & 2276\\
$(63,26)$ & 9 & $(5,2)$ & 3 & 1 & YES & YES & YES & $1.12$ & $(2,2)$ & NO & 2277\\
$(63,17)$ & 9 & $(7,2)$ & 4 & 7 & YES & YES & YES & $0.88$ & $(2,2)$ & -- & 2278\\
$(63,17)$ & 9 & $(7,3)$ & 4 & 7 & YES & YES & YES & $1.25$ & $(2,2)$ & NO & 2279\\
$(63,26)$ & 9 & $(8,3)$ & 4 & 1 & YES & YES & YES & $1.25$ & $(2,2)$ & NO & 2280\\
$(63,17)$ & 9 & $(10,3)$ & 5 & 1 & YES & YES & YES & $0.88$ & $(2,2)$ & NO & 2281\\
$(63,26)$ & 9 & $(12,5)$ & 5 & 3 & YES & YES & YES & $1.00$ & $(2,2)$ & NO & 2282\\
$(63,17)$ & 9 & $(18,5)$ & 6 & 9 & YES & YES & YES & $1.00$ & $(2,2)$ & NO & 2283\\
$(63,17)$ & 9 & $(19,5)$ & 7 & 1 & YES & YES & YES & $0.88$ & $(2,2)$ & NO & 2284\\
$(63,26)$ & 9 & $(29,12)$ & 7 & 1 & YES & YES & YES & $1.12$ & $(2,2)$ & 2694 & 2285\\
$(63,17)$ & 9 & $(41,11)$ & 8 & 1 & YES & YES & YES & $0.88$ & $(2,2)$ & NO & 2286\\
$(63,26)$ & 9 & $(46,19)$ & 8 & 1 & YES & YES & YES & $1.12$ & $(2,2)$ & NO & 2287\\
$(63,26)$ & 9 & $(63,26)$ & 9 & 63 & YES & YES & YES & $1.12$ & $(2,2)$ & NO & 2288\\
$(64,19)$ & 9 & $(2,1)$ & 1 & 2 & YES & YES & YES & $1.12$ & $(2,2)$ & -- & 2289\\
$(64,23)$ & 9 & $(2,1)$ & 1 & 2 & YES & YES & YES & $1.00$ & $(2,2)$ & -- & 2290\\
$(64,23)$ & 9 & $(2,1)$ & 1 & 2 & YES & YES & YES & $1.12$ & $(2,2)$ & NO & 2291\\
$(64,25)$ & 9 & $(2,1)$ & 1 & 2 & YES & YES & YES & $1.12$ & $(2,2)$ & -- & 2292\\
$(64,25)$ & 9 & $(2,1)$ & 1 & 2 & YES & YES & YES & $1.12$ & $(2,2)$ & NO & 2293\\
$(64,27)$ & 9 & $(2,1)$ & 1 & 2 & YES & YES & YES & $0.88$ & $(2,2)$ & -- & 2294\\
$(64,27)$ & 9 & $(2,1)$ & 1 & 2 & YES & YES & YES & $1.12$ & $(2,2)$ & 1268 & 2295\\
$(64,15)$ & 10 & $(3,1)$ & 2 & 1 & YES & YES & YES & $0.88$ & $(2,2)$ & -- & 2296\\
$(64,15)$ & 10 & $(3,1)$ & 2 & 1 & YES & YES & YES & $0.88$ & $(2,2)$ & NO & 2297\\
$(64,19)$ & 9 & $(3,1)$ & 2 & 1 & YES & YES & YES & $1.12$ & $(2,2)$ & NO & 2298\\
$(64,19)$ & 9 & $(3,1)$ & 2 & 1 & YES & YES & YES & $1.12$ & $(2,2)$ & -- & 2299\\
$(64,23)$ & 9 & $(3,1)$ & 2 & 1 & YES & YES & YES & $0.88$ & $(2,2)$ & -- & 2300\\
$(64,23)$ & 9 & $(3,1)$ & 2 & 1 & YES & YES & YES & $1.00$ & $(2,2)$ & 1760 & 2301\\
$(64,25)$ & 9 & $(3,1)$ & 2 & 1 & YES & YES & YES & $1.12$ & $(2,2)$ & NO & 2302\\
$(64,25)$ & 9 & $(3,1)$ & 2 & 1 & YES & YES & YES & $1.12$ & $(2,2)$ & -- & 2303\\
$(64,27)$ & 9 & $(3,1)$ & 2 & 1 & YES & YES & YES & $1.00$ & $(2,2)$ & NO & 2304\\
$(64,27)$ & 9 & $(3,1)$ & 2 & 1 & YES & YES & YES & $1.00$ & $(2,2)$ & -- & 2305\\
$(64,27)$ & 9 & $(3,1)$ & 2 & 1 & YES & YES & YES & $1.00$ & $(2,2)$ & NO & 2306\\
$(64,17)$ & 10 & $(4,1)$ & 3 & 4 & YES & YES & YES & $0.88$ & $(2,2)$ & -- & 2307\\
$(64,23)$ & 9 & $(4,1)$ & 3 & 4 & YES & YES & YES & $1.00$ & $(2,2)$ & -- & 2308\\
$(64,25)$ & 9 & $(4,1)$ & 3 & 4 & YES & YES & YES & $1.12$ & $(2,2)$ & NO & 2309\\
$(64,25)$ & 9 & $(4,1)$ & 3 & 4 & YES & YES & YES & $1.12$ & $(2,2)$ & -- & 2310\\
$(64,25)$ & 9 & $(4,1)$ & 3 & 4 & YES & YES & YES & $1.12$ & $(2,2)$ & NO & 2311\\
$(64,27)$ & 9 & $(4,1)$ & 3 & 4 & YES & YES & YES & $1.00$ & $(2,2)$ & NO & 2312\\
$(64,27)$ & 9 & $(4,1)$ & 3 & 4 & YES & YES & YES & $1.00$ & $(2,2)$ & NO & 2313\\
$(64,27)$ & 9 & $(4,1)$ & 3 & 4 & YES & YES & YES & $1.00$ & $(2,2)$ & -- & 2314\\
$(64,15)$ & 10 & $(5,2)$ & 3 & 1 & YES & YES & YES & $0.88$ & $(2,2)$ & NO & 2315\\
$(64,19)$ & 9 & $(5,1)$ & 4 & 1 & YES & YES & YES & $1.12$ & $(2,2)$ & NO & 2316\\
$(64,19)$ & 9 & $(5,1)$ & 4 & 1 & YES & YES & YES & $1.12$ & $(2,2)$ & -- & 2317\\
$(64,19)$ & 9 & $(5,2)$ & 3 & 1 & YES & YES & YES & $1.12$ & $(2,2)$ & -- & 2318\\
$(64,19)$ & 9 & $(5,2)$ & 3 & 1 & YES & YES & YES & $1.12$ & $(2,2)$ & NO & 2319\\
$(64,23)$ & 9 & $(5,2)$ & 3 & 1 & YES & YES & YES & $1.00$ & $(2,2)$ & 1435 & 2320\\
$(64,25)$ & 9 & $(5,1)$ & 4 & 1 & YES & YES & YES & $1.12$ & $(2,2)$ & NO & 2321\\
$(64,27)$ & 9 & $(5,1)$ & 4 & 1 & YES & YES & YES & $1.00$ & $(2,2)$ & NO & 2322\\
$(64,27)$ & 9 & $(5,1)$ & 4 & 1 & YES & YES & YES & $1.00$ & $(2,2)$ & NO & 2323\\
$(64,27)$ & 9 & $(5,2)$ & 3 & 1 & YES & YES & YES & $1.00$ & $(2,2)$ & NO & 2324\\
$(64,27)$ & 9 & $(5,2)$ & 3 & 1 & YES & YES & YES & $1.25$ & $(2,2)$ & -- & 2325\\
$(64,15)$ & 10 & $(7,3)$ & 4 & 1 & YES & YES & YES & $1.25$ & $(2,2)$ & -- & 2326\\
$(64,25)$ & 9 & $(7,3)$ & 4 & 1 & YES & YES & YES & $1.25$ & $(2,2)$ & NO & 2327\\
$(64,27)$ & 9 & $(7,3)$ & 4 & 1 & YES & YES & YES & $1.12$ & $(2,2)$ & NO & 2328\\
$(64,25)$ & 9 & $(8,3)$ & 4 & 8 & YES & YES & YES & $1.12$ & $(2,2)$ & NO & 2329\\
$(64,17)$ & 10 & $(9,2)$ & 5 & 1 & YES & YES & YES & $1.00$ & $(2,2)$ & -- & 2330\\
$(64,19)$ & 9 & $(9,2)$ & 5 & 1 & YES & YES & YES & $0.88$ & $(2,2)$ & NO & 2331\\
$(64,19)$ & 9 & $(10,3)$ & 5 & 2 & YES & YES & YES & $1.12$ & $(2,2)$ & 1746 & 2332\\
$(64,17)$ & 10 & $(11,3)$ & 5 & 1 & YES & YES & YES & $0.88$ & $(2,2)$ & NO & 2333\\
$(64,19)$ & 9 & $(13,4)$ & 6 & 1 & YES & YES & YES & $0.88$ & $(2,2)$ & NO & 2334\\
$(64,25)$ & 9 & $(13,5)$ & 5 & 1 & YES & YES & YES & $1.00$ & $(2,2)$ & 2835 & 2335\\
$(64,23)$ & 9 & $(14,5)$ & 6 & 2 & YES & YES & YES & $1.00$ & $(2,2)$ & 1962 & 2336\\
$(64,25)$ & 9 & $(18,7)$ & 6 & 2 & YES & YES & YES & $1.12$ & $(2,2)$ & 2170 & 2337\\
$(64,25)$ & 9 & $(23,9)$ & 7 & 1 & YES & YES & YES & $1.25$ & $(2,2)$ & NO & 2338\\
$(64,23)$ & 9 & $(25,9)$ & 7 & 1 & YES & YES & YES & $1.00$ & $(2,2)$ & NO & 2339\\
$(64,17)$ & 10 & $(26,7)$ & 7 & 2 & YES & YES & YES & $1.12$ & $(2,2)$ & NO & 2340\\
$(64,27)$ & 9 & $(26,11)$ & 7 & 2 & YES & YES & YES & $0.88$ & $(2,2)$ & 2595 & 2341\\
$(64,19)$ & 9 & $(27,8)$ & 7 & 1 & YES & YES & YES & $1.00$ & $(2,2)$ & NO & 2342\\
$(64,15)$ & 10 & $(38,9)$ & 9 & 2 & YES & YES & YES & $1.25$ & $(2,2)$ & NO & 2343\\
$(64,23)$ & 9 & $(39,14)$ & 8 & 1 & YES & YES & YES & $0.88$ & $(2,2)$ & NO & 2344\\
$(64,25)$ & 9 & $(41,16)$ & 8 & 1 & YES & YES & YES & $1.12$ & $(2,2)$ & NO & 2345\\
$(64,27)$ & 9 & $(45,19)$ & 8 & 1 & YES & YES & YES & $1.00$ & $(2,2)$ & NO & 2346\\
$(64,19)$ & 9 & $(47,14)$ & 9 & 1 & YES & YES & YES & $0.88$ & $(2,2)$ & NO & 2347\\
$(64,25)$ & 9 & $(64,25)$ & 9 & 64 & YES & YES & YES & $1.12$ & $(2,2)$ & NO & 2348\\
$(64,27)$ & 9 & $(64,27)$ & 9 & 64 & YES & YES & YES & $1.00$ & $(2,2)$ & NO & 2349\\
$(65,17)$ & 10 & $(2,1)$ & 1 & 1 & YES & YES & YES & $1.00$ & $(2,2)$ & -- & 2350\\
$(65,17)$ & 10 & $(2,1)$ & 1 & 1 & YES & YES & YES & $1.12$ & $(2,2)$ & NO & 2351\\
$(65,19)$ & 9 & $(2,1)$ & 1 & 1 & YES & YES & YES & $1.12$ & $(2,2)$ & NO & 2352\\
$(65,19)$ & 9 & $(2,1)$ & 1 & 1 & YES & YES & YES & $1.12$ & $(2,2)$ & -- & 2353\\
$(65,23)$ & 10 & $(2,1)$ & 1 & 1 & YES & YES & YES & $1.25$ & $(2,2)$ & NO & 2354\\
$(65,24)$ & 9 & $(2,1)$ & 1 & 1 & YES & YES & YES & $1.00$ & $(2,2)$ & -- & 2355\\
$(65,24)$ & 9 & $(2,1)$ & 1 & 1 & YES & YES & YES & $1.12$ & $(2,2)$ & NO & 2356\\
$(65,14)$ & 10 & $(3,1)$ & 2 & 1 & YES & YES & YES & $1.12$ & $(2,2)$ & -- & 2357\\
$(65,14)$ & 10 & $(3,1)$ & 2 & 1 & YES & YES & YES & $1.12$ & $(2,2)$ & NO & 2358\\
$(65,19)$ & 9 & $(3,1)$ & 2 & 1 & NO & YES & YES & $1.25$ & $(2,2)$ & -- & 2359\\
$(65,24)$ & 9 & $(3,1)$ & 2 & 1 & YES & YES & YES & $1.00$ & $(2,2)$ & -- & 2360\\
$(65,24)$ & 9 & $(3,1)$ & 2 & 1 & YES & YES & YES & $1.12$ & $(2,2)$ & NO & 2361\\
$(65,24)$ & 9 & $(3,1)$ & 2 & 1 & YES & YES & YES & $1.12$ & $(2,2)$ & NO & 2362\\
$(65,14)$ & 10 & $(4,1)$ & 3 & 1 & YES & YES & YES & $1.12$ & $(2,2)$ & -- & 2363\\
$(65,14)$ & 10 & $(4,1)$ & 3 & 1 & YES & YES & YES & $1.25$ & $(2,2)$ & NO & 2364\\
$(65,17)$ & 10 & $(4,1)$ & 3 & 1 & YES & YES & YES & $1.00$ & $(2,2)$ & -- & 2365\\
$(65,24)$ & 9 & $(4,1)$ & 3 & 1 & YES & YES & YES & $1.25$ & $(2,2)$ & NO & 2366\\
$(65,24)$ & 9 & $(4,1)$ & 3 & 1 & YES & YES & YES & $1.25$ & $(2,2)$ & -- & 2367\\
$(65,27)$ & 10 & $(4,1)$ & 3 & 1 & YES & YES & YES & $1.12$ & $(2,2)$ & -- & 2368\\
$(65,14)$ & 10 & $(5,2)$ & 3 & 5 & YES & YES & YES & $1.00$ & $(2,2)$ & NO & 2369\\
$(65,14)$ & 10 & $(5,2)$ & 3 & 5 & YES & YES & YES & $1.25$ & $(2,2)$ & NO & 2370\\
$(65,19)$ & 9 & $(5,1)$ & 4 & 5 & YES & YES & YES & $1.00$ & $(2,2)$ & NO & 2371\\
$(65,19)$ & 9 & $(5,1)$ & 4 & 5 & YES & YES & YES & $1.00$ & $(2,2)$ & -- & 2372\\
$(65,19)$ & 9 & $(5,2)$ & 3 & 5 & YES & YES & YES & $1.12$ & $(2,2)$ & -- & 2373\\
$(65,24)$ & 9 & $(5,1)$ & 4 & 5 & YES & YES & YES & $1.00$ & $(2,2)$ & -- & 2374\\
$(65,24)$ & 9 & $(5,2)$ & 3 & 5 & YES & YES & YES & $1.12$ & $(2,2)$ & 1739 & 2375\\
$(65,27)$ & 10 & $(5,1)$ & 4 & 5 & YES & YES & YES & $1.25$ & $(2,2)$ & NO & 2376\\
$(65,27)$ & 10 & $(5,1)$ & 4 & 5 & YES & YES & YES & $1.25$ & $(2,2)$ & NO & 2377\\
$(65,27)$ & 10 & $(5,1)$ & 4 & 5 & YES & YES & YES & $1.25$ & $(2,2)$ & -- & 2378\\
$(65,23)$ & 10 & $(6,1)$ & 5 & 1 & YES & YES & YES & $1.12$ & $(2,2)$ & NO & 2379\\
$(65,23)$ & 10 & $(6,1)$ & 5 & 1 & YES & YES & YES & $1.25$ & $(2,2)$ & -- & 2380\\
$(65,27)$ & 10 & $(6,1)$ & 5 & 1 & YES & YES & YES & $1.12$ & $(2,2)$ & NO & 2381\\
$(65,14)$ & 10 & $(7,2)$ & 4 & 1 & YES & YES & YES & $0.88$ & $(2,2)$ & NO & 2382\\
$(65,19)$ & 9 & $(7,2)$ & 4 & 1 & YES & YES & YES & $1.12$ & $(2,2)$ & NO & 2383\\
$(65,24)$ & 9 & $(7,2)$ & 4 & 1 & YES & YES & YES & $1.12$ & $(2,2)$ & NO & 2384\\
$(65,24)$ & 9 & $(8,3)$ & 4 & 1 & YES & YES & YES & $1.12$ & $(2,2)$ & NO & 2385\\
$(65,18)$ & 9 & $(9,2)$ & 5 & 1 & YES & YES & YES & $1.12$ & $(2,2)$ & NO & 2386\\
$(65,14)$ & 10 & $(11,2)$ & 6 & 1 & YES & YES & YES & $0.88$ & $(2,2)$ & NO & 2387\\
$(65,18)$ & 9 & $(11,2)$ & 6 & 1 & YES & YES & YES & $1.12$ & $(2,2)$ & NO & 2388\\
$(65,24)$ & 9 & $(11,4)$ & 5 & 1 & YES & YES & YES & $1.12$ & $(2,2)$ & NO & 2389\\
$(65,19)$ & 9 & $(13,4)$ & 6 & 13 & YES & YES & YES & $1.00$ & $(2,2)$ & 3129 & 2390\\
$(65,23)$ & 10 & $(14,5)$ & 6 & 1 & YES & YES & YES & $1.12$ & $(2,2)$ & NO & 2391\\
$(65,19)$ & 9 & $(17,5)$ & 6 & 1 & YES & YES & YES & $1.00$ & $(2,2)$ & 2103 & 2392\\
$(65,17)$ & 10 & $(19,5)$ & 7 & 1 & YES & YES & YES & $1.00$ & $(2,2)$ & 2229 & 2393\\
$(65,14)$ & 10 & $(24,5)$ & 8 & 1 & YES & YES & YES & $1.25$ & $(2,2)$ & NO & 2394\\
$(65,19)$ & 9 & $(24,7)$ & 7 & 1 & YES & YES & YES & $1.12$ & $(2,2)$ & NO & 2395\\
$(65,24)$ & 9 & $(27,10)$ & 7 & 1 & YES & YES & YES & $1.00$ & $(2,2)$ & 2638 & 2396\\
$(65,27)$ & 10 & $(29,12)$ & 7 & 1 & YES & YES & YES & $1.12$ & $(2,2)$ & NO & 2397\\
$(65,19)$ & 9 & $(31,9)$ & 8 & 1 & YES & YES & YES & $1.00$ & $(2,2)$ & NO & 2398\\
$(65,27)$ & 10 & $(41,17)$ & 8 & 1 & YES & YES & YES & $1.25$ & $(2,2)$ & 3036 & 2399\\
$(65,17)$ & 10 & $(42,11)$ & 9 & 1 & YES & YES & YES & $1.00$ & $(2,2)$ & NO & 2400\\
$(65,27)$ & 10 & $(53,22)$ & 9 & 1 & YES & YES & YES & $1.12$ & $(2,2)$ & NO & 2401\\
$(65,19)$ & 9 & $(58,17)$ & 9 & 1 & YES & YES & YES & $1.00$ & $(2,2)$ & NO & 2402\\
$(65,24)$ & 9 & $(65,24)$ & 9 & 65 & YES & YES & YES & $1.00$ & $(2,2)$ & NO & 2403\\
$(66,25)$ & 9 & $(2,1)$ & 1 & 2 & YES & YES & YES & $1.25$ & $(2,2)$ & -- & 2404\\
$(66,25)$ & 9 & $(2,1)$ & 1 & 2 & YES & YES & YES & $1.25$ & $(2,2)$ & NO & 2405\\
$(66,29)$ & 9 & $(2,1)$ & 1 & 2 & YES & YES & YES & $1.00$ & $(2,2)$ & -- & 2406\\
$(66,25)$ & 9 & $(3,1)$ & 2 & 3 & YES & YES & YES & $1.12$ & $(2,2)$ & NO & 2407\\
$(66,25)$ & 9 & $(3,1)$ & 2 & 3 & YES & YES & YES & $1.00$ & $(2,2)$ & -- & 2408\\
$(66,25)$ & 9 & $(3,1)$ & 2 & 3 & YES & YES & YES & $1.12$ & $(2,2)$ & NO & 2409\\
$(66,29)$ & 9 & $(3,1)$ & 2 & 3 & YES & YES & YES & $1.00$ & $(2,2)$ & -- & 2410\\
$(66,29)$ & 9 & $(3,1)$ & 2 & 3 & YES & YES & YES & $1.00$ & $(2,2)$ & NO & 2411\\
$(66,25)$ & 9 & $(4,1)$ & 3 & 2 & YES & YES & YES & $1.12$ & $(2,2)$ & NO & 2412\\
$(66,25)$ & 9 & $(4,1)$ & 3 & 2 & YES & YES & YES & $1.12$ & $(2,2)$ & -- & 2413\\
$(66,25)$ & 9 & $(5,1)$ & 4 & 1 & YES & YES & YES & $1.12$ & $(2,2)$ & NO & 2414\\
$(66,25)$ & 9 & $(5,1)$ & 4 & 1 & YES & YES & YES & $1.12$ & $(2,2)$ & -- & 2415\\
$(66,25)$ & 9 & $(5,1)$ & 4 & 1 & YES & YES & YES & $1.12$ & $(2,2)$ & NO & 2416\\
$(66,25)$ & 9 & $(5,2)$ & 3 & 1 & YES & YES & YES & $1.25$ & $(2,2)$ & NO & 2417\\
$(66,25)$ & 9 & $(8,3)$ & 4 & 2 & YES & YES & YES & $1.12$ & $(2,2)$ & 1675 & 2418\\
$(66,25)$ & 9 & $(11,4)$ & 5 & 11 & YES & YES & YES & $1.12$ & $(2,2)$ & NO & 2419\\
$(66,25)$ & 9 & $(13,5)$ & 5 & 1 & YES & YES & YES & $1.12$ & $(2,2)$ & 1391 & 2420\\
$(66,29)$ & 9 & $(16,7)$ & 6 & 2 & YES & YES & YES & $1.00$ & $(2,2)$ & 2082 & 2421\\
$(66,25)$ & 9 & $(21,8)$ & 6 & 3 & YES & YES & YES & $1.00$ & $(2,2)$ & NO & 2422\\
$(66,25)$ & 9 & $(29,11)$ & 7 & 1 & YES & YES & YES & $1.12$ & $(2,2)$ & NO & 2423\\
$(66,25)$ & 9 & $(37,14)$ & 8 & 1 & YES & YES & YES & $1.25$ & $(2,2)$ & NO & 2424\\
$(66,29)$ & 9 & $(41,18)$ & 8 & 1 & YES & YES & YES & $1.00$ & $(2,2)$ & NO & 2425\\
$(66,25)$ & 9 & $(66,25)$ & 9 & 66 & YES & YES & YES & $1.00$ & $(2,2)$ & NO & 2426\\
$(67,18)$ & 9 & $(2,1)$ & 1 & 1 & YES & YES & YES & $0.88$ & $(2,2)$ & -- & 2427\\
$(67,21)$ & 11 & $(2,1)$ & 1 & 1 & YES & YES & YES & $1.12$ & $(2,2)$ & -- & 2428\\
$(67,21)$ & 11 & $(2,1)$ & 1 & 1 & YES & YES & YES & $1.25$ & $(2,2)$ & NO & 2429\\
$(67,26)$ & 9 & $(2,1)$ & 1 & 1 & YES & YES & YES & $1.12$ & $(2,2)$ & NO & 2430\\
$(67,26)$ & 9 & $(2,1)$ & 1 & 1 & YES & YES & YES & $1.25$ & $(2,2)$ & -- & 2431\\
$(67,29)$ & 10 & $(2,1)$ & 1 & 1 & YES & YES & YES & $1.12$ & $(2,2)$ & -- & 2432\\
$(67,26)$ & 9 & $(3,1)$ & 2 & 1 & YES & YES & YES & $1.25$ & $(2,2)$ & NO & 2433\\
$(67,26)$ & 9 & $(3,1)$ & 2 & 1 & YES & YES & YES & $1.25$ & $(2,2)$ & -- & 2434\\
$(67,29)$ & 10 & $(3,1)$ & 2 & 1 & YES & YES & YES & $1.12$ & $(2,2)$ & NO & 2435\\
$(67,29)$ & 10 & $(3,1)$ & 2 & 1 & YES & YES & YES & $1.12$ & $(2,2)$ & -- & 2436\\
$(67,29)$ & 10 & $(3,1)$ & 2 & 1 & YES & YES & YES & $1.25$ & $(2,2)$ & NO & 2437\\
$(67,21)$ & 11 & $(4,1)$ & 3 & 1 & YES & YES & YES & $1.12$ & $(2,2)$ & 924 & 2438\\
$(67,29)$ & 10 & $(4,1)$ & 3 & 1 & YES & YES & YES & $1.12$ & $(2,2)$ & -- & 2439\\
$(67,29)$ & 10 & $(4,1)$ & 3 & 1 & YES & YES & YES & $1.25$ & $(2,2)$ & NO & 2440\\
$(67,26)$ & 9 & $(5,2)$ & 3 & 1 & YES & YES & YES & $1.12$ & $(2,2)$ & NO & 2441\\
$(67,29)$ & 10 & $(5,1)$ & 4 & 1 & YES & YES & YES & $1.12$ & $(2,2)$ & NO & 2442\\
$(67,29)$ & 10 & $(5,1)$ & 4 & 1 & YES & YES & YES & $1.12$ & $(2,2)$ & -- & 2443\\
$(67,26)$ & 9 & $(6,1)$ & 5 & 1 & YES & YES & YES & $1.00$ & $(2,2)$ & NO & 2444\\
$(67,18)$ & 9 & $(7,2)$ & 4 & 1 & YES & YES & YES & $0.88$ & $(2,2)$ & NO & 2445\\
$(67,18)$ & 9 & $(7,3)$ & 4 & 1 & YES & YES & YES & $1.00$ & $(2,2)$ & -- & 2446\\
$(67,21)$ & 11 & $(7,2)$ & 4 & 1 & YES & YES & YES & $1.12$ & $(2,2)$ & NO & 2447\\
$(67,26)$ & 9 & $(7,3)$ & 4 & 1 & YES & YES & YES & $1.25$ & $(2,2)$ & NO & 2448\\
$(67,29)$ & 10 & $(9,4)$ & 5 & 1 & YES & YES & YES & $1.12$ & $(2,2)$ & NO & 2449\\
$(67,26)$ & 9 & $(13,5)$ & 5 & 1 & YES & YES & YES & $1.00$ & $(2,2)$ & NO & 2450\\
$(67,29)$ & 10 & $(16,7)$ & 6 & 1 & YES & YES & YES & $1.12$ & $(2,2)$ & 1445 & 2451\\
$(67,26)$ & 9 & $(18,7)$ & 6 & 1 & YES & YES & YES & $1.12$ & $(2,2)$ & NO & 2452\\
$(67,18)$ & 9 & $(19,5)$ & 7 & 1 & YES & YES & YES & $1.00$ & $(2,2)$ & NO & 2453\\
$(67,29)$ & 10 & $(23,10)$ & 7 & 1 & YES & YES & YES & $1.12$ & $(2,2)$ & NO & 2454\\
$(67,29)$ & 10 & $(30,13)$ & 8 & 1 & YES & YES & YES & $1.00$ & $(2,2)$ & NO & 2455\\
$(67,18)$ & 9 & $(34,9)$ & 8 & 1 & YES & YES & YES & $1.00$ & $(2,2)$ & NO & 2456\\
$(67,29)$ & 10 & $(67,29)$ & 10 & 67 & YES & YES & YES & $1.00$ & $(2,2)$ & NO & 2457\\
$(68,25)$ & 9 & $(2,1)$ & 1 & 2 & YES & YES & YES & $1.00$ & $(2,2)$ & NO & 2458\\
$(68,25)$ & 9 & $(2,1)$ & 1 & 2 & YES & YES & YES & $1.00$ & $(2,2)$ & -- & 2459\\
$(68,25)$ & 9 & $(3,1)$ & 2 & 1 & YES & YES & NO(2) & $0.88$ & $(4,1)$ & -- & 2460\\
$(68,25)$ & 9 & $(3,1)$ & 2 & 1 & YES & YES & YES & $1.12$ & $(2,2)$ & NO & 2461\\
$(68,25)$ & 9 & $(3,1)$ & 2 & 1 & YES & YES & YES & $1.00$ & $(2,2)$ & NO & 2462\\
$(68,21)$ & 11 & $(4,1)$ & 3 & 4 & YES & YES & YES & $1.12$ & $(2,2)$ & NO & 2463\\
$(68,21)$ & 11 & $(4,1)$ & 3 & 4 & YES & YES & YES & $1.12$ & $(2,2)$ & -- & 2464\\
$(68,25)$ & 9 & $(4,1)$ & 3 & 4 & YES & YES & YES & $1.00$ & $(2,2)$ & NO & 2465\\
$(68,19)$ & 9 & $(5,2)$ & 3 & 1 & YES & YES & YES & $1.25$ & $(2,2)$ & NO & 2466\\
$(68,19)$ & 9 & $(5,2)$ & 3 & 1 & YES & YES & YES & $1.12$ & $(2,2)$ & -- & 2467\\
$(68,21)$ & 11 & $(5,1)$ & 4 & 1 & YES & YES & YES & $1.12$ & $(2,2)$ & NO & 2468\\
$(68,25)$ & 9 & $(5,1)$ & 4 & 1 & YES & YES & YES & $1.00$ & $(2,2)$ & NO & 2469\\
$(68,25)$ & 9 & $(5,1)$ & 4 & 1 & YES & YES & YES & $1.25$ & $(2,2)$ & -- & 2470\\
$(68,25)$ & 9 & $(5,2)$ & 3 & 1 & YES & YES & YES & $1.12$ & $(2,2)$ & -- & 2471\\
$(68,25)$ & 9 & $(5,2)$ & 3 & 1 & YES & YES & YES & $1.00$ & $(2,2)$ & NO & 2472\\
$(68,21)$ & 11 & $(6,1)$ & 5 & 2 & YES & YES & YES & $1.12$ & $(2,2)$ & NO & 2473\\
$(68,25)$ & 9 & $(6,1)$ & 5 & 2 & YES & YES & YES & $1.12$ & $(2,2)$ & NO & 2474\\
$(68,25)$ & 9 & $(6,1)$ & 5 & 2 & YES & YES & YES & $1.12$ & $(2,2)$ & NO & 2475\\
$(68,19)$ & 9 & $(9,2)$ & 5 & 1 & YES & YES & YES & $1.12$ & $(2,2)$ & NO & 2476\\
$(68,25)$ & 9 & $(11,4)$ & 5 & 1 & YES & YES & YES & $1.00$ & $(2,2)$ & NO & 2477\\
$(68,19)$ & 9 & $(15,4)$ & 6 & 1 & YES & YES & YES & $0.88$ & $(2,2)$ & 3161 & 2478\\
$(68,21)$ & 11 & $(16,5)$ & 7 & 4 & YES & YES & YES & $1.12$ & $(2,2)$ & NO & 2479\\
$(68,25)$ & 9 & $(19,7)$ & 6 & 1 & YES & YES & YES & $1.12$ & $(2,2)$ & NO & 2480\\
$(68,21)$ & 11 & $(29,9)$ & 8 & 1 & YES & YES & YES & $1.12$ & $(2,2)$ & NO & 2481\\
$(68,25)$ & 9 & $(30,11)$ & 7 & 2 & YES & YES & NO(2) & $1.00$ & $(4,1)$ & 2816 & 2482\\
$(68,21)$ & 11 & $(42,13)$ & 9 & 2 & YES & YES & YES & $1.12$ & $(2,2)$ & 3068 & 2483\\
$(68,21)$ & 11 & $(55,17)$ & 10 & 1 & YES & YES & YES & $1.12$ & $(2,2)$ & NO & 2484\\
$(68,25)$ & 9 & $(68,25)$ & 9 & 68 & YES & YES & YES & $0.88$ & $(2,2)$ & NO & 2485\\
$(69,19)$ & 9 & $(2,1)$ & 1 & 1 & YES & YES & YES & $0.88$ & $(2,2)$ & -- & 2486\\
$(69,19)$ & 9 & $(2,1)$ & 1 & 1 & YES & YES & YES & $1.00$ & $(2,2)$ & NO & 2487\\
$(69,29)$ & 9 & $(2,1)$ & 1 & 1 & YES & YES & YES & $1.12$ & $(2,2)$ & -- & 2488\\
$(69,31)$ & 10 & $(2,1)$ & 1 & 1 & YES & YES & YES & $1.12$ & $(2,2)$ & -- & 2489\\
$(69,19)$ & 9 & $(3,1)$ & 2 & 3 & YES & YES & YES & $0.88$ & $(2,2)$ & -- & 2490\\
$(69,19)$ & 9 & $(3,1)$ & 2 & 3 & YES & YES & YES & $1.00$ & $(2,2)$ & NO & 2491\\
$(69,16)$ & 11 & $(4,1)$ & 3 & 1 & YES & YES & YES & $1.12$ & $(2,2)$ & NO & 2492\\
$(69,16)$ & 11 & $(4,1)$ & 3 & 1 & YES & YES & YES & $1.25$ & $(2,2)$ & -- & 2493\\
$(69,19)$ & 9 & $(4,1)$ & 3 & 1 & YES & YES & YES & $1.12$ & $(2,2)$ & -- & 2494\\
$(69,19)$ & 9 & $(4,1)$ & 3 & 1 & YES & YES & YES & $1.25$ & $(2,2)$ & NO & 2495\\
$(69,16)$ & 11 & $(5,2)$ & 3 & 1 & YES & YES & YES & $1.25$ & $(2,2)$ & NO & 2496\\
$(69,16)$ & 11 & $(5,2)$ & 3 & 1 & YES & YES & YES & $1.25$ & $(2,2)$ & -- & 2497\\
$(69,19)$ & 9 & $(5,2)$ & 3 & 1 & YES & YES & YES & $1.12$ & $(2,2)$ & NO & 2498\\
$(69,19)$ & 9 & $(5,2)$ & 3 & 1 & YES & YES & YES & $1.12$ & $(2,2)$ & -- & 2499\\
$(69,19)$ & 9 & $(5,2)$ & 3 & 1 & YES & YES & YES & $1.12$ & $(2,2)$ & NO & 2500\\
$(69,13)$ & 11 & $(7,2)$ & 4 & 1 & YES & YES & YES & $0.88$ & $(2,2)$ & NO & 2501\\
$(69,16)$ & 11 & $(7,2)$ & 4 & 1 & YES & YES & YES & $1.25$ & $(2,2)$ & 2154 & 2502\\
$(69,19)$ & 9 & $(7,2)$ & 4 & 1 & YES & YES & YES & $0.88$ & $(2,2)$ & 1457 & 2503\\
$(69,29)$ & 9 & $(7,3)$ & 4 & 1 & YES & YES & YES & $1.12$ & $(2,2)$ & NO & 2504\\
$(69,19)$ & 9 & $(9,2)$ & 5 & 3 & YES & YES & YES & $1.00$ & $(2,2)$ & NO & 2505\\
$(69,31)$ & 10 & $(9,4)$ & 5 & 3 & YES & YES & YES & $1.12$ & $(2,2)$ & NO & 2506\\
$(69,16)$ & 11 & $(11,2)$ & 6 & 1 & YES & YES & YES & $1.12$ & $(2,2)$ & NO & 2507\\
$(69,19)$ & 9 & $(15,4)$ & 6 & 3 & YES & YES & YES & $1.00$ & $(2,2)$ & NO & 2508\\
$(69,19)$ & 9 & $(18,5)$ & 6 & 3 & YES & YES & YES & $0.88$ & $(2,2)$ & NO & 2509\\
$(69,16)$ & 11 & $(21,5)$ & 8 & 3 & YES & YES & YES & $1.12$ & $(2,2)$ & NO & 2510\\
$(69,19)$ & 9 & $(51,14)$ & 9 & 3 & YES & YES & YES & $1.00$ & $(2,2)$ & NO & 2511\\
$(70,29)$ & 9 & $(3,1)$ & 2 & 1 & YES & YES & YES & $1.12$ & $(2,2)$ & NO & 2512\\
$(70,29)$ & 9 & $(3,1)$ & 2 & 1 & YES & YES & YES & $1.00$ & $(2,2)$ & NO & 2513\\
$(70,29)$ & 9 & $(3,1)$ & 2 & 1 & YES & YES & YES & $1.00$ & $(2,2)$ & -- & 2514\\
$(70,29)$ & 9 & $(4,1)$ & 3 & 2 & YES & YES & YES & $1.00$ & $(2,2)$ & -- & 2515\\
$(70,29)$ & 9 & $(5,1)$ & 4 & 5 & YES & YES & YES & $1.12$ & $(2,2)$ & -- & 2516\\
$(70,29)$ & 9 & $(5,2)$ & 3 & 5 & YES & YES & YES & $1.12$ & $(2,2)$ & NO & 2517\\
$(70,29)$ & 9 & $(5,2)$ & 3 & 5 & YES & YES & YES & $1.25$ & $(2,2)$ & -- & 2518\\
$(70,29)$ & 9 & $(5,2)$ & 3 & 5 & YES & YES & YES & $1.12$ & $(2,2)$ & 1539 & 2519\\
$(70,29)$ & 9 & $(8,3)$ & 4 & 2 & YES & YES & YES & $1.12$ & $(2,2)$ & NO & 2520\\
$(70,29)$ & 9 & $(12,5)$ & 5 & 2 & YES & YES & YES & $1.25$ & $(2,2)$ & 1963 & 2521\\
$(70,29)$ & 9 & $(17,7)$ & 6 & 1 & YES & YES & YES & $1.00$ & $(2,2)$ & NO & 2522\\
$(70,29)$ & 9 & $(41,17)$ & 8 & 1 & YES & YES & YES & $1.00$ & $(2,2)$ & NO & 2523\\
$(71,13)$ & 12 & $(2,1)$ & 1 & 1 & YES & YES & YES & $1.00$ & $(2,2)$ & NO & 2524\\
$(71,13)$ & 12 & $(2,1)$ & 1 & 1 & YES & YES & YES & $1.00$ & $(2,2)$ & -- & 2525\\
$(71,19)$ & 10 & $(2,1)$ & 1 & 1 & YES & YES & YES & $1.00$ & $(2,2)$ & -- & 2526\\
$(71,19)$ & 10 & $(2,1)$ & 1 & 1 & YES & YES & YES & $1.12$ & $(2,2)$ & NO & 2527\\
$(71,21)$ & 9 & $(2,1)$ & 1 & 1 & YES & YES & YES & $1.00$ & $(2,2)$ & NO & 2528\\
$(71,22)$ & 10 & $(2,1)$ & 1 & 1 & YES & YES & YES & $1.12$ & $(2,2)$ & -- & 2529\\
$(71,22)$ & 10 & $(2,1)$ & 1 & 1 & YES & YES & YES & $1.12$ & $(2,2)$ & NO & 2530\\
$(71,26)$ & 9 & $(2,1)$ & 1 & 1 & YES & YES & YES & $1.00$ & $(2,2)$ & NO & 2531\\
$(71,26)$ & 9 & $(2,1)$ & 1 & 1 & YES & YES & YES & $1.00$ & $(2,2)$ & -- & 2532\\
$(71,27)$ & 9 & $(2,1)$ & 1 & 1 & YES & YES & YES & $1.00$ & $(2,2)$ & NO & 2533\\
$(71,27)$ & 9 & $(2,1)$ & 1 & 1 & YES & YES & YES & $1.12$ & $(2,2)$ & -- & 2534\\
$(71,29)$ & 10 & $(2,1)$ & 1 & 1 & YES & YES & YES & $1.00$ & $(2,2)$ & -- & 2535\\
$(71,29)$ & 10 & $(2,1)$ & 1 & 1 & YES & YES & YES & $1.12$ & $(2,2)$ & NO & 2536\\
$(71,30)$ & 9 & $(2,1)$ & 1 & 1 & YES & YES & YES & $1.00$ & $(2,2)$ & -- & 2537\\
$(71,31)$ & 10 & $(2,1)$ & 1 & 1 & YES & YES & YES & $1.25$ & $(2,2)$ & -- & 2538\\
$(71,32)$ & 10 & $(2,1)$ & 1 & 1 & NO & YES & YES & $1.12$ & $(2,2)$ & -- & 2539\\
$(71,13)$ & 12 & $(3,1)$ & 2 & 1 & YES & YES & YES & $1.00$ & $(2,2)$ & NO & 2540\\
$(71,13)$ & 12 & $(3,1)$ & 2 & 1 & YES & YES & YES & $1.00$ & $(2,2)$ & -- & 2541\\
$(71,19)$ & 10 & $(3,1)$ & 2 & 1 & YES & YES & YES & $1.00$ & $(2,2)$ & -- & 2542\\
$(71,19)$ & 10 & $(3,1)$ & 2 & 1 & YES & YES & YES & $1.25$ & $(2,2)$ & NO & 2543\\
$(71,22)$ & 10 & $(3,1)$ & 2 & 1 & YES & YES & YES & $1.00$ & $(2,2)$ & -- & 2544\\
$(71,22)$ & 10 & $(3,1)$ & 2 & 1 & YES & YES & YES & $1.12$ & $(2,2)$ & 1559 & 2545\\
$(71,26)$ & 9 & $(3,1)$ & 2 & 1 & YES & YES & YES & $1.00$ & $(2,2)$ & -- & 2546\\
$(71,26)$ & 9 & $(3,1)$ & 2 & 1 & YES & YES & YES & $1.12$ & $(2,2)$ & NO & 2547\\
$(71,27)$ & 9 & $(3,1)$ & 2 & 1 & YES & YES & YES & $1.12$ & $(2,2)$ & 1533 & 2548\\
$(71,27)$ & 9 & $(3,1)$ & 2 & 1 & YES & YES & YES & $1.12$ & $(2,2)$ & -- & 2549\\
$(71,29)$ & 10 & $(3,1)$ & 2 & 1 & YES & YES & YES & $1.00$ & $(2,2)$ & -- & 2550\\
$(71,30)$ & 9 & $(3,1)$ & 2 & 1 & YES & YES & YES & $1.12$ & $(2,2)$ & NO & 2551\\
$(71,30)$ & 9 & $(3,1)$ & 2 & 1 & YES & YES & YES & $1.12$ & $(2,2)$ & -- & 2552\\
$(71,30)$ & 9 & $(3,1)$ & 2 & 1 & YES & YES & YES & $1.00$ & $(2,2)$ & NO & 2553\\
$(71,31)$ & 10 & $(3,1)$ & 2 & 1 & YES & YES & YES & $1.25$ & $(2,2)$ & NO & 2554\\
$(71,31)$ & 10 & $(3,1)$ & 2 & 1 & YES & YES & YES & $1.25$ & $(2,2)$ & -- & 2555\\
$(71,13)$ & 12 & $(4,1)$ & 3 & 1 & YES & YES & YES & $1.00$ & $(2,2)$ & NO & 2556\\
$(71,19)$ & 10 & $(4,1)$ & 3 & 1 & YES & YES & YES & $1.12$ & $(2,2)$ & -- & 2557\\
$(71,26)$ & 9 & $(4,1)$ & 3 & 1 & YES & YES & YES & $1.00$ & $(2,2)$ & -- & 2558\\
$(71,26)$ & 9 & $(4,1)$ & 3 & 1 & YES & YES & YES & $1.00$ & $(2,2)$ & NO & 2559\\
$(71,29)$ & 10 & $(4,1)$ & 3 & 1 & YES & YES & YES & $1.00$ & $(2,2)$ & -- & 2560\\
$(71,30)$ & 9 & $(4,1)$ & 3 & 1 & YES & YES & YES & $0.88$ & $(2,2)$ & -- & 2561\\
$(71,30)$ & 9 & $(4,1)$ & 3 & 1 & YES & YES & YES & $1.00$ & $(2,2)$ & NO & 2562\\
$(71,31)$ & 10 & $(4,1)$ & 3 & 1 & YES & YES & YES & $1.25$ & $(2,2)$ & -- & 2563\\
$(71,31)$ & 10 & $(4,1)$ & 3 & 1 & YES & YES & YES & $1.25$ & $(2,2)$ & NO & 2564\\
$(71,16)$ & 10 & $(5,2)$ & 3 & 1 & YES & YES & YES & $1.12$ & $(2,2)$ & NO & 2565\\
$(71,19)$ & 10 & $(5,1)$ & 4 & 1 & YES & YES & YES & $1.00$ & $(2,2)$ & NO & 2566\\
$(71,26)$ & 9 & $(5,1)$ & 4 & 1 & YES & YES & YES & $1.00$ & $(2,2)$ & -- & 2567\\
$(71,26)$ & 9 & $(5,2)$ & 3 & 1 & YES & YES & YES & $1.12$ & $(2,2)$ & NO & 2568\\
$(71,27)$ & 9 & $(5,2)$ & 3 & 1 & YES & YES & YES & $1.00$ & $(2,2)$ & NO & 2569\\
$(71,29)$ & 10 & $(5,2)$ & 3 & 1 & YES & YES & YES & $1.12$ & $(2,2)$ & NO & 2570\\
$(71,30)$ & 9 & $(5,1)$ & 4 & 1 & YES & YES & YES & $1.00$ & $(2,2)$ & NO & 2571\\
$(71,30)$ & 9 & $(5,1)$ & 4 & 1 & YES & YES & YES & $1.00$ & $(2,2)$ & NO & 2572\\
$(71,30)$ & 9 & $(5,2)$ & 3 & 1 & YES & YES & YES & $1.00$ & $(2,2)$ & NO & 2573\\
$(71,31)$ & 10 & $(5,1)$ & 4 & 1 & YES & YES & YES & $1.12$ & $(2,2)$ & NO & 2574\\
$(71,31)$ & 10 & $(5,1)$ & 4 & 1 & YES & YES & YES & $1.25$ & $(2,2)$ & -- & 2575\\
$(71,31)$ & 10 & $(5,1)$ & 4 & 1 & YES & YES & YES & $1.38$ & $(2,2)$ & NO & 2576\\
$(71,19)$ & 10 & $(6,1)$ & 5 & 1 & YES & YES & YES & $0.88$ & $(2,2)$ & NO & 2577\\
$(71,31)$ & 10 & $(6,1)$ & 5 & 1 & YES & YES & YES & $1.25$ & $(2,2)$ & NO & 2578\\
$(71,31)$ & 10 & $(6,1)$ & 5 & 1 & YES & YES & YES & $1.25$ & $(2,2)$ & NO & 2579\\
$(71,13)$ & 12 & $(7,1)$ & 6 & 1 & YES & YES & YES & $1.00$ & $(2,2)$ & NO & 2580\\
$(71,30)$ & 9 & $(7,3)$ & 4 & 1 & YES & YES & YES & $1.12$ & $(2,2)$ & NO & 2581\\
$(71,26)$ & 9 & $(8,3)$ & 4 & 1 & YES & YES & YES & $1.25$ & $(2,2)$ & 1544 & 2582\\
$(71,27)$ & 9 & $(8,3)$ & 4 & 1 & YES & YES & YES & $1.12$ & $(2,2)$ & NO & 2583\\
$(71,22)$ & 10 & $(10,3)$ & 5 & 1 & YES & YES & YES & $1.00$ & $(2,2)$ & NO & 2584\\
$(71,19)$ & 10 & $(11,3)$ & 5 & 1 & YES & YES & YES & $1.00$ & $(2,2)$ & NO & 2585\\
$(71,26)$ & 9 & $(11,4)$ & 5 & 1 & YES & YES & YES & $1.12$ & $(2,2)$ & 1928 & 2586\\
$(71,29)$ & 10 & $(12,5)$ & 5 & 1 & YES & YES & YES & $1.00$ & $(2,2)$ & NO & 2587\\
$(71,30)$ & 9 & $(12,5)$ & 5 & 1 & YES & YES & YES & $1.12$ & $(2,2)$ & 2848 & 2588\\
$(71,22)$ & 10 & $(13,4)$ & 6 & 1 & YES & YES & YES & $1.12$ & $(2,2)$ & 2011 & 2589\\
$(71,26)$ & 9 & $(14,5)$ & 6 & 1 & YES & YES & YES & $1.12$ & $(2,2)$ & NO & 2590\\
$(71,13)$ & 12 & $(16,3)$ & 7 & 1 & YES & YES & YES & $1.00$ & $(2,2)$ & NO & 2591\\
$(71,31)$ & 10 & $(16,7)$ & 6 & 1 & YES & YES & YES & $1.25$ & $(2,2)$ & NO & 2592\\
$(71,19)$ & 10 & $(19,5)$ & 7 & 1 & YES & YES & YES & $1.25$ & $(2,2)$ & NO & 2593\\
$(71,26)$ & 9 & $(19,7)$ & 6 & 1 & YES & YES & YES & $1.00$ & $(2,2)$ & NO & 2594\\
$(71,30)$ & 9 & $(19,8)$ & 6 & 1 & YES & YES & YES & $0.88$ & $(2,2)$ & 2341 & 2595\\
$(71,27)$ & 9 & $(21,8)$ & 6 & 1 & YES & YES & YES & $1.12$ & $(2,2)$ & NO & 2596\\
$(71,19)$ & 10 & $(26,7)$ & 7 & 1 & YES & YES & YES & $1.00$ & $(2,2)$ & NO & 2597\\
$(71,30)$ & 9 & $(26,11)$ & 7 & 1 & YES & YES & YES & $1.12$ & $(2,2)$ & NO & 2598\\
$(71,29)$ & 10 & $(27,11)$ & 8 & 1 & YES & YES & YES & $1.00$ & $(2,2)$ & 2731 & 2599\\
$(71,26)$ & 9 & $(30,11)$ & 7 & 1 & YES & YES & YES & $1.00$ & $(2,2)$ & NO & 2600\\
$(71,31)$ & 10 & $(39,17)$ & 8 & 1 & YES & YES & YES & $1.25$ & $(2,2)$ & 3037 & 2601\\
$(71,26)$ & 9 & $(41,15)$ & 8 & 1 & YES & YES & YES & $1.00$ & $(2,2)$ & NO & 2602\\
$(71,22)$ & 10 & $(42,13)$ & 9 & 1 & YES & YES & YES & $1.00$ & $(2,2)$ & NO & 2603\\
$(71,30)$ & 9 & $(45,19)$ & 8 & 1 & YES & YES & YES & $0.88$ & $(2,2)$ & NO & 2604\\
$(71,29)$ & 10 & $(49,20)$ & 9 & 1 & YES & YES & YES & $1.00$ & $(2,2)$ & NO & 2605\\
$(71,22)$ & 10 & $(55,17)$ & 10 & 1 & YES & YES & YES & $1.00$ & $(2,2)$ & NO & 2606\\
$(71,31)$ & 10 & $(55,24)$ & 9 & 1 & YES & YES & YES & $1.12$ & $(2,2)$ & NO & 2607\\
$(71,19)$ & 10 & $(56,15)$ & 9 & 1 & YES & YES & YES & $0.88$ & $(2,2)$ & NO & 2608\\
$(71,13)$ & 12 & $(60,11)$ & 11 & 1 & YES & YES & YES & $1.00$ & $(2,2)$ & NO & 2609\\
$(71,19)$ & 10 & $(71,19)$ & 10 & 71 & YES & YES & YES & $1.00$ & $(2,2)$ & NO & 2610\\
$(71,26)$ & 9 & $(71,26)$ & 9 & 71 & YES & YES & YES & $1.00$ & $(2,2)$ & NO & 2611\\
$(71,27)$ & 9 & $(71,27)$ & 9 & 71 & YES & YES & YES & $1.25$ & $(2,2)$ & NO & 2612\\
$(71,30)$ & 9 & $(71,30)$ & 9 & 71 & YES & YES & YES & $0.88$ & $(2,2)$ & NO & 2613\\
$(71,31)$ & 10 & $(71,31)$ & 10 & 71 & YES & YES & YES & $1.12$ & $(2,2)$ & NO & 2614\\
$(72,13)$ & 12 & $(2,1)$ & 1 & 2 & YES & YES & YES & $0.88$ & $(2,2)$ & NO & 2615\\
$(72,13)$ & 12 & $(3,1)$ & 2 & 3 & YES & YES & YES & $0.88$ & $(2,2)$ & NO & 2616\\
$(72,19)$ & 10 & $(3,1)$ & 2 & 3 & YES & YES & YES & $1.12$ & $(2,2)$ & NO & 2617\\
$(72,13)$ & 12 & $(4,1)$ & 3 & 4 & YES & YES & YES & $0.88$ & $(2,2)$ & NO & 2618\\
$(72,17)$ & 11 & $(4,1)$ & 3 & 4 & YES & YES & YES & $1.12$ & $(2,2)$ & NO & 2619\\
$(72,17)$ & 11 & $(4,1)$ & 3 & 4 & YES & YES & YES & $1.12$ & $(2,2)$ & -- & 2620\\
$(72,19)$ & 10 & $(4,1)$ & 3 & 4 & YES & YES & YES & $1.12$ & $(2,2)$ & -- & 2621\\
$(72,17)$ & 11 & $(5,2)$ & 3 & 1 & YES & YES & YES & $1.12$ & $(2,2)$ & NO & 2622\\
$(72,17)$ & 11 & $(5,2)$ & 3 & 1 & YES & YES & YES & $1.12$ & $(2,2)$ & -- & 2623\\
$(72,19)$ & 10 & $(11,3)$ & 5 & 1 & YES & YES & YES & $1.00$ & $(2,2)$ & NO & 2624\\
$(72,17)$ & 11 & $(14,3)$ & 6 & 2 & YES & YES & YES & $1.12$ & $(2,2)$ & NO & 2625\\
$(72,19)$ & 10 & $(15,4)$ & 6 & 3 & YES & YES & YES & $1.12$ & $(2,2)$ & NO & 2626\\
$(72,17)$ & 11 & $(30,7)$ & 8 & 6 & YES & YES & YES & $1.12$ & $(2,2)$ & NO & 2627\\
$(72,17)$ & 11 & $(59,14)$ & 10 & 1 & YES & YES & YES & $1.12$ & $(2,2)$ & 3213 & 2628\\
$(73,27)$ & 9 & $(2,1)$ & 1 & 1 & YES & YES & YES & $1.12$ & $(2,2)$ & -- & 2629\\
$(73,27)$ & 9 & $(3,1)$ & 2 & 1 & YES & YES & YES & $1.12$ & $(2,2)$ & NO & 2630\\
$(73,27)$ & 9 & $(3,1)$ & 2 & 1 & YES & YES & YES & $1.12$ & $(2,2)$ & -- & 2631\\
$(73,17)$ & 10 & $(4,1)$ & 3 & 1 & YES & YES & YES & $1.12$ & $(2,2)$ & -- & 2632\\
$(73,17)$ & 10 & $(4,1)$ & 3 & 1 & YES & YES & YES & $1.12$ & $(2,2)$ & 1604 & 2633\\
$(73,17)$ & 10 & $(4,1)$ & 3 & 1 & YES & YES & YES & $1.25$ & $(2,2)$ & NO & 2634\\
$(73,27)$ & 9 & $(4,1)$ & 3 & 1 & YES & YES & YES & $1.12$ & $(2,2)$ & -- & 2635\\
$(73,27)$ & 9 & $(5,1)$ & 4 & 1 & YES & YES & YES & $1.00$ & $(2,2)$ & -- & 2636\\
$(73,17)$ & 10 & $(7,3)$ & 4 & 1 & YES & YES & YES & $1.12$ & $(2,2)$ & -- & 2637\\
$(73,27)$ & 9 & $(19,7)$ & 6 & 1 & YES & YES & YES & $1.00$ & $(2,2)$ & 2396 & 2638\\
$(73,27)$ & 9 & $(27,10)$ & 7 & 1 & YES & YES & YES & $1.12$ & $(2,2)$ & NO & 2639\\
$(73,27)$ & 9 & $(46,17)$ & 8 & 1 & YES & YES & YES & $1.00$ & $(2,2)$ & NO & 2640\\
$(73,27)$ & 9 & $(73,27)$ & 9 & 73 & YES & YES & YES & $1.12$ & $(2,2)$ & NO & 2641\\
$(74,29)$ & 10 & $(2,1)$ & 1 & 2 & YES & YES & YES & $1.12$ & $(2,2)$ & -- & 2642\\
$(74,31)$ & 9 & $(2,1)$ & 1 & 2 & YES & YES & NO(2) & $1.12$ & $(4,1)$ & -- & 2643\\
$(74,17)$ & 11 & $(3,1)$ & 2 & 1 & YES & YES & YES & $1.00$ & $(2,2)$ & NO & 2644\\
$(74,17)$ & 11 & $(3,1)$ & 2 & 1 & YES & YES & YES & $1.00$ & $(2,2)$ & -- & 2645\\
$(74,31)$ & 9 & $(3,1)$ & 2 & 1 & YES & YES & YES & $1.12$ & $(2,2)$ & -- & 2646\\
$(74,31)$ & 9 & $(3,1)$ & 2 & 1 & YES & YES & YES & $1.12$ & $(2,2)$ & NO & 2647\\
$(74,31)$ & 9 & $(4,1)$ & 3 & 2 & YES & YES & YES & $1.25$ & $(2,2)$ & NO & 2648\\
$(74,31)$ & 9 & $(4,1)$ & 3 & 2 & YES & YES & YES & $1.25$ & $(2,2)$ & -- & 2649\\
$(74,31)$ & 9 & $(4,1)$ & 3 & 2 & YES & YES & YES & $1.25$ & $(2,2)$ & NO & 2650\\
$(74,17)$ & 11 & $(5,1)$ & 4 & 1 & YES & YES & YES & $1.12$ & $(2,2)$ & NO & 2651\\
$(74,29)$ & 10 & $(5,2)$ & 3 & 1 & YES & YES & YES & $1.12$ & $(2,2)$ & NO & 2652\\
$(74,31)$ & 9 & $(5,1)$ & 4 & 1 & YES & YES & YES & $0.88$ & $(2,2)$ & NO & 2653\\
$(74,31)$ & 9 & $(5,1)$ & 4 & 1 & YES & YES & YES & $1.25$ & $(2,2)$ & NO & 2654\\
$(74,31)$ & 9 & $(5,1)$ & 4 & 1 & YES & YES & NO(2) & $1.12$ & $(4,1)$ & -- & 2655\\
$(74,31)$ & 9 & $(5,2)$ & 3 & 1 & YES & YES & YES & $1.12$ & $(2,2)$ & NO & 2656\\
$(74,17)$ & 11 & $(6,1)$ & 5 & 2 & YES & YES & YES & $1.12$ & $(2,2)$ & NO & 2657\\
$(74,31)$ & 9 & $(7,3)$ & 4 & 1 & YES & YES & YES & $1.12$ & $(2,2)$ & 1606 & 2658\\
$(74,17)$ & 11 & $(9,2)$ & 5 & 1 & YES & YES & YES & $1.12$ & $(2,2)$ & NO & 2659\\
$(74,31)$ & 9 & $(12,5)$ & 5 & 2 & YES & YES & YES & $1.00$ & $(2,2)$ & 2014 & 2660\\
$(74,17)$ & 11 & $(22,5)$ & 7 & 2 & YES & YES & YES & $1.12$ & $(2,2)$ & NO & 2661\\
$(74,31)$ & 9 & $(43,18)$ & 8 & 1 & YES & YES & YES & $1.12$ & $(2,2)$ & NO & 2662\\
$(74,31)$ & 9 & $(74,31)$ & 9 & 74 & YES & YES & YES & $1.00$ & $(2,2)$ & NO & 2663\\
$(75,22)$ & 10 & $(2,1)$ & 1 & 1 & YES & YES & YES & $1.25$ & $(2,2)$ & -- & 2664\\
$(75,22)$ & 10 & $(2,1)$ & 1 & 1 & YES & YES & YES & $1.25$ & $(2,2)$ & NO & 2665\\
$(75,29)$ & 9 & $(2,1)$ & 1 & 1 & YES & YES & YES & $1.00$ & $(2,2)$ & NO & 2666\\
$(75,29)$ & 9 & $(2,1)$ & 1 & 1 & YES & YES & YES & $1.38$ & $(2,2)$ & -- & 2667\\
$(75,31)$ & 9 & $(2,1)$ & 1 & 1 & YES & YES & YES & $1.12$ & $(2,2)$ & -- & 2668\\
$(75,29)$ & 9 & $(3,1)$ & 2 & 3 & YES & YES & YES & $1.12$ & $(2,2)$ & NO & 2669\\
$(75,29)$ & 9 & $(3,1)$ & 2 & 3 & YES & YES & YES & $1.00$ & $(2,2)$ & -- & 2670\\
$(75,31)$ & 9 & $(3,1)$ & 2 & 3 & YES & YES & YES & $1.12$ & $(2,2)$ & -- & 2671\\
$(75,31)$ & 9 & $(3,1)$ & 2 & 3 & YES & YES & YES & $1.12$ & $(2,2)$ & NO & 2672\\
$(75,31)$ & 9 & $(3,1)$ & 2 & 3 & YES & YES & YES & $1.00$ & $(2,2)$ & NO & 2673\\
$(75,22)$ & 10 & $(4,1)$ & 3 & 1 & YES & YES & YES & $1.12$ & $(2,2)$ & NO & 2674\\
$(75,22)$ & 10 & $(4,1)$ & 3 & 1 & YES & YES & YES & $1.12$ & $(2,2)$ & -- & 2675\\
$(75,29)$ & 9 & $(4,1)$ & 3 & 1 & YES & YES & YES & $1.12$ & $(2,2)$ & NO & 2676\\
$(75,29)$ & 9 & $(4,1)$ & 3 & 1 & YES & YES & YES & $1.12$ & $(2,2)$ & -- & 2677\\
$(75,31)$ & 9 & $(4,1)$ & 3 & 1 & YES & YES & YES & $0.88$ & $(2,2)$ & NO & 2678\\
$(75,31)$ & 9 & $(4,1)$ & 3 & 1 & YES & YES & YES & $1.12$ & $(2,2)$ & -- & 2679\\
$(75,17)$ & 10 & $(5,2)$ & 3 & 5 & YES & YES & YES & $1.12$ & $(2,2)$ & NO & 2680\\
$(75,29)$ & 9 & $(5,1)$ & 4 & 5 & YES & YES & YES & $1.25$ & $(2,2)$ & NO & 2681\\
$(75,29)$ & 9 & $(5,1)$ & 4 & 5 & YES & YES & YES & $1.25$ & $(2,2)$ & -- & 2682\\
$(75,29)$ & 9 & $(5,2)$ & 3 & 5 & YES & YES & YES & $1.25$ & $(2,2)$ & 1637 & 2683\\
$(75,31)$ & 9 & $(5,1)$ & 4 & 5 & YES & YES & YES & $0.88$ & $(2,2)$ & NO & 2684\\
$(75,31)$ & 9 & $(5,2)$ & 3 & 5 & YES & YES & YES & $1.12$ & $(2,2)$ & 2222 & 2685\\
$(75,22)$ & 10 & $(6,1)$ & 5 & 3 & YES & YES & YES & $1.00$ & $(2,2)$ & NO & 2686\\
$(75,29)$ & 9 & $(7,3)$ & 4 & 1 & YES & YES & YES & $1.12$ & $(2,2)$ & NO & 2687\\
$(75,14)$ & 11 & $(9,2)$ & 5 & 3 & YES & YES & YES & $1.25$ & $(2,2)$ & NO & 2688\\
$(75,22)$ & 10 & $(10,3)$ & 5 & 5 & YES & YES & YES & $1.12$ & $(2,2)$ & NO & 2689\\
$(75,16)$ & 11 & $(11,2)$ & 6 & 1 & YES & YES & YES & $1.00$ & $(2,2)$ & 2933 & 2690\\
$(75,31)$ & 9 & $(12,5)$ & 5 & 3 & YES & YES & YES & $1.12$ & $(2,2)$ & NO & 2691\\
$(75,29)$ & 9 & $(13,5)$ & 5 & 1 & YES & YES & YES & $1.12$ & $(2,2)$ & 2084 & 2692\\
$(75,14)$ & 11 & $(17,3)$ & 7 & 1 & YES & YES & YES & $1.00$ & $(2,2)$ & NO & 2693\\
$(75,31)$ & 9 & $(17,7)$ & 6 & 1 & YES & YES & YES & $1.12$ & $(2,2)$ & 2285 & 2694\\
$(75,29)$ & 9 & $(18,7)$ & 6 & 3 & YES & YES & YES & $1.12$ & $(2,2)$ & NO & 2695\\
$(75,31)$ & 9 & $(22,9)$ & 7 & 1 & YES & YES & YES & $1.12$ & $(2,2)$ & NO & 2696\\
$(75,16)$ & 11 & $(24,5)$ & 8 & 3 & YES & YES & YES & $1.00$ & $(2,2)$ & NO & 2697\\
$(75,22)$ & 10 & $(24,7)$ & 7 & 3 & YES & YES & YES & $1.12$ & $(2,2)$ & NO & 2698\\
$(75,31)$ & 9 & $(29,12)$ & 7 & 1 & YES & YES & YES & $1.12$ & $(2,2)$ & NO & 2699\\
$(75,17)$ & 10 & $(35,8)$ & 8 & 5 & YES & YES & YES & $1.00$ & $(2,2)$ & NO & 2700\\
$(75,31)$ & 9 & $(46,19)$ & 8 & 1 & YES & YES & YES & $0.88$ & $(2,2)$ & NO & 2701\\
$(75,22)$ & 10 & $(58,17)$ & 9 & 1 & YES & YES & YES & $1.12$ & $(2,2)$ & NO & 2702\\
$(75,31)$ & 9 & $(75,31)$ & 9 & 75 & YES & YES & YES & $0.88$ & $(2,2)$ & NO & 2703\\
$(76,23)$ & 10 & $(2,1)$ & 1 & 2 & YES & YES & YES & $1.00$ & $(2,2)$ & -- & 2704\\
$(76,23)$ & 10 & $(2,1)$ & 1 & 2 & YES & YES & YES & $1.12$ & $(2,2)$ & 1504 & 2705\\
$(76,29)$ & 9 & $(2,1)$ & 1 & 2 & YES & YES & YES & $1.00$ & $(2,2)$ & -- & 2706\\
$(76,31)$ & 10 & $(2,1)$ & 1 & 2 & YES & YES & YES & $1.12$ & $(2,2)$ & NO & 2707\\
$(76,33)$ & 10 & $(2,1)$ & 1 & 2 & YES & YES & YES & $1.12$ & $(2,2)$ & -- & 2708\\
$(76,23)$ & 10 & $(3,1)$ & 2 & 1 & YES & YES & YES & $1.00$ & $(2,2)$ & -- & 2709\\
$(76,23)$ & 10 & $(3,1)$ & 2 & 1 & YES & YES & YES & $1.12$ & $(2,2)$ & NO & 2710\\
$(76,27)$ & 10 & $(3,1)$ & 2 & 1 & YES & YES & YES & $1.00$ & $(2,2)$ & 1671 & 2711\\
$(76,29)$ & 9 & $(3,1)$ & 2 & 1 & YES & YES & YES & $1.12$ & $(2,2)$ & NO & 2712\\
$(76,29)$ & 9 & $(3,1)$ & 2 & 1 & YES & YES & YES & $1.12$ & $(2,2)$ & -- & 2713\\
$(76,29)$ & 9 & $(3,1)$ & 2 & 1 & YES & YES & YES & $1.12$ & $(2,2)$ & NO & 2714\\
$(76,33)$ & 10 & $(3,1)$ & 2 & 1 & YES & YES & YES & $1.12$ & $(2,2)$ & -- & 2715\\
$(76,33)$ & 10 & $(3,1)$ & 2 & 1 & YES & YES & YES & $1.25$ & $(2,2)$ & NO & 2716\\
$(76,23)$ & 10 & $(4,1)$ & 3 & 4 & YES & YES & YES & $1.00$ & $(2,2)$ & -- & 2717\\
$(76,23)$ & 10 & $(4,1)$ & 3 & 4 & YES & YES & YES & $1.25$ & $(2,2)$ & NO & 2718\\
$(76,33)$ & 10 & $(4,1)$ & 3 & 4 & YES & YES & YES & $1.12$ & $(2,2)$ & -- & 2719\\
$(76,33)$ & 10 & $(4,1)$ & 3 & 4 & YES & YES & YES & $1.25$ & $(2,2)$ & NO & 2720\\
$(76,33)$ & 10 & $(4,1)$ & 3 & 4 & YES & YES & YES & $1.25$ & $(2,2)$ & NO & 2721\\
$(76,23)$ & 10 & $(5,2)$ & 3 & 1 & YES & YES & YES & $1.25$ & $(2,2)$ & NO & 2722\\
$(76,33)$ & 10 & $(5,1)$ & 4 & 1 & YES & YES & YES & $1.12$ & $(2,2)$ & -- & 2723\\
$(76,33)$ & 10 & $(5,1)$ & 4 & 1 & YES & YES & YES & $1.25$ & $(2,2)$ & NO & 2724\\
$(76,31)$ & 10 & $(6,1)$ & 5 & 2 & YES & YES & YES & $1.00$ & $(2,2)$ & NO & 2725\\
$(76,33)$ & 10 & $(7,3)$ & 4 & 1 & YES & YES & YES & $1.12$ & $(2,2)$ & NO & 2726\\
$(76,29)$ & 9 & $(8,3)$ & 4 & 4 & YES & YES & YES & $1.12$ & $(2,2)$ & NO & 2727\\
$(76,23)$ & 10 & $(10,3)$ & 5 & 2 & YES & YES & YES & $1.00$ & $(2,2)$ & 1965 & 2728\\
$(76,23)$ & 10 & $(13,4)$ & 6 & 1 & YES & YES & YES & $1.00$ & $(2,2)$ & 1609 & 2729\\
$(76,29)$ & 9 & $(21,8)$ & 6 & 1 & YES & YES & YES & $1.12$ & $(2,2)$ & NO & 2730\\
$(76,31)$ & 10 & $(22,9)$ & 7 & 2 & YES & YES & YES & $1.00$ & $(2,2)$ & 2599 & 2731\\
$(76,23)$ & 10 & $(23,7)$ & 7 & 1 & YES & YES & YES & $1.00$ & $(2,2)$ & NO & 2732\\
$(76,33)$ & 10 & $(30,13)$ & 8 & 2 & YES & YES & YES & $1.12$ & $(2,2)$ & 2893 & 2733\\
$(76,23)$ & 10 & $(33,10)$ & 8 & 1 & YES & YES & YES & $1.00$ & $(2,2)$ & NO & 2734\\
$(76,23)$ & 10 & $(43,13)$ & 9 & 1 & YES & YES & YES & $1.12$ & $(2,2)$ & NO & 2735\\
$(76,33)$ & 10 & $(53,23)$ & 9 & 1 & YES & YES & YES & $1.12$ & $(2,2)$ & NO & 2736\\
$(76,29)$ & 9 & $(55,21)$ & 8 & 1 & YES & YES & YES & $1.12$ & $(2,2)$ & NO & 2737\\
$(76,23)$ & 10 & $(76,23)$ & 10 & 76 & YES & YES & YES & $1.00$ & $(2,2)$ & NO & 2738\\
$(76,29)$ & 9 & $(76,29)$ & 9 & 76 & YES & YES & YES & $1.25$ & $(2,2)$ & NO & 2739\\
$(76,33)$ & 10 & $(76,33)$ & 10 & 76 & YES & YES & YES & $1.12$ & $(2,2)$ & NO & 2740\\
$(77,18)$ & 10 & $(5,2)$ & 3 & 1 & YES & YES & YES & $1.00$ & $(2,2)$ & -- & 2741\\
$(77,18)$ & 10 & $(5,2)$ & 3 & 1 & YES & YES & YES & $1.12$ & $(2,2)$ & NO & 2742\\
$(77,18)$ & 10 & $(5,2)$ & 3 & 1 & YES & YES & YES & $1.12$ & $(2,2)$ & NO & 2743\\
$(77,18)$ & 10 & $(22,5)$ & 7 & 11 & YES & YES & YES & $0.88$ & $(2,2)$ & NO & 2744\\
$(77,18)$ & 10 & $(38,9)$ & 9 & 1 & YES & YES & YES & $1.12$ & $(2,2)$ & NO & 2745\\
$(78,23)$ & 10 & $(2,1)$ & 1 & 2 & YES & YES & YES & $1.12$ & $(2,2)$ & -- & 2746\\
$(78,29)$ & 10 & $(2,1)$ & 1 & 2 & YES & YES & YES & $1.25$ & $(2,2)$ & -- & 2747\\
$(78,23)$ & 10 & $(3,1)$ & 2 & 3 & YES & YES & YES & $1.12$ & $(2,2)$ & NO & 2748\\
$(78,23)$ & 10 & $(3,1)$ & 2 & 3 & YES & YES & YES & $1.12$ & $(2,2)$ & -- & 2749\\
$(78,29)$ & 10 & $(3,1)$ & 2 & 3 & YES & YES & YES & $1.25$ & $(2,2)$ & NO & 2750\\
$(78,29)$ & 10 & $(3,1)$ & 2 & 3 & YES & YES & YES & $1.25$ & $(2,2)$ & -- & 2751\\
$(78,17)$ & 10 & $(4,1)$ & 3 & 2 & YES & YES & YES & $1.12$ & $(2,2)$ & -- & 2752\\
$(78,17)$ & 10 & $(4,1)$ & 3 & 2 & YES & YES & YES & $1.25$ & $(2,2)$ & NO & 2753\\
$(78,23)$ & 10 & $(4,1)$ & 3 & 2 & YES & YES & YES & $1.12$ & $(2,2)$ & NO & 2754\\
$(78,29)$ & 10 & $(4,1)$ & 3 & 2 & YES & YES & YES & $1.12$ & $(2,2)$ & -- & 2755\\
$(78,29)$ & 10 & $(4,1)$ & 3 & 2 & YES & YES & YES & $1.25$ & $(2,2)$ & NO & 2756\\
$(78,29)$ & 10 & $(4,1)$ & 3 & 2 & YES & YES & YES & $1.38$ & $(2,2)$ & NO & 2757\\
$(78,23)$ & 10 & $(5,1)$ & 4 & 1 & YES & YES & YES & $1.12$ & $(2,2)$ & NO & 2758\\
$(78,29)$ & 10 & $(5,1)$ & 4 & 1 & YES & YES & YES & $1.25$ & $(2,2)$ & NO & 2759\\
$(78,29)$ & 10 & $(5,1)$ & 4 & 1 & YES & YES & YES & $1.25$ & $(2,2)$ & -- & 2760\\
$(78,23)$ & 10 & $(10,3)$ & 5 & 2 & YES & YES & YES & $1.12$ & $(2,2)$ & NO & 2761\\
$(78,29)$ & 10 & $(11,4)$ & 5 & 1 & YES & YES & YES & $1.25$ & $(2,2)$ & NO & 2762\\
$(78,17)$ & 10 & $(19,4)$ & 7 & 1 & YES & YES & YES & $0.88$ & $(2,2)$ & NO & 2763\\
$(78,29)$ & 10 & $(19,7)$ & 6 & 1 & YES & YES & YES & $1.25$ & $(2,2)$ & 1714 & 2764\\
$(78,23)$ & 10 & $(27,8)$ & 7 & 3 & YES & YES & YES & $1.00$ & $(2,2)$ & NO & 2765\\
$(78,29)$ & 10 & $(35,13)$ & 8 & 1 & YES & YES & YES & $1.25$ & $(2,2)$ & NO & 2766\\
$(78,17)$ & 10 & $(37,8)$ & 8 & 1 & YES & YES & YES & $0.88$ & $(2,2)$ & NO & 2767\\
$(78,23)$ & 10 & $(44,13)$ & 8 & 2 & YES & YES & YES & $1.12$ & $(2,2)$ & 3124 & 2768\\
$(78,23)$ & 10 & $(61,18)$ & 9 & 1 & YES & YES & YES & $1.12$ & $(2,2)$ & NO & 2769\\
$(78,29)$ & 10 & $(78,29)$ & 10 & 78 & YES & YES & YES & $1.12$ & $(2,2)$ & NO & 2770\\
$(79,17)$ & 11 & $(2,1)$ & 1 & 1 & YES & YES & YES & $1.00$ & $(2,2)$ & -- & 2771\\
$(79,17)$ & 11 & $(2,1)$ & 1 & 1 & YES & YES & YES & $1.12$ & $(2,2)$ & NO & 2772\\
$(79,21)$ & 11 & $(2,1)$ & 1 & 1 & YES & YES & YES & $1.00$ & $(2,2)$ & -- & 2773\\
$(79,21)$ & 11 & $(2,1)$ & 1 & 1 & YES & YES & YES & $1.12$ & $(2,2)$ & NO & 2774\\
$(79,22)$ & 10 & $(2,1)$ & 1 & 1 & YES & YES & YES & $1.12$ & $(2,2)$ & NO & 2775\\
$(79,23)$ & 10 & $(2,1)$ & 1 & 1 & YES & YES & YES & $0.88$ & $(2,2)$ & -- & 2776\\
$(79,23)$ & 10 & $(2,1)$ & 1 & 1 & YES & YES & YES & $1.00$ & $(2,2)$ & NO & 2777\\
$(79,24)$ & 10 & $(2,1)$ & 1 & 1 & YES & YES & YES & $1.12$ & $(2,2)$ & -- & 2778\\
$(79,28)$ & 10 & $(2,1)$ & 1 & 1 & YES & YES & YES & $1.12$ & $(2,2)$ & NO & 2779\\
$(79,29)$ & 9 & $(2,1)$ & 1 & 1 & YES & YES & YES & $1.00$ & $(2,2)$ & NO & 2780\\
$(79,29)$ & 9 & $(2,1)$ & 1 & 1 & YES & YES & YES & $1.12$ & $(2,2)$ & -- & 2781\\
$(79,30)$ & 9 & $(2,1)$ & 1 & 1 & YES & YES & YES & $1.00$ & $(2,2)$ & NO & 2782\\
$(79,30)$ & 9 & $(2,1)$ & 1 & 1 & YES & YES & YES & $1.12$ & $(2,2)$ & -- & 2783\\
$(79,17)$ & 11 & $(3,1)$ & 2 & 1 & YES & YES & YES & $1.00$ & $(2,2)$ & -- & 2784\\
$(79,21)$ & 11 & $(3,1)$ & 2 & 1 & YES & YES & YES & $1.00$ & $(2,2)$ & -- & 2785\\
$(79,22)$ & 10 & $(3,1)$ & 2 & 1 & YES & YES & YES & $1.12$ & $(2,2)$ & NO & 2786\\
$(79,23)$ & 10 & $(3,1)$ & 2 & 1 & YES & YES & YES & $1.00$ & $(2,2)$ & 1199 & 2787\\
$(79,23)$ & 10 & $(3,1)$ & 2 & 1 & YES & YES & YES & $1.12$ & $(2,2)$ & NO & 2788\\
$(79,23)$ & 10 & $(3,1)$ & 2 & 1 & YES & YES & YES & $1.12$ & $(2,2)$ & -- & 2789\\
$(79,24)$ & 10 & $(3,1)$ & 2 & 1 & NO & YES & YES & $1.12$ & $(2,2)$ & -- & 2790\\
$(79,24)$ & 10 & $(3,1)$ & 2 & 1 & YES & YES & YES & $1.12$ & $(2,2)$ & NO & 2791\\
$(79,29)$ & 9 & $(3,1)$ & 2 & 1 & YES & YES & YES & $1.12$ & $(2,2)$ & NO & 2792\\
$(79,29)$ & 9 & $(3,1)$ & 2 & 1 & YES & YES & YES & $1.12$ & $(2,2)$ & -- & 2793\\
$(79,29)$ & 9 & $(3,1)$ & 2 & 1 & YES & YES & YES & $1.12$ & $(2,2)$ & NO & 2794\\
$(79,30)$ & 9 & $(3,1)$ & 2 & 1 & YES & YES & YES & $1.12$ & $(2,2)$ & NO & 2795\\
$(79,30)$ & 9 & $(3,1)$ & 2 & 1 & YES & YES & YES & $1.25$ & $(2,2)$ & -- & 2796\\
$(79,23)$ & 10 & $(4,1)$ & 3 & 1 & YES & YES & YES & $0.88$ & $(2,2)$ & -- & 2797\\
$(79,23)$ & 10 & $(4,1)$ & 3 & 1 & YES & YES & YES & $1.12$ & $(2,2)$ & NO & 2798\\
$(79,29)$ & 9 & $(4,1)$ & 3 & 1 & YES & YES & YES & $1.12$ & $(2,2)$ & -- & 2799\\
$(79,21)$ & 11 & $(5,1)$ & 4 & 1 & YES & YES & YES & $1.00$ & $(2,2)$ & 1825 & 2800\\
$(79,24)$ & 10 & $(5,2)$ & 3 & 1 & YES & YES & YES & $1.12$ & $(2,2)$ & NO & 2801\\
$(79,24)$ & 10 & $(5,2)$ & 3 & 1 & YES & YES & YES & $1.12$ & $(2,2)$ & -- & 2802\\
$(79,29)$ & 9 & $(5,1)$ & 4 & 1 & YES & YES & YES & $1.12$ & $(2,2)$ & -- & 2803\\
$(79,29)$ & 9 & $(5,2)$ & 3 & 1 & YES & YES & YES & $1.00$ & $(2,2)$ & 2968 & 2804\\
$(79,21)$ & 11 & $(6,1)$ & 5 & 1 & YES & YES & YES & $1.00$ & $(2,2)$ & NO & 2805\\
$(79,18)$ & 10 & $(7,2)$ & 4 & 1 & YES & YES & YES & $0.88$ & $(2,2)$ & NO & 2806\\
$(79,22)$ & 10 & $(7,2)$ & 4 & 1 & YES & YES & YES & $1.12$ & $(2,2)$ & NO & 2807\\
$(79,24)$ & 10 & $(7,2)$ & 4 & 1 & YES & YES & YES & $1.00$ & $(2,2)$ & 2223 & 2808\\
$(79,30)$ & 9 & $(8,3)$ & 4 & 1 & YES & YES & YES & $1.12$ & $(2,2)$ & NO & 2809\\
$(79,17)$ & 11 & $(9,2)$ & 5 & 1 & YES & YES & YES & $1.00$ & $(2,2)$ & NO & 2810\\
$(79,23)$ & 10 & $(10,3)$ & 5 & 1 & YES & YES & YES & $0.88$ & $(2,2)$ & NO & 2811\\
$(79,29)$ & 9 & $(11,4)$ & 5 & 1 & YES & YES & YES & $1.00$ & $(2,2)$ & NO & 2812\\
$(79,24)$ & 10 & $(13,4)$ & 6 & 1 & YES & YES & YES & $1.12$ & $(2,2)$ & NO & 2813\\
$(79,30)$ & 9 & $(13,5)$ & 5 & 1 & YES & YES & YES & $1.12$ & $(2,2)$ & 2975 & 2814\\
$(79,21)$ & 11 & $(19,5)$ & 7 & 1 & YES & YES & YES & $1.00$ & $(2,2)$ & NO & 2815\\
$(79,29)$ & 9 & $(19,7)$ & 6 & 1 & YES & YES & NO(2) & $1.00$ & $(4,1)$ & 2482 & 2816\\
$(79,17)$ & 11 & $(23,5)$ & 7 & 1 & YES & YES & YES & $1.00$ & $(2,2)$ & NO & 2817\\
$(79,23)$ & 10 & $(24,7)$ & 7 & 1 & YES & YES & YES & $1.12$ & $(2,2)$ & NO & 2818\\
$(79,22)$ & 10 & $(25,7)$ & 7 & 1 & YES & YES & YES & $1.12$ & $(2,2)$ & NO & 2819\\
$(79,30)$ & 9 & $(29,11)$ & 7 & 1 & YES & YES & YES & $1.25$ & $(2,2)$ & NO & 2820\\
$(79,29)$ & 9 & $(30,11)$ & 7 & 1 & YES & YES & NO(2) & $0.88$ & $(4,1)$ & NO & 2821\\
$(79,24)$ & 10 & $(33,10)$ & 8 & 1 & YES & YES & YES & $1.00$ & $(2,2)$ & 2980 & 2822\\
$(79,21)$ & 11 & $(34,9)$ & 8 & 1 & YES & YES & YES & $1.00$ & $(2,2)$ & NO & 2823\\
$(79,29)$ & 9 & $(49,18)$ & 8 & 1 & YES & YES & YES & $1.00$ & $(2,2)$ & NO & 2824\\
$(79,30)$ & 9 & $(50,19)$ & 8 & 1 & YES & YES & YES & $0.88$ & $(2,2)$ & NO & 2825\\
$(79,22)$ & 10 & $(61,17)$ & 9 & 1 & YES & YES & YES & $1.12$ & $(2,2)$ & NO & 2826\\
$(79,23)$ & 10 & $(79,23)$ & 10 & 79 & YES & YES & YES & $1.25$ & $(2,2)$ & NO & 2827\\
$(79,24)$ & 10 & $(79,24)$ & 10 & 79 & YES & YES & YES & $1.12$ & $(2,2)$ & NO & 2828\\
$(79,29)$ & 9 & $(79,29)$ & 9 & 79 & YES & YES & YES & $1.12$ & $(2,2)$ & NO & 2829\\
$(80,31)$ & 9 & $(2,1)$ & 1 & 2 & YES & YES & YES & $0.88$ & $(2,2)$ & NO & 2830\\
$(80,31)$ & 9 & $(2,1)$ & 1 & 2 & YES & YES & YES & $1.00$ & $(2,2)$ & -- & 2831\\
$(80,31)$ & 9 & $(3,1)$ & 2 & 1 & YES & YES & YES & $1.12$ & $(2,2)$ & 1829 & 2832\\
$(80,31)$ & 9 & $(3,1)$ & 2 & 1 & YES & YES & YES & $1.12$ & $(2,2)$ & -- & 2833\\
$(80,31)$ & 9 & $(4,1)$ & 3 & 4 & YES & YES & YES & $1.12$ & $(2,2)$ & -- & 2834\\
$(80,31)$ & 9 & $(5,2)$ & 3 & 5 & YES & YES & YES & $1.00$ & $(2,2)$ & 2335 & 2835\\
$(80,31)$ & 9 & $(13,5)$ & 5 & 1 & YES & YES & YES & $1.00$ & $(2,2)$ & NO & 2836\\
$(80,31)$ & 9 & $(31,12)$ & 7 & 1 & YES & YES & YES & $0.88$ & $(2,2)$ & NO & 2837\\
$(80,31)$ & 9 & $(49,19)$ & 8 & 1 & YES & YES & YES & $1.12$ & $(2,2)$ & NO & 2838\\
$(80,31)$ & 9 & $(80,31)$ & 9 & 80 & YES & YES & YES & $1.00$ & $(2,2)$ & NO & 2839\\
$(81,31)$ & 9 & $(2,1)$ & 1 & 1 & YES & YES & YES & $1.25$ & $(2,2)$ & -- & 2840\\
$(81,34)$ & 9 & $(2,1)$ & 1 & 1 & YES & YES & YES & $1.12$ & $(2,2)$ & -- & 2841\\
$(81,19)$ & 11 & $(3,1)$ & 2 & 3 & YES & YES & YES & $1.12$ & $(2,2)$ & NO & 2842\\
$(81,19)$ & 11 & $(3,1)$ & 2 & 3 & YES & YES & YES & $1.00$ & $(2,2)$ & NO & 2843\\
$(81,19)$ & 11 & $(3,1)$ & 2 & 3 & YES & YES & YES & $1.00$ & $(2,2)$ & -- & 2844\\
$(81,31)$ & 9 & $(5,1)$ & 4 & 1 & YES & YES & YES & $1.00$ & $(2,2)$ & NO & 2845\\
$(81,34)$ & 9 & $(5,2)$ & 3 & 1 & YES & YES & YES & $1.12$ & $(2,2)$ & NO & 2846\\
$(81,19)$ & 11 & $(6,1)$ & 5 & 3 & YES & YES & YES & $0.88$ & $(2,2)$ & NO & 2847\\
$(81,34)$ & 9 & $(7,3)$ & 4 & 1 & YES & YES & YES & $1.12$ & $(2,2)$ & 2588 & 2848\\
$(81,19)$ & 11 & $(9,2)$ & 5 & 9 & YES & YES & YES & $1.12$ & $(2,2)$ & NO & 2849\\
$(81,34)$ & 9 & $(12,5)$ & 5 & 3 & YES & YES & YES & $1.12$ & $(2,2)$ & NO & 2850\\
$(81,31)$ & 9 & $(13,5)$ & 5 & 1 & YES & YES & YES & $1.12$ & $(2,2)$ & 2187 & 2851\\
$(81,19)$ & 11 & $(21,5)$ & 8 & 3 & YES & YES & YES & $1.00$ & $(2,2)$ & NO & 2852\\
$(81,19)$ & 11 & $(30,7)$ & 8 & 3 & YES & YES & YES & $0.88$ & $(2,2)$ & NO & 2853\\
$(81,19)$ & 11 & $(81,19)$ & 11 & 81 & YES & YES & YES & $0.88$ & $(2,2)$ & NO & 2854\\
$(82,23)$ & 10 & $(2,1)$ & 1 & 2 & YES & YES & YES & $1.00$ & $(2,2)$ & NO & 2855\\
$(82,23)$ & 10 & $(2,1)$ & 1 & 2 & YES & YES & YES & $1.00$ & $(2,2)$ & -- & 2856\\
$(82,25)$ & 10 & $(2,1)$ & 1 & 2 & YES & YES & YES & $1.12$ & $(2,2)$ & -- & 2857\\
$(82,25)$ & 10 & $(2,1)$ & 1 & 2 & YES & YES & YES & $1.12$ & $(2,2)$ & NO & 2858\\
$(82,23)$ & 10 & $(3,1)$ & 2 & 1 & YES & YES & YES & $1.00$ & $(2,2)$ & NO & 2859\\
$(82,25)$ & 10 & $(3,1)$ & 2 & 1 & YES & YES & YES & $1.12$ & $(2,2)$ & -- & 2860\\
$(82,23)$ & 10 & $(4,1)$ & 3 & 2 & YES & YES & YES & $1.12$ & $(2,2)$ & 1214 & 2861\\
$(82,25)$ & 10 & $(4,1)$ & 3 & 2 & YES & YES & YES & $0.88$ & $(2,2)$ & NO & 2862\\
$(82,25)$ & 10 & $(6,1)$ & 5 & 2 & YES & YES & YES & $1.12$ & $(2,2)$ & NO & 2863\\
$(82,23)$ & 10 & $(7,2)$ & 4 & 1 & YES & YES & YES & $1.00$ & $(2,2)$ & NO & 2864\\
$(82,25)$ & 10 & $(13,4)$ & 6 & 1 & YES & YES & YES & $1.00$ & $(2,2)$ & NO & 2865\\
$(82,25)$ & 10 & $(23,7)$ & 7 & 1 & YES & YES & YES & $1.12$ & $(2,2)$ & NO & 2866\\
$(82,23)$ & 10 & $(25,7)$ & 7 & 1 & YES & YES & YES & $1.12$ & $(2,2)$ & NO & 2867\\
$(82,25)$ & 10 & $(59,18)$ & 9 & 1 & YES & YES & YES & $1.00$ & $(2,2)$ & NO & 2868\\
$(83,22)$ & 10 & $(2,1)$ & 1 & 1 & YES & YES & YES & $1.00$ & $(2,2)$ & -- & 2869\\
$(83,22)$ & 10 & $(2,1)$ & 1 & 1 & YES & YES & YES & $1.12$ & $(2,2)$ & NO & 2870\\
$(83,23)$ & 10 & $(2,1)$ & 1 & 1 & YES & YES & YES & $1.12$ & $(2,2)$ & NO & 2871\\
$(83,34)$ & 10 & $(2,1)$ & 1 & 1 & NO & YES & YES & $1.25$ & $(2,2)$ & -- & 2872\\
$(83,36)$ & 10 & $(2,1)$ & 1 & 1 & YES & YES & YES & $1.25$ & $(2,2)$ & -- & 2873\\
$(83,23)$ & 10 & $(3,1)$ & 2 & 1 & YES & YES & YES & $1.12$ & $(2,2)$ & NO & 2874\\
$(83,30)$ & 10 & $(3,1)$ & 2 & 1 & YES & YES & YES & $1.25$ & $(2,2)$ & NO & 2875\\
$(83,30)$ & 10 & $(3,1)$ & 2 & 1 & YES & YES & YES & $1.25$ & $(2,2)$ & -- & 2876\\
$(83,36)$ & 10 & $(3,1)$ & 2 & 1 & YES & YES & YES & $1.25$ & $(2,2)$ & NO & 2877\\
$(83,36)$ & 10 & $(3,1)$ & 2 & 1 & YES & YES & YES & $1.12$ & $(2,2)$ & -- & 2878\\
$(83,19)$ & 10 & $(4,1)$ & 3 & 1 & YES & YES & YES & $1.12$ & $(2,2)$ & -- & 2879\\
$(83,30)$ & 10 & $(4,1)$ & 3 & 1 & YES & YES & YES & $1.25$ & $(2,2)$ & -- & 2880\\
$(83,36)$ & 10 & $(4,1)$ & 3 & 1 & YES & YES & YES & $1.12$ & $(2,2)$ & -- & 2881\\
$(83,19)$ & 10 & $(5,2)$ & 3 & 1 & YES & YES & YES & $1.12$ & $(2,2)$ & NO & 2882\\
$(83,19)$ & 10 & $(5,2)$ & 3 & 1 & YES & YES & YES & $1.12$ & $(2,2)$ & -- & 2883\\
$(83,30)$ & 10 & $(5,2)$ & 3 & 1 & YES & YES & YES & $1.25$ & $(2,2)$ & NO & 2884\\
$(83,36)$ & 10 & $(5,1)$ & 4 & 1 & YES & YES & YES & $1.12$ & $(2,2)$ & NO & 2885\\
$(83,36)$ & 10 & $(5,1)$ & 4 & 1 & YES & YES & YES & $1.12$ & $(2,2)$ & -- & 2886\\
$(83,19)$ & 10 & $(7,2)$ & 4 & 1 & YES & YES & YES & $1.12$ & $(2,2)$ & NO & 2887\\
$(83,36)$ & 10 & $(7,3)$ & 4 & 1 & YES & YES & YES & $1.12$ & $(2,2)$ & NO & 2888\\
$(83,30)$ & 10 & $(8,3)$ & 4 & 1 & YES & YES & YES & $1.38$ & $(2,2)$ & NO & 2889\\
$(83,36)$ & 10 & $(16,7)$ & 6 & 1 & YES & YES & YES & $1.12$ & $(2,2)$ & 3108 & 2890\\
$(83,18)$ & 10 & $(19,4)$ & 7 & 1 & YES & YES & YES & $1.00$ & $(2,2)$ & 3171 & 2891\\
$(83,19)$ & 10 & $(21,5)$ & 8 & 1 & YES & YES & YES & $1.12$ & $(2,2)$ & NO & 2892\\
$(83,36)$ & 10 & $(23,10)$ & 7 & 1 & YES & YES & YES & $1.12$ & $(2,2)$ & 2733 & 2893\\
$(83,36)$ & 10 & $(30,13)$ & 8 & 1 & YES & YES & YES & $1.25$ & $(2,2)$ & NO & 2894\\
$(83,18)$ & 10 & $(32,7)$ & 8 & 1 & YES & YES & YES & $1.00$ & $(2,2)$ & NO & 2895\\
$(83,22)$ & 10 & $(34,9)$ & 8 & 1 & YES & YES & YES & $1.00$ & $(2,2)$ & NO & 2896\\
$(83,30)$ & 10 & $(47,17)$ & 9 & 1 & YES & YES & YES & $1.38$ & $(2,2)$ & NO & 2897\\
$(83,36)$ & 10 & $(53,23)$ & 9 & 1 & YES & YES & YES & $1.25$ & $(2,2)$ & NO & 2898\\
$(83,22)$ & 10 & $(83,22)$ & 10 & 83 & YES & YES & YES & $1.00$ & $(2,2)$ & NO & 2899\\
$(83,30)$ & 10 & $(83,30)$ & 10 & 83 & YES & YES & YES & $1.12$ & $(2,2)$ & NO & 2900\\
$(83,36)$ & 10 & $(83,36)$ & 10 & 83 & YES & YES & YES & $1.25$ & $(2,2)$ & NO & 2901\\
$(84,25)$ & 10 & $(2,1)$ & 1 & 2 & YES & YES & YES & $1.00$ & $(2,2)$ & -- & 2902\\
$(84,25)$ & 10 & $(2,1)$ & 1 & 2 & YES & YES & YES & $1.12$ & $(2,2)$ & NO & 2903\\
$(84,37)$ & 10 & $(2,1)$ & 1 & 2 & NO & YES & YES & $1.12$ & $(2,2)$ & -- & 2904\\
$(84,13)$ & 13 & $(3,1)$ & 2 & 3 & YES & YES & YES & $1.12$ & $(2,2)$ & NO & 2905\\
$(84,13)$ & 13 & $(3,1)$ & 2 & 3 & YES & YES & YES & $1.12$ & $(2,2)$ & -- & 2906\\
$(84,13)$ & 13 & $(3,1)$ & 2 & 3 & YES & YES & YES & $1.25$ & $(2,2)$ & NO & 2907\\
$(84,25)$ & 10 & $(3,1)$ & 2 & 3 & YES & YES & YES & $1.00$ & $(2,2)$ & NO & 2908\\
$(84,25)$ & 10 & $(3,1)$ & 2 & 3 & YES & YES & YES & $1.00$ & $(2,2)$ & -- & 2909\\
$(84,31)$ & 10 & $(3,1)$ & 2 & 3 & YES & YES & YES & $1.25$ & $(2,2)$ & NO & 2910\\
$(84,25)$ & 10 & $(4,1)$ & 3 & 4 & YES & YES & YES & $1.00$ & $(2,2)$ & -- & 2911\\
$(84,25)$ & 10 & $(7,2)$ & 4 & 7 & YES & YES & YES & $1.00$ & $(2,2)$ & NO & 2912\\
$(84,25)$ & 10 & $(17,5)$ & 6 & 1 & YES & YES & YES & $1.00$ & $(2,2)$ & 1749 & 2913\\
$(84,31)$ & 10 & $(19,7)$ & 6 & 1 & YES & YES & YES & $1.25$ & $(2,2)$ & NO & 2914\\
$(84,13)$ & 13 & $(20,3)$ & 8 & 4 & YES & YES & YES & $1.12$ & $(2,2)$ & NO & 2915\\
$(84,25)$ & 10 & $(27,8)$ & 7 & 3 & YES & YES & YES & $1.00$ & $(2,2)$ & NO & 2916\\
$(85,23)$ & 10 & $(2,1)$ & 1 & 1 & YES & YES & YES & $0.88$ & $(2,2)$ & -- & 2917\\
$(85,23)$ & 10 & $(2,1)$ & 1 & 1 & YES & YES & YES & $1.00$ & $(2,2)$ & NO & 2918\\
$(85,26)$ & 10 & $(2,1)$ & 1 & 1 & YES & YES & YES & $1.12$ & $(2,2)$ & -- & 2919\\
$(85,36)$ & 10 & $(2,1)$ & 1 & 1 & NO & YES & YES & $1.12$ & $(2,2)$ & -- & 2920\\
$(85,37)$ & 10 & $(2,1)$ & 1 & 1 & YES & YES & YES & $1.38$ & $(2,2)$ & -- & 2921\\
$(85,23)$ & 10 & $(3,1)$ & 2 & 1 & YES & YES & YES & $0.88$ & $(2,2)$ & -- & 2922\\
$(85,23)$ & 10 & $(3,1)$ & 2 & 1 & YES & YES & YES & $1.12$ & $(2,2)$ & NO & 2923\\
$(85,23)$ & 10 & $(3,1)$ & 2 & 1 & YES & YES & YES & $1.25$ & $(2,2)$ & NO & 2924\\
$(85,26)$ & 10 & $(3,1)$ & 2 & 1 & YES & YES & YES & $1.25$ & $(2,2)$ & NO & 2925\\
$(85,26)$ & 10 & $(3,1)$ & 2 & 1 & YES & YES & YES & $1.25$ & $(2,2)$ & -- & 2926\\
$(85,16)$ & 12 & $(4,1)$ & 3 & 1 & YES & YES & YES & $1.25$ & $(2,2)$ & -- & 2927\\
$(85,16)$ & 12 & $(5,2)$ & 3 & 5 & YES & YES & YES & $1.12$ & $(2,2)$ & NO & 2928\\
$(85,16)$ & 12 & $(5,2)$ & 3 & 5 & YES & YES & YES & $1.25$ & $(2,2)$ & NO & 2929\\
$(85,26)$ & 10 & $(5,1)$ & 4 & 5 & YES & YES & YES & $1.12$ & $(2,2)$ & NO & 2930\\
$(85,26)$ & 10 & $(7,2)$ & 4 & 1 & YES & YES & YES & $1.12$ & $(2,2)$ & NO & 2931\\
$(85,37)$ & 10 & $(7,3)$ & 4 & 1 & YES & YES & YES & $1.38$ & $(2,2)$ & NO & 2932\\
$(85,16)$ & 12 & $(9,2)$ & 5 & 1 & YES & YES & YES & $1.00$ & $(2,2)$ & 2690 & 2933\\
$(85,23)$ & 10 & $(9,2)$ & 5 & 1 & YES & YES & YES & $1.12$ & $(2,2)$ & NO & 2934\\
$(85,26)$ & 10 & $(10,3)$ & 5 & 5 & YES & YES & YES & $1.00$ & $(2,2)$ & 1846 & 2935\\
$(85,26)$ & 10 & $(13,4)$ & 6 & 1 & YES & YES & YES & $1.12$ & $(2,2)$ & 2257 & 2936\\
$(85,23)$ & 10 & $(15,4)$ & 6 & 5 & YES & YES & YES & $0.88$ & $(2,2)$ & 1768 & 2937\\
$(85,26)$ & 10 & $(23,7)$ & 7 & 1 & YES & YES & YES & $1.12$ & $(2,2)$ & NO & 2938\\
$(85,16)$ & 12 & $(26,5)$ & 9 & 1 & YES & YES & YES & $1.25$ & $(2,2)$ & NO & 2939\\
$(85,23)$ & 10 & $(26,7)$ & 7 & 1 & YES & YES & YES & $0.88$ & $(2,2)$ & NO & 2940\\
$(85,23)$ & 10 & $(59,16)$ & 10 & 1 & YES & YES & YES & $1.12$ & $(2,2)$ & NO & 2941\\
$(85,26)$ & 10 & $(85,26)$ & 10 & 85 & YES & YES & YES & $1.12$ & $(2,2)$ & NO & 2942\\
$(86,25)$ & 10 & $(2,1)$ & 1 & 2 & YES & YES & YES & $1.12$ & $(2,2)$ & NO & 2943\\
$(86,31)$ & 10 & $(2,1)$ & 1 & 2 & YES & YES & YES & $1.25$ & $(2,2)$ & -- & 2944\\
$(86,31)$ & 10 & $(2,1)$ & 1 & 2 & YES & YES & YES & $1.25$ & $(2,2)$ & NO & 2945\\
$(86,25)$ & 10 & $(4,1)$ & 3 & 2 & YES & YES & YES & $1.12$ & $(2,2)$ & NO & 2946\\
$(86,31)$ & 10 & $(4,1)$ & 3 & 2 & YES & YES & YES & $1.12$ & $(2,2)$ & -- & 2947\\
$(86,31)$ & 10 & $(5,1)$ & 4 & 1 & YES & YES & YES & $1.12$ & $(2,2)$ & -- & 2948\\
$(86,31)$ & 10 & $(11,4)$ & 5 & 1 & YES & YES & YES & $1.25$ & $(2,2)$ & NO & 2949\\
$(86,31)$ & 10 & $(25,9)$ & 7 & 1 & YES & YES & YES & $1.25$ & $(2,2)$ & NO & 2950\\
$(86,25)$ & 10 & $(31,9)$ & 8 & 1 & YES & YES & YES & $1.12$ & $(2,2)$ & NO & 2951\\
$(86,31)$ & 10 & $(61,22)$ & 9 & 1 & YES & YES & YES & $1.12$ & $(2,2)$ & NO & 2952\\
$(86,31)$ & 10 & $(86,31)$ & 10 & 86 & YES & YES & YES & $1.12$ & $(2,2)$ & NO & 2953\\
$(89,25)$ & 10 & $(2,1)$ & 1 & 1 & YES & YES & YES & $1.12$ & $(2,2)$ & NO & 2954\\
$(89,25)$ & 10 & $(2,1)$ & 1 & 1 & YES & YES & YES & $1.12$ & $(2,2)$ & -- & 2955\\
$(89,26)$ & 10 & $(2,1)$ & 1 & 1 & YES & YES & YES & $1.12$ & $(2,2)$ & -- & 2956\\
$(89,27)$ & 10 & $(2,1)$ & 1 & 1 & YES & YES & NO(2) & $1.00$ & $(4,1)$ & -- & 2957\\
$(89,27)$ & 10 & $(2,1)$ & 1 & 1 & YES & YES & YES & $1.12$ & $(2,2)$ & NO & 2958\\
$(89,34)$ & 9 & $(2,1)$ & 1 & 1 & YES & YES & YES & $1.00$ & $(2,2)$ & -- & 2959\\
$(89,21)$ & 12 & $(3,1)$ & 2 & 1 & YES & YES & YES & $1.25$ & $(2,2)$ & -- & 2960\\
$(89,25)$ & 10 & $(3,1)$ & 2 & 1 & YES & YES & YES & $1.12$ & $(2,2)$ & NO & 2961\\
$(89,25)$ & 10 & $(3,1)$ & 2 & 1 & NO & YES & YES & $1.12$ & $(2,2)$ & -- & 2962\\
$(89,26)$ & 10 & $(3,1)$ & 2 & 1 & YES & YES & YES & $1.12$ & $(2,2)$ & NO & 2963\\
$(89,26)$ & 10 & $(3,1)$ & 2 & 1 & YES & YES & YES & $1.12$ & $(2,2)$ & -- & 2964\\
$(89,26)$ & 10 & $(3,1)$ & 2 & 1 & YES & YES & YES & $1.00$ & $(2,2)$ & NO & 2965\\
$(89,27)$ & 10 & $(3,1)$ & 2 & 1 & YES & YES & YES & $1.12$ & $(2,2)$ & NO & 2966\\
$(89,27)$ & 10 & $(3,1)$ & 2 & 1 & YES & YES & YES & $1.12$ & $(2,2)$ & -- & 2967\\
$(89,34)$ & 9 & $(3,1)$ & 2 & 1 & YES & YES & YES & $1.00$ & $(2,2)$ & 2804 & 2968\\
$(89,21)$ & 12 & $(4,1)$ & 3 & 1 & YES & YES & YES & $1.12$ & $(2,2)$ & -- & 2969\\
$(89,34)$ & 9 & $(5,2)$ & 3 & 1 & YES & YES & YES & $1.12$ & $(2,2)$ & NO & 2970\\
$(89,21)$ & 12 & $(6,1)$ & 5 & 1 & YES & YES & YES & $1.12$ & $(2,2)$ & NO & 2971\\
$(89,26)$ & 10 & $(6,1)$ & 5 & 1 & YES & YES & YES & $1.00$ & $(2,2)$ & NO & 2972\\
$(89,26)$ & 10 & $(7,2)$ & 4 & 1 & YES & YES & YES & $1.12$ & $(2,2)$ & NO & 2973\\
$(89,27)$ & 10 & $(7,2)$ & 4 & 1 & YES & YES & YES & $1.12$ & $(2,2)$ & NO & 2974\\
$(89,34)$ & 9 & $(8,3)$ & 4 & 1 & YES & YES & YES & $1.12$ & $(2,2)$ & 2814 & 2975\\
$(89,21)$ & 12 & $(9,2)$ & 5 & 1 & YES & YES & YES & $1.12$ & $(2,2)$ & NO & 2976\\
$(89,27)$ & 10 & $(10,3)$ & 5 & 1 & YES & YES & YES & $1.00$ & $(2,2)$ & NO & 2977\\
$(89,21)$ & 12 & $(13,3)$ & 6 & 1 & YES & YES & YES & $1.25$ & $(2,2)$ & NO & 2978\\
$(89,21)$ & 12 & $(21,5)$ & 8 & 1 & YES & YES & YES & $1.25$ & $(2,2)$ & NO & 2979\\
$(89,27)$ & 10 & $(23,7)$ & 7 & 1 & YES & YES & YES & $1.00$ & $(2,2)$ & 2822 & 2980\\
$(89,26)$ & 10 & $(24,7)$ & 7 & 1 & YES & YES & YES & $1.12$ & $(2,2)$ & NO & 2981\\
$(89,25)$ & 10 & $(32,9)$ & 8 & 1 & YES & YES & YES & $1.12$ & $(2,2)$ & NO & 2982\\
$(89,27)$ & 10 & $(33,10)$ & 8 & 1 & YES & YES & YES & $1.00$ & $(2,2)$ & NO & 2983\\
$(89,21)$ & 12 & $(38,9)$ & 9 & 1 & YES & YES & YES & $1.12$ & $(2,2)$ & NO & 2984\\
$(89,21)$ & 12 & $(72,17)$ & 11 & 1 & YES & YES & YES & $1.12$ & $(2,2)$ & NO & 2985\\
$(89,21)$ & 12 & $(89,21)$ & 12 & 89 & YES & YES & YES & $1.25$ & $(2,2)$ & NO & 2986\\
$(89,26)$ & 10 & $(89,26)$ & 10 & 89 & YES & YES & YES & $1.25$ & $(2,2)$ & NO & 2987\\
$(89,34)$ & 9 & $(89,34)$ & 9 & 89 & YES & YES & YES & $0.88$ & $(2,2)$ & NO & 2988\\
$(90,19)$ & 11 & $(3,1)$ & 2 & 3 & YES & YES & YES & $1.12$ & $(2,2)$ & NO & 2989\\
$(90,19)$ & 11 & $(3,1)$ & 2 & 3 & YES & YES & YES & $1.12$ & $(2,2)$ & NO & 2990\\
$(90,19)$ & 11 & $(3,1)$ & 2 & 3 & YES & YES & YES & $1.12$ & $(2,2)$ & -- & 2991\\
$(90,19)$ & 11 & $(6,1)$ & 5 & 6 & YES & YES & YES & $1.12$ & $(2,2)$ & NO & 2992\\
$(90,19)$ & 11 & $(9,2)$ & 5 & 9 & YES & YES & YES & $1.00$ & $(2,2)$ & NO & 2993\\
$(90,19)$ & 11 & $(24,5)$ & 8 & 6 & YES & YES & YES & $1.12$ & $(2,2)$ & NO & 2994\\
$(90,19)$ & 11 & $(90,19)$ & 11 & 90 & YES & YES & YES & $1.12$ & $(2,2)$ & NO & 2995\\
$(91,17)$ & 12 & $(2,1)$ & 1 & 1 & YES & YES & YES & $0.88$ & $(2,2)$ & NO & 2996\\
$(91,27)$ & 10 & $(2,1)$ & 1 & 1 & YES & YES & YES & $1.00$ & $(2,2)$ & -- & 2997\\
$(91,17)$ & 12 & $(3,1)$ & 2 & 1 & YES & YES & YES & $0.88$ & $(2,2)$ & NO & 2998\\
$(91,27)$ & 10 & $(3,1)$ & 2 & 1 & YES & YES & YES & $1.00$ & $(2,2)$ & 1835 & 2999\\
$(91,25)$ & 10 & $(4,1)$ & 3 & 1 & YES & YES & YES & $1.00$ & $(2,2)$ & -- & 3000\\
$(91,27)$ & 10 & $(4,1)$ & 3 & 1 & YES & YES & YES & $1.00$ & $(2,2)$ & -- & 3001\\
$(91,27)$ & 10 & $(10,3)$ & 5 & 1 & YES & YES & YES & $0.88$ & $(2,2)$ & NO & 3002\\
$(91,27)$ & 10 & $(27,8)$ & 7 & 1 & YES & YES & YES & $1.12$ & $(2,2)$ & NO & 3003\\
$(91,25)$ & 10 & $(51,14)$ & 9 & 1 & YES & YES & YES & $1.12$ & $(2,2)$ & NO & 3004\\
$(91,27)$ & 10 & $(64,19)$ & 9 & 1 & YES & YES & YES & $1.00$ & $(2,2)$ & NO & 3005\\
$(92,19)$ & 12 & $(2,1)$ & 1 & 2 & YES & YES & YES & $1.00$ & $(2,2)$ & -- & 3006\\
$(92,19)$ & 12 & $(2,1)$ & 1 & 2 & YES & YES & YES & $1.12$ & $(2,2)$ & NO & 3007\\
$(92,39)$ & 10 & $(2,1)$ & 1 & 2 & NO & YES & YES & $1.12$ & $(2,2)$ & -- & 3008\\
$(92,19)$ & 12 & $(4,1)$ & 3 & 4 & YES & YES & YES & $1.00$ & $(2,2)$ & -- & 3009\\
$(92,19)$ & 12 & $(4,1)$ & 3 & 4 & YES & YES & YES & $1.12$ & $(2,2)$ & NO & 3010\\
$(92,21)$ & 10 & $(17,4)$ & 7 & 1 & YES & YES & YES & $1.12$ & $(2,2)$ & NO & 3011\\
$(92,19)$ & 12 & $(63,13)$ & 11 & 1 & YES & YES & YES & $1.00$ & $(2,2)$ & NO & 3012\\
$(93,25)$ & 10 & $(2,1)$ & 1 & 1 & YES & YES & YES & $0.88$ & $(2,2)$ & NO & 3013\\
$(93,25)$ & 10 & $(2,1)$ & 1 & 1 & YES & YES & NO(2) & $0.75$ & $(4,1)$ & -- & 3014\\
$(93,26)$ & 10 & $(2,1)$ & 1 & 1 & YES & YES & YES & $1.12$ & $(2,2)$ & -- & 3015\\
$(93,25)$ & 10 & $(3,1)$ & 2 & 3 & YES & YES & NO(2) & $1.12$ & $(4,1)$ & -- & 3016\\
$(93,22)$ & 11 & $(4,1)$ & 3 & 1 & NO & YES & YES & $1.12$ & $(2,2)$ & -- & 3017\\
$(93,26)$ & 10 & $(4,1)$ & 3 & 1 & YES & YES & YES & $1.25$ & $(2,2)$ & NO & 3018\\
$(93,26)$ & 10 & $(6,1)$ & 5 & 3 & YES & YES & YES & $1.00$ & $(2,2)$ & NO & 3019\\
$(93,26)$ & 10 & $(7,2)$ & 4 & 1 & YES & YES & YES & $1.12$ & $(2,2)$ & NO & 3020\\
$(93,26)$ & 10 & $(25,7)$ & 7 & 1 & YES & YES & YES & $1.12$ & $(2,2)$ & NO & 3021\\
$(93,25)$ & 10 & $(26,7)$ & 7 & 1 & YES & YES & YES & $0.88$ & $(2,2)$ & NO & 3022\\
$(93,22)$ & 11 & $(30,7)$ & 8 & 3 & YES & YES & YES & $1.00$ & $(2,2)$ & NO & 3023\\
$(93,25)$ & 10 & $(41,11)$ & 8 & 1 & YES & YES & YES & $1.00$ & $(2,2)$ & 3144 & 3024\\
$(94,39)$ & 10 & $(2,1)$ & 1 & 2 & YES & YES & YES & $1.25$ & $(2,2)$ & NO & 3025\\
$(94,41)$ & 10 & $(2,1)$ & 1 & 2 & YES & YES & YES & $1.25$ & $(2,2)$ & -- & 3026\\
$(94,41)$ & 10 & $(2,1)$ & 1 & 2 & YES & YES & YES & $1.12$ & $(2,2)$ & NO & 3027\\
$(94,41)$ & 10 & $(4,1)$ & 3 & 2 & YES & YES & YES & $1.12$ & $(2,2)$ & -- & 3028\\
$(94,41)$ & 10 & $(4,1)$ & 3 & 2 & YES & YES & YES & $1.25$ & $(2,2)$ & NO & 3029\\
$(94,39)$ & 10 & $(5,1)$ & 4 & 1 & YES & YES & YES & $1.12$ & $(2,2)$ & NO & 3030\\
$(94,39)$ & 10 & $(5,1)$ & 4 & 1 & YES & YES & YES & $1.12$ & $(2,2)$ & NO & 3031\\
$(94,39)$ & 10 & $(5,1)$ & 4 & 1 & YES & YES & YES & $1.12$ & $(2,2)$ & -- & 3032\\
$(94,41)$ & 10 & $(5,1)$ & 4 & 1 & YES & YES & YES & $1.12$ & $(2,2)$ & NO & 3033\\
$(94,41)$ & 10 & $(5,1)$ & 4 & 1 & YES & YES & YES & $1.25$ & $(2,2)$ & NO & 3034\\
$(94,41)$ & 10 & $(7,3)$ & 4 & 1 & YES & YES & YES & $1.25$ & $(2,2)$ & 2013 & 3035\\
$(94,39)$ & 10 & $(12,5)$ & 5 & 2 & YES & YES & YES & $1.25$ & $(2,2)$ & 2399 & 3036\\
$(94,41)$ & 10 & $(16,7)$ & 6 & 2 & YES & YES & YES & $1.25$ & $(2,2)$ & 2601 & 3037\\
$(94,39)$ & 10 & $(41,17)$ & 8 & 1 & YES & YES & YES & $1.12$ & $(2,2)$ & NO & 3038\\
$(94,41)$ & 10 & $(55,24)$ & 9 & 1 & YES & YES & YES & $1.12$ & $(2,2)$ & NO & 3039\\
$(95,29)$ & 10 & $(2,1)$ & 1 & 1 & YES & YES & YES & $1.12$ & $(2,2)$ & NO & 3040\\
$(95,29)$ & 10 & $(2,1)$ & 1 & 1 & YES & YES & YES & $1.12$ & $(2,2)$ & -- & 3041\\
$(95,39)$ & 10 & $(2,1)$ & 1 & 1 & YES & YES & YES & $1.25$ & $(2,2)$ & -- & 3042\\
$(95,39)$ & 10 & $(2,1)$ & 1 & 1 & YES & YES & YES & $1.25$ & $(2,2)$ & NO & 3043\\
$(95,39)$ & 10 & $(3,1)$ & 2 & 1 & YES & YES & YES & $1.12$ & $(2,2)$ & NO & 3044\\
$(95,39)$ & 10 & $(3,1)$ & 2 & 1 & YES & YES & YES & $1.00$ & $(2,2)$ & -- & 3045\\
$(95,39)$ & 10 & $(5,2)$ & 3 & 5 & YES & YES & YES & $1.12$ & $(2,2)$ & 2054 & 3046\\
$(95,29)$ & 10 & $(13,4)$ & 6 & 1 & YES & YES & YES & $1.12$ & $(2,2)$ & NO & 3047\\
$(95,39)$ & 10 & $(39,16)$ & 8 & 1 & YES & YES & YES & $1.25$ & $(2,2)$ & NO & 3048\\
$(95,39)$ & 10 & $(95,39)$ & 10 & 95 & YES & YES & YES & $1.12$ & $(2,2)$ & NO & 3049\\
$(96,17)$ & 12 & $(2,1)$ & 1 & 2 & YES & YES & YES & $1.00$ & $(2,2)$ & NO & 3050\\
$(96,29)$ & 11 & $(2,1)$ & 1 & 2 & YES & YES & YES & $1.25$ & $(2,2)$ & NO & 3051\\
$(96,17)$ & 12 & $(3,1)$ & 2 & 3 & YES & YES & YES & $1.00$ & $(2,2)$ & NO & 3052\\
$(96,29)$ & 11 & $(3,1)$ & 2 & 3 & YES & YES & YES & $1.12$ & $(2,2)$ & NO & 3053\\
$(96,17)$ & 12 & $(4,1)$ & 3 & 4 & YES & YES & YES & $1.00$ & $(2,2)$ & NO & 3054\\
$(96,29)$ & 11 & $(4,1)$ & 3 & 4 & YES & YES & YES & $1.25$ & $(2,2)$ & NO & 3055\\
$(96,17)$ & 12 & $(5,1)$ & 4 & 1 & YES & YES & YES & $1.00$ & $(2,2)$ & NO & 3056\\
$(96,29)$ & 11 & $(6,1)$ & 5 & 6 & YES & YES & YES & $1.00$ & $(2,2)$ & NO & 3057\\
$(96,29)$ & 11 & $(23,7)$ & 7 & 1 & YES & YES & YES & $1.12$ & $(2,2)$ & 2059 & 3058\\
$(96,29)$ & 11 & $(43,13)$ & 9 & 1 & YES & YES & YES & $1.12$ & $(2,2)$ & NO & 3059\\
$(97,26)$ & 10 & $(2,1)$ & 1 & 1 & YES & YES & YES & $0.88$ & $(2,2)$ & -- & 3060\\
$(97,26)$ & 10 & $(2,1)$ & 1 & 1 & YES & YES & YES & $1.00$ & $(2,2)$ & NO & 3061\\
$(97,35)$ & 10 & $(2,1)$ & 1 & 1 & YES & YES & YES & $1.12$ & $(2,2)$ & NO & 3062\\
$(97,35)$ & 10 & $(2,1)$ & 1 & 1 & YES & YES & YES & $1.25$ & $(2,2)$ & -- & 3063\\
$(97,38)$ & 11 & $(2,1)$ & 1 & 1 & NO & YES & YES & $1.25$ & $(2,2)$ & -- & 3064\\
$(97,26)$ & 10 & $(3,1)$ & 2 & 1 & YES & YES & YES & $0.88$ & $(2,2)$ & -- & 3065\\
$(97,36)$ & 10 & $(3,1)$ & 2 & 1 & YES & YES & YES & $1.38$ & $(2,2)$ & NO & 3066\\
$(97,35)$ & 10 & $(11,4)$ & 5 & 1 & YES & YES & YES & $1.25$ & $(2,2)$ & NO & 3067\\
$(97,30)$ & 11 & $(13,4)$ & 6 & 1 & YES & YES & YES & $1.12$ & $(2,2)$ & 2483 & 3068\\
$(97,36)$ & 10 & $(62,23)$ & 9 & 1 & YES & YES & YES & $1.25$ & $(2,2)$ & NO & 3069\\
$(98,41)$ & 10 & $(2,1)$ & 1 & 2 & NO & YES & YES & $1.25$ & $(2,2)$ & -- & 3070\\
$(98,29)$ & 10 & $(27,8)$ & 7 & 1 & YES & YES & YES & $1.12$ & $(2,2)$ & NO & 3071\\
$(99,29)$ & 10 & $(2,1)$ & 1 & 1 & YES & YES & YES & $1.00$ & $(2,2)$ & NO & 3072\\
$(99,41)$ & 10 & $(2,1)$ & 1 & 1 & YES & YES & YES & $1.00$ & $(2,2)$ & -- & 3073\\
$(99,29)$ & 10 & $(3,1)$ & 2 & 3 & YES & YES & YES & $1.00$ & $(2,2)$ & NO & 3074\\
$(99,23)$ & 11 & $(4,1)$ & 3 & 1 & YES & YES & YES & $1.12$ & $(2,2)$ & -- & 3075\\
$(99,23)$ & 11 & $(4,1)$ & 3 & 1 & YES & YES & YES & $1.25$ & $(2,2)$ & NO & 3076\\
$(99,23)$ & 11 & $(5,2)$ & 3 & 1 & YES & YES & YES & $1.12$ & $(2,2)$ & -- & 3077\\
$(99,23)$ & 11 & $(5,2)$ & 3 & 1 & YES & YES & YES & $1.25$ & $(2,2)$ & NO & 3078\\
$(99,41)$ & 10 & $(5,2)$ & 3 & 1 & YES & YES & YES & $1.12$ & $(2,2)$ & NO & 3079\\
$(99,23)$ & 11 & $(69,16)$ & 11 & 3 & YES & YES & YES & $1.00$ & $(2,2)$ & NO & 3080\\
$(100,19)$ & 12 & $(2,1)$ & 1 & 2 & YES & YES & YES & $1.00$ & $(2,2)$ & NO & 3081\\
$(100,27)$ & 10 & $(2,1)$ & 1 & 2 & YES & YES & YES & $1.00$ & $(2,2)$ & NO & 3082\\
$(100,27)$ & 10 & $(2,1)$ & 1 & 2 & YES & YES & NO(2) & $1.12$ & $(4,1)$ & -- & 3083\\
$(100,41)$ & 10 & $(2,1)$ & 1 & 2 & NO & YES & YES & $1.12$ & $(2,2)$ & -- & 3084\\
$(100,19)$ & 12 & $(3,1)$ & 2 & 1 & YES & YES & YES & $0.88$ & $(2,2)$ & NO & 3085\\
$(100,21)$ & 12 & $(3,1)$ & 2 & 1 & YES & YES & YES & $1.12$ & $(2,2)$ & -- & 3086\\
$(100,21)$ & 12 & $(3,1)$ & 2 & 1 & YES & YES & YES & $1.25$ & $(2,2)$ & NO & 3087\\
$(100,31)$ & 11 & $(3,1)$ & 2 & 1 & NO & YES & YES & $1.12$ & $(2,2)$ & -- & 3088\\
$(100,37)$ & 10 & $(3,1)$ & 2 & 1 & NO & YES & YES & $1.25$ & $(2,2)$ & -- & 3089\\
$(100,19)$ & 12 & $(4,1)$ & 3 & 4 & YES & YES & YES & $1.00$ & $(2,2)$ & NO & 3090\\
$(100,21)$ & 12 & $(4,1)$ & 3 & 4 & YES & YES & YES & $1.25$ & $(2,2)$ & -- & 3091\\
$(100,27)$ & 10 & $(4,1)$ & 3 & 4 & YES & YES & YES & $1.00$ & $(2,2)$ & -- & 3092\\
$(100,19)$ & 12 & $(6,1)$ & 5 & 2 & YES & YES & YES & $1.12$ & $(2,2)$ & NO & 3093\\
$(100,21)$ & 12 & $(6,1)$ & 5 & 2 & YES & YES & YES & $1.25$ & $(2,2)$ & 2186 & 3094\\
$(100,27)$ & 10 & $(7,2)$ & 4 & 1 & YES & YES & YES & $1.00$ & $(2,2)$ & NO & 3095\\
$(100,21)$ & 12 & $(9,2)$ & 5 & 1 & YES & YES & YES & $1.25$ & $(2,2)$ & NO & 3096\\
$(100,27)$ & 10 & $(11,3)$ & 5 & 1 & YES & YES & YES & $0.88$ & $(2,2)$ & NO & 3097\\
$(100,21)$ & 12 & $(14,3)$ & 6 & 2 & YES & YES & YES & $1.12$ & $(2,2)$ & NO & 3098\\
$(100,19)$ & 12 & $(16,3)$ & 7 & 4 & YES & YES & YES & $1.12$ & $(2,2)$ & NO & 3099\\
$(100,21)$ & 12 & $(24,5)$ & 8 & 4 & YES & YES & YES & $1.25$ & $(2,2)$ & NO & 3100\\
$(100,27)$ & 10 & $(37,10)$ & 8 & 1 & YES & YES & YES & $0.88$ & $(2,2)$ & NO & 3101\\
$(101,23)$ & 11 & $(2,1)$ & 1 & 1 & YES & YES & YES & $1.12$ & $(2,2)$ & -- & 3102\\
$(101,44)$ & 10 & $(2,1)$ & 1 & 1 & YES & YES & YES & $1.12$ & $(2,2)$ & -- & 3103\\
$(101,23)$ & 11 & $(3,1)$ & 2 & 1 & YES & YES & NO(2) & $1.00$ & $(4,1)$ & -- & 3104\\
$(101,23)$ & 11 & $(3,1)$ & 2 & 1 & YES & YES & YES & $1.00$ & $(2,2)$ & NO & 3105\\
$(101,30)$ & 10 & $(3,1)$ & 2 & 1 & YES & YES & YES & $1.12$ & $(2,2)$ & NO & 3106\\
$(101,23)$ & 11 & $(4,1)$ & 3 & 1 & YES & YES & YES & $1.00$ & $(2,2)$ & NO & 3107\\
$(101,44)$ & 10 & $(7,3)$ & 4 & 1 & YES & YES & YES & $1.12$ & $(2,2)$ & 2890 & 3108\\
$(101,23)$ & 11 & $(9,2)$ & 5 & 1 & YES & YES & YES & $1.00$ & $(2,2)$ & NO & 3109\\
$(101,30)$ & 10 & $(10,3)$ & 5 & 1 & YES & YES & YES & $1.12$ & $(2,2)$ & NO & 3110\\
$(102,31)$ & 11 & $(23,7)$ & 7 & 1 & YES & YES & YES & $1.25$ & $(2,2)$ & NO & 3111\\
$(102,31)$ & 11 & $(79,24)$ & 10 & 1 & YES & YES & YES & $1.12$ & $(2,2)$ & NO & 3112\\
$(103,24)$ & 11 & $(3,1)$ & 2 & 1 & YES & YES & YES & $1.00$ & $(2,2)$ & -- & 3113\\
$(103,24)$ & 11 & $(3,1)$ & 2 & 1 & YES & YES & YES & $1.12$ & $(2,2)$ & NO & 3114\\
$(103,24)$ & 11 & $(17,4)$ & 7 & 1 & YES & YES & YES & $1.00$ & $(2,2)$ & NO & 3115\\
$(105,23)$ & 11 & $(2,1)$ & 1 & 1 & YES & YES & YES & $0.88$ & $(2,2)$ & -- & 3116\\
$(105,23)$ & 11 & $(2,1)$ & 1 & 1 & YES & YES & YES & $1.00$ & $(2,2)$ & NO & 3117\\
$(105,41)$ & 10 & $(2,1)$ & 1 & 1 & NO & YES & YES & $1.12$ & $(2,2)$ & -- & 3118\\
$(105,23)$ & 11 & $(3,1)$ & 2 & 3 & YES & YES & YES & $0.88$ & $(2,2)$ & -- & 3119\\
$(105,23)$ & 11 & $(3,1)$ & 2 & 3 & YES & YES & YES & $1.12$ & $(2,2)$ & NO & 3120\\
$(105,23)$ & 11 & $(4,1)$ & 3 & 1 & YES & YES & YES & $1.12$ & $(2,2)$ & NO & 3121\\
$(105,31)$ & 10 & $(4,1)$ & 3 & 1 & YES & YES & YES & $1.00$ & $(2,2)$ & NO & 3122\\
$(105,23)$ & 11 & $(14,3)$ & 6 & 7 & YES & YES & YES & $0.88$ & $(2,2)$ & NO & 3123\\
$(105,31)$ & 10 & $(17,5)$ & 6 & 1 & YES & YES & YES & $1.12$ & $(2,2)$ & 2768 & 3124\\
$(105,23)$ & 11 & $(23,5)$ & 7 & 1 & YES & YES & YES & $0.88$ & $(2,2)$ & NO & 3125\\
$(106,23)$ & 11 & $(2,1)$ & 1 & 2 & YES & YES & YES & $1.00$ & $(2,2)$ & -- & 3126\\
$(106,41)$ & 10 & $(2,1)$ & 1 & 2 & NO & YES & YES & $1.12$ & $(2,2)$ & -- & 3127\\
$(106,23)$ & 11 & $(3,1)$ & 2 & 1 & YES & YES & YES & $1.00$ & $(2,2)$ & -- & 3128\\
$(106,31)$ & 10 & $(3,1)$ & 2 & 1 & YES & YES & YES & $1.00$ & $(2,2)$ & 2390 & 3129\\
$(106,23)$ & 11 & $(5,1)$ & 4 & 1 & YES & YES & YES & $0.88$ & $(2,2)$ & NO & 3130\\
$(106,23)$ & 11 & $(9,2)$ & 5 & 1 & YES & YES & YES & $0.88$ & $(2,2)$ & NO & 3131\\
$(106,23)$ & 11 & $(37,8)$ & 8 & 1 & YES & YES & YES & $0.88$ & $(2,2)$ & NO & 3132\\
$(107,25)$ & 11 & $(2,1)$ & 1 & 1 & YES & YES & YES & $1.12$ & $(2,2)$ & NO & 3133\\
$(107,25)$ & 11 & $(2,1)$ & 1 & 1 & YES & YES & YES & $1.12$ & $(2,2)$ & -- & 3134\\
$(107,25)$ & 11 & $(3,1)$ & 2 & 1 & YES & YES & YES & $1.00$ & $(2,2)$ & NO & 3135\\
$(107,25)$ & 11 & $(3,1)$ & 2 & 1 & YES & YES & YES & $1.00$ & $(2,2)$ & -- & 3136\\
$(107,25)$ & 11 & $(9,2)$ & 5 & 1 & YES & YES & YES & $1.12$ & $(2,2)$ & NO & 3137\\
$(107,25)$ & 11 & $(13,3)$ & 6 & 1 & YES & YES & YES & $1.00$ & $(2,2)$ & NO & 3138\\
$(107,25)$ & 11 & $(30,7)$ & 8 & 1 & YES & YES & YES & $1.12$ & $(2,2)$ & NO & 3139\\
$(107,25)$ & 11 & $(47,11)$ & 9 & 1 & YES & YES & YES & $1.00$ & $(2,2)$ & 3202 & 3140\\
$(108,29)$ & 10 & $(2,1)$ & 1 & 2 & YES & YES & YES & $1.00$ & $(2,2)$ & NO & 3141\\
$(108,29)$ & 10 & $(2,1)$ & 1 & 2 & YES & YES & YES & $1.00$ & $(2,2)$ & -- & 3142\\
$(108,29)$ & 10 & $(15,4)$ & 6 & 3 & YES & YES & YES & $1.00$ & $(2,2)$ & NO & 3143\\
$(108,29)$ & 10 & $(26,7)$ & 7 & 2 & YES & YES & YES & $1.00$ & $(2,2)$ & 3024 & 3144\\
$(108,29)$ & 10 & $(41,11)$ & 8 & 1 & YES & YES & YES & $1.00$ & $(2,2)$ & NO & 3145\\
$(109,30)$ & 10 & $(4,1)$ & 3 & 1 & YES & YES & YES & $1.00$ & $(2,2)$ & NO & 3146\\
$(109,30)$ & 10 & $(11,3)$ & 5 & 1 & YES & YES & YES & $1.12$ & $(2,2)$ & NO & 3147\\
$(110,21)$ & 13 & $(3,1)$ & 2 & 1 & YES & YES & YES & $1.12$ & $(2,2)$ & NO & 3148\\
$(110,21)$ & 13 & $(3,1)$ & 2 & 1 & YES & YES & YES & $1.12$ & $(2,2)$ & -- & 3149\\
$(110,21)$ & 13 & $(3,1)$ & 2 & 1 & YES & YES & YES & $1.25$ & $(2,2)$ & NO & 3150\\
$(110,21)$ & 13 & $(4,1)$ & 3 & 2 & YES & YES & YES & $1.25$ & $(2,2)$ & 1881 & 3151\\
$(110,21)$ & 13 & $(4,1)$ & 3 & 2 & YES & YES & YES & $1.25$ & $(2,2)$ & -- & 3152\\
$(110,21)$ & 13 & $(11,2)$ & 6 & 11 & YES & YES & YES & $1.25$ & $(2,2)$ & NO & 3153\\
$(110,21)$ & 13 & $(16,3)$ & 7 & 2 & YES & YES & YES & $1.12$ & $(2,2)$ & NO & 3154\\
$(111,41)$ & 10 & $(2,1)$ & 1 & 1 & YES & YES & YES & $1.12$ & $(2,2)$ & NO & 3155\\
$(111,41)$ & 10 & $(2,1)$ & 1 & 1 & YES & YES & YES & $1.12$ & $(2,2)$ & -- & 3156\\
$(111,26)$ & 11 & $(3,1)$ & 2 & 3 & YES & YES & YES & $1.12$ & $(2,2)$ & NO & 3157\\
$(111,26)$ & 11 & $(3,1)$ & 2 & 3 & YES & YES & YES & $1.12$ & $(2,2)$ & -- & 3158\\
$(111,26)$ & 11 & $(3,1)$ & 2 & 3 & YES & YES & YES & $1.25$ & $(2,2)$ & NO & 3159\\
$(111,41)$ & 10 & $(3,1)$ & 2 & 3 & YES & YES & YES & $1.12$ & $(2,2)$ & NO & 3160\\
$(111,31)$ & 10 & $(4,1)$ & 3 & 1 & YES & YES & YES & $0.88$ & $(2,2)$ & 2478 & 3161\\
$(111,26)$ & 11 & $(9,2)$ & 5 & 3 & YES & YES & YES & $1.00$ & $(2,2)$ & NO & 3162\\
$(111,26)$ & 11 & $(21,5)$ & 8 & 3 & YES & YES & YES & $1.12$ & $(2,2)$ & NO & 3163\\
$(111,26)$ & 11 & $(30,7)$ & 8 & 3 & YES & YES & YES & $1.00$ & $(2,2)$ & NO & 3164\\
$(111,41)$ & 10 & $(111,41)$ & 10 & 111 & YES & YES & YES & $1.12$ & $(2,2)$ & NO & 3165\\
$(112,31)$ & 10 & $(5,1)$ & 4 & 1 & YES & YES & YES & $1.12$ & $(2,2)$ & NO & 3166\\
$(112,31)$ & 10 & $(65,18)$ & 9 & 1 & YES & YES & YES & $1.00$ & $(2,2)$ & NO & 3167\\
$(113,24)$ & 11 & $(3,1)$ & 2 & 1 & YES & YES & YES & $1.00$ & $(2,2)$ & NO & 3168\\
$(113,24)$ & 11 & $(3,1)$ & 2 & 1 & YES & YES & YES & $1.00$ & $(2,2)$ & -- & 3169\\
$(113,24)$ & 11 & $(3,1)$ & 2 & 1 & YES & YES & YES & $1.12$ & $(2,2)$ & NO & 3170\\
$(113,24)$ & 11 & $(9,2)$ & 5 & 1 & YES & YES & YES & $1.00$ & $(2,2)$ & 2891 & 3171\\
$(113,24)$ & 11 & $(19,4)$ & 7 & 1 & YES & YES & YES & $1.12$ & $(2,2)$ & NO & 3172\\
$(115,26)$ & 11 & $(2,1)$ & 1 & 1 & YES & YES & YES & $1.00$ & $(2,2)$ & -- & 3173\\
$(115,26)$ & 11 & $(6,1)$ & 5 & 1 & YES & YES & YES & $1.00$ & $(2,2)$ & NO & 3174\\
$(115,26)$ & 11 & $(9,2)$ & 5 & 1 & YES & YES & YES & $1.00$ & $(2,2)$ & NO & 3175\\
$(115,26)$ & 11 & $(31,7)$ & 8 & 1 & YES & YES & YES & $1.00$ & $(2,2)$ & NO & 3176\\
$(116,25)$ & 11 & $(2,1)$ & 1 & 2 & YES & YES & YES & $0.88$ & $(2,2)$ & -- & 3177\\
$(116,25)$ & 11 & $(2,1)$ & 1 & 2 & YES & YES & YES & $1.00$ & $(2,2)$ & NO & 3178\\
$(116,25)$ & 11 & $(3,1)$ & 2 & 1 & YES & YES & YES & $0.88$ & $(2,2)$ & -- & 3179\\
$(116,25)$ & 11 & $(4,1)$ & 3 & 4 & YES & YES & YES & $1.12$ & $(2,2)$ & NO & 3180\\
$(116,25)$ & 11 & $(65,14)$ & 10 & 1 & YES & YES & YES & $1.00$ & $(2,2)$ & NO & 3181\\
$(116,25)$ & 11 & $(116,25)$ & 11 & 116 & YES & YES & YES & $0.88$ & $(2,2)$ & NO & 3182\\
$(117,31)$ & 11 & $(34,9)$ & 8 & 1 & YES & YES & YES & $1.00$ & $(2,2)$ & NO & 3183\\
$(118,27)$ & 11 & $(2,1)$ & 1 & 2 & YES & YES & YES & $1.12$ & $(2,2)$ & NO & 3184\\
$(118,53)$ & 11 & $(2,1)$ & 1 & 2 & NO & YES & YES & $1.12$ & $(2,2)$ & -- & 3185\\
$(118,27)$ & 11 & $(35,8)$ & 8 & 1 & YES & YES & YES & $1.12$ & $(2,2)$ & NO & 3186\\
$(119,46)$ & 10 & $(2,1)$ & 1 & 1 & NO & YES & YES & $1.12$ & $(2,2)$ & -- & 3187\\
$(119,22)$ & 12 & $(3,1)$ & 2 & 1 & YES & YES & YES & $1.00$ & $(2,2)$ & NO & 3188\\
$(119,22)$ & 12 & $(4,1)$ & 3 & 1 & YES & YES & YES & $1.00$ & $(2,2)$ & NO & 3189\\
$(119,27)$ & 12 & $(4,1)$ & 3 & 1 & YES & YES & YES & $1.00$ & $(2,2)$ & NO & 3190\\
$(121,50)$ & 10 & $(2,1)$ & 1 & 1 & NO & YES & YES & $1.12$ & $(2,2)$ & -- & 3191\\
$(122,33)$ & 11 & $(4,1)$ & 3 & 2 & YES & YES & YES & $1.12$ & $(2,2)$ & 1767 & 3192\\
$(123,22)$ & 12 & $(2,1)$ & 1 & 1 & YES & YES & YES & $1.12$ & $(2,2)$ & NO & 3193\\
$(123,22)$ & 12 & $(3,1)$ & 2 & 3 & YES & YES & YES & $1.00$ & $(2,2)$ & NO & 3194\\
$(123,22)$ & 12 & $(4,1)$ & 3 & 1 & YES & YES & YES & $1.00$ & $(2,2)$ & NO & 3195\\
$(123,22)$ & 12 & $(5,1)$ & 4 & 1 & YES & YES & YES & $1.12$ & $(2,2)$ & NO & 3196\\
$(124,23)$ & 12 & $(2,1)$ & 1 & 2 & YES & YES & YES & $1.00$ & $(2,2)$ & -- & 3197\\
$(124,29)$ & 11 & $(5,1)$ & 4 & 1 & YES & YES & YES & $1.00$ & $(2,2)$ & NO & 3198\\
$(124,23)$ & 12 & $(6,1)$ & 5 & 2 & YES & YES & YES & $1.00$ & $(2,2)$ & NO & 3199\\
$(124,23)$ & 12 & $(11,2)$ & 6 & 1 & YES & YES & YES & $1.00$ & $(2,2)$ & NO & 3200\\
$(124,23)$ & 12 & $(16,3)$ & 7 & 4 & YES & YES & YES & $1.00$ & $(2,2)$ & NO & 3201\\
$(124,29)$ & 11 & $(30,7)$ & 8 & 2 & YES & YES & YES & $1.00$ & $(2,2)$ & 3140 & 3202\\
$(125,29)$ & 12 & $(2,1)$ & 1 & 1 & YES & YES & YES & $1.25$ & $(2,2)$ & NO & 3203\\
$(125,29)$ & 12 & $(4,1)$ & 3 & 1 & YES & YES & YES & $1.12$ & $(2,2)$ & NO & 3204\\
$(125,29)$ & 12 & $(5,1)$ & 4 & 5 & YES & YES & YES & $1.12$ & $(2,2)$ & NO & 3205\\
$(125,29)$ & 12 & $(17,4)$ & 7 & 1 & YES & YES & YES & $1.12$ & $(2,2)$ & NO & 3206\\
$(125,29)$ & 12 & $(43,10)$ & 9 & 1 & YES & YES & YES & $1.00$ & $(2,2)$ & NO & 3207\\
$(125,29)$ & 12 & $(56,13)$ & 10 & 1 & YES & YES & YES & $1.25$ & $(2,2)$ & NO & 3208\\
$(126,55)$ & 11 & $(2,1)$ & 1 & 2 & NO & YES & YES & $1.25$ & $(2,2)$ & -- & 3209\\
$(127,30)$ & 12 & $(3,1)$ & 2 & 1 & YES & YES & YES & $1.12$ & $(2,2)$ & -- & 3210\\
$(127,30)$ & 12 & $(3,1)$ & 2 & 1 & YES & YES & YES & $1.25$ & $(2,2)$ & NO & 3211\\
$(127,30)$ & 12 & $(13,3)$ & 6 & 1 & YES & YES & YES & $1.12$ & $(2,2)$ & NO & 3212\\
$(127,30)$ & 12 & $(21,5)$ & 8 & 1 & YES & YES & YES & $1.12$ & $(2,2)$ & 2628 & 3213\\
$(129,56)$ & 11 & $(2,1)$ & 1 & 1 & NO & YES & YES & $1.25$ & $(2,2)$ & -- & 3214\\
$(131,31)$ & 12 & $(2,1)$ & 1 & 1 & YES & YES & YES & $1.12$ & $(2,2)$ & NO & 3215\\
$(131,31)$ & 12 & $(38,9)$ & 9 & 1 & YES & YES & YES & $1.12$ & $(2,2)$ & NO & 3216\\
$(133,31)$ & 12 & $(4,1)$ & 3 & 1 & YES & YES & YES & $1.12$ & $(2,2)$ & NO & 3217\\
$(133,31)$ & 12 & $(30,7)$ & 8 & 1 & YES & YES & YES & $1.00$ & $(2,2)$ & NO & 3218\\
$(133,30)$ & 12 & $(31,7)$ & 8 & 1 & YES & YES & YES & $1.25$ & $(2,2)$ & NO & 3219\\
$(133,31)$ & 12 & $(73,17)$ & 10 & 1 & YES & YES & YES & $1.12$ & $(2,2)$ & 3245 & 3220\\
$(134,55)$ & 11 & $(2,1)$ & 1 & 2 & NO & YES & YES & $1.25$ & $(2,2)$ & -- & 3221\\
$(134,29)$ & 11 & $(5,1)$ & 4 & 1 & YES & YES & YES & $1.00$ & $(2,2)$ & NO & 3222\\
$(134,29)$ & 11 & $(37,8)$ & 8 & 1 & YES & YES & YES & $1.00$ & $(2,2)$ & NO & 3223\\
$(137,30)$ & 12 & $(32,7)$ & 8 & 1 & YES & YES & YES & $1.25$ & $(2,2)$ & NO & 3224\\
$(142,33)$ & 12 & $(4,1)$ & 3 & 2 & YES & YES & YES & $1.25$ & $(2,2)$ & 2057 & 3225\\
$(146,27)$ & 13 & $(2,1)$ & 1 & 2 & YES & YES & YES & $1.25$ & $(2,2)$ & -- & 3226\\
$(146,31)$ & 12 & $(5,1)$ & 4 & 1 & YES & YES & YES & $1.00$ & $(2,2)$ & NO & 3227\\
$(146,27)$ & 13 & $(11,2)$ & 6 & 1 & YES & YES & YES & $1.25$ & $(2,2)$ & NO & 3228\\
$(146,31)$ & 12 & $(33,7)$ & 8 & 1 & YES & YES & YES & $1.00$ & $(2,2)$ & NO & 3229\\
$(151,27)$ & 13 & $(2,1)$ & 1 & 1 & YES & YES & YES & $1.12$ & $(2,2)$ & -- & 3230\\
$(151,27)$ & 13 & $(6,1)$ & 5 & 1 & YES & YES & YES & $1.00$ & $(2,2)$ & NO & 3231\\
$(151,27)$ & 13 & $(11,2)$ & 6 & 1 & YES & YES & YES & $1.12$ & $(2,2)$ & NO & 3232\\
$(154,29)$ & 13 & $(3,1)$ & 2 & 1 & YES & YES & YES & $1.12$ & $(2,2)$ & NO & 3233\\
$(154,29)$ & 13 & $(3,1)$ & 2 & 1 & YES & YES & YES & $1.25$ & $(2,2)$ & NO & 3234\\
$(154,29)$ & 13 & $(3,1)$ & 2 & 1 & YES & YES & YES & $1.12$ & $(2,2)$ & -- & 3235\\
$(154,29)$ & 13 & $(4,1)$ & 3 & 2 & YES & YES & YES & $1.12$ & $(2,2)$ & NO & 3236\\
$(154,29)$ & 13 & $(154,29)$ & 13 & 154 & YES & YES & YES & $1.12$ & $(2,2)$ & NO & 3237\\
$(155,36)$ & 12 & $(2,1)$ & 1 & 1 & YES & YES & YES & $1.12$ & $(2,2)$ & NO & 3238\\
$(155,36)$ & 12 & $(4,1)$ & 3 & 1 & YES & YES & YES & $1.12$ & $(2,2)$ & NO & 3239\\
$(155,36)$ & 12 & $(13,3)$ & 6 & 1 & YES & YES & YES & $1.00$ & $(2,2)$ & NO & 3240\\
$(155,36)$ & 12 & $(56,13)$ & 10 & 1 & YES & YES & YES & $1.12$ & $(2,2)$ & NO & 3241\\
$(164,31)$ & 13 & $(5,1)$ & 4 & 1 & YES & YES & YES & $1.25$ & $(2,2)$ & NO & 3242\\
$(169,70)$ & 11 & $(2,1)$ & 1 & 1 & NO & YES & YES & $1.12$ & $(2,2)$ & -- & 3243\\
$(176,41)$ & 12 & $(4,1)$ & 3 & 4 & YES & YES & YES & $1.12$ & $(2,2)$ & NO & 3244\\
$(176,41)$ & 12 & $(30,7)$ & 8 & 2 & YES & YES & YES & $1.12$ & $(2,2)$ & 3220 & 3245\\
$(a;0,0,0;3)$ & 4 & $(11,3)$ & 5 & 1 & YES & YES & YES & $1.00$ & $(2,2)$ & -- & 3246\\
$(a;0,0,0;3)$ & 4 & $(12,5)$ & 5 & 3 & YES & YES & YES & $1.00$ & $(2,2)$ & -- & 3247\\
$(a;0,0,0;3)$ & 4 & $(13,5)$ & 5 & 1 & YES & YES & YES & $1.00$ & $(2,2)$ & -- & 3248\\
$(a;0,0,0;3)$ & 4 & $(18,7)$ & 6 & 3 & YES & YES & YES & $0.88$ & $(2,2)$ & -- & 3249\\
$(a;0,0,0;3)$ & 4 & $(19,7)$ & 6 & 1 & YES & YES & YES & $1.00$ & $(2,2)$ & -- & 3250\\
$(a;0,0,0;3)$ & 4 & $(23,6)$ & 8 & 1 & YES & YES & YES & $1.25$ & $(2,2)$ & -- & 3251\\
$(a;0,0,0;3)$ & 4 & $(23,7)$ & 7 & 1 & YES & YES & YES & $0.88$ & $(2,2)$ & -- & 3252\\
$(a;0,0,0;3)$ & 4 & $(23,10)$ & 7 & 1 & YES & YES & YES & $1.25$ & $(2,2)$ & -- & 3253\\
$(a;1,0,0;13)$ & 5 & $(8,3)$ & 4 & 1 & YES & YES & YES & $0.88$ & $(2,2)$ & -- & 3254\\
$(a;1,0,0;13)$ & 5 & $(10,3)$ & 5 & 1 & YES & YES & YES & $1.12$ & $(2,2)$ & -- & 3255\\
$(a;1,0,0;13)$ & 5 & $(11,4)$ & 5 & 1 & YES & YES & YES & $1.00$ & $(2,2)$ & -- & 3256\\
$(a;1,0,0;13)$ & 5 & $(12,5)$ & 5 & 1 & YES & YES & YES & $0.88$ & $(2,2)$ & -- & 3257\\
$(a;1,0,0;13)$ & 5 & $(13,5)$ & 5 & 13 & YES & YES & YES & $1.00$ & $(2,2)$ & -- & 3258\\
$(a;1,0,0;13)$ & 5 & $(14,5)$ & 6 & 1 & YES & YES & YES & $1.00$ & $(2,2)$ & -- & 3259\\
$(a;1,0,0;13)$ & 5 & $(16,7)$ & 6 & 1 & YES & YES & YES & $1.25$ & $(2,2)$ & -- & 3260\\
$(a;1,0,0;13)$ & 5 & $(19,8)$ & 6 & 1 & YES & YES & YES & $1.00$ & $(2,2)$ & -- & 3261\\
$(a;1,0,0;13)$ & 5 & $(27,8)$ & 7 & 1 & YES & YES & YES & $1.12$ & $(2,2)$ & -- & 3262\\
$(a;1,1,0;19)$ & 6 & $(7,2)$ & 4 & 1 & YES & YES & YES & $1.12$ & $(2,2)$ & -- & 3263\\
$(a;1,1,0;19)$ & 6 & $(7,3)$ & 4 & 1 & YES & YES & YES & $1.25$ & $(2,2)$ & -- & 3264\\
$(a;1,1,0;19)$ & 6 & $(8,3)$ & 4 & 1 & YES & YES & YES & $1.00$ & $(2,2)$ & -- & 3265\\
$(a;1,1,0;19)$ & 6 & $(9,4)$ & 5 & 1 & YES & YES & YES & $1.25$ & $(2,2)$ & -- & 3266\\
$(a;1,1,0;19)$ & 6 & $(11,4)$ & 5 & 1 & YES & YES & YES & $1.00$ & $(2,2)$ & -- & 3267\\
$(a;1,1,0;19)$ & 6 & $(13,4)$ & 6 & 1 & YES & YES & YES & $1.25$ & $(2,2)$ & -- & 3268\\
$(a;1,1,0;19)$ & 6 & $(17,5)$ & 6 & 1 & YES & YES & YES & $1.00$ & $(2,2)$ & -- & 3269\\
$(a;1,1,1;4)$ & 7 & $(5,2)$ & 3 & 1 & YES & YES & YES & $1.12$ & $(2,2)$ & -- & 3270\\
$(a;2,0,0;17)$ & 6 & $(7,3)$ & 4 & 1 & YES & YES & YES & $1.00$ & $(2,2)$ & -- & 3271\\
$(a;2,0,0;17)$ & 6 & $(8,3)$ & 4 & 1 & YES & YES & YES & $0.88$ & $(2,2)$ & -- & 3272\\
$(a;2,0,0;17)$ & 6 & $(9,2)$ & 5 & 1 & YES & YES & YES & $1.12$ & $(2,2)$ & -- & 3273\\
$(a;2,0,0;17)$ & 6 & $(9,4)$ & 5 & 1 & YES & YES & YES & $1.12$ & $(2,2)$ & -- & 3274\\
$(a;2,0,0;17)$ & 6 & $(10,3)$ & 5 & 1 & YES & YES & YES & $1.00$ & $(2,2)$ & -- & 3275\\
$(a;2,0,0;17)$ & 6 & $(12,5)$ & 5 & 1 & YES & YES & YES & $1.12$ & $(2,2)$ & -- & 3276\\
$(a;2,0,0;17)$ & 6 & $(13,5)$ & 5 & 1 & YES & YES & YES & $1.12$ & $(2,2)$ & -- & 3277\\
$(a;2,0,0;17)$ & 6 & $(16,7)$ & 6 & 1 & YES & YES & YES & $1.12$ & $(2,2)$ & -- & 3278\\
$(a;2,0,0;17)$ & 6 & $(21,5)$ & 8 & 1 & YES & YES & YES & $1.25$ & $(2,2)$ & -- & 3279\\
$(a;2,0,1;25)$ & 7 & $(5,2)$ & 3 & 5 & YES & YES & YES & $1.25$ & $(2,2)$ & -- & 3280\\
$(a;2,0,1;25)$ & 7 & $(7,2)$ & 4 & 1 & YES & YES & YES & $1.00$ & $(2,2)$ & -- & 3281\\
$(a;2,0,1;25)$ & 7 & $(8,3)$ & 4 & 1 & YES & YES & YES & $1.12$ & $(2,2)$ & -- & 3282\\
$(a;2,1,0;5)$ & 7 & $(5,2)$ & 3 & 5 & YES & YES & YES & $1.00$ & $(2,2)$ & -- & 3283\\
$(a;2,1,0;5)$ & 7 & $(7,3)$ & 4 & 1 & YES & YES & YES & $1.00$ & $(2,2)$ & -- & 3284\\
$(a;2,1,0;5)$ & 7 & $(11,3)$ & 5 & 1 & YES & YES & YES & $1.00$ & $(2,2)$ & -- & 3285\\
$(a;2,1,1;37)$ & 8 & $(2,1)$ & 1 & 1 & YES & YES & YES & $1.12$ & $(2,2)$ & -- & 3286\\
$(a;2,1,1;37)$ & 8 & $(3,1)$ & 2 & 1 & YES & YES & YES & $1.12$ & $(2,2)$ & -- & 3287\\
$(a;2,1,1;37)$ & 8 & $(4,1)$ & 3 & 1 & YES & YES & YES & $1.12$ & $(2,2)$ & -- & 3288\\
$(a;2,2,0;33)$ & 8 & $(4,1)$ & 3 & 1 & YES & YES & YES & $1.12$ & $(2,2)$ & -- & 3289\\
$(a;2,2,0;33)$ & 8 & $(5,2)$ & 3 & 1 & YES & YES & YES & $1.25$ & $(2,2)$ & -- & 3290\\
$(a;2,2,0;33)$ & 8 & $(7,2)$ & 4 & 1 & YES & YES & YES & $1.25$ & $(2,2)$ & -- & 3291\\
$(a;2,2,0;33)$ & 8 & $(13,3)$ & 6 & 1 & YES & YES & YES & $1.12$ & $(2,2)$ & -- & 3292\\
$(a;3,0,0;7)$ & 7 & $(7,2)$ & 4 & 7 & YES & YES & YES & $0.88$ & $(2,2)$ & -- & 3293\\
$(a;3,0,1;31)$ & 8 & $(5,2)$ & 3 & 1 & YES & YES & YES & $1.12$ & $(2,2)$ & -- & 3294\\
$(a;3,0,1;31)$ & 8 & $(9,2)$ & 5 & 1 & YES & YES & YES & $1.00$ & $(2,2)$ & -- & 3295\\
$(a;3,0,2;41)$ & 9 & $(3,1)$ & 2 & 1 & YES & YES & YES & $1.25$ & $(2,2)$ & -- & 3296\\
$(a;3,0,2;41)$ & 9 & $(4,1)$ & 3 & 1 & YES & YES & YES & $1.12$ & $(2,2)$ & -- & 3297\\
$(a;3,0,2;41)$ & 9 & $(11,2)$ & 6 & 1 & YES & YES & YES & $1.12$ & $(2,2)$ & -- & 3298\\
$(a;3,1,0;31)$ & 8 & $(2,1)$ & 1 & 1 & YES & YES & YES & $1.00$ & $(2,2)$ & -- & 3299\\
$(a;3,1,0;31)$ & 8 & $(3,1)$ & 2 & 1 & YES & YES & YES & $1.00$ & $(2,2)$ & -- & 3300\\
$(a;3,1,0;31)$ & 8 & $(4,1)$ & 3 & 1 & YES & YES & YES & $1.00$ & $(2,2)$ & -- & 3301\\
$(a;4,0,0;25)$ & 8 & $(5,2)$ & 3 & 5 & YES & YES & YES & $1.25$ & $(2,2)$ & -- & 3302\\
$(a;4,0,1;37)$ & 9 & $(2,1)$ & 1 & 1 & YES & YES & YES & $1.12$ & $(2,2)$ & -- & 3303\\
$(a;4,0,1;37)$ & 9 & $(5,1)$ & 4 & 1 & YES & YES & YES & $1.12$ & $(2,2)$ & -- & 3304\\
$(a;4,0,1;37)$ & 9 & $(6,1)$ & 5 & 1 & YES & YES & YES & $1.12$ & $(2,2)$ & -- & 3305\\
$(b;0,0,0;14)$ & 5 & $(7,2)$ & 4 & 7 & YES & YES & NO(2) & $1.00$ & $(2,2)$ & -- & 3306\\
$(b;0,0,0;14)$ & 5 & $(8,3)$ & 4 & 2 & YES & YES & YES & $1.11$ & $(2,2)$ & -- & 3307\\
$(b;0,0,0;14)$ & 5 & $(9,2)$ & 5 & 1 & YES & YES & YES & $1.25$ & $(2,2)$ & -- & 3308\\
$(b;0,0,0;14)$ & 5 & $(12,5)$ & 5 & 2 & YES & YES & YES & $1.00$ & $(2,2)$ & -- & 3309\\
$(b;0,0,0;14)$ & 5 & $(13,5)$ & 5 & 1 & YES & YES & YES & $1.12$ & $(2,2)$ & -- & 3310\\
$(b;0,0,0;14)$ & 5 & $(19,7)$ & 6 & 1 & YES & YES & YES & $1.38$ & $(2,2)$ & -- & 3311\\
$(b;0,0,0;14)$ & 5 & $(19,8)$ & 6 & 1 & YES & YES & YES & $1.00$ & $(2,2)$ & -- & 3312\\
$(b;0,0,1;4)$ & 6 & $(5,2)$ & 3 & 1 & YES & YES & NO(2) & $1.22$ & $(2,2)$ & -- & 3313\\
$(b;0,0,1;4)$ & 6 & $(7,3)$ & 4 & 1 & YES & YES & YES & $1.12$ & $(2,2)$ & -- & 3314\\
$(b;0,0,1;4)$ & 6 & $(12,5)$ & 5 & 4 & YES & YES & YES & $1.00$ & $(2,2)$ & -- & 3315\\
$(b;0,0,2;26)$ & 7 & $(3,1)$ & 2 & 1 & YES & YES & YES & $1.12$ & $(2,2)$ & -- & 3316\\
$(b;0,0,2;26)$ & 7 & $(4,1)$ & 3 & 2 & YES & YES & YES & $1.12$ & $(2,2)$ & -- & 3317\\
$(b;0,0,2;26)$ & 7 & $(5,2)$ & 3 & 1 & YES & YES & NO(2) & $1.00$ & $(4,1)$ & -- & 3318\\
$(b;0,0,2;26)$ & 7 & $(7,2)$ & 4 & 1 & YES & YES & NO(2) & $1.00$ & $(4,1)$ & -- & 3319\\
$(b;0,0,2;26)$ & 7 & $(7,3)$ & 4 & 1 & YES & YES & YES & $1.12$ & $(2,2)$ & -- & 3320\\
$(b;0,0,3;32)$ & 8 & $(2,1)$ & 1 & 2 & YES & YES & YES & $1.00$ & $(2,2)$ & -- & 3321\\
$(b;0,0,3;32)$ & 8 & $(3,1)$ & 2 & 1 & YES & YES & YES & $1.12$ & $(2,2)$ & -- & 3322\\
$(b;0,0,3;32)$ & 8 & $(4,1)$ & 3 & 4 & YES & YES & YES & $1.00$ & $(2,2)$ & -- & 3323\\
$(b;0,0,3;32)$ & 8 & $(5,1)$ & 4 & 1 & YES & YES & YES & $1.00$ & $(2,2)$ & -- & 3324\\
$(b;0,1,0;19)$ & 6 & $(7,2)$ & 4 & 1 & YES & YES & YES & $1.12$ & $(2,2)$ & -- & 3325\\
$(b;0,1,0;19)$ & 6 & $(7,3)$ & 4 & 1 & YES & YES & YES & $1.12$ & $(2,2)$ & -- & 3326\\
$(b;0,1,0;19)$ & 6 & $(8,3)$ & 4 & 1 & YES & YES & YES & $1.00$ & $(2,2)$ & -- & 3327\\
$(b;0,1,0;19)$ & 6 & $(10,3)$ & 5 & 1 & YES & YES & YES & $1.12$ & $(2,2)$ & -- & 3328\\
$(b;0,1,0;19)$ & 6 & $(11,3)$ & 5 & 1 & YES & YES & YES & $1.25$ & $(2,2)$ & -- & 3329\\
$(b;0,1,0;19)$ & 6 & $(12,5)$ & 5 & 1 & YES & YES & YES & $1.12$ & $(2,2)$ & -- & 3330\\
$(b;0,1,0;19)$ & 6 & $(16,7)$ & 6 & 1 & YES & YES & YES & $1.12$ & $(2,2)$ & -- & 3331\\
$(b;0,1,0;19)$ & 6 & $(17,4)$ & 7 & 1 & YES & YES & YES & $1.25$ & $(2,2)$ & -- & 3332\\
$(b;0,1,1;27)$ & 7 & $(2,1)$ & 1 & 1 & YES & YES & NO(3) & $0.75$ & $(2,2)$ & -- & 3333\\
$(b;0,1,1;27)$ & 7 & $(3,1)$ & 2 & 3 & YES & YES & NO(3) & $0.75$ & $(2,2)$ & -- & 3334\\
$(b;0,1,3;43)$ & 9 & $(2,1)$ & 1 & 1 & YES & YES & YES & $1.00$ & $(2,2)$ & -- & 3335\\
$(b;0,1,3;43)$ & 9 & $(11,2)$ & 6 & 1 & YES & YES & YES & $1.12$ & $(2,2)$ & -- & 3336\\
$(b;0,2,0;8)$ & 7 & $(3,1)$ & 2 & 1 & YES & YES & YES & $1.12$ & $(2,2)$ & -- & 3337\\
$(b;0,2,0;8)$ & 7 & $(5,2)$ & 3 & 1 & YES & YES & YES & $1.12$ & $(2,2)$ & -- & 3338\\
$(b;0,2,0;8)$ & 7 & $(7,2)$ & 4 & 1 & YES & YES & YES & $0.88$ & $(2,2)$ & -- & 3339\\
$(b;0,2,0;8)$ & 7 & $(7,3)$ & 4 & 1 & YES & YES & YES & $1.25$ & $(2,2)$ & -- & 3340\\
$(b;0,2,1;34)$ & 8 & $(2,1)$ & 1 & 2 & YES & YES & YES & $1.00$ & $(2,2)$ & -- & 3341\\
$(b;0,2,1;34)$ & 8 & $(3,1)$ & 2 & 1 & YES & YES & YES & $1.00$ & $(2,2)$ & -- & 3342\\
$(b;0,2,1;34)$ & 8 & $(5,1)$ & 4 & 1 & YES & YES & YES & $1.12$ & $(2,2)$ & -- & 3343\\
$(b;0,2,2;44)$ & 9 & $(5,1)$ & 4 & 1 & YES & YES & YES & $1.25$ & $(2,2)$ & -- & 3344\\
$(b;0,3,0;29)$ & 8 & $(5,1)$ & 4 & 1 & YES & YES & YES & $1.12$ & $(2,2)$ & -- & 3345\\
$(b;1,0,0;5)$ & 6 & $(5,2)$ & 3 & 5 & YES & YES & NO(2) & $1.11$ & $(2,2)$ & -- & 3346\\
$(b;1,0,0;5)$ & 6 & $(7,3)$ & 4 & 1 & YES & YES & YES & $1.00$ & $(2,2)$ & -- & 3347\\
$(b;1,0,0;5)$ & 6 & $(8,3)$ & 4 & 1 & YES & YES & YES & $1.00$ & $(2,2)$ & -- & 3348\\
$(b;1,0,1;29)$ & 7 & $(2,1)$ & 1 & 1 & YES & YES & YES & $1.00$ & $(2,2)$ & -- & 3349\\
$(b;1,0,1;29)$ & 7 & $(5,2)$ & 3 & 1 & YES & YES & YES & $1.00$ & $(2,2)$ & -- & 3350\\
$(b;1,0,2;19)$ & 8 & $(2,1)$ & 1 & 1 & YES & YES & YES & $1.00$ & $(2,2)$ & -- & 3351\\
$(b;1,0,2;19)$ & 8 & $(3,1)$ & 2 & 1 & YES & YES & NO(2) & $0.88$ & $(4,1)$ & -- & 3352\\
$(b;1,0,2;19)$ & 8 & $(4,1)$ & 3 & 1 & YES & YES & YES & $1.00$ & $(2,2)$ & -- & 3353\\
$(b;1,1,0;27)$ & 7 & $(3,1)$ & 2 & 3 & YES & YES & YES & $1.00$ & $(2,2)$ & -- & 3354\\
$(b;1,1,0;27)$ & 7 & $(4,1)$ & 3 & 1 & YES & YES & YES & $1.00$ & $(2,2)$ & -- & 3355\\
$(b;1,1,0;27)$ & 7 & $(5,2)$ & 3 & 1 & YES & YES & YES & $1.00$ & $(2,2)$ & -- & 3356\\
$(b;1,1,0;27)$ & 7 & $(7,2)$ & 4 & 1 & YES & YES & YES & $0.88$ & $(2,2)$ & -- & 3357\\
$(b;1,1,0;27)$ & 7 & $(7,3)$ & 4 & 1 & YES & YES & YES & $1.12$ & $(2,2)$ & -- & 3358\\
$(b;1,1,0;27)$ & 7 & $(13,3)$ & 6 & 1 & YES & YES & YES & $1.12$ & $(2,2)$ & -- & 3359\\
$(b;1,1,1;39)$ & 8 & $(2,1)$ & 1 & 1 & YES & YES & YES & $1.00$ & $(2,2)$ & -- & 3360\\
$(b;1,1,1;39)$ & 8 & $(3,1)$ & 2 & 3 & YES & YES & YES & $0.88$ & $(2,2)$ & -- & 3361\\
$(b;1,1,1;39)$ & 8 & $(4,1)$ & 3 & 1 & YES & YES & YES & $1.00$ & $(2,2)$ & -- & 3362\\
$(b;1,1,2;51)$ & 9 & $(4,1)$ & 3 & 1 & YES & YES & YES & $1.00$ & $(2,2)$ & -- & 3363\\
$(b;1,2,0;17)$ & 8 & $(2,1)$ & 1 & 1 & YES & YES & YES & $1.00$ & $(2,2)$ & -- & 3364\\
$(b;1,2,0;17)$ & 8 & $(3,1)$ & 2 & 1 & YES & YES & YES & $1.00$ & $(2,2)$ & -- & 3365\\
$(b;1,2,0;17)$ & 8 & $(4,1)$ & 3 & 1 & YES & YES & YES & $1.00$ & $(2,2)$ & -- & 3366\\
$(b;1,2,0;17)$ & 8 & $(5,1)$ & 4 & 1 & YES & YES & YES & $1.00$ & $(2,2)$ & -- & 3367\\
$(b;2,0,0;26)$ & 7 & $(4,1)$ & 3 & 2 & YES & YES & YES & $1.12$ & $(2,2)$ & -- & 3368\\
$(b;2,0,0;26)$ & 7 & $(5,2)$ & 3 & 1 & YES & YES & YES & $1.12$ & $(2,2)$ & -- & 3369\\
$(b;2,0,0;26)$ & 7 & $(7,2)$ & 4 & 1 & YES & YES & YES & $0.88$ & $(2,2)$ & -- & 3370\\
$(b;2,0,0;26)$ & 7 & $(7,3)$ & 4 & 1 & YES & YES & YES & $1.12$ & $(2,2)$ & -- & 3371\\
$(b;2,0,0;26)$ & 7 & $(13,3)$ & 6 & 13 & YES & YES & YES & $1.12$ & $(2,2)$ & -- & 3372\\
$(b;2,0,1;38)$ & 8 & $(2,1)$ & 1 & 2 & YES & YES & YES & $1.12$ & $(2,2)$ & -- & 3373\\
$(b;2,0,1;38)$ & 8 & $(3,1)$ & 2 & 1 & YES & YES & YES & $1.12$ & $(2,2)$ & -- & 3374\\
$(b;2,0,1;38)$ & 8 & $(4,1)$ & 3 & 2 & YES & YES & YES & $0.88$ & $(2,2)$ & -- & 3375\\
$(b;2,0,2;50)$ & 9 & $(5,1)$ & 4 & 5 & YES & YES & YES & $1.12$ & $(2,2)$ & -- & 3376\\
$(b;3,0,0;16)$ & 8 & $(2,1)$ & 1 & 2 & YES & YES & YES & $1.00$ & $(2,2)$ & -- & 3377\\
$(b;3,0,0;16)$ & 8 & $(4,1)$ & 3 & 4 & YES & YES & YES & $1.00$ & $(2,2)$ & -- & 3378\\
$(b;3,0,0;16)$ & 8 & $(5,1)$ & 4 & 1 & YES & YES & YES & $1.00$ & $(2,2)$ & -- & 3379\\
$(c;0,0,0;4)$ & 4 & $(11,3)$ & 5 & 1 & YES & YES & YES & $0.88$ & $(2,2)$ & -- & 3380\\
$(c;0,0,0;4)$ & 4 & $(14,3)$ & 6 & 2 & YES & YES & YES & $1.00$ & $(2,2)$ & -- & 3381\\
$(c;0,0,0;4)$ & 4 & $(19,7)$ & 6 & 1 & YES & YES & YES & $1.12$ & $(2,2)$ & -- & 3382\\
$(c;0,0,0;4)$ & 4 & $(20,9)$ & 7 & 4 & YES & YES & YES & $1.12$ & $(2,2)$ & -- & 3383\\
$(c;0,0,0;4)$ & 4 & $(23,10)$ & 7 & 1 & YES & YES & YES & $1.00$ & $(2,2)$ & -- & 3384\\
$(c;0,0,0;4)$ & 4 & $(25,9)$ & 7 & 1 & YES & YES & YES & $1.00$ & $(2,2)$ & -- & 3385\\
$(c;0,0,0;4)$ & 4 & $(27,10)$ & 7 & 1 & YES & YES & YES & $1.00$ & $(2,2)$ & -- & 3386\\
$(c;0,0,0;4)$ & 4 & $(29,12)$ & 7 & 1 & YES & YES & YES & $1.25$ & $(2,2)$ & -- & 3387\\
$(c;0,0,0;4)$ & 4 & $(30,11)$ & 7 & 2 & YES & YES & YES & $1.12$ & $(2,2)$ & -- & 3388\\
$(c;0,0,0;4)$ & 4 & $(31,12)$ & 7 & 1 & YES & YES & YES & $1.25$ & $(2,2)$ & -- & 3389\\
$(c;0,0,0;4)$ & 4 & $(31,13)$ & 7 & 1 & YES & YES & YES & $1.25$ & $(2,2)$ & -- & 3390\\
$(c;0,0,0;4)$ & 4 & $(34,13)$ & 7 & 2 & YES & YES & YES & $1.12$ & $(2,2)$ & -- & 3391\\
$(c;0,1,0;11)$ & 5 & $(9,2)$ & 5 & 1 & YES & YES & YES & $1.12$ & $(2,2)$ & -- & 3392\\
$(c;0,1,0;11)$ & 5 & $(10,3)$ & 5 & 1 & YES & YES & YES & $1.00$ & $(2,2)$ & -- & 3393\\
$(c;0,1,0;11)$ & 5 & $(11,4)$ & 5 & 11 & YES & YES & YES & $0.88$ & $(2,2)$ & -- & 3394\\
$(c;0,1,0;11)$ & 5 & $(12,5)$ & 5 & 1 & YES & YES & YES & $1.00$ & $(2,2)$ & -- & 3395\\
$(c;0,1,0;11)$ & 5 & $(13,5)$ & 5 & 1 & YES & YES & YES & $1.00$ & $(2,2)$ & -- & 3396\\
$(c;0,1,0;11)$ & 5 & $(17,5)$ & 6 & 1 & YES & YES & NO(3) & $0.75$ & $(2,2)$ & -- & 3397\\
$(c;0,1,0;11)$ & 5 & $(17,7)$ & 6 & 1 & YES & YES & YES & $1.00$ & $(2,2)$ & -- & 3398\\
$(c;0,1,0;11)$ & 5 & $(19,7)$ & 6 & 1 & YES & YES & YES & $1.12$ & $(2,2)$ & -- & 3399\\
$(c;0,1,0;11)$ & 5 & $(19,8)$ & 6 & 1 & YES & YES & YES & $1.12$ & $(2,2)$ & -- & 3400\\
$(c;0,1,0;11)$ & 5 & $(21,8)$ & 6 & 1 & YES & YES & YES & $1.00$ & $(2,2)$ & -- & 3401\\
$(c;0,1,0;11)$ & 5 & $(23,7)$ & 7 & 1 & YES & YES & YES & $1.00$ & $(2,2)$ & -- & 3402\\
$(c;0,1,0;11)$ & 5 & $(23,10)$ & 7 & 1 & YES & YES & YES & $1.25$ & $(2,2)$ & -- & 3403\\
$(c;0,1,0;11)$ & 5 & $(24,7)$ & 7 & 1 & YES & YES & YES & $1.00$ & $(2,2)$ & -- & 3404\\
$(c;0,1,0;11)$ & 5 & $(30,11)$ & 7 & 1 & YES & YES & YES & $1.00$ & $(2,2)$ & -- & 3405\\
$(c;0,1,1;5)$ & 6 & $(10,3)$ & 5 & 5 & YES & YES & YES & $1.00$ & $(2,2)$ & -- & 3406\\
$(c;0,1,1;5)$ & 6 & $(12,5)$ & 5 & 1 & YES & YES & YES & $1.00$ & $(2,2)$ & -- & 3407\\
$(c;0,1,1;5)$ & 6 & $(13,5)$ & 5 & 1 & YES & YES & YES & $1.25$ & $(2,2)$ & -- & 3408\\
$(c;0,2,0;7)$ & 6 & $(7,2)$ & 4 & 7 & YES & YES & YES & $1.00$ & $(2,2)$ & -- & 3409\\
$(c;0,2,0;7)$ & 6 & $(8,3)$ & 4 & 1 & YES & YES & YES & $1.11$ & $(2,2)$ & -- & 3410\\
$(c;0,2,0;7)$ & 6 & $(9,2)$ & 5 & 1 & YES & YES & YES & $1.00$ & $(2,2)$ & -- & 3411\\
$(c;0,2,0;7)$ & 6 & $(9,4)$ & 5 & 1 & YES & YES & YES & $1.00$ & $(2,2)$ & -- & 3412\\
$(c;0,2,0;7)$ & 6 & $(10,3)$ & 5 & 1 & YES & YES & YES & $1.00$ & $(2,2)$ & -- & 3413\\
$(c;0,2,0;7)$ & 6 & $(12,5)$ & 5 & 1 & YES & YES & YES & $1.12$ & $(2,2)$ & -- & 3414\\
$(c;0,2,0;7)$ & 6 & $(13,3)$ & 6 & 1 & YES & YES & YES & $1.00$ & $(2,2)$ & -- & 3415\\
$(c;0,2,0;7)$ & 6 & $(13,5)$ & 5 & 1 & YES & YES & YES & $1.12$ & $(2,2)$ & -- & 3416\\
$(c;0,2,0;7)$ & 6 & $(18,5)$ & 6 & 1 & YES & YES & YES & $1.12$ & $(2,2)$ & -- & 3417\\
$(c;0,2,0;7)$ & 6 & $(23,5)$ & 7 & 1 & YES & YES & YES & $1.00$ & $(2,2)$ & -- & 3418\\
$(c;0,2,1;19)$ & 7 & $(5,2)$ & 3 & 1 & YES & YES & YES & $1.00$ & $(2,2)$ & -- & 3419\\
$(c;0,2,1;19)$ & 7 & $(7,2)$ & 4 & 1 & YES & YES & YES & $1.12$ & $(2,2)$ & -- & 3420\\
$(c;0,2,1;19)$ & 7 & $(7,3)$ & 4 & 1 & YES & YES & YES & $1.00$ & $(2,2)$ & -- & 3421\\
$(c;0,2,1;19)$ & 7 & $(8,3)$ & 4 & 1 & YES & YES & YES & $1.00$ & $(2,2)$ & -- & 3422\\
$(c;0,2,1;19)$ & 7 & $(9,4)$ & 5 & 1 & YES & YES & YES & $1.25$ & $(2,2)$ & -- & 3423\\
$(c;0,2,1;19)$ & 7 & $(11,3)$ & 5 & 1 & YES & YES & YES & $1.00$ & $(2,2)$ & -- & 3424\\
$(c;0,2,1;19)$ & 7 & $(13,3)$ & 6 & 1 & YES & YES & YES & $1.00$ & $(2,2)$ & -- & 3425\\
$(c;0,2,2;6)$ & 8 & $(5,2)$ & 3 & 1 & YES & YES & YES & $1.00$ & $(2,2)$ & -- & 3426\\
$(c;0,2,2;6)$ & 8 & $(7,2)$ & 4 & 1 & YES & YES & YES & $0.88$ & $(2,2)$ & -- & 3427\\
$(c;0,3,1;23)$ & 8 & $(5,2)$ & 3 & 1 & YES & YES & YES & $1.12$ & $(2,2)$ & -- & 3428\\
$(c;0,3,1;23)$ & 8 & $(7,2)$ & 4 & 1 & YES & YES & YES & $1.12$ & $(2,2)$ & -- & 3429\\
$(d;0,0,0;5)$ & 5 & $(8,3)$ & 4 & 1 & YES & YES & YES & $0.88$ & $(2,2)$ & -- & 3430\\
$(d;0,0,0;5)$ & 5 & $(9,2)$ & 5 & 1 & YES & YES & YES & $1.12$ & $(2,2)$ & -- & 3431\\
$(d;0,0,0;5)$ & 5 & $(11,4)$ & 5 & 1 & YES & YES & YES & $0.88$ & $(2,2)$ & -- & 3432\\
$(d;0,0,0;5)$ & 5 & $(12,5)$ & 5 & 1 & YES & YES & YES & $1.00$ & $(2,2)$ & -- & 3433\\
$(d;0,0,0;5)$ & 5 & $(13,5)$ & 5 & 1 & YES & YES & YES & $0.88$ & $(2,2)$ & -- & 3434\\
$(d;0,0,0;5)$ & 5 & $(19,8)$ & 6 & 1 & YES & YES & YES & $1.00$ & $(2,2)$ & -- & 3435\\
$(d;0,0,1;14)$ & 6 & $(7,2)$ & 4 & 7 & YES & YES & YES & $1.00$ & $(2,2)$ & -- & 3436\\
$(d;0,0,1;14)$ & 6 & $(8,3)$ & 4 & 2 & YES & YES & YES & $1.00$ & $(2,2)$ & -- & 3437\\
$(d;0,0,1;14)$ & 6 & $(10,3)$ & 5 & 2 & YES & YES & YES & $1.00$ & $(2,2)$ & -- & 3438\\
$(d;0,0,1;14)$ & 6 & $(11,4)$ & 5 & 1 & YES & YES & YES & $1.12$ & $(2,2)$ & -- & 3439\\
$(d;0,0,1;14)$ & 6 & $(12,5)$ & 5 & 2 & YES & YES & YES & $1.00$ & $(2,2)$ & -- & 3440\\
$(d;0,0,1;14)$ & 6 & $(17,5)$ & 6 & 1 & YES & YES & YES & $0.88$ & $(2,2)$ & -- & 3441\\
$(d;0,0,1;14)$ & 6 & $(18,5)$ & 6 & 2 & YES & YES & YES & $1.00$ & $(2,2)$ & -- & 3442\\
$(d;0,0,2;9)$ & 7 & $(5,2)$ & 3 & 1 & YES & YES & YES & $1.00$ & $(2,2)$ & -- & 3443\\
$(d;0,0,2;9)$ & 7 & $(7,2)$ & 4 & 1 & YES & YES & YES & $0.88$ & $(2,2)$ & -- & 3444\\
$(d;0,0,2;9)$ & 7 & $(7,3)$ & 4 & 1 & YES & YES & YES & $1.12$ & $(2,2)$ & -- & 3445\\
$(d;0,0,2;9)$ & 7 & $(8,3)$ & 4 & 1 & YES & YES & YES & $1.12$ & $(2,2)$ & -- & 3446\\
$(d;0,0,2;9)$ & 7 & $(9,4)$ & 5 & 9 & YES & YES & YES & $1.25$ & $(2,2)$ & -- & 3447\\
$(d;0,1,0;6)$ & 6 & $(7,2)$ & 4 & 1 & YES & YES & NO(2) & $1.11$ & $(2,2)$ & -- & 3448\\
$(d;0,1,0;6)$ & 6 & $(8,3)$ & 4 & 2 & YES & YES & YES & $0.88$ & $(2,2)$ & -- & 3449\\
$(d;0,1,0;6)$ & 6 & $(9,2)$ & 5 & 3 & YES & YES & YES & $1.12$ & $(2,2)$ & -- & 3450\\
$(d;0,1,0;6)$ & 6 & $(9,4)$ & 5 & 3 & YES & YES & YES & $1.00$ & $(2,2)$ & -- & 3451\\
$(d;0,1,0;6)$ & 6 & $(12,5)$ & 5 & 6 & YES & YES & YES & $1.12$ & $(2,2)$ & -- & 3452\\
$(d;0,1,0;6)$ & 6 & $(13,3)$ & 6 & 1 & YES & YES & YES & $1.00$ & $(2,2)$ & -- & 3453\\
$(d;0,1,0;6)$ & 6 & $(13,5)$ & 5 & 1 & YES & YES & YES & $1.12$ & $(2,2)$ & -- & 3454\\
$(d;0,1,0;6)$ & 6 & $(22,5)$ & 7 & 2 & YES & YES & YES & $0.88$ & $(2,2)$ & -- & 3455\\
$(d;0,1,0;6)$ & 6 & $(23,5)$ & 7 & 1 & YES & YES & YES & $1.00$ & $(2,2)$ & -- & 3456\\
$(d;0,1,1;17)$ & 7 & $(5,2)$ & 3 & 1 & YES & YES & YES & $1.00$ & $(2,2)$ & -- & 3457\\
$(d;0,1,1;17)$ & 7 & $(7,2)$ & 4 & 1 & YES & YES & YES & $1.00$ & $(2,2)$ & -- & 3458\\
$(d;0,1,1;17)$ & 7 & $(10,3)$ & 5 & 1 & YES & YES & YES & $0.88$ & $(2,2)$ & -- & 3459\\
$(d;0,1,2;11)$ & 8 & $(5,2)$ & 3 & 1 & YES & YES & YES & $1.00$ & $(2,2)$ & -- & 3460\\
$(d;0,1,2;11)$ & 8 & $(7,2)$ & 4 & 1 & YES & YES & YES & $1.00$ & $(2,2)$ & -- & 3461\\
$(d;0,2,0;7)$ & 7 & $(7,3)$ & 4 & 7 & YES & YES & YES & $1.12$ & $(2,2)$ & -- & 3462\\
$(d;0,2,0;7)$ & 7 & $(8,3)$ & 4 & 1 & YES & YES & YES & $1.00$ & $(2,2)$ & -- & 3463\\
$(e;0,0,0;4)$ & 5 & $(8,3)$ & 4 & 4 & YES & YES & YES & $0.88$ & $(2,2)$ & -- & 3464\\
$(e;0,0,0;4)$ & 5 & $(12,5)$ & 5 & 4 & YES & YES & YES & $1.00$ & $(2,2)$ & -- & 3465\\
$(e;0,0,0;4)$ & 5 & $(13,5)$ & 5 & 1 & YES & YES & YES & $1.12$ & $(2,2)$ & -- & 3466\\
$(e;0,1,0;5)$ & 6 & $(7,2)$ & 4 & 1 & YES & YES & YES & $1.00$ & $(2,2)$ & -- & 3467\\
$(e;0,1,0;5)$ & 6 & $(8,3)$ & 4 & 1 & YES & YES & YES & $0.88$ & $(2,2)$ & -- & 3468\\
$(e;0,1,0;5)$ & 6 & $(10,3)$ & 5 & 5 & YES & YES & YES & $1.00$ & $(2,2)$ & -- & 3469\\
$(e;0,1,0;5)$ & 6 & $(22,5)$ & 7 & 1 & YES & YES & YES & $1.00$ & $(2,2)$ & -- & 3470\\
$(e;0,2,0;6)$ & 7 & $(5,2)$ & 3 & 1 & YES & YES & YES & $1.12$ & $(2,2)$ & -- & 3471\\
$(e;0,2,0;6)$ & 7 & $(7,2)$ & 4 & 1 & YES & YES & YES & $1.00$ & $(2,2)$ & -- & 3472\\
$(e;1,0,0;18)$ & 6 & $(3,1)$ & 2 & 3 & YES & YES & YES & $0.88$ & $(2,2)$ & -- & 3473\\
$(e;1,0,0;18)$ & 6 & $(7,3)$ & 4 & 1 & YES & YES & YES & $1.12$ & $(2,2)$ & -- & 3474\\
$(e;1,0,0;18)$ & 6 & $(8,3)$ & 4 & 2 & YES & YES & YES & $1.00$ & $(2,2)$ & -- & 3475\\
$(e;1,0,0;18)$ & 6 & $(10,3)$ & 5 & 2 & YES & YES & YES & $1.00$ & $(2,2)$ & -- & 3476\\
$(e;1,1,0;23)$ & 7 & $(5,2)$ & 3 & 1 & YES & YES & YES & $1.12$ & $(2,2)$ & -- & 3477\\
$(e;1,2,0;28)$ & 8 & $(2,1)$ & 1 & 2 & YES & YES & YES & $1.00$ & $(2,2)$ & -- & 3478\\
$(e;1,2,0;28)$ & 8 & $(3,1)$ & 2 & 1 & YES & YES & YES & $1.00$ & $(2,2)$ & -- & 3479\\
$(e;1,2,0;28)$ & 8 & $(4,1)$ & 3 & 4 & YES & YES & YES & $1.00$ & $(2,2)$ & -- & 3480\\
$(e;2,0,0;24)$ & 7 & $(3,1)$ & 2 & 3 & YES & YES & YES & $1.00$ & $(2,2)$ & -- & 3481\\
$(e;2,0,0;24)$ & 7 & $(4,1)$ & 3 & 4 & YES & YES & YES & $1.12$ & $(2,2)$ & -- & 3482\\
$(e;2,0,0;24)$ & 7 & $(5,2)$ & 3 & 1 & YES & YES & YES & $1.12$ & $(2,2)$ & -- & 3483\\
$(e;2,0,0;24)$ & 7 & $(7,2)$ & 4 & 1 & YES & YES & YES & $0.88$ & $(2,2)$ & -- & 3484\\
$(e;2,0,0;24)$ & 7 & $(7,3)$ & 4 & 1 & YES & YES & YES & $1.25$ & $(2,2)$ & -- & 3485\\
$(e;2,1,0;31)$ & 8 & $(3,1)$ & 2 & 1 & YES & YES & YES & $1.00$ & $(2,2)$ & -- & 3486\\
$(e;2,2,0;38)$ & 9 & $(5,1)$ & 4 & 1 & YES & YES & YES & $1.25$ & $(2,2)$ & -- & 3487\\
$(e;3,0,0;10)$ & 8 & $(2,1)$ & 1 & 2 & YES & YES & YES & $1.12$ & $(2,2)$ & -- & 3488\\
$(e;3,0,0;10)$ & 8 & $(3,1)$ & 2 & 1 & YES & YES & YES & $1.12$ & $(2,2)$ & -- & 3489\\
$(e;3,1,0;13)$ & 9 & $(2,1)$ & 1 & 1 & YES & YES & YES & $1.12$ & $(2,2)$ & -- & 3490\\
$(f;0,0,0;6)$ & 4 & $(21,8)$ & 6 & 3 & YES & YES & YES & $1.12$ & $(2,2)$ & -- & 3491\\
$(f;0,0,0;6)$ & 4 & $(24,7)$ & 7 & 6 & YES & YES & YES & $1.12$ & $(2,2)$ & -- & 3492\\
$(f;0,0,0;6)$ & 4 & $(25,11)$ & 7 & 1 & YES & YES & YES & $1.12$ & $(2,2)$ & -- & 3493\\
$(f;0,0,0;6)$ & 4 & $(26,11)$ & 7 & 2 & YES & YES & YES & $1.00$ & $(2,2)$ & -- & 3494\\
$(f;0,0,0;6)$ & 4 & $(27,8)$ & 7 & 3 & YES & YES & YES & $1.12$ & $(2,2)$ & -- & 3495\\
$(f;0,0,0;6)$ & 4 & $(29,11)$ & 7 & 1 & YES & YES & YES & $1.12$ & $(2,2)$ & -- & 3496\\
$(f;0,0,0;6)$ & 4 & $(29,12)$ & 7 & 1 & YES & YES & YES & $1.12$ & $(2,2)$ & -- & 3497\\
$(f;0,0,0;6)$ & 4 & $(30,11)$ & 7 & 6 & YES & YES & YES & $1.12$ & $(2,2)$ & -- & 3498\\
$(f;0,0,0;6)$ & 4 & $(31,7)$ & 8 & 1 & YES & YES & YES & $1.12$ & $(2,2)$ & -- & 3499\\
$(f;0,0,0;6)$ & 4 & $(31,12)$ & 7 & 1 & YES & YES & YES & $1.25$ & $(2,2)$ & -- & 3500\\
$(f;0,0,0;6)$ & 4 & $(31,13)$ & 7 & 1 & YES & YES & YES & $0.88$ & $(2,2)$ & -- & 3501\\
$(f;0,0,0;6)$ & 4 & $(33,10)$ & 8 & 3 & YES & YES & YES & $1.12$ & $(2,2)$ & -- & 3502\\
$(f;0,0,0;6)$ & 4 & $(34,9)$ & 8 & 2 & YES & YES & YES & $1.12$ & $(2,2)$ & -- & 3503\\
$(f;0,0,0;6)$ & 4 & $(34,13)$ & 7 & 2 & YES & YES & YES & $1.00$ & $(2,2)$ & -- & 3504\\
$(f;0,0,0;6)$ & 4 & $(36,11)$ & 8 & 6 & YES & YES & YES & $1.12$ & $(2,2)$ & -- & 3505\\
$(f;0,0,0;6)$ & 4 & $(37,10)$ & 8 & 1 & YES & YES & YES & $1.12$ & $(2,2)$ & -- & 3506\\
$(f;0,0,0;6)$ & 4 & $(39,14)$ & 8 & 3 & YES & YES & YES & $1.12$ & $(2,2)$ & -- & 3507\\
$(f;0,0,0;6)$ & 4 & $(39,17)$ & 8 & 3 & YES & YES & YES & $1.12$ & $(2,2)$ & -- & 3508\\
$(f;0,0,0;6)$ & 4 & $(40,9)$ & 9 & 2 & YES & YES & YES & $1.12$ & $(2,2)$ & -- & 3509\\
$(f;0,0,0;6)$ & 4 & $(40,11)$ & 8 & 2 & YES & YES & YES & $1.00$ & $(2,2)$ & -- & 3510\\
$(f;0,0,0;6)$ & 4 & $(41,11)$ & 8 & 1 & YES & YES & YES & $1.12$ & $(2,2)$ & -- & 3511\\
$(f;0,0,0;6)$ & 4 & $(41,15)$ & 8 & 1 & YES & YES & YES & $1.12$ & $(2,2)$ & -- & 3512\\
$(f;0,0,0;6)$ & 4 & $(41,17)$ & 8 & 1 & YES & YES & YES & $1.12$ & $(2,2)$ & -- & 3513\\
$(f;0,0,0;6)$ & 4 & $(43,10)$ & 9 & 1 & YES & YES & YES & $1.12$ & $(2,2)$ & -- & 3514\\
$(f;0,0,0;6)$ & 4 & $(46,17)$ & 8 & 2 & YES & YES & YES & $1.12$ & $(2,2)$ & -- & 3515\\
$(f;0,0,0;6)$ & 4 & $(47,13)$ & 8 & 1 & YES & YES & YES & $1.00$ & $(2,2)$ & -- & 3516\\
$(f;0,1,0;7)$ & 5 & $(17,5)$ & 6 & 1 & YES & YES & YES & $1.00$ & $(2,2)$ & -- & 3517\\
$(f;0,1,0;7)$ & 5 & $(17,7)$ & 6 & 1 & YES & YES & YES & $1.12$ & $(2,2)$ & -- & 3518\\
$(f;0,1,0;7)$ & 5 & $(18,5)$ & 6 & 1 & YES & YES & YES & $1.00$ & $(2,2)$ & -- & 3519\\
$(f;0,1,0;7)$ & 5 & $(19,7)$ & 6 & 1 & YES & YES & YES & $1.12$ & $(2,2)$ & -- & 3520\\
$(f;0,1,0;7)$ & 5 & $(22,5)$ & 7 & 1 & YES & YES & YES & $1.12$ & $(2,2)$ & -- & 3521\\
$(g;0,0,0;19)$ & 6 & $(3,1)$ & 2 & 1 & YES & YES & YES & $0.88$ & $(2,2)$ & -- & 3522\\
$(g;0,0,0;19)$ & 6 & $(7,3)$ & 4 & 1 & YES & YES & YES & $1.00$ & $(2,2)$ & -- & 3523\\
$(g;0,0,1;26)$ & 7 & $(5,2)$ & 3 & 1 & YES & YES & YES & $0.88$ & $(2,2)$ & -- & 3524\\
$(g;0,0,2;11)$ & 8 & $(2,1)$ & 1 & 1 & YES & YES & YES & $1.12$ & $(2,2)$ & -- & 3525\\
$(g;0,0,2;11)$ & 8 & $(3,1)$ & 2 & 1 & YES & YES & YES & $1.12$ & $(2,2)$ & -- & 3526\\
$(g;0,0,2;11)$ & 8 & $(4,1)$ & 3 & 1 & YES & YES & YES & $1.00$ & $(2,2)$ & -- & 3527\\
$(g;0,0,2;11)$ & 8 & $(5,1)$ & 4 & 1 & YES & YES & YES & $1.00$ & $(2,2)$ & -- & 3528\\
$(g;0,1,0;24)$ & 7 & $(2,1)$ & 1 & 2 & YES & YES & YES & $0.88$ & $(2,2)$ & -- & 3529\\
$(g;0,1,0;24)$ & 7 & $(3,1)$ & 2 & 3 & YES & YES & YES & $0.88$ & $(2,2)$ & -- & 3530\\
$(g;0,1,0;24)$ & 7 & $(4,1)$ & 3 & 4 & YES & YES & YES & $0.88$ & $(2,2)$ & -- & 3531\\
$(g;0,1,0;24)$ & 7 & $(5,2)$ & 3 & 1 & YES & YES & YES & $0.88$ & $(2,2)$ & -- & 3532\\
$(g;0,2,0;29)$ & 8 & $(2,1)$ & 1 & 1 & YES & YES & YES & $1.00$ & $(2,2)$ & -- & 3533\\
$(g;0,2,0;29)$ & 8 & $(3,1)$ & 2 & 1 & YES & YES & YES & $1.00$ & $(2,2)$ & -- & 3534\\
$(g;0,2,0;29)$ & 8 & $(4,1)$ & 3 & 1 & YES & YES & YES & $1.00$ & $(2,2)$ & -- & 3535\\
$(g;0,2,0;29)$ & 8 & $(5,1)$ & 4 & 1 & YES & YES & YES & $1.00$ & $(2,2)$ & -- & 3536\\
$(g;1,0,0;7)$ & 7 & $(2,1)$ & 1 & 1 & YES & YES & YES & $1.00$ & $(2,2)$ & -- & 3537\\
$(g;1,0,1;38)$ & 8 & $(3,1)$ & 2 & 1 & YES & YES & YES & $1.12$ & $(2,2)$ & -- & 3538\\
$(g;2,0,0;37)$ & 8 & $(2,1)$ & 1 & 1 & YES & YES & YES & $1.12$ & $(2,2)$ & -- & 3539\\
$(g;2,1,0;48)$ & 9 & $(4,1)$ & 3 & 4 & YES & YES & YES & $1.00$ & $(2,2)$ & -- & 3540\\
$(h;0,0,0;6)$ & 5 & $(12,5)$ & 5 & 6 & YES & YES & YES & $0.88$ & $(2,2)$ & -- & 3541\\
$(h;0,1,0;8)$ & 6 & $(7,3)$ & 4 & 1 & YES & YES & YES & $1.00$ & $(2,2)$ & -- & 3542\\
$(h;0,2,0;10)$ & 7 & $(3,1)$ & 2 & 1 & YES & YES & YES & $0.88$ & $(2,2)$ & -- & 3543\\
$(h;0,2,0;10)$ & 7 & $(5,2)$ & 3 & 5 & YES & YES & YES & $0.88$ & $(2,2)$ & -- & 3544\\
$(i;0,0,0;9)$ & 5 & $(10,3)$ & 5 & 1 & YES & YES & YES & $1.12$ & $(2,2)$ & -- & 3545\\
$(i;0,0,0;9)$ & 5 & $(12,5)$ & 5 & 3 & YES & YES & YES & $1.00$ & $(2,2)$ & -- & 3546\\
$(i;0,0,0;9)$ & 5 & $(13,5)$ & 5 & 1 & YES & YES & YES & $1.00$ & $(2,2)$ & -- & 3547\\
$(i;0,0,0;9)$ & 5 & $(14,5)$ & 6 & 1 & YES & YES & YES & $1.12$ & $(2,2)$ & -- & 3548\\
$(i;0,0,0;9)$ & 5 & $(16,7)$ & 6 & 1 & YES & YES & YES & $1.38$ & $(2,2)$ & -- & 3549\\
$(i;0,0,0;9)$ & 5 & $(17,5)$ & 6 & 1 & YES & YES & YES & $1.00$ & $(2,2)$ & -- & 3550\\
$(i;0,0,0;9)$ & 5 & $(18,5)$ & 6 & 9 & YES & YES & YES & $1.00$ & $(2,2)$ & -- & 3551\\
$(i;0,0,0;9)$ & 5 & $(22,5)$ & 7 & 1 & YES & YES & YES & $1.12$ & $(2,2)$ & -- & 3552\\
$(i;0,1,0;12)$ & 6 & $(7,2)$ & 4 & 1 & YES & YES & YES & $1.00$ & $(2,2)$ & -- & 3553\\
$(i;0,1,0;12)$ & 6 & $(7,3)$ & 4 & 1 & YES & YES & YES & $1.12$ & $(2,2)$ & -- & 3554\\
$(i;0,1,0;12)$ & 6 & $(8,3)$ & 4 & 4 & YES & YES & YES & $1.00$ & $(2,2)$ & -- & 3555\\
$(i;0,1,0;12)$ & 6 & $(9,4)$ & 5 & 3 & YES & YES & YES & $1.12$ & $(2,2)$ & -- & 3556\\
$(i;0,1,0;12)$ & 6 & $(10,3)$ & 5 & 2 & YES & YES & YES & $1.12$ & $(2,2)$ & -- & 3557\\
$(i;0,1,0;12)$ & 6 & $(11,3)$ & 5 & 1 & YES & YES & YES & $1.12$ & $(2,2)$ & -- & 3558\\
$(i;0,1,0;12)$ & 6 & $(13,3)$ & 6 & 1 & YES & YES & YES & $1.00$ & $(2,2)$ & -- & 3559\\
$(i;0,2,0;15)$ & 7 & $(7,2)$ & 4 & 1 & YES & YES & YES & $1.00$ & $(2,2)$ & -- & 3560\\
$(j;0,0,0;8)$ & 5 & $(10,3)$ & 5 & 2 & YES & YES & YES & $1.12$ & $(2,2)$ & -- & 3561\\
$(j;0,0,0;8)$ & 5 & $(11,3)$ & 5 & 1 & YES & YES & NO(2) & $1.22$ & $(2,2)$ & -- & 3562\\
$(j;0,0,0;8)$ & 5 & $(13,5)$ & 5 & 1 & YES & YES & YES & $1.12$ & $(2,2)$ & -- & 3563\\
$(j;0,0,0;8)$ & 5 & $(14,5)$ & 6 & 2 & YES & YES & YES & $1.00$ & $(2,2)$ & -- & 3564\\
$(j;0,0,0;8)$ & 5 & $(16,7)$ & 6 & 8 & YES & YES & YES & $1.00$ & $(2,2)$ & -- & 3565\\
$(j;0,0,0;8)$ & 5 & $(17,5)$ & 6 & 1 & YES & YES & YES & $1.12$ & $(2,2)$ & -- & 3566\\
$(j;0,0,0;8)$ & 5 & $(17,7)$ & 6 & 1 & YES & YES & YES & $1.12$ & $(2,2)$ & -- & 3567\\
$(j;0,0,0;8)$ & 5 & $(18,7)$ & 6 & 2 & YES & YES & YES & $1.12$ & $(2,2)$ & -- & 3568\\
$(j;0,0,0;8)$ & 5 & $(19,7)$ & 6 & 1 & YES & YES & YES & $1.00$ & $(2,2)$ & -- & 3569\\
$(j;0,0,0;8)$ & 5 & $(19,8)$ & 6 & 1 & YES & YES & YES & $1.12$ & $(2,2)$ & -- & 3570\\
$(j;0,0,0;8)$ & 5 & $(21,8)$ & 6 & 1 & YES & YES & YES & $1.12$ & $(2,2)$ & -- & 3571\\
$(j;0,0,0;8)$ & 5 & $(23,7)$ & 7 & 1 & YES & YES & YES & $1.12$ & $(2,2)$ & -- & 3572\\
$(j;0,0,0;8)$ & 5 & $(25,11)$ & 7 & 1 & YES & YES & YES & $1.12$ & $(2,2)$ & -- & 3573\\
$(j;0,0,0;8)$ & 5 & $(26,7)$ & 7 & 2 & YES & YES & YES & $1.00$ & $(2,2)$ & -- & 3574\\
$(j;0,0,0;8)$ & 5 & $(29,8)$ & 7 & 1 & YES & YES & YES & $1.12$ & $(2,2)$ & -- & 3575\\
$(j;0,0,0;8)$ & 5 & $(30,7)$ & 8 & 2 & YES & YES & YES & $1.12$ & $(2,2)$ & -- & 3576\\
$(j;0,0,0;8)$ & 5 & $(31,7)$ & 8 & 1 & YES & YES & YES & $1.12$ & $(2,2)$ & -- & 3577\\
$(j;0,1,0;10)$ & 6 & $(12,5)$ & 5 & 2 & YES & YES & YES & $1.12$ & $(2,2)$ & -- & 3578\\
$(j;0,1,0;10)$ & 6 & $(13,3)$ & 6 & 1 & YES & YES & YES & $1.12$ & $(2,2)$ & -- & 3579\\
$(j;0,1,0;10)$ & 6 & $(13,4)$ & 6 & 1 & YES & YES & YES & $1.12$ & $(2,2)$ & -- & 3580\\
$(j;0,1,0;10)$ & 6 & $(13,5)$ & 5 & 1 & YES & YES & YES & $1.00$ & $(2,2)$ & -- & 3581\\
$(j;0,1,0;10)$ & 6 & $(14,5)$ & 6 & 2 & YES & YES & YES & $1.12$ & $(2,2)$ & -- & 3582\\
$(j;0,1,0;10)$ & 6 & $(16,7)$ & 6 & 2 & YES & YES & YES & $1.12$ & $(2,2)$ & -- & 3583\\
$(j;0,1,0;10)$ & 6 & $(17,7)$ & 6 & 1 & YES & YES & YES & $1.12$ & $(2,2)$ & -- & 3584\\
$(j;0,1,0;10)$ & 6 & $(19,7)$ & 6 & 1 & YES & YES & YES & $1.25$ & $(2,2)$ & -- & 3585\\
$(j;0,1,0;10)$ & 6 & $(19,8)$ & 6 & 1 & YES & YES & YES & $1.12$ & $(2,2)$ & -- & 3586
\end{longtable}
\subsection{2 chains, $K^2 = 3$}
\begin{longtable}{|c|c|c|c|c|c|c|c|c|c|c|c|}
\hline
\multicolumn{12}{|c|}{2 chains, $K^2 = 3$}\\
\hline
$(n,a)$ & Len & $(n,a)$ & Len & GCD & Nef & $\mathbb Q$-ef & Obs 0 & $\overline c_1^2 / \overline c_2$ & $(P,K)$ & WH & Index\\
\hline
\endfirsthead

\hline
$(n,a)$ & Len & $(n,a)$ & Len & GCD & Nef & $\mathbb Q$-ef & Obs 0 & $\overline c_1^2 / \overline c_2$ & $(P,K)$ & WH & Index\\
\hline
\endhead
\hline
\endfoot

$(19,8)$ & 6 & $(14,5)$ & 6 & 1 & YES & YES & NO(2) & $1.50$ & $(2,3)$ & -- & 3587\\
$(21,8)$ & 6 & $(17,5)$ & 6 & 1 & YES & YES & YES & $1.57$ & $(2,3)$ & -- & 3588\\
$(21,8)$ & 6 & $(18,5)$ & 6 & 3 & YES & YES & YES & $1.57$ & $(2,3)$ & -- & 3589\\
$(22,9)$ & 7 & $(21,8)$ & 6 & 1 & YES & YES & YES & $1.71$ & $(2,3)$ & -- & 3590\\
$(23,5)$ & 7 & $(17,5)$ & 6 & 1 & YES & YES & YES & $1.43$ & $(2,3)$ & NO & 3591\\
$(23,5)$ & 7 & $(17,5)$ & 6 & 1 & YES & YES & YES & $1.43$ & $(2,3)$ & -- & 3592\\
$(23,7)$ & 7 & $(19,7)$ & 6 & 1 & YES & YES & NO(2) & $1.67$ & $(2,3)$ & -- & 3593\\
$(23,9)$ & 7 & $(23,7)$ & 7 & 23 & YES & YES & YES & $1.71$ & $(2,3)$ & NO & 3594\\
$(24,7)$ & 7 & $(19,8)$ & 6 & 1 & YES & YES & YES & $1.57$ & $(2,3)$ & -- & 3595\\
$(24,7)$ & 7 & $(22,9)$ & 7 & 2 & YES & YES & YES & $1.57$ & $(2,3)$ & NO & 3596\\
$(24,7)$ & 7 & $(22,9)$ & 7 & 2 & YES & YES & YES & $1.57$ & $(2,3)$ & -- & 3597\\
$(24,7)$ & 7 & $(23,7)$ & 7 & 1 & YES & YES & YES & $1.57$ & $(2,3)$ & -- & 3598\\
$(24,7)$ & 7 & $(23,7)$ & 7 & 1 & YES & YES & YES & $1.57$ & $(2,3)$ & NO & 3599\\
$(24,7)$ & 7 & $(23,9)$ & 7 & 1 & YES & YES & YES & $1.57$ & $(2,3)$ & NO & 3600\\
$(24,7)$ & 7 & $(23,9)$ & 7 & 1 & YES & YES & YES & $1.57$ & $(2,3)$ & -- & 3601\\
$(25,7)$ & 7 & $(17,7)$ & 6 & 1 & YES & YES & YES & $1.57$ & $(2,3)$ & -- & 3602\\
$(25,7)$ & 7 & $(17,7)$ & 6 & 1 & YES & YES & YES & $1.57$ & $(2,3)$ & NO & 3603\\
$(25,7)$ & 7 & $(19,8)$ & 6 & 1 & YES & YES & YES & $1.57$ & $(2,3)$ & -- & 3604\\
$(25,7)$ & 7 & $(19,8)$ & 6 & 1 & YES & YES & YES & $1.57$ & $(2,3)$ & NO & 3605\\
$(25,9)$ & 7 & $(19,8)$ & 6 & 1 & YES & YES & NO(2) & $1.50$ & $(2,3)$ & -- & 3606\\
$(25,7)$ & 7 & $(22,5)$ & 7 & 1 & YES & YES & YES & $1.57$ & $(2,3)$ & NO & 3607\\
$(25,7)$ & 7 & $(22,5)$ & 7 & 1 & YES & YES & YES & $1.57$ & $(2,3)$ & -- & 3608\\
$(25,9)$ & 7 & $(25,7)$ & 7 & 25 & YES & YES & YES & $1.57$ & $(2,3)$ & -- & 3609\\
$(26,11)$ & 7 & $(17,7)$ & 6 & 1 & YES & YES & YES & $1.57$ & $(2,3)$ & -- & 3610\\
$(26,7)$ & 7 & $(18,7)$ & 6 & 2 & YES & YES & YES & $1.57$ & $(2,3)$ & -- & 3611\\
$(26,7)$ & 7 & $(19,8)$ & 6 & 1 & YES & YES & YES & $1.57$ & $(2,3)$ & NO & 3612\\
$(26,7)$ & 7 & $(19,8)$ & 6 & 1 & YES & YES & YES & $1.57$ & $(2,3)$ & -- & 3613\\
$(26,11)$ & 7 & $(19,8)$ & 6 & 1 & YES & YES & YES & $1.43$ & $(2,3)$ & -- & 3614\\
$(26,7)$ & 7 & $(21,8)$ & 6 & 1 & YES & YES & YES & $1.57$ & $(2,3)$ & NO & 3615\\
$(26,7)$ & 7 & $(21,8)$ & 6 & 1 & YES & YES & YES & $1.57$ & $(2,3)$ & -- & 3616\\
$(27,8)$ & 7 & $(13,5)$ & 5 & 1 & YES & YES & YES & $1.43$ & $(2,3)$ & -- & 3617\\
$(27,8)$ & 7 & $(19,8)$ & 6 & 1 & YES & YES & YES & $1.43$ & $(2,3)$ & -- & 3618\\
$(27,10)$ & 7 & $(19,7)$ & 6 & 1 & YES & YES & YES & $1.57$ & $(2,3)$ & -- & 3619\\
$(27,8)$ & 7 & $(21,8)$ & 6 & 3 & YES & YES & YES & $1.43$ & $(2,3)$ & NO & 3620\\
$(27,8)$ & 7 & $(21,8)$ & 6 & 3 & YES & YES & YES & $1.43$ & $(2,3)$ & -- & 3621\\
$(27,10)$ & 7 & $(21,8)$ & 6 & 3 & YES & YES & YES & $1.57$ & $(2,3)$ & -- & 3622\\
$(27,8)$ & 7 & $(22,9)$ & 7 & 1 & YES & YES & NO(2) & $1.50$ & $(2,3)$ & -- & 3623\\
$(27,10)$ & 7 & $(25,9)$ & 7 & 1 & YES & YES & NO(2) & $1.75$ & $(2,3)$ & NO & 3624\\
$(27,10)$ & 7 & $(25,9)$ & 7 & 1 & YES & YES & NO(2) & $1.75$ & $(2,3)$ & -- & 3625\\
$(27,8)$ & 7 & $(26,11)$ & 7 & 1 & YES & YES & YES & $1.57$ & $(2,3)$ & -- & 3626\\
$(28,11)$ & 8 & $(17,5)$ & 6 & 1 & YES & YES & YES & $1.71$ & $(2,3)$ & NO & 3627\\
$(28,11)$ & 8 & $(17,5)$ & 6 & 1 & YES & YES & YES & $1.71$ & $(2,3)$ & -- & 3628\\
$(28,11)$ & 8 & $(19,8)$ & 6 & 1 & YES & YES & YES & $1.43$ & $(2,3)$ & -- & 3629\\
$(28,11)$ & 8 & $(21,8)$ & 6 & 7 & YES & YES & YES & $1.43$ & $(2,3)$ & -- & 3630\\
$(28,11)$ & 8 & $(23,7)$ & 7 & 1 & YES & YES & YES & $1.71$ & $(2,3)$ & -- & 3631\\
$(28,11)$ & 8 & $(25,9)$ & 7 & 1 & YES & YES & YES & $1.57$ & $(2,3)$ & -- & 3632\\
$(29,8)$ & 7 & $(13,5)$ & 5 & 1 & YES & YES & YES & $1.57$ & $(2,3)$ & -- & 3633\\
$(29,8)$ & 7 & $(13,5)$ & 5 & 1 & YES & YES & YES & $1.43$ & $(2,3)$ & NO & 3634\\
$(29,12)$ & 7 & $(15,4)$ & 6 & 1 & YES & YES & YES & $1.57$ & $(2,3)$ & NO & 3635\\
$(29,12)$ & 7 & $(15,4)$ & 6 & 1 & YES & YES & YES & $1.57$ & $(2,3)$ & -- & 3636\\
$(29,8)$ & 7 & $(16,5)$ & 7 & 1 & YES & YES & YES & $1.57$ & $(2,3)$ & NO & 3637\\
$(29,8)$ & 7 & $(16,5)$ & 7 & 1 & YES & YES & YES & $1.57$ & $(2,3)$ & -- & 3638\\
$(29,8)$ & 7 & $(17,7)$ & 6 & 1 & YES & YES & YES & $1.57$ & $(2,3)$ & -- & 3639\\
$(29,12)$ & 7 & $(17,5)$ & 6 & 1 & YES & YES & YES & $1.57$ & $(2,3)$ & -- & 3640\\
$(29,11)$ & 7 & $(18,5)$ & 6 & 1 & YES & YES & YES & $1.43$ & $(2,3)$ & -- & 3641\\
$(29,11)$ & 7 & $(18,5)$ & 6 & 1 & YES & YES & YES & $1.57$ & $(2,3)$ & NO & 3642\\
$(29,11)$ & 7 & $(18,7)$ & 6 & 1 & YES & YES & YES & $1.57$ & $(2,3)$ & -- & 3643\\
$(29,12)$ & 7 & $(18,5)$ & 6 & 1 & YES & YES & YES & $1.57$ & $(2,3)$ & -- & 3644\\
$(29,12)$ & 7 & $(18,5)$ & 6 & 1 & YES & YES & YES & $1.57$ & $(2,3)$ & NO & 3645\\
$(29,8)$ & 7 & $(19,7)$ & 6 & 1 & YES & YES & YES & $1.57$ & $(2,3)$ & -- & 3646\\
$(29,8)$ & 7 & $(19,8)$ & 6 & 1 & YES & YES & YES & $1.43$ & $(2,3)$ & -- & 3647\\
$(29,8)$ & 7 & $(19,8)$ & 6 & 1 & YES & YES & YES & $1.43$ & $(2,3)$ & NO & 3648\\
$(29,11)$ & 7 & $(19,8)$ & 6 & 1 & YES & YES & YES & $1.57$ & $(2,3)$ & -- & 3649\\
$(29,12)$ & 7 & $(19,8)$ & 6 & 1 & YES & YES & YES & $1.43$ & $(2,3)$ & -- & 3650\\
$(29,8)$ & 7 & $(21,8)$ & 6 & 1 & YES & YES & YES & $1.43$ & $(2,3)$ & NO & 3651\\
$(29,8)$ & 7 & $(21,8)$ & 6 & 1 & YES & YES & YES & $1.43$ & $(2,3)$ & -- & 3652\\
$(29,11)$ & 7 & $(21,8)$ & 6 & 1 & YES & YES & YES & $1.71$ & $(2,3)$ & -- & 3653\\
$(29,11)$ & 7 & $(21,8)$ & 6 & 1 & YES & YES & YES & $1.43$ & $(2,3)$ & NO & 3654\\
$(29,8)$ & 7 & $(22,9)$ & 7 & 1 & YES & YES & YES & $1.57$ & $(2,3)$ & NO & 3655\\
$(29,8)$ & 7 & $(22,9)$ & 7 & 1 & YES & YES & YES & $1.57$ & $(2,3)$ & -- & 3656\\
$(29,11)$ & 7 & $(22,5)$ & 7 & 1 & YES & YES & YES & $1.57$ & $(2,3)$ & NO & 3657\\
$(29,11)$ & 7 & $(22,5)$ & 7 & 1 & YES & YES & YES & $1.57$ & $(2,3)$ & -- & 3658\\
$(29,12)$ & 7 & $(22,5)$ & 7 & 1 & YES & YES & YES & $1.57$ & $(2,3)$ & NO & 3659\\
$(29,11)$ & 7 & $(23,10)$ & 7 & 1 & YES & YES & YES & $1.71$ & $(2,3)$ & -- & 3660\\
$(29,12)$ & 7 & $(23,9)$ & 7 & 1 & YES & YES & YES & $1.57$ & $(2,3)$ & -- & 3661\\
$(29,11)$ & 7 & $(25,7)$ & 7 & 1 & YES & YES & YES & $1.57$ & $(2,3)$ & NO & 3662\\
$(29,11)$ & 7 & $(25,7)$ & 7 & 1 & YES & YES & YES & $1.57$ & $(2,3)$ & -- & 3663\\
$(29,12)$ & 7 & $(26,7)$ & 7 & 1 & YES & YES & NO(2) & $1.50$ & $(2,3)$ & -- & 3664\\
$(29,12)$ & 7 & $(26,11)$ & 7 & 1 & YES & YES & YES & $1.43$ & $(2,3)$ & -- & 3665\\
$(29,12)$ & 7 & $(27,8)$ & 7 & 1 & YES & YES & YES & $1.57$ & $(2,3)$ & -- & 3666\\
$(29,11)$ & 7 & $(29,8)$ & 7 & 29 & YES & YES & YES & $1.43$ & $(2,3)$ & -- & 3667\\
$(29,11)$ & 7 & $(29,11)$ & 7 & 29 & YES & YES & YES & $1.43$ & $(2,3)$ & -- & 3668\\
$(29,11)$ & 7 & $(29,11)$ & 7 & 29 & YES & YES & YES & $1.57$ & $(2,3)$ & NO & 3669\\
$(29,12)$ & 7 & $(29,8)$ & 7 & 29 & YES & YES & YES & $1.43$ & $(2,3)$ & -- & 3670\\
$(29,12)$ & 7 & $(29,11)$ & 7 & 29 & YES & YES & YES & $1.43$ & $(2,3)$ & -- & 3671\\
$(29,12)$ & 7 & $(29,12)$ & 7 & 29 & YES & YES & YES & $1.57$ & $(2,3)$ & -- & 3672\\
$(29,12)$ & 7 & $(29,12)$ & 7 & 29 & YES & YES & YES & $1.57$ & $(2,3)$ & NO & 3673\\
$(30,11)$ & 7 & $(13,5)$ & 5 & 1 & YES & YES & NO(2) & $1.38$ & $(2,3)$ & -- & 3674\\
$(30,11)$ & 7 & $(13,5)$ & 5 & 1 & YES & YES & YES & $1.43$ & $(2,3)$ & NO & 3675\\
$(30,11)$ & 7 & $(17,7)$ & 6 & 1 & YES & YES & NO(2) & $1.62$ & $(2,3)$ & -- & 3676\\
$(30,11)$ & 7 & $(18,5)$ & 6 & 6 & YES & YES & YES & $1.43$ & $(2,3)$ & NO & 3677\\
$(30,11)$ & 7 & $(18,5)$ & 6 & 6 & YES & YES & YES & $1.43$ & $(2,3)$ & -- & 3678\\
$(30,11)$ & 7 & $(19,8)$ & 6 & 1 & YES & YES & NO(2) & $1.62$ & $(2,3)$ & -- & 3679\\
$(30,13)$ & 8 & $(19,8)$ & 6 & 1 & YES & YES & NO(2) & $1.88$ & $(2,3)$ & -- & 3680\\
$(30,11)$ & 7 & $(21,8)$ & 6 & 3 & YES & YES & NO(2) & $1.38$ & $(2,3)$ & -- & 3681\\
$(30,7)$ & 8 & $(22,9)$ & 7 & 2 & YES & YES & NO(2) & $1.62$ & $(2,3)$ & NO & 3682\\
$(30,7)$ & 8 & $(22,9)$ & 7 & 2 & YES & YES & NO(2) & $1.62$ & $(2,3)$ & -- & 3683\\
$(30,11)$ & 7 & $(22,9)$ & 7 & 2 & YES & YES & NO(2) & $1.50$ & $(2,3)$ & -- & 3684\\
$(30,11)$ & 7 & $(23,9)$ & 7 & 1 & YES & YES & YES & $1.43$ & $(2,3)$ & -- & 3685\\
$(30,11)$ & 7 & $(25,11)$ & 7 & 5 & YES & YES & YES & $1.43$ & $(2,3)$ & -- & 3686\\
$(30,11)$ & 7 & $(26,11)$ & 7 & 2 & YES & YES & YES & $1.43$ & $(2,3)$ & -- & 3687\\
$(30,11)$ & 7 & $(26,11)$ & 7 & 2 & YES & YES & YES & $1.57$ & $(2,3)$ & NO & 3688\\
$(30,11)$ & 7 & $(27,10)$ & 7 & 3 & YES & YES & NO(2) & $1.62$ & $(2,3)$ & -- & 3689\\
$(30,11)$ & 7 & $(29,11)$ & 7 & 1 & YES & YES & YES & $1.57$ & $(2,3)$ & -- & 3690\\
$(30,11)$ & 7 & $(29,11)$ & 7 & 1 & YES & YES & YES & $1.57$ & $(2,3)$ & NO & 3691\\
$(30,11)$ & 7 & $(29,12)$ & 7 & 1 & YES & YES & YES & $1.71$ & $(2,3)$ & -- & 3692\\
$(30,13)$ & 8 & $(29,12)$ & 7 & 1 & YES & YES & YES & $1.57$ & $(2,3)$ & -- & 3693\\
$(30,13)$ & 8 & $(30,7)$ & 8 & 30 & YES & YES & NO(2) & $1.75$ & $(2,3)$ & NO & 3694\\
$(30,13)$ & 8 & $(30,11)$ & 7 & 30 & YES & YES & YES & $1.57$ & $(2,3)$ & -- & 3695\\
$(31,13)$ & 7 & $(10,3)$ & 5 & 1 & YES & YES & NO(2) & $1.62$ & $(2,3)$ & -- & 3696\\
$(31,13)$ & 7 & $(11,4)$ & 5 & 1 & YES & YES & NO(2) & $1.50$ & $(2,3)$ & -- & 3697\\
$(31,13)$ & 7 & $(14,5)$ & 6 & 1 & YES & YES & NO(2) & $1.50$ & $(2,3)$ & -- & 3698\\
$(31,12)$ & 7 & $(15,4)$ & 6 & 1 & YES & YES & YES & $1.71$ & $(2,3)$ & -- & 3699\\
$(31,9)$ & 8 & $(17,5)$ & 6 & 1 & YES & YES & YES & $1.57$ & $(2,3)$ & -- & 3700\\
$(31,13)$ & 7 & $(17,7)$ & 6 & 1 & YES & YES & YES & $1.43$ & $(2,3)$ & -- & 3701\\
$(31,9)$ & 8 & $(18,5)$ & 6 & 1 & YES & YES & YES & $1.57$ & $(2,3)$ & -- & 3702\\
$(31,13)$ & 7 & $(18,5)$ & 6 & 1 & YES & YES & YES & $1.43$ & $(2,3)$ & NO & 3703\\
$(31,13)$ & 7 & $(18,5)$ & 6 & 1 & YES & YES & YES & $1.43$ & $(2,3)$ & -- & 3704\\
$(31,13)$ & 7 & $(18,7)$ & 6 & 1 & YES & YES & YES & $1.43$ & $(2,3)$ & -- & 3705\\
$(31,14)$ & 8 & $(18,7)$ & 6 & 1 & YES & YES & YES & $1.57$ & $(2,3)$ & -- & 3706\\
$(31,9)$ & 8 & $(19,8)$ & 6 & 1 & YES & YES & YES & $1.43$ & $(2,3)$ & -- & 3707\\
$(31,13)$ & 7 & $(19,7)$ & 6 & 1 & YES & YES & NO(2) & $1.50$ & $(2,3)$ & -- & 3708\\
$(31,13)$ & 7 & $(19,8)$ & 6 & 1 & YES & YES & YES & $1.43$ & $(2,3)$ & -- & 3709\\
$(31,9)$ & 8 & $(20,9)$ & 7 & 1 & YES & YES & YES & $1.57$ & $(2,3)$ & -- & 3710\\
$(31,9)$ & 8 & $(22,5)$ & 7 & 1 & YES & YES & NO(2) & $1.62$ & $(2,3)$ & -- & 3711\\
$(31,12)$ & 7 & $(22,9)$ & 7 & 1 & YES & YES & YES & $1.57$ & $(2,3)$ & -- & 3712\\
$(31,12)$ & 7 & $(22,9)$ & 7 & 1 & YES & YES & YES & $1.71$ & $(2,3)$ & NO & 3713\\
$(31,12)$ & 7 & $(23,7)$ & 7 & 1 & YES & YES & YES & $1.57$ & $(2,3)$ & NO & 3714\\
$(31,12)$ & 7 & $(23,7)$ & 7 & 1 & YES & YES & YES & $1.57$ & $(2,3)$ & -- & 3715\\
$(31,14)$ & 8 & $(24,7)$ & 7 & 1 & YES & YES & YES & $1.71$ & $(2,3)$ & -- & 3716\\
$(31,12)$ & 7 & $(25,7)$ & 7 & 1 & YES & YES & YES & $1.57$ & $(2,3)$ & NO & 3717\\
$(31,12)$ & 7 & $(25,7)$ & 7 & 1 & YES & YES & YES & $1.57$ & $(2,3)$ & -- & 3718\\
$(31,9)$ & 8 & $(26,11)$ & 7 & 1 & YES & YES & YES & $1.43$ & $(2,3)$ & -- & 3719\\
$(31,12)$ & 7 & $(27,8)$ & 7 & 1 & YES & YES & YES & $1.43$ & $(2,3)$ & NO & 3720\\
$(31,12)$ & 7 & $(27,8)$ & 7 & 1 & YES & YES & YES & $1.43$ & $(2,3)$ & -- & 3721\\
$(31,12)$ & 7 & $(27,8)$ & 7 & 1 & YES & YES & YES & $1.57$ & $(2,3)$ & NO & 3722\\
$(31,14)$ & 8 & $(27,8)$ & 7 & 1 & YES & YES & YES & $1.57$ & $(2,3)$ & 6571 & 3723\\
$(31,9)$ & 8 & $(29,9)$ & 8 & 1 & YES & YES & YES & $1.57$ & $(2,3)$ & -- & 3724\\
$(31,9)$ & 8 & $(29,11)$ & 7 & 1 & YES & YES & YES & $1.43$ & $(2,3)$ & -- & 3725\\
$(31,9)$ & 8 & $(29,12)$ & 7 & 1 & YES & YES & YES & $1.43$ & $(2,3)$ & -- & 3726\\
$(31,13)$ & 7 & $(29,12)$ & 7 & 1 & YES & YES & YES & $1.57$ & $(2,3)$ & -- & 3727\\
$(31,13)$ & 7 & $(29,12)$ & 7 & 1 & YES & YES & YES & $1.71$ & $(2,3)$ & NO & 3728\\
$(31,14)$ & 8 & $(29,11)$ & 7 & 1 & YES & YES & YES & $1.57$ & $(2,3)$ & NO & 3729\\
$(31,9)$ & 8 & $(30,11)$ & 7 & 1 & YES & YES & YES & $1.43$ & $(2,3)$ & -- & 3730\\
$(31,9)$ & 8 & $(30,11)$ & 7 & 1 & YES & YES & YES & $1.71$ & $(2,3)$ & NO & 3731\\
$(31,12)$ & 7 & $(30,11)$ & 7 & 1 & YES & YES & YES & $1.57$ & $(2,3)$ & -- & 3732\\
$(31,12)$ & 7 & $(30,13)$ & 8 & 1 & YES & YES & YES & $1.57$ & $(2,3)$ & -- & 3733\\
$(31,13)$ & 7 & $(30,11)$ & 7 & 1 & YES & YES & NO(2) & $1.38$ & $(2,3)$ & NO & 3734\\
$(31,14)$ & 8 & $(30,11)$ & 7 & 1 & YES & YES & YES & $1.57$ & $(2,3)$ & NO & 3735\\
$(31,9)$ & 8 & $(31,9)$ & 8 & 31 & YES & YES & YES & $1.43$ & $(2,3)$ & -- & 3736\\
$(31,13)$ & 7 & $(31,9)$ & 8 & 31 & YES & YES & YES & $1.57$ & $(2,3)$ & -- & 3737\\
$(31,14)$ & 8 & $(31,12)$ & 7 & 31 & YES & YES & YES & $1.57$ & $(2,3)$ & -- & 3738\\
$(32,7)$ & 8 & $(10,3)$ & 5 & 2 & YES & YES & YES & $1.43$ & $(2,3)$ & NO & 3739\\
$(32,7)$ & 8 & $(10,3)$ & 5 & 2 & YES & YES & YES & $1.43$ & $(2,3)$ & -- & 3740\\
$(32,9)$ & 8 & $(13,5)$ & 5 & 1 & YES & YES & YES & $1.57$ & $(2,3)$ & -- & 3741\\
$(32,9)$ & 8 & $(17,5)$ & 6 & 1 & YES & YES & YES & $1.57$ & $(2,3)$ & -- & 3742\\
$(32,13)$ & 9 & $(17,5)$ & 6 & 1 & YES & YES & YES & $1.71$ & $(2,3)$ & -- & 3743\\
$(32,13)$ & 9 & $(18,5)$ & 6 & 2 & YES & YES & YES & $1.57$ & $(2,3)$ & -- & 3744\\
$(32,7)$ & 8 & $(22,9)$ & 7 & 2 & YES & YES & YES & $1.57$ & $(2,3)$ & NO & 3745\\
$(32,7)$ & 8 & $(22,9)$ & 7 & 2 & YES & YES & YES & $1.57$ & $(2,3)$ & -- & 3746\\
$(32,13)$ & 9 & $(23,5)$ & 7 & 1 & YES & YES & YES & $1.57$ & $(2,3)$ & NO & 3747\\
$(32,13)$ & 9 & $(25,7)$ & 7 & 1 & YES & YES & YES & $1.71$ & $(2,3)$ & -- & 3748\\
$(32,9)$ & 8 & $(26,11)$ & 7 & 2 & YES & YES & YES & $1.57$ & $(2,3)$ & -- & 3749\\
$(32,9)$ & 8 & $(26,11)$ & 7 & 2 & YES & YES & YES & $1.71$ & $(2,3)$ & NO & 3750\\
$(32,7)$ & 8 & $(27,11)$ & 8 & 1 & YES & YES & YES & $1.57$ & $(2,3)$ & NO & 3751\\
$(32,9)$ & 8 & $(29,11)$ & 7 & 1 & YES & YES & YES & $1.57$ & $(2,3)$ & -- & 3752\\
$(32,9)$ & 8 & $(31,9)$ & 8 & 1 & YES & YES & YES & $1.57$ & $(2,3)$ & -- & 3753\\
$(32,9)$ & 8 & $(31,12)$ & 7 & 1 & YES & YES & YES & $1.57$ & $(2,3)$ & -- & 3754\\
$(33,10)$ & 8 & $(13,4)$ & 6 & 1 & YES & YES & NO(2) & $1.56$ & $(2,3)$ & NO & 3755\\
$(33,10)$ & 8 & $(13,4)$ & 6 & 1 & YES & YES & NO(2) & $1.56$ & $(2,3)$ & -- & 3756\\
$(33,10)$ & 8 & $(15,4)$ & 6 & 3 & YES & YES & YES & $1.71$ & $(2,3)$ & NO & 3757\\
$(33,10)$ & 8 & $(22,9)$ & 7 & 11 & YES & YES & NO(2) & $1.62$ & $(2,3)$ & -- & 3758\\
$(33,14)$ & 8 & $(22,9)$ & 7 & 11 & YES & YES & YES & $1.57$ & $(2,3)$ & -- & 3759\\
$(33,10)$ & 8 & $(23,9)$ & 7 & 1 & YES & YES & NO(2) & $1.75$ & $(2,3)$ & -- & 3760\\
$(33,10)$ & 8 & $(26,11)$ & 7 & 1 & YES & YES & YES & $1.57$ & $(2,3)$ & -- & 3761\\
$(33,10)$ & 8 & $(29,11)$ & 7 & 1 & YES & YES & YES & $1.43$ & $(2,3)$ & -- & 3762\\
$(33,10)$ & 8 & $(30,11)$ & 7 & 3 & YES & YES & NO(2) & $1.62$ & $(2,3)$ & NO & 3763\\
$(33,10)$ & 8 & $(30,11)$ & 7 & 3 & YES & YES & NO(2) & $1.62$ & $(2,3)$ & -- & 3764\\
$(34,13)$ & 7 & $(13,5)$ & 5 & 1 & YES & YES & YES & $1.43$ & $(2,3)$ & -- & 3765\\
$(34,13)$ & 7 & $(14,5)$ & 6 & 2 & YES & YES & YES & $1.43$ & $(2,3)$ & -- & 3766\\
$(34,13)$ & 7 & $(15,4)$ & 6 & 1 & YES & YES & YES & $1.43$ & $(2,3)$ & NO & 3767\\
$(34,13)$ & 7 & $(15,4)$ & 6 & 1 & YES & YES & YES & $1.43$ & $(2,3)$ & -- & 3768\\
$(34,13)$ & 7 & $(16,7)$ & 6 & 2 & YES & YES & YES & $1.43$ & $(2,3)$ & -- & 3769\\
$(34,13)$ & 7 & $(17,5)$ & 6 & 17 & YES & YES & NO(2) & $1.50$ & $(2,3)$ & -- & 3770\\
$(34,13)$ & 7 & $(17,5)$ & 6 & 17 & YES & YES & YES & $1.57$ & $(2,3)$ & NO & 3771\\
$(34,15)$ & 8 & $(17,5)$ & 6 & 17 & YES & YES & YES & $1.71$ & $(2,3)$ & -- & 3772\\
$(34,13)$ & 7 & $(18,5)$ & 6 & 2 & YES & YES & YES & $1.43$ & $(2,3)$ & NO & 3773\\
$(34,13)$ & 7 & $(18,5)$ & 6 & 2 & YES & YES & YES & $1.43$ & $(2,3)$ & -- & 3774\\
$(34,13)$ & 7 & $(19,7)$ & 6 & 1 & YES & YES & YES & $1.43$ & $(2,3)$ & -- & 3775\\
$(34,13)$ & 7 & $(19,8)$ & 6 & 1 & YES & YES & NO(2) & $1.62$ & $(2,3)$ & -- & 3776\\
$(34,15)$ & 8 & $(19,7)$ & 6 & 1 & YES & YES & YES & $1.57$ & $(2,3)$ & -- & 3777\\
$(34,15)$ & 8 & $(19,8)$ & 6 & 1 & YES & YES & YES & $1.57$ & $(2,3)$ & -- & 3778\\
$(34,13)$ & 7 & $(20,9)$ & 7 & 2 & YES & YES & YES & $1.43$ & $(2,3)$ & -- & 3779\\
$(34,9)$ & 8 & $(22,9)$ & 7 & 2 & YES & YES & NO(2) & $1.62$ & $(2,3)$ & -- & 3780\\
$(34,9)$ & 8 & $(22,9)$ & 7 & 2 & YES & YES & NO(2) & $1.62$ & $(2,3)$ & NO & 3781\\
$(34,13)$ & 7 & $(22,9)$ & 7 & 2 & YES & YES & YES & $1.43$ & $(2,3)$ & -- & 3782\\
$(34,15)$ & 8 & $(23,7)$ & 7 & 1 & YES & YES & YES & $1.71$ & $(2,3)$ & -- & 3783\\
$(34,13)$ & 7 & $(25,11)$ & 7 & 1 & YES & YES & YES & $1.71$ & $(2,3)$ & -- & 3784\\
$(34,15)$ & 8 & $(25,9)$ & 7 & 1 & YES & YES & YES & $1.57$ & $(2,3)$ & -- & 3785\\
$(34,15)$ & 8 & $(27,8)$ & 7 & 1 & YES & YES & YES & $1.57$ & $(2,3)$ & -- & 3786\\
$(34,15)$ & 8 & $(27,8)$ & 7 & 1 & YES & YES & YES & $1.57$ & $(2,3)$ & NO & 3787\\
$(34,9)$ & 8 & $(29,12)$ & 7 & 1 & YES & YES & YES & $1.57$ & $(2,3)$ & -- & 3788\\
$(34,13)$ & 7 & $(29,8)$ & 7 & 1 & YES & YES & YES & $1.43$ & $(2,3)$ & -- & 3789\\
$(34,13)$ & 7 & $(29,12)$ & 7 & 1 & YES & YES & YES & $1.57$ & $(2,3)$ & -- & 3790\\
$(34,15)$ & 8 & $(30,11)$ & 7 & 2 & YES & YES & YES & $1.57$ & $(2,3)$ & NO & 3791\\
$(34,15)$ & 8 & $(30,11)$ & 7 & 2 & YES & YES & YES & $1.57$ & $(2,3)$ & -- & 3792\\
$(34,13)$ & 7 & $(31,9)$ & 8 & 1 & YES & YES & YES & $1.57$ & $(2,3)$ & -- & 3793\\
$(34,13)$ & 7 & $(31,13)$ & 7 & 1 & YES & YES & YES & $1.43$ & $(2,3)$ & -- & 3794\\
$(34,15)$ & 8 & $(31,12)$ & 7 & 1 & YES & YES & YES & $1.71$ & $(2,3)$ & NO & 3795\\
$(34,13)$ & 7 & $(34,13)$ & 7 & 34 & YES & YES & YES & $1.43$ & $(2,3)$ & -- & 3796\\
$(35,13)$ & 8 & $(18,5)$ & 6 & 1 & YES & YES & YES & $1.57$ & $(2,3)$ & -- & 3797\\
$(35,13)$ & 8 & $(19,7)$ & 6 & 1 & YES & YES & YES & $1.71$ & $(2,3)$ & -- & 3798\\
$(35,8)$ & 8 & $(22,9)$ & 7 & 1 & YES & YES & NO(2) & $1.50$ & $(2,3)$ & NO & 3799\\
$(35,8)$ & 8 & $(22,9)$ & 7 & 1 & YES & YES & YES & $1.57$ & $(2,3)$ & NO & 3800\\
$(35,8)$ & 8 & $(22,9)$ & 7 & 1 & YES & YES & YES & $1.57$ & $(2,3)$ & -- & 3801\\
$(35,13)$ & 8 & $(22,5)$ & 7 & 1 & YES & YES & YES & $1.43$ & $(2,3)$ & NO & 3802\\
$(35,13)$ & 8 & $(22,5)$ & 7 & 1 & YES & YES & NO(2) & $1.43$ & $(4,2)$ & -- & 3803\\
$(35,13)$ & 8 & $(23,5)$ & 7 & 1 & YES & YES & YES & $1.43$ & $(2,3)$ & NO & 3804\\
$(35,13)$ & 8 & $(26,11)$ & 7 & 1 & YES & YES & YES & $1.57$ & $(2,3)$ & -- & 3805\\
$(35,8)$ & 8 & $(29,12)$ & 7 & 1 & YES & YES & YES & $1.43$ & $(2,3)$ & NO & 3806\\
$(35,8)$ & 8 & $(29,12)$ & 7 & 1 & YES & YES & YES & $1.43$ & $(2,3)$ & -- & 3807\\
$(35,13)$ & 8 & $(29,12)$ & 7 & 1 & YES & YES & YES & $1.71$ & $(2,3)$ & -- & 3808\\
$(35,13)$ & 8 & $(30,7)$ & 8 & 5 & YES & YES & NO(2) & $1.75$ & $(2,3)$ & NO & 3809\\
$(35,13)$ & 8 & $(30,7)$ & 8 & 5 & YES & YES & YES & $1.57$ & $(2,3)$ & -- & 3810\\
$(35,8)$ & 8 & $(31,14)$ & 8 & 1 & YES & YES & YES & $1.57$ & $(2,3)$ & NO & 3811\\
$(35,13)$ & 8 & $(32,7)$ & 8 & 1 & YES & YES & YES & $1.57$ & $(2,3)$ & NO & 3812\\
$(35,11)$ & 9 & $(35,8)$ & 8 & 35 & YES & YES & YES & $1.57$ & $(2,3)$ & NO & 3813\\
$(35,13)$ & 8 & $(35,8)$ & 8 & 35 & YES & YES & YES & $1.57$ & $(2,3)$ & NO & 3814\\
$(36,11)$ & 8 & $(17,6)$ & 7 & 1 & YES & YES & YES & $1.43$ & $(2,3)$ & -- & 3815\\
$(36,11)$ & 8 & $(19,6)$ & 8 & 1 & YES & YES & YES & $1.57$ & $(2,3)$ & -- & 3816\\
$(36,11)$ & 8 & $(29,8)$ & 7 & 1 & YES & YES & YES & $1.43$ & $(2,3)$ & NO & 3817\\
$(36,11)$ & 8 & $(29,11)$ & 7 & 1 & YES & YES & YES & $1.57$ & $(2,3)$ & -- & 3818\\
$(36,13)$ & 8 & $(29,12)$ & 7 & 1 & YES & YES & NO(2) & $1.62$ & $(2,3)$ & NO & 3819\\
$(36,11)$ & 8 & $(30,11)$ & 7 & 6 & YES & YES & YES & $1.57$ & $(2,3)$ & -- & 3820\\
$(36,13)$ & 8 & $(30,11)$ & 7 & 6 & YES & YES & YES & $1.71$ & $(2,3)$ & -- & 3821\\
$(36,11)$ & 8 & $(31,12)$ & 7 & 1 & YES & YES & YES & $1.57$ & $(2,3)$ & -- & 3822\\
$(36,13)$ & 8 & $(34,9)$ & 8 & 2 & YES & YES & YES & $1.71$ & $(2,3)$ & -- & 3823\\
$(36,13)$ & 8 & $(36,13)$ & 8 & 36 & YES & YES & YES & $1.71$ & $(2,3)$ & -- & 3824\\
$(37,8)$ & 8 & $(10,3)$ & 5 & 1 & YES & YES & YES & $1.43$ & $(2,3)$ & NO & 3825\\
$(37,8)$ & 8 & $(10,3)$ & 5 & 1 & YES & YES & YES & $1.43$ & $(2,3)$ & -- & 3826\\
$(37,11)$ & 8 & $(11,2)$ & 6 & 1 & YES & YES & YES & $1.57$ & $(2,3)$ & NO & 3827\\
$(37,11)$ & 8 & $(11,2)$ & 6 & 1 & YES & YES & YES & $1.57$ & $(2,3)$ & -- & 3828\\
$(37,11)$ & 8 & $(11,2)$ & 6 & 1 & YES & YES & YES & $1.57$ & $(2,3)$ & NO & 3829\\
$(37,11)$ & 8 & $(12,5)$ & 5 & 1 & YES & YES & YES & $1.57$ & $(2,3)$ & -- & 3830\\
$(37,10)$ & 8 & $(13,5)$ & 5 & 1 & YES & YES & YES & $1.57$ & $(2,3)$ & NO & 3831\\
$(37,10)$ & 8 & $(13,5)$ & 5 & 1 & YES & YES & YES & $1.57$ & $(2,3)$ & -- & 3832\\
$(37,11)$ & 8 & $(14,5)$ & 6 & 1 & YES & YES & YES & $1.57$ & $(2,3)$ & NO & 3833\\
$(37,11)$ & 8 & $(14,5)$ & 6 & 1 & YES & YES & YES & $1.57$ & $(2,3)$ & -- & 3834\\
$(37,14)$ & 8 & $(14,3)$ & 6 & 1 & YES & YES & YES & $1.43$ & $(2,3)$ & NO & 3835\\
$(37,14)$ & 8 & $(14,3)$ & 6 & 1 & YES & YES & YES & $1.43$ & $(2,3)$ & -- & 3836\\
$(37,11)$ & 8 & $(16,7)$ & 6 & 1 & YES & YES & YES & $1.57$ & $(2,3)$ & NO & 3837\\
$(37,11)$ & 8 & $(16,7)$ & 6 & 1 & YES & YES & YES & $1.57$ & $(2,3)$ & -- & 3838\\
$(37,10)$ & 8 & $(17,7)$ & 6 & 1 & YES & YES & YES & $1.71$ & $(2,3)$ & -- & 3839\\
$(37,14)$ & 8 & $(17,5)$ & 6 & 1 & YES & YES & YES & $1.57$ & $(2,3)$ & -- & 3840\\
$(37,16)$ & 9 & $(18,7)$ & 6 & 1 & YES & YES & YES & $1.71$ & $(2,3)$ & -- & 3841\\
$(37,14)$ & 8 & $(19,8)$ & 6 & 1 & YES & YES & YES & $1.71$ & $(2,3)$ & -- & 3842\\
$(37,11)$ & 8 & $(20,9)$ & 7 & 1 & YES & YES & YES & $1.57$ & $(2,3)$ & -- & 3843\\
$(37,11)$ & 8 & $(20,9)$ & 7 & 1 & YES & YES & YES & $1.57$ & $(2,3)$ & NO & 3844\\
$(37,8)$ & 8 & $(22,9)$ & 7 & 1 & YES & YES & YES & $1.43$ & $(2,3)$ & -- & 3845\\
$(37,10)$ & 8 & $(22,9)$ & 7 & 1 & YES & YES & NO(2) & $1.75$ & $(2,3)$ & -- & 3846\\
$(37,11)$ & 8 & $(22,5)$ & 7 & 1 & YES & YES & YES & $1.57$ & $(2,3)$ & NO & 3847\\
$(37,11)$ & 8 & $(22,5)$ & 7 & 1 & YES & YES & YES & $1.57$ & $(2,3)$ & -- & 3848\\
$(37,11)$ & 8 & $(22,9)$ & 7 & 1 & YES & YES & YES & $1.57$ & $(2,3)$ & -- & 3849\\
$(37,14)$ & 8 & $(22,5)$ & 7 & 1 & YES & YES & YES & $1.57$ & $(2,3)$ & -- & 3850\\
$(37,10)$ & 8 & $(23,9)$ & 7 & 1 & YES & YES & NO(2) & $1.75$ & $(2,3)$ & -- & 3851\\
$(37,11)$ & 8 & $(23,7)$ & 7 & 1 & YES & YES & YES & $1.57$ & $(2,3)$ & -- & 3852\\
$(37,14)$ & 8 & $(23,10)$ & 7 & 1 & YES & YES & YES & $1.71$ & $(2,3)$ & -- & 3853\\
$(37,10)$ & 8 & $(25,11)$ & 7 & 1 & YES & YES & YES & $1.71$ & $(2,3)$ & NO & 3854\\
$(37,10)$ & 8 & $(25,11)$ & 7 & 1 & YES & YES & YES & $1.71$ & $(2,3)$ & -- & 3855\\
$(37,8)$ & 8 & $(26,11)$ & 7 & 1 & YES & YES & YES & $1.43$ & $(2,3)$ & -- & 3856\\
$(37,11)$ & 8 & $(26,11)$ & 7 & 1 & YES & YES & YES & $1.57$ & $(2,3)$ & -- & 3857\\
$(37,8)$ & 8 & $(27,11)$ & 8 & 1 & YES & YES & YES & $1.57$ & $(2,3)$ & NO & 3858\\
$(37,11)$ & 8 & $(29,11)$ & 7 & 1 & YES & YES & YES & $1.57$ & $(2,3)$ & -- & 3859\\
$(37,11)$ & 8 & $(29,12)$ & 7 & 1 & YES & YES & YES & $1.57$ & $(2,3)$ & -- & 3860\\
$(37,10)$ & 8 & $(30,11)$ & 7 & 1 & YES & YES & YES & $1.57$ & $(2,3)$ & NO & 3861\\
$(37,11)$ & 8 & $(30,11)$ & 7 & 1 & YES & YES & YES & $1.57$ & $(2,3)$ & -- & 3862\\
$(37,11)$ & 8 & $(30,11)$ & 7 & 1 & YES & YES & YES & $1.57$ & $(2,3)$ & NO & 3863\\
$(37,10)$ & 8 & $(31,12)$ & 7 & 1 & YES & YES & YES & $1.57$ & $(2,3)$ & NO & 3864\\
$(37,10)$ & 8 & $(31,13)$ & 7 & 1 & YES & YES & YES & $1.57$ & $(2,3)$ & NO & 3865\\
$(37,11)$ & 8 & $(32,9)$ & 8 & 1 & YES & YES & YES & $1.57$ & $(2,3)$ & -- & 3866\\
$(37,10)$ & 8 & $(34,13)$ & 7 & 1 & YES & YES & YES & $1.57$ & $(2,3)$ & -- & 3867\\
$(37,11)$ & 8 & $(35,11)$ & 9 & 1 & YES & YES & YES & $1.57$ & $(2,3)$ & -- & 3868\\
$(38,17)$ & 9 & $(17,5)$ & 6 & 1 & YES & YES & YES & $1.71$ & $(2,3)$ & -- & 3869\\
$(38,9)$ & 9 & $(18,7)$ & 6 & 2 & YES & YES & YES & $1.71$ & $(2,3)$ & -- & 3870\\
$(38,9)$ & 9 & $(22,9)$ & 7 & 2 & YES & YES & NO(2) & $1.62$ & $(2,3)$ & -- & 3871\\
$(38,11)$ & 9 & $(23,7)$ & 7 & 1 & YES & YES & YES & $1.57$ & $(2,3)$ & -- & 3872\\
$(38,9)$ & 9 & $(29,12)$ & 7 & 1 & YES & YES & NO(2) & $1.50$ & $(2,3)$ & -- & 3873\\
$(38,11)$ & 9 & $(30,11)$ & 7 & 2 & YES & YES & YES & $1.57$ & $(2,3)$ & NO & 3874\\
$(39,11)$ & 9 & $(11,3)$ & 5 & 1 & YES & YES & YES & $1.43$ & $(2,3)$ & NO & 3875\\
$(39,11)$ & 9 & $(11,3)$ & 5 & 1 & YES & YES & YES & $1.43$ & $(2,3)$ & -- & 3876\\
$(39,16)$ & 8 & $(12,5)$ & 5 & 3 & YES & YES & NO(2) & $1.50$ & $(2,3)$ & -- & 3877\\
$(39,11)$ & 9 & $(13,5)$ & 5 & 13 & YES & YES & YES & $1.43$ & $(2,3)$ & -- & 3878\\
$(39,11)$ & 9 & $(13,5)$ & 5 & 13 & YES & YES & YES & $1.43$ & $(2,3)$ & NO & 3879\\
$(39,16)$ & 8 & $(16,7)$ & 6 & 1 & YES & YES & NO(2) & $1.62$ & $(2,3)$ & -- & 3880\\
$(39,14)$ & 8 & $(17,5)$ & 6 & 1 & YES & YES & YES & $1.57$ & $(2,3)$ & NO & 3881\\
$(39,16)$ & 8 & $(17,5)$ & 6 & 1 & YES & YES & YES & $1.71$ & $(2,3)$ & -- & 3882\\
$(39,16)$ & 8 & $(18,7)$ & 6 & 3 & YES & YES & YES & $1.57$ & $(2,3)$ & -- & 3883\\
$(39,16)$ & 8 & $(18,7)$ & 6 & 3 & YES & YES & YES & $1.57$ & $(2,3)$ & NO & 3884\\
$(39,16)$ & 8 & $(19,7)$ & 6 & 1 & YES & YES & NO(2) & $1.62$ & $(2,3)$ & NO & 3885\\
$(39,16)$ & 8 & $(19,7)$ & 6 & 1 & YES & YES & NO(2) & $1.62$ & $(2,3)$ & -- & 3886\\
$(39,16)$ & 8 & $(19,8)$ & 6 & 1 & YES & YES & NO(2) & $1.62$ & $(2,3)$ & -- & 3887\\
$(39,16)$ & 8 & $(22,9)$ & 7 & 1 & YES & YES & YES & $1.71$ & $(2,3)$ & -- & 3888\\
$(39,16)$ & 8 & $(23,9)$ & 7 & 1 & YES & YES & YES & $1.57$ & $(2,3)$ & -- & 3889\\
$(39,16)$ & 8 & $(24,7)$ & 7 & 3 & YES & YES & YES & $1.43$ & $(2,3)$ & -- & 3890\\
$(39,16)$ & 8 & $(25,7)$ & 7 & 1 & YES & YES & YES & $1.57$ & $(2,3)$ & -- & 3891\\
$(39,16)$ & 8 & $(25,9)$ & 7 & 1 & YES & YES & YES & $1.57$ & $(2,3)$ & -- & 3892\\
$(39,16)$ & 8 & $(26,7)$ & 7 & 13 & YES & YES & NO(2) & $1.75$ & $(2,3)$ & NO & 3893\\
$(39,16)$ & 8 & $(29,12)$ & 7 & 1 & YES & YES & YES & $1.71$ & $(2,3)$ & -- & 3894\\
$(39,16)$ & 8 & $(30,7)$ & 8 & 3 & YES & YES & NO(2) & $1.50$ & $(2,3)$ & NO & 3895\\
$(39,16)$ & 8 & $(30,7)$ & 8 & 3 & YES & YES & NO(2) & $1.62$ & $(2,3)$ & -- & 3896\\
$(39,17)$ & 8 & $(30,7)$ & 8 & 3 & YES & YES & NO(2) & $1.75$ & $(2,3)$ & NO & 3897\\
$(39,17)$ & 8 & $(31,12)$ & 7 & 1 & YES & YES & YES & $1.57$ & $(2,3)$ & -- & 3898\\
$(39,16)$ & 8 & $(34,15)$ & 8 & 1 & YES & YES & YES & $1.57$ & $(2,3)$ & NO & 3899\\
$(39,11)$ & 9 & $(36,11)$ & 8 & 3 & YES & YES & YES & $1.43$ & $(2,3)$ & NO & 3900\\
$(39,17)$ & 8 & $(39,16)$ & 8 & 39 & YES & YES & NO(2) & $1.62$ & $(2,3)$ & NO & 3901\\
$(40,11)$ & 8 & $(12,5)$ & 5 & 4 & YES & YES & YES & $1.57$ & $(2,3)$ & -- & 3902\\
$(40,11)$ & 8 & $(12,5)$ & 5 & 4 & YES & YES & YES & $1.71$ & $(2,3)$ & NO & 3903\\
$(40,11)$ & 8 & $(13,3)$ & 6 & 1 & YES & YES & YES & $1.43$ & $(2,3)$ & NO & 3904\\
$(40,11)$ & 8 & $(13,3)$ & 6 & 1 & YES & YES & YES & $1.43$ & $(2,3)$ & -- & 3905\\
$(40,11)$ & 8 & $(13,5)$ & 5 & 1 & YES & YES & YES & $1.57$ & $(2,3)$ & -- & 3906\\
$(40,11)$ & 8 & $(13,5)$ & 5 & 1 & YES & YES & YES & $1.57$ & $(2,3)$ & NO & 3907\\
$(40,11)$ & 8 & $(16,7)$ & 6 & 8 & YES & YES & YES & $1.57$ & $(2,3)$ & NO & 3908\\
$(40,11)$ & 8 & $(16,7)$ & 6 & 8 & YES & YES & YES & $1.57$ & $(2,3)$ & -- & 3909\\
$(40,11)$ & 8 & $(21,8)$ & 6 & 1 & YES & YES & YES & $1.43$ & $(2,3)$ & -- & 3910\\
$(40,11)$ & 8 & $(22,9)$ & 7 & 2 & YES & YES & YES & $1.57$ & $(2,3)$ & -- & 3911\\
$(40,11)$ & 8 & $(23,7)$ & 7 & 1 & YES & YES & YES & $1.57$ & $(2,3)$ & NO & 3912\\
$(40,11)$ & 8 & $(23,7)$ & 7 & 1 & YES & YES & YES & $1.57$ & $(2,3)$ & -- & 3913\\
$(40,11)$ & 8 & $(29,11)$ & 7 & 1 & YES & YES & YES & $1.57$ & $(2,3)$ & -- & 3914\\
$(40,9)$ & 9 & $(31,9)$ & 8 & 1 & YES & YES & YES & $1.57$ & $(2,3)$ & -- & 3915\\
$(40,11)$ & 8 & $(31,9)$ & 8 & 1 & YES & YES & YES & $1.57$ & $(2,3)$ & -- & 3916\\
$(40,11)$ & 8 & $(31,12)$ & 7 & 1 & YES & YES & YES & $1.57$ & $(2,3)$ & NO & 3917\\
$(40,9)$ & 9 & $(32,9)$ & 8 & 8 & YES & YES & YES & $1.57$ & $(2,3)$ & -- & 3918\\
$(40,11)$ & 8 & $(32,9)$ & 8 & 8 & YES & YES & YES & $1.57$ & $(2,3)$ & -- & 3919\\
$(41,11)$ & 8 & $(12,5)$ & 5 & 1 & YES & YES & YES & $1.43$ & $(2,3)$ & NO & 3920\\
$(41,11)$ & 8 & $(12,5)$ & 5 & 1 & YES & YES & YES & $1.43$ & $(2,3)$ & -- & 3921\\
$(41,12)$ & 8 & $(12,5)$ & 5 & 1 & YES & YES & NO(2) & $1.62$ & $(2,3)$ & -- & 3922\\
$(41,15)$ & 8 & $(12,5)$ & 5 & 1 & YES & YES & YES & $1.43$ & $(2,3)$ & -- & 3923\\
$(41,11)$ & 8 & $(13,5)$ & 5 & 1 & YES & YES & YES & $1.43$ & $(2,3)$ & NO & 3924\\
$(41,11)$ & 8 & $(13,5)$ & 5 & 1 & YES & YES & YES & $1.43$ & $(2,3)$ & -- & 3925\\
$(41,12)$ & 8 & $(13,5)$ & 5 & 1 & YES & YES & YES & $1.43$ & $(2,3)$ & -- & 3926\\
$(41,15)$ & 8 & $(13,5)$ & 5 & 1 & YES & YES & YES & $1.43$ & $(2,3)$ & -- & 3927\\
$(41,12)$ & 8 & $(16,5)$ & 7 & 1 & YES & YES & YES & $1.57$ & $(2,3)$ & NO & 3928\\
$(41,12)$ & 8 & $(16,5)$ & 7 & 1 & YES & YES & YES & $1.57$ & $(2,3)$ & -- & 3929\\
$(41,15)$ & 8 & $(16,7)$ & 6 & 1 & YES & YES & YES & $1.57$ & $(2,3)$ & -- & 3930\\
$(41,11)$ & 8 & $(17,5)$ & 6 & 1 & YES & YES & YES & $1.43$ & $(2,3)$ & -- & 3931\\
$(41,15)$ & 8 & $(17,7)$ & 6 & 1 & YES & YES & YES & $1.43$ & $(2,3)$ & -- & 3932\\
$(41,18)$ & 8 & $(18,5)$ & 6 & 1 & YES & YES & YES & $1.57$ & $(2,3)$ & -- & 3933\\
$(41,11)$ & 8 & $(19,6)$ & 8 & 1 & YES & YES & YES & $1.71$ & $(2,3)$ & -- & 3934\\
$(41,15)$ & 8 & $(19,7)$ & 6 & 1 & YES & YES & NO(2) & $1.75$ & $(2,3)$ & NO & 3935\\
$(41,15)$ & 8 & $(19,7)$ & 6 & 1 & YES & YES & NO(2) & $1.75$ & $(2,3)$ & -- & 3936\\
$(41,16)$ & 8 & $(19,7)$ & 6 & 1 & YES & YES & NO(2) & $1.62$ & $(2,3)$ & NO & 3937\\
$(41,16)$ & 8 & $(19,7)$ & 6 & 1 & YES & YES & NO(2) & $1.62$ & $(2,3)$ & -- & 3938\\
$(41,12)$ & 8 & $(20,9)$ & 7 & 1 & YES & YES & YES & $1.43$ & $(2,3)$ & -- & 3939\\
$(41,15)$ & 8 & $(20,9)$ & 7 & 1 & YES & YES & YES & $1.57$ & $(2,3)$ & NO & 3940\\
$(41,15)$ & 8 & $(21,8)$ & 6 & 1 & YES & YES & YES & $1.71$ & $(2,3)$ & NO & 3941\\
$(41,15)$ & 8 & $(21,8)$ & 6 & 1 & YES & YES & YES & $1.71$ & $(2,3)$ & -- & 3942\\
$(41,11)$ & 8 & $(22,9)$ & 7 & 1 & YES & YES & NO(2) & $1.50$ & $(2,3)$ & NO & 3943\\
$(41,12)$ & 8 & $(22,9)$ & 7 & 1 & YES & YES & YES & $1.43$ & $(2,3)$ & -- & 3944\\
$(41,12)$ & 8 & $(22,9)$ & 7 & 1 & YES & YES & NO(2) & $1.50$ & $(2,3)$ & NO & 3945\\
$(41,15)$ & 8 & $(22,9)$ & 7 & 1 & YES & YES & YES & $1.57$ & $(2,3)$ & -- & 3946\\
$(41,15)$ & 8 & $(22,9)$ & 7 & 1 & YES & YES & YES & $1.71$ & $(2,3)$ & NO & 3947\\
$(41,16)$ & 8 & $(22,9)$ & 7 & 1 & YES & YES & YES & $1.71$ & $(2,3)$ & -- & 3948\\
$(41,11)$ & 8 & $(23,9)$ & 7 & 1 & YES & YES & YES & $1.57$ & $(2,3)$ & -- & 3949\\
$(41,12)$ & 8 & $(23,7)$ & 7 & 1 & YES & YES & YES & $1.43$ & $(2,3)$ & -- & 3950\\
$(41,15)$ & 8 & $(23,9)$ & 7 & 1 & YES & YES & YES & $1.57$ & $(2,3)$ & -- & 3951\\
$(41,16)$ & 8 & $(23,7)$ & 7 & 1 & YES & YES & YES & $1.57$ & $(2,3)$ & -- & 3952\\
$(41,16)$ & 8 & $(23,9)$ & 7 & 1 & YES & YES & YES & $1.57$ & $(2,3)$ & -- & 3953\\
$(41,12)$ & 8 & $(25,7)$ & 7 & 1 & YES & YES & YES & $1.57$ & $(2,3)$ & -- & 3954\\
$(41,17)$ & 8 & $(25,9)$ & 7 & 1 & YES & YES & NO(2) & $1.75$ & $(2,3)$ & NO & 3955\\
$(41,16)$ & 8 & $(26,7)$ & 7 & 1 & YES & YES & YES & $1.57$ & $(2,3)$ & -- & 3956\\
$(41,16)$ & 8 & $(26,7)$ & 7 & 1 & YES & YES & NO(2) & $1.75$ & $(2,3)$ & NO & 3957\\
$(41,11)$ & 8 & $(27,8)$ & 7 & 1 & YES & YES & YES & $1.43$ & $(2,3)$ & NO & 3958\\
$(41,12)$ & 8 & $(27,8)$ & 7 & 1 & YES & YES & YES & $1.57$ & $(2,3)$ & -- & 3959\\
$(41,15)$ & 8 & $(27,10)$ & 7 & 1 & YES & YES & YES & $1.71$ & $(2,3)$ & NO & 3960\\
$(41,15)$ & 8 & $(27,10)$ & 7 & 1 & YES & YES & YES & $1.71$ & $(2,3)$ & -- & 3961\\
$(41,11)$ & 8 & $(29,11)$ & 7 & 1 & YES & YES & YES & $1.71$ & $(2,3)$ & -- & 3962\\
$(41,12)$ & 8 & $(29,12)$ & 7 & 1 & YES & YES & YES & $1.57$ & $(2,3)$ & -- & 3963\\
$(41,15)$ & 8 & $(29,12)$ & 7 & 1 & YES & YES & NO(2) & $1.75$ & $(2,3)$ & NO & 3964\\
$(41,16)$ & 8 & $(29,8)$ & 7 & 1 & YES & YES & YES & $1.43$ & $(2,3)$ & NO & 3965\\
$(41,16)$ & 8 & $(29,8)$ & 7 & 1 & YES & YES & YES & $1.57$ & $(2,3)$ & -- & 3966\\
$(41,12)$ & 8 & $(30,11)$ & 7 & 1 & YES & YES & YES & $1.43$ & $(2,3)$ & -- & 3967\\
$(41,16)$ & 8 & $(30,7)$ & 8 & 1 & YES & YES & NO(2) & $1.50$ & $(2,3)$ & NO & 3968\\
$(41,17)$ & 8 & $(30,7)$ & 8 & 1 & YES & YES & YES & $1.57$ & $(2,3)$ & -- & 3969\\
$(41,18)$ & 8 & $(30,7)$ & 8 & 1 & YES & YES & YES & $1.71$ & $(2,3)$ & NO & 3970\\
$(41,11)$ & 8 & $(31,9)$ & 8 & 1 & YES & YES & YES & $1.43$ & $(2,3)$ & -- & 3971\\
$(41,11)$ & 8 & $(31,13)$ & 7 & 1 & YES & YES & YES & $1.57$ & $(2,3)$ & NO & 3972\\
$(41,12)$ & 8 & $(31,9)$ & 8 & 1 & YES & YES & YES & $1.57$ & $(2,3)$ & -- & 3973\\
$(41,15)$ & 8 & $(31,13)$ & 7 & 1 & YES & YES & YES & $1.71$ & $(2,3)$ & NO & 3974\\
$(41,15)$ & 8 & $(32,7)$ & 8 & 1 & YES & YES & YES & $1.57$ & $(2,3)$ & NO & 3975\\
$(41,11)$ & 8 & $(35,11)$ & 9 & 1 & YES & YES & YES & $1.57$ & $(2,3)$ & NO & 3976\\
$(41,11)$ & 8 & $(35,13)$ & 8 & 1 & YES & YES & YES & $1.86$ & $(2,3)$ & NO & 3977\\
$(41,16)$ & 8 & $(35,8)$ & 8 & 1 & YES & YES & YES & $1.43$ & $(2,3)$ & NO & 3978\\
$(41,9)$ & 9 & $(37,11)$ & 8 & 1 & YES & YES & YES & $1.57$ & $(2,3)$ & -- & 3979\\
$(41,16)$ & 8 & $(37,10)$ & 8 & 1 & YES & YES & YES & $1.71$ & $(2,3)$ & NO & 3980\\
$(41,11)$ & 8 & $(38,11)$ & 9 & 1 & YES & YES & YES & $1.57$ & $(2,3)$ & NO & 3981\\
$(41,12)$ & 8 & $(40,11)$ & 8 & 1 & YES & YES & YES & $1.57$ & $(2,3)$ & NO & 3982\\
$(41,16)$ & 8 & $(40,9)$ & 9 & 1 & YES & YES & YES & $1.57$ & $(2,3)$ & -- & 3983\\
$(41,12)$ & 8 & $(41,11)$ & 8 & 41 & YES & YES & YES & $1.43$ & $(2,3)$ & -- & 3984\\
$(41,17)$ & 8 & $(41,15)$ & 8 & 41 & YES & YES & YES & $1.71$ & $(2,3)$ & NO & 3985\\
$(42,13)$ & 9 & $(19,6)$ & 8 & 1 & YES & YES & YES & $1.57$ & $(2,3)$ & -- & 3986\\
$(42,13)$ & 9 & $(19,8)$ & 6 & 1 & YES & YES & YES & $1.57$ & $(2,3)$ & -- & 3987\\
$(42,13)$ & 9 & $(21,8)$ & 6 & 21 & YES & YES & YES & $1.71$ & $(2,3)$ & -- & 3988\\
$(42,13)$ & 9 & $(23,6)$ & 8 & 1 & YES & YES & YES & $1.57$ & $(2,3)$ & -- & 3989\\
$(42,13)$ & 9 & $(23,10)$ & 7 & 1 & YES & YES & YES & $1.57$ & $(2,3)$ & -- & 3990\\
$(42,19)$ & 9 & $(33,14)$ & 8 & 3 & YES & YES & YES & $1.57$ & $(2,3)$ & NO & 3991\\
$(43,12)$ & 8 & $(7,2)$ & 4 & 1 & YES & YES & YES & $1.57$ & $(2,3)$ & -- & 3992\\
$(43,12)$ & 8 & $(11,4)$ & 5 & 1 & YES & YES & YES & $1.43$ & $(2,3)$ & -- & 3993\\
$(43,12)$ & 8 & $(11,4)$ & 5 & 1 & YES & YES & YES & $1.57$ & $(2,3)$ & NO & 3994\\
$(43,18)$ & 8 & $(11,3)$ & 5 & 1 & YES & YES & YES & $1.43$ & $(2,3)$ & -- & 3995\\
$(43,18)$ & 8 & $(11,4)$ & 5 & 1 & YES & YES & YES & $1.43$ & $(2,3)$ & -- & 3996\\
$(43,12)$ & 8 & $(12,5)$ & 5 & 1 & YES & YES & YES & $1.57$ & $(2,3)$ & NO & 3997\\
$(43,12)$ & 8 & $(12,5)$ & 5 & 1 & YES & YES & YES & $1.43$ & $(2,3)$ & -- & 3998\\
$(43,18)$ & 8 & $(12,5)$ & 5 & 1 & YES & YES & NO(2) & $1.43$ & $(4,2)$ & -- & 3999\\
$(43,13)$ & 9 & $(13,5)$ & 5 & 1 & YES & YES & YES & $1.43$ & $(2,3)$ & -- & 4000\\
$(43,18)$ & 8 & $(13,3)$ & 6 & 1 & YES & YES & NO(2) & $1.50$ & $(2,3)$ & NO & 4001\\
$(43,18)$ & 8 & $(13,3)$ & 6 & 1 & YES & YES & NO(2) & $1.50$ & $(2,3)$ & -- & 4002\\
$(43,18)$ & 8 & $(13,5)$ & 5 & 1 & YES & YES & YES & $1.43$ & $(2,3)$ & -- & 4003\\
$(43,13)$ & 9 & $(16,7)$ & 6 & 1 & YES & YES & NO(2) & $1.62$ & $(2,3)$ & -- & 4004\\
$(43,18)$ & 8 & $(16,7)$ & 6 & 1 & YES & YES & NO(2) & $1.62$ & $(2,3)$ & -- & 4005\\
$(43,13)$ & 9 & $(17,5)$ & 6 & 1 & YES & YES & YES & $1.57$ & $(2,3)$ & NO & 4006\\
$(43,13)$ & 9 & $(17,5)$ & 6 & 1 & YES & YES & YES & $1.57$ & $(2,3)$ & -- & 4007\\
$(43,13)$ & 9 & $(17,7)$ & 6 & 1 & YES & YES & NO(2) & $1.62$ & $(2,3)$ & NO & 4008\\
$(43,13)$ & 9 & $(17,7)$ & 6 & 1 & YES & YES & NO(2) & $1.62$ & $(2,3)$ & -- & 4009\\
$(43,18)$ & 8 & $(17,7)$ & 6 & 1 & YES & YES & YES & $1.71$ & $(2,3)$ & -- & 4010\\
$(43,13)$ & 9 & $(18,7)$ & 6 & 1 & YES & YES & YES & $1.57$ & $(2,3)$ & NO & 4011\\
$(43,13)$ & 9 & $(18,7)$ & 6 & 1 & YES & YES & YES & $1.57$ & $(2,3)$ & -- & 4012\\
$(43,18)$ & 8 & $(18,5)$ & 6 & 1 & YES & YES & YES & $1.43$ & $(2,3)$ & NO & 4013\\
$(43,18)$ & 8 & $(18,7)$ & 6 & 1 & YES & YES & YES & $1.57$ & $(2,3)$ & -- & 4014\\
$(43,13)$ & 9 & $(19,7)$ & 6 & 1 & YES & YES & NO(2) & $1.75$ & $(2,3)$ & -- & 4015\\
$(43,13)$ & 9 & $(19,8)$ & 6 & 1 & YES & YES & YES & $1.43$ & $(2,3)$ & -- & 4016\\
$(43,18)$ & 8 & $(19,7)$ & 6 & 1 & YES & YES & YES & $1.43$ & $(2,3)$ & -- & 4017\\
$(43,18)$ & 8 & $(19,8)$ & 6 & 1 & YES & YES & YES & $1.43$ & $(2,3)$ & -- & 4018\\
$(43,13)$ & 9 & $(21,8)$ & 6 & 1 & YES & YES & YES & $1.43$ & $(2,3)$ & -- & 4019\\
$(43,18)$ & 8 & $(21,8)$ & 6 & 1 & YES & YES & YES & $1.43$ & $(2,3)$ & -- & 4020\\
$(43,12)$ & 8 & $(22,9)$ & 7 & 1 & YES & YES & YES & $1.57$ & $(2,3)$ & NO & 4021\\
$(43,12)$ & 8 & $(22,9)$ & 7 & 1 & YES & YES & YES & $1.57$ & $(2,3)$ & -- & 4022\\
$(43,13)$ & 9 & $(22,9)$ & 7 & 1 & YES & YES & YES & $1.57$ & $(2,3)$ & -- & 4023\\
$(43,18)$ & 8 & $(22,5)$ & 7 & 1 & YES & YES & YES & $1.43$ & $(2,3)$ & -- & 4024\\
$(43,18)$ & 8 & $(22,9)$ & 7 & 1 & YES & YES & YES & $1.71$ & $(2,3)$ & -- & 4025\\
$(43,10)$ & 9 & $(23,7)$ & 7 & 1 & YES & YES & YES & $1.86$ & $(2,3)$ & NO & 4026\\
$(43,10)$ & 9 & $(23,7)$ & 7 & 1 & YES & YES & YES & $1.86$ & $(2,3)$ & -- & 4027\\
$(43,10)$ & 9 & $(23,9)$ & 7 & 1 & YES & YES & NO(2) & $1.75$ & $(2,3)$ & -- & 4028\\
$(43,12)$ & 8 & $(23,7)$ & 7 & 1 & YES & YES & YES & $1.43$ & $(2,3)$ & -- & 4029\\
$(43,12)$ & 8 & $(23,9)$ & 7 & 1 & YES & YES & NO(2) & $1.50$ & $(2,3)$ & NO & 4030\\
$(43,13)$ & 9 & $(23,5)$ & 7 & 1 & YES & YES & YES & $1.43$ & $(2,3)$ & NO & 4031\\
$(43,12)$ & 8 & $(24,7)$ & 7 & 1 & YES & YES & YES & $1.43$ & $(2,3)$ & -- & 4032\\
$(43,10)$ & 9 & $(25,9)$ & 7 & 1 & YES & YES & NO(2) & $1.75$ & $(2,3)$ & NO & 4033\\
$(43,10)$ & 9 & $(25,9)$ & 7 & 1 & YES & YES & NO(2) & $1.75$ & $(2,3)$ & -- & 4034\\
$(43,13)$ & 9 & $(25,9)$ & 7 & 1 & YES & YES & YES & $1.57$ & $(2,3)$ & -- & 4035\\
$(43,12)$ & 8 & $(26,7)$ & 7 & 1 & YES & YES & NO(2) & $1.50$ & $(2,3)$ & -- & 4036\\
$(43,16)$ & 9 & $(27,8)$ & 7 & 1 & YES & YES & YES & $1.71$ & $(2,3)$ & NO & 4037\\
$(43,12)$ & 8 & $(29,12)$ & 7 & 1 & YES & YES & YES & $1.57$ & $(2,3)$ & NO & 4038\\
$(43,12)$ & 8 & $(29,12)$ & 7 & 1 & YES & YES & YES & $1.57$ & $(2,3)$ & -- & 4039\\
$(43,18)$ & 8 & $(29,8)$ & 7 & 1 & YES & YES & YES & $1.43$ & $(2,3)$ & -- & 4040\\
$(43,16)$ & 9 & $(30,7)$ & 8 & 1 & YES & YES & YES & $1.57$ & $(2,3)$ & -- & 4041\\
$(43,18)$ & 8 & $(30,11)$ & 7 & 1 & YES & YES & YES & $1.43$ & $(2,3)$ & NO & 4042\\
$(43,10)$ & 9 & $(31,9)$ & 8 & 1 & YES & YES & YES & $1.43$ & $(2,3)$ & -- & 4043\\
$(43,12)$ & 8 & $(31,9)$ & 8 & 1 & YES & YES & YES & $1.57$ & $(2,3)$ & -- & 4044\\
$(43,18)$ & 8 & $(31,12)$ & 7 & 1 & YES & YES & YES & $1.71$ & $(2,3)$ & NO & 4045\\
$(43,10)$ & 9 & $(32,9)$ & 8 & 1 & YES & YES & YES & $1.43$ & $(2,3)$ & -- & 4046\\
$(43,13)$ & 9 & $(34,9)$ & 8 & 1 & YES & YES & NO(2) & $1.50$ & $(2,3)$ & NO & 4047\\
$(43,8)$ & 9 & $(35,11)$ & 9 & 1 & YES & YES & YES & $1.57$ & $(2,3)$ & NO & 4048\\
$(43,13)$ & 9 & $(35,8)$ & 8 & 1 & YES & YES & YES & $1.57$ & $(2,3)$ & NO & 4049\\
$(43,10)$ & 9 & $(41,12)$ & 8 & 1 & YES & YES & YES & $1.57$ & $(2,3)$ & -- & 4050\\
$(43,13)$ & 9 & $(43,12)$ & 8 & 43 & YES & YES & NO(2) & $1.75$ & $(2,3)$ & NO & 4051\\
$(44,13)$ & 8 & $(7,2)$ & 4 & 1 & YES & YES & YES & $1.43$ & $(2,3)$ & -- & 4052\\
$(44,13)$ & 8 & $(11,2)$ & 6 & 11 & YES & YES & YES & $1.43$ & $(2,3)$ & NO & 4053\\
$(44,13)$ & 8 & $(11,2)$ & 6 & 11 & YES & YES & YES & $1.43$ & $(2,3)$ & -- & 4054\\
$(44,13)$ & 8 & $(13,4)$ & 6 & 1 & YES & YES & YES & $1.43$ & $(2,3)$ & -- & 4055\\
$(44,13)$ & 8 & $(13,5)$ & 5 & 1 & YES & YES & YES & $1.43$ & $(2,3)$ & NO & 4056\\
$(44,13)$ & 8 & $(13,5)$ & 5 & 1 & YES & YES & YES & $1.43$ & $(2,3)$ & -- & 4057\\
$(44,13)$ & 8 & $(16,7)$ & 6 & 4 & YES & YES & YES & $1.43$ & $(2,3)$ & -- & 4058\\
$(44,17)$ & 8 & $(18,5)$ & 6 & 2 & YES & YES & YES & $1.57$ & $(2,3)$ & -- & 4059\\
$(44,17)$ & 8 & $(18,5)$ & 6 & 2 & YES & YES & YES & $1.57$ & $(2,3)$ & NO & 4060\\
$(44,17)$ & 8 & $(25,9)$ & 7 & 1 & YES & YES & YES & $1.57$ & $(2,3)$ & NO & 4061\\
$(44,13)$ & 8 & $(29,8)$ & 7 & 1 & YES & YES & YES & $1.43$ & $(2,3)$ & NO & 4062\\
$(45,17)$ & 9 & $(11,3)$ & 5 & 1 & YES & YES & YES & $1.57$ & $(2,3)$ & NO & 4063\\
$(45,17)$ & 9 & $(11,3)$ & 5 & 1 & YES & YES & YES & $1.57$ & $(2,3)$ & -- & 4064\\
$(45,19)$ & 8 & $(11,5)$ & 6 & 1 & YES & YES & YES & $1.71$ & $(2,3)$ & -- & 4065\\
$(45,19)$ & 8 & $(12,5)$ & 5 & 3 & YES & YES & NO(2) & $1.57$ & $(4,2)$ & -- & 4066\\
$(45,13)$ & 10 & $(13,5)$ & 5 & 1 & YES & YES & YES & $1.57$ & $(2,3)$ & -- & 4067\\
$(45,17)$ & 9 & $(13,5)$ & 5 & 1 & YES & YES & YES & $1.57$ & $(2,3)$ & -- & 4068\\
$(45,13)$ & 10 & $(17,5)$ & 6 & 1 & YES & YES & YES & $1.57$ & $(2,3)$ & -- & 4069\\
$(45,13)$ & 10 & $(17,7)$ & 6 & 1 & YES & YES & YES & $1.57$ & $(2,3)$ & -- & 4070\\
$(45,17)$ & 9 & $(17,5)$ & 6 & 1 & YES & YES & YES & $1.57$ & $(2,3)$ & -- & 4071\\
$(45,19)$ & 8 & $(17,5)$ & 6 & 1 & YES & YES & YES & $1.43$ & $(2,3)$ & NO & 4072\\
$(45,19)$ & 8 & $(17,5)$ & 6 & 1 & YES & YES & YES & $1.43$ & $(2,3)$ & -- & 4073\\
$(45,17)$ & 9 & $(23,5)$ & 7 & 1 & YES & YES & YES & $1.57$ & $(2,3)$ & NO & 4074\\
$(45,13)$ & 10 & $(27,5)$ & 8 & 9 & YES & YES & YES & $1.57$ & $(2,3)$ & NO & 4075\\
$(45,17)$ & 9 & $(27,5)$ & 8 & 9 & YES & YES & YES & $1.57$ & $(2,3)$ & NO & 4076\\
$(45,17)$ & 9 & $(27,5)$ & 8 & 9 & YES & YES & YES & $1.71$ & $(2,3)$ & -- & 4077\\
$(45,13)$ & 10 & $(28,5)$ & 8 & 1 & YES & YES & YES & $1.57$ & $(2,3)$ & NO & 4078\\
$(45,19)$ & 8 & $(30,11)$ & 7 & 15 & YES & YES & YES & $1.57$ & $(2,3)$ & NO & 4079\\
$(45,13)$ & 10 & $(32,5)$ & 9 & 1 & YES & YES & YES & $1.57$ & $(2,3)$ & NO & 4080\\
$(45,19)$ & 8 & $(32,13)$ & 9 & 1 & YES & YES & YES & $1.71$ & $(2,3)$ & NO & 4081\\
$(45,19)$ & 8 & $(34,13)$ & 7 & 1 & YES & YES & YES & $1.57$ & $(2,3)$ & NO & 4082\\
$(45,14)$ & 9 & $(35,8)$ & 8 & 5 & YES & YES & YES & $1.43$ & $(2,3)$ & 7845 & 4083\\
$(45,13)$ & 10 & $(36,11)$ & 8 & 9 & YES & YES & YES & $1.57$ & $(2,3)$ & NO & 4084\\
$(45,13)$ & 10 & $(42,13)$ & 9 & 3 & YES & YES & YES & $1.57$ & $(2,3)$ & NO & 4085\\
$(46,17)$ & 8 & $(10,3)$ & 5 & 2 & YES & YES & YES & $1.43$ & $(2,3)$ & -- & 4086\\
$(46,19)$ & 8 & $(11,3)$ & 5 & 1 & YES & YES & YES & $1.43$ & $(2,3)$ & NO & 4087\\
$(46,19)$ & 8 & $(11,3)$ & 5 & 1 & YES & YES & YES & $1.43$ & $(2,3)$ & -- & 4088\\
$(46,19)$ & 8 & $(12,5)$ & 5 & 2 & YES & YES & NO(2) & $1.50$ & $(2,3)$ & -- & 4089\\
$(46,17)$ & 8 & $(13,5)$ & 5 & 1 & YES & YES & YES & $1.43$ & $(2,3)$ & -- & 4090\\
$(46,19)$ & 8 & $(13,5)$ & 5 & 1 & YES & YES & NO(2) & $1.43$ & $(4,2)$ & -- & 4091\\
$(46,19)$ & 8 & $(13,5)$ & 5 & 1 & YES & YES & YES & $1.43$ & $(2,3)$ & NO & 4092\\
$(46,17)$ & 8 & $(14,5)$ & 6 & 2 & YES & YES & NO(2) & $1.75$ & $(2,3)$ & -- & 4093\\
$(46,19)$ & 8 & $(14,5)$ & 6 & 2 & YES & YES & YES & $1.57$ & $(2,3)$ & -- & 4094\\
$(46,19)$ & 8 & $(16,7)$ & 6 & 2 & YES & YES & NO(2) & $1.62$ & $(2,3)$ & -- & 4095\\
$(46,19)$ & 8 & $(17,5)$ & 6 & 1 & YES & YES & YES & $1.43$ & $(2,3)$ & -- & 4096\\
$(46,19)$ & 8 & $(17,5)$ & 6 & 1 & YES & YES & NO(2) & $1.50$ & $(2,3)$ & NO & 4097\\
$(46,19)$ & 8 & $(17,7)$ & 6 & 1 & YES & YES & YES & $1.71$ & $(2,3)$ & NO & 4098\\
$(46,19)$ & 8 & $(17,7)$ & 6 & 1 & YES & YES & YES & $1.71$ & $(2,3)$ & -- & 4099\\
$(46,19)$ & 8 & $(18,7)$ & 6 & 2 & YES & YES & YES & $1.57$ & $(2,3)$ & -- & 4100\\
$(46,17)$ & 8 & $(19,7)$ & 6 & 1 & YES & YES & NO(2) & $1.50$ & $(2,3)$ & -- & 4101\\
$(46,17)$ & 8 & $(19,8)$ & 6 & 1 & YES & YES & YES & $1.43$ & $(2,3)$ & -- & 4102\\
$(46,19)$ & 8 & $(19,7)$ & 6 & 1 & YES & YES & YES & $1.43$ & $(2,3)$ & -- & 4103\\
$(46,19)$ & 8 & $(19,8)$ & 6 & 1 & YES & YES & YES & $1.57$ & $(2,3)$ & -- & 4104\\
$(46,17)$ & 8 & $(22,5)$ & 7 & 2 & YES & YES & YES & $1.43$ & $(2,3)$ & -- & 4105\\
$(46,17)$ & 8 & $(26,7)$ & 7 & 2 & YES & YES & YES & $1.57$ & $(2,3)$ & -- & 4106\\
$(46,13)$ & 10 & $(28,5)$ & 8 & 2 & YES & YES & YES & $1.71$ & $(2,3)$ & NO & 4107\\
$(46,17)$ & 8 & $(30,7)$ & 8 & 2 & YES & YES & NO(2) & $1.62$ & $(2,3)$ & NO & 4108\\
$(46,17)$ & 8 & $(30,7)$ & 8 & 2 & YES & YES & YES & $1.57$ & $(2,3)$ & -- & 4109\\
$(46,19)$ & 8 & $(30,11)$ & 7 & 2 & YES & YES & YES & $1.43$ & $(2,3)$ & NO & 4110\\
$(46,17)$ & 8 & $(31,13)$ & 7 & 1 & YES & YES & YES & $1.43$ & $(2,3)$ & NO & 4111\\
$(46,19)$ & 8 & $(31,14)$ & 8 & 1 & YES & YES & YES & $1.43$ & $(2,3)$ & NO & 4112\\
$(46,17)$ & 8 & $(35,8)$ & 8 & 1 & YES & YES & YES & $1.57$ & $(2,3)$ & NO & 4113\\
$(46,19)$ & 8 & $(35,13)$ & 8 & 1 & YES & YES & YES & $1.57$ & $(2,3)$ & NO & 4114\\
$(46,17)$ & 8 & $(41,16)$ & 8 & 1 & YES & YES & NO(2) & $1.75$ & $(2,3)$ & NO & 4115\\
$(47,13)$ & 8 & $(8,3)$ & 4 & 1 & YES & YES & YES & $1.29$ & $(2,3)$ & -- & 4116\\
$(47,18)$ & 8 & $(8,3)$ & 4 & 1 & YES & YES & YES & $1.43$ & $(2,3)$ & -- & 4117\\
$(47,18)$ & 8 & $(10,3)$ & 5 & 1 & YES & YES & YES & $1.57$ & $(2,3)$ & -- & 4118\\
$(47,13)$ & 8 & $(11,2)$ & 6 & 1 & YES & YES & YES & $1.43$ & $(2,3)$ & NO & 4119\\
$(47,13)$ & 8 & $(11,2)$ & 6 & 1 & YES & YES & YES & $1.43$ & $(2,3)$ & -- & 4120\\
$(47,14)$ & 9 & $(11,3)$ & 5 & 1 & YES & YES & YES & $1.43$ & $(2,3)$ & -- & 4121\\
$(47,18)$ & 8 & $(11,3)$ & 5 & 1 & YES & YES & YES & $1.71$ & $(2,3)$ & -- & 4122\\
$(47,13)$ & 8 & $(12,5)$ & 5 & 1 & YES & YES & YES & $1.43$ & $(2,3)$ & NO & 4123\\
$(47,13)$ & 8 & $(12,5)$ & 5 & 1 & YES & YES & YES & $1.43$ & $(2,3)$ & -- & 4124\\
$(47,13)$ & 8 & $(12,5)$ & 5 & 1 & YES & YES & YES & $1.43$ & $(2,3)$ & NO & 4125\\
$(47,14)$ & 9 & $(12,5)$ & 5 & 1 & YES & YES & YES & $1.43$ & $(2,3)$ & -- & 4126\\
$(47,14)$ & 9 & $(13,5)$ & 5 & 1 & YES & YES & YES & $1.43$ & $(2,3)$ & -- & 4127\\
$(47,18)$ & 8 & $(13,3)$ & 6 & 1 & YES & YES & YES & $1.43$ & $(2,3)$ & -- & 4128\\
$(47,18)$ & 8 & $(13,4)$ & 6 & 1 & YES & YES & YES & $1.57$ & $(2,3)$ & -- & 4129\\
$(47,18)$ & 8 & $(13,5)$ & 5 & 1 & YES & YES & YES & $1.43$ & $(2,3)$ & -- & 4130\\
$(47,14)$ & 9 & $(16,7)$ & 6 & 1 & YES & YES & YES & $1.43$ & $(2,3)$ & -- & 4131\\
$(47,18)$ & 8 & $(16,7)$ & 6 & 1 & YES & YES & YES & $1.57$ & $(2,3)$ & NO & 4132\\
$(47,18)$ & 8 & $(16,7)$ & 6 & 1 & YES & YES & YES & $1.57$ & $(2,3)$ & -- & 4133\\
$(47,11)$ & 9 & $(17,7)$ & 6 & 1 & YES & YES & NO(2) & $1.50$ & $(2,3)$ & -- & 4134\\
$(47,13)$ & 8 & $(17,5)$ & 6 & 1 & YES & YES & YES & $1.57$ & $(2,3)$ & -- & 4135\\
$(47,13)$ & 8 & $(17,5)$ & 6 & 1 & YES & YES & YES & $1.57$ & $(2,3)$ & NO & 4136\\
$(47,18)$ & 8 & $(18,7)$ & 6 & 1 & YES & YES & YES & $1.57$ & $(2,3)$ & -- & 4137\\
$(47,14)$ & 9 & $(19,7)$ & 6 & 1 & YES & YES & YES & $1.43$ & $(2,3)$ & -- & 4138\\
$(47,18)$ & 8 & $(19,7)$ & 6 & 1 & YES & YES & YES & $1.57$ & $(2,3)$ & NO & 4139\\
$(47,18)$ & 8 & $(19,7)$ & 6 & 1 & YES & YES & YES & $1.57$ & $(2,3)$ & -- & 4140\\
$(47,13)$ & 8 & $(21,8)$ & 6 & 1 & YES & YES & YES & $1.43$ & $(2,3)$ & -- & 4141\\
$(47,18)$ & 8 & $(21,8)$ & 6 & 1 & YES & YES & YES & $1.43$ & $(2,3)$ & -- & 4142\\
$(47,10)$ & 9 & $(22,9)$ & 7 & 1 & YES & YES & YES & $1.43$ & $(2,3)$ & -- & 4143\\
$(47,11)$ & 9 & $(22,9)$ & 7 & 1 & YES & YES & YES & $1.57$ & $(2,3)$ & NO & 4144\\
$(47,11)$ & 9 & $(22,9)$ & 7 & 1 & YES & YES & YES & $1.57$ & $(2,3)$ & -- & 4145\\
$(47,13)$ & 8 & $(22,9)$ & 7 & 1 & YES & YES & YES & $1.43$ & $(2,3)$ & -- & 4146\\
$(47,13)$ & 8 & $(22,9)$ & 7 & 1 & YES & YES & YES & $1.43$ & $(2,3)$ & NO & 4147\\
$(47,18)$ & 8 & $(22,5)$ & 7 & 1 & YES & YES & YES & $1.43$ & $(2,3)$ & -- & 4148\\
$(47,10)$ & 9 & $(23,9)$ & 7 & 1 & YES & YES & NO(2) & $1.75$ & $(2,3)$ & -- & 4149\\
$(47,10)$ & 9 & $(23,9)$ & 7 & 1 & YES & YES & YES & $1.71$ & $(2,3)$ & NO & 4150\\
$(47,13)$ & 8 & $(23,9)$ & 7 & 1 & YES & YES & YES & $1.43$ & $(2,3)$ & NO & 4151\\
$(47,17)$ & 9 & $(23,7)$ & 7 & 1 & YES & YES & YES & $1.71$ & $(2,3)$ & -- & 4152\\
$(47,17)$ & 9 & $(23,10)$ & 7 & 1 & YES & YES & YES & $1.86$ & $(2,3)$ & -- & 4153\\
$(47,18)$ & 8 & $(23,5)$ & 7 & 1 & YES & YES & YES & $1.43$ & $(2,3)$ & NO & 4154\\
$(47,11)$ & 9 & $(25,11)$ & 7 & 1 & YES & YES & YES & $1.57$ & $(2,3)$ & NO & 4155\\
$(47,11)$ & 9 & $(25,11)$ & 7 & 1 & YES & YES & YES & $1.57$ & $(2,3)$ & -- & 4156\\
$(47,17)$ & 9 & $(25,9)$ & 7 & 1 & YES & YES & YES & $1.86$ & $(2,3)$ & -- & 4157\\
$(47,10)$ & 9 & $(26,11)$ & 7 & 1 & YES & YES & YES & $1.43$ & $(2,3)$ & -- & 4158\\
$(47,13)$ & 8 & $(26,11)$ & 7 & 1 & YES & YES & YES & $1.43$ & $(2,3)$ & -- & 4159\\
$(47,13)$ & 8 & $(26,11)$ & 7 & 1 & YES & YES & YES & $1.43$ & $(2,3)$ & NO & 4160\\
$(47,13)$ & 8 & $(27,8)$ & 7 & 1 & YES & YES & YES & $1.43$ & $(2,3)$ & NO & 4161\\
$(47,11)$ & 9 & $(29,12)$ & 7 & 1 & YES & YES & YES & $1.43$ & $(2,3)$ & -- & 4162\\
$(47,10)$ & 9 & $(31,9)$ & 8 & 1 & YES & YES & YES & $1.43$ & $(2,3)$ & -- & 4163\\
$(47,11)$ & 9 & $(34,13)$ & 7 & 1 & YES & YES & YES & $1.43$ & $(2,3)$ & -- & 4164\\
$(47,18)$ & 8 & $(37,8)$ & 8 & 1 & YES & YES & YES & $1.43$ & $(2,3)$ & -- & 4165\\
$(47,17)$ & 9 & $(39,16)$ & 8 & 1 & YES & YES & YES & $1.71$ & $(2,3)$ & NO & 4166\\
$(47,14)$ & 9 & $(41,11)$ & 8 & 1 & YES & YES & YES & $1.43$ & $(2,3)$ & NO & 4167\\
$(47,17)$ & 9 & $(41,16)$ & 8 & 1 & YES & YES & YES & $1.71$ & $(2,3)$ & NO & 4168\\
$(47,13)$ & 8 & $(43,13)$ & 9 & 1 & YES & YES & YES & $1.43$ & $(2,3)$ & NO & 4169\\
$(47,11)$ & 9 & $(44,17)$ & 8 & 1 & YES & YES & YES & $1.71$ & $(2,3)$ & -- & 4170\\
$(47,14)$ & 9 & $(47,13)$ & 8 & 47 & YES & YES & YES & $1.43$ & $(2,3)$ & NO & 4171\\
$(48,13)$ & 9 & $(11,3)$ & 5 & 1 & YES & YES & YES & $1.43$ & $(2,3)$ & -- & 4172\\
$(48,13)$ & 9 & $(11,3)$ & 5 & 1 & YES & YES & YES & $1.57$ & $(2,3)$ & NO & 4173\\
$(48,17)$ & 9 & $(15,4)$ & 6 & 3 & YES & YES & YES & $1.43$ & $(2,3)$ & -- & 4174\\
$(48,17)$ & 9 & $(17,5)$ & 6 & 1 & YES & YES & YES & $1.57$ & $(2,3)$ & -- & 4175\\
$(48,17)$ & 9 & $(19,5)$ & 7 & 1 & YES & YES & YES & $1.57$ & $(2,3)$ & -- & 4176\\
$(48,17)$ & 9 & $(21,5)$ & 8 & 3 & YES & YES & YES & $1.71$ & $(2,3)$ & -- & 4177\\
$(48,11)$ & 9 & $(22,9)$ & 7 & 2 & YES & YES & YES & $1.57$ & $(2,3)$ & -- & 4178\\
$(48,13)$ & 9 & $(23,5)$ & 7 & 1 & YES & YES & YES & $1.29$ & $(2,3)$ & -- & 4179\\
$(48,11)$ & 9 & $(31,9)$ & 8 & 1 & YES & YES & YES & $1.57$ & $(2,3)$ & -- & 4180\\
$(48,11)$ & 9 & $(32,9)$ & 8 & 16 & YES & YES & YES & $1.57$ & $(2,3)$ & -- & 4181\\
$(48,17)$ & 9 & $(37,14)$ & 8 & 1 & YES & YES & YES & $1.57$ & $(2,3)$ & NO & 4182\\
$(48,11)$ & 9 & $(41,9)$ & 9 & 1 & YES & YES & YES & $1.43$ & $(2,3)$ & -- & 4183\\
$(48,17)$ & 9 & $(43,16)$ & 9 & 1 & YES & YES & YES & $1.71$ & $(2,3)$ & NO & 4184\\
$(49,18)$ & 8 & $(8,3)$ & 4 & 1 & YES & YES & YES & $1.29$ & $(2,3)$ & -- & 4185\\
$(49,18)$ & 8 & $(9,2)$ & 5 & 1 & YES & YES & YES & $1.43$ & $(2,3)$ & -- & 4186\\
$(49,18)$ & 8 & $(11,3)$ & 5 & 1 & YES & YES & YES & $1.57$ & $(2,3)$ & -- & 4187\\
$(49,18)$ & 8 & $(11,4)$ & 5 & 1 & YES & YES & NO(2) & $1.43$ & $(4,2)$ & NO & 4188\\
$(49,18)$ & 8 & $(11,4)$ & 5 & 1 & YES & YES & NO(2) & $1.43$ & $(4,2)$ & -- & 4189\\
$(49,19)$ & 8 & $(11,4)$ & 5 & 1 & YES & YES & YES & $1.43$ & $(2,3)$ & -- & 4190\\
$(49,18)$ & 8 & $(12,5)$ & 5 & 1 & YES & YES & YES & $1.29$ & $(2,3)$ & -- & 4191\\
$(49,19)$ & 8 & $(12,5)$ & 5 & 1 & YES & YES & NO(2) & $1.50$ & $(2,3)$ & -- & 4192\\
$(49,19)$ & 8 & $(12,5)$ & 5 & 1 & YES & YES & YES & $1.57$ & $(2,3)$ & NO & 4193\\
$(49,13)$ & 9 & $(13,5)$ & 5 & 1 & YES & YES & YES & $1.57$ & $(2,3)$ & -- & 4194\\
$(49,13)$ & 9 & $(13,5)$ & 5 & 1 & YES & YES & YES & $1.57$ & $(2,3)$ & NO & 4195\\
$(49,18)$ & 8 & $(13,5)$ & 5 & 1 & YES & YES & YES & $1.43$ & $(2,3)$ & -- & 4196\\
$(49,13)$ & 9 & $(17,5)$ & 6 & 1 & YES & YES & YES & $1.57$ & $(2,3)$ & -- & 4197\\
$(49,13)$ & 9 & $(17,7)$ & 6 & 1 & YES & YES & YES & $1.57$ & $(2,3)$ & -- & 4198\\
$(49,20)$ & 9 & $(17,5)$ & 6 & 1 & YES & YES & YES & $1.43$ & $(2,3)$ & -- & 4199\\
$(49,22)$ & 9 & $(18,5)$ & 6 & 1 & YES & YES & YES & $1.57$ & $(2,3)$ & -- & 4200\\
$(49,13)$ & 9 & $(19,6)$ & 8 & 1 & YES & YES & YES & $1.71$ & $(2,3)$ & -- & 4201\\
$(49,13)$ & 9 & $(19,8)$ & 6 & 1 & YES & YES & YES & $1.57$ & $(2,3)$ & -- & 4202\\
$(49,13)$ & 9 & $(19,8)$ & 6 & 1 & YES & YES & YES & $1.71$ & $(2,3)$ & NO & 4203\\
$(49,18)$ & 8 & $(19,8)$ & 6 & 1 & YES & YES & YES & $1.43$ & $(2,3)$ & NO & 4204\\
$(49,19)$ & 8 & $(19,7)$ & 6 & 1 & YES & YES & YES & $1.57$ & $(2,3)$ & NO & 4205\\
$(49,19)$ & 8 & $(19,7)$ & 6 & 1 & YES & YES & YES & $1.57$ & $(2,3)$ & -- & 4206\\
$(49,19)$ & 8 & $(19,8)$ & 6 & 1 & YES & YES & YES & $1.71$ & $(2,3)$ & -- & 4207\\
$(49,18)$ & 8 & $(21,8)$ & 6 & 7 & YES & YES & YES & $1.43$ & $(2,3)$ & -- & 4208\\
$(49,15)$ & 9 & $(23,5)$ & 7 & 1 & YES & YES & YES & $1.29$ & $(2,3)$ & -- & 4209\\
$(49,19)$ & 8 & $(23,7)$ & 7 & 1 & YES & YES & YES & $1.57$ & $(2,3)$ & -- & 4210\\
$(49,19)$ & 8 & $(23,10)$ & 7 & 1 & YES & YES & YES & $1.71$ & $(2,3)$ & -- & 4211\\
$(49,18)$ & 8 & $(24,7)$ & 7 & 1 & YES & YES & YES & $1.71$ & $(2,3)$ & -- & 4212\\
$(49,18)$ & 8 & $(26,11)$ & 7 & 1 & YES & YES & YES & $1.71$ & $(2,3)$ & NO & 4213\\
$(49,19)$ & 8 & $(26,7)$ & 7 & 1 & YES & YES & YES & $1.57$ & $(2,3)$ & -- & 4214\\
$(49,19)$ & 8 & $(29,12)$ & 7 & 1 & YES & YES & YES & $1.57$ & $(2,3)$ & -- & 4215\\
$(49,22)$ & 9 & $(29,12)$ & 7 & 1 & YES & YES & YES & $1.71$ & $(2,3)$ & NO & 4216\\
$(49,18)$ & 8 & $(30,7)$ & 8 & 1 & YES & YES & YES & $1.71$ & $(2,3)$ & NO & 4217\\
$(49,22)$ & 9 & $(31,13)$ & 7 & 1 & YES & YES & YES & $1.43$ & $(2,3)$ & NO & 4218\\
$(49,18)$ & 8 & $(34,13)$ & 7 & 1 & YES & YES & NO(2) & $1.38$ & $(2,3)$ & NO & 4219\\
$(49,13)$ & 9 & $(35,13)$ & 8 & 7 & YES & YES & YES & $1.86$ & $(2,3)$ & NO & 4220\\
$(49,19)$ & 8 & $(37,14)$ & 8 & 1 & YES & YES & YES & $1.57$ & $(2,3)$ & NO & 4221\\
$(49,15)$ & 9 & $(41,11)$ & 8 & 1 & YES & YES & YES & $1.57$ & $(2,3)$ & NO & 4222\\
$(49,18)$ & 8 & $(41,16)$ & 8 & 1 & YES & YES & YES & $1.43$ & $(2,3)$ & NO & 4223\\
$(49,20)$ & 9 & $(43,18)$ & 8 & 1 & YES & YES & NO(2) & $1.75$ & $(2,3)$ & NO & 4224\\
$(49,18)$ & 8 & $(44,17)$ & 8 & 1 & YES & YES & YES & $1.57$ & $(2,3)$ & NO & 4225\\
$(49,19)$ & 8 & $(46,17)$ & 8 & 1 & YES & YES & YES & $1.43$ & $(2,3)$ & NO & 4226\\
$(50,21)$ & 8 & $(8,3)$ & 4 & 2 & YES & YES & YES & $1.14$ & $(2,3)$ & -- & 4227\\
$(50,19)$ & 8 & $(10,3)$ & 5 & 10 & YES & YES & YES & $1.43$ & $(2,3)$ & -- & 4228\\
$(50,19)$ & 8 & $(11,5)$ & 6 & 1 & YES & YES & YES & $1.71$ & $(2,3)$ & -- & 4229\\
$(50,21)$ & 8 & $(11,3)$ & 5 & 1 & YES & YES & YES & $1.57$ & $(2,3)$ & NO & 4230\\
$(50,21)$ & 8 & $(11,3)$ & 5 & 1 & YES & YES & YES & $1.57$ & $(2,3)$ & -- & 4231\\
$(50,21)$ & 8 & $(11,5)$ & 6 & 1 & YES & YES & YES & $1.57$ & $(2,3)$ & -- & 4232\\
$(50,19)$ & 8 & $(12,5)$ & 5 & 2 & YES & YES & YES & $1.57$ & $(2,3)$ & -- & 4233\\
$(50,21)$ & 8 & $(12,5)$ & 5 & 2 & YES & YES & YES & $1.43$ & $(2,3)$ & -- & 4234\\
$(50,19)$ & 8 & $(13,5)$ & 5 & 1 & YES & YES & YES & $1.43$ & $(2,3)$ & -- & 4235\\
$(50,21)$ & 8 & $(13,5)$ & 5 & 1 & YES & YES & YES & $1.43$ & $(2,3)$ & -- & 4236\\
$(50,19)$ & 8 & $(16,7)$ & 6 & 2 & YES & YES & YES & $1.57$ & $(2,3)$ & NO & 4237\\
$(50,19)$ & 8 & $(16,7)$ & 6 & 2 & YES & YES & YES & $1.57$ & $(2,3)$ & -- & 4238\\
$(50,19)$ & 8 & $(17,7)$ & 6 & 1 & YES & YES & YES & $1.57$ & $(2,3)$ & -- & 4239\\
$(50,21)$ & 8 & $(17,7)$ & 6 & 1 & YES & YES & YES & $1.43$ & $(2,3)$ & -- & 4240\\
$(50,19)$ & 8 & $(19,7)$ & 6 & 1 & YES & YES & YES & $1.57$ & $(2,3)$ & -- & 4241\\
$(50,19)$ & 8 & $(19,7)$ & 6 & 1 & YES & YES & YES & $1.57$ & $(2,3)$ & NO & 4242\\
$(50,19)$ & 8 & $(19,8)$ & 6 & 1 & YES & YES & YES & $1.57$ & $(2,3)$ & -- & 4243\\
$(50,21)$ & 8 & $(19,7)$ & 6 & 1 & YES & YES & YES & $1.43$ & $(2,3)$ & -- & 4244\\
$(50,21)$ & 8 & $(19,7)$ & 6 & 1 & YES & YES & NO(2) & $1.62$ & $(2,3)$ & NO & 4245\\
$(50,19)$ & 8 & $(21,8)$ & 6 & 1 & YES & YES & YES & $1.43$ & $(2,3)$ & -- & 4246\\
$(50,19)$ & 8 & $(22,5)$ & 7 & 2 & YES & YES & YES & $1.43$ & $(2,3)$ & -- & 4247\\
$(50,19)$ & 8 & $(23,7)$ & 7 & 1 & YES & YES & YES & $1.57$ & $(2,3)$ & -- & 4248\\
$(50,21)$ & 8 & $(23,9)$ & 7 & 1 & YES & YES & YES & $1.43$ & $(2,3)$ & NO & 4249\\
$(50,19)$ & 8 & $(29,8)$ & 7 & 1 & YES & YES & YES & $1.57$ & $(2,3)$ & -- & 4250\\
$(50,19)$ & 8 & $(29,12)$ & 7 & 1 & YES & YES & YES & $1.57$ & $(2,3)$ & NO & 4251\\
$(50,21)$ & 8 & $(29,11)$ & 7 & 1 & YES & YES & YES & $1.57$ & $(2,3)$ & NO & 4252\\
$(50,21)$ & 8 & $(29,13)$ & 8 & 1 & YES & YES & YES & $1.43$ & $(2,3)$ & NO & 4253\\
$(50,19)$ & 8 & $(31,13)$ & 7 & 1 & YES & YES & YES & $1.43$ & $(2,3)$ & NO & 4254\\
$(50,21)$ & 8 & $(32,13)$ & 9 & 2 & YES & YES & YES & $1.57$ & $(2,3)$ & NO & 4255\\
$(50,21)$ & 8 & $(34,15)$ & 8 & 2 & YES & YES & YES & $1.57$ & $(2,3)$ & NO & 4256\\
$(50,19)$ & 8 & $(39,17)$ & 8 & 1 & YES & YES & YES & $1.57$ & $(2,3)$ & NO & 4257\\
$(50,21)$ & 8 & $(39,16)$ & 8 & 1 & YES & YES & YES & $1.57$ & $(2,3)$ & NO & 4258\\
$(50,19)$ & 8 & $(44,17)$ & 8 & 2 & YES & YES & YES & $1.57$ & $(2,3)$ & NO & 4259\\
$(50,21)$ & 8 & $(46,19)$ & 8 & 2 & YES & YES & NO(2) & $1.62$ & $(2,3)$ & NO & 4260\\
$(51,11)$ & 9 & $(5,2)$ & 3 & 1 & YES & YES & YES & $1.43$ & $(2,3)$ & -- & 4261\\
$(51,20)$ & 9 & $(8,3)$ & 4 & 1 & YES & YES & YES & $1.29$ & $(2,3)$ & -- & 4262\\
$(51,14)$ & 9 & $(11,3)$ & 5 & 1 & YES & YES & YES & $1.43$ & $(2,3)$ & -- & 4263\\
$(51,14)$ & 9 & $(12,5)$ & 5 & 3 & YES & YES & NO(2) & $1.57$ & $(4,2)$ & -- & 4264\\
$(51,16)$ & 10 & $(12,5)$ & 5 & 3 & YES & YES & YES & $1.57$ & $(2,3)$ & -- & 4265\\
$(51,16)$ & 10 & $(13,5)$ & 5 & 1 & YES & YES & YES & $1.71$ & $(2,3)$ & -- & 4266\\
$(51,11)$ & 9 & $(14,5)$ & 6 & 1 & YES & YES & YES & $1.57$ & $(2,3)$ & -- & 4267\\
$(51,23)$ & 9 & $(15,4)$ & 6 & 3 & YES & YES & YES & $1.57$ & $(2,3)$ & NO & 4268\\
$(51,23)$ & 9 & $(15,4)$ & 6 & 3 & YES & YES & YES & $1.57$ & $(2,3)$ & -- & 4269\\
$(51,14)$ & 9 & $(17,7)$ & 6 & 17 & YES & YES & YES & $1.71$ & $(2,3)$ & -- & 4270\\
$(51,16)$ & 10 & $(17,7)$ & 6 & 17 & YES & YES & YES & $1.71$ & $(2,3)$ & -- & 4271\\
$(51,14)$ & 9 & $(19,7)$ & 6 & 1 & YES & YES & YES & $1.57$ & $(2,3)$ & -- & 4272\\
$(51,14)$ & 9 & $(19,8)$ & 6 & 1 & YES & YES & YES & $1.71$ & $(2,3)$ & NO & 4273\\
$(51,16)$ & 10 & $(19,5)$ & 7 & 1 & YES & YES & YES & $1.71$ & $(2,3)$ & -- & 4274\\
$(51,14)$ & 9 & $(21,8)$ & 6 & 3 & YES & YES & YES & $1.57$ & $(2,3)$ & NO & 4275\\
$(51,11)$ & 9 & $(22,9)$ & 7 & 1 & YES & YES & YES & $1.57$ & $(2,3)$ & -- & 4276\\
$(51,23)$ & 9 & $(29,12)$ & 7 & 1 & YES & YES & YES & $1.71$ & $(2,3)$ & NO & 4277\\
$(51,11)$ & 9 & $(31,9)$ & 8 & 1 & YES & YES & YES & $1.71$ & $(2,3)$ & -- & 4278\\
$(51,11)$ & 9 & $(32,7)$ & 8 & 1 & YES & YES & YES & $1.43$ & $(2,3)$ & 5079 & 4279\\
$(51,11)$ & 9 & $(41,9)$ & 9 & 1 & YES & YES & YES & $1.57$ & $(2,3)$ & -- & 4280\\
$(51,14)$ & 9 & $(44,13)$ & 8 & 1 & YES & YES & YES & $1.71$ & $(2,3)$ & NO & 4281\\
$(52,19)$ & 9 & $(12,5)$ & 5 & 4 & YES & YES & NO(2) & $1.62$ & $(2,3)$ & -- & 4282\\
$(52,19)$ & 9 & $(13,4)$ & 6 & 13 & YES & YES & NO(2) & $1.62$ & $(2,3)$ & -- & 4283\\
$(52,19)$ & 9 & $(13,5)$ & 5 & 13 & YES & YES & YES & $1.57$ & $(2,3)$ & -- & 4284\\
$(52,19)$ & 9 & $(15,4)$ & 6 & 1 & YES & YES & NO(2) & $1.62$ & $(2,3)$ & -- & 4285\\
$(52,19)$ & 9 & $(19,7)$ & 6 & 1 & YES & YES & YES & $1.57$ & $(2,3)$ & -- & 4286\\
$(52,19)$ & 9 & $(23,7)$ & 7 & 1 & YES & YES & YES & $1.57$ & $(2,3)$ & -- & 4287\\
$(52,19)$ & 9 & $(29,12)$ & 7 & 1 & YES & YES & YES & $1.57$ & $(2,3)$ & NO & 4288\\
$(52,19)$ & 9 & $(41,16)$ & 8 & 1 & YES & YES & YES & $1.57$ & $(2,3)$ & NO & 4289\\
$(53,19)$ & 9 & $(8,3)$ & 4 & 1 & YES & YES & YES & $1.29$ & $(2,3)$ & -- & 4290\\
$(53,22)$ & 9 & $(10,3)$ & 5 & 1 & YES & YES & YES & $1.43$ & $(2,3)$ & -- & 4291\\
$(53,12)$ & 9 & $(12,5)$ & 5 & 1 & YES & YES & YES & $1.57$ & $(2,3)$ & NO & 4292\\
$(53,12)$ & 9 & $(12,5)$ & 5 & 1 & YES & YES & YES & $1.57$ & $(2,3)$ & -- & 4293\\
$(53,16)$ & 10 & $(12,5)$ & 5 & 1 & YES & YES & NO(2) & $1.62$ & $(2,3)$ & -- & 4294\\
$(53,23)$ & 9 & $(12,5)$ & 5 & 1 & YES & YES & NO(2) & $1.50$ & $(2,3)$ & -- & 4295\\
$(53,14)$ & 9 & $(13,5)$ & 5 & 1 & YES & YES & YES & $1.57$ & $(2,3)$ & NO & 4296\\
$(53,14)$ & 9 & $(13,5)$ & 5 & 1 & YES & YES & YES & $1.57$ & $(2,3)$ & -- & 4297\\
$(53,16)$ & 10 & $(13,5)$ & 5 & 1 & YES & YES & NO(2) & $1.62$ & $(2,3)$ & -- & 4298\\
$(53,19)$ & 9 & $(13,4)$ & 6 & 1 & YES & YES & YES & $1.71$ & $(2,3)$ & -- & 4299\\
$(53,22)$ & 9 & $(13,5)$ & 5 & 1 & YES & YES & YES & $1.71$ & $(2,3)$ & -- & 4300\\
$(53,23)$ & 9 & $(13,4)$ & 6 & 1 & YES & YES & NO(2) & $1.75$ & $(2,3)$ & -- & 4301\\
$(53,19)$ & 9 & $(14,5)$ & 6 & 1 & YES & YES & YES & $1.57$ & $(2,3)$ & -- & 4302\\
$(53,12)$ & 9 & $(16,5)$ & 7 & 1 & YES & YES & YES & $1.57$ & $(2,3)$ & -- & 4303\\
$(53,11)$ & 10 & $(17,7)$ & 6 & 1 & YES & YES & YES & $1.57$ & $(2,3)$ & NO & 4304\\
$(53,12)$ & 9 & $(17,5)$ & 6 & 1 & YES & YES & YES & $1.57$ & $(2,3)$ & NO & 4305\\
$(53,12)$ & 9 & $(17,5)$ & 6 & 1 & YES & YES & YES & $1.57$ & $(2,3)$ & -- & 4306\\
$(53,19)$ & 9 & $(17,5)$ & 6 & 1 & YES & YES & YES & $1.43$ & $(2,3)$ & -- & 4307\\
$(53,22)$ & 9 & $(17,5)$ & 6 & 1 & YES & YES & YES & $1.57$ & $(2,3)$ & -- & 4308\\
$(53,22)$ & 9 & $(17,5)$ & 6 & 1 & YES & YES & YES & $1.57$ & $(2,3)$ & NO & 4309\\
$(53,23)$ & 9 & $(17,5)$ & 6 & 1 & YES & YES & YES & $1.43$ & $(2,3)$ & -- & 4310\\
$(53,11)$ & 10 & $(19,8)$ & 6 & 1 & YES & YES & YES & $1.57$ & $(2,3)$ & NO & 4311\\
$(53,19)$ & 9 & $(19,5)$ & 7 & 1 & YES & YES & YES & $1.57$ & $(2,3)$ & -- & 4312\\
$(53,11)$ & 10 & $(21,8)$ & 6 & 1 & YES & YES & YES & $1.57$ & $(2,3)$ & NO & 4313\\
$(53,19)$ & 9 & $(21,5)$ & 8 & 1 & YES & YES & YES & $1.71$ & $(2,3)$ & -- & 4314\\
$(53,16)$ & 10 & $(22,5)$ & 7 & 1 & YES & YES & NO(2) & $1.75$ & $(2,3)$ & NO & 4315\\
$(53,16)$ & 10 & $(29,8)$ & 7 & 1 & YES & YES & YES & $1.57$ & $(2,3)$ & NO & 4316\\
$(53,19)$ & 9 & $(29,11)$ & 7 & 1 & YES & YES & YES & $1.57$ & $(2,3)$ & NO & 4317\\
$(53,12)$ & 9 & $(32,9)$ & 8 & 1 & YES & YES & YES & $1.57$ & $(2,3)$ & -- & 4318\\
$(53,19)$ & 9 & $(34,13)$ & 7 & 1 & YES & YES & YES & $1.43$ & $(2,3)$ & NO & 4319\\
$(53,12)$ & 9 & $(41,12)$ & 8 & 1 & YES & YES & YES & $1.43$ & $(2,3)$ & -- & 4320\\
$(53,12)$ & 9 & $(43,12)$ & 8 & 1 & YES & YES & YES & $1.43$ & $(2,3)$ & -- & 4321\\
$(53,19)$ & 9 & $(43,16)$ & 9 & 1 & YES & YES & YES & $1.71$ & $(2,3)$ & NO & 4322\\
$(55,21)$ & 8 & $(7,2)$ & 4 & 1 & YES & YES & YES & $1.43$ & $(2,3)$ & -- & 4323\\
$(55,21)$ & 8 & $(7,2)$ & 4 & 1 & YES & YES & YES & $1.57$ & $(2,3)$ & NO & 4324\\
$(55,16)$ & 9 & $(8,3)$ & 4 & 1 & YES & YES & NO(2) & $1.62$ & $(2,3)$ & -- & 4325\\
$(55,21)$ & 8 & $(8,3)$ & 4 & 1 & YES & YES & YES & $1.29$ & $(2,3)$ & -- & 4326\\
$(55,21)$ & 8 & $(9,2)$ & 5 & 1 & YES & YES & YES & $1.43$ & $(2,3)$ & NO & 4327\\
$(55,21)$ & 8 & $(9,2)$ & 5 & 1 & YES & YES & YES & $1.43$ & $(2,3)$ & -- & 4328\\
$(55,16)$ & 9 & $(10,3)$ & 5 & 5 & YES & YES & YES & $1.57$ & $(2,3)$ & -- & 4329\\
$(55,21)$ & 8 & $(10,3)$ & 5 & 5 & YES & YES & NO(2) & $1.50$ & $(2,3)$ & NO & 4330\\
$(55,21)$ & 8 & $(10,3)$ & 5 & 5 & YES & YES & NO(2) & $1.50$ & $(2,3)$ & -- & 4331\\
$(55,21)$ & 8 & $(11,5)$ & 6 & 11 & YES & YES & YES & $1.71$ & $(2,3)$ & -- & 4332\\
$(55,23)$ & 9 & $(11,3)$ & 5 & 11 & YES & YES & YES & $1.43$ & $(2,3)$ & -- & 4333\\
$(55,21)$ & 8 & $(12,5)$ & 5 & 1 & YES & YES & YES & $1.43$ & $(2,3)$ & -- & 4334\\
$(55,23)$ & 9 & $(12,5)$ & 5 & 1 & YES & YES & YES & $1.57$ & $(2,3)$ & NO & 4335\\
$(55,23)$ & 9 & $(12,5)$ & 5 & 1 & YES & YES & YES & $1.57$ & $(2,3)$ & -- & 4336\\
$(55,24)$ & 9 & $(12,5)$ & 5 & 1 & YES & YES & YES & $1.57$ & $(2,3)$ & -- & 4337\\
$(55,16)$ & 9 & $(13,3)$ & 6 & 1 & YES & YES & NO(2) & $1.67$ & $(2,3)$ & -- & 4338\\
$(55,21)$ & 8 & $(13,5)$ & 5 & 1 & YES & YES & YES & $1.43$ & $(2,3)$ & -- & 4339\\
$(55,16)$ & 9 & $(14,5)$ & 6 & 1 & YES & YES & NO(2) & $1.62$ & $(2,3)$ & -- & 4340\\
$(55,16)$ & 9 & $(16,7)$ & 6 & 1 & YES & YES & NO(2) & $1.62$ & $(2,3)$ & NO & 4341\\
$(55,21)$ & 8 & $(16,7)$ & 6 & 1 & YES & YES & YES & $1.43$ & $(2,3)$ & -- & 4342\\
$(55,17)$ & 10 & $(17,5)$ & 6 & 1 & YES & YES & YES & $1.43$ & $(2,3)$ & -- & 4343\\
$(55,23)$ & 9 & $(17,5)$ & 6 & 1 & YES & YES & YES & $1.57$ & $(2,3)$ & -- & 4344\\
$(55,23)$ & 9 & $(17,7)$ & 6 & 1 & YES & YES & YES & $1.71$ & $(2,3)$ & -- & 4345\\
$(55,23)$ & 9 & $(18,5)$ & 6 & 1 & YES & YES & YES & $1.43$ & $(2,3)$ & NO & 4346\\
$(55,23)$ & 9 & $(18,5)$ & 6 & 1 & YES & YES & YES & $1.57$ & $(2,3)$ & -- & 4347\\
$(55,21)$ & 8 & $(19,8)$ & 6 & 1 & YES & YES & YES & $1.43$ & $(2,3)$ & -- & 4348\\
$(55,21)$ & 8 & $(20,9)$ & 7 & 5 & YES & YES & YES & $1.43$ & $(2,3)$ & NO & 4349\\
$(55,23)$ & 9 & $(20,9)$ & 7 & 5 & YES & YES & YES & $1.57$ & $(2,3)$ & NO & 4350\\
$(55,16)$ & 9 & $(21,5)$ & 8 & 1 & YES & YES & NO(2) & $1.62$ & $(2,3)$ & -- & 4351\\
$(55,23)$ & 9 & $(22,5)$ & 7 & 11 & YES & YES & YES & $1.57$ & $(2,3)$ & -- & 4352\\
$(55,16)$ & 9 & $(23,7)$ & 7 & 1 & YES & YES & NO(2) & $1.67$ & $(2,3)$ & NO & 4353\\
$(55,23)$ & 9 & $(23,5)$ & 7 & 1 & YES & YES & YES & $1.57$ & $(2,3)$ & -- & 4354\\
$(55,16)$ & 9 & $(24,7)$ & 7 & 1 & YES & YES & YES & $1.57$ & $(2,3)$ & -- & 4355\\
$(55,21)$ & 8 & $(25,9)$ & 7 & 5 & YES & YES & YES & $1.43$ & $(2,3)$ & NO & 4356\\
$(55,21)$ & 8 & $(26,11)$ & 7 & 1 & YES & YES & YES & $1.71$ & $(2,3)$ & NO & 4357\\
$(55,21)$ & 8 & $(30,11)$ & 7 & 5 & YES & YES & NO(2) & $1.38$ & $(2,3)$ & NO & 4358\\
$(55,17)$ & 10 & $(31,9)$ & 8 & 1 & YES & YES & NO(2) & $1.62$ & $(2,3)$ & NO & 4359\\
$(55,24)$ & 9 & $(31,14)$ & 8 & 1 & YES & YES & YES & $1.57$ & $(2,3)$ & NO & 4360\\
$(55,16)$ & 9 & $(37,11)$ & 8 & 1 & YES & YES & YES & $1.57$ & $(2,3)$ & 4562 & 4361\\
$(55,17)$ & 10 & $(41,12)$ & 8 & 1 & YES & YES & YES & $1.43$ & $(2,3)$ & NO & 4362\\
$(55,21)$ & 8 & $(41,15)$ & 8 & 1 & YES & YES & YES & $1.71$ & $(2,3)$ & NO & 4363\\
$(55,16)$ & 9 & $(43,13)$ & 9 & 1 & YES & YES & NO(2) & $1.50$ & $(2,3)$ & NO & 4364\\
$(55,16)$ & 9 & $(44,13)$ & 8 & 11 & YES & YES & YES & $1.43$ & $(2,3)$ & NO & 4365\\
$(56,17)$ & 9 & $(8,3)$ & 4 & 8 & YES & YES & YES & $1.29$ & $(2,3)$ & -- & 4366\\
$(56,17)$ & 9 & $(8,3)$ & 4 & 8 & YES & YES & YES & $1.43$ & $(2,3)$ & NO & 4367\\
$(56,23)$ & 9 & $(10,3)$ & 5 & 2 & YES & YES & YES & $1.57$ & $(2,3)$ & -- & 4368\\
$(56,15)$ & 9 & $(12,5)$ & 5 & 4 & YES & YES & YES & $1.29$ & $(2,3)$ & -- & 4369\\
$(56,17)$ & 9 & $(12,5)$ & 5 & 4 & YES & YES & YES & $1.29$ & $(2,3)$ & -- & 4370\\
$(56,23)$ & 9 & $(12,5)$ & 5 & 4 & YES & YES & NO(2) & $1.75$ & $(2,3)$ & -- & 4371\\
$(56,13)$ & 10 & $(13,5)$ & 5 & 1 & YES & YES & YES & $1.43$ & $(2,3)$ & NO & 4372\\
$(56,13)$ & 10 & $(13,5)$ & 5 & 1 & YES & YES & YES & $1.43$ & $(2,3)$ & -- & 4373\\
$(56,23)$ & 9 & $(13,4)$ & 6 & 1 & YES & YES & YES & $1.57$ & $(2,3)$ & -- & 4374\\
$(56,23)$ & 9 & $(13,5)$ & 5 & 1 & YES & YES & NO(2) & $1.75$ & $(2,3)$ & -- & 4375\\
$(56,17)$ & 9 & $(15,4)$ & 6 & 1 & YES & YES & YES & $1.43$ & $(2,3)$ & NO & 4376\\
$(56,23)$ & 9 & $(15,4)$ & 6 & 1 & YES & YES & YES & $1.57$ & $(2,3)$ & -- & 4377\\
$(56,23)$ & 9 & $(15,4)$ & 6 & 1 & YES & YES & YES & $1.71$ & $(2,3)$ & NO & 4378\\
$(56,13)$ & 10 & $(16,7)$ & 6 & 8 & YES & YES & NO(2) & $1.50$ & $(2,3)$ & NO & 4379\\
$(56,13)$ & 10 & $(16,7)$ & 6 & 8 & YES & YES & NO(2) & $1.50$ & $(2,3)$ & -- & 4380\\
$(56,23)$ & 9 & $(16,7)$ & 6 & 8 & YES & YES & YES & $1.71$ & $(2,3)$ & -- & 4381\\
$(56,15)$ & 9 & $(17,4)$ & 7 & 1 & YES & YES & NO(2) & $1.62$ & $(2,3)$ & -- & 4382\\
$(56,15)$ & 9 & $(17,5)$ & 6 & 1 & YES & YES & YES & $1.43$ & $(2,3)$ & -- & 4383\\
$(56,23)$ & 9 & $(17,5)$ & 6 & 1 & YES & YES & YES & $1.71$ & $(2,3)$ & -- & 4384\\
$(56,23)$ & 9 & $(17,5)$ & 6 & 1 & YES & YES & NO(2) & $1.75$ & $(2,3)$ & NO & 4385\\
$(56,23)$ & 9 & $(17,7)$ & 6 & 1 & YES & YES & YES & $1.71$ & $(2,3)$ & -- & 4386\\
$(56,13)$ & 10 & $(19,7)$ & 6 & 1 & YES & YES & NO(2) & $1.50$ & $(2,3)$ & NO & 4387\\
$(56,13)$ & 10 & $(19,8)$ & 6 & 1 & YES & YES & YES & $1.43$ & $(2,3)$ & NO & 4388\\
$(56,15)$ & 9 & $(19,7)$ & 6 & 1 & YES & YES & YES & $1.57$ & $(2,3)$ & -- & 4389\\
$(56,15)$ & 9 & $(19,8)$ & 6 & 1 & YES & YES & YES & $1.57$ & $(2,3)$ & -- & 4390\\
$(56,23)$ & 9 & $(19,7)$ & 6 & 1 & YES & YES & YES & $1.71$ & $(2,3)$ & -- & 4391\\
$(56,13)$ & 10 & $(21,8)$ & 6 & 7 & YES & YES & YES & $1.43$ & $(2,3)$ & NO & 4392\\
$(56,15)$ & 9 & $(21,8)$ & 6 & 7 & YES & YES & YES & $1.57$ & $(2,3)$ & NO & 4393\\
$(56,17)$ & 9 & $(21,8)$ & 6 & 7 & YES & YES & YES & $1.57$ & $(2,3)$ & -- & 4394\\
$(56,23)$ & 9 & $(21,5)$ & 8 & 7 & YES & YES & YES & $1.57$ & $(2,3)$ & -- & 4395\\
$(56,13)$ & 10 & $(23,7)$ & 7 & 1 & YES & YES & NO(2) & $1.50$ & $(2,3)$ & NO & 4396\\
$(56,17)$ & 9 & $(26,7)$ & 7 & 2 & YES & YES & YES & $1.57$ & $(2,3)$ & -- & 4397\\
$(56,17)$ & 9 & $(26,7)$ & 7 & 2 & YES & YES & NO(2) & $1.38$ & $(2,3)$ & NO & 4398\\
$(56,17)$ & 9 & $(40,11)$ & 8 & 8 & YES & YES & YES & $1.57$ & $(2,3)$ & NO & 4399\\
$(56,23)$ & 9 & $(50,21)$ & 8 & 2 & YES & YES & YES & $1.43$ & $(2,3)$ & NO & 4400\\
$(56,13)$ & 10 & $(55,12)$ & 9 & 1 & YES & YES & NO(2) & $1.50$ & $(2,3)$ & NO & 4401\\
$(57,16)$ & 9 & $(7,3)$ & 4 & 1 & YES & YES & YES & $1.57$ & $(2,3)$ & -- & 4402\\
$(57,16)$ & 9 & $(10,3)$ & 5 & 1 & YES & YES & YES & $1.57$ & $(2,3)$ & -- & 4403\\
$(57,17)$ & 10 & $(10,3)$ & 5 & 1 & YES & YES & YES & $1.57$ & $(2,3)$ & -- & 4404\\
$(57,16)$ & 9 & $(11,3)$ & 5 & 1 & YES & YES & YES & $1.43$ & $(2,3)$ & -- & 4405\\
$(57,16)$ & 9 & $(11,3)$ & 5 & 1 & YES & YES & YES & $1.57$ & $(2,3)$ & NO & 4406\\
$(57,25)$ & 9 & $(12,5)$ & 5 & 3 & YES & YES & YES & $1.43$ & $(2,3)$ & -- & 4407\\
$(57,17)$ & 10 & $(13,5)$ & 5 & 1 & YES & YES & YES & $1.43$ & $(2,3)$ & -- & 4408\\
$(57,22)$ & 9 & $(14,3)$ & 6 & 1 & YES & YES & NO(2) & $1.75$ & $(2,3)$ & NO & 4409\\
$(57,22)$ & 9 & $(14,3)$ & 6 & 1 & YES & YES & NO(2) & $1.75$ & $(2,3)$ & -- & 4410\\
$(57,22)$ & 9 & $(15,4)$ & 6 & 3 & YES & YES & YES & $1.57$ & $(2,3)$ & -- & 4411\\
$(57,22)$ & 9 & $(15,4)$ & 6 & 3 & YES & YES & YES & $1.57$ & $(2,3)$ & NO & 4412\\
$(57,16)$ & 9 & $(16,5)$ & 7 & 1 & YES & YES & YES & $1.57$ & $(2,3)$ & -- & 4413\\
$(57,13)$ & 9 & $(18,7)$ & 6 & 3 & YES & YES & YES & $1.57$ & $(2,3)$ & -- & 4414\\
$(57,22)$ & 9 & $(18,5)$ & 6 & 3 & YES & YES & YES & $1.57$ & $(2,3)$ & -- & 4415\\
$(57,22)$ & 9 & $(18,5)$ & 6 & 3 & YES & YES & YES & $1.71$ & $(2,3)$ & NO & 4416\\
$(57,25)$ & 9 & $(18,7)$ & 6 & 3 & YES & YES & YES & $1.71$ & $(2,3)$ & NO & 4417\\
$(57,17)$ & 10 & $(22,5)$ & 7 & 1 & YES & YES & YES & $1.57$ & $(2,3)$ & NO & 4418\\
$(57,22)$ & 9 & $(26,5)$ & 9 & 1 & YES & YES & YES & $1.71$ & $(2,3)$ & -- & 4419\\
$(57,13)$ & 9 & $(33,10)$ & 8 & 3 & YES & YES & YES & $1.57$ & $(2,3)$ & NO & 4420\\
$(57,25)$ & 9 & $(33,14)$ & 8 & 3 & YES & YES & YES & $1.57$ & $(2,3)$ & NO & 4421\\
$(57,16)$ & 9 & $(40,11)$ & 8 & 1 & YES & YES & YES & $1.57$ & $(2,3)$ & 4684 & 4422\\
$(57,17)$ & 10 & $(42,13)$ & 9 & 3 & YES & YES & YES & $1.57$ & $(2,3)$ & NO & 4423\\
$(57,16)$ & 9 & $(47,13)$ & 8 & 1 & YES & YES & YES & $1.43$ & $(2,3)$ & NO & 4424\\
$(57,25)$ & 9 & $(53,23)$ & 9 & 1 & YES & YES & NO(2) & $1.50$ & $(2,3)$ & NO & 4425\\
$(57,17)$ & 10 & $(55,16)$ & 9 & 1 & YES & YES & YES & $1.57$ & $(2,3)$ & NO & 4426\\
$(58,13)$ & 11 & $(13,5)$ & 5 & 1 & YES & YES & YES & $1.71$ & $(2,3)$ & NO & 4427\\
$(58,17)$ & 9 & $(14,5)$ & 6 & 2 & YES & YES & YES & $1.71$ & $(2,3)$ & -- & 4428\\
$(58,17)$ & 9 & $(16,7)$ & 6 & 2 & YES & YES & YES & $1.57$ & $(2,3)$ & -- & 4429\\
$(58,13)$ & 11 & $(17,5)$ & 6 & 1 & YES & YES & YES & $1.71$ & $(2,3)$ & NO & 4430\\
$(58,13)$ & 11 & $(17,7)$ & 6 & 1 & YES & YES & YES & $1.71$ & $(2,3)$ & NO & 4431\\
$(58,13)$ & 11 & $(19,8)$ & 6 & 1 & YES & YES & YES & $1.71$ & $(2,3)$ & -- & 4432\\
$(58,17)$ & 9 & $(19,7)$ & 6 & 1 & YES & YES & YES & $1.57$ & $(2,3)$ & -- & 4433\\
$(58,17)$ & 9 & $(19,7)$ & 6 & 1 & YES & YES & YES & $1.57$ & $(2,3)$ & NO & 4434\\
$(58,13)$ & 11 & $(25,7)$ & 7 & 1 & YES & YES & YES & $1.71$ & $(2,3)$ & NO & 4435\\
$(58,17)$ & 9 & $(30,7)$ & 8 & 2 & YES & YES & YES & $1.57$ & $(2,3)$ & -- & 4436\\
$(58,17)$ & 9 & $(43,13)$ & 9 & 1 & YES & YES & YES & $1.57$ & $(2,3)$ & NO & 4437\\
$(59,13)$ & 11 & $(12,5)$ & 5 & 1 & YES & YES & YES & $1.43$ & $(2,3)$ & NO & 4438\\
$(59,25)$ & 9 & $(12,5)$ & 5 & 1 & YES & YES & NO(2) & $1.50$ & $(2,3)$ & -- & 4439\\
$(59,26)$ & 9 & $(12,5)$ & 5 & 1 & YES & YES & YES & $1.57$ & $(2,3)$ & -- & 4440\\
$(59,18)$ & 9 & $(13,5)$ & 5 & 1 & YES & YES & NO(2) & $1.50$ & $(2,3)$ & -- & 4441\\
$(59,23)$ & 9 & $(13,4)$ & 6 & 1 & YES & YES & NO(2) & $1.62$ & $(2,3)$ & -- & 4442\\
$(59,26)$ & 9 & $(13,4)$ & 6 & 1 & YES & YES & YES & $1.71$ & $(2,3)$ & -- & 4443\\
$(59,16)$ & 10 & $(14,3)$ & 6 & 1 & YES & YES & YES & $1.57$ & $(2,3)$ & -- & 4444\\
$(59,18)$ & 9 & $(18,7)$ & 6 & 1 & YES & YES & YES & $1.57$ & $(2,3)$ & -- & 4445\\
$(59,18)$ & 9 & $(18,7)$ & 6 & 1 & YES & YES & YES & $1.57$ & $(2,3)$ & NO & 4446\\
$(59,26)$ & 9 & $(18,7)$ & 6 & 1 & YES & YES & YES & $1.57$ & $(2,3)$ & NO & 4447\\
$(59,11)$ & 10 & $(19,6)$ & 8 & 1 & YES & YES & YES & $1.71$ & $(2,3)$ & -- & 4448\\
$(59,26)$ & 9 & $(22,9)$ & 7 & 1 & YES & YES & YES & $1.57$ & $(2,3)$ & NO & 4449\\
$(59,26)$ & 9 & $(26,7)$ & 7 & 1 & YES & YES & YES & $1.71$ & $(2,3)$ & NO & 4450\\
$(59,23)$ & 9 & $(30,11)$ & 7 & 1 & YES & YES & NO(2) & $1.62$ & $(2,3)$ & NO & 4451\\
$(59,11)$ & 10 & $(32,9)$ & 8 & 1 & YES & YES & YES & $1.43$ & $(2,3)$ & -- & 4452\\
$(59,13)$ & 11 & $(32,5)$ & 9 & 1 & YES & YES & YES & $1.57$ & $(2,3)$ & NO & 4453\\
$(59,25)$ & 9 & $(39,16)$ & 8 & 1 & YES & YES & NO(2) & $1.75$ & $(2,3)$ & NO & 4454\\
$(59,25)$ & 9 & $(41,17)$ & 8 & 1 & YES & YES & NO(2) & $1.75$ & $(2,3)$ & NO & 4455\\
$(59,18)$ & 9 & $(47,14)$ & 9 & 1 & YES & YES & YES & $1.43$ & $(2,3)$ & NO & 4456\\
$(59,18)$ & 9 & $(51,16)$ & 10 & 1 & YES & YES & YES & $1.57$ & $(2,3)$ & NO & 4457\\
$(60,13)$ & 9 & $(5,2)$ & 3 & 5 & YES & YES & YES & $1.57$ & $(2,3)$ & NO & 4458\\
$(60,13)$ & 9 & $(5,2)$ & 3 & 5 & YES & YES & YES & $1.57$ & $(2,3)$ & -- & 4459\\
$(60,23)$ & 9 & $(11,3)$ & 5 & 1 & YES & YES & YES & $1.43$ & $(2,3)$ & -- & 4460\\
$(60,23)$ & 9 & $(12,5)$ & 5 & 12 & YES & YES & NO(2) & $1.75$ & $(2,3)$ & -- & 4461\\
$(60,23)$ & 9 & $(14,3)$ & 6 & 2 & YES & YES & YES & $1.43$ & $(2,3)$ & NO & 4462\\
$(60,23)$ & 9 & $(17,5)$ & 6 & 1 & YES & YES & YES & $1.57$ & $(2,3)$ & -- & 4463\\
$(60,23)$ & 9 & $(18,5)$ & 6 & 6 & YES & YES & YES & $1.57$ & $(2,3)$ & NO & 4464\\
$(60,23)$ & 9 & $(18,5)$ & 6 & 6 & YES & YES & YES & $1.71$ & $(2,3)$ & -- & 4465\\
$(60,23)$ & 9 & $(22,5)$ & 7 & 2 & YES & YES & YES & $1.71$ & $(2,3)$ & -- & 4466\\
$(60,23)$ & 9 & $(23,5)$ & 7 & 1 & YES & YES & YES & $1.71$ & $(2,3)$ & -- & 4467\\
$(60,23)$ & 9 & $(23,5)$ & 7 & 1 & YES & YES & YES & $1.57$ & $(2,3)$ & NO & 4468\\
$(60,23)$ & 9 & $(27,5)$ & 8 & 3 & YES & YES & YES & $1.57$ & $(2,3)$ & NO & 4469\\
$(60,23)$ & 9 & $(27,5)$ & 8 & 3 & YES & YES & YES & $1.71$ & $(2,3)$ & -- & 4470\\
$(61,22)$ & 9 & $(8,3)$ & 4 & 1 & YES & YES & YES & $1.43$ & $(2,3)$ & -- & 4471\\
$(61,22)$ & 9 & $(10,3)$ & 5 & 1 & YES & YES & NO(2) & $1.62$ & $(2,3)$ & -- & 4472\\
$(61,25)$ & 9 & $(12,5)$ & 5 & 1 & YES & YES & YES & $1.57$ & $(2,3)$ & -- & 4473\\
$(61,17)$ & 9 & $(13,5)$ & 5 & 1 & YES & YES & YES & $1.57$ & $(2,3)$ & NO & 4474\\
$(61,17)$ & 9 & $(13,5)$ & 5 & 1 & YES & YES & YES & $1.57$ & $(2,3)$ & -- & 4475\\
$(61,25)$ & 9 & $(13,5)$ & 5 & 1 & YES & YES & YES & $1.57$ & $(2,3)$ & -- & 4476\\
$(61,25)$ & 9 & $(14,5)$ & 6 & 1 & YES & YES & YES & $1.57$ & $(2,3)$ & -- & 4477\\
$(61,17)$ & 9 & $(17,7)$ & 6 & 1 & YES & YES & YES & $1.71$ & $(2,3)$ & -- & 4478\\
$(61,18)$ & 9 & $(17,4)$ & 7 & 1 & YES & YES & YES & $1.43$ & $(2,3)$ & -- & 4479\\
$(61,25)$ & 9 & $(18,5)$ & 6 & 1 & YES & YES & YES & $1.57$ & $(2,3)$ & NO & 4480\\
$(61,18)$ & 9 & $(19,8)$ & 6 & 1 & YES & YES & YES & $1.43$ & $(2,3)$ & -- & 4481\\
$(61,22)$ & 9 & $(19,5)$ & 7 & 1 & YES & YES & YES & $1.57$ & $(2,3)$ & -- & 4482\\
$(61,16)$ & 10 & $(21,5)$ & 8 & 1 & YES & YES & YES & $1.57$ & $(2,3)$ & -- & 4483\\
$(61,25)$ & 9 & $(21,5)$ & 8 & 1 & YES & YES & YES & $1.57$ & $(2,3)$ & -- & 4484\\
$(61,18)$ & 9 & $(23,5)$ & 7 & 1 & YES & YES & YES & $1.43$ & $(2,3)$ & NO & 4485\\
$(61,25)$ & 9 & $(31,13)$ & 7 & 1 & YES & YES & YES & $1.43$ & $(2,3)$ & NO & 4486\\
$(61,22)$ & 9 & $(37,14)$ & 8 & 1 & YES & YES & YES & $1.57$ & $(2,3)$ & NO & 4487\\
$(61,18)$ & 9 & $(41,11)$ & 8 & 1 & YES & YES & YES & $1.43$ & $(2,3)$ & NO & 4488\\
$(61,18)$ & 9 & $(47,13)$ & 8 & 1 & YES & YES & YES & $1.43$ & $(2,3)$ & NO & 4489\\
$(61,22)$ & 9 & $(53,19)$ & 9 & 1 & YES & YES & YES & $1.43$ & $(2,3)$ & NO & 4490\\
$(62,17)$ & 10 & $(7,2)$ & 4 & 1 & YES & YES & YES & $1.57$ & $(2,3)$ & NO & 4491\\
$(62,17)$ & 10 & $(10,3)$ & 5 & 2 & YES & YES & YES & $1.71$ & $(2,3)$ & -- & 4492\\
$(62,23)$ & 9 & $(11,3)$ & 5 & 1 & YES & YES & YES & $1.43$ & $(2,3)$ & -- & 4493\\
$(62,23)$ & 9 & $(11,4)$ & 5 & 1 & YES & YES & YES & $1.43$ & $(2,3)$ & -- & 4494\\
$(62,27)$ & 9 & $(11,5)$ & 6 & 1 & YES & YES & YES & $1.57$ & $(2,3)$ & -- & 4495\\
$(62,17)$ & 10 & $(12,5)$ & 5 & 2 & YES & YES & YES & $1.57$ & $(2,3)$ & -- & 4496\\
$(62,17)$ & 10 & $(12,5)$ & 5 & 2 & YES & YES & YES & $1.57$ & $(2,3)$ & NO & 4497\\
$(62,17)$ & 10 & $(13,5)$ & 5 & 1 & YES & YES & YES & $1.57$ & $(2,3)$ & -- & 4498\\
$(62,17)$ & 10 & $(13,5)$ & 5 & 1 & YES & YES & YES & $1.57$ & $(2,3)$ & NO & 4499\\
$(62,19)$ & 10 & $(13,5)$ & 5 & 1 & YES & YES & YES & $1.43$ & $(2,3)$ & -- & 4500\\
$(62,27)$ & 9 & $(13,4)$ & 6 & 1 & YES & YES & NO(2) & $1.62$ & $(2,3)$ & -- & 4501\\
$(62,23)$ & 9 & $(15,4)$ & 6 & 1 & YES & YES & NO(2) & $1.62$ & $(2,3)$ & -- & 4502\\
$(62,27)$ & 9 & $(15,4)$ & 6 & 1 & YES & YES & YES & $1.57$ & $(2,3)$ & -- & 4503\\
$(62,17)$ & 10 & $(16,7)$ & 6 & 2 & YES & YES & YES & $1.57$ & $(2,3)$ & NO & 4504\\
$(62,23)$ & 9 & $(17,4)$ & 7 & 1 & YES & YES & NO(2) & $1.62$ & $(2,3)$ & NO & 4505\\
$(62,23)$ & 9 & $(17,4)$ & 7 & 1 & YES & YES & NO(2) & $1.62$ & $(2,3)$ & -- & 4506\\
$(62,11)$ & 10 & $(19,6)$ & 8 & 1 & YES & YES & YES & $1.57$ & $(2,3)$ & -- & 4507\\
$(62,23)$ & 9 & $(19,7)$ & 6 & 1 & YES & YES & YES & $1.71$ & $(2,3)$ & -- & 4508\\
$(62,23)$ & 9 & $(19,8)$ & 6 & 1 & YES & YES & YES & $1.71$ & $(2,3)$ & -- & 4509\\
$(62,27)$ & 9 & $(22,9)$ & 7 & 2 & YES & YES & NO(2) & $1.50$ & $(2,3)$ & NO & 4510\\
$(62,17)$ & 10 & $(23,7)$ & 7 & 1 & YES & YES & YES & $1.57$ & $(2,3)$ & NO & 4511\\
$(62,19)$ & 10 & $(23,5)$ & 7 & 1 & YES & YES & YES & $1.43$ & $(2,3)$ & NO & 4512\\
$(62,23)$ & 9 & $(29,12)$ & 7 & 1 & YES & YES & YES & $1.57$ & $(2,3)$ & NO & 4513\\
$(62,27)$ & 9 & $(29,11)$ & 7 & 1 & YES & YES & YES & $1.71$ & $(2,3)$ & NO & 4514\\
$(62,13)$ & 10 & $(31,9)$ & 8 & 31 & YES & YES & YES & $1.71$ & $(2,3)$ & NO & 4515\\
$(62,17)$ & 10 & $(41,12)$ & 8 & 1 & YES & YES & YES & $1.57$ & $(2,3)$ & NO & 4516\\
$(62,23)$ & 9 & $(41,15)$ & 8 & 1 & YES & YES & YES & $1.43$ & $(2,3)$ & 4752 & 4517\\
$(62,23)$ & 9 & $(41,16)$ & 8 & 1 & YES & YES & YES & $1.57$ & $(2,3)$ & NO & 4518\\
$(62,17)$ & 10 & $(49,13)$ & 9 & 1 & YES & YES & YES & $1.71$ & $(2,3)$ & NO & 4519\\
$(62,23)$ & 9 & $(52,19)$ & 9 & 2 & YES & YES & NO(2) & $1.75$ & $(2,3)$ & NO & 4520\\
$(62,17)$ & 10 & $(57,16)$ & 9 & 1 & YES & YES & YES & $1.57$ & $(2,3)$ & NO & 4521\\
$(63,17)$ & 9 & $(7,2)$ & 4 & 7 & YES & YES & YES & $1.29$ & $(2,3)$ & -- & 4522\\
$(63,17)$ & 9 & $(7,2)$ & 4 & 7 & YES & YES & YES & $1.43$ & $(2,3)$ & NO & 4523\\
$(63,17)$ & 9 & $(10,3)$ & 5 & 1 & YES & YES & YES & $1.57$ & $(2,3)$ & NO & 4524\\
$(63,26)$ & 9 & $(11,4)$ & 5 & 1 & YES & YES & YES & $1.57$ & $(2,3)$ & -- & 4525\\
$(63,26)$ & 9 & $(12,5)$ & 5 & 3 & YES & YES & YES & $1.43$ & $(2,3)$ & -- & 4526\\
$(63,17)$ & 9 & $(13,3)$ & 6 & 1 & YES & YES & YES & $1.43$ & $(2,3)$ & -- & 4527\\
$(63,17)$ & 9 & $(13,3)$ & 6 & 1 & YES & YES & YES & $1.57$ & $(2,3)$ & NO & 4528\\
$(63,26)$ & 9 & $(13,4)$ & 6 & 1 & YES & YES & YES & $1.71$ & $(2,3)$ & -- & 4529\\
$(63,26)$ & 9 & $(13,5)$ & 5 & 1 & YES & YES & YES & $1.57$ & $(2,3)$ & -- & 4530\\
$(63,26)$ & 9 & $(14,5)$ & 6 & 7 & YES & YES & YES & $1.71$ & $(2,3)$ & -- & 4531\\
$(63,26)$ & 9 & $(15,4)$ & 6 & 3 & YES & YES & YES & $1.57$ & $(2,3)$ & -- & 4532\\
$(63,26)$ & 9 & $(16,7)$ & 6 & 1 & YES & YES & YES & $1.57$ & $(2,3)$ & -- & 4533\\
$(63,17)$ & 9 & $(17,7)$ & 6 & 1 & YES & YES & YES & $1.57$ & $(2,3)$ & -- & 4534\\
$(63,26)$ & 9 & $(17,5)$ & 6 & 1 & YES & YES & YES & $1.43$ & $(2,3)$ & NO & 4535\\
$(63,26)$ & 9 & $(17,5)$ & 6 & 1 & YES & YES & YES & $1.43$ & $(2,3)$ & -- & 4536\\
$(63,26)$ & 9 & $(18,5)$ & 6 & 9 & YES & YES & YES & $1.57$ & $(2,3)$ & -- & 4537\\
$(63,26)$ & 9 & $(18,7)$ & 6 & 9 & YES & YES & YES & $1.71$ & $(2,3)$ & -- & 4538\\
$(63,26)$ & 9 & $(20,9)$ & 7 & 1 & YES & YES & YES & $1.43$ & $(2,3)$ & NO & 4539\\
$(63,17)$ & 9 & $(21,8)$ & 6 & 21 & YES & YES & YES & $1.43$ & $(2,3)$ & -- & 4540\\
$(63,17)$ & 9 & $(23,9)$ & 7 & 1 & YES & YES & YES & $1.57$ & $(2,3)$ & NO & 4541\\
$(63,26)$ & 9 & $(23,5)$ & 7 & 1 & YES & YES & YES & $1.43$ & $(2,3)$ & NO & 4542\\
$(63,17)$ & 9 & $(44,13)$ & 8 & 1 & YES & YES & YES & $1.57$ & $(2,3)$ & NO & 4543\\
$(64,19)$ & 9 & $(7,2)$ & 4 & 1 & YES & YES & NO(2) & $1.62$ & $(2,3)$ & -- & 4544\\
$(64,19)$ & 9 & $(7,3)$ & 4 & 1 & YES & YES & YES & $1.57$ & $(2,3)$ & -- & 4545\\
$(64,27)$ & 9 & $(7,3)$ & 4 & 1 & YES & YES & YES & $1.57$ & $(2,3)$ & -- & 4546\\
$(64,19)$ & 9 & $(11,5)$ & 6 & 1 & YES & YES & YES & $1.57$ & $(2,3)$ & -- & 4547\\
$(64,17)$ & 10 & $(13,4)$ & 6 & 1 & YES & YES & NO(2) & $1.62$ & $(2,3)$ & -- & 4548\\
$(64,19)$ & 9 & $(13,3)$ & 6 & 1 & YES & YES & YES & $1.57$ & $(2,3)$ & NO & 4549\\
$(64,19)$ & 9 & $(13,3)$ & 6 & 1 & YES & YES & YES & $1.57$ & $(2,3)$ & -- & 4550\\
$(64,19)$ & 9 & $(13,4)$ & 6 & 1 & YES & YES & YES & $1.57$ & $(2,3)$ & -- & 4551\\
$(64,25)$ & 9 & $(13,4)$ & 6 & 1 & YES & YES & YES & $1.71$ & $(2,3)$ & -- & 4552\\
$(64,25)$ & 9 & $(13,4)$ & 6 & 1 & YES & YES & YES & $1.57$ & $(2,3)$ & NO & 4553\\
$(64,27)$ & 9 & $(13,5)$ & 5 & 1 & YES & YES & YES & $1.71$ & $(2,3)$ & -- & 4554\\
$(64,23)$ & 9 & $(15,4)$ & 6 & 1 & YES & YES & YES & $1.71$ & $(2,3)$ & -- & 4555\\
$(64,23)$ & 9 & $(15,4)$ & 6 & 1 & YES & YES & YES & $1.57$ & $(2,3)$ & NO & 4556\\
$(64,25)$ & 9 & $(15,4)$ & 6 & 1 & YES & YES & YES & $1.71$ & $(2,3)$ & -- & 4557\\
$(64,25)$ & 9 & $(17,6)$ & 7 & 1 & YES & YES & YES & $1.57$ & $(2,3)$ & NO & 4558\\
$(64,19)$ & 9 & $(19,6)$ & 8 & 1 & YES & YES & YES & $1.57$ & $(2,3)$ & -- & 4559\\
$(64,27)$ & 9 & $(22,9)$ & 7 & 2 & YES & YES & YES & $1.57$ & $(2,3)$ & NO & 4560\\
$(64,19)$ & 9 & $(25,7)$ & 7 & 1 & YES & YES & YES & $1.57$ & $(2,3)$ & NO & 4561\\
$(64,19)$ & 9 & $(31,9)$ & 8 & 1 & YES & YES & YES & $1.57$ & $(2,3)$ & 4361 & 4562\\
$(64,25)$ & 9 & $(37,14)$ & 8 & 1 & YES & YES & YES & $1.71$ & $(2,3)$ & NO & 4563\\
$(64,19)$ & 9 & $(41,11)$ & 8 & 1 & YES & YES & YES & $1.43$ & $(2,3)$ & NO & 4564\\
$(64,19)$ & 9 & $(45,13)$ & 10 & 1 & YES & YES & YES & $1.57$ & $(2,3)$ & NO & 4565\\
$(64,25)$ & 9 & $(47,18)$ & 8 & 1 & YES & YES & YES & $1.71$ & $(2,3)$ & NO & 4566\\
$(64,25)$ & 9 & $(50,19)$ & 8 & 2 & YES & YES & YES & $1.57$ & $(2,3)$ & 5426 & 4567\\
$(64,15)$ & 10 & $(55,12)$ & 9 & 1 & YES & YES & YES & $1.71$ & $(2,3)$ & NO & 4568\\
$(64,19)$ & 9 & $(61,17)$ & 9 & 1 & YES & YES & YES & $1.57$ & $(2,3)$ & NO & 4569\\
$(65,18)$ & 9 & $(7,2)$ & 4 & 1 & YES & YES & YES & $1.43$ & $(2,3)$ & -- & 4570\\
$(65,18)$ & 9 & $(7,2)$ & 4 & 1 & YES & YES & YES & $1.57$ & $(2,3)$ & NO & 4571\\
$(65,18)$ & 9 & $(7,3)$ & 4 & 1 & YES & YES & YES & $1.57$ & $(2,3)$ & NO & 4572\\
$(65,18)$ & 9 & $(7,3)$ & 4 & 1 & YES & YES & YES & $1.57$ & $(2,3)$ & -- & 4573\\
$(65,24)$ & 9 & $(8,3)$ & 4 & 1 & YES & YES & YES & $1.57$ & $(2,3)$ & -- & 4574\\
$(65,18)$ & 9 & $(9,2)$ & 5 & 1 & YES & YES & YES & $1.57$ & $(2,3)$ & NO & 4575\\
$(65,18)$ & 9 & $(9,2)$ & 5 & 1 & YES & YES & YES & $1.57$ & $(2,3)$ & -- & 4576\\
$(65,14)$ & 10 & $(12,5)$ & 5 & 1 & YES & YES & YES & $1.29$ & $(2,3)$ & -- & 4577\\
$(65,19)$ & 9 & $(12,5)$ & 5 & 1 & YES & YES & YES & $1.57$ & $(2,3)$ & -- & 4578\\
$(65,19)$ & 9 & $(13,4)$ & 6 & 13 & YES & YES & YES & $1.43$ & $(2,3)$ & -- & 4579\\
$(65,19)$ & 9 & $(15,4)$ & 6 & 5 & YES & YES & YES & $1.43$ & $(2,3)$ & -- & 4580\\
$(65,18)$ & 9 & $(17,4)$ & 7 & 1 & YES & YES & YES & $1.57$ & $(2,3)$ & -- & 4581\\
$(65,19)$ & 9 & $(17,5)$ & 6 & 1 & YES & YES & YES & $1.57$ & $(2,3)$ & -- & 4582\\
$(65,27)$ & 10 & $(17,5)$ & 6 & 1 & YES & YES & YES & $1.71$ & $(2,3)$ & -- & 4583\\
$(65,18)$ & 9 & $(18,7)$ & 6 & 1 & YES & YES & YES & $1.71$ & $(2,3)$ & NO & 4584\\
$(65,27)$ & 10 & $(18,5)$ & 6 & 1 & YES & YES & YES & $1.71$ & $(2,3)$ & -- & 4585\\
$(65,19)$ & 9 & $(19,7)$ & 6 & 1 & YES & YES & YES & $1.57$ & $(2,3)$ & -- & 4586\\
$(65,19)$ & 9 & $(27,8)$ & 7 & 1 & YES & YES & YES & $1.43$ & $(2,3)$ & NO & 4587\\
$(65,24)$ & 9 & $(31,12)$ & 7 & 1 & YES & YES & YES & $1.57$ & $(2,3)$ & NO & 4588\\
$(65,18)$ & 9 & $(33,10)$ & 8 & 1 & YES & YES & YES & $1.71$ & $(2,3)$ & NO & 4589\\
$(65,19)$ & 9 & $(35,11)$ & 9 & 5 & YES & YES & YES & $1.57$ & $(2,3)$ & NO & 4590\\
$(65,24)$ & 9 & $(41,16)$ & 8 & 1 & YES & YES & YES & $1.71$ & $(2,3)$ & NO & 4591\\
$(65,19)$ & 9 & $(64,19)$ & 9 & 1 & YES & YES & YES & $1.57$ & $(2,3)$ & NO & 4592\\
$(66,25)$ & 9 & $(5,2)$ & 3 & 1 & YES & YES & NO(2) & $1.56$ & $(2,3)$ & NO & 4593\\
$(66,25)$ & 9 & $(5,2)$ & 3 & 1 & YES & YES & NO(2) & $1.56$ & $(2,3)$ & -- & 4594\\
$(66,25)$ & 9 & $(7,2)$ & 4 & 1 & YES & YES & YES & $1.43$ & $(2,3)$ & -- & 4595\\
$(66,29)$ & 9 & $(8,3)$ & 4 & 2 & YES & YES & YES & $1.57$ & $(2,3)$ & -- & 4596\\
$(66,25)$ & 9 & $(10,3)$ & 5 & 2 & YES & YES & YES & $1.57$ & $(2,3)$ & -- & 4597\\
$(66,29)$ & 9 & $(10,3)$ & 5 & 2 & YES & YES & YES & $1.57$ & $(2,3)$ & -- & 4598\\
$(66,25)$ & 9 & $(13,4)$ & 6 & 1 & YES & YES & YES & $1.57$ & $(2,3)$ & -- & 4599\\
$(66,25)$ & 9 & $(13,5)$ & 5 & 1 & YES & YES & YES & $1.57$ & $(2,3)$ & -- & 4600\\
$(66,25)$ & 9 & $(15,4)$ & 6 & 3 & YES & YES & YES & $1.43$ & $(2,3)$ & -- & 4601\\
$(66,25)$ & 9 & $(16,7)$ & 6 & 2 & YES & YES & YES & $1.71$ & $(2,3)$ & -- & 4602\\
$(66,25)$ & 9 & $(17,5)$ & 6 & 1 & YES & YES & YES & $1.43$ & $(2,3)$ & -- & 4603\\
$(66,25)$ & 9 & $(19,8)$ & 6 & 1 & YES & YES & YES & $1.43$ & $(2,3)$ & 7643 & 4604\\
$(66,25)$ & 9 & $(21,5)$ & 8 & 3 & YES & YES & YES & $1.71$ & $(2,3)$ & -- & 4605\\
$(66,25)$ & 9 & $(29,12)$ & 7 & 1 & YES & YES & YES & $1.57$ & $(2,3)$ & NO & 4606\\
$(66,25)$ & 9 & $(36,13)$ & 8 & 6 & YES & YES & YES & $1.71$ & $(2,3)$ & NO & 4607\\
$(66,25)$ & 9 & $(41,16)$ & 8 & 1 & YES & YES & YES & $1.43$ & $(2,3)$ & NO & 4608\\
$(66,25)$ & 9 & $(49,19)$ & 8 & 1 & YES & YES & YES & $1.57$ & $(2,3)$ & NO & 4609\\
$(66,25)$ & 9 & $(55,21)$ & 8 & 11 & YES & YES & YES & $1.43$ & $(2,3)$ & NO & 4610\\
$(66,25)$ & 9 & $(60,23)$ & 9 & 6 & YES & YES & YES & $1.57$ & $(2,3)$ & NO & 4611\\
$(67,18)$ & 9 & $(5,2)$ & 3 & 1 & YES & YES & YES & $1.29$ & $(2,3)$ & -- & 4612\\
$(67,18)$ & 9 & $(5,2)$ & 3 & 1 & YES & YES & YES & $1.43$ & $(2,3)$ & NO & 4613\\
$(67,18)$ & 9 & $(7,3)$ & 4 & 1 & YES & YES & YES & $1.43$ & $(2,3)$ & -- & 4614\\
$(67,26)$ & 9 & $(7,3)$ & 4 & 1 & YES & YES & YES & $1.57$ & $(2,3)$ & -- & 4615\\
$(67,26)$ & 9 & $(7,3)$ & 4 & 1 & YES & YES & YES & $1.57$ & $(2,3)$ & NO & 4616\\
$(67,18)$ & 9 & $(8,3)$ & 4 & 1 & YES & YES & NO(2) & $1.38$ & $(2,3)$ & NO & 4617\\
$(67,28)$ & 10 & $(8,3)$ & 4 & 1 & YES & YES & YES & $1.71$ & $(2,3)$ & -- & 4618\\
$(67,18)$ & 9 & $(11,4)$ & 5 & 1 & YES & YES & YES & $1.43$ & $(2,3)$ & -- & 4619\\
$(67,28)$ & 10 & $(11,3)$ & 5 & 1 & YES & YES & YES & $1.43$ & $(2,3)$ & -- & 4620\\
$(67,26)$ & 9 & $(12,5)$ & 5 & 1 & YES & YES & YES & $1.71$ & $(2,3)$ & -- & 4621\\
$(67,18)$ & 9 & $(13,5)$ & 5 & 1 & YES & YES & YES & $1.43$ & $(2,3)$ & -- & 4622\\
$(67,26)$ & 9 & $(13,4)$ & 6 & 1 & YES & YES & YES & $1.71$ & $(2,3)$ & -- & 4623\\
$(67,26)$ & 9 & $(14,5)$ & 6 & 1 & YES & YES & YES & $1.57$ & $(2,3)$ & -- & 4624\\
$(67,26)$ & 9 & $(14,5)$ & 6 & 1 & YES & YES & YES & $1.57$ & $(2,3)$ & NO & 4625\\
$(67,28)$ & 10 & $(14,3)$ & 6 & 1 & YES & YES & YES & $1.43$ & $(2,3)$ & NO & 4626\\
$(67,26)$ & 9 & $(15,4)$ & 6 & 1 & YES & YES & YES & $1.57$ & $(2,3)$ & -- & 4627\\
$(67,26)$ & 9 & $(16,7)$ & 6 & 1 & YES & YES & YES & $1.71$ & $(2,3)$ & -- & 4628\\
$(67,18)$ & 9 & $(17,7)$ & 6 & 1 & YES & YES & YES & $1.57$ & $(2,3)$ & -- & 4629\\
$(67,18)$ & 9 & $(17,7)$ & 6 & 1 & YES & YES & YES & $1.57$ & $(2,3)$ & NO & 4630\\
$(67,18)$ & 9 & $(18,7)$ & 6 & 1 & YES & YES & YES & $1.57$ & $(2,3)$ & NO & 4631\\
$(67,18)$ & 9 & $(18,7)$ & 6 & 1 & YES & YES & YES & $1.71$ & $(2,3)$ & -- & 4632\\
$(67,26)$ & 9 & $(18,5)$ & 6 & 1 & YES & YES & YES & $1.57$ & $(2,3)$ & NO & 4633\\
$(67,26)$ & 9 & $(19,5)$ & 7 & 1 & YES & YES & YES & $1.57$ & $(2,3)$ & -- & 4634\\
$(67,28)$ & 10 & $(20,3)$ & 8 & 1 & YES & YES & YES & $1.71$ & $(2,3)$ & NO & 4635\\
$(67,26)$ & 9 & $(21,5)$ & 8 & 1 & YES & YES & YES & $1.71$ & $(2,3)$ & -- & 4636\\
$(67,26)$ & 9 & $(26,5)$ & 9 & 1 & YES & YES & YES & $1.71$ & $(2,3)$ & -- & 4637\\
$(67,26)$ & 9 & $(30,11)$ & 7 & 1 & YES & YES & YES & $1.57$ & $(2,3)$ & NO & 4638\\
$(67,18)$ & 9 & $(37,11)$ & 8 & 1 & YES & YES & YES & $1.57$ & $(2,3)$ & NO & 4639\\
$(67,18)$ & 9 & $(48,13)$ & 9 & 1 & YES & YES & YES & $1.43$ & $(2,3)$ & NO & 4640\\
$(67,18)$ & 9 & $(61,17)$ & 9 & 1 & YES & YES & YES & $1.43$ & $(2,3)$ & NO & 4641\\
$(68,19)$ & 9 & $(5,2)$ & 3 & 1 & YES & YES & YES & $1.43$ & $(2,3)$ & -- & 4642\\
$(68,19)$ & 9 & $(5,2)$ & 3 & 1 & YES & YES & YES & $1.57$ & $(2,3)$ & NO & 4643\\
$(68,19)$ & 9 & $(7,2)$ & 4 & 1 & YES & YES & YES & $1.43$ & $(2,3)$ & -- & 4644\\
$(68,19)$ & 9 & $(7,3)$ & 4 & 1 & YES & YES & YES & $1.71$ & $(2,3)$ & NO & 4645\\
$(68,25)$ & 9 & $(8,3)$ & 4 & 4 & YES & YES & NO(3) & $1.29$ & $(2,3)$ & -- & 4646\\
$(68,13)$ & 11 & $(9,4)$ & 5 & 1 & YES & YES & YES & $1.86$ & $(2,3)$ & -- & 4647\\
$(68,19)$ & 9 & $(10,3)$ & 5 & 2 & YES & YES & YES & $1.57$ & $(2,3)$ & -- & 4648\\
$(68,19)$ & 9 & $(11,5)$ & 6 & 1 & YES & YES & YES & $1.71$ & $(2,3)$ & NO & 4649\\
$(68,19)$ & 9 & $(11,5)$ & 6 & 1 & YES & YES & YES & $1.71$ & $(2,3)$ & -- & 4650\\
$(68,19)$ & 9 & $(12,5)$ & 5 & 4 & YES & YES & YES & $1.71$ & $(2,3)$ & -- & 4651\\
$(68,25)$ & 9 & $(12,5)$ & 5 & 4 & YES & YES & NO(2) & $1.75$ & $(2,3)$ & -- & 4652\\
$(68,19)$ & 9 & $(13,4)$ & 6 & 1 & YES & YES & YES & $1.43$ & $(2,3)$ & -- & 4653\\
$(68,19)$ & 9 & $(13,5)$ & 5 & 1 & YES & YES & YES & $1.57$ & $(2,3)$ & NO & 4654\\
$(68,19)$ & 9 & $(13,5)$ & 5 & 1 & YES & YES & YES & $1.57$ & $(2,3)$ & -- & 4655\\
$(68,25)$ & 9 & $(13,5)$ & 5 & 1 & YES & YES & YES & $1.57$ & $(2,3)$ & -- & 4656\\
$(68,19)$ & 9 & $(15,4)$ & 6 & 1 & YES & YES & YES & $1.43$ & $(2,3)$ & -- & 4657\\
$(68,19)$ & 9 & $(17,5)$ & 6 & 17 & YES & YES & YES & $1.43$ & $(2,3)$ & -- & 4658\\
$(68,19)$ & 9 & $(18,5)$ & 6 & 2 & YES & YES & YES & $1.57$ & $(2,3)$ & -- & 4659\\
$(68,25)$ & 9 & $(19,8)$ & 6 & 1 & YES & YES & YES & $1.57$ & $(2,3)$ & NO & 4660\\
$(68,25)$ & 9 & $(21,8)$ & 6 & 1 & YES & YES & YES & $1.57$ & $(2,3)$ & NO & 4661\\
$(68,19)$ & 9 & $(26,7)$ & 7 & 2 & YES & YES & YES & $1.57$ & $(2,3)$ & NO & 4662\\
$(68,19)$ & 9 & $(29,8)$ & 7 & 1 & YES & YES & YES & $1.43$ & $(2,3)$ & NO & 4663\\
$(68,25)$ & 9 & $(29,11)$ & 7 & 1 & YES & YES & YES & $1.57$ & $(2,3)$ & NO & 4664\\
$(68,25)$ & 9 & $(31,12)$ & 7 & 1 & YES & YES & YES & $1.71$ & $(2,3)$ & NO & 4665\\
$(68,25)$ & 9 & $(34,13)$ & 7 & 34 & YES & YES & YES & $1.57$ & $(2,3)$ & NO & 4666\\
$(69,19)$ & 9 & $(7,2)$ & 4 & 1 & YES & YES & YES & $1.29$ & $(2,3)$ & -- & 4667\\
$(69,19)$ & 9 & $(7,2)$ & 4 & 1 & YES & YES & YES & $1.57$ & $(2,3)$ & NO & 4668\\
$(69,19)$ & 9 & $(7,3)$ & 4 & 1 & YES & YES & YES & $1.57$ & $(2,3)$ & -- & 4669\\
$(69,29)$ & 9 & $(8,3)$ & 4 & 1 & YES & YES & YES & $1.71$ & $(2,3)$ & -- & 4670\\
$(69,29)$ & 9 & $(9,4)$ & 5 & 3 & YES & YES & YES & $1.71$ & $(2,3)$ & -- & 4671\\
$(69,16)$ & 11 & $(12,5)$ & 5 & 3 & YES & YES & NO(2) & $1.50$ & $(2,3)$ & NO & 4672\\
$(69,19)$ & 9 & $(13,3)$ & 6 & 1 & YES & YES & YES & $1.43$ & $(2,3)$ & -- & 4673\\
$(69,19)$ & 9 & $(13,3)$ & 6 & 1 & YES & YES & YES & $1.57$ & $(2,3)$ & NO & 4674\\
$(69,29)$ & 9 & $(13,3)$ & 6 & 1 & YES & YES & YES & $1.43$ & $(2,3)$ & -- & 4675\\
$(69,29)$ & 9 & $(13,5)$ & 5 & 1 & YES & YES & YES & $1.57$ & $(2,3)$ & -- & 4676\\
$(69,19)$ & 9 & $(17,7)$ & 6 & 1 & YES & YES & YES & $1.57$ & $(2,3)$ & -- & 4677\\
$(69,16)$ & 11 & $(19,7)$ & 6 & 1 & YES & YES & YES & $1.71$ & $(2,3)$ & -- & 4678\\
$(69,19)$ & 9 & $(19,8)$ & 6 & 1 & YES & YES & YES & $1.71$ & $(2,3)$ & -- & 4679\\
$(69,16)$ & 11 & $(21,4)$ & 8 & 3 & YES & YES & NO(2) & $1.50$ & $(2,3)$ & -- & 4680\\
$(69,20)$ & 10 & $(21,4)$ & 8 & 3 & YES & YES & NO(2) & $1.75$ & $(2,3)$ & NO & 4681\\
$(69,29)$ & 9 & $(21,8)$ & 6 & 3 & YES & YES & YES & $1.57$ & $(2,3)$ & NO & 4682\\
$(69,29)$ & 9 & $(30,13)$ & 8 & 3 & YES & YES & YES & $1.71$ & $(2,3)$ & NO & 4683\\
$(69,19)$ & 9 & $(32,9)$ & 8 & 1 & YES & YES & YES & $1.57$ & $(2,3)$ & 4422 & 4684\\
$(69,29)$ & 9 & $(32,13)$ & 9 & 1 & YES & YES & YES & $1.71$ & $(2,3)$ & NO & 4685\\
$(69,20)$ & 10 & $(33,10)$ & 8 & 3 & YES & YES & NO(2) & $1.75$ & $(2,3)$ & 7800 & 4686\\
$(69,29)$ & 9 & $(39,16)$ & 8 & 3 & YES & YES & YES & $1.57$ & $(2,3)$ & NO & 4687\\
$(69,19)$ & 9 & $(44,13)$ & 8 & 1 & YES & YES & YES & $1.71$ & $(2,3)$ & NO & 4688\\
$(70,29)$ & 9 & $(7,3)$ & 4 & 7 & YES & YES & YES & $1.43$ & $(2,3)$ & -- & 4689\\
$(70,29)$ & 9 & $(8,3)$ & 4 & 2 & YES & YES & YES & $1.57$ & $(2,3)$ & NO & 4690\\
$(70,29)$ & 9 & $(9,2)$ & 5 & 1 & YES & YES & YES & $1.57$ & $(2,3)$ & NO & 4691\\
$(70,29)$ & 9 & $(9,2)$ & 5 & 1 & YES & YES & YES & $1.57$ & $(2,3)$ & -- & 4692\\
$(70,29)$ & 9 & $(9,2)$ & 5 & 1 & YES & YES & YES & $1.57$ & $(2,3)$ & NO & 4693\\
$(70,29)$ & 9 & $(10,3)$ & 5 & 10 & YES & YES & YES & $1.57$ & $(2,3)$ & -- & 4694\\
$(70,29)$ & 9 & $(12,5)$ & 5 & 2 & YES & YES & YES & $1.57$ & $(2,3)$ & -- & 4695\\
$(70,29)$ & 9 & $(13,4)$ & 6 & 1 & YES & YES & NO(2) & $1.62$ & $(2,3)$ & -- & 4696\\
$(70,29)$ & 9 & $(13,5)$ & 5 & 1 & YES & YES & NO(2) & $1.56$ & $(2,3)$ & NO & 4697\\
$(70,27)$ & 10 & $(19,4)$ & 7 & 1 & YES & YES & YES & $1.71$ & $(2,3)$ & -- & 4698\\
$(70,29)$ & 9 & $(19,7)$ & 6 & 1 & YES & YES & NO(2) & $1.62$ & $(2,3)$ & NO & 4699\\
$(70,29)$ & 9 & $(30,13)$ & 8 & 10 & YES & YES & YES & $1.57$ & $(2,3)$ & NO & 4700\\
$(70,29)$ & 9 & $(39,17)$ & 8 & 1 & YES & YES & YES & $1.57$ & $(2,3)$ & NO & 4701\\
$(70,29)$ & 9 & $(45,19)$ & 8 & 5 & YES & YES & YES & $1.71$ & $(2,3)$ & NO & 4702\\
$(70,29)$ & 9 & $(56,23)$ & 9 & 14 & YES & YES & NO(2) & $1.50$ & $(2,3)$ & NO & 4703\\
$(71,21)$ & 9 & $(7,3)$ & 4 & 1 & YES & YES & YES & $1.43$ & $(2,3)$ & -- & 4704\\
$(71,26)$ & 9 & $(7,3)$ & 4 & 1 & YES & YES & YES & $1.43$ & $(2,3)$ & -- & 4705\\
$(71,27)$ & 9 & $(7,2)$ & 4 & 1 & YES & YES & YES & $1.57$ & $(2,3)$ & NO & 4706\\
$(71,27)$ & 9 & $(7,2)$ & 4 & 1 & YES & YES & YES & $1.57$ & $(2,3)$ & -- & 4707\\
$(71,27)$ & 9 & $(7,3)$ & 4 & 1 & YES & YES & YES & $1.57$ & $(2,3)$ & -- & 4708\\
$(71,30)$ & 9 & $(7,3)$ & 4 & 1 & YES & YES & NO(2) & $1.62$ & $(2,3)$ & -- & 4709\\
$(71,21)$ & 9 & $(8,3)$ & 4 & 1 & YES & YES & NO(2) & $1.50$ & $(2,3)$ & NO & 4710\\
$(71,21)$ & 9 & $(8,3)$ & 4 & 1 & YES & YES & NO(2) & $1.50$ & $(2,3)$ & -- & 4711\\
$(71,26)$ & 9 & $(9,4)$ & 5 & 1 & YES & YES & YES & $1.71$ & $(2,3)$ & -- & 4712\\
$(71,22)$ & 10 & $(10,3)$ & 5 & 1 & YES & YES & YES & $1.57$ & $(2,3)$ & NO & 4713\\
$(71,22)$ & 10 & $(10,3)$ & 5 & 1 & YES & YES & YES & $1.57$ & $(2,3)$ & -- & 4714\\
$(71,26)$ & 9 & $(10,3)$ & 5 & 1 & YES & YES & NO(2) & $1.62$ & $(2,3)$ & -- & 4715\\
$(71,30)$ & 9 & $(10,3)$ & 5 & 1 & YES & YES & NO(2) & $1.62$ & $(2,3)$ & NO & 4716\\
$(71,30)$ & 9 & $(10,3)$ & 5 & 1 & YES & YES & YES & $1.57$ & $(2,3)$ & NO & 4717\\
$(71,30)$ & 9 & $(10,3)$ & 5 & 1 & YES & YES & YES & $1.57$ & $(2,3)$ & -- & 4718\\
$(71,32)$ & 10 & $(10,3)$ & 5 & 1 & YES & YES & YES & $1.57$ & $(2,3)$ & -- & 4719\\
$(71,21)$ & 9 & $(11,3)$ & 5 & 1 & YES & YES & YES & $1.43$ & $(2,3)$ & -- & 4720\\
$(71,21)$ & 9 & $(11,5)$ & 6 & 1 & YES & YES & YES & $1.43$ & $(2,3)$ & -- & 4721\\
$(71,26)$ & 9 & $(11,5)$ & 6 & 1 & YES & YES & YES & $1.71$ & $(2,3)$ & NO & 4722\\
$(71,30)$ & 9 & $(11,3)$ & 5 & 1 & YES & YES & NO(2) & $1.62$ & $(2,3)$ & -- & 4723\\
$(71,30)$ & 9 & $(11,4)$ & 5 & 1 & YES & YES & YES & $1.43$ & $(2,3)$ & -- & 4724\\
$(71,21)$ & 9 & $(13,5)$ & 5 & 1 & YES & YES & YES & $1.43$ & $(2,3)$ & -- & 4725\\
$(71,26)$ & 9 & $(13,4)$ & 6 & 1 & YES & YES & NO(2) & $1.50$ & $(2,3)$ & -- & 4726\\
$(71,26)$ & 9 & $(13,5)$ & 5 & 1 & YES & YES & YES & $1.43$ & $(2,3)$ & -- & 4727\\
$(71,27)$ & 9 & $(13,5)$ & 5 & 1 & YES & YES & YES & $1.71$ & $(2,3)$ & -- & 4728\\
$(71,16)$ & 10 & $(14,5)$ & 6 & 1 & YES & YES & NO(2) & $1.62$ & $(2,3)$ & -- & 4729\\
$(71,32)$ & 10 & $(14,3)$ & 6 & 1 & YES & YES & YES & $1.57$ & $(2,3)$ & NO & 4730\\
$(71,32)$ & 10 & $(14,3)$ & 6 & 1 & YES & YES & YES & $1.57$ & $(2,3)$ & -- & 4731\\
$(71,32)$ & 10 & $(14,3)$ & 6 & 1 & YES & YES & YES & $1.57$ & $(2,3)$ & NO & 4732\\
$(71,21)$ & 9 & $(15,4)$ & 6 & 1 & YES & YES & YES & $1.43$ & $(2,3)$ & NO & 4733\\
$(71,16)$ & 10 & $(16,5)$ & 7 & 1 & YES & YES & YES & $1.57$ & $(2,3)$ & -- & 4734\\
$(71,16)$ & 10 & $(17,7)$ & 6 & 1 & YES & YES & YES & $1.71$ & $(2,3)$ & NO & 4735\\
$(71,16)$ & 10 & $(17,7)$ & 6 & 1 & YES & YES & YES & $1.71$ & $(2,3)$ & -- & 4736\\
$(71,21)$ & 9 & $(17,4)$ & 7 & 1 & YES & YES & YES & $1.57$ & $(2,3)$ & -- & 4737\\
$(71,21)$ & 9 & $(17,7)$ & 6 & 1 & YES & YES & YES & $1.43$ & $(2,3)$ & -- & 4738\\
$(71,21)$ & 9 & $(17,7)$ & 6 & 1 & YES & YES & YES & $1.57$ & $(2,3)$ & NO & 4739\\
$(71,27)$ & 9 & $(17,5)$ & 6 & 1 & YES & YES & YES & $1.57$ & $(2,3)$ & -- & 4740\\
$(71,13)$ & 12 & $(18,5)$ & 6 & 1 & YES & YES & YES & $1.57$ & $(2,3)$ & NO & 4741\\
$(71,21)$ & 9 & $(18,5)$ & 6 & 1 & YES & YES & YES & $1.57$ & $(2,3)$ & NO & 4742\\
$(71,27)$ & 9 & $(18,5)$ & 6 & 1 & YES & YES & YES & $1.57$ & $(2,3)$ & -- & 4743\\
$(71,26)$ & 9 & $(21,8)$ & 6 & 1 & YES & YES & YES & $1.43$ & $(2,3)$ & NO & 4744\\
$(71,16)$ & 10 & $(22,9)$ & 7 & 1 & YES & YES & YES & $1.71$ & $(2,3)$ & -- & 4745\\
$(71,27)$ & 9 & $(22,5)$ & 7 & 1 & YES & YES & YES & $1.57$ & $(2,3)$ & -- & 4746\\
$(71,30)$ & 9 & $(22,9)$ & 7 & 1 & YES & YES & NO(2) & $1.50$ & $(2,3)$ & NO & 4747\\
$(71,26)$ & 9 & $(23,9)$ & 7 & 1 & YES & YES & NO(2) & $1.50$ & $(2,3)$ & NO & 4748\\
$(71,27)$ & 9 & $(23,9)$ & 7 & 1 & YES & YES & YES & $1.57$ & $(2,3)$ & NO & 4749\\
$(71,32)$ & 10 & $(23,10)$ & 7 & 1 & YES & YES & YES & $1.71$ & $(2,3)$ & NO & 4750\\
$(71,16)$ & 10 & $(25,7)$ & 7 & 1 & YES & YES & YES & $1.71$ & $(2,3)$ & NO & 4751\\
$(71,26)$ & 9 & $(35,13)$ & 8 & 1 & YES & YES & YES & $1.43$ & $(2,3)$ & 4517 & 4752\\
$(71,21)$ & 9 & $(45,13)$ & 10 & 1 & YES & YES & YES & $1.57$ & $(2,3)$ & NO & 4753\\
$(71,27)$ & 9 & $(45,17)$ & 9 & 1 & YES & YES & YES & $1.43$ & $(2,3)$ & NO & 4754\\
$(71,21)$ & 9 & $(49,15)$ & 9 & 1 & YES & YES & YES & $1.57$ & $(2,3)$ & NO & 4755\\
$(71,30)$ & 9 & $(50,21)$ & 8 & 1 & YES & YES & NO(2) & $1.62$ & $(2,3)$ & NO & 4756\\
$(71,22)$ & 10 & $(51,16)$ & 10 & 1 & YES & YES & YES & $1.71$ & $(2,3)$ & NO & 4757\\
$(71,16)$ & 10 & $(55,13)$ & 10 & 1 & YES & YES & YES & $1.71$ & $(2,3)$ & NO & 4758\\
$(71,30)$ & 9 & $(55,23)$ & 9 & 1 & YES & YES & YES & $1.43$ & $(2,3)$ & NO & 4759\\
$(71,16)$ & 10 & $(56,13)$ & 10 & 1 & YES & YES & NO(2) & $1.50$ & $(2,3)$ & NO & 4760\\
$(71,16)$ & 10 & $(64,15)$ & 10 & 1 & YES & YES & YES & $1.71$ & $(2,3)$ & NO & 4761\\
$(72,19)$ & 10 & $(8,3)$ & 4 & 8 & YES & YES & YES & $1.43$ & $(2,3)$ & NO & 4762\\
$(72,19)$ & 10 & $(8,3)$ & 4 & 8 & YES & YES & YES & $1.43$ & $(2,3)$ & -- & 4763\\
$(72,17)$ & 11 & $(12,5)$ & 5 & 12 & YES & YES & NO(2) & $1.50$ & $(2,3)$ & -- & 4764\\
$(72,19)$ & 10 & $(12,5)$ & 5 & 12 & YES & YES & YES & $1.57$ & $(2,3)$ & -- & 4765\\
$(72,17)$ & 11 & $(13,5)$ & 5 & 1 & YES & YES & NO(2) & $1.50$ & $(2,3)$ & -- & 4766\\
$(72,19)$ & 10 & $(13,3)$ & 6 & 1 & YES & YES & YES & $1.43$ & $(2,3)$ & -- & 4767\\
$(72,19)$ & 10 & $(13,4)$ & 6 & 1 & YES & YES & YES & $1.57$ & $(2,3)$ & -- & 4768\\
$(72,19)$ & 10 & $(13,5)$ & 5 & 1 & YES & YES & YES & $1.57$ & $(2,3)$ & -- & 4769\\
$(72,19)$ & 10 & $(17,5)$ & 6 & 1 & YES & YES & YES & $1.43$ & $(2,3)$ & NO & 4770\\
$(72,19)$ & 10 & $(37,10)$ & 8 & 1 & YES & YES & YES & $1.43$ & $(2,3)$ & NO & 4771\\
$(73,27)$ & 9 & $(7,2)$ & 4 & 1 & YES & YES & YES & $1.43$ & $(2,3)$ & NO & 4772\\
$(73,27)$ & 9 & $(7,2)$ & 4 & 1 & YES & YES & YES & $1.43$ & $(2,3)$ & -- & 4773\\
$(73,27)$ & 9 & $(8,3)$ & 4 & 1 & YES & YES & YES & $1.43$ & $(2,3)$ & NO & 4774\\
$(73,27)$ & 9 & $(8,3)$ & 4 & 1 & YES & YES & YES & $1.43$ & $(2,3)$ & -- & 4775\\
$(73,28)$ & 10 & $(8,3)$ & 4 & 1 & YES & YES & YES & $1.71$ & $(2,3)$ & -- & 4776\\
$(73,27)$ & 9 & $(9,4)$ & 5 & 1 & YES & YES & YES & $1.57$ & $(2,3)$ & -- & 4777\\
$(73,17)$ & 10 & $(11,3)$ & 5 & 1 & YES & YES & YES & $1.43$ & $(2,3)$ & -- & 4778\\
$(73,17)$ & 10 & $(11,3)$ & 5 & 1 & YES & YES & YES & $1.57$ & $(2,3)$ & NO & 4779\\
$(73,28)$ & 10 & $(11,3)$ & 5 & 1 & YES & YES & NO(2) & $1.75$ & $(2,3)$ & NO & 4780\\
$(73,28)$ & 10 & $(11,3)$ & 5 & 1 & YES & YES & YES & $1.71$ & $(2,3)$ & -- & 4781\\
$(73,30)$ & 10 & $(11,3)$ & 5 & 1 & YES & YES & NO(2) & $1.75$ & $(2,3)$ & -- & 4782\\
$(73,27)$ & 9 & $(12,5)$ & 5 & 1 & YES & YES & YES & $1.57$ & $(2,3)$ & -- & 4783\\
$(73,32)$ & 10 & $(13,5)$ & 5 & 1 & YES & YES & YES & $1.71$ & $(2,3)$ & NO & 4784\\
$(73,28)$ & 10 & $(14,3)$ & 6 & 1 & YES & YES & YES & $1.43$ & $(2,3)$ & -- & 4785\\
$(73,30)$ & 10 & $(15,4)$ & 6 & 1 & YES & YES & YES & $1.57$ & $(2,3)$ & -- & 4786\\
$(73,28)$ & 10 & $(16,3)$ & 7 & 1 & YES & YES & YES & $1.57$ & $(2,3)$ & -- & 4787\\
$(73,17)$ & 10 & $(17,7)$ & 6 & 1 & YES & YES & YES & $1.57$ & $(2,3)$ & -- & 4788\\
$(73,17)$ & 10 & $(17,7)$ & 6 & 1 & YES & YES & YES & $1.71$ & $(2,3)$ & NO & 4789\\
$(73,17)$ & 10 & $(18,7)$ & 6 & 1 & YES & YES & YES & $1.57$ & $(2,3)$ & -- & 4790\\
$(73,17)$ & 10 & $(18,7)$ & 6 & 1 & YES & YES & YES & $1.71$ & $(2,3)$ & NO & 4791\\
$(73,17)$ & 10 & $(19,8)$ & 6 & 1 & YES & YES & YES & $1.43$ & $(2,3)$ & -- & 4792\\
$(73,27)$ & 9 & $(19,8)$ & 6 & 1 & YES & YES & YES & $1.57$ & $(2,3)$ & NO & 4793\\
$(73,30)$ & 10 & $(19,4)$ & 7 & 1 & YES & YES & YES & $1.71$ & $(2,3)$ & -- & 4794\\
$(73,17)$ & 10 & $(24,7)$ & 7 & 1 & YES & YES & YES & $1.57$ & $(2,3)$ & -- & 4795\\
$(73,27)$ & 9 & $(31,12)$ & 7 & 1 & YES & YES & YES & $1.43$ & $(2,3)$ & NO & 4796\\
$(73,27)$ & 9 & $(41,15)$ & 8 & 1 & YES & YES & YES & $1.43$ & $(2,3)$ & NO & 4797\\
$(73,27)$ & 9 & $(52,19)$ & 9 & 1 & YES & YES & NO(2) & $1.75$ & $(2,3)$ & NO & 4798\\
$(73,32)$ & 10 & $(59,26)$ & 9 & 1 & YES & YES & YES & $1.57$ & $(2,3)$ & 7073 & 4799\\
$(74,29)$ & 10 & $(7,2)$ & 4 & 1 & YES & YES & YES & $1.29$ & $(2,3)$ & -- & 4800\\
$(74,31)$ & 9 & $(7,2)$ & 4 & 1 & YES & YES & YES & $1.43$ & $(2,3)$ & -- & 4801\\
$(74,31)$ & 9 & $(9,4)$ & 5 & 1 & YES & YES & YES & $1.71$ & $(2,3)$ & -- & 4802\\
$(74,31)$ & 9 & $(10,3)$ & 5 & 2 & YES & YES & YES & $1.29$ & $(2,3)$ & -- & 4803\\
$(74,31)$ & 9 & $(10,3)$ & 5 & 2 & YES & YES & NO(2) & $1.62$ & $(2,3)$ & NO & 4804\\
$(74,31)$ & 9 & $(11,4)$ & 5 & 1 & YES & YES & YES & $1.43$ & $(2,3)$ & -- & 4805\\
$(74,31)$ & 9 & $(12,5)$ & 5 & 2 & YES & YES & YES & $1.43$ & $(2,3)$ & -- & 4806\\
$(74,31)$ & 9 & $(13,4)$ & 6 & 1 & YES & YES & YES & $1.71$ & $(2,3)$ & -- & 4807\\
$(74,31)$ & 9 & $(18,7)$ & 6 & 2 & YES & YES & YES & $1.57$ & $(2,3)$ & NO & 4808\\
$(74,17)$ & 11 & $(19,4)$ & 7 & 1 & YES & YES & YES & $1.43$ & $(2,3)$ & -- & 4809\\
$(74,31)$ & 9 & $(19,7)$ & 6 & 1 & YES & YES & YES & $1.57$ & $(2,3)$ & NO & 4810\\
$(74,31)$ & 9 & $(20,9)$ & 7 & 2 & YES & YES & YES & $1.43$ & $(2,3)$ & NO & 4811\\
$(74,17)$ & 11 & $(21,4)$ & 8 & 1 & YES & YES & YES & $1.57$ & $(2,3)$ & -- & 4812\\
$(74,29)$ & 10 & $(21,8)$ & 6 & 1 & YES & YES & YES & $1.57$ & $(2,3)$ & NO & 4813\\
$(74,17)$ & 11 & $(27,5)$ & 8 & 1 & YES & YES & YES & $1.57$ & $(2,3)$ & NO & 4814\\
$(74,17)$ & 11 & $(71,16)$ & 10 & 1 & YES & YES & YES & $1.57$ & $(2,3)$ & NO & 4815\\
$(74,31)$ & 9 & $(71,30)$ & 9 & 1 & YES & YES & YES & $1.43$ & $(2,3)$ & NO & 4816\\
$(75,31)$ & 9 & $(7,3)$ & 4 & 1 & YES & YES & NO(2) & $1.50$ & $(2,3)$ & -- & 4817\\
$(75,31)$ & 9 & $(10,3)$ & 5 & 5 & YES & YES & YES & $1.43$ & $(2,3)$ & -- & 4818\\
$(75,31)$ & 9 & $(10,3)$ & 5 & 5 & YES & YES & NO(2) & $1.50$ & $(2,3)$ & NO & 4819\\
$(75,29)$ & 9 & $(11,4)$ & 5 & 1 & YES & YES & YES & $1.43$ & $(2,3)$ & NO & 4820\\
$(75,31)$ & 9 & $(11,4)$ & 5 & 1 & YES & YES & NO(2) & $1.43$ & $(4,2)$ & NO & 4821\\
$(75,22)$ & 10 & $(12,5)$ & 5 & 3 & YES & YES & YES & $1.57$ & $(2,3)$ & -- & 4822\\
$(75,31)$ & 9 & $(12,5)$ & 5 & 3 & YES & YES & YES & $1.71$ & $(2,3)$ & -- & 4823\\
$(75,29)$ & 9 & $(13,4)$ & 6 & 1 & YES & YES & NO(2) & $1.50$ & $(2,3)$ & -- & 4824\\
$(75,29)$ & 9 & $(16,7)$ & 6 & 1 & YES & YES & YES & $1.57$ & $(2,3)$ & -- & 4825\\
$(75,22)$ & 10 & $(19,6)$ & 8 & 1 & YES & YES & YES & $1.71$ & $(2,3)$ & NO & 4826\\
$(75,29)$ & 9 & $(19,5)$ & 7 & 1 & YES & YES & YES & $1.57$ & $(2,3)$ & NO & 4827\\
$(75,31)$ & 9 & $(22,5)$ & 7 & 1 & YES & YES & YES & $1.43$ & $(2,3)$ & -- & 4828\\
$(75,31)$ & 9 & $(31,13)$ & 7 & 1 & YES & YES & NO(2) & $1.62$ & $(2,3)$ & NO & 4829\\
$(75,31)$ & 9 & $(32,13)$ & 9 & 1 & YES & YES & YES & $1.57$ & $(2,3)$ & NO & 4830\\
$(75,31)$ & 9 & $(49,20)$ & 9 & 1 & YES & YES & YES & $1.43$ & $(2,3)$ & NO & 4831\\
$(75,31)$ & 9 & $(50,21)$ & 8 & 25 & YES & YES & YES & $1.57$ & $(2,3)$ & NO & 4832\\
$(75,29)$ & 9 & $(59,23)$ & 9 & 1 & YES & YES & NO(2) & $1.62$ & $(2,3)$ & NO & 4833\\
$(76,21)$ & 9 & $(4,1)$ & 3 & 4 & YES & YES & YES & $1.43$ & $(2,3)$ & NO & 4834\\
$(76,21)$ & 9 & $(4,1)$ & 3 & 4 & YES & YES & YES & $1.57$ & $(2,3)$ & -- & 4835\\
$(76,29)$ & 9 & $(5,1)$ & 4 & 1 & YES & YES & YES & $1.71$ & $(2,3)$ & -- & 4836\\
$(76,21)$ & 9 & $(7,3)$ & 4 & 1 & YES & YES & YES & $1.57$ & $(2,3)$ & NO & 4837\\
$(76,21)$ & 9 & $(7,3)$ & 4 & 1 & YES & YES & YES & $1.57$ & $(2,3)$ & -- & 4838\\
$(76,21)$ & 9 & $(7,3)$ & 4 & 1 & YES & YES & YES & $1.57$ & $(2,3)$ & NO & 4839\\
$(76,23)$ & 10 & $(7,3)$ & 4 & 1 & YES & YES & NO(2) & $1.38$ & $(2,3)$ & -- & 4840\\
$(76,21)$ & 9 & $(8,3)$ & 4 & 4 & YES & YES & NO(2) & $1.50$ & $(2,3)$ & NO & 4841\\
$(76,21)$ & 9 & $(8,3)$ & 4 & 4 & YES & YES & NO(2) & $1.50$ & $(2,3)$ & -- & 4842\\
$(76,29)$ & 9 & $(8,3)$ & 4 & 4 & YES & YES & YES & $1.57$ & $(2,3)$ & -- & 4843\\
$(76,21)$ & 9 & $(9,2)$ & 5 & 1 & YES & YES & YES & $1.57$ & $(2,3)$ & NO & 4844\\
$(76,21)$ & 9 & $(9,2)$ & 5 & 1 & YES & YES & YES & $1.57$ & $(2,3)$ & -- & 4845\\
$(76,29)$ & 9 & $(10,3)$ & 5 & 2 & YES & YES & YES & $1.57$ & $(2,3)$ & -- & 4846\\
$(76,29)$ & 9 & $(10,3)$ & 5 & 2 & YES & YES & YES & $1.71$ & $(2,3)$ & NO & 4847\\
$(76,21)$ & 9 & $(11,5)$ & 6 & 1 & YES & YES & YES & $1.57$ & $(2,3)$ & -- & 4848\\
$(76,21)$ & 9 & $(11,5)$ & 6 & 1 & YES & YES & YES & $1.71$ & $(2,3)$ & NO & 4849\\
$(76,29)$ & 9 & $(12,5)$ & 5 & 4 & YES & YES & YES & $1.57$ & $(2,3)$ & -- & 4850\\
$(76,33)$ & 10 & $(13,5)$ & 5 & 1 & YES & YES & YES & $1.71$ & $(2,3)$ & -- & 4851\\
$(76,21)$ & 9 & $(17,5)$ & 6 & 1 & YES & YES & YES & $1.57$ & $(2,3)$ & NO & 4852\\
$(76,21)$ & 9 & $(17,7)$ & 6 & 1 & YES & YES & YES & $1.71$ & $(2,3)$ & NO & 4853\\
$(76,21)$ & 9 & $(17,7)$ & 6 & 1 & YES & YES & YES & $1.57$ & $(2,3)$ & -- & 4854\\
$(76,33)$ & 10 & $(17,4)$ & 7 & 1 & YES & YES & YES & $1.71$ & $(2,3)$ & NO & 4855\\
$(76,33)$ & 10 & $(17,4)$ & 7 & 1 & YES & YES & YES & $1.71$ & $(2,3)$ & -- & 4856\\
$(76,21)$ & 9 & $(18,7)$ & 6 & 2 & YES & YES & YES & $1.57$ & $(2,3)$ & NO & 4857\\
$(76,29)$ & 9 & $(19,8)$ & 6 & 19 & YES & YES & YES & $1.57$ & $(2,3)$ & NO & 4858\\
$(76,29)$ & 9 & $(22,9)$ & 7 & 2 & YES & YES & YES & $1.71$ & $(2,3)$ & NO & 4859\\
$(76,23)$ & 10 & $(29,8)$ & 7 & 1 & YES & YES & YES & $1.43$ & $(2,3)$ & NO & 4860\\
$(76,29)$ & 9 & $(30,11)$ & 7 & 2 & YES & YES & YES & $1.43$ & $(2,3)$ & NO & 4861\\
$(76,21)$ & 9 & $(31,9)$ & 8 & 1 & YES & YES & YES & $1.71$ & $(2,3)$ & NO & 4862\\
$(76,21)$ & 9 & $(32,9)$ & 8 & 4 & YES & YES & YES & $1.43$ & $(2,3)$ & NO & 4863\\
$(76,21)$ & 9 & $(33,10)$ & 8 & 1 & YES & YES & YES & $1.57$ & $(2,3)$ & NO & 4864\\
$(76,21)$ & 9 & $(37,11)$ & 8 & 1 & YES & YES & YES & $1.57$ & $(2,3)$ & NO & 4865\\
$(76,21)$ & 9 & $(46,13)$ & 10 & 2 & YES & YES & YES & $1.71$ & $(2,3)$ & NO & 4866\\
$(76,21)$ & 9 & $(51,14)$ & 9 & 1 & YES & YES & YES & $1.57$ & $(2,3)$ & NO & 4867\\
$(76,21)$ & 9 & $(65,18)$ & 9 & 1 & YES & YES & YES & $1.43$ & $(2,3)$ & NO & 4868\\
$(76,29)$ & 9 & $(66,25)$ & 9 & 2 & YES & YES & YES & $1.43$ & $(2,3)$ & NO & 4869\\
$(77,30)$ & 10 & $(9,2)$ & 5 & 1 & YES & YES & YES & $1.43$ & $(2,3)$ & -- & 4870\\
$(77,18)$ & 10 & $(11,4)$ & 5 & 11 & YES & YES & NO(2) & $1.50$ & $(2,3)$ & -- & 4871\\
$(77,30)$ & 10 & $(11,3)$ & 5 & 11 & YES & YES & NO(2) & $1.75$ & $(2,3)$ & -- & 4872\\
$(77,18)$ & 10 & $(17,7)$ & 6 & 1 & YES & YES & YES & $1.57$ & $(2,3)$ & -- & 4873\\
$(77,18)$ & 10 & $(18,7)$ & 6 & 1 & YES & YES & YES & $1.71$ & $(2,3)$ & -- & 4874\\
$(77,18)$ & 10 & $(18,7)$ & 6 & 1 & YES & YES & YES & $1.57$ & $(2,3)$ & NO & 4875\\
$(77,18)$ & 10 & $(19,8)$ & 6 & 1 & YES & YES & YES & $1.43$ & $(2,3)$ & -- & 4876\\
$(77,30)$ & 10 & $(19,4)$ & 7 & 1 & YES & YES & YES & $1.57$ & $(2,3)$ & -- & 4877\\
$(77,18)$ & 10 & $(21,8)$ & 6 & 7 & YES & YES & YES & $1.71$ & $(2,3)$ & -- & 4878\\
$(78,29)$ & 10 & $(10,3)$ & 5 & 2 & YES & YES & YES & $1.71$ & $(2,3)$ & -- & 4879\\
$(78,29)$ & 10 & $(10,3)$ & 5 & 2 & YES & YES & YES & $1.71$ & $(2,3)$ & NO & 4880\\
$(78,23)$ & 10 & $(11,4)$ & 5 & 1 & YES & YES & YES & $1.43$ & $(2,3)$ & NO & 4881\\
$(78,35)$ & 10 & $(11,3)$ & 5 & 1 & YES & YES & YES & $1.43$ & $(2,3)$ & -- & 4882\\
$(78,29)$ & 10 & $(13,4)$ & 6 & 13 & YES & YES & YES & $1.71$ & $(2,3)$ & -- & 4883\\
$(78,23)$ & 10 & $(19,6)$ & 8 & 1 & YES & YES & YES & $1.57$ & $(2,3)$ & NO & 4884\\
$(78,35)$ & 10 & $(19,8)$ & 6 & 1 & YES & YES & YES & $1.57$ & $(2,3)$ & NO & 4885\\
$(78,23)$ & 10 & $(23,5)$ & 7 & 1 & YES & YES & YES & $1.43$ & $(2,3)$ & NO & 4886\\
$(78,23)$ & 10 & $(29,9)$ & 8 & 1 & YES & YES & YES & $1.43$ & $(2,3)$ & NO & 4887\\
$(78,17)$ & 10 & $(78,17)$ & 10 & 78 & YES & YES & YES & $1.43$ & $(2,3)$ & NO & 4888\\
$(79,18)$ & 10 & $(4,1)$ & 3 & 1 & YES & YES & YES & $1.57$ & $(2,3)$ & NO & 4889\\
$(79,30)$ & 9 & $(5,2)$ & 3 & 1 & YES & YES & NO(2) & $1.67$ & $(2,3)$ & NO & 4890\\
$(79,30)$ & 9 & $(5,2)$ & 3 & 1 & YES & YES & YES & $1.57$ & $(2,3)$ & -- & 4891\\
$(79,29)$ & 9 & $(7,2)$ & 4 & 1 & YES & YES & YES & $1.29$ & $(2,3)$ & NO & 4892\\
$(79,29)$ & 9 & $(7,2)$ & 4 & 1 & YES & YES & YES & $1.29$ & $(2,3)$ & -- & 4893\\
$(79,29)$ & 9 & $(7,3)$ & 4 & 1 & YES & YES & YES & $1.43$ & $(2,3)$ & -- & 4894\\
$(79,30)$ & 9 & $(7,2)$ & 4 & 1 & YES & YES & YES & $1.43$ & $(2,3)$ & NO & 4895\\
$(79,30)$ & 9 & $(7,2)$ & 4 & 1 & YES & YES & YES & $1.43$ & $(2,3)$ & -- & 4896\\
$(79,30)$ & 9 & $(7,3)$ & 4 & 1 & YES & YES & YES & $1.71$ & $(2,3)$ & -- & 4897\\
$(79,30)$ & 9 & $(8,3)$ & 4 & 1 & YES & YES & YES & $1.43$ & $(2,3)$ & NO & 4898\\
$(79,30)$ & 9 & $(8,3)$ & 4 & 1 & YES & YES & YES & $1.43$ & $(2,3)$ & -- & 4899\\
$(79,18)$ & 10 & $(9,2)$ & 5 & 1 & YES & YES & YES & $1.43$ & $(2,3)$ & -- & 4900\\
$(79,18)$ & 10 & $(9,2)$ & 5 & 1 & YES & YES & YES & $1.57$ & $(2,3)$ & NO & 4901\\
$(79,23)$ & 10 & $(9,2)$ & 5 & 1 & YES & YES & YES & $1.57$ & $(2,3)$ & -- & 4902\\
$(79,23)$ & 10 & $(9,2)$ & 5 & 1 & YES & YES & YES & $1.71$ & $(2,3)$ & NO & 4903\\
$(79,30)$ & 9 & $(9,4)$ & 5 & 1 & YES & YES & YES & $1.57$ & $(2,3)$ & -- & 4904\\
$(79,28)$ & 10 & $(10,3)$ & 5 & 1 & YES & YES & YES & $1.57$ & $(2,3)$ & -- & 4905\\
$(79,17)$ & 11 & $(11,4)$ & 5 & 1 & YES & YES & YES & $1.43$ & $(2,3)$ & NO & 4906\\
$(79,29)$ & 9 & $(11,4)$ & 5 & 1 & YES & YES & YES & $1.43$ & $(2,3)$ & -- & 4907\\
$(79,30)$ & 9 & $(11,4)$ & 5 & 1 & YES & YES & YES & $1.43$ & $(2,3)$ & -- & 4908\\
$(79,33)$ & 11 & $(11,2)$ & 6 & 1 & YES & YES & YES & $1.57$ & $(2,3)$ & -- & 4909\\
$(79,17)$ & 11 & $(12,5)$ & 5 & 1 & YES & YES & YES & $1.43$ & $(2,3)$ & NO & 4910\\
$(79,23)$ & 10 & $(12,5)$ & 5 & 1 & YES & YES & YES & $1.57$ & $(2,3)$ & -- & 4911\\
$(79,23)$ & 10 & $(12,5)$ & 5 & 1 & YES & YES & YES & $1.71$ & $(2,3)$ & NO & 4912\\
$(79,24)$ & 10 & $(12,5)$ & 5 & 1 & YES & YES & YES & $1.57$ & $(2,3)$ & NO & 4913\\
$(79,29)$ & 9 & $(12,5)$ & 5 & 1 & YES & YES & YES & $1.71$ & $(2,3)$ & -- & 4914\\
$(79,30)$ & 9 & $(12,5)$ & 5 & 1 & YES & YES & YES & $1.43$ & $(2,3)$ & -- & 4915\\
$(79,17)$ & 11 & $(13,4)$ & 6 & 1 & YES & YES & YES & $1.57$ & $(2,3)$ & NO & 4916\\
$(79,18)$ & 10 & $(13,5)$ & 5 & 1 & YES & YES & YES & $1.57$ & $(2,3)$ & NO & 4917\\
$(79,23)$ & 10 & $(13,4)$ & 6 & 1 & YES & YES & YES & $1.57$ & $(2,3)$ & -- & 4918\\
$(79,23)$ & 10 & $(13,5)$ & 5 & 1 & YES & YES & YES & $1.57$ & $(2,3)$ & -- & 4919\\
$(79,23)$ & 10 & $(13,5)$ & 5 & 1 & YES & YES & YES & $1.57$ & $(2,3)$ & NO & 4920\\
$(79,24)$ & 10 & $(13,5)$ & 5 & 1 & YES & YES & YES & $1.57$ & $(2,3)$ & NO & 4921\\
$(79,30)$ & 9 & $(13,4)$ & 6 & 1 & YES & YES & YES & $1.57$ & $(2,3)$ & -- & 4922\\
$(79,30)$ & 9 & $(14,5)$ & 6 & 1 & YES & YES & YES & $1.71$ & $(2,3)$ & -- & 4923\\
$(79,23)$ & 10 & $(15,4)$ & 6 & 1 & YES & YES & YES & $1.57$ & $(2,3)$ & -- & 4924\\
$(79,18)$ & 10 & $(17,4)$ & 7 & 1 & YES & YES & YES & $1.43$ & $(2,3)$ & -- & 4925\\
$(79,17)$ & 11 & $(18,5)$ & 6 & 1 & YES & YES & YES & $1.57$ & $(2,3)$ & NO & 4926\\
$(79,23)$ & 10 & $(18,5)$ & 6 & 1 & YES & YES & YES & $1.57$ & $(2,3)$ & -- & 4927\\
$(79,23)$ & 10 & $(18,7)$ & 6 & 1 & YES & YES & YES & $1.71$ & $(2,3)$ & NO & 4928\\
$(79,28)$ & 10 & $(18,7)$ & 6 & 1 & YES & YES & YES & $1.57$ & $(2,3)$ & NO & 4929\\
$(79,17)$ & 11 & $(19,7)$ & 6 & 1 & YES & YES & YES & $1.71$ & $(2,3)$ & -- & 4930\\
$(79,30)$ & 9 & $(19,8)$ & 6 & 1 & YES & YES & YES & $1.43$ & $(2,3)$ & NO & 4931\\
$(79,29)$ & 9 & $(21,8)$ & 6 & 1 & YES & YES & NO(2) & $1.50$ & $(2,3)$ & NO & 4932\\
$(79,30)$ & 9 & $(22,5)$ & 7 & 1 & YES & YES & YES & $1.43$ & $(2,3)$ & NO & 4933\\
$(79,29)$ & 9 & $(23,9)$ & 7 & 1 & YES & YES & YES & $1.43$ & $(2,3)$ & NO & 4934\\
$(79,23)$ & 10 & $(34,9)$ & 8 & 1 & YES & YES & YES & $1.57$ & $(2,3)$ & NO & 4935\\
$(79,30)$ & 9 & $(34,13)$ & 7 & 1 & YES & YES & YES & $1.43$ & $(2,3)$ & NO & 4936\\
$(79,29)$ & 9 & $(35,13)$ & 8 & 1 & YES & YES & YES & $1.43$ & $(2,3)$ & NO & 4937\\
$(79,30)$ & 9 & $(37,14)$ & 8 & 1 & YES & YES & YES & $1.43$ & $(2,3)$ & NO & 4938\\
$(79,30)$ & 9 & $(41,16)$ & 8 & 1 & YES & YES & YES & $1.43$ & $(2,3)$ & 5961 & 4939\\
$(79,30)$ & 9 & $(44,17)$ & 8 & 1 & YES & YES & YES & $1.71$ & $(2,3)$ & NO & 4940\\
$(79,17)$ & 11 & $(53,12)$ & 9 & 1 & YES & YES & YES & $1.57$ & $(2,3)$ & NO & 4941\\
$(79,30)$ & 9 & $(55,21)$ & 8 & 1 & YES & YES & YES & $1.57$ & $(2,3)$ & NO & 4942\\
$(79,30)$ & 9 & $(60,23)$ & 9 & 1 & YES & YES & YES & $1.43$ & $(2,3)$ & NO & 4943\\
$(79,17)$ & 11 & $(73,16)$ & 10 & 1 & YES & YES & YES & $1.57$ & $(2,3)$ & NO & 4944\\
$(80,31)$ & 9 & $(7,2)$ & 4 & 1 & YES & YES & YES & $1.29$ & $(2,3)$ & -- & 4945\\
$(80,31)$ & 9 & $(7,3)$ & 4 & 1 & YES & YES & NO(2) & $1.56$ & $(2,3)$ & NO & 4946\\
$(80,31)$ & 9 & $(7,3)$ & 4 & 1 & YES & YES & NO(2) & $1.56$ & $(2,3)$ & -- & 4947\\
$(80,33)$ & 10 & $(7,2)$ & 4 & 1 & YES & YES & NO(2) & $1.62$ & $(2,3)$ & -- & 4948\\
$(80,31)$ & 9 & $(8,3)$ & 4 & 8 & YES & YES & YES & $1.71$ & $(2,3)$ & -- & 4949\\
$(80,33)$ & 10 & $(9,2)$ & 5 & 1 & YES & YES & NO(2) & $1.57$ & $(4,2)$ & NO & 4950\\
$(80,33)$ & 10 & $(9,2)$ & 5 & 1 & YES & YES & NO(2) & $1.57$ & $(4,2)$ & -- & 4951\\
$(80,31)$ & 9 & $(11,3)$ & 5 & 1 & YES & YES & YES & $1.43$ & $(2,3)$ & -- & 4952\\
$(80,33)$ & 10 & $(11,3)$ & 5 & 1 & YES & YES & YES & $1.43$ & $(2,3)$ & -- & 4953\\
$(80,31)$ & 9 & $(12,5)$ & 5 & 4 & YES & YES & YES & $1.43$ & $(2,3)$ & -- & 4954\\
$(80,33)$ & 10 & $(14,3)$ & 6 & 2 & YES & YES & YES & $1.57$ & $(2,3)$ & NO & 4955\\
$(80,33)$ & 10 & $(14,3)$ & 6 & 2 & YES & YES & YES & $1.71$ & $(2,3)$ & -- & 4956\\
$(80,31)$ & 9 & $(15,4)$ & 6 & 5 & YES & YES & YES & $1.43$ & $(2,3)$ & -- & 4957\\
$(80,33)$ & 10 & $(16,3)$ & 7 & 16 & YES & YES & YES & $1.57$ & $(2,3)$ & -- & 4958\\
$(80,31)$ & 9 & $(55,21)$ & 8 & 5 & YES & YES & YES & $1.43$ & $(2,3)$ & NO & 4959\\
$(80,33)$ & 10 & $(75,31)$ & 9 & 5 & YES & YES & NO(2) & $1.50$ & $(2,3)$ & 6245 & 4960\\
$(81,31)$ & 9 & $(3,1)$ & 2 & 3 & YES & YES & YES & $1.57$ & $(2,3)$ & NO & 4961\\
$(81,31)$ & 9 & $(5,2)$ & 3 & 1 & YES & YES & YES & $1.43$ & $(2,3)$ & -- & 4962\\
$(81,34)$ & 9 & $(5,2)$ & 3 & 1 & YES & YES & NO(3) & $1.14$ & $(2,3)$ & -- & 4963\\
$(81,31)$ & 9 & $(7,2)$ & 4 & 1 & YES & YES & YES & $1.43$ & $(2,3)$ & -- & 4964\\
$(81,31)$ & 9 & $(7,2)$ & 4 & 1 & YES & YES & YES & $1.43$ & $(2,3)$ & NO & 4965\\
$(81,34)$ & 9 & $(7,3)$ & 4 & 1 & YES & YES & YES & $1.57$ & $(2,3)$ & -- & 4966\\
$(81,31)$ & 9 & $(8,3)$ & 4 & 1 & YES & YES & YES & $1.71$ & $(2,3)$ & -- & 4967\\
$(81,31)$ & 9 & $(9,4)$ & 5 & 9 & YES & YES & YES & $1.71$ & $(2,3)$ & -- & 4968\\
$(81,34)$ & 9 & $(9,4)$ & 5 & 9 & YES & YES & YES & $1.57$ & $(2,3)$ & -- & 4969\\
$(81,31)$ & 9 & $(10,3)$ & 5 & 1 & YES & YES & YES & $1.43$ & $(2,3)$ & -- & 4970\\
$(81,31)$ & 9 & $(12,5)$ & 5 & 3 & YES & YES & YES & $1.43$ & $(2,3)$ & -- & 4971\\
$(81,19)$ & 11 & $(13,5)$ & 5 & 1 & YES & YES & YES & $1.43$ & $(2,3)$ & NO & 4972\\
$(81,34)$ & 9 & $(22,9)$ & 7 & 1 & YES & YES & YES & $1.57$ & $(2,3)$ & NO & 4973\\
$(81,31)$ & 9 & $(29,11)$ & 7 & 1 & YES & YES & YES & $1.43$ & $(2,3)$ & NO & 4974\\
$(81,34)$ & 9 & $(29,12)$ & 7 & 1 & YES & YES & NO(2) & $1.43$ & $(4,2)$ & NO & 4975\\
$(81,31)$ & 9 & $(41,16)$ & 8 & 1 & YES & YES & YES & $1.57$ & $(2,3)$ & NO & 4976\\
$(81,34)$ & 9 & $(46,19)$ & 8 & 1 & YES & YES & YES & $1.57$ & $(2,3)$ & NO & 4977\\
$(81,31)$ & 9 & $(55,21)$ & 8 & 1 & YES & YES & YES & $1.43$ & $(2,3)$ & NO & 4978\\
$(81,31)$ & 9 & $(67,26)$ & 9 & 1 & YES & YES & YES & $1.71$ & $(2,3)$ & NO & 4979\\
$(81,34)$ & 9 & $(67,28)$ & 10 & 1 & YES & YES & YES & $1.57$ & $(2,3)$ & 6871 & 4980\\
$(81,34)$ & 9 & $(74,31)$ & 9 & 1 & YES & YES & YES & $1.71$ & $(2,3)$ & NO & 4981\\
$(82,23)$ & 10 & $(11,4)$ & 5 & 1 & YES & YES & YES & $1.57$ & $(2,3)$ & -- & 4982\\
$(82,31)$ & 10 & $(11,3)$ & 5 & 1 & YES & YES & YES & $1.43$ & $(2,3)$ & -- & 4983\\
$(82,23)$ & 10 & $(12,5)$ & 5 & 2 & YES & YES & YES & $1.57$ & $(2,3)$ & -- & 4984\\
$(82,23)$ & 10 & $(13,4)$ & 6 & 1 & YES & YES & YES & $1.57$ & $(2,3)$ & -- & 4985\\
$(82,23)$ & 10 & $(13,5)$ & 5 & 1 & YES & YES & YES & $1.57$ & $(2,3)$ & -- & 4986\\
$(82,31)$ & 10 & $(13,3)$ & 6 & 1 & YES & YES & YES & $1.57$ & $(2,3)$ & -- & 4987\\
$(82,37)$ & 10 & $(13,3)$ & 6 & 1 & YES & YES & YES & $1.57$ & $(2,3)$ & NO & 4988\\
$(82,23)$ & 10 & $(15,4)$ & 6 & 1 & YES & YES & YES & $1.57$ & $(2,3)$ & -- & 4989\\
$(82,23)$ & 10 & $(15,4)$ & 6 & 1 & YES & YES & YES & $1.57$ & $(2,3)$ & NO & 4990\\
$(82,37)$ & 10 & $(17,7)$ & 6 & 1 & YES & YES & YES & $1.43$ & $(2,3)$ & NO & 4991\\
$(82,23)$ & 10 & $(18,5)$ & 6 & 2 & YES & YES & YES & $1.57$ & $(2,3)$ & -- & 4992\\
$(82,37)$ & 10 & $(19,8)$ & 6 & 1 & YES & YES & YES & $1.43$ & $(2,3)$ & NO & 4993\\
$(82,31)$ & 10 & $(55,21)$ & 8 & 1 & YES & YES & YES & $1.43$ & $(2,3)$ & NO & 4994\\
$(82,23)$ & 10 & $(69,19)$ & 9 & 1 & YES & YES & YES & $1.71$ & $(2,3)$ & NO & 4995\\
$(82,31)$ & 10 & $(71,27)$ & 9 & 1 & YES & YES & YES & $1.57$ & $(2,3)$ & 5819 & 4996\\
$(82,23)$ & 10 & $(76,21)$ & 9 & 2 & YES & YES & YES & $1.71$ & $(2,3)$ & NO & 4997\\
$(83,19)$ & 10 & $(5,2)$ & 3 & 1 & YES & YES & YES & $1.57$ & $(2,3)$ & -- & 4998\\
$(83,19)$ & 10 & $(7,2)$ & 4 & 1 & YES & YES & YES & $1.57$ & $(2,3)$ & -- & 4999\\
$(83,30)$ & 10 & $(7,3)$ & 4 & 1 & YES & YES & NO(2) & $1.75$ & $(2,3)$ & -- & 5000\\
$(83,34)$ & 10 & $(7,3)$ & 4 & 1 & YES & YES & YES & $1.57$ & $(2,3)$ & -- & 5001\\
$(83,35)$ & 10 & $(7,2)$ & 4 & 1 & YES & YES & YES & $1.29$ & $(2,3)$ & -- & 5002\\
$(83,36)$ & 10 & $(7,3)$ & 4 & 1 & YES & YES & NO(2) & $1.75$ & $(2,3)$ & -- & 5003\\
$(83,30)$ & 10 & $(8,3)$ & 4 & 1 & YES & YES & NO(2) & $1.75$ & $(2,3)$ & NO & 5004\\
$(83,30)$ & 10 & $(8,3)$ & 4 & 1 & YES & YES & NO(2) & $1.75$ & $(2,3)$ & -- & 5005\\
$(83,23)$ & 10 & $(9,4)$ & 5 & 1 & YES & YES & YES & $1.71$ & $(2,3)$ & NO & 5006\\
$(83,34)$ & 10 & $(9,2)$ & 5 & 1 & YES & YES & NO(2) & $1.62$ & $(2,3)$ & NO & 5007\\
$(83,35)$ & 10 & $(9,2)$ & 5 & 1 & YES & YES & NO(2) & $1.43$ & $(4,2)$ & -- & 5008\\
$(83,34)$ & 10 & $(10,3)$ & 5 & 1 & YES & YES & YES & $1.57$ & $(2,3)$ & -- & 5009\\
$(83,36)$ & 10 & $(10,3)$ & 5 & 1 & YES & YES & YES & $1.57$ & $(2,3)$ & -- & 5010\\
$(83,19)$ & 10 & $(11,5)$ & 6 & 1 & YES & YES & YES & $1.57$ & $(2,3)$ & -- & 5011\\
$(83,22)$ & 10 & $(11,4)$ & 5 & 1 & YES & YES & YES & $1.43$ & $(2,3)$ & -- & 5012\\
$(83,23)$ & 10 & $(11,4)$ & 5 & 1 & YES & YES & YES & $1.71$ & $(2,3)$ & NO & 5013\\
$(83,26)$ & 12 & $(11,3)$ & 5 & 1 & YES & YES & YES & $1.71$ & $(2,3)$ & -- & 5014\\
$(83,34)$ & 10 & $(11,5)$ & 6 & 1 & YES & YES & YES & $1.57$ & $(2,3)$ & NO & 5015\\
$(83,22)$ & 10 & $(12,5)$ & 5 & 1 & YES & YES & YES & $1.57$ & $(2,3)$ & NO & 5016\\
$(83,22)$ & 10 & $(12,5)$ & 5 & 1 & YES & YES & YES & $1.57$ & $(2,3)$ & -- & 5017\\
$(83,22)$ & 10 & $(13,5)$ & 5 & 1 & YES & YES & YES & $1.43$ & $(2,3)$ & -- & 5018\\
$(83,26)$ & 12 & $(13,3)$ & 6 & 1 & YES & YES & YES & $1.71$ & $(2,3)$ & -- & 5019\\
$(83,30)$ & 10 & $(13,4)$ & 6 & 1 & YES & YES & YES & $1.57$ & $(2,3)$ & -- & 5020\\
$(83,30)$ & 10 & $(13,5)$ & 5 & 1 & YES & YES & YES & $1.71$ & $(2,3)$ & -- & 5021\\
$(83,34)$ & 10 & $(13,3)$ & 6 & 1 & YES & YES & YES & $1.57$ & $(2,3)$ & NO & 5022\\
$(83,34)$ & 10 & $(13,3)$ & 6 & 1 & YES & YES & YES & $1.57$ & $(2,3)$ & -- & 5023\\
$(83,30)$ & 10 & $(14,5)$ & 6 & 1 & YES & YES & YES & $1.86$ & $(2,3)$ & -- & 5024\\
$(83,34)$ & 10 & $(14,3)$ & 6 & 1 & YES & YES & YES & $1.71$ & $(2,3)$ & -- & 5025\\
$(83,19)$ & 10 & $(19,6)$ & 8 & 1 & YES & YES & YES & $1.57$ & $(2,3)$ & -- & 5026\\
$(83,30)$ & 10 & $(22,9)$ & 7 & 1 & YES & YES & YES & $1.86$ & $(2,3)$ & NO & 5027\\
$(83,30)$ & 10 & $(23,9)$ & 7 & 1 & YES & YES & YES & $1.57$ & $(2,3)$ & NO & 5028\\
$(83,19)$ & 10 & $(31,7)$ & 8 & 1 & YES & YES & YES & $1.57$ & $(2,3)$ & NO & 5029\\
$(83,36)$ & 10 & $(34,15)$ & 8 & 1 & YES & YES & YES & $1.57$ & $(2,3)$ & NO & 5030\\
$(83,24)$ & 11 & $(37,11)$ & 8 & 1 & YES & YES & YES & $1.57$ & $(2,3)$ & NO & 5031\\
$(83,24)$ & 11 & $(79,23)$ & 10 & 1 & YES & YES & YES & $1.57$ & $(2,3)$ & 6093 & 5032\\
$(84,25)$ & 10 & $(7,3)$ & 4 & 7 & YES & YES & NO(2) & $1.62$ & $(2,3)$ & -- & 5033\\
$(84,25)$ & 10 & $(7,3)$ & 4 & 7 & YES & YES & YES & $1.71$ & $(2,3)$ & NO & 5034\\
$(84,37)$ & 10 & $(7,2)$ & 4 & 7 & YES & YES & YES & $1.71$ & $(2,3)$ & -- & 5035\\
$(84,31)$ & 10 & $(8,3)$ & 4 & 4 & YES & YES & YES & $1.57$ & $(2,3)$ & -- & 5036\\
$(84,25)$ & 10 & $(22,5)$ & 7 & 2 & YES & YES & YES & $1.43$ & $(2,3)$ & NO & 5037\\
$(84,31)$ & 10 & $(62,23)$ & 9 & 2 & YES & YES & YES & $1.57$ & $(2,3)$ & 7603 & 5038\\
$(84,25)$ & 10 & $(65,19)$ & 9 & 1 & YES & YES & YES & $1.43$ & $(2,3)$ & NO & 5039\\
$(85,36)$ & 10 & $(7,3)$ & 4 & 1 & YES & YES & NO(2) & $1.75$ & $(2,3)$ & -- & 5040\\
$(85,37)$ & 10 & $(9,4)$ & 5 & 1 & YES & YES & YES & $1.71$ & $(2,3)$ & -- & 5041\\
$(85,36)$ & 10 & $(10,3)$ & 5 & 5 & YES & YES & YES & $1.57$ & $(2,3)$ & NO & 5042\\
$(85,36)$ & 10 & $(10,3)$ & 5 & 5 & YES & YES & YES & $1.57$ & $(2,3)$ & -- & 5043\\
$(85,23)$ & 10 & $(12,5)$ & 5 & 1 & YES & YES & NO(2) & $1.50$ & $(2,3)$ & NO & 5044\\
$(85,36)$ & 10 & $(13,3)$ & 6 & 1 & YES & YES & NO(2) & $1.62$ & $(2,3)$ & -- & 5045\\
$(85,37)$ & 10 & $(13,4)$ & 6 & 1 & YES & YES & YES & $1.57$ & $(2,3)$ & -- & 5046\\
$(85,36)$ & 10 & $(14,3)$ & 6 & 1 & YES & YES & YES & $1.57$ & $(2,3)$ & NO & 5047\\
$(85,36)$ & 10 & $(14,3)$ & 6 & 1 & YES & YES & YES & $1.57$ & $(2,3)$ & -- & 5048\\
$(85,37)$ & 10 & $(22,9)$ & 7 & 1 & YES & YES & YES & $1.57$ & $(2,3)$ & NO & 5049\\
$(85,37)$ & 10 & $(26,11)$ & 7 & 1 & YES & YES & YES & $1.71$ & $(2,3)$ & NO & 5050\\
$(85,23)$ & 10 & $(63,17)$ & 9 & 1 & YES & YES & YES & $1.43$ & $(2,3)$ & NO & 5051\\
$(85,36)$ & 10 & $(64,27)$ & 9 & 1 & YES & YES & NO(2) & $1.75$ & $(2,3)$ & NO & 5052\\
$(86,31)$ & 10 & $(5,2)$ & 3 & 1 & YES & YES & YES & $1.43$ & $(2,3)$ & -- & 5053\\
$(86,25)$ & 10 & $(11,4)$ & 5 & 1 & YES & YES & YES & $1.43$ & $(2,3)$ & NO & 5054\\
$(86,33)$ & 11 & $(11,2)$ & 6 & 1 & YES & YES & YES & $1.57$ & $(2,3)$ & NO & 5055\\
$(86,33)$ & 11 & $(11,2)$ & 6 & 1 & YES & YES & YES & $1.71$ & $(2,3)$ & -- & 5056\\
$(86,25)$ & 10 & $(12,5)$ & 5 & 2 & YES & YES & YES & $1.57$ & $(2,3)$ & -- & 5057\\
$(86,31)$ & 10 & $(13,3)$ & 6 & 1 & YES & YES & YES & $1.71$ & $(2,3)$ & -- & 5058\\
$(86,31)$ & 10 & $(13,4)$ & 6 & 1 & YES & YES & YES & $1.71$ & $(2,3)$ & -- & 5059\\
$(86,25)$ & 10 & $(17,4)$ & 7 & 1 & YES & YES & YES & $1.43$ & $(2,3)$ & -- & 5060\\
$(86,25)$ & 10 & $(19,6)$ & 8 & 1 & YES & YES & YES & $1.71$ & $(2,3)$ & NO & 5061\\
$(86,25)$ & 10 & $(33,10)$ & 8 & 1 & YES & YES & YES & $1.57$ & $(2,3)$ & NO & 5062\\
$(87,32)$ & 10 & $(4,1)$ & 3 & 1 & YES & YES & NO(2) & $1.56$ & $(2,3)$ & -- & 5063\\
$(87,32)$ & 10 & $(7,3)$ & 4 & 1 & YES & YES & NO(2) & $1.75$ & $(2,3)$ & NO & 5064\\
$(87,32)$ & 10 & $(7,3)$ & 4 & 1 & YES & YES & NO(2) & $1.75$ & $(2,3)$ & -- & 5065\\
$(87,20)$ & 12 & $(9,4)$ & 5 & 3 & YES & YES & YES & $1.57$ & $(2,3)$ & NO & 5066\\
$(87,34)$ & 10 & $(9,2)$ & 5 & 3 & YES & YES & YES & $1.43$ & $(2,3)$ & -- & 5067\\
$(87,32)$ & 10 & $(10,3)$ & 5 & 1 & YES & YES & YES & $1.71$ & $(2,3)$ & -- & 5068\\
$(87,23)$ & 10 & $(12,5)$ & 5 & 3 & YES & YES & YES & $1.43$ & $(2,3)$ & -- & 5069\\
$(87,23)$ & 10 & $(12,5)$ & 5 & 3 & YES & YES & YES & $1.43$ & $(2,3)$ & NO & 5070\\
$(88,19)$ & 10 & $(2,1)$ & 1 & 2 & YES & YES & YES & $1.43$ & $(2,3)$ & -- & 5071\\
$(88,19)$ & 10 & $(2,1)$ & 1 & 2 & YES & YES & YES & $1.57$ & $(2,3)$ & NO & 5072\\
$(88,19)$ & 10 & $(3,1)$ & 2 & 1 & YES & YES & YES & $1.43$ & $(2,3)$ & -- & 5073\\
$(88,19)$ & 10 & $(3,1)$ & 2 & 1 & YES & YES & YES & $1.57$ & $(2,3)$ & NO & 5074\\
$(88,19)$ & 10 & $(4,1)$ & 3 & 4 & YES & YES & YES & $1.57$ & $(2,3)$ & NO & 5075\\
$(88,19)$ & 10 & $(4,1)$ & 3 & 4 & YES & YES & YES & $1.57$ & $(2,3)$ & -- & 5076\\
$(88,37)$ & 10 & $(8,3)$ & 4 & 8 & YES & YES & YES & $1.71$ & $(2,3)$ & NO & 5077\\
$(88,37)$ & 10 & $(8,3)$ & 4 & 8 & YES & YES & YES & $1.71$ & $(2,3)$ & -- & 5078\\
$(88,19)$ & 10 & $(9,2)$ & 5 & 1 & YES & YES & YES & $1.43$ & $(2,3)$ & 4279 & 5079\\
$(88,37)$ & 10 & $(10,3)$ & 5 & 2 & YES & YES & YES & $1.71$ & $(2,3)$ & -- & 5080\\
$(88,19)$ & 10 & $(11,5)$ & 6 & 11 & YES & YES & YES & $1.57$ & $(2,3)$ & -- & 5081\\
$(88,37)$ & 10 & $(11,3)$ & 5 & 11 & YES & YES & YES & $1.71$ & $(2,3)$ & NO & 5082\\
$(88,37)$ & 10 & $(13,5)$ & 5 & 1 & YES & YES & NO(2) & $1.75$ & $(2,3)$ & NO & 5083\\
$(88,19)$ & 10 & $(19,5)$ & 7 & 1 & YES & YES & YES & $1.71$ & $(2,3)$ & NO & 5084\\
$(88,37)$ & 10 & $(19,3)$ & 8 & 1 & YES & YES & YES & $1.71$ & $(2,3)$ & NO & 5085\\
$(88,37)$ & 10 & $(19,3)$ & 8 & 1 & YES & YES & YES & $1.71$ & $(2,3)$ & NO & 5086\\
$(88,37)$ & 10 & $(22,9)$ & 7 & 22 & YES & YES & NO(2) & $1.75$ & $(2,3)$ & NO & 5087\\
$(88,19)$ & 10 & $(23,5)$ & 7 & 1 & YES & YES & YES & $1.43$ & $(2,3)$ & NO & 5088\\
$(88,19)$ & 10 & $(59,13)$ & 11 & 1 & YES & YES & YES & $1.57$ & $(2,3)$ & NO & 5089\\
$(88,37)$ & 10 & $(74,31)$ & 9 & 2 & YES & YES & YES & $1.57$ & $(2,3)$ & 7708 & 5090\\
$(89,27)$ & 10 & $(3,1)$ & 2 & 1 & YES & YES & NO(2) & $1.38$ & $(2,3)$ & -- & 5091\\
$(89,27)$ & 10 & $(3,1)$ & 2 & 1 & YES & YES & NO(2) & $1.50$ & $(2,3)$ & NO & 5092\\
$(89,34)$ & 9 & $(4,1)$ & 3 & 1 & YES & YES & YES & $1.57$ & $(2,3)$ & -- & 5093\\
$(89,24)$ & 10 & $(5,1)$ & 4 & 1 & YES & YES & YES & $1.57$ & $(2,3)$ & NO & 5094\\
$(89,24)$ & 10 & $(5,1)$ & 4 & 1 & YES & YES & YES & $1.57$ & $(2,3)$ & -- & 5095\\
$(89,24)$ & 10 & $(5,1)$ & 4 & 1 & YES & YES & YES & $1.57$ & $(2,3)$ & NO & 5096\\
$(89,24)$ & 10 & $(5,2)$ & 3 & 1 & YES & YES & YES & $1.57$ & $(2,3)$ & -- & 5097\\
$(89,24)$ & 10 & $(5,2)$ & 3 & 1 & YES & YES & YES & $1.57$ & $(2,3)$ & NO & 5098\\
$(89,33)$ & 10 & $(5,2)$ & 3 & 1 & YES & YES & YES & $1.57$ & $(2,3)$ & -- & 5099\\
$(89,33)$ & 10 & $(5,2)$ & 3 & 1 & YES & YES & YES & $1.57$ & $(2,3)$ & NO & 5100\\
$(89,34)$ & 9 & $(5,1)$ & 4 & 1 & YES & YES & YES & $1.57$ & $(2,3)$ & NO & 5101\\
$(89,34)$ & 9 & $(5,1)$ & 4 & 1 & YES & YES & YES & $1.57$ & $(2,3)$ & -- & 5102\\
$(89,34)$ & 9 & $(5,2)$ & 3 & 1 & YES & YES & YES & $1.57$ & $(2,3)$ & -- & 5103\\
$(89,33)$ & 10 & $(7,3)$ & 4 & 1 & YES & YES & YES & $1.57$ & $(2,3)$ & -- & 5104\\
$(89,34)$ & 9 & $(7,2)$ & 4 & 1 & YES & YES & YES & $1.57$ & $(2,3)$ & NO & 5105\\
$(89,34)$ & 9 & $(7,2)$ & 4 & 1 & YES & YES & YES & $1.57$ & $(2,3)$ & -- & 5106\\
$(89,34)$ & 9 & $(7,3)$ & 4 & 1 & YES & YES & YES & $1.57$ & $(2,3)$ & -- & 5107\\
$(89,34)$ & 9 & $(8,3)$ & 4 & 1 & YES & YES & YES & $1.43$ & $(2,3)$ & -- & 5108\\
$(89,27)$ & 10 & $(9,4)$ & 5 & 1 & YES & YES & NO(2) & $1.62$ & $(2,3)$ & -- & 5109\\
$(89,32)$ & 10 & $(9,2)$ & 5 & 1 & YES & YES & YES & $1.43$ & $(2,3)$ & -- & 5110\\
$(89,34)$ & 9 & $(9,4)$ & 5 & 1 & YES & YES & YES & $1.57$ & $(2,3)$ & -- & 5111\\
$(89,33)$ & 10 & $(10,3)$ & 5 & 1 & YES & YES & YES & $1.57$ & $(2,3)$ & -- & 5112\\
$(89,27)$ & 10 & $(11,3)$ & 5 & 1 & YES & YES & YES & $1.43$ & $(2,3)$ & -- & 5113\\
$(89,27)$ & 10 & $(11,4)$ & 5 & 1 & YES & YES & YES & $1.43$ & $(2,3)$ & -- & 5114\\
$(89,34)$ & 9 & $(11,5)$ & 6 & 1 & YES & YES & YES & $1.71$ & $(2,3)$ & NO & 5115\\
$(89,25)$ & 10 & $(12,5)$ & 5 & 1 & YES & YES & YES & $1.57$ & $(2,3)$ & -- & 5116\\
$(89,26)$ & 10 & $(12,5)$ & 5 & 1 & YES & YES & YES & $1.57$ & $(2,3)$ & NO & 5117\\
$(89,27)$ & 10 & $(12,5)$ & 5 & 1 & YES & YES & YES & $1.43$ & $(2,3)$ & -- & 5118\\
$(89,34)$ & 9 & $(12,5)$ & 5 & 1 & YES & YES & YES & $1.43$ & $(2,3)$ & -- & 5119\\
$(89,20)$ & 11 & $(13,4)$ & 6 & 1 & YES & YES & NO(2) & $1.75$ & $(2,3)$ & NO & 5120\\
$(89,26)$ & 10 & $(13,4)$ & 6 & 1 & YES & YES & YES & $1.57$ & $(2,3)$ & -- & 5121\\
$(89,26)$ & 10 & $(13,5)$ & 5 & 1 & YES & YES & YES & $1.57$ & $(2,3)$ & -- & 5122\\
$(89,27)$ & 10 & $(13,4)$ & 6 & 1 & YES & YES & YES & $1.43$ & $(2,3)$ & -- & 5123\\
$(89,27)$ & 10 & $(13,5)$ & 5 & 1 & YES & YES & YES & $1.57$ & $(2,3)$ & NO & 5124\\
$(89,27)$ & 10 & $(17,4)$ & 7 & 1 & YES & YES & YES & $1.57$ & $(2,3)$ & -- & 5125\\
$(89,34)$ & 9 & $(17,7)$ & 6 & 1 & YES & YES & YES & $1.71$ & $(2,3)$ & NO & 5126\\
$(89,26)$ & 10 & $(18,7)$ & 6 & 1 & YES & YES & YES & $1.71$ & $(2,3)$ & NO & 5127\\
$(89,33)$ & 10 & $(25,9)$ & 7 & 1 & YES & YES & NO(2) & $1.75$ & $(2,3)$ & NO & 5128\\
$(89,33)$ & 10 & $(41,15)$ & 8 & 1 & YES & YES & YES & $1.43$ & $(2,3)$ & NO & 5129\\
$(89,27)$ & 10 & $(42,13)$ & 9 & 1 & YES & YES & YES & $1.43$ & $(2,3)$ & NO & 5130\\
$(89,34)$ & 9 & $(45,17)$ & 9 & 1 & YES & YES & YES & $1.57$ & $(2,3)$ & NO & 5131\\
$(89,34)$ & 9 & $(47,18)$ & 8 & 1 & YES & YES & YES & $1.57$ & $(2,3)$ & NO & 5132\\
$(89,24)$ & 10 & $(48,13)$ & 9 & 1 & YES & YES & YES & $1.57$ & $(2,3)$ & 6390 & 5133\\
$(89,34)$ & 9 & $(49,19)$ & 8 & 1 & YES & YES & YES & $1.43$ & $(2,3)$ & NO & 5134\\
$(89,34)$ & 9 & $(50,19)$ & 8 & 1 & YES & YES & YES & $1.71$ & $(2,3)$ & NO & 5135\\
$(89,33)$ & 10 & $(52,19)$ & 9 & 1 & YES & YES & YES & $1.57$ & $(2,3)$ & NO & 5136\\
$(89,27)$ & 10 & $(56,17)$ & 9 & 1 & YES & YES & NO(2) & $1.50$ & $(2,3)$ & NO & 5137\\
$(89,34)$ & 9 & $(60,23)$ & 9 & 1 & YES & YES & YES & $1.43$ & $(2,3)$ & NO & 5138\\
$(89,27)$ & 10 & $(62,19)$ & 10 & 1 & YES & YES & YES & $1.43$ & $(2,3)$ & NO & 5139\\
$(89,34)$ & 9 & $(73,28)$ & 10 & 1 & YES & YES & YES & $1.71$ & $(2,3)$ & 7207 & 5140\\
$(90,37)$ & 11 & $(7,2)$ & 4 & 1 & YES & YES & NO(2) & $1.62$ & $(2,3)$ & -- & 5141\\
$(90,37)$ & 11 & $(7,2)$ & 4 & 1 & YES & YES & NO(2) & $1.75$ & $(2,3)$ & NO & 5142\\
$(90,37)$ & 11 & $(9,2)$ & 5 & 9 & YES & YES & NO(2) & $1.75$ & $(2,3)$ & -- & 5143\\
$(90,37)$ & 11 & $(9,2)$ & 5 & 9 & YES & YES & NO(2) & $1.62$ & $(2,3)$ & NO & 5144\\
$(90,37)$ & 11 & $(11,2)$ & 6 & 1 & YES & YES & NO(2) & $1.62$ & $(2,3)$ & -- & 5145\\
$(90,37)$ & 11 & $(14,3)$ & 6 & 2 & YES & YES & YES & $1.57$ & $(2,3)$ & -- & 5146\\
$(91,40)$ & 10 & $(7,2)$ & 4 & 7 & YES & YES & YES & $1.57$ & $(2,3)$ & -- & 5147\\
$(91,40)$ & 10 & $(8,3)$ & 4 & 1 & YES & YES & YES & $1.57$ & $(2,3)$ & -- & 5148\\
$(91,25)$ & 10 & $(9,2)$ & 5 & 1 & YES & YES & YES & $1.43$ & $(2,3)$ & -- & 5149\\
$(91,25)$ & 10 & $(9,2)$ & 5 & 1 & YES & YES & YES & $1.57$ & $(2,3)$ & NO & 5150\\
$(91,40)$ & 10 & $(9,2)$ & 5 & 1 & YES & YES & YES & $1.43$ & $(2,3)$ & NO & 5151\\
$(91,40)$ & 10 & $(9,2)$ & 5 & 1 & YES & YES & YES & $1.43$ & $(2,3)$ & NO & 5152\\
$(91,40)$ & 10 & $(9,2)$ & 5 & 1 & YES & YES & YES & $1.43$ & $(2,3)$ & -- & 5153\\
$(91,40)$ & 10 & $(11,3)$ & 5 & 1 & YES & YES & YES & $1.71$ & $(2,3)$ & NO & 5154\\
$(91,41)$ & 11 & $(11,2)$ & 6 & 1 & YES & YES & YES & $1.57$ & $(2,3)$ & NO & 5155\\
$(91,41)$ & 11 & $(11,2)$ & 6 & 1 & YES & YES & YES & $1.57$ & $(2,3)$ & -- & 5156\\
$(91,25)$ & 10 & $(13,4)$ & 6 & 13 & YES & YES & YES & $1.57$ & $(2,3)$ & -- & 5157\\
$(91,40)$ & 10 & $(13,5)$ & 5 & 13 & YES & YES & YES & $1.71$ & $(2,3)$ & NO & 5158\\
$(91,25)$ & 10 & $(69,19)$ & 9 & 1 & YES & YES & YES & $1.43$ & $(2,3)$ & NO & 5159\\
$(92,21)$ & 10 & $(3,1)$ & 2 & 1 & YES & YES & YES & $1.43$ & $(2,3)$ & -- & 5160\\
$(92,21)$ & 10 & $(4,1)$ & 3 & 4 & YES & YES & YES & $1.43$ & $(2,3)$ & -- & 5161\\
$(92,35)$ & 10 & $(5,1)$ & 4 & 1 & YES & YES & YES & $1.43$ & $(2,3)$ & NO & 5162\\
$(92,35)$ & 10 & $(5,1)$ & 4 & 1 & YES & YES & YES & $1.43$ & $(2,3)$ & -- & 5163\\
$(92,17)$ & 11 & $(11,5)$ & 6 & 1 & YES & YES & YES & $1.71$ & $(2,3)$ & NO & 5164\\
$(92,35)$ & 10 & $(11,2)$ & 6 & 1 & YES & YES & YES & $1.71$ & $(2,3)$ & -- & 5165\\
$(92,35)$ & 10 & $(13,3)$ & 6 & 1 & YES & YES & YES & $1.57$ & $(2,3)$ & -- & 5166\\
$(92,35)$ & 10 & $(14,3)$ & 6 & 2 & YES & YES & YES & $1.43$ & $(2,3)$ & NO & 5167\\
$(92,35)$ & 10 & $(14,3)$ & 6 & 2 & YES & YES & YES & $1.43$ & $(2,3)$ & -- & 5168\\
$(92,27)$ & 11 & $(23,7)$ & 7 & 23 & YES & YES & YES & $1.71$ & $(2,3)$ & NO & 5169\\
$(92,21)$ & 10 & $(57,13)$ & 9 & 1 & YES & YES & YES & $1.43$ & $(2,3)$ & NO & 5170\\
$(92,21)$ & 10 & $(92,21)$ & 10 & 92 & YES & YES & YES & $1.43$ & $(2,3)$ & NO & 5171\\
$(93,25)$ & 10 & $(4,1)$ & 3 & 1 & YES & YES & NO(2) & $1.38$ & $(2,3)$ & NO & 5172\\
$(93,25)$ & 10 & $(4,1)$ & 3 & 1 & YES & YES & NO(2) & $1.38$ & $(2,3)$ & -- & 5173\\
$(93,34)$ & 10 & $(7,3)$ & 4 & 1 & YES & YES & NO(2) & $1.62$ & $(2,3)$ & -- & 5174\\
$(93,34)$ & 10 & $(8,3)$ & 4 & 1 & YES & YES & NO(2) & $1.62$ & $(2,3)$ & -- & 5175\\
$(93,20)$ & 12 & $(9,4)$ & 5 & 3 & YES & YES & YES & $1.71$ & $(2,3)$ & NO & 5176\\
$(93,41)$ & 10 & $(11,4)$ & 5 & 1 & YES & YES & YES & $1.43$ & $(2,3)$ & NO & 5177\\
$(93,26)$ & 10 & $(13,5)$ & 5 & 1 & YES & YES & YES & $1.71$ & $(2,3)$ & -- & 5178\\
$(93,34)$ & 10 & $(13,4)$ & 6 & 1 & YES & YES & YES & $1.57$ & $(2,3)$ & -- & 5179\\
$(93,20)$ & 12 & $(15,4)$ & 6 & 3 & YES & YES & YES & $1.57$ & $(2,3)$ & NO & 5180\\
$(93,41)$ & 10 & $(19,8)$ & 6 & 1 & YES & YES & YES & $1.43$ & $(2,3)$ & NO & 5181\\
$(93,22)$ & 11 & $(21,8)$ & 6 & 3 & YES & YES & YES & $1.57$ & $(2,3)$ & -- & 5182\\
$(93,34)$ & 10 & $(23,9)$ & 7 & 1 & YES & YES & YES & $1.57$ & $(2,3)$ & NO & 5183\\
$(93,38)$ & 11 & $(29,12)$ & 7 & 1 & YES & YES & NO(2) & $1.50$ & $(2,3)$ & NO & 5184\\
$(93,34)$ & 10 & $(35,13)$ & 8 & 1 & YES & YES & NO(2) & $1.75$ & $(2,3)$ & NO & 5185\\
$(93,25)$ & 10 & $(67,18)$ & 9 & 1 & YES & YES & YES & $1.29$ & $(2,3)$ & NO & 5186\\
$(94,41)$ & 10 & $(2,1)$ & 1 & 2 & YES & YES & YES & $1.29$ & $(2,3)$ & -- & 5187\\
$(94,39)$ & 10 & $(7,2)$ & 4 & 1 & YES & YES & YES & $1.43$ & $(2,3)$ & -- & 5188\\
$(94,39)$ & 10 & $(10,3)$ & 5 & 2 & YES & YES & YES & $1.57$ & $(2,3)$ & -- & 5189\\
$(94,39)$ & 10 & $(10,3)$ & 5 & 2 & YES & YES & YES & $1.57$ & $(2,3)$ & NO & 5190\\
$(94,39)$ & 10 & $(13,3)$ & 6 & 1 & YES & YES & NO(2) & $1.62$ & $(2,3)$ & -- & 5191\\
$(94,41)$ & 10 & $(13,5)$ & 5 & 1 & YES & YES & YES & $1.57$ & $(2,3)$ & -- & 5192\\
$(94,39)$ & 10 & $(17,4)$ & 7 & 1 & YES & YES & YES & $1.57$ & $(2,3)$ & -- & 5193\\
$(94,39)$ & 10 & $(19,7)$ & 6 & 1 & YES & YES & YES & $1.71$ & $(2,3)$ & NO & 5194\\
$(94,39)$ & 10 & $(21,8)$ & 6 & 1 & YES & YES & YES & $1.71$ & $(2,3)$ & NO & 5195\\
$(94,39)$ & 10 & $(22,9)$ & 7 & 2 & YES & YES & NO(2) & $1.62$ & $(2,3)$ & NO & 5196\\
$(94,39)$ & 10 & $(39,16)$ & 8 & 1 & YES & YES & NO(2) & $1.62$ & $(2,3)$ & NO & 5197\\
$(94,39)$ & 10 & $(46,19)$ & 8 & 2 & YES & YES & NO(2) & $1.62$ & $(2,3)$ & NO & 5198\\
$(94,41)$ & 10 & $(59,26)$ & 9 & 1 & YES & YES & YES & $1.71$ & $(2,3)$ & NO & 5199\\
$(94,39)$ & 10 & $(63,26)$ & 9 & 1 & YES & YES & YES & $1.57$ & $(2,3)$ & 5554 & 5200\\
$(95,36)$ & 10 & $(4,1)$ & 3 & 1 & YES & YES & YES & $1.43$ & $(2,3)$ & NO & 5201\\
$(95,36)$ & 10 & $(4,1)$ & 3 & 1 & YES & YES & YES & $1.43$ & $(2,3)$ & -- & 5202\\
$(95,36)$ & 10 & $(7,3)$ & 4 & 1 & YES & YES & NO(2) & $1.62$ & $(2,3)$ & -- & 5203\\
$(95,39)$ & 10 & $(7,2)$ & 4 & 1 & YES & YES & YES & $1.29$ & $(2,3)$ & -- & 5204\\
$(95,39)$ & 10 & $(7,2)$ & 4 & 1 & YES & YES & NO(2) & $1.43$ & $(4,2)$ & NO & 5205\\
$(95,39)$ & 10 & $(7,3)$ & 4 & 1 & YES & YES & NO(2) & $1.75$ & $(2,3)$ & -- & 5206\\
$(95,39)$ & 10 & $(9,4)$ & 5 & 1 & YES & YES & YES & $1.57$ & $(2,3)$ & -- & 5207\\
$(95,36)$ & 10 & $(10,3)$ & 5 & 5 & YES & YES & YES & $1.57$ & $(2,3)$ & -- & 5208\\
$(95,39)$ & 10 & $(10,3)$ & 5 & 5 & YES & YES & YES & $1.43$ & $(2,3)$ & -- & 5209\\
$(95,39)$ & 10 & $(10,3)$ & 5 & 5 & YES & YES & NO(2) & $1.50$ & $(2,3)$ & NO & 5210\\
$(95,28)$ & 11 & $(11,4)$ & 5 & 1 & YES & YES & YES & $1.71$ & $(2,3)$ & NO & 5211\\
$(95,39)$ & 10 & $(11,3)$ & 5 & 1 & YES & YES & YES & $1.43$ & $(2,3)$ & -- & 5212\\
$(95,39)$ & 10 & $(11,4)$ & 5 & 1 & YES & YES & YES & $1.71$ & $(2,3)$ & -- & 5213\\
$(95,28)$ & 11 & $(13,2)$ & 7 & 1 & YES & YES & YES & $1.43$ & $(2,3)$ & NO & 5214\\
$(95,39)$ & 10 & $(17,4)$ & 7 & 1 & YES & YES & YES & $1.57$ & $(2,3)$ & -- & 5215\\
$(95,36)$ & 10 & $(18,7)$ & 6 & 1 & YES & YES & YES & $1.57$ & $(2,3)$ & NO & 5216\\
$(95,39)$ & 10 & $(23,10)$ & 7 & 1 & YES & YES & YES & $1.57$ & $(2,3)$ & NO & 5217\\
$(95,39)$ & 10 & $(26,11)$ & 7 & 1 & YES & YES & YES & $1.43$ & $(2,3)$ & NO & 5218\\
$(95,39)$ & 10 & $(31,13)$ & 7 & 1 & YES & YES & YES & $1.43$ & $(2,3)$ & NO & 5219\\
$(95,39)$ & 10 & $(53,22)$ & 9 & 1 & YES & YES & YES & $1.57$ & $(2,3)$ & NO & 5220\\
$(95,39)$ & 10 & $(73,30)$ & 10 & 1 & YES & YES & NO(2) & $1.62$ & $(2,3)$ & NO & 5221\\
$(96,29)$ & 11 & $(7,3)$ & 4 & 1 & YES & YES & NO(2) & $1.75$ & $(2,3)$ & -- & 5222\\
$(97,35)$ & 10 & $(5,2)$ & 3 & 1 & YES & YES & NO(2) & $1.50$ & $(2,3)$ & NO & 5223\\
$(97,35)$ & 10 & $(5,2)$ & 3 & 1 & YES & YES & NO(2) & $1.50$ & $(2,3)$ & -- & 5224\\
$(97,36)$ & 10 & $(5,2)$ & 3 & 1 & YES & YES & YES & $1.57$ & $(2,3)$ & NO & 5225\\
$(97,36)$ & 10 & $(5,2)$ & 3 & 1 & YES & YES & YES & $1.57$ & $(2,3)$ & -- & 5226\\
$(97,35)$ & 10 & $(7,3)$ & 4 & 1 & YES & YES & NO(2) & $1.50$ & $(2,3)$ & NO & 5227\\
$(97,36)$ & 10 & $(7,3)$ & 4 & 1 & YES & YES & NO(2) & $1.62$ & $(2,3)$ & -- & 5228\\
$(97,36)$ & 10 & $(7,3)$ & 4 & 1 & YES & YES & NO(2) & $1.62$ & $(2,3)$ & NO & 5229\\
$(97,37)$ & 10 & $(7,3)$ & 4 & 1 & YES & YES & YES & $1.57$ & $(2,3)$ & -- & 5230\\
$(97,41)$ & 10 & $(7,3)$ & 4 & 1 & YES & YES & YES & $1.71$ & $(2,3)$ & -- & 5231\\
$(97,35)$ & 10 & $(8,3)$ & 4 & 1 & YES & YES & YES & $1.57$ & $(2,3)$ & NO & 5232\\
$(97,35)$ & 10 & $(8,3)$ & 4 & 1 & YES & YES & YES & $1.57$ & $(2,3)$ & -- & 5233\\
$(97,37)$ & 10 & $(8,3)$ & 4 & 1 & YES & YES & YES & $1.57$ & $(2,3)$ & -- & 5234\\
$(97,36)$ & 10 & $(9,4)$ & 5 & 1 & YES & YES & YES & $1.71$ & $(2,3)$ & -- & 5235\\
$(97,26)$ & 10 & $(10,3)$ & 5 & 1 & YES & YES & YES & $1.43$ & $(2,3)$ & -- & 5236\\
$(97,36)$ & 10 & $(10,3)$ & 5 & 1 & YES & YES & YES & $1.71$ & $(2,3)$ & NO & 5237\\
$(97,36)$ & 10 & $(10,3)$ & 5 & 1 & YES & YES & YES & $1.71$ & $(2,3)$ & -- & 5238\\
$(97,37)$ & 10 & $(10,3)$ & 5 & 1 & YES & YES & YES & $1.71$ & $(2,3)$ & NO & 5239\\
$(97,41)$ & 10 & $(10,3)$ & 5 & 1 & YES & YES & YES & $1.57$ & $(2,3)$ & NO & 5240\\
$(97,41)$ & 10 & $(10,3)$ & 5 & 1 & YES & YES & YES & $1.57$ & $(2,3)$ & -- & 5241\\
$(97,30)$ & 11 & $(11,3)$ & 5 & 1 & YES & YES & YES & $1.43$ & $(2,3)$ & -- & 5242\\
$(97,22)$ & 11 & $(12,5)$ & 5 & 1 & YES & YES & YES & $1.57$ & $(2,3)$ & NO & 5243\\
$(97,23)$ & 11 & $(12,5)$ & 5 & 1 & YES & YES & YES & $1.57$ & $(2,3)$ & -- & 5244\\
$(97,41)$ & 10 & $(13,3)$ & 6 & 1 & YES & YES & YES & $1.57$ & $(2,3)$ & NO & 5245\\
$(97,37)$ & 10 & $(14,3)$ & 6 & 1 & YES & YES & YES & $1.43$ & $(2,3)$ & NO & 5246\\
$(97,37)$ & 10 & $(14,3)$ & 6 & 1 & YES & YES & YES & $1.43$ & $(2,3)$ & -- & 5247\\
$(97,41)$ & 10 & $(14,5)$ & 6 & 1 & YES & YES & YES & $1.57$ & $(2,3)$ & NO & 5248\\
$(97,22)$ & 11 & $(17,4)$ & 7 & 1 & YES & YES & YES & $1.43$ & $(2,3)$ & -- & 5249\\
$(97,22)$ & 11 & $(17,5)$ & 6 & 1 & YES & YES & YES & $1.57$ & $(2,3)$ & NO & 5250\\
$(97,35)$ & 10 & $(17,7)$ & 6 & 1 & YES & YES & YES & $1.86$ & $(2,3)$ & NO & 5251\\
$(97,22)$ & 11 & $(19,4)$ & 7 & 1 & YES & YES & YES & $1.43$ & $(2,3)$ & -- & 5252\\
$(97,41)$ & 10 & $(22,9)$ & 7 & 1 & YES & YES & YES & $1.57$ & $(2,3)$ & NO & 5253\\
$(97,26)$ & 10 & $(23,7)$ & 7 & 1 & YES & YES & YES & $1.71$ & $(2,3)$ & NO & 5254\\
$(97,38)$ & 11 & $(31,12)$ & 7 & 1 & YES & YES & YES & $1.57$ & $(2,3)$ & NO & 5255\\
$(97,35)$ & 10 & $(35,13)$ & 8 & 1 & YES & YES & YES & $1.86$ & $(2,3)$ & NO & 5256\\
$(97,41)$ & 10 & $(43,18)$ & 8 & 1 & YES & YES & YES & $1.57$ & $(2,3)$ & NO & 5257\\
$(97,36)$ & 10 & $(46,17)$ & 8 & 1 & YES & YES & YES & $1.43$ & $(2,3)$ & NO & 5258\\
$(97,30)$ & 11 & $(61,19)$ & 10 & 1 & YES & YES & YES & $1.43$ & $(2,3)$ & 5451 & 5259\\
$(97,37)$ & 10 & $(81,31)$ & 9 & 1 & YES & YES & YES & $1.57$ & $(2,3)$ & 8032 & 5260\\
$(97,35)$ & 10 & $(86,31)$ & 10 & 1 & YES & YES & YES & $1.29$ & $(2,3)$ & NO & 5261\\
$(97,37)$ & 10 & $(89,34)$ & 9 & 1 & YES & YES & YES & $1.57$ & $(2,3)$ & 6804 & 5262\\
$(98,27)$ & 10 & $(4,1)$ & 3 & 2 & YES & YES & YES & $1.43$ & $(2,3)$ & NO & 5263\\
$(98,27)$ & 10 & $(4,1)$ & 3 & 2 & YES & YES & YES & $1.43$ & $(2,3)$ & -- & 5264\\
$(98,27)$ & 10 & $(7,3)$ & 4 & 7 & YES & YES & YES & $1.57$ & $(2,3)$ & NO & 5265\\
$(98,29)$ & 10 & $(7,3)$ & 4 & 7 & YES & YES & YES & $1.71$ & $(2,3)$ & NO & 5266\\
$(98,29)$ & 10 & $(7,3)$ & 4 & 7 & YES & YES & YES & $1.71$ & $(2,3)$ & -- & 5267\\
$(98,41)$ & 10 & $(7,3)$ & 4 & 7 & YES & YES & NO(2) & $1.62$ & $(2,3)$ & -- & 5268\\
$(98,43)$ & 10 & $(7,3)$ & 4 & 7 & YES & YES & YES & $1.57$ & $(2,3)$ & -- & 5269\\
$(98,29)$ & 10 & $(10,3)$ & 5 & 2 & YES & YES & YES & $1.57$ & $(2,3)$ & -- & 5270\\
$(98,41)$ & 10 & $(11,3)$ & 5 & 1 & YES & YES & YES & $1.57$ & $(2,3)$ & NO & 5271\\
$(98,41)$ & 10 & $(13,3)$ & 6 & 1 & YES & YES & YES & $1.57$ & $(2,3)$ & NO & 5272\\
$(98,41)$ & 10 & $(13,3)$ & 6 & 1 & YES & YES & YES & $1.57$ & $(2,3)$ & -- & 5273\\
$(98,41)$ & 10 & $(13,5)$ & 5 & 1 & YES & YES & NO(2) & $1.62$ & $(2,3)$ & NO & 5274\\
$(98,27)$ & 10 & $(17,5)$ & 6 & 1 & YES & YES & YES & $1.57$ & $(2,3)$ & NO & 5275\\
$(98,27)$ & 10 & $(17,7)$ & 6 & 1 & YES & YES & YES & $1.57$ & $(2,3)$ & -- & 5276\\
$(98,27)$ & 10 & $(17,7)$ & 6 & 1 & YES & YES & YES & $1.71$ & $(2,3)$ & NO & 5277\\
$(98,29)$ & 10 & $(18,5)$ & 6 & 2 & YES & YES & YES & $1.57$ & $(2,3)$ & NO & 5278\\
$(98,41)$ & 10 & $(22,9)$ & 7 & 2 & YES & YES & NO(2) & $1.62$ & $(2,3)$ & NO & 5279\\
$(98,29)$ & 10 & $(29,8)$ & 7 & 1 & YES & YES & YES & $1.43$ & $(2,3)$ & NO & 5280\\
$(98,29)$ & 10 & $(31,9)$ & 8 & 1 & YES & YES & YES & $1.57$ & $(2,3)$ & NO & 5281\\
$(98,27)$ & 10 & $(40,11)$ & 8 & 2 & YES & YES & YES & $1.43$ & $(2,3)$ & 5549 & 5282\\
$(98,41)$ & 10 & $(45,19)$ & 8 & 1 & YES & YES & YES & $1.57$ & $(2,3)$ & NO & 5283\\
$(98,29)$ & 10 & $(55,16)$ & 9 & 1 & YES & YES & YES & $1.43$ & $(2,3)$ & NO & 5284\\
$(98,27)$ & 10 & $(69,19)$ & 9 & 1 & YES & YES & YES & $1.43$ & $(2,3)$ & NO & 5285\\
$(98,41)$ & 10 & $(69,29)$ & 9 & 1 & YES & YES & YES & $1.57$ & $(2,3)$ & 5777 & 5286\\
$(98,41)$ & 10 & $(79,33)$ & 11 & 1 & YES & YES & YES & $1.57$ & $(2,3)$ & 6284 & 5287\\
$(99,29)$ & 10 & $(3,1)$ & 2 & 3 & YES & YES & YES & $1.43$ & $(2,3)$ & -- & 5288\\
$(99,41)$ & 10 & $(5,2)$ & 3 & 1 & YES & YES & YES & $1.57$ & $(2,3)$ & -- & 5289\\
$(99,41)$ & 10 & $(7,3)$ & 4 & 1 & YES & YES & YES & $1.57$ & $(2,3)$ & -- & 5290\\
$(99,29)$ & 10 & $(8,3)$ & 4 & 1 & YES & YES & YES & $1.57$ & $(2,3)$ & -- & 5291\\
$(99,26)$ & 12 & $(10,3)$ & 5 & 1 & YES & YES & YES & $1.57$ & $(2,3)$ & -- & 5292\\
$(99,29)$ & 10 & $(12,5)$ & 5 & 3 & YES & YES & YES & $1.57$ & $(2,3)$ & -- & 5293\\
$(99,29)$ & 10 & $(12,5)$ & 5 & 3 & YES & YES & YES & $1.71$ & $(2,3)$ & NO & 5294\\
$(99,29)$ & 10 & $(12,5)$ & 5 & 3 & YES & YES & YES & $1.71$ & $(2,3)$ & NO & 5295\\
$(99,26)$ & 12 & $(13,3)$ & 6 & 1 & YES & YES & YES & $1.57$ & $(2,3)$ & -- & 5296\\
$(99,29)$ & 10 & $(13,4)$ & 6 & 1 & YES & YES & YES & $1.57$ & $(2,3)$ & -- & 5297\\
$(99,41)$ & 10 & $(22,5)$ & 7 & 11 & YES & YES & YES & $1.57$ & $(2,3)$ & -- & 5298\\
$(99,29)$ & 10 & $(24,7)$ & 7 & 3 & YES & YES & YES & $1.57$ & $(2,3)$ & NO & 5299\\
$(99,29)$ & 10 & $(26,7)$ & 7 & 1 & YES & YES & YES & $1.57$ & $(2,3)$ & NO & 5300\\
$(99,29)$ & 10 & $(64,19)$ & 9 & 1 & YES & YES & YES & $1.71$ & $(2,3)$ & NO & 5301\\
$(99,29)$ & 10 & $(79,23)$ & 10 & 1 & YES & YES & YES & $1.57$ & $(2,3)$ & NO & 5302\\
$(100,27)$ & 10 & $(2,1)$ & 1 & 2 & YES & YES & YES & $1.43$ & $(2,3)$ & -- & 5303\\
$(100,27)$ & 10 & $(2,1)$ & 1 & 2 & YES & YES & YES & $1.57$ & $(2,3)$ & NO & 5304\\
$(100,41)$ & 10 & $(2,1)$ & 1 & 2 & YES & YES & NO(2) & $1.38$ & $(2,3)$ & -- & 5305\\
$(100,27)$ & 10 & $(3,1)$ & 2 & 1 & YES & YES & YES & $1.43$ & $(2,3)$ & -- & 5306\\
$(100,39)$ & 10 & $(7,3)$ & 4 & 1 & YES & YES & NO(2) & $1.62$ & $(2,3)$ & -- & 5307\\
$(100,41)$ & 10 & $(7,3)$ & 4 & 1 & YES & YES & YES & $1.43$ & $(2,3)$ & -- & 5308\\
$(100,27)$ & 10 & $(8,3)$ & 4 & 4 & YES & YES & YES & $1.29$ & $(2,3)$ & NO & 5309\\
$(100,37)$ & 10 & $(8,3)$ & 4 & 4 & YES & YES & YES & $1.57$ & $(2,3)$ & NO & 5310\\
$(100,37)$ & 10 & $(8,3)$ & 4 & 4 & YES & YES & YES & $1.57$ & $(2,3)$ & -- & 5311\\
$(100,27)$ & 10 & $(9,2)$ & 5 & 1 & YES & YES & YES & $1.43$ & $(2,3)$ & -- & 5312\\
$(100,27)$ & 10 & $(9,2)$ & 5 & 1 & YES & YES & YES & $1.43$ & $(2,3)$ & NO & 5313\\
$(100,27)$ & 10 & $(9,4)$ & 5 & 1 & YES & YES & YES & $1.57$ & $(2,3)$ & NO & 5314\\
$(100,27)$ & 10 & $(12,5)$ & 5 & 4 & YES & YES & YES & $1.43$ & $(2,3)$ & NO & 5315\\
$(100,37)$ & 10 & $(13,5)$ & 5 & 1 & YES & YES & YES & $1.43$ & $(2,3)$ & NO & 5316\\
$(100,39)$ & 10 & $(17,4)$ & 7 & 1 & YES & YES & YES & $1.57$ & $(2,3)$ & -- & 5317\\
$(100,29)$ & 11 & $(19,6)$ & 8 & 1 & YES & YES & YES & $1.71$ & $(2,3)$ & NO & 5318\\
$(100,39)$ & 10 & $(19,7)$ & 6 & 1 & YES & YES & NO(2) & $1.62$ & $(2,3)$ & NO & 5319\\
$(100,41)$ & 10 & $(29,12)$ & 7 & 1 & YES & YES & YES & $1.43$ & $(2,3)$ & NO & 5320\\
$(100,41)$ & 10 & $(31,13)$ & 7 & 1 & YES & YES & YES & $1.57$ & $(2,3)$ & NO & 5321\\
$(100,39)$ & 10 & $(44,17)$ & 8 & 4 & YES & YES & NO(2) & $1.62$ & $(2,3)$ & NO & 5322\\
$(100,41)$ & 10 & $(46,19)$ & 8 & 2 & YES & YES & YES & $1.43$ & $(2,3)$ & NO & 5323\\
$(100,27)$ & 10 & $(59,16)$ & 10 & 1 & YES & YES & YES & $1.57$ & $(2,3)$ & NO & 5324\\
$(100,27)$ & 10 & $(63,17)$ & 9 & 1 & YES & YES & YES & $1.43$ & $(2,3)$ & NO & 5325\\
$(100,37)$ & 10 & $(78,29)$ & 10 & 2 & YES & YES & YES & $1.57$ & $(2,3)$ & NO & 5326\\
$(101,30)$ & 10 & $(4,1)$ & 3 & 1 & YES & YES & YES & $1.71$ & $(2,3)$ & NO & 5327\\
$(101,23)$ & 11 & $(7,3)$ & 4 & 1 & YES & YES & NO(2) & $1.62$ & $(2,3)$ & NO & 5328\\
$(101,30)$ & 10 & $(7,2)$ & 4 & 1 & YES & YES & YES & $1.57$ & $(2,3)$ & NO & 5329\\
$(101,30)$ & 10 & $(7,2)$ & 4 & 1 & YES & YES & YES & $1.57$ & $(2,3)$ & -- & 5330\\
$(101,30)$ & 10 & $(7,3)$ & 4 & 1 & YES & YES & YES & $1.57$ & $(2,3)$ & -- & 5331\\
$(101,39)$ & 10 & $(7,2)$ & 4 & 1 & YES & YES & YES & $1.43$ & $(2,3)$ & -- & 5332\\
$(101,28)$ & 11 & $(8,3)$ & 4 & 1 & YES & YES & YES & $1.71$ & $(2,3)$ & NO & 5333\\
$(101,39)$ & 10 & $(8,3)$ & 4 & 1 & YES & YES & YES & $1.71$ & $(2,3)$ & -- & 5334\\
$(101,44)$ & 10 & $(8,3)$ & 4 & 1 & YES & YES & YES & $1.57$ & $(2,3)$ & -- & 5335\\
$(101,30)$ & 10 & $(9,2)$ & 5 & 1 & YES & YES & YES & $1.43$ & $(2,3)$ & -- & 5336\\
$(101,30)$ & 10 & $(9,2)$ & 5 & 1 & YES & YES & YES & $1.43$ & $(2,3)$ & NO & 5337\\
$(101,30)$ & 10 & $(10,3)$ & 5 & 1 & YES & YES & YES & $1.43$ & $(2,3)$ & -- & 5338\\
$(101,30)$ & 10 & $(11,3)$ & 5 & 1 & YES & YES & YES & $1.29$ & $(2,3)$ & -- & 5339\\
$(101,30)$ & 10 & $(11,4)$ & 5 & 1 & YES & YES & YES & $1.43$ & $(2,3)$ & -- & 5340\\
$(101,44)$ & 10 & $(12,5)$ & 5 & 1 & YES & YES & YES & $1.57$ & $(2,3)$ & -- & 5341\\
$(101,28)$ & 11 & $(13,2)$ & 7 & 1 & YES & YES & YES & $1.57$ & $(2,3)$ & NO & 5342\\
$(101,28)$ & 11 & $(13,4)$ & 6 & 1 & YES & YES & YES & $1.71$ & $(2,3)$ & NO & 5343\\
$(101,39)$ & 10 & $(13,3)$ & 6 & 1 & YES & YES & NO(2) & $1.62$ & $(2,3)$ & -- & 5344\\
$(101,39)$ & 10 & $(16,7)$ & 6 & 1 & YES & YES & YES & $1.71$ & $(2,3)$ & NO & 5345\\
$(101,39)$ & 10 & $(17,4)$ & 7 & 1 & YES & YES & YES & $1.57$ & $(2,3)$ & -- & 5346\\
$(101,39)$ & 10 & $(19,8)$ & 6 & 1 & YES & YES & YES & $1.71$ & $(2,3)$ & NO & 5347\\
$(101,44)$ & 10 & $(20,9)$ & 7 & 1 & YES & YES & YES & $1.43$ & $(2,3)$ & NO & 5348\\
$(101,30)$ & 10 & $(22,5)$ & 7 & 1 & YES & YES & YES & $1.43$ & $(2,3)$ & NO & 5349\\
$(101,39)$ & 10 & $(23,9)$ & 7 & 1 & YES & YES & YES & $1.43$ & $(2,3)$ & NO & 5350\\
$(101,30)$ & 10 & $(26,7)$ & 7 & 1 & YES & YES & YES & $1.43$ & $(2,3)$ & NO & 5351\\
$(101,39)$ & 10 & $(41,16)$ & 8 & 1 & YES & YES & NO(2) & $1.62$ & $(2,3)$ & NO & 5352\\
$(101,22)$ & 11 & $(47,10)$ & 9 & 1 & YES & YES & NO(2) & $1.75$ & $(2,3)$ & NO & 5353\\
$(101,30)$ & 10 & $(91,27)$ & 10 & 1 & YES & YES & YES & $1.43$ & $(2,3)$ & NO & 5354\\
$(102,31)$ & 11 & $(8,3)$ & 4 & 2 & YES & YES & YES & $1.71$ & $(2,3)$ & -- & 5355\\
$(102,31)$ & 11 & $(10,3)$ & 5 & 2 & YES & YES & YES & $1.71$ & $(2,3)$ & -- & 5356\\
$(102,19)$ & 11 & $(11,5)$ & 6 & 1 & YES & YES & YES & $1.57$ & $(2,3)$ & -- & 5357\\
$(102,23)$ & 11 & $(15,4)$ & 6 & 3 & YES & YES & YES & $1.57$ & $(2,3)$ & -- & 5358\\
$(103,39)$ & 10 & $(5,2)$ & 3 & 1 & YES & YES & YES & $1.57$ & $(2,3)$ & -- & 5359\\
$(103,39)$ & 10 & $(5,2)$ & 3 & 1 & YES & YES & YES & $1.57$ & $(2,3)$ & NO & 5360\\
$(103,30)$ & 11 & $(7,2)$ & 4 & 1 & YES & YES & NO(2) & $1.75$ & $(2,3)$ & -- & 5361\\
$(103,32)$ & 11 & $(7,3)$ & 4 & 1 & YES & YES & YES & $1.43$ & $(2,3)$ & -- & 5362\\
$(103,37)$ & 10 & $(7,2)$ & 4 & 1 & YES & YES & YES & $1.29$ & $(2,3)$ & -- & 5363\\
$(103,37)$ & 10 & $(7,3)$ & 4 & 1 & YES & YES & YES & $1.57$ & $(2,3)$ & -- & 5364\\
$(103,39)$ & 10 & $(7,2)$ & 4 & 1 & YES & YES & YES & $1.43$ & $(2,3)$ & -- & 5365\\
$(103,39)$ & 10 & $(7,3)$ & 4 & 1 & YES & YES & YES & $1.57$ & $(2,3)$ & -- & 5366\\
$(103,39)$ & 10 & $(9,2)$ & 5 & 1 & YES & YES & YES & $1.43$ & $(2,3)$ & NO & 5367\\
$(103,39)$ & 10 & $(9,2)$ & 5 & 1 & YES & YES & YES & $1.43$ & $(2,3)$ & -- & 5368\\
$(103,30)$ & 11 & $(10,3)$ & 5 & 1 & YES & YES & YES & $1.57$ & $(2,3)$ & -- & 5369\\
$(103,39)$ & 10 & $(10,3)$ & 5 & 1 & YES & YES & YES & $1.57$ & $(2,3)$ & -- & 5370\\
$(103,19)$ & 11 & $(11,5)$ & 6 & 1 & YES & YES & YES & $1.57$ & $(2,3)$ & -- & 5371\\
$(103,24)$ & 11 & $(11,4)$ & 5 & 1 & YES & YES & NO(2) & $1.50$ & $(2,3)$ & -- & 5372\\
$(103,37)$ & 10 & $(12,5)$ & 5 & 1 & YES & YES & YES & $1.57$ & $(2,3)$ & NO & 5373\\
$(103,37)$ & 10 & $(13,5)$ & 5 & 1 & YES & YES & YES & $1.43$ & $(2,3)$ & 6660 & 5374\\
$(103,38)$ & 11 & $(14,3)$ & 6 & 1 & YES & YES & YES & $1.71$ & $(2,3)$ & NO & 5375\\
$(103,24)$ & 11 & $(17,4)$ & 7 & 1 & YES & YES & YES & $1.43$ & $(2,3)$ & -- & 5376\\
$(103,29)$ & 11 & $(17,4)$ & 7 & 1 & YES & YES & YES & $1.71$ & $(2,3)$ & NO & 5377\\
$(103,29)$ & 11 & $(23,6)$ & 8 & 1 & YES & YES & YES & $1.71$ & $(2,3)$ & NO & 5378\\
$(103,39)$ & 10 & $(23,9)$ & 7 & 1 & YES & YES & YES & $1.43$ & $(2,3)$ & NO & 5379\\
$(103,38)$ & 11 & $(25,9)$ & 7 & 1 & YES & YES & YES & $1.71$ & $(2,3)$ & NO & 5380\\
$(103,39)$ & 10 & $(34,13)$ & 7 & 1 & YES & YES & YES & $1.43$ & $(2,3)$ & 6714 & 5381\\
$(103,30)$ & 11 & $(37,11)$ & 8 & 1 & YES & YES & YES & $1.43$ & $(2,3)$ & NO & 5382\\
$(103,30)$ & 11 & $(44,13)$ & 8 & 1 & YES & YES & YES & $1.43$ & $(2,3)$ & NO & 5383\\
$(103,39)$ & 10 & $(50,19)$ & 8 & 1 & YES & YES & YES & $1.43$ & $(2,3)$ & NO & 5384\\
$(103,32)$ & 11 & $(51,16)$ & 10 & 1 & YES & YES & YES & $1.71$ & $(2,3)$ & NO & 5385\\
$(103,39)$ & 10 & $(55,21)$ & 8 & 1 & YES & YES & YES & $1.43$ & $(2,3)$ & NO & 5386\\
$(103,39)$ & 10 & $(71,27)$ & 9 & 1 & YES & YES & YES & $1.57$ & $(2,3)$ & NO & 5387\\
$(103,30)$ & 11 & $(86,25)$ & 10 & 1 & YES & YES & YES & $1.71$ & $(2,3)$ & 6994 & 5388\\
$(103,39)$ & 10 & $(95,36)$ & 10 & 1 & YES & YES & YES & $1.57$ & $(2,3)$ & NO & 5389\\
$(104,29)$ & 10 & $(4,1)$ & 3 & 4 & YES & YES & YES & $1.43$ & $(2,3)$ & NO & 5390\\
$(104,29)$ & 10 & $(4,1)$ & 3 & 4 & YES & YES & YES & $1.43$ & $(2,3)$ & -- & 5391\\
$(104,43)$ & 10 & $(4,1)$ & 3 & 4 & YES & YES & YES & $1.57$ & $(2,3)$ & -- & 5392\\
$(104,43)$ & 10 & $(5,2)$ & 3 & 1 & YES & YES & YES & $1.57$ & $(2,3)$ & -- & 5393\\
$(104,43)$ & 10 & $(9,2)$ & 5 & 1 & YES & YES & YES & $1.57$ & $(2,3)$ & NO & 5394\\
$(104,43)$ & 10 & $(9,2)$ & 5 & 1 & YES & YES & YES & $1.57$ & $(2,3)$ & -- & 5395\\
$(104,43)$ & 10 & $(9,2)$ & 5 & 1 & YES & YES & YES & $1.57$ & $(2,3)$ & NO & 5396\\
$(104,43)$ & 10 & $(11,3)$ & 5 & 1 & YES & YES & YES & $1.43$ & $(2,3)$ & -- & 5397\\
$(104,31)$ & 11 & $(13,3)$ & 6 & 13 & YES & YES & YES & $1.57$ & $(2,3)$ & NO & 5398\\
$(104,43)$ & 10 & $(13,5)$ & 5 & 13 & YES & YES & YES & $1.57$ & $(2,3)$ & NO & 5399\\
$(104,43)$ & 10 & $(14,3)$ & 6 & 2 & YES & YES & YES & $1.57$ & $(2,3)$ & NO & 5400\\
$(104,43)$ & 10 & $(14,3)$ & 6 & 2 & YES & YES & YES & $1.71$ & $(2,3)$ & -- & 5401\\
$(104,43)$ & 10 & $(16,3)$ & 7 & 8 & YES & YES & YES & $1.57$ & $(2,3)$ & -- & 5402\\
$(104,31)$ & 11 & $(17,4)$ & 7 & 1 & YES & YES & YES & $1.57$ & $(2,3)$ & NO & 5403\\
$(104,43)$ & 10 & $(22,9)$ & 7 & 2 & YES & YES & YES & $1.57$ & $(2,3)$ & 6081 & 5404\\
$(104,43)$ & 10 & $(27,11)$ & 8 & 1 & YES & YES & YES & $1.57$ & $(2,3)$ & NO & 5405\\
$(104,31)$ & 11 & $(31,9)$ & 8 & 1 & YES & YES & YES & $1.57$ & $(2,3)$ & NO & 5406\\
$(104,43)$ & 10 & $(31,13)$ & 7 & 1 & YES & YES & YES & $1.57$ & $(2,3)$ & NO & 5407\\
$(105,32)$ & 11 & $(2,1)$ & 1 & 1 & YES & YES & NO(2) & $1.25$ & $(2,3)$ & -- & 5408\\
$(105,29)$ & 10 & $(4,1)$ & 3 & 1 & YES & YES & YES & $1.57$ & $(2,3)$ & -- & 5409\\
$(105,31)$ & 10 & $(4,1)$ & 3 & 1 & YES & YES & YES & $1.57$ & $(2,3)$ & NO & 5410\\
$(105,31)$ & 10 & $(4,1)$ & 3 & 1 & YES & YES & YES & $1.57$ & $(2,3)$ & -- & 5411\\
$(105,31)$ & 10 & $(4,1)$ & 3 & 1 & YES & YES & YES & $1.57$ & $(2,3)$ & NO & 5412\\
$(105,38)$ & 11 & $(6,1)$ & 5 & 3 & YES & YES & YES & $1.86$ & $(2,3)$ & NO & 5413\\
$(105,38)$ & 11 & $(6,1)$ & 5 & 3 & YES & YES & YES & $1.86$ & $(2,3)$ & -- & 5414\\
$(105,38)$ & 11 & $(6,1)$ & 5 & 3 & YES & YES & YES & $1.86$ & $(2,3)$ & NO & 5415\\
$(105,41)$ & 10 & $(7,2)$ & 4 & 7 & YES & YES & NO(2) & $1.38$ & $(2,3)$ & NO & 5416\\
$(105,41)$ & 10 & $(7,3)$ & 4 & 7 & YES & YES & YES & $1.57$ & $(2,3)$ & -- & 5417\\
$(105,44)$ & 10 & $(7,3)$ & 4 & 7 & YES & YES & YES & $1.57$ & $(2,3)$ & -- & 5418\\
$(105,44)$ & 10 & $(8,3)$ & 4 & 1 & YES & YES & YES & $1.57$ & $(2,3)$ & -- & 5419\\
$(105,31)$ & 10 & $(10,3)$ & 5 & 5 & YES & YES & YES & $1.29$ & $(2,3)$ & -- & 5420\\
$(105,31)$ & 10 & $(11,4)$ & 5 & 1 & YES & YES & YES & $1.43$ & $(2,3)$ & -- & 5421\\
$(105,38)$ & 11 & $(11,4)$ & 5 & 1 & YES & YES & YES & $1.71$ & $(2,3)$ & -- & 5422\\
$(105,41)$ & 10 & $(11,4)$ & 5 & 1 & YES & YES & NO(2) & $1.38$ & $(2,3)$ & NO & 5423\\
$(105,44)$ & 10 & $(17,7)$ & 6 & 1 & YES & YES & YES & $1.43$ & $(2,3)$ & NO & 5424\\
$(105,32)$ & 11 & $(18,5)$ & 6 & 3 & YES & YES & YES & $1.57$ & $(2,3)$ & NO & 5425\\
$(105,41)$ & 10 & $(29,11)$ & 7 & 1 & YES & YES & YES & $1.57$ & $(2,3)$ & 4567 & 5426\\
$(105,44)$ & 10 & $(43,18)$ & 8 & 1 & YES & YES & NO(2) & $1.38$ & $(2,3)$ & 5702 & 5427\\
$(105,43)$ & 11 & $(46,19)$ & 8 & 1 & YES & YES & YES & $1.57$ & $(2,3)$ & 8313 & 5428\\
$(105,31)$ & 10 & $(55,16)$ & 9 & 5 & YES & YES & YES & $1.43$ & $(2,3)$ & NO & 5429\\
$(105,29)$ & 10 & $(76,21)$ & 9 & 1 & YES & YES & YES & $1.57$ & $(2,3)$ & NO & 5430\\
$(105,31)$ & 10 & $(95,28)$ & 11 & 5 & YES & YES & YES & $1.43$ & $(2,3)$ & 6807 & 5431\\
$(105,31)$ & 10 & $(98,29)$ & 10 & 7 & YES & YES & YES & $1.43$ & $(2,3)$ & NO & 5432\\
$(106,31)$ & 10 & $(2,1)$ & 1 & 2 & YES & YES & YES & $1.43$ & $(2,3)$ & -- & 5433\\
$(106,41)$ & 10 & $(3,1)$ & 2 & 1 & YES & YES & YES & $1.57$ & $(2,3)$ & -- & 5434\\
$(106,31)$ & 10 & $(4,1)$ & 3 & 2 & YES & YES & YES & $1.43$ & $(2,3)$ & -- & 5435\\
$(106,31)$ & 10 & $(4,1)$ & 3 & 2 & YES & YES & YES & $1.57$ & $(2,3)$ & NO & 5436\\
$(106,31)$ & 10 & $(5,2)$ & 3 & 1 & YES & YES & YES & $1.57$ & $(2,3)$ & NO & 5437\\
$(106,31)$ & 10 & $(5,2)$ & 3 & 1 & YES & YES & YES & $1.57$ & $(2,3)$ & -- & 5438\\
$(106,41)$ & 10 & $(5,2)$ & 3 & 1 & YES & YES & YES & $1.57$ & $(2,3)$ & -- & 5439\\
$(106,31)$ & 10 & $(7,3)$ & 4 & 1 & YES & YES & YES & $1.43$ & $(2,3)$ & -- & 5440\\
$(106,33)$ & 11 & $(7,3)$ & 4 & 1 & YES & YES & YES & $1.57$ & $(2,3)$ & -- & 5441\\
$(106,31)$ & 10 & $(9,4)$ & 5 & 1 & YES & YES & YES & $1.57$ & $(2,3)$ & -- & 5442\\
$(106,31)$ & 10 & $(10,3)$ & 5 & 2 & YES & YES & YES & $1.43$ & $(2,3)$ & -- & 5443\\
$(106,31)$ & 10 & $(11,3)$ & 5 & 1 & YES & YES & YES & $1.29$ & $(2,3)$ & -- & 5444\\
$(106,31)$ & 10 & $(11,3)$ & 5 & 1 & YES & YES & YES & $1.57$ & $(2,3)$ & NO & 5445\\
$(106,33)$ & 11 & $(11,3)$ & 5 & 1 & YES & YES & YES & $1.43$ & $(2,3)$ & -- & 5446\\
$(106,41)$ & 10 & $(11,4)$ & 5 & 1 & YES & YES & YES & $1.57$ & $(2,3)$ & NO & 5447\\
$(106,31)$ & 10 & $(13,5)$ & 5 & 1 & YES & YES & YES & $1.57$ & $(2,3)$ & -- & 5448\\
$(106,41)$ & 10 & $(21,8)$ & 6 & 1 & YES & YES & YES & $1.57$ & $(2,3)$ & NO & 5449\\
$(106,31)$ & 10 & $(31,9)$ & 8 & 1 & YES & YES & YES & $1.57$ & $(2,3)$ & NO & 5450\\
$(106,33)$ & 11 & $(55,17)$ & 10 & 1 & YES & YES & YES & $1.43$ & $(2,3)$ & 5259 & 5451\\
$(106,31)$ & 10 & $(65,19)$ & 9 & 1 & YES & YES & YES & $1.43$ & $(2,3)$ & NO & 5452\\
$(106,31)$ & 10 & $(106,31)$ & 10 & 106 & YES & YES & YES & $1.43$ & $(2,3)$ & NO & 5453\\
$(107,30)$ & 11 & $(5,1)$ & 4 & 1 & YES & YES & YES & $1.57$ & $(2,3)$ & NO & 5454\\
$(107,30)$ & 11 & $(5,1)$ & 4 & 1 & YES & YES & YES & $1.57$ & $(2,3)$ & -- & 5455\\
$(107,41)$ & 10 & $(5,2)$ & 3 & 1 & YES & YES & NO(2) & $1.50$ & $(2,3)$ & -- & 5456\\
$(107,48)$ & 11 & $(5,2)$ & 3 & 1 & YES & YES & YES & $1.57$ & $(2,3)$ & -- & 5457\\
$(107,30)$ & 11 & $(7,3)$ & 4 & 1 & YES & YES & YES & $1.57$ & $(2,3)$ & -- & 5458\\
$(107,31)$ & 11 & $(7,2)$ & 4 & 1 & YES & YES & YES & $1.43$ & $(2,3)$ & -- & 5459\\
$(107,38)$ & 11 & $(7,2)$ & 4 & 1 & YES & YES & YES & $1.57$ & $(2,3)$ & -- & 5460\\
$(107,41)$ & 10 & $(7,2)$ & 4 & 1 & YES & YES & YES & $1.29$ & $(2,3)$ & -- & 5461\\
$(107,47)$ & 10 & $(7,2)$ & 4 & 1 & YES & YES & YES & $1.29$ & $(2,3)$ & -- & 5462\\
$(107,47)$ & 10 & $(7,2)$ & 4 & 1 & YES & YES & YES & $1.57$ & $(2,3)$ & NO & 5463\\
$(107,47)$ & 10 & $(7,3)$ & 4 & 1 & YES & YES & YES & $1.57$ & $(2,3)$ & -- & 5464\\
$(107,48)$ & 11 & $(7,2)$ & 4 & 1 & YES & YES & YES & $1.71$ & $(2,3)$ & -- & 5465\\
$(107,25)$ & 11 & $(8,3)$ & 4 & 1 & YES & YES & YES & $1.29$ & $(2,3)$ & -- & 5466\\
$(107,38)$ & 11 & $(9,2)$ & 5 & 1 & YES & YES & YES & $1.43$ & $(2,3)$ & -- & 5467\\
$(107,41)$ & 10 & $(9,2)$ & 5 & 1 & YES & YES & YES & $1.43$ & $(2,3)$ & -- & 5468\\
$(107,41)$ & 10 & $(9,2)$ & 5 & 1 & YES & YES & YES & $1.43$ & $(2,3)$ & NO & 5469\\
$(107,30)$ & 11 & $(10,3)$ & 5 & 1 & YES & YES & YES & $1.57$ & $(2,3)$ & -- & 5470\\
$(107,41)$ & 10 & $(10,3)$ & 5 & 1 & YES & YES & YES & $1.57$ & $(2,3)$ & -- & 5471\\
$(107,41)$ & 10 & $(11,3)$ & 5 & 1 & YES & YES & YES & $1.43$ & $(2,3)$ & -- & 5472\\
$(107,41)$ & 10 & $(11,3)$ & 5 & 1 & YES & YES & YES & $1.57$ & $(2,3)$ & 8567 & 5473\\
$(107,41)$ & 10 & $(12,5)$ & 5 & 1 & YES & YES & NO(2) & $1.75$ & $(2,3)$ & NO & 5474\\
$(107,47)$ & 10 & $(12,5)$ & 5 & 1 & YES & YES & YES & $1.43$ & $(2,3)$ & NO & 5475\\
$(107,48)$ & 11 & $(12,5)$ & 5 & 1 & YES & YES & YES & $1.71$ & $(2,3)$ & NO & 5476\\
$(107,38)$ & 11 & $(13,5)$ & 5 & 1 & YES & YES & YES & $1.57$ & $(2,3)$ & NO & 5477\\
$(107,41)$ & 10 & $(13,3)$ & 6 & 1 & YES & YES & YES & $1.57$ & $(2,3)$ & NO & 5478\\
$(107,41)$ & 10 & $(14,5)$ & 6 & 1 & YES & YES & NO(2) & $1.62$ & $(2,3)$ & NO & 5479\\
$(107,31)$ & 11 & $(15,4)$ & 6 & 1 & YES & YES & YES & $1.57$ & $(2,3)$ & NO & 5480\\
$(107,31)$ & 11 & $(19,5)$ & 7 & 1 & YES & YES & YES & $1.57$ & $(2,3)$ & NO & 5481\\
$(107,38)$ & 11 & $(19,7)$ & 6 & 1 & YES & YES & YES & $1.57$ & $(2,3)$ & NO & 5482\\
$(107,47)$ & 10 & $(19,8)$ & 6 & 1 & YES & YES & YES & $1.43$ & $(2,3)$ & NO & 5483\\
$(107,41)$ & 10 & $(28,11)$ & 8 & 1 & YES & YES & YES & $1.71$ & $(2,3)$ & NO & 5484\\
$(107,25)$ & 11 & $(35,8)$ & 8 & 1 & YES & YES & NO(2) & $1.38$ & $(2,3)$ & NO & 5485\\
$(107,30)$ & 11 & $(40,11)$ & 8 & 1 & YES & YES & YES & $1.57$ & $(2,3)$ & NO & 5486\\
$(107,30)$ & 11 & $(47,13)$ & 8 & 1 & YES & YES & YES & $1.43$ & $(2,3)$ & NO & 5487\\
$(107,41)$ & 10 & $(50,19)$ & 8 & 1 & YES & YES & YES & $1.57$ & $(2,3)$ & NO & 5488\\
$(107,41)$ & 10 & $(55,21)$ & 8 & 1 & YES & YES & YES & $1.43$ & $(2,3)$ & NO & 5489\\
$(107,41)$ & 10 & $(76,29)$ & 9 & 1 & YES & YES & YES & $1.57$ & $(2,3)$ & 6017 & 5490\\
$(107,31)$ & 11 & $(83,24)$ & 11 & 1 & YES & YES & YES & $1.57$ & $(2,3)$ & NO & 5491\\
$(107,41)$ & 10 & $(86,33)$ & 11 & 1 & YES & YES & YES & $1.57$ & $(2,3)$ & 6541 & 5492\\
$(107,41)$ & 10 & $(89,34)$ & 9 & 1 & YES & YES & YES & $1.43$ & $(2,3)$ & NO & 5493\\
$(107,47)$ & 10 & $(91,40)$ & 10 & 1 & YES & YES & YES & $1.43$ & $(2,3)$ & NO & 5494\\
$(108,41)$ & 10 & $(5,1)$ & 4 & 1 & YES & YES & YES & $1.43$ & $(2,3)$ & NO & 5495\\
$(108,41)$ & 10 & $(5,1)$ & 4 & 1 & YES & YES & YES & $1.43$ & $(2,3)$ & -- & 5496\\
$(108,41)$ & 10 & $(7,3)$ & 4 & 1 & YES & YES & YES & $1.71$ & $(2,3)$ & -- & 5497\\
$(108,41)$ & 10 & $(8,3)$ & 4 & 4 & YES & YES & YES & $1.57$ & $(2,3)$ & -- & 5498\\
$(108,47)$ & 11 & $(10,3)$ & 5 & 2 & YES & YES & YES & $1.71$ & $(2,3)$ & -- & 5499\\
$(108,41)$ & 10 & $(11,3)$ & 5 & 1 & YES & YES & YES & $1.43$ & $(2,3)$ & -- & 5500\\
$(108,41)$ & 10 & $(11,3)$ & 5 & 1 & YES & YES & YES & $1.43$ & $(2,3)$ & NO & 5501\\
$(108,41)$ & 10 & $(13,3)$ & 6 & 1 & YES & YES & YES & $1.57$ & $(2,3)$ & -- & 5502\\
$(108,41)$ & 10 & $(13,3)$ & 6 & 1 & YES & YES & YES & $1.57$ & $(2,3)$ & NO & 5503\\
$(108,41)$ & 10 & $(14,3)$ & 6 & 2 & YES & YES & YES & $1.43$ & $(2,3)$ & NO & 5504\\
$(108,41)$ & 10 & $(14,3)$ & 6 & 2 & YES & YES & YES & $1.43$ & $(2,3)$ & -- & 5505\\
$(108,41)$ & 10 & $(19,7)$ & 6 & 1 & YES & YES & YES & $1.57$ & $(2,3)$ & NO & 5506\\
$(108,41)$ & 10 & $(47,18)$ & 8 & 1 & YES & YES & YES & $1.71$ & $(2,3)$ & NO & 5507\\
$(108,41)$ & 10 & $(108,41)$ & 10 & 108 & YES & YES & YES & $1.57$ & $(2,3)$ & NO & 5508\\
$(109,30)$ & 10 & $(2,1)$ & 1 & 1 & YES & YES & YES & $1.43$ & $(2,3)$ & -- & 5509\\
$(109,30)$ & 10 & $(3,1)$ & 2 & 1 & YES & YES & YES & $1.57$ & $(2,3)$ & -- & 5510\\
$(109,40)$ & 10 & $(3,1)$ & 2 & 1 & YES & YES & YES & $1.57$ & $(2,3)$ & -- & 5511\\
$(109,45)$ & 10 & $(3,1)$ & 2 & 1 & YES & YES & NO(2) & $1.62$ & $(2,3)$ & -- & 5512\\
$(109,30)$ & 10 & $(4,1)$ & 3 & 1 & YES & YES & YES & $1.43$ & $(2,3)$ & NO & 5513\\
$(109,30)$ & 10 & $(4,1)$ & 3 & 1 & YES & YES & YES & $1.43$ & $(2,3)$ & -- & 5514\\
$(109,33)$ & 11 & $(4,1)$ & 3 & 1 & YES & YES & NO(2) & $1.50$ & $(2,3)$ & -- & 5515\\
$(109,40)$ & 10 & $(5,2)$ & 3 & 1 & YES & YES & NO(2) & $1.62$ & $(2,3)$ & -- & 5516\\
$(109,33)$ & 11 & $(7,3)$ & 4 & 1 & YES & YES & NO(2) & $1.50$ & $(2,3)$ & -- & 5517\\
$(109,40)$ & 10 & $(7,2)$ & 4 & 1 & YES & YES & YES & $1.29$ & $(2,3)$ & -- & 5518\\
$(109,40)$ & 10 & $(7,2)$ & 4 & 1 & YES & YES & YES & $1.71$ & $(2,3)$ & NO & 5519\\
$(109,40)$ & 10 & $(7,3)$ & 4 & 1 & YES & YES & YES & $1.57$ & $(2,3)$ & -- & 5520\\
$(109,45)$ & 10 & $(7,2)$ & 4 & 1 & YES & YES & YES & $1.29$ & $(2,3)$ & -- & 5521\\
$(109,45)$ & 10 & $(7,3)$ & 4 & 1 & YES & YES & YES & $1.57$ & $(2,3)$ & -- & 5522\\
$(109,45)$ & 10 & $(7,3)$ & 4 & 1 & YES & YES & YES & $1.57$ & $(2,3)$ & NO & 5523\\
$(109,46)$ & 10 & $(7,2)$ & 4 & 1 & YES & YES & YES & $1.57$ & $(2,3)$ & NO & 5524\\
$(109,46)$ & 10 & $(7,2)$ & 4 & 1 & YES & YES & YES & $1.57$ & $(2,3)$ & -- & 5525\\
$(109,46)$ & 10 & $(7,3)$ & 4 & 1 & YES & YES & YES & $1.71$ & $(2,3)$ & -- & 5526\\
$(109,40)$ & 10 & $(8,3)$ & 4 & 1 & YES & YES & YES & $1.57$ & $(2,3)$ & -- & 5527\\
$(109,40)$ & 10 & $(9,4)$ & 5 & 1 & YES & YES & YES & $1.57$ & $(2,3)$ & NO & 5528\\
$(109,45)$ & 10 & $(9,2)$ & 5 & 1 & YES & YES & NO(2) & $1.29$ & $(4,2)$ & -- & 5529\\
$(109,45)$ & 10 & $(9,4)$ & 5 & 1 & YES & YES & YES & $1.71$ & $(2,3)$ & -- & 5530\\
$(109,46)$ & 10 & $(9,2)$ & 5 & 1 & YES & YES & NO(2) & $1.43$ & $(4,2)$ & NO & 5531\\
$(109,46)$ & 10 & $(9,2)$ & 5 & 1 & YES & YES & NO(2) & $1.43$ & $(4,2)$ & -- & 5532\\
$(109,40)$ & 10 & $(10,3)$ & 5 & 1 & YES & YES & YES & $1.57$ & $(2,3)$ & -- & 5533\\
$(109,45)$ & 10 & $(10,3)$ & 5 & 1 & YES & YES & YES & $1.57$ & $(2,3)$ & NO & 5534\\
$(109,46)$ & 10 & $(10,3)$ & 5 & 1 & YES & YES & NO(2) & $1.62$ & $(2,3)$ & NO & 5535\\
$(109,30)$ & 10 & $(11,4)$ & 5 & 1 & YES & YES & YES & $1.43$ & $(2,3)$ & -- & 5536\\
$(109,45)$ & 10 & $(11,3)$ & 5 & 1 & YES & YES & YES & $1.43$ & $(2,3)$ & -- & 5537\\
$(109,45)$ & 10 & $(11,5)$ & 6 & 1 & YES & YES & YES & $1.57$ & $(2,3)$ & NO & 5538\\
$(109,40)$ & 10 & $(12,5)$ & 5 & 1 & YES & YES & YES & $1.57$ & $(2,3)$ & NO & 5539\\
$(109,25)$ & 11 & $(13,4)$ & 6 & 1 & YES & YES & YES & $1.57$ & $(2,3)$ & -- & 5540\\
$(109,45)$ & 10 & $(13,3)$ & 6 & 1 & YES & YES & YES & $1.57$ & $(2,3)$ & -- & 5541\\
$(109,40)$ & 10 & $(14,5)$ & 6 & 1 & YES & YES & YES & $1.57$ & $(2,3)$ & NO & 5542\\
$(109,33)$ & 11 & $(15,4)$ & 6 & 1 & YES & YES & YES & $1.43$ & $(2,3)$ & NO & 5543\\
$(109,33)$ & 11 & $(17,4)$ & 7 & 1 & YES & YES & YES & $1.71$ & $(2,3)$ & NO & 5544\\
$(109,40)$ & 10 & $(18,7)$ & 6 & 1 & YES & YES & YES & $1.57$ & $(2,3)$ & NO & 5545\\
$(109,40)$ & 10 & $(21,8)$ & 6 & 1 & YES & YES & YES & $1.57$ & $(2,3)$ & NO & 5546\\
$(109,45)$ & 10 & $(22,9)$ & 7 & 1 & YES & YES & NO(2) & $1.62$ & $(2,3)$ & NO & 5547\\
$(109,46)$ & 10 & $(22,9)$ & 7 & 1 & YES & YES & YES & $1.43$ & $(2,3)$ & NO & 5548\\
$(109,30)$ & 10 & $(29,8)$ & 7 & 1 & YES & YES & YES & $1.43$ & $(2,3)$ & 5282 & 5549\\
$(109,45)$ & 10 & $(31,13)$ & 7 & 1 & YES & YES & YES & $1.71$ & $(2,3)$ & NO & 5550\\
$(109,40)$ & 10 & $(35,13)$ & 8 & 1 & YES & YES & YES & $1.57$ & $(2,3)$ & NO & 5551\\
$(109,45)$ & 10 & $(39,16)$ & 8 & 1 & YES & YES & NO(2) & $1.50$ & $(2,3)$ & NO & 5552\\
$(109,40)$ & 10 & $(52,19)$ & 9 & 1 & YES & YES & NO(2) & $1.75$ & $(2,3)$ & NO & 5553\\
$(109,45)$ & 10 & $(53,22)$ & 9 & 1 & YES & YES & YES & $1.57$ & $(2,3)$ & 5200 & 5554\\
$(109,46)$ & 10 & $(59,25)$ & 9 & 1 & YES & YES & NO(2) & $1.75$ & $(2,3)$ & NO & 5555\\
$(109,30)$ & 10 & $(61,17)$ & 9 & 1 & YES & YES & YES & $1.43$ & $(2,3)$ & NO & 5556\\
$(109,40)$ & 10 & $(68,25)$ & 9 & 1 & YES & YES & NO(2) & $1.62$ & $(2,3)$ & 7109 & 5557\\
$(109,30)$ & 10 & $(69,19)$ & 9 & 1 & YES & YES & YES & $1.43$ & $(2,3)$ & NO & 5558\\
$(109,45)$ & 10 & $(80,33)$ & 10 & 1 & YES & YES & NO(2) & $1.57$ & $(4,2)$ & NO & 5559\\
$(109,46)$ & 10 & $(83,35)$ & 10 & 1 & YES & YES & NO(2) & $1.57$ & $(4,2)$ & NO & 5560\\
$(109,40)$ & 10 & $(87,32)$ & 10 & 1 & YES & YES & YES & $1.71$ & $(2,3)$ & NO & 5561\\
$(109,45)$ & 10 & $(104,43)$ & 10 & 1 & YES & YES & YES & $1.57$ & $(2,3)$ & NO & 5562\\
$(109,30)$ & 10 & $(109,30)$ & 10 & 109 & YES & YES & YES & $1.57$ & $(2,3)$ & NO & 5563\\
$(109,40)$ & 10 & $(109,40)$ & 10 & 109 & YES & YES & YES & $1.57$ & $(2,3)$ & NO & 5564\\
$(110,39)$ & 11 & $(25,9)$ & 7 & 5 & YES & YES & YES & $1.57$ & $(2,3)$ & NO & 5565\\
$(110,31)$ & 11 & $(26,7)$ & 7 & 2 & YES & YES & YES & $1.57$ & $(2,3)$ & NO & 5566\\
$(110,43)$ & 11 & $(49,19)$ & 8 & 1 & YES & YES & YES & $1.57$ & $(2,3)$ & 8480 & 5567\\
$(110,43)$ & 11 & $(105,41)$ & 10 & 5 & YES & YES & YES & $1.57$ & $(2,3)$ & 7299 & 5568\\
$(111,31)$ & 10 & $(3,1)$ & 2 & 3 & YES & YES & YES & $1.43$ & $(2,3)$ & -- & 5569\\
$(111,43)$ & 10 & $(3,1)$ & 2 & 3 & YES & YES & YES & $1.43$ & $(2,3)$ & NO & 5570\\
$(111,31)$ & 10 & $(4,1)$ & 3 & 1 & YES & YES & YES & $1.43$ & $(2,3)$ & -- & 5571\\
$(111,46)$ & 10 & $(4,1)$ & 3 & 1 & YES & YES & YES & $1.57$ & $(2,3)$ & -- & 5572\\
$(111,41)$ & 10 & $(5,2)$ & 3 & 1 & YES & YES & NO(2) & $1.43$ & $(4,2)$ & -- & 5573\\
$(111,46)$ & 10 & $(5,2)$ & 3 & 1 & YES & YES & NO(2) & $1.62$ & $(2,3)$ & -- & 5574\\
$(111,31)$ & 10 & $(7,3)$ & 4 & 1 & YES & YES & YES & $1.43$ & $(2,3)$ & -- & 5575\\
$(111,41)$ & 10 & $(7,2)$ & 4 & 1 & YES & YES & YES & $1.57$ & $(2,3)$ & -- & 5576\\
$(111,41)$ & 10 & $(7,3)$ & 4 & 1 & YES & YES & YES & $1.57$ & $(2,3)$ & -- & 5577\\
$(111,46)$ & 10 & $(7,2)$ & 4 & 1 & YES & YES & YES & $1.71$ & $(2,3)$ & NO & 5578\\
$(111,46)$ & 10 & $(7,2)$ & 4 & 1 & YES & YES & YES & $1.71$ & $(2,3)$ & -- & 5579\\
$(111,46)$ & 10 & $(7,2)$ & 4 & 1 & YES & YES & YES & $1.43$ & $(2,3)$ & NO & 5580\\
$(111,31)$ & 10 & $(8,3)$ & 4 & 1 & YES & YES & YES & $1.43$ & $(2,3)$ & NO & 5581\\
$(111,31)$ & 10 & $(8,3)$ & 4 & 1 & YES & YES & YES & $1.43$ & $(2,3)$ & -- & 5582\\
$(111,41)$ & 10 & $(8,3)$ & 4 & 1 & YES & YES & YES & $1.57$ & $(2,3)$ & -- & 5583\\
$(111,46)$ & 10 & $(8,3)$ & 4 & 1 & YES & YES & YES & $1.57$ & $(2,3)$ & -- & 5584\\
$(111,41)$ & 10 & $(9,4)$ & 5 & 3 & YES & YES & YES & $1.57$ & $(2,3)$ & -- & 5585\\
$(111,43)$ & 10 & $(9,2)$ & 5 & 3 & YES & YES & YES & $1.43$ & $(2,3)$ & -- & 5586\\
$(111,43)$ & 10 & $(9,2)$ & 5 & 3 & YES & YES & YES & $1.43$ & $(2,3)$ & NO & 5587\\
$(111,31)$ & 10 & $(10,3)$ & 5 & 1 & YES & YES & YES & $1.43$ & $(2,3)$ & -- & 5588\\
$(111,46)$ & 10 & $(10,3)$ & 5 & 1 & YES & YES & YES & $1.43$ & $(2,3)$ & NO & 5589\\
$(111,46)$ & 10 & $(10,3)$ & 5 & 1 & YES & YES & YES & $1.43$ & $(2,3)$ & -- & 5590\\
$(111,31)$ & 10 & $(11,3)$ & 5 & 1 & YES & YES & YES & $1.43$ & $(2,3)$ & -- & 5591\\
$(111,31)$ & 10 & $(12,5)$ & 5 & 3 & YES & YES & YES & $1.43$ & $(2,3)$ & NO & 5592\\
$(111,46)$ & 10 & $(13,4)$ & 6 & 1 & YES & YES & YES & $1.71$ & $(2,3)$ & -- & 5593\\
$(111,46)$ & 10 & $(13,5)$ & 5 & 1 & YES & YES & YES & $1.57$ & $(2,3)$ & NO & 5594\\
$(111,41)$ & 10 & $(21,8)$ & 6 & 3 & YES & YES & NO(2) & $1.50$ & $(2,3)$ & NO & 5595\\
$(111,46)$ & 10 & $(22,9)$ & 7 & 1 & YES & YES & YES & $1.57$ & $(2,3)$ & 6955 & 5596\\
$(111,41)$ & 10 & $(23,9)$ & 7 & 1 & YES & YES & YES & $1.57$ & $(2,3)$ & NO & 5597\\
$(111,46)$ & 10 & $(26,11)$ & 7 & 1 & YES & YES & YES & $1.57$ & $(2,3)$ & NO & 5598\\
$(111,41)$ & 10 & $(35,13)$ & 8 & 1 & YES & YES & NO(2) & $1.43$ & $(4,2)$ & NO & 5599\\
$(111,31)$ & 10 & $(40,11)$ & 8 & 1 & YES & YES & YES & $1.43$ & $(2,3)$ & NO & 5600\\
$(111,46)$ & 10 & $(46,19)$ & 8 & 1 & YES & YES & YES & $1.57$ & $(2,3)$ & NO & 5601\\
$(111,46)$ & 10 & $(56,23)$ & 9 & 1 & YES & YES & YES & $1.57$ & $(2,3)$ & NO & 5602\\
$(111,43)$ & 10 & $(57,22)$ & 9 & 3 & YES & YES & YES & $1.57$ & $(2,3)$ & NO & 5603\\
$(111,31)$ & 10 & $(68,19)$ & 9 & 1 & YES & YES & YES & $1.43$ & $(2,3)$ & NO & 5604\\
$(111,31)$ & 10 & $(111,31)$ & 10 & 111 & YES & YES & YES & $1.43$ & $(2,3)$ & NO & 5605\\
$(112,31)$ & 10 & $(2,1)$ & 1 & 2 & YES & YES & YES & $1.57$ & $(2,3)$ & NO & 5606\\
$(112,31)$ & 10 & $(3,1)$ & 2 & 1 & YES & YES & YES & $1.43$ & $(2,3)$ & -- & 5607\\
$(112,31)$ & 10 & $(3,1)$ & 2 & 1 & YES & YES & YES & $1.57$ & $(2,3)$ & NO & 5608\\
$(112,31)$ & 10 & $(4,1)$ & 3 & 4 & YES & YES & YES & $1.43$ & $(2,3)$ & NO & 5609\\
$(112,31)$ & 10 & $(4,1)$ & 3 & 4 & YES & YES & YES & $1.43$ & $(2,3)$ & -- & 5610\\
$(112,31)$ & 10 & $(5,2)$ & 3 & 1 & YES & YES & YES & $1.57$ & $(2,3)$ & -- & 5611\\
$(112,41)$ & 10 & $(5,2)$ & 3 & 1 & YES & YES & NO(2) & $1.50$ & $(2,3)$ & -- & 5612\\
$(112,31)$ & 10 & $(7,2)$ & 4 & 7 & YES & YES & YES & $1.43$ & $(2,3)$ & -- & 5613\\
$(112,31)$ & 10 & $(7,2)$ & 4 & 7 & YES & YES & YES & $1.43$ & $(2,3)$ & NO & 5614\\
$(112,41)$ & 10 & $(7,2)$ & 4 & 7 & YES & YES & YES & $1.43$ & $(2,3)$ & -- & 5615\\
$(112,41)$ & 10 & $(7,3)$ & 4 & 7 & YES & YES & YES & $1.57$ & $(2,3)$ & -- & 5616\\
$(112,41)$ & 10 & $(7,3)$ & 4 & 7 & YES & YES & NO(2) & $1.50$ & $(2,3)$ & NO & 5617\\
$(112,47)$ & 10 & $(7,3)$ & 4 & 7 & YES & YES & YES & $1.43$ & $(2,3)$ & -- & 5618\\
$(112,41)$ & 10 & $(9,2)$ & 5 & 1 & YES & YES & YES & $1.43$ & $(2,3)$ & NO & 5619\\
$(112,41)$ & 10 & $(9,2)$ & 5 & 1 & YES & YES & YES & $1.43$ & $(2,3)$ & -- & 5620\\
$(112,41)$ & 10 & $(10,3)$ & 5 & 2 & YES & YES & YES & $1.57$ & $(2,3)$ & -- & 5621\\
$(112,31)$ & 10 & $(11,4)$ & 5 & 1 & YES & YES & YES & $1.57$ & $(2,3)$ & -- & 5622\\
$(112,47)$ & 10 & $(11,5)$ & 6 & 1 & YES & YES & YES & $1.57$ & $(2,3)$ & NO & 5623\\
$(112,31)$ & 10 & $(13,5)$ & 5 & 1 & YES & YES & YES & $1.43$ & $(2,3)$ & NO & 5624\\
$(112,47)$ & 10 & $(13,5)$ & 5 & 1 & YES & YES & YES & $1.57$ & $(2,3)$ & NO & 5625\\
$(112,41)$ & 10 & $(21,8)$ & 6 & 7 & YES & YES & YES & $1.71$ & $(2,3)$ & NO & 5626\\
$(112,47)$ & 10 & $(22,9)$ & 7 & 2 & YES & YES & YES & $1.57$ & $(2,3)$ & NO & 5627\\
$(112,31)$ & 10 & $(23,7)$ & 7 & 1 & YES & YES & YES & $1.43$ & $(2,3)$ & NO & 5628\\
$(112,41)$ & 10 & $(35,13)$ & 8 & 7 & YES & YES & YES & $1.57$ & $(2,3)$ & NO & 5629\\
$(112,31)$ & 10 & $(65,18)$ & 9 & 1 & YES & YES & YES & $1.43$ & $(2,3)$ & NO & 5630\\
$(112,41)$ & 10 & $(93,34)$ & 10 & 1 & YES & YES & NO(2) & $1.75$ & $(2,3)$ & NO & 5631\\
$(112,31)$ & 10 & $(112,31)$ & 10 & 112 & YES & YES & YES & $1.43$ & $(2,3)$ & NO & 5632\\
$(113,33)$ & 11 & $(4,1)$ & 3 & 1 & YES & YES & YES & $1.57$ & $(2,3)$ & NO & 5633\\
$(113,33)$ & 11 & $(4,1)$ & 3 & 1 & YES & YES & YES & $1.57$ & $(2,3)$ & -- & 5634\\
$(113,49)$ & 11 & $(5,1)$ & 4 & 1 & YES & YES & YES & $1.86$ & $(2,3)$ & -- & 5635\\
$(113,51)$ & 11 & $(5,2)$ & 3 & 1 & YES & YES & YES & $1.71$ & $(2,3)$ & -- & 5636\\
$(113,33)$ & 11 & $(7,2)$ & 4 & 1 & YES & YES & NO(2) & $1.62$ & $(2,3)$ & -- & 5637\\
$(113,33)$ & 11 & $(7,3)$ & 4 & 1 & YES & YES & NO(2) & $1.75$ & $(2,3)$ & -- & 5638\\
$(113,51)$ & 11 & $(7,2)$ & 4 & 1 & YES & YES & YES & $1.57$ & $(2,3)$ & NO & 5639\\
$(113,51)$ & 11 & $(8,3)$ & 4 & 1 & YES & YES & YES & $1.57$ & $(2,3)$ & NO & 5640\\
$(113,42)$ & 11 & $(9,2)$ & 5 & 1 & YES & YES & YES & $1.57$ & $(2,3)$ & -- & 5641\\
$(113,42)$ & 11 & $(9,4)$ & 5 & 1 & YES & YES & YES & $1.71$ & $(2,3)$ & -- & 5642\\
$(113,30)$ & 11 & $(10,3)$ & 5 & 1 & YES & YES & YES & $1.43$ & $(2,3)$ & -- & 5643\\
$(113,33)$ & 11 & $(13,4)$ & 6 & 1 & YES & YES & NO(2) & $1.50$ & $(2,3)$ & NO & 5644\\
$(113,51)$ & 11 & $(16,7)$ & 6 & 1 & YES & YES & YES & $1.57$ & $(2,3)$ & NO & 5645\\
$(113,24)$ & 11 & $(19,4)$ & 7 & 1 & YES & YES & YES & $1.43$ & $(2,3)$ & -- & 5646\\
$(113,33)$ & 11 & $(23,7)$ & 7 & 1 & YES & YES & YES & $1.43$ & $(2,3)$ & NO & 5647\\
$(113,30)$ & 11 & $(25,7)$ & 7 & 1 & YES & YES & YES & $1.71$ & $(2,3)$ & NO & 5648\\
$(113,35)$ & 11 & $(33,10)$ & 8 & 1 & YES & YES & YES & $1.71$ & $(2,3)$ & NO & 5649\\
$(113,33)$ & 11 & $(45,13)$ & 10 & 1 & YES & YES & YES & $1.57$ & $(2,3)$ & NO & 5650\\
$(113,49)$ & 11 & $(62,27)$ & 9 & 1 & YES & YES & YES & $1.71$ & $(2,3)$ & NO & 5651\\
$(113,33)$ & 11 & $(65,19)$ & 9 & 1 & YES & YES & YES & $1.57$ & $(2,3)$ & 6426 & 5652\\
$(113,51)$ & 11 & $(71,32)$ & 10 & 1 & YES & YES & YES & $1.57$ & $(2,3)$ & 7240 & 5653\\
$(113,51)$ & 11 & $(73,33)$ & 10 & 1 & YES & YES & YES & $1.71$ & $(2,3)$ & NO & 5654\\
$(113,35)$ & 11 & $(74,23)$ & 10 & 1 & YES & YES & YES & $1.43$ & $(2,3)$ & NO & 5655\\
$(114,41)$ & 11 & $(5,2)$ & 3 & 1 & YES & YES & NO(2) & $1.50$ & $(2,3)$ & -- & 5656\\
$(114,41)$ & 11 & $(7,3)$ & 4 & 1 & YES & YES & NO(2) & $1.50$ & $(2,3)$ & NO & 5657\\
$(115,44)$ & 10 & $(3,1)$ & 2 & 1 & YES & YES & YES & $1.57$ & $(2,3)$ & -- & 5658\\
$(115,31)$ & 11 & $(5,1)$ & 4 & 5 & YES & YES & YES & $1.57$ & $(2,3)$ & NO & 5659\\
$(115,31)$ & 11 & $(5,1)$ & 4 & 5 & YES & YES & YES & $1.57$ & $(2,3)$ & -- & 5660\\
$(115,44)$ & 10 & $(5,1)$ & 4 & 5 & YES & YES & YES & $1.71$ & $(2,3)$ & NO & 5661\\
$(115,44)$ & 10 & $(5,1)$ & 4 & 5 & YES & YES & YES & $1.71$ & $(2,3)$ & -- & 5662\\
$(115,31)$ & 11 & $(7,3)$ & 4 & 1 & YES & YES & NO(2) & $1.75$ & $(2,3)$ & -- & 5663\\
$(115,44)$ & 10 & $(7,3)$ & 4 & 1 & YES & YES & YES & $1.57$ & $(2,3)$ & NO & 5664\\
$(115,44)$ & 10 & $(8,3)$ & 4 & 1 & YES & YES & YES & $1.71$ & $(2,3)$ & -- & 5665\\
$(115,44)$ & 10 & $(8,3)$ & 4 & 1 & YES & YES & YES & $1.71$ & $(2,3)$ & NO & 5666\\
$(115,26)$ & 11 & $(15,4)$ & 6 & 5 & YES & YES & YES & $1.43$ & $(2,3)$ & -- & 5667\\
$(115,26)$ & 11 & $(17,4)$ & 7 & 1 & YES & YES & YES & $1.43$ & $(2,3)$ & -- & 5668\\
$(115,26)$ & 11 & $(19,4)$ & 7 & 1 & YES & YES & YES & $1.43$ & $(2,3)$ & -- & 5669\\
$(115,44)$ & 10 & $(23,9)$ & 7 & 23 & YES & YES & YES & $1.71$ & $(2,3)$ & NO & 5670\\
$(115,31)$ & 11 & $(89,24)$ & 10 & 1 & YES & YES & YES & $1.57$ & $(2,3)$ & NO & 5671\\
$(115,31)$ & 11 & $(115,31)$ & 11 & 115 & YES & YES & YES & $1.57$ & $(2,3)$ & NO & 5672\\
$(116,43)$ & 11 & $(5,2)$ & 3 & 1 & YES & YES & YES & $1.71$ & $(2,3)$ & -- & 5673\\
$(116,49)$ & 10 & $(5,2)$ & 3 & 1 & YES & YES & YES & $1.29$ & $(2,3)$ & -- & 5674\\
$(116,45)$ & 10 & $(7,3)$ & 4 & 1 & YES & YES & YES & $1.57$ & $(2,3)$ & -- & 5675\\
$(116,45)$ & 10 & $(7,3)$ & 4 & 1 & YES & YES & YES & $1.43$ & $(2,3)$ & NO & 5676\\
$(116,45)$ & 10 & $(9,2)$ & 5 & 1 & YES & YES & YES & $1.43$ & $(2,3)$ & NO & 5677\\
$(116,45)$ & 10 & $(9,2)$ & 5 & 1 & YES & YES & YES & $1.43$ & $(2,3)$ & -- & 5678\\
$(116,43)$ & 11 & $(10,3)$ & 5 & 2 & YES & YES & YES & $1.71$ & $(2,3)$ & -- & 5679\\
$(116,27)$ & 11 & $(11,4)$ & 5 & 1 & YES & YES & YES & $1.57$ & $(2,3)$ & -- & 5680\\
$(116,43)$ & 11 & $(16,3)$ & 7 & 4 & YES & YES & YES & $1.71$ & $(2,3)$ & NO & 5681\\
$(116,45)$ & 10 & $(19,7)$ & 6 & 1 & YES & YES & YES & $1.71$ & $(2,3)$ & NO & 5682\\
$(116,45)$ & 10 & $(44,17)$ & 8 & 4 & YES & YES & YES & $1.57$ & $(2,3)$ & NO & 5683\\
$(116,49)$ & 10 & $(83,35)$ & 10 & 1 & YES & YES & YES & $1.71$ & $(2,3)$ & NO & 5684\\
$(117,49)$ & 10 & $(2,1)$ & 1 & 1 & YES & YES & NO(2) & $1.50$ & $(2,3)$ & -- & 5685\\
$(117,49)$ & 10 & $(3,1)$ & 2 & 3 & YES & YES & YES & $1.43$ & $(2,3)$ & -- & 5686\\
$(117,34)$ & 11 & $(5,2)$ & 3 & 1 & YES & YES & YES & $1.43$ & $(2,3)$ & -- & 5687\\
$(117,43)$ & 10 & $(5,2)$ & 3 & 1 & YES & YES & NO(2) & $1.50$ & $(2,3)$ & NO & 5688\\
$(117,43)$ & 10 & $(5,2)$ & 3 & 1 & YES & YES & NO(2) & $1.50$ & $(2,3)$ & -- & 5689\\
$(117,34)$ & 11 & $(7,3)$ & 4 & 1 & YES & YES & YES & $1.57$ & $(2,3)$ & -- & 5690\\
$(117,43)$ & 10 & $(7,2)$ & 4 & 1 & YES & YES & YES & $1.29$ & $(2,3)$ & -- & 5691\\
$(117,49)$ & 10 & $(7,2)$ & 4 & 1 & YES & YES & YES & $1.43$ & $(2,3)$ & NO & 5692\\
$(117,49)$ & 10 & $(7,2)$ & 4 & 1 & YES & YES & YES & $1.29$ & $(2,3)$ & -- & 5693\\
$(117,49)$ & 10 & $(7,3)$ & 4 & 1 & YES & YES & YES & $1.57$ & $(2,3)$ & -- & 5694\\
$(117,49)$ & 10 & $(11,4)$ & 5 & 1 & YES & YES & YES & $1.43$ & $(2,3)$ & NO & 5695\\
$(117,49)$ & 10 & $(11,5)$ & 6 & 1 & YES & YES & YES & $1.57$ & $(2,3)$ & NO & 5696\\
$(117,49)$ & 10 & $(13,5)$ & 5 & 13 & YES & YES & YES & $1.57$ & $(2,3)$ & NO & 5697\\
$(117,43)$ & 10 & $(14,5)$ & 6 & 1 & YES & YES & YES & $1.43$ & $(2,3)$ & NO & 5698\\
$(117,49)$ & 10 & $(17,7)$ & 6 & 1 & YES & YES & YES & $1.43$ & $(2,3)$ & NO & 5699\\
$(117,49)$ & 10 & $(26,11)$ & 7 & 13 & YES & YES & YES & $1.43$ & $(2,3)$ & 7163 & 5700\\
$(117,43)$ & 10 & $(30,11)$ & 7 & 3 & YES & YES & YES & $1.43$ & $(2,3)$ & NO & 5701\\
$(117,49)$ & 10 & $(31,13)$ & 7 & 1 & YES & YES & NO(2) & $1.38$ & $(2,3)$ & 5427 & 5702\\
$(117,34)$ & 11 & $(37,11)$ & 8 & 1 & YES & YES & YES & $1.57$ & $(2,3)$ & NO & 5703\\
$(117,34)$ & 11 & $(44,13)$ & 8 & 1 & YES & YES & YES & $1.43$ & $(2,3)$ & NO & 5704\\
$(117,49)$ & 10 & $(45,19)$ & 8 & 9 & YES & YES & YES & $1.57$ & $(2,3)$ & NO & 5705\\
$(117,49)$ & 10 & $(67,28)$ & 10 & 1 & YES & YES & YES & $1.57$ & $(2,3)$ & NO & 5706\\
$(117,49)$ & 10 & $(74,31)$ & 9 & 1 & YES & YES & YES & $1.29$ & $(2,3)$ & NO & 5707\\
$(117,34)$ & 11 & $(79,23)$ & 10 & 1 & YES & YES & YES & $1.43$ & $(2,3)$ & 7393 & 5708\\
$(117,43)$ & 10 & $(109,40)$ & 10 & 1 & YES & YES & YES & $1.71$ & $(2,3)$ & NO & 5709\\
$(118,27)$ & 11 & $(3,1)$ & 2 & 1 & YES & YES & YES & $1.43$ & $(2,3)$ & -- & 5710\\
$(118,27)$ & 11 & $(4,1)$ & 3 & 2 & YES & YES & YES & $1.43$ & $(2,3)$ & -- & 5711\\
$(118,49)$ & 11 & $(5,1)$ & 4 & 1 & YES & YES & YES & $1.71$ & $(2,3)$ & -- & 5712\\
$(118,27)$ & 11 & $(7,3)$ & 4 & 1 & YES & YES & YES & $1.57$ & $(2,3)$ & NO & 5713\\
$(118,27)$ & 11 & $(7,3)$ & 4 & 1 & YES & YES & YES & $1.57$ & $(2,3)$ & -- & 5714\\
$(118,33)$ & 11 & $(7,2)$ & 4 & 1 & YES & YES & NO(2) & $1.62$ & $(2,3)$ & -- & 5715\\
$(118,33)$ & 11 & $(7,3)$ & 4 & 1 & YES & YES & YES & $1.57$ & $(2,3)$ & -- & 5716\\
$(118,35)$ & 11 & $(7,3)$ & 4 & 1 & YES & YES & YES & $1.43$ & $(2,3)$ & -- & 5717\\
$(118,49)$ & 11 & $(11,2)$ & 6 & 1 & YES & YES & NO(2) & $1.62$ & $(2,3)$ & NO & 5718\\
$(118,33)$ & 11 & $(13,4)$ & 6 & 1 & YES & YES & NO(2) & $1.50$ & $(2,3)$ & NO & 5719\\
$(118,27)$ & 11 & $(15,4)$ & 6 & 1 & YES & YES & YES & $1.57$ & $(2,3)$ & -- & 5720\\
$(118,33)$ & 11 & $(16,3)$ & 7 & 2 & YES & YES & YES & $1.57$ & $(2,3)$ & -- & 5721\\
$(118,27)$ & 11 & $(17,4)$ & 7 & 1 & YES & YES & YES & $1.57$ & $(2,3)$ & -- & 5722\\
$(118,49)$ & 11 & $(22,9)$ & 7 & 2 & YES & YES & NO(2) & $1.62$ & $(2,3)$ & NO & 5723\\
$(118,35)$ & 11 & $(31,9)$ & 8 & 1 & YES & YES & YES & $1.57$ & $(2,3)$ & NO & 5724\\
$(118,49)$ & 11 & $(39,16)$ & 8 & 1 & YES & YES & YES & $1.57$ & $(2,3)$ & NO & 5725\\
$(118,27)$ & 11 & $(48,11)$ & 9 & 2 & YES & YES & YES & $1.43$ & $(2,3)$ & 6013 & 5726\\
$(118,27)$ & 11 & $(71,16)$ & 10 & 1 & YES & YES & YES & $1.71$ & $(2,3)$ & NO & 5727\\
$(118,27)$ & 11 & $(83,19)$ & 10 & 1 & YES & YES & YES & $1.43$ & $(2,3)$ & NO & 5728\\
$(118,35)$ & 11 & $(84,25)$ & 10 & 2 & YES & YES & YES & $1.71$ & $(2,3)$ & 8598 & 5729\\
$(118,33)$ & 11 & $(111,31)$ & 10 & 1 & YES & YES & NO(2) & $1.50$ & $(2,3)$ & 7554 & 5730\\
$(119,44)$ & 10 & $(3,1)$ & 2 & 1 & YES & YES & YES & $1.43$ & $(2,3)$ & -- & 5731\\
$(119,45)$ & 11 & $(3,1)$ & 2 & 1 & YES & YES & YES & $1.57$ & $(2,3)$ & NO & 5732\\
$(119,45)$ & 11 & $(3,1)$ & 2 & 1 & YES & YES & YES & $1.57$ & $(2,3)$ & -- & 5733\\
$(119,44)$ & 10 & $(4,1)$ & 3 & 1 & YES & YES & YES & $1.43$ & $(2,3)$ & NO & 5734\\
$(119,44)$ & 10 & $(4,1)$ & 3 & 1 & YES & YES & YES & $1.43$ & $(2,3)$ & -- & 5735\\
$(119,44)$ & 10 & $(4,1)$ & 3 & 1 & YES & YES & YES & $1.57$ & $(2,3)$ & NO & 5736\\
$(119,45)$ & 11 & $(4,1)$ & 3 & 1 & YES & YES & YES & $1.57$ & $(2,3)$ & NO & 5737\\
$(119,45)$ & 11 & $(4,1)$ & 3 & 1 & YES & YES & YES & $1.57$ & $(2,3)$ & -- & 5738\\
$(119,50)$ & 10 & $(4,1)$ & 3 & 1 & YES & YES & YES & $1.57$ & $(2,3)$ & -- & 5739\\
$(119,43)$ & 11 & $(5,2)$ & 3 & 1 & YES & YES & NO(2) & $1.75$ & $(2,3)$ & NO & 5740\\
$(119,43)$ & 11 & $(5,2)$ & 3 & 1 & YES & YES & NO(2) & $1.75$ & $(2,3)$ & -- & 5741\\
$(119,44)$ & 10 & $(5,2)$ & 3 & 1 & YES & YES & YES & $1.43$ & $(2,3)$ & NO & 5742\\
$(119,44)$ & 10 & $(5,2)$ & 3 & 1 & YES & YES & YES & $1.43$ & $(2,3)$ & -- & 5743\\
$(119,45)$ & 11 & $(5,2)$ & 3 & 1 & YES & YES & YES & $1.57$ & $(2,3)$ & -- & 5744\\
$(119,50)$ & 10 & $(5,2)$ & 3 & 1 & YES & YES & YES & $1.57$ & $(2,3)$ & -- & 5745\\
$(119,32)$ & 11 & $(7,3)$ & 4 & 7 & YES & YES & NO(2) & $1.62$ & $(2,3)$ & -- & 5746\\
$(119,32)$ & 11 & $(7,3)$ & 4 & 7 & YES & YES & NO(2) & $1.50$ & $(2,3)$ & NO & 5747\\
$(119,44)$ & 10 & $(7,3)$ & 4 & 7 & YES & YES & YES & $1.43$ & $(2,3)$ & -- & 5748\\
$(119,45)$ & 11 & $(7,2)$ & 4 & 7 & YES & YES & YES & $1.57$ & $(2,3)$ & -- & 5749\\
$(119,46)$ & 10 & $(7,2)$ & 4 & 7 & YES & YES & YES & $1.57$ & $(2,3)$ & NO & 5750\\
$(119,46)$ & 10 & $(7,2)$ & 4 & 7 & YES & YES & YES & $1.57$ & $(2,3)$ & -- & 5751\\
$(119,36)$ & 11 & $(8,3)$ & 4 & 1 & YES & YES & YES & $1.71$ & $(2,3)$ & -- & 5752\\
$(119,43)$ & 11 & $(9,4)$ & 5 & 1 & YES & YES & YES & $1.71$ & $(2,3)$ & -- & 5753\\
$(119,43)$ & 11 & $(10,3)$ & 5 & 1 & YES & YES & YES & $1.71$ & $(2,3)$ & -- & 5754\\
$(119,50)$ & 10 & $(10,3)$ & 5 & 1 & YES & YES & YES & $1.57$ & $(2,3)$ & NO & 5755\\
$(119,26)$ & 11 & $(11,4)$ & 5 & 1 & YES & YES & YES & $1.71$ & $(2,3)$ & NO & 5756\\
$(119,50)$ & 10 & $(11,4)$ & 5 & 1 & YES & YES & YES & $1.57$ & $(2,3)$ & NO & 5757\\
$(119,50)$ & 10 & $(11,5)$ & 6 & 1 & YES & YES & YES & $1.57$ & $(2,3)$ & NO & 5758\\
$(119,44)$ & 10 & $(12,5)$ & 5 & 1 & YES & YES & YES & $1.43$ & $(2,3)$ & NO & 5759\\
$(119,26)$ & 11 & $(13,4)$ & 6 & 1 & YES & YES & YES & $1.71$ & $(2,3)$ & NO & 5760\\
$(119,43)$ & 11 & $(13,5)$ & 5 & 1 & YES & YES & NO(2) & $1.62$ & $(2,3)$ & NO & 5761\\
$(119,44)$ & 10 & $(13,5)$ & 5 & 1 & YES & YES & YES & $1.29$ & $(2,3)$ & NO & 5762\\
$(119,50)$ & 10 & $(13,5)$ & 5 & 1 & YES & YES & YES & $1.57$ & $(2,3)$ & NO & 5763\\
$(119,32)$ & 11 & $(17,5)$ & 6 & 17 & YES & YES & YES & $1.71$ & $(2,3)$ & NO & 5764\\
$(119,43)$ & 11 & $(17,7)$ & 6 & 17 & YES & YES & YES & $1.86$ & $(2,3)$ & NO & 5765\\
$(119,43)$ & 11 & $(18,7)$ & 6 & 1 & YES & YES & YES & $1.71$ & $(2,3)$ & NO & 5766\\
$(119,44)$ & 10 & $(18,7)$ & 6 & 1 & YES & YES & YES & $1.57$ & $(2,3)$ & NO & 5767\\
$(119,45)$ & 11 & $(21,8)$ & 6 & 7 & YES & YES & YES & $1.57$ & $(2,3)$ & NO & 5768\\
$(119,50)$ & 10 & $(22,9)$ & 7 & 1 & YES & YES & YES & $1.57$ & $(2,3)$ & NO & 5769\\
$(119,46)$ & 10 & $(23,9)$ & 7 & 1 & YES & YES & YES & $1.43$ & $(2,3)$ & 7224 & 5770\\
$(119,36)$ & 11 & $(27,8)$ & 7 & 1 & YES & YES & YES & $1.43$ & $(2,3)$ & NO & 5771\\
$(119,36)$ & 11 & $(29,9)$ & 8 & 1 & YES & YES & YES & $1.71$ & $(2,3)$ & NO & 5772\\
$(119,50)$ & 10 & $(29,12)$ & 7 & 1 & YES & YES & YES & $1.57$ & $(2,3)$ & NO & 5773\\
$(119,45)$ & 11 & $(34,13)$ & 7 & 17 & YES & YES & YES & $1.57$ & $(2,3)$ & NO & 5774\\
$(119,43)$ & 11 & $(35,13)$ & 8 & 7 & YES & YES & YES & $1.86$ & $(2,3)$ & NO & 5775\\
$(119,44)$ & 10 & $(46,17)$ & 8 & 1 & YES & YES & YES & $1.43$ & $(2,3)$ & NO & 5776\\
$(119,50)$ & 10 & $(55,23)$ & 9 & 1 & YES & YES & YES & $1.57$ & $(2,3)$ & 5286 & 5777\\
$(119,46)$ & 10 & $(57,22)$ & 9 & 1 & YES & YES & NO(2) & $1.75$ & $(2,3)$ & NO & 5778\\
$(119,37)$ & 11 & $(71,22)$ & 10 & 1 & YES & YES & YES & $1.57$ & $(2,3)$ & NO & 5779\\
$(119,44)$ & 10 & $(73,27)$ & 9 & 1 & YES & YES & YES & $1.43$ & $(2,3)$ & NO & 5780\\
$(119,50)$ & 10 & $(74,31)$ & 9 & 1 & YES & YES & YES & $1.43$ & $(2,3)$ & NO & 5781\\
$(119,45)$ & 11 & $(82,31)$ & 10 & 1 & YES & YES & YES & $1.57$ & $(2,3)$ & NO & 5782\\
$(119,36)$ & 11 & $(96,29)$ & 11 & 1 & YES & YES & NO(2) & $1.75$ & $(2,3)$ & NO & 5783\\
$(119,37)$ & 11 & $(106,33)$ & 11 & 1 & YES & YES & YES & $1.57$ & $(2,3)$ & NO & 5784\\
$(119,46)$ & 10 & $(106,41)$ & 10 & 1 & YES & YES & YES & $1.57$ & $(2,3)$ & NO & 5785\\
$(119,50)$ & 10 & $(112,47)$ & 10 & 7 & YES & YES & YES & $1.57$ & $(2,3)$ & NO & 5786\\
$(120,47)$ & 12 & $(5,2)$ & 3 & 5 & YES & YES & YES & $1.71$ & $(2,3)$ & -- & 5787\\
$(120,47)$ & 12 & $(7,2)$ & 4 & 1 & YES & YES & YES & $1.71$ & $(2,3)$ & -- & 5788\\
$(121,46)$ & 10 & $(2,1)$ & 1 & 1 & YES & YES & YES & $1.43$ & $(2,3)$ & -- & 5789\\
$(121,46)$ & 10 & $(3,1)$ & 2 & 1 & YES & YES & YES & $1.43$ & $(2,3)$ & -- & 5790\\
$(121,46)$ & 10 & $(3,1)$ & 2 & 1 & YES & YES & YES & $1.43$ & $(2,3)$ & NO & 5791\\
$(121,50)$ & 10 & $(3,1)$ & 2 & 1 & YES & YES & YES & $1.43$ & $(2,3)$ & -- & 5792\\
$(121,50)$ & 10 & $(3,1)$ & 2 & 1 & YES & YES & YES & $1.43$ & $(2,3)$ & NO & 5793\\
$(121,35)$ & 12 & $(5,2)$ & 3 & 1 & YES & YES & YES & $1.57$ & $(2,3)$ & -- & 5794\\
$(121,45)$ & 11 & $(5,2)$ & 3 & 1 & YES & YES & YES & $1.57$ & $(2,3)$ & -- & 5795\\
$(121,46)$ & 10 & $(5,2)$ & 3 & 1 & YES & YES & YES & $1.71$ & $(2,3)$ & -- & 5796\\
$(121,46)$ & 10 & $(5,2)$ & 3 & 1 & YES & YES & YES & $1.57$ & $(2,3)$ & NO & 5797\\
$(121,50)$ & 10 & $(5,2)$ & 3 & 1 & YES & YES & YES & $1.57$ & $(2,3)$ & -- & 5798\\
$(121,32)$ & 11 & $(7,2)$ & 4 & 1 & YES & YES & YES & $1.43$ & $(2,3)$ & -- & 5799\\
$(121,36)$ & 11 & $(7,3)$ & 4 & 1 & YES & YES & YES & $1.43$ & $(2,3)$ & -- & 5800\\
$(121,36)$ & 11 & $(7,3)$ & 4 & 1 & YES & YES & NO(2) & $1.75$ & $(2,3)$ & NO & 5801\\
$(121,37)$ & 11 & $(7,3)$ & 4 & 1 & YES & YES & YES & $1.57$ & $(2,3)$ & -- & 5802\\
$(121,46)$ & 10 & $(7,3)$ & 4 & 1 & YES & YES & NO(2) & $1.75$ & $(2,3)$ & NO & 5803\\
$(121,50)$ & 10 & $(7,3)$ & 4 & 1 & YES & YES & YES & $1.57$ & $(2,3)$ & -- & 5804\\
$(121,50)$ & 10 & $(8,3)$ & 4 & 1 & YES & YES & YES & $1.57$ & $(2,3)$ & -- & 5805\\
$(121,34)$ & 11 & $(9,2)$ & 5 & 1 & YES & YES & NO(2) & $1.62$ & $(2,3)$ & NO & 5806\\
$(121,50)$ & 10 & $(9,2)$ & 5 & 1 & YES & YES & YES & $1.29$ & $(2,3)$ & NO & 5807\\
$(121,50)$ & 10 & $(9,2)$ & 5 & 1 & YES & YES & YES & $1.57$ & $(2,3)$ & -- & 5808\\
$(121,46)$ & 10 & $(12,5)$ & 5 & 1 & YES & YES & YES & $1.57$ & $(2,3)$ & NO & 5809\\
$(121,36)$ & 11 & $(13,3)$ & 6 & 1 & YES & YES & YES & $1.57$ & $(2,3)$ & NO & 5810\\
$(121,43)$ & 11 & $(13,5)$ & 5 & 1 & YES & YES & YES & $1.57$ & $(2,3)$ & NO & 5811\\
$(121,50)$ & 10 & $(13,5)$ & 5 & 1 & YES & YES & YES & $1.71$ & $(2,3)$ & NO & 5812\\
$(121,32)$ & 11 & $(14,3)$ & 6 & 1 & YES & YES & YES & $1.57$ & $(2,3)$ & NO & 5813\\
$(121,50)$ & 10 & $(22,9)$ & 7 & 11 & YES & YES & YES & $1.43$ & $(2,3)$ & NO & 5814\\
$(121,50)$ & 10 & $(26,11)$ & 7 & 1 & YES & YES & YES & $1.43$ & $(2,3)$ & NO & 5815\\
$(121,50)$ & 10 & $(27,11)$ & 8 & 1 & YES & YES & YES & $1.57$ & $(2,3)$ & NO & 5816\\
$(121,46)$ & 10 & $(29,11)$ & 7 & 1 & YES & YES & YES & $1.57$ & $(2,3)$ & NO & 5817\\
$(121,50)$ & 10 & $(39,16)$ & 8 & 1 & YES & YES & NO(2) & $1.50$ & $(2,3)$ & NO & 5818\\
$(121,46)$ & 10 & $(45,17)$ & 9 & 1 & YES & YES & YES & $1.57$ & $(2,3)$ & 4996 & 5819\\
$(121,34)$ & 11 & $(47,13)$ & 8 & 1 & YES & YES & YES & $1.43$ & $(2,3)$ & NO & 5820\\
$(121,50)$ & 10 & $(53,22)$ & 9 & 1 & YES & YES & YES & $1.43$ & $(2,3)$ & NO & 5821\\
$(121,46)$ & 10 & $(66,25)$ & 9 & 11 & YES & YES & YES & $1.43$ & $(2,3)$ & NO & 5822\\
$(121,46)$ & 10 & $(79,30)$ & 9 & 1 & YES & YES & YES & $1.57$ & $(2,3)$ & NO & 5823\\
$(121,50)$ & 10 & $(80,33)$ & 10 & 1 & YES & YES & YES & $1.43$ & $(2,3)$ & NO & 5824\\
$(121,46)$ & 10 & $(92,35)$ & 10 & 1 & YES & YES & YES & $1.57$ & $(2,3)$ & NO & 5825\\
$(121,50)$ & 10 & $(104,43)$ & 10 & 1 & YES & YES & YES & $1.57$ & $(2,3)$ & NO & 5826\\
$(121,43)$ & 11 & $(107,38)$ & 11 & 1 & YES & YES & YES & $1.71$ & $(2,3)$ & NO & 5827\\
$(121,50)$ & 10 & $(109,45)$ & 10 & 1 & YES & YES & YES & $1.71$ & $(2,3)$ & NO & 5828\\
$(121,45)$ & 11 & $(113,42)$ & 11 & 1 & YES & YES & YES & $1.57$ & $(2,3)$ & NO & 5829\\
$(121,34)$ & 11 & $(121,34)$ & 11 & 121 & YES & YES & YES & $1.71$ & $(2,3)$ & NO & 5830\\
$(122,33)$ & 11 & $(2,1)$ & 1 & 2 & YES & YES & YES & $1.43$ & $(2,3)$ & -- & 5831\\
$(122,33)$ & 11 & $(2,1)$ & 1 & 2 & YES & YES & YES & $1.57$ & $(2,3)$ & NO & 5832\\
$(122,33)$ & 11 & $(3,1)$ & 2 & 1 & YES & YES & YES & $1.57$ & $(2,3)$ & NO & 5833\\
$(122,33)$ & 11 & $(3,1)$ & 2 & 1 & YES & YES & YES & $1.57$ & $(2,3)$ & -- & 5834\\
$(122,33)$ & 11 & $(3,1)$ & 2 & 1 & YES & YES & YES & $1.57$ & $(2,3)$ & NO & 5835\\
$(122,33)$ & 11 & $(7,2)$ & 4 & 1 & YES & YES & YES & $1.29$ & $(2,3)$ & -- & 5836\\
$(122,37)$ & 11 & $(7,3)$ & 4 & 1 & YES & YES & YES & $1.57$ & $(2,3)$ & -- & 5837\\
$(122,37)$ & 11 & $(7,3)$ & 4 & 1 & YES & YES & YES & $1.86$ & $(2,3)$ & NO & 5838\\
$(122,55)$ & 11 & $(9,2)$ & 5 & 1 & YES & YES & YES & $1.71$ & $(2,3)$ & NO & 5839\\
$(122,55)$ & 11 & $(9,2)$ & 5 & 1 & YES & YES & YES & $1.43$ & $(2,3)$ & -- & 5840\\
$(122,33)$ & 11 & $(10,3)$ & 5 & 2 & YES & YES & YES & $1.57$ & $(2,3)$ & -- & 5841\\
$(122,37)$ & 11 & $(10,3)$ & 5 & 2 & YES & YES & YES & $1.57$ & $(2,3)$ & -- & 5842\\
$(122,37)$ & 11 & $(11,2)$ & 6 & 1 & YES & YES & YES & $1.57$ & $(2,3)$ & -- & 5843\\
$(122,55)$ & 11 & $(12,5)$ & 5 & 2 & YES & YES & YES & $1.71$ & $(2,3)$ & NO & 5844\\
$(122,33)$ & 11 & $(13,4)$ & 6 & 1 & YES & YES & YES & $1.57$ & $(2,3)$ & NO & 5845\\
$(122,55)$ & 11 & $(16,7)$ & 6 & 2 & YES & YES & YES & $1.57$ & $(2,3)$ & NO & 5846\\
$(122,37)$ & 11 & $(19,6)$ & 8 & 1 & YES & YES & YES & $1.57$ & $(2,3)$ & NO & 5847\\
$(122,37)$ & 11 & $(27,8)$ & 7 & 1 & YES & YES & YES & $1.57$ & $(2,3)$ & NO & 5848\\
$(122,33)$ & 11 & $(48,13)$ & 9 & 2 & YES & YES & YES & $1.43$ & $(2,3)$ & 6052 & 5849\\
$(122,33)$ & 11 & $(67,18)$ & 9 & 1 & YES & YES & YES & $1.57$ & $(2,3)$ & NO & 5850\\
$(122,33)$ & 11 & $(85,23)$ & 10 & 1 & YES & YES & YES & $1.43$ & $(2,3)$ & NO & 5851\\
$(122,37)$ & 11 & $(102,31)$ & 11 & 2 & YES & YES & YES & $1.71$ & $(2,3)$ & NO & 5852\\
$(123,47)$ & 10 & $(3,1)$ & 2 & 3 & YES & YES & NO(2) & $1.38$ & $(2,3)$ & -- & 5853\\
$(123,47)$ & 10 & $(5,2)$ & 3 & 1 & YES & YES & YES & $1.71$ & $(2,3)$ & -- & 5854\\
$(123,47)$ & 10 & $(7,2)$ & 4 & 1 & YES & YES & YES & $1.29$ & $(2,3)$ & -- & 5855\\
$(123,47)$ & 10 & $(7,3)$ & 4 & 1 & YES & YES & YES & $1.57$ & $(2,3)$ & -- & 5856\\
$(123,28)$ & 12 & $(9,4)$ & 5 & 3 & YES & YES & YES & $1.71$ & $(2,3)$ & -- & 5857\\
$(123,47)$ & 10 & $(9,2)$ & 5 & 3 & YES & YES & YES & $1.43$ & $(2,3)$ & NO & 5858\\
$(123,47)$ & 10 & $(12,5)$ & 5 & 3 & YES & YES & YES & $1.57$ & $(2,3)$ & NO & 5859\\
$(123,28)$ & 12 & $(13,2)$ & 7 & 1 & YES & YES & YES & $1.43$ & $(2,3)$ & NO & 5860\\
$(123,28)$ & 12 & $(24,5)$ & 8 & 3 & YES & YES & YES & $1.71$ & $(2,3)$ & NO & 5861\\
$(123,47)$ & 10 & $(55,21)$ & 8 & 1 & YES & YES & NO(2) & $1.38$ & $(2,3)$ & 6263 & 5862\\
$(124,47)$ & 11 & $(11,2)$ & 6 & 1 & YES & YES & YES & $1.43$ & $(2,3)$ & NO & 5863\\
$(125,49)$ & 11 & $(2,1)$ & 1 & 1 & YES & YES & YES & $1.29$ & $(2,3)$ & -- & 5864\\
$(125,37)$ & 11 & $(4,1)$ & 3 & 1 & YES & YES & YES & $1.57$ & $(2,3)$ & -- & 5865\\
$(125,53)$ & 11 & $(5,2)$ & 3 & 5 & YES & YES & YES & $1.71$ & $(2,3)$ & -- & 5866\\
$(125,37)$ & 11 & $(8,3)$ & 4 & 1 & YES & YES & YES & $1.71$ & $(2,3)$ & NO & 5867\\
$(125,29)$ & 12 & $(10,3)$ & 5 & 5 & YES & YES & NO(2) & $1.62$ & $(2,3)$ & -- & 5868\\
$(125,29)$ & 12 & $(11,3)$ & 5 & 1 & YES & YES & YES & $1.43$ & $(2,3)$ & -- & 5869\\
$(125,37)$ & 11 & $(18,5)$ & 6 & 1 & YES & YES & YES & $1.57$ & $(2,3)$ & NO & 5870\\
$(125,38)$ & 12 & $(18,5)$ & 6 & 1 & YES & YES & YES & $1.71$ & $(2,3)$ & NO & 5871\\
$(125,37)$ & 11 & $(31,9)$ & 8 & 1 & YES & YES & YES & $1.57$ & $(2,3)$ & NO & 5872\\
$(125,49)$ & 11 & $(51,20)$ & 9 & 1 & YES & YES & YES & $1.43$ & $(2,3)$ & NO & 5873\\
$(125,49)$ & 11 & $(74,29)$ & 10 & 1 & YES & YES & YES & $1.29$ & $(2,3)$ & NO & 5874\\
$(125,53)$ & 11 & $(85,36)$ & 10 & 5 & YES & YES & YES & $1.57$ & $(2,3)$ & 7632 & 5875\\
$(125,29)$ & 12 & $(116,27)$ & 11 & 1 & YES & YES & NO(2) & $1.50$ & $(2,3)$ & NO & 5876\\
$(126,55)$ & 11 & $(8,3)$ & 4 & 2 & YES & YES & YES & $1.71$ & $(2,3)$ & -- & 5877\\
$(126,55)$ & 11 & $(10,3)$ & 5 & 2 & YES & YES & YES & $1.57$ & $(2,3)$ & NO & 5878\\
$(127,35)$ & 11 & $(7,3)$ & 4 & 1 & YES & YES & YES & $1.57$ & $(2,3)$ & -- & 5879\\
$(127,55)$ & 12 & $(7,3)$ & 4 & 1 & YES & YES & YES & $1.71$ & $(2,3)$ & -- & 5880\\
$(127,57)$ & 11 & $(7,2)$ & 4 & 1 & YES & YES & YES & $1.57$ & $(2,3)$ & NO & 5881\\
$(127,34)$ & 11 & $(8,3)$ & 4 & 1 & YES & YES & YES & $1.71$ & $(2,3)$ & -- & 5882\\
$(127,37)$ & 12 & $(9,2)$ & 5 & 1 & YES & YES & YES & $1.57$ & $(2,3)$ & -- & 5883\\
$(127,45)$ & 11 & $(9,2)$ & 5 & 1 & YES & YES & YES & $1.43$ & $(2,3)$ & -- & 5884\\
$(127,49)$ & 11 & $(10,3)$ & 5 & 1 & YES & YES & YES & $1.71$ & $(2,3)$ & -- & 5885\\
$(127,30)$ & 12 & $(11,4)$ & 5 & 1 & YES & YES & YES & $1.71$ & $(2,3)$ & -- & 5886\\
$(127,37)$ & 12 & $(11,2)$ & 6 & 1 & YES & YES & YES & $1.57$ & $(2,3)$ & -- & 5887\\
$(127,34)$ & 11 & $(13,3)$ & 6 & 1 & YES & YES & YES & $1.57$ & $(2,3)$ & -- & 5888\\
$(127,49)$ & 11 & $(28,11)$ & 8 & 1 & YES & YES & YES & $1.71$ & $(2,3)$ & NO & 5889\\
$(127,35)$ & 11 & $(32,9)$ & 8 & 1 & YES & YES & YES & $1.57$ & $(2,3)$ & NO & 5890\\
$(127,29)$ & 11 & $(37,8)$ & 8 & 1 & YES & YES & YES & $1.43$ & $(2,3)$ & NO & 5891\\
$(127,34)$ & 11 & $(63,17)$ & 9 & 1 & YES & YES & YES & $1.71$ & $(2,3)$ & NO & 5892\\
$(127,49)$ & 11 & $(67,26)$ & 9 & 1 & YES & YES & YES & $1.71$ & $(2,3)$ & NO & 5893\\
$(127,29)$ & 11 & $(71,16)$ & 10 & 1 & YES & YES & YES & $1.43$ & $(2,3)$ & NO & 5894\\
$(128,49)$ & 10 & $(2,1)$ & 1 & 2 & YES & YES & YES & $1.57$ & $(2,3)$ & -- & 5895\\
$(128,53)$ & 11 & $(3,1)$ & 2 & 1 & YES & YES & YES & $1.71$ & $(2,3)$ & NO & 5896\\
$(128,53)$ & 11 & $(3,1)$ & 2 & 1 & YES & YES & YES & $1.57$ & $(2,3)$ & -- & 5897\\
$(128,49)$ & 10 & $(4,1)$ & 3 & 4 & YES & YES & YES & $1.43$ & $(2,3)$ & -- & 5898\\
$(128,53)$ & 11 & $(4,1)$ & 3 & 4 & YES & YES & YES & $1.57$ & $(2,3)$ & NO & 5899\\
$(128,47)$ & 10 & $(5,2)$ & 3 & 1 & YES & YES & YES & $1.57$ & $(2,3)$ & -- & 5900\\
$(128,49)$ & 10 & $(5,2)$ & 3 & 1 & YES & YES & YES & $1.43$ & $(2,3)$ & -- & 5901\\
$(128,53)$ & 11 & $(5,2)$ & 3 & 1 & YES & YES & YES & $1.57$ & $(2,3)$ & -- & 5902\\
$(128,47)$ & 10 & $(7,3)$ & 4 & 1 & YES & YES & YES & $1.57$ & $(2,3)$ & -- & 5903\\
$(128,49)$ & 10 & $(7,2)$ & 4 & 1 & YES & YES & YES & $1.43$ & $(2,3)$ & -- & 5904\\
$(128,49)$ & 10 & $(7,2)$ & 4 & 1 & YES & YES & YES & $1.57$ & $(2,3)$ & NO & 5905\\
$(128,49)$ & 10 & $(7,3)$ & 4 & 1 & YES & YES & YES & $1.57$ & $(2,3)$ & -- & 5906\\
$(128,47)$ & 10 & $(8,3)$ & 4 & 8 & YES & YES & YES & $1.57$ & $(2,3)$ & NO & 5907\\
$(128,49)$ & 10 & $(8,3)$ & 4 & 8 & YES & YES & YES & $1.57$ & $(2,3)$ & -- & 5908\\
$(128,53)$ & 11 & $(8,3)$ & 4 & 8 & YES & YES & YES & $1.57$ & $(2,3)$ & NO & 5909\\
$(128,29)$ & 11 & $(9,4)$ & 5 & 1 & YES & YES & YES & $1.57$ & $(2,3)$ & -- & 5910\\
$(128,49)$ & 10 & $(9,2)$ & 5 & 1 & YES & YES & NO(3) & $1.29$ & $(2,3)$ & NO & 5911\\
$(128,53)$ & 11 & $(9,2)$ & 5 & 1 & YES & YES & YES & $1.57$ & $(2,3)$ & NO & 5912\\
$(128,53)$ & 11 & $(9,2)$ & 5 & 1 & YES & YES & YES & $1.43$ & $(2,3)$ & -- & 5913\\
$(128,47)$ & 10 & $(12,5)$ & 5 & 4 & YES & YES & YES & $1.57$ & $(2,3)$ & NO & 5914\\
$(128,47)$ & 10 & $(13,5)$ & 5 & 1 & YES & YES & YES & $1.57$ & $(2,3)$ & NO & 5915\\
$(128,47)$ & 10 & $(18,7)$ & 6 & 2 & YES & YES & YES & $1.57$ & $(2,3)$ & NO & 5916\\
$(128,47)$ & 10 & $(19,7)$ & 6 & 1 & YES & YES & YES & $1.57$ & $(2,3)$ & NO & 5917\\
$(128,49)$ & 10 & $(29,11)$ & 7 & 1 & YES & YES & YES & $1.43$ & $(2,3)$ & 7519 & 5918\\
$(128,53)$ & 11 & $(39,16)$ & 8 & 1 & YES & YES & YES & $1.57$ & $(2,3)$ & NO & 5919\\
$(128,47)$ & 10 & $(49,18)$ & 8 & 1 & YES & YES & YES & $1.57$ & $(2,3)$ & NO & 5920\\
$(128,49)$ & 10 & $(55,21)$ & 8 & 1 & YES & YES & YES & $1.43$ & $(2,3)$ & NO & 5921\\
$(128,53)$ & 11 & $(63,26)$ & 9 & 1 & YES & YES & YES & $1.71$ & $(2,3)$ & NO & 5922\\
$(128,49)$ & 10 & $(73,28)$ & 10 & 1 & YES & YES & YES & $1.57$ & $(2,3)$ & NO & 5923\\
$(128,47)$ & 10 & $(79,29)$ & 9 & 1 & YES & YES & YES & $1.43$ & $(2,3)$ & NO & 5924\\
$(128,49)$ & 10 & $(81,31)$ & 9 & 1 & YES & YES & YES & $1.43$ & $(2,3)$ & NO & 5925\\
$(128,49)$ & 10 & $(107,41)$ & 10 & 1 & YES & YES & YES & $1.71$ & $(2,3)$ & NO & 5926\\
$(128,47)$ & 10 & $(109,40)$ & 10 & 1 & YES & YES & YES & $1.43$ & $(2,3)$ & NO & 5927\\
$(128,49)$ & 10 & $(128,49)$ & 10 & 128 & YES & YES & YES & $1.57$ & $(2,3)$ & NO & 5928\\
$(129,49)$ & 10 & $(2,1)$ & 1 & 1 & YES & YES & YES & $1.43$ & $(2,3)$ & -- & 5929\\
$(129,49)$ & 10 & $(3,1)$ & 2 & 3 & YES & YES & YES & $1.43$ & $(2,3)$ & -- & 5930\\
$(129,49)$ & 10 & $(3,1)$ & 2 & 3 & YES & YES & YES & $1.43$ & $(2,3)$ & NO & 5931\\
$(129,50)$ & 10 & $(3,1)$ & 2 & 3 & YES & YES & NO(2) & $1.75$ & $(2,3)$ & NO & 5932\\
$(129,50)$ & 10 & $(3,1)$ & 2 & 3 & YES & YES & YES & $1.43$ & $(2,3)$ & -- & 5933\\
$(129,53)$ & 11 & $(3,1)$ & 2 & 3 & YES & YES & NO(2) & $1.62$ & $(2,3)$ & -- & 5934\\
$(129,49)$ & 10 & $(4,1)$ & 3 & 1 & YES & YES & YES & $1.43$ & $(2,3)$ & -- & 5935\\
$(129,50)$ & 10 & $(4,1)$ & 3 & 1 & YES & YES & YES & $1.43$ & $(2,3)$ & -- & 5936\\
$(129,49)$ & 10 & $(5,2)$ & 3 & 1 & YES & YES & NO(2) & $1.38$ & $(2,3)$ & -- & 5937\\
$(129,50)$ & 10 & $(5,2)$ & 3 & 1 & YES & YES & YES & $1.57$ & $(2,3)$ & -- & 5938\\
$(129,53)$ & 11 & $(5,2)$ & 3 & 1 & YES & YES & NO(2) & $1.75$ & $(2,3)$ & NO & 5939\\
$(129,56)$ & 11 & $(5,2)$ & 3 & 1 & YES & YES & NO(2) & $1.62$ & $(2,3)$ & -- & 5940\\
$(129,49)$ & 10 & $(7,2)$ & 4 & 1 & YES & YES & YES & $1.29$ & $(2,3)$ & -- & 5941\\
$(129,49)$ & 10 & $(7,2)$ & 4 & 1 & YES & YES & YES & $1.29$ & $(2,3)$ & NO & 5942\\
$(129,49)$ & 10 & $(7,3)$ & 4 & 1 & YES & YES & YES & $1.43$ & $(2,3)$ & -- & 5943\\
$(129,50)$ & 10 & $(7,3)$ & 4 & 1 & YES & YES & YES & $1.57$ & $(2,3)$ & -- & 5944\\
$(129,53)$ & 11 & $(7,2)$ & 4 & 1 & YES & YES & YES & $1.57$ & $(2,3)$ & -- & 5945\\
$(129,53)$ & 11 & $(7,3)$ & 4 & 1 & YES & YES & YES & $1.71$ & $(2,3)$ & -- & 5946\\
$(129,56)$ & 11 & $(7,3)$ & 4 & 1 & YES & YES & YES & $1.71$ & $(2,3)$ & -- & 5947\\
$(129,50)$ & 10 & $(8,3)$ & 4 & 1 & YES & YES & NO(2) & $1.50$ & $(2,3)$ & 6256 & 5948\\
$(129,56)$ & 11 & $(8,3)$ & 4 & 1 & YES & YES & YES & $1.71$ & $(2,3)$ & -- & 5949\\
$(129,50)$ & 10 & $(9,2)$ & 5 & 3 & YES & YES & NO(3) & $1.29$ & $(2,3)$ & NO & 5950\\
$(129,50)$ & 10 & $(9,2)$ & 5 & 3 & YES & YES & YES & $1.43$ & $(2,3)$ & -- & 5951\\
$(129,53)$ & 11 & $(9,2)$ & 5 & 3 & YES & YES & NO(2) & $1.62$ & $(2,3)$ & -- & 5952\\
$(129,53)$ & 11 & $(9,2)$ & 5 & 3 & YES & YES & NO(2) & $1.50$ & $(2,3)$ & NO & 5953\\
$(129,53)$ & 11 & $(9,4)$ & 5 & 3 & YES & YES & YES & $1.71$ & $(2,3)$ & -- & 5954\\
$(129,53)$ & 11 & $(9,4)$ & 5 & 3 & YES & YES & NO(2) & $1.62$ & $(2,3)$ & NO & 5955\\
$(129,49)$ & 10 & $(11,4)$ & 5 & 1 & YES & YES & YES & $1.29$ & $(2,3)$ & NO & 5956\\
$(129,49)$ & 10 & $(12,5)$ & 5 & 3 & YES & YES & YES & $1.43$ & $(2,3)$ & NO & 5957\\
$(129,28)$ & 12 & $(13,2)$ & 7 & 1 & YES & YES & YES & $1.29$ & $(2,3)$ & NO & 5958\\
$(129,53)$ & 11 & $(13,3)$ & 6 & 1 & YES & YES & YES & $1.57$ & $(2,3)$ & -- & 5959\\
$(129,53)$ & 11 & $(14,5)$ & 6 & 1 & YES & YES & YES & $1.86$ & $(2,3)$ & NO & 5960\\
$(129,49)$ & 10 & $(23,9)$ & 7 & 1 & YES & YES & YES & $1.43$ & $(2,3)$ & 4939 & 5961\\
$(129,53)$ & 11 & $(29,12)$ & 7 & 1 & YES & YES & NO(2) & $1.50$ & $(2,3)$ & NO & 5962\\
$(129,53)$ & 11 & $(39,16)$ & 8 & 3 & YES & YES & NO(2) & $1.50$ & $(2,3)$ & NO & 5963\\
$(129,56)$ & 11 & $(39,17)$ & 8 & 3 & YES & YES & NO(2) & $1.62$ & $(2,3)$ & NO & 5964\\
$(129,53)$ & 11 & $(41,17)$ & 8 & 1 & YES & YES & YES & $1.86$ & $(2,3)$ & NO & 5965\\
$(129,50)$ & 10 & $(44,17)$ & 8 & 1 & YES & YES & YES & $1.57$ & $(2,3)$ & NO & 5966\\
$(129,49)$ & 10 & $(45,17)$ & 9 & 3 & YES & YES & YES & $1.71$ & $(2,3)$ & NO & 5967\\
$(129,49)$ & 10 & $(50,19)$ & 8 & 1 & YES & YES & YES & $1.43$ & $(2,3)$ & NO & 5968\\
$(129,49)$ & 10 & $(55,21)$ & 8 & 1 & YES & YES & YES & $1.43$ & $(2,3)$ & NO & 5969\\
$(129,49)$ & 10 & $(79,30)$ & 9 & 1 & YES & YES & YES & $1.43$ & $(2,3)$ & NO & 5970\\
$(129,50)$ & 10 & $(80,31)$ & 9 & 1 & YES & YES & YES & $1.43$ & $(2,3)$ & NO & 5971\\
$(129,53)$ & 11 & $(90,37)$ & 11 & 3 & YES & YES & NO(2) & $1.75$ & $(2,3)$ & NO & 5972\\
$(129,49)$ & 10 & $(92,35)$ & 10 & 1 & YES & YES & YES & $1.43$ & $(2,3)$ & NO & 5973\\
$(129,50)$ & 10 & $(111,43)$ & 10 & 3 & YES & YES & YES & $1.43$ & $(2,3)$ & NO & 5974\\
$(129,49)$ & 10 & $(129,49)$ & 10 & 129 & YES & YES & YES & $1.29$ & $(2,3)$ & NO & 5975\\
$(129,50)$ & 10 & $(129,50)$ & 10 & 129 & YES & YES & YES & $1.43$ & $(2,3)$ & NO & 5976\\
$(130,47)$ & 11 & $(5,2)$ & 3 & 5 & YES & YES & NO(2) & $1.50$ & $(2,3)$ & -- & 5977\\
$(130,47)$ & 11 & $(7,3)$ & 4 & 1 & YES & YES & NO(2) & $1.50$ & $(2,3)$ & NO & 5978\\
$(130,47)$ & 11 & $(7,3)$ & 4 & 1 & YES & YES & YES & $1.71$ & $(2,3)$ & -- & 5979\\
$(130,47)$ & 11 & $(8,3)$ & 4 & 2 & YES & YES & YES & $1.86$ & $(2,3)$ & -- & 5980\\
$(130,57)$ & 11 & $(9,2)$ & 5 & 1 & YES & YES & YES & $1.43$ & $(2,3)$ & -- & 5981\\
$(130,57)$ & 11 & $(9,2)$ & 5 & 1 & YES & YES & YES & $1.71$ & $(2,3)$ & NO & 5982\\
$(130,57)$ & 11 & $(34,15)$ & 8 & 2 & YES & YES & YES & $1.57$ & $(2,3)$ & NO & 5983\\
$(130,47)$ & 11 & $(61,22)$ & 9 & 1 & YES & YES & YES & $1.43$ & $(2,3)$ & NO & 5984\\
$(131,36)$ & 11 & $(2,1)$ & 1 & 1 & YES & YES & YES & $1.43$ & $(2,3)$ & -- & 5985\\
$(131,40)$ & 11 & $(2,1)$ & 1 & 1 & YES & YES & YES & $1.57$ & $(2,3)$ & NO & 5986\\
$(131,50)$ & 10 & $(2,1)$ & 1 & 1 & YES & YES & YES & $1.43$ & $(2,3)$ & -- & 5987\\
$(131,36)$ & 11 & $(3,1)$ & 2 & 1 & YES & YES & YES & $1.57$ & $(2,3)$ & NO & 5988\\
$(131,36)$ & 11 & $(3,1)$ & 2 & 1 & YES & YES & YES & $1.57$ & $(2,3)$ & -- & 5989\\
$(131,50)$ & 10 & $(3,1)$ & 2 & 1 & YES & YES & YES & $1.57$ & $(2,3)$ & -- & 5990\\
$(131,50)$ & 10 & $(3,1)$ & 2 & 1 & YES & YES & YES & $1.57$ & $(2,3)$ & NO & 5991\\
$(131,30)$ & 11 & $(4,1)$ & 3 & 1 & YES & YES & YES & $1.43$ & $(2,3)$ & NO & 5992\\
$(131,30)$ & 11 & $(4,1)$ & 3 & 1 & YES & YES & YES & $1.43$ & $(2,3)$ & -- & 5993\\
$(131,40)$ & 11 & $(4,1)$ & 3 & 1 & YES & YES & YES & $1.57$ & $(2,3)$ & NO & 5994\\
$(131,40)$ & 11 & $(4,1)$ & 3 & 1 & YES & YES & YES & $1.57$ & $(2,3)$ & -- & 5995\\
$(131,50)$ & 10 & $(4,1)$ & 3 & 1 & YES & YES & YES & $1.43$ & $(2,3)$ & -- & 5996\\
$(131,55)$ & 10 & $(4,1)$ & 3 & 1 & YES & YES & YES & $1.43$ & $(2,3)$ & NO & 5997\\
$(131,55)$ & 10 & $(4,1)$ & 3 & 1 & YES & YES & YES & $1.43$ & $(2,3)$ & -- & 5998\\
$(131,36)$ & 11 & $(5,1)$ & 4 & 1 & YES & YES & YES & $1.71$ & $(2,3)$ & NO & 5999\\
$(131,50)$ & 10 & $(5,2)$ & 3 & 1 & YES & YES & YES & $1.57$ & $(2,3)$ & -- & 6000\\
$(131,50)$ & 10 & $(5,2)$ & 3 & 1 & YES & YES & YES & $1.57$ & $(2,3)$ & NO & 6001\\
$(131,50)$ & 10 & $(7,3)$ & 4 & 1 & YES & YES & YES & $1.71$ & $(2,3)$ & -- & 6002\\
$(131,40)$ & 11 & $(8,3)$ & 4 & 1 & YES & YES & YES & $1.71$ & $(2,3)$ & NO & 6003\\
$(131,48)$ & 11 & $(9,2)$ & 5 & 1 & YES & YES & YES & $1.57$ & $(2,3)$ & -- & 6004\\
$(131,30)$ & 11 & $(13,5)$ & 5 & 1 & YES & YES & YES & $1.57$ & $(2,3)$ & NO & 6005\\
$(131,47)$ & 11 & $(13,5)$ & 5 & 1 & YES & YES & YES & $1.71$ & $(2,3)$ & NO & 6006\\
$(131,30)$ & 11 & $(14,5)$ & 6 & 1 & YES & YES & YES & $1.71$ & $(2,3)$ & -- & 6007\\
$(131,39)$ & 11 & $(15,4)$ & 6 & 1 & YES & YES & YES & $1.57$ & $(2,3)$ & NO & 6008\\
$(131,39)$ & 11 & $(18,5)$ & 6 & 1 & YES & YES & YES & $1.71$ & $(2,3)$ & NO & 6009\\
$(131,40)$ & 11 & $(19,6)$ & 8 & 1 & YES & YES & YES & $1.71$ & $(2,3)$ & NO & 6010\\
$(131,55)$ & 10 & $(22,9)$ & 7 & 1 & YES & YES & YES & $1.43$ & $(2,3)$ & NO & 6011\\
$(131,55)$ & 10 & $(29,12)$ & 7 & 1 & YES & YES & YES & $1.57$ & $(2,3)$ & NO & 6012\\
$(131,30)$ & 11 & $(35,8)$ & 8 & 1 & YES & YES & YES & $1.43$ & $(2,3)$ & 5726 & 6013\\
$(131,30)$ & 11 & $(47,11)$ & 9 & 1 & YES & YES & YES & $1.57$ & $(2,3)$ & NO & 6014\\
$(131,50)$ & 10 & $(55,21)$ & 8 & 1 & YES & YES & YES & $1.43$ & $(2,3)$ & NO & 6015\\
$(131,55)$ & 10 & $(55,23)$ & 9 & 1 & YES & YES & YES & $1.57$ & $(2,3)$ & NO & 6016\\
$(131,50)$ & 10 & $(60,23)$ & 9 & 1 & YES & YES & YES & $1.57$ & $(2,3)$ & 5490 & 6017\\
$(131,50)$ & 10 & $(76,29)$ & 9 & 1 & YES & YES & YES & $1.57$ & $(2,3)$ & NO & 6018\\
$(131,50)$ & 10 & $(89,34)$ & 9 & 1 & YES & YES & YES & $1.43$ & $(2,3)$ & NO & 6019\\
$(131,36)$ & 11 & $(91,25)$ & 10 & 1 & YES & YES & YES & $1.43$ & $(2,3)$ & NO & 6020\\
$(131,39)$ & 11 & $(104,31)$ & 11 & 1 & YES & YES & YES & $1.57$ & $(2,3)$ & NO & 6021\\
$(131,36)$ & 11 & $(131,36)$ & 11 & 131 & YES & YES & YES & $1.57$ & $(2,3)$ & NO & 6022\\
$(131,48)$ & 11 & $(131,48)$ & 11 & 131 & YES & YES & YES & $1.57$ & $(2,3)$ & NO & 6023\\
$(131,50)$ & 10 & $(131,50)$ & 10 & 131 & YES & YES & YES & $1.43$ & $(2,3)$ & NO & 6024\\
$(132,35)$ & 11 & $(7,3)$ & 4 & 1 & YES & YES & YES & $1.71$ & $(2,3)$ & NO & 6025\\
$(132,35)$ & 11 & $(7,3)$ & 4 & 1 & YES & YES & YES & $1.71$ & $(2,3)$ & -- & 6026\\
$(132,49)$ & 11 & $(11,2)$ & 6 & 11 & YES & YES & NO(2) & $1.62$ & $(2,3)$ & NO & 6027\\
$(132,49)$ & 11 & $(11,3)$ & 5 & 11 & YES & YES & YES & $1.71$ & $(2,3)$ & NO & 6028\\
$(132,49)$ & 11 & $(16,3)$ & 7 & 4 & YES & YES & YES & $1.71$ & $(2,3)$ & NO & 6029\\
$(133,31)$ & 12 & $(2,1)$ & 1 & 1 & YES & YES & YES & $1.57$ & $(2,3)$ & -- & 6030\\
$(133,31)$ & 12 & $(2,1)$ & 1 & 1 & YES & YES & YES & $1.71$ & $(2,3)$ & NO & 6031\\
$(133,39)$ & 11 & $(4,1)$ & 3 & 1 & YES & YES & YES & $1.43$ & $(2,3)$ & -- & 6032\\
$(133,39)$ & 11 & $(4,1)$ & 3 & 1 & YES & YES & YES & $1.57$ & $(2,3)$ & NO & 6033\\
$(133,39)$ & 11 & $(5,2)$ & 3 & 1 & YES & YES & NO(2) & $1.62$ & $(2,3)$ & -- & 6034\\
$(133,51)$ & 11 & $(5,2)$ & 3 & 1 & YES & YES & YES & $1.71$ & $(2,3)$ & -- & 6035\\
$(133,55)$ & 11 & $(5,1)$ & 4 & 1 & YES & YES & NO(2) & $1.62$ & $(2,3)$ & NO & 6036\\
$(133,58)$ & 11 & $(5,2)$ & 3 & 1 & YES & YES & YES & $1.57$ & $(2,3)$ & -- & 6037\\
$(133,36)$ & 11 & $(7,3)$ & 4 & 7 & YES & YES & YES & $1.57$ & $(2,3)$ & NO & 6038\\
$(133,55)$ & 11 & $(7,2)$ & 4 & 7 & YES & YES & YES & $1.71$ & $(2,3)$ & NO & 6039\\
$(133,36)$ & 11 & $(8,3)$ & 4 & 1 & YES & YES & YES & $1.57$ & $(2,3)$ & NO & 6040\\
$(133,48)$ & 11 & $(9,4)$ & 5 & 1 & YES & YES & YES & $1.57$ & $(2,3)$ & NO & 6041\\
$(133,30)$ & 12 & $(10,3)$ & 5 & 1 & YES & YES & YES & $1.57$ & $(2,3)$ & -- & 6042\\
$(133,58)$ & 11 & $(11,5)$ & 6 & 1 & YES & YES & YES & $1.71$ & $(2,3)$ & NO & 6043\\
$(133,31)$ & 12 & $(14,3)$ & 6 & 7 & YES & YES & YES & $1.57$ & $(2,3)$ & -- & 6044\\
$(133,60)$ & 11 & $(16,7)$ & 6 & 1 & YES & YES & YES & $1.43$ & $(2,3)$ & NO & 6045\\
$(133,36)$ & 11 & $(17,5)$ & 6 & 1 & YES & YES & YES & $1.57$ & $(2,3)$ & NO & 6046\\
$(133,39)$ & 11 & $(17,4)$ & 7 & 1 & YES & YES & YES & $1.57$ & $(2,3)$ & NO & 6047\\
$(133,36)$ & 11 & $(25,7)$ & 7 & 1 & YES & YES & YES & $1.57$ & $(2,3)$ & NO & 6048\\
$(133,58)$ & 11 & $(25,11)$ & 7 & 1 & YES & YES & YES & $1.71$ & $(2,3)$ & NO & 6049\\
$(133,48)$ & 11 & $(30,11)$ & 7 & 1 & YES & YES & YES & $1.57$ & $(2,3)$ & NO & 6050\\
$(133,58)$ & 11 & $(30,13)$ & 8 & 1 & YES & YES & YES & $1.57$ & $(2,3)$ & NO & 6051\\
$(133,36)$ & 11 & $(37,10)$ & 8 & 1 & YES & YES & YES & $1.43$ & $(2,3)$ & 5849 & 6052\\
$(133,30)$ & 12 & $(48,11)$ & 9 & 1 & YES & YES & YES & $1.57$ & $(2,3)$ & NO & 6053\\
$(133,39)$ & 11 & $(55,16)$ & 9 & 1 & YES & YES & YES & $1.43$ & $(2,3)$ & NO & 6054\\
$(133,55)$ & 11 & $(63,26)$ & 9 & 7 & YES & YES & YES & $1.71$ & $(2,3)$ & NO & 6055\\
$(133,39)$ & 11 & $(65,19)$ & 9 & 1 & YES & YES & YES & $1.43$ & $(2,3)$ & NO & 6056\\
$(133,31)$ & 12 & $(69,16)$ & 11 & 1 & YES & YES & NO(2) & $1.50$ & $(2,3)$ & NO & 6057\\
$(133,39)$ & 11 & $(75,22)$ & 10 & 1 & YES & YES & YES & $1.57$ & $(2,3)$ & NO & 6058\\
$(133,55)$ & 11 & $(75,31)$ & 9 & 1 & YES & YES & NO(2) & $1.62$ & $(2,3)$ & 6916 & 6059\\
$(133,39)$ & 11 & $(89,26)$ & 10 & 1 & YES & YES & YES & $1.43$ & $(2,3)$ & 6435 & 6060\\
$(133,55)$ & 11 & $(104,43)$ & 10 & 1 & YES & YES & NO(2) & $1.62$ & $(2,3)$ & NO & 6061\\
$(133,51)$ & 11 & $(107,41)$ & 10 & 1 & YES & YES & YES & $1.57$ & $(2,3)$ & NO & 6062\\
$(133,39)$ & 11 & $(109,32)$ & 12 & 1 & YES & YES & YES & $1.57$ & $(2,3)$ & 7347 & 6063\\
$(134,39)$ & 11 & $(2,1)$ & 1 & 2 & YES & YES & NO(2) & $1.62$ & $(2,3)$ & -- & 6064\\
$(134,39)$ & 11 & $(3,1)$ & 2 & 1 & YES & YES & NO(2) & $1.67$ & $(2,3)$ & NO & 6065\\
$(134,39)$ & 11 & $(3,1)$ & 2 & 1 & YES & YES & NO(2) & $1.67$ & $(2,3)$ & -- & 6066\\
$(134,55)$ & 11 & $(3,1)$ & 2 & 1 & YES & YES & YES & $1.57$ & $(2,3)$ & -- & 6067\\
$(134,41)$ & 11 & $(5,2)$ & 3 & 1 & YES & YES & YES & $1.57$ & $(2,3)$ & -- & 6068\\
$(134,55)$ & 11 & $(5,2)$ & 3 & 1 & YES & YES & NO(2) & $1.62$ & $(2,3)$ & -- & 6069\\
$(134,37)$ & 11 & $(7,3)$ & 4 & 1 & YES & YES & YES & $1.57$ & $(2,3)$ & -- & 6070\\
$(134,37)$ & 11 & $(7,3)$ & 4 & 1 & YES & YES & YES & $1.57$ & $(2,3)$ & NO & 6071\\
$(134,39)$ & 11 & $(7,3)$ & 4 & 1 & YES & YES & YES & $1.57$ & $(2,3)$ & -- & 6072\\
$(134,39)$ & 11 & $(7,3)$ & 4 & 1 & YES & YES & YES & $1.57$ & $(2,3)$ & NO & 6073\\
$(134,49)$ & 11 & $(7,3)$ & 4 & 1 & YES & YES & YES & $1.57$ & $(2,3)$ & -- & 6074\\
$(134,55)$ & 11 & $(7,3)$ & 4 & 1 & YES & YES & YES & $1.71$ & $(2,3)$ & -- & 6075\\
$(134,37)$ & 11 & $(8,3)$ & 4 & 2 & YES & YES & YES & $1.57$ & $(2,3)$ & NO & 6076\\
$(134,39)$ & 11 & $(9,4)$ & 5 & 1 & YES & YES & YES & $1.57$ & $(2,3)$ & NO & 6077\\
$(134,55)$ & 11 & $(9,2)$ & 5 & 1 & YES & YES & YES & $1.57$ & $(2,3)$ & -- & 6078\\
$(134,39)$ & 11 & $(10,3)$ & 5 & 2 & YES & YES & YES & $1.57$ & $(2,3)$ & -- & 6079\\
$(134,41)$ & 11 & $(11,3)$ & 5 & 1 & YES & YES & NO(2) & $1.50$ & $(2,3)$ & NO & 6080\\
$(134,55)$ & 11 & $(12,5)$ & 5 & 2 & YES & YES & YES & $1.57$ & $(2,3)$ & 5404 & 6081\\
$(134,37)$ & 11 & $(13,4)$ & 6 & 1 & YES & YES & YES & $1.43$ & $(2,3)$ & NO & 6082\\
$(134,37)$ & 11 & $(17,5)$ & 6 & 1 & YES & YES & YES & $1.57$ & $(2,3)$ & NO & 6083\\
$(134,41)$ & 11 & $(17,5)$ & 6 & 1 & YES & YES & NO(2) & $1.50$ & $(2,3)$ & NO & 6084\\
$(134,39)$ & 11 & $(19,5)$ & 7 & 1 & YES & YES & YES & $1.57$ & $(2,3)$ & NO & 6085\\
$(134,39)$ & 11 & $(27,8)$ & 7 & 1 & YES & YES & YES & $1.57$ & $(2,3)$ & NO & 6086\\
$(134,55)$ & 11 & $(29,12)$ & 7 & 1 & YES & YES & NO(2) & $1.62$ & $(2,3)$ & NO & 6087\\
$(134,37)$ & 11 & $(32,9)$ & 8 & 2 & YES & YES & YES & $1.57$ & $(2,3)$ & NO & 6088\\
$(134,41)$ & 11 & $(33,10)$ & 8 & 1 & YES & YES & YES & $1.43$ & $(2,3)$ & 7654 & 6089\\
$(134,49)$ & 11 & $(35,13)$ & 8 & 1 & YES & YES & YES & $1.71$ & $(2,3)$ & NO & 6090\\
$(134,39)$ & 11 & $(37,11)$ & 8 & 1 & YES & YES & YES & $1.57$ & $(2,3)$ & NO & 6091\\
$(134,39)$ & 11 & $(44,13)$ & 8 & 2 & YES & YES & YES & $1.43$ & $(2,3)$ & NO & 6092\\
$(134,39)$ & 11 & $(45,13)$ & 10 & 1 & YES & YES & YES & $1.57$ & $(2,3)$ & 5032 & 6093\\
$(134,55)$ & 11 & $(100,41)$ & 10 & 2 & YES & YES & YES & $1.71$ & $(2,3)$ & NO & 6094\\
$(134,39)$ & 11 & $(127,37)$ & 12 & 1 & YES & YES & YES & $1.57$ & $(2,3)$ & 7794 & 6095\\
$(135,41)$ & 11 & $(7,3)$ & 4 & 1 & YES & YES & YES & $1.71$ & $(2,3)$ & -- & 6096\\
$(135,56)$ & 11 & $(7,2)$ & 4 & 1 & YES & YES & YES & $1.43$ & $(2,3)$ & -- & 6097\\
$(135,56)$ & 11 & $(7,2)$ & 4 & 1 & YES & YES & YES & $1.71$ & $(2,3)$ & NO & 6098\\
$(135,56)$ & 11 & $(7,3)$ & 4 & 1 & YES & YES & YES & $1.57$ & $(2,3)$ & -- & 6099\\
$(135,56)$ & 11 & $(9,4)$ & 5 & 9 & YES & YES & YES & $1.71$ & $(2,3)$ & -- & 6100\\
$(135,32)$ & 12 & $(22,5)$ & 7 & 1 & YES & YES & YES & $1.43$ & $(2,3)$ & NO & 6101\\
$(135,56)$ & 11 & $(22,9)$ & 7 & 1 & YES & YES & NO(2) & $1.75$ & $(2,3)$ & NO & 6102\\
$(135,56)$ & 11 & $(111,46)$ & 10 & 3 & YES & YES & YES & $1.57$ & $(2,3)$ & NO & 6103\\
$(136,31)$ & 11 & $(4,1)$ & 3 & 4 & NO & YES & YES & $1.57$ & $(2,3)$ & -- & 6104\\
$(136,53)$ & 11 & $(5,2)$ & 3 & 1 & YES & YES & NO(2) & $1.62$ & $(2,3)$ & -- & 6105\\
$(136,53)$ & 11 & $(5,2)$ & 3 & 1 & YES & YES & NO(2) & $1.75$ & $(2,3)$ & NO & 6106\\
$(136,57)$ & 11 & $(5,1)$ & 4 & 1 & YES & YES & YES & $1.29$ & $(2,3)$ & NO & 6107\\
$(136,57)$ & 11 & $(6,1)$ & 5 & 2 & YES & YES & NO(2) & $1.38$ & $(2,3)$ & NO & 6108\\
$(136,61)$ & 12 & $(6,1)$ & 5 & 2 & YES & YES & YES & $1.71$ & $(2,3)$ & -- & 6109\\
$(136,53)$ & 11 & $(7,3)$ & 4 & 1 & YES & YES & YES & $1.57$ & $(2,3)$ & -- & 6110\\
$(136,59)$ & 11 & $(8,3)$ & 4 & 8 & YES & YES & YES & $1.86$ & $(2,3)$ & -- & 6111\\
$(136,53)$ & 11 & $(14,5)$ & 6 & 2 & YES & YES & YES & $1.71$ & $(2,3)$ & NO & 6112\\
$(136,31)$ & 11 & $(57,13)$ & 9 & 1 & YES & YES & YES & $1.57$ & $(2,3)$ & NO & 6113\\
$(136,59)$ & 11 & $(99,43)$ & 11 & 1 & YES & YES & YES & $1.71$ & $(2,3)$ & NO & 6114\\
$(136,53)$ & 11 & $(100,39)$ & 10 & 4 & YES & YES & NO(2) & $1.75$ & $(2,3)$ & NO & 6115\\
$(136,59)$ & 11 & $(129,56)$ & 11 & 1 & YES & YES & YES & $1.71$ & $(2,3)$ & NO & 6116\\
$(136,61)$ & 12 & $(136,61)$ & 12 & 136 & YES & YES & YES & $1.71$ & $(2,3)$ & NO & 6117\\
$(137,37)$ & 11 & $(3,1)$ & 2 & 1 & YES & YES & YES & $1.57$ & $(2,3)$ & -- & 6118\\
$(137,37)$ & 11 & $(4,1)$ & 3 & 1 & YES & YES & NO(2) & $1.50$ & $(2,3)$ & NO & 6119\\
$(137,37)$ & 11 & $(4,1)$ & 3 & 1 & YES & YES & NO(2) & $1.50$ & $(2,3)$ & -- & 6120\\
$(137,37)$ & 11 & $(5,1)$ & 4 & 1 & YES & YES & YES & $1.57$ & $(2,3)$ & NO & 6121\\
$(137,37)$ & 11 & $(5,1)$ & 4 & 1 & YES & YES & YES & $1.57$ & $(2,3)$ & -- & 6122\\
$(137,52)$ & 11 & $(5,1)$ & 4 & 1 & YES & YES & YES & $1.43$ & $(2,3)$ & -- & 6123\\
$(137,30)$ & 12 & $(7,2)$ & 4 & 1 & YES & YES & YES & $1.29$ & $(2,3)$ & -- & 6124\\
$(137,30)$ & 12 & $(7,3)$ & 4 & 1 & YES & YES & YES & $1.57$ & $(2,3)$ & NO & 6125\\
$(137,37)$ & 11 & $(7,3)$ & 4 & 1 & YES & YES & YES & $1.71$ & $(2,3)$ & -- & 6126\\
$(137,37)$ & 11 & $(13,4)$ & 6 & 1 & YES & YES & YES & $1.43$ & $(2,3)$ & NO & 6127\\
$(137,32)$ & 12 & $(23,5)$ & 7 & 1 & YES & YES & YES & $1.71$ & $(2,3)$ & NO & 6128\\
$(137,37)$ & 11 & $(26,7)$ & 7 & 1 & YES & YES & YES & $1.57$ & $(2,3)$ & NO & 6129\\
$(137,53)$ & 11 & $(41,16)$ & 8 & 1 & YES & YES & YES & $1.57$ & $(2,3)$ & NO & 6130\\
$(137,30)$ & 12 & $(51,11)$ & 9 & 1 & YES & YES & YES & $1.57$ & $(2,3)$ & NO & 6131\\
$(137,37)$ & 11 & $(63,17)$ & 9 & 1 & YES & YES & YES & $1.57$ & $(2,3)$ & 6599 & 6132\\
$(137,37)$ & 11 & $(67,18)$ & 9 & 1 & YES & YES & YES & $1.71$ & $(2,3)$ & NO & 6133\\
$(137,53)$ & 11 & $(67,26)$ & 9 & 1 & YES & YES & YES & $1.71$ & $(2,3)$ & NO & 6134\\
$(137,52)$ & 11 & $(79,30)$ & 9 & 1 & YES & YES & YES & $1.71$ & $(2,3)$ & 7078 & 6135\\
$(137,37)$ & 11 & $(100,27)$ & 10 & 1 & YES & YES & YES & $1.57$ & $(2,3)$ & NO & 6136\\
$(137,53)$ & 11 & $(106,41)$ & 10 & 1 & YES & YES & NO(2) & $1.62$ & $(2,3)$ & NO & 6137\\
$(137,43)$ & 12 & $(121,38)$ & 12 & 1 & YES & YES & YES & $1.71$ & $(2,3)$ & NO & 6138\\
$(137,53)$ & 11 & $(137,53)$ & 11 & 137 & YES & YES & YES & $1.57$ & $(2,3)$ & NO & 6139\\
$(138,37)$ & 11 & $(7,2)$ & 4 & 1 & YES & YES & YES & $1.29$ & $(2,3)$ & -- & 6140\\
$(138,41)$ & 11 & $(13,3)$ & 6 & 1 & YES & YES & YES & $1.57$ & $(2,3)$ & NO & 6141\\
$(138,41)$ & 11 & $(23,7)$ & 7 & 23 & YES & YES & YES & $1.71$ & $(2,3)$ & NO & 6142\\
$(138,41)$ & 11 & $(25,7)$ & 7 & 1 & YES & YES & YES & $1.57$ & $(2,3)$ & NO & 6143\\
$(138,41)$ & 11 & $(31,9)$ & 8 & 1 & YES & YES & YES & $1.57$ & $(2,3)$ & NO & 6144\\
$(138,37)$ & 11 & $(67,18)$ & 9 & 1 & YES & YES & YES & $1.43$ & $(2,3)$ & NO & 6145\\
$(138,37)$ & 11 & $(127,34)$ & 11 & 1 & YES & YES & YES & $1.57$ & $(2,3)$ & NO & 6146\\
$(139,57)$ & 11 & $(3,1)$ & 2 & 1 & YES & YES & YES & $1.43$ & $(2,3)$ & NO & 6147\\
$(139,57)$ & 11 & $(3,1)$ & 2 & 1 & YES & YES & YES & $1.43$ & $(2,3)$ & -- & 6148\\
$(139,51)$ & 11 & $(4,1)$ & 3 & 1 & YES & YES & YES & $1.57$ & $(2,3)$ & -- & 6149\\
$(139,57)$ & 11 & $(4,1)$ & 3 & 1 & YES & YES & YES & $1.57$ & $(2,3)$ & NO & 6150\\
$(139,57)$ & 11 & $(4,1)$ & 3 & 1 & YES & YES & YES & $1.57$ & $(2,3)$ & -- & 6151\\
$(139,57)$ & 11 & $(4,1)$ & 3 & 1 & YES & YES & YES & $1.57$ & $(2,3)$ & NO & 6152\\
$(139,39)$ & 11 & $(5,2)$ & 3 & 1 & YES & YES & YES & $1.43$ & $(2,3)$ & -- & 6153\\
$(139,39)$ & 11 & $(7,3)$ & 4 & 1 & YES & YES & YES & $1.57$ & $(2,3)$ & -- & 6154\\
$(139,57)$ & 11 & $(7,2)$ & 4 & 1 & YES & YES & YES & $1.43$ & $(2,3)$ & NO & 6155\\
$(139,42)$ & 12 & $(8,3)$ & 4 & 1 & YES & YES & YES & $1.57$ & $(2,3)$ & -- & 6156\\
$(139,39)$ & 11 & $(11,3)$ & 5 & 1 & YES & YES & YES & $1.57$ & $(2,3)$ & -- & 6157\\
$(139,41)$ & 11 & $(13,3)$ & 6 & 1 & YES & YES & YES & $1.43$ & $(2,3)$ & NO & 6158\\
$(139,41)$ & 11 & $(13,3)$ & 6 & 1 & YES & YES & YES & $1.43$ & $(2,3)$ & -- & 6159\\
$(139,41)$ & 11 & $(15,4)$ & 6 & 1 & YES & YES & YES & $1.43$ & $(2,3)$ & NO & 6160\\
$(139,41)$ & 11 & $(16,5)$ & 7 & 1 & YES & YES & YES & $1.57$ & $(2,3)$ & NO & 6161\\
$(139,43)$ & 12 & $(17,5)$ & 6 & 1 & YES & YES & YES & $1.71$ & $(2,3)$ & NO & 6162\\
$(139,39)$ & 11 & $(26,7)$ & 7 & 1 & YES & YES & YES & $1.57$ & $(2,3)$ & NO & 6163\\
$(139,39)$ & 11 & $(29,8)$ & 7 & 1 & YES & YES & YES & $1.43$ & $(2,3)$ & NO & 6164\\
$(139,39)$ & 11 & $(61,17)$ & 9 & 1 & YES & YES & YES & $1.57$ & $(2,3)$ & NO & 6165\\
$(139,41)$ & 11 & $(61,18)$ & 9 & 1 & YES & YES & YES & $1.43$ & $(2,3)$ & NO & 6166\\
$(139,57)$ & 11 & $(61,25)$ & 9 & 1 & YES & YES & YES & $1.71$ & $(2,3)$ & 6567 & 6167\\
$(139,57)$ & 11 & $(73,30)$ & 10 & 1 & YES & YES & YES & $1.57$ & $(2,3)$ & NO & 6168\\
$(139,57)$ & 11 & $(83,34)$ & 10 & 1 & YES & YES & YES & $1.43$ & $(2,3)$ & 7904 & 6169\\
$(139,57)$ & 11 & $(100,41)$ & 10 & 1 & YES & YES & YES & $1.71$ & $(2,3)$ & NO & 6170\\
$(139,42)$ & 12 & $(109,33)$ & 11 & 1 & YES & YES & YES & $1.71$ & $(2,3)$ & NO & 6171\\
$(140,39)$ & 11 & $(2,1)$ & 1 & 2 & YES & YES & YES & $1.71$ & $(2,3)$ & -- & 6172\\
$(140,41)$ & 11 & $(3,1)$ & 2 & 1 & YES & YES & YES & $1.57$ & $(2,3)$ & NO & 6173\\
$(140,41)$ & 11 & $(3,1)$ & 2 & 1 & YES & YES & YES & $1.57$ & $(2,3)$ & -- & 6174\\
$(140,37)$ & 11 & $(5,2)$ & 3 & 5 & YES & YES & YES & $1.57$ & $(2,3)$ & NO & 6175\\
$(140,37)$ & 11 & $(5,2)$ & 3 & 5 & YES & YES & YES & $1.57$ & $(2,3)$ & -- & 6176\\
$(140,61)$ & 11 & $(5,2)$ & 3 & 5 & YES & YES & YES & $1.57$ & $(2,3)$ & NO & 6177\\
$(140,61)$ & 11 & $(5,2)$ & 3 & 5 & YES & YES & YES & $1.57$ & $(2,3)$ & -- & 6178\\
$(140,37)$ & 11 & $(7,3)$ & 4 & 7 & YES & YES & YES & $1.57$ & $(2,3)$ & NO & 6179\\
$(140,37)$ & 11 & $(7,3)$ & 4 & 7 & YES & YES & YES & $1.57$ & $(2,3)$ & -- & 6180\\
$(140,39)$ & 11 & $(7,3)$ & 4 & 7 & YES & YES & YES & $1.57$ & $(2,3)$ & -- & 6181\\
$(140,41)$ & 11 & $(7,2)$ & 4 & 7 & YES & YES & YES & $1.57$ & $(2,3)$ & NO & 6182\\
$(140,61)$ & 11 & $(9,2)$ & 5 & 1 & YES & YES & YES & $1.57$ & $(2,3)$ & -- & 6183\\
$(140,37)$ & 11 & $(10,3)$ & 5 & 10 & YES & YES & YES & $1.57$ & $(2,3)$ & NO & 6184\\
$(140,39)$ & 11 & $(11,3)$ & 5 & 1 & YES & YES & YES & $1.71$ & $(2,3)$ & NO & 6185\\
$(140,53)$ & 11 & $(11,2)$ & 6 & 1 & YES & YES & YES & $1.43$ & $(2,3)$ & NO & 6186\\
$(140,39)$ & 11 & $(26,7)$ & 7 & 2 & YES & YES & YES & $1.43$ & $(2,3)$ & NO & 6187\\
$(140,39)$ & 11 & $(61,17)$ & 9 & 1 & YES & YES & YES & $1.57$ & $(2,3)$ & NO & 6188\\
$(140,37)$ & 11 & $(121,32)$ & 11 & 1 & YES & YES & YES & $1.43$ & $(2,3)$ & NO & 6189\\
$(141,59)$ & 11 & $(2,1)$ & 1 & 1 & YES & YES & YES & $1.43$ & $(2,3)$ & -- & 6190\\
$(141,50)$ & 12 & $(3,1)$ & 2 & 3 & YES & YES & YES & $1.43$ & $(2,3)$ & -- & 6191\\
$(141,55)$ & 11 & $(3,1)$ & 2 & 3 & YES & YES & YES & $1.57$ & $(2,3)$ & NO & 6192\\
$(141,55)$ & 11 & $(3,1)$ & 2 & 3 & YES & YES & YES & $1.57$ & $(2,3)$ & -- & 6193\\
$(141,59)$ & 11 & $(3,1)$ & 2 & 3 & YES & YES & YES & $1.43$ & $(2,3)$ & -- & 6194\\
$(141,41)$ & 11 & $(4,1)$ & 3 & 1 & YES & YES & YES & $1.57$ & $(2,3)$ & NO & 6195\\
$(141,41)$ & 11 & $(4,1)$ & 3 & 1 & YES & YES & YES & $1.57$ & $(2,3)$ & -- & 6196\\
$(141,50)$ & 12 & $(5,1)$ & 4 & 1 & YES & YES & YES & $1.57$ & $(2,3)$ & -- & 6197\\
$(141,50)$ & 12 & $(5,2)$ & 3 & 1 & YES & YES & YES & $1.57$ & $(2,3)$ & NO & 6198\\
$(141,55)$ & 11 & $(5,2)$ & 3 & 1 & YES & YES & NO(2) & $1.62$ & $(2,3)$ & -- & 6199\\
$(141,59)$ & 11 & $(5,2)$ & 3 & 1 & YES & YES & YES & $1.71$ & $(2,3)$ & -- & 6200\\
$(141,50)$ & 12 & $(6,1)$ & 5 & 3 & YES & YES & YES & $1.57$ & $(2,3)$ & -- & 6201\\
$(141,41)$ & 11 & $(7,3)$ & 4 & 1 & YES & YES & YES & $1.43$ & $(2,3)$ & -- & 6202\\
$(141,55)$ & 11 & $(7,2)$ & 4 & 1 & YES & YES & YES & $1.57$ & $(2,3)$ & -- & 6203\\
$(141,55)$ & 11 & $(7,3)$ & 4 & 1 & YES & YES & YES & $1.57$ & $(2,3)$ & -- & 6204\\
$(141,59)$ & 11 & $(7,2)$ & 4 & 1 & YES & YES & YES & $1.43$ & $(2,3)$ & NO & 6205\\
$(141,41)$ & 11 & $(8,3)$ & 4 & 1 & YES & YES & YES & $1.57$ & $(2,3)$ & -- & 6206\\
$(141,41)$ & 11 & $(8,3)$ & 4 & 1 & YES & YES & YES & $1.57$ & $(2,3)$ & NO & 6207\\
$(141,41)$ & 11 & $(9,2)$ & 5 & 3 & YES & YES & YES & $1.57$ & $(2,3)$ & -- & 6208\\
$(141,50)$ & 12 & $(11,4)$ & 5 & 1 & YES & YES & YES & $1.57$ & $(2,3)$ & NO & 6209\\
$(141,55)$ & 11 & $(11,4)$ & 5 & 1 & YES & YES & NO(2) & $1.62$ & $(2,3)$ & NO & 6210\\
$(141,41)$ & 11 & $(15,4)$ & 6 & 3 & YES & YES & YES & $1.43$ & $(2,3)$ & NO & 6211\\
$(141,59)$ & 11 & $(26,11)$ & 7 & 1 & YES & YES & YES & $1.43$ & $(2,3)$ & NO & 6212\\
$(141,55)$ & 11 & $(31,12)$ & 7 & 1 & YES & YES & NO(2) & $1.62$ & $(2,3)$ & NO & 6213\\
$(141,59)$ & 11 & $(55,23)$ & 9 & 1 & YES & YES & YES & $1.43$ & $(2,3)$ & 6411 & 6214\\
$(141,41)$ & 11 & $(69,20)$ & 10 & 3 & YES & YES & YES & $1.43$ & $(2,3)$ & NO & 6215\\
$(141,32)$ & 13 & $(128,29)$ & 11 & 1 & YES & YES & YES & $1.57$ & $(2,3)$ & NO & 6216\\
$(141,50)$ & 12 & $(141,50)$ & 12 & 141 & YES & YES & YES & $1.43$ & $(2,3)$ & NO & 6217\\
$(141,59)$ & 11 & $(141,59)$ & 11 & 141 & YES & YES & YES & $1.43$ & $(2,3)$ & NO & 6218\\
$(142,55)$ & 11 & $(4,1)$ & 3 & 2 & YES & YES & NO(2) & $1.62$ & $(2,3)$ & -- & 6219\\
$(142,39)$ & 11 & $(5,2)$ & 3 & 1 & YES & YES & YES & $1.57$ & $(2,3)$ & -- & 6220\\
$(142,51)$ & 11 & $(5,2)$ & 3 & 1 & YES & YES & YES & $1.57$ & $(2,3)$ & -- & 6221\\
$(142,55)$ & 11 & $(5,2)$ & 3 & 1 & YES & YES & YES & $1.71$ & $(2,3)$ & -- & 6222\\
$(142,39)$ & 11 & $(7,3)$ & 4 & 1 & YES & YES & YES & $1.71$ & $(2,3)$ & -- & 6223\\
$(142,43)$ & 12 & $(7,3)$ & 4 & 1 & YES & YES & YES & $1.71$ & $(2,3)$ & -- & 6224\\
$(142,51)$ & 11 & $(7,2)$ & 4 & 1 & YES & YES & YES & $1.57$ & $(2,3)$ & NO & 6225\\
$(142,51)$ & 11 & $(7,2)$ & 4 & 1 & YES & YES & YES & $1.57$ & $(2,3)$ & -- & 6226\\
$(142,55)$ & 11 & $(7,2)$ & 4 & 1 & YES & YES & YES & $1.57$ & $(2,3)$ & NO & 6227\\
$(142,55)$ & 11 & $(7,3)$ & 4 & 1 & YES & YES & YES & $1.71$ & $(2,3)$ & -- & 6228\\
$(142,33)$ & 12 & $(11,4)$ & 5 & 1 & YES & YES & YES & $1.57$ & $(2,3)$ & -- & 6229\\
$(142,55)$ & 11 & $(11,2)$ & 6 & 1 & YES & YES & YES & $1.43$ & $(2,3)$ & NO & 6230\\
$(142,39)$ & 11 & $(13,4)$ & 6 & 1 & YES & YES & YES & $1.71$ & $(2,3)$ & NO & 6231\\
$(142,55)$ & 11 & $(67,26)$ & 9 & 1 & YES & YES & YES & $1.71$ & $(2,3)$ & NO & 6232\\
$(142,33)$ & 12 & $(77,18)$ & 10 & 1 & YES & YES & YES & $1.71$ & $(2,3)$ & NO & 6233\\
$(142,55)$ & 11 & $(80,31)$ & 9 & 2 & YES & YES & NO(2) & $1.62$ & $(2,3)$ & 7169 & 6234\\
$(143,63)$ & 11 & $(2,1)$ & 1 & 1 & YES & YES & YES & $1.29$ & $(2,3)$ & -- & 6235\\
$(143,59)$ & 11 & $(3,1)$ & 2 & 1 & YES & YES & NO(2) & $1.62$ & $(2,3)$ & NO & 6236\\
$(143,59)$ & 11 & $(3,1)$ & 2 & 1 & YES & YES & NO(2) & $1.62$ & $(2,3)$ & -- & 6237\\
$(143,59)$ & 11 & $(4,1)$ & 3 & 1 & YES & YES & NO(2) & $1.62$ & $(2,3)$ & -- & 6238\\
$(143,59)$ & 11 & $(4,1)$ & 3 & 1 & YES & YES & NO(2) & $1.62$ & $(2,3)$ & NO & 6239\\
$(143,59)$ & 11 & $(4,1)$ & 3 & 1 & YES & YES & YES & $1.71$ & $(2,3)$ & NO & 6240\\
$(143,59)$ & 11 & $(5,2)$ & 3 & 1 & YES & YES & YES & $1.57$ & $(2,3)$ & -- & 6241\\
$(143,59)$ & 11 & $(7,2)$ & 4 & 1 & YES & YES & YES & $1.71$ & $(2,3)$ & NO & 6242\\
$(143,59)$ & 11 & $(8,3)$ & 4 & 1 & YES & YES & NO(2) & $1.62$ & $(2,3)$ & NO & 6243\\
$(143,59)$ & 11 & $(22,9)$ & 7 & 11 & YES & YES & NO(2) & $1.62$ & $(2,3)$ & NO & 6244\\
$(143,59)$ & 11 & $(29,12)$ & 7 & 1 & YES & YES & NO(2) & $1.50$ & $(2,3)$ & 4960 & 6245\\
$(143,59)$ & 11 & $(41,17)$ & 8 & 1 & YES & YES & YES & $1.71$ & $(2,3)$ & NO & 6246\\
$(143,59)$ & 11 & $(46,19)$ & 8 & 1 & YES & YES & NO(2) & $1.50$ & $(2,3)$ & NO & 6247\\
$(143,59)$ & 11 & $(75,31)$ & 9 & 1 & YES & YES & YES & $1.71$ & $(2,3)$ & NO & 6248\\
$(143,59)$ & 11 & $(109,45)$ & 10 & 1 & YES & YES & YES & $1.43$ & $(2,3)$ & NO & 6249\\
$(143,63)$ & 11 & $(109,48)$ & 11 & 1 & YES & YES & YES & $1.43$ & $(2,3)$ & NO & 6250\\
$(143,59)$ & 11 & $(143,59)$ & 11 & 143 & YES & YES & NO(2) & $1.62$ & $(2,3)$ & NO & 6251\\
$(144,55)$ & 10 & $(2,1)$ & 1 & 2 & YES & YES & NO(2) & $1.50$ & $(2,3)$ & -- & 6252\\
$(144,65)$ & 12 & $(4,1)$ & 3 & 4 & YES & YES & YES & $1.71$ & $(2,3)$ & NO & 6253\\
$(144,65)$ & 12 & $(4,1)$ & 3 & 4 & YES & YES & YES & $1.71$ & $(2,3)$ & -- & 6254\\
$(144,55)$ & 10 & $(5,1)$ & 4 & 1 & YES & YES & NO(2) & $1.50$ & $(2,3)$ & -- & 6255\\
$(144,55)$ & 10 & $(5,2)$ & 3 & 1 & YES & YES & NO(2) & $1.50$ & $(2,3)$ & 5948 & 6256\\
$(144,55)$ & 10 & $(7,3)$ & 4 & 1 & YES & YES & YES & $1.43$ & $(2,3)$ & -- & 6257\\
$(144,55)$ & 10 & $(8,3)$ & 4 & 8 & YES & YES & NO(2) & $1.50$ & $(2,3)$ & NO & 6258\\
$(144,55)$ & 10 & $(11,4)$ & 5 & 1 & YES & YES & YES & $1.57$ & $(2,3)$ & NO & 6259\\
$(144,55)$ & 10 & $(29,11)$ & 7 & 1 & YES & YES & YES & $1.43$ & $(2,3)$ & NO & 6260\\
$(144,59)$ & 11 & $(29,12)$ & 7 & 1 & YES & YES & YES & $1.43$ & $(2,3)$ & NO & 6261\\
$(144,55)$ & 10 & $(31,12)$ & 7 & 1 & YES & YES & YES & $1.57$ & $(2,3)$ & NO & 6262\\
$(144,55)$ & 10 & $(34,13)$ & 7 & 2 & YES & YES & NO(2) & $1.38$ & $(2,3)$ & 5862 & 6263\\
$(144,55)$ & 10 & $(50,19)$ & 8 & 2 & YES & YES & YES & $1.71$ & $(2,3)$ & NO & 6264\\
$(144,65)$ & 12 & $(51,23)$ & 9 & 3 & YES & YES & YES & $1.71$ & $(2,3)$ & NO & 6265\\
$(144,55)$ & 10 & $(60,23)$ & 9 & 12 & YES & YES & YES & $1.57$ & $(2,3)$ & NO & 6266\\
$(144,65)$ & 12 & $(113,51)$ & 11 & 1 & YES & YES & YES & $1.71$ & $(2,3)$ & NO & 6267\\
$(145,56)$ & 11 & $(4,1)$ & 3 & 1 & YES & YES & NO(2) & $1.62$ & $(2,3)$ & -- & 6268\\
$(145,56)$ & 11 & $(4,1)$ & 3 & 1 & YES & YES & NO(2) & $1.62$ & $(2,3)$ & NO & 6269\\
$(145,43)$ & 12 & $(7,3)$ & 4 & 1 & YES & YES & YES & $1.71$ & $(2,3)$ & NO & 6270\\
$(145,53)$ & 11 & $(9,2)$ & 5 & 1 & YES & YES & YES & $1.43$ & $(2,3)$ & -- & 6271\\
$(145,56)$ & 11 & $(9,4)$ & 5 & 1 & YES & YES & YES & $1.71$ & $(2,3)$ & -- & 6272\\
$(145,33)$ & 13 & $(11,2)$ & 6 & 1 & YES & YES & YES & $1.57$ & $(2,3)$ & NO & 6273\\
$(145,56)$ & 11 & $(23,9)$ & 7 & 1 & YES & YES & NO(2) & $1.75$ & $(2,3)$ & NO & 6274\\
$(145,56)$ & 11 & $(31,12)$ & 7 & 1 & YES & YES & YES & $1.57$ & $(2,3)$ & NO & 6275\\
$(145,34)$ & 12 & $(32,7)$ & 8 & 1 & YES & YES & YES & $1.57$ & $(2,3)$ & NO & 6276\\
$(145,34)$ & 12 & $(40,9)$ & 9 & 5 & YES & YES & YES & $1.57$ & $(2,3)$ & NO & 6277\\
$(145,44)$ & 11 & $(135,41)$ & 11 & 5 & YES & YES & YES & $1.57$ & $(2,3)$ & NO & 6278\\
$(146,41)$ & 11 & $(2,1)$ & 1 & 2 & YES & YES & YES & $1.57$ & $(2,3)$ & NO & 6279\\
$(146,41)$ & 11 & $(3,1)$ & 2 & 1 & YES & YES & YES & $1.43$ & $(2,3)$ & -- & 6280\\
$(146,57)$ & 11 & $(5,2)$ & 3 & 1 & YES & YES & YES & $1.57$ & $(2,3)$ & -- & 6281\\
$(146,61)$ & 12 & $(7,1)$ & 6 & 1 & YES & YES & YES & $1.71$ & $(2,3)$ & NO & 6282\\
$(146,57)$ & 11 & $(11,4)$ & 5 & 1 & YES & YES & YES & $1.57$ & $(2,3)$ & NO & 6283\\
$(146,61)$ & 12 & $(43,18)$ & 8 & 1 & YES & YES & YES & $1.57$ & $(2,3)$ & 5287 & 6284\\
$(146,61)$ & 12 & $(55,23)$ & 9 & 1 & YES & YES & YES & $1.57$ & $(2,3)$ & NO & 6285\\
$(146,41)$ & 11 & $(57,16)$ & 9 & 1 & YES & YES & YES & $1.43$ & $(2,3)$ & NO & 6286\\
$(146,57)$ & 11 & $(87,34)$ & 10 & 1 & YES & YES & YES & $1.43$ & $(2,3)$ & 8066 & 6287\\
$(147,61)$ & 11 & $(2,1)$ & 1 & 1 & YES & YES & YES & $1.43$ & $(2,3)$ & -- & 6288\\
$(147,62)$ & 11 & $(2,1)$ & 1 & 1 & YES & YES & YES & $1.43$ & $(2,3)$ & -- & 6289\\
$(147,61)$ & 11 & $(3,1)$ & 2 & 3 & YES & YES & YES & $1.43$ & $(2,3)$ & -- & 6290\\
$(147,62)$ & 11 & $(3,1)$ & 2 & 3 & YES & YES & YES & $1.43$ & $(2,3)$ & NO & 6291\\
$(147,41)$ & 11 & $(4,1)$ & 3 & 1 & YES & YES & YES & $1.57$ & $(2,3)$ & NO & 6292\\
$(147,41)$ & 11 & $(4,1)$ & 3 & 1 & YES & YES & YES & $1.57$ & $(2,3)$ & -- & 6293\\
$(147,41)$ & 11 & $(4,1)$ & 3 & 1 & YES & YES & YES & $1.57$ & $(2,3)$ & NO & 6294\\
$(147,61)$ & 11 & $(4,1)$ & 3 & 1 & YES & YES & YES & $1.43$ & $(2,3)$ & -- & 6295\\
$(147,62)$ & 11 & $(4,1)$ & 3 & 1 & YES & YES & NO(2) & $1.57$ & $(4,2)$ & NO & 6296\\
$(147,62)$ & 11 & $(4,1)$ & 3 & 1 & YES & YES & NO(2) & $1.57$ & $(4,2)$ & -- & 6297\\
$(147,41)$ & 11 & $(5,2)$ & 3 & 1 & YES & YES & YES & $1.57$ & $(2,3)$ & -- & 6298\\
$(147,61)$ & 11 & $(5,2)$ & 3 & 1 & YES & YES & NO(2) & $1.50$ & $(2,3)$ & NO & 6299\\
$(147,64)$ & 11 & $(5,2)$ & 3 & 1 & YES & YES & YES & $1.57$ & $(2,3)$ & NO & 6300\\
$(147,64)$ & 11 & $(5,2)$ & 3 & 1 & YES & YES & YES & $1.57$ & $(2,3)$ & -- & 6301\\
$(147,43)$ & 11 & $(7,3)$ & 4 & 7 & YES & YES & YES & $1.43$ & $(2,3)$ & -- & 6302\\
$(147,61)$ & 11 & $(7,2)$ & 4 & 7 & YES & YES & YES & $1.71$ & $(2,3)$ & NO & 6303\\
$(147,61)$ & 11 & $(7,3)$ & 4 & 7 & YES & YES & YES & $1.57$ & $(2,3)$ & -- & 6304\\
$(147,61)$ & 11 & $(17,7)$ & 6 & 1 & YES & YES & NO(2) & $1.50$ & $(2,3)$ & NO & 6305\\
$(147,53)$ & 11 & $(21,8)$ & 6 & 21 & YES & YES & YES & $1.57$ & $(2,3)$ & NO & 6306\\
$(147,61)$ & 11 & $(22,9)$ & 7 & 1 & YES & YES & YES & $1.57$ & $(2,3)$ & NO & 6307\\
$(147,61)$ & 11 & $(29,12)$ & 7 & 1 & YES & YES & YES & $1.43$ & $(2,3)$ & 6964 & 6308\\
$(147,62)$ & 11 & $(45,19)$ & 8 & 3 & YES & YES & NO(2) & $1.57$ & $(4,2)$ & NO & 6309\\
$(147,61)$ & 11 & $(46,19)$ & 8 & 1 & YES & YES & YES & $1.57$ & $(2,3)$ & 7932 & 6310\\
$(147,61)$ & 11 & $(53,22)$ & 9 & 1 & YES & YES & YES & $1.43$ & $(2,3)$ & NO & 6311\\
$(147,61)$ & 11 & $(94,39)$ & 10 & 1 & YES & YES & YES & $1.43$ & $(2,3)$ & NO & 6312\\
$(147,64)$ & 11 & $(101,44)$ & 10 & 1 & YES & YES & YES & $1.57$ & $(2,3)$ & NO & 6313\\
$(147,61)$ & 11 & $(135,56)$ & 11 & 3 & YES & YES & YES & $1.57$ & $(2,3)$ & NO & 6314\\
$(147,61)$ & 11 & $(147,61)$ & 11 & 147 & YES & YES & YES & $1.43$ & $(2,3)$ & NO & 6315\\
$(148,41)$ & 11 & $(3,1)$ & 2 & 1 & YES & YES & YES & $1.57$ & $(2,3)$ & NO & 6316\\
$(148,41)$ & 11 & $(3,1)$ & 2 & 1 & YES & YES & YES & $1.57$ & $(2,3)$ & -- & 6317\\
$(148,65)$ & 11 & $(4,1)$ & 3 & 4 & YES & YES & YES & $1.43$ & $(2,3)$ & NO & 6318\\
$(148,41)$ & 11 & $(5,1)$ & 4 & 1 & YES & YES & YES & $1.57$ & $(2,3)$ & NO & 6319\\
$(148,41)$ & 11 & $(5,1)$ & 4 & 1 & YES & YES & YES & $1.57$ & $(2,3)$ & -- & 6320\\
$(148,65)$ & 11 & $(11,5)$ & 6 & 1 & YES & YES & YES & $1.57$ & $(2,3)$ & NO & 6321\\
$(148,41)$ & 11 & $(13,3)$ & 6 & 1 & YES & YES & YES & $1.57$ & $(2,3)$ & NO & 6322\\
$(148,41)$ & 11 & $(65,18)$ & 9 & 1 & YES & YES & YES & $1.57$ & $(2,3)$ & NO & 6323\\
$(148,65)$ & 11 & $(66,29)$ & 9 & 2 & YES & YES & YES & $1.57$ & $(2,3)$ & 6806 & 6324\\
$(148,65)$ & 11 & $(107,47)$ & 10 & 1 & YES & YES & YES & $1.57$ & $(2,3)$ & NO & 6325\\
$(148,65)$ & 11 & $(148,65)$ & 11 & 148 & YES & YES & YES & $1.57$ & $(2,3)$ & NO & 6326\\
$(149,41)$ & 11 & $(5,1)$ & 4 & 1 & YES & YES & YES & $1.71$ & $(2,3)$ & -- & 6327\\
$(149,55)$ & 11 & $(5,2)$ & 3 & 1 & YES & YES & YES & $1.57$ & $(2,3)$ & NO & 6328\\
$(149,55)$ & 11 & $(5,2)$ & 3 & 1 & YES & YES & YES & $1.57$ & $(2,3)$ & -- & 6329\\
$(149,40)$ & 11 & $(7,3)$ & 4 & 1 & YES & YES & YES & $1.71$ & $(2,3)$ & NO & 6330\\
$(149,41)$ & 11 & $(7,3)$ & 4 & 1 & YES & YES & YES & $1.57$ & $(2,3)$ & -- & 6331\\
$(149,41)$ & 11 & $(7,3)$ & 4 & 1 & YES & YES & YES & $1.71$ & $(2,3)$ & NO & 6332\\
$(149,55)$ & 11 & $(7,3)$ & 4 & 1 & YES & YES & YES & $1.57$ & $(2,3)$ & -- & 6333\\
$(149,40)$ & 11 & $(9,2)$ & 5 & 1 & YES & YES & YES & $1.43$ & $(2,3)$ & NO & 6334\\
$(149,45)$ & 12 & $(9,2)$ & 5 & 1 & YES & YES & YES & $1.43$ & $(2,3)$ & -- & 6335\\
$(149,41)$ & 11 & $(32,9)$ & 8 & 1 & YES & YES & YES & $1.57$ & $(2,3)$ & NO & 6336\\
$(149,65)$ & 11 & $(34,15)$ & 8 & 1 & YES & YES & YES & $1.57$ & $(2,3)$ & NO & 6337\\
$(149,55)$ & 11 & $(35,13)$ & 8 & 1 & YES & YES & YES & $1.57$ & $(2,3)$ & NO & 6338\\
$(149,55)$ & 11 & $(43,16)$ & 9 & 1 & YES & YES & YES & $1.57$ & $(2,3)$ & NO & 6339\\
$(149,40)$ & 11 & $(63,17)$ & 9 & 1 & YES & YES & YES & $1.71$ & $(2,3)$ & NO & 6340\\
$(149,40)$ & 11 & $(93,25)$ & 10 & 1 & YES & YES & YES & $1.43$ & $(2,3)$ & 8143 & 6341\\
$(150,47)$ & 13 & $(5,2)$ & 3 & 5 & YES & YES & YES & $1.71$ & $(2,3)$ & -- & 6342\\
$(151,56)$ & 11 & $(3,1)$ & 2 & 1 & YES & YES & NO(2) & $1.57$ & $(4,2)$ & -- & 6343\\
$(151,62)$ & 11 & $(3,1)$ & 2 & 1 & YES & YES & NO(2) & $1.43$ & $(4,2)$ & NO & 6344\\
$(151,62)$ & 11 & $(3,1)$ & 2 & 1 & YES & YES & YES & $1.43$ & $(2,3)$ & -- & 6345\\
$(151,62)$ & 11 & $(4,1)$ & 3 & 1 & YES & YES & YES & $1.57$ & $(2,3)$ & NO & 6346\\
$(151,62)$ & 11 & $(4,1)$ & 3 & 1 & YES & YES & YES & $1.57$ & $(2,3)$ & -- & 6347\\
$(151,62)$ & 11 & $(4,1)$ & 3 & 1 & YES & YES & YES & $1.57$ & $(2,3)$ & NO & 6348\\
$(151,34)$ & 12 & $(5,2)$ & 3 & 1 & YES & YES & YES & $1.43$ & $(2,3)$ & NO & 6349\\
$(151,56)$ & 11 & $(5,2)$ & 3 & 1 & YES & YES & NO(2) & $1.62$ & $(2,3)$ & -- & 6350\\
$(151,56)$ & 11 & $(5,2)$ & 3 & 1 & YES & YES & NO(2) & $1.62$ & $(2,3)$ & NO & 6351\\
$(151,59)$ & 11 & $(5,2)$ & 3 & 1 & YES & YES & YES & $1.57$ & $(2,3)$ & -- & 6352\\
$(151,62)$ & 11 & $(5,2)$ & 3 & 1 & YES & YES & YES & $1.57$ & $(2,3)$ & -- & 6353\\
$(151,28)$ & 13 & $(7,3)$ & 4 & 1 & YES & YES & YES & $1.71$ & $(2,3)$ & NO & 6354\\
$(151,28)$ & 13 & $(7,3)$ & 4 & 1 & YES & YES & YES & $1.71$ & $(2,3)$ & -- & 6355\\
$(151,56)$ & 11 & $(7,3)$ & 4 & 1 & YES & YES & YES & $1.71$ & $(2,3)$ & -- & 6356\\
$(151,62)$ & 11 & $(7,3)$ & 4 & 1 & YES & YES & YES & $1.71$ & $(2,3)$ & -- & 6357\\
$(151,62)$ & 11 & $(8,3)$ & 4 & 1 & YES & YES & YES & $1.57$ & $(2,3)$ & NO & 6358\\
$(151,56)$ & 11 & $(9,4)$ & 5 & 1 & YES & YES & YES & $1.57$ & $(2,3)$ & NO & 6359\\
$(151,62)$ & 11 & $(9,2)$ & 5 & 1 & YES & YES & YES & $1.43$ & $(2,3)$ & -- & 6360\\
$(151,32)$ & 12 & $(10,3)$ & 5 & 1 & YES & YES & YES & $1.71$ & $(2,3)$ & NO & 6361\\
$(151,62)$ & 11 & $(11,4)$ & 5 & 1 & YES & YES & YES & $1.57$ & $(2,3)$ & NO & 6362\\
$(151,40)$ & 12 & $(13,4)$ & 6 & 1 & YES & YES & YES & $1.57$ & $(2,3)$ & NO & 6363\\
$(151,62)$ & 11 & $(17,7)$ & 6 & 1 & YES & YES & NO(2) & $1.50$ & $(2,3)$ & NO & 6364\\
$(151,32)$ & 12 & $(22,5)$ & 7 & 1 & YES & YES & YES & $1.71$ & $(2,3)$ & NO & 6365\\
$(151,62)$ & 11 & $(22,9)$ & 7 & 1 & YES & YES & YES & $1.71$ & $(2,3)$ & 6564 & 6366\\
$(151,62)$ & 11 & $(29,12)$ & 7 & 1 & YES & YES & YES & $1.43$ & $(2,3)$ & NO & 6367\\
$(151,34)$ & 12 & $(35,8)$ & 8 & 1 & YES & YES & YES & $1.43$ & $(2,3)$ & NO & 6368\\
$(151,33)$ & 12 & $(37,8)$ & 8 & 1 & YES & YES & NO(3) & $1.29$ & $(2,3)$ & NO & 6369\\
$(151,56)$ & 11 & $(46,17)$ & 8 & 1 & YES & YES & NO(2) & $1.62$ & $(2,3)$ & NO & 6370\\
$(151,62)$ & 11 & $(73,30)$ & 10 & 1 & YES & YES & NO(2) & $1.75$ & $(2,3)$ & NO & 6371\\
$(151,62)$ & 11 & $(90,37)$ & 11 & 1 & YES & YES & YES & $1.57$ & $(2,3)$ & NO & 6372\\
$(151,44)$ & 13 & $(103,30)$ & 11 & 1 & YES & YES & YES & $1.71$ & $(2,3)$ & 7793 & 6373\\
$(151,62)$ & 11 & $(129,53)$ & 11 & 1 & YES & YES & YES & $1.57$ & $(2,3)$ & NO & 6374\\
$(151,44)$ & 13 & $(151,44)$ & 13 & 151 & YES & YES & YES & $1.71$ & $(2,3)$ & NO & 6375\\
$(151,62)$ & 11 & $(151,62)$ & 11 & 151 & YES & YES & YES & $1.43$ & $(2,3)$ & NO & 6376\\
$(152,41)$ & 11 & $(2,1)$ & 1 & 2 & YES & YES & YES & $1.57$ & $(2,3)$ & -- & 6377\\
$(152,67)$ & 11 & $(3,1)$ & 2 & 1 & YES & YES & YES & $1.43$ & $(2,3)$ & NO & 6378\\
$(152,67)$ & 11 & $(3,1)$ & 2 & 1 & YES & YES & YES & $1.43$ & $(2,3)$ & -- & 6379\\
$(152,41)$ & 11 & $(4,1)$ & 3 & 4 & YES & YES & YES & $1.43$ & $(2,3)$ & -- & 6380\\
$(152,41)$ & 11 & $(4,1)$ & 3 & 4 & YES & YES & YES & $1.57$ & $(2,3)$ & NO & 6381\\
$(152,47)$ & 12 & $(5,2)$ & 3 & 1 & YES & YES & YES & $1.57$ & $(2,3)$ & -- & 6382\\
$(152,55)$ & 12 & $(7,3)$ & 4 & 1 & YES & YES & YES & $1.71$ & $(2,3)$ & -- & 6383\\
$(152,59)$ & 11 & $(7,2)$ & 4 & 1 & YES & YES & YES & $1.71$ & $(2,3)$ & NO & 6384\\
$(152,59)$ & 11 & $(7,3)$ & 4 & 1 & YES & YES & NO(2) & $1.62$ & $(2,3)$ & NO & 6385\\
$(152,63)$ & 11 & $(7,2)$ & 4 & 1 & YES & YES & YES & $1.57$ & $(2,3)$ & -- & 6386\\
$(152,55)$ & 12 & $(8,3)$ & 4 & 8 & YES & YES & YES & $1.86$ & $(2,3)$ & -- & 6387\\
$(152,63)$ & 11 & $(9,2)$ & 5 & 1 & YES & YES & YES & $1.43$ & $(2,3)$ & -- & 6388\\
$(152,63)$ & 11 & $(9,2)$ & 5 & 1 & YES & YES & YES & $1.57$ & $(2,3)$ & NO & 6389\\
$(152,41)$ & 11 & $(11,3)$ & 5 & 1 & YES & YES & YES & $1.57$ & $(2,3)$ & 5133 & 6390\\
$(152,63)$ & 11 & $(12,5)$ & 5 & 4 & YES & YES & YES & $1.43$ & $(2,3)$ & NO & 6391\\
$(152,63)$ & 11 & $(22,9)$ & 7 & 2 & YES & YES & YES & $1.57$ & $(2,3)$ & NO & 6392\\
$(152,59)$ & 11 & $(28,11)$ & 8 & 4 & YES & YES & YES & $1.71$ & $(2,3)$ & NO & 6393\\
$(152,41)$ & 11 & $(37,10)$ & 8 & 1 & YES & YES & YES & $1.57$ & $(2,3)$ & NO & 6394\\
$(152,47)$ & 12 & $(45,14)$ & 9 & 1 & YES & YES & YES & $1.43$ & $(2,3)$ & 8033 & 6395\\
$(152,63)$ & 11 & $(94,39)$ & 10 & 2 & YES & YES & YES & $1.57$ & $(2,3)$ & NO & 6396\\
$(152,55)$ & 12 & $(119,43)$ & 11 & 1 & YES & YES & YES & $1.71$ & $(2,3)$ & NO & 6397\\
$(153,35)$ & 12 & $(4,1)$ & 3 & 1 & YES & YES & NO(2) & $1.62$ & $(2,3)$ & -- & 6398\\
$(153,59)$ & 12 & $(4,1)$ & 3 & 1 & YES & YES & YES & $1.57$ & $(2,3)$ & -- & 6399\\
$(153,56)$ & 11 & $(5,2)$ & 3 & 1 & YES & YES & YES & $1.71$ & $(2,3)$ & NO & 6400\\
$(153,56)$ & 11 & $(5,2)$ & 3 & 1 & YES & YES & YES & $1.71$ & $(2,3)$ & -- & 6401\\
$(153,58)$ & 12 & $(5,2)$ & 3 & 1 & YES & YES & YES & $1.71$ & $(2,3)$ & -- & 6402\\
$(153,64)$ & 11 & $(5,2)$ & 3 & 1 & YES & YES & NO(2) & $1.50$ & $(2,3)$ & NO & 6403\\
$(153,64)$ & 11 & $(5,2)$ & 3 & 1 & YES & YES & YES & $1.57$ & $(2,3)$ & -- & 6404\\
$(153,35)$ & 12 & $(7,3)$ & 4 & 1 & YES & YES & YES & $1.57$ & $(2,3)$ & -- & 6405\\
$(153,41)$ & 11 & $(7,2)$ & 4 & 1 & YES & YES & YES & $1.29$ & $(2,3)$ & -- & 6406\\
$(153,56)$ & 11 & $(7,2)$ & 4 & 1 & YES & YES & YES & $1.57$ & $(2,3)$ & -- & 6407\\
$(153,41)$ & 11 & $(13,4)$ & 6 & 1 & YES & YES & YES & $1.43$ & $(2,3)$ & NO & 6408\\
$(153,35)$ & 12 & $(16,3)$ & 7 & 1 & YES & YES & YES & $1.57$ & $(2,3)$ & NO & 6409\\
$(153,35)$ & 12 & $(40,9)$ & 9 & 1 & YES & YES & YES & $1.57$ & $(2,3)$ & NO & 6410\\
$(153,64)$ & 11 & $(43,18)$ & 8 & 1 & YES & YES & YES & $1.43$ & $(2,3)$ & 6214 & 6411\\
$(153,41)$ & 11 & $(67,18)$ & 9 & 1 & YES & YES & YES & $1.43$ & $(2,3)$ & NO & 6412\\
$(153,55)$ & 11 & $(89,32)$ & 10 & 1 & YES & YES & YES & $1.43$ & $(2,3)$ & NO & 6413\\
$(154,59)$ & 11 & $(3,1)$ & 2 & 1 & YES & YES & YES & $1.57$ & $(2,3)$ & NO & 6414\\
$(154,59)$ & 11 & $(3,1)$ & 2 & 1 & YES & YES & YES & $1.57$ & $(2,3)$ & -- & 6415\\
$(154,65)$ & 11 & $(3,1)$ & 2 & 1 & YES & YES & YES & $1.57$ & $(2,3)$ & NO & 6416\\
$(154,65)$ & 11 & $(3,1)$ & 2 & 1 & YES & YES & YES & $1.57$ & $(2,3)$ & -- & 6417\\
$(154,59)$ & 11 & $(4,1)$ & 3 & 2 & YES & YES & YES & $1.43$ & $(2,3)$ & NO & 6418\\
$(154,43)$ & 11 & $(5,2)$ & 3 & 1 & YES & YES & NO(3) & $1.29$ & $(2,3)$ & NO & 6419\\
$(154,59)$ & 11 & $(5,2)$ & 3 & 1 & YES & YES & YES & $1.57$ & $(2,3)$ & -- & 6420\\
$(154,59)$ & 11 & $(7,2)$ & 4 & 7 & YES & YES & YES & $1.57$ & $(2,3)$ & NO & 6421\\
$(154,43)$ & 11 & $(9,2)$ & 5 & 1 & YES & YES & YES & $1.57$ & $(2,3)$ & NO & 6422\\
$(154,45)$ & 11 & $(11,3)$ & 5 & 11 & YES & YES & YES & $1.43$ & $(2,3)$ & -- & 6423\\
$(154,43)$ & 11 & $(16,3)$ & 7 & 2 & YES & YES & YES & $1.57$ & $(2,3)$ & NO & 6424\\
$(154,59)$ & 11 & $(21,8)$ & 6 & 7 & YES & YES & YES & $1.57$ & $(2,3)$ & NO & 6425\\
$(154,45)$ & 11 & $(24,7)$ & 7 & 2 & YES & YES & YES & $1.57$ & $(2,3)$ & 5652 & 6426\\
$(154,65)$ & 11 & $(26,11)$ & 7 & 2 & YES & YES & YES & $1.57$ & $(2,3)$ & NO & 6427\\
$(154,43)$ & 11 & $(29,8)$ & 7 & 1 & YES & YES & YES & $1.43$ & $(2,3)$ & NO & 6428\\
$(154,59)$ & 11 & $(29,11)$ & 7 & 1 & YES & YES & YES & $1.43$ & $(2,3)$ & NO & 6429\\
$(154,45)$ & 11 & $(38,11)$ & 9 & 2 & YES & YES & YES & $1.43$ & $(2,3)$ & NO & 6430\\
$(154,45)$ & 11 & $(44,13)$ & 8 & 22 & YES & YES & YES & $1.57$ & $(2,3)$ & NO & 6431\\
$(154,59)$ & 11 & $(60,23)$ & 9 & 2 & YES & YES & YES & $1.43$ & $(2,3)$ & 6673 & 6432\\
$(154,57)$ & 12 & $(73,27)$ & 9 & 1 & YES & YES & YES & $1.57$ & $(2,3)$ & NO & 6433\\
$(154,59)$ & 11 & $(73,28)$ & 10 & 1 & YES & YES & YES & $1.43$ & $(2,3)$ & 8182 & 6434\\
$(154,45)$ & 11 & $(75,22)$ & 10 & 1 & YES & YES & YES & $1.43$ & $(2,3)$ & 6060 & 6435\\
$(154,59)$ & 11 & $(107,41)$ & 10 & 1 & YES & YES & YES & $1.43$ & $(2,3)$ & NO & 6436\\
$(154,59)$ & 11 & $(154,59)$ & 11 & 154 & YES & YES & YES & $1.43$ & $(2,3)$ & NO & 6437\\
$(155,47)$ & 12 & $(3,1)$ & 2 & 1 & YES & YES & NO(2) & $1.50$ & $(2,3)$ & -- & 6438\\
$(155,64)$ & 11 & $(3,1)$ & 2 & 1 & YES & YES & YES & $1.43$ & $(2,3)$ & -- & 6439\\
$(155,68)$ & 11 & $(3,1)$ & 2 & 1 & YES & YES & YES & $1.43$ & $(2,3)$ & -- & 6440\\
$(155,56)$ & 12 & $(4,1)$ & 3 & 1 & YES & YES & NO(2) & $1.75$ & $(2,3)$ & -- & 6441\\
$(155,47)$ & 12 & $(5,1)$ & 4 & 5 & YES & YES & YES & $1.71$ & $(2,3)$ & -- & 6442\\
$(155,48)$ & 12 & $(5,2)$ & 3 & 5 & YES & YES & YES & $1.57$ & $(2,3)$ & -- & 6443\\
$(155,56)$ & 12 & $(5,1)$ & 4 & 5 & YES & YES & NO(2) & $1.75$ & $(2,3)$ & -- & 6444\\
$(155,57)$ & 11 & $(5,2)$ & 3 & 5 & YES & YES & YES & $1.57$ & $(2,3)$ & NO & 6445\\
$(155,57)$ & 11 & $(5,2)$ & 3 & 5 & YES & YES & NO(2) & $1.50$ & $(2,3)$ & -- & 6446\\
$(155,68)$ & 11 & $(5,2)$ & 3 & 5 & YES & YES & YES & $1.57$ & $(2,3)$ & NO & 6447\\
$(155,47)$ & 12 & $(7,2)$ & 4 & 1 & YES & YES & NO(2) & $1.50$ & $(2,3)$ & NO & 6448\\
$(155,47)$ & 12 & $(7,3)$ & 4 & 1 & YES & YES & YES & $1.71$ & $(2,3)$ & NO & 6449\\
$(155,48)$ & 12 & $(7,2)$ & 4 & 1 & YES & YES & YES & $1.57$ & $(2,3)$ & -- & 6450\\
$(155,64)$ & 11 & $(7,3)$ & 4 & 1 & YES & YES & YES & $1.71$ & $(2,3)$ & -- & 6451\\
$(155,68)$ & 11 & $(7,2)$ & 4 & 1 & YES & YES & YES & $1.57$ & $(2,3)$ & -- & 6452\\
$(155,64)$ & 11 & $(8,3)$ & 4 & 1 & YES & YES & YES & $1.71$ & $(2,3)$ & NO & 6453\\
$(155,64)$ & 11 & $(8,3)$ & 4 & 1 & YES & YES & YES & $1.71$ & $(2,3)$ & -- & 6454\\
$(155,64)$ & 11 & $(9,2)$ & 5 & 1 & YES & YES & YES & $1.57$ & $(2,3)$ & -- & 6455\\
$(155,46)$ & 11 & $(13,3)$ & 6 & 1 & YES & YES & YES & $1.57$ & $(2,3)$ & NO & 6456\\
$(155,47)$ & 12 & $(13,2)$ & 7 & 1 & YES & YES & YES & $1.57$ & $(2,3)$ & NO & 6457\\
$(155,57)$ & 11 & $(14,5)$ & 6 & 1 & YES & YES & NO(2) & $1.62$ & $(2,3)$ & NO & 6458\\
$(155,64)$ & 11 & $(17,7)$ & 6 & 1 & YES & YES & YES & $1.57$ & $(2,3)$ & NO & 6459\\
$(155,64)$ & 11 & $(22,9)$ & 7 & 1 & YES & YES & NO(2) & $1.75$ & $(2,3)$ & NO & 6460\\
$(155,68)$ & 11 & $(25,11)$ & 7 & 5 & YES & YES & YES & $1.43$ & $(2,3)$ & 6799 & 6461\\
$(155,46)$ & 11 & $(31,9)$ & 8 & 31 & YES & YES & YES & $1.57$ & $(2,3)$ & NO & 6462\\
$(155,68)$ & 11 & $(34,15)$ & 8 & 1 & YES & YES & YES & $1.57$ & $(2,3)$ & 8087 & 6463\\
$(155,57)$ & 11 & $(35,13)$ & 8 & 5 & YES & YES & YES & $1.57$ & $(2,3)$ & NO & 6464\\
$(155,34)$ & 12 & $(37,8)$ & 8 & 1 & YES & YES & YES & $1.29$ & $(2,3)$ & NO & 6465\\
$(155,34)$ & 12 & $(51,11)$ & 9 & 1 & YES & YES & YES & $1.57$ & $(2,3)$ & NO & 6466\\
$(155,64)$ & 11 & $(63,26)$ & 9 & 1 & YES & YES & YES & $1.43$ & $(2,3)$ & 6775 & 6467\\
$(155,41)$ & 12 & $(72,19)$ & 10 & 1 & YES & YES & YES & $1.43$ & $(2,3)$ & NO & 6468\\
$(155,56)$ & 12 & $(86,31)$ & 10 & 1 & YES & YES & YES & $1.86$ & $(2,3)$ & NO & 6469\\
$(155,68)$ & 11 & $(98,43)$ & 10 & 1 & YES & YES & YES & $1.29$ & $(2,3)$ & NO & 6470\\
$(155,56)$ & 12 & $(105,38)$ & 11 & 5 & YES & YES & YES & $1.71$ & $(2,3)$ & 8985 & 6471\\
$(155,64)$ & 11 & $(109,45)$ & 10 & 1 & YES & YES & YES & $1.43$ & $(2,3)$ & NO & 6472\\
$(155,56)$ & 12 & $(130,47)$ & 11 & 5 & YES & YES & YES & $1.71$ & $(2,3)$ & 8376 & 6473\\
$(155,68)$ & 11 & $(155,68)$ & 11 & 155 & YES & YES & YES & $1.43$ & $(2,3)$ & NO & 6474\\
$(156,37)$ & 12 & $(5,2)$ & 3 & 1 & YES & YES & YES & $1.57$ & $(2,3)$ & NO & 6475\\
$(156,37)$ & 12 & $(10,3)$ & 5 & 2 & YES & YES & YES & $1.57$ & $(2,3)$ & NO & 6476\\
$(156,59)$ & 12 & $(66,25)$ & 9 & 6 & YES & YES & YES & $1.71$ & $(2,3)$ & NO & 6477\\
$(156,59)$ & 12 & $(156,59)$ & 12 & 156 & YES & YES & YES & $1.57$ & $(2,3)$ & NO & 6478\\
$(157,46)$ & 11 & $(3,1)$ & 2 & 1 & YES & YES & YES & $1.57$ & $(2,3)$ & NO & 6479\\
$(157,46)$ & 11 & $(3,1)$ & 2 & 1 & YES & YES & YES & $1.57$ & $(2,3)$ & -- & 6480\\
$(157,69)$ & 11 & $(3,1)$ & 2 & 1 & YES & YES & YES & $1.57$ & $(2,3)$ & -- & 6481\\
$(157,69)$ & 11 & $(3,1)$ & 2 & 1 & YES & YES & YES & $1.57$ & $(2,3)$ & NO & 6482\\
$(157,69)$ & 11 & $(4,1)$ & 3 & 1 & YES & YES & YES & $1.43$ & $(2,3)$ & NO & 6483\\
$(157,69)$ & 11 & $(4,1)$ & 3 & 1 & YES & YES & YES & $1.57$ & $(2,3)$ & -- & 6484\\
$(157,58)$ & 11 & $(5,2)$ & 3 & 1 & YES & YES & NO(2) & $1.62$ & $(2,3)$ & -- & 6485\\
$(157,58)$ & 11 & $(5,2)$ & 3 & 1 & YES & YES & YES & $1.57$ & $(2,3)$ & NO & 6486\\
$(157,60)$ & 11 & $(5,1)$ & 4 & 1 & YES & YES & YES & $1.57$ & $(2,3)$ & NO & 6487\\
$(157,60)$ & 11 & $(5,2)$ & 3 & 1 & YES & YES & YES & $1.57$ & $(2,3)$ & -- & 6488\\
$(157,66)$ & 11 & $(5,1)$ & 4 & 1 & YES & YES & YES & $1.57$ & $(2,3)$ & -- & 6489\\
$(157,66)$ & 11 & $(5,2)$ & 3 & 1 & YES & YES & NO(2) & $1.62$ & $(2,3)$ & NO & 6490\\
$(157,58)$ & 11 & $(7,2)$ & 4 & 1 & YES & YES & YES & $1.57$ & $(2,3)$ & -- & 6491\\
$(157,58)$ & 11 & $(8,3)$ & 4 & 1 & YES & YES & YES & $1.57$ & $(2,3)$ & NO & 6492\\
$(157,69)$ & 11 & $(11,5)$ & 6 & 1 & YES & YES & YES & $1.57$ & $(2,3)$ & NO & 6493\\
$(157,46)$ & 11 & $(17,5)$ & 6 & 1 & YES & YES & YES & $1.57$ & $(2,3)$ & NO & 6494\\
$(157,28)$ & 13 & $(20,3)$ & 8 & 1 & YES & YES & YES & $1.57$ & $(2,3)$ & NO & 6495\\
$(157,46)$ & 11 & $(23,7)$ & 7 & 1 & YES & YES & YES & $1.57$ & $(2,3)$ & NO & 6496\\
$(157,60)$ & 11 & $(29,11)$ & 7 & 1 & YES & YES & YES & $1.57$ & $(2,3)$ & NO & 6497\\
$(157,46)$ & 11 & $(37,11)$ & 8 & 1 & YES & YES & YES & $1.43$ & $(2,3)$ & NO & 6498\\
$(157,66)$ & 11 & $(69,29)$ & 9 & 1 & YES & YES & YES & $1.57$ & $(2,3)$ & NO & 6499\\
$(157,58)$ & 11 & $(84,31)$ & 10 & 1 & YES & YES & YES & $1.71$ & $(2,3)$ & 8306 & 6500\\
$(157,60)$ & 11 & $(89,34)$ & 9 & 1 & YES & YES & YES & $1.57$ & $(2,3)$ & 7574 & 6501\\
$(157,69)$ & 11 & $(91,40)$ & 10 & 1 & YES & YES & YES & $1.57$ & $(2,3)$ & NO & 6502\\
$(157,46)$ & 11 & $(106,31)$ & 10 & 1 & YES & YES & YES & $1.57$ & $(2,3)$ & NO & 6503\\
$(157,60)$ & 11 & $(123,47)$ & 10 & 1 & YES & YES & YES & $1.57$ & $(2,3)$ & NO & 6504\\
$(157,46)$ & 11 & $(133,39)$ & 11 & 1 & YES & YES & YES & $1.57$ & $(2,3)$ & NO & 6505\\
$(157,44)$ & 13 & $(157,44)$ & 13 & 157 & YES & YES & YES & $1.57$ & $(2,3)$ & NO & 6506\\
$(157,48)$ & 12 & $(157,48)$ & 12 & 157 & YES & YES & YES & $1.57$ & $(2,3)$ & NO & 6507\\
$(157,69)$ & 11 & $(157,69)$ & 11 & 157 & YES & YES & YES & $1.57$ & $(2,3)$ & NO & 6508\\
$(158,57)$ & 11 & $(2,1)$ & 1 & 2 & YES & YES & NO(2) & $1.62$ & $(2,3)$ & -- & 6509\\
$(158,57)$ & 11 & $(2,1)$ & 1 & 2 & YES & YES & NO(2) & $1.75$ & $(2,3)$ & NO & 6510\\
$(158,57)$ & 11 & $(3,1)$ & 2 & 1 & YES & YES & NO(2) & $1.62$ & $(2,3)$ & -- & 6511\\
$(158,61)$ & 11 & $(3,1)$ & 2 & 1 & YES & YES & YES & $1.43$ & $(2,3)$ & -- & 6512\\
$(158,61)$ & 11 & $(4,1)$ & 3 & 2 & YES & YES & NO(2) & $1.50$ & $(2,3)$ & -- & 6513\\
$(158,61)$ & 11 & $(5,2)$ & 3 & 1 & YES & YES & YES & $1.57$ & $(2,3)$ & -- & 6514\\
$(158,37)$ & 12 & $(7,3)$ & 4 & 1 & YES & YES & YES & $1.57$ & $(2,3)$ & NO & 6515\\
$(158,61)$ & 11 & $(7,2)$ & 4 & 1 & YES & YES & YES & $1.57$ & $(2,3)$ & NO & 6516\\
$(158,61)$ & 11 & $(7,3)$ & 4 & 1 & YES & YES & NO(2) & $1.62$ & $(2,3)$ & NO & 6517\\
$(158,61)$ & 11 & $(7,3)$ & 4 & 1 & YES & YES & YES & $1.71$ & $(2,3)$ & -- & 6518\\
$(158,61)$ & 11 & $(9,2)$ & 5 & 1 & YES & YES & YES & $1.57$ & $(2,3)$ & NO & 6519\\
$(158,61)$ & 11 & $(12,5)$ & 5 & 2 & YES & YES & YES & $1.57$ & $(2,3)$ & NO & 6520\\
$(158,61)$ & 11 & $(18,7)$ & 6 & 2 & YES & YES & NO(2) & $1.50$ & $(2,3)$ & NO & 6521\\
$(158,61)$ & 11 & $(23,9)$ & 7 & 1 & YES & YES & YES & $1.57$ & $(2,3)$ & NO & 6522\\
$(158,47)$ & 12 & $(27,8)$ & 7 & 1 & YES & YES & NO(2) & $1.62$ & $(2,3)$ & NO & 6523\\
$(158,37)$ & 12 & $(35,8)$ & 8 & 1 & YES & YES & YES & $1.57$ & $(2,3)$ & NO & 6524\\
$(158,61)$ & 11 & $(49,19)$ & 8 & 1 & YES & YES & YES & $1.71$ & $(2,3)$ & 8161 & 6525\\
$(158,61)$ & 11 & $(57,22)$ & 9 & 1 & YES & YES & YES & $1.43$ & $(2,3)$ & NO & 6526\\
$(158,57)$ & 11 & $(61,22)$ & 9 & 1 & YES & YES & NO(2) & $1.75$ & $(2,3)$ & NO & 6527\\
$(158,61)$ & 11 & $(75,29)$ & 9 & 1 & YES & YES & YES & $1.57$ & $(2,3)$ & NO & 6528\\
$(158,57)$ & 11 & $(86,31)$ & 10 & 2 & YES & YES & YES & $1.71$ & $(2,3)$ & NO & 6529\\
$(158,61)$ & 11 & $(158,61)$ & 11 & 158 & YES & YES & YES & $1.43$ & $(2,3)$ & NO & 6530\\
$(159,47)$ & 11 & $(3,1)$ & 2 & 3 & YES & YES & YES & $1.43$ & $(2,3)$ & -- & 6531\\
$(159,62)$ & 11 & $(5,2)$ & 3 & 1 & YES & YES & YES & $1.43$ & $(2,3)$ & NO & 6532\\
$(159,61)$ & 12 & $(6,1)$ & 5 & 3 & YES & YES & YES & $1.71$ & $(2,3)$ & NO & 6533\\
$(159,61)$ & 12 & $(7,1)$ & 6 & 1 & YES & YES & YES & $1.71$ & $(2,3)$ & NO & 6534\\
$(159,37)$ & 12 & $(10,3)$ & 5 & 1 & YES & YES & YES & $1.57$ & $(2,3)$ & -- & 6535\\
$(159,44)$ & 11 & $(11,3)$ & 5 & 1 & YES & YES & YES & $1.57$ & $(2,3)$ & NO & 6536\\
$(159,62)$ & 11 & $(11,4)$ & 5 & 1 & YES & YES & YES & $1.43$ & $(2,3)$ & NO & 6537\\
$(159,37)$ & 12 & $(19,4)$ & 7 & 1 & YES & YES & YES & $1.43$ & $(2,3)$ & NO & 6538\\
$(159,62)$ & 11 & $(23,9)$ & 7 & 1 & YES & YES & YES & $1.43$ & $(2,3)$ & 6718 & 6539\\
$(159,46)$ & 13 & $(31,9)$ & 8 & 1 & YES & YES & YES & $1.57$ & $(2,3)$ & NO & 6540\\
$(159,61)$ & 12 & $(47,18)$ & 8 & 1 & YES & YES & YES & $1.57$ & $(2,3)$ & 5492 & 6541\\
$(159,61)$ & 12 & $(60,23)$ & 9 & 3 & YES & YES & YES & $1.71$ & $(2,3)$ & NO & 6542\\
$(159,59)$ & 11 & $(62,23)$ & 9 & 1 & YES & YES & YES & $1.43$ & $(2,3)$ & NO & 6543\\
$(159,37)$ & 12 & $(69,16)$ & 11 & 3 & YES & YES & NO(2) & $1.50$ & $(2,3)$ & NO & 6544\\
$(159,37)$ & 12 & $(77,18)$ & 10 & 1 & YES & YES & YES & $1.57$ & $(2,3)$ & NO & 6545\\
$(159,62)$ & 11 & $(77,30)$ & 10 & 1 & YES & YES & NO(2) & $1.75$ & $(2,3)$ & NO & 6546\\
$(159,46)$ & 13 & $(159,46)$ & 13 & 159 & YES & YES & YES & $1.71$ & $(2,3)$ & NO & 6547\\
$(160,49)$ & 12 & $(5,2)$ & 3 & 5 & YES & YES & YES & $1.57$ & $(2,3)$ & -- & 6548\\
$(160,67)$ & 11 & $(5,2)$ & 3 & 5 & YES & YES & YES & $1.71$ & $(2,3)$ & -- & 6549\\
$(160,67)$ & 11 & $(6,1)$ & 5 & 2 & YES & YES & NO(2) & $1.38$ & $(2,3)$ & NO & 6550\\
$(160,49)$ & 12 & $(33,10)$ & 8 & 1 & YES & YES & YES & $1.43$ & $(2,3)$ & NO & 6551\\
$(160,67)$ & 11 & $(43,18)$ & 8 & 1 & YES & YES & NO(2) & $1.50$ & $(2,3)$ & NO & 6552\\
$(160,43)$ & 11 & $(56,15)$ & 9 & 8 & YES & YES & YES & $1.29$ & $(2,3)$ & NO & 6553\\
$(161,66)$ & 11 & $(2,1)$ & 1 & 1 & YES & YES & YES & $1.71$ & $(2,3)$ & -- & 6554\\
$(161,68)$ & 11 & $(2,1)$ & 1 & 1 & YES & YES & NO(2) & $1.62$ & $(2,3)$ & -- & 6555\\
$(161,61)$ & 11 & $(3,1)$ & 2 & 1 & YES & YES & YES & $1.57$ & $(2,3)$ & -- & 6556\\
$(161,66)$ & 11 & $(3,1)$ & 2 & 1 & YES & YES & YES & $1.71$ & $(2,3)$ & -- & 6557\\
$(161,68)$ & 11 & $(4,1)$ & 3 & 1 & YES & YES & YES & $1.29$ & $(2,3)$ & -- & 6558\\
$(161,61)$ & 11 & $(5,2)$ & 3 & 1 & YES & YES & YES & $1.57$ & $(2,3)$ & -- & 6559\\
$(161,66)$ & 11 & $(5,2)$ & 3 & 1 & YES & YES & YES & $1.57$ & $(2,3)$ & -- & 6560\\
$(161,68)$ & 11 & $(5,1)$ & 4 & 1 & YES & YES & YES & $1.43$ & $(2,3)$ & NO & 6561\\
$(161,68)$ & 11 & $(5,1)$ & 4 & 1 & YES & YES & YES & $1.43$ & $(2,3)$ & NO & 6562\\
$(161,61)$ & 11 & $(11,4)$ & 5 & 1 & YES & YES & NO(2) & $1.50$ & $(2,3)$ & NO & 6563\\
$(161,66)$ & 11 & $(17,7)$ & 6 & 1 & YES & YES & YES & $1.71$ & $(2,3)$ & 6366 & 6564\\
$(161,61)$ & 11 & $(18,7)$ & 6 & 1 & YES & YES & YES & $1.71$ & $(2,3)$ & NO & 6565\\
$(161,66)$ & 11 & $(22,9)$ & 7 & 1 & YES & YES & NO(2) & $1.62$ & $(2,3)$ & NO & 6566\\
$(161,66)$ & 11 & $(39,16)$ & 8 & 1 & YES & YES & YES & $1.71$ & $(2,3)$ & 6167 & 6567\\
$(161,68)$ & 11 & $(45,19)$ & 8 & 1 & YES & YES & NO(2) & $1.62$ & $(2,3)$ & NO & 6568\\
$(161,68)$ & 11 & $(116,49)$ & 10 & 1 & YES & YES & YES & $1.29$ & $(2,3)$ & NO & 6569\\
$(161,68)$ & 11 & $(161,68)$ & 11 & 161 & YES & YES & YES & $1.43$ & $(2,3)$ & NO & 6570\\
$(162,73)$ & 12 & $(4,1)$ & 3 & 2 & YES & YES & YES & $1.57$ & $(2,3)$ & 3723 & 6571\\
$(162,37)$ & 12 & $(23,5)$ & 7 & 1 & YES & YES & YES & $1.57$ & $(2,3)$ & NO & 6572\\
$(162,37)$ & 12 & $(40,9)$ & 9 & 2 & YES & YES & YES & $1.57$ & $(2,3)$ & NO & 6573\\
$(162,43)$ & 12 & $(83,22)$ & 10 & 1 & YES & YES & YES & $1.43$ & $(2,3)$ & NO & 6574\\
$(162,73)$ & 12 & $(91,41)$ & 11 & 1 & YES & YES & YES & $1.57$ & $(2,3)$ & NO & 6575\\
$(162,73)$ & 12 & $(162,73)$ & 12 & 162 & YES & YES & YES & $1.43$ & $(2,3)$ & NO & 6576\\
$(163,62)$ & 11 & $(2,1)$ & 1 & 1 & YES & YES & YES & $1.57$ & $(2,3)$ & -- & 6577\\
$(163,44)$ & 11 & $(3,1)$ & 2 & 1 & YES & YES & YES & $1.43$ & $(2,3)$ & NO & 6578\\
$(163,44)$ & 11 & $(3,1)$ & 2 & 1 & YES & YES & YES & $1.43$ & $(2,3)$ & -- & 6579\\
$(163,67)$ & 12 & $(3,1)$ & 2 & 1 & YES & YES & NO(2) & $1.62$ & $(2,3)$ & NO & 6580\\
$(163,67)$ & 12 & $(3,1)$ & 2 & 1 & YES & YES & NO(2) & $1.62$ & $(2,3)$ & -- & 6581\\
$(163,71)$ & 11 & $(3,1)$ & 2 & 1 & YES & YES & YES & $1.57$ & $(2,3)$ & NO & 6582\\
$(163,71)$ & 11 & $(3,1)$ & 2 & 1 & YES & YES & YES & $1.57$ & $(2,3)$ & -- & 6583\\
$(163,62)$ & 11 & $(4,1)$ & 3 & 1 & YES & YES & YES & $1.71$ & $(2,3)$ & -- & 6584\\
$(163,67)$ & 12 & $(4,1)$ & 3 & 1 & YES & YES & NO(2) & $1.75$ & $(2,3)$ & -- & 6585\\
$(163,67)$ & 12 & $(4,1)$ & 3 & 1 & YES & YES & NO(2) & $1.62$ & $(2,3)$ & NO & 6586\\
$(163,71)$ & 11 & $(4,1)$ & 3 & 1 & YES & YES & YES & $1.57$ & $(2,3)$ & -- & 6587\\
$(163,44)$ & 11 & $(5,2)$ & 3 & 1 & YES & YES & YES & $1.57$ & $(2,3)$ & -- & 6588\\
$(163,62)$ & 11 & $(5,2)$ & 3 & 1 & YES & YES & YES & $1.71$ & $(2,3)$ & NO & 6589\\
$(163,62)$ & 11 & $(5,2)$ & 3 & 1 & YES & YES & YES & $1.71$ & $(2,3)$ & -- & 6590\\
$(163,67)$ & 12 & $(5,2)$ & 3 & 1 & YES & YES & YES & $1.71$ & $(2,3)$ & -- & 6591\\
$(163,71)$ & 11 & $(5,2)$ & 3 & 1 & YES & YES & YES & $1.43$ & $(2,3)$ & -- & 6592\\
$(163,67)$ & 12 & $(7,3)$ & 4 & 1 & YES & YES & NO(2) & $1.75$ & $(2,3)$ & NO & 6593\\
$(163,63)$ & 11 & $(8,3)$ & 4 & 1 & YES & YES & YES & $1.43$ & $(2,3)$ & NO & 6594\\
$(163,44)$ & 11 & $(10,3)$ & 5 & 1 & YES & YES & YES & $1.57$ & $(2,3)$ & NO & 6595\\
$(163,67)$ & 12 & $(11,4)$ & 5 & 1 & YES & YES & YES & $1.71$ & $(2,3)$ & NO & 6596\\
$(163,63)$ & 11 & $(13,5)$ & 5 & 1 & YES & YES & YES & $1.43$ & $(2,3)$ & NO & 6597\\
$(163,39)$ & 13 & $(14,3)$ & 6 & 1 & YES & YES & YES & $1.57$ & $(2,3)$ & NO & 6598\\
$(163,44)$ & 11 & $(37,10)$ & 8 & 1 & YES & YES & YES & $1.57$ & $(2,3)$ & 6132 & 6599\\
$(163,67)$ & 12 & $(56,23)$ & 9 & 1 & YES & YES & NO(2) & $1.75$ & $(2,3)$ & NO & 6600\\
$(163,62)$ & 11 & $(71,27)$ & 9 & 1 & YES & YES & YES & $1.57$ & $(2,3)$ & NO & 6601\\
$(163,71)$ & 11 & $(85,37)$ & 10 & 1 & YES & YES & YES & $1.43$ & $(2,3)$ & NO & 6602\\
$(163,63)$ & 11 & $(101,39)$ & 10 & 1 & YES & YES & YES & $1.57$ & $(2,3)$ & NO & 6603\\
$(163,71)$ & 11 & $(101,44)$ & 10 & 1 & YES & YES & YES & $1.57$ & $(2,3)$ & NO & 6604\\
$(163,62)$ & 11 & $(121,46)$ & 10 & 1 & YES & YES & YES & $1.43$ & $(2,3)$ & NO & 6605\\
$(163,67)$ & 12 & $(129,53)$ & 11 & 1 & YES & YES & YES & $1.57$ & $(2,3)$ & NO & 6606\\
$(163,62)$ & 11 & $(163,62)$ & 11 & 163 & YES & YES & YES & $1.57$ & $(2,3)$ & NO & 6607\\
$(163,67)$ & 12 & $(163,67)$ & 12 & 163 & YES & YES & NO(2) & $1.50$ & $(2,3)$ & NO & 6608\\
$(163,71)$ & 11 & $(163,71)$ & 11 & 163 & YES & YES & YES & $1.43$ & $(2,3)$ & NO & 6609\\
$(164,51)$ & 12 & $(3,1)$ & 2 & 1 & YES & YES & YES & $1.57$ & $(2,3)$ & -- & 6610\\
$(164,67)$ & 12 & $(3,1)$ & 2 & 1 & YES & YES & NO(2) & $1.75$ & $(2,3)$ & NO & 6611\\
$(164,67)$ & 12 & $(3,1)$ & 2 & 1 & YES & YES & NO(2) & $1.75$ & $(2,3)$ & -- & 6612\\
$(164,51)$ & 12 & $(4,1)$ & 3 & 4 & YES & YES & YES & $1.57$ & $(2,3)$ & -- & 6613\\
$(164,37)$ & 13 & $(5,2)$ & 3 & 1 & YES & YES & NO(2) & $1.75$ & $(2,3)$ & -- & 6614\\
$(164,51)$ & 12 & $(5,2)$ & 3 & 1 & YES & YES & YES & $1.57$ & $(2,3)$ & -- & 6615\\
$(164,61)$ & 12 & $(5,1)$ & 4 & 1 & YES & YES & YES & $1.57$ & $(2,3)$ & NO & 6616\\
$(164,37)$ & 13 & $(8,3)$ & 4 & 4 & YES & YES & YES & $1.71$ & $(2,3)$ & -- & 6617\\
$(164,51)$ & 12 & $(10,3)$ & 5 & 2 & YES & YES & YES & $1.57$ & $(2,3)$ & NO & 6618\\
$(164,37)$ & 13 & $(17,4)$ & 7 & 1 & YES & YES & NO(2) & $1.62$ & $(2,3)$ & NO & 6619\\
$(164,61)$ & 12 & $(19,7)$ & 6 & 1 & YES & YES & YES & $1.57$ & $(2,3)$ & NO & 6620\\
$(164,37)$ & 13 & $(30,7)$ & 8 & 2 & YES & YES & YES & $1.57$ & $(2,3)$ & NO & 6621\\
$(164,67)$ & 12 & $(93,38)$ & 11 & 1 & YES & YES & NO(2) & $1.75$ & $(2,3)$ & NO & 6622\\
$(164,51)$ & 12 & $(103,32)$ & 11 & 1 & YES & YES & YES & $1.57$ & $(2,3)$ & 8446 & 6623\\
$(165,49)$ & 11 & $(2,1)$ & 1 & 1 & YES & YES & YES & $1.57$ & $(2,3)$ & NO & 6624\\
$(165,61)$ & 11 & $(2,1)$ & 1 & 1 & YES & YES & YES & $1.57$ & $(2,3)$ & -- & 6625\\
$(165,49)$ & 11 & $(3,1)$ & 2 & 3 & YES & YES & YES & $1.57$ & $(2,3)$ & -- & 6626\\
$(165,49)$ & 11 & $(3,1)$ & 2 & 3 & YES & YES & YES & $1.57$ & $(2,3)$ & NO & 6627\\
$(165,46)$ & 11 & $(4,1)$ & 3 & 1 & YES & YES & YES & $1.57$ & $(2,3)$ & NO & 6628\\
$(165,46)$ & 11 & $(4,1)$ & 3 & 1 & YES & YES & YES & $1.57$ & $(2,3)$ & -- & 6629\\
$(165,46)$ & 11 & $(4,1)$ & 3 & 1 & YES & YES & YES & $1.57$ & $(2,3)$ & NO & 6630\\
$(165,61)$ & 11 & $(5,2)$ & 3 & 5 & YES & YES & YES & $1.43$ & $(2,3)$ & -- & 6631\\
$(165,46)$ & 11 & $(7,3)$ & 4 & 1 & YES & YES & YES & $1.57$ & $(2,3)$ & -- & 6632\\
$(165,49)$ & 11 & $(7,3)$ & 4 & 1 & YES & YES & YES & $1.57$ & $(2,3)$ & -- & 6633\\
$(165,49)$ & 11 & $(17,5)$ & 6 & 1 & YES & YES & YES & $1.71$ & $(2,3)$ & NO & 6634\\
$(165,64)$ & 11 & $(23,9)$ & 7 & 1 & YES & YES & NO(2) & $1.75$ & $(2,3)$ & NO & 6635\\
$(165,64)$ & 11 & $(116,45)$ & 10 & 1 & YES & YES & YES & $1.43$ & $(2,3)$ & NO & 6636\\
$(166,61)$ & 11 & $(2,1)$ & 1 & 2 & YES & YES & YES & $1.57$ & $(2,3)$ & -- & 6637\\
$(166,61)$ & 11 & $(3,1)$ & 2 & 1 & YES & YES & YES & $1.57$ & $(2,3)$ & -- & 6638\\
$(166,59)$ & 12 & $(4,1)$ & 3 & 2 & YES & YES & YES & $1.43$ & $(2,3)$ & NO & 6639\\
$(166,45)$ & 12 & $(5,2)$ & 3 & 1 & YES & YES & YES & $1.71$ & $(2,3)$ & NO & 6640\\
$(166,45)$ & 12 & $(5,2)$ & 3 & 1 & YES & YES & YES & $1.71$ & $(2,3)$ & -- & 6641\\
$(166,61)$ & 11 & $(5,2)$ & 3 & 1 & YES & YES & YES & $1.57$ & $(2,3)$ & -- & 6642\\
$(166,49)$ & 11 & $(7,2)$ & 4 & 1 & YES & YES & YES & $1.43$ & $(2,3)$ & -- & 6643\\
$(166,49)$ & 11 & $(9,2)$ & 5 & 1 & YES & YES & NO(3) & $1.29$ & $(2,3)$ & NO & 6644\\
$(166,61)$ & 11 & $(13,5)$ & 5 & 1 & YES & YES & YES & $1.57$ & $(2,3)$ & NO & 6645\\
$(166,49)$ & 11 & $(31,9)$ & 8 & 1 & YES & YES & YES & $1.43$ & $(2,3)$ & NO & 6646\\
$(166,75)$ & 13 & $(73,33)$ & 10 & 1 & YES & YES & YES & $1.71$ & $(2,3)$ & NO & 6647\\
$(166,75)$ & 13 & $(135,61)$ & 12 & 1 & YES & YES & YES & $1.71$ & $(2,3)$ & NO & 6648\\
$(167,69)$ & 11 & $(2,1)$ & 1 & 1 & YES & YES & YES & $1.43$ & $(2,3)$ & -- & 6649\\
$(167,62)$ & 12 & $(3,1)$ & 2 & 1 & YES & YES & NO(2) & $1.75$ & $(2,3)$ & -- & 6650\\
$(167,64)$ & 11 & $(3,1)$ & 2 & 1 & YES & YES & YES & $1.57$ & $(2,3)$ & -- & 6651\\
$(167,69)$ & 11 & $(3,1)$ & 2 & 1 & YES & YES & YES & $1.43$ & $(2,3)$ & -- & 6652\\
$(167,69)$ & 11 & $(3,1)$ & 2 & 1 & YES & YES & YES & $1.57$ & $(2,3)$ & NO & 6653\\
$(167,46)$ & 11 & $(4,1)$ & 3 & 1 & YES & YES & YES & $1.57$ & $(2,3)$ & NO & 6654\\
$(167,46)$ & 11 & $(4,1)$ & 3 & 1 & YES & YES & YES & $1.57$ & $(2,3)$ & -- & 6655\\
$(167,60)$ & 11 & $(4,1)$ & 3 & 1 & YES & YES & YES & $1.57$ & $(2,3)$ & -- & 6656\\
$(167,64)$ & 11 & $(4,1)$ & 3 & 1 & YES & YES & YES & $1.43$ & $(2,3)$ & -- & 6657\\
$(167,69)$ & 11 & $(4,1)$ & 3 & 1 & YES & YES & YES & $1.43$ & $(2,3)$ & NO & 6658\\
$(167,69)$ & 11 & $(4,1)$ & 3 & 1 & YES & YES & YES & $1.57$ & $(2,3)$ & -- & 6659\\
$(167,60)$ & 11 & $(5,2)$ & 3 & 1 & YES & YES & YES & $1.43$ & $(2,3)$ & 5374 & 6660\\
$(167,64)$ & 11 & $(5,2)$ & 3 & 1 & YES & YES & YES & $1.71$ & $(2,3)$ & -- & 6661\\
$(167,69)$ & 11 & $(5,1)$ & 4 & 1 & YES & YES & YES & $1.29$ & $(2,3)$ & NO & 6662\\
$(167,69)$ & 11 & $(5,2)$ & 3 & 1 & YES & YES & YES & $1.71$ & $(2,3)$ & -- & 6663\\
$(167,75)$ & 12 & $(5,2)$ & 3 & 1 & YES & YES & YES & $1.57$ & $(2,3)$ & NO & 6664\\
$(167,46)$ & 11 & $(7,3)$ & 4 & 1 & YES & YES & YES & $1.57$ & $(2,3)$ & -- & 6665\\
$(167,64)$ & 11 & $(7,3)$ & 4 & 1 & YES & YES & YES & $1.57$ & $(2,3)$ & NO & 6666\\
$(167,69)$ & 11 & $(8,3)$ & 4 & 1 & YES & YES & YES & $1.71$ & $(2,3)$ & NO & 6667\\
$(167,64)$ & 11 & $(9,2)$ & 5 & 1 & YES & YES & YES & $1.43$ & $(2,3)$ & -- & 6668\\
$(167,64)$ & 11 & $(21,8)$ & 6 & 1 & YES & YES & YES & $1.57$ & $(2,3)$ & NO & 6669\\
$(167,39)$ & 13 & $(22,5)$ & 7 & 1 & YES & YES & YES & $1.57$ & $(2,3)$ & NO & 6670\\
$(167,69)$ & 11 & $(22,9)$ & 7 & 1 & YES & YES & YES & $1.71$ & $(2,3)$ & NO & 6671\\
$(167,64)$ & 11 & $(34,13)$ & 7 & 1 & YES & YES & YES & $1.43$ & $(2,3)$ & 7524 & 6672\\
$(167,64)$ & 11 & $(47,18)$ & 8 & 1 & YES & YES & YES & $1.43$ & $(2,3)$ & 6432 & 6673\\
$(167,69)$ & 11 & $(63,26)$ & 9 & 1 & YES & YES & YES & $1.71$ & $(2,3)$ & NO & 6674\\
$(167,64)$ & 11 & $(73,28)$ & 10 & 1 & YES & YES & YES & $1.71$ & $(2,3)$ & NO & 6675\\
$(167,69)$ & 11 & $(75,31)$ & 9 & 1 & YES & YES & YES & $1.43$ & $(2,3)$ & 7266 & 6676\\
$(167,62)$ & 12 & $(89,33)$ & 10 & 1 & YES & YES & YES & $1.71$ & $(2,3)$ & NO & 6677\\
$(167,62)$ & 12 & $(97,36)$ & 10 & 1 & YES & YES & NO(2) & $1.62$ & $(2,3)$ & 7774 & 6678\\
$(167,69)$ & 11 & $(121,50)$ & 10 & 1 & YES & YES & YES & $1.43$ & $(2,3)$ & NO & 6679\\
$(167,62)$ & 12 & $(132,49)$ & 11 & 1 & YES & YES & NO(2) & $1.75$ & $(2,3)$ & NO & 6680\\
$(167,69)$ & 11 & $(167,69)$ & 11 & 167 & YES & YES & YES & $1.43$ & $(2,3)$ & NO & 6681\\
$(168,65)$ & 12 & $(9,2)$ & 5 & 3 & YES & YES & YES & $1.71$ & $(2,3)$ & NO & 6682\\
$(168,65)$ & 12 & $(119,46)$ & 10 & 7 & YES & YES & YES & $1.71$ & $(2,3)$ & NO & 6683\\
$(169,64)$ & 11 & $(2,1)$ & 1 & 1 & YES & YES & YES & $1.43$ & $(2,3)$ & NO & 6684\\
$(169,66)$ & 11 & $(2,1)$ & 1 & 1 & YES & YES & YES & $1.57$ & $(2,3)$ & -- & 6685\\
$(169,64)$ & 11 & $(3,1)$ & 2 & 1 & YES & YES & YES & $1.43$ & $(2,3)$ & -- & 6686\\
$(169,66)$ & 11 & $(3,1)$ & 2 & 1 & YES & YES & NO(2) & $1.62$ & $(2,3)$ & NO & 6687\\
$(169,66)$ & 11 & $(3,1)$ & 2 & 1 & YES & YES & YES & $1.43$ & $(2,3)$ & -- & 6688\\
$(169,70)$ & 11 & $(3,1)$ & 2 & 1 & YES & YES & NO(2) & $1.50$ & $(2,3)$ & NO & 6689\\
$(169,70)$ & 11 & $(3,1)$ & 2 & 1 & YES & YES & NO(2) & $1.50$ & $(2,3)$ & -- & 6690\\
$(169,62)$ & 12 & $(4,1)$ & 3 & 1 & YES & YES & YES & $1.71$ & $(2,3)$ & -- & 6691\\
$(169,69)$ & 12 & $(4,1)$ & 3 & 1 & YES & YES & YES & $1.57$ & $(2,3)$ & NO & 6692\\
$(169,70)$ & 11 & $(4,1)$ & 3 & 1 & YES & YES & YES & $1.57$ & $(2,3)$ & -- & 6693\\
$(169,70)$ & 11 & $(4,1)$ & 3 & 1 & YES & YES & YES & $1.71$ & $(2,3)$ & NO & 6694\\
$(169,50)$ & 11 & $(5,2)$ & 3 & 1 & YES & YES & YES & $1.43$ & $(2,3)$ & -- & 6695\\
$(169,61)$ & 12 & $(5,2)$ & 3 & 1 & YES & YES & YES & $1.71$ & $(2,3)$ & NO & 6696\\
$(169,61)$ & 12 & $(5,2)$ & 3 & 1 & YES & YES & YES & $1.71$ & $(2,3)$ & -- & 6697\\
$(169,62)$ & 12 & $(5,1)$ & 4 & 1 & YES & YES & YES & $1.71$ & $(2,3)$ & NO & 6698\\
$(169,64)$ & 11 & $(5,2)$ & 3 & 1 & YES & YES & YES & $1.57$ & $(2,3)$ & -- & 6699\\
$(169,70)$ & 11 & $(5,2)$ & 3 & 1 & YES & YES & YES & $1.43$ & $(2,3)$ & -- & 6700\\
$(169,70)$ & 11 & $(5,2)$ & 3 & 1 & YES & YES & YES & $1.43$ & $(2,3)$ & NO & 6701\\
$(169,40)$ & 13 & $(6,1)$ & 5 & 1 & YES & YES & YES & $1.71$ & $(2,3)$ & -- & 6702\\
$(169,64)$ & 11 & $(7,3)$ & 4 & 1 & YES & YES & YES & $1.57$ & $(2,3)$ & NO & 6703\\
$(169,66)$ & 11 & $(7,3)$ & 4 & 1 & YES & YES & YES & $1.57$ & $(2,3)$ & NO & 6704\\
$(169,70)$ & 11 & $(7,3)$ & 4 & 1 & YES & YES & YES & $1.43$ & $(2,3)$ & NO & 6705\\
$(169,66)$ & 11 & $(8,3)$ & 4 & 1 & YES & YES & YES & $1.43$ & $(2,3)$ & NO & 6706\\
$(169,71)$ & 11 & $(8,3)$ & 4 & 1 & YES & YES & YES & $1.71$ & $(2,3)$ & NO & 6707\\
$(169,64)$ & 11 & $(9,2)$ & 5 & 1 & YES & YES & YES & $1.43$ & $(2,3)$ & NO & 6708\\
$(169,64)$ & 11 & $(9,2)$ & 5 & 1 & YES & YES & YES & $1.43$ & $(2,3)$ & -- & 6709\\
$(169,66)$ & 11 & $(9,2)$ & 5 & 1 & YES & YES & YES & $1.43$ & $(2,3)$ & NO & 6710\\
$(169,70)$ & 11 & $(9,2)$ & 5 & 1 & YES & YES & YES & $1.57$ & $(2,3)$ & NO & 6711\\
$(169,40)$ & 13 & $(10,3)$ & 5 & 1 & YES & YES & YES & $1.57$ & $(2,3)$ & -- & 6712\\
$(169,69)$ & 12 & $(12,5)$ & 5 & 1 & YES & YES & YES & $1.57$ & $(2,3)$ & NO & 6713\\
$(169,64)$ & 11 & $(13,5)$ & 5 & 13 & YES & YES & YES & $1.43$ & $(2,3)$ & 5381 & 6714\\
$(169,66)$ & 11 & $(13,5)$ & 5 & 13 & YES & YES & YES & $1.43$ & $(2,3)$ & 7517 & 6715\\
$(169,71)$ & 11 & $(17,7)$ & 6 & 1 & YES & YES & YES & $1.71$ & $(2,3)$ & NO & 6716\\
$(169,64)$ & 11 & $(18,7)$ & 6 & 1 & YES & YES & YES & $1.57$ & $(2,3)$ & NO & 6717\\
$(169,66)$ & 11 & $(18,7)$ & 6 & 1 & YES & YES & YES & $1.43$ & $(2,3)$ & 6539 & 6718\\
$(169,70)$ & 11 & $(19,8)$ & 6 & 1 & YES & YES & YES & $1.57$ & $(2,3)$ & NO & 6719\\
$(169,64)$ & 11 & $(21,8)$ & 6 & 1 & YES & YES & YES & $1.43$ & $(2,3)$ & NO & 6720\\
$(169,66)$ & 11 & $(23,9)$ & 7 & 1 & YES & YES & YES & $1.43$ & $(2,3)$ & NO & 6721\\
$(169,50)$ & 11 & $(24,7)$ & 7 & 1 & YES & YES & YES & $1.43$ & $(2,3)$ & NO & 6722\\
$(169,64)$ & 11 & $(29,11)$ & 7 & 1 & YES & YES & YES & $1.57$ & $(2,3)$ & NO & 6723\\
$(169,64)$ & 11 & $(45,17)$ & 9 & 1 & YES & YES & YES & $1.57$ & $(2,3)$ & NO & 6724\\
$(169,64)$ & 11 & $(66,25)$ & 9 & 1 & YES & YES & YES & $1.43$ & $(2,3)$ & NO & 6725\\
$(169,71)$ & 11 & $(69,29)$ & 9 & 1 & YES & YES & YES & $1.43$ & $(2,3)$ & 7112 & 6726\\
$(169,40)$ & 13 & $(72,17)$ & 11 & 1 & YES & YES & NO(2) & $1.62$ & $(2,3)$ & NO & 6727\\
$(169,61)$ & 12 & $(86,31)$ & 10 & 1 & YES & YES & YES & $1.71$ & $(2,3)$ & NO & 6728\\
$(169,71)$ & 11 & $(119,50)$ & 10 & 1 & YES & YES & YES & $1.57$ & $(2,3)$ & NO & 6729\\
$(169,70)$ & 11 & $(128,53)$ & 11 & 1 & YES & YES & YES & $1.43$ & $(2,3)$ & NO & 6730\\
$(169,64)$ & 11 & $(140,53)$ & 11 & 1 & YES & YES & YES & $1.43$ & $(2,3)$ & NO & 6731\\
$(169,40)$ & 13 & $(169,40)$ & 13 & 169 & YES & YES & YES & $1.71$ & $(2,3)$ & NO & 6732\\
$(169,53)$ & 13 & $(169,53)$ & 13 & 169 & YES & YES & YES & $1.71$ & $(2,3)$ & NO & 6733\\
$(169,64)$ & 11 & $(169,64)$ & 11 & 169 & YES & YES & YES & $1.43$ & $(2,3)$ & NO & 6734\\
$(170,39)$ & 12 & $(5,2)$ & 3 & 5 & YES & YES & YES & $1.57$ & $(2,3)$ & NO & 6735\\
$(170,47)$ & 11 & $(7,3)$ & 4 & 1 & YES & YES & YES & $1.57$ & $(2,3)$ & -- & 6736\\
$(170,61)$ & 12 & $(25,9)$ & 7 & 5 & YES & YES & YES & $1.57$ & $(2,3)$ & NO & 6737\\
$(170,47)$ & 11 & $(32,9)$ & 8 & 2 & YES & YES & YES & $1.43$ & $(2,3)$ & NO & 6738\\
$(170,61)$ & 12 & $(170,61)$ & 12 & 170 & YES & YES & YES & $1.57$ & $(2,3)$ & NO & 6739\\
$(171,65)$ & 11 & $(2,1)$ & 1 & 1 & YES & YES & YES & $1.43$ & $(2,3)$ & -- & 6740\\
$(171,65)$ & 11 & $(2,1)$ & 1 & 1 & YES & YES & YES & $1.57$ & $(2,3)$ & NO & 6741\\
$(171,65)$ & 11 & $(3,1)$ & 2 & 3 & YES & YES & YES & $1.57$ & $(2,3)$ & NO & 6742\\
$(171,65)$ & 11 & $(3,1)$ & 2 & 3 & YES & YES & YES & $1.57$ & $(2,3)$ & -- & 6743\\
$(171,65)$ & 11 & $(3,1)$ & 2 & 3 & YES & YES & YES & $1.57$ & $(2,3)$ & NO & 6744\\
$(171,65)$ & 11 & $(4,1)$ & 3 & 1 & YES & YES & YES & $1.43$ & $(2,3)$ & -- & 6745\\
$(171,65)$ & 11 & $(5,2)$ & 3 & 1 & YES & YES & YES & $1.57$ & $(2,3)$ & -- & 6746\\
$(171,71)$ & 12 & $(5,2)$ & 3 & 1 & YES & YES & YES & $1.71$ & $(2,3)$ & -- & 6747\\
$(171,40)$ & 12 & $(7,3)$ & 4 & 1 & YES & YES & YES & $1.57$ & $(2,3)$ & -- & 6748\\
$(171,65)$ & 11 & $(7,2)$ & 4 & 1 & YES & YES & YES & $1.57$ & $(2,3)$ & -- & 6749\\
$(171,71)$ & 12 & $(7,3)$ & 4 & 1 & YES & YES & YES & $1.57$ & $(2,3)$ & NO & 6750\\
$(171,50)$ & 11 & $(9,2)$ & 5 & 9 & YES & YES & NO(3) & $1.29$ & $(2,3)$ & NO & 6751\\
$(171,40)$ & 12 & $(13,3)$ & 6 & 1 & YES & YES & YES & $1.57$ & $(2,3)$ & NO & 6752\\
$(171,53)$ & 12 & $(16,5)$ & 7 & 1 & YES & YES & YES & $1.57$ & $(2,3)$ & NO & 6753\\
$(171,37)$ & 12 & $(22,5)$ & 7 & 1 & YES & YES & YES & $1.57$ & $(2,3)$ & NO & 6754\\
$(171,40)$ & 12 & $(22,5)$ & 7 & 1 & YES & YES & YES & $1.29$ & $(2,3)$ & NO & 6755\\
$(171,65)$ & 11 & $(71,27)$ & 9 & 1 & YES & YES & YES & $1.43$ & $(2,3)$ & 7184 & 6756\\
$(171,40)$ & 12 & $(73,17)$ & 10 & 1 & YES & YES & YES & $1.57$ & $(2,3)$ & NO & 6757\\
$(171,65)$ & 11 & $(92,35)$ & 10 & 1 & YES & YES & YES & $1.57$ & $(2,3)$ & 8540 & 6758\\
$(171,67)$ & 12 & $(97,38)$ & 11 & 1 & YES & YES & YES & $1.57$ & $(2,3)$ & NO & 6759\\
$(171,50)$ & 11 & $(99,29)$ & 10 & 9 & YES & YES & YES & $1.57$ & $(2,3)$ & NO & 6760\\
$(171,65)$ & 11 & $(121,46)$ & 10 & 1 & YES & YES & YES & $1.43$ & $(2,3)$ & NO & 6761\\
$(171,65)$ & 11 & $(171,65)$ & 11 & 171 & YES & YES & YES & $1.43$ & $(2,3)$ & NO & 6762\\
$(172,71)$ & 11 & $(2,1)$ & 1 & 2 & YES & YES & YES & $1.43$ & $(2,3)$ & -- & 6763\\
$(172,61)$ & 13 & $(3,1)$ & 2 & 1 & YES & YES & YES & $1.57$ & $(2,3)$ & -- & 6764\\
$(172,71)$ & 11 & $(3,1)$ & 2 & 1 & YES & YES & YES & $1.43$ & $(2,3)$ & -- & 6765\\
$(172,71)$ & 11 & $(5,2)$ & 3 & 1 & YES & YES & YES & $1.57$ & $(2,3)$ & -- & 6766\\
$(172,71)$ & 11 & $(5,2)$ & 3 & 1 & YES & YES & NO(2) & $1.62$ & $(2,3)$ & NO & 6767\\
$(172,63)$ & 11 & $(9,2)$ & 5 & 1 & YES & YES & YES & $1.43$ & $(2,3)$ & -- & 6768\\
$(172,71)$ & 11 & $(9,2)$ & 5 & 1 & YES & YES & YES & $1.57$ & $(2,3)$ & -- & 6769\\
$(172,71)$ & 11 & $(9,2)$ & 5 & 1 & YES & YES & YES & $1.57$ & $(2,3)$ & NO & 6770\\
$(172,71)$ & 11 & $(17,7)$ & 6 & 1 & YES & YES & NO(2) & $1.50$ & $(2,3)$ & NO & 6771\\
$(172,71)$ & 11 & $(22,9)$ & 7 & 2 & YES & YES & YES & $1.57$ & $(2,3)$ & NO & 6772\\
$(172,71)$ & 11 & $(29,12)$ & 7 & 1 & YES & YES & YES & $1.43$ & $(2,3)$ & 7263 & 6773\\
$(172,71)$ & 11 & $(41,17)$ & 8 & 1 & YES & YES & YES & $1.43$ & $(2,3)$ & 8431 & 6774\\
$(172,71)$ & 11 & $(46,19)$ & 8 & 2 & YES & YES & YES & $1.43$ & $(2,3)$ & 6467 & 6775\\
$(172,71)$ & 11 & $(109,45)$ & 10 & 1 & YES & YES & YES & $1.43$ & $(2,3)$ & NO & 6776\\
$(172,63)$ & 11 & $(131,48)$ & 11 & 1 & YES & YES & YES & $1.57$ & $(2,3)$ & NO & 6777\\
$(172,61)$ & 13 & $(141,50)$ & 12 & 1 & YES & YES & YES & $1.57$ & $(2,3)$ & NO & 6778\\
$(172,71)$ & 11 & $(155,64)$ & 11 & 1 & YES & YES & YES & $1.57$ & $(2,3)$ & NO & 6779\\
$(173,66)$ & 11 & $(2,1)$ & 1 & 1 & YES & YES & YES & $1.71$ & $(2,3)$ & -- & 6780\\
$(173,73)$ & 11 & $(2,1)$ & 1 & 1 & YES & YES & YES & $1.43$ & $(2,3)$ & -- & 6781\\
$(173,76)$ & 11 & $(2,1)$ & 1 & 1 & YES & YES & YES & $1.57$ & $(2,3)$ & -- & 6782\\
$(173,71)$ & 12 & $(3,1)$ & 2 & 1 & YES & YES & YES & $1.57$ & $(2,3)$ & NO & 6783\\
$(173,71)$ & 12 & $(3,1)$ & 2 & 1 & YES & YES & YES & $1.57$ & $(2,3)$ & -- & 6784\\
$(173,73)$ & 11 & $(3,1)$ & 2 & 1 & YES & YES & YES & $1.57$ & $(2,3)$ & -- & 6785\\
$(173,73)$ & 11 & $(3,1)$ & 2 & 1 & YES & YES & YES & $1.43$ & $(2,3)$ & NO & 6786\\
$(173,76)$ & 11 & $(3,1)$ & 2 & 1 & YES & YES & YES & $1.29$ & $(2,3)$ & -- & 6787\\
$(173,76)$ & 11 & $(3,1)$ & 2 & 1 & YES & YES & YES & $1.43$ & $(2,3)$ & NO & 6788\\
$(173,51)$ & 12 & $(4,1)$ & 3 & 1 & YES & YES & YES & $1.43$ & $(2,3)$ & -- & 6789\\
$(173,66)$ & 11 & $(4,1)$ & 3 & 1 & YES & YES & YES & $1.43$ & $(2,3)$ & -- & 6790\\
$(173,76)$ & 11 & $(4,1)$ & 3 & 1 & YES & YES & YES & $1.43$ & $(2,3)$ & -- & 6791\\
$(173,64)$ & 11 & $(5,2)$ & 3 & 1 & YES & YES & YES & $1.43$ & $(2,3)$ & -- & 6792\\
$(173,66)$ & 11 & $(5,2)$ & 3 & 1 & YES & YES & YES & $1.71$ & $(2,3)$ & -- & 6793\\
$(173,71)$ & 12 & $(5,1)$ & 4 & 1 & YES & YES & NO(2) & $1.62$ & $(2,3)$ & NO & 6794\\
$(173,64)$ & 11 & $(7,2)$ & 4 & 1 & YES & YES & YES & $1.57$ & $(2,3)$ & -- & 6795\\
$(173,71)$ & 12 & $(7,3)$ & 4 & 1 & YES & YES & YES & $1.71$ & $(2,3)$ & -- & 6796\\
$(173,76)$ & 11 & $(11,5)$ & 6 & 1 & YES & YES & YES & $1.43$ & $(2,3)$ & NO & 6797\\
$(173,71)$ & 12 & $(12,5)$ & 5 & 1 & YES & YES & NO(2) & $1.75$ & $(2,3)$ & NO & 6798\\
$(173,76)$ & 11 & $(16,7)$ & 6 & 1 & YES & YES & YES & $1.43$ & $(2,3)$ & 6461 & 6799\\
$(173,73)$ & 11 & $(19,8)$ & 6 & 1 & YES & YES & NO(2) & $1.62$ & $(2,3)$ & NO & 6800\\
$(173,76)$ & 11 & $(25,11)$ & 7 & 1 & YES & YES & YES & $1.43$ & $(2,3)$ & NO & 6801\\
$(173,73)$ & 11 & $(26,11)$ & 7 & 1 & YES & YES & NO(2) & $1.50$ & $(2,3)$ & 7077 & 6802\\
$(173,64)$ & 11 & $(30,11)$ & 7 & 1 & YES & YES & YES & $1.57$ & $(2,3)$ & NO & 6803\\
$(173,66)$ & 11 & $(34,13)$ & 7 & 1 & YES & YES & YES & $1.57$ & $(2,3)$ & 5262 & 6804\\
$(173,64)$ & 11 & $(35,13)$ & 8 & 1 & YES & YES & YES & $1.43$ & $(2,3)$ & NO & 6805\\
$(173,76)$ & 11 & $(41,18)$ & 8 & 1 & YES & YES & YES & $1.57$ & $(2,3)$ & 6324 & 6806\\
$(173,51)$ & 12 & $(44,13)$ & 8 & 1 & YES & YES & YES & $1.43$ & $(2,3)$ & 5431 & 6807\\
$(173,66)$ & 11 & $(47,18)$ & 8 & 1 & YES & YES & YES & $1.71$ & $(2,3)$ & NO & 6808\\
$(173,78)$ & 12 & $(51,23)$ & 9 & 1 & YES & YES & YES & $1.57$ & $(2,3)$ & NO & 6809\\
$(173,71)$ & 12 & $(56,23)$ & 9 & 1 & YES & YES & YES & $1.71$ & $(2,3)$ & NO & 6810\\
$(173,71)$ & 12 & $(61,25)$ & 9 & 1 & YES & YES & YES & $1.57$ & $(2,3)$ & NO & 6811\\
$(173,66)$ & 11 & $(76,29)$ & 9 & 1 & YES & YES & YES & $1.57$ & $(2,3)$ & NO & 6812\\
$(173,51)$ & 12 & $(95,28)$ & 11 & 1 & YES & YES & YES & $1.57$ & $(2,3)$ & NO & 6813\\
$(173,32)$ & 14 & $(103,19)$ & 11 & 1 & YES & YES & YES & $1.57$ & $(2,3)$ & NO & 6814\\
$(173,76)$ & 11 & $(107,47)$ & 10 & 1 & YES & YES & YES & $1.43$ & $(2,3)$ & NO & 6815\\
$(173,71)$ & 12 & $(134,55)$ & 11 & 1 & YES & YES & NO(2) & $1.62$ & $(2,3)$ & NO & 6816\\
$(173,76)$ & 11 & $(173,76)$ & 11 & 173 & YES & YES & YES & $1.29$ & $(2,3)$ & NO & 6817\\
$(174,47)$ & 12 & $(3,1)$ & 2 & 3 & YES & YES & NO(2) & $1.71$ & $(4,2)$ & -- & 6818\\
$(175,62)$ & 12 & $(3,1)$ & 2 & 1 & YES & YES & YES & $1.57$ & $(2,3)$ & -- & 6819\\
$(175,62)$ & 12 & $(3,1)$ & 2 & 1 & YES & YES & YES & $1.71$ & $(2,3)$ & NO & 6820\\
$(175,64)$ & 12 & $(3,1)$ & 2 & 1 & YES & YES & NO(2) & $1.62$ & $(2,3)$ & -- & 6821\\
$(175,62)$ & 12 & $(4,1)$ & 3 & 1 & YES & YES & YES & $1.57$ & $(2,3)$ & -- & 6822\\
$(175,76)$ & 12 & $(4,1)$ & 3 & 1 & YES & YES & NO(2) & $1.50$ & $(2,3)$ & -- & 6823\\
$(175,62)$ & 12 & $(5,1)$ & 4 & 5 & YES & YES & YES & $1.57$ & $(2,3)$ & -- & 6824\\
$(175,62)$ & 12 & $(5,2)$ & 3 & 5 & YES & YES & YES & $1.57$ & $(2,3)$ & NO & 6825\\
$(175,64)$ & 12 & $(5,1)$ & 4 & 5 & YES & YES & NO(2) & $1.62$ & $(2,3)$ & NO & 6826\\
$(175,64)$ & 12 & $(5,2)$ & 3 & 5 & YES & YES & YES & $1.57$ & $(2,3)$ & -- & 6827\\
$(175,62)$ & 12 & $(6,1)$ & 5 & 1 & YES & YES & YES & $1.57$ & $(2,3)$ & -- & 6828\\
$(175,53)$ & 13 & $(7,3)$ & 4 & 7 & YES & YES & YES & $1.71$ & $(2,3)$ & NO & 6829\\
$(175,62)$ & 12 & $(8,3)$ & 4 & 1 & YES & YES & YES & $1.57$ & $(2,3)$ & NO & 6830\\
$(175,62)$ & 12 & $(11,4)$ & 5 & 1 & YES & YES & YES & $1.57$ & $(2,3)$ & NO & 6831\\
$(175,76)$ & 12 & $(16,7)$ & 6 & 1 & YES & YES & NO(2) & $1.62$ & $(2,3)$ & NO & 6832\\
$(175,62)$ & 12 & $(79,28)$ & 10 & 1 & YES & YES & YES & $1.57$ & $(2,3)$ & 7442 & 6833\\
$(175,64)$ & 12 & $(93,34)$ & 10 & 1 & YES & YES & NO(2) & $1.62$ & $(2,3)$ & 7743 & 6834\\
$(175,76)$ & 12 & $(99,43)$ & 11 & 1 & YES & YES & NO(2) & $1.62$ & $(2,3)$ & NO & 6835\\
$(175,62)$ & 12 & $(127,45)$ & 11 & 1 & YES & YES & YES & $1.57$ & $(2,3)$ & NO & 6836\\
$(175,53)$ & 13 & $(142,43)$ & 12 & 1 & YES & YES & NO(2) & $1.62$ & $(2,3)$ & NO & 6837\\
$(175,62)$ & 12 & $(175,62)$ & 12 & 175 & YES & YES & YES & $1.57$ & $(2,3)$ & NO & 6838\\
$(175,64)$ & 12 & $(175,64)$ & 12 & 175 & YES & YES & NO(2) & $1.62$ & $(2,3)$ & NO & 6839\\
$(176,41)$ & 12 & $(3,1)$ & 2 & 1 & YES & YES & YES & $1.57$ & $(2,3)$ & NO & 6840\\
$(176,41)$ & 12 & $(3,1)$ & 2 & 1 & YES & YES & YES & $1.57$ & $(2,3)$ & -- & 6841\\
$(176,51)$ & 12 & $(3,1)$ & 2 & 1 & YES & YES & YES & $1.43$ & $(2,3)$ & -- & 6842\\
$(176,79)$ & 12 & $(3,1)$ & 2 & 1 & YES & YES & YES & $1.57$ & $(2,3)$ & -- & 6843\\
$(176,51)$ & 12 & $(4,1)$ & 3 & 4 & YES & YES & YES & $1.43$ & $(2,3)$ & -- & 6844\\
$(176,65)$ & 11 & $(4,1)$ & 3 & 4 & YES & YES & YES & $1.43$ & $(2,3)$ & -- & 6845\\
$(176,51)$ & 12 & $(5,2)$ & 3 & 1 & YES & YES & YES & $1.57$ & $(2,3)$ & -- & 6846\\
$(176,65)$ & 11 & $(5,2)$ & 3 & 1 & YES & YES & YES & $1.43$ & $(2,3)$ & -- & 6847\\
$(176,79)$ & 12 & $(6,1)$ & 5 & 2 & YES & YES & YES & $1.71$ & $(2,3)$ & -- & 6848\\
$(176,41)$ & 12 & $(7,3)$ & 4 & 1 & YES & YES & NO(2) & $1.62$ & $(2,3)$ & -- & 6849\\
$(176,65)$ & 11 & $(7,2)$ & 4 & 1 & YES & YES & YES & $1.57$ & $(2,3)$ & -- & 6850\\
$(176,49)$ & 12 & $(8,3)$ & 4 & 8 & YES & YES & YES & $1.71$ & $(2,3)$ & NO & 6851\\
$(176,41)$ & 12 & $(17,4)$ & 7 & 1 & YES & YES & YES & $1.57$ & $(2,3)$ & NO & 6852\\
$(176,51)$ & 12 & $(45,13)$ & 10 & 1 & YES & YES & YES & $1.57$ & $(2,3)$ & NO & 6853\\
$(176,65)$ & 11 & $(103,38)$ & 11 & 1 & YES & YES & YES & $1.71$ & $(2,3)$ & NO & 6854\\
$(176,51)$ & 12 & $(107,31)$ & 11 & 1 & YES & YES & YES & $1.43$ & $(2,3)$ & NO & 6855\\
$(176,65)$ & 11 & $(157,58)$ & 11 & 1 & YES & YES & YES & $1.57$ & $(2,3)$ & NO & 6856\\
$(176,51)$ & 12 & $(176,51)$ & 12 & 176 & YES & YES & YES & $1.43$ & $(2,3)$ & NO & 6857\\
$(176,79)$ & 12 & $(176,79)$ & 12 & 176 & YES & YES & YES & $1.57$ & $(2,3)$ & NO & 6858\\
$(177,65)$ & 11 & $(2,1)$ & 1 & 1 & YES & YES & YES & $1.57$ & $(2,3)$ & -- & 6859\\
$(177,74)$ & 12 & $(2,1)$ & 1 & 1 & YES & YES & NO(2) & $1.75$ & $(2,3)$ & -- & 6860\\
$(177,73)$ & 12 & $(3,1)$ & 2 & 3 & YES & YES & YES & $1.71$ & $(2,3)$ & NO & 6861\\
$(177,73)$ & 12 & $(3,1)$ & 2 & 3 & YES & YES & YES & $1.71$ & $(2,3)$ & -- & 6862\\
$(177,74)$ & 12 & $(3,1)$ & 2 & 3 & YES & YES & YES & $1.43$ & $(2,3)$ & -- & 6863\\
$(177,64)$ & 12 & $(5,2)$ & 3 & 1 & YES & YES & YES & $1.71$ & $(2,3)$ & NO & 6864\\
$(177,64)$ & 12 & $(5,2)$ & 3 & 1 & YES & YES & YES & $1.71$ & $(2,3)$ & -- & 6865\\
$(177,67)$ & 12 & $(5,1)$ & 4 & 1 & YES & YES & YES & $1.57$ & $(2,3)$ & NO & 6866\\
$(177,67)$ & 12 & $(5,1)$ & 4 & 1 & YES & YES & YES & $1.71$ & $(2,3)$ & -- & 6867\\
$(177,49)$ & 11 & $(8,3)$ & 4 & 1 & YES & YES & YES & $1.57$ & $(2,3)$ & NO & 6868\\
$(177,74)$ & 12 & $(12,5)$ & 5 & 3 & YES & YES & YES & $1.57$ & $(2,3)$ & NO & 6869\\
$(177,49)$ & 11 & $(13,4)$ & 6 & 1 & YES & YES & YES & $1.57$ & $(2,3)$ & NO & 6870\\
$(177,74)$ & 12 & $(19,8)$ & 6 & 1 & YES & YES & YES & $1.57$ & $(2,3)$ & 4980 & 6871\\
$(177,64)$ & 12 & $(61,22)$ & 9 & 1 & YES & YES & YES & $1.86$ & $(2,3)$ & 8786 & 6872\\
$(177,73)$ & 12 & $(63,26)$ & 9 & 3 & YES & YES & YES & $1.57$ & $(2,3)$ & NO & 6873\\
$(177,47)$ & 12 & $(83,22)$ & 10 & 1 & YES & YES & YES & $1.43$ & $(2,3)$ & NO & 6874\\
$(177,67)$ & 12 & $(103,39)$ & 10 & 1 & YES & YES & YES & $1.57$ & $(2,3)$ & 7956 & 6875\\
$(177,65)$ & 11 & $(109,40)$ & 10 & 1 & YES & YES & YES & $1.43$ & $(2,3)$ & 8631 & 6876\\
$(177,73)$ & 12 & $(177,73)$ & 12 & 177 & YES & YES & YES & $1.57$ & $(2,3)$ & NO & 6877\\
$(178,69)$ & 11 & $(2,1)$ & 1 & 2 & YES & YES & YES & $1.57$ & $(2,3)$ & NO & 6878\\
$(178,53)$ & 12 & $(3,1)$ & 2 & 1 & YES & YES & YES & $1.57$ & $(2,3)$ & -- & 6879\\
$(178,69)$ & 11 & $(3,1)$ & 2 & 1 & YES & YES & YES & $1.71$ & $(2,3)$ & NO & 6880\\
$(178,49)$ & 11 & $(4,1)$ & 3 & 2 & NO & YES & YES & $1.57$ & $(2,3)$ & -- & 6881\\
$(178,53)$ & 12 & $(4,1)$ & 3 & 2 & YES & YES & YES & $1.57$ & $(2,3)$ & -- & 6882\\
$(178,69)$ & 11 & $(4,1)$ & 3 & 2 & YES & YES & YES & $1.71$ & $(2,3)$ & -- & 6883\\
$(178,53)$ & 12 & $(5,2)$ & 3 & 1 & YES & YES & YES & $1.71$ & $(2,3)$ & NO & 6884\\
$(178,69)$ & 11 & $(5,1)$ & 4 & 1 & YES & YES & YES & $1.57$ & $(2,3)$ & -- & 6885\\
$(178,69)$ & 11 & $(7,3)$ & 4 & 1 & YES & YES & YES & $1.71$ & $(2,3)$ & NO & 6886\\
$(178,69)$ & 11 & $(11,2)$ & 6 & 1 & YES & YES & YES & $1.43$ & $(2,3)$ & NO & 6887\\
$(178,53)$ & 12 & $(13,4)$ & 6 & 1 & YES & YES & YES & $1.71$ & $(2,3)$ & NO & 6888\\
$(178,53)$ & 12 & $(17,5)$ & 6 & 1 & YES & YES & YES & $1.57$ & $(2,3)$ & NO & 6889\\
$(178,53)$ & 12 & $(27,8)$ & 7 & 1 & YES & YES & YES & $1.57$ & $(2,3)$ & NO & 6890\\
$(178,69)$ & 11 & $(31,12)$ & 7 & 1 & YES & YES & YES & $1.57$ & $(2,3)$ & NO & 6891\\
$(178,33)$ & 14 & $(59,11)$ & 10 & 1 & YES & YES & YES & $1.57$ & $(2,3)$ & NO & 6892\\
$(178,69)$ & 11 & $(80,31)$ & 9 & 2 & YES & YES & YES & $1.57$ & $(2,3)$ & 7532 & 6893\\
$(178,69)$ & 11 & $(178,69)$ & 11 & 178 & YES & YES & YES & $1.57$ & $(2,3)$ & NO & 6894\\
$(179,75)$ & 11 & $(2,1)$ & 1 & 1 & YES & YES & YES & $1.29$ & $(2,3)$ & -- & 6895\\
$(179,73)$ & 12 & $(3,1)$ & 2 & 1 & YES & YES & YES & $1.57$ & $(2,3)$ & -- & 6896\\
$(179,73)$ & 12 & $(3,1)$ & 2 & 1 & YES & YES & YES & $1.71$ & $(2,3)$ & NO & 6897\\
$(179,74)$ & 11 & $(3,1)$ & 2 & 1 & YES & YES & YES & $1.29$ & $(2,3)$ & -- & 6898\\
$(179,75)$ & 11 & $(3,1)$ & 2 & 1 & YES & YES & YES & $1.29$ & $(2,3)$ & -- & 6899\\
$(179,75)$ & 11 & $(3,1)$ & 2 & 1 & YES & YES & YES & $1.43$ & $(2,3)$ & NO & 6900\\
$(179,50)$ & 11 & $(5,2)$ & 3 & 1 & YES & YES & YES & $1.43$ & $(2,3)$ & -- & 6901\\
$(179,73)$ & 12 & $(5,1)$ & 4 & 1 & YES & YES & YES & $1.57$ & $(2,3)$ & NO & 6902\\
$(179,73)$ & 12 & $(5,1)$ & 4 & 1 & YES & YES & YES & $1.57$ & $(2,3)$ & -- & 6903\\
$(179,74)$ & 11 & $(5,1)$ & 4 & 1 & YES & YES & NO(2) & $1.57$ & $(4,2)$ & -- & 6904\\
$(179,74)$ & 11 & $(5,2)$ & 3 & 1 & YES & YES & YES & $1.57$ & $(2,3)$ & -- & 6905\\
$(179,75)$ & 11 & $(5,2)$ & 3 & 1 & YES & YES & YES & $1.43$ & $(2,3)$ & NO & 6906\\
$(179,75)$ & 11 & $(5,2)$ & 3 & 1 & YES & YES & YES & $1.57$ & $(2,3)$ & -- & 6907\\
$(179,38)$ & 13 & $(7,3)$ & 4 & 1 & YES & YES & YES & $1.71$ & $(2,3)$ & -- & 6908\\
$(179,73)$ & 12 & $(7,3)$ & 4 & 1 & YES & YES & YES & $1.71$ & $(2,3)$ & NO & 6909\\
$(179,75)$ & 11 & $(7,2)$ & 4 & 1 & YES & YES & YES & $1.57$ & $(2,3)$ & NO & 6910\\
$(179,50)$ & 11 & $(9,2)$ & 5 & 1 & YES & YES & YES & $1.43$ & $(2,3)$ & NO & 6911\\
$(179,41)$ & 12 & $(16,3)$ & 7 & 1 & YES & YES & YES & $1.57$ & $(2,3)$ & NO & 6912\\
$(179,73)$ & 12 & $(17,7)$ & 6 & 1 & YES & YES & YES & $1.57$ & $(2,3)$ & NO & 6913\\
$(179,75)$ & 11 & $(17,7)$ & 6 & 1 & YES & YES & YES & $1.57$ & $(2,3)$ & NO & 6914\\
$(179,50)$ & 11 & $(29,8)$ & 7 & 1 & YES & YES & YES & $1.43$ & $(2,3)$ & NO & 6915\\
$(179,74)$ & 11 & $(29,12)$ & 7 & 1 & YES & YES & NO(2) & $1.62$ & $(2,3)$ & 6059 & 6916\\
$(179,41)$ & 12 & $(40,9)$ & 9 & 1 & YES & YES & YES & $1.57$ & $(2,3)$ & NO & 6917\\
$(179,74)$ & 11 & $(41,17)$ & 8 & 1 & YES & YES & YES & $1.57$ & $(2,3)$ & NO & 6918\\
$(179,73)$ & 12 & $(49,20)$ & 9 & 1 & YES & YES & YES & $1.43$ & $(2,3)$ & NO & 6919\\
$(179,75)$ & 11 & $(55,23)$ & 9 & 1 & YES & YES & YES & $1.57$ & $(2,3)$ & NO & 6920\\
$(179,41)$ & 12 & $(74,17)$ & 11 & 1 & YES & YES & YES & $1.57$ & $(2,3)$ & NO & 6921\\
$(179,74)$ & 11 & $(75,31)$ & 9 & 1 & YES & YES & NO(3) & $1.29$ & $(2,3)$ & NO & 6922\\
$(179,75)$ & 11 & $(105,44)$ & 10 & 1 & YES & YES & YES & $1.29$ & $(2,3)$ & NO & 6923\\
$(179,74)$ & 11 & $(179,74)$ & 11 & 179 & YES & YES & YES & $1.43$ & $(2,3)$ & NO & 6924\\
$(179,75)$ & 11 & $(179,75)$ & 11 & 179 & YES & YES & YES & $1.29$ & $(2,3)$ & NO & 6925\\
$(180,43)$ & 13 & $(5,2)$ & 3 & 5 & YES & YES & YES & $1.57$ & $(2,3)$ & NO & 6926\\
$(180,79)$ & 12 & $(5,1)$ & 4 & 5 & YES & YES & YES & $1.71$ & $(2,3)$ & -- & 6927\\
$(180,41)$ & 12 & $(7,3)$ & 4 & 1 & YES & YES & YES & $1.57$ & $(2,3)$ & NO & 6928\\
$(180,41)$ & 12 & $(8,3)$ & 4 & 4 & YES & YES & YES & $1.71$ & $(2,3)$ & NO & 6929\\
$(180,41)$ & 12 & $(10,3)$ & 5 & 10 & YES & YES & YES & $1.43$ & $(2,3)$ & 8472 & 6930\\
$(180,79)$ & 12 & $(98,43)$ & 10 & 2 & YES & YES & YES & $1.71$ & $(2,3)$ & 7880 & 6931\\
$(181,65)$ & 12 & $(2,1)$ & 1 & 1 & YES & YES & YES & $1.57$ & $(2,3)$ & -- & 6932\\
$(181,70)$ & 11 & $(2,1)$ & 1 & 1 & YES & YES & NO(2) & $1.75$ & $(2,3)$ & -- & 6933\\
$(181,75)$ & 11 & $(2,1)$ & 1 & 1 & YES & YES & YES & $1.43$ & $(2,3)$ & -- & 6934\\
$(181,50)$ & 11 & $(3,1)$ & 2 & 1 & NO & YES & YES & $1.57$ & $(2,3)$ & -- & 6935\\
$(181,53)$ & 12 & $(3,1)$ & 2 & 1 & YES & YES & YES & $1.43$ & $(2,3)$ & NO & 6936\\
$(181,65)$ & 12 & $(3,1)$ & 2 & 1 & YES & YES & YES & $1.57$ & $(2,3)$ & -- & 6937\\
$(181,70)$ & 11 & $(3,1)$ & 2 & 1 & YES & YES & YES & $1.57$ & $(2,3)$ & -- & 6938\\
$(181,70)$ & 11 & $(3,1)$ & 2 & 1 & YES & YES & YES & $1.57$ & $(2,3)$ & NO & 6939\\
$(181,75)$ & 11 & $(3,1)$ & 2 & 1 & YES & YES & YES & $1.43$ & $(2,3)$ & -- & 6940\\
$(181,75)$ & 11 & $(3,1)$ & 2 & 1 & YES & YES & YES & $1.43$ & $(2,3)$ & NO & 6941\\
$(181,53)$ & 12 & $(4,1)$ & 3 & 1 & YES & YES & YES & $1.57$ & $(2,3)$ & -- & 6942\\
$(181,55)$ & 12 & $(4,1)$ & 3 & 1 & YES & YES & YES & $1.57$ & $(2,3)$ & -- & 6943\\
$(181,65)$ & 12 & $(4,1)$ & 3 & 1 & YES & YES & YES & $1.57$ & $(2,3)$ & NO & 6944\\
$(181,65)$ & 12 & $(4,1)$ & 3 & 1 & YES & YES & YES & $1.57$ & $(2,3)$ & -- & 6945\\
$(181,70)$ & 11 & $(4,1)$ & 3 & 1 & YES & YES & YES & $1.57$ & $(2,3)$ & -- & 6946\\
$(181,75)$ & 11 & $(4,1)$ & 3 & 1 & YES & YES & YES & $1.43$ & $(2,3)$ & NO & 6947\\
$(181,75)$ & 11 & $(4,1)$ & 3 & 1 & YES & YES & YES & $1.43$ & $(2,3)$ & -- & 6948\\
$(181,42)$ & 13 & $(5,1)$ & 4 & 1 & YES & YES & YES & $1.71$ & $(2,3)$ & NO & 6949\\
$(181,42)$ & 13 & $(5,1)$ & 4 & 1 & YES & YES & YES & $1.71$ & $(2,3)$ & -- & 6950\\
$(181,55)$ & 12 & $(5,2)$ & 3 & 1 & YES & YES & NO(2) & $1.62$ & $(2,3)$ & -- & 6951\\
$(181,65)$ & 12 & $(5,1)$ & 4 & 1 & YES & YES & YES & $1.57$ & $(2,3)$ & NO & 6952\\
$(181,70)$ & 11 & $(5,2)$ & 3 & 1 & YES & YES & YES & $1.43$ & $(2,3)$ & NO & 6953\\
$(181,75)$ & 11 & $(5,2)$ & 3 & 1 & YES & YES & YES & $1.57$ & $(2,3)$ & -- & 6954\\
$(181,75)$ & 11 & $(5,2)$ & 3 & 1 & YES & YES & YES & $1.57$ & $(2,3)$ & 5596 & 6955\\
$(181,76)$ & 11 & $(5,2)$ & 3 & 1 & YES & YES & YES & $1.57$ & $(2,3)$ & -- & 6956\\
$(181,65)$ & 12 & $(6,1)$ & 5 & 1 & YES & YES & YES & $1.57$ & $(2,3)$ & NO & 6957\\
$(181,55)$ & 12 & $(7,2)$ & 4 & 1 & YES & YES & YES & $1.57$ & $(2,3)$ & -- & 6958\\
$(181,75)$ & 11 & $(7,2)$ & 4 & 1 & YES & YES & YES & $1.43$ & $(2,3)$ & NO & 6959\\
$(181,75)$ & 11 & $(7,3)$ & 4 & 1 & YES & YES & YES & $1.43$ & $(2,3)$ & NO & 6960\\
$(181,75)$ & 11 & $(9,2)$ & 5 & 1 & YES & YES & YES & $1.43$ & $(2,3)$ & NO & 6961\\
$(181,79)$ & 12 & $(9,4)$ & 5 & 1 & YES & YES & YES & $1.71$ & $(2,3)$ & NO & 6962\\
$(181,55)$ & 12 & $(10,3)$ & 5 & 1 & YES & YES & NO(2) & $1.50$ & $(2,3)$ & NO & 6963\\
$(181,75)$ & 11 & $(12,5)$ & 5 & 1 & YES & YES & YES & $1.43$ & $(2,3)$ & 6308 & 6964\\
$(181,55)$ & 12 & $(13,3)$ & 6 & 1 & YES & YES & YES & $1.57$ & $(2,3)$ & NO & 6965\\
$(181,70)$ & 11 & $(18,7)$ & 6 & 1 & YES & YES & YES & $1.43$ & $(2,3)$ & NO & 6966\\
$(181,75)$ & 11 & $(19,8)$ & 6 & 1 & YES & YES & YES & $1.57$ & $(2,3)$ & NO & 6967\\
$(181,53)$ & 12 & $(24,7)$ & 7 & 1 & YES & YES & YES & $1.57$ & $(2,3)$ & NO & 6968\\
$(181,65)$ & 12 & $(25,9)$ & 7 & 1 & YES & YES & YES & $1.57$ & $(2,3)$ & NO & 6969\\
$(181,53)$ & 12 & $(31,9)$ & 8 & 1 & YES & YES & YES & $1.57$ & $(2,3)$ & NO & 6970\\
$(181,55)$ & 12 & $(43,13)$ & 9 & 1 & YES & YES & YES & $1.57$ & $(2,3)$ & NO & 6971\\
$(181,70)$ & 11 & $(44,17)$ & 8 & 1 & YES & YES & YES & $1.71$ & $(2,3)$ & NO & 6972\\
$(181,65)$ & 12 & $(53,19)$ & 9 & 1 & YES & YES & YES & $1.71$ & $(2,3)$ & NO & 6973\\
$(181,65)$ & 12 & $(64,23)$ & 9 & 1 & YES & YES & YES & $1.57$ & $(2,3)$ & NO & 6974\\
$(181,55)$ & 12 & $(102,31)$ & 11 & 1 & YES & YES & YES & $1.57$ & $(2,3)$ & NO & 6975\\
$(181,65)$ & 12 & $(103,37)$ & 10 & 1 & YES & YES & YES & $1.57$ & $(2,3)$ & 7977 & 6976\\
$(181,70)$ & 11 & $(106,41)$ & 10 & 1 & YES & YES & YES & $1.57$ & $(2,3)$ & NO & 6977\\
$(181,65)$ & 12 & $(142,51)$ & 11 & 1 & YES & YES & YES & $1.57$ & $(2,3)$ & NO & 6978\\
$(181,75)$ & 11 & $(152,63)$ & 11 & 1 & YES & YES & YES & $1.57$ & $(2,3)$ & NO & 6979\\
$(181,65)$ & 12 & $(181,65)$ & 12 & 181 & YES & YES & YES & $1.57$ & $(2,3)$ & NO & 6980\\
$(181,70)$ & 11 & $(181,70)$ & 11 & 181 & YES & YES & YES & $1.57$ & $(2,3)$ & NO & 6981\\
$(182,55)$ & 13 & $(3,1)$ & 2 & 1 & YES & YES & NO(2) & $1.62$ & $(2,3)$ & -- & 6982\\
$(182,53)$ & 12 & $(4,1)$ & 3 & 2 & YES & YES & YES & $1.71$ & $(2,3)$ & -- & 6983\\
$(182,55)$ & 13 & $(4,1)$ & 3 & 2 & YES & YES & NO(2) & $1.62$ & $(2,3)$ & -- & 6984\\
$(182,71)$ & 12 & $(4,1)$ & 3 & 2 & YES & YES & NO(2) & $1.62$ & $(2,3)$ & -- & 6985\\
$(182,53)$ & 12 & $(5,2)$ & 3 & 1 & YES & YES & YES & $1.57$ & $(2,3)$ & -- & 6986\\
$(182,55)$ & 13 & $(5,1)$ & 4 & 1 & YES & YES & NO(2) & $1.62$ & $(2,3)$ & -- & 6987\\
$(182,71)$ & 12 & $(5,1)$ & 4 & 1 & YES & YES & NO(2) & $1.62$ & $(2,3)$ & NO & 6988\\
$(182,71)$ & 12 & $(5,2)$ & 3 & 1 & YES & YES & YES & $1.71$ & $(2,3)$ & NO & 6989\\
$(182,71)$ & 12 & $(8,3)$ & 4 & 2 & YES & YES & YES & $1.71$ & $(2,3)$ & NO & 6990\\
$(182,53)$ & 12 & $(9,2)$ & 5 & 1 & YES & YES & YES & $1.43$ & $(2,3)$ & -- & 6991\\
$(182,71)$ & 12 & $(13,5)$ & 5 & 13 & YES & YES & YES & $1.71$ & $(2,3)$ & NO & 6992\\
$(182,79)$ & 12 & $(30,13)$ & 8 & 2 & YES & YES & YES & $1.57$ & $(2,3)$ & NO & 6993\\
$(182,53)$ & 12 & $(31,9)$ & 8 & 1 & YES & YES & YES & $1.71$ & $(2,3)$ & 5388 & 6994\\
$(182,55)$ & 13 & $(33,10)$ & 8 & 1 & YES & YES & NO(2) & $1.62$ & $(2,3)$ & NO & 6995\\
$(182,55)$ & 13 & $(96,29)$ & 11 & 2 & YES & YES & NO(2) & $1.62$ & $(2,3)$ & 7848 & 6996\\
$(182,71)$ & 12 & $(100,39)$ & 10 & 2 & YES & YES & NO(2) & $1.62$ & $(2,3)$ & 7934 & 6997\\
$(182,53)$ & 12 & $(127,37)$ & 12 & 1 & YES & YES & YES & $1.57$ & $(2,3)$ & NO & 6998\\
$(183,67)$ & 11 & $(2,1)$ & 1 & 1 & YES & YES & YES & $1.43$ & $(2,3)$ & -- & 6999\\
$(183,67)$ & 11 & $(2,1)$ & 1 & 1 & YES & YES & YES & $1.57$ & $(2,3)$ & NO & 7000\\
$(183,67)$ & 11 & $(3,1)$ & 2 & 3 & YES & YES & YES & $1.43$ & $(2,3)$ & -- & 7001\\
$(183,71)$ & 11 & $(3,1)$ & 2 & 3 & YES & YES & YES & $1.57$ & $(2,3)$ & -- & 7002\\
$(183,71)$ & 11 & $(8,3)$ & 4 & 1 & YES & YES & YES & $1.57$ & $(2,3)$ & NO & 7003\\
$(183,67)$ & 11 & $(9,2)$ & 5 & 3 & YES & YES & YES & $1.43$ & $(2,3)$ & -- & 7004\\
$(183,71)$ & 11 & $(11,4)$ & 5 & 1 & YES & YES & YES & $1.43$ & $(2,3)$ & NO & 7005\\
$(183,65)$ & 12 & $(17,6)$ & 7 & 1 & YES & YES & YES & $1.43$ & $(2,3)$ & NO & 7006\\
$(183,71)$ & 11 & $(18,7)$ & 6 & 3 & YES & YES & YES & $1.57$ & $(2,3)$ & NO & 7007\\
$(183,67)$ & 11 & $(30,11)$ & 7 & 3 & YES & YES & YES & $1.57$ & $(2,3)$ & NO & 7008\\
$(183,71)$ & 11 & $(31,12)$ & 7 & 1 & YES & YES & YES & $1.43$ & $(2,3)$ & 7528 & 7009\\
$(183,67)$ & 11 & $(71,26)$ & 9 & 1 & YES & YES & YES & $1.43$ & $(2,3)$ & NO & 7010\\
$(183,65)$ & 12 & $(107,38)$ & 11 & 1 & YES & YES & YES & $1.43$ & $(2,3)$ & NO & 7011\\
$(183,67)$ & 11 & $(112,41)$ & 10 & 1 & YES & YES & YES & $1.43$ & $(2,3)$ & NO & 7012\\
$(183,56)$ & 12 & $(121,37)$ & 11 & 1 & YES & YES & YES & $1.57$ & $(2,3)$ & 8703 & 7013\\
$(183,67)$ & 11 & $(183,67)$ & 11 & 183 & YES & YES & YES & $1.29$ & $(2,3)$ & NO & 7014\\
$(184,57)$ & 12 & $(10,3)$ & 5 & 2 & YES & YES & YES & $1.43$ & $(2,3)$ & NO & 7015\\
$(184,43)$ & 12 & $(13,3)$ & 6 & 1 & YES & YES & YES & $1.57$ & $(2,3)$ & NO & 7016\\
$(184,71)$ & 12 & $(49,19)$ & 8 & 1 & YES & YES & YES & $1.71$ & $(2,3)$ & NO & 7017\\
$(184,83)$ & 12 & $(82,37)$ & 10 & 2 & YES & YES & YES & $1.71$ & $(2,3)$ & 7619 & 7018\\
$(184,77)$ & 12 & $(141,59)$ & 11 & 1 & YES & YES & YES & $1.71$ & $(2,3)$ & NO & 7019\\
$(185,56)$ & 12 & $(3,1)$ & 2 & 1 & NO & YES & NO(2) & $1.67$ & $(2,3)$ & -- & 7020\\
$(185,58)$ & 13 & $(3,1)$ & 2 & 1 & YES & YES & YES & $1.71$ & $(2,3)$ & -- & 7021\\
$(185,76)$ & 12 & $(3,1)$ & 2 & 1 & YES & YES & NO(2) & $1.62$ & $(2,3)$ & -- & 7022\\
$(185,83)$ & 12 & $(3,1)$ & 2 & 1 & YES & YES & YES & $1.57$ & $(2,3)$ & -- & 7023\\
$(185,83)$ & 12 & $(3,1)$ & 2 & 1 & YES & YES & YES & $1.71$ & $(2,3)$ & NO & 7024\\
$(185,58)$ & 13 & $(4,1)$ & 3 & 1 & YES & YES & YES & $1.57$ & $(2,3)$ & NO & 7025\\
$(185,76)$ & 12 & $(4,1)$ & 3 & 1 & YES & YES & NO(2) & $1.62$ & $(2,3)$ & -- & 7026\\
$(185,43)$ & 13 & $(5,2)$ & 3 & 5 & YES & YES & YES & $1.43$ & $(2,3)$ & -- & 7027\\
$(185,56)$ & 12 & $(5,2)$ & 3 & 5 & YES & YES & NO(2) & $1.62$ & $(2,3)$ & -- & 7028\\
$(185,68)$ & 11 & $(5,2)$ & 3 & 5 & YES & YES & YES & $1.43$ & $(2,3)$ & -- & 7029\\
$(185,83)$ & 12 & $(5,2)$ & 3 & 5 & YES & YES & YES & $1.57$ & $(2,3)$ & NO & 7030\\
$(185,76)$ & 12 & $(7,3)$ & 4 & 1 & YES & YES & YES & $1.86$ & $(2,3)$ & -- & 7031\\
$(185,76)$ & 12 & $(7,3)$ & 4 & 1 & YES & YES & NO(2) & $1.75$ & $(2,3)$ & NO & 7032\\
$(185,76)$ & 12 & $(11,4)$ & 5 & 1 & YES & YES & YES & $1.71$ & $(2,3)$ & NO & 7033\\
$(185,76)$ & 12 & $(12,5)$ & 5 & 1 & YES & YES & NO(2) & $1.62$ & $(2,3)$ & NO & 7034\\
$(185,58)$ & 13 & $(13,4)$ & 6 & 1 & YES & YES & YES & $1.57$ & $(2,3)$ & NO & 7035\\
$(185,43)$ & 13 & $(16,3)$ & 7 & 1 & YES & YES & YES & $1.57$ & $(2,3)$ & NO & 7036\\
$(185,76)$ & 12 & $(29,12)$ & 7 & 1 & YES & YES & YES & $1.71$ & $(2,3)$ & NO & 7037\\
$(185,43)$ & 13 & $(47,11)$ & 9 & 1 & YES & YES & YES & $1.43$ & $(2,3)$ & NO & 7038\\
$(185,33)$ & 14 & $(62,11)$ & 10 & 1 & YES & YES & YES & $1.43$ & $(2,3)$ & NO & 7039\\
$(185,76)$ & 12 & $(73,30)$ & 10 & 1 & YES & YES & NO(2) & $1.62$ & $(2,3)$ & 7371 & 7040\\
$(185,58)$ & 13 & $(83,26)$ & 12 & 1 & YES & YES & YES & $1.71$ & $(2,3)$ & NO & 7041\\
$(185,76)$ & 12 & $(90,37)$ & 11 & 5 & YES & YES & YES & $1.57$ & $(2,3)$ & 8675 & 7042\\
$(185,83)$ & 12 & $(107,48)$ & 11 & 1 & YES & YES & YES & $1.57$ & $(2,3)$ & NO & 7043\\
$(185,43)$ & 13 & $(125,29)$ & 12 & 5 & YES & YES & YES & $1.57$ & $(2,3)$ & 9050 & 7044\\
$(185,76)$ & 12 & $(129,53)$ & 11 & 1 & YES & YES & NO(2) & $1.50$ & $(2,3)$ & NO & 7045\\
$(185,58)$ & 13 & $(185,58)$ & 13 & 185 & YES & YES & YES & $1.71$ & $(2,3)$ & NO & 7046\\
$(185,83)$ & 12 & $(185,83)$ & 12 & 185 & YES & YES & YES & $1.57$ & $(2,3)$ & NO & 7047\\
$(186,55)$ & 11 & $(2,1)$ & 1 & 2 & YES & YES & YES & $1.43$ & $(2,3)$ & NO & 7048\\
$(186,71)$ & 11 & $(2,1)$ & 1 & 2 & YES & YES & YES & $1.57$ & $(2,3)$ & -- & 7049\\
$(186,71)$ & 11 & $(5,2)$ & 3 & 1 & YES & YES & YES & $1.57$ & $(2,3)$ & NO & 7050\\
$(186,71)$ & 11 & $(7,2)$ & 4 & 1 & YES & YES & YES & $1.43$ & $(2,3)$ & NO & 7051\\
$(186,55)$ & 11 & $(10,3)$ & 5 & 2 & YES & YES & YES & $1.43$ & $(2,3)$ & NO & 7052\\
$(186,55)$ & 11 & $(71,21)$ & 9 & 1 & YES & YES & YES & $1.43$ & $(2,3)$ & NO & 7053\\
$(186,71)$ & 11 & $(97,37)$ & 10 & 1 & YES & YES & YES & $1.43$ & $(2,3)$ & 8702 & 7054\\
$(187,79)$ & 11 & $(2,1)$ & 1 & 1 & YES & YES & YES & $1.43$ & $(2,3)$ & -- & 7055\\
$(187,82)$ & 12 & $(2,1)$ & 1 & 1 & YES & YES & NO(2) & $1.75$ & $(2,3)$ & -- & 7056\\
$(187,71)$ & 11 & $(3,1)$ & 2 & 1 & YES & YES & YES & $1.57$ & $(2,3)$ & NO & 7057\\
$(187,71)$ & 11 & $(3,1)$ & 2 & 1 & YES & YES & YES & $1.57$ & $(2,3)$ & -- & 7058\\
$(187,79)$ & 11 & $(3,1)$ & 2 & 1 & YES & YES & NO(2) & $1.50$ & $(2,3)$ & NO & 7059\\
$(187,79)$ & 11 & $(3,1)$ & 2 & 1 & YES & YES & YES & $1.43$ & $(2,3)$ & NO & 7060\\
$(187,79)$ & 11 & $(3,1)$ & 2 & 1 & YES & YES & YES & $1.43$ & $(2,3)$ & -- & 7061\\
$(187,71)$ & 11 & $(4,1)$ & 3 & 1 & YES & YES & YES & $1.57$ & $(2,3)$ & NO & 7062\\
$(187,71)$ & 11 & $(4,1)$ & 3 & 1 & YES & YES & YES & $1.57$ & $(2,3)$ & -- & 7063\\
$(187,73)$ & 12 & $(4,1)$ & 3 & 1 & YES & YES & YES & $1.57$ & $(2,3)$ & -- & 7064\\
$(187,84)$ & 12 & $(4,1)$ & 3 & 1 & YES & YES & YES & $1.43$ & $(2,3)$ & NO & 7065\\
$(187,57)$ & 12 & $(5,2)$ & 3 & 1 & YES & YES & YES & $1.57$ & $(2,3)$ & -- & 7066\\
$(187,71)$ & 11 & $(5,2)$ & 3 & 1 & YES & YES & YES & $1.57$ & $(2,3)$ & -- & 7067\\
$(187,71)$ & 11 & $(5,2)$ & 3 & 1 & YES & YES & YES & $1.71$ & $(2,3)$ & NO & 7068\\
$(187,79)$ & 11 & $(5,1)$ & 4 & 1 & YES & YES & YES & $1.43$ & $(2,3)$ & NO & 7069\\
$(187,79)$ & 11 & $(5,2)$ & 3 & 1 & YES & YES & YES & $1.43$ & $(2,3)$ & -- & 7070\\
$(187,71)$ & 11 & $(7,2)$ & 4 & 1 & YES & YES & YES & $1.57$ & $(2,3)$ & -- & 7071\\
$(187,79)$ & 11 & $(7,2)$ & 4 & 1 & YES & YES & YES & $1.43$ & $(2,3)$ & -- & 7072\\
$(187,82)$ & 12 & $(9,4)$ & 5 & 1 & YES & YES & YES & $1.57$ & $(2,3)$ & 4799 & 7073\\
$(187,84)$ & 12 & $(11,5)$ & 6 & 11 & YES & YES & YES & $1.71$ & $(2,3)$ & NO & 7074\\
$(187,73)$ & 12 & $(13,5)$ & 5 & 1 & YES & YES & YES & $1.57$ & $(2,3)$ & NO & 7075\\
$(187,79)$ & 11 & $(17,7)$ & 6 & 17 & YES & YES & YES & $1.43$ & $(2,3)$ & 8618 & 7076\\
$(187,79)$ & 11 & $(19,8)$ & 6 & 1 & YES & YES & NO(2) & $1.50$ & $(2,3)$ & 6802 & 7077\\
$(187,71)$ & 11 & $(29,11)$ & 7 & 1 & YES & YES & YES & $1.71$ & $(2,3)$ & 6135 & 7078\\
$(187,79)$ & 11 & $(31,13)$ & 7 & 1 & YES & YES & YES & $1.43$ & $(2,3)$ & NO & 7079\\
$(187,79)$ & 11 & $(64,27)$ & 9 & 1 & YES & YES & YES & $1.57$ & $(2,3)$ & NO & 7080\\
$(187,71)$ & 11 & $(71,27)$ & 9 & 1 & YES & YES & YES & $1.57$ & $(2,3)$ & NO & 7081\\
$(187,79)$ & 11 & $(71,30)$ & 9 & 1 & YES & YES & YES & $1.57$ & $(2,3)$ & NO & 7082\\
$(187,79)$ & 11 & $(97,41)$ & 10 & 1 & YES & YES & YES & $1.57$ & $(2,3)$ & NO & 7083\\
$(187,71)$ & 11 & $(129,49)$ & 10 & 1 & YES & YES & YES & $1.71$ & $(2,3)$ & NO & 7084\\
$(187,79)$ & 11 & $(187,79)$ & 11 & 187 & YES & YES & YES & $1.43$ & $(2,3)$ & NO & 7085\\
$(187,84)$ & 12 & $(187,84)$ & 12 & 187 & YES & YES & YES & $1.57$ & $(2,3)$ & NO & 7086\\
$(188,55)$ & 12 & $(2,1)$ & 1 & 2 & YES & YES & YES & $1.43$ & $(2,3)$ & -- & 7087\\
$(188,69)$ & 11 & $(2,1)$ & 1 & 2 & NO & YES & YES & $1.57$ & $(2,3)$ & -- & 7088\\
$(188,79)$ & 11 & $(2,1)$ & 1 & 2 & YES & YES & YES & $1.57$ & $(2,3)$ & -- & 7089\\
$(188,79)$ & 11 & $(2,1)$ & 1 & 2 & YES & YES & YES & $1.43$ & $(2,3)$ & NO & 7090\\
$(188,55)$ & 12 & $(3,1)$ & 2 & 1 & YES & YES & YES & $1.43$ & $(2,3)$ & -- & 7091\\
$(188,79)$ & 11 & $(3,1)$ & 2 & 1 & YES & YES & YES & $1.43$ & $(2,3)$ & -- & 7092\\
$(188,59)$ & 13 & $(4,1)$ & 3 & 4 & YES & YES & YES & $1.57$ & $(2,3)$ & NO & 7093\\
$(188,59)$ & 13 & $(4,1)$ & 3 & 4 & YES & YES & YES & $1.71$ & $(2,3)$ & -- & 7094\\
$(188,55)$ & 12 & $(5,1)$ & 4 & 1 & YES & YES & YES & $1.43$ & $(2,3)$ & NO & 7095\\
$(188,55)$ & 12 & $(5,2)$ & 3 & 1 & YES & YES & YES & $1.57$ & $(2,3)$ & NO & 7096\\
$(188,55)$ & 12 & $(5,2)$ & 3 & 1 & YES & YES & YES & $1.57$ & $(2,3)$ & -- & 7097\\
$(188,57)$ & 13 & $(5,1)$ & 4 & 1 & YES & YES & YES & $1.71$ & $(2,3)$ & NO & 7098\\
$(188,69)$ & 11 & $(5,2)$ & 3 & 1 & YES & YES & YES & $1.57$ & $(2,3)$ & NO & 7099\\
$(188,79)$ & 11 & $(5,2)$ & 3 & 1 & YES & YES & YES & $1.43$ & $(2,3)$ & NO & 7100\\
$(188,55)$ & 12 & $(6,1)$ & 5 & 2 & YES & YES & YES & $1.43$ & $(2,3)$ & NO & 7101\\
$(188,55)$ & 12 & $(7,2)$ & 4 & 1 & YES & YES & YES & $1.43$ & $(2,3)$ & -- & 7102\\
$(188,59)$ & 13 & $(7,2)$ & 4 & 1 & YES & YES & YES & $1.57$ & $(2,3)$ & NO & 7103\\
$(188,55)$ & 12 & $(9,2)$ & 5 & 1 & YES & YES & YES & $1.57$ & $(2,3)$ & NO & 7104\\
$(188,59)$ & 13 & $(10,3)$ & 5 & 2 & YES & YES & YES & $1.57$ & $(2,3)$ & NO & 7105\\
$(188,55)$ & 12 & $(11,3)$ & 5 & 1 & YES & YES & YES & $1.57$ & $(2,3)$ & NO & 7106\\
$(188,59)$ & 13 & $(13,4)$ & 6 & 1 & YES & YES & YES & $1.57$ & $(2,3)$ & NO & 7107\\
$(188,69)$ & 11 & $(13,5)$ & 5 & 1 & YES & YES & YES & $1.71$ & $(2,3)$ & NO & 7108\\
$(188,69)$ & 11 & $(19,7)$ & 6 & 1 & YES & YES & NO(2) & $1.62$ & $(2,3)$ & 5557 & 7109\\
$(188,69)$ & 11 & $(27,10)$ & 7 & 1 & YES & YES & YES & $1.71$ & $(2,3)$ & NO & 7110\\
$(188,55)$ & 12 & $(31,9)$ & 8 & 1 & YES & YES & YES & $1.57$ & $(2,3)$ & NO & 7111\\
$(188,79)$ & 11 & $(50,21)$ & 8 & 2 & YES & YES & YES & $1.43$ & $(2,3)$ & 6726 & 7112\\
$(188,55)$ & 12 & $(65,19)$ & 9 & 1 & YES & YES & YES & $1.43$ & $(2,3)$ & NO & 7113\\
$(188,55)$ & 12 & $(89,26)$ & 10 & 1 & YES & YES & YES & $1.57$ & $(2,3)$ & NO & 7114\\
$(188,57)$ & 13 & $(89,27)$ & 10 & 1 & YES & YES & YES & $1.57$ & $(2,3)$ & NO & 7115\\
$(188,55)$ & 12 & $(106,31)$ & 10 & 2 & YES & YES & YES & $1.43$ & $(2,3)$ & 8092 & 7116\\
$(188,57)$ & 13 & $(122,37)$ & 11 & 2 & YES & YES & YES & $1.71$ & $(2,3)$ & 8402 & 7117\\
$(188,59)$ & 13 & $(137,43)$ & 12 & 1 & YES & YES & YES & $1.71$ & $(2,3)$ & NO & 7118\\
$(188,57)$ & 13 & $(155,47)$ & 12 & 1 & YES & YES & YES & $1.71$ & $(2,3)$ & NO & 7119\\
$(188,55)$ & 12 & $(188,55)$ & 12 & 188 & YES & YES & YES & $1.43$ & $(2,3)$ & NO & 7120\\
$(189,82)$ & 12 & $(2,1)$ & 1 & 1 & YES & YES & NO(2) & $1.62$ & $(2,3)$ & -- & 7121\\
$(189,83)$ & 12 & $(4,1)$ & 3 & 1 & YES & YES & YES & $1.71$ & $(2,3)$ & -- & 7122\\
$(189,55)$ & 12 & $(5,2)$ & 3 & 1 & YES & YES & YES & $1.71$ & $(2,3)$ & -- & 7123\\
$(189,73)$ & 12 & $(5,1)$ & 4 & 1 & YES & YES & YES & $1.71$ & $(2,3)$ & -- & 7124\\
$(189,73)$ & 12 & $(5,1)$ & 4 & 1 & YES & YES & YES & $1.71$ & $(2,3)$ & NO & 7125\\
$(189,79)$ & 12 & $(5,1)$ & 4 & 1 & YES & YES & YES & $1.71$ & $(2,3)$ & NO & 7126\\
$(189,79)$ & 12 & $(5,1)$ & 4 & 1 & YES & YES & YES & $1.71$ & $(2,3)$ & -- & 7127\\
$(189,79)$ & 12 & $(5,2)$ & 3 & 1 & YES & YES & YES & $1.57$ & $(2,3)$ & NO & 7128\\
$(189,40)$ & 12 & $(7,2)$ & 4 & 7 & YES & YES & YES & $1.29$ & $(2,3)$ & NO & 7129\\
$(189,44)$ & 12 & $(7,3)$ & 4 & 7 & YES & YES & YES & $1.57$ & $(2,3)$ & NO & 7130\\
$(189,53)$ & 12 & $(11,3)$ & 5 & 1 & YES & YES & NO(2) & $1.62$ & $(2,3)$ & NO & 7131\\
$(189,40)$ & 12 & $(13,3)$ & 6 & 1 & YES & YES & YES & $1.43$ & $(2,3)$ & NO & 7132\\
$(189,67)$ & 12 & $(14,5)$ & 6 & 7 & YES & YES & YES & $1.43$ & $(2,3)$ & NO & 7133\\
$(189,55)$ & 12 & $(41,12)$ & 8 & 1 & YES & YES & YES & $1.57$ & $(2,3)$ & NO & 7134\\
$(189,82)$ & 12 & $(53,23)$ & 9 & 1 & YES & YES & NO(2) & $1.62$ & $(2,3)$ & NO & 7135\\
$(189,79)$ & 12 & $(67,28)$ & 10 & 1 & YES & YES & YES & $1.71$ & $(2,3)$ & NO & 7136\\
$(189,82)$ & 12 & $(76,33)$ & 10 & 1 & YES & YES & YES & $1.71$ & $(2,3)$ & NO & 7137\\
$(189,44)$ & 12 & $(77,18)$ & 10 & 7 & YES & YES & YES & $1.43$ & $(2,3)$ & NO & 7138\\
$(189,83)$ & 12 & $(148,65)$ & 11 & 1 & YES & YES & YES & $1.71$ & $(2,3)$ & NO & 7139\\
$(190,43)$ & 12 & $(5,2)$ & 3 & 5 & YES & YES & NO(3) & $1.29$ & $(2,3)$ & NO & 7140\\
$(190,43)$ & 12 & $(5,2)$ & 3 & 5 & YES & YES & YES & $1.57$ & $(2,3)$ & -- & 7141\\
$(190,41)$ & 12 & $(7,3)$ & 4 & 1 & YES & YES & YES & $1.43$ & $(2,3)$ & NO & 7142\\
$(190,41)$ & 12 & $(7,3)$ & 4 & 1 & YES & YES & YES & $1.57$ & $(2,3)$ & -- & 7143\\
$(190,43)$ & 12 & $(7,2)$ & 4 & 1 & YES & YES & YES & $1.43$ & $(2,3)$ & NO & 7144\\
$(190,41)$ & 12 & $(11,3)$ & 5 & 1 & YES & YES & YES & $1.43$ & $(2,3)$ & NO & 7145\\
$(190,43)$ & 12 & $(17,4)$ & 7 & 1 & YES & YES & YES & $1.57$ & $(2,3)$ & NO & 7146\\
$(190,53)$ & 12 & $(25,7)$ & 7 & 5 & YES & YES & YES & $1.57$ & $(2,3)$ & NO & 7147\\
$(190,41)$ & 12 & $(41,9)$ & 9 & 1 & YES & YES & YES & $1.57$ & $(2,3)$ & NO & 7148\\
$(191,80)$ & 11 & $(2,1)$ & 1 & 1 & YES & YES & YES & $1.29$ & $(2,3)$ & -- & 7149\\
$(191,80)$ & 11 & $(2,1)$ & 1 & 1 & YES & YES & YES & $1.43$ & $(2,3)$ & NO & 7150\\
$(191,50)$ & 13 & $(3,1)$ & 2 & 1 & YES & YES & YES & $1.57$ & $(2,3)$ & -- & 7151\\
$(191,50)$ & 13 & $(3,1)$ & 2 & 1 & YES & YES & YES & $1.57$ & $(2,3)$ & NO & 7152\\
$(191,56)$ & 12 & $(3,1)$ & 2 & 1 & NO & YES & YES & $1.57$ & $(2,3)$ & -- & 7153\\
$(191,80)$ & 11 & $(3,1)$ & 2 & 1 & YES & YES & YES & $1.29$ & $(2,3)$ & -- & 7154\\
$(191,80)$ & 11 & $(3,1)$ & 2 & 1 & YES & YES & YES & $1.43$ & $(2,3)$ & NO & 7155\\
$(191,80)$ & 11 & $(4,1)$ & 3 & 1 & YES & YES & YES & $1.43$ & $(2,3)$ & -- & 7156\\
$(191,35)$ & 14 & $(5,2)$ & 3 & 1 & YES & YES & YES & $1.57$ & $(2,3)$ & NO & 7157\\
$(191,50)$ & 13 & $(5,2)$ & 3 & 1 & YES & YES & YES & $1.57$ & $(2,3)$ & -- & 7158\\
$(191,69)$ & 13 & $(5,2)$ & 3 & 1 & YES & YES & YES & $1.86$ & $(2,3)$ & -- & 7159\\
$(191,74)$ & 11 & $(5,2)$ & 3 & 1 & YES & YES & YES & $1.57$ & $(2,3)$ & -- & 7160\\
$(191,80)$ & 11 & $(5,2)$ & 3 & 1 & YES & YES & YES & $1.57$ & $(2,3)$ & -- & 7161\\
$(191,35)$ & 14 & $(7,2)$ & 4 & 1 & YES & YES & YES & $1.57$ & $(2,3)$ & -- & 7162\\
$(191,80)$ & 11 & $(7,3)$ & 4 & 1 & YES & YES & YES & $1.43$ & $(2,3)$ & 5700 & 7163\\
$(191,80)$ & 11 & $(8,3)$ & 4 & 1 & YES & YES & YES & $1.71$ & $(2,3)$ & NO & 7164\\
$(191,50)$ & 13 & $(15,4)$ & 6 & 1 & YES & YES & YES & $1.57$ & $(2,3)$ & NO & 7165\\
$(191,80)$ & 11 & $(17,7)$ & 6 & 1 & YES & YES & YES & $1.57$ & $(2,3)$ & NO & 7166\\
$(191,78)$ & 12 & $(27,11)$ & 8 & 1 & YES & YES & YES & $1.43$ & $(2,3)$ & 7343 & 7167\\
$(191,56)$ & 12 & $(31,9)$ & 8 & 1 & YES & YES & YES & $1.43$ & $(2,3)$ & NO & 7168\\
$(191,74)$ & 11 & $(31,12)$ & 7 & 1 & YES & YES & NO(2) & $1.62$ & $(2,3)$ & 6234 & 7169\\
$(191,74)$ & 11 & $(44,17)$ & 8 & 1 & YES & YES & YES & $1.57$ & $(2,3)$ & NO & 7170\\
$(191,80)$ & 11 & $(55,23)$ & 9 & 1 & YES & YES & YES & $1.57$ & $(2,3)$ & NO & 7171\\
$(191,50)$ & 13 & $(61,16)$ & 10 & 1 & YES & YES & YES & $1.57$ & $(2,3)$ & NO & 7172\\
$(191,80)$ & 11 & $(117,49)$ & 10 & 1 & YES & YES & YES & $1.29$ & $(2,3)$ & NO & 7173\\
$(191,80)$ & 11 & $(191,80)$ & 11 & 191 & YES & YES & YES & $1.29$ & $(2,3)$ & NO & 7174\\
$(192,71)$ & 11 & $(2,1)$ & 1 & 2 & YES & YES & YES & $1.57$ & $(2,3)$ & NO & 7175\\
$(192,73)$ & 11 & $(2,1)$ & 1 & 2 & NO & YES & YES & $1.57$ & $(2,3)$ & -- & 7176\\
$(192,73)$ & 11 & $(3,1)$ & 2 & 3 & YES & YES & YES & $1.57$ & $(2,3)$ & NO & 7177\\
$(192,73)$ & 11 & $(3,1)$ & 2 & 3 & YES & YES & YES & $1.57$ & $(2,3)$ & -- & 7178\\
$(192,71)$ & 11 & $(5,2)$ & 3 & 1 & YES & YES & YES & $1.57$ & $(2,3)$ & NO & 7179\\
$(192,73)$ & 11 & $(5,2)$ & 3 & 1 & YES & YES & YES & $1.57$ & $(2,3)$ & -- & 7180\\
$(192,73)$ & 11 & $(9,2)$ & 5 & 3 & YES & YES & YES & $1.43$ & $(2,3)$ & -- & 7181\\
$(192,73)$ & 11 & $(9,2)$ & 5 & 3 & YES & YES & YES & $1.43$ & $(2,3)$ & NO & 7182\\
$(192,73)$ & 11 & $(34,13)$ & 7 & 2 & YES & YES & YES & $1.43$ & $(2,3)$ & NO & 7183\\
$(192,73)$ & 11 & $(50,19)$ & 8 & 2 & YES & YES & YES & $1.43$ & $(2,3)$ & 6756 & 7184\\
$(192,73)$ & 11 & $(79,30)$ & 9 & 1 & YES & YES & YES & $1.43$ & $(2,3)$ & NO & 7185\\
$(192,73)$ & 11 & $(92,35)$ & 10 & 4 & YES & YES & YES & $1.43$ & $(2,3)$ & NO & 7186\\
$(192,73)$ & 11 & $(171,65)$ & 11 & 3 & YES & YES & YES & $1.57$ & $(2,3)$ & NO & 7187\\
$(192,73)$ & 11 & $(192,73)$ & 11 & 192 & YES & YES & YES & $1.57$ & $(2,3)$ & NO & 7188\\
$(193,74)$ & 12 & $(2,1)$ & 1 & 1 & YES & YES & YES & $1.71$ & $(2,3)$ & -- & 7189\\
$(193,74)$ & 12 & $(2,1)$ & 1 & 1 & YES & YES & YES & $1.71$ & $(2,3)$ & NO & 7190\\
$(193,81)$ & 11 & $(2,1)$ & 1 & 1 & YES & YES & YES & $1.43$ & $(2,3)$ & NO & 7191\\
$(193,81)$ & 11 & $(2,1)$ & 1 & 1 & YES & YES & YES & $1.43$ & $(2,3)$ & -- & 7192\\
$(193,74)$ & 12 & $(3,1)$ & 2 & 1 & YES & YES & NO(2) & $1.75$ & $(2,3)$ & NO & 7193\\
$(193,74)$ & 12 & $(3,1)$ & 2 & 1 & YES & YES & YES & $1.57$ & $(2,3)$ & -- & 7194\\
$(193,53)$ & 12 & $(4,1)$ & 3 & 1 & YES & YES & YES & $1.57$ & $(2,3)$ & -- & 7195\\
$(193,73)$ & 13 & $(4,1)$ & 3 & 1 & YES & YES & YES & $1.57$ & $(2,3)$ & -- & 7196\\
$(193,85)$ & 12 & $(4,1)$ & 3 & 1 & YES & YES & YES & $1.57$ & $(2,3)$ & -- & 7197\\
$(193,44)$ & 12 & $(5,2)$ & 3 & 1 & YES & YES & NO(3) & $1.29$ & $(2,3)$ & NO & 7198\\
$(193,59)$ & 13 & $(5,1)$ & 4 & 1 & YES & YES & YES & $1.43$ & $(2,3)$ & -- & 7199\\
$(193,81)$ & 11 & $(5,2)$ & 3 & 1 & YES & YES & YES & $1.57$ & $(2,3)$ & NO & 7200\\
$(193,80)$ & 12 & $(6,1)$ & 5 & 1 & YES & YES & YES & $1.43$ & $(2,3)$ & NO & 7201\\
$(193,87)$ & 12 & $(9,4)$ & 5 & 1 & YES & YES & YES & $1.57$ & $(2,3)$ & NO & 7202\\
$(193,80)$ & 12 & $(12,5)$ & 5 & 1 & YES & YES & NO(2) & $1.62$ & $(2,3)$ & NO & 7203\\
$(193,74)$ & 12 & $(13,5)$ & 5 & 1 & YES & YES & YES & $1.71$ & $(2,3)$ & NO & 7204\\
$(193,85)$ & 12 & $(16,7)$ & 6 & 1 & YES & YES & YES & $1.71$ & $(2,3)$ & NO & 7205\\
$(193,81)$ & 11 & $(17,7)$ & 6 & 1 & YES & YES & YES & $1.57$ & $(2,3)$ & NO & 7206\\
$(193,74)$ & 12 & $(21,8)$ & 6 & 1 & YES & YES & YES & $1.71$ & $(2,3)$ & 5140 & 7207\\
$(193,57)$ & 12 & $(61,18)$ & 9 & 1 & YES & YES & YES & $1.43$ & $(2,3)$ & NO & 7208\\
$(193,87)$ & 12 & $(71,32)$ & 10 & 1 & YES & YES & YES & $1.57$ & $(2,3)$ & NO & 7209\\
$(193,85)$ & 12 & $(109,48)$ & 11 & 1 & YES & YES & YES & $1.71$ & $(2,3)$ & NO & 7210\\
$(193,80)$ & 12 & $(111,46)$ & 10 & 1 & YES & YES & YES & $1.43$ & $(2,3)$ & 8221 & 7211\\
$(193,53)$ & 12 & $(142,39)$ & 11 & 1 & YES & YES & YES & $1.57$ & $(2,3)$ & NO & 7212\\
$(193,73)$ & 13 & $(156,59)$ & 12 & 1 & YES & YES & YES & $1.57$ & $(2,3)$ & NO & 7213\\
$(193,80)$ & 12 & $(193,80)$ & 12 & 193 & YES & YES & YES & $1.57$ & $(2,3)$ & NO & 7214\\
$(193,85)$ & 12 & $(193,85)$ & 12 & 193 & YES & YES & YES & $1.43$ & $(2,3)$ & NO & 7215\\
$(193,87)$ & 12 & $(193,87)$ & 12 & 193 & YES & YES & YES & $1.43$ & $(2,3)$ & NO & 7216\\
$(194,75)$ & 11 & $(2,1)$ & 1 & 2 & YES & YES & YES & $1.57$ & $(2,3)$ & -- & 7217\\
$(194,75)$ & 11 & $(3,1)$ & 2 & 1 & YES & YES & NO(3) & $1.29$ & $(2,3)$ & -- & 7218\\
$(194,75)$ & 11 & $(3,1)$ & 2 & 1 & YES & YES & YES & $1.43$ & $(2,3)$ & NO & 7219\\
$(194,75)$ & 11 & $(4,1)$ & 3 & 2 & YES & YES & YES & $1.71$ & $(2,3)$ & NO & 7220\\
$(194,75)$ & 11 & $(4,1)$ & 3 & 2 & YES & YES & YES & $1.71$ & $(2,3)$ & -- & 7221\\
$(194,45)$ & 13 & $(5,2)$ & 3 & 1 & YES & YES & YES & $1.57$ & $(2,3)$ & NO & 7222\\
$(194,75)$ & 11 & $(5,2)$ & 3 & 1 & YES & YES & YES & $1.57$ & $(2,3)$ & -- & 7223\\
$(194,75)$ & 11 & $(5,2)$ & 3 & 1 & YES & YES & YES & $1.43$ & $(2,3)$ & 5770 & 7224\\
$(194,75)$ & 11 & $(7,2)$ & 4 & 1 & YES & YES & YES & $1.43$ & $(2,3)$ & NO & 7225\\
$(194,75)$ & 11 & $(12,5)$ & 5 & 2 & YES & YES & YES & $1.71$ & $(2,3)$ & NO & 7226\\
$(194,75)$ & 11 & $(18,7)$ & 6 & 2 & YES & YES & YES & $1.57$ & $(2,3)$ & NO & 7227\\
$(194,75)$ & 11 & $(21,8)$ & 6 & 1 & YES & YES & YES & $1.57$ & $(2,3)$ & NO & 7228\\
$(194,75)$ & 11 & $(23,9)$ & 7 & 1 & YES & YES & YES & $1.57$ & $(2,3)$ & NO & 7229\\
$(194,75)$ & 11 & $(31,12)$ & 7 & 1 & YES & YES & YES & $1.57$ & $(2,3)$ & NO & 7230\\
$(194,75)$ & 11 & $(57,22)$ & 9 & 1 & YES & YES & YES & $1.43$ & $(2,3)$ & NO & 7231\\
$(195,59)$ & 12 & $(2,1)$ & 1 & 1 & YES & YES & NO(2) & $1.62$ & $(2,3)$ & NO & 7232\\
$(195,88)$ & 12 & $(2,1)$ & 1 & 1 & YES & YES & YES & $1.71$ & $(2,3)$ & -- & 7233\\
$(195,76)$ & 12 & $(3,1)$ & 2 & 3 & YES & YES & NO(2) & $1.75$ & $(2,3)$ & -- & 7234\\
$(195,82)$ & 12 & $(3,1)$ & 2 & 3 & YES & YES & YES & $1.71$ & $(2,3)$ & NO & 7235\\
$(195,88)$ & 12 & $(3,1)$ & 2 & 3 & YES & YES & YES & $1.57$ & $(2,3)$ & NO & 7236\\
$(195,88)$ & 12 & $(9,4)$ & 5 & 3 & YES & YES & YES & $1.71$ & $(2,3)$ & NO & 7237\\
$(195,82)$ & 12 & $(12,5)$ & 5 & 3 & YES & YES & YES & $1.71$ & $(2,3)$ & NO & 7238\\
$(195,76)$ & 12 & $(13,5)$ & 5 & 13 & YES & YES & NO(2) & $1.62$ & $(2,3)$ & NO & 7239\\
$(195,88)$ & 12 & $(20,9)$ & 7 & 5 & YES & YES & YES & $1.57$ & $(2,3)$ & 5653 & 7240\\
$(195,59)$ & 12 & $(33,10)$ & 8 & 3 & YES & YES & NO(2) & $1.50$ & $(2,3)$ & NO & 7241\\
$(195,76)$ & 12 & $(77,30)$ & 10 & 1 & YES & YES & NO(2) & $1.62$ & $(2,3)$ & 7591 & 7242\\
$(195,88)$ & 12 & $(82,37)$ & 10 & 1 & YES & YES & YES & $1.71$ & $(2,3)$ & NO & 7243\\
$(195,82)$ & 12 & $(88,37)$ & 10 & 1 & YES & YES & YES & $1.71$ & $(2,3)$ & NO & 7244\\
$(196,75)$ & 11 & $(2,1)$ & 1 & 2 & YES & YES & YES & $1.43$ & $(2,3)$ & NO & 7245\\
$(196,81)$ & 11 & $(2,1)$ & 1 & 2 & YES & YES & YES & $1.29$ & $(2,3)$ & -- & 7246\\
$(196,83)$ & 12 & $(2,1)$ & 1 & 2 & YES & YES & NO(2) & $1.75$ & $(2,3)$ & -- & 7247\\
$(196,75)$ & 11 & $(3,1)$ & 2 & 1 & YES & YES & YES & $1.43$ & $(2,3)$ & -- & 7248\\
$(196,75)$ & 11 & $(3,1)$ & 2 & 1 & YES & YES & NO(2) & $1.62$ & $(2,3)$ & NO & 7249\\
$(196,81)$ & 11 & $(3,1)$ & 2 & 1 & YES & YES & YES & $1.29$ & $(2,3)$ & -- & 7250\\
$(196,57)$ & 12 & $(4,1)$ & 3 & 4 & YES & YES & YES & $1.57$ & $(2,3)$ & -- & 7251\\
$(196,75)$ & 11 & $(4,1)$ & 3 & 4 & YES & YES & YES & $1.71$ & $(2,3)$ & -- & 7252\\
$(196,75)$ & 11 & $(4,1)$ & 3 & 4 & YES & YES & YES & $1.57$ & $(2,3)$ & NO & 7253\\
$(196,55)$ & 12 & $(5,2)$ & 3 & 1 & YES & YES & YES & $1.71$ & $(2,3)$ & -- & 7254\\
$(196,75)$ & 11 & $(5,2)$ & 3 & 1 & YES & YES & YES & $1.71$ & $(2,3)$ & NO & 7255\\
$(196,75)$ & 11 & $(5,2)$ & 3 & 1 & YES & YES & YES & $1.71$ & $(2,3)$ & -- & 7256\\
$(196,81)$ & 11 & $(5,1)$ & 4 & 1 & YES & YES & YES & $1.57$ & $(2,3)$ & NO & 7257\\
$(196,81)$ & 11 & $(5,2)$ & 3 & 1 & YES & YES & YES & $1.57$ & $(2,3)$ & -- & 7258\\
$(196,75)$ & 11 & $(7,2)$ & 4 & 7 & YES & YES & YES & $1.71$ & $(2,3)$ & NO & 7259\\
$(196,81)$ & 11 & $(8,3)$ & 4 & 4 & YES & YES & YES & $1.57$ & $(2,3)$ & NO & 7260\\
$(196,75)$ & 11 & $(11,4)$ & 5 & 1 & YES & YES & YES & $1.57$ & $(2,3)$ & NO & 7261\\
$(196,55)$ & 12 & $(15,4)$ & 6 & 1 & YES & YES & YES & $1.57$ & $(2,3)$ & NO & 7262\\
$(196,81)$ & 11 & $(17,7)$ & 6 & 1 & YES & YES & YES & $1.43$ & $(2,3)$ & 6773 & 7263\\
$(196,81)$ & 11 & $(41,17)$ & 8 & 1 & YES & YES & YES & $1.43$ & $(2,3)$ & NO & 7264\\
$(196,55)$ & 12 & $(43,12)$ & 8 & 1 & YES & YES & YES & $1.57$ & $(2,3)$ & NO & 7265\\
$(196,81)$ & 11 & $(46,19)$ & 8 & 2 & YES & YES & YES & $1.43$ & $(2,3)$ & 6676 & 7266\\
$(196,75)$ & 11 & $(60,23)$ & 9 & 4 & YES & YES & YES & $1.71$ & $(2,3)$ & NO & 7267\\
$(196,81)$ & 11 & $(63,26)$ & 9 & 7 & YES & YES & YES & $1.57$ & $(2,3)$ & NO & 7268\\
$(196,83)$ & 12 & $(85,36)$ & 10 & 1 & YES & YES & NO(2) & $1.75$ & $(2,3)$ & NO & 7269\\
$(196,81)$ & 11 & $(104,43)$ & 10 & 4 & YES & YES & YES & $1.57$ & $(2,3)$ & NO & 7270\\
$(196,75)$ & 11 & $(128,49)$ & 10 & 4 & YES & YES & YES & $1.57$ & $(2,3)$ & NO & 7271\\
$(196,57)$ & 12 & $(141,41)$ & 11 & 1 & YES & YES & YES & $1.57$ & $(2,3)$ & NO & 7272\\
$(197,55)$ & 12 & $(2,1)$ & 1 & 1 & YES & YES & YES & $1.43$ & $(2,3)$ & -- & 7273\\
$(197,55)$ & 12 & $(2,1)$ & 1 & 1 & YES & YES & YES & $1.57$ & $(2,3)$ & NO & 7274\\
$(197,86)$ & 12 & $(2,1)$ & 1 & 1 & YES & YES & NO(2) & $1.75$ & $(2,3)$ & -- & 7275\\
$(197,55)$ & 12 & $(3,1)$ & 2 & 1 & YES & YES & YES & $1.43$ & $(2,3)$ & -- & 7276\\
$(197,57)$ & 14 & $(3,1)$ & 2 & 1 & YES & YES & YES & $1.57$ & $(2,3)$ & -- & 7277\\
$(197,70)$ & 12 & $(3,1)$ & 2 & 1 & YES & YES & YES & $1.57$ & $(2,3)$ & -- & 7278\\
$(197,77)$ & 12 & $(3,1)$ & 2 & 1 & YES & YES & YES & $1.71$ & $(2,3)$ & NO & 7279\\
$(197,77)$ & 12 & $(3,1)$ & 2 & 1 & YES & YES & YES & $1.71$ & $(2,3)$ & -- & 7280\\
$(197,72)$ & 12 & $(4,1)$ & 3 & 1 & YES & YES & YES & $1.57$ & $(2,3)$ & -- & 7281\\
$(197,77)$ & 12 & $(4,1)$ & 3 & 1 & YES & YES & YES & $1.71$ & $(2,3)$ & -- & 7282\\
$(197,55)$ & 12 & $(5,1)$ & 4 & 1 & YES & YES & YES & $1.43$ & $(2,3)$ & NO & 7283\\
$(197,55)$ & 12 & $(5,2)$ & 3 & 1 & YES & YES & YES & $1.57$ & $(2,3)$ & -- & 7284\\
$(197,57)$ & 14 & $(5,1)$ & 4 & 1 & YES & YES & YES & $1.57$ & $(2,3)$ & -- & 7285\\
$(197,86)$ & 12 & $(5,1)$ & 4 & 1 & YES & YES & NO(2) & $1.62$ & $(2,3)$ & NO & 7286\\
$(197,55)$ & 12 & $(6,1)$ & 5 & 1 & YES & YES & YES & $1.43$ & $(2,3)$ & NO & 7287\\
$(197,43)$ & 12 & $(7,3)$ & 4 & 1 & YES & YES & YES & $1.57$ & $(2,3)$ & -- & 7288\\
$(197,45)$ & 13 & $(7,1)$ & 6 & 1 & YES & YES & YES & $1.43$ & $(2,3)$ & NO & 7289\\
$(197,86)$ & 12 & $(7,3)$ & 4 & 1 & YES & YES & NO(2) & $1.62$ & $(2,3)$ & NO & 7290\\
$(197,77)$ & 12 & $(8,3)$ & 4 & 1 & YES & YES & YES & $1.71$ & $(2,3)$ & NO & 7291\\
$(197,55)$ & 12 & $(10,3)$ & 5 & 1 & YES & YES & YES & $1.57$ & $(2,3)$ & NO & 7292\\
$(197,43)$ & 12 & $(17,4)$ & 7 & 1 & YES & YES & YES & $1.71$ & $(2,3)$ & NO & 7293\\
$(197,57)$ & 14 & $(17,5)$ & 6 & 1 & YES & YES & YES & $1.57$ & $(2,3)$ & NO & 7294\\
$(197,72)$ & 12 & $(30,11)$ & 7 & 1 & YES & YES & YES & $1.57$ & $(2,3)$ & NO & 7295\\
$(197,57)$ & 14 & $(31,9)$ & 8 & 1 & YES & YES & YES & $1.57$ & $(2,3)$ & NO & 7296\\
$(197,55)$ & 12 & $(32,9)$ & 8 & 1 & YES & YES & YES & $1.57$ & $(2,3)$ & NO & 7297\\
$(197,43)$ & 12 & $(37,8)$ & 8 & 1 & YES & YES & NO(3) & $1.29$ & $(2,3)$ & NO & 7298\\
$(197,77)$ & 12 & $(41,16)$ & 8 & 1 & YES & YES & YES & $1.57$ & $(2,3)$ & 5568 & 7299\\
$(197,55)$ & 12 & $(68,19)$ & 9 & 1 & YES & YES & YES & $1.43$ & $(2,3)$ & NO & 7300\\
$(197,45)$ & 13 & $(92,21)$ & 10 & 1 & YES & YES & YES & $1.43$ & $(2,3)$ & NO & 7301\\
$(197,55)$ & 12 & $(111,31)$ & 10 & 1 & YES & YES & YES & $1.43$ & $(2,3)$ & 8254 & 7302\\
$(197,45)$ & 13 & $(197,45)$ & 13 & 197 & YES & YES & YES & $1.57$ & $(2,3)$ & NO & 7303\\
$(197,55)$ & 12 & $(197,55)$ & 12 & 197 & YES & YES & YES & $1.43$ & $(2,3)$ & NO & 7304\\
$(197,61)$ & 13 & $(197,61)$ & 13 & 197 & YES & YES & YES & $1.71$ & $(2,3)$ & NO & 7305\\
$(197,72)$ & 12 & $(197,72)$ & 12 & 197 & YES & YES & YES & $1.57$ & $(2,3)$ & NO & 7306\\
$(198,59)$ & 13 & $(5,1)$ & 4 & 1 & YES & YES & YES & $1.43$ & $(2,3)$ & -- & 7307\\
$(198,59)$ & 13 & $(37,11)$ & 8 & 1 & YES & YES & YES & $1.57$ & $(2,3)$ & NO & 7308\\
$(199,76)$ & 11 & $(2,1)$ & 1 & 1 & YES & YES & YES & $1.57$ & $(2,3)$ & -- & 7309\\
$(199,76)$ & 11 & $(4,1)$ & 3 & 1 & YES & YES & YES & $1.57$ & $(2,3)$ & -- & 7310\\
$(199,47)$ & 13 & $(7,3)$ & 4 & 1 & YES & YES & YES & $1.57$ & $(2,3)$ & -- & 7311\\
$(199,76)$ & 11 & $(7,3)$ & 4 & 1 & YES & YES & YES & $1.57$ & $(2,3)$ & NO & 7312\\
$(199,76)$ & 11 & $(8,3)$ & 4 & 1 & YES & YES & YES & $1.43$ & $(2,3)$ & NO & 7313\\
$(199,76)$ & 11 & $(21,8)$ & 6 & 1 & YES & YES & YES & $1.43$ & $(2,3)$ & NO & 7314\\
$(199,78)$ & 12 & $(23,9)$ & 7 & 1 & YES & YES & YES & $1.57$ & $(2,3)$ & NO & 7315\\
$(199,38)$ & 14 & $(89,17)$ & 12 & 1 & YES & YES & YES & $1.86$ & $(2,3)$ & NO & 7316\\
$(199,56)$ & 14 & $(135,38)$ & 12 & 1 & YES & YES & YES & $1.71$ & $(2,3)$ & 8612 & 7317\\
$(199,76)$ & 11 & $(144,55)$ & 10 & 1 & YES & YES & YES & $1.57$ & $(2,3)$ & NO & 7318\\
$(199,84)$ & 12 & $(199,84)$ & 12 & 199 & YES & YES & YES & $1.57$ & $(2,3)$ & NO & 7319\\
$(200,59)$ & 12 & $(2,1)$ & 1 & 2 & YES & YES & NO(2) & $1.62$ & $(2,3)$ & -- & 7320\\
$(200,59)$ & 12 & $(2,1)$ & 1 & 2 & YES & YES & NO(2) & $1.62$ & $(2,3)$ & NO & 7321\\
$(200,53)$ & 12 & $(3,1)$ & 2 & 1 & YES & YES & YES & $1.43$ & $(2,3)$ & NO & 7322\\
$(200,59)$ & 12 & $(3,1)$ & 2 & 1 & NO & YES & YES & $1.57$ & $(2,3)$ & -- & 7323\\
$(200,61)$ & 12 & $(200,61)$ & 12 & 200 & YES & YES & YES & $1.43$ & $(2,3)$ & NO & 7324\\
$(201,83)$ & 12 & $(2,1)$ & 1 & 1 & YES & YES & YES & $1.57$ & $(2,3)$ & -- & 7325\\
$(201,61)$ & 12 & $(3,1)$ & 2 & 3 & YES & YES & YES & $1.43$ & $(2,3)$ & -- & 7326\\
$(201,61)$ & 12 & $(3,1)$ & 2 & 3 & YES & YES & YES & $1.57$ & $(2,3)$ & NO & 7327\\
$(201,76)$ & 12 & $(3,1)$ & 2 & 3 & YES & YES & YES & $1.43$ & $(2,3)$ & -- & 7328\\
$(201,82)$ & 12 & $(3,1)$ & 2 & 3 & YES & YES & YES & $1.57$ & $(2,3)$ & NO & 7329\\
$(201,82)$ & 12 & $(3,1)$ & 2 & 3 & YES & YES & YES & $1.57$ & $(2,3)$ & -- & 7330\\
$(201,83)$ & 12 & $(3,1)$ & 2 & 3 & YES & YES & YES & $1.57$ & $(2,3)$ & -- & 7331\\
$(201,56)$ & 12 & $(4,1)$ & 3 & 1 & YES & YES & YES & $1.71$ & $(2,3)$ & NO & 7332\\
$(201,59)$ & 13 & $(4,1)$ & 3 & 1 & YES & YES & YES & $1.57$ & $(2,3)$ & -- & 7333\\
$(201,77)$ & 12 & $(4,1)$ & 3 & 1 & YES & YES & YES & $1.57$ & $(2,3)$ & -- & 7334\\
$(201,83)$ & 12 & $(4,1)$ & 3 & 1 & YES & YES & YES & $1.57$ & $(2,3)$ & -- & 7335\\
$(201,59)$ & 13 & $(5,1)$ & 4 & 1 & YES & YES & YES & $1.57$ & $(2,3)$ & -- & 7336\\
$(201,61)$ & 12 & $(5,1)$ & 4 & 1 & YES & YES & YES & $1.29$ & $(2,3)$ & NO & 7337\\
$(201,83)$ & 12 & $(5,1)$ & 4 & 1 & YES & YES & YES & $1.57$ & $(2,3)$ & NO & 7338\\
$(201,83)$ & 12 & $(6,1)$ & 5 & 3 & YES & YES & YES & $1.43$ & $(2,3)$ & NO & 7339\\
$(201,82)$ & 12 & $(7,3)$ & 4 & 1 & YES & YES & YES & $1.57$ & $(2,3)$ & NO & 7340\\
$(201,76)$ & 12 & $(13,5)$ & 5 & 1 & YES & YES & YES & $1.43$ & $(2,3)$ & NO & 7341\\
$(201,76)$ & 12 & $(21,8)$ & 6 & 3 & YES & YES & YES & $1.43$ & $(2,3)$ & NO & 7342\\
$(201,82)$ & 12 & $(22,9)$ & 7 & 1 & YES & YES & YES & $1.43$ & $(2,3)$ & 7167 & 7343\\
$(201,82)$ & 12 & $(27,11)$ & 8 & 3 & YES & YES & YES & $1.43$ & $(2,3)$ & NO & 7344\\
$(201,59)$ & 13 & $(41,12)$ & 8 & 1 & YES & YES & YES & $1.57$ & $(2,3)$ & NO & 7345\\
$(201,76)$ & 12 & $(45,17)$ & 9 & 3 & YES & YES & YES & $1.57$ & $(2,3)$ & NO & 7346\\
$(201,59)$ & 13 & $(58,17)$ & 9 & 1 & YES & YES & YES & $1.57$ & $(2,3)$ & 6063 & 7347\\
$(201,47)$ & 12 & $(73,17)$ & 10 & 1 & YES & YES & YES & $1.43$ & $(2,3)$ & NO & 7348\\
$(201,77)$ & 12 & $(107,41)$ & 10 & 1 & YES & YES & YES & $1.57$ & $(2,3)$ & 8186 & 7349\\
$(201,77)$ & 12 & $(154,59)$ & 11 & 1 & YES & YES & YES & $1.57$ & $(2,3)$ & NO & 7350\\
$(201,83)$ & 12 & $(155,64)$ & 11 & 1 & YES & YES & YES & $1.71$ & $(2,3)$ & NO & 7351\\
$(201,77)$ & 12 & $(201,77)$ & 12 & 201 & YES & YES & YES & $1.43$ & $(2,3)$ & NO & 7352\\
$(201,83)$ & 12 & $(201,83)$ & 12 & 201 & YES & YES & YES & $1.57$ & $(2,3)$ & NO & 7353\\
$(202,73)$ & 12 & $(2,1)$ & 1 & 2 & YES & YES & NO(2) & $1.75$ & $(2,3)$ & NO & 7354\\
$(202,73)$ & 12 & $(2,1)$ & 1 & 2 & YES & YES & NO(2) & $1.75$ & $(2,3)$ & -- & 7355\\
$(202,83)$ & 12 & $(2,1)$ & 1 & 2 & YES & YES & NO(2) & $1.62$ & $(2,3)$ & NO & 7356\\
$(202,47)$ & 13 & $(3,1)$ & 2 & 1 & YES & YES & NO(2) & $1.57$ & $(4,2)$ & NO & 7357\\
$(202,59)$ & 12 & $(3,1)$ & 2 & 1 & YES & YES & NO(2) & $1.50$ & $(2,3)$ & NO & 7358\\
$(202,61)$ & 13 & $(3,1)$ & 2 & 1 & YES & YES & YES & $1.57$ & $(2,3)$ & -- & 7359\\
$(202,61)$ & 13 & $(4,1)$ & 3 & 2 & YES & YES & YES & $1.57$ & $(2,3)$ & -- & 7360\\
$(202,83)$ & 12 & $(4,1)$ & 3 & 2 & YES & YES & YES & $1.57$ & $(2,3)$ & -- & 7361\\
$(202,53)$ & 13 & $(5,1)$ & 4 & 1 & YES & YES & YES & $1.57$ & $(2,3)$ & NO & 7362\\
$(202,59)$ & 12 & $(5,1)$ & 4 & 1 & YES & YES & YES & $1.57$ & $(2,3)$ & NO & 7363\\
$(202,59)$ & 12 & $(5,1)$ & 4 & 1 & YES & YES & YES & $1.57$ & $(2,3)$ & -- & 7364\\
$(202,83)$ & 12 & $(5,2)$ & 3 & 1 & YES & YES & YES & $1.71$ & $(2,3)$ & -- & 7365\\
$(202,91)$ & 13 & $(6,1)$ & 5 & 2 & YES & YES & YES & $1.71$ & $(2,3)$ & -- & 7366\\
$(202,61)$ & 13 & $(23,7)$ & 7 & 1 & YES & YES & YES & $1.57$ & $(2,3)$ & NO & 7367\\
$(202,61)$ & 13 & $(33,10)$ & 8 & 1 & YES & YES & YES & $1.57$ & $(2,3)$ & NO & 7368\\
$(202,83)$ & 12 & $(39,16)$ & 8 & 1 & YES & YES & YES & $1.57$ & $(2,3)$ & 7996 & 7369\\
$(202,61)$ & 13 & $(43,13)$ & 9 & 1 & YES & YES & YES & $1.57$ & $(2,3)$ & NO & 7370\\
$(202,83)$ & 12 & $(56,23)$ & 9 & 2 & YES & YES & NO(2) & $1.62$ & $(2,3)$ & 7040 & 7371\\
$(202,59)$ & 12 & $(58,17)$ & 9 & 2 & YES & YES & YES & $1.43$ & $(2,3)$ & NO & 7372\\
$(202,73)$ & 12 & $(83,30)$ & 10 & 1 & YES & YES & NO(2) & $1.62$ & $(2,3)$ & NO & 7373\\
$(202,91)$ & 13 & $(91,41)$ & 11 & 1 & YES & YES & YES & $1.71$ & $(2,3)$ & NO & 7374\\
$(202,61)$ & 13 & $(96,29)$ & 11 & 2 & YES & YES & YES & $1.57$ & $(2,3)$ & 7978 & 7375\\
$(202,83)$ & 12 & $(129,53)$ & 11 & 1 & YES & YES & YES & $1.57$ & $(2,3)$ & NO & 7376\\
$(202,61)$ & 13 & $(149,45)$ & 12 & 1 & YES & YES & YES & $1.57$ & $(2,3)$ & NO & 7377\\
$(202,61)$ & 13 & $(202,61)$ & 13 & 202 & YES & YES & YES & $1.57$ & $(2,3)$ & NO & 7378\\
$(203,59)$ & 12 & $(2,1)$ & 1 & 1 & YES & YES & YES & $1.43$ & $(2,3)$ & -- & 7379\\
$(203,86)$ & 12 & $(2,1)$ & 1 & 1 & YES & YES & NO(2) & $1.75$ & $(2,3)$ & -- & 7380\\
$(203,59)$ & 12 & $(3,1)$ & 2 & 1 & YES & YES & YES & $1.43$ & $(2,3)$ & -- & 7381\\
$(203,59)$ & 12 & $(4,1)$ & 3 & 1 & YES & YES & YES & $1.43$ & $(2,3)$ & -- & 7382\\
$(203,60)$ & 12 & $(5,2)$ & 3 & 1 & YES & YES & YES & $1.57$ & $(2,3)$ & -- & 7383\\
$(203,86)$ & 12 & $(5,1)$ & 4 & 1 & YES & YES & NO(2) & $1.62$ & $(2,3)$ & -- & 7384\\
$(203,48)$ & 13 & $(6,1)$ & 5 & 1 & YES & YES & YES & $1.71$ & $(2,3)$ & NO & 7385\\
$(203,48)$ & 13 & $(6,1)$ & 5 & 1 & YES & YES & YES & $1.71$ & $(2,3)$ & -- & 7386\\
$(203,75)$ & 12 & $(8,3)$ & 4 & 1 & YES & YES & YES & $1.71$ & $(2,3)$ & NO & 7387\\
$(203,59)$ & 12 & $(10,3)$ & 5 & 1 & YES & YES & YES & $1.43$ & $(2,3)$ & NO & 7388\\
$(203,60)$ & 12 & $(11,3)$ & 5 & 1 & YES & YES & YES & $1.57$ & $(2,3)$ & NO & 7389\\
$(203,55)$ & 13 & $(15,4)$ & 6 & 1 & YES & YES & NO(2) & $1.75$ & $(2,3)$ & NO & 7390\\
$(203,48)$ & 13 & $(17,4)$ & 7 & 1 & YES & YES & YES & $1.71$ & $(2,3)$ & NO & 7391\\
$(203,59)$ & 12 & $(17,5)$ & 6 & 1 & YES & YES & YES & $1.43$ & $(2,3)$ & NO & 7392\\
$(203,59)$ & 12 & $(24,7)$ & 7 & 1 & YES & YES & YES & $1.43$ & $(2,3)$ & 5708 & 7393\\
$(203,55)$ & 13 & $(37,10)$ & 8 & 1 & YES & YES & NO(2) & $1.75$ & $(2,3)$ & NO & 7394\\
$(203,60)$ & 12 & $(37,11)$ & 8 & 1 & YES & YES & YES & $1.57$ & $(2,3)$ & NO & 7395\\
$(203,59)$ & 12 & $(41,12)$ & 8 & 1 & YES & YES & YES & $1.43$ & $(2,3)$ & NO & 7396\\
$(203,48)$ & 13 & $(55,13)$ & 10 & 1 & YES & YES & YES & $1.71$ & $(2,3)$ & NO & 7397\\
$(203,86)$ & 12 & $(85,36)$ & 10 & 1 & YES & YES & NO(2) & $1.62$ & $(2,3)$ & 7776 & 7398\\
$(203,75)$ & 12 & $(157,58)$ & 11 & 1 & YES & YES & NO(2) & $1.62$ & $(2,3)$ & NO & 7399\\
$(203,48)$ & 13 & $(169,40)$ & 13 & 1 & YES & YES & YES & $1.57$ & $(2,3)$ & NO & 7400\\
$(205,61)$ & 12 & $(2,1)$ & 1 & 1 & YES & YES & YES & $1.57$ & $(2,3)$ & NO & 7401\\
$(205,62)$ & 13 & $(2,1)$ & 1 & 1 & YES & YES & NO(2) & $1.62$ & $(2,3)$ & -- & 7402\\
$(205,92)$ & 12 & $(3,1)$ & 2 & 1 & YES & YES & YES & $1.71$ & $(2,3)$ & NO & 7403\\
$(205,92)$ & 12 & $(3,1)$ & 2 & 1 & YES & YES & YES & $1.71$ & $(2,3)$ & -- & 7404\\
$(205,62)$ & 13 & $(4,1)$ & 3 & 1 & YES & YES & NO(2) & $1.62$ & $(2,3)$ & NO & 7405\\
$(205,76)$ & 12 & $(4,1)$ & 3 & 1 & YES & YES & NO(2) & $1.62$ & $(2,3)$ & -- & 7406\\
$(205,76)$ & 12 & $(5,2)$ & 3 & 5 & YES & YES & YES & $1.71$ & $(2,3)$ & -- & 7407\\
$(205,78)$ & 12 & $(5,2)$ & 3 & 5 & YES & YES & YES & $1.71$ & $(2,3)$ & -- & 7408\\
$(205,84)$ & 12 & $(5,1)$ & 4 & 5 & YES & YES & YES & $1.57$ & $(2,3)$ & -- & 7409\\
$(205,84)$ & 12 & $(5,2)$ & 3 & 5 & YES & YES & YES & $1.57$ & $(2,3)$ & -- & 7410\\
$(205,84)$ & 12 & $(7,2)$ & 4 & 1 & YES & YES & YES & $1.57$ & $(2,3)$ & NO & 7411\\
$(205,47)$ & 13 & $(11,2)$ & 6 & 1 & YES & YES & YES & $1.43$ & $(2,3)$ & NO & 7412\\
$(205,92)$ & 12 & $(11,5)$ & 6 & 1 & YES & YES & YES & $1.57$ & $(2,3)$ & 7615 & 7413\\
$(205,74)$ & 13 & $(25,9)$ & 7 & 5 & YES & YES & YES & $1.71$ & $(2,3)$ & NO & 7414\\
$(205,61)$ & 12 & $(27,8)$ & 7 & 1 & YES & YES & YES & $1.29$ & $(2,3)$ & NO & 7415\\
$(205,62)$ & 13 & $(33,10)$ & 8 & 1 & YES & YES & NO(2) & $1.50$ & $(2,3)$ & NO & 7416\\
$(205,84)$ & 12 & $(39,16)$ & 8 & 1 & YES & YES & YES & $1.57$ & $(2,3)$ & NO & 7417\\
$(205,47)$ & 13 & $(74,17)$ & 11 & 1 & YES & YES & YES & $1.43$ & $(2,3)$ & NO & 7418\\
$(205,84)$ & 12 & $(83,34)$ & 10 & 1 & YES & YES & YES & $1.57$ & $(2,3)$ & 7744 & 7419\\
$(205,62)$ & 13 & $(109,33)$ & 11 & 1 & YES & YES & YES & $1.71$ & $(2,3)$ & NO & 7420\\
$(205,62)$ & 13 & $(119,36)$ & 11 & 1 & YES & YES & NO(2) & $1.50$ & $(2,3)$ & 8434 & 7421\\
$(205,92)$ & 12 & $(127,57)$ & 11 & 1 & YES & YES & YES & $1.57$ & $(2,3)$ & NO & 7422\\
$(205,84)$ & 12 & $(205,84)$ & 12 & 205 & YES & YES & YES & $1.71$ & $(2,3)$ & NO & 7423\\
$(206,85)$ & 12 & $(2,1)$ & 1 & 2 & YES & YES & YES & $1.57$ & $(2,3)$ & -- & 7424\\
$(206,63)$ & 12 & $(3,1)$ & 2 & 1 & YES & YES & YES & $1.43$ & $(2,3)$ & -- & 7425\\
$(206,73)$ & 12 & $(3,1)$ & 2 & 1 & YES & YES & YES & $1.43$ & $(2,3)$ & -- & 7426\\
$(206,85)$ & 12 & $(3,1)$ & 2 & 1 & YES & YES & YES & $1.57$ & $(2,3)$ & NO & 7427\\
$(206,85)$ & 12 & $(3,1)$ & 2 & 1 & YES & YES & YES & $1.57$ & $(2,3)$ & -- & 7428\\
$(206,85)$ & 12 & $(3,1)$ & 2 & 1 & YES & YES & YES & $1.57$ & $(2,3)$ & NO & 7429\\
$(206,85)$ & 12 & $(4,1)$ & 3 & 2 & YES & YES & YES & $1.57$ & $(2,3)$ & -- & 7430\\
$(206,85)$ & 12 & $(4,1)$ & 3 & 2 & YES & YES & YES & $1.71$ & $(2,3)$ & NO & 7431\\
$(206,57)$ & 12 & $(5,2)$ & 3 & 1 & YES & YES & YES & $1.57$ & $(2,3)$ & -- & 7432\\
$(206,63)$ & 12 & $(5,2)$ & 3 & 1 & YES & YES & YES & $1.43$ & $(2,3)$ & -- & 7433\\
$(206,73)$ & 12 & $(5,1)$ & 4 & 1 & YES & YES & YES & $1.71$ & $(2,3)$ & -- & 7434\\
$(206,85)$ & 12 & $(5,1)$ & 4 & 1 & YES & YES & YES & $1.57$ & $(2,3)$ & -- & 7435\\
$(206,47)$ & 12 & $(7,3)$ & 4 & 1 & YES & YES & YES & $1.57$ & $(2,3)$ & -- & 7436\\
$(206,47)$ & 12 & $(17,4)$ & 7 & 1 & YES & YES & YES & $1.57$ & $(2,3)$ & NO & 7437\\
$(206,85)$ & 12 & $(22,9)$ & 7 & 2 & YES & YES & YES & $1.57$ & $(2,3)$ & NO & 7438\\
$(206,37)$ & 14 & $(23,4)$ & 8 & 1 & YES & YES & YES & $1.43$ & $(2,3)$ & NO & 7439\\
$(206,39)$ & 14 & $(27,5)$ & 8 & 1 & YES & YES & YES & $1.71$ & $(2,3)$ & NO & 7440\\
$(206,85)$ & 12 & $(29,12)$ & 7 & 1 & YES & YES & YES & $1.71$ & $(2,3)$ & NO & 7441\\
$(206,73)$ & 12 & $(48,17)$ & 9 & 2 & YES & YES & YES & $1.57$ & $(2,3)$ & 6833 & 7442\\
$(206,85)$ & 12 & $(80,33)$ & 10 & 2 & YES & YES & YES & $1.57$ & $(2,3)$ & 7686 & 7443\\
$(206,85)$ & 12 & $(143,59)$ & 11 & 1 & YES & YES & YES & $1.57$ & $(2,3)$ & NO & 7444\\
$(206,85)$ & 12 & $(206,85)$ & 12 & 206 & YES & YES & YES & $1.43$ & $(2,3)$ & NO & 7445\\
$(207,79)$ & 11 & $(2,1)$ & 1 & 1 & YES & YES & YES & $1.57$ & $(2,3)$ & NO & 7446\\
$(207,91)$ & 12 & $(2,1)$ & 1 & 1 & YES & YES & YES & $1.71$ & $(2,3)$ & -- & 7447\\
$(207,79)$ & 11 & $(3,1)$ & 2 & 3 & YES & YES & YES & $1.43$ & $(2,3)$ & NO & 7448\\
$(207,79)$ & 11 & $(3,1)$ & 2 & 3 & YES & YES & YES & $1.43$ & $(2,3)$ & -- & 7449\\
$(207,85)$ & 12 & $(3,1)$ & 2 & 3 & YES & YES & YES & $1.71$ & $(2,3)$ & -- & 7450\\
$(207,85)$ & 12 & $(3,1)$ & 2 & 3 & YES & YES & YES & $1.57$ & $(2,3)$ & NO & 7451\\
$(207,79)$ & 11 & $(4,1)$ & 3 & 1 & YES & YES & YES & $1.29$ & $(2,3)$ & NO & 7452\\
$(207,79)$ & 11 & $(4,1)$ & 3 & 1 & YES & YES & YES & $1.43$ & $(2,3)$ & -- & 7453\\
$(207,85)$ & 12 & $(4,1)$ & 3 & 1 & YES & YES & YES & $1.57$ & $(2,3)$ & NO & 7454\\
$(207,85)$ & 12 & $(4,1)$ & 3 & 1 & YES & YES & YES & $1.57$ & $(2,3)$ & -- & 7455\\
$(207,79)$ & 11 & $(5,2)$ & 3 & 1 & YES & YES & YES & $1.57$ & $(2,3)$ & -- & 7456\\
$(207,79)$ & 11 & $(5,2)$ & 3 & 1 & YES & YES & YES & $1.57$ & $(2,3)$ & NO & 7457\\
$(207,61)$ & 13 & $(7,1)$ & 6 & 1 & YES & YES & YES & $1.43$ & $(2,3)$ & NO & 7458\\
$(207,79)$ & 11 & $(11,4)$ & 5 & 1 & YES & YES & YES & $1.43$ & $(2,3)$ & NO & 7459\\
$(207,91)$ & 12 & $(16,7)$ & 6 & 1 & YES & YES & YES & $1.71$ & $(2,3)$ & NO & 7460\\
$(207,85)$ & 12 & $(17,7)$ & 6 & 1 & YES & YES & NO(2) & $1.50$ & $(2,3)$ & NO & 7461\\
$(207,85)$ & 12 & $(22,9)$ & 7 & 1 & YES & YES & YES & $1.57$ & $(2,3)$ & NO & 7462\\
$(207,79)$ & 11 & $(29,11)$ & 7 & 1 & YES & YES & YES & $1.57$ & $(2,3)$ & NO & 7463\\
$(207,79)$ & 11 & $(34,13)$ & 7 & 1 & YES & YES & YES & $1.43$ & $(2,3)$ & 7824 & 7464\\
$(207,79)$ & 11 & $(47,18)$ & 8 & 1 & YES & YES & YES & $1.43$ & $(2,3)$ & 8841 & 7465\\
$(207,85)$ & 12 & $(56,23)$ & 9 & 1 & YES & YES & YES & $1.57$ & $(2,3)$ & NO & 7466\\
$(207,85)$ & 12 & $(73,30)$ & 10 & 1 & YES & YES & YES & $1.71$ & $(2,3)$ & NO & 7467\\
$(207,91)$ & 12 & $(91,40)$ & 10 & 1 & YES & YES & YES & $1.71$ & $(2,3)$ & NO & 7468\\
$(207,61)$ & 13 & $(95,28)$ & 11 & 1 & YES & YES & YES & $1.57$ & $(2,3)$ & NO & 7469\\
$(207,85)$ & 12 & $(95,39)$ & 10 & 1 & YES & YES & YES & $1.57$ & $(2,3)$ & 8000 & 7470\\
$(207,79)$ & 11 & $(97,37)$ & 10 & 1 & YES & YES & YES & $1.57$ & $(2,3)$ & NO & 7471\\
$(207,65)$ & 13 & $(121,38)$ & 12 & 1 & YES & YES & YES & $1.57$ & $(2,3)$ & NO & 7472\\
$(207,79)$ & 11 & $(131,50)$ & 10 & 1 & YES & YES & YES & $1.43$ & $(2,3)$ & NO & 7473\\
$(207,85)$ & 12 & $(151,62)$ & 11 & 1 & YES & YES & YES & $1.71$ & $(2,3)$ & NO & 7474\\
$(207,79)$ & 11 & $(207,79)$ & 11 & 207 & YES & YES & YES & $1.43$ & $(2,3)$ & NO & 7475\\
$(207,85)$ & 12 & $(207,85)$ & 12 & 207 & YES & YES & YES & $1.71$ & $(2,3)$ & NO & 7476\\
$(208,79)$ & 11 & $(2,1)$ & 1 & 2 & YES & YES & NO(2) & $1.38$ & $(2,3)$ & NO & 7477\\
$(208,55)$ & 12 & $(3,1)$ & 2 & 1 & YES & YES & YES & $1.43$ & $(2,3)$ & -- & 7478\\
$(208,79)$ & 11 & $(3,1)$ & 2 & 1 & YES & YES & YES & $1.43$ & $(2,3)$ & NO & 7479\\
$(208,79)$ & 11 & $(3,1)$ & 2 & 1 & YES & YES & YES & $1.43$ & $(2,3)$ & -- & 7480\\
$(208,79)$ & 11 & $(4,1)$ & 3 & 4 & YES & YES & YES & $1.43$ & $(2,3)$ & -- & 7481\\
$(208,79)$ & 11 & $(4,1)$ & 3 & 4 & YES & YES & YES & $1.43$ & $(2,3)$ & NO & 7482\\
$(208,87)$ & 12 & $(4,1)$ & 3 & 4 & YES & YES & YES & $1.57$ & $(2,3)$ & NO & 7483\\
$(208,37)$ & 13 & $(5,2)$ & 3 & 1 & YES & YES & YES & $1.57$ & $(2,3)$ & NO & 7484\\
$(208,37)$ & 13 & $(5,2)$ & 3 & 1 & YES & YES & YES & $1.57$ & $(2,3)$ & NO & 7485\\
$(208,79)$ & 11 & $(5,2)$ & 3 & 1 & YES & YES & YES & $1.57$ & $(2,3)$ & -- & 7486\\
$(208,45)$ & 13 & $(6,1)$ & 5 & 2 & YES & YES & YES & $1.43$ & $(2,3)$ & NO & 7487\\
$(208,45)$ & 13 & $(7,1)$ & 6 & 1 & YES & YES & YES & $1.57$ & $(2,3)$ & NO & 7488\\
$(208,79)$ & 11 & $(7,3)$ & 4 & 1 & YES & YES & YES & $1.57$ & $(2,3)$ & NO & 7489\\
$(208,55)$ & 12 & $(9,2)$ & 5 & 1 & YES & YES & YES & $1.57$ & $(2,3)$ & NO & 7490\\
$(208,79)$ & 11 & $(13,5)$ & 5 & 13 & YES & YES & YES & $1.43$ & $(2,3)$ & 7821 & 7491\\
$(208,61)$ & 12 & $(24,7)$ & 7 & 8 & YES & YES & YES & $1.43$ & $(2,3)$ & NO & 7492\\
$(208,79)$ & 11 & $(71,27)$ & 9 & 1 & YES & YES & YES & $1.57$ & $(2,3)$ & NO & 7493\\
$(208,79)$ & 11 & $(108,41)$ & 10 & 4 & YES & YES & YES & $1.57$ & $(2,3)$ & NO & 7494\\
$(208,55)$ & 12 & $(121,32)$ & 11 & 1 & YES & YES & YES & $1.43$ & $(2,3)$ & NO & 7495\\
$(208,79)$ & 11 & $(129,49)$ & 10 & 1 & YES & YES & YES & $1.43$ & $(2,3)$ & NO & 7496\\
$(208,87)$ & 12 & $(153,64)$ & 11 & 1 & YES & YES & YES & $1.57$ & $(2,3)$ & NO & 7497\\
$(208,45)$ & 13 & $(171,37)$ & 12 & 1 & YES & YES & YES & $1.57$ & $(2,3)$ & NO & 7498\\
$(208,45)$ & 13 & $(208,45)$ & 13 & 208 & YES & YES & YES & $1.43$ & $(2,3)$ & NO & 7499\\
$(208,55)$ & 12 & $(208,55)$ & 12 & 208 & YES & YES & YES & $1.43$ & $(2,3)$ & NO & 7500\\
$(208,79)$ & 11 & $(208,79)$ & 11 & 208 & YES & YES & YES & $1.43$ & $(2,3)$ & NO & 7501\\
$(209,75)$ & 13 & $(2,1)$ & 1 & 1 & YES & YES & YES & $1.57$ & $(2,3)$ & -- & 7502\\
$(209,80)$ & 11 & $(2,1)$ & 1 & 1 & YES & YES & YES & $1.43$ & $(2,3)$ & -- & 7503\\
$(209,80)$ & 11 & $(2,1)$ & 1 & 1 & YES & YES & YES & $1.43$ & $(2,3)$ & NO & 7504\\
$(209,81)$ & 11 & $(2,1)$ & 1 & 1 & YES & YES & YES & $1.57$ & $(2,3)$ & -- & 7505\\
$(209,81)$ & 11 & $(2,1)$ & 1 & 1 & YES & YES & NO(2) & $1.50$ & $(2,3)$ & NO & 7506\\
$(209,80)$ & 11 & $(3,1)$ & 2 & 1 & YES & YES & NO(3) & $1.29$ & $(2,3)$ & -- & 7507\\
$(209,80)$ & 11 & $(3,1)$ & 2 & 1 & YES & YES & YES & $1.57$ & $(2,3)$ & NO & 7508\\
$(209,80)$ & 11 & $(3,1)$ & 2 & 1 & YES & YES & YES & $1.71$ & $(2,3)$ & NO & 7509\\
$(209,81)$ & 11 & $(3,1)$ & 2 & 1 & YES & YES & YES & $1.43$ & $(2,3)$ & NO & 7510\\
$(209,81)$ & 11 & $(3,1)$ & 2 & 1 & YES & YES & YES & $1.43$ & $(2,3)$ & -- & 7511\\
$(209,65)$ & 13 & $(4,1)$ & 3 & 1 & YES & YES & YES & $1.57$ & $(2,3)$ & NO & 7512\\
$(209,80)$ & 11 & $(4,1)$ & 3 & 1 & YES & YES & YES & $1.57$ & $(2,3)$ & NO & 7513\\
$(209,80)$ & 11 & $(4,1)$ & 3 & 1 & YES & YES & YES & $1.43$ & $(2,3)$ & -- & 7514\\
$(209,64)$ & 13 & $(5,2)$ & 3 & 1 & YES & YES & YES & $1.71$ & $(2,3)$ & -- & 7515\\
$(209,80)$ & 11 & $(5,2)$ & 3 & 1 & YES & YES & YES & $1.71$ & $(2,3)$ & -- & 7516\\
$(209,81)$ & 11 & $(5,2)$ & 3 & 1 & YES & YES & YES & $1.43$ & $(2,3)$ & 6715 & 7517\\
$(209,75)$ & 13 & $(6,1)$ & 5 & 1 & YES & YES & YES & $1.57$ & $(2,3)$ & NO & 7518\\
$(209,80)$ & 11 & $(8,3)$ & 4 & 1 & YES & YES & YES & $1.43$ & $(2,3)$ & 5918 & 7519\\
$(209,82)$ & 12 & $(8,3)$ & 4 & 1 & YES & YES & YES & $1.57$ & $(2,3)$ & NO & 7520\\
$(209,91)$ & 12 & $(8,3)$ & 4 & 1 & YES & YES & YES & $1.71$ & $(2,3)$ & NO & 7521\\
$(209,62)$ & 12 & $(10,3)$ & 5 & 1 & YES & YES & NO(2) & $1.62$ & $(2,3)$ & NO & 7522\\
$(209,65)$ & 13 & $(13,4)$ & 6 & 1 & YES & YES & YES & $1.57$ & $(2,3)$ & NO & 7523\\
$(209,80)$ & 11 & $(13,5)$ & 5 & 1 & YES & YES & YES & $1.43$ & $(2,3)$ & 6672 & 7524\\
$(209,81)$ & 11 & $(13,5)$ & 5 & 1 & YES & YES & YES & $1.57$ & $(2,3)$ & NO & 7525\\
$(209,49)$ & 13 & $(14,3)$ & 6 & 1 & YES & YES & YES & $1.57$ & $(2,3)$ & NO & 7526\\
$(209,80)$ & 11 & $(18,7)$ & 6 & 1 & YES & YES & YES & $1.71$ & $(2,3)$ & NO & 7527\\
$(209,81)$ & 11 & $(18,7)$ & 6 & 1 & YES & YES & YES & $1.43$ & $(2,3)$ & 7009 & 7528\\
$(209,81)$ & 11 & $(21,8)$ & 6 & 1 & YES & YES & YES & $1.57$ & $(2,3)$ & NO & 7529\\
$(209,62)$ & 12 & $(24,7)$ & 7 & 1 & YES & YES & YES & $1.57$ & $(2,3)$ & NO & 7530\\
$(209,65)$ & 13 & $(29,9)$ & 8 & 1 & YES & YES & YES & $1.57$ & $(2,3)$ & NO & 7531\\
$(209,81)$ & 11 & $(49,19)$ & 8 & 1 & YES & YES & YES & $1.57$ & $(2,3)$ & 6893 & 7532\\
$(209,75)$ & 13 & $(53,19)$ & 9 & 1 & YES & YES & YES & $1.57$ & $(2,3)$ & NO & 7533\\
$(209,80)$ & 11 & $(60,23)$ & 9 & 1 & YES & YES & YES & $1.71$ & $(2,3)$ & NO & 7534\\
$(209,49)$ & 13 & $(64,15)$ & 10 & 1 & YES & YES & YES & $1.71$ & $(2,3)$ & NO & 7535\\
$(209,80)$ & 11 & $(128,49)$ & 10 & 1 & YES & YES & YES & $1.43$ & $(2,3)$ & NO & 7536\\
$(209,80)$ & 11 & $(175,67)$ & 11 & 1 & YES & YES & YES & $1.43$ & $(2,3)$ & NO & 7537\\
$(209,80)$ & 11 & $(209,80)$ & 11 & 209 & YES & YES & YES & $1.43$ & $(2,3)$ & NO & 7538\\
$(210,59)$ & 12 & $(5,2)$ & 3 & 5 & YES & YES & YES & $1.57$ & $(2,3)$ & -- & 7539\\
$(210,59)$ & 12 & $(43,12)$ & 8 & 1 & YES & YES & YES & $1.43$ & $(2,3)$ & NO & 7540\\
$(211,89)$ & 12 & $(3,1)$ & 2 & 1 & YES & YES & YES & $1.71$ & $(2,3)$ & -- & 7541\\
$(211,49)$ & 13 & $(4,1)$ & 3 & 1 & YES & YES & YES & $1.57$ & $(2,3)$ & -- & 7542\\
$(211,49)$ & 13 & $(4,1)$ & 3 & 1 & YES & YES & YES & $1.71$ & $(2,3)$ & NO & 7543\\
$(211,59)$ & 12 & $(4,1)$ & 3 & 1 & YES & YES & NO(2) & $1.50$ & $(2,3)$ & NO & 7544\\
$(211,89)$ & 12 & $(4,1)$ & 3 & 1 & YES & YES & YES & $1.71$ & $(2,3)$ & -- & 7545\\
$(211,49)$ & 13 & $(5,1)$ & 4 & 1 & YES & YES & YES & $1.71$ & $(2,3)$ & -- & 7546\\
$(211,49)$ & 13 & $(5,2)$ & 3 & 1 & YES & YES & YES & $1.57$ & $(2,3)$ & NO & 7547\\
$(211,49)$ & 13 & $(5,2)$ & 3 & 1 & YES & YES & YES & $1.57$ & $(2,3)$ & -- & 7548\\
$(211,64)$ & 12 & $(5,2)$ & 3 & 1 & YES & YES & YES & $1.71$ & $(2,3)$ & -- & 7549\\
$(211,64)$ & 12 & $(10,3)$ & 5 & 1 & YES & YES & NO(2) & $1.50$ & $(2,3)$ & NO & 7550\\
$(211,89)$ & 12 & $(12,5)$ & 5 & 1 & YES & YES & YES & $1.57$ & $(2,3)$ & NO & 7551\\
$(211,49)$ & 13 & $(16,3)$ & 7 & 1 & YES & YES & YES & $1.57$ & $(2,3)$ & NO & 7552\\
$(211,49)$ & 13 & $(17,4)$ & 7 & 1 & YES & YES & NO(2) & $1.50$ & $(2,3)$ & NO & 7553\\
$(211,59)$ & 12 & $(43,12)$ & 8 & 1 & YES & YES & NO(2) & $1.50$ & $(2,3)$ & 5730 & 7554\\
$(211,64)$ & 12 & $(43,13)$ & 9 & 1 & YES & YES & YES & $1.43$ & $(2,3)$ & NO & 7555\\
$(211,93)$ & 12 & $(59,26)$ & 9 & 1 & YES & YES & YES & $1.57$ & $(2,3)$ & NO & 7556\\
$(211,56)$ & 13 & $(64,17)$ & 10 & 1 & YES & YES & NO(2) & $1.75$ & $(2,3)$ & NO & 7557\\
$(211,49)$ & 13 & $(73,17)$ & 10 & 1 & YES & YES & YES & $1.57$ & $(2,3)$ & 8979 & 7558\\
$(211,64)$ & 12 & $(79,24)$ & 10 & 1 & YES & YES & YES & $1.57$ & $(2,3)$ & NO & 7559\\
$(211,89)$ & 12 & $(83,35)$ & 10 & 1 & YES & YES & YES & $1.71$ & $(2,3)$ & 7795 & 7560\\
$(211,78)$ & 12 & $(119,44)$ & 10 & 1 & YES & YES & YES & $1.57$ & $(2,3)$ & 8463 & 7561\\
$(211,89)$ & 12 & $(147,62)$ & 11 & 1 & YES & YES & YES & $1.57$ & $(2,3)$ & NO & 7562\\
$(211,56)$ & 13 & $(211,56)$ & 13 & 211 & YES & YES & NO(2) & $1.62$ & $(2,3)$ & NO & 7563\\
$(211,93)$ & 12 & $(211,93)$ & 12 & 211 & YES & YES & YES & $1.43$ & $(2,3)$ & NO & 7564\\
$(212,81)$ & 11 & $(2,1)$ & 1 & 2 & YES & YES & YES & $1.43$ & $(2,3)$ & -- & 7565\\
$(212,81)$ & 11 & $(3,1)$ & 2 & 1 & YES & YES & YES & $1.29$ & $(2,3)$ & -- & 7566\\
$(212,93)$ & 12 & $(3,1)$ & 2 & 1 & YES & YES & YES & $1.71$ & $(2,3)$ & NO & 7567\\
$(212,81)$ & 11 & $(4,1)$ & 3 & 4 & YES & YES & YES & $1.57$ & $(2,3)$ & -- & 7568\\
$(212,81)$ & 11 & $(4,1)$ & 3 & 4 & YES & YES & YES & $1.43$ & $(2,3)$ & NO & 7569\\
$(212,81)$ & 11 & $(5,2)$ & 3 & 1 & YES & YES & YES & $1.43$ & $(2,3)$ & NO & 7570\\
$(212,87)$ & 13 & $(5,1)$ & 4 & 1 & YES & YES & YES & $1.71$ & $(2,3)$ & NO & 7571\\
$(212,81)$ & 11 & $(13,5)$ & 5 & 1 & YES & YES & YES & $1.43$ & $(2,3)$ & NO & 7572\\
$(212,89)$ & 11 & $(17,7)$ & 6 & 1 & YES & YES & YES & $1.57$ & $(2,3)$ & NO & 7573\\
$(212,81)$ & 11 & $(34,13)$ & 7 & 2 & YES & YES & YES & $1.57$ & $(2,3)$ & 6501 & 7574\\
$(212,89)$ & 11 & $(43,18)$ & 8 & 1 & YES & YES & YES & $1.43$ & $(2,3)$ & NO & 7575\\
$(212,87)$ & 13 & $(56,23)$ & 9 & 4 & YES & YES & YES & $1.71$ & $(2,3)$ & NO & 7576\\
$(212,87)$ & 13 & $(95,39)$ & 10 & 1 & YES & YES & YES & $1.71$ & $(2,3)$ & NO & 7577\\
$(212,81)$ & 11 & $(123,47)$ & 10 & 1 & YES & YES & YES & $1.29$ & $(2,3)$ & NO & 7578\\
$(212,81)$ & 11 & $(212,81)$ & 11 & 212 & YES & YES & YES & $1.29$ & $(2,3)$ & NO & 7579\\
$(213,65)$ & 12 & $(2,1)$ & 1 & 1 & YES & YES & YES & $1.57$ & $(2,3)$ & -- & 7580\\
$(213,77)$ & 12 & $(2,1)$ & 1 & 1 & YES & YES & NO(2) & $1.62$ & $(2,3)$ & NO & 7581\\
$(213,83)$ & 12 & $(2,1)$ & 1 & 1 & YES & YES & NO(2) & $1.75$ & $(2,3)$ & NO & 7582\\
$(213,83)$ & 12 & $(2,1)$ & 1 & 1 & YES & YES & NO(2) & $1.62$ & $(2,3)$ & -- & 7583\\
$(213,79)$ & 12 & $(3,1)$ & 2 & 3 & YES & YES & NO(2) & $1.75$ & $(2,3)$ & NO & 7584\\
$(213,79)$ & 12 & $(3,1)$ & 2 & 3 & YES & YES & NO(2) & $1.75$ & $(2,3)$ & -- & 7585\\
$(213,89)$ & 13 & $(3,1)$ & 2 & 3 & YES & YES & YES & $1.57$ & $(2,3)$ & -- & 7586\\
$(213,59)$ & 12 & $(5,2)$ & 3 & 1 & YES & YES & YES & $1.57$ & $(2,3)$ & NO & 7587\\
$(213,83)$ & 12 & $(5,1)$ & 4 & 1 & YES & YES & NO(2) & $1.50$ & $(2,3)$ & NO & 7588\\
$(213,77)$ & 12 & $(8,3)$ & 4 & 1 & YES & YES & YES & $1.57$ & $(2,3)$ & NO & 7589\\
$(213,62)$ & 12 & $(17,5)$ & 6 & 1 & YES & YES & YES & $1.57$ & $(2,3)$ & NO & 7590\\
$(213,83)$ & 12 & $(59,23)$ & 9 & 1 & YES & YES & NO(2) & $1.62$ & $(2,3)$ & 7242 & 7591\\
$(213,83)$ & 12 & $(77,30)$ & 10 & 1 & YES & YES & NO(2) & $1.50$ & $(2,3)$ & NO & 7592\\
$(213,77)$ & 12 & $(83,30)$ & 10 & 1 & YES & YES & NO(2) & $1.75$ & $(2,3)$ & NO & 7593\\
$(214,83)$ & 12 & $(2,1)$ & 1 & 2 & YES & YES & YES & $1.57$ & $(2,3)$ & -- & 7594\\
$(214,83)$ & 12 & $(2,1)$ & 1 & 2 & YES & YES & YES & $1.71$ & $(2,3)$ & NO & 7595\\
$(214,65)$ & 12 & $(3,1)$ & 2 & 1 & NO & YES & YES & $1.57$ & $(2,3)$ & -- & 7596\\
$(214,83)$ & 12 & $(3,1)$ & 2 & 1 & YES & YES & YES & $1.57$ & $(2,3)$ & -- & 7597\\
$(214,83)$ & 12 & $(4,1)$ & 3 & 2 & YES & YES & YES & $1.57$ & $(2,3)$ & -- & 7598\\
$(214,83)$ & 12 & $(4,1)$ & 3 & 2 & YES & YES & YES & $1.57$ & $(2,3)$ & NO & 7599\\
$(214,83)$ & 12 & $(5,1)$ & 4 & 1 & YES & YES & YES & $1.57$ & $(2,3)$ & NO & 7600\\
$(214,83)$ & 12 & $(6,1)$ & 5 & 2 & YES & YES & YES & $1.57$ & $(2,3)$ & NO & 7601\\
$(214,47)$ & 14 & $(7,2)$ & 4 & 1 & YES & YES & YES & $1.71$ & $(2,3)$ & NO & 7602\\
$(214,79)$ & 12 & $(8,3)$ & 4 & 2 & YES & YES & YES & $1.57$ & $(2,3)$ & 5038 & 7603\\
$(214,83)$ & 12 & $(18,7)$ & 6 & 2 & YES & YES & YES & $1.71$ & $(2,3)$ & NO & 7604\\
$(214,83)$ & 12 & $(31,12)$ & 7 & 1 & YES & YES & YES & $1.57$ & $(2,3)$ & NO & 7605\\
$(214,83)$ & 12 & $(116,45)$ & 10 & 2 & YES & YES & YES & $1.71$ & $(2,3)$ & 8435 & 7606\\
$(214,83)$ & 12 & $(214,83)$ & 12 & 214 & YES & YES & YES & $1.57$ & $(2,3)$ & NO & 7607\\
$(215,97)$ & 12 & $(2,1)$ & 1 & 1 & YES & YES & YES & $1.57$ & $(2,3)$ & -- & 7608\\
$(215,51)$ & 13 & $(3,1)$ & 2 & 1 & YES & YES & YES & $1.43$ & $(2,3)$ & -- & 7609\\
$(215,63)$ & 12 & $(3,1)$ & 2 & 1 & YES & YES & YES & $1.43$ & $(2,3)$ & -- & 7610\\
$(215,82)$ & 12 & $(3,1)$ & 2 & 1 & YES & YES & YES & $1.71$ & $(2,3)$ & NO & 7611\\
$(215,59)$ & 13 & $(5,1)$ & 4 & 5 & YES & YES & YES & $1.57$ & $(2,3)$ & -- & 7612\\
$(215,82)$ & 12 & $(5,1)$ & 4 & 5 & YES & YES & YES & $1.57$ & $(2,3)$ & -- & 7613\\
$(215,51)$ & 13 & $(9,2)$ & 5 & 1 & YES & YES & YES & $1.43$ & $(2,3)$ & NO & 7614\\
$(215,97)$ & 12 & $(9,4)$ & 5 & 1 & YES & YES & YES & $1.57$ & $(2,3)$ & 7413 & 7615\\
$(215,51)$ & 13 & $(13,3)$ & 6 & 1 & YES & YES & YES & $1.43$ & $(2,3)$ & NO & 7616\\
$(215,63)$ & 12 & $(31,9)$ & 8 & 1 & YES & YES & YES & $1.71$ & $(2,3)$ & NO & 7617\\
$(215,59)$ & 13 & $(40,11)$ & 8 & 5 & YES & YES & YES & $1.57$ & $(2,3)$ & NO & 7618\\
$(215,97)$ & 12 & $(51,23)$ & 9 & 1 & YES & YES & YES & $1.71$ & $(2,3)$ & 7018 & 7619\\
$(215,82)$ & 12 & $(76,29)$ & 9 & 1 & YES & YES & YES & $1.57$ & $(2,3)$ & NO & 7620\\
$(215,51)$ & 13 & $(135,32)$ & 12 & 5 & YES & YES & YES & $1.43$ & $(2,3)$ & 8929 & 7621\\
$(215,79)$ & 12 & $(215,79)$ & 12 & 215 & YES & YES & YES & $1.57$ & $(2,3)$ & NO & 7622\\
$(216,49)$ & 13 & $(49,11)$ & 10 & 1 & YES & YES & YES & $1.57$ & $(2,3)$ & NO & 7623\\
$(217,92)$ & 12 & $(2,1)$ & 1 & 1 & YES & YES & YES & $1.71$ & $(2,3)$ & -- & 7624\\
$(217,68)$ & 14 & $(3,1)$ & 2 & 1 & YES & YES & YES & $1.71$ & $(2,3)$ & -- & 7625\\
$(217,85)$ & 13 & $(3,1)$ & 2 & 1 & YES & YES & YES & $1.71$ & $(2,3)$ & NO & 7626\\
$(217,60)$ & 12 & $(5,2)$ & 3 & 1 & YES & YES & YES & $1.57$ & $(2,3)$ & -- & 7627\\
$(217,64)$ & 12 & $(5,2)$ & 3 & 1 & YES & YES & YES & $1.57$ & $(2,3)$ & -- & 7628\\
$(217,60)$ & 12 & $(10,3)$ & 5 & 1 & YES & YES & YES & $1.57$ & $(2,3)$ & NO & 7629\\
$(217,68)$ & 14 & $(13,4)$ & 6 & 1 & YES & YES & YES & $1.71$ & $(2,3)$ & NO & 7630\\
$(217,78)$ & 12 & $(25,9)$ & 7 & 1 & YES & YES & NO(2) & $1.75$ & $(2,3)$ & NO & 7631\\
$(217,92)$ & 12 & $(26,11)$ & 7 & 1 & YES & YES & YES & $1.57$ & $(2,3)$ & 5875 & 7632\\
$(217,68)$ & 14 & $(83,26)$ & 12 & 1 & YES & YES & YES & $1.71$ & $(2,3)$ & 7828 & 7633\\
$(218,49)$ & 13 & $(7,2)$ & 4 & 1 & YES & YES & YES & $1.57$ & $(2,3)$ & -- & 7634\\
$(218,47)$ & 13 & $(9,2)$ & 5 & 1 & YES & YES & YES & $1.43$ & $(2,3)$ & NO & 7635\\
$(218,33)$ & 15 & $(11,2)$ & 6 & 1 & YES & YES & YES & $1.71$ & $(2,3)$ & NO & 7636\\
$(218,85)$ & 12 & $(18,7)$ & 6 & 2 & YES & YES & NO(2) & $1.50$ & $(2,3)$ & NO & 7637\\
$(218,47)$ & 13 & $(37,8)$ & 8 & 1 & YES & YES & YES & $1.43$ & $(2,3)$ & NO & 7638\\
$(218,49)$ & 13 & $(58,13)$ & 11 & 2 & YES & YES & YES & $1.57$ & $(2,3)$ & NO & 7639\\
$(218,85)$ & 12 & $(59,23)$ & 9 & 1 & YES & YES & NO(2) & $1.62$ & $(2,3)$ & NO & 7640\\
$(218,85)$ & 12 & $(77,30)$ & 10 & 1 & YES & YES & YES & $1.71$ & $(2,3)$ & NO & 7641\\
$(219,67)$ & 12 & $(2,1)$ & 1 & 1 & YES & YES & YES & $1.57$ & $(2,3)$ & NO & 7642\\
$(219,83)$ & 12 & $(2,1)$ & 1 & 1 & YES & YES & YES & $1.43$ & $(2,3)$ & 4604 & 7643\\
$(219,83)$ & 12 & $(3,1)$ & 2 & 3 & YES & YES & YES & $1.57$ & $(2,3)$ & -- & 7644\\
$(219,85)$ & 12 & $(3,1)$ & 2 & 3 & YES & YES & YES & $1.57$ & $(2,3)$ & NO & 7645\\
$(219,95)$ & 12 & $(3,1)$ & 2 & 3 & YES & YES & YES & $1.57$ & $(2,3)$ & -- & 7646\\
$(219,61)$ & 12 & $(4,1)$ & 3 & 1 & YES & YES & NO(2) & $1.50$ & $(2,3)$ & NO & 7647\\
$(219,83)$ & 12 & $(4,1)$ & 3 & 1 & YES & YES & YES & $1.43$ & $(2,3)$ & NO & 7648\\
$(219,83)$ & 12 & $(4,1)$ & 3 & 1 & YES & YES & YES & $1.57$ & $(2,3)$ & -- & 7649\\
$(219,61)$ & 12 & $(5,2)$ & 3 & 1 & YES & YES & YES & $1.57$ & $(2,3)$ & NO & 7650\\
$(219,95)$ & 12 & $(5,1)$ & 4 & 1 & YES & YES & YES & $1.43$ & $(2,3)$ & NO & 7651\\
$(219,95)$ & 12 & $(5,2)$ & 3 & 1 & YES & YES & YES & $1.57$ & $(2,3)$ & NO & 7652\\
$(219,61)$ & 12 & $(10,3)$ & 5 & 1 & YES & YES & YES & $1.57$ & $(2,3)$ & NO & 7653\\
$(219,67)$ & 12 & $(10,3)$ & 5 & 1 & YES & YES & YES & $1.43$ & $(2,3)$ & 6089 & 7654\\
$(219,79)$ & 12 & $(11,4)$ & 5 & 1 & YES & YES & NO(2) & $1.75$ & $(2,3)$ & NO & 7655\\
$(219,61)$ & 12 & $(25,7)$ & 7 & 1 & YES & YES & YES & $1.43$ & $(2,3)$ & NO & 7656\\
$(219,64)$ & 12 & $(31,9)$ & 8 & 1 & YES & YES & YES & $1.43$ & $(2,3)$ & NO & 7657\\
$(219,95)$ & 12 & $(83,36)$ & 10 & 1 & YES & YES & YES & $1.57$ & $(2,3)$ & NO & 7658\\
$(219,67)$ & 12 & $(85,26)$ & 10 & 1 & YES & YES & YES & $1.57$ & $(2,3)$ & NO & 7659\\
$(219,79)$ & 12 & $(133,48)$ & 11 & 1 & YES & YES & YES & $1.57$ & $(2,3)$ & 8947 & 7660\\
$(220,61)$ & 13 & $(7,1)$ & 6 & 1 & YES & YES & YES & $1.57$ & $(2,3)$ & NO & 7661\\
$(221,47)$ & 13 & $(3,1)$ & 2 & 1 & YES & YES & NO(2) & $1.71$ & $(4,2)$ & NO & 7662\\
$(221,47)$ & 13 & $(3,1)$ & 2 & 1 & YES & YES & NO(2) & $1.71$ & $(4,2)$ & -- & 7663\\
$(221,58)$ & 13 & $(3,1)$ & 2 & 1 & YES & YES & YES & $1.57$ & $(2,3)$ & -- & 7664\\
$(221,58)$ & 13 & $(3,1)$ & 2 & 1 & YES & YES & YES & $1.57$ & $(2,3)$ & NO & 7665\\
$(221,82)$ & 12 & $(4,1)$ & 3 & 1 & YES & YES & YES & $1.57$ & $(2,3)$ & -- & 7666\\
$(221,80)$ & 13 & $(5,1)$ & 4 & 1 & YES & YES & YES & $1.71$ & $(2,3)$ & NO & 7667\\
$(221,62)$ & 12 & $(11,3)$ & 5 & 1 & YES & YES & YES & $1.43$ & $(2,3)$ & NO & 7668\\
$(221,47)$ & 13 & $(13,2)$ & 7 & 13 & YES & YES & YES & $1.57$ & $(2,3)$ & NO & 7669\\
$(221,58)$ & 13 & $(15,4)$ & 6 & 1 & YES & YES & YES & $1.57$ & $(2,3)$ & NO & 7670\\
$(221,58)$ & 13 & $(99,26)$ & 12 & 1 & YES & YES & YES & $1.57$ & $(2,3)$ & NO & 7671\\
$(221,80)$ & 13 & $(105,38)$ & 11 & 1 & YES & YES & YES & $1.71$ & $(2,3)$ & 8307 & 7672\\
$(221,47)$ & 13 & $(113,24)$ & 11 & 1 & YES & YES & YES & $1.57$ & $(2,3)$ & NO & 7673\\
$(221,84)$ & 12 & $(221,84)$ & 12 & 221 & YES & YES & YES & $1.71$ & $(2,3)$ & NO & 7674\\
$(222,91)$ & 12 & $(2,1)$ & 1 & 2 & YES & YES & YES & $1.57$ & $(2,3)$ & NO & 7675\\
$(222,91)$ & 12 & $(22,9)$ & 7 & 2 & YES & YES & YES & $1.71$ & $(2,3)$ & NO & 7676\\
$(222,91)$ & 12 & $(39,16)$ & 8 & 3 & YES & YES & YES & $1.43$ & $(2,3)$ & NO & 7677\\
$(223,82)$ & 12 & $(2,1)$ & 1 & 1 & YES & YES & NO(2) & $1.62$ & $(2,3)$ & NO & 7678\\
$(223,82)$ & 12 & $(2,1)$ & 1 & 1 & YES & YES & NO(2) & $1.62$ & $(2,3)$ & -- & 7679\\
$(223,92)$ & 12 & $(2,1)$ & 1 & 1 & YES & YES & YES & $1.71$ & $(2,3)$ & -- & 7680\\
$(223,66)$ & 12 & $(3,1)$ & 2 & 1 & NO & YES & YES & $1.57$ & $(2,3)$ & -- & 7681\\
$(223,98)$ & 12 & $(3,1)$ & 2 & 1 & YES & YES & YES & $1.71$ & $(2,3)$ & NO & 7682\\
$(223,80)$ & 12 & $(5,2)$ & 3 & 1 & YES & YES & YES & $1.43$ & $(2,3)$ & NO & 7683\\
$(223,92)$ & 12 & $(12,5)$ & 5 & 1 & YES & YES & YES & $1.71$ & $(2,3)$ & NO & 7684\\
$(223,92)$ & 12 & $(46,19)$ & 8 & 1 & YES & YES & YES & $1.43$ & $(2,3)$ & 8432 & 7685\\
$(223,92)$ & 12 & $(63,26)$ & 9 & 1 & YES & YES & YES & $1.57$ & $(2,3)$ & 7443 & 7686\\
$(223,98)$ & 12 & $(91,40)$ & 10 & 1 & YES & YES & YES & $1.71$ & $(2,3)$ & 8019 & 7687\\
$(223,68)$ & 12 & $(223,68)$ & 12 & 223 & YES & YES & YES & $1.43$ & $(2,3)$ & NO & 7688\\
$(224,93)$ & 13 & $(4,1)$ & 3 & 4 & YES & YES & YES & $1.71$ & $(2,3)$ & -- & 7689\\
$(224,93)$ & 13 & $(171,71)$ & 12 & 1 & YES & YES & YES & $1.71$ & $(2,3)$ & NO & 7690\\
$(224,93)$ & 13 & $(224,93)$ & 13 & 224 & YES & YES & YES & $1.71$ & $(2,3)$ & NO & 7691\\
$(225,98)$ & 12 & $(4,1)$ & 3 & 1 & YES & YES & YES & $1.57$ & $(2,3)$ & -- & 7692\\
$(225,98)$ & 12 & $(7,3)$ & 4 & 1 & YES & YES & YES & $1.71$ & $(2,3)$ & NO & 7693\\
$(225,61)$ & 13 & $(15,4)$ & 6 & 15 & YES & YES & YES & $1.71$ & $(2,3)$ & NO & 7694\\
$(225,49)$ & 13 & $(17,4)$ & 7 & 1 & YES & YES & YES & $1.57$ & $(2,3)$ & NO & 7695\\
$(225,43)$ & 14 & $(21,4)$ & 8 & 3 & YES & YES & YES & $1.86$ & $(2,3)$ & NO & 7696\\
$(225,98)$ & 12 & $(62,27)$ & 9 & 1 & YES & YES & YES & $1.71$ & $(2,3)$ & NO & 7697\\
$(225,98)$ & 12 & $(225,98)$ & 12 & 225 & YES & YES & YES & $1.57$ & $(2,3)$ & NO & 7698\\
$(226,95)$ & 12 & $(2,1)$ & 1 & 2 & YES & YES & YES & $1.57$ & $(2,3)$ & NO & 7699\\
$(226,95)$ & 12 & $(2,1)$ & 1 & 2 & YES & YES & YES & $1.57$ & $(2,3)$ & -- & 7700\\
$(226,61)$ & 12 & $(3,1)$ & 2 & 1 & YES & YES & YES & $1.43$ & $(2,3)$ & -- & 7701\\
$(226,69)$ & 12 & $(3,1)$ & 2 & 1 & YES & YES & YES & $1.43$ & $(2,3)$ & -- & 7702\\
$(226,95)$ & 12 & $(3,1)$ & 2 & 1 & YES & YES & YES & $1.43$ & $(2,3)$ & -- & 7703\\
$(226,95)$ & 12 & $(3,1)$ & 2 & 1 & YES & YES & YES & $1.57$ & $(2,3)$ & NO & 7704\\
$(226,53)$ & 13 & $(5,2)$ & 3 & 1 & YES & YES & YES & $1.57$ & $(2,3)$ & -- & 7705\\
$(226,53)$ & 13 & $(5,2)$ & 3 & 1 & YES & YES & YES & $1.71$ & $(2,3)$ & NO & 7706\\
$(226,61)$ & 12 & $(5,1)$ & 4 & 1 & YES & YES & YES & $1.29$ & $(2,3)$ & NO & 7707\\
$(226,95)$ & 12 & $(12,5)$ & 5 & 2 & YES & YES & YES & $1.57$ & $(2,3)$ & 5090 & 7708\\
$(226,95)$ & 12 & $(19,8)$ & 6 & 1 & YES & YES & YES & $1.57$ & $(2,3)$ & NO & 7709\\
$(226,95)$ & 12 & $(88,37)$ & 10 & 2 & YES & YES & YES & $1.71$ & $(2,3)$ & 7979 & 7710\\
$(227,83)$ & 12 & $(2,1)$ & 1 & 1 & YES & YES & NO(2) & $1.62$ & $(2,3)$ & NO & 7711\\
$(227,83)$ & 12 & $(2,1)$ & 1 & 1 & YES & YES & NO(2) & $1.62$ & $(2,3)$ & -- & 7712\\
$(227,88)$ & 12 & $(2,1)$ & 1 & 1 & YES & YES & YES & $1.57$ & $(2,3)$ & -- & 7713\\
$(227,93)$ & 12 & $(2,1)$ & 1 & 1 & YES & YES & YES & $1.71$ & $(2,3)$ & -- & 7714\\
$(227,100)$ & 12 & $(2,1)$ & 1 & 1 & YES & YES & YES & $1.43$ & $(2,3)$ & -- & 7715\\
$(227,84)$ & 12 & $(3,1)$ & 2 & 1 & YES & YES & YES & $1.43$ & $(2,3)$ & -- & 7716\\
$(227,86)$ & 12 & $(3,1)$ & 2 & 1 & YES & YES & NO(2) & $1.62$ & $(2,3)$ & NO & 7717\\
$(227,87)$ & 12 & $(3,1)$ & 2 & 1 & YES & YES & YES & $1.57$ & $(2,3)$ & -- & 7718\\
$(227,88)$ & 12 & $(3,1)$ & 2 & 1 & YES & YES & YES & $1.57$ & $(2,3)$ & -- & 7719\\
$(227,87)$ & 12 & $(4,1)$ & 3 & 1 & YES & YES & YES & $1.57$ & $(2,3)$ & -- & 7720\\
$(227,88)$ & 12 & $(4,1)$ & 3 & 1 & YES & YES & YES & $1.57$ & $(2,3)$ & -- & 7721\\
$(227,88)$ & 12 & $(4,1)$ & 3 & 1 & YES & YES & YES & $1.57$ & $(2,3)$ & NO & 7722\\
$(227,94)$ & 12 & $(4,1)$ & 3 & 1 & YES & YES & YES & $1.57$ & $(2,3)$ & NO & 7723\\
$(227,100)$ & 12 & $(4,1)$ & 3 & 1 & YES & YES & YES & $1.57$ & $(2,3)$ & -- & 7724\\
$(227,66)$ & 12 & $(5,2)$ & 3 & 1 & YES & YES & YES & $1.57$ & $(2,3)$ & -- & 7725\\
$(227,67)$ & 12 & $(5,2)$ & 3 & 1 & YES & YES & YES & $1.57$ & $(2,3)$ & NO & 7726\\
$(227,67)$ & 12 & $(5,2)$ & 3 & 1 & YES & YES & YES & $1.57$ & $(2,3)$ & -- & 7727\\
$(227,88)$ & 12 & $(5,1)$ & 4 & 1 & YES & YES & YES & $1.57$ & $(2,3)$ & NO & 7728\\
$(227,88)$ & 12 & $(5,2)$ & 3 & 1 & YES & YES & YES & $1.71$ & $(2,3)$ & NO & 7729\\
$(227,93)$ & 12 & $(5,1)$ & 4 & 1 & YES & YES & YES & $1.57$ & $(2,3)$ & -- & 7730\\
$(227,69)$ & 13 & $(6,1)$ & 5 & 1 & YES & YES & NO(2) & $1.62$ & $(2,3)$ & NO & 7731\\
$(227,88)$ & 12 & $(6,1)$ & 5 & 1 & YES & YES & YES & $1.57$ & $(2,3)$ & NO & 7732\\
$(227,61)$ & 12 & $(7,2)$ & 4 & 1 & YES & YES & YES & $1.57$ & $(2,3)$ & -- & 7733\\
$(227,83)$ & 12 & $(8,3)$ & 4 & 1 & YES & YES & NO(2) & $1.75$ & $(2,3)$ & NO & 7734\\
$(227,84)$ & 12 & $(8,3)$ & 4 & 1 & YES & YES & NO(2) & $1.62$ & $(2,3)$ & NO & 7735\\
$(227,66)$ & 12 & $(9,2)$ & 5 & 1 & YES & YES & YES & $1.43$ & $(2,3)$ & NO & 7736\\
$(227,99)$ & 12 & $(9,4)$ & 5 & 1 & YES & YES & YES & $1.57$ & $(2,3)$ & NO & 7737\\
$(227,61)$ & 12 & $(10,3)$ & 5 & 1 & YES & YES & YES & $1.57$ & $(2,3)$ & NO & 7738\\
$(227,69)$ & 13 & $(10,3)$ & 5 & 1 & YES & YES & NO(2) & $1.62$ & $(2,3)$ & NO & 7739\\
$(227,93)$ & 12 & $(17,7)$ & 6 & 1 & YES & YES & YES & $1.71$ & $(2,3)$ & NO & 7740\\
$(227,100)$ & 12 & $(25,11)$ & 7 & 1 & YES & YES & YES & $1.57$ & $(2,3)$ & NO & 7741\\
$(227,88)$ & 12 & $(31,12)$ & 7 & 1 & YES & YES & YES & $1.57$ & $(2,3)$ & NO & 7742\\
$(227,83)$ & 12 & $(41,15)$ & 8 & 1 & YES & YES & NO(2) & $1.62$ & $(2,3)$ & 6834 & 7743\\
$(227,93)$ & 12 & $(61,25)$ & 9 & 1 & YES & YES & YES & $1.57$ & $(2,3)$ & 7419 & 7744\\
$(227,88)$ & 12 & $(67,26)$ & 9 & 1 & YES & YES & YES & $1.71$ & $(2,3)$ & NO & 7745\\
$(227,66)$ & 12 & $(79,23)$ & 10 & 1 & YES & YES & YES & $1.43$ & $(2,3)$ & NO & 7746\\
$(227,88)$ & 12 & $(80,31)$ & 9 & 1 & YES & YES & YES & $1.57$ & $(2,3)$ & NO & 7747\\
$(227,83)$ & 12 & $(93,34)$ & 10 & 1 & YES & YES & NO(2) & $1.62$ & $(2,3)$ & NO & 7748\\
$(227,87)$ & 12 & $(107,41)$ & 10 & 1 & YES & YES & YES & $1.57$ & $(2,3)$ & 8377 & 7749\\
$(227,88)$ & 12 & $(129,50)$ & 10 & 1 & YES & YES & YES & $1.57$ & $(2,3)$ & 8641 & 7750\\
$(227,67)$ & 12 & $(166,49)$ & 11 & 1 & YES & YES & NO(3) & $1.29$ & $(2,3)$ & NO & 7751\\
$(227,88)$ & 12 & $(178,69)$ & 11 & 1 & YES & YES & YES & $1.57$ & $(2,3)$ & NO & 7752\\
$(227,88)$ & 12 & $(227,88)$ & 12 & 227 & YES & YES & YES & $1.57$ & $(2,3)$ & NO & 7753\\
$(227,94)$ & 12 & $(227,94)$ & 12 & 227 & YES & YES & YES & $1.43$ & $(2,3)$ & NO & 7754\\
$(229,87)$ & 12 & $(2,1)$ & 1 & 1 & YES & YES & YES & $1.57$ & $(2,3)$ & -- & 7755\\
$(229,94)$ & 12 & $(2,1)$ & 1 & 1 & YES & YES & NO(2) & $1.75$ & $(2,3)$ & -- & 7756\\
$(229,97)$ & 12 & $(2,1)$ & 1 & 1 & YES & YES & NO(2) & $1.62$ & $(2,3)$ & -- & 7757\\
$(229,68)$ & 12 & $(3,1)$ & 2 & 1 & YES & YES & NO(3) & $1.29$ & $(2,3)$ & -- & 7758\\
$(229,68)$ & 12 & $(3,1)$ & 2 & 1 & YES & YES & YES & $1.43$ & $(2,3)$ & NO & 7759\\
$(229,87)$ & 12 & $(3,1)$ & 2 & 1 & YES & YES & YES & $1.57$ & $(2,3)$ & NO & 7760\\
$(229,87)$ & 12 & $(3,1)$ & 2 & 1 & YES & YES & YES & $1.57$ & $(2,3)$ & -- & 7761\\
$(229,95)$ & 12 & $(3,1)$ & 2 & 1 & YES & YES & YES & $1.71$ & $(2,3)$ & NO & 7762\\
$(229,95)$ & 12 & $(3,1)$ & 2 & 1 & YES & YES & YES & $1.71$ & $(2,3)$ & -- & 7763\\
$(229,95)$ & 12 & $(3,1)$ & 2 & 1 & YES & YES & YES & $1.71$ & $(2,3)$ & NO & 7764\\
$(229,87)$ & 12 & $(4,1)$ & 3 & 1 & YES & YES & YES & $1.71$ & $(2,3)$ & -- & 7765\\
$(229,95)$ & 12 & $(4,1)$ & 3 & 1 & YES & YES & YES & $1.57$ & $(2,3)$ & -- & 7766\\
$(229,68)$ & 12 & $(5,1)$ & 4 & 1 & YES & YES & YES & $1.43$ & $(2,3)$ & NO & 7767\\
$(229,95)$ & 12 & $(5,2)$ & 3 & 1 & YES & YES & NO(2) & $1.62$ & $(2,3)$ & NO & 7768\\
$(229,95)$ & 12 & $(8,3)$ & 4 & 1 & YES & YES & YES & $1.57$ & $(2,3)$ & NO & 7769\\
$(229,94)$ & 12 & $(12,5)$ & 5 & 1 & YES & YES & YES & $1.71$ & $(2,3)$ & NO & 7770\\
$(229,95)$ & 12 & $(17,7)$ & 6 & 1 & YES & YES & YES & $1.57$ & $(2,3)$ & NO & 7771\\
$(229,87)$ & 12 & $(29,11)$ & 7 & 1 & YES & YES & YES & $1.57$ & $(2,3)$ & NO & 7772\\
$(229,64)$ & 12 & $(32,9)$ & 8 & 1 & YES & YES & YES & $1.43$ & $(2,3)$ & NO & 7773\\
$(229,85)$ & 12 & $(35,13)$ & 8 & 1 & YES & YES & NO(2) & $1.62$ & $(2,3)$ & 6678 & 7774\\
$(229,62)$ & 13 & $(37,10)$ & 8 & 1 & YES & YES & NO(2) & $1.75$ & $(2,3)$ & NO & 7775\\
$(229,97)$ & 12 & $(59,25)$ & 9 & 1 & YES & YES & NO(2) & $1.62$ & $(2,3)$ & 7398 & 7776\\
$(229,87)$ & 12 & $(79,30)$ & 9 & 1 & YES & YES & YES & $1.57$ & $(2,3)$ & NO & 7777\\
$(229,97)$ & 12 & $(85,36)$ & 10 & 1 & YES & YES & YES & $1.57$ & $(2,3)$ & NO & 7778\\
$(229,68)$ & 12 & $(101,30)$ & 10 & 1 & YES & YES & YES & $1.29$ & $(2,3)$ & 8270 & 7779\\
$(229,87)$ & 12 & $(129,49)$ & 10 & 1 & YES & YES & YES & $1.71$ & $(2,3)$ & 8649 & 7780\\
$(229,94)$ & 12 & $(134,55)$ & 11 & 1 & YES & YES & YES & $1.57$ & $(2,3)$ & NO & 7781\\
$(229,87)$ & 12 & $(179,68)$ & 11 & 1 & YES & YES & YES & $1.71$ & $(2,3)$ & NO & 7782\\
$(229,68)$ & 12 & $(229,68)$ & 12 & 229 & YES & YES & YES & $1.43$ & $(2,3)$ & NO & 7783\\
$(229,87)$ & 12 & $(229,87)$ & 12 & 229 & YES & YES & YES & $1.43$ & $(2,3)$ & NO & 7784\\
$(229,95)$ & 12 & $(229,95)$ & 12 & 229 & YES & YES & YES & $1.71$ & $(2,3)$ & NO & 7785\\
$(229,97)$ & 12 & $(229,97)$ & 12 & 229 & YES & YES & YES & $1.43$ & $(2,3)$ & NO & 7786\\
$(230,67)$ & 13 & $(3,1)$ & 2 & 1 & YES & YES & NO(2) & $1.75$ & $(2,3)$ & NO & 7787\\
$(230,67)$ & 13 & $(3,1)$ & 2 & 1 & YES & YES & YES & $1.57$ & $(2,3)$ & -- & 7788\\
$(230,97)$ & 12 & $(3,1)$ & 2 & 1 & YES & YES & YES & $1.57$ & $(2,3)$ & -- & 7789\\
$(230,67)$ & 13 & $(4,1)$ & 3 & 2 & YES & YES & YES & $1.57$ & $(2,3)$ & -- & 7790\\
$(230,61)$ & 13 & $(5,2)$ & 3 & 5 & YES & YES & YES & $1.57$ & $(2,3)$ & -- & 7791\\
$(230,61)$ & 13 & $(10,3)$ & 5 & 10 & YES & YES & YES & $1.57$ & $(2,3)$ & NO & 7792\\
$(230,67)$ & 13 & $(24,7)$ & 7 & 2 & YES & YES & YES & $1.71$ & $(2,3)$ & 6373 & 7793\\
$(230,67)$ & 13 & $(55,16)$ & 9 & 5 & YES & YES & YES & $1.57$ & $(2,3)$ & 6095 & 7794\\
$(230,97)$ & 12 & $(64,27)$ & 9 & 2 & YES & YES & YES & $1.71$ & $(2,3)$ & 7560 & 7795\\
$(230,67)$ & 13 & $(79,23)$ & 10 & 1 & YES & YES & YES & $1.57$ & $(2,3)$ & NO & 7796\\
$(230,67)$ & 13 & $(103,30)$ & 11 & 1 & YES & YES & YES & $1.71$ & $(2,3)$ & NO & 7797\\
$(231,53)$ & 13 & $(3,1)$ & 2 & 3 & YES & YES & YES & $1.43$ & $(2,3)$ & -- & 7798\\
$(231,61)$ & 13 & $(3,1)$ & 2 & 3 & YES & YES & YES & $1.43$ & $(2,3)$ & -- & 7799\\
$(231,67)$ & 13 & $(3,1)$ & 2 & 3 & YES & YES & NO(2) & $1.75$ & $(2,3)$ & 4686 & 7800\\
$(231,67)$ & 13 & $(3,1)$ & 2 & 3 & YES & YES & YES & $1.43$ & $(2,3)$ & -- & 7801\\
$(231,67)$ & 13 & $(4,1)$ & 3 & 1 & YES & YES & YES & $1.43$ & $(2,3)$ & -- & 7802\\
$(231,67)$ & 13 & $(7,2)$ & 4 & 7 & YES & YES & YES & $1.57$ & $(2,3)$ & NO & 7803\\
$(231,53)$ & 13 & $(11,2)$ & 6 & 11 & YES & YES & YES & $1.43$ & $(2,3)$ & NO & 7804\\
$(231,67)$ & 13 & $(17,5)$ & 6 & 1 & YES & YES & YES & $1.43$ & $(2,3)$ & NO & 7805\\
$(231,53)$ & 13 & $(22,5)$ & 7 & 11 & YES & YES & YES & $1.43$ & $(2,3)$ & NO & 7806\\
$(231,61)$ & 13 & $(34,9)$ & 8 & 1 & YES & YES & YES & $1.57$ & $(2,3)$ & NO & 7807\\
$(231,53)$ & 13 & $(35,8)$ & 8 & 7 & YES & YES & YES & $1.43$ & $(2,3)$ & NO & 7808\\
$(233,71)$ & 13 & $(2,1)$ & 1 & 1 & YES & YES & YES & $1.57$ & $(2,3)$ & -- & 7809\\
$(233,84)$ & 12 & $(2,1)$ & 1 & 1 & YES & YES & YES & $1.71$ & $(2,3)$ & -- & 7810\\
$(233,86)$ & 12 & $(2,1)$ & 1 & 1 & YES & YES & NO(2) & $1.62$ & $(2,3)$ & -- & 7811\\
$(233,86)$ & 12 & $(2,1)$ & 1 & 1 & YES & YES & NO(2) & $1.62$ & $(2,3)$ & NO & 7812\\
$(233,89)$ & 11 & $(2,1)$ & 1 & 1 & YES & YES & YES & $1.29$ & $(2,3)$ & -- & 7813\\
$(233,91)$ & 12 & $(2,1)$ & 1 & 1 & NO & YES & YES & $1.57$ & $(2,3)$ & -- & 7814\\
$(233,61)$ & 14 & $(3,1)$ & 2 & 1 & YES & YES & YES & $1.57$ & $(2,3)$ & NO & 7815\\
$(233,73)$ & 14 & $(4,1)$ & 3 & 1 & YES & YES & YES & $1.71$ & $(2,3)$ & NO & 7816\\
$(233,84)$ & 12 & $(4,1)$ & 3 & 1 & YES & YES & YES & $1.71$ & $(2,3)$ & -- & 7817\\
$(233,91)$ & 12 & $(4,1)$ & 3 & 1 & YES & YES & YES & $1.57$ & $(2,3)$ & -- & 7818\\
$(233,91)$ & 12 & $(4,1)$ & 3 & 1 & YES & YES & YES & $1.57$ & $(2,3)$ & NO & 7819\\
$(233,91)$ & 12 & $(5,1)$ & 4 & 1 & YES & YES & YES & $1.57$ & $(2,3)$ & NO & 7820\\
$(233,89)$ & 11 & $(8,3)$ & 4 & 1 & YES & YES & YES & $1.43$ & $(2,3)$ & 7491 & 7821\\
$(233,89)$ & 11 & $(13,5)$ & 5 & 1 & YES & YES & YES & $1.43$ & $(2,3)$ & NO & 7822\\
$(233,91)$ & 12 & $(13,5)$ & 5 & 1 & YES & YES & YES & $1.57$ & $(2,3)$ & NO & 7823\\
$(233,89)$ & 11 & $(21,8)$ & 6 & 1 & YES & YES & YES & $1.43$ & $(2,3)$ & 7464 & 7824\\
$(233,86)$ & 12 & $(27,10)$ & 7 & 1 & YES & YES & YES & $1.57$ & $(2,3)$ & NO & 7825\\
$(233,89)$ & 11 & $(34,13)$ & 7 & 1 & YES & YES & YES & $1.43$ & $(2,3)$ & NO & 7826\\
$(233,90)$ & 13 & $(57,22)$ & 9 & 1 & YES & YES & YES & $1.71$ & $(2,3)$ & NO & 7827\\
$(233,73)$ & 14 & $(67,21)$ & 11 & 1 & YES & YES & YES & $1.71$ & $(2,3)$ & 7633 & 7828\\
$(233,91)$ & 12 & $(169,66)$ & 11 & 1 & YES & YES & YES & $1.57$ & $(2,3)$ & NO & 7829\\
$(233,91)$ & 12 & $(233,91)$ & 12 & 233 & YES & YES & YES & $1.57$ & $(2,3)$ & NO & 7830\\
$(234,89)$ & 12 & $(2,1)$ & 1 & 2 & NO & YES & YES & $1.57$ & $(2,3)$ & -- & 7831\\
$(234,89)$ & 12 & $(3,1)$ & 2 & 3 & YES & YES & YES & $1.57$ & $(2,3)$ & -- & 7832\\
$(234,89)$ & 12 & $(13,5)$ & 5 & 13 & YES & YES & YES & $1.57$ & $(2,3)$ & NO & 7833\\
$(234,71)$ & 12 & $(89,27)$ & 10 & 1 & YES & YES & YES & $1.43$ & $(2,3)$ & NO & 7834\\
$(234,89)$ & 12 & $(92,35)$ & 10 & 2 & YES & YES & YES & $1.71$ & $(2,3)$ & 8122 & 7835\\
$(234,53)$ & 13 & $(97,22)$ & 11 & 1 & YES & YES & YES & $1.43$ & $(2,3)$ & NO & 7836\\
$(235,73)$ & 13 & $(2,1)$ & 1 & 1 & YES & YES & YES & $1.71$ & $(2,3)$ & NO & 7837\\
$(235,97)$ & 12 & $(2,1)$ & 1 & 1 & YES & YES & YES & $1.43$ & $(2,3)$ & -- & 7838\\
$(235,63)$ & 12 & $(3,1)$ & 2 & 1 & YES & YES & YES & $1.29$ & $(2,3)$ & -- & 7839\\
$(235,89)$ & 12 & $(3,1)$ & 2 & 1 & YES & YES & YES & $1.43$ & $(2,3)$ & -- & 7840\\
$(235,97)$ & 12 & $(3,1)$ & 2 & 1 & YES & YES & YES & $1.71$ & $(2,3)$ & NO & 7841\\
$(235,97)$ & 12 & $(3,1)$ & 2 & 1 & YES & YES & YES & $1.71$ & $(2,3)$ & -- & 7842\\
$(235,73)$ & 13 & $(4,1)$ & 3 & 1 & YES & YES & YES & $1.57$ & $(2,3)$ & NO & 7843\\
$(235,97)$ & 12 & $(4,1)$ & 3 & 1 & YES & YES & YES & $1.57$ & $(2,3)$ & NO & 7844\\
$(235,73)$ & 13 & $(5,1)$ & 4 & 5 & YES & YES & YES & $1.43$ & $(2,3)$ & 4083 & 7845\\
$(235,97)$ & 12 & $(6,1)$ & 5 & 1 & YES & YES & YES & $1.43$ & $(2,3)$ & NO & 7846\\
$(235,63)$ & 12 & $(26,7)$ & 7 & 1 & YES & YES & YES & $1.43$ & $(2,3)$ & NO & 7847\\
$(235,71)$ & 13 & $(43,13)$ & 9 & 1 & YES & YES & NO(2) & $1.62$ & $(2,3)$ & 6996 & 7848\\
$(235,97)$ & 12 & $(46,19)$ & 8 & 1 & YES & YES & YES & $1.57$ & $(2,3)$ & NO & 7849\\
$(235,97)$ & 12 & $(63,26)$ & 9 & 1 & YES & YES & YES & $1.57$ & $(2,3)$ & NO & 7850\\
$(235,73)$ & 13 & $(74,23)$ & 10 & 1 & YES & YES & YES & $1.43$ & $(2,3)$ & NO & 7851\\
$(235,66)$ & 12 & $(82,23)$ & 10 & 1 & YES & YES & YES & $1.57$ & $(2,3)$ & NO & 7852\\
$(235,73)$ & 13 & $(103,32)$ & 11 & 1 & YES & YES & YES & $1.57$ & $(2,3)$ & NO & 7853\\
$(235,97)$ & 12 & $(109,45)$ & 10 & 1 & YES & YES & YES & $1.43$ & $(2,3)$ & 8436 & 7854\\
$(235,97)$ & 12 & $(172,71)$ & 11 & 1 & YES & YES & YES & $1.57$ & $(2,3)$ & NO & 7855\\
$(236,69)$ & 12 & $(3,1)$ & 2 & 1 & YES & YES & YES & $1.29$ & $(2,3)$ & -- & 7856\\
$(236,69)$ & 12 & $(3,1)$ & 2 & 1 & YES & YES & YES & $1.43$ & $(2,3)$ & NO & 7857\\
$(236,69)$ & 12 & $(4,1)$ & 3 & 4 & YES & YES & YES & $1.43$ & $(2,3)$ & -- & 7858\\
$(236,55)$ & 13 & $(5,2)$ & 3 & 1 & YES & YES & NO(2) & $1.62$ & $(2,3)$ & NO & 7859\\
$(236,69)$ & 12 & $(5,1)$ & 4 & 1 & YES & YES & YES & $1.43$ & $(2,3)$ & NO & 7860\\
$(236,69)$ & 12 & $(5,2)$ & 3 & 1 & YES & YES & YES & $1.57$ & $(2,3)$ & -- & 7861\\
$(236,97)$ & 13 & $(5,1)$ & 4 & 1 & YES & YES & YES & $1.71$ & $(2,3)$ & NO & 7862\\
$(236,97)$ & 13 & $(5,1)$ & 4 & 1 & YES & YES & YES & $1.71$ & $(2,3)$ & NO & 7863\\
$(236,97)$ & 13 & $(5,2)$ & 3 & 1 & YES & YES & YES & $1.71$ & $(2,3)$ & -- & 7864\\
$(236,69)$ & 12 & $(6,1)$ & 5 & 2 & YES & YES & YES & $1.43$ & $(2,3)$ & NO & 7865\\
$(236,69)$ & 12 & $(10,3)$ & 5 & 2 & YES & YES & YES & $1.43$ & $(2,3)$ & NO & 7866\\
$(236,69)$ & 12 & $(11,3)$ & 5 & 1 & YES & YES & YES & $1.57$ & $(2,3)$ & NO & 7867\\
$(236,69)$ & 12 & $(31,9)$ & 8 & 1 & YES & YES & YES & $1.57$ & $(2,3)$ & NO & 7868\\
$(236,73)$ & 13 & $(55,17)$ & 10 & 1 & YES & YES & YES & $1.57$ & $(2,3)$ & NO & 7869\\
$(236,69)$ & 12 & $(65,19)$ & 9 & 1 & YES & YES & YES & $1.43$ & $(2,3)$ & NO & 7870\\
$(236,69)$ & 12 & $(171,50)$ & 11 & 1 & YES & YES & NO(3) & $1.29$ & $(2,3)$ & NO & 7871\\
$(236,69)$ & 12 & $(236,69)$ & 12 & 236 & YES & YES & YES & $1.43$ & $(2,3)$ & NO & 7872\\
$(237,85)$ & 12 & $(2,1)$ & 1 & 1 & YES & YES & YES & $1.43$ & $(2,3)$ & NO & 7873\\
$(237,100)$ & 12 & $(4,1)$ & 3 & 1 & YES & YES & YES & $1.57$ & $(2,3)$ & -- & 7874\\
$(237,104)$ & 12 & $(4,1)$ & 3 & 1 & YES & YES & YES & $1.43$ & $(2,3)$ & -- & 7875\\
$(237,98)$ & 12 & $(5,1)$ & 4 & 1 & YES & YES & YES & $1.57$ & $(2,3)$ & NO & 7876\\
$(237,98)$ & 12 & $(5,1)$ & 4 & 1 & YES & YES & YES & $1.57$ & $(2,3)$ & -- & 7877\\
$(237,104)$ & 12 & $(5,1)$ & 4 & 1 & YES & YES & YES & $1.57$ & $(2,3)$ & -- & 7878\\
$(237,70)$ & 13 & $(11,3)$ & 5 & 1 & YES & YES & YES & $1.71$ & $(2,3)$ & NO & 7879\\
$(237,104)$ & 12 & $(41,18)$ & 8 & 1 & YES & YES & YES & $1.71$ & $(2,3)$ & 6931 & 7880\\
$(237,92)$ & 12 & $(49,19)$ & 8 & 1 & YES & YES & YES & $1.57$ & $(2,3)$ & 8592 & 7881\\
$(237,56)$ & 14 & $(72,17)$ & 11 & 3 & YES & YES & NO(2) & $1.62$ & $(2,3)$ & NO & 7882\\
$(237,98)$ & 12 & $(104,43)$ & 10 & 1 & YES & YES & YES & $1.43$ & $(2,3)$ & NO & 7883\\
$(237,56)$ & 14 & $(237,56)$ & 14 & 237 & YES & YES & NO(2) & $1.50$ & $(2,3)$ & NO & 7884\\
$(237,98)$ & 12 & $(237,98)$ & 12 & 237 & YES & YES & YES & $1.43$ & $(2,3)$ & NO & 7885\\
$(238,93)$ & 12 & $(2,1)$ & 1 & 2 & YES & YES & YES & $1.71$ & $(2,3)$ & NO & 7886\\
$(238,103)$ & 13 & $(2,1)$ & 1 & 2 & YES & YES & YES & $1.57$ & $(2,3)$ & NO & 7887\\
$(238,69)$ & 13 & $(7,2)$ & 4 & 7 & YES & YES & YES & $1.57$ & $(2,3)$ & NO & 7888\\
$(238,93)$ & 12 & $(18,7)$ & 6 & 2 & YES & YES & YES & $1.57$ & $(2,3)$ & NO & 7889\\
$(239,64)$ & 13 & $(2,1)$ & 1 & 1 & YES & YES & NO(2) & $1.75$ & $(2,3)$ & -- & 7890\\
$(239,98)$ & 12 & $(2,1)$ & 1 & 1 & YES & YES & YES & $1.71$ & $(2,3)$ & -- & 7891\\
$(239,101)$ & 12 & $(2,1)$ & 1 & 1 & YES & YES & YES & $1.43$ & $(2,3)$ & -- & 7892\\
$(239,100)$ & 12 & $(3,1)$ & 2 & 1 & YES & YES & YES & $1.43$ & $(2,3)$ & NO & 7893\\
$(239,100)$ & 12 & $(3,1)$ & 2 & 1 & YES & YES & YES & $1.57$ & $(2,3)$ & -- & 7894\\
$(239,101)$ & 12 & $(3,1)$ & 2 & 1 & YES & YES & YES & $1.43$ & $(2,3)$ & -- & 7895\\
$(239,66)$ & 12 & $(5,2)$ & 3 & 1 & YES & YES & YES & $1.57$ & $(2,3)$ & NO & 7896\\
$(239,67)$ & 13 & $(5,1)$ & 4 & 1 & YES & YES & YES & $1.57$ & $(2,3)$ & -- & 7897\\
$(239,98)$ & 12 & $(5,1)$ & 4 & 1 & YES & YES & YES & $1.57$ & $(2,3)$ & -- & 7898\\
$(239,99)$ & 12 & $(5,2)$ & 3 & 1 & YES & YES & YES & $1.57$ & $(2,3)$ & NO & 7899\\
$(239,100)$ & 12 & $(7,3)$ & 4 & 1 & YES & YES & YES & $1.57$ & $(2,3)$ & NO & 7900\\
$(239,66)$ & 12 & $(10,3)$ & 5 & 1 & YES & YES & YES & $1.57$ & $(2,3)$ & NO & 7901\\
$(239,71)$ & 12 & $(10,3)$ & 5 & 1 & YES & YES & YES & $1.43$ & $(2,3)$ & NO & 7902\\
$(239,64)$ & 13 & $(11,3)$ & 5 & 1 & YES & YES & NO(2) & $1.75$ & $(2,3)$ & NO & 7903\\
$(239,98)$ & 12 & $(22,9)$ & 7 & 1 & YES & YES & YES & $1.43$ & $(2,3)$ & 6169 & 7904\\
$(239,70)$ & 12 & $(31,9)$ & 8 & 1 & YES & YES & YES & $1.57$ & $(2,3)$ & NO & 7905\\
$(239,101)$ & 12 & $(71,30)$ & 9 & 1 & YES & YES & YES & $1.71$ & $(2,3)$ & NO & 7906\\
$(239,104)$ & 13 & $(108,47)$ & 11 & 1 & YES & YES & YES & $1.71$ & $(2,3)$ & NO & 7907\\
$(240,71)$ & 12 & $(3,1)$ & 2 & 3 & YES & YES & YES & $1.43$ & $(2,3)$ & -- & 7908\\
$(240,89)$ & 12 & $(3,1)$ & 2 & 3 & YES & YES & YES & $1.43$ & $(2,3)$ & -- & 7909\\
$(240,89)$ & 12 & $(4,1)$ & 3 & 4 & YES & YES & YES & $1.57$ & $(2,3)$ & -- & 7910\\
$(240,71)$ & 12 & $(7,2)$ & 4 & 1 & YES & YES & NO(3) & $1.29$ & $(2,3)$ & NO & 7911\\
$(240,89)$ & 12 & $(27,10)$ & 7 & 3 & YES & YES & NO(2) & $1.62$ & $(2,3)$ & NO & 7912\\
$(240,71)$ & 12 & $(71,21)$ & 9 & 1 & YES & YES & YES & $1.57$ & $(2,3)$ & NO & 7913\\
$(240,89)$ & 12 & $(116,43)$ & 11 & 4 & YES & YES & YES & $1.71$ & $(2,3)$ & NO & 7914\\
$(240,89)$ & 12 & $(151,56)$ & 11 & 1 & YES & YES & YES & $1.57$ & $(2,3)$ & NO & 7915\\
$(240,71)$ & 12 & $(240,71)$ & 12 & 240 & YES & YES & YES & $1.43$ & $(2,3)$ & NO & 7916\\
$(241,94)$ & 12 & $(2,1)$ & 1 & 1 & YES & YES & NO(2) & $1.75$ & $(2,3)$ & -- & 7917\\
$(241,94)$ & 12 & $(3,1)$ & 2 & 1 & YES & YES & NO(2) & $1.62$ & $(2,3)$ & NO & 7918\\
$(241,94)$ & 12 & $(3,1)$ & 2 & 1 & YES & YES & YES & $1.57$ & $(2,3)$ & -- & 7919\\
$(241,100)$ & 12 & $(3,1)$ & 2 & 1 & YES & YES & YES & $1.57$ & $(2,3)$ & NO & 7920\\
$(241,100)$ & 12 & $(3,1)$ & 2 & 1 & YES & YES & YES & $1.71$ & $(2,3)$ & -- & 7921\\
$(241,100)$ & 12 & $(3,1)$ & 2 & 1 & YES & YES & YES & $1.71$ & $(2,3)$ & NO & 7922\\
$(241,101)$ & 12 & $(3,1)$ & 2 & 1 & YES & YES & YES & $1.57$ & $(2,3)$ & -- & 7923\\
$(241,99)$ & 13 & $(4,1)$ & 3 & 1 & YES & YES & YES & $1.57$ & $(2,3)$ & -- & 7924\\
$(241,65)$ & 12 & $(5,2)$ & 3 & 1 & YES & YES & YES & $1.43$ & $(2,3)$ & -- & 7925\\
$(241,65)$ & 12 & $(5,2)$ & 3 & 1 & YES & YES & YES & $1.57$ & $(2,3)$ & NO & 7926\\
$(241,94)$ & 12 & $(5,1)$ & 4 & 1 & YES & YES & NO(2) & $1.62$ & $(2,3)$ & NO & 7927\\
$(241,101)$ & 12 & $(5,1)$ & 4 & 1 & YES & YES & YES & $1.57$ & $(2,3)$ & -- & 7928\\
$(241,56)$ & 13 & $(7,3)$ & 4 & 1 & YES & YES & YES & $1.71$ & $(2,3)$ & NO & 7929\\
$(241,100)$ & 12 & $(7,3)$ & 4 & 1 & YES & YES & YES & $1.57$ & $(2,3)$ & NO & 7930\\
$(241,65)$ & 12 & $(10,3)$ & 5 & 1 & YES & YES & YES & $1.43$ & $(2,3)$ & NO & 7931\\
$(241,100)$ & 12 & $(17,7)$ & 6 & 1 & YES & YES & YES & $1.57$ & $(2,3)$ & 6310 & 7932\\
$(241,100)$ & 12 & $(29,12)$ & 7 & 1 & YES & YES & YES & $1.57$ & $(2,3)$ & NO & 7933\\
$(241,94)$ & 12 & $(41,16)$ & 8 & 1 & YES & YES & NO(2) & $1.62$ & $(2,3)$ & 6997 & 7934\\
$(241,101)$ & 12 & $(105,44)$ & 10 & 1 & YES & YES & YES & $1.57$ & $(2,3)$ & NO & 7935\\
$(241,94)$ & 12 & $(141,55)$ & 11 & 1 & YES & YES & YES & $1.57$ & $(2,3)$ & NO & 7936\\
$(241,100)$ & 12 & $(147,61)$ & 11 & 1 & YES & YES & YES & $1.57$ & $(2,3)$ & NO & 7937\\
$(241,100)$ & 12 & $(241,100)$ & 12 & 241 & YES & YES & YES & $1.43$ & $(2,3)$ & NO & 7938\\
$(242,45)$ & 14 & $(4,1)$ & 3 & 2 & YES & YES & YES & $1.57$ & $(2,3)$ & NO & 7939\\
$(242,89)$ & 12 & $(4,1)$ & 3 & 2 & YES & YES & YES & $1.57$ & $(2,3)$ & NO & 7940\\
$(242,67)$ & 12 & $(242,67)$ & 12 & 242 & YES & YES & YES & $1.43$ & $(2,3)$ & NO & 7941\\
$(242,89)$ & 12 & $(242,89)$ & 12 & 242 & YES & YES & YES & $1.43$ & $(2,3)$ & NO & 7942\\
$(243,71)$ & 12 & $(3,1)$ & 2 & 3 & YES & YES & YES & $1.57$ & $(2,3)$ & -- & 7943\\
$(243,89)$ & 12 & $(3,1)$ & 2 & 3 & YES & YES & YES & $1.43$ & $(2,3)$ & -- & 7944\\
$(243,92)$ & 12 & $(3,1)$ & 2 & 3 & YES & YES & YES & $1.57$ & $(2,3)$ & NO & 7945\\
$(243,71)$ & 12 & $(4,1)$ & 3 & 1 & YES & YES & YES & $1.57$ & $(2,3)$ & NO & 7946\\
$(243,71)$ & 12 & $(4,1)$ & 3 & 1 & YES & YES & YES & $1.57$ & $(2,3)$ & -- & 7947\\
$(243,89)$ & 12 & $(4,1)$ & 3 & 1 & YES & YES & YES & $1.43$ & $(2,3)$ & -- & 7948\\
$(243,94)$ & 12 & $(4,1)$ & 3 & 1 & YES & YES & YES & $1.57$ & $(2,3)$ & NO & 7949\\
$(243,92)$ & 12 & $(5,1)$ & 4 & 1 & YES & YES & YES & $1.57$ & $(2,3)$ & -- & 7950\\
$(243,94)$ & 12 & $(5,1)$ & 4 & 1 & YES & YES & YES & $1.57$ & $(2,3)$ & NO & 7951\\
$(243,94)$ & 12 & $(5,1)$ & 4 & 1 & YES & YES & YES & $1.57$ & $(2,3)$ & -- & 7952\\
$(243,94)$ & 12 & $(5,2)$ & 3 & 1 & YES & YES & NO(2) & $1.62$ & $(2,3)$ & NO & 7953\\
$(243,94)$ & 12 & $(9,2)$ & 5 & 9 & YES & YES & YES & $1.57$ & $(2,3)$ & NO & 7954\\
$(243,89)$ & 12 & $(11,4)$ & 5 & 1 & YES & YES & NO(2) & $1.62$ & $(2,3)$ & NO & 7955\\
$(243,92)$ & 12 & $(37,14)$ & 8 & 1 & YES & YES & YES & $1.57$ & $(2,3)$ & 6875 & 7956\\
$(243,89)$ & 12 & $(101,37)$ & 10 & 1 & YES & YES & YES & $1.43$ & $(2,3)$ & 8354 & 7957\\
$(244,71)$ & 13 & $(2,1)$ & 1 & 2 & YES & YES & YES & $1.71$ & $(2,3)$ & NO & 7958\\
$(244,71)$ & 13 & $(3,1)$ & 2 & 1 & YES & YES & YES & $1.43$ & $(2,3)$ & NO & 7959\\
$(244,71)$ & 13 & $(3,1)$ & 2 & 1 & YES & YES & YES & $1.57$ & $(2,3)$ & -- & 7960\\
$(244,55)$ & 13 & $(5,2)$ & 3 & 1 & YES & YES & YES & $1.57$ & $(2,3)$ & -- & 7961\\
$(244,71)$ & 13 & $(5,1)$ & 4 & 1 & YES & YES & YES & $1.57$ & $(2,3)$ & NO & 7962\\
$(244,57)$ & 13 & $(7,2)$ & 4 & 1 & YES & YES & YES & $1.57$ & $(2,3)$ & NO & 7963\\
$(244,71)$ & 13 & $(10,3)$ & 5 & 2 & YES & YES & YES & $1.71$ & $(2,3)$ & NO & 7964\\
$(244,55)$ & 13 & $(17,4)$ & 7 & 1 & YES & YES & YES & $1.57$ & $(2,3)$ & NO & 7965\\
$(244,71)$ & 13 & $(31,9)$ & 8 & 1 & YES & YES & YES & $1.57$ & $(2,3)$ & NO & 7966\\
$(244,57)$ & 13 & $(56,13)$ & 10 & 4 & YES & YES & YES & $1.57$ & $(2,3)$ & NO & 7967\\
$(244,71)$ & 13 & $(134,39)$ & 11 & 2 & YES & YES & YES & $1.57$ & $(2,3)$ & 8761 & 7968\\
$(244,71)$ & 13 & $(189,55)$ & 12 & 1 & YES & YES & YES & $1.57$ & $(2,3)$ & NO & 7969\\
$(244,71)$ & 13 & $(244,71)$ & 13 & 244 & YES & YES & YES & $1.71$ & $(2,3)$ & NO & 7970\\
$(245,93)$ & 12 & $(2,1)$ & 1 & 1 & YES & YES & YES & $1.57$ & $(2,3)$ & NO & 7971\\
$(245,88)$ & 12 & $(3,1)$ & 2 & 1 & YES & YES & YES & $1.57$ & $(2,3)$ & -- & 7972\\
$(245,88)$ & 12 & $(3,1)$ & 2 & 1 & YES & YES & YES & $1.57$ & $(2,3)$ & NO & 7973\\
$(245,93)$ & 12 & $(3,1)$ & 2 & 1 & YES & YES & YES & $1.71$ & $(2,3)$ & -- & 7974\\
$(245,103)$ & 12 & $(3,1)$ & 2 & 1 & YES & YES & YES & $1.57$ & $(2,3)$ & NO & 7975\\
$(245,103)$ & 12 & $(12,5)$ & 5 & 1 & YES & YES & YES & $1.57$ & $(2,3)$ & NO & 7976\\
$(245,88)$ & 12 & $(39,14)$ & 8 & 1 & YES & YES & YES & $1.57$ & $(2,3)$ & 6976 & 7977\\
$(245,74)$ & 13 & $(53,16)$ & 10 & 1 & YES & YES & YES & $1.57$ & $(2,3)$ & 7375 & 7978\\
$(245,103)$ & 12 & $(69,29)$ & 9 & 1 & YES & YES & YES & $1.71$ & $(2,3)$ & 7710 & 7979\\
$(245,103)$ & 12 & $(88,37)$ & 10 & 1 & YES & YES & YES & $1.71$ & $(2,3)$ & NO & 7980\\
$(245,93)$ & 12 & $(108,41)$ & 10 & 1 & YES & YES & YES & $1.43$ & $(2,3)$ & NO & 7981\\
$(245,88)$ & 12 & $(142,51)$ & 11 & 1 & YES & YES & YES & $1.57$ & $(2,3)$ & NO & 7982\\
$(245,88)$ & 12 & $(245,88)$ & 12 & 245 & YES & YES & YES & $1.71$ & $(2,3)$ & NO & 7983\\
$(246,65)$ & 13 & $(2,1)$ & 1 & 2 & YES & YES & YES & $1.57$ & $(2,3)$ & -- & 7984\\
$(246,65)$ & 13 & $(2,1)$ & 1 & 2 & YES & YES & YES & $1.57$ & $(2,3)$ & NO & 7985\\
$(246,95)$ & 12 & $(2,1)$ & 1 & 2 & YES & YES & YES & $1.71$ & $(2,3)$ & -- & 7986\\
$(246,101)$ & 12 & $(2,1)$ & 1 & 2 & YES & YES & YES & $1.43$ & $(2,3)$ & -- & 7987\\
$(246,65)$ & 13 & $(3,1)$ & 2 & 3 & YES & YES & YES & $1.57$ & $(2,3)$ & NO & 7988\\
$(246,73)$ & 12 & $(3,1)$ & 2 & 3 & YES & YES & YES & $1.43$ & $(2,3)$ & -- & 7989\\
$(246,101)$ & 12 & $(3,1)$ & 2 & 3 & YES & YES & YES & $1.43$ & $(2,3)$ & -- & 7990\\
$(246,65)$ & 13 & $(4,1)$ & 3 & 2 & YES & YES & YES & $1.57$ & $(2,3)$ & NO & 7991\\
$(246,91)$ & 12 & $(4,1)$ & 3 & 2 & YES & YES & YES & $1.57$ & $(2,3)$ & NO & 7992\\
$(246,95)$ & 12 & $(5,2)$ & 3 & 1 & YES & YES & NO(2) & $1.62$ & $(2,3)$ & NO & 7993\\
$(246,53)$ & 13 & $(9,2)$ & 5 & 3 & YES & YES & YES & $1.43$ & $(2,3)$ & NO & 7994\\
$(246,65)$ & 13 & $(15,4)$ & 6 & 3 & YES & YES & YES & $1.43$ & $(2,3)$ & NO & 7995\\
$(246,101)$ & 12 & $(17,7)$ & 6 & 1 & YES & YES & YES & $1.57$ & $(2,3)$ & 7369 & 7996\\
$(246,95)$ & 12 & $(18,7)$ & 6 & 6 & YES & YES & YES & $1.57$ & $(2,3)$ & NO & 7997\\
$(246,65)$ & 13 & $(34,9)$ & 8 & 2 & YES & YES & YES & $1.57$ & $(2,3)$ & NO & 7998\\
$(246,53)$ & 13 & $(37,8)$ & 8 & 1 & YES & YES & YES & $1.43$ & $(2,3)$ & NO & 7999\\
$(246,101)$ & 12 & $(56,23)$ & 9 & 2 & YES & YES & YES & $1.57$ & $(2,3)$ & 7470 & 8000\\
$(246,91)$ & 12 & $(173,64)$ & 11 & 1 & YES & YES & YES & $1.57$ & $(2,3)$ & NO & 8001\\
$(247,69)$ & 12 & $(2,1)$ & 1 & 1 & YES & YES & YES & $1.57$ & $(2,3)$ & NO & 8002\\
$(247,69)$ & 12 & $(2,1)$ & 1 & 1 & YES & YES & YES & $1.57$ & $(2,3)$ & -- & 8003\\
$(247,69)$ & 12 & $(3,1)$ & 2 & 1 & YES & YES & YES & $1.43$ & $(2,3)$ & -- & 8004\\
$(247,69)$ & 12 & $(4,1)$ & 3 & 1 & YES & YES & YES & $1.43$ & $(2,3)$ & -- & 8005\\
$(247,68)$ & 12 & $(5,1)$ & 4 & 1 & YES & YES & YES & $1.43$ & $(2,3)$ & NO & 8006\\
$(247,69)$ & 12 & $(5,1)$ & 4 & 1 & YES & YES & YES & $1.29$ & $(2,3)$ & NO & 8007\\
$(247,69)$ & 12 & $(6,1)$ & 5 & 1 & YES & YES & YES & $1.43$ & $(2,3)$ & NO & 8008\\
$(247,69)$ & 12 & $(10,3)$ & 5 & 1 & YES & YES & YES & $1.57$ & $(2,3)$ & NO & 8009\\
$(247,69)$ & 12 & $(32,9)$ & 8 & 1 & YES & YES & YES & $1.57$ & $(2,3)$ & NO & 8010\\
$(247,75)$ & 13 & $(33,10)$ & 8 & 1 & YES & YES & YES & $1.71$ & $(2,3)$ & NO & 8011\\
$(247,69)$ & 12 & $(43,12)$ & 8 & 1 & YES & YES & YES & $1.43$ & $(2,3)$ & NO & 8012\\
$(247,69)$ & 12 & $(68,19)$ & 9 & 1 & YES & YES & YES & $1.43$ & $(2,3)$ & NO & 8013\\
$(247,69)$ & 12 & $(111,31)$ & 10 & 1 & YES & YES & YES & $1.43$ & $(2,3)$ & 8512 & 8014\\
$(247,69)$ & 12 & $(179,50)$ & 11 & 1 & YES & YES & YES & $1.43$ & $(2,3)$ & NO & 8015\\
$(248,91)$ & 12 & $(5,1)$ & 4 & 1 & YES & YES & YES & $1.57$ & $(2,3)$ & NO & 8016\\
$(248,91)$ & 12 & $(5,1)$ & 4 & 1 & YES & YES & YES & $1.71$ & $(2,3)$ & -- & 8017\\
$(248,91)$ & 12 & $(5,2)$ & 3 & 1 & YES & YES & YES & $1.71$ & $(2,3)$ & NO & 8018\\
$(248,109)$ & 12 & $(66,29)$ & 9 & 2 & YES & YES & YES & $1.71$ & $(2,3)$ & 7687 & 8019\\
$(248,91)$ & 12 & $(79,29)$ & 9 & 1 & YES & YES & YES & $1.71$ & $(2,3)$ & NO & 8020\\
$(248,91)$ & 12 & $(248,91)$ & 12 & 248 & YES & YES & YES & $1.57$ & $(2,3)$ & NO & 8021\\
$(249,77)$ & 13 & $(2,1)$ & 1 & 1 & YES & YES & YES & $1.57$ & $(2,3)$ & NO & 8022\\
$(249,95)$ & 12 & $(2,1)$ & 1 & 1 & YES & YES & YES & $1.57$ & $(2,3)$ & -- & 8023\\
$(249,95)$ & 12 & $(2,1)$ & 1 & 1 & YES & YES & YES & $1.57$ & $(2,3)$ & NO & 8024\\
$(249,95)$ & 12 & $(3,1)$ & 2 & 3 & YES & YES & YES & $1.57$ & $(2,3)$ & -- & 8025\\
$(249,95)$ & 12 & $(3,1)$ & 2 & 3 & YES & YES & YES & $1.57$ & $(2,3)$ & NO & 8026\\
$(249,56)$ & 15 & $(5,1)$ & 4 & 1 & YES & YES & YES & $1.57$ & $(2,3)$ & -- & 8027\\
$(249,58)$ & 13 & $(5,2)$ & 3 & 1 & YES & YES & YES & $1.57$ & $(2,3)$ & -- & 8028\\
$(249,95)$ & 12 & $(5,2)$ & 3 & 1 & YES & YES & YES & $1.57$ & $(2,3)$ & NO & 8029\\
$(249,53)$ & 14 & $(6,1)$ & 5 & 3 & YES & YES & NO(2) & $1.62$ & $(2,3)$ & NO & 8030\\
$(249,95)$ & 12 & $(8,3)$ & 4 & 1 & YES & YES & YES & $1.43$ & $(2,3)$ & NO & 8031\\
$(249,95)$ & 12 & $(13,5)$ & 5 & 1 & YES & YES & YES & $1.57$ & $(2,3)$ & 5260 & 8032\\
$(249,77)$ & 13 & $(16,5)$ & 7 & 1 & YES & YES & YES & $1.43$ & $(2,3)$ & 6395 & 8033\\
$(249,95)$ & 12 & $(34,13)$ & 7 & 1 & YES & YES & YES & $1.71$ & $(2,3)$ & NO & 8034\\
$(249,77)$ & 13 & $(42,13)$ & 9 & 3 & YES & YES & YES & $1.71$ & $(2,3)$ & NO & 8035\\
$(249,58)$ & 13 & $(133,31)$ & 12 & 1 & YES & YES & YES & $1.57$ & $(2,3)$ & 9021 & 8036\\
$(249,73)$ & 13 & $(133,39)$ & 11 & 1 & YES & YES & YES & $1.43$ & $(2,3)$ & 8772 & 8037\\
$(249,95)$ & 12 & $(249,95)$ & 12 & 249 & YES & YES & YES & $1.57$ & $(2,3)$ & NO & 8038\\
$(250,73)$ & 13 & $(2,1)$ & 1 & 2 & YES & YES & YES & $1.57$ & $(2,3)$ & -- & 8039\\
$(250,57)$ & 13 & $(3,1)$ & 2 & 1 & YES & YES & YES & $1.57$ & $(2,3)$ & NO & 8040\\
$(250,67)$ & 12 & $(3,1)$ & 2 & 1 & YES & YES & YES & $1.57$ & $(2,3)$ & -- & 8041\\
$(250,73)$ & 13 & $(3,1)$ & 2 & 1 & YES & YES & YES & $1.71$ & $(2,3)$ & NO & 8042\\
$(250,97)$ & 12 & $(3,1)$ & 2 & 1 & YES & YES & YES & $1.57$ & $(2,3)$ & NO & 8043\\
$(250,57)$ & 13 & $(5,2)$ & 3 & 5 & YES & YES & YES & $1.43$ & $(2,3)$ & -- & 8044\\
$(250,59)$ & 14 & $(5,2)$ & 3 & 5 & YES & YES & YES & $1.71$ & $(2,3)$ & NO & 8045\\
$(250,59)$ & 14 & $(5,2)$ & 3 & 5 & YES & YES & YES & $1.71$ & $(2,3)$ & -- & 8046\\
$(250,57)$ & 13 & $(48,11)$ & 9 & 2 & YES & YES & YES & $1.43$ & $(2,3)$ & NO & 8047\\
$(250,67)$ & 12 & $(97,26)$ & 10 & 1 & YES & YES & YES & $1.43$ & $(2,3)$ & NO & 8048\\
$(250,57)$ & 13 & $(250,57)$ & 13 & 250 & YES & YES & YES & $1.43$ & $(2,3)$ & NO & 8049\\
$(251,73)$ & 13 & $(2,1)$ & 1 & 1 & YES & YES & YES & $1.71$ & $(2,3)$ & NO & 8050\\
$(251,98)$ & 12 & $(2,1)$ & 1 & 1 & YES & YES & YES & $1.43$ & $(2,3)$ & -- & 8051\\
$(251,105)$ & 12 & $(2,1)$ & 1 & 1 & YES & YES & YES & $1.43$ & $(2,3)$ & -- & 8052\\
$(251,73)$ & 13 & $(3,1)$ & 2 & 1 & YES & YES & YES & $1.43$ & $(2,3)$ & -- & 8053\\
$(251,98)$ & 12 & $(3,1)$ & 2 & 1 & YES & YES & YES & $1.71$ & $(2,3)$ & NO & 8054\\
$(251,73)$ & 13 & $(4,1)$ & 3 & 1 & YES & YES & YES & $1.57$ & $(2,3)$ & NO & 8055\\
$(251,109)$ & 13 & $(4,1)$ & 3 & 1 & YES & YES & YES & $1.57$ & $(2,3)$ & -- & 8056\\
$(251,105)$ & 12 & $(5,2)$ & 3 & 1 & YES & YES & YES & $1.57$ & $(2,3)$ & NO & 8057\\
$(251,109)$ & 13 & $(5,1)$ & 4 & 1 & YES & YES & YES & $1.71$ & $(2,3)$ & NO & 8058\\
$(251,109)$ & 13 & $(5,1)$ & 4 & 1 & YES & YES & YES & $1.71$ & $(2,3)$ & NO & 8059\\
$(251,109)$ & 13 & $(5,1)$ & 4 & 1 & YES & YES & YES & $1.71$ & $(2,3)$ & -- & 8060\\
$(251,48)$ & 14 & $(7,3)$ & 4 & 1 & YES & YES & YES & $1.57$ & $(2,3)$ & -- & 8061\\
$(251,104)$ & 12 & $(7,3)$ & 4 & 1 & YES & YES & YES & $1.57$ & $(2,3)$ & NO & 8062\\
$(251,105)$ & 12 & $(7,3)$ & 4 & 1 & YES & YES & YES & $1.57$ & $(2,3)$ & NO & 8063\\
$(251,76)$ & 13 & $(9,2)$ & 5 & 1 & YES & YES & YES & $1.57$ & $(2,3)$ & NO & 8064\\
$(251,76)$ & 13 & $(10,3)$ & 5 & 1 & YES & YES & NO(2) & $1.50$ & $(2,3)$ & NO & 8065\\
$(251,98)$ & 12 & $(23,9)$ & 7 & 1 & YES & YES & YES & $1.43$ & $(2,3)$ & 6287 & 8066\\
$(251,73)$ & 13 & $(24,7)$ & 7 & 1 & YES & YES & YES & $1.57$ & $(2,3)$ & NO & 8067\\
$(251,73)$ & 13 & $(31,9)$ & 8 & 1 & YES & YES & YES & $1.57$ & $(2,3)$ & NO & 8068\\
$(251,109)$ & 13 & $(99,43)$ & 11 & 1 & YES & YES & YES & $1.71$ & $(2,3)$ & 8378 & 8069\\
$(251,104)$ & 12 & $(181,75)$ & 11 & 1 & YES & YES & YES & $1.43$ & $(2,3)$ & NO & 8070\\
$(252,55)$ & 13 & $(2,1)$ & 1 & 2 & YES & YES & NO(2) & $1.50$ & $(2,3)$ & -- & 8071\\
$(253,74)$ & 12 & $(2,1)$ & 1 & 1 & YES & YES & YES & $1.29$ & $(2,3)$ & NO & 8072\\
$(253,106)$ & 12 & $(2,1)$ & 1 & 1 & YES & YES & YES & $1.57$ & $(2,3)$ & -- & 8073\\
$(253,68)$ & 12 & $(3,1)$ & 2 & 1 & YES & YES & YES & $1.57$ & $(2,3)$ & NO & 8074\\
$(253,74)$ & 12 & $(3,1)$ & 2 & 1 & YES & YES & YES & $1.29$ & $(2,3)$ & -- & 8075\\
$(253,74)$ & 12 & $(3,1)$ & 2 & 1 & YES & YES & YES & $1.29$ & $(2,3)$ & NO & 8076\\
$(253,98)$ & 12 & $(3,1)$ & 2 & 1 & YES & YES & YES & $1.57$ & $(2,3)$ & NO & 8077\\
$(253,68)$ & 12 & $(4,1)$ & 3 & 1 & YES & YES & YES & $1.43$ & $(2,3)$ & NO & 8078\\
$(253,98)$ & 12 & $(4,1)$ & 3 & 1 & YES & YES & YES & $1.43$ & $(2,3)$ & NO & 8079\\
$(253,74)$ & 12 & $(5,2)$ & 3 & 1 & YES & YES & YES & $1.57$ & $(2,3)$ & -- & 8080\\
$(253,93)$ & 12 & $(5,2)$ & 3 & 1 & YES & YES & YES & $1.57$ & $(2,3)$ & -- & 8081\\
$(253,98)$ & 12 & $(5,2)$ & 3 & 1 & YES & YES & YES & $1.71$ & $(2,3)$ & -- & 8082\\
$(253,105)$ & 13 & $(5,1)$ & 4 & 1 & YES & YES & YES & $1.57$ & $(2,3)$ & -- & 8083\\
$(253,74)$ & 12 & $(7,2)$ & 4 & 1 & YES & YES & YES & $1.43$ & $(2,3)$ & NO & 8084\\
$(253,57)$ & 13 & $(9,2)$ & 5 & 1 & YES & YES & YES & $1.57$ & $(2,3)$ & NO & 8085\\
$(253,74)$ & 12 & $(9,2)$ & 5 & 1 & YES & YES & YES & $1.43$ & $(2,3)$ & NO & 8086\\
$(253,111)$ & 12 & $(9,4)$ & 5 & 1 & YES & YES & YES & $1.57$ & $(2,3)$ & 6463 & 8087\\
$(253,74)$ & 12 & $(10,3)$ & 5 & 1 & YES & YES & YES & $1.29$ & $(2,3)$ & NO & 8088\\
$(253,74)$ & 12 & $(11,3)$ & 5 & 11 & YES & YES & YES & $1.57$ & $(2,3)$ & NO & 8089\\
$(253,74)$ & 12 & $(31,9)$ & 8 & 1 & YES & YES & YES & $1.57$ & $(2,3)$ & NO & 8090\\
$(253,106)$ & 12 & $(31,13)$ & 7 & 1 & YES & YES & YES & $1.57$ & $(2,3)$ & NO & 8091\\
$(253,74)$ & 12 & $(41,12)$ & 8 & 1 & YES & YES & YES & $1.43$ & $(2,3)$ & 7116 & 8092\\
$(253,74)$ & 12 & $(58,17)$ & 9 & 1 & YES & YES & YES & $1.57$ & $(2,3)$ & NO & 8093\\
$(253,74)$ & 12 & $(65,19)$ & 9 & 1 & YES & YES & YES & $1.43$ & $(2,3)$ & NO & 8094\\
$(253,74)$ & 12 & $(106,31)$ & 10 & 1 & YES & YES & YES & $1.57$ & $(2,3)$ & NO & 8095\\
$(253,98)$ & 12 & $(253,98)$ & 12 & 253 & YES & YES & YES & $1.57$ & $(2,3)$ & NO & 8096\\
$(254,57)$ & 15 & $(3,1)$ & 2 & 1 & YES & YES & YES & $1.57$ & $(2,3)$ & -- & 8097\\
$(254,71)$ & 12 & $(3,1)$ & 2 & 1 & YES & YES & YES & $1.43$ & $(2,3)$ & -- & 8098\\
$(254,105)$ & 12 & $(3,1)$ & 2 & 1 & YES & YES & YES & $1.43$ & $(2,3)$ & -- & 8099\\
$(254,93)$ & 12 & $(4,1)$ & 3 & 2 & YES & YES & YES & $1.57$ & $(2,3)$ & -- & 8100\\
$(254,105)$ & 12 & $(4,1)$ & 3 & 2 & YES & YES & YES & $1.43$ & $(2,3)$ & -- & 8101\\
$(254,71)$ & 12 & $(11,3)$ & 5 & 1 & YES & YES & YES & $1.29$ & $(2,3)$ & NO & 8102\\
$(254,57)$ & 15 & $(13,3)$ & 6 & 1 & YES & YES & YES & $1.57$ & $(2,3)$ & NO & 8103\\
$(254,105)$ & 12 & $(17,7)$ & 6 & 1 & YES & YES & YES & $1.57$ & $(2,3)$ & NO & 8104\\
$(254,93)$ & 12 & $(41,15)$ & 8 & 1 & YES & YES & YES & $1.57$ & $(2,3)$ & NO & 8105\\
$(254,105)$ & 12 & $(104,43)$ & 10 & 2 & YES & YES & YES & $1.43$ & $(2,3)$ & 8449 & 8106\\
$(254,93)$ & 12 & $(112,41)$ & 10 & 2 & YES & YES & YES & $1.57$ & $(2,3)$ & 8564 & 8107\\
$(254,105)$ & 12 & $(179,74)$ & 11 & 1 & YES & YES & YES & $1.43$ & $(2,3)$ & NO & 8108\\
$(254,93)$ & 12 & $(183,67)$ & 11 & 1 & YES & YES & YES & $1.57$ & $(2,3)$ & NO & 8109\\
$(254,93)$ & 12 & $(254,93)$ & 12 & 254 & YES & YES & YES & $1.57$ & $(2,3)$ & NO & 8110\\
$(254,105)$ & 12 & $(254,105)$ & 12 & 254 & YES & YES & YES & $1.43$ & $(2,3)$ & NO & 8111\\
$(255,97)$ & 12 & $(2,1)$ & 1 & 1 & YES & YES & YES & $1.71$ & $(2,3)$ & NO & 8112\\
$(255,79)$ & 13 & $(3,1)$ & 2 & 3 & NO & YES & YES & $1.57$ & $(2,3)$ & -- & 8113\\
$(255,97)$ & 12 & $(3,1)$ & 2 & 3 & YES & YES & YES & $1.71$ & $(2,3)$ & NO & 8114\\
$(255,97)$ & 12 & $(3,1)$ & 2 & 3 & YES & YES & YES & $1.57$ & $(2,3)$ & NO & 8115\\
$(255,97)$ & 12 & $(3,1)$ & 2 & 3 & YES & YES & YES & $1.57$ & $(2,3)$ & -- & 8116\\
$(255,97)$ & 12 & $(4,1)$ & 3 & 1 & YES & YES & YES & $1.57$ & $(2,3)$ & NO & 8117\\
$(255,97)$ & 12 & $(5,2)$ & 3 & 5 & YES & YES & YES & $1.57$ & $(2,3)$ & -- & 8118\\
$(255,97)$ & 12 & $(8,3)$ & 4 & 1 & YES & YES & YES & $1.71$ & $(2,3)$ & NO & 8119\\
$(255,97)$ & 12 & $(13,5)$ & 5 & 1 & YES & YES & YES & $1.71$ & $(2,3)$ & NO & 8120\\
$(255,97)$ & 12 & $(29,11)$ & 7 & 1 & YES & YES & YES & $1.57$ & $(2,3)$ & NO & 8121\\
$(255,97)$ & 12 & $(71,27)$ & 9 & 1 & YES & YES & YES & $1.71$ & $(2,3)$ & 7835 & 8122\\
$(255,107)$ & 12 & $(112,47)$ & 10 & 1 & YES & YES & YES & $1.57$ & $(2,3)$ & NO & 8123\\
$(255,97)$ & 12 & $(255,97)$ & 12 & 255 & YES & YES & YES & $1.57$ & $(2,3)$ & NO & 8124\\
$(256,97)$ & 12 & $(3,1)$ & 2 & 1 & YES & YES & YES & $1.43$ & $(2,3)$ & -- & 8125\\
$(256,95)$ & 12 & $(5,1)$ & 4 & 1 & YES & YES & YES & $1.57$ & $(2,3)$ & NO & 8126\\
$(256,67)$ & 13 & $(6,1)$ & 5 & 2 & YES & YES & YES & $1.57$ & $(2,3)$ & NO & 8127\\
$(256,67)$ & 13 & $(19,5)$ & 7 & 1 & YES & YES & YES & $1.71$ & $(2,3)$ & NO & 8128\\
$(257,69)$ & 12 & $(2,1)$ & 1 & 1 & YES & YES & YES & $1.43$ & $(2,3)$ & NO & 8129\\
$(257,69)$ & 12 & $(2,1)$ & 1 & 1 & YES & YES & YES & $1.43$ & $(2,3)$ & -- & 8130\\
$(257,76)$ & 12 & $(2,1)$ & 1 & 1 & YES & YES & YES & $1.43$ & $(2,3)$ & -- & 8131\\
$(257,106)$ & 13 & $(2,1)$ & 1 & 1 & YES & YES & YES & $1.71$ & $(2,3)$ & -- & 8132\\
$(257,106)$ & 13 & $(2,1)$ & 1 & 1 & YES & YES & YES & $1.71$ & $(2,3)$ & NO & 8133\\
$(257,108)$ & 12 & $(2,1)$ & 1 & 1 & YES & YES & YES & $1.57$ & $(2,3)$ & -- & 8134\\
$(257,69)$ & 12 & $(3,1)$ & 2 & 1 & YES & YES & YES & $1.43$ & $(2,3)$ & NO & 8135\\
$(257,78)$ & 13 & $(3,1)$ & 2 & 1 & YES & YES & YES & $1.57$ & $(2,3)$ & NO & 8136\\
$(257,108)$ & 12 & $(3,1)$ & 2 & 1 & YES & YES & YES & $1.57$ & $(2,3)$ & -- & 8137\\
$(257,106)$ & 13 & $(4,1)$ & 3 & 1 & YES & YES & YES & $1.71$ & $(2,3)$ & -- & 8138\\
$(257,108)$ & 12 & $(7,3)$ & 4 & 1 & YES & YES & YES & $1.57$ & $(2,3)$ & NO & 8139\\
$(257,69)$ & 12 & $(11,3)$ & 5 & 1 & YES & YES & YES & $1.29$ & $(2,3)$ & NO & 8140\\
$(257,69)$ & 12 & $(15,4)$ & 6 & 1 & YES & YES & YES & $1.43$ & $(2,3)$ & NO & 8141\\
$(257,106)$ & 13 & $(17,7)$ & 6 & 1 & YES & YES & YES & $1.71$ & $(2,3)$ & NO & 8142\\
$(257,69)$ & 12 & $(26,7)$ & 7 & 1 & YES & YES & YES & $1.43$ & $(2,3)$ & 6341 & 8143\\
$(257,78)$ & 13 & $(33,10)$ & 8 & 1 & YES & YES & YES & $1.71$ & $(2,3)$ & NO & 8144\\
$(257,78)$ & 13 & $(145,44)$ & 11 & 1 & YES & YES & YES & $1.57$ & $(2,3)$ & 8869 & 8145\\
$(258,55)$ & 14 & $(19,4)$ & 7 & 1 & YES & YES & YES & $1.43$ & $(2,3)$ & NO & 8146\\
$(258,109)$ & 12 & $(26,11)$ & 7 & 2 & YES & YES & YES & $1.57$ & $(2,3)$ & NO & 8147\\
$(258,55)$ & 14 & $(47,10)$ & 9 & 1 & YES & YES & YES & $1.43$ & $(2,3)$ & NO & 8148\\
$(259,100)$ & 12 & $(2,1)$ & 1 & 1 & YES & YES & YES & $1.57$ & $(2,3)$ & NO & 8149\\
$(259,50)$ & 15 & $(3,1)$ & 2 & 1 & YES & YES & YES & $1.57$ & $(2,3)$ & NO & 8150\\
$(259,50)$ & 15 & $(3,1)$ & 2 & 1 & YES & YES & YES & $1.57$ & $(2,3)$ & -- & 8151\\
$(259,100)$ & 12 & $(3,1)$ & 2 & 1 & YES & YES & YES & $1.57$ & $(2,3)$ & NO & 8152\\
$(259,100)$ & 12 & $(3,1)$ & 2 & 1 & YES & YES & YES & $1.71$ & $(2,3)$ & NO & 8153\\
$(259,100)$ & 12 & $(3,1)$ & 2 & 1 & YES & YES & YES & $1.71$ & $(2,3)$ & -- & 8154\\
$(259,101)$ & 12 & $(3,1)$ & 2 & 1 & YES & YES & YES & $1.71$ & $(2,3)$ & -- & 8155\\
$(259,100)$ & 12 & $(4,1)$ & 3 & 1 & YES & YES & YES & $1.57$ & $(2,3)$ & NO & 8156\\
$(259,100)$ & 12 & $(5,2)$ & 3 & 1 & YES & YES & YES & $1.43$ & $(2,3)$ & NO & 8157\\
$(259,100)$ & 12 & $(8,3)$ & 4 & 1 & YES & YES & YES & $1.57$ & $(2,3)$ & NO & 8158\\
$(259,50)$ & 15 & $(11,2)$ & 6 & 1 & YES & YES & YES & $1.57$ & $(2,3)$ & NO & 8159\\
$(259,73)$ & 13 & $(11,3)$ & 5 & 1 & YES & YES & YES & $1.57$ & $(2,3)$ & NO & 8160\\
$(259,100)$ & 12 & $(18,7)$ & 6 & 1 & YES & YES & YES & $1.71$ & $(2,3)$ & 6525 & 8161\\
$(259,50)$ & 15 & $(26,5)$ & 9 & 1 & YES & YES & YES & $1.57$ & $(2,3)$ & NO & 8162\\
$(259,100)$ & 12 & $(31,12)$ & 7 & 1 & YES & YES & YES & $1.57$ & $(2,3)$ & NO & 8163\\
$(259,50)$ & 15 & $(119,23)$ & 13 & 7 & YES & YES & YES & $1.57$ & $(2,3)$ & NO & 8164\\
$(259,100)$ & 12 & $(259,100)$ & 12 & 259 & YES & YES & YES & $1.57$ & $(2,3)$ & NO & 8165\\
$(260,79)$ & 13 & $(2,1)$ & 1 & 2 & YES & YES & NO(2) & $1.62$ & $(2,3)$ & NO & 8166\\
$(260,73)$ & 13 & $(3,1)$ & 2 & 1 & YES & YES & YES & $1.57$ & $(2,3)$ & NO & 8167\\
$(260,57)$ & 13 & $(5,2)$ & 3 & 5 & YES & YES & YES & $1.43$ & $(2,3)$ & NO & 8168\\
$(260,57)$ & 13 & $(7,2)$ & 4 & 1 & YES & YES & YES & $1.43$ & $(2,3)$ & NO & 8169\\
$(260,73)$ & 13 & $(25,7)$ & 7 & 5 & YES & YES & YES & $1.57$ & $(2,3)$ & NO & 8170\\
$(260,79)$ & 13 & $(260,79)$ & 13 & 260 & YES & YES & YES & $1.57$ & $(2,3)$ & NO & 8171\\
$(261,79)$ & 13 & $(2,1)$ & 1 & 1 & YES & YES & NO(2) & $1.62$ & $(2,3)$ & NO & 8172\\
$(261,100)$ & 12 & $(2,1)$ & 1 & 1 & YES & YES & YES & $1.57$ & $(2,3)$ & -- & 8173\\
$(261,76)$ & 13 & $(3,1)$ & 2 & 3 & YES & YES & YES & $1.57$ & $(2,3)$ & -- & 8174\\
$(261,100)$ & 12 & $(3,1)$ & 2 & 3 & YES & YES & YES & $1.57$ & $(2,3)$ & NO & 8175\\
$(261,100)$ & 12 & $(3,1)$ & 2 & 3 & YES & YES & YES & $1.57$ & $(2,3)$ & -- & 8176\\
$(261,100)$ & 12 & $(3,1)$ & 2 & 3 & YES & YES & YES & $1.71$ & $(2,3)$ & NO & 8177\\
$(261,107)$ & 12 & $(3,1)$ & 2 & 3 & YES & YES & YES & $1.57$ & $(2,3)$ & NO & 8178\\
$(261,76)$ & 13 & $(5,1)$ & 4 & 1 & YES & YES & YES & $1.57$ & $(2,3)$ & NO & 8179\\
$(261,59)$ & 13 & $(7,2)$ & 4 & 1 & YES & YES & YES & $1.57$ & $(2,3)$ & NO & 8180\\
$(261,100)$ & 12 & $(8,3)$ & 4 & 1 & YES & YES & YES & $1.57$ & $(2,3)$ & NO & 8181\\
$(261,100)$ & 12 & $(13,5)$ & 5 & 1 & YES & YES & YES & $1.43$ & $(2,3)$ & 6434 & 8182\\
$(261,76)$ & 13 & $(17,5)$ & 6 & 1 & YES & YES & YES & $1.57$ & $(2,3)$ & NO & 8183\\
$(261,61)$ & 13 & $(22,5)$ & 7 & 1 & YES & YES & YES & $1.57$ & $(2,3)$ & NO & 8184\\
$(261,79)$ & 13 & $(43,13)$ & 9 & 1 & YES & YES & YES & $1.43$ & $(2,3)$ & NO & 8185\\
$(261,100)$ & 12 & $(47,18)$ & 8 & 1 & YES & YES & YES & $1.57$ & $(2,3)$ & 7349 & 8186\\
$(261,61)$ & 13 & $(77,18)$ & 10 & 1 & YES & YES & NO(2) & $1.50$ & $(2,3)$ & NO & 8187\\
$(261,107)$ & 12 & $(100,41)$ & 10 & 1 & YES & YES & YES & $1.57$ & $(2,3)$ & NO & 8188\\
$(261,76)$ & 13 & $(261,76)$ & 13 & 261 & YES & YES & YES & $1.57$ & $(2,3)$ & NO & 8189\\
$(262,61)$ & 13 & $(3,1)$ & 2 & 1 & YES & YES & YES & $1.43$ & $(2,3)$ & -- & 8190\\
$(262,61)$ & 13 & $(3,1)$ & 2 & 1 & YES & YES & YES & $1.57$ & $(2,3)$ & NO & 8191\\
$(262,61)$ & 13 & $(3,1)$ & 2 & 1 & YES & YES & YES & $1.43$ & $(2,3)$ & NO & 8192\\
$(262,61)$ & 13 & $(5,2)$ & 3 & 1 & YES & YES & YES & $1.43$ & $(2,3)$ & -- & 8193\\
$(262,111)$ & 13 & $(5,1)$ & 4 & 1 & YES & YES & YES & $1.71$ & $(2,3)$ & -- & 8194\\
$(262,61)$ & 13 & $(9,2)$ & 5 & 1 & YES & YES & YES & $1.57$ & $(2,3)$ & NO & 8195\\
$(262,61)$ & 13 & $(17,4)$ & 7 & 1 & YES & YES & YES & $1.43$ & $(2,3)$ & NO & 8196\\
$(262,61)$ & 13 & $(47,11)$ & 9 & 1 & YES & YES & YES & $1.43$ & $(2,3)$ & NO & 8197\\
$(262,61)$ & 13 & $(103,24)$ & 11 & 1 & YES & YES & YES & $1.43$ & $(2,3)$ & NO & 8198\\
$(262,61)$ & 13 & $(159,37)$ & 12 & 1 & YES & YES & YES & $1.57$ & $(2,3)$ & 9040 & 8199\\
$(263,60)$ & 13 & $(2,1)$ & 1 & 1 & YES & YES & YES & $1.43$ & $(2,3)$ & -- & 8200\\
$(263,71)$ & 12 & $(2,1)$ & 1 & 1 & YES & YES & YES & $1.57$ & $(2,3)$ & NO & 8201\\
$(263,109)$ & 12 & $(3,1)$ & 2 & 1 & YES & YES & YES & $1.71$ & $(2,3)$ & -- & 8202\\
$(263,109)$ & 12 & $(3,1)$ & 2 & 1 & YES & YES & YES & $1.71$ & $(2,3)$ & NO & 8203\\
$(263,61)$ & 14 & $(4,1)$ & 3 & 1 & YES & YES & YES & $1.43$ & $(2,3)$ & -- & 8204\\
$(263,100)$ & 12 & $(4,1)$ & 3 & 1 & YES & YES & YES & $1.57$ & $(2,3)$ & NO & 8205\\
$(263,102)$ & 13 & $(4,1)$ & 3 & 1 & YES & YES & YES & $1.57$ & $(2,3)$ & NO & 8206\\
$(263,109)$ & 12 & $(4,1)$ & 3 & 1 & YES & YES & YES & $1.43$ & $(2,3)$ & -- & 8207\\
$(263,109)$ & 12 & $(4,1)$ & 3 & 1 & YES & YES & YES & $1.57$ & $(2,3)$ & NO & 8208\\
$(263,57)$ & 13 & $(5,2)$ & 3 & 1 & YES & YES & YES & $1.57$ & $(2,3)$ & -- & 8209\\
$(263,60)$ & 13 & $(5,2)$ & 3 & 1 & YES & YES & YES & $1.57$ & $(2,3)$ & -- & 8210\\
$(263,102)$ & 13 & $(5,1)$ & 4 & 1 & YES & YES & YES & $1.57$ & $(2,3)$ & NO & 8211\\
$(263,102)$ & 13 & $(5,2)$ & 3 & 1 & YES & YES & YES & $1.71$ & $(2,3)$ & NO & 8212\\
$(263,109)$ & 12 & $(5,2)$ & 3 & 1 & YES & YES & YES & $1.57$ & $(2,3)$ & NO & 8213\\
$(263,102)$ & 13 & $(6,1)$ & 5 & 1 & YES & YES & YES & $1.57$ & $(2,3)$ & NO & 8214\\
$(263,111)$ & 12 & $(12,5)$ & 5 & 1 & YES & YES & YES & $1.71$ & $(2,3)$ & NO & 8215\\
$(263,100)$ & 12 & $(13,5)$ & 5 & 1 & YES & YES & YES & $1.71$ & $(2,3)$ & NO & 8216\\
$(263,61)$ & 14 & $(17,4)$ & 7 & 1 & YES & YES & YES & $1.57$ & $(2,3)$ & NO & 8217\\
$(263,109)$ & 12 & $(17,7)$ & 6 & 1 & YES & YES & YES & $1.57$ & $(2,3)$ & NO & 8218\\
$(263,100)$ & 12 & $(29,11)$ & 7 & 1 & YES & YES & YES & $1.57$ & $(2,3)$ & NO & 8219\\
$(263,108)$ & 13 & $(39,16)$ & 8 & 1 & YES & YES & YES & $1.57$ & $(2,3)$ & NO & 8220\\
$(263,109)$ & 12 & $(41,17)$ & 8 & 1 & YES & YES & YES & $1.43$ & $(2,3)$ & 7211 & 8221\\
$(263,57)$ & 13 & $(51,11)$ & 9 & 1 & YES & YES & YES & $1.57$ & $(2,3)$ & NO & 8222\\
$(263,102)$ & 13 & $(67,26)$ & 9 & 1 & YES & YES & YES & $1.57$ & $(2,3)$ & NO & 8223\\
$(263,61)$ & 14 & $(125,29)$ & 12 & 1 & YES & YES & YES & $1.43$ & $(2,3)$ & 8712 & 8224\\
$(263,109)$ & 12 & $(152,63)$ & 11 & 1 & YES & YES & YES & $1.57$ & $(2,3)$ & NO & 8225\\
$(263,100)$ & 12 & $(192,73)$ & 11 & 1 & YES & YES & YES & $1.57$ & $(2,3)$ & NO & 8226\\
$(263,61)$ & 14 & $(194,45)$ & 13 & 1 & YES & YES & YES & $1.57$ & $(2,3)$ & NO & 8227\\
$(263,61)$ & 14 & $(263,61)$ & 14 & 263 & YES & YES & YES & $1.43$ & $(2,3)$ & NO & 8228\\
$(263,109)$ & 12 & $(263,109)$ & 12 & 263 & YES & YES & YES & $1.43$ & $(2,3)$ & NO & 8229\\
$(264,109)$ & 12 & $(2,1)$ & 1 & 2 & YES & YES & YES & $1.57$ & $(2,3)$ & -- & 8230\\
$(264,109)$ & 12 & $(3,1)$ & 2 & 3 & YES & YES & YES & $1.57$ & $(2,3)$ & -- & 8231\\
$(264,101)$ & 12 & $(5,1)$ & 4 & 1 & YES & YES & YES & $1.57$ & $(2,3)$ & -- & 8232\\
$(264,71)$ & 13 & $(11,3)$ & 5 & 11 & YES & YES & YES & $1.71$ & $(2,3)$ & NO & 8233\\
$(264,101)$ & 12 & $(13,5)$ & 5 & 1 & YES & YES & YES & $1.71$ & $(2,3)$ & NO & 8234\\
$(264,109)$ & 12 & $(63,26)$ & 9 & 3 & YES & YES & YES & $1.71$ & $(2,3)$ & NO & 8235\\
$(264,101)$ & 12 & $(115,44)$ & 10 & 1 & YES & YES & YES & $1.57$ & $(2,3)$ & NO & 8236\\
$(265,62)$ & 14 & $(2,1)$ & 1 & 1 & YES & YES & YES & $1.71$ & $(2,3)$ & -- & 8237\\
$(265,97)$ & 12 & $(2,1)$ & 1 & 1 & YES & YES & YES & $1.71$ & $(2,3)$ & NO & 8238\\
$(265,111)$ & 12 & $(2,1)$ & 1 & 1 & YES & YES & YES & $1.43$ & $(2,3)$ & -- & 8239\\
$(265,73)$ & 12 & $(3,1)$ & 2 & 1 & YES & YES & YES & $1.57$ & $(2,3)$ & -- & 8240\\
$(265,74)$ & 12 & $(3,1)$ & 2 & 1 & YES & YES & YES & $1.57$ & $(2,3)$ & -- & 8241\\
$(265,98)$ & 12 & $(3,1)$ & 2 & 1 & YES & YES & YES & $1.43$ & $(2,3)$ & -- & 8242\\
$(265,112)$ & 12 & $(4,1)$ & 3 & 1 & YES & YES & YES & $1.43$ & $(2,3)$ & NO & 8243\\
$(265,62)$ & 14 & $(5,1)$ & 4 & 5 & YES & YES & YES & $1.57$ & $(2,3)$ & NO & 8244\\
$(265,74)$ & 12 & $(5,2)$ & 3 & 5 & YES & YES & YES & $1.57$ & $(2,3)$ & -- & 8245\\
$(265,112)$ & 12 & $(5,1)$ & 4 & 5 & YES & YES & YES & $1.43$ & $(2,3)$ & NO & 8246\\
$(265,112)$ & 12 & $(5,2)$ & 3 & 5 & YES & YES & YES & $1.43$ & $(2,3)$ & NO & 8247\\
$(265,111)$ & 12 & $(7,3)$ & 4 & 1 & YES & YES & YES & $1.43$ & $(2,3)$ & NO & 8248\\
$(265,74)$ & 12 & $(10,3)$ & 5 & 5 & YES & YES & YES & $1.57$ & $(2,3)$ & NO & 8249\\
$(265,97)$ & 12 & $(11,4)$ & 5 & 1 & YES & YES & YES & $1.57$ & $(2,3)$ & NO & 8250\\
$(265,62)$ & 14 & $(13,3)$ & 6 & 1 & YES & YES & YES & $1.57$ & $(2,3)$ & NO & 8251\\
$(265,74)$ & 12 & $(29,8)$ & 7 & 1 & YES & YES & YES & $1.57$ & $(2,3)$ & NO & 8252\\
$(265,74)$ & 12 & $(32,9)$ & 8 & 1 & YES & YES & YES & $1.57$ & $(2,3)$ & NO & 8253\\
$(265,74)$ & 12 & $(43,12)$ & 8 & 1 & YES & YES & YES & $1.43$ & $(2,3)$ & 7302 & 8254\\
$(265,74)$ & 12 & $(68,19)$ & 9 & 1 & YES & YES & YES & $1.57$ & $(2,3)$ & NO & 8255\\
$(265,112)$ & 12 & $(97,41)$ & 10 & 1 & YES & YES & YES & $1.57$ & $(2,3)$ & NO & 8256\\
$(265,97)$ & 12 & $(153,56)$ & 11 & 1 & YES & YES & YES & $1.57$ & $(2,3)$ & NO & 8257\\
$(266,101)$ & 12 & $(2,1)$ & 1 & 2 & YES & YES & YES & $1.57$ & $(2,3)$ & NO & 8258\\
$(266,101)$ & 12 & $(2,1)$ & 1 & 2 & YES & YES & YES & $1.43$ & $(2,3)$ & -- & 8259\\
$(266,101)$ & 12 & $(3,1)$ & 2 & 1 & YES & YES & YES & $1.43$ & $(2,3)$ & NO & 8260\\
$(266,101)$ & 12 & $(3,1)$ & 2 & 1 & YES & YES & YES & $1.43$ & $(2,3)$ & -- & 8261\\
$(266,79)$ & 12 & $(4,1)$ & 3 & 2 & YES & YES & YES & $1.29$ & $(2,3)$ & NO & 8262\\
$(266,101)$ & 12 & $(4,1)$ & 3 & 2 & YES & YES & YES & $1.43$ & $(2,3)$ & NO & 8263\\
$(266,101)$ & 12 & $(4,1)$ & 3 & 2 & YES & YES & YES & $1.43$ & $(2,3)$ & -- & 8264\\
$(266,101)$ & 12 & $(5,1)$ & 4 & 1 & YES & YES & YES & $1.43$ & $(2,3)$ & NO & 8265\\
$(266,73)$ & 14 & $(11,3)$ & 5 & 1 & YES & YES & YES & $1.71$ & $(2,3)$ & NO & 8266\\
$(266,101)$ & 12 & $(21,8)$ & 6 & 7 & YES & YES & YES & $1.57$ & $(2,3)$ & NO & 8267\\
$(266,101)$ & 12 & $(29,11)$ & 7 & 1 & YES & YES & YES & $1.43$ & $(2,3)$ & NO & 8268\\
$(266,109)$ & 13 & $(61,25)$ & 9 & 1 & YES & YES & YES & $1.57$ & $(2,3)$ & NO & 8269\\
$(266,79)$ & 12 & $(64,19)$ & 9 & 2 & YES & YES & YES & $1.29$ & $(2,3)$ & 7779 & 8270\\
$(266,79)$ & 12 & $(101,30)$ & 10 & 1 & YES & YES & YES & $1.43$ & $(2,3)$ & NO & 8271\\
$(266,101)$ & 12 & $(108,41)$ & 10 & 2 & YES & YES & YES & $1.43$ & $(2,3)$ & 8565 & 8272\\
$(266,101)$ & 12 & $(187,71)$ & 11 & 1 & YES & YES & YES & $1.43$ & $(2,3)$ & NO & 8273\\
$(267,62)$ & 14 & $(2,1)$ & 1 & 1 & YES & YES & NO(2) & $1.75$ & $(2,3)$ & NO & 8274\\
$(267,79)$ & 12 & $(2,1)$ & 1 & 1 & YES & YES & YES & $1.57$ & $(2,3)$ & NO & 8275\\
$(267,79)$ & 12 & $(2,1)$ & 1 & 1 & YES & YES & YES & $1.29$ & $(2,3)$ & -- & 8276\\
$(267,98)$ & 12 & $(2,1)$ & 1 & 1 & YES & YES & YES & $1.86$ & $(2,3)$ & -- & 8277\\
$(267,98)$ & 12 & $(2,1)$ & 1 & 1 & YES & YES & YES & $1.86$ & $(2,3)$ & NO & 8278\\
$(267,79)$ & 12 & $(3,1)$ & 2 & 3 & YES & YES & YES & $1.43$ & $(2,3)$ & -- & 8279\\
$(267,98)$ & 12 & $(3,1)$ & 2 & 3 & YES & YES & YES & $1.43$ & $(2,3)$ & -- & 8280\\
$(267,79)$ & 12 & $(4,1)$ & 3 & 1 & YES & YES & YES & $1.29$ & $(2,3)$ & NO & 8281\\
$(267,98)$ & 12 & $(4,1)$ & 3 & 1 & YES & YES & YES & $1.71$ & $(2,3)$ & -- & 8282\\
$(267,98)$ & 12 & $(5,2)$ & 3 & 1 & YES & YES & YES & $1.71$ & $(2,3)$ & NO & 8283\\
$(267,79)$ & 12 & $(7,2)$ & 4 & 1 & YES & YES & YES & $1.43$ & $(2,3)$ & NO & 8284\\
$(267,98)$ & 12 & $(8,3)$ & 4 & 1 & YES & YES & YES & $1.71$ & $(2,3)$ & NO & 8285\\
$(267,98)$ & 12 & $(19,7)$ & 6 & 1 & YES & YES & YES & $1.71$ & $(2,3)$ & NO & 8286\\
$(267,98)$ & 12 & $(30,11)$ & 7 & 3 & YES & YES & YES & $1.57$ & $(2,3)$ & NO & 8287\\
$(267,62)$ & 14 & $(43,10)$ & 9 & 1 & YES & YES & NO(2) & $1.75$ & $(2,3)$ & NO & 8288\\
$(267,98)$ & 12 & $(49,18)$ & 8 & 1 & YES & YES & YES & $1.71$ & $(2,3)$ & NO & 8289\\
$(267,79)$ & 12 & $(98,29)$ & 10 & 1 & YES & YES & YES & $1.57$ & $(2,3)$ & NO & 8290\\
$(267,98)$ & 12 & $(267,98)$ & 12 & 267 & YES & YES & YES & $1.43$ & $(2,3)$ & NO & 8291\\
$(268,83)$ & 13 & $(3,1)$ & 2 & 1 & YES & YES & YES & $1.57$ & $(2,3)$ & -- & 8292\\
$(268,83)$ & 13 & $(3,1)$ & 2 & 1 & YES & YES & YES & $1.57$ & $(2,3)$ & NO & 8293\\
$(268,83)$ & 13 & $(3,1)$ & 2 & 1 & YES & YES & YES & $1.57$ & $(2,3)$ & NO & 8294\\
$(268,71)$ & 13 & $(4,1)$ & 3 & 4 & YES & YES & YES & $1.43$ & $(2,3)$ & -- & 8295\\
$(268,97)$ & 13 & $(5,1)$ & 4 & 1 & YES & YES & YES & $1.57$ & $(2,3)$ & NO & 8296\\
$(268,99)$ & 12 & $(5,2)$ & 3 & 1 & YES & YES & YES & $1.57$ & $(2,3)$ & NO & 8297\\
$(268,113)$ & 13 & $(5,2)$ & 3 & 1 & YES & YES & YES & $1.71$ & $(2,3)$ & NO & 8298\\
$(268,57)$ & 14 & $(6,1)$ & 5 & 2 & YES & YES & YES & $1.43$ & $(2,3)$ & NO & 8299\\
$(268,61)$ & 14 & $(7,1)$ & 6 & 1 & YES & YES & YES & $1.57$ & $(2,3)$ & NO & 8300\\
$(268,111)$ & 12 & $(7,3)$ & 4 & 1 & YES & YES & YES & $1.57$ & $(2,3)$ & NO & 8301\\
$(268,99)$ & 12 & $(8,3)$ & 4 & 4 & YES & YES & YES & $1.57$ & $(2,3)$ & NO & 8302\\
$(268,99)$ & 12 & $(11,4)$ & 5 & 1 & YES & YES & YES & $1.71$ & $(2,3)$ & NO & 8303\\
$(268,111)$ & 12 & $(12,5)$ & 5 & 4 & YES & YES & YES & $1.57$ & $(2,3)$ & NO & 8304\\
$(268,57)$ & 14 & $(19,4)$ & 7 & 1 & YES & YES & YES & $1.43$ & $(2,3)$ & NO & 8305\\
$(268,99)$ & 12 & $(19,7)$ & 6 & 1 & YES & YES & YES & $1.71$ & $(2,3)$ & 6500 & 8306\\
$(268,97)$ & 13 & $(58,21)$ & 10 & 2 & YES & YES & YES & $1.71$ & $(2,3)$ & 7672 & 8307\\
$(268,83)$ & 13 & $(155,48)$ & 12 & 1 & YES & YES & YES & $1.57$ & $(2,3)$ & NO & 8308\\
$(268,99)$ & 12 & $(157,58)$ & 11 & 1 & YES & YES & YES & $1.57$ & $(2,3)$ & NO & 8309\\
$(268,83)$ & 13 & $(268,83)$ & 13 & 268 & YES & YES & YES & $1.71$ & $(2,3)$ & NO & 8310\\
$(268,111)$ & 12 & $(268,111)$ & 12 & 268 & YES & YES & YES & $1.57$ & $(2,3)$ & NO & 8311\\
$(269,63)$ & 13 & $(3,1)$ & 2 & 1 & YES & YES & NO(2) & $1.57$ & $(4,2)$ & NO & 8312\\
$(269,111)$ & 13 & $(5,2)$ & 3 & 1 & YES & YES & YES & $1.57$ & $(2,3)$ & 5428 & 8313\\
$(269,72)$ & 13 & $(11,3)$ & 5 & 1 & YES & YES & YES & $1.71$ & $(2,3)$ & NO & 8314\\
$(269,104)$ & 12 & $(13,5)$ & 5 & 1 & YES & YES & YES & $1.57$ & $(2,3)$ & NO & 8315\\
$(269,111)$ & 13 & $(63,26)$ & 9 & 1 & YES & YES & YES & $1.57$ & $(2,3)$ & NO & 8316\\
$(270,103)$ & 12 & $(2,1)$ & 1 & 2 & YES & YES & YES & $1.71$ & $(2,3)$ & -- & 8317\\
$(270,103)$ & 12 & $(270,103)$ & 12 & 270 & YES & YES & YES & $1.71$ & $(2,3)$ & NO & 8318\\
$(271,112)$ & 12 & $(2,1)$ & 1 & 1 & YES & YES & YES & $1.43$ & $(2,3)$ & -- & 8319\\
$(271,80)$ & 12 & $(3,1)$ & 2 & 1 & YES & YES & YES & $1.43$ & $(2,3)$ & NO & 8320\\
$(271,80)$ & 12 & $(3,1)$ & 2 & 1 & YES & YES & YES & $1.43$ & $(2,3)$ & -- & 8321\\
$(271,82)$ & 13 & $(3,1)$ & 2 & 1 & YES & YES & YES & $1.43$ & $(2,3)$ & NO & 8322\\
$(271,112)$ & 12 & $(3,1)$ & 2 & 1 & YES & YES & YES & $1.57$ & $(2,3)$ & NO & 8323\\
$(271,80)$ & 12 & $(4,1)$ & 3 & 1 & YES & YES & YES & $1.43$ & $(2,3)$ & NO & 8324\\
$(271,80)$ & 12 & $(4,1)$ & 3 & 1 & YES & YES & YES & $1.57$ & $(2,3)$ & -- & 8325\\
$(271,82)$ & 13 & $(4,1)$ & 3 & 1 & YES & YES & YES & $1.57$ & $(2,3)$ & NO & 8326\\
$(271,112)$ & 12 & $(4,1)$ & 3 & 1 & YES & YES & YES & $1.57$ & $(2,3)$ & -- & 8327\\
$(271,76)$ & 13 & $(5,1)$ & 4 & 1 & YES & YES & YES & $1.57$ & $(2,3)$ & NO & 8328\\
$(271,80)$ & 12 & $(5,2)$ & 3 & 1 & YES & YES & YES & $1.43$ & $(2,3)$ & -- & 8329\\
$(271,112)$ & 12 & $(12,5)$ & 5 & 1 & YES & YES & YES & $1.43$ & $(2,3)$ & NO & 8330\\
$(271,80)$ & 12 & $(24,7)$ & 7 & 1 & YES & YES & YES & $1.43$ & $(2,3)$ & NO & 8331\\
$(271,84)$ & 13 & $(29,9)$ & 8 & 1 & YES & YES & YES & $1.57$ & $(2,3)$ & NO & 8332\\
$(271,112)$ & 12 & $(29,12)$ & 7 & 1 & YES & YES & YES & $1.43$ & $(2,3)$ & NO & 8333\\
$(271,82)$ & 13 & $(33,10)$ & 8 & 1 & YES & YES & YES & $1.43$ & $(2,3)$ & NO & 8334\\
$(271,103)$ & 13 & $(71,27)$ & 9 & 1 & YES & YES & YES & $1.57$ & $(2,3)$ & NO & 8335\\
$(271,80)$ & 12 & $(78,23)$ & 10 & 1 & YES & YES & YES & $1.43$ & $(2,3)$ & NO & 8336\\
$(271,76)$ & 13 & $(82,23)$ & 10 & 1 & YES & YES & YES & $1.57$ & $(2,3)$ & NO & 8337\\
$(271,80)$ & 12 & $(105,31)$ & 10 & 1 & YES & YES & YES & $1.57$ & $(2,3)$ & NO & 8338\\
$(271,76)$ & 13 & $(107,30)$ & 11 & 1 & YES & YES & YES & $1.57$ & $(2,3)$ & 8573 & 8339\\
$(271,103)$ & 13 & $(121,46)$ & 10 & 1 & YES & YES & YES & $1.57$ & $(2,3)$ & NO & 8340\\
$(271,80)$ & 12 & $(166,49)$ & 11 & 1 & YES & YES & YES & $1.43$ & $(2,3)$ & NO & 8341\\
$(271,112)$ & 12 & $(196,81)$ & 11 & 1 & YES & YES & YES & $1.57$ & $(2,3)$ & NO & 8342\\
$(271,80)$ & 12 & $(271,80)$ & 12 & 271 & YES & YES & YES & $1.57$ & $(2,3)$ & NO & 8343\\
$(272,103)$ & 12 & $(3,1)$ & 2 & 1 & YES & YES & YES & $1.43$ & $(2,3)$ & -- & 8344\\
$(272,103)$ & 12 & $(5,1)$ & 4 & 1 & YES & YES & YES & $1.57$ & $(2,3)$ & NO & 8345\\
$(272,105)$ & 13 & $(5,1)$ & 4 & 1 & YES & YES & YES & $1.57$ & $(2,3)$ & -- & 8346\\
$(272,105)$ & 13 & $(44,17)$ & 8 & 4 & YES & YES & YES & $1.71$ & $(2,3)$ & NO & 8347\\
$(273,101)$ & 12 & $(3,1)$ & 2 & 3 & YES & YES & YES & $1.43$ & $(2,3)$ & -- & 8348\\
$(273,64)$ & 14 & $(4,1)$ & 3 & 1 & YES & YES & NO(2) & $1.62$ & $(2,3)$ & -- & 8349\\
$(273,80)$ & 13 & $(4,1)$ & 3 & 1 & YES & YES & YES & $1.43$ & $(2,3)$ & -- & 8350\\
$(273,101)$ & 12 & $(4,1)$ & 3 & 1 & YES & YES & YES & $1.57$ & $(2,3)$ & NO & 8351\\
$(273,101)$ & 12 & $(4,1)$ & 3 & 1 & YES & YES & YES & $1.57$ & $(2,3)$ & -- & 8352\\
$(273,64)$ & 14 & $(5,1)$ & 4 & 1 & YES & YES & NO(2) & $1.62$ & $(2,3)$ & NO & 8353\\
$(273,100)$ & 12 & $(71,26)$ & 9 & 1 & YES & YES & YES & $1.43$ & $(2,3)$ & 7957 & 8354\\
$(273,101)$ & 12 & $(173,64)$ & 11 & 1 & YES & YES & YES & $1.57$ & $(2,3)$ & NO & 8355\\
$(273,101)$ & 12 & $(273,101)$ & 12 & 273 & YES & YES & YES & $1.43$ & $(2,3)$ & NO & 8356\\
$(274,81)$ & 12 & $(2,1)$ & 1 & 2 & YES & YES & YES & $1.43$ & $(2,3)$ & -- & 8357\\
$(274,105)$ & 12 & $(2,1)$ & 1 & 2 & NO & YES & YES & $1.57$ & $(2,3)$ & -- & 8358\\
$(274,115)$ & 12 & $(2,1)$ & 1 & 2 & YES & YES & YES & $1.43$ & $(2,3)$ & -- & 8359\\
$(274,119)$ & 13 & $(2,1)$ & 1 & 2 & NO & YES & NO(2) & $1.75$ & $(2,3)$ & -- & 8360\\
$(274,81)$ & 12 & $(3,1)$ & 2 & 1 & YES & YES & YES & $1.43$ & $(2,3)$ & -- & 8361\\
$(274,119)$ & 13 & $(3,1)$ & 2 & 1 & YES & YES & YES & $1.71$ & $(2,3)$ & NO & 8362\\
$(274,81)$ & 12 & $(4,1)$ & 3 & 2 & YES & YES & YES & $1.57$ & $(2,3)$ & -- & 8363\\
$(274,99)$ & 13 & $(4,1)$ & 3 & 2 & YES & YES & YES & $1.71$ & $(2,3)$ & -- & 8364\\
$(274,65)$ & 14 & $(5,1)$ & 4 & 1 & YES & YES & YES & $1.57$ & $(2,3)$ & NO & 8365\\
$(274,105)$ & 12 & $(5,2)$ & 3 & 1 & YES & YES & YES & $1.57$ & $(2,3)$ & NO & 8366\\
$(274,119)$ & 13 & $(5,1)$ & 4 & 1 & YES & YES & YES & $1.71$ & $(2,3)$ & NO & 8367\\
$(274,119)$ & 13 & $(5,1)$ & 4 & 1 & YES & YES & YES & $1.71$ & $(2,3)$ & -- & 8368\\
$(274,115)$ & 12 & $(7,3)$ & 4 & 1 & YES & YES & YES & $1.43$ & $(2,3)$ & NO & 8369\\
$(274,105)$ & 12 & $(8,3)$ & 4 & 2 & YES & YES & YES & $1.57$ & $(2,3)$ & NO & 8370\\
$(274,81)$ & 12 & $(10,3)$ & 5 & 2 & YES & YES & YES & $1.57$ & $(2,3)$ & NO & 8371\\
$(274,81)$ & 12 & $(17,5)$ & 6 & 1 & YES & YES & YES & $1.43$ & $(2,3)$ & NO & 8372\\
$(274,115)$ & 12 & $(19,8)$ & 6 & 1 & YES & YES & YES & $1.43$ & $(2,3)$ & NO & 8373\\
$(274,51)$ & 15 & $(27,5)$ & 8 & 1 & YES & YES & YES & $1.57$ & $(2,3)$ & NO & 8374\\
$(274,65)$ & 14 & $(38,9)$ & 9 & 2 & YES & YES & YES & $1.57$ & $(2,3)$ & NO & 8375\\
$(274,99)$ & 13 & $(47,17)$ & 9 & 1 & YES & YES & YES & $1.71$ & $(2,3)$ & 6473 & 8376\\
$(274,105)$ & 12 & $(60,23)$ & 9 & 2 & YES & YES & YES & $1.57$ & $(2,3)$ & 7749 & 8377\\
$(274,119)$ & 13 & $(76,33)$ & 10 & 2 & YES & YES & YES & $1.71$ & $(2,3)$ & 8069 & 8378\\
$(274,51)$ & 15 & $(102,19)$ & 11 & 2 & YES & YES & YES & $1.57$ & $(2,3)$ & NO & 8379\\
$(274,99)$ & 13 & $(155,56)$ & 12 & 1 & YES & YES & YES & $1.71$ & $(2,3)$ & NO & 8380\\
$(274,51)$ & 15 & $(274,51)$ & 15 & 274 & YES & YES & YES & $1.71$ & $(2,3)$ & NO & 8381\\
$(275,102)$ & 13 & $(2,1)$ & 1 & 1 & NO & YES & YES & $1.71$ & $(2,3)$ & -- & 8382\\
$(275,73)$ & 13 & $(3,1)$ & 2 & 1 & YES & YES & YES & $1.57$ & $(2,3)$ & -- & 8383\\
$(275,102)$ & 13 & $(213,79)$ & 12 & 1 & YES & YES & YES & $1.71$ & $(2,3)$ & NO & 8384\\
$(276,73)$ & 13 & $(3,1)$ & 2 & 3 & YES & YES & YES & $1.57$ & $(2,3)$ & NO & 8385\\
$(277,60)$ & 13 & $(2,1)$ & 1 & 1 & YES & YES & YES & $1.29$ & $(2,3)$ & -- & 8386\\
$(277,81)$ & 12 & $(2,1)$ & 1 & 1 & YES & YES & YES & $1.43$ & $(2,3)$ & -- & 8387\\
$(277,106)$ & 12 & $(2,1)$ & 1 & 1 & YES & YES & YES & $1.57$ & $(2,3)$ & -- & 8388\\
$(277,116)$ & 12 & $(2,1)$ & 1 & 1 & YES & YES & YES & $1.71$ & $(2,3)$ & -- & 8389\\
$(277,84)$ & 13 & $(3,1)$ & 2 & 1 & YES & YES & YES & $1.43$ & $(2,3)$ & -- & 8390\\
$(277,84)$ & 13 & $(3,1)$ & 2 & 1 & YES & YES & YES & $1.57$ & $(2,3)$ & NO & 8391\\
$(277,106)$ & 12 & $(3,1)$ & 2 & 1 & YES & YES & YES & $1.57$ & $(2,3)$ & NO & 8392\\
$(277,116)$ & 12 & $(3,1)$ & 2 & 1 & YES & YES & YES & $1.71$ & $(2,3)$ & NO & 8393\\
$(277,106)$ & 12 & $(4,1)$ & 3 & 1 & YES & YES & YES & $1.57$ & $(2,3)$ & NO & 8394\\
$(277,60)$ & 13 & $(5,2)$ & 3 & 1 & YES & YES & YES & $1.57$ & $(2,3)$ & -- & 8395\\
$(277,84)$ & 13 & $(5,2)$ & 3 & 1 & YES & YES & YES & $1.57$ & $(2,3)$ & NO & 8396\\
$(277,116)$ & 12 & $(5,2)$ & 3 & 1 & YES & YES & YES & $1.57$ & $(2,3)$ & NO & 8397\\
$(277,106)$ & 12 & $(7,3)$ & 4 & 1 & YES & YES & YES & $1.71$ & $(2,3)$ & NO & 8398\\
$(277,84)$ & 13 & $(13,4)$ & 6 & 1 & YES & YES & YES & $1.57$ & $(2,3)$ & NO & 8399\\
$(277,116)$ & 12 & $(19,8)$ & 6 & 1 & YES & YES & YES & $1.71$ & $(2,3)$ & NO & 8400\\
$(277,117)$ & 12 & $(19,8)$ & 6 & 1 & YES & YES & YES & $1.57$ & $(2,3)$ & NO & 8401\\
$(277,84)$ & 13 & $(33,10)$ & 8 & 1 & YES & YES & YES & $1.71$ & $(2,3)$ & 7117 & 8402\\
$(277,81)$ & 12 & $(41,12)$ & 8 & 1 & YES & YES & YES & $1.57$ & $(2,3)$ & NO & 8403\\
$(277,108)$ & 13 & $(41,16)$ & 8 & 1 & YES & YES & YES & $1.57$ & $(2,3)$ & NO & 8404\\
$(277,81)$ & 12 & $(106,31)$ & 10 & 1 & YES & YES & YES & $1.43$ & $(2,3)$ & NO & 8405\\
$(277,106)$ & 12 & $(196,75)$ & 11 & 1 & YES & YES & YES & $1.57$ & $(2,3)$ & NO & 8406\\
$(278,65)$ & 13 & $(2,1)$ & 1 & 2 & YES & YES & YES & $1.29$ & $(2,3)$ & -- & 8407\\
$(278,85)$ & 13 & $(2,1)$ & 1 & 2 & YES & YES & YES & $1.57$ & $(2,3)$ & -- & 8408\\
$(278,65)$ & 13 & $(5,2)$ & 3 & 1 & YES & YES & YES & $1.57$ & $(2,3)$ & -- & 8409\\
$(278,65)$ & 13 & $(17,4)$ & 7 & 1 & YES & YES & YES & $1.29$ & $(2,3)$ & NO & 8410\\
$(278,121)$ & 13 & $(23,10)$ & 7 & 1 & YES & YES & YES & $1.57$ & $(2,3)$ & NO & 8411\\
$(278,63)$ & 13 & $(97,22)$ & 11 & 1 & YES & YES & YES & $1.43$ & $(2,3)$ & NO & 8412\\
$(278,85)$ & 13 & $(121,37)$ & 11 & 1 & YES & YES & YES & $1.57$ & $(2,3)$ & NO & 8413\\
$(279,83)$ & 13 & $(2,1)$ & 1 & 1 & YES & YES & YES & $1.43$ & $(2,3)$ & NO & 8414\\
$(279,121)$ & 13 & $(2,1)$ & 1 & 1 & YES & YES & YES & $1.71$ & $(2,3)$ & -- & 8415\\
$(279,65)$ & 13 & $(5,2)$ & 3 & 1 & YES & YES & YES & $1.57$ & $(2,3)$ & NO & 8416\\
$(279,83)$ & 13 & $(10,3)$ & 5 & 1 & YES & YES & YES & $1.43$ & $(2,3)$ & NO & 8417\\
$(279,121)$ & 13 & $(83,36)$ & 10 & 1 & YES & YES & YES & $1.71$ & $(2,3)$ & NO & 8418\\
$(280,101)$ & 13 & $(2,1)$ & 1 & 2 & YES & YES & YES & $1.71$ & $(2,3)$ & NO & 8419\\
$(280,107)$ & 12 & $(5,2)$ & 3 & 5 & YES & YES & YES & $1.71$ & $(2,3)$ & NO & 8420\\
$(280,101)$ & 13 & $(11,4)$ & 5 & 1 & YES & YES & YES & $1.71$ & $(2,3)$ & NO & 8421\\
$(280,101)$ & 13 & $(36,13)$ & 8 & 4 & YES & YES & YES & $1.71$ & $(2,3)$ & NO & 8422\\
$(281,76)$ & 13 & $(2,1)$ & 1 & 1 & YES & YES & NO(2) & $1.75$ & $(2,3)$ & -- & 8423\\
$(281,109)$ & 12 & $(2,1)$ & 1 & 1 & YES & YES & YES & $1.57$ & $(2,3)$ & -- & 8424\\
$(281,116)$ & 12 & $(2,1)$ & 1 & 1 & YES & YES & YES & $1.43$ & $(2,3)$ & -- & 8425\\
$(281,109)$ & 12 & $(3,1)$ & 2 & 1 & YES & YES & YES & $1.57$ & $(2,3)$ & NO & 8426\\
$(281,109)$ & 12 & $(3,1)$ & 2 & 1 & YES & YES & YES & $1.57$ & $(2,3)$ & -- & 8427\\
$(281,116)$ & 12 & $(3,1)$ & 2 & 1 & YES & YES & YES & $1.71$ & $(2,3)$ & -- & 8428\\
$(281,116)$ & 12 & $(3,1)$ & 2 & 1 & YES & YES & YES & $1.43$ & $(2,3)$ & NO & 8429\\
$(281,109)$ & 12 & $(5,2)$ & 3 & 1 & YES & YES & YES & $1.43$ & $(2,3)$ & NO & 8430\\
$(281,116)$ & 12 & $(12,5)$ & 5 & 1 & YES & YES & YES & $1.43$ & $(2,3)$ & 6774 & 8431\\
$(281,116)$ & 12 & $(17,7)$ & 6 & 1 & YES & YES & YES & $1.43$ & $(2,3)$ & 7685 & 8432\\
$(281,53)$ & 15 & $(21,4)$ & 8 & 1 & YES & YES & NO(2) & $1.50$ & $(2,3)$ & NO & 8433\\
$(281,85)$ & 13 & $(43,13)$ & 9 & 1 & YES & YES & NO(2) & $1.50$ & $(2,3)$ & 7421 & 8434\\
$(281,109)$ & 12 & $(49,19)$ & 8 & 1 & YES & YES & YES & $1.71$ & $(2,3)$ & 7606 & 8435\\
$(281,116)$ & 12 & $(63,26)$ & 9 & 1 & YES & YES & YES & $1.43$ & $(2,3)$ & 7854 & 8436\\
$(281,116)$ & 12 & $(172,71)$ & 11 & 1 & YES & YES & YES & $1.57$ & $(2,3)$ & NO & 8437\\
$(282,59)$ & 15 & $(2,1)$ & 1 & 2 & YES & YES & YES & $1.57$ & $(2,3)$ & NO & 8438\\
$(282,119)$ & 12 & $(3,1)$ & 2 & 3 & YES & YES & YES & $1.43$ & $(2,3)$ & NO & 8439\\
$(283,104)$ & 12 & $(2,1)$ & 1 & 1 & YES & YES & YES & $1.57$ & $(2,3)$ & NO & 8440\\
$(283,117)$ & 12 & $(2,1)$ & 1 & 1 & YES & YES & YES & $1.57$ & $(2,3)$ & -- & 8441\\
$(283,119)$ & 13 & $(2,1)$ & 1 & 1 & NO & YES & YES & $1.71$ & $(2,3)$ & -- & 8442\\
$(283,86)$ & 13 & $(3,1)$ & 2 & 1 & YES & YES & YES & $1.57$ & $(2,3)$ & NO & 8443\\
$(283,102)$ & 13 & $(4,1)$ & 3 & 1 & YES & YES & YES & $1.57$ & $(2,3)$ & -- & 8444\\
$(283,104)$ & 12 & $(8,3)$ & 4 & 1 & YES & YES & YES & $1.71$ & $(2,3)$ & NO & 8445\\
$(283,88)$ & 13 & $(29,9)$ & 8 & 1 & YES & YES & YES & $1.57$ & $(2,3)$ & 6623 & 8446\\
$(283,86)$ & 13 & $(33,10)$ & 8 & 1 & YES & YES & YES & $1.57$ & $(2,3)$ & NO & 8447\\
$(283,84)$ & 13 & $(37,11)$ & 8 & 1 & YES & YES & YES & $1.57$ & $(2,3)$ & NO & 8448\\
$(283,117)$ & 12 & $(75,31)$ & 9 & 1 & YES & YES & YES & $1.43$ & $(2,3)$ & 8106 & 8449\\
$(284,75)$ & 14 & $(2,1)$ & 1 & 2 & YES & YES & YES & $1.71$ & $(2,3)$ & NO & 8450\\
$(284,83)$ & 13 & $(2,1)$ & 1 & 2 & YES & YES & YES & $1.57$ & $(2,3)$ & -- & 8451\\
$(284,119)$ & 12 & $(2,1)$ & 1 & 2 & YES & YES & YES & $1.71$ & $(2,3)$ & -- & 8452\\
$(284,83)$ & 13 & $(3,1)$ & 2 & 1 & YES & YES & YES & $1.57$ & $(2,3)$ & NO & 8453\\
$(284,105)$ & 12 & $(3,1)$ & 2 & 1 & YES & YES & YES & $1.57$ & $(2,3)$ & -- & 8454\\
$(284,105)$ & 12 & $(4,1)$ & 3 & 4 & YES & YES & YES & $1.57$ & $(2,3)$ & -- & 8455\\
$(284,105)$ & 12 & $(5,2)$ & 3 & 1 & YES & YES & YES & $1.57$ & $(2,3)$ & NO & 8456\\
$(284,119)$ & 12 & $(5,1)$ & 4 & 1 & YES & YES & YES & $1.57$ & $(2,3)$ & NO & 8457\\
$(284,119)$ & 12 & $(5,1)$ & 4 & 1 & YES & YES & YES & $1.57$ & $(2,3)$ & NO & 8458\\
$(284,105)$ & 12 & $(11,4)$ & 5 & 1 & YES & YES & YES & $1.57$ & $(2,3)$ & NO & 8459\\
$(284,67)$ & 15 & $(21,5)$ & 8 & 1 & YES & YES & YES & $1.57$ & $(2,3)$ & NO & 8460\\
$(284,119)$ & 12 & $(31,13)$ & 7 & 1 & YES & YES & YES & $1.57$ & $(2,3)$ & NO & 8461\\
$(284,83)$ & 13 & $(41,12)$ & 8 & 1 & YES & YES & YES & $1.43$ & $(2,3)$ & NO & 8462\\
$(284,105)$ & 12 & $(46,17)$ & 8 & 2 & YES & YES & YES & $1.57$ & $(2,3)$ & 7561 & 8463\\
$(284,67)$ & 15 & $(72,17)$ & 11 & 4 & YES & YES & YES & $1.57$ & $(2,3)$ & NO & 8464\\
$(284,83)$ & 13 & $(154,45)$ & 11 & 2 & YES & YES & YES & $1.43$ & $(2,3)$ & 8936 & 8465\\
$(284,105)$ & 12 & $(165,61)$ & 11 & 1 & YES & YES & YES & $1.57$ & $(2,3)$ & NO & 8466\\
$(284,83)$ & 13 & $(284,83)$ & 13 & 284 & YES & YES & YES & $1.71$ & $(2,3)$ & NO & 8467\\
$(285,53)$ & 15 & $(2,1)$ & 1 & 1 & YES & YES & YES & $1.71$ & $(2,3)$ & NO & 8468\\
$(285,83)$ & 13 & $(2,1)$ & 1 & 1 & YES & YES & YES & $1.57$ & $(2,3)$ & NO & 8469\\
$(285,83)$ & 13 & $(3,1)$ & 2 & 3 & YES & YES & YES & $1.57$ & $(2,3)$ & -- & 8470\\
$(285,83)$ & 13 & $(4,1)$ & 3 & 1 & YES & YES & YES & $1.43$ & $(2,3)$ & NO & 8471\\
$(285,83)$ & 13 & $(5,1)$ & 4 & 5 & YES & YES & YES & $1.43$ & $(2,3)$ & 6930 & 8472\\
$(285,53)$ & 15 & $(6,1)$ & 5 & 3 & YES & YES & YES & $1.57$ & $(2,3)$ & NO & 8473\\
$(285,83)$ & 13 & $(24,7)$ & 7 & 3 & YES & YES & YES & $1.57$ & $(2,3)$ & NO & 8474\\
$(285,53)$ & 15 & $(27,5)$ & 8 & 3 & YES & YES & YES & $1.57$ & $(2,3)$ & NO & 8475\\
$(285,83)$ & 13 & $(285,83)$ & 13 & 285 & YES & YES & YES & $1.57$ & $(2,3)$ & NO & 8476\\
$(286,79)$ & 12 & $(2,1)$ & 1 & 2 & YES & YES & YES & $1.43$ & $(2,3)$ & NO & 8477\\
$(286,79)$ & 12 & $(3,1)$ & 2 & 1 & YES & YES & YES & $1.29$ & $(2,3)$ & -- & 8478\\
$(286,105)$ & 12 & $(3,1)$ & 2 & 1 & YES & YES & YES & $1.43$ & $(2,3)$ & -- & 8479\\
$(286,111)$ & 13 & $(5,2)$ & 3 & 1 & YES & YES & YES & $1.57$ & $(2,3)$ & 5567 & 8480\\
$(286,79)$ & 12 & $(7,2)$ & 4 & 1 & YES & YES & YES & $1.43$ & $(2,3)$ & NO & 8481\\
$(286,79)$ & 12 & $(10,3)$ & 5 & 2 & YES & YES & YES & $1.57$ & $(2,3)$ & NO & 8482\\
$(286,61)$ & 14 & $(19,4)$ & 7 & 1 & YES & YES & YES & $1.43$ & $(2,3)$ & NO & 8483\\
$(286,61)$ & 14 & $(47,10)$ & 9 & 1 & YES & YES & YES & $1.43$ & $(2,3)$ & NO & 8484\\
$(286,111)$ & 13 & $(67,26)$ & 9 & 1 & YES & YES & YES & $1.57$ & $(2,3)$ & NO & 8485\\
$(287,79)$ & 12 & $(2,1)$ & 1 & 1 & YES & YES & YES & $1.43$ & $(2,3)$ & -- & 8486\\
$(287,79)$ & 12 & $(2,1)$ & 1 & 1 & YES & YES & YES & $1.43$ & $(2,3)$ & NO & 8487\\
$(287,109)$ & 12 & $(2,1)$ & 1 & 1 & YES & YES & YES & $1.57$ & $(2,3)$ & -- & 8488\\
$(287,80)$ & 13 & $(4,1)$ & 3 & 1 & YES & YES & YES & $1.43$ & $(2,3)$ & -- & 8489\\
$(287,51)$ & 15 & $(28,5)$ & 8 & 7 & YES & YES & YES & $1.57$ & $(2,3)$ & NO & 8490\\
$(287,109)$ & 12 & $(29,11)$ & 7 & 1 & YES & YES & YES & $1.57$ & $(2,3)$ & NO & 8491\\
$(287,106)$ & 12 & $(287,106)$ & 12 & 287 & YES & YES & YES & $1.57$ & $(2,3)$ & NO & 8492\\
$(288,119)$ & 12 & $(2,1)$ & 1 & 2 & YES & YES & YES & $1.57$ & $(2,3)$ & -- & 8493\\
$(288,121)$ & 12 & $(5,2)$ & 3 & 1 & YES & YES & YES & $1.57$ & $(2,3)$ & NO & 8494\\
$(288,121)$ & 12 & $(12,5)$ & 5 & 12 & YES & YES & YES & $1.57$ & $(2,3)$ & NO & 8495\\
$(288,119)$ & 12 & $(17,7)$ & 6 & 1 & YES & YES & YES & $1.57$ & $(2,3)$ & NO & 8496\\
$(289,80)$ & 12 & $(2,1)$ & 1 & 1 & YES & YES & YES & $1.29$ & $(2,3)$ & NO & 8497\\
$(289,80)$ & 12 & $(2,1)$ & 1 & 1 & YES & YES & YES & $1.57$ & $(2,3)$ & -- & 8498\\
$(289,112)$ & 12 & $(2,1)$ & 1 & 1 & YES & YES & YES & $1.57$ & $(2,3)$ & -- & 8499\\
$(289,112)$ & 12 & $(2,1)$ & 1 & 1 & YES & YES & YES & $1.57$ & $(2,3)$ & NO & 8500\\
$(289,80)$ & 12 & $(3,1)$ & 2 & 1 & YES & YES & YES & $1.29$ & $(2,3)$ & NO & 8501\\
$(289,80)$ & 12 & $(7,2)$ & 4 & 1 & YES & YES & YES & $1.29$ & $(2,3)$ & NO & 8502\\
$(289,63)$ & 13 & $(41,9)$ & 9 & 1 & YES & YES & YES & $1.57$ & $(2,3)$ & NO & 8503\\
$(289,80)$ & 12 & $(47,13)$ & 8 & 1 & YES & YES & YES & $1.43$ & $(2,3)$ & NO & 8504\\
$(289,86)$ & 13 & $(84,25)$ & 10 & 1 & YES & YES & YES & $1.57$ & $(2,3)$ & NO & 8505\\
$(289,63)$ & 13 & $(124,27)$ & 12 & 1 & YES & YES & YES & $1.57$ & $(2,3)$ & NO & 8506\\
$(290,81)$ & 12 & $(2,1)$ & 1 & 2 & YES & YES & YES & $1.29$ & $(2,3)$ & -- & 8507\\
$(290,111)$ & 12 & $(2,1)$ & 1 & 2 & YES & YES & YES & $1.43$ & $(2,3)$ & -- & 8508\\
$(290,77)$ & 13 & $(3,1)$ & 2 & 1 & YES & YES & YES & $1.57$ & $(2,3)$ & -- & 8509\\
$(290,111)$ & 12 & $(8,3)$ & 4 & 2 & YES & YES & YES & $1.43$ & $(2,3)$ & NO & 8510\\
$(290,111)$ & 12 & $(34,13)$ & 7 & 2 & YES & YES & YES & $1.43$ & $(2,3)$ & NO & 8511\\
$(290,81)$ & 12 & $(68,19)$ & 9 & 2 & YES & YES & YES & $1.43$ & $(2,3)$ & 8014 & 8512\\
$(290,81)$ & 12 & $(111,31)$ & 10 & 1 & YES & YES & YES & $1.43$ & $(2,3)$ & NO & 8513\\
$(290,77)$ & 13 & $(113,30)$ & 11 & 1 & YES & YES & YES & $1.43$ & $(2,3)$ & NO & 8514\\
$(291,89)$ & 13 & $(2,1)$ & 1 & 1 & YES & YES & YES & $1.57$ & $(2,3)$ & NO & 8515\\
$(291,107)$ & 13 & $(2,1)$ & 1 & 1 & YES & YES & YES & $1.71$ & $(2,3)$ & NO & 8516\\
$(291,107)$ & 13 & $(4,1)$ & 3 & 1 & YES & YES & YES & $1.57$ & $(2,3)$ & -- & 8517\\
$(291,68)$ & 13 & $(5,2)$ & 3 & 1 & YES & YES & YES & $1.43$ & $(2,3)$ & -- & 8518\\
$(291,68)$ & 13 & $(22,5)$ & 7 & 1 & YES & YES & YES & $1.43$ & $(2,3)$ & NO & 8519\\
$(291,62)$ & 14 & $(47,10)$ & 9 & 1 & YES & YES & NO(2) & $1.75$ & $(2,3)$ & NO & 8520\\
$(292,89)$ & 13 & $(2,1)$ & 1 & 2 & YES & YES & YES & $1.57$ & $(2,3)$ & -- & 8521\\
$(292,111)$ & 12 & $(2,1)$ & 1 & 2 & YES & YES & YES & $1.57$ & $(2,3)$ & -- & 8522\\
$(292,121)$ & 12 & $(2,1)$ & 1 & 2 & YES & YES & YES & $1.57$ & $(2,3)$ & -- & 8523\\
$(292,85)$ & 13 & $(3,1)$ & 2 & 1 & YES & YES & YES & $1.43$ & $(2,3)$ & NO & 8524\\
$(292,85)$ & 13 & $(3,1)$ & 2 & 1 & YES & YES & YES & $1.43$ & $(2,3)$ & -- & 8525\\
$(292,111)$ & 12 & $(3,1)$ & 2 & 1 & YES & YES & YES & $1.57$ & $(2,3)$ & -- & 8526\\
$(292,121)$ & 12 & $(3,1)$ & 2 & 1 & YES & YES & YES & $1.57$ & $(2,3)$ & -- & 8527\\
$(292,121)$ & 12 & $(3,1)$ & 2 & 1 & YES & YES & YES & $1.57$ & $(2,3)$ & NO & 8528\\
$(292,85)$ & 13 & $(4,1)$ & 3 & 4 & YES & YES & YES & $1.43$ & $(2,3)$ & -- & 8529\\
$(292,111)$ & 12 & $(4,1)$ & 3 & 4 & YES & YES & YES & $1.57$ & $(2,3)$ & -- & 8530\\
$(292,67)$ & 14 & $(5,1)$ & 4 & 1 & YES & YES & YES & $1.57$ & $(2,3)$ & NO & 8531\\
$(292,79)$ & 13 & $(5,1)$ & 4 & 1 & YES & YES & YES & $1.57$ & $(2,3)$ & NO & 8532\\
$(292,85)$ & 13 & $(5,1)$ & 4 & 1 & YES & YES & YES & $1.57$ & $(2,3)$ & NO & 8533\\
$(292,85)$ & 13 & $(5,2)$ & 3 & 1 & YES & YES & YES & $1.57$ & $(2,3)$ & -- & 8534\\
$(292,111)$ & 12 & $(8,3)$ & 4 & 4 & YES & YES & YES & $1.71$ & $(2,3)$ & NO & 8535\\
$(292,85)$ & 13 & $(10,3)$ & 5 & 2 & YES & YES & YES & $1.43$ & $(2,3)$ & NO & 8536\\
$(292,89)$ & 13 & $(10,3)$ & 5 & 2 & YES & YES & YES & $1.57$ & $(2,3)$ & NO & 8537\\
$(292,85)$ & 13 & $(17,5)$ & 6 & 1 & YES & YES & YES & $1.43$ & $(2,3)$ & NO & 8538\\
$(292,121)$ & 12 & $(17,7)$ & 6 & 1 & YES & YES & YES & $1.57$ & $(2,3)$ & 8699 & 8539\\
$(292,111)$ & 12 & $(21,8)$ & 6 & 1 & YES & YES & YES & $1.57$ & $(2,3)$ & 6758 & 8540\\
$(292,85)$ & 13 & $(24,7)$ & 7 & 4 & YES & YES & YES & $1.43$ & $(2,3)$ & NO & 8541\\
$(292,85)$ & 13 & $(31,9)$ & 8 & 1 & YES & YES & YES & $1.57$ & $(2,3)$ & NO & 8542\\
$(292,111)$ & 12 & $(71,27)$ & 9 & 1 & YES & YES & YES & $1.71$ & $(2,3)$ & NO & 8543\\
$(292,121)$ & 12 & $(181,75)$ & 11 & 1 & YES & YES & YES & $1.43$ & $(2,3)$ & NO & 8544\\
$(292,111)$ & 12 & $(292,111)$ & 12 & 292 & YES & YES & YES & $1.71$ & $(2,3)$ & NO & 8545\\
$(293,62)$ & 14 & $(2,1)$ & 1 & 1 & YES & YES & YES & $1.57$ & $(2,3)$ & -- & 8546\\
$(293,62)$ & 14 & $(2,1)$ & 1 & 1 & YES & YES & YES & $1.57$ & $(2,3)$ & NO & 8547\\
$(293,89)$ & 13 & $(2,1)$ & 1 & 1 & YES & YES & YES & $1.71$ & $(2,3)$ & -- & 8548\\
$(293,52)$ & 14 & $(3,1)$ & 2 & 1 & YES & YES & YES & $1.43$ & $(2,3)$ & NO & 8549\\
$(293,87)$ & 13 & $(3,1)$ & 2 & 1 & YES & YES & YES & $1.43$ & $(2,3)$ & -- & 8550\\
$(293,89)$ & 13 & $(4,1)$ & 3 & 1 & YES & YES & YES & $1.57$ & $(2,3)$ & -- & 8551\\
$(293,52)$ & 14 & $(5,1)$ & 4 & 1 & YES & YES & YES & $1.43$ & $(2,3)$ & NO & 8552\\
$(293,62)$ & 14 & $(14,3)$ & 6 & 1 & YES & YES & YES & $1.43$ & $(2,3)$ & NO & 8553\\
$(293,87)$ & 13 & $(37,11)$ & 8 & 1 & YES & YES & YES & $1.57$ & $(2,3)$ & NO & 8554\\
$(294,67)$ & 13 & $(3,1)$ & 2 & 3 & YES & YES & YES & $1.43$ & $(2,3)$ & NO & 8555\\
$(294,109)$ & 13 & $(3,1)$ & 2 & 3 & YES & YES & YES & $1.57$ & $(2,3)$ & -- & 8556\\
$(294,109)$ & 13 & $(4,1)$ & 3 & 2 & YES & YES & YES & $1.57$ & $(2,3)$ & -- & 8557\\
$(294,89)$ & 13 & $(5,1)$ & 4 & 1 & YES & YES & YES & $1.57$ & $(2,3)$ & NO & 8558\\
$(294,89)$ & 13 & $(185,56)$ & 12 & 1 & YES & YES & YES & $1.57$ & $(2,3)$ & NO & 8559\\
$(294,109)$ & 13 & $(205,76)$ & 12 & 1 & YES & YES & YES & $1.71$ & $(2,3)$ & NO & 8560\\
$(295,108)$ & 12 & $(2,1)$ & 1 & 1 & YES & YES & YES & $1.43$ & $(2,3)$ & NO & 8561\\
$(295,112)$ & 12 & $(2,1)$ & 1 & 1 & YES & YES & YES & $1.57$ & $(2,3)$ & NO & 8562\\
$(295,108)$ & 12 & $(3,1)$ & 2 & 1 & YES & YES & YES & $1.43$ & $(2,3)$ & -- & 8563\\
$(295,108)$ & 12 & $(71,26)$ & 9 & 1 & YES & YES & YES & $1.57$ & $(2,3)$ & 8107 & 8564\\
$(295,112)$ & 12 & $(79,30)$ & 9 & 1 & YES & YES & YES & $1.43$ & $(2,3)$ & 8272 & 8565\\
$(295,112)$ & 12 & $(108,41)$ & 10 & 1 & YES & YES & YES & $1.57$ & $(2,3)$ & NO & 8566\\
$(296,83)$ & 13 & $(2,1)$ & 1 & 2 & YES & YES & YES & $1.57$ & $(2,3)$ & 5473 & 8567\\
$(296,83)$ & 13 & $(2,1)$ & 1 & 2 & YES & YES & YES & $1.43$ & $(2,3)$ & -- & 8568\\
$(296,83)$ & 13 & $(3,1)$ & 2 & 1 & YES & YES & YES & $1.57$ & $(2,3)$ & NO & 8569\\
$(296,83)$ & 13 & $(5,1)$ & 4 & 1 & YES & YES & YES & $1.43$ & $(2,3)$ & NO & 8570\\
$(296,69)$ & 14 & $(17,4)$ & 7 & 1 & YES & YES & YES & $1.57$ & $(2,3)$ & NO & 8571\\
$(296,107)$ & 13 & $(36,13)$ & 8 & 4 & YES & YES & YES & $1.71$ & $(2,3)$ & NO & 8572\\
$(296,83)$ & 13 & $(82,23)$ & 10 & 2 & YES & YES & YES & $1.57$ & $(2,3)$ & 8339 & 8573\\
$(297,113)$ & 13 & $(2,1)$ & 1 & 1 & YES & YES & YES & $1.71$ & $(2,3)$ & NO & 8574\\
$(297,68)$ & 13 & $(3,1)$ & 2 & 3 & YES & YES & YES & $1.29$ & $(2,3)$ & NO & 8575\\
$(297,68)$ & 13 & $(3,1)$ & 2 & 3 & YES & YES & YES & $1.57$ & $(2,3)$ & -- & 8576\\
$(297,109)$ & 12 & $(3,1)$ & 2 & 3 & YES & YES & YES & $1.57$ & $(2,3)$ & -- & 8577\\
$(297,113)$ & 13 & $(3,1)$ & 2 & 3 & YES & YES & YES & $1.71$ & $(2,3)$ & NO & 8578\\
$(297,113)$ & 13 & $(3,1)$ & 2 & 3 & YES & YES & YES & $1.71$ & $(2,3)$ & -- & 8579\\
$(297,68)$ & 13 & $(4,1)$ & 3 & 1 & YES & YES & YES & $1.29$ & $(2,3)$ & NO & 8580\\
$(297,113)$ & 13 & $(4,1)$ & 3 & 1 & YES & YES & YES & $1.71$ & $(2,3)$ & NO & 8581\\
$(297,113)$ & 13 & $(8,3)$ & 4 & 1 & YES & YES & YES & $1.57$ & $(2,3)$ & NO & 8582\\
$(297,109)$ & 12 & $(30,11)$ & 7 & 3 & YES & YES & YES & $1.57$ & $(2,3)$ & NO & 8583\\
$(298,53)$ & 15 & $(2,1)$ & 1 & 2 & YES & YES & YES & $1.57$ & $(2,3)$ & NO & 8584\\
$(298,53)$ & 15 & $(2,1)$ & 1 & 2 & YES & YES & YES & $1.57$ & $(2,3)$ & -- & 8585\\
$(298,53)$ & 15 & $(5,1)$ & 4 & 1 & YES & YES & YES & $1.71$ & $(2,3)$ & NO & 8586\\
$(298,91)$ & 13 & $(10,3)$ & 5 & 2 & YES & YES & YES & $1.57$ & $(2,3)$ & NO & 8587\\
$(298,53)$ & 15 & $(28,5)$ & 8 & 2 & YES & YES & YES & $1.71$ & $(2,3)$ & NO & 8588\\
$(299,87)$ & 14 & $(2,1)$ & 1 & 1 & YES & YES & YES & $1.71$ & $(2,3)$ & NO & 8589\\
$(299,116)$ & 12 & $(2,1)$ & 1 & 1 & YES & YES & YES & $1.43$ & $(2,3)$ & -- & 8590\\
$(299,116)$ & 12 & $(8,3)$ & 4 & 1 & YES & YES & YES & $1.43$ & $(2,3)$ & NO & 8591\\
$(299,116)$ & 12 & $(18,7)$ & 6 & 1 & YES & YES & YES & $1.57$ & $(2,3)$ & 7881 & 8592\\
$(299,111)$ & 13 & $(27,10)$ & 7 & 1 & YES & YES & YES & $1.71$ & $(2,3)$ & NO & 8593\\
$(299,89)$ & 13 & $(37,11)$ & 8 & 1 & YES & YES & YES & $1.57$ & $(2,3)$ & NO & 8594\\
$(299,111)$ & 13 & $(132,49)$ & 11 & 1 & YES & YES & YES & $1.71$ & $(2,3)$ & NO & 8595\\
$(300,89)$ & 13 & $(2,1)$ & 1 & 2 & YES & YES & YES & $1.71$ & $(2,3)$ & NO & 8596\\
$(300,91)$ & 13 & $(2,1)$ & 1 & 2 & YES & YES & YES & $1.57$ & $(2,3)$ & -- & 8597\\
$(300,89)$ & 13 & $(10,3)$ & 5 & 10 & YES & YES & YES & $1.71$ & $(2,3)$ & 5729 & 8598\\
$(300,91)$ & 13 & $(89,27)$ & 10 & 1 & YES & YES & YES & $1.57$ & $(2,3)$ & NO & 8599\\
$(301,88)$ & 13 & $(2,1)$ & 1 & 1 & YES & YES & YES & $1.57$ & $(2,3)$ & -- & 8600\\
$(301,88)$ & 13 & $(2,1)$ & 1 & 1 & YES & YES & YES & $1.57$ & $(2,3)$ & NO & 8601\\
$(301,88)$ & 13 & $(7,2)$ & 4 & 7 & YES & YES & YES & $1.57$ & $(2,3)$ & NO & 8602\\
$(301,88)$ & 13 & $(17,5)$ & 6 & 1 & YES & YES & YES & $1.57$ & $(2,3)$ & NO & 8603\\
$(301,88)$ & 13 & $(41,12)$ & 8 & 1 & YES & YES & YES & $1.43$ & $(2,3)$ & NO & 8604\\
$(301,131)$ & 13 & $(108,47)$ & 11 & 1 & YES & YES & YES & $1.71$ & $(2,3)$ & NO & 8605\\
$(302,109)$ & 13 & $(2,1)$ & 1 & 2 & YES & YES & YES & $1.71$ & $(2,3)$ & NO & 8606\\
$(302,131)$ & 13 & $(2,1)$ & 1 & 2 & NO & YES & YES & $1.71$ & $(2,3)$ & -- & 8607\\
$(302,57)$ & 15 & $(3,1)$ & 2 & 1 & YES & YES & YES & $1.43$ & $(2,3)$ & -- & 8608\\
$(302,111)$ & 12 & $(3,1)$ & 2 & 1 & YES & YES & YES & $1.57$ & $(2,3)$ & -- & 8609\\
$(302,57)$ & 15 & $(4,1)$ & 3 & 2 & YES & YES & YES & $1.57$ & $(2,3)$ & NO & 8610\\
$(302,57)$ & 15 & $(21,4)$ & 8 & 1 & YES & YES & YES & $1.57$ & $(2,3)$ & NO & 8611\\
$(302,85)$ & 14 & $(32,9)$ & 8 & 2 & YES & YES & YES & $1.71$ & $(2,3)$ & 7317 & 8612\\
$(303,82)$ & 13 & $(2,1)$ & 1 & 1 & YES & YES & YES & $1.71$ & $(2,3)$ & -- & 8613\\
$(303,85)$ & 13 & $(2,1)$ & 1 & 1 & YES & YES & YES & $1.57$ & $(2,3)$ & NO & 8614\\
$(303,82)$ & 13 & $(3,1)$ & 2 & 3 & YES & YES & YES & $1.57$ & $(2,3)$ & NO & 8615\\
$(303,85)$ & 13 & $(3,1)$ & 2 & 3 & YES & YES & YES & $1.57$ & $(2,3)$ & -- & 8616\\
$(303,116)$ & 12 & $(4,1)$ & 3 & 1 & YES & YES & YES & $1.43$ & $(2,3)$ & -- & 8617\\
$(303,128)$ & 12 & $(5,2)$ & 3 & 1 & YES & YES & YES & $1.43$ & $(2,3)$ & 7076 & 8618\\
$(303,85)$ & 13 & $(10,3)$ & 5 & 1 & YES & YES & YES & $1.57$ & $(2,3)$ & NO & 8619\\
$(303,85)$ & 13 & $(18,5)$ & 6 & 3 & YES & YES & YES & $1.43$ & $(2,3)$ & NO & 8620\\
$(303,85)$ & 13 & $(25,7)$ & 7 & 1 & YES & YES & YES & $1.43$ & $(2,3)$ & NO & 8621\\
$(303,85)$ & 13 & $(32,9)$ & 8 & 1 & YES & YES & YES & $1.57$ & $(2,3)$ & NO & 8622\\
$(303,116)$ & 12 & $(303,116)$ & 12 & 303 & YES & YES & YES & $1.43$ & $(2,3)$ & NO & 8623\\
$(304,71)$ & 14 & $(13,3)$ & 6 & 1 & YES & YES & YES & $1.57$ & $(2,3)$ & NO & 8624\\
$(304,93)$ & 13 & $(36,11)$ & 8 & 4 & YES & YES & YES & $1.57$ & $(2,3)$ & NO & 8625\\
$(304,85)$ & 13 & $(118,33)$ & 11 & 2 & YES & YES & YES & $1.71$ & $(2,3)$ & 8795 & 8626\\
$(305,93)$ & 13 & $(2,1)$ & 1 & 1 & YES & YES & YES & $1.57$ & $(2,3)$ & NO & 8627\\
$(305,93)$ & 13 & $(2,1)$ & 1 & 1 & YES & YES & YES & $1.71$ & $(2,3)$ & -- & 8628\\
$(305,82)$ & 13 & $(3,1)$ & 2 & 1 & YES & YES & YES & $1.57$ & $(2,3)$ & NO & 8629\\
$(305,112)$ & 12 & $(4,1)$ & 3 & 1 & YES & YES & YES & $1.57$ & $(2,3)$ & -- & 8630\\
$(305,112)$ & 12 & $(30,11)$ & 7 & 5 & YES & YES & YES & $1.43$ & $(2,3)$ & 6876 & 8631\\
$(307,72)$ & 14 & $(2,1)$ & 1 & 1 & YES & YES & YES & $1.71$ & $(2,3)$ & NO & 8632\\
$(307,126)$ & 13 & $(2,1)$ & 1 & 1 & YES & YES & YES & $1.71$ & $(2,3)$ & -- & 8633\\
$(307,126)$ & 13 & $(3,1)$ & 2 & 1 & YES & YES & YES & $1.71$ & $(2,3)$ & NO & 8634\\
$(307,119)$ & 12 & $(5,1)$ & 4 & 1 & YES & YES & YES & $1.71$ & $(2,3)$ & NO & 8635\\
$(307,126)$ & 13 & $(5,1)$ & 4 & 1 & YES & YES & YES & $1.71$ & $(2,3)$ & NO & 8636\\
$(307,129)$ & 12 & $(7,3)$ & 4 & 1 & YES & YES & YES & $1.43$ & $(2,3)$ & NO & 8637\\
$(307,134)$ & 13 & $(7,3)$ & 4 & 1 & YES & YES & YES & $1.57$ & $(2,3)$ & NO & 8638\\
$(307,69)$ & 14 & $(9,2)$ & 5 & 1 & YES & YES & YES & $1.57$ & $(2,3)$ & NO & 8639\\
$(307,72)$ & 14 & $(13,3)$ & 6 & 1 & YES & YES & YES & $1.57$ & $(2,3)$ & NO & 8640\\
$(307,119)$ & 12 & $(49,19)$ & 8 & 1 & YES & YES & YES & $1.57$ & $(2,3)$ & 7750 & 8641\\
$(307,111)$ & 13 & $(130,47)$ & 11 & 1 & YES & YES & YES & $1.86$ & $(2,3)$ & NO & 8642\\
$(308,117)$ & 12 & $(2,1)$ & 1 & 2 & YES & YES & YES & $1.71$ & $(2,3)$ & -- & 8643\\
$(308,73)$ & 14 & $(3,1)$ & 2 & 1 & YES & YES & YES & $1.57$ & $(2,3)$ & NO & 8644\\
$(308,73)$ & 14 & $(3,1)$ & 2 & 1 & YES & YES & YES & $1.57$ & $(2,3)$ & -- & 8645\\
$(308,73)$ & 14 & $(3,1)$ & 2 & 1 & YES & YES & YES & $1.71$ & $(2,3)$ & NO & 8646\\
$(308,73)$ & 14 & $(13,3)$ & 6 & 1 & YES & YES & YES & $1.57$ & $(2,3)$ & NO & 8647\\
$(308,117)$ & 12 & $(21,8)$ & 6 & 7 & YES & YES & YES & $1.57$ & $(2,3)$ & NO & 8648\\
$(308,117)$ & 12 & $(50,19)$ & 8 & 2 & YES & YES & YES & $1.71$ & $(2,3)$ & 7780 & 8649\\
$(309,113)$ & 13 & $(2,1)$ & 1 & 1 & YES & YES & YES & $1.71$ & $(2,3)$ & NO & 8650\\
$(309,92)$ & 13 & $(4,1)$ & 3 & 1 & YES & YES & YES & $1.57$ & $(2,3)$ & NO & 8651\\
$(309,113)$ & 13 & $(4,1)$ & 3 & 1 & YES & YES & YES & $1.86$ & $(2,3)$ & -- & 8652\\
$(311,67)$ & 14 & $(2,1)$ & 1 & 1 & YES & YES & YES & $1.71$ & $(2,3)$ & NO & 8653\\
$(311,84)$ & 13 & $(2,1)$ & 1 & 1 & YES & YES & YES & $1.71$ & $(2,3)$ & NO & 8654\\
$(311,71)$ & 13 & $(3,1)$ & 2 & 1 & YES & YES & YES & $1.43$ & $(2,3)$ & NO & 8655\\
$(311,84)$ & 13 & $(3,1)$ & 2 & 1 & YES & YES & YES & $1.43$ & $(2,3)$ & NO & 8656\\
$(311,120)$ & 13 & $(3,1)$ & 2 & 1 & YES & YES & YES & $1.57$ & $(2,3)$ & NO & 8657\\
$(311,65)$ & 14 & $(4,1)$ & 3 & 1 & YES & YES & YES & $1.57$ & $(2,3)$ & NO & 8658\\
$(311,119)$ & 12 & $(4,1)$ & 3 & 1 & YES & YES & YES & $1.43$ & $(2,3)$ & NO & 8659\\
$(311,84)$ & 13 & $(26,7)$ & 7 & 1 & YES & YES & YES & $1.57$ & $(2,3)$ & NO & 8660\\
$(313,91)$ & 13 & $(2,1)$ & 1 & 1 & YES & YES & YES & $1.57$ & $(2,3)$ & NO & 8661\\
$(313,119)$ & 12 & $(2,1)$ & 1 & 1 & YES & YES & YES & $1.57$ & $(2,3)$ & NO & 8662\\
$(313,121)$ & 12 & $(2,1)$ & 1 & 1 & YES & YES & YES & $1.71$ & $(2,3)$ & -- & 8663\\
$(313,93)$ & 13 & $(4,1)$ & 3 & 1 & YES & YES & YES & $1.43$ & $(2,3)$ & -- & 8664\\
$(313,121)$ & 12 & $(5,2)$ & 3 & 1 & YES & YES & YES & $1.57$ & $(2,3)$ & NO & 8665\\
$(313,121)$ & 12 & $(8,3)$ & 4 & 1 & YES & YES & YES & $1.71$ & $(2,3)$ & NO & 8666\\
$(313,71)$ & 14 & $(9,2)$ & 5 & 1 & YES & YES & YES & $1.71$ & $(2,3)$ & NO & 8667\\
$(313,121)$ & 12 & $(13,5)$ & 5 & 1 & YES & YES & YES & $1.57$ & $(2,3)$ & NO & 8668\\
$(313,83)$ & 13 & $(15,4)$ & 6 & 1 & YES & YES & YES & $1.43$ & $(2,3)$ & NO & 8669\\
$(314,87)$ & 13 & $(2,1)$ & 1 & 2 & YES & YES & YES & $1.57$ & $(2,3)$ & -- & 8670\\
$(314,129)$ & 13 & $(2,1)$ & 1 & 2 & YES & YES & YES & $1.71$ & $(2,3)$ & -- & 8671\\
$(314,129)$ & 13 & $(3,1)$ & 2 & 1 & YES & YES & YES & $1.71$ & $(2,3)$ & NO & 8672\\
$(314,129)$ & 13 & $(4,1)$ & 3 & 2 & YES & YES & YES & $1.57$ & $(2,3)$ & -- & 8673\\
$(314,129)$ & 13 & $(5,1)$ & 4 & 1 & YES & YES & YES & $1.57$ & $(2,3)$ & NO & 8674\\
$(314,129)$ & 13 & $(17,7)$ & 6 & 1 & YES & YES & YES & $1.57$ & $(2,3)$ & 7042 & 8675\\
$(315,71)$ & 14 & $(2,1)$ & 1 & 1 & YES & YES & NO(2) & $1.62$ & $(2,3)$ & -- & 8676\\
$(315,88)$ & 13 & $(2,1)$ & 1 & 1 & YES & YES & YES & $1.57$ & $(2,3)$ & -- & 8677\\
$(315,92)$ & 13 & $(2,1)$ & 1 & 1 & YES & YES & YES & $1.71$ & $(2,3)$ & NO & 8678\\
$(315,88)$ & 13 & $(3,1)$ & 2 & 3 & YES & YES & YES & $1.57$ & $(2,3)$ & NO & 8679\\
$(315,92)$ & 13 & $(4,1)$ & 3 & 1 & YES & YES & YES & $1.57$ & $(2,3)$ & NO & 8680\\
$(315,68)$ & 13 & $(5,1)$ & 4 & 5 & YES & YES & YES & $1.43$ & $(2,3)$ & NO & 8681\\
$(315,71)$ & 14 & $(102,23)$ & 11 & 3 & YES & YES & YES & $1.57$ & $(2,3)$ & NO & 8682\\
$(316,87)$ & 13 & $(2,1)$ & 1 & 2 & YES & YES & YES & $1.43$ & $(2,3)$ & NO & 8683\\
$(316,69)$ & 13 & $(3,1)$ & 2 & 1 & YES & YES & YES & $1.43$ & $(2,3)$ & NO & 8684\\
$(316,69)$ & 13 & $(3,1)$ & 2 & 1 & YES & YES & YES & $1.43$ & $(2,3)$ & -- & 8685\\
$(316,87)$ & 13 & $(3,1)$ & 2 & 1 & YES & YES & YES & $1.43$ & $(2,3)$ & NO & 8686\\
$(316,69)$ & 13 & $(5,1)$ & 4 & 1 & YES & YES & YES & $1.57$ & $(2,3)$ & NO & 8687\\
$(316,87)$ & 13 & $(7,2)$ & 4 & 1 & YES & YES & YES & $1.43$ & $(2,3)$ & NO & 8688\\
$(316,69)$ & 13 & $(14,3)$ & 6 & 2 & YES & YES & YES & $1.57$ & $(2,3)$ & NO & 8689\\
$(316,69)$ & 13 & $(41,9)$ & 9 & 1 & YES & YES & YES & $1.43$ & $(2,3)$ & NO & 8690\\
$(316,61)$ & 16 & $(259,50)$ & 15 & 1 & YES & YES & YES & $1.57$ & $(2,3)$ & NO & 8691\\
$(317,85)$ & 13 & $(2,1)$ & 1 & 1 & YES & YES & YES & $1.57$ & $(2,3)$ & NO & 8692\\
$(317,131)$ & 12 & $(2,1)$ & 1 & 1 & YES & YES & YES & $1.43$ & $(2,3)$ & -- & 8693\\
$(317,85)$ & 13 & $(3,1)$ & 2 & 1 & YES & YES & YES & $1.43$ & $(2,3)$ & NO & 8694\\
$(317,121)$ & 12 & $(3,1)$ & 2 & 1 & YES & YES & YES & $1.43$ & $(2,3)$ & NO & 8695\\
$(317,84)$ & 13 & $(4,1)$ & 3 & 1 & YES & YES & YES & $1.43$ & $(2,3)$ & NO & 8696\\
$(317,96)$ & 14 & $(4,1)$ & 3 & 1 & YES & YES & YES & $1.71$ & $(2,3)$ & NO & 8697\\
$(317,97)$ & 13 & $(4,1)$ & 3 & 1 & YES & YES & YES & $1.57$ & $(2,3)$ & NO & 8698\\
$(317,131)$ & 12 & $(12,5)$ & 5 & 1 & YES & YES & YES & $1.57$ & $(2,3)$ & 8539 & 8699\\
$(317,138)$ & 13 & $(16,7)$ & 6 & 1 & YES & YES & YES & $1.71$ & $(2,3)$ & NO & 8700\\
$(317,131)$ & 12 & $(17,7)$ & 6 & 1 & YES & YES & YES & $1.43$ & $(2,3)$ & NO & 8701\\
$(317,121)$ & 12 & $(21,8)$ & 6 & 1 & YES & YES & YES & $1.43$ & $(2,3)$ & 7054 & 8702\\
$(317,97)$ & 13 & $(36,11)$ & 8 & 1 & YES & YES & YES & $1.57$ & $(2,3)$ & 7013 & 8703\\
$(317,97)$ & 13 & $(183,56)$ & 12 & 1 & YES & YES & YES & $1.57$ & $(2,3)$ & NO & 8704\\
$(317,121)$ & 12 & $(317,121)$ & 12 & 317 & YES & YES & YES & $1.57$ & $(2,3)$ & NO & 8705\\
$(318,73)$ & 15 & $(2,1)$ & 1 & 2 & YES & YES & YES & $1.71$ & $(2,3)$ & -- & 8706\\
$(318,73)$ & 15 & $(13,3)$ & 6 & 1 & YES & YES & YES & $1.71$ & $(2,3)$ & NO & 8707\\
$(319,74)$ & 14 & $(2,1)$ & 1 & 1 & YES & YES & YES & $1.43$ & $(2,3)$ & -- & 8708\\
$(319,86)$ & 13 & $(3,1)$ & 2 & 1 & YES & YES & YES & $1.57$ & $(2,3)$ & NO & 8709\\
$(319,115)$ & 13 & $(4,1)$ & 3 & 1 & YES & YES & YES & $1.71$ & $(2,3)$ & -- & 8710\\
$(319,115)$ & 13 & $(5,2)$ & 3 & 1 & YES & YES & YES & $1.57$ & $(2,3)$ & NO & 8711\\
$(319,74)$ & 14 & $(69,16)$ & 11 & 1 & YES & YES & YES & $1.43$ & $(2,3)$ & 8224 & 8712\\
$(320,93)$ & 13 & $(4,1)$ & 3 & 4 & YES & YES & YES & $1.57$ & $(2,3)$ & -- & 8713\\
$(320,117)$ & 13 & $(8,3)$ & 4 & 8 & YES & YES & YES & $1.71$ & $(2,3)$ & NO & 8714\\
$(320,139)$ & 14 & $(76,33)$ & 10 & 4 & YES & YES & YES & $1.86$ & $(2,3)$ & NO & 8715\\
$(320,117)$ & 13 & $(93,34)$ & 10 & 1 & YES & YES & YES & $1.57$ & $(2,3)$ & NO & 8716\\
$(321,95)$ & 13 & $(2,1)$ & 1 & 1 & YES & YES & YES & $1.57$ & $(2,3)$ & NO & 8717\\
$(321,116)$ & 13 & $(2,1)$ & 1 & 1 & YES & YES & YES & $1.71$ & $(2,3)$ & NO & 8718\\
$(321,119)$ & 13 & $(2,1)$ & 1 & 1 & YES & YES & YES & $1.71$ & $(2,3)$ & NO & 8719\\
$(321,62)$ & 15 & $(3,1)$ & 2 & 3 & YES & YES & YES & $1.57$ & $(2,3)$ & -- & 8720\\
$(321,94)$ & 13 & $(3,1)$ & 2 & 3 & YES & YES & YES & $1.57$ & $(2,3)$ & NO & 8721\\
$(321,94)$ & 13 & $(3,1)$ & 2 & 3 & YES & YES & YES & $1.57$ & $(2,3)$ & NO & 8722\\
$(321,94)$ & 13 & $(3,1)$ & 2 & 3 & YES & YES & YES & $1.57$ & $(2,3)$ & -- & 8723\\
$(321,119)$ & 13 & $(3,1)$ & 2 & 3 & YES & YES & YES & $1.71$ & $(2,3)$ & NO & 8724\\
$(321,94)$ & 13 & $(4,1)$ & 3 & 1 & YES & YES & YES & $1.43$ & $(2,3)$ & -- & 8725\\
$(321,95)$ & 13 & $(4,1)$ & 3 & 1 & YES & YES & YES & $1.57$ & $(2,3)$ & NO & 8726\\
$(321,95)$ & 13 & $(7,2)$ & 4 & 1 & YES & YES & YES & $1.57$ & $(2,3)$ & NO & 8727\\
$(321,119)$ & 13 & $(8,3)$ & 4 & 1 & YES & YES & YES & $1.71$ & $(2,3)$ & NO & 8728\\
$(321,94)$ & 13 & $(17,5)$ & 6 & 1 & YES & YES & YES & $1.57$ & $(2,3)$ & NO & 8729\\
$(321,62)$ & 15 & $(26,5)$ & 9 & 1 & YES & YES & YES & $1.57$ & $(2,3)$ & NO & 8730\\
$(321,116)$ & 13 & $(36,13)$ & 8 & 3 & YES & YES & YES & $1.71$ & $(2,3)$ & NO & 8731\\
$(321,119)$ & 13 & $(62,23)$ & 9 & 1 & YES & YES & YES & $1.71$ & $(2,3)$ & 8980 & 8732\\
$(321,94)$ & 13 & $(99,29)$ & 10 & 3 & YES & YES & YES & $1.57$ & $(2,3)$ & NO & 8733\\
$(321,97)$ & 14 & $(139,42)$ & 12 & 1 & YES & YES & YES & $1.57$ & $(2,3)$ & NO & 8734\\
$(322,75)$ & 14 & $(2,1)$ & 1 & 2 & YES & YES & YES & $1.43$ & $(2,3)$ & -- & 8735\\
$(322,75)$ & 14 & $(2,1)$ & 1 & 2 & YES & YES & YES & $1.57$ & $(2,3)$ & NO & 8736\\
$(322,59)$ & 15 & $(3,1)$ & 2 & 1 & YES & YES & YES & $1.57$ & $(2,3)$ & NO & 8737\\
$(322,59)$ & 15 & $(3,1)$ & 2 & 1 & YES & YES & YES & $1.57$ & $(2,3)$ & -- & 8738\\
$(322,59)$ & 15 & $(3,1)$ & 2 & 1 & YES & YES & YES & $1.71$ & $(2,3)$ & NO & 8739\\
$(322,75)$ & 14 & $(3,1)$ & 2 & 1 & YES & YES & YES & $1.57$ & $(2,3)$ & NO & 8740\\
$(322,59)$ & 15 & $(4,1)$ & 3 & 2 & YES & YES & YES & $1.71$ & $(2,3)$ & NO & 8741\\
$(322,73)$ & 14 & $(4,1)$ & 3 & 2 & YES & YES & YES & $1.43$ & $(2,3)$ & -- & 8742\\
$(322,75)$ & 14 & $(17,4)$ & 7 & 1 & YES & YES & YES & $1.57$ & $(2,3)$ & NO & 8743\\
$(322,75)$ & 14 & $(30,7)$ & 8 & 2 & YES & YES & YES & $1.57$ & $(2,3)$ & NO & 8744\\
$(322,75)$ & 14 & $(43,10)$ & 9 & 1 & YES & YES & YES & $1.43$ & $(2,3)$ & NO & 8745\\
$(322,75)$ & 14 & $(103,24)$ & 11 & 1 & YES & YES & YES & $1.57$ & $(2,3)$ & NO & 8746\\
$(322,59)$ & 15 & $(191,35)$ & 14 & 1 & YES & YES & YES & $1.57$ & $(2,3)$ & NO & 8747\\
$(322,59)$ & 15 & $(322,59)$ & 15 & 322 & YES & YES & YES & $1.71$ & $(2,3)$ & NO & 8748\\
$(323,89)$ & 13 & $(2,1)$ & 1 & 1 & YES & YES & YES & $1.43$ & $(2,3)$ & -- & 8749\\
$(323,89)$ & 13 & $(2,1)$ & 1 & 1 & YES & YES & YES & $1.57$ & $(2,3)$ & NO & 8750\\
$(323,94)$ & 13 & $(2,1)$ & 1 & 1 & YES & YES & YES & $1.71$ & $(2,3)$ & NO & 8751\\
$(323,94)$ & 13 & $(2,1)$ & 1 & 1 & YES & YES & YES & $1.71$ & $(2,3)$ & -- & 8752\\
$(323,126)$ & 13 & $(2,1)$ & 1 & 1 & YES & YES & YES & $1.71$ & $(2,3)$ & -- & 8753\\
$(323,89)$ & 13 & $(3,1)$ & 2 & 1 & YES & YES & YES & $1.57$ & $(2,3)$ & -- & 8754\\
$(323,94)$ & 13 & $(3,1)$ & 2 & 1 & YES & YES & YES & $1.57$ & $(2,3)$ & -- & 8755\\
$(323,126)$ & 13 & $(3,1)$ & 2 & 1 & YES & YES & YES & $1.71$ & $(2,3)$ & NO & 8756\\
$(323,141)$ & 13 & $(3,1)$ & 2 & 1 & YES & YES & YES & $1.71$ & $(2,3)$ & NO & 8757\\
$(323,94)$ & 13 & $(5,1)$ & 4 & 1 & YES & YES & YES & $1.57$ & $(2,3)$ & NO & 8758\\
$(323,94)$ & 13 & $(5,1)$ & 4 & 1 & YES & YES & YES & $1.57$ & $(2,3)$ & -- & 8759\\
$(323,60)$ & 14 & $(38,7)$ & 9 & 19 & YES & YES & YES & $1.57$ & $(2,3)$ & NO & 8760\\
$(323,94)$ & 13 & $(55,16)$ & 9 & 1 & YES & YES & YES & $1.57$ & $(2,3)$ & 7968 & 8761\\
$(323,60)$ & 14 & $(59,11)$ & 10 & 1 & YES & YES & YES & $1.57$ & $(2,3)$ & NO & 8762\\
$(324,71)$ & 14 & $(2,1)$ & 1 & 2 & YES & YES & NO(2) & $1.50$ & $(2,3)$ & -- & 8763\\
$(324,89)$ & 13 & $(2,1)$ & 1 & 2 & YES & YES & YES & $1.43$ & $(2,3)$ & -- & 8764\\
$(324,89)$ & 13 & $(2,1)$ & 1 & 2 & YES & YES & YES & $1.57$ & $(2,3)$ & NO & 8765\\
$(324,91)$ & 13 & $(2,1)$ & 1 & 2 & YES & YES & YES & $1.57$ & $(2,3)$ & NO & 8766\\
$(324,89)$ & 13 & $(3,1)$ & 2 & 3 & YES & YES & YES & $1.43$ & $(2,3)$ & -- & 8767\\
$(324,89)$ & 13 & $(3,1)$ & 2 & 3 & YES & YES & YES & $1.57$ & $(2,3)$ & NO & 8768\\
$(324,133)$ & 13 & $(3,1)$ & 2 & 3 & YES & YES & YES & $1.71$ & $(2,3)$ & NO & 8769\\
$(324,89)$ & 13 & $(5,1)$ & 4 & 1 & YES & YES & YES & $1.57$ & $(2,3)$ & -- & 8770\\
$(324,89)$ & 13 & $(40,11)$ & 8 & 4 & YES & YES & YES & $1.57$ & $(2,3)$ & NO & 8771\\
$(324,95)$ & 13 & $(58,17)$ & 9 & 2 & YES & YES & YES & $1.43$ & $(2,3)$ & 8037 & 8772\\
$(324,89)$ & 13 & $(91,25)$ & 10 & 1 & YES & YES & YES & $1.57$ & $(2,3)$ & NO & 8773\\
$(324,133)$ & 13 & $(95,39)$ & 10 & 1 & YES & YES & YES & $1.57$ & $(2,3)$ & NO & 8774\\
$(324,71)$ & 14 & $(324,71)$ & 14 & 324 & YES & YES & YES & $1.71$ & $(2,3)$ & NO & 8775\\
$(325,76)$ & 13 & $(5,2)$ & 3 & 5 & YES & YES & YES & $1.43$ & $(2,3)$ & -- & 8776\\
$(326,135)$ & 13 & $(2,1)$ & 1 & 2 & YES & YES & YES & $1.71$ & $(2,3)$ & -- & 8777\\
$(326,71)$ & 14 & $(5,2)$ & 3 & 1 & YES & YES & YES & $1.71$ & $(2,3)$ & -- & 8778\\
$(326,71)$ & 14 & $(7,2)$ & 4 & 1 & YES & YES & YES & $1.71$ & $(2,3)$ & NO & 8779\\
$(326,97)$ & 13 & $(37,11)$ & 8 & 1 & YES & YES & YES & $1.43$ & $(2,3)$ & NO & 8780\\
$(327,62)$ & 15 & $(2,1)$ & 1 & 1 & YES & YES & YES & $1.71$ & $(2,3)$ & NO & 8781\\
$(327,97)$ & 13 & $(2,1)$ & 1 & 1 & YES & YES & YES & $1.57$ & $(2,3)$ & NO & 8782\\
$(327,100)$ & 13 & $(2,1)$ & 1 & 1 & YES & YES & YES & $1.43$ & $(2,3)$ & NO & 8783\\
$(327,121)$ & 13 & $(2,1)$ & 1 & 1 & YES & YES & YES & $1.71$ & $(2,3)$ & NO & 8784\\
$(327,97)$ & 13 & $(7,2)$ & 4 & 1 & YES & YES & YES & $1.57$ & $(2,3)$ & NO & 8785\\
$(327,118)$ & 13 & $(11,4)$ & 5 & 1 & YES & YES & YES & $1.86$ & $(2,3)$ & 6872 & 8786\\
$(327,62)$ & 15 & $(16,3)$ & 7 & 1 & YES & YES & YES & $1.57$ & $(2,3)$ & NO & 8787\\
$(327,142)$ & 14 & $(23,10)$ & 7 & 1 & YES & YES & YES & $1.86$ & $(2,3)$ & NO & 8788\\
$(329,89)$ & 13 & $(2,1)$ & 1 & 1 & YES & YES & YES & $1.57$ & $(2,3)$ & -- & 8789\\
$(329,89)$ & 13 & $(3,1)$ & 2 & 1 & YES & YES & YES & $1.43$ & $(2,3)$ & NO & 8790\\
$(329,92)$ & 13 & $(3,1)$ & 2 & 1 & YES & YES & YES & $1.57$ & $(2,3)$ & NO & 8791\\
$(329,61)$ & 15 & $(7,1)$ & 6 & 7 & YES & YES & YES & $1.57$ & $(2,3)$ & NO & 8792\\
$(329,61)$ & 15 & $(16,3)$ & 7 & 1 & YES & YES & YES & $1.57$ & $(2,3)$ & NO & 8793\\
$(329,122)$ & 13 & $(89,33)$ & 10 & 1 & YES & YES & YES & $1.57$ & $(2,3)$ & NO & 8794\\
$(329,92)$ & 13 & $(93,26)$ & 10 & 1 & YES & YES & YES & $1.71$ & $(2,3)$ & 8626 & 8795\\
$(329,89)$ & 13 & $(122,33)$ & 11 & 1 & YES & YES & YES & $1.57$ & $(2,3)$ & NO & 8796\\
$(330,59)$ & 15 & $(4,1)$ & 3 & 2 & YES & YES & YES & $1.43$ & $(2,3)$ & -- & 8797\\
$(331,123)$ & 14 & $(2,1)$ & 1 & 1 & YES & YES & YES & $1.86$ & $(2,3)$ & NO & 8798\\
$(331,129)$ & 13 & $(5,1)$ & 4 & 1 & YES & YES & YES & $1.57$ & $(2,3)$ & NO & 8799\\
$(331,89)$ & 13 & $(11,3)$ & 5 & 1 & YES & YES & YES & $1.43$ & $(2,3)$ & NO & 8800\\
$(332,97)$ & 13 & $(3,1)$ & 2 & 1 & YES & YES & YES & $1.57$ & $(2,3)$ & NO & 8801\\
$(332,97)$ & 13 & $(4,1)$ & 3 & 4 & YES & YES & YES & $1.57$ & $(2,3)$ & -- & 8802\\
$(332,97)$ & 13 & $(7,2)$ & 4 & 1 & YES & YES & YES & $1.57$ & $(2,3)$ & NO & 8803\\
$(333,76)$ & 13 & $(2,1)$ & 1 & 1 & YES & YES & YES & $1.43$ & $(2,3)$ & -- & 8804\\
$(333,97)$ & 14 & $(2,1)$ & 1 & 1 & YES & YES & YES & $1.71$ & $(2,3)$ & NO & 8805\\
$(333,101)$ & 13 & $(2,1)$ & 1 & 1 & YES & YES & YES & $1.57$ & $(2,3)$ & -- & 8806\\
$(333,122)$ & 13 & $(2,1)$ & 1 & 1 & YES & YES & YES & $1.71$ & $(2,3)$ & NO & 8807\\
$(333,122)$ & 13 & $(4,1)$ & 3 & 1 & YES & YES & YES & $1.57$ & $(2,3)$ & -- & 8808\\
$(333,101)$ & 13 & $(5,1)$ & 4 & 1 & YES & YES & YES & $1.43$ & $(2,3)$ & NO & 8809\\
$(333,101)$ & 13 & $(23,7)$ & 7 & 1 & YES & YES & YES & $1.57$ & $(2,3)$ & NO & 8810\\
$(333,101)$ & 13 & $(122,37)$ & 11 & 1 & YES & YES & YES & $1.57$ & $(2,3)$ & NO & 8811\\
$(334,129)$ & 13 & $(5,1)$ & 4 & 1 & YES & YES & YES & $1.57$ & $(2,3)$ & NO & 8812\\
$(334,129)$ & 13 & $(5,1)$ & 4 & 1 & YES & YES & YES & $1.57$ & $(2,3)$ & -- & 8813\\
$(334,129)$ & 13 & $(13,5)$ & 5 & 1 & YES & YES & YES & $1.71$ & $(2,3)$ & NO & 8814\\
$(335,78)$ & 14 & $(2,1)$ & 1 & 1 & YES & YES & YES & $1.43$ & $(2,3)$ & -- & 8815\\
$(335,94)$ & 13 & $(3,1)$ & 2 & 1 & NO & YES & YES & $1.57$ & $(2,3)$ & -- & 8816\\
$(335,73)$ & 14 & $(4,1)$ & 3 & 1 & YES & YES & YES & $1.57$ & $(2,3)$ & -- & 8817\\
$(335,94)$ & 13 & $(4,1)$ & 3 & 1 & YES & YES & YES & $1.57$ & $(2,3)$ & NO & 8818\\
$(335,94)$ & 13 & $(11,3)$ & 5 & 1 & YES & YES & YES & $1.57$ & $(2,3)$ & NO & 8819\\
$(335,98)$ & 13 & $(17,5)$ & 6 & 1 & YES & YES & YES & $1.57$ & $(2,3)$ & NO & 8820\\
$(335,78)$ & 14 & $(43,10)$ & 9 & 1 & YES & YES & YES & $1.43$ & $(2,3)$ & NO & 8821\\
$(335,78)$ & 14 & $(262,61)$ & 13 & 1 & YES & YES & YES & $1.57$ & $(2,3)$ & NO & 8822\\
$(337,91)$ & 13 & $(2,1)$ & 1 & 1 & YES & YES & YES & $1.57$ & $(2,3)$ & NO & 8823\\
$(337,94)$ & 13 & $(2,1)$ & 1 & 1 & YES & YES & YES & $1.57$ & $(2,3)$ & NO & 8824\\
$(337,100)$ & 13 & $(2,1)$ & 1 & 1 & YES & YES & YES & $1.43$ & $(2,3)$ & -- & 8825\\
$(337,102)$ & 14 & $(2,1)$ & 1 & 1 & YES & YES & YES & $1.71$ & $(2,3)$ & NO & 8826\\
$(337,91)$ & 13 & $(3,1)$ & 2 & 1 & YES & YES & YES & $1.57$ & $(2,3)$ & -- & 8827\\
$(337,94)$ & 13 & $(4,1)$ & 3 & 1 & YES & YES & YES & $1.43$ & $(2,3)$ & -- & 8828\\
$(337,98)$ & 13 & $(4,1)$ & 3 & 1 & YES & YES & YES & $1.43$ & $(2,3)$ & NO & 8829\\
$(337,102)$ & 14 & $(4,1)$ & 3 & 1 & YES & YES & YES & $1.57$ & $(2,3)$ & -- & 8830\\
$(337,91)$ & 13 & $(5,1)$ & 4 & 1 & YES & YES & YES & $1.43$ & $(2,3)$ & NO & 8831\\
$(337,129)$ & 12 & $(5,2)$ & 3 & 1 & YES & YES & YES & $1.57$ & $(2,3)$ & NO & 8832\\
$(337,94)$ & 13 & $(7,2)$ & 4 & 1 & YES & YES & YES & $1.57$ & $(2,3)$ & NO & 8833\\
$(337,80)$ & 14 & $(38,9)$ & 9 & 1 & YES & YES & YES & $1.43$ & $(2,3)$ & NO & 8834\\
$(338,79)$ & 14 & $(2,1)$ & 1 & 2 & YES & YES & YES & $1.57$ & $(2,3)$ & -- & 8835\\
$(338,129)$ & 12 & $(2,1)$ & 1 & 2 & YES & YES & YES & $1.43$ & $(2,3)$ & -- & 8836\\
$(338,131)$ & 12 & $(2,1)$ & 1 & 2 & NO & YES & YES & $1.43$ & $(2,3)$ & -- & 8837\\
$(338,79)$ & 14 & $(4,1)$ & 3 & 2 & YES & YES & YES & $1.43$ & $(2,3)$ & -- & 8838\\
$(338,129)$ & 12 & $(5,2)$ & 3 & 1 & YES & YES & YES & $1.43$ & $(2,3)$ & NO & 8839\\
$(338,79)$ & 14 & $(13,3)$ & 6 & 13 & YES & YES & YES & $1.57$ & $(2,3)$ & NO & 8840\\
$(338,129)$ & 12 & $(13,5)$ & 5 & 13 & YES & YES & YES & $1.43$ & $(2,3)$ & 7465 & 8841\\
$(339,65)$ & 15 & $(2,1)$ & 1 & 1 & YES & YES & YES & $1.71$ & $(2,3)$ & NO & 8842\\
$(339,73)$ & 15 & $(2,1)$ & 1 & 1 & YES & YES & YES & $1.71$ & $(2,3)$ & NO & 8843\\
$(339,80)$ & 15 & $(2,1)$ & 1 & 1 & YES & YES & YES & $1.71$ & $(2,3)$ & -- & 8844\\
$(339,100)$ & 13 & $(2,1)$ & 1 & 1 & YES & YES & YES & $1.43$ & $(2,3)$ & -- & 8845\\
$(339,91)$ & 13 & $(3,1)$ & 2 & 3 & YES & YES & YES & $1.57$ & $(2,3)$ & NO & 8846\\
$(339,73)$ & 15 & $(4,1)$ & 3 & 1 & YES & YES & YES & $1.57$ & $(2,3)$ & NO & 8847\\
$(339,79)$ & 14 & $(4,1)$ & 3 & 1 & YES & YES & YES & $1.43$ & $(2,3)$ & -- & 8848\\
$(339,100)$ & 13 & $(4,1)$ & 3 & 1 & YES & YES & YES & $1.43$ & $(2,3)$ & NO & 8849\\
$(339,100)$ & 13 & $(10,3)$ & 5 & 1 & YES & YES & YES & $1.71$ & $(2,3)$ & NO & 8850\\
$(339,80)$ & 15 & $(17,4)$ & 7 & 1 & YES & YES & YES & $1.71$ & $(2,3)$ & NO & 8851\\
$(339,79)$ & 14 & $(103,24)$ & 11 & 1 & YES & YES & NO(2) & $1.50$ & $(2,3)$ & NO & 8852\\
$(339,100)$ & 13 & $(139,41)$ & 11 & 1 & YES & YES & YES & $1.43$ & $(2,3)$ & NO & 8853\\
$(340,101)$ & 13 & $(2,1)$ & 1 & 2 & YES & YES & YES & $1.43$ & $(2,3)$ & -- & 8854\\
$(340,79)$ & 14 & $(4,1)$ & 3 & 4 & YES & YES & YES & $1.43$ & $(2,3)$ & -- & 8855\\
$(341,100)$ & 13 & $(2,1)$ & 1 & 1 & YES & YES & YES & $1.57$ & $(2,3)$ & -- & 8856\\
$(341,100)$ & 13 & $(4,1)$ & 3 & 1 & YES & YES & YES & $1.57$ & $(2,3)$ & NO & 8857\\
$(341,133)$ & 13 & $(100,39)$ & 10 & 1 & YES & YES & YES & $1.57$ & $(2,3)$ & NO & 8858\\
$(342,77)$ & 14 & $(13,3)$ & 6 & 1 & YES & YES & YES & $1.57$ & $(2,3)$ & NO & 8859\\
$(343,104)$ & 14 & $(5,1)$ & 4 & 1 & YES & YES & YES & $1.57$ & $(2,3)$ & -- & 8860\\
$(344,95)$ & 13 & $(3,1)$ & 2 & 1 & YES & YES & YES & $1.57$ & $(2,3)$ & NO & 8861\\
$(344,95)$ & 13 & $(3,1)$ & 2 & 1 & YES & YES & YES & $1.57$ & $(2,3)$ & -- & 8862\\
$(345,143)$ & 13 & $(152,63)$ & 11 & 1 & YES & YES & YES & $1.57$ & $(2,3)$ & NO & 8863\\
$(346,93)$ & 13 & $(2,1)$ & 1 & 2 & YES & YES & YES & $1.43$ & $(2,3)$ & NO & 8864\\
$(346,125)$ & 14 & $(2,1)$ & 1 & 2 & YES & YES & YES & $1.71$ & $(2,3)$ & NO & 8865\\
$(346,93)$ & 13 & $(3,1)$ & 2 & 1 & YES & YES & YES & $1.43$ & $(2,3)$ & NO & 8866\\
$(346,105)$ & 13 & $(4,1)$ & 3 & 2 & YES & YES & YES & $1.57$ & $(2,3)$ & -- & 8867\\
$(346,93)$ & 13 & $(11,3)$ & 5 & 1 & YES & YES & YES & $1.43$ & $(2,3)$ & NO & 8868\\
$(346,105)$ & 13 & $(56,17)$ & 9 & 2 & YES & YES & YES & $1.57$ & $(2,3)$ & 8145 & 8869\\
$(346,125)$ & 14 & $(155,56)$ & 12 & 1 & YES & YES & YES & $1.71$ & $(2,3)$ & NO & 8870\\
$(346,105)$ & 13 & $(201,61)$ & 12 & 1 & YES & YES & YES & $1.57$ & $(2,3)$ & NO & 8871\\
$(347,97)$ & 13 & $(2,1)$ & 1 & 1 & YES & YES & YES & $1.57$ & $(2,3)$ & NO & 8872\\
$(347,101)$ & 13 & $(2,1)$ & 1 & 1 & YES & YES & YES & $1.57$ & $(2,3)$ & NO & 8873\\
$(347,101)$ & 13 & $(2,1)$ & 1 & 1 & YES & YES & YES & $1.71$ & $(2,3)$ & -- & 8874\\
$(347,103)$ & 14 & $(2,1)$ & 1 & 1 & YES & YES & YES & $1.71$ & $(2,3)$ & NO & 8875\\
$(347,151)$ & 14 & $(2,1)$ & 1 & 1 & NO & YES & YES & $1.71$ & $(2,3)$ & -- & 8876\\
$(347,78)$ & 14 & $(4,1)$ & 3 & 1 & YES & YES & YES & $1.43$ & $(2,3)$ & -- & 8877\\
$(349,102)$ & 14 & $(2,1)$ & 1 & 1 & YES & YES & YES & $1.57$ & $(2,3)$ & -- & 8878\\
$(349,102)$ & 14 & $(2,1)$ & 1 & 1 & YES & YES & YES & $1.57$ & $(2,3)$ & NO & 8879\\
$(349,143)$ & 13 & $(3,1)$ & 2 & 1 & YES & YES & YES & $1.57$ & $(2,3)$ & NO & 8880\\
$(349,143)$ & 13 & $(12,5)$ & 5 & 1 & YES & YES & YES & $1.71$ & $(2,3)$ & NO & 8881\\
$(350,107)$ & 14 & $(10,3)$ & 5 & 10 & YES & YES & YES & $1.71$ & $(2,3)$ & NO & 8882\\
$(350,93)$ & 14 & $(143,38)$ & 12 & 1 & YES & YES & YES & $1.71$ & $(2,3)$ & NO & 8883\\
$(351,76)$ & 13 & $(2,1)$ & 1 & 1 & YES & YES & YES & $1.57$ & $(2,3)$ & -- & 8884\\
$(351,127)$ & 14 & $(152,55)$ & 12 & 1 & YES & YES & YES & $1.86$ & $(2,3)$ & NO & 8885\\
$(353,74)$ & 15 & $(2,1)$ & 1 & 1 & YES & YES & YES & $1.57$ & $(2,3)$ & NO & 8886\\
$(353,75)$ & 14 & $(2,1)$ & 1 & 1 & YES & YES & YES & $1.43$ & $(2,3)$ & -- & 8887\\
$(353,75)$ & 14 & $(4,1)$ & 3 & 1 & YES & YES & YES & $1.57$ & $(2,3)$ & -- & 8888\\
$(353,82)$ & 14 & $(5,1)$ & 4 & 1 & YES & YES & YES & $1.57$ & $(2,3)$ & NO & 8889\\
$(353,75)$ & 14 & $(14,3)$ & 6 & 1 & YES & YES & YES & $1.71$ & $(2,3)$ & NO & 8890\\
$(353,75)$ & 14 & $(19,4)$ & 7 & 1 & YES & YES & YES & $1.43$ & $(2,3)$ & NO & 8891\\
$(353,75)$ & 14 & $(47,10)$ & 9 & 1 & YES & YES & YES & $1.71$ & $(2,3)$ & NO & 8892\\
$(355,81)$ & 13 & $(2,1)$ & 1 & 1 & YES & YES & YES & $1.43$ & $(2,3)$ & -- & 8893\\
$(355,77)$ & 14 & $(4,1)$ & 3 & 1 & YES & YES & YES & $1.57$ & $(2,3)$ & NO & 8894\\
$(355,62)$ & 15 & $(5,1)$ & 4 & 5 & YES & YES & YES & $1.71$ & $(2,3)$ & NO & 8895\\
$(356,155)$ & 13 & $(3,1)$ & 2 & 1 & YES & YES & YES & $1.57$ & $(2,3)$ & NO & 8896\\
$(359,75)$ & 15 & $(2,1)$ & 1 & 1 & YES & YES & YES & $1.71$ & $(2,3)$ & NO & 8897\\
$(359,100)$ & 13 & $(3,1)$ & 2 & 1 & YES & YES & YES & $1.57$ & $(2,3)$ & NO & 8898\\
$(359,105)$ & 13 & $(253,74)$ & 12 & 1 & YES & YES & YES & $1.43$ & $(2,3)$ & NO & 8899\\
$(360,109)$ & 14 & $(2,1)$ & 1 & 2 & YES & YES & YES & $1.57$ & $(2,3)$ & -- & 8900\\
$(360,109)$ & 14 & $(109,33)$ & 11 & 1 & YES & YES & YES & $1.57$ & $(2,3)$ & NO & 8901\\
$(360,109)$ & 14 & $(251,76)$ & 13 & 1 & YES & YES & YES & $1.57$ & $(2,3)$ & NO & 8902\\
$(361,84)$ & 14 & $(3,1)$ & 2 & 1 & YES & YES & YES & $1.43$ & $(2,3)$ & -- & 8903\\
$(361,84)$ & 14 & $(3,1)$ & 2 & 1 & YES & YES & YES & $1.57$ & $(2,3)$ & NO & 8904\\
$(361,84)$ & 14 & $(13,3)$ & 6 & 1 & YES & YES & NO(2) & $1.50$ & $(2,3)$ & NO & 8905\\
$(361,84)$ & 14 & $(17,4)$ & 7 & 1 & YES & YES & YES & $1.43$ & $(2,3)$ & NO & 8906\\
$(362,131)$ & 14 & $(2,1)$ & 1 & 2 & YES & YES & YES & $1.86$ & $(2,3)$ & -- & 8907\\
$(363,85)$ & 14 & $(2,1)$ & 1 & 1 & YES & YES & NO(2) & $1.62$ & $(2,3)$ & NO & 8908\\
$(365,98)$ & 13 & $(4,1)$ & 3 & 1 & YES & YES & YES & $1.57$ & $(2,3)$ & -- & 8909\\
$(367,78)$ & 14 & $(2,1)$ & 1 & 1 & YES & YES & YES & $1.43$ & $(2,3)$ & -- & 8910\\
$(367,83)$ & 14 & $(2,1)$ & 1 & 1 & YES & YES & YES & $1.57$ & $(2,3)$ & -- & 8911\\
$(367,101)$ & 13 & $(2,1)$ & 1 & 1 & YES & YES & YES & $1.43$ & $(2,3)$ & -- & 8912\\
$(367,101)$ & 13 & $(2,1)$ & 1 & 1 & YES & YES & YES & $1.57$ & $(2,3)$ & NO & 8913\\
$(367,112)$ & 13 & $(2,1)$ & 1 & 1 & YES & YES & YES & $1.43$ & $(2,3)$ & -- & 8914\\
$(367,83)$ & 14 & $(3,1)$ & 2 & 1 & YES & YES & YES & $1.57$ & $(2,3)$ & -- & 8915\\
$(367,87)$ & 14 & $(4,1)$ & 3 & 1 & YES & YES & YES & $1.43$ & $(2,3)$ & NO & 8916\\
$(367,78)$ & 14 & $(14,3)$ & 6 & 1 & YES & YES & YES & $1.43$ & $(2,3)$ & NO & 8917\\
$(367,78)$ & 14 & $(47,10)$ & 9 & 1 & YES & YES & YES & $1.43$ & $(2,3)$ & NO & 8918\\
$(367,83)$ & 14 & $(115,26)$ & 11 & 1 & YES & YES & YES & $1.43$ & $(2,3)$ & NO & 8919\\
$(367,84)$ & 14 & $(118,27)$ & 11 & 1 & YES & YES & YES & $1.57$ & $(2,3)$ & NO & 8920\\
$(367,84)$ & 14 & $(367,84)$ & 14 & 367 & YES & YES & YES & $1.43$ & $(2,3)$ & NO & 8921\\
$(368,133)$ & 14 & $(2,1)$ & 1 & 2 & YES & YES & YES & $1.86$ & $(2,3)$ & NO & 8922\\
$(368,133)$ & 14 & $(83,30)$ & 10 & 1 & YES & YES & YES & $1.86$ & $(2,3)$ & NO & 8923\\
$(370,59)$ & 16 & $(3,1)$ & 2 & 1 & YES & YES & YES & $1.43$ & $(2,3)$ & NO & 8924\\
$(370,59)$ & 16 & $(5,1)$ & 4 & 5 & YES & YES & YES & $1.43$ & $(2,3)$ & NO & 8925\\
$(371,87)$ & 14 & $(2,1)$ & 1 & 1 & YES & YES & YES & $1.57$ & $(2,3)$ & NO & 8926\\
$(371,88)$ & 14 & $(2,1)$ & 1 & 1 & YES & YES & YES & $1.57$ & $(2,3)$ & -- & 8927\\
$(371,144)$ & 13 & $(2,1)$ & 1 & 1 & NO & YES & YES & $1.57$ & $(2,3)$ & -- & 8928\\
$(371,88)$ & 14 & $(38,9)$ & 9 & 1 & YES & YES & YES & $1.43$ & $(2,3)$ & 7621 & 8929\\
$(372,109)$ & 13 & $(4,1)$ & 3 & 4 & YES & YES & YES & $1.43$ & $(2,3)$ & -- & 8930\\
$(372,109)$ & 13 & $(7,2)$ & 4 & 1 & YES & YES & YES & $1.57$ & $(2,3)$ & NO & 8931\\
$(373,79)$ & 14 & $(2,1)$ & 1 & 1 & YES & YES & YES & $1.43$ & $(2,3)$ & -- & 8932\\
$(373,79)$ & 14 & $(4,1)$ & 3 & 1 & YES & YES & YES & $1.57$ & $(2,3)$ & -- & 8933\\
$(373,109)$ & 13 & $(4,1)$ & 3 & 1 & YES & YES & YES & $1.43$ & $(2,3)$ & -- & 8934\\
$(373,79)$ & 14 & $(14,3)$ & 6 & 1 & YES & YES & YES & $1.43$ & $(2,3)$ & NO & 8935\\
$(373,109)$ & 13 & $(65,19)$ & 9 & 1 & YES & YES & YES & $1.43$ & $(2,3)$ & 8465 & 8936\\
$(374,101)$ & 13 & $(2,1)$ & 1 & 2 & YES & YES & YES & $1.43$ & $(2,3)$ & -- & 8937\\
$(374,101)$ & 13 & $(2,1)$ & 1 & 2 & YES & YES & YES & $1.43$ & $(2,3)$ & NO & 8938\\
$(374,101)$ & 13 & $(3,1)$ & 2 & 1 & YES & YES & YES & $1.57$ & $(2,3)$ & NO & 8939\\
$(374,155)$ & 13 & $(3,1)$ & 2 & 1 & YES & YES & YES & $1.57$ & $(2,3)$ & NO & 8940\\
$(374,111)$ & 13 & $(17,5)$ & 6 & 17 & YES & YES & YES & $1.57$ & $(2,3)$ & NO & 8941\\
$(374,155)$ & 13 & $(152,63)$ & 11 & 2 & YES & YES & YES & $1.57$ & $(2,3)$ & 9010 & 8942\\
$(375,139)$ & 14 & $(2,1)$ & 1 & 1 & YES & YES & YES & $1.71$ & $(2,3)$ & NO & 8943\\
$(375,139)$ & 14 & $(8,3)$ & 4 & 1 & YES & YES & YES & $1.71$ & $(2,3)$ & NO & 8944\\
$(377,79)$ & 14 & $(5,1)$ & 4 & 1 & YES & YES & YES & $1.57$ & $(2,3)$ & -- & 8945\\
$(377,79)$ & 14 & $(19,4)$ & 7 & 1 & YES & YES & YES & $1.71$ & $(2,3)$ & NO & 8946\\
$(377,136)$ & 13 & $(36,13)$ & 8 & 1 & YES & YES & YES & $1.57$ & $(2,3)$ & 7660 & 8947\\
$(379,147)$ & 13 & $(4,1)$ & 3 & 1 & YES & YES & YES & $1.71$ & $(2,3)$ & -- & 8948\\
$(379,111)$ & 13 & $(10,3)$ & 5 & 1 & YES & YES & YES & $1.43$ & $(2,3)$ & NO & 8949\\
$(380,83)$ & 14 & $(2,1)$ & 1 & 2 & YES & YES & YES & $1.57$ & $(2,3)$ & -- & 8950\\
$(380,87)$ & 14 & $(3,1)$ & 2 & 1 & YES & YES & YES & $1.43$ & $(2,3)$ & -- & 8951\\
$(380,83)$ & 14 & $(32,7)$ & 8 & 4 & YES & YES & YES & $1.57$ & $(2,3)$ & NO & 8952\\
$(380,87)$ & 14 & $(48,11)$ & 9 & 4 & YES & YES & YES & $1.57$ & $(2,3)$ & NO & 8953\\
$(380,87)$ & 14 & $(297,68)$ & 13 & 1 & YES & YES & YES & $1.57$ & $(2,3)$ & NO & 8954\\
$(380,83)$ & 14 & $(380,83)$ & 14 & 380 & YES & YES & YES & $1.43$ & $(2,3)$ & NO & 8955\\
$(381,113)$ & 14 & $(4,1)$ & 3 & 1 & YES & YES & YES & $1.57$ & $(2,3)$ & NO & 8956\\
$(381,113)$ & 14 & $(7,2)$ & 4 & 1 & YES & YES & YES & $1.71$ & $(2,3)$ & NO & 8957\\
$(382,141)$ & 13 & $(2,1)$ & 1 & 2 & YES & YES & YES & $1.57$ & $(2,3)$ & NO & 8958\\
$(382,89)$ & 14 & $(3,1)$ & 2 & 1 & YES & YES & YES & $1.57$ & $(2,3)$ & NO & 8959\\
$(382,87)$ & 14 & $(5,1)$ & 4 & 1 & YES & YES & YES & $1.57$ & $(2,3)$ & -- & 8960\\
$(382,141)$ & 13 & $(8,3)$ & 4 & 2 & YES & YES & YES & $1.57$ & $(2,3)$ & NO & 8961\\
$(382,89)$ & 14 & $(17,4)$ & 7 & 1 & YES & YES & YES & $1.57$ & $(2,3)$ & NO & 8962\\
$(382,87)$ & 14 & $(382,87)$ & 14 & 382 & YES & YES & YES & $1.43$ & $(2,3)$ & NO & 8963\\
$(383,89)$ & 14 & $(4,1)$ & 3 & 1 & YES & YES & YES & $1.57$ & $(2,3)$ & -- & 8964\\
$(383,89)$ & 14 & $(17,4)$ & 7 & 1 & YES & YES & YES & $1.57$ & $(2,3)$ & NO & 8965\\
$(384,91)$ & 15 & $(2,1)$ & 1 & 2 & YES & YES & YES & $1.57$ & $(2,3)$ & NO & 8966\\
$(385,82)$ & 14 & $(5,1)$ & 4 & 5 & YES & YES & YES & $1.43$ & $(2,3)$ & NO & 8967\\
$(386,87)$ & 15 & $(2,1)$ & 1 & 2 & YES & YES & YES & $1.71$ & $(2,3)$ & NO & 8968\\
$(386,87)$ & 15 & $(4,1)$ & 3 & 2 & YES & YES & YES & $1.57$ & $(2,3)$ & NO & 8969\\
$(388,89)$ & 14 & $(3,1)$ & 2 & 1 & YES & YES & YES & $1.43$ & $(2,3)$ & -- & 8970\\
$(388,89)$ & 14 & $(5,1)$ & 4 & 1 & YES & YES & YES & $1.57$ & $(2,3)$ & NO & 8971\\
$(388,89)$ & 14 & $(109,25)$ & 11 & 1 & YES & YES & YES & $1.57$ & $(2,3)$ & NO & 8972\\
$(389,91)$ & 14 & $(2,1)$ & 1 & 1 & YES & YES & YES & $1.57$ & $(2,3)$ & -- & 8973\\
$(389,91)$ & 14 & $(2,1)$ & 1 & 1 & YES & YES & YES & $1.57$ & $(2,3)$ & NO & 8974\\
$(389,84)$ & 14 & $(51,11)$ & 9 & 1 & YES & YES & YES & $1.71$ & $(2,3)$ & NO & 8975\\
$(389,84)$ & 14 & $(389,84)$ & 14 & 389 & YES & YES & YES & $1.71$ & $(2,3)$ & NO & 8976\\
$(391,75)$ & 16 & $(2,1)$ & 1 & 1 & YES & YES & YES & $1.71$ & $(2,3)$ & NO & 8977\\
$(391,91)$ & 14 & $(4,1)$ & 3 & 1 & YES & YES & YES & $1.43$ & $(2,3)$ & -- & 8978\\
$(391,91)$ & 14 & $(13,3)$ & 6 & 1 & YES & YES & YES & $1.57$ & $(2,3)$ & 7558 & 8979\\
$(391,145)$ & 13 & $(27,10)$ & 7 & 1 & YES & YES & YES & $1.71$ & $(2,3)$ & 8732 & 8980\\
$(391,91)$ & 14 & $(116,27)$ & 11 & 1 & YES & YES & YES & $1.57$ & $(2,3)$ & NO & 8981\\
$(393,106)$ & 13 & $(2,1)$ & 1 & 1 & YES & YES & YES & $1.57$ & $(2,3)$ & NO & 8982\\
$(393,142)$ & 14 & $(2,1)$ & 1 & 1 & YES & YES & YES & $1.71$ & $(2,3)$ & NO & 8983\\
$(393,106)$ & 13 & $(3,1)$ & 2 & 3 & YES & YES & YES & $1.57$ & $(2,3)$ & NO & 8984\\
$(393,142)$ & 14 & $(11,4)$ & 5 & 1 & YES & YES & YES & $1.71$ & $(2,3)$ & 6471 & 8985\\
$(393,142)$ & 14 & $(36,13)$ & 8 & 3 & YES & YES & YES & $1.71$ & $(2,3)$ & NO & 8986\\
$(393,106)$ & 13 & $(63,17)$ & 9 & 3 & YES & YES & YES & $1.57$ & $(2,3)$ & NO & 8987\\
$(397,90)$ & 15 & $(2,1)$ & 1 & 1 & YES & YES & YES & $1.57$ & $(2,3)$ & -- & 8988\\
$(397,87)$ & 15 & $(5,1)$ & 4 & 1 & YES & YES & YES & $1.71$ & $(2,3)$ & NO & 8989\\
$(397,90)$ & 15 & $(9,2)$ & 5 & 1 & YES & YES & YES & $1.57$ & $(2,3)$ & NO & 8990\\
$(397,87)$ & 15 & $(32,7)$ & 8 & 1 & YES & YES & YES & $1.71$ & $(2,3)$ & NO & 8991\\
$(398,93)$ & 14 & $(13,3)$ & 6 & 1 & YES & YES & YES & $1.57$ & $(2,3)$ & NO & 8992\\
$(398,93)$ & 14 & $(398,93)$ & 14 & 398 & YES & YES & YES & $1.43$ & $(2,3)$ & NO & 8993\\
$(399,86)$ & 14 & $(5,1)$ & 4 & 1 & YES & YES & YES & $1.57$ & $(2,3)$ & NO & 8994\\
$(401,87)$ & 14 & $(4,1)$ & 3 & 1 & YES & YES & YES & $1.57$ & $(2,3)$ & NO & 8995\\
$(403,111)$ & 13 & $(2,1)$ & 1 & 1 & YES & YES & YES & $1.43$ & $(2,3)$ & -- & 8996\\
$(403,87)$ & 14 & $(3,1)$ & 2 & 1 & YES & YES & YES & $1.57$ & $(2,3)$ & -- & 8997\\
$(403,87)$ & 14 & $(51,11)$ & 9 & 1 & YES & YES & YES & $1.71$ & $(2,3)$ & NO & 8998\\
$(404,91)$ & 14 & $(2,1)$ & 1 & 2 & YES & YES & YES & $1.57$ & $(2,3)$ & NO & 8999\\
$(404,91)$ & 14 & $(5,1)$ & 4 & 1 & YES & YES & YES & $1.57$ & $(2,3)$ & NO & 9000\\
$(406,93)$ & 14 & $(5,1)$ & 4 & 1 & YES & YES & YES & $1.43$ & $(2,3)$ & NO & 9001\\
$(407,92)$ & 14 & $(4,1)$ & 3 & 1 & YES & YES & YES & $1.43$ & $(2,3)$ & NO & 9002\\
$(408,169)$ & 13 & $(3,1)$ & 2 & 3 & YES & YES & YES & $1.57$ & $(2,3)$ & NO & 9003\\
$(408,169)$ & 13 & $(99,41)$ & 10 & 3 & YES & YES & YES & $1.57$ & $(2,3)$ & NO & 9004\\
$(409,121)$ & 13 & $(2,1)$ & 1 & 1 & YES & YES & YES & $1.71$ & $(2,3)$ & NO & 9005\\
$(409,121)$ & 13 & $(7,2)$ & 4 & 1 & YES & YES & YES & $1.43$ & $(2,3)$ & NO & 9006\\
$(410,113)$ & 14 & $(3,1)$ & 2 & 1 & YES & YES & YES & $1.57$ & $(2,3)$ & -- & 9007\\
$(410,113)$ & 14 & $(3,1)$ & 2 & 1 & YES & YES & YES & $1.71$ & $(2,3)$ & NO & 9008\\
$(415,172)$ & 13 & $(29,12)$ & 7 & 1 & YES & YES & YES & $1.57$ & $(2,3)$ & NO & 9009\\
$(415,172)$ & 13 & $(111,46)$ & 10 & 1 & YES & YES & YES & $1.57$ & $(2,3)$ & 8942 & 9010\\
$(419,89)$ & 14 & $(2,1)$ & 1 & 1 & YES & YES & YES & $1.57$ & $(2,3)$ & -- & 9011\\
$(419,98)$ & 14 & $(3,1)$ & 2 & 1 & YES & YES & YES & $1.43$ & $(2,3)$ & -- & 9012\\
$(419,116)$ & 13 & $(7,2)$ & 4 & 1 & YES & YES & YES & $1.57$ & $(2,3)$ & NO & 9013\\
$(419,89)$ & 14 & $(14,3)$ & 6 & 1 & YES & YES & YES & $1.57$ & $(2,3)$ & NO & 9014\\
$(419,89)$ & 14 & $(19,4)$ & 7 & 1 & YES & YES & YES & $1.43$ & $(2,3)$ & NO & 9015\\
$(419,89)$ & 14 & $(113,24)$ & 11 & 1 & YES & YES & YES & $1.43$ & $(2,3)$ & NO & 9016\\
$(421,89)$ & 14 & $(19,4)$ & 7 & 1 & YES & YES & YES & $1.57$ & $(2,3)$ & NO & 9017\\
$(425,99)$ & 14 & $(2,1)$ & 1 & 1 & YES & YES & YES & $1.57$ & $(2,3)$ & NO & 9018\\
$(425,99)$ & 14 & $(3,1)$ & 2 & 1 & YES & YES & YES & $1.57$ & $(2,3)$ & NO & 9019\\
$(425,99)$ & 14 & $(17,4)$ & 7 & 17 & YES & YES & YES & $1.57$ & $(2,3)$ & NO & 9020\\
$(425,99)$ & 14 & $(30,7)$ & 8 & 5 & YES & YES & YES & $1.57$ & $(2,3)$ & 8036 & 9021\\
$(425,99)$ & 14 & $(43,10)$ & 9 & 1 & YES & YES & YES & $1.57$ & $(2,3)$ & NO & 9022\\
$(429,97)$ & 14 & $(115,26)$ & 11 & 1 & YES & YES & YES & $1.43$ & $(2,3)$ & NO & 9023\\
$(431,120)$ & 14 & $(3,1)$ & 2 & 1 & YES & YES & YES & $1.57$ & $(2,3)$ & NO & 9024\\
$(433,80)$ & 15 & $(3,1)$ & 2 & 1 & YES & YES & YES & $1.57$ & $(2,3)$ & NO & 9025\\
$(433,80)$ & 15 & $(3,1)$ & 2 & 1 & YES & YES & YES & $1.57$ & $(2,3)$ & -- & 9026\\
$(433,80)$ & 15 & $(38,7)$ & 9 & 1 & YES & YES & YES & $1.57$ & $(2,3)$ & NO & 9027\\
$(434,101)$ & 14 & $(3,1)$ & 2 & 1 & YES & YES & YES & $1.43$ & $(2,3)$ & NO & 9028\\
$(434,101)$ & 14 & $(3,1)$ & 2 & 1 & YES & YES & YES & $1.43$ & $(2,3)$ & -- & 9029\\
$(434,101)$ & 14 & $(17,4)$ & 7 & 1 & YES & YES & YES & $1.57$ & $(2,3)$ & NO & 9030\\
$(436,103)$ & 15 & $(55,13)$ & 10 & 1 & YES & YES & YES & $1.57$ & $(2,3)$ & NO & 9031\\
$(436,103)$ & 15 & $(127,30)$ & 12 & 1 & YES & YES & YES & $1.57$ & $(2,3)$ & NO & 9032\\
$(439,99)$ & 15 & $(2,1)$ & 1 & 1 & YES & YES & YES & $1.57$ & $(2,3)$ & -- & 9033\\
$(439,93)$ & 14 & $(4,1)$ & 3 & 1 & YES & YES & YES & $1.57$ & $(2,3)$ & -- & 9034\\
$(439,93)$ & 14 & $(14,3)$ & 6 & 1 & YES & YES & YES & $1.43$ & $(2,3)$ & NO & 9035\\
$(439,99)$ & 15 & $(102,23)$ & 11 & 1 & YES & YES & YES & $1.57$ & $(2,3)$ & NO & 9036\\
$(445,104)$ & 14 & $(2,1)$ & 1 & 1 & YES & YES & YES & $1.57$ & $(2,3)$ & -- & 9037\\
$(445,104)$ & 14 & $(13,3)$ & 6 & 1 & YES & YES & YES & $1.57$ & $(2,3)$ & NO & 9038\\
$(447,80)$ & 15 & $(6,1)$ & 5 & 3 & YES & YES & YES & $1.43$ & $(2,3)$ & NO & 9039\\
$(451,105)$ & 14 & $(43,10)$ & 9 & 1 & YES & YES & YES & $1.57$ & $(2,3)$ & 8199 & 9040\\
$(452,99)$ & 15 & $(2,1)$ & 1 & 2 & YES & YES & YES & $1.57$ & $(2,3)$ & -- & 9041\\
$(452,99)$ & 15 & $(105,23)$ & 11 & 1 & YES & YES & YES & $1.71$ & $(2,3)$ & NO & 9042\\
$(453,104)$ & 15 & $(13,3)$ & 6 & 1 & YES & YES & YES & $1.57$ & $(2,3)$ & NO & 9043\\
$(458,85)$ & 15 & $(16,3)$ & 7 & 2 & YES & YES & YES & $1.57$ & $(2,3)$ & NO & 9044\\
$(461,82)$ & 15 & $(2,1)$ & 1 & 1 & YES & YES & YES & $1.71$ & $(2,3)$ & NO & 9045\\
$(461,109)$ & 15 & $(2,1)$ & 1 & 1 & YES & YES & YES & $1.57$ & $(2,3)$ & NO & 9046\\
$(461,105)$ & 14 & $(9,2)$ & 5 & 1 & YES & YES & YES & $1.43$ & $(2,3)$ & NO & 9047\\
$(467,87)$ & 15 & $(2,1)$ & 1 & 1 & YES & YES & YES & $1.71$ & $(2,3)$ & -- & 9048\\
$(469,109)$ & 15 & $(4,1)$ & 3 & 1 & YES & YES & YES & $1.57$ & $(2,3)$ & NO & 9049\\
$(469,109)$ & 15 & $(13,3)$ & 6 & 1 & YES & YES & YES & $1.57$ & $(2,3)$ & 7044 & 9050\\
$(469,109)$ & 15 & $(142,33)$ & 12 & 1 & YES & YES & YES & $1.57$ & $(2,3)$ & NO & 9051\\
$(475,111)$ & 14 & $(77,18)$ & 10 & 1 & YES & YES & YES & $1.57$ & $(2,3)$ & NO & 9052\\
$(476,103)$ & 14 & $(2,1)$ & 1 & 2 & YES & YES & YES & $1.43$ & $(2,3)$ & NO & 9053\\
$(477,88)$ & 15 & $(2,1)$ & 1 & 1 & YES & YES & YES & $1.57$ & $(2,3)$ & -- & 9054\\
$(477,103)$ & 15 & $(9,2)$ & 5 & 9 & YES & YES & YES & $1.57$ & $(2,3)$ & NO & 9055\\
$(478,111)$ & 15 & $(5,1)$ & 4 & 1 & YES & YES & YES & $1.71$ & $(2,3)$ & NO & 9056\\
$(478,111)$ & 15 & $(43,10)$ & 9 & 1 & YES & YES & YES & $1.71$ & $(2,3)$ & NO & 9057\\
$(478,111)$ & 15 & $(211,49)$ & 13 & 1 & YES & YES & YES & $1.57$ & $(2,3)$ & NO & 9058\\
$(482,109)$ & 14 & $(3,1)$ & 2 & 1 & YES & YES & YES & $1.43$ & $(2,3)$ & -- & 9059\\
$(489,112)$ & 14 & $(5,1)$ & 4 & 1 & YES & YES & YES & $1.43$ & $(2,3)$ & NO & 9060\\
$(491,88)$ & 15 & $(2,1)$ & 1 & 1 & YES & YES & YES & $1.43$ & $(2,3)$ & NO & 9061\\
$(491,103)$ & 15 & $(5,1)$ & 4 & 1 & YES & YES & YES & $1.57$ & $(2,3)$ & -- & 9062\\
$(491,88)$ & 15 & $(11,2)$ & 6 & 1 & YES & YES & YES & $1.43$ & $(2,3)$ & NO & 9063\\
$(491,103)$ & 15 & $(19,4)$ & 7 & 1 & YES & YES & YES & $1.71$ & $(2,3)$ & NO & 9064\\
$(491,116)$ & 15 & $(55,13)$ & 10 & 1 & YES & YES & YES & $1.71$ & $(2,3)$ & NO & 9065\\
$(494,113)$ & 15 & $(3,1)$ & 2 & 1 & YES & YES & YES & $1.57$ & $(2,3)$ & -- & 9066\\
$(494,113)$ & 15 & $(5,1)$ & 4 & 1 & YES & YES & YES & $1.71$ & $(2,3)$ & NO & 9067\\
$(503,90)$ & 16 & $(11,2)$ & 6 & 1 & YES & YES & YES & $1.71$ & $(2,3)$ & NO & 9068\\
$(507,97)$ & 16 & $(21,4)$ & 8 & 3 & YES & YES & YES & $1.57$ & $(2,3)$ & NO & 9069\\
$(512,119)$ & 15 & $(4,1)$ & 3 & 4 & YES & YES & YES & $1.71$ & $(2,3)$ & NO & 9070\\
$(522,119)$ & 14 & $(22,5)$ & 7 & 2 & YES & YES & YES & $1.43$ & $(2,3)$ & NO & 9071\\
$(526,119)$ & 14 & $(3,1)$ & 2 & 1 & YES & YES & YES & $1.43$ & $(2,3)$ & -- & 9072\\
$(545,103)$ & 16 & $(2,1)$ & 1 & 1 & YES & YES & YES & $1.71$ & $(2,3)$ & -- & 9073\\
$(545,103)$ & 16 & $(2,1)$ & 1 & 1 & YES & YES & YES & $1.71$ & $(2,3)$ & NO & 9074\\
$(545,103)$ & 16 & $(16,3)$ & 7 & 1 & YES & YES & YES & $1.57$ & $(2,3)$ & NO & 9075\\
$(551,120)$ & 15 & $(4,1)$ & 3 & 1 & YES & YES & YES & $1.57$ & $(2,3)$ & NO & 9076\\
$(579,137)$ & 15 & $(93,22)$ & 11 & 3 & YES & YES & YES & $1.57$ & $(2,3)$ & NO & 9077\\
$(a;0,0,0;3)$ & 4 & $(39,14)$ & 8 & 3 & YES & YES & YES & $1.43$ & $(2,3)$ & -- & 9078\\
$(a;0,0,0;3)$ & 4 & $(55,21)$ & 8 & 1 & YES & YES & YES & $1.57$ & $(2,3)$ & -- & 9079\\
$(a;0,0,0;3)$ & 4 & $(63,17)$ & 9 & 3 & YES & YES & NO(3) & $1.29$ & $(2,3)$ & -- & 9080\\
$(a;0,0,0;3)$ & 4 & $(64,15)$ & 10 & 1 & YES & YES & NO(3) & $1.29$ & $(2,3)$ & -- & 9081\\
$(a;0,0,0;3)$ & 4 & $(79,18)$ & 10 & 1 & YES & YES & YES & $1.43$ & $(2,3)$ & -- & 9082\\
$(a;0,0,0;3)$ & 4 & $(95,39)$ & 10 & 1 & YES & YES & YES & $1.71$ & $(2,3)$ & -- & 9083\\
$(a;1,0,0;13)$ & 5 & $(19,8)$ & 6 & 1 & YES & YES & NO(2) & $1.38$ & $(2,3)$ & -- & 9084\\
$(a;1,0,0;13)$ & 5 & $(25,7)$ & 7 & 1 & YES & YES & YES & $1.57$ & $(2,3)$ & -- & 9085\\
$(a;1,0,0;13)$ & 5 & $(26,11)$ & 7 & 13 & YES & YES & YES & $1.57$ & $(2,3)$ & -- & 9086\\
$(a;1,0,0;13)$ & 5 & $(27,8)$ & 7 & 1 & YES & YES & YES & $1.43$ & $(2,3)$ & -- & 9087\\
$(a;1,0,0;13)$ & 5 & $(29,8)$ & 7 & 1 & YES & YES & YES & $1.43$ & $(2,3)$ & -- & 9088\\
$(a;1,0,0;13)$ & 5 & $(29,11)$ & 7 & 1 & YES & YES & YES & $1.57$ & $(2,3)$ & -- & 9089\\
$(a;1,0,0;13)$ & 5 & $(29,12)$ & 7 & 1 & YES & YES & YES & $1.43$ & $(2,3)$ & -- & 9090\\
$(a;1,0,0;13)$ & 5 & $(30,11)$ & 7 & 1 & YES & YES & YES & $1.29$ & $(2,3)$ & -- & 9091\\
$(a;1,0,0;13)$ & 5 & $(31,13)$ & 7 & 1 & YES & YES & YES & $1.43$ & $(2,3)$ & -- & 9092\\
$(a;1,0,0;13)$ & 5 & $(37,10)$ & 8 & 1 & YES & YES & YES & $1.71$ & $(2,3)$ & -- & 9093\\
$(a;1,0,0;13)$ & 5 & $(39,11)$ & 9 & 13 & YES & YES & YES & $1.43$ & $(2,3)$ & -- & 9094\\
$(a;1,0,0;13)$ & 5 & $(39,16)$ & 8 & 13 & YES & YES & NO(2) & $1.50$ & $(2,3)$ & -- & 9095\\
$(a;1,0,0;13)$ & 5 & $(41,15)$ & 8 & 1 & YES & YES & YES & $1.71$ & $(2,3)$ & -- & 9096\\
$(a;1,0,0;13)$ & 5 & $(41,16)$ & 8 & 1 & YES & YES & NO(2) & $1.50$ & $(2,3)$ & -- & 9097\\
$(a;1,0,0;13)$ & 5 & $(43,13)$ & 9 & 1 & YES & YES & YES & $1.43$ & $(2,3)$ & -- & 9098\\
$(a;1,0,0;13)$ & 5 & $(43,18)$ & 8 & 1 & YES & YES & YES & $1.43$ & $(2,3)$ & -- & 9099\\
$(a;1,0,0;13)$ & 5 & $(45,14)$ & 9 & 1 & YES & YES & YES & $1.57$ & $(2,3)$ & -- & 9100\\
$(a;1,0,0;13)$ & 5 & $(46,19)$ & 8 & 1 & YES & YES & YES & $1.43$ & $(2,3)$ & -- & 9101\\
$(a;1,0,0;13)$ & 5 & $(47,14)$ & 9 & 1 & YES & YES & YES & $1.43$ & $(2,3)$ & -- & 9102\\
$(a;1,0,0;13)$ & 5 & $(49,13)$ & 9 & 1 & YES & YES & YES & $1.57$ & $(2,3)$ & -- & 9103\\
$(a;1,0,0;13)$ & 5 & $(49,15)$ & 9 & 1 & YES & YES & YES & $1.43$ & $(2,3)$ & -- & 9104\\
$(a;1,0,0;13)$ & 5 & $(50,19)$ & 8 & 1 & YES & YES & YES & $1.43$ & $(2,3)$ & -- & 9105\\
$(a;1,0,0;13)$ & 5 & $(50,21)$ & 8 & 1 & YES & YES & YES & $1.57$ & $(2,3)$ & -- & 9106\\
$(a;1,0,0;13)$ & 5 & $(51,14)$ & 9 & 1 & YES & YES & YES & $1.71$ & $(2,3)$ & -- & 9107\\
$(a;1,0,0;13)$ & 5 & $(55,16)$ & 9 & 1 & YES & YES & YES & $1.71$ & $(2,3)$ & -- & 9108\\
$(a;1,0,0;13)$ & 5 & $(55,21)$ & 8 & 1 & YES & YES & YES & $1.43$ & $(2,3)$ & -- & 9109\\
$(a;1,0,0;13)$ & 5 & $(56,23)$ & 9 & 1 & YES & YES & YES & $1.71$ & $(2,3)$ & -- & 9110\\
$(a;1,0,0;13)$ & 5 & $(61,18)$ & 9 & 1 & YES & YES & YES & $1.43$ & $(2,3)$ & -- & 9111\\
$(a;1,0,0;13)$ & 5 & $(64,27)$ & 9 & 1 & YES & YES & YES & $1.71$ & $(2,3)$ & -- & 9112\\
$(a;1,0,0;13)$ & 5 & $(70,29)$ & 9 & 1 & YES & YES & YES & $1.57$ & $(2,3)$ & -- & 9113\\
$(a;1,0,0;13)$ & 5 & $(75,29)$ & 9 & 1 & YES & YES & YES & $1.57$ & $(2,3)$ & -- & 9114\\
$(a;1,1,0;19)$ & 6 & $(17,5)$ & 6 & 1 & YES & YES & YES & $1.43$ & $(2,3)$ & -- & 9115\\
$(a;1,1,0;19)$ & 6 & $(18,5)$ & 6 & 1 & YES & YES & YES & $1.43$ & $(2,3)$ & -- & 9116\\
$(a;1,1,0;19)$ & 6 & $(18,7)$ & 6 & 1 & YES & YES & YES & $1.71$ & $(2,3)$ & -- & 9117\\
$(a;1,1,0;19)$ & 6 & $(19,8)$ & 6 & 19 & YES & YES & YES & $1.43$ & $(2,3)$ & -- & 9118\\
$(a;1,1,0;19)$ & 6 & $(21,8)$ & 6 & 1 & YES & YES & YES & $1.43$ & $(2,3)$ & -- & 9119\\
$(a;1,1,0;19)$ & 6 & $(26,11)$ & 7 & 1 & YES & YES & YES & $1.43$ & $(2,3)$ & -- & 9120\\
$(a;1,1,0;19)$ & 6 & $(29,11)$ & 7 & 1 & YES & YES & YES & $1.43$ & $(2,3)$ & -- & 9121\\
$(a;1,1,0;19)$ & 6 & $(29,12)$ & 7 & 1 & YES & YES & YES & $1.43$ & $(2,3)$ & -- & 9122\\
$(a;1,1,0;19)$ & 6 & $(30,11)$ & 7 & 1 & YES & YES & YES & $1.43$ & $(2,3)$ & -- & 9123\\
$(a;1,1,0;19)$ & 6 & $(31,9)$ & 8 & 1 & YES & YES & YES & $1.43$ & $(2,3)$ & -- & 9124\\
$(a;1,1,0;19)$ & 6 & $(31,13)$ & 7 & 1 & YES & YES & YES & $1.57$ & $(2,3)$ & -- & 9125\\
$(a;1,1,0;19)$ & 6 & $(34,13)$ & 7 & 1 & YES & YES & YES & $1.57$ & $(2,3)$ & -- & 9126\\
$(a;1,1,0;19)$ & 6 & $(41,12)$ & 8 & 1 & YES & YES & YES & $1.57$ & $(2,3)$ & -- & 9127\\
$(a;1,1,0;19)$ & 6 & $(41,17)$ & 8 & 1 & YES & YES & YES & $1.71$ & $(2,3)$ & -- & 9128\\
$(a;1,1,0;19)$ & 6 & $(44,17)$ & 8 & 1 & YES & YES & YES & $1.57$ & $(2,3)$ & -- & 9129\\
$(a;1,1,1;4)$ & 7 & $(19,8)$ & 6 & 1 & YES & YES & YES & $1.57$ & $(2,3)$ & -- & 9130\\
$(a;2,0,0;17)$ & 6 & $(19,8)$ & 6 & 1 & YES & YES & NO(2) & $1.50$ & $(2,3)$ & -- & 9131\\
$(a;2,0,0;17)$ & 6 & $(26,7)$ & 7 & 1 & YES & YES & YES & $1.57$ & $(2,3)$ & -- & 9132\\
$(a;2,0,0;17)$ & 6 & $(28,11)$ & 8 & 1 & YES & YES & YES & $1.57$ & $(2,3)$ & -- & 9133\\
$(a;2,0,0;17)$ & 6 & $(39,16)$ & 8 & 1 & YES & YES & YES & $1.57$ & $(2,3)$ & -- & 9134\\
$(a;2,0,0;17)$ & 6 & $(43,13)$ & 9 & 1 & YES & YES & YES & $1.57$ & $(2,3)$ & -- & 9135\\
$(a;2,0,1;25)$ & 7 & $(13,5)$ & 5 & 1 & YES & YES & YES & $1.43$ & $(2,3)$ & -- & 9136\\
$(a;2,0,1;25)$ & 7 & $(15,4)$ & 6 & 5 & YES & YES & YES & $1.57$ & $(2,3)$ & -- & 9137\\
$(a;2,0,1;25)$ & 7 & $(16,7)$ & 6 & 1 & YES & YES & NO(2) & $1.62$ & $(2,3)$ & -- & 9138\\
$(a;2,0,1;25)$ & 7 & $(17,5)$ & 6 & 1 & YES & YES & YES & $1.57$ & $(2,3)$ & -- & 9139\\
$(a;2,0,1;25)$ & 7 & $(17,7)$ & 6 & 1 & YES & YES & NO(2) & $1.62$ & $(2,3)$ & -- & 9140\\
$(a;2,0,1;25)$ & 7 & $(18,7)$ & 6 & 1 & YES & YES & YES & $1.57$ & $(2,3)$ & -- & 9141\\
$(a;2,0,1;25)$ & 7 & $(19,7)$ & 6 & 1 & YES & YES & NO(2) & $1.62$ & $(2,3)$ & -- & 9142\\
$(a;2,0,1;25)$ & 7 & $(19,8)$ & 6 & 1 & YES & YES & YES & $1.43$ & $(2,3)$ & -- & 9143\\
$(a;2,0,1;25)$ & 7 & $(21,8)$ & 6 & 1 & YES & YES & YES & $1.43$ & $(2,3)$ & -- & 9144\\
$(a;2,0,1;25)$ & 7 & $(22,9)$ & 7 & 1 & YES & YES & YES & $1.57$ & $(2,3)$ & -- & 9145\\
$(a;2,0,1;25)$ & 7 & $(23,7)$ & 7 & 1 & YES & YES & NO(2) & $1.75$ & $(2,3)$ & -- & 9146\\
$(a;2,0,1;25)$ & 7 & $(25,9)$ & 7 & 25 & YES & YES & YES & $1.57$ & $(2,3)$ & -- & 9147\\
$(a;2,0,1;25)$ & 7 & $(26,7)$ & 7 & 1 & YES & YES & NO(2) & $1.50$ & $(2,3)$ & -- & 9148\\
$(a;2,0,1;25)$ & 7 & $(27,8)$ & 7 & 1 & YES & YES & YES & $1.71$ & $(2,3)$ & -- & 9149\\
$(a;2,0,1;25)$ & 7 & $(35,8)$ & 8 & 5 & YES & YES & YES & $1.43$ & $(2,3)$ & -- & 9150\\
$(a;2,0,1;25)$ & 7 & $(47,11)$ & 9 & 1 & YES & YES & YES & $1.57$ & $(2,3)$ & -- & 9151\\
$(a;2,1,0;5)$ & 7 & $(13,5)$ & 5 & 1 & YES & YES & YES & $1.43$ & $(2,3)$ & -- & 9152\\
$(a;2,1,0;5)$ & 7 & $(19,8)$ & 6 & 1 & YES & YES & YES & $1.43$ & $(2,3)$ & -- & 9153\\
$(a;2,1,0;5)$ & 7 & $(21,8)$ & 6 & 1 & YES & YES & NO(2) & $1.75$ & $(2,3)$ & -- & 9154\\
$(a;2,1,0;5)$ & 7 & $(23,7)$ & 7 & 1 & YES & YES & NO(2) & $1.62$ & $(2,3)$ & -- & 9155\\
$(a;2,1,0;5)$ & 7 & $(27,8)$ & 7 & 1 & YES & YES & YES & $1.43$ & $(2,3)$ & -- & 9156\\
$(a;2,1,1;37)$ & 8 & $(7,2)$ & 4 & 1 & YES & YES & NO(2) & $1.50$ & $(2,3)$ & -- & 9157\\
$(a;2,1,1;37)$ & 8 & $(12,5)$ & 5 & 1 & YES & YES & YES & $1.43$ & $(2,3)$ & -- & 9158\\
$(a;2,1,1;37)$ & 8 & $(13,5)$ & 5 & 1 & YES & YES & YES & $1.43$ & $(2,3)$ & -- & 9159\\
$(a;2,1,1;37)$ & 8 & $(16,7)$ & 6 & 1 & YES & YES & YES & $1.71$ & $(2,3)$ & -- & 9160\\
$(a;2,1,1;37)$ & 8 & $(17,5)$ & 6 & 1 & YES & YES & YES & $1.43$ & $(2,3)$ & -- & 9161\\
$(a;2,2,0;33)$ & 8 & $(8,3)$ & 4 & 1 & YES & YES & NO(2) & $1.67$ & $(2,3)$ & -- & 9162\\
$(a;2,2,0;33)$ & 8 & $(12,5)$ & 5 & 3 & YES & YES & NO(2) & $1.50$ & $(2,3)$ & -- & 9163\\
$(a;2,2,0;33)$ & 8 & $(19,7)$ & 6 & 1 & YES & YES & YES & $1.71$ & $(2,3)$ & -- & 9164\\
$(a;2,2,0;33)$ & 8 & $(23,7)$ & 7 & 1 & YES & YES & YES & $1.57$ & $(2,3)$ & -- & 9165\\
$(a;2,2,1;49)$ & 9 & $(7,3)$ & 4 & 7 & YES & YES & NO(2) & $1.50$ & $(2,3)$ & -- & 9166\\
$(a;2,2,2;5)$ & 10 & $(7,3)$ & 4 & 1 & YES & YES & YES & $1.57$ & $(2,3)$ & -- & 9167\\
$(a;3,0,0;7)$ & 7 & $(27,8)$ & 7 & 1 & YES & YES & YES & $1.43$ & $(2,3)$ & -- & 9168\\
$(a;3,0,1;31)$ & 8 & $(15,4)$ & 6 & 1 & YES & YES & NO(2) & $1.62$ & $(2,3)$ & -- & 9169\\
$(a;3,0,1;31)$ & 8 & $(17,5)$ & 6 & 1 & YES & YES & YES & $1.43$ & $(2,3)$ & -- & 9170\\
$(a;3,0,1;31)$ & 8 & $(18,5)$ & 6 & 1 & YES & YES & YES & $1.43$ & $(2,3)$ & -- & 9171\\
$(a;3,1,0;31)$ & 8 & $(17,5)$ & 6 & 1 & YES & YES & YES & $1.57$ & $(2,3)$ & -- & 9172\\
$(a;3,1,0;31)$ & 8 & $(18,5)$ & 6 & 1 & YES & YES & YES & $1.43$ & $(2,3)$ & -- & 9173\\
$(a;3,2,1;61)$ & 10 & $(7,2)$ & 4 & 1 & YES & YES & YES & $1.57$ & $(2,3)$ & -- & 9174\\
$(a;4,1,1;55)$ & 10 & $(7,2)$ & 4 & 1 & YES & YES & YES & $1.57$ & $(2,3)$ & -- & 9175\\
$(b;0,0,0;14)$ & 5 & $(14,3)$ & 6 & 14 & YES & YES & YES & $1.43$ & $(2,3)$ & -- & 9176\\
$(b;0,0,0;14)$ & 5 & $(17,5)$ & 6 & 1 & YES & YES & YES & $1.43$ & $(2,3)$ & -- & 9177\\
$(b;0,0,0;14)$ & 5 & $(21,8)$ & 6 & 7 & YES & YES & YES & $1.43$ & $(2,3)$ & -- & 9178\\
$(b;0,0,0;14)$ & 5 & $(29,11)$ & 7 & 1 & YES & YES & YES & $1.57$ & $(2,3)$ & -- & 9179\\
$(b;0,0,0;14)$ & 5 & $(29,12)$ & 7 & 1 & YES & YES & NO(2) & $1.62$ & $(2,3)$ & -- & 9180\\
$(b;0,0,0;14)$ & 5 & $(30,11)$ & 7 & 2 & YES & YES & YES & $1.29$ & $(2,3)$ & -- & 9181\\
$(b;0,0,0;14)$ & 5 & $(31,13)$ & 7 & 1 & YES & YES & YES & $1.57$ & $(2,3)$ & -- & 9182\\
$(b;0,0,0;14)$ & 5 & $(33,10)$ & 8 & 1 & YES & YES & YES & $1.43$ & $(2,3)$ & -- & 9183\\
$(b;0,0,0;14)$ & 5 & $(34,13)$ & 7 & 2 & YES & YES & YES & $1.57$ & $(2,3)$ & -- & 9184\\
$(b;0,0,0;14)$ & 5 & $(40,9)$ & 9 & 2 & YES & YES & YES & $1.29$ & $(2,3)$ & -- & 9185\\
$(b;0,0,0;14)$ & 5 & $(41,17)$ & 8 & 1 & YES & YES & YES & $1.57$ & $(2,3)$ & -- & 9186\\
$(b;0,0,0;14)$ & 5 & $(44,17)$ & 8 & 2 & YES & YES & YES & $1.57$ & $(2,3)$ & -- & 9187\\
$(b;0,0,0;14)$ & 5 & $(46,19)$ & 8 & 2 & YES & YES & YES & $1.43$ & $(2,3)$ & -- & 9188\\
$(b;0,0,0;14)$ & 5 & $(47,18)$ & 8 & 1 & YES & YES & YES & $1.57$ & $(2,3)$ & -- & 9189\\
$(b;0,0,0;14)$ & 5 & $(49,19)$ & 8 & 7 & YES & YES & YES & $1.71$ & $(2,3)$ & -- & 9190\\
$(b;0,0,0;14)$ & 5 & $(50,19)$ & 8 & 2 & YES & YES & YES & $1.43$ & $(2,3)$ & -- & 9191\\
$(b;0,0,0;14)$ & 5 & $(56,17)$ & 9 & 14 & YES & YES & YES & $1.57$ & $(2,3)$ & -- & 9192\\
$(b;0,0,1;4)$ & 6 & $(13,5)$ & 5 & 1 & YES & YES & NO(2) & $1.50$ & $(2,3)$ & -- & 9193\\
$(b;0,0,1;4)$ & 6 & $(19,8)$ & 6 & 1 & YES & YES & YES & $1.57$ & $(2,3)$ & -- & 9194\\
$(b;0,0,1;4)$ & 6 & $(22,9)$ & 7 & 2 & YES & YES & YES & $1.57$ & $(2,3)$ & -- & 9195\\
$(b;0,0,1;4)$ & 6 & $(29,11)$ & 7 & 1 & YES & YES & YES & $1.43$ & $(2,3)$ & -- & 9196\\
$(b;0,0,1;4)$ & 6 & $(29,12)$ & 7 & 1 & YES & YES & YES & $1.57$ & $(2,3)$ & -- & 9197\\
$(b;0,0,1;4)$ & 6 & $(30,11)$ & 7 & 2 & YES & YES & YES & $1.43$ & $(2,3)$ & -- & 9198\\
$(b;0,0,1;4)$ & 6 & $(31,9)$ & 8 & 1 & YES & YES & YES & $1.57$ & $(2,3)$ & -- & 9199\\
$(b;0,0,2;26)$ & 7 & $(12,5)$ & 5 & 2 & YES & YES & YES & $1.29$ & $(2,3)$ & -- & 9200\\
$(b;0,0,2;26)$ & 7 & $(13,5)$ & 5 & 13 & YES & YES & YES & $1.43$ & $(2,3)$ & -- & 9201\\
$(b;0,0,2;26)$ & 7 & $(17,5)$ & 6 & 1 & YES & YES & YES & $1.43$ & $(2,3)$ & -- & 9202\\
$(b;0,0,2;26)$ & 7 & $(24,7)$ & 7 & 2 & YES & YES & YES & $1.57$ & $(2,3)$ & -- & 9203\\
$(b;0,0,2;26)$ & 7 & $(27,8)$ & 7 & 1 & YES & YES & YES & $1.57$ & $(2,3)$ & -- & 9204\\
$(b;0,1,0;19)$ & 6 & $(15,4)$ & 6 & 1 & YES & YES & YES & $1.57$ & $(2,3)$ & -- & 9205\\
$(b;0,1,0;19)$ & 6 & $(22,9)$ & 7 & 1 & YES & YES & YES & $1.71$ & $(2,3)$ & -- & 9206\\
$(b;0,1,0;19)$ & 6 & $(25,11)$ & 7 & 1 & YES & YES & YES & $1.43$ & $(2,3)$ & -- & 9207\\
$(b;0,1,0;19)$ & 6 & $(29,11)$ & 7 & 1 & YES & YES & YES & $1.71$ & $(2,3)$ & -- & 9208\\
$(b;0,1,0;19)$ & 6 & $(47,13)$ & 8 & 1 & YES & YES & YES & $1.57$ & $(2,3)$ & -- & 9209\\
$(b;0,1,1;27)$ & 7 & $(13,5)$ & 5 & 1 & YES & YES & YES & $1.57$ & $(2,3)$ & -- & 9210\\
$(b;0,1,1;27)$ & 7 & $(17,7)$ & 6 & 1 & YES & YES & YES & $1.71$ & $(2,3)$ & -- & 9211\\
$(b;0,1,1;27)$ & 7 & $(27,8)$ & 7 & 27 & YES & YES & YES & $1.43$ & $(2,3)$ & -- & 9212\\
$(b;0,1,2;5)$ & 8 & $(8,3)$ & 4 & 1 & YES & YES & NO(3) & $1.29$ & $(2,3)$ & -- & 9213\\
$(b;0,1,2;5)$ & 8 & $(13,4)$ & 6 & 1 & YES & YES & YES & $1.57$ & $(2,3)$ & -- & 9214\\
$(b;0,2,0;8)$ & 7 & $(13,4)$ & 6 & 1 & YES & YES & YES & $1.71$ & $(2,3)$ & -- & 9215\\
$(b;0,2,0;8)$ & 7 & $(22,9)$ & 7 & 2 & YES & YES & YES & $1.57$ & $(2,3)$ & -- & 9216\\
$(b;0,2,0;8)$ & 7 & $(25,9)$ & 7 & 1 & YES & YES & YES & $1.71$ & $(2,3)$ & -- & 9217\\
$(b;0,2,1;34)$ & 8 & $(17,5)$ & 6 & 17 & YES & YES & YES & $1.71$ & $(2,3)$ & -- & 9218\\
$(b;0,2,1;34)$ & 8 & $(18,5)$ & 6 & 2 & YES & YES & YES & $1.57$ & $(2,3)$ & -- & 9219\\
$(b;0,2,1;34)$ & 8 & $(22,5)$ & 7 & 2 & YES & YES & YES & $1.57$ & $(2,3)$ & -- & 9220\\
$(b;0,2,2;44)$ & 9 & $(7,2)$ & 4 & 1 & YES & YES & YES & $1.29$ & $(2,3)$ & -- & 9221\\
$(b;0,2,2;44)$ & 9 & $(7,3)$ & 4 & 1 & YES & YES & NO(2) & $1.62$ & $(2,3)$ & -- & 9222\\
$(b;0,2,2;44)$ & 9 & $(10,3)$ & 5 & 2 & YES & YES & YES & $1.57$ & $(2,3)$ & -- & 9223\\
$(b;0,2,3;6)$ & 10 & $(5,2)$ & 3 & 1 & YES & YES & YES & $1.43$ & $(2,3)$ & -- & 9224\\
$(b;0,3,0;29)$ & 8 & $(8,3)$ & 4 & 1 & YES & YES & YES & $1.43$ & $(2,3)$ & -- & 9225\\
$(b;1,0,0;5)$ & 6 & $(19,7)$ & 6 & 1 & YES & YES & NO(2) & $1.50$ & $(2,3)$ & -- & 9226\\
$(b;1,0,0;5)$ & 6 & $(21,8)$ & 6 & 1 & YES & YES & NO(2) & $1.50$ & $(2,3)$ & -- & 9227\\
$(b;1,0,1;29)$ & 7 & $(12,5)$ & 5 & 1 & YES & YES & NO(2) & $1.62$ & $(2,3)$ & -- & 9228\\
$(b;1,0,1;29)$ & 7 & $(13,5)$ & 5 & 1 & YES & YES & YES & $1.43$ & $(2,3)$ & -- & 9229\\
$(b;1,0,1;29)$ & 7 & $(19,7)$ & 6 & 1 & YES & YES & YES & $1.71$ & $(2,3)$ & -- & 9230\\
$(b;1,0,1;29)$ & 7 & $(19,8)$ & 6 & 1 & YES & YES & YES & $1.43$ & $(2,3)$ & -- & 9231\\
$(b;1,0,1;29)$ & 7 & $(21,8)$ & 6 & 1 & YES & YES & YES & $1.43$ & $(2,3)$ & -- & 9232\\
$(b;1,0,1;29)$ & 7 & $(27,8)$ & 7 & 1 & YES & YES & YES & $1.43$ & $(2,3)$ & -- & 9233\\
$(b;1,0,1;29)$ & 7 & $(27,10)$ & 7 & 1 & YES & YES & YES & $1.57$ & $(2,3)$ & -- & 9234\\
$(b;1,0,2;19)$ & 8 & $(7,3)$ & 4 & 1 & YES & YES & NO(2) & $1.50$ & $(2,3)$ & -- & 9235\\
$(b;1,0,2;19)$ & 8 & $(9,4)$ & 5 & 1 & YES & YES & NO(2) & $1.75$ & $(2,3)$ & -- & 9236\\
$(b;1,0,3;47)$ & 9 & $(7,2)$ & 4 & 1 & YES & YES & YES & $1.29$ & $(2,3)$ & -- & 9237\\
$(b;1,0,3;47)$ & 9 & $(7,3)$ & 4 & 1 & YES & YES & YES & $1.71$ & $(2,3)$ & -- & 9238\\
$(b;1,0,3;47)$ & 9 & $(13,3)$ & 6 & 1 & YES & YES & YES & $1.57$ & $(2,3)$ & -- & 9239\\
$(b;1,1,0;27)$ & 7 & $(12,5)$ & 5 & 3 & YES & YES & NO(2) & $1.57$ & $(4,2)$ & -- & 9240\\
$(b;1,1,0;27)$ & 7 & $(19,7)$ & 6 & 1 & YES & YES & YES & $1.71$ & $(2,3)$ & -- & 9241\\
$(b;1,1,0;27)$ & 7 & $(19,8)$ & 6 & 1 & YES & YES & YES & $1.43$ & $(2,3)$ & -- & 9242\\
$(b;1,1,0;27)$ & 7 & $(25,9)$ & 7 & 1 & YES & YES & YES & $1.57$ & $(2,3)$ & -- & 9243\\
$(b;1,1,0;27)$ & 7 & $(27,8)$ & 7 & 27 & YES & YES & YES & $1.57$ & $(2,3)$ & -- & 9244\\
$(b;1,1,0;27)$ & 7 & $(29,8)$ & 7 & 1 & YES & YES & YES & $1.57$ & $(2,3)$ & -- & 9245\\
$(b;1,1,1;39)$ & 8 & $(12,5)$ & 5 & 3 & YES & YES & YES & $1.43$ & $(2,3)$ & -- & 9246\\
$(b;1,1,1;39)$ & 8 & $(17,5)$ & 6 & 1 & YES & YES & YES & $1.43$ & $(2,3)$ & -- & 9247\\
$(b;1,1,2;51)$ & 9 & $(4,1)$ & 3 & 1 & YES & YES & NO(2) & $1.50$ & $(2,3)$ & -- & 9248\\
$(b;1,1,2;51)$ & 9 & $(7,3)$ & 4 & 1 & YES & YES & YES & $1.57$ & $(2,3)$ & -- & 9249\\
$(b;1,2,0;17)$ & 8 & $(11,4)$ & 5 & 1 & YES & YES & NO(2) & $1.75$ & $(2,3)$ & -- & 9250\\
$(b;1,2,0;17)$ & 8 & $(12,5)$ & 5 & 1 & YES & YES & NO(2) & $1.50$ & $(2,3)$ & -- & 9251\\
$(b;1,2,1;7)$ & 9 & $(7,3)$ & 4 & 7 & YES & YES & NO(2) & $1.62$ & $(2,3)$ & -- & 9252\\
$(b;2,0,1;38)$ & 8 & $(7,2)$ & 4 & 1 & YES & YES & YES & $1.43$ & $(2,3)$ & -- & 9253\\
$(b;2,0,1;38)$ & 8 & $(13,5)$ & 5 & 1 & YES & YES & YES & $1.43$ & $(2,3)$ & -- & 9254\\
$(b;2,0,2;50)$ & 9 & $(10,3)$ & 5 & 10 & YES & YES & YES & $1.57$ & $(2,3)$ & -- & 9255\\
$(b;2,0,3;62)$ & 10 & $(5,2)$ & 3 & 1 & YES & YES & YES & $1.43$ & $(2,3)$ & -- & 9256\\
$(b;2,1,0;7)$ & 8 & $(12,5)$ & 5 & 1 & YES & YES & YES & $1.43$ & $(2,3)$ & -- & 9257\\
$(b;2,1,1;17)$ & 9 & $(7,3)$ & 4 & 1 & YES & YES & YES & $1.57$ & $(2,3)$ & -- & 9258\\
$(b;2,2,1;64)$ & 10 & $(5,2)$ & 3 & 1 & YES & YES & YES & $1.43$ & $(2,3)$ & -- & 9259\\
$(b;2,2,2;84)$ & 11 & $(4,1)$ & 3 & 4 & YES & YES & NO(2) & $1.62$ & $(2,3)$ & -- & 9260\\
$(b;3,0,1;47)$ & 9 & $(7,2)$ & 4 & 1 & YES & YES & YES & $1.43$ & $(2,3)$ & -- & 9261\\
$(b;3,1,0;43)$ & 9 & $(15,4)$ & 6 & 1 & YES & YES & YES & $1.57$ & $(2,3)$ & -- & 9262\\
$(b;3,1,3;103)$ & 12 & $(2,1)$ & 1 & 1 & YES & YES & YES & $1.57$ & $(2,3)$ & -- & 9263\\
$(c;0,0,0;4)$ & 4 & $(32,7)$ & 8 & 4 & YES & YES & YES & $1.43$ & $(2,3)$ & -- & 9264\\
$(c;0,0,0;4)$ & 4 & $(35,8)$ & 8 & 1 & YES & YES & YES & $1.29$ & $(2,3)$ & -- & 9265\\
$(c;0,0,0;4)$ & 4 & $(37,8)$ & 8 & 1 & YES & YES & YES & $1.43$ & $(2,3)$ & -- & 9266\\
$(c;0,0,0;4)$ & 4 & $(38,7)$ & 9 & 2 & YES & YES & YES & $1.57$ & $(2,3)$ & -- & 9267\\
$(c;0,0,0;4)$ & 4 & $(47,17)$ & 9 & 1 & YES & YES & NO(2) & $1.75$ & $(2,3)$ & -- & 9268\\
$(c;0,0,0;4)$ & 4 & $(49,18)$ & 8 & 1 & YES & YES & YES & $1.43$ & $(2,3)$ & -- & 9269\\
$(c;0,0,0;4)$ & 4 & $(50,19)$ & 8 & 2 & YES & YES & NO(2) & $1.62$ & $(2,3)$ & -- & 9270\\
$(c;0,0,0;4)$ & 4 & $(55,21)$ & 8 & 1 & YES & YES & YES & $1.43$ & $(2,3)$ & -- & 9271\\
$(c;0,0,0;4)$ & 4 & $(62,27)$ & 9 & 2 & YES & YES & YES & $1.29$ & $(2,3)$ & -- & 9272\\
$(c;0,0,0;4)$ & 4 & $(64,19)$ & 9 & 4 & YES & YES & YES & $1.57$ & $(2,3)$ & -- & 9273\\
$(c;0,0,0;4)$ & 4 & $(65,19)$ & 9 & 1 & YES & YES & YES & $1.43$ & $(2,3)$ & -- & 9274\\
$(c;0,0,0;4)$ & 4 & $(66,25)$ & 9 & 2 & YES & YES & YES & $1.43$ & $(2,3)$ & -- & 9275\\
$(c;0,0,0;4)$ & 4 & $(68,25)$ & 9 & 4 & YES & YES & YES & $1.43$ & $(2,3)$ & -- & 9276\\
$(c;0,0,0;4)$ & 4 & $(71,21)$ & 9 & 1 & YES & YES & YES & $1.57$ & $(2,3)$ & -- & 9277\\
$(c;0,0,0;4)$ & 4 & $(71,30)$ & 9 & 1 & YES & YES & YES & $1.43$ & $(2,3)$ & -- & 9278\\
$(c;0,0,0;4)$ & 4 & $(74,31)$ & 9 & 2 & YES & YES & YES & $1.57$ & $(2,3)$ & -- & 9279\\
$(c;0,0,0;4)$ & 4 & $(75,29)$ & 9 & 1 & YES & YES & NO(2) & $1.50$ & $(2,3)$ & -- & 9280\\
$(c;0,0,0;4)$ & 4 & $(79,29)$ & 9 & 1 & YES & YES & YES & $1.43$ & $(2,3)$ & -- & 9281\\
$(c;0,0,0;4)$ & 4 & $(79,30)$ & 9 & 1 & YES & YES & YES & $1.43$ & $(2,3)$ & -- & 9282\\
$(c;0,0,0;4)$ & 4 & $(81,31)$ & 9 & 1 & YES & YES & YES & $1.71$ & $(2,3)$ & -- & 9283\\
$(c;0,0,0;4)$ & 4 & $(82,25)$ & 10 & 2 & YES & YES & YES & $1.43$ & $(2,3)$ & -- & 9284\\
$(c;0,0,0;4)$ & 4 & $(86,31)$ & 10 & 2 & YES & YES & YES & $1.57$ & $(2,3)$ & -- & 9285\\
$(c;0,0,0;4)$ & 4 & $(86,35)$ & 11 & 2 & YES & YES & YES & $1.71$ & $(2,3)$ & -- & 9286\\
$(c;0,0,0;4)$ & 4 & $(100,37)$ & 10 & 4 & YES & YES & YES & $1.57$ & $(2,3)$ & -- & 9287\\
$(c;0,0,0;4)$ & 4 & $(100,39)$ & 10 & 4 & YES & YES & YES & $1.71$ & $(2,3)$ & -- & 9288\\
$(c;0,0,0;4)$ & 4 & $(100,41)$ & 10 & 4 & YES & YES & YES & $1.57$ & $(2,3)$ & -- & 9289\\
$(c;0,0,0;4)$ & 4 & $(101,39)$ & 10 & 1 & YES & YES & YES & $1.57$ & $(2,3)$ & -- & 9290\\
$(c;0,0,0;4)$ & 4 & $(106,41)$ & 10 & 2 & YES & YES & YES & $1.57$ & $(2,3)$ & -- & 9291\\
$(c;0,0,0;4)$ & 4 & $(109,45)$ & 10 & 1 & YES & YES & YES & $1.57$ & $(2,3)$ & -- & 9292\\
$(c;0,0,0;4)$ & 4 & $(116,45)$ & 10 & 4 & YES & YES & YES & $1.57$ & $(2,3)$ & -- & 9293\\
$(c;0,0,0;4)$ & 4 & $(116,49)$ & 10 & 4 & YES & YES & YES & $1.71$ & $(2,3)$ & -- & 9294\\
$(c;0,1,0;11)$ & 5 & $(29,8)$ & 7 & 1 & YES & YES & YES & $1.43$ & $(2,3)$ & -- & 9295\\
$(c;0,1,0;11)$ & 5 & $(31,7)$ & 8 & 1 & YES & YES & YES & $1.57$ & $(2,3)$ & -- & 9296\\
$(c;0,1,0;11)$ & 5 & $(31,13)$ & 7 & 1 & YES & YES & YES & $1.57$ & $(2,3)$ & -- & 9297\\
$(c;0,1,0;11)$ & 5 & $(34,13)$ & 7 & 1 & YES & YES & YES & $1.43$ & $(2,3)$ & -- & 9298\\
$(c;0,1,0;11)$ & 5 & $(37,10)$ & 8 & 1 & YES & YES & YES & $1.57$ & $(2,3)$ & -- & 9299\\
$(c;0,1,0;11)$ & 5 & $(39,16)$ & 8 & 1 & YES & YES & NO(2) & $1.50$ & $(2,3)$ & -- & 9300\\
$(c;0,1,0;11)$ & 5 & $(39,17)$ & 8 & 1 & YES & YES & YES & $1.43$ & $(2,3)$ & -- & 9301\\
$(c;0,1,0;11)$ & 5 & $(41,17)$ & 8 & 1 & YES & YES & YES & $1.57$ & $(2,3)$ & -- & 9302\\
$(c;0,1,0;11)$ & 5 & $(44,17)$ & 8 & 11 & YES & YES & YES & $1.57$ & $(2,3)$ & -- & 9303\\
$(c;0,1,0;11)$ & 5 & $(45,19)$ & 8 & 1 & YES & YES & YES & $1.57$ & $(2,3)$ & -- & 9304\\
$(c;0,1,0;11)$ & 5 & $(46,19)$ & 8 & 1 & YES & YES & YES & $1.43$ & $(2,3)$ & -- & 9305\\
$(c;0,1,0;11)$ & 5 & $(49,19)$ & 8 & 1 & YES & YES & YES & $1.43$ & $(2,3)$ & -- & 9306\\
$(c;0,1,0;11)$ & 5 & $(51,16)$ & 10 & 1 & YES & YES & YES & $1.71$ & $(2,3)$ & -- & 9307\\
$(c;0,1,0;11)$ & 5 & $(53,14)$ & 9 & 1 & YES & YES & YES & $1.43$ & $(2,3)$ & -- & 9308\\
$(c;0,1,0;11)$ & 5 & $(55,16)$ & 9 & 11 & YES & YES & YES & $1.57$ & $(2,3)$ & -- & 9309\\
$(c;0,1,0;11)$ & 5 & $(55,21)$ & 8 & 11 & YES & YES & YES & $1.43$ & $(2,3)$ & -- & 9310\\
$(c;0,1,0;11)$ & 5 & $(56,23)$ & 9 & 1 & YES & YES & YES & $1.57$ & $(2,3)$ & -- & 9311\\
$(c;0,1,0;11)$ & 5 & $(58,13)$ & 11 & 1 & YES & YES & YES & $1.57$ & $(2,3)$ & -- & 9312\\
$(c;0,1,0;11)$ & 5 & $(59,26)$ & 9 & 1 & YES & YES & YES & $1.57$ & $(2,3)$ & -- & 9313\\
$(c;0,1,0;11)$ & 5 & $(61,18)$ & 9 & 1 & YES & YES & YES & $1.43$ & $(2,3)$ & -- & 9314\\
$(c;0,1,0;11)$ & 5 & $(63,26)$ & 9 & 1 & YES & YES & YES & $1.43$ & $(2,3)$ & -- & 9315\\
$(c;0,1,0;11)$ & 5 & $(64,19)$ & 9 & 1 & YES & YES & YES & $1.57$ & $(2,3)$ & -- & 9316\\
$(c;0,1,0;11)$ & 5 & $(65,24)$ & 9 & 1 & YES & YES & YES & $1.57$ & $(2,3)$ & -- & 9317\\
$(c;0,1,0;11)$ & 5 & $(69,19)$ & 9 & 1 & YES & YES & YES & $1.57$ & $(2,3)$ & -- & 9318\\
$(c;0,1,0;11)$ & 5 & $(70,27)$ & 10 & 1 & YES & YES & YES & $1.71$ & $(2,3)$ & -- & 9319\\
$(c;0,1,0;11)$ & 5 & $(70,29)$ & 9 & 1 & YES & YES & YES & $1.57$ & $(2,3)$ & -- & 9320\\
$(c;0,1,0;11)$ & 5 & $(73,27)$ & 9 & 1 & YES & YES & YES & $1.43$ & $(2,3)$ & -- & 9321\\
$(c;0,1,0;11)$ & 5 & $(73,30)$ & 10 & 1 & YES & YES & YES & $1.86$ & $(2,3)$ & -- & 9322\\
$(c;0,1,0;11)$ & 5 & $(74,31)$ & 9 & 1 & YES & YES & YES & $1.43$ & $(2,3)$ & -- & 9323\\
$(c;0,1,0;11)$ & 5 & $(75,31)$ & 9 & 1 & YES & YES & YES & $1.57$ & $(2,3)$ & -- & 9324\\
$(c;0,1,0;11)$ & 5 & $(76,29)$ & 9 & 1 & YES & YES & YES & $1.57$ & $(2,3)$ & -- & 9325\\
$(c;0,1,0;11)$ & 5 & $(79,18)$ & 10 & 1 & YES & YES & YES & $1.43$ & $(2,3)$ & -- & 9326\\
$(c;0,1,0;11)$ & 5 & $(79,23)$ & 10 & 1 & YES & YES & YES & $1.57$ & $(2,3)$ & -- & 9327\\
$(c;0,1,0;11)$ & 5 & $(79,29)$ & 9 & 1 & YES & YES & YES & $1.71$ & $(2,3)$ & -- & 9328\\
$(c;0,1,0;11)$ & 5 & $(82,23)$ & 10 & 1 & YES & YES & YES & $1.57$ & $(2,3)$ & -- & 9329\\
$(c;0,1,0;11)$ & 5 & $(83,19)$ & 10 & 1 & YES & YES & YES & $1.57$ & $(2,3)$ & -- & 9330\\
$(c;0,1,0;11)$ & 5 & $(89,26)$ & 10 & 1 & YES & YES & YES & $1.57$ & $(2,3)$ & -- & 9331\\
$(c;0,1,0;11)$ & 5 & $(100,27)$ & 10 & 1 & YES & YES & YES & $1.43$ & $(2,3)$ & -- & 9332\\
$(c;0,1,1;5)$ & 6 & $(19,8)$ & 6 & 1 & YES & YES & NO(2) & $1.29$ & $(4,2)$ & -- & 9333\\
$(c;0,1,1;5)$ & 6 & $(30,7)$ & 8 & 5 & YES & YES & YES & $1.71$ & $(2,3)$ & -- & 9334\\
$(c;0,1,1;5)$ & 6 & $(31,13)$ & 7 & 1 & YES & YES & YES & $1.43$ & $(2,3)$ & -- & 9335\\
$(c;0,1,1;5)$ & 6 & $(39,17)$ & 8 & 1 & YES & YES & YES & $1.57$ & $(2,3)$ & -- & 9336\\
$(c;0,1,1;5)$ & 6 & $(43,13)$ & 9 & 1 & YES & YES & YES & $1.43$ & $(2,3)$ & -- & 9337\\
$(c;0,1,1;5)$ & 6 & $(46,19)$ & 8 & 1 & YES & YES & YES & $1.43$ & $(2,3)$ & -- & 9338\\
$(c;0,1,1;5)$ & 6 & $(47,14)$ & 9 & 1 & YES & YES & YES & $1.43$ & $(2,3)$ & -- & 9339\\
$(c;0,1,1;5)$ & 6 & $(56,13)$ & 10 & 1 & YES & YES & YES & $1.43$ & $(2,3)$ & -- & 9340\\
$(c;0,2,0;7)$ & 6 & $(26,11)$ & 7 & 1 & YES & YES & YES & $1.43$ & $(2,3)$ & -- & 9341\\
$(c;0,2,0;7)$ & 6 & $(28,11)$ & 8 & 7 & YES & YES & YES & $1.43$ & $(2,3)$ & -- & 9342\\
$(c;0,2,0;7)$ & 6 & $(29,11)$ & 7 & 1 & YES & YES & YES & $1.43$ & $(2,3)$ & -- & 9343\\
$(c;0,2,0;7)$ & 6 & $(29,12)$ & 7 & 1 & YES & YES & YES & $1.43$ & $(2,3)$ & -- & 9344\\
$(c;0,2,0;7)$ & 6 & $(31,9)$ & 8 & 1 & YES & YES & YES & $1.43$ & $(2,3)$ & -- & 9345\\
$(c;0,2,0;7)$ & 6 & $(34,13)$ & 7 & 1 & YES & YES & YES & $1.43$ & $(2,3)$ & -- & 9346\\
$(c;0,2,0;7)$ & 6 & $(34,15)$ & 8 & 1 & YES & YES & YES & $1.43$ & $(2,3)$ & -- & 9347\\
$(c;0,2,0;7)$ & 6 & $(37,14)$ & 8 & 1 & YES & YES & YES & $1.57$ & $(2,3)$ & -- & 9348\\
$(c;0,2,0;7)$ & 6 & $(39,16)$ & 8 & 1 & YES & YES & YES & $1.43$ & $(2,3)$ & -- & 9349\\
$(c;0,2,0;7)$ & 6 & $(42,13)$ & 9 & 7 & YES & YES & YES & $1.43$ & $(2,3)$ & -- & 9350\\
$(c;0,2,0;7)$ & 6 & $(43,13)$ & 9 & 1 & YES & YES & YES & $1.43$ & $(2,3)$ & -- & 9351\\
$(c;0,2,0;7)$ & 6 & $(44,17)$ & 8 & 1 & YES & YES & YES & $1.57$ & $(2,3)$ & -- & 9352\\
$(c;0,2,0;7)$ & 6 & $(49,13)$ & 9 & 7 & YES & YES & YES & $1.43$ & $(2,3)$ & -- & 9353\\
$(c;0,2,0;7)$ & 6 & $(49,19)$ & 8 & 7 & YES & YES & YES & $1.57$ & $(2,3)$ & -- & 9354\\
$(c;0,2,0;7)$ & 6 & $(50,19)$ & 8 & 1 & YES & YES & YES & $1.57$ & $(2,3)$ & -- & 9355\\
$(c;0,2,0;7)$ & 6 & $(51,14)$ & 9 & 1 & YES & YES & YES & $1.57$ & $(2,3)$ & -- & 9356\\
$(c;0,2,0;7)$ & 6 & $(56,13)$ & 10 & 7 & YES & YES & YES & $1.43$ & $(2,3)$ & -- & 9357\\
$(c;0,2,0;7)$ & 6 & $(57,13)$ & 9 & 1 & YES & YES & YES & $1.43$ & $(2,3)$ & -- & 9358\\
$(c;0,2,1;19)$ & 7 & $(17,7)$ & 6 & 1 & YES & YES & YES & $1.43$ & $(2,3)$ & -- & 9359\\
$(c;0,2,1;19)$ & 7 & $(19,8)$ & 6 & 19 & YES & YES & NO(2) & $1.43$ & $(4,2)$ & -- & 9360\\
$(c;0,2,1;19)$ & 7 & $(23,7)$ & 7 & 1 & YES & YES & NO(2) & $1.50$ & $(2,3)$ & -- & 9361\\
$(c;0,2,1;19)$ & 7 & $(30,13)$ & 8 & 1 & YES & YES & YES & $1.57$ & $(2,3)$ & -- & 9362\\
$(c;0,2,1;19)$ & 7 & $(34,9)$ & 8 & 1 & YES & YES & YES & $1.57$ & $(2,3)$ & -- & 9363\\
$(c;0,2,1;19)$ & 7 & $(53,12)$ & 9 & 1 & YES & YES & YES & $1.43$ & $(2,3)$ & -- & 9364\\
$(c;0,2,2;6)$ & 8 & $(23,9)$ & 7 & 1 & YES & YES & YES & $1.71$ & $(2,3)$ & -- & 9365\\
$(c;0,3,0;17)$ & 7 & $(27,8)$ & 7 & 1 & YES & YES & YES & $1.43$ & $(2,3)$ & -- & 9366\\
$(c;0,3,0;17)$ & 7 & $(30,11)$ & 7 & 1 & YES & YES & YES & $1.57$ & $(2,3)$ & -- & 9367\\
$(c;0,3,0;17)$ & 7 & $(37,8)$ & 8 & 1 & YES & YES & YES & $1.43$ & $(2,3)$ & -- & 9368\\
$(c;0,3,1;23)$ & 8 & $(30,7)$ & 8 & 1 & YES & YES & YES & $1.71$ & $(2,3)$ & -- & 9369\\
$(d;0,0,0;5)$ & 5 & $(29,8)$ & 7 & 1 & YES & YES & YES & $1.29$ & $(2,3)$ & -- & 9370\\
$(d;0,0,0;5)$ & 5 & $(31,13)$ & 7 & 1 & YES & YES & YES & $1.29$ & $(2,3)$ & -- & 9371\\
$(d;0,0,0;5)$ & 5 & $(34,13)$ & 7 & 1 & YES & YES & YES & $1.29$ & $(2,3)$ & -- & 9372\\
$(d;0,0,0;5)$ & 5 & $(36,11)$ & 8 & 1 & YES & YES & YES & $1.29$ & $(2,3)$ & -- & 9373\\
$(d;0,0,0;5)$ & 5 & $(41,18)$ & 8 & 1 & YES & YES & YES & $1.43$ & $(2,3)$ & -- & 9374\\
$(d;0,0,0;5)$ & 5 & $(42,13)$ & 9 & 1 & YES & YES & YES & $1.57$ & $(2,3)$ & -- & 9375\\
$(d;0,0,0;5)$ & 5 & $(44,17)$ & 8 & 1 & YES & YES & YES & $1.57$ & $(2,3)$ & -- & 9376\\
$(d;0,0,0;5)$ & 5 & $(45,14)$ & 9 & 5 & YES & YES & YES & $1.29$ & $(2,3)$ & -- & 9377\\
$(d;0,0,0;5)$ & 5 & $(46,17)$ & 8 & 1 & YES & YES & NO(3) & $1.29$ & $(2,3)$ & -- & 9378\\
$(d;0,0,0;5)$ & 5 & $(46,19)$ & 8 & 1 & YES & YES & YES & $1.29$ & $(2,3)$ & -- & 9379\\
$(d;0,0,0;5)$ & 5 & $(47,18)$ & 8 & 1 & YES & YES & YES & $1.43$ & $(2,3)$ & -- & 9380\\
$(d;0,0,0;5)$ & 5 & $(49,19)$ & 8 & 1 & YES & YES & YES & $1.57$ & $(2,3)$ & -- & 9381\\
$(d;0,0,0;5)$ & 5 & $(55,21)$ & 8 & 5 & YES & YES & YES & $1.29$ & $(2,3)$ & -- & 9382\\
$(d;0,0,0;5)$ & 5 & $(56,23)$ & 9 & 1 & YES & YES & YES & $1.43$ & $(2,3)$ & -- & 9383\\
$(d;0,0,0;5)$ & 5 & $(62,23)$ & 9 & 1 & YES & YES & YES & $1.57$ & $(2,3)$ & -- & 9384\\
$(d;0,0,0;5)$ & 5 & $(68,19)$ & 9 & 1 & YES & YES & YES & $1.43$ & $(2,3)$ & -- & 9385\\
$(d;0,0,0;5)$ & 5 & $(70,29)$ & 9 & 5 & YES & YES & YES & $1.57$ & $(2,3)$ & -- & 9386\\
$(d;0,0,0;5)$ & 5 & $(71,21)$ & 9 & 1 & YES & YES & YES & $1.43$ & $(2,3)$ & -- & 9387\\
$(d;0,0,0;5)$ & 5 & $(76,23)$ & 10 & 1 & YES & YES & YES & $1.57$ & $(2,3)$ & -- & 9388\\
$(d;0,0,0;5)$ & 5 & $(82,23)$ & 10 & 1 & YES & YES & YES & $1.43$ & $(2,3)$ & -- & 9389\\
$(d;0,0,1;14)$ & 6 & $(19,8)$ & 6 & 1 & YES & YES & NO(2) & $1.50$ & $(2,3)$ & -- & 9390\\
$(d;0,0,1;14)$ & 6 & $(21,8)$ & 6 & 7 & YES & YES & YES & $1.43$ & $(2,3)$ & -- & 9391\\
$(d;0,0,1;14)$ & 6 & $(23,9)$ & 7 & 1 & YES & YES & YES & $1.43$ & $(2,3)$ & -- & 9392\\
$(d;0,0,1;14)$ & 6 & $(26,7)$ & 7 & 2 & YES & YES & YES & $1.43$ & $(2,3)$ & -- & 9393\\
$(d;0,0,1;14)$ & 6 & $(26,11)$ & 7 & 2 & YES & YES & YES & $1.43$ & $(2,3)$ & -- & 9394\\
$(d;0,0,1;14)$ & 6 & $(27,10)$ & 7 & 1 & YES & YES & YES & $1.57$ & $(2,3)$ & -- & 9395\\
$(d;0,0,1;14)$ & 6 & $(29,12)$ & 7 & 1 & YES & YES & YES & $1.43$ & $(2,3)$ & -- & 9396\\
$(d;0,0,1;14)$ & 6 & $(31,9)$ & 8 & 1 & YES & YES & YES & $1.43$ & $(2,3)$ & -- & 9397\\
$(d;0,0,1;14)$ & 6 & $(31,14)$ & 8 & 1 & YES & YES & YES & $1.43$ & $(2,3)$ & -- & 9398\\
$(d;0,0,1;14)$ & 6 & $(33,10)$ & 8 & 1 & YES & YES & YES & $1.57$ & $(2,3)$ & -- & 9399\\
$(d;0,0,1;14)$ & 6 & $(33,14)$ & 8 & 1 & YES & YES & YES & $1.43$ & $(2,3)$ & -- & 9400\\
$(d;0,0,1;14)$ & 6 & $(36,11)$ & 8 & 2 & YES & YES & YES & $1.57$ & $(2,3)$ & -- & 9401\\
$(d;0,0,1;14)$ & 6 & $(39,16)$ & 8 & 1 & YES & YES & YES & $1.43$ & $(2,3)$ & -- & 9402\\
$(d;0,0,1;14)$ & 6 & $(43,13)$ & 9 & 1 & YES & YES & YES & $1.43$ & $(2,3)$ & -- & 9403\\
$(d;0,0,1;14)$ & 6 & $(46,19)$ & 8 & 2 & YES & YES & YES & $1.57$ & $(2,3)$ & -- & 9404\\
$(d;0,0,1;14)$ & 6 & $(49,18)$ & 8 & 7 & YES & YES & YES & $1.43$ & $(2,3)$ & -- & 9405\\
$(d;0,0,1;14)$ & 6 & $(50,19)$ & 8 & 2 & YES & YES & YES & $1.43$ & $(2,3)$ & -- & 9406\\
$(d;0,0,1;14)$ & 6 & $(55,16)$ & 9 & 1 & YES & YES & YES & $1.43$ & $(2,3)$ & -- & 9407\\
$(d;0,0,2;9)$ & 7 & $(17,5)$ & 6 & 1 & YES & YES & YES & $1.57$ & $(2,3)$ & -- & 9408\\
$(d;0,0,2;9)$ & 7 & $(30,13)$ & 8 & 3 & YES & YES & YES & $1.57$ & $(2,3)$ & -- & 9409\\
$(d;0,0,3;22)$ & 8 & $(23,7)$ & 7 & 1 & YES & YES & YES & $1.57$ & $(2,3)$ & -- & 9410\\
$(d;0,0,3;22)$ & 8 & $(29,9)$ & 8 & 1 & YES & YES & YES & $1.57$ & $(2,3)$ & -- & 9411\\
$(d;0,1,0;6)$ & 6 & $(28,11)$ & 8 & 2 & YES & YES & YES & $1.43$ & $(2,3)$ & -- & 9412\\
$(d;0,1,0;6)$ & 6 & $(29,11)$ & 7 & 1 & YES & YES & YES & $1.43$ & $(2,3)$ & -- & 9413\\
$(d;0,1,0;6)$ & 6 & $(31,9)$ & 8 & 1 & YES & YES & YES & $1.57$ & $(2,3)$ & -- & 9414\\
$(d;0,1,0;6)$ & 6 & $(42,13)$ & 9 & 6 & YES & YES & YES & $1.57$ & $(2,3)$ & -- & 9415\\
$(d;0,1,0;6)$ & 6 & $(43,18)$ & 8 & 1 & YES & YES & YES & $1.43$ & $(2,3)$ & -- & 9416\\
$(d;0,1,0;6)$ & 6 & $(49,13)$ & 9 & 1 & YES & YES & YES & $1.57$ & $(2,3)$ & -- & 9417\\
$(d;0,1,0;6)$ & 6 & $(49,18)$ & 8 & 1 & YES & YES & YES & $1.43$ & $(2,3)$ & -- & 9418\\
$(d;0,1,1;17)$ & 7 & $(23,7)$ & 7 & 1 & YES & YES & NO(2) & $1.50$ & $(2,3)$ & -- & 9419\\
$(d;0,1,1;17)$ & 7 & $(30,13)$ & 8 & 1 & YES & YES & YES & $1.57$ & $(2,3)$ & -- & 9420\\
$(d;0,1,2;11)$ & 8 & $(19,7)$ & 6 & 1 & YES & YES & NO(2) & $1.62$ & $(2,3)$ & -- & 9421\\
$(d;0,1,2;11)$ & 8 & $(23,9)$ & 7 & 1 & YES & YES & YES & $1.71$ & $(2,3)$ & -- & 9422\\
$(d;0,2,1;20)$ & 8 & $(19,8)$ & 6 & 1 & YES & YES & YES & $1.71$ & $(2,3)$ & -- & 9423\\
$(d;0,2,1;20)$ & 8 & $(23,7)$ & 7 & 1 & YES & YES & YES & $1.57$ & $(2,3)$ & -- & 9424\\
$(e;0,0,0;4)$ & 5 & $(29,11)$ & 7 & 1 & YES & YES & YES & $1.57$ & $(2,3)$ & -- & 9425\\
$(e;0,0,0;4)$ & 5 & $(30,11)$ & 7 & 2 & YES & YES & NO(2) & $1.50$ & $(2,3)$ & -- & 9426\\
$(e;0,0,0;4)$ & 5 & $(32,9)$ & 8 & 4 & YES & YES & YES & $1.29$ & $(2,3)$ & -- & 9427\\
$(e;0,0,0;4)$ & 5 & $(33,10)$ & 8 & 1 & YES & YES & YES & $1.71$ & $(2,3)$ & -- & 9428\\
$(e;0,0,0;4)$ & 5 & $(37,11)$ & 8 & 1 & YES & YES & YES & $1.29$ & $(2,3)$ & -- & 9429\\
$(e;0,0,0;4)$ & 5 & $(39,16)$ & 8 & 1 & YES & YES & YES & $1.71$ & $(2,3)$ & -- & 9430\\
$(e;0,0,0;4)$ & 5 & $(40,9)$ & 9 & 4 & YES & YES & NO(2) & $1.38$ & $(2,3)$ & -- & 9431\\
$(e;0,0,0;4)$ & 5 & $(41,17)$ & 8 & 1 & YES & YES & YES & $1.57$ & $(2,3)$ & -- & 9432\\
$(e;0,0,0;4)$ & 5 & $(46,17)$ & 8 & 2 & YES & YES & YES & $1.43$ & $(2,3)$ & -- & 9433\\
$(e;0,0,0;4)$ & 5 & $(46,19)$ & 8 & 2 & YES & YES & YES & $1.43$ & $(2,3)$ & -- & 9434\\
$(e;1,0,0;18)$ & 6 & $(19,7)$ & 6 & 1 & YES & YES & YES & $1.57$ & $(2,3)$ & -- & 9435\\
$(e;1,0,0;18)$ & 6 & $(19,8)$ & 6 & 1 & YES & YES & NO(2) & $1.50$ & $(2,3)$ & -- & 9436\\
$(e;1,0,0;18)$ & 6 & $(22,9)$ & 7 & 2 & YES & YES & YES & $1.57$ & $(2,3)$ & -- & 9437\\
$(e;1,0,0;18)$ & 6 & $(23,10)$ & 7 & 1 & YES & YES & YES & $1.57$ & $(2,3)$ & -- & 9438\\
$(e;1,0,0;18)$ & 6 & $(29,11)$ & 7 & 1 & YES & YES & YES & $1.57$ & $(2,3)$ & -- & 9439\\
$(e;1,0,0;18)$ & 6 & $(31,9)$ & 8 & 1 & YES & YES & YES & $1.57$ & $(2,3)$ & -- & 9440\\
$(e;1,0,0;18)$ & 6 & $(31,12)$ & 7 & 1 & YES & YES & YES & $1.57$ & $(2,3)$ & -- & 9441\\
$(e;1,0,0;18)$ & 6 & $(31,13)$ & 7 & 1 & YES & YES & YES & $1.57$ & $(2,3)$ & -- & 9442\\
$(e;1,0,0;18)$ & 6 & $(39,17)$ & 8 & 3 & YES & YES & YES & $1.57$ & $(2,3)$ & -- & 9443\\
$(e;1,1,0;23)$ & 7 & $(13,5)$ & 5 & 1 & YES & YES & YES & $1.43$ & $(2,3)$ & -- & 9444\\
$(e;1,1,0;23)$ & 7 & $(17,7)$ & 6 & 1 & YES & YES & YES & $1.57$ & $(2,3)$ & -- & 9445\\
$(e;1,1,0;23)$ & 7 & $(19,8)$ & 6 & 1 & YES & YES & YES & $1.43$ & $(2,3)$ & -- & 9446\\
$(e;1,1,0;23)$ & 7 & $(27,8)$ & 7 & 1 & YES & YES & YES & $1.43$ & $(2,3)$ & -- & 9447\\
$(e;2,0,0;24)$ & 7 & $(8,3)$ & 4 & 8 & YES & YES & YES & $1.29$ & $(2,3)$ & -- & 9448\\
$(e;2,0,0;24)$ & 7 & $(12,5)$ & 5 & 12 & YES & YES & YES & $1.29$ & $(2,3)$ & -- & 9449\\
$(e;2,0,0;24)$ & 7 & $(18,5)$ & 6 & 6 & YES & YES & NO(2) & $1.38$ & $(2,3)$ & -- & 9450\\
$(e;2,0,0;24)$ & 7 & $(24,7)$ & 7 & 24 & YES & YES & YES & $1.57$ & $(2,3)$ & -- & 9451\\
$(e;2,1,0;31)$ & 8 & $(7,2)$ & 4 & 1 & YES & YES & NO(2) & $1.50$ & $(2,3)$ & -- & 9452\\
$(e;2,1,0;31)$ & 8 & $(11,4)$ & 5 & 1 & YES & YES & YES & $1.43$ & $(2,3)$ & -- & 9453\\
$(e;2,1,0;31)$ & 8 & $(12,5)$ & 5 & 1 & YES & YES & YES & $1.43$ & $(2,3)$ & -- & 9454\\
$(e;2,2,0;38)$ & 9 & $(7,2)$ & 4 & 1 & YES & YES & NO(2) & $1.50$ & $(2,3)$ & -- & 9455\\
$(e;2,2,0;38)$ & 9 & $(8,3)$ & 4 & 2 & YES & YES & NO(2) & $1.50$ & $(2,3)$ & -- & 9456\\
$(e;3,2,0;16)$ & 10 & $(7,3)$ & 4 & 1 & YES & YES & YES & $1.57$ & $(2,3)$ & -- & 9457\\
$(f;0,0,0;6)$ & 4 & $(32,7)$ & 8 & 2 & YES & YES & YES & $1.57$ & $(2,3)$ & -- & 9458\\
$(f;0,0,0;6)$ & 4 & $(41,11)$ & 8 & 1 & YES & YES & YES & $1.43$ & $(2,3)$ & -- & 9459\\
$(f;0,0,0;6)$ & 4 & $(43,12)$ & 8 & 1 & YES & YES & YES & $1.57$ & $(2,3)$ & -- & 9460\\
$(f;0,0,0;6)$ & 4 & $(49,18)$ & 8 & 1 & YES & YES & NO(2) & $1.56$ & $(2,3)$ & -- & 9461\\
$(f;0,0,0;6)$ & 4 & $(55,21)$ & 8 & 1 & YES & YES & YES & $1.57$ & $(2,3)$ & -- & 9462\\
$(f;0,0,0;6)$ & 4 & $(61,17)$ & 9 & 1 & YES & YES & YES & $1.57$ & $(2,3)$ & -- & 9463\\
$(f;0,0,0;6)$ & 4 & $(61,18)$ & 9 & 1 & YES & YES & YES & $1.43$ & $(2,3)$ & -- & 9464\\
$(f;0,0,0;6)$ & 4 & $(64,19)$ & 9 & 2 & YES & YES & YES & $1.43$ & $(2,3)$ & -- & 9465\\
$(f;0,0,0;6)$ & 4 & $(65,18)$ & 9 & 1 & YES & YES & YES & $1.43$ & $(2,3)$ & -- & 9466\\
$(f;0,0,0;6)$ & 4 & $(68,25)$ & 9 & 2 & YES & YES & NO(2) & $1.50$ & $(2,3)$ & -- & 9467\\
$(f;0,0,0;6)$ & 4 & $(73,32)$ & 10 & 1 & YES & YES & NO(2) & $1.62$ & $(2,3)$ & -- & 9468\\
$(f;0,0,0;6)$ & 4 & $(74,31)$ & 9 & 2 & YES & YES & YES & $1.57$ & $(2,3)$ & -- & 9469\\
$(f;0,0,0;6)$ & 4 & $(75,31)$ & 9 & 3 & YES & YES & NO(2) & $1.71$ & $(4,2)$ & -- & 9470\\
$(f;0,0,0;6)$ & 4 & $(78,23)$ & 10 & 6 & YES & YES & NO(2) & $1.62$ & $(2,3)$ & -- & 9471\\
$(f;0,0,0;6)$ & 4 & $(79,29)$ & 9 & 1 & YES & YES & YES & $1.29$ & $(2,3)$ & -- & 9472\\
$(f;0,0,0;6)$ & 4 & $(83,23)$ & 10 & 1 & YES & YES & YES & $1.43$ & $(2,3)$ & -- & 9473\\
$(f;0,0,0;6)$ & 4 & $(83,36)$ & 10 & 1 & YES & YES & NO(2) & $1.62$ & $(2,3)$ & -- & 9474\\
$(f;0,0,0;6)$ & 4 & $(86,25)$ & 10 & 2 & YES & YES & YES & $1.71$ & $(2,3)$ & -- & 9475\\
$(f;0,0,0;6)$ & 4 & $(87,32)$ & 10 & 3 & YES & YES & NO(2) & $1.50$ & $(2,3)$ & -- & 9476\\
$(f;0,0,0;6)$ & 4 & $(93,34)$ & 10 & 3 & YES & YES & NO(2) & $1.75$ & $(2,3)$ & -- & 9477\\
$(f;0,0,0;6)$ & 4 & $(95,36)$ & 10 & 1 & YES & YES & NO(2) & $1.50$ & $(2,3)$ & -- & 9478\\
$(f;0,0,0;6)$ & 4 & $(95,39)$ & 10 & 1 & YES & YES & NO(2) & $1.62$ & $(2,3)$ & -- & 9479\\
$(f;0,0,0;6)$ & 4 & $(97,22)$ & 11 & 1 & YES & YES & NO(2) & $1.50$ & $(2,3)$ & -- & 9480\\
$(f;0,0,0;6)$ & 4 & $(97,36)$ & 10 & 1 & YES & YES & NO(2) & $1.50$ & $(2,3)$ & -- & 9481\\
$(f;0,0,0;6)$ & 4 & $(97,37)$ & 10 & 1 & YES & YES & YES & $1.57$ & $(2,3)$ & -- & 9482\\
$(f;0,0,0;6)$ & 4 & $(99,29)$ & 10 & 3 & YES & YES & YES & $1.57$ & $(2,3)$ & -- & 9483\\
$(f;0,0,0;6)$ & 4 & $(100,37)$ & 10 & 2 & YES & YES & YES & $1.57$ & $(2,3)$ & -- & 9484\\
$(f;0,0,0;6)$ & 4 & $(100,39)$ & 10 & 2 & YES & YES & NO(2) & $1.62$ & $(2,3)$ & -- & 9485\\
$(f;0,0,0;6)$ & 4 & $(102,31)$ & 11 & 6 & YES & YES & NO(2) & $1.62$ & $(2,3)$ & -- & 9486\\
$(f;0,0,0;6)$ & 4 & $(103,37)$ & 10 & 1 & YES & YES & YES & $1.71$ & $(2,3)$ & -- & 9487\\
$(f;0,0,0;6)$ & 4 & $(104,43)$ & 10 & 2 & YES & YES & YES & $1.57$ & $(2,3)$ & -- & 9488\\
$(f;0,0,0;6)$ & 4 & $(107,41)$ & 10 & 1 & YES & YES & YES & $1.71$ & $(2,3)$ & -- & 9489\\
$(f;0,0,0;6)$ & 4 & $(109,40)$ & 10 & 1 & YES & YES & YES & $1.71$ & $(2,3)$ & -- & 9490\\
$(f;0,0,0;6)$ & 4 & $(111,41)$ & 10 & 3 & YES & YES & YES & $1.43$ & $(2,3)$ & -- & 9491\\
$(f;0,0,0;6)$ & 4 & $(113,31)$ & 11 & 1 & YES & YES & YES & $1.43$ & $(2,3)$ & -- & 9492\\
$(f;0,0,0;6)$ & 4 & $(117,34)$ & 11 & 3 & YES & YES & YES & $1.43$ & $(2,3)$ & -- & 9493\\
$(f;0,0,0;6)$ & 4 & $(118,49)$ & 11 & 2 & YES & YES & YES & $1.57$ & $(2,3)$ & -- & 9494\\
$(f;0,0,0;6)$ & 4 & $(124,47)$ & 11 & 2 & YES & YES & YES & $1.57$ & $(2,3)$ & -- & 9495\\
$(f;0,0,0;6)$ & 4 & $(127,47)$ & 11 & 1 & YES & YES & YES & $1.57$ & $(2,3)$ & -- & 9496\\
$(f;0,0,0;6)$ & 4 & $(127,49)$ & 11 & 1 & YES & YES & YES & $1.57$ & $(2,3)$ & -- & 9497\\
$(f;0,0,0;6)$ & 4 & $(128,53)$ & 11 & 2 & YES & YES & YES & $1.57$ & $(2,3)$ & -- & 9498\\
$(f;0,0,0;6)$ & 4 & $(129,53)$ & 11 & 3 & YES & YES & YES & $1.71$ & $(2,3)$ & -- & 9499\\
$(f;0,0,0;6)$ & 4 & $(130,47)$ & 11 & 2 & YES & YES & YES & $1.86$ & $(2,3)$ & -- & 9500\\
$(f;0,0,0;6)$ & 4 & $(131,36)$ & 11 & 1 & YES & YES & YES & $1.43$ & $(2,3)$ & -- & 9501\\
$(f;0,0,0;6)$ & 4 & $(131,48)$ & 11 & 1 & YES & YES & YES & $1.71$ & $(2,3)$ & -- & 9502\\
$(f;0,0,0;6)$ & 4 & $(131,50)$ & 10 & 1 & YES & YES & YES & $1.57$ & $(2,3)$ & -- & 9503\\
$(f;0,0,0;6)$ & 4 & $(133,39)$ & 11 & 1 & YES & YES & YES & $1.57$ & $(2,3)$ & -- & 9504\\
$(f;0,0,0;6)$ & 4 & $(134,39)$ & 11 & 2 & YES & YES & YES & $1.57$ & $(2,3)$ & -- & 9505\\
$(f;0,0,0;6)$ & 4 & $(140,39)$ & 11 & 2 & YES & YES & YES & $1.43$ & $(2,3)$ & -- & 9506\\
$(f;0,0,0;6)$ & 4 & $(145,56)$ & 11 & 1 & YES & YES & YES & $1.57$ & $(2,3)$ & -- & 9507\\
$(f;0,0,0;6)$ & 4 & $(149,41)$ & 11 & 1 & YES & YES & YES & $1.57$ & $(2,3)$ & -- & 9508\\
$(f;0,0,0;6)$ & 4 & $(151,28)$ & 13 & 1 & YES & YES & YES & $1.57$ & $(2,3)$ & -- & 9509\\
$(f;0,0,0;6)$ & 4 & $(152,55)$ & 12 & 2 & YES & YES & YES & $1.86$ & $(2,3)$ & -- & 9510\\
$(f;0,0,0;6)$ & 4 & $(154,43)$ & 11 & 2 & YES & YES & YES & $1.43$ & $(2,3)$ & -- & 9511\\
$(f;0,0,0;6)$ & 4 & $(155,56)$ & 12 & 1 & YES & YES & YES & $1.71$ & $(2,3)$ & -- & 9512\\
$(f;0,0,0;6)$ & 4 & $(159,44)$ & 11 & 3 & YES & YES & YES & $1.43$ & $(2,3)$ & -- & 9513\\
$(f;0,0,0;6)$ & 4 & $(164,37)$ & 13 & 2 & YES & YES & YES & $1.71$ & $(2,3)$ & -- & 9514\\
$(f;0,0,0;6)$ & 4 & $(167,46)$ & 11 & 1 & YES & YES & YES & $1.43$ & $(2,3)$ & -- & 9515\\
$(g;0,0,0;19)$ & 6 & $(17,5)$ & 6 & 1 & YES & YES & YES & $1.71$ & $(2,3)$ & -- & 9516\\
$(g;0,0,0;19)$ & 6 & $(29,12)$ & 7 & 1 & YES & YES & YES & $1.71$ & $(2,3)$ & -- & 9517\\
$(g;0,0,1;26)$ & 7 & $(8,3)$ & 4 & 2 & YES & YES & YES & $1.29$ & $(2,3)$ & -- & 9518\\
$(g;0,0,1;26)$ & 7 & $(12,5)$ & 5 & 2 & YES & YES & YES & $1.29$ & $(2,3)$ & -- & 9519\\
$(g;0,0,3;40)$ & 9 & $(7,3)$ & 4 & 1 & YES & YES & YES & $1.71$ & $(2,3)$ & -- & 9520\\
$(g;0,0,3;40)$ & 9 & $(8,3)$ & 4 & 8 & YES & YES & YES & $1.57$ & $(2,3)$ & -- & 9521\\
$(g;0,1,0;24)$ & 7 & $(8,3)$ & 4 & 8 & YES & YES & YES & $1.29$ & $(2,3)$ & -- & 9522\\
$(g;0,1,0;24)$ & 7 & $(12,5)$ & 5 & 12 & YES & YES & YES & $1.29$ & $(2,3)$ & -- & 9523\\
$(g;0,1,0;24)$ & 7 & $(13,5)$ & 5 & 1 & YES & YES & YES & $1.43$ & $(2,3)$ & -- & 9524\\
$(g;0,1,0;24)$ & 7 & $(17,5)$ & 6 & 1 & YES & YES & YES & $1.43$ & $(2,3)$ & -- & 9525\\
$(g;0,1,0;24)$ & 7 & $(17,7)$ & 6 & 1 & YES & YES & YES & $1.57$ & $(2,3)$ & -- & 9526\\
$(g;0,1,1;33)$ & 8 & $(7,2)$ & 4 & 1 & YES & YES & YES & $1.43$ & $(2,3)$ & -- & 9527\\
$(g;0,1,1;33)$ & 8 & $(8,3)$ & 4 & 1 & YES & YES & NO(3) & $1.29$ & $(2,3)$ & -- & 9528\\
$(g;0,1,1;33)$ & 8 & $(10,3)$ & 5 & 1 & YES & YES & YES & $1.71$ & $(2,3)$ & -- & 9529\\
$(g;0,1,1;33)$ & 8 & $(11,4)$ & 5 & 11 & YES & YES & YES & $1.43$ & $(2,3)$ & -- & 9530\\
$(g;0,1,1;33)$ & 8 & $(12,5)$ & 5 & 3 & YES & YES & YES & $1.43$ & $(2,3)$ & -- & 9531\\
$(g;0,1,1;33)$ & 8 & $(14,5)$ & 6 & 1 & YES & YES & YES & $1.71$ & $(2,3)$ & -- & 9532\\
$(g;0,1,2;14)$ & 9 & $(7,2)$ & 4 & 7 & YES & YES & YES & $1.43$ & $(2,3)$ & -- & 9533\\
$(g;0,1,2;14)$ & 9 & $(10,3)$ & 5 & 2 & YES & YES & YES & $1.57$ & $(2,3)$ & -- & 9534\\
$(g;0,2,1;40)$ & 9 & $(7,2)$ & 4 & 1 & YES & YES & YES & $1.43$ & $(2,3)$ & -- & 9535\\
$(g;0,3,0;34)$ & 9 & $(7,3)$ & 4 & 1 & YES & YES & YES & $1.71$ & $(2,3)$ & -- & 9536\\
$(g;1,0,1;38)$ & 8 & $(4,1)$ & 3 & 2 & YES & YES & NO(2) & $1.62$ & $(2,3)$ & -- & 9537\\
$(g;1,0,1;38)$ & 8 & $(7,2)$ & 4 & 1 & YES & YES & YES & $1.43$ & $(2,3)$ & -- & 9538\\
$(g;1,0,1;38)$ & 8 & $(13,5)$ & 5 & 1 & YES & YES & YES & $1.43$ & $(2,3)$ & -- & 9539\\
$(g;1,0,2;24)$ & 9 & $(8,3)$ & 4 & 8 & YES & YES & YES & $1.57$ & $(2,3)$ & -- & 9540\\
$(g;1,1,0;9)$ & 8 & $(12,5)$ & 5 & 3 & YES & YES & YES & $1.43$ & $(2,3)$ & -- & 9541\\
$(g;1,1,1;49)$ & 9 & $(7,3)$ & 4 & 7 & YES & YES & YES & $1.43$ & $(2,3)$ & -- & 9542\\
$(g;1,1,2;31)$ & 10 & $(3,1)$ & 2 & 1 & YES & YES & YES & $1.29$ & $(2,3)$ & -- & 9543\\
$(g;1,1,2;31)$ & 10 & $(4,1)$ & 3 & 1 & YES & YES & YES & $1.29$ & $(2,3)$ & -- & 9544\\
$(g;1,2,0;11)$ & 9 & $(8,3)$ & 4 & 1 & YES & YES & YES & $1.43$ & $(2,3)$ & -- & 9545\\
$(g;1,2,1;60)$ & 10 & $(5,2)$ & 3 & 5 & YES & YES & YES & $1.43$ & $(2,3)$ & -- & 9546\\
$(g;2,0,1;10)$ & 9 & $(5,2)$ & 3 & 5 & YES & YES & YES & $1.43$ & $(2,3)$ & -- & 9547\\
$(g;2,1,0;48)$ & 9 & $(5,2)$ & 3 & 1 & YES & YES & YES & $1.29$ & $(2,3)$ & -- & 9548\\
$(g;2,1,0;48)$ & 9 & $(7,2)$ & 4 & 1 & YES & YES & YES & $1.43$ & $(2,3)$ & -- & 9549\\
$(g;2,1,0;48)$ & 9 & $(8,3)$ & 4 & 8 & YES & YES & YES & $1.57$ & $(2,3)$ & -- & 9550\\
$(g;2,1,0;48)$ & 9 & $(10,3)$ & 5 & 2 & YES & YES & YES & $1.57$ & $(2,3)$ & -- & 9551\\
$(g;2,1,2;82)$ & 11 & $(2,1)$ & 1 & 2 & YES & YES & YES & $1.57$ & $(2,3)$ & -- & 9552\\
$(g;3,0,0;23)$ & 9 & $(8,3)$ & 4 & 1 & YES & YES & YES & $1.57$ & $(2,3)$ & -- & 9553\\
$(g;3,1,0;30)$ & 10 & $(3,1)$ & 2 & 3 & YES & YES & YES & $1.29$ & $(2,3)$ & -- & 9554\\
$(g;3,1,0;30)$ & 10 & $(4,1)$ & 3 & 2 & YES & YES & YES & $1.29$ & $(2,3)$ & -- & 9555\\
$(g;3,2,0;37)$ & 11 & $(2,1)$ & 1 & 1 & YES & YES & YES & $1.57$ & $(2,3)$ & -- & 9556\\
$(h;0,0,0;6)$ & 5 & $(39,16)$ & 8 & 3 & YES & YES & YES & $1.71$ & $(2,3)$ & -- & 9557\\
$(h;0,0,0;6)$ & 5 & $(44,17)$ & 8 & 2 & YES & YES & YES & $1.57$ & $(2,3)$ & -- & 9558\\
$(h;0,1,0;8)$ & 6 & $(29,12)$ & 7 & 1 & YES & YES & YES & $1.71$ & $(2,3)$ & -- & 9559\\
$(h;0,3,0;12)$ & 8 & $(12,5)$ & 5 & 12 & YES & YES & YES & $1.43$ & $(2,3)$ & -- & 9560\\
$(i;0,0,0;9)$ & 5 & $(29,8)$ & 7 & 1 & YES & YES & YES & $1.57$ & $(2,3)$ & -- & 9561\\
$(i;0,0,0;9)$ & 5 & $(29,12)$ & 7 & 1 & YES & YES & YES & $1.43$ & $(2,3)$ & -- & 9562\\
$(i;0,0,0;9)$ & 5 & $(31,13)$ & 7 & 1 & YES & YES & NO(2) & $1.62$ & $(2,3)$ & -- & 9563\\
$(i;0,0,0;9)$ & 5 & $(43,18)$ & 8 & 1 & YES & YES & YES & $1.43$ & $(2,3)$ & -- & 9564\\
$(i;0,0,0;9)$ & 5 & $(46,17)$ & 8 & 1 & YES & YES & NO(2) & $1.50$ & $(2,3)$ & -- & 9565\\
$(i;0,0,0;9)$ & 5 & $(46,19)$ & 8 & 1 & YES & YES & YES & $1.43$ & $(2,3)$ & -- & 9566\\
$(i;0,0,0;9)$ & 5 & $(47,14)$ & 9 & 1 & YES & YES & YES & $1.43$ & $(2,3)$ & -- & 9567\\
$(i;0,0,0;9)$ & 5 & $(50,21)$ & 8 & 1 & YES & YES & YES & $1.43$ & $(2,3)$ & -- & 9568\\
$(i;0,0,0;9)$ & 5 & $(51,14)$ & 9 & 3 & YES & YES & YES & $1.57$ & $(2,3)$ & -- & 9569\\
$(i;0,0,0;9)$ & 5 & $(56,15)$ & 9 & 1 & YES & YES & YES & $1.57$ & $(2,3)$ & -- & 9570\\
$(i;0,0,0;9)$ & 5 & $(59,18)$ & 9 & 1 & YES & YES & YES & $1.43$ & $(2,3)$ & -- & 9571\\
$(i;0,0,0;9)$ & 5 & $(62,23)$ & 9 & 1 & YES & YES & YES & $1.71$ & $(2,3)$ & -- & 9572\\
$(i;0,0,0;9)$ & 5 & $(63,17)$ & 9 & 9 & YES & YES & YES & $1.71$ & $(2,3)$ & -- & 9573\\
$(i;0,0,0;9)$ & 5 & $(69,16)$ & 11 & 3 & YES & YES & YES & $1.71$ & $(2,3)$ & -- & 9574\\
$(i;0,0,0;9)$ & 5 & $(77,18)$ & 10 & 1 & YES & YES & YES & $1.57$ & $(2,3)$ & -- & 9575\\
$(i;0,0,0;9)$ & 5 & $(93,22)$ & 11 & 3 & YES & YES & YES & $1.57$ & $(2,3)$ & -- & 9576\\
$(i;0,1,0;12)$ & 6 & $(18,5)$ & 6 & 6 & YES & YES & YES & $1.43$ & $(2,3)$ & -- & 9577\\
$(i;0,1,0;12)$ & 6 & $(23,9)$ & 7 & 1 & YES & YES & YES & $1.43$ & $(2,3)$ & -- & 9578\\
$(i;0,1,0;12)$ & 6 & $(25,11)$ & 7 & 1 & YES & YES & YES & $1.43$ & $(2,3)$ & -- & 9579\\
$(i;0,1,0;12)$ & 6 & $(29,12)$ & 7 & 1 & YES & YES & YES & $1.71$ & $(2,3)$ & -- & 9580\\
$(i;0,1,0;12)$ & 6 & $(31,9)$ & 8 & 1 & YES & YES & YES & $1.71$ & $(2,3)$ & -- & 9581\\
$(i;0,1,0;12)$ & 6 & $(33,10)$ & 8 & 3 & YES & YES & YES & $1.57$ & $(2,3)$ & -- & 9582\\
$(i;0,1,0;12)$ & 6 & $(36,11)$ & 8 & 12 & YES & YES & YES & $1.57$ & $(2,3)$ & -- & 9583\\
$(i;0,2,0;15)$ & 7 & $(21,8)$ & 6 & 3 & YES & YES & YES & $1.43$ & $(2,3)$ & -- & 9584\\
$(j;0,0,0;8)$ & 5 & $(30,11)$ & 7 & 2 & YES & YES & NO(2) & $1.67$ & $(2,3)$ & -- & 9585\\
$(j;0,0,0;8)$ & 5 & $(40,11)$ & 8 & 8 & YES & YES & YES & $1.57$ & $(2,3)$ & -- & 9586\\
$(j;0,0,0;8)$ & 5 & $(41,12)$ & 8 & 1 & YES & YES & YES & $1.43$ & $(2,3)$ & -- & 9587\\
$(j;0,0,0;8)$ & 5 & $(45,13)$ & 10 & 1 & YES & YES & YES & $1.57$ & $(2,3)$ & -- & 9588\\
$(j;0,0,0;8)$ & 5 & $(45,17)$ & 9 & 1 & YES & YES & YES & $1.71$ & $(2,3)$ & -- & 9589\\
$(j;0,0,0;8)$ & 5 & $(46,19)$ & 8 & 2 & YES & YES & NO(2) & $1.43$ & $(4,2)$ & -- & 9590\\
$(j;0,0,0;8)$ & 5 & $(47,14)$ & 9 & 1 & YES & YES & YES & $1.43$ & $(2,3)$ & -- & 9591\\
$(j;0,0,0;8)$ & 5 & $(52,19)$ & 9 & 4 & YES & YES & NO(2) & $1.75$ & $(2,3)$ & -- & 9592\\
$(j;0,0,0;8)$ & 5 & $(53,16)$ & 10 & 1 & YES & YES & NO(2) & $1.62$ & $(2,3)$ & -- & 9593\\
$(j;0,0,0;8)$ & 5 & $(53,22)$ & 9 & 1 & YES & YES & NO(2) & $1.75$ & $(2,3)$ & -- & 9594\\
$(j;0,0,0;8)$ & 5 & $(55,21)$ & 8 & 1 & YES & YES & YES & $1.43$ & $(2,3)$ & -- & 9595\\
$(j;0,0,0;8)$ & 5 & $(55,23)$ & 9 & 1 & YES & YES & NO(2) & $1.75$ & $(2,3)$ & -- & 9596\\
$(j;0,0,0;8)$ & 5 & $(56,23)$ & 9 & 8 & YES & YES & NO(2) & $1.62$ & $(2,3)$ & -- & 9597\\
$(j;0,0,0;8)$ & 5 & $(57,17)$ & 10 & 1 & YES & YES & YES & $1.43$ & $(2,3)$ & -- & 9598\\
$(j;0,0,0;8)$ & 5 & $(59,26)$ & 9 & 1 & YES & YES & YES & $1.57$ & $(2,3)$ & -- & 9599\\
$(j;0,0,0;8)$ & 5 & $(60,23)$ & 9 & 4 & YES & YES & YES & $1.71$ & $(2,3)$ & -- & 9600\\
$(j;0,0,0;8)$ & 5 & $(61,25)$ & 9 & 1 & YES & YES & YES & $1.57$ & $(2,3)$ & -- & 9601\\
$(j;0,0,0;8)$ & 5 & $(62,17)$ & 10 & 2 & YES & YES & YES & $1.57$ & $(2,3)$ & -- & 9602\\
$(j;0,0,0;8)$ & 5 & $(63,26)$ & 9 & 1 & YES & YES & YES & $1.57$ & $(2,3)$ & -- & 9603\\
$(j;0,0,0;8)$ & 5 & $(64,27)$ & 9 & 8 & YES & YES & YES & $1.71$ & $(2,3)$ & -- & 9604\\
$(j;0,0,0;8)$ & 5 & $(65,19)$ & 9 & 1 & YES & YES & YES & $1.57$ & $(2,3)$ & -- & 9605\\
$(j;0,0,0;8)$ & 5 & $(66,25)$ & 9 & 2 & YES & YES & YES & $1.57$ & $(2,3)$ & -- & 9606\\
$(j;0,0,0;8)$ & 5 & $(71,27)$ & 9 & 1 & YES & YES & YES & $1.71$ & $(2,3)$ & -- & 9607\\
$(j;0,0,0;8)$ & 5 & $(72,17)$ & 11 & 8 & YES & YES & NO(2) & $1.50$ & $(2,3)$ & -- & 9608\\
$(j;0,0,0;8)$ & 5 & $(72,19)$ & 10 & 8 & YES & YES & YES & $1.57$ & $(2,3)$ & -- & 9609\\
$(j;0,0,0;8)$ & 5 & $(76,23)$ & 10 & 4 & YES & YES & YES & $1.71$ & $(2,3)$ & -- & 9610\\
$(j;0,0,0;8)$ & 5 & $(76,29)$ & 9 & 4 & YES & YES & YES & $1.57$ & $(2,3)$ & -- & 9611\\
$(j;0,0,0;8)$ & 5 & $(76,33)$ & 10 & 4 & YES & YES & YES & $1.71$ & $(2,3)$ & -- & 9612\\
$(j;0,0,0;8)$ & 5 & $(79,17)$ & 11 & 1 & YES & YES & YES & $1.71$ & $(2,3)$ & -- & 9613\\
$(j;0,0,0;8)$ & 5 & $(79,23)$ & 10 & 1 & YES & YES & YES & $1.57$ & $(2,3)$ & -- & 9614\\
$(j;0,0,0;8)$ & 5 & $(83,30)$ & 10 & 1 & YES & YES & YES & $1.71$ & $(2,3)$ & -- & 9615\\
$(j;0,0,0;8)$ & 5 & $(89,26)$ & 10 & 1 & YES & YES & YES & $1.57$ & $(2,3)$ & -- & 9616\\
$(j;0,0,0;8)$ & 5 & $(94,41)$ & 10 & 2 & YES & YES & YES & $1.57$ & $(2,3)$ & -- & 9617\\
$(j;0,0,0;8)$ & 5 & $(131,30)$ & 11 & 1 & YES & YES & YES & $1.57$ & $(2,3)$ & -- & 9618\\
$(j;0,1,0;10)$ & 6 & $(29,12)$ & 7 & 1 & YES & YES & NO(2) & $1.50$ & $(2,3)$ & -- & 9619\\
$(j;0,1,0;10)$ & 6 & $(31,9)$ & 8 & 1 & YES & YES & NO(2) & $1.50$ & $(2,3)$ & -- & 9620\\
$(j;0,1,0;10)$ & 6 & $(39,16)$ & 8 & 1 & YES & YES & NO(2) & $1.50$ & $(2,3)$ & -- & 9621\\
$(j;0,1,0;10)$ & 6 & $(45,16)$ & 9 & 5 & YES & YES & YES & $1.71$ & $(2,3)$ & -- & 9622\\
$(j;0,1,0;10)$ & 6 & $(45,17)$ & 9 & 5 & YES & YES & YES & $1.57$ & $(2,3)$ & -- & 9623\\
$(j;0,1,0;10)$ & 6 & $(53,19)$ & 9 & 1 & YES & YES & YES & $1.57$ & $(2,3)$ & -- & 9624\\
$(j;0,1,0;10)$ & 6 & $(53,22)$ & 9 & 1 & YES & YES & YES & $1.71$ & $(2,3)$ & -- & 9625\\
$(j;0,1,0;10)$ & 6 & $(56,23)$ & 9 & 2 & YES & YES & YES & $1.57$ & $(2,3)$ & -- & 9626\\
$(j;0,1,0;10)$ & 6 & $(64,19)$ & 9 & 2 & YES & YES & YES & $1.43$ & $(2,3)$ & -- & 9627\\
$(j;0,2,0;12)$ & 7 & $(31,9)$ & 8 & 1 & YES & YES & YES & $1.57$ & $(2,3)$ & -- & 9628\\
$(j;0,2,0;12)$ & 7 & $(37,11)$ & 8 & 1 & YES & YES & YES & $1.43$ & $(2,3)$ & -- & 9629\\
$(j;0,2,0;12)$ & 7 & $(41,12)$ & 8 & 1 & YES & YES & YES & $1.43$ & $(2,3)$ & -- & 9630
\end{longtable}
\subsection{2 chains, $K^2 = 4$}
\begin{longtable}{|c|c|c|c|c|c|c|c|c|c|c|c|}
\hline
\multicolumn{12}{|c|}{2 chains, $K^2 = 4$}\\
\hline
$(n,a)$ & Len & $(n,a)$ & Len & GCD & Nef & $\mathbb Q$-ef & Obs 0 & $\overline c_1^2 / \overline c_2$ & $(P,K)$ & WH & Index\\
\hline
\endfirsthead

\hline
$(n,a)$ & Len & $(n,a)$ & Len & GCD & Nef & $\mathbb Q$-ef & Obs 0 & $\overline c_1^2 / \overline c_2$ & $(P,K)$ & WH & Index\\
\hline
\endhead
\hline
\endfoot

$(208,79)$ & 11 & $(7,2)$ & 4 & 1 & YES & YES & NO(3) & $1.83$ & $(2,4)$ & -- & 9631
\end{longtable}



% %%%%%%%%%%%%%%%%%%%%%%%%%%%%%%%%%%%%%%%%%%%
% \section{Extra: $I_8 + 4I_1$}

% Input:
% \lstinputlisting[language=config]{../Tests/81111.txt}
% Result:
% \input{../summary/81111.tex}
\end{document}
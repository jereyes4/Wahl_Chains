\documentclass{article}
\usepackage[utf8]{inputenc}
\usepackage{longtable}
\usepackage{verbatim}
\usepackage{amsmath}
\usepackage{amssymb}
\usepackage{hyperref}
\usepackage{geometry}

\usepackage{listings}
\usepackage{xcolor}

\geometry{left=7em, bottom=7em, right=7em, tmargin=7em, headheight=7em}
%New colors defined below
\definecolor{codegreen}{rgb}{0,0.6,0}
\definecolor{codegray}{rgb}{0.5,0.5,0.5}
\definecolor{codepurple}{rgb}{0.58,0,0.82}
\definecolor{backcolour}{rgb}{0.95,0.95,0.92}

%Code listing style named "mystyle"

\lstdefinelanguage{config}{
  morekeywords={Try,Fix,Ign,Dis,Di1,Di2,Di3,Di4,Di5,Di6,Di7,Di8,Di9,global,print},
  morecomment=[l]{\#},
}
\lstdefinestyle{mystyle}{
  backgroundcolor=\color{backcolour},
  commentstyle=\color{codegreen},
  keywordstyle=\color{magenta},
  numberstyle=\tiny\color{codegray},
  stringstyle=\color{codegray},
  basicstyle=\ttfamily\footnotesize,
  breakatwhitespace=false,         
  breaklines=true,                 
  captionpos=b,                    
  keepspaces=true,                 
  numbers=left,                    
  numbersep=5pt,                  
  showspaces=false,                
  showstringspaces=false,
  showtabs=false,                  
  tabsize=2,
  inputencoding=utf8,
  extendedchars=true,
  literate={á}{{\'a}}1 {é}{{\'e}}1 {í}{{\'i}}1 {ó}{{\'o}}1 {ú}{{\'u}}1
}
\lstset{style=mystyle}


\setlength{\parindent}{0pt}
\title{Index of single and double Wahl singularities by blowing up extremal elliptic surfaces}
\author{\ }
\date{\today}

\newcommand{\C}{\mathbb{C}}

\begin{document}

\maketitle

\tableofcontents

%%%%%%%%%%%%%%%%%%%%%%%%%%%%%%%%%%%%%%%%%%%%%
\section{$I_9 + 3I_1$}
Fibration given by the pencil
\[F_{\lambda} = y^3 - zx^2 + z^2x + 3\lambda xyz.\]
The nine exceptionals are as follows:
\begin{itemize}
  \item $E_1$ - $E_4$ at $[0,0,1]$.
  \item $E_5$ - $E_8$ at $[1,0,0]$.
  \item $E_9$ at $[1,0,1]$.
\end{itemize}
Let $w$ be a primite third root of unity, then singular fibers are as follows:
\begin{itemize}
  \item $\lambda = \infty$: $I_9$ fiber given by $y$, $E_1$, $E_2$, $E_3$, $x$, $z$, $E_7$, $E_6$ $E_5$ in order.
  \item $\lambda = -1$: $I_1$ fiber called $F_3$ with node at $[-1,-1,1]$.
  \item $\lambda = -w$: $I_1$ fiber called $F_1$ with node at $[-1,-w^2,1]$.
  \item $\lambda = -w^2$: $I_1$ fiber called $F_2$ with node at $[-1,-w,1]$.
\end{itemize}
Extra curves:
\begin{itemize}
  \item $H = x-z$, a triple section that passes through all nodes of the $I_1$'s and through the intersection of $x$ and $z$.
  \item $K = x+z$, double section through $x \cap z$ and $[1,0,1]$.
  \item $T_i = y - w^{2i}x$, $i=1,2,3$, double section through $[-1,-w^{2i},1]$ and $[0,0,1]$.
  \item $S_i = y + w^{2i}z$, $i=1,2,3$, double section through $[-1,-w^{2i},1]$ and $[1,0,0]$.
  \item $R_i = 2y + w^{2i}(z-x)$, $i=1,2,3$, double section through $[-1,-w^{2i},1]$ and $[1,0,1]$.
\end{itemize}
Input:
%%\lstinputlisting[language=config]{Tests/9111_alt.txt}
%\lstinputlisting[language=config]{../Tests/9111.txt}
Result:
%%\input{summary/Exp_9111_new}
%%\usepackage{longtable}
\subsection{1 chain, $K^2 = 1$}
\begin{longtable}{|c|c|c|c|c|c|c|c|}
\hline
\multicolumn{8}{|c|}{1 chain, $K^2 = 1$}\\
\hline
$(n,a)$ & Len & Nef & $\mathbb Q$-ef & Obs 0 & $\overline c_1^2 / \overline c_2$ & $(P,K)$ & Index\\
\hline
\endfirsthead

\hline
$(n,a)$ & Len & Nef & $\mathbb Q$-ef & Obs 0 & $\overline c_1^2 / \overline c_2$ & $(P,K)$ & Index\\
\hline
\endhead
\hline
\endfoot

$(13,5)$ & 5 & YES & YES & YES & $0.64$ & $(1,1)$ & 1\\
$(13,4)$ & 6 & YES & YES & YES & $0.67$ & $(3,0)$ & 2\\
$(14,5)$ & 6 & YES & YES & YES & $0.75$ & $(1,1)$ & 3\\
$(16,5)$ & 7 & YES & YES & YES & $0.93$ & $(1,1)$ & 4\\
$(16,7)$ & 6 & YES & YES & YES & $0.83$ & $(1,1)$ & 5\\
$(17,7)$ & 6 & YES & YES & YES & $0.85$ & $(1,1)$ & 6\\
$(19,5)$ & 7 & YES & YES & YES & $0.85$ & $(1,1)$ & 7\\
$(19,8)$ & 6 & YES & YES & YES & $0.85$ & $(1,1)$ & 8\\
$(21,5)$ & 8 & YES & YES & YES & $0.85$ & $(1,1)$ & 9\\
$(24,5)$ & 8 & YES & YES & YES & $0.75$ & $(1,1)$ & 10\\
$(26,7)$ & 7 & YES & YES & YES & $0.77$ & $(1,1)$ & 11\\
$(30,7)$ & 8 & YES & YES & YES & $0.67$ & $(1,1)$ & 12\\
$(a;1,0,0;13)$ & 5 & YES & YES & YES & $0.85$ & $(1,1)$ & 13\\
$(j;0,0,0;8)$ & 5 & YES & YES & YES & $0.55$ & $(1,1)$ & 14\\
$(j;0,1,0;10)$ & 6 & YES & YES & YES & $0.67$ & $(1,1)$ & 15
\end{longtable}
\subsection{1 chain, $K^2 = 2$}
\begin{longtable}{|c|c|c|c|c|c|c|c|}
\hline
\multicolumn{8}{|c|}{1 chain, $K^2 = 2$}\\
\hline
$(n,a)$ & Len & Nef & $\mathbb Q$-ef & Obs 0 & $\overline c_1^2 / \overline c_2$ & $(P,K)$ & Index\\
\hline
\endfirsthead

\hline
$(n,a)$ & Len & Nef & $\mathbb Q$-ef & Obs 0 & $\overline c_1^2 / \overline c_2$ & $(P,K)$ & Index\\
\hline
\endhead
\hline
\endfoot

$(27,5)$ & 8 & YES & YES & YES & $0.78$ & $(1,2)$ & 16\\
$(28,5)$ & 8 & YES & YES & YES & $0.78$ & $(1,2)$ & 17\\
$(32,7)$ & 8 & YES & YES & YES & $0.67$ & $(5,0)$ & 18\\
$(34,9)$ & 8 & YES & YES & YES & $0.89$ & $(1,2)$ & 19\\
$(36,11)$ & 8 & YES & YES & YES & $1.09$ & $(1,2)$ & 20\\
$(39,14)$ & 8 & YES & YES & YES & $0.89$ & $(1,2)$ & 21\\
$(40,17)$ & 9 & YES & YES & YES & $1.31$ & $(1,2)$ & 22\\
$(41,19)$ & 10 & YES & YES & YES & $0.89$ & $(1,2)$ & 23\\
$(41,15)$ & 8 & YES & YES & YES & $0.90$ & $(3,1)$ & 24\\
$(42,19)$ & 9 & YES & YES & YES & $0.89$ & $(3,1)$ & 25\\
$(44,17)$ & 8 & YES & YES & YES & $1.08$ & $(1,2)$ & 26\\
$(48,17)$ & 9 & YES & YES & YES & $0.78$ & $(3,1)$ & 27\\
$(49,18)$ & 8 & YES & YES & YES & $1.08$ & $(1,2)$ & 28\\
$(49,20)$ & 9 & YES & YES & YES & $1.09$ & $(1,2)$ & 29\\
$(51,20)$ & 9 & YES & YES & YES & $1.09$ & $(1,2)$ & 30\\
$(53,19)$ & 9 & YES & YES & YES & $1.09$ & $(1,2)$ & 31\\
$(64,17)$ & 10 & YES & YES & YES & $1.15$ & $(1,2)$ & 32\\
$(64,23)$ & 9 & YES & YES & YES & $0.67$ & $(3,1)$ & 33\\
$(72,13)$ & 12 & YES & YES & YES & $0.67$ & $(3,1)$ & 34\\
$(79,14)$ & 11 & YES & YES & YES & $1.08$ & $(1,2)$ & 35\\
$(89,17)$ & 12 & YES & YES & YES & $1.00$ & $(1,2)$ & 36\\
$(a;3,0,1;31)$ & 8 & YES & YES & YES & $1.09$ & $(1,2)$ & 37\\
$(c;0,3,1;23)$ & 8 & YES & YES & YES & $0.80$ & $(3,1)$ & 38\\
$(c;0,3,2;29)$ & 9 & YES & YES & YES & $1.00$ & $(1,2)$ & 39\\
$(d;0,0,3;22)$ & 8 & YES & YES & YES & $0.90$ & $(1,2)$ & 40\\
$(d;0,2,2;13)$ & 9 & YES & YES & YES & $1.00$ & $(1,2)$ & 41\\
$(i;0,3,0;18)$ & 8 & YES & YES & YES & $0.67$ & $(3,1)$ & 42
\end{longtable}
\subsection{1 chain, $K^2 = 3$}
\begin{longtable}{|c|c|c|c|c|c|c|c|}
\hline
\multicolumn{8}{|c|}{1 chain, $K^2 = 3$}\\
\hline
$(n,a)$ & Len & Nef & $\mathbb Q$-ef & Obs 0 & $\overline c_1^2 / \overline c_2$ & $(P,K)$ & Index\\
\hline
\endfirsthead

\hline
$(n,a)$ & Len & Nef & $\mathbb Q$-ef & Obs 0 & $\overline c_1^2 / \overline c_2$ & $(P,K)$ & Index\\
\hline
\endhead
\hline
\endfoot

$(71,21)$ & 9 & YES & YES & NO(2) & $1.27$ & $(3,2)$ & 43\\
$(73,11)$ & 11 & YES & YES & YES & $1.12$ & $(3,2)$ & 44\\
$(83,24)$ & 11 & YES & YES & YES & $1.25$ & $(5,1)$ & 45\\
$(84,25)$ & 10 & YES & YES & YES & $1.40$ & $(1,3)$ & 46\\
$(85,24)$ & 11 & YES & YES & YES & $1.25$ & $(5,1)$ & 47\\
$(91,25)$ & 10 & YES & YES & YES & $1.40$ & $(1,3)$ & 48\\
$(95,43)$ & 11 & YES & YES & YES & $1.12$ & $(1,3)$ & 49\\
$(97,41)$ & 10 & YES & YES & NO(2) & $1.36$ & $(3,2)$ & 50\\
$(101,37)$ & 10 & YES & YES & NO(2) & $1.27$ & $(3,2)$ & 51\\
$(104,45)$ & 11 & YES & YES & YES & $1.40$ & $(1,3)$ & 52\\
$(111,32)$ & 13 & YES & YES & YES & $1.25$ & $(5,1)$ & 53\\
$(113,32)$ & 13 & YES & YES & YES & $1.25$ & $(5,1)$ & 54\\
$(113,48)$ & 11 & YES & YES & YES & $1.40$ & $(1,3)$ & 55\\
$(115,52)$ & 11 & YES & YES & YES & $1.12$ & $(1,3)$ & 56\\
$(119,45)$ & 11 & YES & YES & YES & $1.12$ & $(3,2)$ & 57\\
$(124,35)$ & 12 & YES & YES & YES & $1.42$ & $(1,3)$ & 58\\
$(125,46)$ & 12 & YES & YES & YES & $1.55$ & $(1,3)$ & 59\\
$(127,48)$ & 11 & YES & YES & YES & $1.33$ & $(1,3)$ & 60\\
$(129,59)$ & 12 & YES & YES & YES & $1.50$ & $(1,3)$ & 61\\
$(132,47)$ & 12 & YES & YES & YES & $1.12$ & $(3,2)$ & 62\\
$(138,49)$ & 12 & YES & YES & YES & $1.45$ & $(1,3)$ & 63\\
$(140,53)$ & 11 & YES & YES & YES & $1.40$ & $(1,3)$ & 64\\
$(144,61)$ & 11 & YES & YES & YES & $1.30$ & $(1,3)$ & 65\\
$(145,42)$ & 12 & YES & YES & YES & $1.45$ & $(1,3)$ & 66\\
$(148,65)$ & 11 & YES & YES & YES & $1.36$ & $(1,3)$ & 67\\
$(151,53)$ & 12 & YES & YES & YES & $1.12$ & $(5,1)$ & 68\\
$(154,57)$ & 12 & YES & YES & YES & $1.50$ & $(1,3)$ & 69\\
$(161,51)$ & 13 & YES & YES & YES & $1.12$ & $(5,1)$ & 70\\
$(163,43)$ & 12 & YES & YES & YES & $1.40$ & $(1,3)$ & 71\\
$(175,62)$ & 12 & YES & YES & YES & $1.50$ & $(1,3)$ & 72\\
$(177,47)$ & 12 & YES & YES & YES & $1.40$ & $(1,3)$ & 73\\
$(178,47)$ & 12 & YES & YES & YES & $1.36$ & $(1,3)$ & 74\\
$(181,65)$ & 12 & YES & YES & YES & $1.45$ & $(1,3)$ & 75\\
$(187,42)$ & 13 & YES & YES & YES & $1.22$ & $(1,3)$ & 76\\
$(188,59)$ & 13 & YES & YES & YES & $1.50$ & $(1,3)$ & 77\\
$(196,45)$ & 13 & YES & YES & YES & $1.36$ & $(1,3)$ & 78\\
$(197,61)$ & 13 & YES & YES & YES & $1.45$ & $(1,3)$ & 79\\
$(239,32)$ & 17 & YES & YES & YES & $1.12$ & $(5,1)$ & 80\\
$(251,46)$ & 15 & YES & YES & YES & $1.45$ & $(1,3)$ & 81\\
$(257,40)$ & 15 & YES & YES & YES & $1.22$ & $(1,3)$ & 82\\
$(265,41)$ & 16 & YES & YES & YES & $1.50$ & $(1,3)$ & 83\\
$(b;0,4,1;48)$ & 10 & YES & YES & YES & $1.42$ & $(1,3)$ & 84\\
$(g;0,4,0;39)$ & 10 & YES & YES & YES & $1.40$ & $(1,3)$ & 85
\end{longtable}
\subsection{1 chain, $K^2 = 4$}
\begin{longtable}{|c|c|c|c|c|c|c|c|}
\hline
\multicolumn{8}{|c|}{1 chain, $K^2 = 4$}\\
\hline
$(n,a)$ & Len & Nef & $\mathbb Q$-ef & Obs 0 & $\overline c_1^2 / \overline c_2$ & $(P,K)$ & Index\\
\hline
\endfirsthead

\hline
$(n,a)$ & Len & Nef & $\mathbb Q$-ef & Obs 0 & $\overline c_1^2 / \overline c_2$ & $(P,K)$ & Index\\
\hline
\endhead
\hline
\endfoot

$(251,74)$ & 13 & YES & YES & YES & $1.78$ & $(1,4)$ & 86\\
$(289,101)$ & 15 & YES & YES & YES & $1.75$ & $(3,3)$ & 87\\
$(294,85)$ & 14 & YES & YES & YES & $1.67$ & $(1,4)$ & 88\\
$(311,132)$ & 14 & YES & YES & YES & $1.78$ & $(1,4)$ & 89\\
$(336,137)$ & 14 & YES & YES & YES & $1.57$ & $(3,3)$ & 90\\
$(337,138)$ & 14 & YES & YES & YES & $1.78$ & $(1,4)$ & 91\\
$(392,53)$ & 20 & YES & YES & YES & $1.71$ & $(3,3)$ & 92\\
$(404,107)$ & 15 & YES & YES & YES & $1.90$ & $(1,4)$ & 93\\
$(563,91)$ & 18 & YES & YES & YES & $1.82$ & $(1,4)$ & 94\\
$(870,269)$ & 16 & YES & YES & YES & $2.00$ & $(1,4)$ & 95\\
$(945,388)$ & 15 & YES & YES & YES & $2.00$ & $(1,4)$ & 96
\end{longtable}
\subsection{2 chains, $K^2 = 1$}
\begin{longtable}{|c|c|c|c|c|c|c|c|c|c|c|c|}
\hline
\multicolumn{12}{|c|}{2 chains, $K^2 = 1$}\\
\hline
$(n,a)$ & Len & $(n,a)$ & Len & GCD & Nef & $\mathbb Q$-ef & Obs 0 & $\overline c_1^2 / \overline c_2$ & $(P,K)$ & WH & Index\\
\hline
\endfirsthead

\hline
$(n,a)$ & Len & $(n,a)$ & Len & GCD & Nef & $\mathbb Q$-ef & Obs 0 & $\overline c_1^2 / \overline c_2$ & $(P,K)$ & WH & Index\\
\hline
\endhead
\hline
\endfoot

$(7,3)$ & 4 & $(5,1)$ & 4 & 1 & YES & YES & YES & $0.56$ & $(4,0)$ & NO & 97\\
$(7,3)$ & 4 & $(5,1)$ & 4 & 1 & YES & YES & YES & $0.56$ & $(4,0)$ & NO & 98\\
$(7,3)$ & 4 & $(7,2)$ & 4 & 7 & YES & YES & YES & $0.82$ & $(2,1)$ & NO & 99\\
$(7,3)$ & 4 & $(7,2)$ & 4 & 7 & YES & YES & YES & $0.82$ & $(2,1)$ & -- & 100\\
$(7,3)$ & 4 & $(7,2)$ & 4 & 7 & YES & YES & YES & $0.82$ & $(2,1)$ & NO & 101\\
$(7,3)$ & 4 & $(7,3)$ & 4 & 7 & YES & YES & YES & $0.91$ & $(2,1)$ & NO & 102\\
$(8,3)$ & 4 & $(7,3)$ & 4 & 1 & YES & YES & YES & $0.82$ & $(2,1)$ & NO & 103\\
$(8,3)$ & 4 & $(7,3)$ & 4 & 1 & YES & YES & YES & $0.82$ & $(2,1)$ & -- & 104\\
$(8,3)$ & 4 & $(7,3)$ & 4 & 1 & YES & YES & YES & $0.82$ & $(2,1)$ & NO & 105\\
$(9,2)$ & 5 & $(4,1)$ & 3 & 1 & YES & YES & YES & $0.44$ & $(2,1)$ & -- & 106\\
$(9,2)$ & 5 & $(4,1)$ & 3 & 1 & YES & YES & YES & $0.56$ & $(2,1)$ & NO & 107\\
$(9,4)$ & 5 & $(4,1)$ & 3 & 1 & YES & YES & YES & $1.00$ & $(2,1)$ & NO & 108\\
$(9,4)$ & 5 & $(4,1)$ & 3 & 1 & YES & YES & YES & $1.00$ & $(2,1)$ & NO & 109\\
$(9,2)$ & 5 & $(5,1)$ & 4 & 1 & YES & YES & YES & $0.56$ & $(2,1)$ & NO & 110\\
$(9,2)$ & 5 & $(5,1)$ & 4 & 1 & YES & YES & YES & $0.56$ & $(2,1)$ & NO & 111\\
$(9,2)$ & 5 & $(5,1)$ & 4 & 1 & YES & YES & YES & $0.56$ & $(2,1)$ & -- & 112\\
$(9,4)$ & 5 & $(5,2)$ & 3 & 1 & YES & YES & YES & $0.80$ & $(2,1)$ & NO & 113\\
$(9,2)$ & 5 & $(7,3)$ & 4 & 1 & YES & YES & YES & $0.82$ & $(2,1)$ & NO & 114\\
$(9,2)$ & 5 & $(7,3)$ & 4 & 1 & YES & YES & YES & $0.82$ & $(2,1)$ & -- & 115\\
$(9,4)$ & 5 & $(7,2)$ & 4 & 1 & YES & YES & YES & $0.82$ & $(2,1)$ & NO & 116\\
$(9,4)$ & 5 & $(8,3)$ & 4 & 1 & YES & YES & YES & $0.82$ & $(2,1)$ & 185 & 117\\
$(10,3)$ & 5 & $(4,1)$ & 3 & 2 & YES & YES & YES & $0.60$ & $(2,1)$ & 128 & 118\\
$(10,3)$ & 5 & $(4,1)$ & 3 & 2 & YES & YES & YES & $0.60$ & $(2,1)$ & -- & 119\\
$(10,3)$ & 5 & $(5,1)$ & 4 & 5 & YES & YES & YES & $0.60$ & $(2,1)$ & -- & 120\\
$(10,3)$ & 5 & $(5,1)$ & 4 & 5 & YES & YES & YES & $0.70$ & $(2,1)$ & NO & 121\\
$(10,3)$ & 5 & $(5,2)$ & 3 & 5 & YES & YES & YES & $0.83$ & $(2,1)$ & -- & 122\\
$(11,2)$ & 6 & $(2,1)$ & 1 & 1 & YES & YES & YES & $0.67$ & $(2,1)$ & NO & 123\\
$(11,3)$ & 5 & $(2,1)$ & 1 & 1 & YES & YES & YES & $0.60$ & $(4,0)$ & -- & 124\\
$(11,3)$ & 5 & $(2,1)$ & 1 & 1 & YES & YES & YES & $0.70$ & $(2,1)$ & NO & 125\\
$(11,4)$ & 5 & $(2,1)$ & 1 & 1 & YES & YES & YES & $0.73$ & $(2,1)$ & -- & 126\\
$(11,5)$ & 6 & $(2,1)$ & 1 & 1 & YES & YES & YES & $0.73$ & $(2,1)$ & -- & 127\\
$(11,3)$ & 5 & $(3,1)$ & 2 & 1 & YES & YES & YES & $0.60$ & $(2,1)$ & 118 & 128\\
$(11,3)$ & 5 & $(3,1)$ & 2 & 1 & YES & YES & YES & $0.60$ & $(2,1)$ & -- & 129\\
$(11,4)$ & 5 & $(3,1)$ & 2 & 1 & YES & YES & YES & $0.92$ & $(2,1)$ & -- & 130\\
$(11,4)$ & 5 & $(3,1)$ & 2 & 1 & YES & YES & YES & $0.92$ & $(2,1)$ & NO & 131\\
$(11,4)$ & 5 & $(3,1)$ & 2 & 1 & YES & YES & YES & $0.82$ & $(2,1)$ & NO & 132\\
$(11,5)$ & 6 & $(3,1)$ & 2 & 1 & YES & YES & YES & $0.70$ & $(2,1)$ & -- & 133\\
$(11,5)$ & 6 & $(3,1)$ & 2 & 1 & YES & YES & YES & $0.70$ & $(2,1)$ & NO & 134\\
$(11,5)$ & 6 & $(3,1)$ & 2 & 1 & YES & YES & YES & $0.91$ & $(2,1)$ & NO & 135\\
$(11,3)$ & 5 & $(4,1)$ & 3 & 1 & YES & YES & YES & $0.60$ & $(4,0)$ & NO & 136\\
$(11,3)$ & 5 & $(4,1)$ & 3 & 1 & YES & YES & YES & $0.60$ & $(4,0)$ & -- & 137\\
$(11,3)$ & 5 & $(4,1)$ & 3 & 1 & YES & YES & YES & $0.60$ & $(4,0)$ & NO & 138\\
$(11,4)$ & 5 & $(4,1)$ & 3 & 1 & YES & YES & YES & $0.82$ & $(2,1)$ & NO & 139\\
$(11,4)$ & 5 & $(4,1)$ & 3 & 1 & YES & YES & YES & $0.82$ & $(2,1)$ & -- & 140\\
$(11,5)$ & 6 & $(4,1)$ & 3 & 1 & YES & YES & YES & $0.80$ & $(2,1)$ & NO & 141\\
$(11,5)$ & 6 & $(4,1)$ & 3 & 1 & YES & YES & YES & $0.80$ & $(2,1)$ & -- & 142\\
$(11,5)$ & 6 & $(4,1)$ & 3 & 1 & YES & YES & YES & $0.80$ & $(2,1)$ & NO & 143\\
$(11,2)$ & 6 & $(5,1)$ & 4 & 1 & YES & YES & YES & $0.56$ & $(2,1)$ & NO & 144\\
$(11,2)$ & 6 & $(5,1)$ & 4 & 1 & YES & YES & YES & $0.56$ & $(2,1)$ & NO & 145\\
$(11,2)$ & 6 & $(5,1)$ & 4 & 1 & YES & YES & YES & $0.56$ & $(2,1)$ & -- & 146\\
$(11,3)$ & 5 & $(5,1)$ & 4 & 1 & YES & YES & YES & $0.60$ & $(2,1)$ & -- & 147\\
$(11,3)$ & 5 & $(5,1)$ & 4 & 1 & YES & YES & YES & $0.70$ & $(2,1)$ & NO & 148\\
$(11,3)$ & 5 & $(5,2)$ & 3 & 1 & YES & YES & YES & $0.92$ & $(2,1)$ & NO & 149\\
$(11,3)$ & 5 & $(5,2)$ & 3 & 1 & YES & YES & YES & $0.92$ & $(2,1)$ & -- & 150\\
$(11,4)$ & 5 & $(5,2)$ & 3 & 1 & YES & YES & YES & $0.92$ & $(2,1)$ & 174 & 151\\
$(11,4)$ & 5 & $(5,2)$ & 3 & 1 & YES & YES & YES & $0.92$ & $(2,1)$ & -- & 152\\
$(11,5)$ & 6 & $(5,1)$ & 4 & 1 & YES & YES & YES & $0.91$ & $(2,1)$ & NO & 153\\
$(11,5)$ & 6 & $(5,1)$ & 4 & 1 & YES & YES & YES & $0.91$ & $(2,1)$ & NO & 154\\
$(11,5)$ & 6 & $(5,2)$ & 3 & 1 & YES & YES & YES & $0.80$ & $(2,1)$ & NO & 155\\
$(11,5)$ & 6 & $(5,2)$ & 3 & 1 & YES & YES & YES & $0.80$ & $(2,1)$ & -- & 156\\
$(11,5)$ & 6 & $(6,1)$ & 5 & 1 & YES & YES & YES & $0.80$ & $(2,1)$ & NO & 157\\
$(11,5)$ & 6 & $(6,1)$ & 5 & 1 & YES & YES & YES & $0.80$ & $(2,1)$ & NO & 158\\
$(11,5)$ & 6 & $(7,3)$ & 4 & 1 & YES & YES & YES & $0.91$ & $(2,1)$ & 182 & 159\\
$(11,4)$ & 5 & $(8,3)$ & 4 & 1 & YES & YES & YES & $0.82$ & $(2,1)$ & NO & 160\\
$(11,2)$ & 6 & $(9,4)$ & 5 & 1 & YES & YES & YES & $0.82$ & $(2,1)$ & NO & 161\\
$(11,5)$ & 6 & $(9,4)$ & 5 & 1 & YES & YES & YES & $0.80$ & $(2,1)$ & NO & 162\\
$(11,4)$ & 5 & $(11,4)$ & 5 & 11 & YES & YES & YES & $0.92$ & $(2,1)$ & NO & 163\\
$(11,5)$ & 6 & $(11,5)$ & 6 & 11 & YES & YES & YES & $0.70$ & $(2,1)$ & NO & 164\\
$(12,5)$ & 5 & $(3,1)$ & 2 & 3 & YES & YES & YES & $0.83$ & $(2,1)$ & -- & 165\\
$(12,5)$ & 5 & $(3,1)$ & 2 & 3 & YES & YES & YES & $0.92$ & $(2,1)$ & NO & 166\\
$(12,5)$ & 5 & $(3,1)$ & 2 & 3 & YES & YES & YES & $0.92$ & $(2,1)$ & NO & 167\\
$(13,3)$ & 6 & $(2,1)$ & 1 & 1 & YES & YES & YES & $0.60$ & $(4,0)$ & -- & 168\\
$(13,5)$ & 5 & $(2,1)$ & 1 & 1 & YES & YES & YES & $0.92$ & $(2,1)$ & NO & 169\\
$(13,4)$ & 6 & $(3,1)$ & 2 & 1 & YES & YES & YES & $1.00$ & $(2,1)$ & NO & 170\\
$(13,4)$ & 6 & $(3,1)$ & 2 & 1 & YES & YES & YES & $1.00$ & $(2,1)$ & -- & 171\\
$(13,5)$ & 5 & $(3,1)$ & 2 & 1 & YES & YES & YES & $0.92$ & $(2,1)$ & NO & 172\\
$(13,5)$ & 5 & $(3,1)$ & 2 & 1 & YES & YES & YES & $0.92$ & $(2,1)$ & -- & 173\\
$(13,5)$ & 5 & $(3,1)$ & 2 & 1 & YES & YES & YES & $0.92$ & $(2,1)$ & 151 & 174\\
$(13,3)$ & 6 & $(4,1)$ & 3 & 1 & YES & YES & YES & $0.60$ & $(4,0)$ & NO & 175\\
$(13,3)$ & 6 & $(4,1)$ & 3 & 1 & YES & YES & YES & $0.60$ & $(4,0)$ & -- & 176\\
$(13,3)$ & 6 & $(11,3)$ & 5 & 1 & YES & YES & YES & $0.60$ & $(2,1)$ & NO & 177\\
$(14,5)$ & 6 & $(3,1)$ & 2 & 1 & NO & YES & YES & $0.82$ & $(2,1)$ & -- & 178\\
$(14,5)$ & 6 & $(3,1)$ & 2 & 1 & YES & YES & YES & $0.82$ & $(2,1)$ & NO & 179\\
$(14,3)$ & 6 & $(5,1)$ & 4 & 1 & NO & YES & YES & $0.56$ & $(2,1)$ & -- & 180\\
$(15,4)$ & 6 & $(4,1)$ & 3 & 1 & NO & YES & YES & $0.60$ & $(4,0)$ & -- & 181\\
$(16,7)$ & 6 & $(2,1)$ & 1 & 2 & YES & YES & YES & $0.91$ & $(2,1)$ & 159 & 182\\
$(16,5)$ & 7 & $(3,1)$ & 2 & 1 & YES & YES & YES & $0.60$ & $(2,1)$ & NO & 183\\
$(16,5)$ & 7 & $(3,1)$ & 2 & 1 & NO & YES & YES & $1.00$ & $(2,1)$ & -- & 184\\
$(16,7)$ & 6 & $(3,1)$ & 2 & 1 & YES & YES & YES & $0.82$ & $(2,1)$ & 117 & 185\\
$(16,3)$ & 7 & $(5,1)$ & 4 & 1 & NO & YES & YES & $0.56$ & $(2,1)$ & NO & 186\\
$(16,3)$ & 7 & $(5,1)$ & 4 & 1 & NO & YES & YES & $0.56$ & $(2,1)$ & -- & 187\\
$(16,7)$ & 6 & $(5,1)$ & 4 & 1 & YES & YES & YES & $0.82$ & $(2,1)$ & NO & 188\\
$(16,7)$ & 6 & $(5,1)$ & 4 & 1 & YES & YES & YES & $0.82$ & $(2,1)$ & -- & 189\\
$(16,7)$ & 6 & $(5,1)$ & 4 & 1 & YES & YES & YES & $0.82$ & $(2,1)$ & NO & 190\\
$(16,5)$ & 7 & $(7,1)$ & 6 & 1 & YES & YES & YES & $0.60$ & $(2,1)$ & NO & 191\\
$(16,7)$ & 6 & $(7,3)$ & 4 & 1 & YES & YES & YES & $0.82$ & $(2,1)$ & NO & 192\\
$(16,7)$ & 6 & $(9,4)$ & 5 & 1 & YES & YES & YES & $0.82$ & $(2,1)$ & NO & 193\\
$(16,5)$ & 7 & $(13,4)$ & 6 & 1 & YES & YES & YES & $0.60$ & $(2,1)$ & NO & 194\\
$(17,7)$ & 6 & $(2,1)$ & 1 & 1 & YES & YES & YES & $0.70$ & $(2,1)$ & NO & 195\\
$(17,5)$ & 6 & $(3,1)$ & 2 & 1 & NO & YES & YES & $0.60$ & $(2,1)$ & -- & 196\\
$(17,4)$ & 7 & $(4,1)$ & 3 & 1 & NO & YES & YES & $0.60$ & $(4,0)$ & NO & 197\\
$(17,4)$ & 7 & $(4,1)$ & 3 & 1 & NO & YES & YES & $0.60$ & $(4,0)$ & -- & 198\\
$(19,4)$ & 7 & $(2,1)$ & 1 & 1 & YES & YES & YES & $0.50$ & $(2,1)$ & -- & 199\\
$(19,4)$ & 7 & $(2,1)$ & 1 & 1 & YES & YES & YES & $0.60$ & $(2,1)$ & NO & 200\\
$(19,8)$ & 6 & $(2,1)$ & 1 & 1 & YES & YES & YES & $0.70$ & $(2,1)$ & NO & 201\\
$(19,8)$ & 6 & $(2,1)$ & 1 & 1 & NO & YES & YES & $0.92$ & $(2,1)$ & -- & 202\\
$(19,5)$ & 7 & $(4,1)$ & 3 & 1 & YES & YES & YES & $0.70$ & $(2,1)$ & NO & 203\\
$(19,5)$ & 7 & $(7,1)$ & 6 & 1 & YES & YES & YES & $0.60$ & $(2,1)$ & NO & 204\\
$(19,4)$ & 7 & $(11,2)$ & 6 & 1 & YES & YES & YES & $0.73$ & $(2,1)$ & NO & 205\\
$(19,5)$ & 7 & $(11,3)$ & 5 & 1 & YES & YES & YES & $0.60$ & $(2,1)$ & 215 & 206\\
$(20,9)$ & 7 & $(2,1)$ & 1 & 2 & NO & YES & YES & $0.91$ & $(2,1)$ & -- & 207\\
$(21,8)$ & 6 & $(2,1)$ & 1 & 1 & NO & YES & YES & $0.70$ & $(2,1)$ & -- & 208\\
$(21,5)$ & 8 & $(4,1)$ & 3 & 1 & YES & YES & YES & $0.92$ & $(2,1)$ & NO & 209\\
$(23,10)$ & 7 & $(2,1)$ & 1 & 1 & NO & YES & YES & $0.70$ & $(2,1)$ & -- & 210\\
$(24,5)$ & 8 & $(5,1)$ & 4 & 1 & YES & YES & YES & $0.82$ & $(2,1)$ & NO & 211\\
$(24,5)$ & 8 & $(7,1)$ & 6 & 1 & YES & YES & YES & $0.73$ & $(2,1)$ & NO & 212\\
$(24,5)$ & 8 & $(19,4)$ & 7 & 1 & YES & YES & YES & $0.73$ & $(2,1)$ & NO & 213\\
$(25,9)$ & 7 & $(2,1)$ & 1 & 1 & NO & YES & YES & $0.67$ & $(2,1)$ & -- & 214\\
$(26,7)$ & 7 & $(4,1)$ & 3 & 2 & YES & YES & YES & $0.60$ & $(2,1)$ & 206 & 215\\
$(a;1,0,0;13)$ & 5 & $(2,1)$ & 1 & 1 & YES & YES & YES & $0.70$ & $(2,1)$ & -- & 216\\
$(a;2,0,0;17)$ & 6 & $(5,1)$ & 4 & 1 & YES & YES & YES & $0.82$ & $(2,1)$ & -- & 217\\
$(c;0,0,0;4)$ & 4 & $(3,1)$ & 2 & 1 & YES & YES & NO(3) & $0.33$ & $(2,1)$ & -- & 218\\
$(c;0,1,0;11)$ & 5 & $(2,1)$ & 1 & 1 & YES & YES & YES & $0.50$ & $(2,1)$ & -- & 219\\
$(c;0,1,1;5)$ & 6 & $(2,1)$ & 1 & 1 & YES & YES & YES & $0.73$ & $(2,1)$ & -- & 220\\
$(c;0,2,0;7)$ & 6 & $(2,1)$ & 1 & 1 & YES & YES & YES & $0.83$ & $(2,1)$ & -- & 221\\
$(f;0,0,0;6)$ & 4 & $(4,1)$ & 3 & 2 & YES & YES & YES & $0.44$ & $(4,0)$ & -- & 222\\
$(f;0,0,0;6)$ & 4 & $(5,2)$ & 3 & 1 & YES & YES & YES & $0.82$ & $(2,1)$ & -- & 223\\
$(f;0,0,0;6)$ & 4 & $(7,3)$ & 4 & 1 & YES & YES & YES & $0.82$ & $(2,1)$ & -- & 224\\
$(f;0,0,0;6)$ & 4 & $(9,2)$ & 5 & 3 & YES & YES & YES & $0.82$ & $(2,1)$ & -- & 225\\
$(f;0,1,0;7)$ & 5 & $(2,1)$ & 1 & 1 & YES & YES & YES & $0.73$ & $(2,1)$ & -- & 226\\
$(f;0,1,0;7)$ & 5 & $(3,1)$ & 2 & 1 & YES & YES & YES & $0.92$ & $(2,1)$ & -- & 227\\
$(f;0,1,0;7)$ & 5 & $(4,1)$ & 3 & 1 & YES & YES & YES & $0.82$ & $(2,1)$ & -- & 228\\
$(f;0,1,0;7)$ & 5 & $(5,1)$ & 4 & 1 & YES & YES & YES & $0.92$ & $(2,1)$ & -- & 229\\
$(j;0,0,0;8)$ & 5 & $(3,1)$ & 2 & 1 & YES & YES & YES & $0.83$ & $(2,1)$ & -- & 230\\
$(j;0,0,0;8)$ & 5 & $(5,1)$ & 4 & 1 & YES & YES & YES & $0.83$ & $(2,1)$ & -- & 231
\end{longtable}
\subsection{2 chains, $K^2 = 2$}
\begin{longtable}{|c|c|c|c|c|c|c|c|c|c|c|c|}
\hline
\multicolumn{12}{|c|}{2 chains, $K^2 = 2$}\\
\hline
$(n,a)$ & Len & $(n,a)$ & Len & GCD & Nef & $\mathbb Q$-ef & Obs 0 & $\overline c_1^2 / \overline c_2$ & $(P,K)$ & WH & Index\\
\hline
\endfirsthead

\hline
$(n,a)$ & Len & $(n,a)$ & Len & GCD & Nef & $\mathbb Q$-ef & Obs 0 & $\overline c_1^2 / \overline c_2$ & $(P,K)$ & WH & Index\\
\hline
\endhead
\hline
\endfoot

$(10,3)$ & 5 & $(8,3)$ & 4 & 2 & YES & YES & NO(2) & $1.08$ & $(2,2)$ & -- & 232\\
$(11,2)$ & 6 & $(8,3)$ & 4 & 1 & YES & YES & YES & $0.75$ & $(4,1)$ & NO & 233\\
$(11,2)$ & 6 & $(8,3)$ & 4 & 1 & YES & YES & YES & $0.75$ & $(4,1)$ & -- & 234\\
$(11,4)$ & 5 & $(9,2)$ & 5 & 1 & YES & YES & YES & $0.89$ & $(4,1)$ & -- & 235\\
$(11,2)$ & 6 & $(10,3)$ & 5 & 1 & YES & YES & YES & $0.89$ & $(6,0)$ & NO & 236\\
$(11,2)$ & 6 & $(10,3)$ & 5 & 1 & YES & YES & YES & $0.89$ & $(6,0)$ & -- & 237\\
$(11,4)$ & 5 & $(10,3)$ & 5 & 1 & YES & YES & YES & $1.20$ & $(2,2)$ & NO & 238\\
$(11,5)$ & 6 & $(10,3)$ & 5 & 1 & YES & YES & YES & $0.88$ & $(4,1)$ & NO & 239\\
$(13,4)$ & 6 & $(8,3)$ & 4 & 1 & YES & YES & YES & $1.27$ & $(2,2)$ & NO & 240\\
$(13,3)$ & 6 & $(9,4)$ & 5 & 1 & YES & YES & YES & $0.88$ & $(4,1)$ & NO & 241\\
$(13,3)$ & 6 & $(9,4)$ & 5 & 1 & YES & YES & YES & $0.88$ & $(4,1)$ & -- & 242\\
$(13,3)$ & 6 & $(9,4)$ & 5 & 1 & YES & YES & YES & $0.88$ & $(4,1)$ & NO & 243\\
$(13,4)$ & 6 & $(9,2)$ & 5 & 1 & YES & YES & YES & $1.10$ & $(2,2)$ & NO & 244\\
$(13,4)$ & 6 & $(9,2)$ & 5 & 1 & YES & YES & YES & $1.10$ & $(2,2)$ & -- & 245\\
$(13,5)$ & 5 & $(10,3)$ & 5 & 1 & YES & YES & YES & $0.89$ & $(2,2)$ & -- & 246\\
$(13,3)$ & 6 & $(11,4)$ & 5 & 1 & YES & YES & YES & $1.20$ & $(2,2)$ & NO & 247\\
$(13,3)$ & 6 & $(11,4)$ & 5 & 1 & YES & YES & YES & $1.20$ & $(2,2)$ & -- & 248\\
$(13,3)$ & 6 & $(11,4)$ & 5 & 1 & YES & YES & YES & $1.20$ & $(2,2)$ & 396 & 249\\
$(13,4)$ & 6 & $(11,5)$ & 6 & 1 & YES & YES & YES & $1.42$ & $(2,2)$ & NO & 250\\
$(13,4)$ & 6 & $(11,5)$ & 6 & 1 & YES & YES & YES & $1.42$ & $(2,2)$ & -- & 251\\
$(13,6)$ & 7 & $(11,2)$ & 6 & 1 & YES & YES & YES & $0.88$ & $(2,2)$ & NO & 252\\
$(13,6)$ & 7 & $(11,4)$ & 5 & 1 & YES & YES & YES & $0.88$ & $(4,1)$ & NO & 253\\
$(13,6)$ & 7 & $(11,4)$ & 5 & 1 & YES & YES & YES & $0.88$ & $(4,1)$ & -- & 254\\
$(14,5)$ & 6 & $(7,2)$ & 4 & 7 & YES & YES & YES & $1.00$ & $(4,1)$ & -- & 255\\
$(14,5)$ & 6 & $(9,2)$ & 5 & 1 & YES & YES & YES & $1.00$ & $(4,1)$ & NO & 256\\
$(14,5)$ & 6 & $(9,2)$ & 5 & 1 & YES & YES & YES & $1.00$ & $(4,1)$ & -- & 257\\
$(14,5)$ & 6 & $(9,2)$ & 5 & 1 & YES & YES & YES & $1.00$ & $(4,1)$ & NO & 258\\
$(14,5)$ & 6 & $(13,4)$ & 6 & 1 & YES & YES & YES & $1.12$ & $(2,2)$ & NO & 259\\
$(14,5)$ & 6 & $(13,4)$ & 6 & 1 & YES & YES & YES & $1.12$ & $(2,2)$ & -- & 260\\
$(15,4)$ & 6 & $(9,2)$ & 5 & 3 & YES & YES & YES & $1.10$ & $(2,2)$ & NO & 261\\
$(15,4)$ & 6 & $(9,2)$ & 5 & 3 & YES & YES & YES & $1.10$ & $(2,2)$ & -- & 262\\
$(15,7)$ & 8 & $(11,2)$ & 6 & 1 & YES & YES & YES & $0.88$ & $(2,2)$ & -- & 263\\
$(15,4)$ & 6 & $(13,6)$ & 7 & 1 & YES & YES & YES & $0.88$ & $(4,1)$ & NO & 264\\
$(16,5)$ & 7 & $(9,2)$ & 5 & 1 & YES & YES & YES & $1.11$ & $(2,2)$ & NO & 265\\
$(16,5)$ & 7 & $(9,2)$ & 5 & 1 & YES & YES & YES & $1.11$ & $(2,2)$ & -- & 266\\
$(16,5)$ & 7 & $(9,2)$ & 5 & 1 & YES & YES & YES & $1.11$ & $(2,2)$ & NO & 267\\
$(16,5)$ & 7 & $(11,3)$ & 5 & 1 & YES & YES & YES & $1.11$ & $(2,2)$ & NO & 268\\
$(16,5)$ & 7 & $(11,3)$ & 5 & 1 & YES & YES & YES & $1.11$ & $(2,2)$ & -- & 269\\
$(16,7)$ & 6 & $(11,4)$ & 5 & 1 & YES & YES & YES & $1.10$ & $(2,2)$ & 312 & 270\\
$(16,7)$ & 6 & $(13,6)$ & 7 & 1 & YES & YES & YES & $0.88$ & $(2,2)$ & 367 & 271\\
$(16,3)$ & 7 & $(14,5)$ & 6 & 2 & YES & YES & YES & $0.75$ & $(4,1)$ & NO & 272\\
$(16,3)$ & 7 & $(14,5)$ & 6 & 2 & YES & YES & YES & $0.75$ & $(4,1)$ & -- & 273\\
$(17,3)$ & 7 & $(3,1)$ & 2 & 1 & YES & YES & YES & $1.00$ & $(2,2)$ & NO & 274\\
$(17,7)$ & 6 & $(6,1)$ & 5 & 1 & YES & YES & YES & $1.33$ & $(2,2)$ & NO & 275\\
$(17,7)$ & 6 & $(6,1)$ & 5 & 1 & YES & YES & YES & $1.33$ & $(2,2)$ & -- & 276\\
$(17,4)$ & 7 & $(7,3)$ & 4 & 1 & YES & YES & YES & $1.27$ & $(2,2)$ & NO & 277\\
$(17,4)$ & 7 & $(7,3)$ & 4 & 1 & YES & YES & YES & $1.27$ & $(2,2)$ & -- & 278\\
$(17,4)$ & 7 & $(7,3)$ & 4 & 1 & YES & YES & YES & $1.27$ & $(2,2)$ & NO & 279\\
$(17,5)$ & 6 & $(7,3)$ & 4 & 1 & YES & YES & NO(2) & $1.09$ & $(4,1)$ & -- & 280\\
$(17,4)$ & 7 & $(8,3)$ & 4 & 1 & YES & YES & YES & $1.27$ & $(2,2)$ & NO & 281\\
$(17,4)$ & 7 & $(8,3)$ & 4 & 1 & YES & YES & YES & $1.27$ & $(2,2)$ & NO & 282\\
$(17,4)$ & 7 & $(8,3)$ & 4 & 1 & YES & YES & YES & $1.27$ & $(2,2)$ & -- & 283\\
$(17,5)$ & 6 & $(8,3)$ & 4 & 1 & YES & YES & YES & $0.89$ & $(2,2)$ & -- & 284\\
$(17,3)$ & 7 & $(11,5)$ & 6 & 1 & YES & YES & YES & $0.88$ & $(4,1)$ & NO & 285\\
$(17,3)$ & 7 & $(11,5)$ & 6 & 1 & YES & YES & YES & $0.88$ & $(4,1)$ & -- & 286\\
$(17,5)$ & 6 & $(11,5)$ & 6 & 1 & YES & YES & YES & $1.10$ & $(2,2)$ & -- & 287\\
$(17,7)$ & 6 & $(11,5)$ & 6 & 1 & YES & YES & YES & $0.88$ & $(4,1)$ & -- & 288\\
$(17,7)$ & 6 & $(11,5)$ & 6 & 1 & YES & YES & YES & $0.88$ & $(4,1)$ & NO & 289\\
$(17,6)$ & 7 & $(13,3)$ & 6 & 1 & YES & YES & YES & $1.20$ & $(2,2)$ & NO & 290\\
$(17,7)$ & 6 & $(13,4)$ & 6 & 1 & YES & YES & YES & $1.27$ & $(2,2)$ & NO & 291\\
$(18,7)$ & 6 & $(5,1)$ & 4 & 1 & YES & YES & YES & $1.33$ & $(2,2)$ & NO & 292\\
$(18,7)$ & 6 & $(5,1)$ & 4 & 1 & YES & YES & YES & $1.33$ & $(2,2)$ & -- & 293\\
$(18,7)$ & 6 & $(6,1)$ & 5 & 6 & YES & YES & YES & $1.33$ & $(2,2)$ & NO & 294\\
$(18,7)$ & 6 & $(6,1)$ & 5 & 6 & YES & YES & YES & $1.33$ & $(2,2)$ & -- & 295\\
$(18,7)$ & 6 & $(6,1)$ & 5 & 6 & YES & YES & YES & $1.33$ & $(2,2)$ & NO & 296\\
$(18,5)$ & 6 & $(7,3)$ & 4 & 1 & YES & YES & NO(2) & $1.00$ & $(4,1)$ & -- & 297\\
$(18,7)$ & 6 & $(9,4)$ & 5 & 9 & YES & YES & YES & $1.20$ & $(2,2)$ & NO & 298\\
$(18,7)$ & 6 & $(9,4)$ & 5 & 9 & YES & YES & YES & $1.20$ & $(2,2)$ & -- & 299\\
$(18,5)$ & 6 & $(11,5)$ & 6 & 1 & YES & YES & YES & $1.10$ & $(2,2)$ & -- & 300\\
$(18,7)$ & 6 & $(13,6)$ & 7 & 1 & YES & YES & YES & $0.88$ & $(4,1)$ & 591 & 301\\
$(19,3)$ & 8 & $(4,1)$ & 3 & 1 & YES & YES & YES & $1.00$ & $(2,2)$ & -- & 302\\
$(19,3)$ & 8 & $(4,1)$ & 3 & 1 & YES & YES & YES & $1.11$ & $(2,2)$ & NO & 303\\
$(19,4)$ & 7 & $(4,1)$ & 3 & 1 & YES & YES & YES & $0.75$ & $(2,2)$ & -- & 304\\
$(19,4)$ & 7 & $(4,1)$ & 3 & 1 & YES & YES & YES & $0.88$ & $(2,2)$ & NO & 305\\
$(19,6)$ & 8 & $(5,1)$ & 4 & 1 & YES & YES & YES & $1.00$ & $(2,2)$ & NO & 306\\
$(19,6)$ & 8 & $(7,3)$ & 4 & 1 & YES & YES & YES & $1.20$ & $(4,1)$ & NO & 307\\
$(19,6)$ & 8 & $(7,3)$ & 4 & 1 & YES & YES & YES & $1.20$ & $(4,1)$ & -- & 308\\
$(19,7)$ & 6 & $(7,3)$ & 4 & 1 & YES & YES & YES & $1.10$ & $(2,2)$ & NO & 309\\
$(19,7)$ & 6 & $(7,3)$ & 4 & 1 & YES & YES & YES & $1.10$ & $(2,2)$ & -- & 310\\
$(19,6)$ & 8 & $(8,3)$ & 4 & 1 & YES & YES & YES & $1.27$ & $(2,2)$ & -- & 311\\
$(19,7)$ & 6 & $(9,4)$ & 5 & 1 & YES & YES & YES & $1.10$ & $(2,2)$ & 270 & 312\\
$(19,8)$ & 6 & $(11,4)$ & 5 & 1 & YES & YES & YES & $1.10$ & $(2,2)$ & NO & 313\\
$(19,3)$ & 8 & $(13,6)$ & 7 & 1 & YES & YES & YES & $0.88$ & $(2,2)$ & NO & 314\\
$(19,4)$ & 7 & $(14,3)$ & 6 & 1 & YES & YES & YES & $0.88$ & $(2,2)$ & NO & 315\\
$(19,6)$ & 8 & $(14,3)$ & 6 & 1 & YES & YES & YES & $1.10$ & $(2,2)$ & NO & 316\\
$(19,6)$ & 8 & $(18,5)$ & 6 & 1 & YES & YES & YES & $1.10$ & $(2,2)$ & NO & 317\\
$(20,3)$ & 8 & $(4,1)$ & 3 & 4 & YES & YES & YES & $1.00$ & $(2,2)$ & -- & 318\\
$(20,3)$ & 8 & $(4,1)$ & 3 & 4 & YES & YES & YES & $1.11$ & $(2,2)$ & NO & 319\\
$(20,9)$ & 7 & $(4,1)$ & 3 & 4 & YES & YES & YES & $1.00$ & $(2,2)$ & NO & 320\\
$(20,9)$ & 7 & $(4,1)$ & 3 & 4 & YES & YES & YES & $1.00$ & $(2,2)$ & -- & 321\\
$(20,9)$ & 7 & $(4,1)$ & 3 & 4 & YES & YES & YES & $1.00$ & $(2,2)$ & NO & 322\\
$(20,9)$ & 7 & $(5,2)$ & 3 & 5 & YES & YES & YES & $0.88$ & $(4,1)$ & -- & 323\\
$(20,7)$ & 8 & $(7,2)$ & 4 & 1 & YES & YES & YES & $1.00$ & $(4,1)$ & -- & 324\\
$(20,9)$ & 7 & $(7,3)$ & 4 & 1 & YES & YES & YES & $0.88$ & $(2,2)$ & -- & 325\\
$(20,7)$ & 8 & $(9,2)$ & 5 & 1 & YES & YES & YES & $0.89$ & $(4,1)$ & -- & 326\\
$(20,7)$ & 8 & $(9,2)$ & 5 & 1 & YES & YES & YES & $1.00$ & $(4,1)$ & NO & 327\\
$(20,9)$ & 7 & $(13,3)$ & 6 & 1 & YES & YES & YES & $1.10$ & $(2,2)$ & NO & 328\\
$(20,9)$ & 7 & $(13,6)$ & 7 & 1 & YES & YES & YES & $0.88$ & $(2,2)$ & NO & 329\\
$(20,7)$ & 8 & $(15,2)$ & 8 & 5 & YES & YES & YES & $1.00$ & $(4,1)$ & NO & 330\\
$(20,9)$ & 7 & $(15,7)$ & 8 & 5 & YES & YES & YES & $0.88$ & $(2,2)$ & 411 & 331\\
$(20,3)$ & 8 & $(17,6)$ & 7 & 1 & YES & YES & YES & $1.20$ & $(2,2)$ & NO & 332\\
$(20,9)$ & 7 & $(17,7)$ & 6 & 1 & YES & YES & YES & $1.10$ & $(2,2)$ & 422 & 333\\
$(20,7)$ & 8 & $(19,7)$ & 6 & 1 & YES & YES & YES & $1.00$ & $(4,1)$ & NO & 334\\
$(20,3)$ & 8 & $(20,3)$ & 8 & 20 & YES & YES & YES & $1.11$ & $(2,2)$ & NO & 335\\
$(21,8)$ & 6 & $(2,1)$ & 1 & 1 & YES & YES & NO(2) & $1.17$ & $(2,2)$ & -- & 336\\
$(21,8)$ & 6 & $(7,3)$ & 4 & 7 & YES & YES & YES & $1.10$ & $(2,2)$ & NO & 337\\
$(21,8)$ & 6 & $(7,3)$ & 4 & 7 & YES & YES & YES & $1.10$ & $(2,2)$ & -- & 338\\
$(21,8)$ & 6 & $(9,4)$ & 5 & 3 & YES & YES & YES & $1.10$ & $(2,2)$ & NO & 339\\
$(21,5)$ & 8 & $(21,4)$ & 8 & 21 & YES & YES & YES & $1.18$ & $(2,2)$ & NO & 340\\
$(22,9)$ & 7 & $(5,1)$ & 4 & 1 & YES & YES & YES & $1.27$ & $(2,2)$ & NO & 341\\
$(22,9)$ & 7 & $(5,1)$ & 4 & 1 & YES & YES & YES & $1.27$ & $(2,2)$ & -- & 342\\
$(23,4)$ & 8 & $(3,1)$ & 2 & 1 & YES & YES & YES & $1.11$ & $(2,2)$ & NO & 343\\
$(23,4)$ & 8 & $(3,1)$ & 2 & 1 & YES & YES & YES & $1.11$ & $(2,2)$ & NO & 344\\
$(23,4)$ & 8 & $(3,1)$ & 2 & 1 & YES & YES & YES & $1.11$ & $(2,2)$ & -- & 345\\
$(23,4)$ & 8 & $(4,1)$ & 3 & 1 & YES & YES & YES & $1.00$ & $(2,2)$ & NO & 346\\
$(23,4)$ & 8 & $(4,1)$ & 3 & 1 & YES & YES & YES & $1.00$ & $(2,2)$ & NO & 347\\
$(23,4)$ & 8 & $(4,1)$ & 3 & 1 & YES & YES & YES & $1.00$ & $(2,2)$ & -- & 348\\
$(23,5)$ & 7 & $(4,1)$ & 3 & 1 & YES & YES & YES & $1.00$ & $(2,2)$ & -- & 349\\
$(23,5)$ & 7 & $(4,1)$ & 3 & 1 & YES & YES & YES & $1.10$ & $(2,2)$ & NO & 350\\
$(23,9)$ & 7 & $(4,1)$ & 3 & 1 & YES & YES & YES & $1.27$ & $(2,2)$ & NO & 351\\
$(23,9)$ & 7 & $(4,1)$ & 3 & 1 & YES & YES & YES & $1.27$ & $(2,2)$ & -- & 352\\
$(23,6)$ & 8 & $(5,1)$ & 4 & 1 & YES & YES & YES & $1.00$ & $(2,2)$ & NO & 353\\
$(23,6)$ & 8 & $(5,1)$ & 4 & 1 & YES & YES & YES & $1.00$ & $(2,2)$ & -- & 354\\
$(23,9)$ & 7 & $(5,1)$ & 4 & 1 & YES & YES & YES & $1.20$ & $(2,2)$ & NO & 355\\
$(23,9)$ & 7 & $(5,1)$ & 4 & 1 & YES & YES & YES & $1.27$ & $(2,2)$ & NO & 356\\
$(23,9)$ & 7 & $(5,1)$ & 4 & 1 & YES & YES & YES & $1.27$ & $(2,2)$ & -- & 357\\
$(23,6)$ & 8 & $(7,3)$ & 4 & 1 & YES & YES & YES & $1.20$ & $(2,2)$ & -- & 358\\
$(23,7)$ & 7 & $(7,3)$ & 4 & 1 & YES & YES & NO(2) & $1.00$ & $(4,1)$ & -- & 359\\
$(23,6)$ & 8 & $(8,3)$ & 4 & 1 & YES & YES & YES & $1.33$ & $(2,2)$ & -- & 360\\
$(23,4)$ & 8 & $(14,5)$ & 6 & 1 & YES & YES & YES & $1.11$ & $(2,2)$ & NO & 361\\
$(23,6)$ & 8 & $(15,4)$ & 6 & 1 & YES & YES & YES & $1.00$ & $(2,2)$ & 492 & 362\\
$(23,6)$ & 8 & $(17,5)$ & 6 & 1 & YES & YES & YES & $1.25$ & $(2,2)$ & NO & 363\\
$(23,4)$ & 8 & $(21,5)$ & 8 & 1 & YES & YES & YES & $1.00$ & $(2,2)$ & NO & 364\\
$(24,11)$ & 8 & $(4,1)$ & 3 & 4 & YES & YES & YES & $0.88$ & $(2,2)$ & -- & 365\\
$(24,11)$ & 8 & $(5,2)$ & 3 & 1 & YES & YES & YES & $1.00$ & $(4,1)$ & -- & 366\\
$(24,11)$ & 8 & $(7,3)$ & 4 & 1 & YES & YES & YES & $0.88$ & $(2,2)$ & 271 & 367\\
$(24,5)$ & 8 & $(9,4)$ & 5 & 3 & YES & YES & YES & $1.27$ & $(2,2)$ & -- & 368\\
$(24,5)$ & 8 & $(11,4)$ & 5 & 1 & YES & YES & YES & $1.11$ & $(2,2)$ & -- & 369\\
$(24,5)$ & 8 & $(21,5)$ & 8 & 3 & YES & YES & YES & $1.18$ & $(2,2)$ & NO & 370\\
$(25,9)$ & 7 & $(3,1)$ & 2 & 1 & YES & YES & YES & $0.88$ & $(4,1)$ & NO & 371\\
$(25,11)$ & 7 & $(3,1)$ & 2 & 1 & YES & YES & YES & $1.20$ & $(2,2)$ & NO & 372\\
$(25,9)$ & 7 & $(4,1)$ & 3 & 1 & YES & YES & YES & $0.88$ & $(4,1)$ & NO & 373\\
$(25,9)$ & 7 & $(4,1)$ & 3 & 1 & YES & YES & YES & $0.88$ & $(4,1)$ & -- & 374\\
$(25,9)$ & 7 & $(4,1)$ & 3 & 1 & YES & YES & YES & $0.88$ & $(4,1)$ & NO & 375\\
$(25,4)$ & 9 & $(5,1)$ & 4 & 5 & YES & YES & YES & $1.11$ & $(2,2)$ & NO & 376\\
$(25,4)$ & 9 & $(5,1)$ & 4 & 5 & YES & YES & YES & $1.11$ & $(2,2)$ & NO & 377\\
$(25,4)$ & 9 & $(5,1)$ & 4 & 5 & YES & YES & YES & $1.11$ & $(2,2)$ & -- & 378\\
$(25,8)$ & 10 & $(5,1)$ & 4 & 5 & YES & YES & YES & $1.00$ & $(2,2)$ & -- & 379\\
$(25,9)$ & 7 & $(5,2)$ & 3 & 5 & YES & YES & YES & $1.20$ & $(2,2)$ & -- & 380\\
$(25,6)$ & 9 & $(8,3)$ & 4 & 1 & YES & YES & YES & $1.27$ & $(2,2)$ & NO & 381\\
$(25,9)$ & 7 & $(13,3)$ & 6 & 1 & YES & YES & YES & $1.10$ & $(2,2)$ & NO & 382\\
$(25,8)$ & 10 & $(19,6)$ & 8 & 1 & YES & YES & YES & $1.00$ & $(2,2)$ & 544 & 383\\
$(25,4)$ & 9 & $(24,5)$ & 8 & 1 & YES & YES & YES & $1.18$ & $(2,2)$ & NO & 384\\
$(26,11)$ & 7 & $(3,1)$ & 2 & 1 & YES & YES & NO(2) & $1.09$ & $(4,1)$ & -- & 385\\
$(26,11)$ & 7 & $(12,5)$ & 5 & 2 & YES & YES & NO(2) & $1.09$ & $(4,1)$ & 441 & 386\\
$(26,5)$ & 9 & $(15,4)$ & 6 & 1 & YES & YES & YES & $1.10$ & $(2,2)$ & NO & 387\\
$(26,7)$ & 7 & $(15,4)$ & 6 & 1 & YES & YES & YES & $1.10$ & $(2,2)$ & NO & 388\\
$(27,5)$ & 8 & $(2,1)$ & 1 & 1 & YES & YES & YES & $0.75$ & $(4,1)$ & NO & 389\\
$(27,10)$ & 7 & $(3,1)$ & 2 & 3 & YES & YES & YES & $1.18$ & $(2,2)$ & NO & 390\\
$(27,10)$ & 7 & $(4,1)$ & 3 & 1 & YES & YES & YES & $1.18$ & $(2,2)$ & NO & 391\\
$(27,10)$ & 7 & $(4,1)$ & 3 & 1 & YES & YES & YES & $1.18$ & $(2,2)$ & -- & 392\\
$(27,10)$ & 7 & $(4,1)$ & 3 & 1 & YES & YES & YES & $1.18$ & $(2,2)$ & NO & 393\\
$(27,11)$ & 8 & $(4,1)$ & 3 & 1 & YES & YES & YES & $1.33$ & $(2,2)$ & NO & 394\\
$(27,11)$ & 8 & $(4,1)$ & 3 & 1 & YES & YES & YES & $1.33$ & $(2,2)$ & -- & 395\\
$(27,10)$ & 7 & $(5,1)$ & 4 & 1 & YES & YES & YES & $1.20$ & $(2,2)$ & 249 & 396\\
$(27,10)$ & 7 & $(5,1)$ & 4 & 1 & YES & YES & YES & $1.20$ & $(2,2)$ & -- & 397\\
$(27,11)$ & 8 & $(7,2)$ & 4 & 1 & YES & YES & YES & $1.27$ & $(2,2)$ & NO & 398\\
$(27,11)$ & 8 & $(9,4)$ & 5 & 9 & YES & YES & YES & $1.27$ & $(2,2)$ & NO & 399\\
$(27,5)$ & 8 & $(11,2)$ & 6 & 1 & YES & YES & YES & $0.75$ & $(4,1)$ & NO & 400\\
$(27,10)$ & 7 & $(17,6)$ & 7 & 1 & YES & YES & YES & $1.20$ & $(2,2)$ & NO & 401\\
$(28,13)$ & 9 & $(3,1)$ & 2 & 1 & YES & YES & YES & $1.00$ & $(2,2)$ & NO & 402\\
$(28,13)$ & 9 & $(3,1)$ & 2 & 1 & YES & YES & YES & $1.00$ & $(2,2)$ & -- & 403\\
$(28,13)$ & 9 & $(4,1)$ & 3 & 4 & YES & YES & YES & $1.00$ & $(2,2)$ & NO & 404\\
$(28,5)$ & 8 & $(5,1)$ & 4 & 1 & YES & YES & YES & $0.89$ & $(6,0)$ & NO & 405\\
$(28,5)$ & 8 & $(5,1)$ & 4 & 1 & YES & YES & YES & $0.89$ & $(6,0)$ & NO & 406\\
$(28,5)$ & 8 & $(5,1)$ & 4 & 1 & YES & YES & YES & $0.89$ & $(6,0)$ & -- & 407\\
$(28,11)$ & 8 & $(5,2)$ & 3 & 1 & YES & YES & YES & $1.11$ & $(2,2)$ & NO & 408\\
$(28,13)$ & 9 & $(5,2)$ & 3 & 1 & YES & YES & YES & $1.00$ & $(2,2)$ & NO & 409\\
$(28,13)$ & 9 & $(6,1)$ & 5 & 2 & YES & YES & YES & $0.88$ & $(2,2)$ & NO & 410\\
$(28,13)$ & 9 & $(9,4)$ & 5 & 1 & YES & YES & YES & $0.88$ & $(2,2)$ & 331 & 411\\
$(28,5)$ & 8 & $(11,2)$ & 6 & 1 & YES & YES & YES & $0.89$ & $(6,0)$ & NO & 412\\
$(28,5)$ & 8 & $(11,5)$ & 6 & 1 & YES & YES & YES & $1.10$ & $(2,2)$ & -- & 413\\
$(28,13)$ & 9 & $(11,5)$ & 6 & 1 & YES & YES & YES & $0.88$ & $(2,2)$ & NO & 414\\
$(28,5)$ & 8 & $(14,5)$ & 6 & 14 & YES & YES & YES & $1.10$ & $(2,2)$ & -- & 415\\
$(29,4)$ & 10 & $(2,1)$ & 1 & 1 & YES & YES & YES & $1.00$ & $(4,1)$ & NO & 416\\
$(29,12)$ & 7 & $(3,1)$ & 2 & 1 & YES & YES & NO(2) & $1.00$ & $(4,1)$ & -- & 417\\
$(29,9)$ & 8 & $(4,1)$ & 3 & 1 & YES & YES & YES & $1.20$ & $(2,2)$ & NO & 418\\
$(29,9)$ & 8 & $(4,1)$ & 3 & 1 & YES & YES & YES & $1.20$ & $(2,2)$ & -- & 419\\
$(29,9)$ & 8 & $(4,1)$ & 3 & 1 & YES & YES & YES & $1.20$ & $(2,2)$ & NO & 420\\
$(29,12)$ & 7 & $(7,3)$ & 4 & 1 & YES & YES & NO(2) & $1.09$ & $(4,1)$ & NO & 421\\
$(29,12)$ & 7 & $(11,5)$ & 6 & 1 & YES & YES & YES & $1.10$ & $(2,2)$ & 333 & 422\\
$(29,4)$ & 10 & $(13,6)$ & 7 & 1 & YES & YES & YES & $0.88$ & $(4,1)$ & NO & 423\\
$(29,13)$ & 8 & $(13,6)$ & 7 & 1 & YES & YES & YES & $0.88$ & $(2,2)$ & 606 & 424\\
$(29,11)$ & 7 & $(29,11)$ & 7 & 29 & YES & YES & YES & $1.00$ & $(2,2)$ & NO & 425\\
$(30,11)$ & 7 & $(3,1)$ & 2 & 3 & YES & YES & YES & $1.00$ & $(2,2)$ & NO & 426\\
$(30,11)$ & 7 & $(3,1)$ & 2 & 3 & YES & YES & YES & $1.00$ & $(2,2)$ & -- & 427\\
$(30,11)$ & 7 & $(5,2)$ & 3 & 5 & YES & YES & YES & $1.00$ & $(2,2)$ & 490 & 428\\
$(31,7)$ & 8 & $(2,1)$ & 1 & 1 & YES & YES & YES & $0.89$ & $(4,1)$ & -- & 429\\
$(31,7)$ & 8 & $(3,1)$ & 2 & 1 & YES & YES & YES & $1.18$ & $(2,2)$ & -- & 430\\
$(31,13)$ & 7 & $(3,1)$ & 2 & 1 & YES & YES & YES & $1.18$ & $(2,2)$ & NO & 431\\
$(31,13)$ & 7 & $(3,1)$ & 2 & 1 & YES & YES & YES & $1.18$ & $(2,2)$ & -- & 432\\
$(31,14)$ & 8 & $(3,1)$ & 2 & 1 & YES & YES & YES & $1.20$ & $(2,2)$ & NO & 433\\
$(31,14)$ & 8 & $(3,1)$ & 2 & 1 & YES & YES & YES & $1.20$ & $(2,2)$ & -- & 434\\
$(31,14)$ & 8 & $(3,1)$ & 2 & 1 & YES & YES & YES & $1.20$ & $(2,2)$ & NO & 435\\
$(31,7)$ & 8 & $(5,1)$ & 4 & 1 & YES & YES & YES & $0.89$ & $(4,1)$ & NO & 436\\
$(31,7)$ & 8 & $(5,1)$ & 4 & 1 & YES & YES & YES & $0.89$ & $(4,1)$ & -- & 437\\
$(31,9)$ & 8 & $(5,2)$ & 3 & 1 & YES & YES & YES & $1.10$ & $(2,2)$ & -- & 438\\
$(31,11)$ & 8 & $(5,2)$ & 3 & 1 & YES & YES & YES & $0.88$ & $(4,1)$ & NO & 439\\
$(31,11)$ & 8 & $(7,2)$ & 4 & 1 & YES & YES & YES & $0.89$ & $(4,1)$ & NO & 440\\
$(31,13)$ & 7 & $(7,3)$ & 4 & 1 & YES & YES & NO(2) & $1.09$ & $(4,1)$ & 386 & 441\\
$(31,6)$ & 10 & $(9,4)$ & 5 & 1 & YES & YES & YES & $1.20$ & $(2,2)$ & NO & 442\\
$(31,11)$ & 8 & $(9,2)$ & 5 & 1 & YES & YES & YES & $0.89$ & $(4,1)$ & NO & 443\\
$(31,6)$ & 10 & $(20,3)$ & 8 & 1 & YES & YES & YES & $1.10$ & $(2,2)$ & NO & 444\\
$(31,11)$ & 8 & $(20,7)$ & 8 & 1 & YES & YES & YES & $0.89$ & $(4,1)$ & NO & 445\\
$(31,7)$ & 8 & $(24,5)$ & 8 & 1 & YES & YES & YES & $1.18$ & $(2,2)$ & NO & 446\\
$(31,6)$ & 10 & $(28,5)$ & 8 & 1 & YES & YES & YES & $1.10$ & $(2,2)$ & NO & 447\\
$(32,13)$ & 9 & $(2,1)$ & 1 & 2 & YES & YES & YES & $1.18$ & $(2,2)$ & -- & 448\\
$(32,13)$ & 9 & $(3,1)$ & 2 & 1 & YES & YES & YES & $1.20$ & $(2,2)$ & -- & 449\\
$(32,13)$ & 9 & $(4,1)$ & 3 & 4 & YES & YES & YES & $1.27$ & $(2,2)$ & -- & 450\\
$(32,7)$ & 8 & $(5,1)$ & 4 & 1 & YES & YES & YES & $0.89$ & $(4,1)$ & NO & 451\\
$(32,7)$ & 8 & $(5,1)$ & 4 & 1 & YES & YES & YES & $0.89$ & $(4,1)$ & -- & 452\\
$(32,9)$ & 8 & $(5,2)$ & 3 & 1 & YES & YES & YES & $1.25$ & $(2,2)$ & -- & 453\\
$(32,9)$ & 8 & $(7,2)$ & 4 & 1 & YES & YES & YES & $1.10$ & $(2,2)$ & NO & 454\\
$(32,13)$ & 9 & $(7,1)$ & 6 & 1 & YES & YES & YES & $1.27$ & $(2,2)$ & NO & 455\\
$(32,13)$ & 9 & $(8,1)$ & 7 & 8 & YES & YES & YES & $1.20$ & $(2,2)$ & NO & 456\\
$(32,13)$ & 9 & $(8,1)$ & 7 & 8 & YES & YES & YES & $1.20$ & $(2,2)$ & NO & 457\\
$(32,7)$ & 8 & $(9,2)$ & 5 & 1 & YES & YES & YES & $1.18$ & $(2,2)$ & NO & 458\\
$(32,13)$ & 9 & $(12,5)$ & 5 & 4 & YES & YES & YES & $1.27$ & $(2,2)$ & NO & 459\\
$(32,9)$ & 8 & $(17,5)$ & 6 & 1 & YES & YES & YES & $1.25$ & $(2,2)$ & NO & 460\\
$(32,7)$ & 8 & $(21,5)$ & 8 & 1 & YES & YES & YES & $1.00$ & $(2,2)$ & NO & 461\\
$(32,13)$ & 9 & $(27,11)$ & 8 & 1 & YES & YES & YES & $1.27$ & $(2,2)$ & NO & 462\\
$(32,13)$ & 9 & $(32,13)$ & 9 & 32 & YES & YES & YES & $1.20$ & $(2,2)$ & NO & 463\\
$(33,13)$ & 9 & $(2,1)$ & 1 & 1 & YES & YES & YES & $1.18$ & $(2,2)$ & -- & 464\\
$(33,13)$ & 9 & $(2,1)$ & 1 & 1 & YES & YES & YES & $1.36$ & $(2,2)$ & NO & 465\\
$(33,13)$ & 9 & $(3,1)$ & 2 & 3 & YES & YES & YES & $1.33$ & $(2,2)$ & NO & 466\\
$(33,13)$ & 9 & $(3,1)$ & 2 & 3 & YES & YES & YES & $1.20$ & $(2,2)$ & -- & 467\\
$(33,10)$ & 8 & $(4,1)$ & 3 & 1 & YES & YES & YES & $1.18$ & $(2,2)$ & NO & 468\\
$(33,10)$ & 8 & $(4,1)$ & 3 & 1 & YES & YES & YES & $1.18$ & $(2,2)$ & -- & 469\\
$(33,10)$ & 8 & $(4,1)$ & 3 & 1 & YES & YES & YES & $1.18$ & $(2,2)$ & NO & 470\\
$(33,13)$ & 9 & $(4,1)$ & 3 & 1 & YES & YES & YES & $1.33$ & $(2,2)$ & NO & 471\\
$(33,13)$ & 9 & $(5,1)$ & 4 & 1 & YES & YES & YES & $1.33$ & $(2,2)$ & -- & 472\\
$(33,13)$ & 9 & $(5,1)$ & 4 & 1 & YES & YES & YES & $1.27$ & $(2,2)$ & NO & 473\\
$(33,14)$ & 8 & $(5,1)$ & 4 & 1 & YES & YES & NO(2) & $1.00$ & $(4,1)$ & NO & 474\\
$(33,13)$ & 9 & $(6,1)$ & 5 & 3 & YES & YES & YES & $1.20$ & $(2,2)$ & NO & 475\\
$(33,10)$ & 8 & $(7,3)$ & 4 & 1 & YES & YES & YES & $1.18$ & $(2,2)$ & NO & 476\\
$(33,13)$ & 9 & $(7,1)$ & 6 & 1 & YES & YES & YES & $1.27$ & $(2,2)$ & NO & 477\\
$(33,13)$ & 9 & $(8,3)$ & 4 & 1 & YES & YES & YES & $1.33$ & $(2,2)$ & NO & 478\\
$(33,13)$ & 9 & $(13,5)$ & 5 & 1 & YES & YES & YES & $1.27$ & $(2,2)$ & NO & 479\\
$(33,13)$ & 9 & $(18,7)$ & 6 & 3 & YES & YES & YES & $1.20$ & $(2,2)$ & NO & 480\\
$(33,14)$ & 8 & $(19,8)$ & 6 & 1 & YES & YES & NO(2) & $1.00$ & $(4,1)$ & 587 & 481\\
$(33,13)$ & 9 & $(23,9)$ & 7 & 1 & YES & YES & YES & $1.33$ & $(2,2)$ & 619 & 482\\
$(33,13)$ & 9 & $(33,13)$ & 9 & 33 & YES & YES & YES & $1.20$ & $(2,2)$ & NO & 483\\
$(34,9)$ & 8 & $(2,1)$ & 1 & 2 & YES & YES & YES & $0.88$ & $(4,1)$ & NO & 484\\
$(34,13)$ & 7 & $(2,1)$ & 1 & 2 & YES & YES & YES & $1.18$ & $(2,2)$ & -- & 485\\
$(34,13)$ & 7 & $(2,1)$ & 1 & 2 & YES & YES & YES & $1.18$ & $(2,2)$ & NO & 486\\
$(34,9)$ & 8 & $(3,1)$ & 2 & 1 & YES & YES & YES & $0.88$ & $(4,1)$ & NO & 487\\
$(34,9)$ & 8 & $(3,1)$ & 2 & 1 & YES & YES & YES & $0.88$ & $(4,1)$ & -- & 488\\
$(34,13)$ & 7 & $(3,1)$ & 2 & 1 & YES & YES & YES & $0.89$ & $(2,2)$ & -- & 489\\
$(34,13)$ & 7 & $(3,1)$ & 2 & 1 & YES & YES & YES & $1.00$ & $(2,2)$ & 428 & 490\\
$(34,7)$ & 10 & $(4,1)$ & 3 & 2 & YES & YES & YES & $0.75$ & $(2,2)$ & -- & 491\\
$(34,9)$ & 8 & $(4,1)$ & 3 & 2 & YES & YES & YES & $1.00$ & $(2,2)$ & 362 & 492\\
$(34,9)$ & 8 & $(4,1)$ & 3 & 2 & YES & YES & YES & $1.00$ & $(2,2)$ & -- & 493\\
$(34,9)$ & 8 & $(4,1)$ & 3 & 2 & YES & YES & YES & $1.00$ & $(2,2)$ & NO & 494\\
$(34,9)$ & 8 & $(5,1)$ & 4 & 1 & YES & YES & YES & $0.88$ & $(2,2)$ & NO & 495\\
$(34,9)$ & 8 & $(5,1)$ & 4 & 1 & YES & YES & YES & $0.88$ & $(2,2)$ & -- & 496\\
$(34,7)$ & 10 & $(14,3)$ & 6 & 2 & YES & YES & YES & $0.88$ & $(2,2)$ & NO & 497\\
$(35,13)$ & 8 & $(3,1)$ & 2 & 1 & YES & YES & YES & $1.25$ & $(2,2)$ & -- & 498\\
$(35,11)$ & 9 & $(4,1)$ & 3 & 1 & YES & YES & YES & $1.10$ & $(2,2)$ & -- & 499\\
$(35,16)$ & 9 & $(4,1)$ & 3 & 1 & YES & YES & YES & $0.89$ & $(4,1)$ & -- & 500\\
$(35,11)$ & 9 & $(5,2)$ & 3 & 5 & YES & YES & YES & $1.27$ & $(2,2)$ & -- & 501\\
$(35,16)$ & 9 & $(5,1)$ & 4 & 5 & YES & YES & YES & $1.00$ & $(4,1)$ & NO & 502\\
$(35,16)$ & 9 & $(5,1)$ & 4 & 5 & YES & YES & YES & $1.00$ & $(4,1)$ & NO & 503\\
$(35,16)$ & 9 & $(13,6)$ & 7 & 1 & YES & YES & YES & $0.88$ & $(2,2)$ & 529 & 504\\
$(35,13)$ & 8 & $(14,5)$ & 6 & 7 & YES & YES & YES & $1.11$ & $(2,2)$ & NO & 505\\
$(35,6)$ & 10 & $(19,3)$ & 8 & 1 & YES & YES & YES & $0.88$ & $(2,2)$ & NO & 506\\
$(35,16)$ & 9 & $(24,11)$ & 8 & 1 & YES & YES & YES & $0.89$ & $(4,1)$ & NO & 507\\
$(35,16)$ & 9 & $(35,16)$ & 9 & 35 & YES & YES & YES & $1.00$ & $(4,1)$ & NO & 508\\
$(36,11)$ & 8 & $(3,1)$ & 2 & 3 & YES & YES & YES & $1.20$ & $(2,2)$ & NO & 509\\
$(36,11)$ & 8 & $(3,1)$ & 2 & 3 & YES & YES & YES & $1.20$ & $(2,2)$ & -- & 510\\
$(36,13)$ & 8 & $(5,1)$ & 4 & 1 & YES & YES & YES & $0.75$ & $(4,1)$ & NO & 511\\
$(36,13)$ & 8 & $(5,1)$ & 4 & 1 & YES & YES & YES & $0.75$ & $(4,1)$ & -- & 512\\
$(36,13)$ & 8 & $(7,3)$ & 4 & 1 & YES & YES & YES & $1.18$ & $(2,2)$ & NO & 513\\
$(36,13)$ & 8 & $(14,5)$ & 6 & 2 & YES & YES & YES & $0.75$ & $(4,1)$ & 534 & 514\\
$(37,10)$ & 8 & $(2,1)$ & 1 & 1 & YES & YES & YES & $1.18$ & $(2,2)$ & NO & 515\\
$(37,14)$ & 8 & $(2,1)$ & 1 & 1 & YES & YES & YES & $1.20$ & $(2,2)$ & -- & 516\\
$(37,14)$ & 8 & $(2,1)$ & 1 & 1 & YES & YES & YES & $1.20$ & $(2,2)$ & NO & 517\\
$(37,17)$ & 9 & $(2,1)$ & 1 & 1 & YES & YES & YES & $1.00$ & $(2,2)$ & NO & 518\\
$(37,10)$ & 8 & $(3,1)$ & 2 & 1 & YES & YES & YES & $1.18$ & $(2,2)$ & NO & 519\\
$(37,10)$ & 8 & $(3,1)$ & 2 & 1 & YES & YES & YES & $1.18$ & $(2,2)$ & -- & 520\\
$(37,14)$ & 8 & $(3,1)$ & 2 & 1 & YES & YES & YES & $1.10$ & $(2,2)$ & -- & 521\\
$(37,14)$ & 8 & $(3,1)$ & 2 & 1 & YES & YES & YES & $1.20$ & $(2,2)$ & NO & 522\\
$(37,17)$ & 9 & $(3,1)$ & 2 & 1 & YES & YES & YES & $1.00$ & $(2,2)$ & -- & 523\\
$(37,13)$ & 9 & $(4,1)$ & 3 & 1 & YES & YES & YES & $0.89$ & $(4,1)$ & -- & 524\\
$(37,16)$ & 9 & $(6,1)$ & 5 & 1 & YES & YES & YES & $1.18$ & $(2,2)$ & -- & 525\\
$(37,16)$ & 9 & $(7,1)$ & 6 & 1 & YES & YES & YES & $1.27$ & $(2,2)$ & NO & 526\\
$(37,16)$ & 9 & $(7,1)$ & 6 & 1 & YES & YES & YES & $1.27$ & $(2,2)$ & NO & 527\\
$(37,17)$ & 9 & $(9,4)$ & 5 & 1 & YES & YES & YES & $0.88$ & $(2,2)$ & 615 & 528\\
$(37,17)$ & 9 & $(11,5)$ & 6 & 1 & YES & YES & YES & $0.88$ & $(2,2)$ & 504 & 529\\
$(37,13)$ & 9 & $(20,7)$ & 8 & 1 & YES & YES & YES & $1.00$ & $(4,1)$ & NO & 530\\
$(37,16)$ & 9 & $(30,13)$ & 8 & 1 & YES & YES & YES & $1.18$ & $(2,2)$ & NO & 531\\
$(37,16)$ & 9 & $(37,16)$ & 9 & 37 & YES & YES & YES & $1.27$ & $(2,2)$ & NO & 532\\
$(39,14)$ & 8 & $(2,1)$ & 1 & 1 & YES & YES & YES & $0.88$ & $(4,1)$ & -- & 533\\
$(39,14)$ & 8 & $(11,4)$ & 5 & 1 & YES & YES & YES & $0.75$ & $(4,1)$ & 514 & 534\\
$(39,14)$ & 8 & $(14,5)$ & 6 & 1 & YES & YES & YES & $0.88$ & $(4,1)$ & NO & 535\\
$(40,17)$ & 9 & $(2,1)$ & 1 & 2 & YES & YES & YES & $1.30$ & $(2,2)$ & -- & 536\\
$(40,17)$ & 9 & $(2,1)$ & 1 & 2 & YES & YES & YES & $1.30$ & $(2,2)$ & NO & 537\\
$(40,17)$ & 9 & $(3,1)$ & 2 & 1 & YES & YES & YES & $1.20$ & $(2,2)$ & NO & 538\\
$(40,17)$ & 9 & $(4,1)$ & 3 & 4 & YES & YES & YES & $1.33$ & $(2,2)$ & -- & 539\\
$(40,17)$ & 9 & $(5,2)$ & 3 & 5 & YES & YES & YES & $1.20$ & $(2,2)$ & NO & 540\\
$(40,17)$ & 9 & $(8,1)$ & 7 & 8 & YES & YES & YES & $1.20$ & $(2,2)$ & NO & 541\\
$(40,17)$ & 9 & $(8,1)$ & 7 & 8 & YES & YES & YES & $1.20$ & $(2,2)$ & NO & 542\\
$(40,17)$ & 9 & $(40,17)$ & 9 & 40 & YES & YES & YES & $1.20$ & $(2,2)$ & NO & 543\\
$(41,13)$ & 10 & $(3,1)$ & 2 & 1 & YES & YES & YES & $1.00$ & $(2,2)$ & 383 & 544\\
$(41,16)$ & 8 & $(3,1)$ & 2 & 1 & YES & YES & YES & $1.20$ & $(2,2)$ & NO & 545\\
$(41,16)$ & 8 & $(3,1)$ & 2 & 1 & YES & YES & YES & $1.20$ & $(2,2)$ & -- & 546\\
$(41,16)$ & 8 & $(3,1)$ & 2 & 1 & YES & YES & YES & $1.20$ & $(2,2)$ & NO & 547\\
$(41,13)$ & 10 & $(5,1)$ & 4 & 1 & YES & YES & YES & $0.88$ & $(2,2)$ & -- & 548\\
$(41,16)$ & 8 & $(5,1)$ & 4 & 1 & YES & YES & YES & $1.10$ & $(2,2)$ & -- & 549\\
$(41,15)$ & 8 & $(7,2)$ & 4 & 1 & YES & YES & YES & $1.00$ & $(2,2)$ & NO & 550\\
$(41,15)$ & 8 & $(7,3)$ & 4 & 1 & YES & YES & YES & $1.00$ & $(2,2)$ & NO & 551\\
$(41,19)$ & 10 & $(7,1)$ & 6 & 1 & YES & YES & YES & $0.88$ & $(4,1)$ & NO & 552\\
$(41,8)$ & 12 & $(9,1)$ & 8 & 1 & YES & YES & YES & $0.88$ & $(2,2)$ & NO & 553\\
$(41,19)$ & 10 & $(11,5)$ & 6 & 1 & YES & YES & YES & $0.88$ & $(4,1)$ & NO & 554\\
$(41,19)$ & 10 & $(13,6)$ & 7 & 1 & YES & YES & YES & $0.88$ & $(4,1)$ & NO & 555\\
$(41,8)$ & 12 & $(21,4)$ & 8 & 1 & YES & YES & YES & $0.88$ & $(2,2)$ & NO & 556\\
$(41,8)$ & 12 & $(41,8)$ & 12 & 41 & YES & YES & YES & $0.88$ & $(2,2)$ & NO & 557\\
$(42,13)$ & 9 & $(2,1)$ & 1 & 2 & YES & YES & YES & $1.20$ & $(2,2)$ & NO & 558\\
$(42,19)$ & 9 & $(2,1)$ & 1 & 2 & YES & YES & YES & $1.42$ & $(2,2)$ & -- & 559\\
$(42,19)$ & 9 & $(3,1)$ & 2 & 3 & YES & YES & YES & $1.20$ & $(2,2)$ & NO & 560\\
$(42,19)$ & 9 & $(6,1)$ & 5 & 6 & YES & YES & YES & $0.88$ & $(4,1)$ & NO & 561\\
$(42,19)$ & 9 & $(6,1)$ & 5 & 6 & YES & YES & YES & $0.88$ & $(4,1)$ & -- & 562\\
$(42,19)$ & 9 & $(9,4)$ & 5 & 3 & YES & YES & YES & $1.20$ & $(2,2)$ & NO & 563\\
$(43,19)$ & 9 & $(2,1)$ & 1 & 1 & YES & YES & YES & $1.36$ & $(2,2)$ & -- & 564\\
$(43,19)$ & 9 & $(2,1)$ & 1 & 1 & YES & YES & YES & $1.36$ & $(2,2)$ & NO & 565\\
$(43,20)$ & 10 & $(2,1)$ & 1 & 1 & NO & YES & YES & $1.00$ & $(2,2)$ & -- & 566\\
$(43,16)$ & 9 & $(3,1)$ & 2 & 1 & YES & YES & YES & $1.20$ & $(2,2)$ & NO & 567\\
$(43,16)$ & 9 & $(3,1)$ & 2 & 1 & YES & YES & YES & $1.20$ & $(2,2)$ & -- & 568\\
$(43,19)$ & 9 & $(3,1)$ & 2 & 1 & YES & YES & YES & $1.27$ & $(2,2)$ & NO & 569\\
$(43,16)$ & 9 & $(4,1)$ & 3 & 1 & YES & YES & YES & $1.11$ & $(2,2)$ & -- & 570\\
$(43,8)$ & 9 & $(5,1)$ & 4 & 1 & NO & YES & YES & $0.89$ & $(6,0)$ & -- & 571\\
$(43,16)$ & 9 & $(6,1)$ & 5 & 1 & YES & YES & YES & $1.11$ & $(2,2)$ & -- & 572\\
$(43,19)$ & 9 & $(7,1)$ & 6 & 1 & YES & YES & YES & $1.27$ & $(2,2)$ & NO & 573\\
$(43,19)$ & 9 & $(7,1)$ & 6 & 1 & YES & YES & YES & $1.27$ & $(2,2)$ & NO & 574\\
$(43,19)$ & 9 & $(7,3)$ & 4 & 1 & YES & YES & YES & $1.27$ & $(2,2)$ & NO & 575\\
$(43,16)$ & 9 & $(19,7)$ & 6 & 1 & YES & YES & YES & $1.11$ & $(2,2)$ & NO & 576\\
$(43,16)$ & 9 & $(35,13)$ & 8 & 1 & YES & YES & YES & $1.11$ & $(2,2)$ & NO & 577\\
$(43,19)$ & 9 & $(43,19)$ & 9 & 43 & YES & YES & YES & $1.27$ & $(2,2)$ & NO & 578\\
$(44,19)$ & 10 & $(2,1)$ & 1 & 2 & YES & YES & YES & $1.30$ & $(2,2)$ & NO & 579\\
$(44,19)$ & 10 & $(3,1)$ & 2 & 1 & YES & YES & YES & $1.20$ & $(2,2)$ & NO & 580\\
$(44,19)$ & 10 & $(9,4)$ & 5 & 1 & YES & YES & YES & $1.20$ & $(2,2)$ & NO & 581\\
$(45,14)$ & 9 & $(3,1)$ & 2 & 3 & NO & YES & YES & $1.20$ & $(2,2)$ & -- & 582\\
$(45,17)$ & 9 & $(3,1)$ & 2 & 3 & YES & YES & YES & $1.20$ & $(2,2)$ & NO & 583\\
$(45,13)$ & 10 & $(4,1)$ & 3 & 1 & YES & YES & YES & $1.25$ & $(2,2)$ & NO & 584\\
$(45,19)$ & 8 & $(5,1)$ & 4 & 5 & YES & YES & NO(2) & $0.90$ & $(4,1)$ & NO & 585\\
$(45,13)$ & 10 & $(7,2)$ & 4 & 1 & YES & YES & YES & $1.10$ & $(2,2)$ & NO & 586\\
$(45,19)$ & 8 & $(7,3)$ & 4 & 1 & YES & YES & NO(2) & $1.00$ & $(4,1)$ & 481 & 587\\
$(45,17)$ & 9 & $(8,3)$ & 4 & 1 & YES & YES & YES & $1.33$ & $(2,2)$ & NO & 588\\
$(46,21)$ & 10 & $(2,1)$ & 1 & 2 & YES & YES & YES & $1.00$ & $(4,1)$ & NO & 589\\
$(46,13)$ & 10 & $(3,1)$ & 2 & 1 & YES & YES & YES & $1.25$ & $(2,2)$ & NO & 590\\
$(46,21)$ & 10 & $(3,1)$ & 2 & 1 & YES & YES & YES & $0.88$ & $(4,1)$ & 301 & 591\\
$(46,13)$ & 10 & $(7,2)$ & 4 & 1 & YES & YES & YES & $1.00$ & $(4,1)$ & NO & 592\\
$(46,21)$ & 10 & $(13,6)$ & 7 & 1 & YES & YES & YES & $0.88$ & $(4,1)$ & NO & 593\\
$(47,20)$ & 10 & $(2,1)$ & 1 & 1 & NO & YES & YES & $1.36$ & $(2,2)$ & -- & 594\\
$(47,8)$ & 12 & $(7,1)$ & 6 & 1 & YES & YES & YES & $1.00$ & $(2,2)$ & NO & 595\\
$(47,8)$ & 12 & $(9,1)$ & 8 & 1 & YES & YES & YES & $0.88$ & $(2,2)$ & NO & 596\\
$(47,8)$ & 12 & $(35,6)$ & 10 & 1 & YES & YES & YES & $0.88$ & $(2,2)$ & 656 & 597\\
$(47,8)$ & 12 & $(47,8)$ & 12 & 47 & YES & YES & YES & $1.00$ & $(2,2)$ & NO & 598\\
$(48,17)$ & 9 & $(3,1)$ & 2 & 3 & YES & YES & YES & $1.00$ & $(4,1)$ & NO & 599\\
$(48,7)$ & 12 & $(4,1)$ & 3 & 4 & YES & YES & YES & $0.88$ & $(2,2)$ & NO & 600\\
$(48,17)$ & 9 & $(5,1)$ & 4 & 1 & YES & YES & YES & $1.10$ & $(2,2)$ & -- & 601\\
$(48,17)$ & 9 & $(17,6)$ & 7 & 1 & YES & YES & YES & $1.20$ & $(2,2)$ & NO & 602\\
$(48,17)$ & 9 & $(31,11)$ & 8 & 1 & YES & YES & YES & $0.89$ & $(4,1)$ & NO & 603\\
$(49,19)$ & 8 & $(2,1)$ & 1 & 1 & YES & YES & NO(2) & $1.00$ & $(4,1)$ & -- & 604\\
$(49,20)$ & 9 & $(2,1)$ & 1 & 1 & YES & YES & YES & $1.27$ & $(2,2)$ & NO & 605\\
$(49,22)$ & 9 & $(2,1)$ & 1 & 1 & YES & YES & YES & $0.88$ & $(2,2)$ & 424 & 606\\
$(49,8)$ & 13 & $(5,1)$ & 4 & 1 & YES & YES & YES & $0.88$ & $(2,2)$ & NO & 607\\
$(49,22)$ & 9 & $(11,5)$ & 6 & 1 & YES & YES & YES & $1.10$ & $(2,2)$ & NO & 608\\
$(49,9)$ & 10 & $(28,5)$ & 8 & 7 & YES & YES & YES & $1.10$ & $(2,2)$ & NO & 609\\
$(50,21)$ & 8 & $(2,1)$ & 1 & 2 & NO & YES & YES & $1.18$ & $(2,2)$ & -- & 610\\
$(50,9)$ & 10 & $(5,1)$ & 4 & 5 & NO & YES & YES & $1.11$ & $(2,2)$ & -- & 611\\
$(51,16)$ & 10 & $(2,1)$ & 1 & 1 & YES & YES & YES & $1.27$ & $(2,2)$ & NO & 612\\
$(51,20)$ & 9 & $(2,1)$ & 1 & 1 & YES & YES & YES & $1.11$ & $(2,2)$ & -- & 613\\
$(51,20)$ & 9 & $(2,1)$ & 1 & 1 & YES & YES & YES & $1.11$ & $(2,2)$ & NO & 614\\
$(51,23)$ & 9 & $(2,1)$ & 1 & 1 & YES & YES & YES & $0.88$ & $(2,2)$ & 528 & 615\\
$(51,11)$ & 9 & $(4,1)$ & 3 & 1 & NO & YES & YES & $1.00$ & $(2,2)$ & -- & 616\\
$(51,20)$ & 9 & $(5,1)$ & 4 & 1 & YES & YES & YES & $1.25$ & $(2,2)$ & -- & 617\\
$(51,20)$ & 9 & $(5,1)$ & 4 & 1 & YES & YES & YES & $1.18$ & $(2,2)$ & NO & 618\\
$(51,20)$ & 9 & $(5,2)$ & 3 & 1 & YES & YES & YES & $1.33$ & $(2,2)$ & 482 & 619\\
$(52,23)$ & 10 & $(2,1)$ & 1 & 2 & NO & YES & YES & $1.30$ & $(2,2)$ & -- & 620\\
$(52,19)$ & 9 & $(3,1)$ & 2 & 1 & YES & YES & YES & $1.27$ & $(2,2)$ & NO & 621\\
$(52,19)$ & 9 & $(11,4)$ & 5 & 1 & YES & YES & YES & $1.27$ & $(2,2)$ & NO & 622\\
$(53,19)$ & 9 & $(2,1)$ & 1 & 1 & YES & YES & YES & $1.11$ & $(2,2)$ & NO & 623\\
$(53,24)$ & 10 & $(2,1)$ & 1 & 1 & NO & YES & YES & $1.36$ & $(2,2)$ & -- & 624\\
$(53,19)$ & 9 & $(3,1)$ & 2 & 1 & YES & YES & YES & $1.20$ & $(2,2)$ & NO & 625\\
$(53,11)$ & 10 & $(4,1)$ & 3 & 1 & YES & YES & YES & $1.00$ & $(2,2)$ & -- & 626\\
$(53,19)$ & 9 & $(5,1)$ & 4 & 1 & YES & YES & YES & $1.00$ & $(2,2)$ & NO & 627\\
$(53,19)$ & 9 & $(25,9)$ & 7 & 1 & YES & YES & YES & $1.10$ & $(2,2)$ & 641 & 628\\
$(54,25)$ & 11 & $(2,1)$ & 1 & 2 & NO & YES & YES & $1.00$ & $(4,1)$ & -- & 629\\
$(55,23)$ & 9 & $(2,1)$ & 1 & 1 & NO & YES & YES & $1.20$ & $(2,2)$ & -- & 630\\
$(55,24)$ & 9 & $(2,1)$ & 1 & 1 & YES & YES & YES & $0.89$ & $(4,1)$ & NO & 631\\
$(55,16)$ & 9 & $(3,1)$ & 2 & 1 & NO & YES & YES & $0.88$ & $(4,1)$ & -- & 632\\
$(56,25)$ & 11 & $(2,1)$ & 1 & 2 & NO & YES & YES & $1.30$ & $(2,2)$ & -- & 633\\
$(56,15)$ & 9 & $(3,1)$ & 2 & 1 & NO & YES & YES & $1.11$ & $(2,2)$ & -- & 634\\
$(57,25)$ & 9 & $(2,1)$ & 1 & 1 & YES & YES & YES & $0.89$ & $(4,1)$ & NO & 635\\
$(58,13)$ & 11 & $(2,1)$ & 1 & 2 & YES & YES & YES & $1.18$ & $(2,2)$ & -- & 636\\
$(58,17)$ & 9 & $(3,1)$ & 2 & 1 & NO & YES & YES & $1.18$ & $(2,2)$ & -- & 637\\
$(59,13)$ & 11 & $(9,2)$ & 5 & 1 & YES & YES & YES & $0.89$ & $(4,1)$ & NO & 638\\
$(61,22)$ & 9 & $(2,1)$ & 1 & 1 & NO & YES & YES & $0.75$ & $(4,1)$ & -- & 639\\
$(64,17)$ & 10 & $(3,1)$ & 2 & 1 & YES & YES & YES & $1.10$ & $(2,2)$ & NO & 640\\
$(64,23)$ & 9 & $(14,5)$ & 6 & 2 & YES & YES & YES & $1.10$ & $(2,2)$ & 628 & 641\\
$(65,24)$ & 9 & $(2,1)$ & 1 & 1 & YES & YES & YES & $1.00$ & $(2,2)$ & NO & 642\\
$(65,24)$ & 9 & $(3,1)$ & 2 & 1 & YES & YES & YES & $1.00$ & $(2,2)$ & NO & 643\\
$(66,25)$ & 9 & $(2,1)$ & 1 & 2 & NO & YES & YES & $0.88$ & $(4,1)$ & -- & 644\\
$(67,16)$ & 11 & $(4,1)$ & 3 & 1 & YES & YES & YES & $0.89$ & $(4,1)$ & NO & 645\\
$(67,16)$ & 11 & $(21,5)$ & 8 & 1 & YES & YES & YES & $1.18$ & $(2,2)$ & NO & 646\\
$(71,22)$ & 10 & $(2,1)$ & 1 & 1 & YES & YES & YES & $1.10$ & $(2,2)$ & NO & 647\\
$(71,25)$ & 11 & $(2,1)$ & 1 & 1 & NO & YES & YES & $1.00$ & $(4,1)$ & -- & 648\\
$(71,27)$ & 9 & $(2,1)$ & 1 & 1 & NO & YES & YES & $1.18$ & $(2,2)$ & -- & 649\\
$(71,17)$ & 11 & $(4,1)$ & 3 & 1 & YES & YES & YES & $0.89$ & $(4,1)$ & NO & 650\\
$(71,19)$ & 10 & $(4,1)$ & 3 & 1 & YES & YES & YES & $1.10$ & $(2,2)$ & NO & 651\\
$(71,13)$ & 12 & $(6,1)$ & 5 & 1 & YES & YES & YES & $1.10$ & $(2,2)$ & NO & 652\\
$(71,19)$ & 10 & $(15,4)$ & 6 & 1 & YES & YES & YES & $1.10$ & $(2,2)$ & NO & 653\\
$(72,13)$ & 12 & $(5,1)$ & 4 & 1 & YES & YES & YES & $1.25$ & $(2,2)$ & NO & 654\\
$(74,29)$ & 10 & $(2,1)$ & 1 & 2 & NO & YES & YES & $1.20$ & $(2,2)$ & -- & 655\\
$(76,13)$ & 12 & $(6,1)$ & 5 & 2 & YES & YES & YES & $0.88$ & $(2,2)$ & 597 & 656\\
$(77,16)$ & 11 & $(5,1)$ & 4 & 1 & YES & YES & YES & $0.89$ & $(4,1)$ & NO & 657\\
$(91,19)$ & 11 & $(5,1)$ & 4 & 1 & YES & YES & YES & $0.89$ & $(4,1)$ & NO & 658\\
$(91,19)$ & 11 & $(24,5)$ & 8 & 1 & YES & YES & YES & $1.18$ & $(2,2)$ & NO & 659\\
$(99,17)$ & 12 & $(6,1)$ & 5 & 3 & YES & YES & YES & $0.88$ & $(2,2)$ & NO & 660\\
$(101,16)$ & 13 & $(6,1)$ & 5 & 1 & YES & YES & YES & $1.18$ & $(2,2)$ & NO & 661\\
$(a;1,0,0;13)$ & 5 & $(16,5)$ & 7 & 1 & YES & YES & YES & $1.27$ & $(2,2)$ & -- & 662\\
$(a;3,0,0;7)$ & 7 & $(3,1)$ & 2 & 1 & YES & YES & YES & $0.88$ & $(4,1)$ & -- & 663\\
$(a;3,1,0;31)$ & 8 & $(3,1)$ & 2 & 1 & YES & YES & YES & $1.10$ & $(2,2)$ & -- & 664\\
$(a;4,0,0;25)$ & 8 & $(4,1)$ & 3 & 1 & YES & YES & YES & $0.89$ & $(4,1)$ & -- & 665\\
$(a;4,0,1;37)$ & 9 & $(7,1)$ & 6 & 1 & YES & YES & YES & $1.27$ & $(2,2)$ & -- & 666\\
$(b;0,3,0;29)$ & 8 & $(2,1)$ & 1 & 1 & YES & YES & YES & $1.11$ & $(2,2)$ & -- & 667\\
$(c;0,0,0;4)$ & 4 & $(13,6)$ & 7 & 1 & YES & YES & YES & $1.00$ & $(4,1)$ & -- & 668\\
$(c;0,0,0;4)$ & 4 & $(17,6)$ & 7 & 1 & YES & YES & YES & $1.00$ & $(4,1)$ & -- & 669\\
$(c;0,1,0;11)$ & 5 & $(14,5)$ & 6 & 1 & YES & YES & YES & $0.89$ & $(4,1)$ & -- & 670\\
$(c;0,1,0;11)$ & 5 & $(19,5)$ & 7 & 1 & YES & YES & YES & $1.25$ & $(2,2)$ & -- & 671\\
$(c;0,1,1;5)$ & 6 & $(13,4)$ & 6 & 1 & YES & YES & YES & $1.18$ & $(2,2)$ & -- & 672\\
$(c;0,2,0;7)$ & 6 & $(8,3)$ & 4 & 1 & YES & YES & YES & $0.75$ & $(4,1)$ & -- & 673\\
$(c;0,2,1;19)$ & 7 & $(11,3)$ & 5 & 1 & YES & YES & YES & $1.10$ & $(2,2)$ & -- & 674\\
$(c;0,3,0;17)$ & 7 & $(4,1)$ & 3 & 1 & YES & YES & YES & $1.10$ & $(2,2)$ & -- & 675\\
$(c;0,3,2;29)$ & 9 & $(6,1)$ & 5 & 1 & YES & YES & YES & $1.18$ & $(2,2)$ & -- & 676\\
$(c;0,4,0;10)$ & 8 & $(4,1)$ & 3 & 2 & YES & YES & YES & $1.10$ & $(2,2)$ & -- & 677\\
$(c;0,4,1;9)$ & 9 & $(4,1)$ & 3 & 1 & YES & YES & YES & $0.89$ & $(4,1)$ & -- & 678\\
$(c;0,4,1;9)$ & 9 & $(7,1)$ & 6 & 1 & YES & YES & YES & $1.18$ & $(2,2)$ & -- & 679\\
$(d;0,0,4;13)$ & 9 & $(4,1)$ & 3 & 1 & YES & YES & YES & $0.89$ & $(4,1)$ & -- & 680\\
$(d;0,1,0;6)$ & 6 & $(4,1)$ & 3 & 2 & YES & YES & YES & $1.10$ & $(2,2)$ & -- & 681\\
$(d;0,1,0;6)$ & 6 & $(7,3)$ & 4 & 1 & YES & YES & YES & $1.10$ & $(2,2)$ & -- & 682\\
$(d;0,1,2;11)$ & 8 & $(5,1)$ & 4 & 1 & YES & YES & YES & $1.10$ & $(2,2)$ & -- & 683\\
$(d;0,1,3;27)$ & 9 & $(2,1)$ & 1 & 1 & YES & YES & YES & $1.00$ & $(2,2)$ & -- & 684\\
$(d;0,1,3;27)$ & 9 & $(5,1)$ & 4 & 1 & YES & YES & YES & $1.18$ & $(2,2)$ & -- & 685\\
$(d;0,2,0;7)$ & 7 & $(4,1)$ & 3 & 1 & YES & YES & YES & $1.10$ & $(2,2)$ & -- & 686\\
$(d;0,3,1;23)$ & 9 & $(7,1)$ & 6 & 1 & YES & YES & YES & $1.18$ & $(2,2)$ & -- & 687\\
$(e;0,3,0;7)$ & 8 & $(2,1)$ & 1 & 1 & YES & YES & YES & $1.11$ & $(2,2)$ & -- & 688\\
$(e;0,3,0;7)$ & 8 & $(6,1)$ & 5 & 1 & YES & YES & YES & $1.00$ & $(2,2)$ & -- & 689\\
$(e;2,0,0;24)$ & 7 & $(2,1)$ & 1 & 2 & YES & YES & NO(2) & $0.90$ & $(4,1)$ & -- & 690\\
$(f;0,0,0;6)$ & 4 & $(19,5)$ & 7 & 1 & YES & YES & YES & $1.33$ & $(2,2)$ & -- & 691\\
$(f;0,1,0;7)$ & 5 & $(10,3)$ & 5 & 1 & YES & YES & YES & $0.88$ & $(4,1)$ & -- & 692\\
$(f;0,1,0;7)$ & 5 & $(12,5)$ & 5 & 1 & YES & YES & YES & $1.10$ & $(2,2)$ & -- & 693\\
$(f;0,1,0;7)$ & 5 & $(13,5)$ & 5 & 1 & YES & YES & YES & $1.10$ & $(2,2)$ & -- & 694\\
$(f;0,2,0;8)$ & 6 & $(17,3)$ & 7 & 1 & YES & YES & YES & $0.88$ & $(4,1)$ & -- & 695\\
$(i;0,0,0;9)$ & 5 & $(6,1)$ & 5 & 3 & YES & YES & YES & $1.00$ & $(4,1)$ & -- & 696\\
$(i;0,0,0;9)$ & 5 & $(9,4)$ & 5 & 9 & YES & YES & YES & $1.00$ & $(4,1)$ & -- & 697\\
$(i;0,0,0;9)$ & 5 & $(19,4)$ & 7 & 1 & YES & YES & YES & $1.27$ & $(2,2)$ & -- & 698\\
$(i;0,3,0;18)$ & 8 & $(5,1)$ & 4 & 1 & YES & YES & YES & $1.10$ & $(2,2)$ & -- & 699\\
$(j;0,0,0;8)$ & 5 & $(11,5)$ & 6 & 1 & YES & YES & YES & $1.27$ & $(2,2)$ & -- & 700\\
$(j;0,4,0;16)$ & 9 & $(6,1)$ & 5 & 2 & YES & YES & YES & $0.88$ & $(2,2)$ & -- & 701
\end{longtable}
\subsection{2 chains, $K^2 = 3$}
\begin{longtable}{|c|c|c|c|c|c|c|c|c|c|c|c|}
\hline
\multicolumn{12}{|c|}{2 chains, $K^2 = 3$}\\
\hline
$(n,a)$ & Len & $(n,a)$ & Len & GCD & Nef & $\mathbb Q$-ef & Obs 0 & $\overline c_1^2 / \overline c_2$ & $(P,K)$ & WH & Index\\
\hline
\endfirsthead

\hline
$(n,a)$ & Len & $(n,a)$ & Len & GCD & Nef & $\mathbb Q$-ef & Obs 0 & $\overline c_1^2 / \overline c_2$ & $(P,K)$ & WH & Index\\
\hline
\endhead
\hline
\endfoot

$(17,3)$ & 7 & $(14,5)$ & 6 & 1 & YES & YES & YES & $1.38$ & $(4,2)$ & -- & 702\\
$(19,6)$ & 8 & $(19,6)$ & 8 & 19 & YES & YES & YES & $1.82$ & $(2,3)$ & -- & 703\\
$(20,7)$ & 8 & $(17,3)$ & 7 & 1 & YES & YES & YES & $1.14$ & $(6,1)$ & NO & 704\\
$(20,7)$ & 8 & $(17,3)$ & 7 & 1 & YES & YES & YES & $1.14$ & $(6,1)$ & -- & 705\\
$(20,9)$ & 7 & $(18,7)$ & 6 & 2 & YES & YES & YES & $1.14$ & $(4,2)$ & -- & 706\\
$(23,10)$ & 7 & $(15,4)$ & 6 & 1 & YES & YES & YES & $1.56$ & $(2,3)$ & 759 & 707\\
$(23,10)$ & 7 & $(15,4)$ & 6 & 1 & YES & YES & YES & $1.56$ & $(2,3)$ & -- & 708\\
$(23,10)$ & 7 & $(18,7)$ & 6 & 1 & YES & YES & YES & $1.56$ & $(2,3)$ & NO & 709\\
$(23,10)$ & 7 & $(18,7)$ & 6 & 1 & YES & YES & YES & $1.56$ & $(2,3)$ & -- & 710\\
$(23,7)$ & 7 & $(20,7)$ & 8 & 1 & YES & YES & YES & $1.60$ & $(2,3)$ & -- & 711\\
$(24,11)$ & 8 & $(19,5)$ & 7 & 1 & YES & YES & YES & $1.38$ & $(4,2)$ & NO & 712\\
$(24,11)$ & 8 & $(21,5)$ & 8 & 3 & YES & YES & YES & $1.50$ & $(2,3)$ & NO & 713\\
$(24,11)$ & 8 & $(23,9)$ & 7 & 1 & YES & YES & YES & $1.38$ & $(4,2)$ & NO & 714\\
$(25,7)$ & 7 & $(9,4)$ & 5 & 1 & YES & YES & NO(2) & $1.58$ & $(2,3)$ & NO & 715\\
$(25,7)$ & 7 & $(9,4)$ & 5 & 1 & YES & YES & NO(2) & $1.58$ & $(2,3)$ & -- & 716\\
$(25,9)$ & 7 & $(13,6)$ & 7 & 1 & YES & YES & YES & $1.56$ & $(4,2)$ & -- & 717\\
$(27,10)$ & 7 & $(11,4)$ & 5 & 1 & YES & YES & YES & $1.56$ & $(2,3)$ & -- & 718\\
$(27,10)$ & 7 & $(11,5)$ & 6 & 1 & YES & YES & YES & $1.29$ & $(4,2)$ & -- & 719\\
$(27,11)$ & 8 & $(13,3)$ & 6 & 1 & YES & YES & YES & $1.70$ & $(2,3)$ & NO & 720\\
$(27,11)$ & 8 & $(13,3)$ & 6 & 1 & YES & YES & YES & $1.70$ & $(2,3)$ & -- & 721\\
$(27,11)$ & 8 & $(13,3)$ & 6 & 1 & YES & YES & YES & $1.70$ & $(2,3)$ & NO & 722\\
$(27,11)$ & 8 & $(24,11)$ & 8 & 3 & YES & YES & YES & $1.50$ & $(2,3)$ & NO & 723\\
$(28,11)$ & 8 & $(13,3)$ & 6 & 1 & YES & YES & YES & $1.70$ & $(2,3)$ & NO & 724\\
$(28,11)$ & 8 & $(13,3)$ & 6 & 1 & YES & YES & YES & $1.70$ & $(2,3)$ & -- & 725\\
$(28,5)$ & 8 & $(17,6)$ & 7 & 1 & YES & YES & YES & $1.55$ & $(2,3)$ & NO & 726\\
$(28,5)$ & 8 & $(17,6)$ & 7 & 1 & YES & YES & YES & $1.55$ & $(2,3)$ & -- & 727\\
$(29,11)$ & 7 & $(11,5)$ & 6 & 1 & YES & YES & YES & $1.75$ & $(2,3)$ & NO & 728\\
$(29,9)$ & 8 & $(12,5)$ & 5 & 1 & YES & YES & YES & $1.56$ & $(2,3)$ & NO & 729\\
$(29,9)$ & 8 & $(12,5)$ & 5 & 1 & YES & YES & YES & $1.56$ & $(2,3)$ & -- & 730\\
$(29,13)$ & 8 & $(13,4)$ & 6 & 1 & YES & YES & YES & $1.29$ & $(4,2)$ & NO & 731\\
$(29,13)$ & 8 & $(13,4)$ & 6 & 1 & YES & YES & YES & $1.29$ & $(4,2)$ & -- & 732\\
$(30,13)$ & 8 & $(9,4)$ & 5 & 3 & YES & YES & NO(2) & $1.50$ & $(2,3)$ & -- & 733\\
$(31,14)$ & 8 & $(6,1)$ & 5 & 1 & YES & YES & YES & $1.50$ & $(2,3)$ & -- & 734\\
$(31,13)$ & 7 & $(13,6)$ & 7 & 1 & YES & YES & YES & $1.50$ & $(2,3)$ & -- & 735\\
$(31,14)$ & 8 & $(13,5)$ & 5 & 1 & YES & YES & YES & $1.14$ & $(4,2)$ & -- & 736\\
$(31,9)$ & 8 & $(17,4)$ & 7 & 1 & YES & YES & YES & $1.43$ & $(4,2)$ & NO & 737\\
$(31,9)$ & 8 & $(17,4)$ & 7 & 1 & YES & YES & YES & $1.43$ & $(4,2)$ & -- & 738\\
$(31,13)$ & 7 & $(19,6)$ & 8 & 1 & YES & YES & YES & $1.57$ & $(2,3)$ & -- & 739\\
$(32,13)$ & 9 & $(28,11)$ & 8 & 4 & YES & YES & YES & $1.57$ & $(4,2)$ & NO & 740\\
$(33,10)$ & 8 & $(9,4)$ & 5 & 3 & YES & YES & YES & $1.29$ & $(4,2)$ & NO & 741\\
$(33,10)$ & 8 & $(9,4)$ & 5 & 3 & YES & YES & YES & $1.29$ & $(4,2)$ & -- & 742\\
$(33,14)$ & 8 & $(16,3)$ & 7 & 1 & YES & YES & YES & $1.44$ & $(2,3)$ & -- & 743\\
$(33,13)$ & 9 & $(19,3)$ & 8 & 1 & YES & YES & YES & $1.14$ & $(4,2)$ & NO & 744\\
$(33,13)$ & 9 & $(27,11)$ & 8 & 3 & YES & YES & YES & $1.57$ & $(4,2)$ & NO & 745\\
$(34,9)$ & 8 & $(12,5)$ & 5 & 2 & YES & YES & YES & $1.56$ & $(2,3)$ & NO & 746\\
$(34,9)$ & 8 & $(12,5)$ & 5 & 2 & YES & YES & YES & $1.56$ & $(2,3)$ & -- & 747\\
$(34,15)$ & 8 & $(14,5)$ & 6 & 2 & YES & YES & YES & $1.56$ & $(2,3)$ & -- & 748\\
$(34,15)$ & 8 & $(27,11)$ & 8 & 1 & YES & YES & YES & $1.56$ & $(2,3)$ & NO & 749\\
$(35,13)$ & 8 & $(9,4)$ & 5 & 1 & YES & YES & NO(2) & $1.55$ & $(2,3)$ & NO & 750\\
$(35,13)$ & 8 & $(17,3)$ & 7 & 1 & YES & YES & YES & $1.29$ & $(4,2)$ & NO & 751\\
$(35,13)$ & 8 & $(17,3)$ & 7 & 1 & YES & YES & YES & $1.29$ & $(4,2)$ & -- & 752\\
$(36,7)$ & 11 & $(34,5)$ & 10 & 2 & YES & YES & YES & $1.38$ & $(4,2)$ & NO & 753\\
$(36,7)$ & 11 & $(36,5)$ & 11 & 36 & YES & YES & YES & $1.50$ & $(2,3)$ & NO & 754\\
$(37,17)$ & 9 & $(5,1)$ & 4 & 1 & YES & YES & YES & $1.43$ & $(2,3)$ & -- & 755\\
$(37,17)$ & 9 & $(7,2)$ & 4 & 1 & YES & YES & YES & $1.44$ & $(2,3)$ & -- & 756\\
$(37,10)$ & 8 & $(9,4)$ & 5 & 1 & YES & YES & YES & $1.29$ & $(4,2)$ & NO & 757\\
$(37,10)$ & 8 & $(9,4)$ & 5 & 1 & YES & YES & YES & $1.29$ & $(4,2)$ & -- & 758\\
$(37,10)$ & 8 & $(9,4)$ & 5 & 1 & YES & YES & YES & $1.56$ & $(2,3)$ & 707 & 759\\
$(37,13)$ & 9 & $(9,4)$ & 5 & 1 & YES & YES & YES & $1.38$ & $(4,2)$ & -- & 760\\
$(37,14)$ & 8 & $(9,4)$ & 5 & 1 & YES & YES & NO(2) & $1.55$ & $(2,3)$ & NO & 761\\
$(37,17)$ & 9 & $(9,4)$ & 5 & 1 & YES & YES & YES & $1.43$ & $(2,3)$ & -- & 762\\
$(37,10)$ & 8 & $(11,4)$ & 5 & 1 & YES & YES & YES & $1.56$ & $(2,3)$ & NO & 763\\
$(37,10)$ & 8 & $(11,4)$ & 5 & 1 & YES & YES & YES & $1.56$ & $(2,3)$ & -- & 764\\
$(37,17)$ & 9 & $(12,5)$ & 5 & 1 & YES & YES & YES & $1.44$ & $(2,3)$ & NO & 765\\
$(37,13)$ & 9 & $(13,4)$ & 6 & 1 & YES & YES & YES & $1.60$ & $(2,3)$ & -- & 766\\
$(37,10)$ & 8 & $(16,5)$ & 7 & 1 & YES & YES & YES & $1.29$ & $(4,2)$ & NO & 767\\
$(37,17)$ & 9 & $(16,7)$ & 6 & 1 & YES & YES & YES & $1.44$ & $(2,3)$ & 863 & 768\\
$(38,11)$ & 9 & $(11,5)$ & 6 & 1 & YES & YES & YES & $1.70$ & $(2,3)$ & NO & 769\\
$(38,11)$ & 9 & $(11,5)$ & 6 & 1 & YES & YES & YES & $1.70$ & $(2,3)$ & -- & 770\\
$(38,17)$ & 9 & $(13,5)$ & 5 & 1 & YES & YES & YES & $1.60$ & $(2,3)$ & NO & 771\\
$(38,17)$ & 9 & $(16,3)$ & 7 & 2 & YES & YES & YES & $1.14$ & $(4,2)$ & -- & 772\\
$(39,11)$ & 9 & $(5,1)$ & 4 & 1 & YES & YES & YES & $1.43$ & $(6,1)$ & NO & 773\\
$(39,14)$ & 8 & $(8,3)$ & 4 & 1 & YES & YES & YES & $1.64$ & $(2,3)$ & -- & 774\\
$(39,14)$ & 8 & $(20,7)$ & 8 & 1 & YES & YES & YES & $1.14$ & $(6,1)$ & NO & 775\\
$(39,16)$ & 8 & $(21,5)$ & 8 & 3 & YES & YES & YES & $1.50$ & $(2,3)$ & NO & 776\\
$(40,11)$ & 8 & $(11,4)$ & 5 & 1 & YES & YES & YES & $1.64$ & $(2,3)$ & NO & 777\\
$(40,11)$ & 8 & $(11,4)$ & 5 & 1 & YES & YES & YES & $1.64$ & $(2,3)$ & -- & 778\\
$(40,17)$ & 9 & $(11,2)$ & 6 & 1 & YES & YES & YES & $1.44$ & $(2,3)$ & NO & 779\\
$(41,16)$ & 8 & $(16,5)$ & 7 & 1 & YES & YES & YES & $1.60$ & $(2,3)$ & NO & 780\\
$(41,16)$ & 8 & $(20,7)$ & 8 & 1 & YES & YES & YES & $1.60$ & $(2,3)$ & NO & 781\\
$(42,19)$ & 9 & $(7,2)$ & 4 & 7 & YES & YES & YES & $1.25$ & $(2,3)$ & -- & 782\\
$(42,19)$ & 9 & $(8,3)$ & 4 & 2 & YES & YES & YES & $1.14$ & $(4,2)$ & -- & 783\\
$(42,11)$ & 9 & $(10,3)$ & 5 & 2 & YES & YES & YES & $1.56$ & $(2,3)$ & NO & 784\\
$(42,19)$ & 9 & $(11,3)$ & 5 & 1 & YES & YES & YES & $1.14$ & $(4,2)$ & NO & 785\\
$(42,19)$ & 9 & $(13,4)$ & 6 & 1 & YES & YES & YES & $1.56$ & $(2,3)$ & NO & 786\\
$(42,19)$ & 9 & $(14,5)$ & 6 & 14 & YES & YES & YES & $1.56$ & $(2,3)$ & NO & 787\\
$(42,19)$ & 9 & $(16,3)$ & 7 & 2 & YES & YES & YES & $1.14$ & $(4,2)$ & NO & 788\\
$(43,16)$ & 9 & $(5,1)$ & 4 & 1 & YES & YES & YES & $1.50$ & $(4,2)$ & -- & 789\\
$(43,18)$ & 8 & $(11,5)$ & 6 & 1 & YES & YES & YES & $1.56$ & $(2,3)$ & -- & 790\\
$(44,17)$ & 8 & $(9,4)$ & 5 & 1 & YES & YES & YES & $1.29$ & $(4,2)$ & NO & 791\\
$(44,17)$ & 8 & $(9,4)$ & 5 & 1 & YES & YES & YES & $1.29$ & $(4,2)$ & -- & 792\\
$(45,13)$ & 10 & $(6,1)$ & 5 & 3 & YES & YES & YES & $1.50$ & $(6,1)$ & NO & 793\\
$(45,17)$ & 9 & $(7,2)$ & 4 & 1 & YES & YES & YES & $1.60$ & $(2,3)$ & NO & 794\\
$(45,17)$ & 9 & $(7,2)$ & 4 & 1 & YES & YES & YES & $1.60$ & $(2,3)$ & -- & 795\\
$(45,13)$ & 10 & $(9,2)$ & 5 & 9 & YES & YES & YES & $1.73$ & $(2,3)$ & NO & 796\\
$(45,13)$ & 10 & $(9,2)$ & 5 & 9 & YES & YES & YES & $1.73$ & $(2,3)$ & -- & 797\\
$(45,17)$ & 9 & $(12,5)$ & 5 & 3 & YES & YES & YES & $1.50$ & $(2,3)$ & -- & 798\\
$(45,8)$ & 9 & $(16,5)$ & 7 & 1 & YES & YES & YES & $1.50$ & $(2,3)$ & NO & 799\\
$(45,13)$ & 10 & $(19,6)$ & 8 & 1 & YES & YES & YES & $1.67$ & $(2,3)$ & NO & 800\\
$(45,13)$ & 10 & $(22,3)$ & 9 & 1 & YES & YES & YES & $1.44$ & $(2,3)$ & -- & 801\\
$(45,19)$ & 8 & $(40,17)$ & 9 & 5 & YES & YES & YES & $1.44$ & $(2,3)$ & 914 & 802\\
$(46,13)$ & 10 & $(23,7)$ & 7 & 23 & YES & YES & YES & $1.60$ & $(2,3)$ & NO & 803\\
$(47,15)$ & 11 & $(7,1)$ & 6 & 1 & YES & YES & YES & $1.50$ & $(2,3)$ & NO & 804\\
$(47,15)$ & 11 & $(7,1)$ & 6 & 1 & YES & YES & YES & $1.50$ & $(2,3)$ & -- & 805\\
$(47,18)$ & 8 & $(11,5)$ & 6 & 1 & YES & YES & YES & $1.64$ & $(2,3)$ & NO & 806\\
$(47,22)$ & 11 & $(12,5)$ & 5 & 1 & YES & YES & YES & $1.57$ & $(4,2)$ & -- & 807\\
$(48,17)$ & 9 & $(5,1)$ & 4 & 1 & YES & YES & YES & $1.56$ & $(2,3)$ & NO & 808\\
$(48,17)$ & 9 & $(5,1)$ & 4 & 1 & YES & YES & YES & $1.56$ & $(2,3)$ & -- & 809\\
$(49,22)$ & 9 & $(42,19)$ & 9 & 7 & YES & YES & YES & $1.14$ & $(4,2)$ & NO & 810\\
$(50,21)$ & 8 & $(11,5)$ & 6 & 1 & YES & YES & YES & $1.57$ & $(2,3)$ & -- & 811\\
$(50,21)$ & 8 & $(24,7)$ & 7 & 2 & YES & YES & YES & $1.71$ & $(2,3)$ & NO & 812\\
$(50,9)$ & 10 & $(36,7)$ & 11 & 2 & YES & YES & YES & $1.38$ & $(4,2)$ & 1064 & 813\\
$(51,16)$ & 10 & $(5,1)$ & 4 & 1 & YES & YES & YES & $1.56$ & $(2,3)$ & NO & 814\\
$(51,23)$ & 9 & $(7,3)$ & 4 & 1 & YES & YES & YES & $1.14$ & $(4,2)$ & -- & 815\\
$(51,16)$ & 10 & $(8,3)$ & 4 & 1 & YES & YES & YES & $1.29$ & $(4,2)$ & -- & 816\\
$(51,23)$ & 9 & $(42,19)$ & 9 & 3 & YES & YES & YES & $1.38$ & $(2,3)$ & NO & 817\\
$(53,11)$ & 10 & $(3,1)$ & 2 & 1 & YES & YES & YES & $1.50$ & $(2,3)$ & -- & 818\\
$(53,11)$ & 10 & $(4,1)$ & 3 & 1 & YES & YES & YES & $1.50$ & $(2,3)$ & NO & 819\\
$(53,11)$ & 10 & $(4,1)$ & 3 & 1 & YES & YES & YES & $1.50$ & $(2,3)$ & -- & 820\\
$(53,15)$ & 11 & $(5,1)$ & 4 & 1 & YES & YES & YES & $1.62$ & $(2,3)$ & NO & 821\\
$(53,20)$ & 10 & $(5,1)$ & 4 & 1 & YES & YES & YES & $1.50$ & $(4,2)$ & -- & 822\\
$(53,20)$ & 10 & $(5,1)$ & 4 & 1 & YES & YES & YES & $1.73$ & $(2,3)$ & NO & 823\\
$(53,20)$ & 10 & $(5,1)$ & 4 & 1 & YES & YES & YES & $1.73$ & $(2,3)$ & NO & 824\\
$(53,24)$ & 10 & $(7,2)$ & 4 & 1 & YES & YES & YES & $1.29$ & $(4,2)$ & NO & 825\\
$(53,14)$ & 9 & $(10,3)$ & 5 & 1 & YES & YES & YES & $1.50$ & $(2,3)$ & -- & 826\\
$(53,14)$ & 9 & $(11,3)$ & 5 & 1 & YES & YES & YES & $1.50$ & $(2,3)$ & -- & 827\\
$(53,11)$ & 10 & $(29,6)$ & 9 & 1 & YES & YES & YES & $1.50$ & $(2,3)$ & NO & 828\\
$(53,24)$ & 10 & $(29,13)$ & 8 & 1 & YES & YES & YES & $1.29$ & $(4,2)$ & NO & 829\\
$(53,24)$ & 10 & $(51,23)$ & 9 & 1 & YES & YES & YES & $1.14$ & $(4,2)$ & 985 & 830\\
$(53,11)$ & 10 & $(53,11)$ & 10 & 53 & YES & YES & YES & $1.50$ & $(2,3)$ & NO & 831\\
$(54,17)$ & 10 & $(5,1)$ & 4 & 1 & YES & YES & YES & $1.38$ & $(6,1)$ & NO & 832\\
$(54,17)$ & 10 & $(7,2)$ & 4 & 1 & YES & YES & YES & $1.44$ & $(2,3)$ & -- & 833\\
$(54,19)$ & 10 & $(7,3)$ & 4 & 1 & YES & YES & YES & $1.38$ & $(4,2)$ & -- & 834\\
$(54,17)$ & 10 & $(9,4)$ & 5 & 9 & YES & YES & YES & $1.43$ & $(2,3)$ & NO & 835\\
$(55,16)$ & 9 & $(14,5)$ & 6 & 1 & YES & YES & YES & $1.60$ & $(2,3)$ & NO & 836\\
$(56,15)$ & 9 & $(3,1)$ & 2 & 1 & YES & YES & YES & $1.60$ & $(2,3)$ & NO & 837\\
$(56,15)$ & 9 & $(7,2)$ & 4 & 7 & YES & YES & YES & $1.60$ & $(2,3)$ & NO & 838\\
$(56,15)$ & 9 & $(7,2)$ & 4 & 7 & YES & YES & YES & $1.60$ & $(2,3)$ & -- & 839\\
$(56,23)$ & 9 & $(28,11)$ & 8 & 28 & YES & YES & YES & $1.71$ & $(4,2)$ & NO & 840\\
$(56,15)$ & 9 & $(36,11)$ & 8 & 4 & YES & YES & YES & $1.57$ & $(2,3)$ & NO & 841\\
$(57,26)$ & 11 & $(13,2)$ & 7 & 1 & YES & YES & YES & $1.43$ & $(2,3)$ & -- & 842\\
$(58,15)$ & 11 & $(7,1)$ & 6 & 1 & YES & YES & YES & $1.50$ & $(2,3)$ & NO & 843\\
$(58,15)$ & 11 & $(7,1)$ & 6 & 1 & YES & YES & YES & $1.50$ & $(2,3)$ & -- & 844\\
$(58,15)$ & 11 & $(23,6)$ & 8 & 1 & YES & YES & YES & $1.50$ & $(2,3)$ & NO & 845\\
$(59,25)$ & 9 & $(7,3)$ & 4 & 1 & YES & YES & YES & $1.29$ & $(4,2)$ & -- & 846\\
$(59,18)$ & 9 & $(11,5)$ & 6 & 1 & YES & YES & YES & $1.57$ & $(2,3)$ & NO & 847\\
$(59,18)$ & 9 & $(11,5)$ & 6 & 1 & YES & YES & YES & $1.57$ & $(2,3)$ & -- & 848\\
$(59,23)$ & 9 & $(27,11)$ & 8 & 1 & YES & YES & YES & $1.71$ & $(4,2)$ & NO & 849\\
$(59,23)$ & 9 & $(33,13)$ & 9 & 1 & YES & YES & YES & $1.14$ & $(4,2)$ & NO & 850\\
$(59,25)$ & 9 & $(40,17)$ & 9 & 1 & YES & YES & YES & $1.44$ & $(2,3)$ & NO & 851\\
$(60,23)$ & 9 & $(4,1)$ & 3 & 4 & YES & YES & YES & $1.60$ & $(2,3)$ & -- & 852\\
$(60,19)$ & 11 & $(6,1)$ & 5 & 6 & YES & YES & YES & $1.29$ & $(4,2)$ & NO & 853\\
$(60,19)$ & 11 & $(6,1)$ & 5 & 6 & YES & YES & YES & $1.29$ & $(4,2)$ & -- & 854\\
$(60,11)$ & 11 & $(11,5)$ & 6 & 1 & YES & YES & YES & $1.50$ & $(2,3)$ & -- & 855\\
$(60,11)$ & 11 & $(14,5)$ & 6 & 2 & YES & YES & YES & $1.50$ & $(2,3)$ & -- & 856\\
$(60,11)$ & 11 & $(36,7)$ & 11 & 12 & YES & YES & YES & $1.50$ & $(2,3)$ & NO & 857\\
$(61,19)$ & 10 & $(3,1)$ & 2 & 1 & YES & YES & YES & $1.73$ & $(2,3)$ & NO & 858\\
$(61,16)$ & 10 & $(4,1)$ & 3 & 1 & YES & YES & YES & $1.25$ & $(4,2)$ & -- & 859\\
$(61,24)$ & 10 & $(4,1)$ & 3 & 1 & YES & YES & YES & $1.38$ & $(4,2)$ & NO & 860\\
$(61,25)$ & 9 & $(5,2)$ & 3 & 1 & YES & YES & YES & $1.56$ & $(2,3)$ & NO & 861\\
$(61,25)$ & 9 & $(5,2)$ & 3 & 1 & YES & YES & YES & $1.56$ & $(2,3)$ & -- & 862\\
$(61,28)$ & 10 & $(7,3)$ & 4 & 1 & YES & YES & YES & $1.44$ & $(2,3)$ & 768 & 863\\
$(61,24)$ & 10 & $(9,4)$ & 5 & 1 & YES & YES & YES & $1.38$ & $(4,2)$ & NO & 864\\
$(61,25)$ & 9 & $(32,13)$ & 9 & 1 & YES & YES & YES & $1.29$ & $(4,2)$ & NO & 865\\
$(62,29)$ & 12 & $(11,3)$ & 5 & 1 & YES & YES & YES & $1.57$ & $(4,2)$ & NO & 866\\
$(63,26)$ & 9 & $(5,1)$ & 4 & 1 & YES & YES & YES & $1.56$ & $(2,3)$ & -- & 867\\
$(64,17)$ & 10 & $(9,2)$ & 5 & 1 & YES & YES & YES & $1.44$ & $(2,3)$ & -- & 868\\
$(64,23)$ & 9 & $(17,6)$ & 7 & 1 & YES & YES & YES & $1.55$ & $(2,3)$ & NO & 869\\
$(65,17)$ & 10 & $(65,17)$ & 10 & 65 & YES & YES & YES & $1.29$ & $(6,1)$ & NO & 870\\
$(67,26)$ & 9 & $(5,2)$ & 3 & 1 & YES & YES & YES & $1.50$ & $(2,3)$ & -- & 871\\
$(67,28)$ & 10 & $(5,1)$ & 4 & 1 & YES & YES & YES & $1.56$ & $(2,3)$ & -- & 872\\
$(67,24)$ & 10 & $(7,2)$ & 4 & 1 & YES & YES & YES & $1.29$ & $(4,2)$ & NO & 873\\
$(67,28)$ & 10 & $(7,3)$ & 4 & 1 & YES & YES & YES & $1.50$ & $(2,3)$ & -- & 874\\
$(67,16)$ & 11 & $(9,4)$ & 5 & 1 & YES & YES & YES & $1.56$ & $(2,3)$ & NO & 875\\
$(67,28)$ & 10 & $(13,5)$ & 5 & 1 & YES & YES & YES & $1.50$ & $(2,3)$ & NO & 876\\
$(67,26)$ & 9 & $(21,8)$ & 6 & 1 & YES & YES & YES & $1.50$ & $(2,3)$ & 1016 & 877\\
$(67,28)$ & 10 & $(43,18)$ & 8 & 1 & YES & YES & YES & $1.56$ & $(2,3)$ & 989 & 878\\
$(68,21)$ & 11 & $(6,1)$ & 5 & 2 & YES & YES & YES & $1.56$ & $(2,3)$ & -- & 879\\
$(69,20)$ & 10 & $(10,3)$ & 5 & 1 & YES & YES & YES & $1.56$ & $(2,3)$ & NO & 880\\
$(69,13)$ & 11 & $(60,11)$ & 11 & 3 & YES & YES & YES & $1.50$ & $(2,3)$ & NO & 881\\
$(70,11)$ & 11 & $(11,5)$ & 6 & 1 & YES & YES & YES & $1.64$ & $(2,3)$ & NO & 882\\
$(70,11)$ & 11 & $(11,5)$ & 6 & 1 & YES & YES & YES & $1.64$ & $(2,3)$ & -- & 883\\
$(71,13)$ & 12 & $(7,3)$ & 4 & 1 & YES & YES & YES & $1.14$ & $(4,2)$ & -- & 884\\
$(71,21)$ & 9 & $(8,3)$ & 4 & 1 & YES & YES & YES & $1.50$ & $(2,3)$ & NO & 885\\
$(71,20)$ & 10 & $(11,3)$ & 5 & 1 & YES & YES & YES & $1.56$ & $(2,3)$ & NO & 886\\
$(71,25)$ & 11 & $(11,2)$ & 6 & 1 & YES & YES & YES & $1.60$ & $(2,3)$ & -- & 887\\
$(71,21)$ & 9 & $(16,5)$ & 7 & 1 & YES & YES & YES & $1.50$ & $(2,3)$ & NO & 888\\
$(71,13)$ & 12 & $(19,3)$ & 8 & 1 & YES & YES & YES & $1.14$ & $(4,2)$ & NO & 889\\
$(72,19)$ & 10 & $(5,1)$ & 4 & 1 & YES & YES & YES & $1.64$ & $(2,3)$ & NO & 890\\
$(72,19)$ & 10 & $(5,1)$ & 4 & 1 & YES & YES & YES & $1.64$ & $(2,3)$ & -- & 891\\
$(72,19)$ & 10 & $(7,2)$ & 4 & 1 & YES & YES & YES & $1.60$ & $(2,3)$ & NO & 892\\
$(72,25)$ & 12 & $(7,3)$ & 4 & 1 & YES & YES & YES & $1.71$ & $(4,2)$ & -- & 893\\
$(72,25)$ & 12 & $(8,3)$ & 4 & 8 & YES & YES & YES & $1.71$ & $(4,2)$ & NO & 894\\
$(72,25)$ & 12 & $(8,3)$ & 4 & 8 & YES & YES & YES & $1.71$ & $(4,2)$ & -- & 895\\
$(72,19)$ & 10 & $(16,5)$ & 7 & 8 & YES & YES & YES & $1.56$ & $(2,3)$ & NO & 896\\
$(73,20)$ & 11 & $(2,1)$ & 1 & 1 & YES & YES & YES & $1.50$ & $(4,2)$ & -- & 897\\
$(73,20)$ & 11 & $(2,1)$ & 1 & 1 & YES & YES & YES & $1.60$ & $(4,2)$ & NO & 898\\
$(73,20)$ & 11 & $(3,1)$ & 2 & 1 & YES & YES & YES & $1.60$ & $(2,3)$ & -- & 899\\
$(73,20)$ & 11 & $(3,1)$ & 2 & 1 & YES & YES & YES & $1.60$ & $(2,3)$ & NO & 900\\
$(73,23)$ & 11 & $(3,1)$ & 2 & 1 & YES & YES & YES & $1.38$ & $(2,3)$ & NO & 901\\
$(73,31)$ & 10 & $(3,1)$ & 2 & 1 & YES & YES & YES & $1.56$ & $(2,3)$ & -- & 902\\
$(73,33)$ & 10 & $(3,1)$ & 2 & 1 & YES & YES & YES & $1.25$ & $(2,3)$ & -- & 903\\
$(73,28)$ & 10 & $(4,1)$ & 3 & 1 & YES & YES & YES & $1.60$ & $(2,3)$ & -- & 904\\
$(73,11)$ & 11 & $(6,1)$ & 5 & 1 & YES & YES & YES & $1.14$ & $(6,1)$ & NO & 905\\
$(73,11)$ & 11 & $(6,1)$ & 5 & 1 & YES & YES & YES & $1.14$ & $(6,1)$ & NO & 906\\
$(73,11)$ & 11 & $(6,1)$ & 5 & 1 & YES & YES & YES & $1.14$ & $(6,1)$ & -- & 907\\
$(73,19)$ & 11 & $(6,1)$ & 5 & 1 & YES & YES & YES & $1.29$ & $(4,2)$ & NO & 908\\
$(73,19)$ & 11 & $(6,1)$ & 5 & 1 & YES & YES & YES & $1.29$ & $(4,2)$ & -- & 909\\
$(73,28)$ & 10 & $(6,1)$ & 5 & 1 & YES & YES & YES & $1.60$ & $(2,3)$ & -- & 910\\
$(73,20)$ & 11 & $(7,2)$ & 4 & 1 & YES & YES & YES & $1.50$ & $(4,2)$ & NO & 911\\
$(73,31)$ & 10 & $(12,5)$ & 5 & 1 & YES & YES & YES & $1.56$ & $(2,3)$ & NO & 912\\
$(73,19)$ & 11 & $(19,5)$ & 7 & 1 & YES & YES & YES & $1.29$ & $(4,2)$ & NO & 913\\
$(73,31)$ & 10 & $(19,8)$ & 6 & 1 & YES & YES & YES & $1.44$ & $(2,3)$ & 802 & 914\\
$(73,33)$ & 10 & $(20,9)$ & 7 & 1 & YES & YES & YES & $1.38$ & $(2,3)$ & NO & 915\\
$(73,28)$ & 10 & $(34,13)$ & 7 & 1 & YES & YES & YES & $1.60$ & $(2,3)$ & NO & 916\\
$(73,26)$ & 11 & $(59,21)$ & 10 & 1 & YES & YES & YES & $1.29$ & $(4,2)$ & NO & 917\\
$(73,28)$ & 10 & $(60,23)$ & 9 & 1 & YES & YES & YES & $1.60$ & $(2,3)$ & NO & 918\\
$(74,29)$ & 10 & $(4,1)$ & 3 & 2 & YES & YES & YES & $1.29$ & $(4,2)$ & NO & 919\\
$(74,29)$ & 10 & $(4,1)$ & 3 & 2 & YES & YES & YES & $1.29$ & $(4,2)$ & -- & 920\\
$(74,29)$ & 10 & $(7,3)$ & 4 & 1 & YES & YES & YES & $1.56$ & $(2,3)$ & NO & 921\\
$(74,29)$ & 10 & $(33,13)$ & 9 & 1 & YES & YES & YES & $1.14$ & $(4,2)$ & 1034 & 922\\
$(76,35)$ & 12 & $(37,17)$ & 9 & 1 & YES & YES & YES & $1.43$ & $(2,3)$ & NO & 923\\
$(77,32)$ & 11 & $(4,1)$ & 3 & 1 & YES & YES & YES & $1.56$ & $(4,2)$ & -- & 924\\
$(77,16)$ & 11 & $(9,4)$ & 5 & 1 & YES & YES & YES & $1.56$ & $(2,3)$ & NO & 925\\
$(77,34)$ & 10 & $(25,11)$ & 7 & 1 & YES & YES & YES & $1.44$ & $(2,3)$ & NO & 926\\
$(79,29)$ & 9 & $(2,1)$ & 1 & 1 & YES & YES & NO(2) & $1.40$ & $(4,2)$ & NO & 927\\
$(79,31)$ & 10 & $(3,1)$ & 2 & 1 & YES & YES & YES & $1.60$ & $(2,3)$ & -- & 928\\
$(79,33)$ & 11 & $(13,2)$ & 7 & 1 & YES & YES & YES & $1.43$ & $(2,3)$ & -- & 929\\
$(79,33)$ & 11 & $(74,31)$ & 9 & 1 & YES & YES & YES & $1.57$ & $(2,3)$ & NO & 930\\
$(80,19)$ & 11 & $(3,1)$ & 2 & 1 & YES & YES & YES & $1.73$ & $(2,3)$ & NO & 931\\
$(80,19)$ & 11 & $(3,1)$ & 2 & 1 & YES & YES & YES & $1.73$ & $(2,3)$ & -- & 932\\
$(82,31)$ & 10 & $(3,1)$ & 2 & 1 & YES & YES & YES & $1.60$ & $(2,3)$ & NO & 933\\
$(82,31)$ & 10 & $(3,1)$ & 2 & 1 & YES & YES & YES & $1.60$ & $(2,3)$ & -- & 934\\
$(82,31)$ & 10 & $(82,31)$ & 10 & 82 & YES & YES & YES & $1.50$ & $(2,3)$ & NO & 935\\
$(83,23)$ & 10 & $(2,1)$ & 1 & 1 & YES & YES & YES & $1.55$ & $(2,3)$ & -- & 936\\
$(83,24)$ & 11 & $(2,1)$ & 1 & 1 & YES & YES & YES & $1.50$ & $(4,2)$ & -- & 937\\
$(83,24)$ & 11 & $(3,1)$ & 2 & 1 & YES & YES & YES & $1.70$ & $(2,3)$ & NO & 938\\
$(83,24)$ & 11 & $(3,1)$ & 2 & 1 & YES & YES & YES & $1.70$ & $(2,3)$ & -- & 939\\
$(83,36)$ & 10 & $(4,1)$ & 3 & 1 & YES & YES & YES & $1.56$ & $(2,3)$ & NO & 940\\
$(83,36)$ & 10 & $(4,1)$ & 3 & 1 & YES & YES & YES & $1.56$ & $(2,3)$ & -- & 941\\
$(83,36)$ & 10 & $(5,2)$ & 3 & 1 & YES & YES & YES & $1.56$ & $(2,3)$ & NO & 942\\
$(83,29)$ & 12 & $(11,3)$ & 5 & 1 & YES & YES & YES & $1.57$ & $(4,2)$ & NO & 943\\
$(83,13)$ & 11 & $(17,6)$ & 7 & 1 & YES & YES & YES & $1.57$ & $(2,3)$ & -- & 944\\
$(83,24)$ & 11 & $(38,11)$ & 9 & 1 & YES & YES & YES & $1.38$ & $(4,2)$ & NO & 945\\
$(83,29)$ & 12 & $(49,17)$ & 11 & 1 & YES & YES & YES & $1.57$ & $(4,2)$ & 1081 & 946\\
$(84,13)$ & 13 & $(5,1)$ & 4 & 1 & YES & YES & YES & $1.56$ & $(2,3)$ & -- & 947\\
$(84,37)$ & 10 & $(5,2)$ & 3 & 1 & YES & YES & YES & $1.50$ & $(2,3)$ & NO & 948\\
$(84,13)$ & 13 & $(7,3)$ & 4 & 7 & YES & YES & YES & $1.29$ & $(4,2)$ & NO & 949\\
$(85,24)$ & 11 & $(4,1)$ & 3 & 1 & YES & YES & YES & $1.70$ & $(2,3)$ & NO & 950\\
$(85,24)$ & 11 & $(4,1)$ & 3 & 1 & YES & YES & YES & $1.70$ & $(2,3)$ & -- & 951\\
$(85,24)$ & 11 & $(4,1)$ & 3 & 1 & YES & YES & YES & $1.70$ & $(2,3)$ & NO & 952\\
$(85,24)$ & 11 & $(5,1)$ & 4 & 5 & YES & YES & YES & $1.25$ & $(2,3)$ & -- & 953\\
$(85,36)$ & 10 & $(33,14)$ & 8 & 1 & YES & YES & YES & $1.44$ & $(2,3)$ & 976 & 954\\
$(86,27)$ & 11 & $(2,1)$ & 1 & 2 & YES & YES & YES & $1.38$ & $(4,2)$ & -- & 955\\
$(86,35)$ & 11 & $(5,2)$ & 3 & 1 & YES & YES & YES & $1.56$ & $(2,3)$ & -- & 956\\
$(86,35)$ & 11 & $(9,4)$ & 5 & 1 & YES & YES & YES & $1.56$ & $(2,3)$ & NO & 957\\
$(86,23)$ & 11 & $(41,11)$ & 8 & 1 & YES & YES & YES & $1.44$ & $(2,3)$ & NO & 958\\
$(86,33)$ & 11 & $(86,33)$ & 11 & 86 & YES & YES & YES & $1.56$ & $(2,3)$ & NO & 959\\
$(87,20)$ & 12 & $(8,1)$ & 7 & 1 & YES & YES & YES & $1.38$ & $(2,3)$ & NO & 960\\
$(87,20)$ & 12 & $(48,11)$ & 9 & 3 & YES & YES & YES & $1.38$ & $(2,3)$ & NO & 961\\
$(88,31)$ & 12 & $(6,1)$ & 5 & 2 & YES & YES & YES & $1.38$ & $(4,2)$ & NO & 962\\
$(88,31)$ & 12 & $(54,19)$ & 10 & 2 & YES & YES & YES & $1.38$ & $(4,2)$ & 1035 & 963\\
$(89,24)$ & 10 & $(2,1)$ & 1 & 1 & YES & YES & NO(2) & $1.40$ & $(4,2)$ & -- & 964\\
$(89,26)$ & 10 & $(3,1)$ & 2 & 1 & YES & YES & NO(2) & $1.50$ & $(4,2)$ & NO & 965\\
$(89,26)$ & 10 & $(3,1)$ & 2 & 1 & YES & YES & NO(2) & $1.50$ & $(4,2)$ & -- & 966\\
$(89,40)$ & 11 & $(3,1)$ & 2 & 1 & YES & YES & YES & $1.60$ & $(2,3)$ & NO & 967\\
$(89,40)$ & 11 & $(3,1)$ & 2 & 1 & YES & YES & YES & $1.60$ & $(2,3)$ & -- & 968\\
$(89,26)$ & 10 & $(4,1)$ & 3 & 1 & YES & YES & NO(2) & $1.40$ & $(4,2)$ & -- & 969\\
$(89,35)$ & 11 & $(6,1)$ & 5 & 1 & YES & YES & YES & $1.29$ & $(4,2)$ & NO & 970\\
$(89,35)$ & 11 & $(6,1)$ & 5 & 1 & YES & YES & YES & $1.29$ & $(4,2)$ & -- & 971\\
$(91,27)$ & 10 & $(2,1)$ & 1 & 1 & YES & YES & NO(2) & $1.40$ & $(4,2)$ & NO & 972\\
$(91,29)$ & 13 & $(7,1)$ & 6 & 7 & YES & YES & YES & $1.60$ & $(2,3)$ & NO & 973\\
$(91,29)$ & 13 & $(13,4)$ & 6 & 13 & YES & YES & YES & $1.60$ & $(2,3)$ & NO & 974\\
$(92,39)$ & 10 & $(2,1)$ & 1 & 2 & YES & YES & YES & $1.56$ & $(2,3)$ & NO & 975\\
$(92,39)$ & 10 & $(26,11)$ & 7 & 2 & YES & YES & YES & $1.44$ & $(2,3)$ & 954 & 976\\
$(93,26)$ & 10 & $(7,3)$ & 4 & 1 & YES & YES & YES & $1.50$ & $(2,3)$ & NO & 977\\
$(93,34)$ & 10 & $(41,15)$ & 8 & 1 & YES & YES & NO(2) & $1.33$ & $(4,2)$ & NO & 978\\
$(94,35)$ & 11 & $(2,1)$ & 1 & 2 & YES & YES & YES & $1.70$ & $(2,3)$ & -- & 979\\
$(94,39)$ & 10 & $(3,1)$ & 2 & 1 & YES & YES & NO(2) & $1.44$ & $(4,2)$ & -- & 980\\
$(94,35)$ & 11 & $(5,2)$ & 3 & 1 & YES & YES & YES & $1.70$ & $(2,3)$ & NO & 981\\
$(94,35)$ & 11 & $(43,16)$ & 9 & 1 & YES & YES & YES & $1.60$ & $(2,3)$ & NO & 982\\
$(94,29)$ & 13 & $(68,21)$ & 11 & 2 & YES & YES & YES & $1.50$ & $(2,3)$ & 1059 & 983\\
$(95,44)$ & 12 & $(4,1)$ & 3 & 1 & YES & YES & YES & $1.60$ & $(2,3)$ & NO & 984\\
$(95,43)$ & 11 & $(20,9)$ & 7 & 5 & YES & YES & YES & $1.14$ & $(4,2)$ & 830 & 985\\
$(95,43)$ & 11 & $(42,19)$ & 9 & 1 & YES & YES & YES & $1.14$ & $(4,2)$ & NO & 986\\
$(97,20)$ & 12 & $(4,1)$ & 3 & 1 & YES & YES & YES & $1.33$ & $(2,3)$ & -- & 987\\
$(98,41)$ & 10 & $(5,1)$ & 4 & 1 & YES & YES & YES & $1.44$ & $(2,3)$ & -- & 988\\
$(98,41)$ & 10 & $(12,5)$ & 5 & 2 & YES & YES & YES & $1.56$ & $(2,3)$ & 878 & 989\\
$(98,29)$ & 10 & $(36,11)$ & 8 & 2 & YES & YES & YES & $1.57$ & $(2,3)$ & NO & 990\\
$(99,46)$ & 12 & $(5,1)$ & 4 & 1 & YES & YES & YES & $1.60$ & $(2,3)$ & -- & 991\\
$(99,31)$ & 13 & $(99,31)$ & 13 & 99 & YES & YES & YES & $1.67$ & $(2,3)$ & NO & 992\\
$(100,29)$ & 11 & $(4,1)$ & 3 & 4 & YES & YES & YES & $1.29$ & $(4,2)$ & NO & 993\\
$(100,29)$ & 11 & $(4,1)$ & 3 & 4 & YES & YES & YES & $1.29$ & $(4,2)$ & -- & 994\\
$(101,41)$ & 12 & $(7,3)$ & 4 & 1 & YES & YES & YES & $1.50$ & $(2,3)$ & NO & 995\\
$(101,41)$ & 12 & $(13,5)$ & 5 & 1 & YES & YES & YES & $1.57$ & $(4,2)$ & NO & 996\\
$(103,37)$ & 10 & $(8,3)$ & 4 & 1 & YES & YES & YES & $1.50$ & $(2,3)$ & 1040 & 997\\
$(103,47)$ & 12 & $(57,26)$ & 11 & 1 & YES & YES & YES & $1.43$ & $(2,3)$ & NO & 998\\
$(104,47)$ & 11 & $(2,1)$ & 1 & 2 & YES & YES & YES & $1.29$ & $(4,2)$ & -- & 999\\
$(104,47)$ & 11 & $(11,5)$ & 6 & 1 & YES & YES & YES & $1.29$ & $(4,2)$ & NO & 1000\\
$(104,41)$ & 12 & $(12,5)$ & 5 & 4 & YES & YES & YES & $1.57$ & $(4,2)$ & NO & 1001\\
$(104,47)$ & 11 & $(42,19)$ & 9 & 2 & YES & YES & YES & $1.14$ & $(4,2)$ & 1020 & 1002\\
$(105,44)$ & 10 & $(2,1)$ & 1 & 1 & YES & YES & YES & $1.50$ & $(2,3)$ & NO & 1003\\
$(106,39)$ & 11 & $(106,39)$ & 11 & 106 & YES & YES & YES & $1.56$ & $(2,3)$ & NO & 1004\\
$(108,41)$ & 10 & $(2,1)$ & 1 & 2 & YES & YES & YES & $1.60$ & $(2,3)$ & -- & 1005\\
$(108,41)$ & 10 & $(7,3)$ & 4 & 1 & YES & YES & YES & $1.50$ & $(2,3)$ & NO & 1006\\
$(109,23)$ & 12 & $(3,1)$ & 2 & 1 & YES & YES & YES & $1.44$ & $(2,3)$ & -- & 1007\\
$(109,40)$ & 10 & $(3,1)$ & 2 & 1 & YES & YES & YES & $1.50$ & $(2,3)$ & NO & 1008\\
$(109,40)$ & 10 & $(3,1)$ & 2 & 1 & YES & YES & YES & $1.50$ & $(2,3)$ & -- & 1009\\
$(109,23)$ & 12 & $(33,7)$ & 8 & 1 & YES & YES & YES & $1.44$ & $(2,3)$ & NO & 1010\\
$(111,32)$ & 13 & $(3,1)$ & 2 & 3 & YES & YES & YES & $1.70$ & $(2,3)$ & NO & 1011\\
$(112,33)$ & 12 & $(3,1)$ & 2 & 1 & NO & YES & YES & $1.60$ & $(2,3)$ & -- & 1012\\
$(113,32)$ & 13 & $(2,1)$ & 1 & 1 & YES & YES & YES & $1.70$ & $(2,3)$ & NO & 1013\\
$(113,20)$ & 13 & $(8,1)$ & 7 & 1 & YES & YES & YES & $1.38$ & $(2,3)$ & NO & 1014\\
$(115,44)$ & 10 & $(2,1)$ & 1 & 1 & YES & YES & YES & $1.50$ & $(2,3)$ & NO & 1015\\
$(115,44)$ & 10 & $(5,2)$ & 3 & 5 & YES & YES & YES & $1.50$ & $(2,3)$ & 877 & 1016\\
$(115,18)$ & 12 & $(6,1)$ & 5 & 1 & NO & YES & YES & $1.14$ & $(6,1)$ & -- & 1017\\
$(115,52)$ & 11 & $(9,4)$ & 5 & 1 & YES & YES & YES & $1.14$ & $(4,2)$ & NO & 1018\\
$(115,24)$ & 12 & $(14,3)$ & 6 & 1 & YES & YES & YES & $1.44$ & $(2,3)$ & NO & 1019\\
$(115,52)$ & 11 & $(31,14)$ & 8 & 1 & YES & YES & YES & $1.14$ & $(4,2)$ & 1002 & 1020\\
$(116,35)$ & 12 & $(3,1)$ & 2 & 1 & NO & YES & YES & $1.70$ & $(2,3)$ & -- & 1021\\
$(118,51)$ & 12 & $(5,2)$ & 3 & 1 & YES & YES & YES & $1.56$ & $(2,3)$ & NO & 1022\\
$(119,37)$ & 11 & $(3,1)$ & 2 & 1 & YES & YES & NO(2) & $1.40$ & $(4,2)$ & -- & 1023\\
$(119,45)$ & 11 & $(8,3)$ & 4 & 1 & YES & YES & YES & $1.29$ & $(4,2)$ & NO & 1024\\
$(119,37)$ & 11 & $(45,14)$ & 9 & 1 & YES & YES & NO(2) & $1.40$ & $(4,2)$ & NO & 1025\\
$(120,53)$ & 11 & $(2,1)$ & 1 & 2 & NO & YES & YES & $1.56$ & $(2,3)$ & -- & 1026\\
$(120,43)$ & 11 & $(3,1)$ & 2 & 3 & YES & YES & YES & $1.29$ & $(4,2)$ & NO & 1027\\
$(121,50)$ & 10 & $(2,1)$ & 1 & 1 & NO & YES & NO(2) & $1.40$ & $(4,2)$ & -- & 1028\\
$(121,35)$ & 12 & $(45,13)$ & 10 & 1 & YES & YES & YES & $1.44$ & $(2,3)$ & 1041 & 1029\\
$(121,32)$ & 11 & $(53,14)$ & 9 & 1 & YES & YES & YES & $1.50$ & $(2,3)$ & 1052 & 1030\\
$(122,37)$ & 11 & $(3,1)$ & 2 & 1 & NO & YES & NO(2) & $1.50$ & $(4,2)$ & -- & 1031\\
$(124,37)$ & 12 & $(7,2)$ & 4 & 1 & YES & YES & YES & $1.44$ & $(2,3)$ & NO & 1032\\
$(125,24)$ & 13 & $(3,1)$ & 2 & 1 & YES & YES & YES & $1.44$ & $(2,3)$ & NO & 1033\\
$(125,49)$ & 11 & $(5,2)$ & 3 & 5 & YES & YES & YES & $1.14$ & $(4,2)$ & 922 & 1034\\
$(125,44)$ & 12 & $(17,6)$ & 7 & 1 & YES & YES & YES & $1.38$ & $(4,2)$ & 963 & 1035\\
$(125,44)$ & 12 & $(71,25)$ & 11 & 1 & YES & YES & YES & $1.60$ & $(2,3)$ & NO & 1036\\
$(126,55)$ & 11 & $(2,1)$ & 1 & 2 & YES & YES & YES & $1.29$ & $(4,2)$ & NO & 1037\\
$(127,46)$ & 12 & $(5,1)$ & 4 & 1 & YES & YES & YES & $1.38$ & $(4,2)$ & -- & 1038\\
$(127,54)$ & 12 & $(7,3)$ & 4 & 1 & YES & YES & YES & $1.43$ & $(2,3)$ & NO & 1039\\
$(128,47)$ & 10 & $(3,1)$ & 2 & 1 & YES & YES & YES & $1.50$ & $(2,3)$ & 997 & 1040\\
$(128,37)$ & 12 & $(38,11)$ & 9 & 2 & YES & YES & YES & $1.44$ & $(2,3)$ & 1029 & 1041\\
$(131,39)$ & 11 & $(10,3)$ & 5 & 1 & YES & YES & YES & $1.50$ & $(2,3)$ & NO & 1042\\
$(132,23)$ & 13 & $(3,1)$ & 2 & 3 & YES & YES & YES & $1.44$ & $(2,3)$ & NO & 1043\\
$(132,47)$ & 12 & $(4,1)$ & 3 & 4 & YES & YES & YES & $1.56$ & $(2,3)$ & NO & 1044\\
$(134,35)$ & 13 & $(65,17)$ & 10 & 1 & YES & YES & YES & $1.43$ & $(2,3)$ & NO & 1045\\
$(137,53)$ & 11 & $(137,53)$ & 11 & 137 & YES & YES & YES & $1.38$ & $(2,3)$ & NO & 1046\\
$(139,53)$ & 12 & $(2,1)$ & 1 & 1 & NO & YES & YES & $1.60$ & $(4,2)$ & -- & 1047\\
$(139,61)$ & 11 & $(2,1)$ & 1 & 1 & NO & YES & YES & $1.50$ & $(2,3)$ & -- & 1048\\
$(140,37)$ & 11 & $(2,1)$ & 1 & 2 & YES & YES & YES & $1.50$ & $(2,3)$ & -- & 1049\\
$(140,37)$ & 11 & $(5,1)$ & 4 & 5 & YES & YES & NO(2) & $1.40$ & $(4,2)$ & NO & 1050\\
$(140,37)$ & 11 & $(19,5)$ & 7 & 1 & YES & YES & YES & $1.50$ & $(2,3)$ & NO & 1051\\
$(140,37)$ & 11 & $(34,9)$ & 8 & 2 & YES & YES & YES & $1.50$ & $(2,3)$ & 1030 & 1052\\
$(140,37)$ & 11 & $(53,14)$ & 9 & 1 & YES & YES & YES & $1.50$ & $(2,3)$ & NO & 1053\\
$(141,59)$ & 11 & $(50,21)$ & 8 & 1 & YES & YES & YES & $1.71$ & $(2,3)$ & NO & 1054\\
$(142,39)$ & 11 & $(11,3)$ & 5 & 1 & YES & YES & YES & $1.40$ & $(2,3)$ & NO & 1055\\
$(143,54)$ & 12 & $(2,1)$ & 1 & 1 & YES & YES & YES & $1.56$ & $(2,3)$ & NO & 1056\\
$(146,61)$ & 12 & $(6,1)$ & 5 & 2 & YES & YES & YES & $1.43$ & $(2,3)$ & -- & 1057\\
$(149,42)$ & 12 & $(3,1)$ & 2 & 1 & NO & YES & YES & $1.73$ & $(2,3)$ & -- & 1058\\
$(149,46)$ & 13 & $(13,4)$ & 6 & 1 & YES & YES & YES & $1.50$ & $(2,3)$ & 983 & 1059\\
$(150,47)$ & 13 & $(2,1)$ & 1 & 2 & YES & YES & YES & $1.56$ & $(2,3)$ & NO & 1060\\
$(151,53)$ & 12 & $(2,1)$ & 1 & 1 & YES & YES & YES & $1.60$ & $(2,3)$ & NO & 1061\\
$(156,73)$ & 14 & $(3,1)$ & 2 & 3 & YES & YES & YES & $1.57$ & $(4,2)$ & NO & 1062\\
$(160,31)$ & 15 & $(2,1)$ & 1 & 2 & YES & YES & YES & $1.38$ & $(4,2)$ & -- & 1063\\
$(160,31)$ & 15 & $(6,1)$ & 5 & 2 & YES & YES & YES & $1.38$ & $(4,2)$ & 813 & 1064\\
$(160,31)$ & 15 & $(36,7)$ & 11 & 4 & YES & YES & YES & $1.50$ & $(2,3)$ & NO & 1065\\
$(160,31)$ & 15 & $(160,31)$ & 15 & 160 & YES & YES & YES & $1.50$ & $(2,3)$ & NO & 1066\\
$(161,51)$ & 13 & $(19,6)$ & 8 & 1 & YES & YES & YES & $1.60$ & $(2,3)$ & NO & 1067\\
$(165,64)$ & 11 & $(2,1)$ & 1 & 1 & NO & YES & NO(2) & $1.40$ & $(4,2)$ & -- & 1068\\
$(167,58)$ & 14 & $(2,1)$ & 1 & 1 & YES & YES & YES & $1.71$ & $(4,2)$ & -- & 1069\\
$(167,58)$ & 14 & $(72,25)$ & 12 & 1 & YES & YES & YES & $1.71$ & $(4,2)$ & NO & 1070\\
$(169,71)$ & 11 & $(2,1)$ & 1 & 1 & NO & YES & YES & $1.50$ & $(2,3)$ & -- & 1071\\
$(170,29)$ & 15 & $(7,1)$ & 6 & 1 & YES & YES & YES & $1.60$ & $(2,3)$ & NO & 1072\\
$(191,31)$ & 16 & $(13,2)$ & 7 & 1 & YES & YES & YES & $1.43$ & $(2,3)$ & NO & 1073\\
$(192,31)$ & 16 & $(37,6)$ & 11 & 1 & YES & YES & YES & $1.56$ & $(2,3)$ & NO & 1074\\
$(192,31)$ & 16 & $(192,31)$ & 16 & 192 & YES & YES & YES & $1.56$ & $(2,3)$ & NO & 1075\\
$(206,73)$ & 12 & $(2,1)$ & 1 & 2 & YES & YES & YES & $1.57$ & $(2,3)$ & NO & 1076\\
$(206,73)$ & 12 & $(6,1)$ & 5 & 2 & YES & YES & YES & $1.57$ & $(2,3)$ & -- & 1077\\
$(208,87)$ & 12 & $(4,1)$ & 3 & 4 & YES & YES & YES & $1.57$ & $(2,3)$ & NO & 1078\\
$(208,87)$ & 12 & $(19,8)$ & 6 & 1 & YES & YES & YES & $1.57$ & $(2,3)$ & NO & 1079\\
$(208,37)$ & 13 & $(45,8)$ & 9 & 1 & YES & YES & YES & $1.50$ & $(2,3)$ & NO & 1080\\
$(209,73)$ & 14 & $(3,1)$ & 2 & 1 & YES & YES & YES & $1.57$ & $(4,2)$ & 946 & 1081\\
$(210,29)$ & 17 & $(2,1)$ & 1 & 2 & YES & YES & YES & $1.38$ & $(4,2)$ & NO & 1082\\
$(210,29)$ & 17 & $(36,5)$ & 11 & 6 & YES & YES & YES & $1.50$ & $(2,3)$ & NO & 1083\\
$(223,70)$ & 13 & $(2,1)$ & 1 & 1 & YES & YES & YES & $1.57$ & $(2,3)$ & NO & 1084\\
$(239,32)$ & 17 & $(7,1)$ & 6 & 1 & YES & YES & YES & $1.60$ & $(2,3)$ & NO & 1085\\
$(267,98)$ & 12 & $(5,2)$ & 3 & 1 & YES & YES & YES & $1.71$ & $(2,3)$ & NO & 1086\\
$(274,115)$ & 12 & $(2,1)$ & 1 & 2 & YES & YES & YES & $1.71$ & $(2,3)$ & -- & 1087\\
$(286,105)$ & 12 & $(6,1)$ & 5 & 2 & YES & YES & YES & $1.57$ & $(2,3)$ & -- & 1088\\
$(293,123)$ & 12 & $(3,1)$ & 2 & 1 & YES & YES & YES & $1.71$ & $(2,3)$ & NO & 1089\\
$(295,87)$ & 13 & $(3,1)$ & 2 & 1 & YES & YES & YES & $1.57$ & $(2,3)$ & NO & 1090\\
$(a;6,1,0;49)$ & 11 & $(7,3)$ & 4 & 7 & YES & YES & YES & $1.57$ & $(4,2)$ & -- & 1091\\
$(b;0,3,0;29)$ & 8 & $(7,2)$ & 4 & 1 & YES & YES & YES & $1.44$ & $(2,3)$ & -- & 1092\\
$(b;0,3,0;29)$ & 8 & $(9,2)$ & 5 & 1 & YES & YES & YES & $1.44$ & $(2,3)$ & -- & 1093\\
$(b;0,4,0;34)$ & 9 & $(4,1)$ & 3 & 2 & YES & YES & YES & $1.38$ & $(4,2)$ & -- & 1094\\
$(b;1,0,0;5)$ & 6 & $(19,6)$ & 8 & 1 & YES & YES & YES & $1.57$ & $(2,3)$ & -- & 1095\\
$(b;1,3,0;41)$ & 9 & $(4,1)$ & 3 & 1 & YES & YES & YES & $1.44$ & $(2,3)$ & -- & 1096\\
$(b;2,2,0;44)$ & 9 & $(4,1)$ & 3 & 4 & YES & YES & YES & $1.33$ & $(2,3)$ & -- & 1097\\
$(b;3,0,0;16)$ & 8 & $(7,3)$ & 4 & 1 & YES & YES & YES & $1.14$ & $(4,2)$ & -- & 1098\\
$(b;3,1,0;43)$ & 9 & $(3,1)$ & 2 & 1 & YES & YES & YES & $1.25$ & $(2,3)$ & -- & 1099\\
$(b;4,0,0;38)$ & 9 & $(4,1)$ & 3 & 2 & YES & YES & YES & $1.44$ & $(2,3)$ & -- & 1100\\
$(c;0,0,0;4)$ & 4 & $(28,13)$ & 9 & 4 & YES & YES & YES & $1.50$ & $(4,2)$ & -- & 1101\\
$(c;0,0,0;4)$ & 4 & $(34,15)$ & 8 & 2 & YES & YES & YES & $1.60$ & $(2,3)$ & -- & 1102\\
$(c;0,0,0;4)$ & 4 & $(115,24)$ & 12 & 1 & YES & YES & YES & $1.71$ & $(2,3)$ & -- & 1103\\
$(c;0,1,0;11)$ & 5 & $(17,7)$ & 6 & 1 & YES & YES & NO(2) & $1.40$ & $(4,2)$ & -- & 1104\\
$(c;0,2,0;7)$ & 6 & $(16,7)$ & 6 & 1 & YES & YES & YES & $1.44$ & $(2,3)$ & -- & 1105\\
$(c;0,2,0;7)$ & 6 & $(35,13)$ & 8 & 7 & YES & YES & YES & $1.60$ & $(2,3)$ & -- & 1106\\
$(c;0,2,0;7)$ & 6 & $(56,15)$ & 9 & 7 & YES & YES & YES & $1.57$ & $(2,3)$ & -- & 1107\\
$(c;0,2,1;19)$ & 7 & $(13,5)$ & 5 & 1 & YES & YES & YES & $1.44$ & $(2,3)$ & -- & 1108\\
$(c;0,2,2;6)$ & 8 & $(21,5)$ & 8 & 3 & YES & YES & YES & $1.50$ & $(2,3)$ & -- & 1109\\
$(c;0,4,0;10)$ & 8 & $(7,2)$ & 4 & 1 & YES & YES & YES & $1.29$ & $(6,1)$ & -- & 1110\\
$(c;0,4,0;10)$ & 8 & $(9,2)$ & 5 & 1 & YES & YES & YES & $1.29$ & $(6,1)$ & -- & 1111\\
$(c;0,5,3;47)$ & 12 & $(4,1)$ & 3 & 1 & YES & YES & YES & $1.43$ & $(2,3)$ & -- & 1112\\
$(d;0,0,0;5)$ & 5 & $(29,11)$ & 7 & 1 & YES & YES & YES & $1.38$ & $(2,3)$ & -- & 1113\\
$(d;0,0,0;5)$ & 5 & $(41,11)$ & 8 & 1 & YES & YES & NO(2) & $1.30$ & $(4,2)$ & -- & 1114\\
$(d;0,0,3;22)$ & 8 & $(16,5)$ & 7 & 2 & YES & YES & YES & $1.56$ & $(2,3)$ & -- & 1115\\
$(d;0,3,0;8)$ & 8 & $(7,2)$ & 4 & 1 & YES & YES & YES & $1.29$ & $(6,1)$ & -- & 1116\\
$(d;0,4,3;42)$ & 12 & $(3,1)$ & 2 & 3 & YES & YES & YES & $1.43$ & $(2,3)$ & -- & 1117\\
$(f;0,0,0;6)$ & 4 & $(52,15)$ & 11 & 2 & YES & YES & YES & $1.60$ & $(2,3)$ & -- & 1118\\
$(f;0,0,0;6)$ & 4 & $(53,15)$ & 11 & 1 & YES & YES & YES & $1.60$ & $(2,3)$ & -- & 1119\\
$(f;0,0,0;6)$ & 4 & $(72,25)$ & 12 & 6 & YES & YES & YES & $1.71$ & $(4,2)$ & -- & 1120\\
$(f;0,0,0;6)$ & 4 & $(84,13)$ & 13 & 6 & YES & YES & YES & $1.14$ & $(4,2)$ & -- & 1121\\
$(f;0,1,0;7)$ & 5 & $(18,7)$ & 6 & 1 & YES & YES & NO(2) & $1.58$ & $(2,3)$ & -- & 1122\\
$(f;0,1,0;7)$ & 5 & $(22,9)$ & 7 & 1 & YES & YES & NO(2) & $1.55$ & $(2,3)$ & -- & 1123\\
$(f;0,1,0;7)$ & 5 & $(23,9)$ & 7 & 1 & YES & YES & NO(2) & $1.55$ & $(2,3)$ & -- & 1124\\
$(f;0,1,0;7)$ & 5 & $(27,10)$ & 7 & 1 & YES & YES & YES & $1.56$ & $(2,3)$ & -- & 1125\\
$(g;0,0,3;40)$ & 9 & $(3,1)$ & 2 & 1 & YES & YES & YES & $1.50$ & $(2,3)$ & -- & 1126\\
$(g;0,3,0;34)$ & 9 & $(3,1)$ & 2 & 1 & YES & YES & YES & $1.50$ & $(2,3)$ & -- & 1127\\
$(g;3,0,0;23)$ & 9 & $(3,1)$ & 2 & 1 & YES & YES & NO(2) & $1.30$ & $(4,2)$ & -- & 1128\\
$(j;0,0,0;8)$ & 5 & $(40,17)$ & 9 & 8 & YES & YES & YES & $1.50$ & $(2,3)$ & -- & 1129\\
$(j;0,1,0;10)$ & 6 & $(27,11)$ & 8 & 1 & YES & YES & YES & $1.50$ & $(2,3)$ & -- & 1130
\end{longtable}
\subsection{2 chains, $K^2 = 4$}
\begin{longtable}{|c|c|c|c|c|c|c|c|c|c|c|c|}
\hline
\multicolumn{12}{|c|}{2 chains, $K^2 = 4$}\\
\hline
$(n,a)$ & Len & $(n,a)$ & Len & GCD & Nef & $\mathbb Q$-ef & Obs 0 & $\overline c_1^2 / \overline c_2$ & $(P,K)$ & WH & Index\\
\hline
\endfirsthead

\hline
$(n,a)$ & Len & $(n,a)$ & Len & GCD & Nef & $\mathbb Q$-ef & Obs 0 & $\overline c_1^2 / \overline c_2$ & $(P,K)$ & WH & Index\\
\hline
\endhead
\hline
\endfoot

$(25,11)$ & 7 & $(25,11)$ & 7 & 25 & YES & YES & YES & $1.75$ & $(2,4)$ & -- & 1131\\
$(31,14)$ & 8 & $(31,13)$ & 7 & 31 & YES & YES & YES & $1.88$ & $(2,4)$ & -- & 1132\\
$(36,11)$ & 8 & $(31,14)$ & 8 & 1 & YES & YES & YES & $1.88$ & $(2,4)$ & -- & 1133\\
$(39,11)$ & 9 & $(16,3)$ & 7 & 1 & YES & YES & NO(2) & $1.82$ & $(2,4)$ & NO & 1134\\
$(39,11)$ & 9 & $(16,3)$ & 7 & 1 & YES & YES & NO(2) & $1.82$ & $(2,4)$ & -- & 1135\\
$(39,11)$ & 9 & $(27,5)$ & 8 & 3 & YES & YES & YES & $1.88$ & $(2,4)$ & NO & 1136\\
$(39,11)$ & 9 & $(27,5)$ & 8 & 3 & YES & YES & YES & $1.88$ & $(2,4)$ & -- & 1137\\
$(41,15)$ & 8 & $(29,13)$ & 8 & 1 & YES & YES & YES & $2.00$ & $(2,4)$ & -- & 1138\\
$(41,15)$ & 8 & $(39,11)$ & 9 & 1 & YES & YES & YES & $1.83$ & $(4,3)$ & NO & 1139\\
$(49,22)$ & 9 & $(28,11)$ & 8 & 7 & YES & YES & YES & $1.86$ & $(4,3)$ & NO & 1140\\
$(61,16)$ & 10 & $(29,11)$ & 7 & 1 & YES & YES & YES & $2.00$ & $(2,4)$ & NO & 1141\\
$(65,18)$ & 9 & $(17,8)$ & 9 & 1 & YES & YES & YES & $2.00$ & $(4,3)$ & NO & 1142\\
$(65,18)$ & 9 & $(52,15)$ & 11 & 13 & YES & YES & YES & $2.00$ & $(4,3)$ & NO & 1143\\
$(73,21)$ & 14 & $(22,3)$ & 9 & 1 & YES & YES & YES & $1.83$ & $(6,2)$ & -- & 1144\\
$(76,31)$ & 10 & $(11,4)$ & 5 & 1 & YES & YES & YES & $1.83$ & $(4,3)$ & -- & 1145\\
$(76,29)$ & 9 & $(17,8)$ & 9 & 1 & YES & YES & YES & $2.00$ & $(4,3)$ & NO & 1146\\
$(79,24)$ & 10 & $(18,7)$ & 6 & 1 & YES & YES & NO(2) & $1.78$ & $(4,3)$ & NO & 1147\\
$(89,39)$ & 11 & $(9,2)$ & 5 & 1 & YES & YES & YES & $1.88$ & $(2,4)$ & -- & 1148\\
$(89,39)$ & 11 & $(11,2)$ & 6 & 1 & YES & YES & YES & $1.88$ & $(2,4)$ & NO & 1149\\
$(94,41)$ & 10 & $(37,10)$ & 8 & 1 & YES & YES & YES & $2.12$ & $(6,2)$ & -- & 1150\\
$(96,17)$ & 12 & $(26,5)$ & 9 & 2 & YES & YES & YES & $1.88$ & $(2,4)$ & -- & 1151\\
$(98,19)$ & 13 & $(19,5)$ & 7 & 1 & YES & YES & YES & $2.00$ & $(2,4)$ & -- & 1152\\
$(103,27)$ & 11 & $(14,5)$ & 6 & 1 & YES & YES & YES & $1.86$ & $(4,3)$ & NO & 1153\\
$(107,38)$ & 11 & $(18,7)$ & 6 & 1 & YES & YES & NO(2) & $1.78$ & $(4,3)$ & NO & 1154\\
$(109,16)$ & 13 & $(23,6)$ & 8 & 1 & YES & YES & YES & $1.83$ & $(4,3)$ & -- & 1155\\
$(113,17)$ & 13 & $(109,16)$ & 13 & 1 & YES & YES & YES & $1.83$ & $(4,3)$ & NO & 1156\\
$(117,41)$ & 13 & $(13,2)$ & 7 & 13 & YES & YES & YES & $1.86$ & $(4,3)$ & NO & 1157\\
$(117,31)$ & 11 & $(49,15)$ & 9 & 1 & YES & YES & YES & $2.00$ & $(6,2)$ & NO & 1158\\
$(128,37)$ & 12 & $(32,9)$ & 8 & 32 & YES & YES & YES & $1.83$ & $(6,2)$ & NO & 1159\\
$(128,37)$ & 12 & $(73,21)$ & 14 & 1 & YES & YES & YES & $1.83$ & $(6,2)$ & NO & 1160\\
$(138,61)$ & 12 & $(10,3)$ & 5 & 2 & YES & YES & YES & $1.83$ & $(4,3)$ & -- & 1161\\
$(145,42)$ & 12 & $(23,7)$ & 7 & 1 & YES & YES & NO(2) & $1.75$ & $(4,3)$ & NO & 1162\\
$(151,45)$ & 12 & $(11,4)$ & 5 & 1 & YES & YES & YES & $2.00$ & $(2,4)$ & NO & 1163\\
$(153,40)$ & 12 & $(5,1)$ & 4 & 1 & YES & YES & YES & $1.88$ & $(2,4)$ & -- & 1164\\
$(157,58)$ & 11 & $(11,5)$ & 6 & 1 & YES & YES & YES & $2.00$ & $(2,4)$ & NO & 1165\\
$(157,28)$ & 13 & $(41,7)$ & 11 & 1 & YES & YES & YES & $2.00$ & $(2,4)$ & NO & 1166\\
$(163,64)$ & 13 & $(74,29)$ & 10 & 1 & YES & YES & YES & $1.83$ & $(4,3)$ & NO & 1167\\
$(164,61)$ & 12 & $(4,1)$ & 3 & 4 & YES & YES & NO(2) & $1.78$ & $(4,3)$ & -- & 1168\\
$(169,70)$ & 11 & $(23,7)$ & 7 & 1 & YES & YES & YES & $2.17$ & $(10,0)$ & -- & 1169\\
$(175,67)$ & 11 & $(2,1)$ & 1 & 1 & YES & YES & NO(2) & $1.90$ & $(2,4)$ & -- & 1170\\
$(183,67)$ & 11 & $(7,2)$ & 4 & 1 & YES & YES & NO(2) & $1.78$ & $(4,3)$ & NO & 1171\\
$(183,38)$ & 13 & $(9,4)$ & 5 & 3 & YES & YES & YES & $1.83$ & $(4,3)$ & NO & 1172\\
$(187,79)$ & 11 & $(3,1)$ & 2 & 1 & YES & YES & NO(2) & $1.78$ & $(4,3)$ & NO & 1173\\
$(187,79)$ & 11 & $(3,1)$ & 2 & 1 & YES & YES & NO(2) & $1.78$ & $(4,3)$ & -- & 1174\\
$(191,75)$ & 14 & $(4,1)$ & 3 & 1 & YES & YES & YES & $1.88$ & $(4,3)$ & -- & 1175\\
$(193,53)$ & 12 & $(4,1)$ & 3 & 1 & YES & YES & YES & $1.89$ & $(2,4)$ & -- & 1176\\
$(208,37)$ & 13 & $(41,7)$ & 11 & 1 & YES & YES & YES & $2.00$ & $(2,4)$ & NO & 1177\\
$(211,41)$ & 16 & $(5,2)$ & 3 & 1 & YES & YES & YES & $1.86$ & $(4,3)$ & -- & 1178\\
$(211,41)$ & 16 & $(13,2)$ & 7 & 1 & YES & YES & YES & $1.86$ & $(4,3)$ & NO & 1179\\
$(211,41)$ & 16 & $(98,19)$ & 13 & 1 & YES & YES & YES & $1.86$ & $(4,3)$ & NO & 1180\\
$(219,65)$ & 12 & $(49,15)$ & 9 & 1 & YES & YES & YES & $2.00$ & $(6,2)$ & NO & 1181\\
$(223,54)$ & 15 & $(3,1)$ & 2 & 1 & YES & YES & NO(3) & $1.83$ & $(2,4)$ & NO & 1182\\
$(227,67)$ & 12 & $(17,5)$ & 6 & 1 & YES & YES & NO(2) & $1.89$ & $(4,3)$ & NO & 1183\\
$(236,65)$ & 12 & $(23,7)$ & 7 & 1 & YES & YES & YES & $2.17$ & $(10,0)$ & -- & 1184\\
$(237,85)$ & 12 & $(5,1)$ & 4 & 1 & YES & YES & NO(2) & $1.78$ & $(4,3)$ & NO & 1185\\
$(238,107)$ & 14 & $(2,1)$ & 1 & 2 & YES & YES & YES & $1.86$ & $(4,3)$ & -- & 1186\\
$(238,107)$ & 14 & $(109,49)$ & 12 & 1 & YES & YES & YES & $1.86$ & $(4,3)$ & NO & 1187\\
$(243,110)$ & 13 & $(3,1)$ & 2 & 3 & YES & YES & YES & $1.88$ & $(2,4)$ & -- & 1188\\
$(243,86)$ & 13 & $(6,1)$ & 5 & 3 & YES & YES & NO(2) & $1.57$ & $(6,2)$ & -- & 1189\\
$(243,110)$ & 13 & $(31,14)$ & 8 & 1 & YES & YES & YES & $1.88$ & $(2,4)$ & NO & 1190\\
$(244,33)$ & 16 & $(163,22)$ & 14 & 1 & YES & YES & YES & $1.83$ & $(6,2)$ & NO & 1191\\
$(252,41)$ & 17 & $(5,2)$ & 3 & 1 & YES & YES & YES & $1.86$ & $(4,3)$ & NO & 1192\\
$(252,107)$ & 13 & $(8,3)$ & 4 & 4 & YES & YES & YES & $1.83$ & $(4,3)$ & NO & 1193\\
$(252,107)$ & 13 & $(9,4)$ & 5 & 9 & YES & YES & YES & $1.83$ & $(4,3)$ & NO & 1194\\
$(263,93)$ & 14 & $(17,6)$ & 7 & 1 & YES & YES & YES & $1.67$ & $(6,2)$ & NO & 1195\\
$(263,82)$ & 14 & $(93,29)$ & 12 & 1 & YES & YES & NO(3) & $1.83$ & $(2,4)$ & NO & 1196\\
$(265,119)$ & 14 & $(265,119)$ & 14 & 265 & YES & YES & YES & $1.86$ & $(4,3)$ & NO & 1197\\
$(277,121)$ & 14 & $(2,1)$ & 1 & 1 & YES & YES & YES & $1.86$ & $(4,3)$ & NO & 1198\\
$(286,89)$ & 15 & $(5,1)$ & 4 & 1 & YES & YES & YES & $2.00$ & $(2,4)$ & -- & 1199\\
$(287,53)$ & 14 & $(26,5)$ & 9 & 1 & YES & YES & YES & $2.00$ & $(10,0)$ & -- & 1200\\
$(289,90)$ & 14 & $(4,1)$ & 3 & 1 & YES & YES & NO(2) & $1.75$ & $(4,3)$ & NO & 1201\\
$(297,79)$ & 15 & $(19,5)$ & 7 & 1 & YES & YES & YES & $2.00$ & $(2,4)$ & NO & 1202\\
$(319,95)$ & 15 & $(37,11)$ & 8 & 1 & YES & YES & YES & $2.00$ & $(2,4)$ & NO & 1203\\
$(325,137)$ & 14 & $(2,1)$ & 1 & 1 & YES & YES & YES & $2.00$ & $(2,4)$ & -- & 1204\\
$(325,137)$ & 14 & $(2,1)$ & 1 & 1 & YES & YES & YES & $2.00$ & $(2,4)$ & NO & 1205\\
$(326,71)$ & 14 & $(13,4)$ & 6 & 1 & YES & YES & YES & $2.00$ & $(6,2)$ & -- & 1206\\
$(328,121)$ & 14 & $(3,1)$ & 2 & 1 & YES & YES & YES & $2.00$ & $(2,4)$ & -- & 1207\\
$(328,63)$ & 16 & $(4,1)$ & 3 & 4 & YES & YES & YES & $1.88$ & $(2,4)$ & -- & 1208\\
$(328,121)$ & 14 & $(27,10)$ & 7 & 1 & YES & YES & YES & $2.00$ & $(2,4)$ & NO & 1209\\
$(328,63)$ & 16 & $(177,34)$ & 15 & 1 & YES & YES & YES & $1.88$ & $(2,4)$ & NO & 1210\\
$(332,89)$ & 13 & $(23,7)$ & 7 & 1 & YES & YES & YES & $2.17$ & $(10,0)$ & NO & 1211\\
$(346,95)$ & 15 & $(9,1)$ & 8 & 1 & YES & YES & YES & $2.00$ & $(2,4)$ & NO & 1212\\
$(356,93)$ & 15 & $(23,6)$ & 8 & 1 & YES & YES & YES & $1.83$ & $(4,3)$ & NO & 1213\\
$(368,141)$ & 13 & $(7,3)$ & 4 & 1 & YES & YES & YES & $2.00$ & $(10,0)$ & -- & 1214\\
$(386,75)$ & 17 & $(3,1)$ & 2 & 1 & YES & YES & YES & $1.86$ & $(4,3)$ & -- & 1215\\
$(386,75)$ & 17 & $(211,41)$ & 16 & 1 & YES & YES & YES & $1.86$ & $(4,3)$ & NO & 1216\\
$(392,53)$ & 20 & $(8,1)$ & 7 & 8 & YES & YES & YES & $1.83$ & $(6,2)$ & NO & 1217\\
$(398,147)$ & 13 & $(306,113)$ & 13 & 2 & YES & YES & YES & $2.12$ & $(4,3)$ & NO & 1218\\
$(411,109)$ & 14 & $(5,1)$ & 4 & 1 & YES & YES & YES & $1.88$ & $(2,4)$ & -- & 1219\\
$(417,161)$ & 13 & $(8,3)$ & 4 & 1 & YES & YES & YES & $2.00$ & $(10,0)$ & -- & 1220\\
$(417,161)$ & 13 & $(41,16)$ & 8 & 1 & YES & YES & YES & $2.00$ & $(10,0)$ & NO & 1221\\
$(455,192)$ & 13 & $(7,3)$ & 4 & 7 & YES & YES & YES & $2.17$ & $(2,4)$ & -- & 1222\\
$(469,76)$ & 18 & $(2,1)$ & 1 & 1 & YES & YES & YES & $1.67$ & $(6,2)$ & NO & 1223\\
$(469,76)$ & 18 & $(3,1)$ & 2 & 1 & YES & YES & YES & $1.83$ & $(4,3)$ & -- & 1224\\
$(494,111)$ & 16 & $(2,1)$ & 1 & 2 & YES & YES & YES & $2.00$ & $(2,4)$ & -- & 1225\\
$(544,223)$ & 14 & $(56,23)$ & 9 & 8 & YES & YES & YES & $2.12$ & $(6,2)$ & NO & 1226\\
$(548,227)$ & 14 & $(5,2)$ & 3 & 1 & YES & YES & YES & $2.12$ & $(4,3)$ & -- & 1227\\
$(551,240)$ & 14 & $(5,2)$ & 3 & 1 & YES & YES & YES & $2.12$ & $(6,2)$ & -- & 1228\\
$(574,155)$ & 14 & $(311,84)$ & 13 & 1 & YES & YES & YES & $2.17$ & $(8,1)$ & NO & 1229\\
$(589,80)$ & 19 & $(9,1)$ & 8 & 1 & YES & YES & YES & $2.00$ & $(2,4)$ & NO & 1230\\
$(606,115)$ & 16 & $(38,7)$ & 9 & 2 & YES & YES & YES & $2.12$ & $(4,3)$ & NO & 1231\\
$(607,256)$ & 14 & $(8,3)$ & 4 & 1 & YES & YES & YES & $2.00$ & $(10,0)$ & NO & 1232\\
$(621,140)$ & 15 & $(7,3)$ & 4 & 1 & YES & YES & YES & $2.11$ & $(6,2)$ & -- & 1233\\
$(623,241)$ & 14 & $(517,200)$ & 14 & 1 & YES & YES & YES & $2.17$ & $(10,0)$ & NO & 1234\\
$(631,191)$ & 15 & $(53,16)$ & 10 & 1 & YES & YES & YES & $2.00$ & $(6,2)$ & NO & 1235\\
$(643,196)$ & 15 & $(33,10)$ & 8 & 1 & YES & YES & YES & $2.00$ & $(6,2)$ & NO & 1236\\
$(666,101)$ & 18 & $(3,1)$ & 2 & 3 & YES & YES & YES & $2.00$ & $(2,4)$ & -- & 1237\\
$(772,279)$ & 15 & $(4,1)$ & 3 & 4 & YES & YES & YES & $1.83$ & $(10,0)$ & NO & 1238\\
$(824,227)$ & 15 & $(236,65)$ & 12 & 4 & YES & YES & YES & $2.17$ & $(10,0)$ & NO & 1239\\
$(835,323)$ & 15 & $(3,1)$ & 2 & 1 & YES & YES & YES & $2.12$ & $(4,3)$ & -- & 1240\\
$(843,326)$ & 15 & $(3,1)$ & 2 & 3 & YES & YES & YES & $2.25$ & $(4,3)$ & -- & 1241\\
$(870,269)$ & 16 & $(42,13)$ & 9 & 6 & YES & YES & YES & $2.00$ & $(10,0)$ & NO & 1242\\
$(907,269)$ & 16 & $(3,1)$ & 2 & 1 & YES & YES & YES & $2.00$ & $(6,2)$ & NO & 1243\\
$(907,335)$ & 15 & $(5,1)$ & 4 & 1 & YES & YES & YES & $2.00$ & $(10,0)$ & -- & 1244\\
$(923,259)$ & 16 & $(11,3)$ & 5 & 1 & YES & YES & YES & $2.11$ & $(6,2)$ & NO & 1245\\
$(945,388)$ & 15 & $(945,388)$ & 15 & 945 & YES & YES & YES & $2.11$ & $(6,2)$ & NO & 1246\\
$(985,289)$ & 16 & $(2,1)$ & 1 & 1 & YES & YES & YES & $2.12$ & $(6,2)$ & NO & 1247\\
$(1057,321)$ & 16 & $(3,1)$ & 2 & 1 & YES & YES & YES & $2.00$ & $(10,0)$ & -- & 1248\\
$(1058,409)$ & 15 & $(2,1)$ & 1 & 2 & YES & YES & YES & $2.22$ & $(6,2)$ & -- & 1249\\
$(1212,217)$ & 18 & $(1212,217)$ & 18 & 1212 & YES & YES & YES & $1.83$ & $(10,0)$ & NO & 1250\\
$(1218,463)$ & 15 & $(50,19)$ & 8 & 2 & YES & YES & YES & $2.17$ & $(10,0)$ & NO & 1251\\
$(1237,345)$ & 16 & $(7,2)$ & 4 & 1 & YES & YES & YES & $2.00$ & $(10,0)$ & NO & 1252\\
$(1783,331)$ & 18 & $(27,5)$ & 8 & 1 & YES & YES & YES & $2.00$ & $(6,2)$ & NO & 1253\\
$(a;0,0,0;3)$ & 4 & $(265,62)$ & 14 & 1 & YES & YES & YES & $2.00$ & $(10,0)$ & -- & 1254\\
$(a;4,1,0;37)$ & 9 & $(9,4)$ & 5 & 1 & YES & YES & NO(2) & $1.57$ & $(6,2)$ & -- & 1255\\
$(a;5,1,3;106)$ & 13 & $(3,1)$ & 2 & 1 & YES & YES & YES & $2.00$ & $(2,4)$ & -- & 1256\\
$(b;0,0,4;38)$ & 9 & $(11,5)$ & 6 & 1 & YES & YES & YES & $1.83$ & $(4,3)$ & -- & 1257\\
$(b;0,2,0;8)$ & 7 & $(124,29)$ & 11 & 4 & YES & YES & YES & $2.00$ & $(10,0)$ & -- & 1258\\
$(b;0,3,5;89)$ & 13 & $(3,1)$ & 2 & 1 & YES & YES & YES & $2.00$ & $(2,4)$ & -- & 1259\\
$(b;1,3,1;59)$ & 10 & $(11,4)$ & 5 & 1 & YES & YES & YES & $2.00$ & $(2,4)$ & -- & 1260\\
$(b;2,0,3;62)$ & 10 & $(9,2)$ & 5 & 1 & YES & YES & YES & $1.75$ & $(2,4)$ & -- & 1261\\
$(b;2,1,4;33)$ & 12 & $(2,1)$ & 1 & 1 & YES & YES & NO(2) & $1.78$ & $(4,3)$ & -- & 1262\\
$(c;0,3,2;29)$ & 9 & $(45,8)$ & 9 & 1 & YES & YES & YES & $1.88$ & $(2,4)$ & -- & 1263\\
$(c;0,5,2;39)$ & 11 & $(37,5)$ & 10 & 1 & YES & YES & YES & $1.86$ & $(4,3)$ & -- & 1264\\
$(d;0,0,0;5)$ & 5 & $(207,80)$ & 12 & 1 & YES & YES & YES & $2.22$ & $(6,2)$ & -- & 1265\\
$(d;0,0,2;9)$ & 7 & $(31,11)$ & 8 & 1 & YES & YES & NO(2) & $1.78$ & $(4,3)$ & -- & 1266\\
$(e;1,1,0;23)$ & 7 & $(99,29)$ & 10 & 1 & YES & YES & YES & $2.00$ & $(10,0)$ & -- & 1267\\
$(f;0,0,0;6)$ & 4 & $(738,137)$ & 16 & 6 & YES & YES & YES & $2.00$ & $(10,0)$ & -- & 1268\\
$(f;0,1,0;7)$ & 5 & $(37,17)$ & 9 & 1 & YES & YES & YES & $1.86$ & $(2,4)$ & -- & 1269\\
$(g;1,0,2;24)$ & 9 & $(14,5)$ & 6 & 2 & YES & YES & YES & $2.00$ & $(2,4)$ & -- & 1270\\
$(g;3,3,0;44)$ & 12 & $(4,1)$ & 3 & 4 & YES & YES & YES & $1.89$ & $(2,4)$ & -- & 1271
\end{longtable}
\subsection{2 chains, $K^2 = 5$}
\begin{longtable}{|c|c|c|c|c|c|c|c|c|c|c|c|}
\hline
\multicolumn{12}{|c|}{2 chains, $K^2 = 5$}\\
\hline
$(n,a)$ & Len & $(n,a)$ & Len & GCD & Nef & $\mathbb Q$-ef & Obs 0 & $\overline c_1^2 / \overline c_2$ & $(P,K)$ & WH & Index\\
\hline
\endfirsthead

\hline
$(n,a)$ & Len & $(n,a)$ & Len & GCD & Nef & $\mathbb Q$-ef & Obs 0 & $\overline c_1^2 / \overline c_2$ & $(P,K)$ & WH & Index\\
\hline
\endhead
\hline
\endfoot

$(158,33)$ & 12 & $(4,1)$ & 3 & 2 & YES & YES & NO(3) & $2.12$ & $(2,5)$ & -- & 1272\\
$(555,199)$ & 15 & $(6,1)$ & 5 & 3 & YES & YES & NO(3) & $2.17$ & $(6,3)$ & -- & 1273\\
$(1099,247)$ & 17 & $(89,20)$ & 11 & 1 & YES & YES & NO(3) & $2.17$ & $(6,3)$ & NO & 1274
\end{longtable}



%%%%%%%%%%%%%%%%%%%%%%%%%%%%%%%%%%%%%%%%%%%
\section{$I_8 + I_2 + 2I_1$}

(2858 examples from 101122048 tests)

Base curves:
\begin{itemize}
  \item $L_1 = y - \sqrt{3}x$.
  \item $L_2 = 2y - 3z$.
  \item $L_3 = y + \sqrt{3}x$.
  \item $C = x^2 + (y-2)^2 - z^2$.
  \item $L = x$.
\end{itemize}
Fibration given by pencil
\[F_{\lambda} = L_1L_2L_3 + \lambda CL.\]
Nine exceptionals are as follows:
\begin{itemize}
  \item $E_1$ - $E_3$ at $L_1 \cap L_2 \cap C = [\sqrt{3},3,2]$.
  \item $E_4$ - $E_5$ at $L_1 \cap L_3 \cap L = [0,0,1]$.
  \item $E_6$ at $L_2 \cap L = [0,3,2]$.
  \item $E_7$ - $E_9$ at $L_3 \cap L_2 \cap C = [-\sqrt{3},3,2]$.
\end{itemize}
Singular fibers are as follows:
\begin{itemize}
  \item $\lambda = \infty$: $I_2$ fiber given by $C$ and $L$. with nodes at $N_{I_2,1} = [0,3,1]$ and $N_{I_2,2} = [0,1,1]$.
  \item $\lambda = 0$: $I_8$ fiber given by $L_2$, $E_7$, $E_8$, $L_3$, $E_4$, $L_1$, $E_2$, $E_1$ in order.
  \item $\lambda = \frac{3\sqrt{3}}{2}$: $I_1$ fiber called $F_1$ with node at $N_{F_1} = [-\sqrt{3},0,1]$.
  \item $\lambda = -\frac{3\sqrt{3}}{2}$: $I_1$ fiber called $F_2$ with node at $N_{F_2} = [\sqrt{3},0,1]$.
\end{itemize}

\begin{center}
Classification of degree 1 double sections by intersections with $I_8$ and $I_2$
\end{center}

\begin{enumerate}
    \item $L_2 + E_4 + 2C$
    \[R_\alpha = y - \alpha x, \quad \alpha \in \C \setminus\{-\sqrt{3},\sqrt{3}\}\]
    Degenerations:
    \begin{itemize}
        \item $\alpha = 0$: $R_\alpha$ intersects $N_{F_1}$ and $N_{F_2}$
    \end{itemize}
    \item $E_1 + L_3 + C + L$
    \[M_\alpha^R = y - \alpha x + \dfrac{\sqrt{3}\alpha - 3}{2}z, \quad \alpha \in \widehat\C \setminus\{0,\sqrt{3}\}\]
    Degenerations:
    \begin{itemize}
        \item $\alpha = -\sqrt{3}$: $M_{\alpha}^R$ intersects $N_{F_2}$ and $N_{I_2,1}$
        \item $\alpha = \frac{1}{\sqrt{3}}$: $M_{\alpha}^R$ intersects $N_{F_1}$ and $N_{I_2,2}$
    \end{itemize}
    \item $E_7 + L_1 + C + L$
    \[M_\alpha^L = y + \alpha x + \dfrac{\sqrt{3}\alpha - 3}{2}z, \quad \alpha \in \widehat\C \setminus\{0,\sqrt{3}\}\]
    Degenerations:
    \begin{itemize}
        \item $\alpha = -\sqrt{3}$: $M_{\alpha}^L$ intersects $N_{F_1}$ and $N_{I_2,1}$
        \item $\alpha = \frac{1}{\sqrt{3}}$: $M_{\alpha}^L$ intersects $N_{F_2}$ and $N_{I_2,2}$
    \end{itemize}
    \item $L_1 + L_2 + 2C$ (also intersects $E_6$)
    \[S_\alpha = 2y - \alpha x - 3z, \quad \alpha \in \C \setminus\{0\}\]
    Degenerations:
    \begin{itemize}
        \item $\alpha = \sqrt{3}$: $S_\alpha$ intersects $N_{F_1}$
        \item $\alpha = -\sqrt{3}$: $S_\alpha$ intersects $N_{F_2}$
    \end{itemize}
    
\end{enumerate}

\begin{center}
Classification of degree 2 double sections by intersections with $I_8$ and $I_2$
\end{center}

\begin{enumerate}
  \item $E_1 + L_1 + 2L$ (also intersects $E_9$)
  \[D_{\alpha}^L = L_3L_2 + \alpha C, \quad \alpha \in \C \setminus\{0\}\]
  Degenerations:
  \begin{itemize}
      \item $\alpha = 3/2$: $D_{\alpha}^L$ intersects $N_{F_2}$
      \item $\alpha = -3/2$: $D_{\alpha}^L$ intersects $N_{F_1}$
  \end{itemize}
  \item $E_1 + E_7 + 2C$ (also intersects $E_5$)
  \[E_\alpha = LL_2 + \alpha L_1L_3, \quad \alpha \in \C \setminus\{0\}\]
  Degenerations:
  \begin{itemize}
      \item $\alpha = \frac{1}{\sqrt{3}}$: $E_{\alpha}$ intersects $N_{F_1}$
      \item $\alpha = -\frac{1}{\sqrt{3}}$: $E_{\alpha}$ intersects $N_{F_2}$
  \end{itemize}
  \item $E_2 + E_8 + 2L$
  \[A_\alpha = L_1L_3 + \alpha C, \quad \alpha \in \C \setminus\{0\}\]
  Degenerations:
  \begin{itemize}
      \item $\alpha = \frac{3}{2}$: $E_{\alpha}$ intersects $N_{F_1}$ and $N_{F_2}$
  \end{itemize}
  \item $2L_1 + C + L$ (also intersects $E_9$ and $E_6$)
  \[B_\alpha^L = C - \frac{1}{3}L_3M_{-\sqrt{3}}^L + \alpha L_2 L_3, \quad \alpha \in \C\]
  Degenerations:
  \begin{itemize}
      \item $\alpha = 0$: $B_\alpha^L$ intersects $N_{I_2,1}$
      \item $\alpha = 2/3$: $B_\alpha^L$ intersects $N_{I_2,2}$
      \item $\alpha = 4/3$: $B_\alpha^L$ intersects $N_{F_2}$
      \item $\alpha = -2/3$: $B_\alpha^L$ intersects $N_{F_1}$
  \end{itemize}
  \item $2L_3 + C + L$ (also intersects $E_3$ and $E_6$)
  \[B_\alpha^R = C - \frac{1}{3}L_1M_{-\sqrt{3}}^R + \alpha L_2 L_1, \quad \alpha \in \C\]
  Degenerations:
  \begin{itemize}
      \item $\alpha = 0$: $B_\alpha^R$ intersects $N_{I_2,1}$
      \item $\alpha = 2/3$: $B_\alpha^R$ intersects $N_{I_2,2}$
      \item $\alpha = 4/3$: $B_\alpha^R$ intersects $N_{F_1}$
      \item $\alpha = -2/3$: $B_\alpha^L$ intersects $N_{F_2}$
  \end{itemize}
  \item $E_7 + L_3 + 2L$ (also intersects $E_3$)
  \[D_{\alpha}^R = L_1L_2 + \alpha C, \quad \alpha \in \C \setminus\{0\}\]
  Degenerations:
  \begin{itemize}
      \item $\alpha = 3/2$: $D_{\alpha}^R$ intersects $N_{F_2}$
      \item $\alpha = -3/2$: $D_{\alpha}^R$ intersects $N_{F_1}$
  \end{itemize}
\end{enumerate}


Input:
\lstinputlisting[language=config]{../Tests/8211.txt}
Result:
%\usepackage{longtable}
\subsection{1 chain, $K^2 = 1$}
\begin{longtable}{|c|c|c|c|c|c|}
\hline
\multicolumn{6}{|c|}{1 chain, $K^2 = 1$}\\
\hline
$(n,a)$ & Length & Nef & $\mathbb Q$-ef & Obstruction 0 & Index\\
\hline
\endfirsthead

\hline
$(n,a)$ & Length & Nef & $\mathbb Q$-ef & Obstruction 0 & Index\\
\hline
\endhead
\hline
\endfoot

$(11, 4)$ & 5 & YES & YES & YES & 1\\
$(13, 4)$ & 6 & YES & YES & YES & 2\\
$(13, 5)$ & 5 & YES & YES & YES & 3\\
$(14, 5)$ & 6 & YES & YES & YES & 4\\
$(16, 5)$ & 7 & YES & YES & YES & 5\\
$(16, 7)$ & 6 & YES & YES & YES & 6\\
$(17, 7)$ & 6 & YES & YES & YES & 7\\
$(19, 5)$ & 7 & YES & YES & YES & 8\\
$(19, 8)$ & 6 & YES & YES & YES & 9\\
$(21, 5)$ & 8 & YES & YES & YES & 10\\
$(24, 5)$ & 8 & YES & YES & YES & 11\\
$(26, 7)$ & 7 & YES & YES & YES & 12\\
$(30, 7)$ & 8 & YES & YES & YES & 13\\
$(a; 1, 0, 0; 13)$ & 5 & YES & YES & YES & 14\\
$(b; 0, 0, 0; 14)$ & 5 & YES & YES & YES & 15\\
$(j; 0, 0, 0; 8)$ & 5 & YES & YES & YES & 16\\
$(j; 0, 1, 0; 10)$ & 6 & YES & YES & YES & 17
\end{longtable}
\subsection{1 chain, $K^2 = 2$}
\begin{longtable}{|c|c|c|c|c|c|}
\hline
\multicolumn{6}{|c|}{1 chain, $K^2 = 2$}\\
\hline
$(n,a)$ & Length & Nef & $\mathbb Q$-ef & Obstruction 0 & Index\\
\hline
\endfirsthead

\hline
$(n,a)$ & Length & Nef & $\mathbb Q$-ef & Obstruction 0 & Index\\
\hline
\endhead
\hline
\endfoot

$(27, 8)$ & 7 & YES & YES & YES & 18\\
$(29, 8)$ & 7 & YES & YES & YES & 19\\
$(31, 7)$ & 8 & YES & YES & NO(2) & 20\\
$(31, 9)$ & 8 & YES & YES & YES & 21\\
$(32, 7)$ & 8 & YES & YES & YES & 22\\
$(32, 9)$ & 8 & YES & YES & YES & 23\\
$(33, 13)$ & 9 & YES & YES & YES & 24\\
$(37, 8)$ & 8 & YES & YES & YES & 25\\
$(37, 10)$ & 8 & YES & YES & YES & 26\\
$(39, 14)$ & 8 & YES & YES & YES & 27\\
$(40, 17)$ & 9 & YES & YES & YES & 28\\
$(41, 11)$ & 8 & YES & YES & NO(2) & 29\\
$(41, 15)$ & 8 & YES & YES & YES & 30\\
$(42, 13)$ & 9 & YES & YES & YES & 31\\
$(44, 19)$ & 10 & YES & YES & YES & 32\\
$(45, 13)$ & 10 & YES & YES & YES & 33\\
$(45, 14)$ & 9 & YES & YES & YES & 34\\
$(46, 21)$ & 10 & YES & YES & YES & 35\\
$(48, 17)$ & 9 & YES & YES & YES & 36\\
$(49, 13)$ & 9 & YES & YES & YES & 37\\
$(49, 15)$ & 9 & YES & YES & NO(2) & 38\\
$(49, 18)$ & 8 & YES & YES & YES & 39\\
$(49, 22)$ & 9 & YES & YES & NO(2) & 40\\
$(50, 19)$ & 8 & YES & YES & NO(2) & 41\\
$(51, 20)$ & 9 & YES & YES & YES & 42\\
$(53, 19)$ & 9 & YES & YES & YES & 43\\
$(55, 24)$ & 9 & YES & YES & YES & 44\\
$(57, 17)$ & 10 & YES & YES & NO(2) & 45\\
$(57, 25)$ & 9 & YES & YES & YES & 46\\
$(59, 13)$ & 11 & YES & YES & YES & 47\\
$(62, 27)$ & 9 & YES & YES & YES & 48\\
$(64, 17)$ & 10 & YES & YES & YES & 49\\
$(64, 23)$ & 9 & YES & YES & YES & 50\\
$(65, 24)$ & 9 & YES & YES & YES & 51\\
$(67, 16)$ & 11 & YES & YES & YES & 52\\
$(71, 13)$ & 12 & YES & YES & NO(2) & 53\\
$(71, 17)$ & 11 & YES & YES & YES & 54\\
$(71, 19)$ & 10 & YES & YES & YES & 55\\
$(71, 22)$ & 10 & YES & YES & NO(2) & 56\\
$(72, 19)$ & 10 & YES & YES & YES & 57\\
$(74, 17)$ & 11 & YES & YES & YES & 58\\
$(77, 16)$ & 11 & YES & YES & YES & 59\\
$(79, 14)$ & 11 & YES & YES & YES & 60\\
$(80, 19)$ & 11 & YES & YES & YES & 61\\
$(81, 19)$ & 11 & YES & YES & NO(2) & 62\\
$(89, 27)$ & 10 & YES & YES & YES & 63\\
$(90, 19)$ & 11 & YES & YES & NO(2) & 64\\
$(91, 19)$ & 11 & YES & YES & YES & 65\\
$(96, 17)$ & 12 & YES & YES & YES & 66\\
$(a; 3, 1, 0; 31)$ & 8 & YES & YES & NO(2) & 67\\
$(b; 0, 0, 3; 32)$ & 8 & YES & YES & YES & 68\\
$(b; 0, 3, 0; 29)$ & 8 & YES & YES & YES & 69\\
$(c; 0, 3, 1; 23)$ & 8 & YES & YES & YES & 70\\
$(c; 0, 4, 1; 9)$ & 9 & YES & YES & YES & 71\\
$(d; 0, 0, 3; 22)$ & 8 & YES & YES & YES & 72\\
$(d; 0, 0, 4; 13)$ & 9 & YES & YES & YES & 73\\
$(d; 0, 1, 3; 27)$ & 9 & YES & YES & YES & 74\\
$(d; 0, 3, 1; 23)$ & 9 & YES & YES & YES & 75\\
$(e; 3, 0, 0; 10)$ & 8 & YES & YES & YES & 76
\end{longtable}
\subsection{1 chain, $K^2 = 3$}
\begin{longtable}{|c|c|c|c|c|c|}
\hline
\multicolumn{6}{|c|}{1 chain, $K^2 = 3$}\\
\hline
$(n,a)$ & Length & Nef & $\mathbb Q$-ef & Obstruction 0 & Index\\
\hline
\endfirsthead

\hline
$(n,a)$ & Length & Nef & $\mathbb Q$-ef & Obstruction 0 & Index\\
\hline
\endhead
\hline
\endfoot

$(64, 25)$ & 9 & YES & YES & NO(2) & 77\\
$(71, 26)$ & 9 & YES & YES & NO(2) & 78\\
$(76, 31)$ & 10 & YES & YES & NO(2) & 79\\
$(92, 39)$ & 10 & YES & YES & YES & 80\\
$(97, 18)$ & 11 & YES & YES & YES & 81\\
$(98, 41)$ & 10 & YES & YES & YES & 82\\
$(101, 22)$ & 11 & YES & YES & NO(2) & 83\\
$(101, 30)$ & 10 & YES & YES & NO(2) & 84\\
$(101, 37)$ & 10 & YES & YES & NO(2) & 85\\
$(104, 31)$ & 11 & YES & YES & NO(2) & 86\\
$(104, 45)$ & 11 & YES & YES & YES & 87\\
$(109, 30)$ & 10 & YES & YES & NO(2) & 88\\
$(113, 35)$ & 11 & YES & YES & NO(2) & 89\\
$(113, 42)$ & 11 & YES & YES & YES & 90\\
$(115, 52)$ & 11 & YES & YES & NO(2) & 91\\
$(119, 37)$ & 11 & YES & YES & NO(2) & 92\\
$(119, 45)$ & 11 & YES & YES & YES & 93\\
$(120, 53)$ & 11 & YES & YES & NO(2) & 94\\
$(125, 46)$ & 12 & YES & YES & YES & 95\\
$(125, 49)$ & 11 & YES & YES & YES & 96\\
$(129, 56)$ & 11 & YES & YES & NO(2) & 97\\
$(135, 32)$ & 12 & YES & YES & YES & 98\\
$(137, 63)$ & 12 & YES & YES & NO(2) & 99\\
$(144, 43)$ & 13 & YES & YES & NO(2) & 100\\
$(145, 51)$ & 12 & YES & YES & YES & 101\\
$(149, 46)$ & 13 & YES & YES & YES & 102\\
$(151, 53)$ & 12 & YES & YES & NO(2) & 103\\
$(151, 62)$ & 11 & YES & YES & YES & 104\\
$(152, 55)$ & 12 & YES & YES & YES & 105\\
$(152, 67)$ & 11 & YES & YES & NO(2) & 106\\
$(153, 64)$ & 11 & YES & YES & YES & 107\\
$(161, 48)$ & 12 & YES & YES & NO(2) & 108\\
$(169, 64)$ & 11 & YES & YES & YES & 109\\
$(171, 71)$ & 12 & YES & YES & YES & 110\\
$(183, 67)$ & 11 & YES & YES & YES & 111\\
$(188, 39)$ & 13 & YES & YES & YES & 112\\
$(201, 37)$ & 14 & YES & YES & NO(2) & 113\\
$(207, 37)$ & 15 & YES & YES & YES & 114\\
$(211, 50)$ & 14 & YES & YES & NO(2) & 115\\
$(213, 38)$ & 15 & YES & YES & NO(2) & 116\\
$(213, 62)$ & 12 & YES & YES & YES & 117\\
$(231, 83)$ & 12 & YES & YES & YES & 118\\
$(241, 63)$ & 13 & YES & YES & NO(2) & 119\\
$(243, 38)$ & 16 & YES & YES & NO(2) & 120\\
$(246, 91)$ & 12 & YES & YES & NO(2) & 121\\
$(272, 59)$ & 13 & YES & YES & YES & 122\\
$(b; 4, 0, 1; 56)$ & 10 & YES & YES & YES & 123
\end{longtable}
\subsection{1 chain, $K^2 = 4$}
\begin{longtable}{|c|c|c|c|c|c|}
\hline
\multicolumn{6}{|c|}{1 chain, $K^2 = 4$}\\
\hline
$(n,a)$ & Length & Nef & $\mathbb Q$-ef & Obstruction 0 & Index\\
\hline
\endfirsthead

\hline
$(n,a)$ & Length & Nef & $\mathbb Q$-ef & Obstruction 0 & Index\\
\hline
\endhead
\hline
\endfoot

$(178, 63)$ & 12 & YES & YES & YES & 124\\
$(252, 107)$ & 13 & YES & YES & YES & 125\\
$(289, 66)$ & 13 & YES & YES & NO(2) & 126\\
$(298, 131)$ & 13 & YES & YES & NO(2) & 127\\
$(323, 116)$ & 13 & YES & YES & NO(2) & 128\\
$(336, 137)$ & 14 & YES & YES & YES & 129\\
$(375, 143)$ & 14 & YES & YES & YES & 130\\
$(379, 165)$ & 13 & YES & YES & YES & 131\\
$(412, 107)$ & 16 & YES & YES & NO(2) & 132\\
$(497, 107)$ & 15 & YES & YES & YES & 133\\
$(539, 200)$ & 14 & YES & YES & NO(2) & 134\\
$(618, 239)$ & 14 & YES & YES & NO(2) & 135\\
$(635, 132)$ & 16 & YES & YES & YES & 136\\
$(636, 179)$ & 16 & YES & YES & NO(2) & 137\\
$(727, 282)$ & 14 & YES & YES & NO(2) & 138\\
$(832, 191)$ & 17 & YES & YES & NO(2) & 139\\
$(1058, 409)$ & 15 & YES & YES & YES & 140\\
$(1190, 349)$ & 16 & YES & YES & YES & 141\\
$(g; 2, 3, 1; 19)$ & 12 & YES & YES & YES & 142
\end{longtable}
\subsection{1 chain, $K^2 = 5$}
\begin{longtable}{|c|c|c|c|c|c|}
\hline
\multicolumn{6}{|c|}{1 chain, $K^2 = 5$}\\
\hline
$(n,a)$ & Length & Nef & $\mathbb Q$-ef & Obstruction 0 & Index\\
\hline
\endfirsthead

\hline
$(n,a)$ & Length & Nef & $\mathbb Q$-ef & Obstruction 0 & Index\\
\hline
\endhead
\hline
\endfoot

$(1005, 412)$ & 15 & YES & YES & NO(2) & 143
\end{longtable}
\subsection{2 chains, $K^2 = 1$}
\begin{longtable}{|c|c|c|c|c|c|c|c|c|c|}
\hline
\multicolumn{10}{|c|}{2 chains, $K^2 = 1$}\\
\hline
$(n,a)$ & Length & $(n,a)$ & Length & GCD & Nef & $\mathbb Q$-ef & Obstruction 0 & WH & Index\\
\hline
\endfirsthead

\hline
$(n,a)$ & Length & $(n,a)$ & Length & GCD & Nef & $\mathbb Q$-ef & Obstruction 0 & WH & Index\\
\hline
\endhead
\hline
\endfoot

$(5, 2)$ & 3 & $(5, 2)$ & 3 & 5 & YES & YES & YES & -- & 144\\
$(7, 3)$ & 4 & $(5, 1)$ & 4 & 1 & YES & YES & YES & NO & 145\\
$(7, 3)$ & 4 & $(5, 1)$ & 4 & 1 & YES & YES & YES & NO & 146\\
$(7, 3)$ & 4 & $(7, 2)$ & 4 & 7 & YES & YES & YES & -- & 147\\
$(7, 3)$ & 4 & $(7, 2)$ & 4 & 7 & YES & YES & YES & NO & 148\\
$(7, 3)$ & 4 & $(7, 2)$ & 4 & 7 & YES & YES & YES & NO & 149\\
$(8, 3)$ & 4 & $(4, 1)$ & 3 & 4 & YES & YES & YES & -- & 150\\
$(8, 3)$ & 4 & $(4, 1)$ & 3 & 4 & YES & YES & YES & NO & 151\\
$(8, 3)$ & 4 & $(5, 1)$ & 4 & 1 & YES & YES & YES & -- & 152\\
$(8, 3)$ & 4 & $(5, 1)$ & 4 & 1 & YES & YES & YES & NO & 153\\
$(8, 3)$ & 4 & $(5, 1)$ & 4 & 1 & YES & YES & YES & NO & 154\\
$(8, 3)$ & 4 & $(5, 2)$ & 3 & 1 & YES & YES & YES & -- & 155\\
$(8, 3)$ & 4 & $(7, 2)$ & 4 & 1 & YES & YES & YES & -- & 156\\
$(8, 3)$ & 4 & $(7, 2)$ & 4 & 1 & YES & YES & YES & NO & 157\\
$(8, 3)$ & 4 & $(7, 3)$ & 4 & 1 & YES & YES & YES & -- & 158\\
$(8, 3)$ & 4 & $(7, 3)$ & 4 & 1 & YES & YES & YES & NO & 159\\
$(8, 3)$ & 4 & $(7, 3)$ & 4 & 1 & YES & YES & YES & NO & 160\\
$(9, 4)$ & 5 & $(4, 1)$ & 3 & 1 & YES & YES & YES & NO & 161\\
$(9, 4)$ & 5 & $(4, 1)$ & 3 & 1 & YES & YES & YES & NO & 162\\
$(9, 4)$ & 5 & $(5, 2)$ & 3 & 1 & YES & YES & YES & NO & 163\\
$(9, 2)$ & 5 & $(7, 3)$ & 4 & 1 & YES & YES & YES & -- & 164\\
$(9, 2)$ & 5 & $(7, 3)$ & 4 & 1 & YES & YES & YES & NO & 165\\
$(9, 4)$ & 5 & $(7, 2)$ & 4 & 1 & YES & YES & NO(2) & -- & 166\\
$(9, 4)$ & 5 & $(7, 2)$ & 4 & 1 & YES & YES & NO(2) & NO & 167\\
$(9, 4)$ & 5 & $(9, 2)$ & 5 & 9 & YES & YES & NO(2) & NO & 168\\
$(10, 3)$ & 5 & $(4, 1)$ & 3 & 2 & YES & YES & YES & -- & 169\\
$(10, 3)$ & 5 & $(4, 1)$ & 3 & 2 & YES & YES & YES & 178 & 170\\
$(10, 3)$ & 5 & $(5, 1)$ & 4 & 5 & YES & YES & YES & -- & 171\\
$(10, 3)$ & 5 & $(5, 1)$ & 4 & 5 & YES & YES & YES & NO & 172\\
$(10, 3)$ & 5 & $(5, 2)$ & 3 & 5 & YES & YES & YES & -- & 173\\
$(11, 3)$ & 5 & $(2, 1)$ & 1 & 1 & YES & YES & YES & -- & 174\\
$(11, 3)$ & 5 & $(2, 1)$ & 1 & 1 & YES & YES & YES & NO & 175\\
$(11, 4)$ & 5 & $(2, 1)$ & 1 & 1 & YES & YES & YES & -- & 176\\
$(11, 3)$ & 5 & $(3, 1)$ & 2 & 1 & YES & YES & YES & -- & 177\\
$(11, 3)$ & 5 & $(3, 1)$ & 2 & 1 & YES & YES & YES & 170 & 178\\
$(11, 4)$ & 5 & $(3, 1)$ & 2 & 1 & YES & YES & YES & -- & 179\\
$(11, 4)$ & 5 & $(3, 1)$ & 2 & 1 & YES & YES & YES & NO & 180\\
$(11, 5)$ & 6 & $(3, 1)$ & 2 & 1 & YES & YES & YES & -- & 181\\
$(11, 5)$ & 6 & $(3, 1)$ & 2 & 1 & YES & YES & YES & NO & 182\\
$(11, 3)$ & 5 & $(4, 1)$ & 3 & 1 & YES & YES & YES & -- & 183\\
$(11, 3)$ & 5 & $(4, 1)$ & 3 & 1 & YES & YES & YES & NO & 184\\
$(11, 3)$ & 5 & $(4, 1)$ & 3 & 1 & YES & YES & YES & NO & 185\\
$(11, 4)$ & 5 & $(4, 1)$ & 3 & 1 & YES & YES & YES & -- & 186\\
$(11, 4)$ & 5 & $(4, 1)$ & 3 & 1 & YES & YES & YES & NO & 187\\
$(11, 5)$ & 6 & $(4, 1)$ & 3 & 1 & YES & YES & YES & -- & 188\\
$(11, 5)$ & 6 & $(4, 1)$ & 3 & 1 & YES & YES & YES & NO & 189\\
$(11, 5)$ & 6 & $(4, 1)$ & 3 & 1 & YES & YES & YES & NO & 190\\
$(11, 3)$ & 5 & $(5, 1)$ & 4 & 1 & YES & YES & YES & -- & 191\\
$(11, 3)$ & 5 & $(5, 1)$ & 4 & 1 & YES & YES & YES & NO & 192\\
$(11, 5)$ & 6 & $(5, 2)$ & 3 & 1 & YES & YES & YES & -- & 193\\
$(11, 5)$ & 6 & $(5, 2)$ & 3 & 1 & YES & YES & YES & NO & 194\\
$(11, 5)$ & 6 & $(6, 1)$ & 5 & 1 & YES & YES & YES & NO & 195\\
$(11, 5)$ & 6 & $(6, 1)$ & 5 & 1 & YES & YES & YES & NO & 196\\
$(11, 4)$ & 5 & $(8, 3)$ & 4 & 1 & YES & YES & YES & NO & 197\\
$(11, 5)$ & 6 & $(9, 4)$ & 5 & 1 & YES & YES & YES & NO & 198\\
$(11, 5)$ & 6 & $(11, 5)$ & 6 & 11 & YES & YES & YES & NO & 199\\
$(12, 5)$ & 5 & $(3, 1)$ & 2 & 3 & YES & YES & YES & -- & 200\\
$(12, 5)$ & 5 & $(3, 1)$ & 2 & 3 & YES & YES & YES & NO & 201\\
$(12, 5)$ & 5 & $(3, 1)$ & 2 & 3 & YES & YES & YES & NO & 202\\
$(12, 5)$ & 5 & $(4, 1)$ & 3 & 4 & YES & YES & YES & -- & 203\\
$(12, 5)$ & 5 & $(4, 1)$ & 3 & 4 & YES & YES & YES & NO & 204\\
$(12, 5)$ & 5 & $(4, 1)$ & 3 & 4 & YES & YES & YES & NO & 205\\
$(12, 5)$ & 5 & $(5, 2)$ & 3 & 1 & YES & YES & NO(2) & -- & 206\\
$(12, 5)$ & 5 & $(5, 2)$ & 3 & 1 & YES & YES & NO(2) & NO & 207\\
$(12, 5)$ & 5 & $(7, 2)$ & 4 & 1 & YES & YES & YES & -- & 208\\
$(12, 5)$ & 5 & $(7, 2)$ & 4 & 1 & YES & YES & YES & NO & 209\\
$(12, 5)$ & 5 & $(9, 2)$ & 5 & 3 & YES & YES & NO(2) & -- & 210\\
$(12, 5)$ & 5 & $(9, 4)$ & 5 & 3 & YES & YES & NO(2) & NO & 211\\
$(13, 3)$ & 6 & $(2, 1)$ & 1 & 1 & YES & YES & YES & -- & 212\\
$(13, 5)$ & 5 & $(2, 1)$ & 1 & 1 & YES & YES & YES & NO & 213\\
$(13, 4)$ & 6 & $(3, 1)$ & 2 & 1 & YES & YES & YES & -- & 214\\
$(13, 4)$ & 6 & $(3, 1)$ & 2 & 1 & YES & YES & YES & NO & 215\\
$(13, 5)$ & 5 & $(3, 1)$ & 2 & 1 & YES & YES & YES & -- & 216\\
$(13, 5)$ & 5 & $(3, 1)$ & 2 & 1 & YES & YES & YES & NO & 217\\
$(13, 3)$ & 6 & $(4, 1)$ & 3 & 1 & YES & YES & YES & -- & 218\\
$(13, 3)$ & 6 & $(4, 1)$ & 3 & 1 & YES & YES & YES & NO & 219\\
$(13, 4)$ & 6 & $(4, 1)$ & 3 & 1 & YES & YES & YES & -- & 220\\
$(13, 4)$ & 6 & $(4, 1)$ & 3 & 1 & YES & YES & YES & NO & 221\\
$(13, 4)$ & 6 & $(7, 2)$ & 4 & 1 & YES & YES & YES & -- & 222\\
$(13, 4)$ & 6 & $(7, 2)$ & 4 & 1 & YES & YES & YES & 246 & 223\\
$(13, 3)$ & 6 & $(11, 3)$ & 5 & 1 & YES & YES & YES & NO & 224\\
$(13, 5)$ & 5 & $(13, 5)$ & 5 & 13 & YES & YES & YES & NO & 225\\
$(14, 5)$ & 6 & $(3, 1)$ & 2 & 1 & NO & YES & YES & -- & 226\\
$(15, 4)$ & 6 & $(4, 1)$ & 3 & 1 & NO & YES & YES & -- & 227\\
$(15, 4)$ & 6 & $(9, 2)$ & 5 & 3 & YES & YES & NO(2) & NO & 228\\
$(16, 5)$ & 7 & $(3, 1)$ & 2 & 1 & NO & YES & YES & -- & 229\\
$(16, 5)$ & 7 & $(3, 1)$ & 2 & 1 & YES & YES & YES & NO & 230\\
$(16, 7)$ & 6 & $(3, 1)$ & 2 & 1 & YES & YES & NO(2) & -- & 231\\
$(16, 7)$ & 6 & $(3, 1)$ & 2 & 1 & YES & YES & NO(2) & NO & 232\\
$(16, 7)$ & 6 & $(4, 1)$ & 3 & 4 & YES & YES & NO(2) & -- & 233\\
$(16, 7)$ & 6 & $(4, 1)$ & 3 & 4 & YES & YES & NO(2) & NO & 234\\
$(16, 5)$ & 7 & $(5, 1)$ & 4 & 1 & YES & YES & NO(2) & NO & 235\\
$(16, 7)$ & 6 & $(5, 1)$ & 4 & 1 & YES & YES & YES & -- & 236\\
$(16, 7)$ & 6 & $(5, 1)$ & 4 & 1 & YES & YES & YES & NO & 237\\
$(16, 7)$ & 6 & $(5, 2)$ & 3 & 1 & YES & YES & NO(2) & NO & 238\\
$(16, 5)$ & 7 & $(7, 1)$ & 6 & 1 & YES & YES & YES & NO & 239\\
$(16, 5)$ & 7 & $(7, 2)$ & 4 & 1 & YES & YES & NO(2) & NO & 240\\
$(16, 7)$ & 6 & $(9, 4)$ & 5 & 1 & YES & YES & NO(2) & NO & 241\\
$(16, 5)$ & 7 & $(13, 4)$ & 6 & 1 & YES & YES & YES & NO & 242\\
$(16, 7)$ & 6 & $(16, 7)$ & 6 & 16 & YES & YES & NO(2) & NO & 243\\
$(17, 7)$ & 6 & $(2, 1)$ & 1 & 1 & YES & YES & YES & NO & 244\\
$(17, 5)$ & 6 & $(3, 1)$ & 2 & 1 & NO & YES & YES & -- & 245\\
$(17, 5)$ & 6 & $(3, 1)$ & 2 & 1 & YES & YES & YES & 223 & 246\\
$(17, 4)$ & 7 & $(4, 1)$ & 3 & 1 & NO & YES & YES & -- & 247\\
$(17, 4)$ & 7 & $(4, 1)$ & 3 & 1 & NO & YES & YES & NO & 248\\
$(17, 7)$ & 6 & $(12, 5)$ & 5 & 1 & YES & YES & NO(2) & NO & 249\\
$(17, 7)$ & 6 & $(17, 7)$ & 6 & 17 & YES & YES & NO(2) & NO & 250\\
$(18, 5)$ & 6 & $(3, 1)$ & 2 & 3 & YES & YES & NO(3) & -- & 251\\
$(19, 8)$ & 6 & $(2, 1)$ & 1 & 1 & NO & YES & YES & -- & 252\\
$(19, 8)$ & 6 & $(2, 1)$ & 1 & 1 & YES & YES & YES & NO & 253\\
$(19, 5)$ & 7 & $(4, 1)$ & 3 & 1 & YES & YES & YES & NO & 254\\
$(19, 8)$ & 6 & $(4, 1)$ & 3 & 1 & YES & YES & YES & -- & 255\\
$(19, 5)$ & 7 & $(5, 1)$ & 4 & 1 & YES & YES & NO(2) & NO & 256\\
$(19, 5)$ & 7 & $(6, 1)$ & 5 & 1 & YES & YES & NO(2) & NO & 257\\
$(19, 5)$ & 7 & $(7, 1)$ & 6 & 1 & YES & YES & YES & NO & 258\\
$(19, 5)$ & 7 & $(11, 3)$ & 5 & 1 & YES & YES & YES & 267 & 259\\
$(19, 5)$ & 7 & $(15, 4)$ & 6 & 1 & YES & YES & NO(2) & NO & 260\\
$(19, 5)$ & 7 & $(19, 5)$ & 7 & 19 & YES & YES & NO(2) & NO & 261\\
$(19, 8)$ & 6 & $(19, 8)$ & 6 & 19 & YES & YES & NO(2) & NO & 262\\
$(21, 8)$ & 6 & $(2, 1)$ & 1 & 1 & NO & YES & YES & -- & 263\\
$(23, 10)$ & 7 & $(2, 1)$ & 1 & 1 & NO & YES & YES & -- & 264\\
$(23, 7)$ & 7 & $(3, 1)$ & 2 & 1 & NO & YES & YES & -- & 265\\
$(25, 11)$ & 7 & $(2, 1)$ & 1 & 1 & NO & YES & NO(2) & -- & 266\\
$(26, 7)$ & 7 & $(4, 1)$ & 3 & 2 & YES & YES & YES & 259 & 267\\
$(26, 7)$ & 7 & $(26, 7)$ & 7 & 26 & YES & YES & NO(2) & NO & 268\\
$(30, 7)$ & 8 & $(3, 1)$ & 2 & 3 & YES & YES & NO(2) & NO & 269\\
$(30, 7)$ & 8 & $(9, 2)$ & 5 & 3 & YES & YES & NO(2) & NO & 270\\
$(a; 1, 0, 0; 13)$ & 5 & $(2, 1)$ & 1 & 1 & YES & YES & YES & -- & 271\\
$(a; 1, 0, 0; 13)$ & 5 & $(5, 2)$ & 3 & 1 & YES & YES & NO(2) & -- & 272\\
$(b; 0, 0, 0; 14)$ & 5 & $(2, 1)$ & 1 & 2 & YES & YES & NO(2) & -- & 273\\
$(c; 0, 1, 1; 5)$ & 6 & $(2, 1)$ & 1 & 1 & YES & YES & YES & -- & 274\\
$(c; 0, 2, 0; 7)$ & 6 & $(2, 1)$ & 1 & 1 & YES & YES & YES & -- & 275\\
$(d; 0, 0, 0; 5)$ & 5 & $(2, 1)$ & 1 & 1 & YES & YES & YES & -- & 276\\
$(d; 0, 0, 0; 5)$ & 5 & $(3, 1)$ & 2 & 1 & YES & YES & YES & -- & 277\\
$(f; 0, 0, 0; 6)$ & 4 & $(4, 1)$ & 3 & 2 & YES & YES & YES & -- & 278\\
$(f; 0, 0, 0; 6)$ & 4 & $(5, 2)$ & 3 & 1 & YES & YES & YES & -- & 279\\
$(f; 0, 0, 0; 6)$ & 4 & $(7, 2)$ & 4 & 1 & YES & YES & YES & -- & 280\\
$(f; 0, 0, 0; 6)$ & 4 & $(9, 2)$ & 5 & 3 & YES & YES & YES & -- & 281\\
$(f; 0, 1, 0; 7)$ & 5 & $(2, 1)$ & 1 & 1 & YES & YES & YES & -- & 282\\
$(f; 0, 1, 0; 7)$ & 5 & $(4, 1)$ & 3 & 1 & YES & YES & YES & -- & 283\\
$(j; 0, 0, 0; 8)$ & 5 & $(3, 1)$ & 2 & 1 & YES & YES & YES & -- & 284\\
$(j; 0, 1, 0; 10)$ & 6 & $(3, 1)$ & 2 & 1 & YES & YES & NO(2) & -- & 285\\
$(j; 0, 1, 0; 10)$ & 6 & $(4, 1)$ & 3 & 2 & YES & YES & NO(2) & -- & 286
\end{longtable}
\subsection{2 chains, $K^2 = 2$}
\begin{longtable}{|c|c|c|c|c|c|c|c|c|c|}
\hline
\multicolumn{10}{|c|}{2 chains, $K^2 = 2$}\\
\hline
$(n,a)$ & Length & $(n,a)$ & Length & GCD & Nef & $\mathbb Q$-ef & Obstruction 0 & WH & Index\\
\hline
\endfirsthead

\hline
$(n,a)$ & Length & $(n,a)$ & Length & GCD & Nef & $\mathbb Q$-ef & Obstruction 0 & WH & Index\\
\hline
\endhead
\hline
\endfoot

$(11, 3)$ & 5 & $(5, 2)$ & 3 & 1 & YES & YES & YES & -- & 287\\
$(11, 4)$ & 5 & $(9, 2)$ & 5 & 1 & YES & YES & YES & -- & 288\\
$(11, 4)$ & 5 & $(9, 2)$ & 5 & 1 & YES & YES & NO(2) & NO & 289\\
$(11, 4)$ & 5 & $(10, 3)$ & 5 & 1 & YES & YES & YES & NO & 290\\
$(12, 5)$ & 5 & $(11, 5)$ & 6 & 1 & YES & YES & YES & -- & 291\\
$(13, 3)$ & 6 & $(9, 4)$ & 5 & 1 & YES & YES & YES & -- & 292\\
$(13, 3)$ & 6 & $(9, 4)$ & 5 & 1 & YES & YES & YES & NO & 293\\
$(13, 3)$ & 6 & $(9, 4)$ & 5 & 1 & YES & YES & YES & NO & 294\\
$(13, 5)$ & 5 & $(9, 2)$ & 5 & 1 & YES & YES & YES & -- & 295\\
$(13, 6)$ & 7 & $(10, 3)$ & 5 & 1 & YES & YES & YES & -- & 296\\
$(13, 6)$ & 7 & $(10, 3)$ & 5 & 1 & YES & YES & YES & NO & 297\\
$(13, 3)$ & 6 & $(11, 4)$ & 5 & 1 & YES & YES & YES & -- & 298\\
$(13, 3)$ & 6 & $(11, 4)$ & 5 & 1 & YES & YES & YES & NO & 299\\
$(13, 3)$ & 6 & $(11, 4)$ & 5 & 1 & YES & YES & YES & 497 & 300\\
$(13, 3)$ & 6 & $(11, 5)$ & 6 & 1 & YES & YES & NO(2) & -- & 301\\
$(13, 4)$ & 6 & $(11, 2)$ & 6 & 1 & YES & YES & NO(2) & -- & 302\\
$(13, 4)$ & 6 & $(11, 2)$ & 6 & 1 & YES & YES & NO(2) & NO & 303\\
$(13, 5)$ & 5 & $(11, 4)$ & 5 & 1 & YES & YES & YES & -- & 304\\
$(13, 5)$ & 5 & $(11, 5)$ & 6 & 1 & YES & YES & YES & -- & 305\\
$(13, 6)$ & 7 & $(13, 3)$ & 6 & 13 & YES & YES & YES & NO & 306\\
$(14, 5)$ & 6 & $(9, 2)$ & 5 & 1 & YES & YES & YES & -- & 307\\
$(14, 5)$ & 6 & $(9, 2)$ & 5 & 1 & YES & YES & YES & NO & 308\\
$(14, 3)$ & 6 & $(10, 3)$ & 5 & 2 & YES & YES & YES & -- & 309\\
$(14, 3)$ & 6 & $(10, 3)$ & 5 & 2 & YES & YES & YES & NO & 310\\
$(14, 5)$ & 6 & $(10, 3)$ & 5 & 2 & YES & YES & YES & -- & 311\\
$(14, 3)$ & 6 & $(11, 3)$ & 5 & 1 & YES & YES & YES & -- & 312\\
$(14, 3)$ & 6 & $(11, 3)$ & 5 & 1 & YES & YES & YES & NO & 313\\
$(14, 5)$ & 6 & $(11, 3)$ & 5 & 1 & YES & YES & YES & -- & 314\\
$(14, 5)$ & 6 & $(11, 3)$ & 5 & 1 & YES & YES & YES & NO & 315\\
$(14, 3)$ & 6 & $(13, 4)$ & 6 & 1 & YES & YES & NO(2) & -- & 316\\
$(14, 3)$ & 6 & $(13, 4)$ & 6 & 1 & YES & YES & NO(2) & NO & 317\\
$(14, 5)$ & 6 & $(13, 3)$ & 6 & 1 & YES & YES & NO(2) & -- & 318\\
$(15, 4)$ & 6 & $(7, 2)$ & 4 & 1 & YES & YES & NO(2) & -- & 319\\
$(15, 4)$ & 6 & $(11, 2)$ & 6 & 1 & YES & YES & NO(2) & -- & 320\\
$(15, 4)$ & 6 & $(11, 2)$ & 6 & 1 & YES & YES & NO(2) & NO & 321\\
$(15, 4)$ & 6 & $(11, 3)$ & 5 & 1 & YES & YES & NO(2) & -- & 322\\
$(15, 4)$ & 6 & $(11, 3)$ & 5 & 1 & YES & YES & NO(2) & NO & 323\\
$(15, 4)$ & 6 & $(12, 5)$ & 5 & 3 & YES & YES & YES & -- & 324\\
$(16, 7)$ & 6 & $(8, 3)$ & 4 & 8 & YES & YES & YES & -- & 325\\
$(16, 5)$ & 7 & $(9, 2)$ & 5 & 1 & YES & YES & YES & -- & 326\\
$(16, 5)$ & 7 & $(9, 2)$ & 5 & 1 & YES & YES & YES & NO & 327\\
$(16, 5)$ & 7 & $(9, 2)$ & 5 & 1 & YES & YES & YES & NO & 328\\
$(16, 5)$ & 7 & $(9, 4)$ & 5 & 1 & YES & YES & YES & -- & 329\\
$(16, 5)$ & 7 & $(9, 4)$ & 5 & 1 & YES & YES & YES & NO & 330\\
$(16, 5)$ & 7 & $(10, 3)$ & 5 & 2 & YES & YES & NO(2) & -- & 331\\
$(16, 5)$ & 7 & $(11, 2)$ & 6 & 1 & YES & YES & NO(2) & -- & 332\\
$(16, 5)$ & 7 & $(11, 3)$ & 5 & 1 & YES & YES & YES & -- & 333\\
$(16, 5)$ & 7 & $(11, 3)$ & 5 & 1 & YES & YES & YES & NO & 334\\
$(16, 5)$ & 7 & $(12, 5)$ & 5 & 4 & YES & YES & YES & -- & 335\\
$(16, 5)$ & 7 & $(12, 5)$ & 5 & 4 & YES & YES & YES & NO & 336\\
$(16, 5)$ & 7 & $(12, 5)$ & 5 & 4 & YES & YES & YES & NO & 337\\
$(16, 7)$ & 6 & $(15, 4)$ & 6 & 1 & YES & YES & YES & NO & 338\\
$(17, 7)$ & 6 & $(5, 1)$ & 4 & 1 & YES & YES & YES & -- & 339\\
$(17, 7)$ & 6 & $(6, 1)$ & 5 & 1 & YES & YES & YES & -- & 340\\
$(17, 7)$ & 6 & $(6, 1)$ & 5 & 1 & YES & YES & YES & NO & 341\\
$(17, 6)$ & 7 & $(7, 2)$ & 4 & 1 & YES & YES & YES & -- & 342\\
$(17, 6)$ & 7 & $(7, 2)$ & 4 & 1 & YES & YES & YES & NO & 343\\
$(17, 7)$ & 6 & $(7, 2)$ & 4 & 1 & YES & YES & NO(2) & -- & 344\\
$(17, 5)$ & 6 & $(9, 4)$ & 5 & 1 & YES & YES & NO(2) & -- & 345\\
$(17, 5)$ & 6 & $(9, 4)$ & 5 & 1 & YES & YES & NO(2) & NO & 346\\
$(17, 4)$ & 7 & $(11, 5)$ & 6 & 1 & YES & YES & YES & NO & 347\\
$(17, 6)$ & 7 & $(13, 3)$ & 6 & 1 & YES & YES & YES & NO & 348\\
$(17, 6)$ & 7 & $(13, 5)$ & 5 & 1 & YES & YES & YES & NO & 349\\
$(17, 7)$ & 6 & $(13, 6)$ & 7 & 1 & YES & YES & YES & NO & 350\\
$(17, 4)$ & 7 & $(14, 5)$ & 6 & 1 & YES & YES & YES & NO & 351\\
$(17, 6)$ & 7 & $(14, 3)$ & 6 & 1 & YES & YES & NO(2) & NO & 352\\
$(17, 4)$ & 7 & $(16, 7)$ & 6 & 1 & YES & YES & YES & NO & 353\\
$(17, 7)$ & 6 & $(16, 7)$ & 6 & 1 & YES & YES & NO(2) & -- & 354\\
$(17, 7)$ & 6 & $(16, 7)$ & 6 & 1 & YES & YES & YES & NO & 355\\
$(18, 7)$ & 6 & $(5, 1)$ & 4 & 1 & YES & YES & YES & -- & 356\\
$(18, 7)$ & 6 & $(5, 1)$ & 4 & 1 & YES & YES & YES & NO & 357\\
$(18, 7)$ & 6 & $(6, 1)$ & 5 & 6 & YES & YES & YES & -- & 358\\
$(18, 7)$ & 6 & $(6, 1)$ & 5 & 6 & YES & YES & YES & NO & 359\\
$(18, 7)$ & 6 & $(6, 1)$ & 5 & 6 & YES & YES & YES & NO & 360\\
$(18, 7)$ & 6 & $(9, 2)$ & 5 & 9 & YES & YES & NO(2) & -- & 361\\
$(18, 7)$ & 6 & $(9, 2)$ & 5 & 9 & YES & YES & NO(2) & NO & 362\\
$(18, 7)$ & 6 & $(9, 4)$ & 5 & 9 & YES & YES & NO(2) & NO & 363\\
$(18, 5)$ & 6 & $(11, 5)$ & 6 & 1 & YES & YES & NO(2) & NO & 364\\
$(18, 5)$ & 6 & $(14, 5)$ & 6 & 2 & YES & YES & NO(2) & NO & 365\\
$(19, 4)$ & 7 & $(5, 2)$ & 3 & 1 & YES & YES & YES & -- & 366\\
$(19, 4)$ & 7 & $(5, 2)$ & 3 & 1 & YES & YES & YES & NO & 367\\
$(19, 5)$ & 7 & $(5, 2)$ & 3 & 1 & YES & YES & YES & NO & 368\\
$(19, 6)$ & 8 & $(5, 2)$ & 3 & 1 & YES & YES & YES & NO & 369\\
$(19, 8)$ & 6 & $(5, 1)$ & 4 & 1 & YES & YES & NO(2) & -- & 370\\
$(19, 8)$ & 6 & $(6, 1)$ & 5 & 1 & YES & YES & NO(2) & -- & 371\\
$(19, 8)$ & 6 & $(6, 1)$ & 5 & 1 & YES & YES & NO(2) & NO & 372\\
$(19, 8)$ & 6 & $(6, 1)$ & 5 & 1 & YES & YES & NO(2) & NO & 373\\
$(19, 4)$ & 7 & $(7, 2)$ & 4 & 1 & YES & YES & YES & -- & 374\\
$(19, 4)$ & 7 & $(7, 2)$ & 4 & 1 & YES & YES & YES & NO & 375\\
$(19, 5)$ & 7 & $(7, 2)$ & 4 & 1 & YES & YES & YES & -- & 376\\
$(19, 5)$ & 7 & $(7, 3)$ & 4 & 1 & YES & YES & YES & -- & 377\\
$(19, 6)$ & 8 & $(7, 3)$ & 4 & 1 & YES & YES & YES & -- & 378\\
$(19, 6)$ & 8 & $(7, 3)$ & 4 & 1 & YES & YES & YES & NO & 379\\
$(19, 7)$ & 6 & $(8, 3)$ & 4 & 1 & YES & YES & YES & -- & 380\\
$(19, 8)$ & 6 & $(8, 3)$ & 4 & 1 & YES & YES & NO(2) & -- & 381\\
$(19, 8)$ & 6 & $(8, 3)$ & 4 & 1 & YES & YES & YES & NO & 382\\
$(19, 4)$ & 7 & $(9, 4)$ & 5 & 1 & YES & YES & YES & -- & 383\\
$(19, 4)$ & 7 & $(9, 4)$ & 5 & 1 & YES & YES & YES & NO & 384\\
$(19, 7)$ & 6 & $(9, 4)$ & 5 & 1 & YES & YES & YES & -- & 385\\
$(19, 7)$ & 6 & $(9, 4)$ & 5 & 1 & YES & YES & YES & NO & 386\\
$(19, 5)$ & 7 & $(10, 3)$ & 5 & 1 & YES & YES & YES & -- & 387\\
$(19, 5)$ & 7 & $(10, 3)$ & 5 & 1 & YES & YES & YES & NO & 388\\
$(19, 7)$ & 6 & $(10, 3)$ & 5 & 1 & YES & YES & YES & -- & 389\\
$(19, 4)$ & 7 & $(11, 4)$ & 5 & 1 & YES & YES & YES & -- & 390\\
$(19, 7)$ & 6 & $(11, 5)$ & 6 & 1 & YES & YES & NO(2) & -- & 391\\
$(19, 7)$ & 6 & $(11, 5)$ & 6 & 1 & YES & YES & NO(2) & NO & 392\\
$(19, 7)$ & 6 & $(14, 5)$ & 6 & 1 & YES & YES & YES & NO & 393\\
$(19, 3)$ & 8 & $(17, 6)$ & 7 & 1 & YES & YES & YES & NO & 394\\
$(19, 7)$ & 6 & $(17, 4)$ & 7 & 1 & YES & YES & YES & NO & 395\\
$(19, 7)$ & 6 & $(17, 6)$ & 7 & 1 & YES & YES & NO(2) & 583 & 396\\
$(19, 7)$ & 6 & $(18, 7)$ & 6 & 1 & YES & YES & YES & NO & 397\\
$(20, 9)$ & 7 & $(5, 2)$ & 3 & 5 & YES & YES & NO(2) & -- & 398\\
$(20, 9)$ & 7 & $(7, 2)$ & 4 & 1 & YES & YES & YES & -- & 399\\
$(20, 9)$ & 7 & $(8, 3)$ & 4 & 4 & YES & YES & YES & -- & 400\\
$(20, 9)$ & 7 & $(10, 3)$ & 5 & 10 & YES & YES & YES & NO & 401\\
$(20, 9)$ & 7 & $(11, 3)$ & 5 & 1 & YES & YES & NO(2) & -- & 402\\
$(20, 9)$ & 7 & $(11, 3)$ & 5 & 1 & YES & YES & YES & NO & 403\\
$(20, 9)$ & 7 & $(11, 4)$ & 5 & 1 & YES & YES & YES & NO & 404\\
$(20, 3)$ & 8 & $(13, 6)$ & 7 & 1 & YES & YES & YES & NO & 405\\
$(20, 7)$ & 8 & $(13, 3)$ & 6 & 1 & YES & YES & NO(2) & -- & 406\\
$(20, 9)$ & 7 & $(13, 3)$ & 6 & 1 & YES & YES & YES & NO & 407\\
$(20, 3)$ & 8 & $(17, 6)$ & 7 & 1 & YES & YES & YES & NO & 408\\
$(20, 9)$ & 7 & $(17, 7)$ & 6 & 1 & YES & YES & YES & 540 & 409\\
$(20, 9)$ & 7 & $(19, 8)$ & 6 & 1 & YES & YES & YES & NO & 410\\
$(21, 8)$ & 6 & $(5, 1)$ & 4 & 1 & YES & YES & NO(2) & NO & 411\\
$(21, 8)$ & 6 & $(6, 1)$ & 5 & 3 & YES & YES & NO(2) & -- & 412\\
$(21, 8)$ & 6 & $(6, 1)$ & 5 & 3 & YES & YES & NO(2) & NO & 413\\
$(21, 8)$ & 6 & $(6, 1)$ & 5 & 3 & YES & YES & YES & NO & 414\\
$(21, 5)$ & 8 & $(7, 3)$ & 4 & 7 & YES & YES & NO(2) & -- & 415\\
$(21, 8)$ & 6 & $(9, 4)$ & 5 & 3 & YES & YES & NO(2) & -- & 416\\
$(21, 8)$ & 6 & $(9, 4)$ & 5 & 3 & YES & YES & NO(2) & NO & 417\\
$(21, 8)$ & 6 & $(9, 4)$ & 5 & 3 & YES & YES & NO(2) & NO & 418\\
$(21, 8)$ & 6 & $(11, 5)$ & 6 & 1 & YES & YES & NO(2) & NO & 419\\
$(21, 4)$ & 8 & $(13, 6)$ & 7 & 1 & YES & YES & NO(2) & NO & 420\\
$(21, 4)$ & 8 & $(13, 6)$ & 7 & 1 & YES & YES & NO(2) & NO & 421\\
$(21, 5)$ & 8 & $(13, 4)$ & 6 & 1 & YES & YES & YES & NO & 422\\
$(21, 8)$ & 6 & $(14, 5)$ & 6 & 7 & YES & YES & NO(2) & NO & 423\\
$(21, 5)$ & 8 & $(21, 4)$ & 8 & 21 & YES & YES & YES & NO & 424\\
$(22, 9)$ & 7 & $(9, 4)$ & 5 & 1 & YES & YES & YES & NO & 425\\
$(22, 5)$ & 7 & $(11, 5)$ & 6 & 11 & YES & YES & YES & NO & 426\\
$(22, 9)$ & 7 & $(11, 5)$ & 6 & 11 & YES & YES & YES & NO & 427\\
$(22, 5)$ & 7 & $(14, 5)$ & 6 & 2 & YES & YES & NO(2) & NO & 428\\
$(23, 5)$ & 7 & $(3, 1)$ & 2 & 1 & YES & YES & YES & NO & 429\\
$(23, 5)$ & 7 & $(4, 1)$ & 3 & 1 & YES & YES & YES & -- & 430\\
$(23, 5)$ & 7 & $(4, 1)$ & 3 & 1 & YES & YES & YES & NO & 431\\
$(23, 9)$ & 7 & $(5, 1)$ & 4 & 1 & YES & YES & YES & NO & 432\\
$(23, 9)$ & 7 & $(5, 2)$ & 3 & 1 & YES & YES & YES & NO & 433\\
$(23, 6)$ & 8 & $(7, 3)$ & 4 & 1 & YES & YES & YES & -- & 434\\
$(23, 9)$ & 7 & $(7, 3)$ & 4 & 1 & YES & YES & YES & -- & 435\\
$(23, 9)$ & 7 & $(7, 3)$ & 4 & 1 & YES & YES & YES & NO & 436\\
$(23, 4)$ & 8 & $(11, 5)$ & 6 & 1 & YES & YES & YES & -- & 437\\
$(23, 4)$ & 8 & $(11, 5)$ & 6 & 1 & YES & YES & YES & NO & 438\\
$(23, 9)$ & 7 & $(11, 4)$ & 5 & 1 & YES & YES & YES & NO & 439\\
$(23, 10)$ & 7 & $(11, 5)$ & 6 & 1 & YES & YES & NO(2) & 728 & 440\\
$(23, 4)$ & 8 & $(13, 6)$ & 7 & 1 & YES & YES & NO(2) & NO & 441\\
$(23, 4)$ & 8 & $(13, 6)$ & 7 & 1 & YES & YES & NO(2) & NO & 442\\
$(23, 6)$ & 8 & $(13, 4)$ & 6 & 1 & YES & YES & YES & NO & 443\\
$(23, 4)$ & 8 & $(14, 5)$ & 6 & 1 & YES & YES & YES & NO & 444\\
$(23, 6)$ & 8 & $(14, 3)$ & 6 & 1 & YES & YES & YES & -- & 445\\
$(23, 10)$ & 7 & $(14, 3)$ & 6 & 1 & YES & YES & NO(2) & NO & 446\\
$(23, 6)$ & 8 & $(16, 3)$ & 7 & 1 & YES & YES & YES & -- & 447\\
$(23, 6)$ & 8 & $(16, 3)$ & 7 & 1 & YES & YES & YES & NO & 448\\
$(23, 6)$ & 8 & $(20, 3)$ & 8 & 1 & YES & YES & YES & NO & 449\\
$(23, 4)$ & 8 & $(21, 5)$ & 8 & 1 & YES & YES & YES & NO & 450\\
$(24, 7)$ & 7 & $(4, 1)$ & 3 & 4 & YES & YES & YES & -- & 451\\
$(24, 7)$ & 7 & $(4, 1)$ & 3 & 4 & YES & YES & YES & NO & 452\\
$(24, 7)$ & 7 & $(5, 1)$ & 4 & 1 & YES & YES & YES & NO & 453\\
$(24, 11)$ & 8 & $(5, 2)$ & 3 & 1 & YES & YES & YES & -- & 454\\
$(24, 7)$ & 7 & $(6, 1)$ & 5 & 6 & YES & YES & YES & -- & 455\\
$(24, 7)$ & 7 & $(6, 1)$ & 5 & 6 & YES & YES & YES & NO & 456\\
$(24, 7)$ & 7 & $(6, 1)$ & 5 & 6 & YES & YES & YES & NO & 457\\
$(24, 11)$ & 8 & $(7, 3)$ & 4 & 1 & YES & YES & NO(2) & -- & 458\\
$(24, 5)$ & 8 & $(9, 4)$ & 5 & 3 & YES & YES & YES & -- & 459\\
$(24, 5)$ & 8 & $(11, 4)$ & 5 & 1 & YES & YES & YES & -- & 460\\
$(24, 5)$ & 8 & $(11, 4)$ & 5 & 1 & YES & YES & YES & NO & 461\\
$(24, 5)$ & 8 & $(13, 4)$ & 6 & 1 & YES & YES & YES & NO & 462\\
$(24, 11)$ & 8 & $(20, 9)$ & 7 & 4 & YES & YES & YES & NO & 463\\
$(24, 5)$ & 8 & $(21, 5)$ & 8 & 3 & YES & YES & YES & NO & 464\\
$(24, 5)$ & 8 & $(23, 4)$ & 8 & 1 & YES & YES & YES & NO & 465\\
$(25, 9)$ & 7 & $(3, 1)$ & 2 & 1 & YES & YES & NO(2) & -- & 466\\
$(25, 9)$ & 7 & $(3, 1)$ & 2 & 1 & YES & YES & YES & NO & 467\\
$(25, 11)$ & 7 & $(3, 1)$ & 2 & 1 & YES & YES & YES & NO & 468\\
$(25, 9)$ & 7 & $(4, 1)$ & 3 & 1 & YES & YES & YES & -- & 469\\
$(25, 9)$ & 7 & $(4, 1)$ & 3 & 1 & YES & YES & YES & NO & 470\\
$(25, 9)$ & 7 & $(4, 1)$ & 3 & 1 & YES & YES & YES & NO & 471\\
$(25, 9)$ & 7 & $(5, 2)$ & 3 & 5 & YES & YES & YES & -- & 472\\
$(25, 6)$ & 9 & $(7, 3)$ & 4 & 1 & YES & YES & YES & NO & 473\\
$(25, 11)$ & 7 & $(7, 2)$ & 4 & 1 & YES & YES & YES & -- & 474\\
$(25, 11)$ & 7 & $(7, 2)$ & 4 & 1 & YES & YES & YES & NO & 475\\
$(25, 11)$ & 7 & $(8, 3)$ & 4 & 1 & YES & YES & YES & -- & 476\\
$(25, 11)$ & 7 & $(8, 3)$ & 4 & 1 & YES & YES & YES & 750 & 477\\
$(25, 9)$ & 7 & $(11, 3)$ & 5 & 1 & YES & YES & NO(2) & NO & 478\\
$(25, 9)$ & 7 & $(13, 3)$ & 6 & 1 & YES & YES & YES & NO & 479\\
$(25, 11)$ & 7 & $(13, 3)$ & 6 & 1 & YES & YES & YES & -- & 480\\
$(25, 11)$ & 7 & $(13, 3)$ & 6 & 1 & YES & YES & YES & NO & 481\\
$(25, 9)$ & 7 & $(19, 7)$ & 6 & 1 & YES & YES & YES & NO & 482\\
$(25, 6)$ & 9 & $(20, 3)$ & 8 & 5 & YES & YES & YES & NO & 483\\
$(25, 4)$ & 9 & $(24, 5)$ & 8 & 1 & YES & YES & YES & NO & 484\\
$(26, 7)$ & 7 & $(3, 1)$ & 2 & 1 & YES & YES & NO(2) & -- & 485\\
$(26, 7)$ & 7 & $(5, 1)$ & 4 & 1 & YES & YES & NO(2) & -- & 486\\
$(26, 7)$ & 7 & $(5, 1)$ & 4 & 1 & YES & YES & NO(2) & NO & 487\\
$(26, 11)$ & 7 & $(5, 1)$ & 4 & 1 & YES & YES & NO(2) & NO & 488\\
$(26, 7)$ & 7 & $(7, 2)$ & 4 & 1 & YES & YES & YES & -- & 489\\
$(26, 11)$ & 7 & $(7, 3)$ & 4 & 1 & YES & YES & NO(2) & -- & 490\\
$(26, 11)$ & 7 & $(7, 3)$ & 4 & 1 & YES & YES & YES & NO & 491\\
$(26, 11)$ & 7 & $(8, 3)$ & 4 & 2 & YES & YES & NO(2) & -- & 492\\
$(27, 11)$ & 8 & $(3, 1)$ & 2 & 3 & YES & YES & YES & -- & 493\\
$(27, 11)$ & 8 & $(4, 1)$ & 3 & 1 & YES & YES & YES & -- & 494\\
$(27, 11)$ & 8 & $(4, 1)$ & 3 & 1 & YES & YES & YES & NO & 495\\
$(27, 10)$ & 7 & $(5, 1)$ & 4 & 1 & YES & YES & YES & -- & 496\\
$(27, 10)$ & 7 & $(5, 1)$ & 4 & 1 & YES & YES & YES & 300 & 497\\
$(27, 11)$ & 8 & $(6, 1)$ & 5 & 3 & YES & YES & YES & -- & 498\\
$(27, 8)$ & 7 & $(7, 3)$ & 4 & 1 & YES & YES & YES & NO & 499\\
$(27, 11)$ & 8 & $(7, 2)$ & 4 & 1 & YES & YES & YES & NO & 500\\
$(27, 11)$ & 8 & $(9, 4)$ & 5 & 9 & YES & YES & YES & NO & 501\\
$(27, 11)$ & 8 & $(12, 5)$ & 5 & 3 & YES & YES & YES & NO & 502\\
$(27, 10)$ & 7 & $(17, 6)$ & 7 & 1 & YES & YES & YES & NO & 503\\
$(27, 11)$ & 8 & $(22, 9)$ & 7 & 1 & YES & YES & YES & NO & 504\\
$(27, 11)$ & 8 & $(27, 11)$ & 8 & 27 & YES & YES & YES & NO & 505\\
$(28, 11)$ & 8 & $(2, 1)$ & 1 & 2 & YES & YES & YES & -- & 506\\
$(28, 11)$ & 8 & $(3, 1)$ & 2 & 1 & YES & YES & YES & -- & 507\\
$(28, 11)$ & 8 & $(4, 1)$ & 3 & 4 & YES & YES & YES & -- & 508\\
$(28, 11)$ & 8 & $(5, 2)$ & 3 & 1 & YES & YES & YES & -- & 509\\
$(28, 11)$ & 8 & $(6, 1)$ & 5 & 2 & YES & YES & YES & -- & 510\\
$(28, 11)$ & 8 & $(7, 2)$ & 4 & 7 & YES & YES & NO(2) & -- & 511\\
$(28, 5)$ & 8 & $(11, 5)$ & 6 & 1 & YES & YES & YES & NO & 512\\
$(28, 11)$ & 8 & $(11, 2)$ & 6 & 1 & YES & YES & NO(2) & -- & 513\\
$(28, 11)$ & 8 & $(13, 5)$ & 5 & 1 & YES & YES & YES & NO & 514\\
$(28, 5)$ & 8 & $(14, 5)$ & 6 & 14 & YES & YES & YES & -- & 515\\
$(28, 5)$ & 8 & $(14, 5)$ & 6 & 14 & YES & YES & NO(2) & NO & 516\\
$(28, 5)$ & 8 & $(21, 5)$ & 8 & 7 & YES & YES & NO(2) & NO & 517\\
$(28, 11)$ & 8 & $(23, 9)$ & 7 & 1 & YES & YES & YES & NO & 518\\
$(28, 11)$ & 8 & $(28, 11)$ & 8 & 28 & YES & YES & YES & NO & 519\\
$(29, 11)$ & 7 & $(3, 1)$ & 2 & 1 & YES & YES & YES & -- & 520\\
$(29, 9)$ & 8 & $(4, 1)$ & 3 & 1 & YES & YES & YES & -- & 521\\
$(29, 9)$ & 8 & $(4, 1)$ & 3 & 1 & YES & YES & YES & NO & 522\\
$(29, 9)$ & 8 & $(4, 1)$ & 3 & 1 & YES & YES & YES & NO & 523\\
$(29, 11)$ & 7 & $(4, 1)$ & 3 & 1 & YES & YES & NO(2) & -- & 524\\
$(29, 13)$ & 8 & $(4, 1)$ & 3 & 1 & YES & YES & YES & -- & 525\\
$(29, 6)$ & 9 & $(5, 2)$ & 3 & 1 & YES & YES & NO(2) & -- & 526\\
$(29, 9)$ & 8 & $(5, 1)$ & 4 & 1 & YES & YES & YES & -- & 527\\
$(29, 9)$ & 8 & $(5, 1)$ & 4 & 1 & YES & YES & YES & NO & 528\\
$(29, 9)$ & 8 & $(5, 2)$ & 3 & 1 & YES & YES & YES & -- & 529\\
$(29, 9)$ & 8 & $(5, 2)$ & 3 & 1 & YES & YES & YES & NO & 530\\
$(29, 11)$ & 7 & $(5, 2)$ & 3 & 1 & YES & YES & YES & -- & 531\\
$(29, 13)$ & 8 & $(5, 2)$ & 3 & 1 & YES & YES & YES & -- & 532\\
$(29, 9)$ & 8 & $(7, 3)$ & 4 & 1 & YES & YES & YES & NO & 533\\
$(29, 11)$ & 7 & $(7, 3)$ & 4 & 1 & YES & YES & YES & 748 & 534\\
$(29, 9)$ & 8 & $(8, 3)$ & 4 & 1 & YES & YES & YES & NO & 535\\
$(29, 13)$ & 8 & $(9, 2)$ & 5 & 1 & YES & YES & YES & -- & 536\\
$(29, 13)$ & 8 & $(9, 2)$ & 5 & 1 & YES & YES & YES & NO & 537\\
$(29, 13)$ & 8 & $(9, 2)$ & 5 & 1 & YES & YES & NO(2) & NO & 538\\
$(29, 12)$ & 7 & $(10, 3)$ & 5 & 1 & YES & YES & NO(2) & -- & 539\\
$(29, 12)$ & 7 & $(11, 5)$ & 6 & 1 & YES & YES & YES & 409 & 540\\
$(29, 13)$ & 8 & $(12, 5)$ & 5 & 1 & YES & YES & YES & NO & 541\\
$(29, 6)$ & 9 & $(13, 3)$ & 6 & 1 & YES & YES & NO(2) & NO & 542\\
$(29, 7)$ & 10 & $(13, 3)$ & 6 & 1 & YES & YES & NO(2) & -- & 543\\
$(29, 6)$ & 9 & $(23, 4)$ & 8 & 1 & YES & YES & NO(2) & NO & 544\\
$(29, 9)$ & 8 & $(23, 7)$ & 7 & 1 & YES & YES & YES & NO & 545\\
$(29, 4)$ & 10 & $(25, 6)$ & 9 & 1 & YES & YES & NO(2) & NO & 546\\
$(29, 12)$ & 7 & $(26, 11)$ & 7 & 1 & YES & YES & NO(2) & NO & 547\\
$(29, 6)$ & 9 & $(29, 4)$ & 10 & 29 & YES & YES & NO(2) & NO & 548\\
$(29, 11)$ & 7 & $(29, 11)$ & 7 & 29 & YES & YES & YES & NO & 549\\
$(30, 11)$ & 7 & $(3, 1)$ & 2 & 3 & YES & YES & YES & -- & 550\\
$(30, 11)$ & 7 & $(5, 1)$ & 4 & 5 & YES & YES & NO(2) & NO & 551\\
$(30, 11)$ & 7 & $(5, 2)$ & 3 & 5 & YES & YES & YES & -- & 552\\
$(30, 11)$ & 7 & $(5, 2)$ & 3 & 5 & YES & YES & YES & 644 & 553\\
$(30, 11)$ & 7 & $(7, 2)$ & 4 & 1 & YES & YES & YES & -- & 554\\
$(30, 11)$ & 7 & $(7, 2)$ & 4 & 1 & YES & YES & YES & NO & 555\\
$(30, 11)$ & 7 & $(7, 3)$ & 4 & 1 & YES & YES & YES & 851 & 556\\
$(30, 13)$ & 8 & $(7, 3)$ & 4 & 1 & YES & YES & NO(2) & -- & 557\\
$(30, 13)$ & 8 & $(7, 3)$ & 4 & 1 & YES & YES & NO(2) & NO & 558\\
$(30, 11)$ & 7 & $(10, 3)$ & 5 & 10 & YES & YES & YES & NO & 559\\
$(30, 11)$ & 7 & $(13, 5)$ & 5 & 1 & YES & YES & YES & NO & 560\\
$(30, 13)$ & 8 & $(13, 6)$ & 7 & 1 & YES & YES & NO(2) & NO & 561\\
$(30, 11)$ & 7 & $(17, 6)$ & 7 & 1 & YES & YES & YES & NO & 562\\
$(30, 11)$ & 7 & $(30, 11)$ & 7 & 30 & YES & YES & NO(2) & NO & 563\\
$(31, 7)$ & 8 & $(2, 1)$ & 1 & 1 & YES & YES & YES & -- & 564\\
$(31, 9)$ & 8 & $(2, 1)$ & 1 & 1 & YES & YES & YES & -- & 565\\
$(31, 7)$ & 8 & $(3, 1)$ & 2 & 1 & YES & YES & YES & -- & 566\\
$(31, 7)$ & 8 & $(3, 1)$ & 2 & 1 & YES & YES & YES & NO & 567\\
$(31, 11)$ & 8 & $(3, 1)$ & 2 & 1 & YES & YES & YES & -- & 568\\
$(31, 11)$ & 8 & $(3, 1)$ & 2 & 1 & YES & YES & YES & NO & 569\\
$(31, 11)$ & 8 & $(4, 1)$ & 3 & 1 & YES & YES & NO(2) & -- & 570\\
$(31, 14)$ & 8 & $(4, 1)$ & 3 & 1 & YES & YES & YES & -- & 571\\
$(31, 14)$ & 8 & $(4, 1)$ & 3 & 1 & YES & YES & YES & NO & 572\\
$(31, 7)$ & 8 & $(5, 1)$ & 4 & 1 & YES & YES & YES & -- & 573\\
$(31, 7)$ & 8 & $(5, 1)$ & 4 & 1 & YES & YES & YES & NO & 574\\
$(31, 7)$ & 8 & $(5, 1)$ & 4 & 1 & YES & YES & YES & NO & 575\\
$(31, 9)$ & 8 & $(5, 2)$ & 3 & 1 & YES & YES & YES & -- & 576\\
$(31, 11)$ & 8 & $(5, 2)$ & 3 & 1 & YES & YES & YES & -- & 577\\
$(31, 13)$ & 7 & $(5, 2)$ & 3 & 1 & YES & YES & YES & -- & 578\\
$(31, 13)$ & 7 & $(5, 2)$ & 3 & 1 & YES & YES & YES & NO & 579\\
$(31, 7)$ & 8 & $(7, 3)$ & 4 & 1 & YES & YES & YES & -- & 580\\
$(31, 11)$ & 8 & $(7, 3)$ & 4 & 1 & YES & YES & YES & NO & 581\\
$(31, 12)$ & 7 & $(7, 3)$ & 4 & 1 & YES & YES & NO(2) & -- & 582\\
$(31, 11)$ & 8 & $(8, 3)$ & 4 & 1 & YES & YES & NO(2) & 396 & 583\\
$(31, 6)$ & 10 & $(9, 4)$ & 5 & 1 & YES & YES & YES & NO & 584\\
$(31, 7)$ & 8 & $(9, 2)$ & 5 & 1 & YES & YES & NO(2) & NO & 585\\
$(31, 14)$ & 8 & $(13, 6)$ & 7 & 1 & YES & YES & YES & NO & 586\\
$(31, 6)$ & 10 & $(19, 3)$ & 8 & 1 & YES & YES & YES & NO & 587\\
$(31, 11)$ & 8 & $(19, 7)$ & 6 & 1 & YES & YES & YES & NO & 588\\
$(31, 14)$ & 8 & $(20, 9)$ & 7 & 1 & YES & YES & YES & NO & 589\\
$(31, 6)$ & 10 & $(23, 4)$ & 8 & 1 & YES & YES & YES & NO & 590\\
$(31, 7)$ & 8 & $(24, 5)$ & 8 & 1 & YES & YES & YES & NO & 591\\
$(31, 11)$ & 8 & $(25, 9)$ & 7 & 1 & YES & YES & NO(2) & NO & 592\\
$(31, 12)$ & 7 & $(28, 11)$ & 8 & 1 & YES & YES & NO(2) & 893 & 593\\
$(31, 11)$ & 8 & $(31, 11)$ & 8 & 31 & YES & YES & YES & NO & 594\\
$(31, 14)$ & 8 & $(31, 14)$ & 8 & 31 & YES & YES & YES & NO & 595\\
$(32, 7)$ & 8 & $(2, 1)$ & 1 & 2 & YES & YES & YES & NO & 596\\
$(32, 13)$ & 9 & $(2, 1)$ & 1 & 2 & YES & YES & YES & -- & 597\\
$(32, 7)$ & 8 & $(3, 1)$ & 2 & 1 & YES & YES & NO(2) & -- & 598\\
$(32, 7)$ & 8 & $(3, 1)$ & 2 & 1 & YES & YES & YES & NO & 599\\
$(32, 9)$ & 8 & $(3, 1)$ & 2 & 1 & YES & YES & NO(2) & NO & 600\\
$(32, 13)$ & 9 & $(3, 1)$ & 2 & 1 & YES & YES & YES & -- & 601\\
$(32, 7)$ & 8 & $(4, 1)$ & 3 & 4 & YES & YES & YES & -- & 602\\
$(32, 7)$ & 8 & $(4, 1)$ & 3 & 4 & YES & YES & YES & NO & 603\\
$(32, 13)$ & 9 & $(4, 1)$ & 3 & 4 & YES & YES & YES & -- & 604\\
$(32, 7)$ & 8 & $(5, 1)$ & 4 & 1 & YES & YES & YES & -- & 605\\
$(32, 7)$ & 8 & $(5, 1)$ & 4 & 1 & YES & YES & YES & NO & 606\\
$(32, 9)$ & 8 & $(5, 2)$ & 3 & 1 & YES & YES & YES & -- & 607\\
$(32, 13)$ & 9 & $(5, 1)$ & 4 & 1 & YES & YES & YES & -- & 608\\
$(32, 13)$ & 9 & $(5, 1)$ & 4 & 1 & YES & YES & YES & NO & 609\\
$(32, 13)$ & 9 & $(6, 1)$ & 5 & 2 & YES & YES & YES & NO & 610\\
$(32, 9)$ & 8 & $(7, 2)$ & 4 & 1 & YES & YES & YES & NO & 611\\
$(32, 13)$ & 9 & $(7, 3)$ & 4 & 1 & YES & YES & YES & NO & 612\\
$(32, 7)$ & 8 & $(9, 2)$ & 5 & 1 & YES & YES & YES & NO & 613\\
$(32, 7)$ & 8 & $(11, 4)$ & 5 & 1 & YES & YES & NO(2) & NO & 614\\
$(32, 9)$ & 8 & $(13, 4)$ & 6 & 1 & YES & YES & YES & NO & 615\\
$(32, 7)$ & 8 & $(14, 3)$ & 6 & 2 & YES & YES & YES & 710 & 616\\
$(32, 13)$ & 9 & $(17, 7)$ & 6 & 1 & YES & YES & YES & NO & 617\\
$(32, 7)$ & 8 & $(21, 5)$ & 8 & 1 & YES & YES & YES & NO & 618\\
$(32, 13)$ & 9 & $(22, 9)$ & 7 & 2 & YES & YES & YES & 868 & 619\\
$(32, 13)$ & 9 & $(27, 11)$ & 8 & 1 & YES & YES & YES & NO & 620\\
$(32, 7)$ & 8 & $(32, 7)$ & 8 & 32 & YES & YES & NO(2) & NO & 621\\
$(33, 13)$ & 9 & $(2, 1)$ & 1 & 1 & YES & YES & YES & -- & 622\\
$(33, 13)$ & 9 & $(2, 1)$ & 1 & 1 & YES & YES & YES & NO & 623\\
$(33, 13)$ & 9 & $(3, 1)$ & 2 & 3 & YES & YES & YES & -- & 624\\
$(33, 14)$ & 8 & $(3, 1)$ & 2 & 3 & YES & YES & YES & -- & 625\\
$(33, 13)$ & 9 & $(4, 1)$ & 3 & 1 & YES & YES & YES & -- & 626\\
$(33, 13)$ & 9 & $(4, 1)$ & 3 & 1 & YES & YES & YES & NO & 627\\
$(33, 13)$ & 9 & $(5, 1)$ & 4 & 1 & YES & YES & YES & -- & 628\\
$(33, 13)$ & 9 & $(6, 1)$ & 5 & 3 & YES & YES & NO(2) & -- & 629\\
$(33, 13)$ & 9 & $(6, 1)$ & 5 & 3 & YES & YES & YES & NO & 630\\
$(33, 13)$ & 9 & $(8, 3)$ & 4 & 1 & YES & YES & YES & NO & 631\\
$(33, 14)$ & 8 & $(8, 3)$ & 4 & 1 & YES & YES & YES & NO & 632\\
$(33, 10)$ & 8 & $(11, 4)$ & 5 & 11 & YES & YES & NO(2) & NO & 633\\
$(33, 10)$ & 8 & $(13, 4)$ & 6 & 1 & YES & YES & NO(2) & 686 & 634\\
$(33, 13)$ & 9 & $(18, 7)$ & 6 & 3 & YES & YES & YES & NO & 635\\
$(33, 13)$ & 9 & $(23, 9)$ & 7 & 1 & YES & YES & YES & 888 & 636\\
$(33, 13)$ & 9 & $(28, 11)$ & 8 & 1 & YES & YES & YES & NO & 637\\
$(33, 14)$ & 8 & $(33, 14)$ & 8 & 33 & YES & YES & YES & NO & 638\\
$(34, 9)$ & 8 & $(2, 1)$ & 1 & 2 & YES & YES & YES & NO & 639\\
$(34, 13)$ & 7 & $(2, 1)$ & 1 & 2 & YES & YES & YES & -- & 640\\
$(34, 9)$ & 8 & $(3, 1)$ & 2 & 1 & YES & YES & YES & -- & 641\\
$(34, 9)$ & 8 & $(3, 1)$ & 2 & 1 & YES & YES & YES & NO & 642\\
$(34, 13)$ & 7 & $(3, 1)$ & 2 & 1 & YES & YES & YES & -- & 643\\
$(34, 13)$ & 7 & $(3, 1)$ & 2 & 1 & YES & YES & YES & 553 & 644\\
$(34, 13)$ & 7 & $(5, 2)$ & 3 & 1 & YES & YES & YES & NO & 645\\
$(34, 15)$ & 8 & $(5, 2)$ & 3 & 1 & YES & YES & YES & -- & 646\\
$(34, 15)$ & 8 & $(5, 2)$ & 3 & 1 & YES & YES & YES & NO & 647\\
$(34, 13)$ & 7 & $(7, 3)$ & 4 & 1 & YES & YES & YES & NO & 648\\
$(34, 9)$ & 8 & $(8, 3)$ & 4 & 2 & YES & YES & NO(2) & -- & 649\\
$(34, 15)$ & 8 & $(8, 3)$ & 4 & 2 & YES & YES & YES & NO & 650\\
$(34, 9)$ & 8 & $(11, 3)$ & 5 & 1 & YES & YES & YES & NO & 651\\
$(34, 15)$ & 8 & $(11, 5)$ & 6 & 1 & YES & YES & YES & NO & 652\\
$(34, 15)$ & 8 & $(12, 5)$ & 5 & 2 & YES & YES & YES & 950 & 653\\
$(34, 15)$ & 8 & $(13, 6)$ & 7 & 1 & YES & YES & NO(2) & NO & 654\\
$(34, 9)$ & 8 & $(19, 5)$ & 7 & 1 & YES & YES & YES & NO & 655\\
$(34, 15)$ & 8 & $(23, 10)$ & 7 & 1 & YES & YES & NO(2) & 937 & 656\\
$(35, 11)$ & 9 & $(2, 1)$ & 1 & 1 & YES & YES & YES & NO & 657\\
$(35, 11)$ & 9 & $(3, 1)$ & 2 & 1 & YES & YES & YES & -- & 658\\
$(35, 11)$ & 9 & $(3, 1)$ & 2 & 1 & YES & YES & YES & NO & 659\\
$(35, 13)$ & 8 & $(3, 1)$ & 2 & 1 & YES & YES & YES & -- & 660\\
$(35, 13)$ & 8 & $(4, 1)$ & 3 & 1 & YES & YES & YES & -- & 661\\
$(35, 6)$ & 10 & $(5, 2)$ & 3 & 5 & YES & YES & NO(2) & -- & 662\\
$(35, 6)$ & 10 & $(5, 2)$ & 3 & 5 & YES & YES & NO(2) & NO & 663\\
$(35, 6)$ & 10 & $(5, 2)$ & 3 & 5 & YES & YES & NO(2) & NO & 664\\
$(35, 13)$ & 8 & $(6, 1)$ & 5 & 1 & YES & YES & YES & -- & 665\\
$(35, 13)$ & 8 & $(6, 1)$ & 5 & 1 & YES & YES & YES & NO & 666\\
$(35, 8)$ & 8 & $(7, 3)$ & 4 & 7 & YES & YES & YES & -- & 667\\
$(35, 8)$ & 8 & $(7, 3)$ & 4 & 7 & YES & YES & YES & NO & 668\\
$(35, 11)$ & 9 & $(7, 2)$ & 4 & 7 & YES & YES & YES & NO & 669\\
$(35, 16)$ & 9 & $(9, 4)$ & 5 & 1 & YES & YES & YES & NO & 670\\
$(35, 16)$ & 9 & $(11, 2)$ & 6 & 1 & YES & YES & NO(2) & NO & 671\\
$(35, 11)$ & 9 & $(13, 4)$ & 6 & 1 & YES & YES & YES & NO & 672\\
$(35, 13)$ & 8 & $(14, 5)$ & 6 & 7 & YES & YES & YES & NO & 673\\
$(35, 16)$ & 9 & $(16, 7)$ & 6 & 1 & YES & YES & NO(2) & NO & 674\\
$(35, 6)$ & 10 & $(20, 3)$ & 8 & 5 & YES & YES & YES & NO & 675\\
$(35, 6)$ & 10 & $(22, 3)$ & 9 & 1 & YES & YES & YES & NO & 676\\
$(35, 8)$ & 8 & $(25, 6)$ & 9 & 5 & YES & YES & YES & 962 & 677\\
$(36, 11)$ & 8 & $(2, 1)$ & 1 & 2 & YES & YES & NO(2) & -- & 678\\
$(36, 11)$ & 8 & $(3, 1)$ & 2 & 3 & YES & YES & YES & -- & 679\\
$(36, 11)$ & 8 & $(3, 1)$ & 2 & 3 & YES & YES & YES & NO & 680\\
$(36, 13)$ & 8 & $(3, 1)$ & 2 & 3 & YES & YES & YES & -- & 681\\
$(36, 13)$ & 8 & $(3, 1)$ & 2 & 3 & YES & YES & YES & NO & 682\\
$(36, 11)$ & 8 & $(5, 2)$ & 3 & 1 & YES & YES & YES & -- & 683\\
$(36, 11)$ & 8 & $(5, 2)$ & 3 & 1 & YES & YES & YES & NO & 684\\
$(36, 11)$ & 8 & $(5, 2)$ & 3 & 1 & YES & YES & YES & NO & 685\\
$(36, 11)$ & 8 & $(10, 3)$ & 5 & 2 & YES & YES & NO(2) & 634 & 686\\
$(36, 11)$ & 8 & $(16, 5)$ & 7 & 4 & YES & YES & YES & NO & 687\\
$(36, 13)$ & 8 & $(36, 13)$ & 8 & 36 & YES & YES & YES & NO & 688\\
$(37, 8)$ & 8 & $(2, 1)$ & 1 & 1 & YES & YES & NO(2) & -- & 689\\
$(37, 14)$ & 8 & $(2, 1)$ & 1 & 1 & YES & YES & YES & -- & 690\\
$(37, 14)$ & 8 & $(2, 1)$ & 1 & 1 & YES & YES & YES & NO & 691\\
$(37, 8)$ & 8 & $(3, 1)$ & 2 & 1 & YES & YES & NO(2) & -- & 692\\
$(37, 8)$ & 8 & $(3, 1)$ & 2 & 1 & YES & YES & NO(2) & NO & 693\\
$(37, 10)$ & 8 & $(3, 1)$ & 2 & 1 & YES & YES & NO(2) & -- & 694\\
$(37, 10)$ & 8 & $(3, 1)$ & 2 & 1 & YES & YES & NO(2) & NO & 695\\
$(37, 14)$ & 8 & $(3, 1)$ & 2 & 1 & YES & YES & YES & -- & 696\\
$(37, 14)$ & 8 & $(3, 1)$ & 2 & 1 & YES & YES & YES & NO & 697\\
$(37, 14)$ & 8 & $(3, 1)$ & 2 & 1 & YES & YES & NO(2) & NO & 698\\
$(37, 17)$ & 9 & $(3, 1)$ & 2 & 1 & YES & YES & YES & NO & 699\\
$(37, 8)$ & 8 & $(5, 1)$ & 4 & 1 & YES & YES & NO(2) & -- & 700\\
$(37, 8)$ & 8 & $(5, 1)$ & 4 & 1 & YES & YES & NO(2) & NO & 701\\
$(37, 8)$ & 8 & $(5, 1)$ & 4 & 1 & YES & YES & NO(2) & NO & 702\\
$(37, 13)$ & 9 & $(5, 2)$ & 3 & 1 & YES & YES & NO(2) & -- & 703\\
$(37, 13)$ & 9 & $(5, 2)$ & 3 & 1 & YES & YES & NO(2) & NO & 704\\
$(37, 10)$ & 8 & $(7, 2)$ & 4 & 1 & YES & YES & NO(2) & NO & 705\\
$(37, 10)$ & 8 & $(7, 3)$ & 4 & 1 & YES & YES & YES & NO & 706\\
$(37, 16)$ & 9 & $(7, 1)$ & 6 & 1 & YES & YES & YES & NO & 707\\
$(37, 16)$ & 9 & $(7, 1)$ & 6 & 1 & YES & YES & YES & NO & 708\\
$(37, 13)$ & 9 & $(8, 3)$ & 4 & 1 & YES & YES & YES & NO & 709\\
$(37, 8)$ & 8 & $(9, 2)$ & 5 & 1 & YES & YES & YES & 616 & 710\\
$(37, 8)$ & 8 & $(11, 4)$ & 5 & 1 & YES & YES & NO(2) & -- & 711\\
$(37, 17)$ & 9 & $(13, 6)$ & 7 & 1 & YES & YES & YES & NO & 712\\
$(37, 8)$ & 8 & $(14, 3)$ & 6 & 1 & YES & YES & YES & NO & 713\\
$(37, 10)$ & 8 & $(14, 3)$ & 6 & 1 & YES & YES & NO(2) & NO & 714\\
$(37, 8)$ & 8 & $(21, 5)$ & 8 & 1 & YES & YES & NO(2) & NO & 715\\
$(37, 10)$ & 8 & $(34, 9)$ & 8 & 1 & YES & YES & NO(2) & NO & 716\\
$(37, 16)$ & 9 & $(37, 16)$ & 9 & 37 & YES & YES & YES & NO & 717\\
$(38, 17)$ & 9 & $(4, 1)$ & 3 & 2 & YES & YES & YES & -- & 718\\
$(38, 17)$ & 9 & $(5, 1)$ & 4 & 1 & YES & YES & YES & -- & 719\\
$(38, 17)$ & 9 & $(5, 2)$ & 3 & 1 & YES & YES & NO(2) & -- & 720\\
$(38, 17)$ & 9 & $(5, 2)$ & 3 & 1 & YES & YES & NO(2) & NO & 721\\
$(38, 17)$ & 9 & $(6, 1)$ & 5 & 2 & YES & YES & YES & NO & 722\\
$(38, 17)$ & 9 & $(6, 1)$ & 5 & 2 & YES & YES & YES & NO & 723\\
$(38, 17)$ & 9 & $(7, 3)$ & 4 & 1 & YES & YES & YES & NO & 724\\
$(38, 17)$ & 9 & $(8, 3)$ & 4 & 2 & YES & YES & NO(2) & NO & 725\\
$(38, 17)$ & 9 & $(29, 13)$ & 8 & 1 & YES & YES & YES & NO & 726\\
$(38, 17)$ & 9 & $(38, 17)$ & 9 & 38 & YES & YES & YES & NO & 727\\
$(39, 17)$ & 8 & $(2, 1)$ & 1 & 1 & YES & YES & NO(2) & 440 & 728\\
$(39, 14)$ & 8 & $(3, 1)$ & 2 & 3 & YES & YES & YES & -- & 729\\
$(39, 14)$ & 8 & $(3, 1)$ & 2 & 3 & YES & YES & YES & NO & 730\\
$(39, 17)$ & 8 & $(3, 1)$ & 2 & 3 & YES & YES & YES & -- & 731\\
$(39, 17)$ & 8 & $(3, 1)$ & 2 & 3 & YES & YES & YES & NO & 732\\
$(39, 14)$ & 8 & $(4, 1)$ & 3 & 1 & YES & YES & NO(2) & -- & 733\\
$(39, 14)$ & 8 & $(4, 1)$ & 3 & 1 & YES & YES & YES & NO & 734\\
$(39, 14)$ & 8 & $(4, 1)$ & 3 & 1 & YES & YES & NO(2) & NO & 735\\
$(39, 14)$ & 8 & $(7, 2)$ & 4 & 1 & YES & YES & YES & NO & 736\\
$(39, 17)$ & 8 & $(9, 4)$ & 5 & 3 & YES & YES & YES & NO & 737\\
$(39, 17)$ & 8 & $(39, 17)$ & 8 & 39 & YES & YES & YES & NO & 738\\
$(40, 11)$ & 8 & $(2, 1)$ & 1 & 2 & YES & YES & NO(2) & -- & 739\\
$(40, 17)$ & 9 & $(2, 1)$ & 1 & 2 & YES & YES & YES & -- & 740\\
$(40, 17)$ & 9 & $(4, 1)$ & 3 & 4 & YES & YES & YES & -- & 741\\
$(40, 11)$ & 8 & $(11, 3)$ & 5 & 1 & YES & YES & NO(2) & NO & 742\\
$(41, 11)$ & 8 & $(2, 1)$ & 1 & 1 & YES & YES & NO(2) & NO & 743\\
$(41, 13)$ & 10 & $(2, 1)$ & 1 & 1 & YES & YES & YES & NO & 744\\
$(41, 15)$ & 8 & $(2, 1)$ & 1 & 1 & YES & YES & YES & NO & 745\\
$(41, 11)$ & 8 & $(3, 1)$ & 2 & 1 & YES & YES & YES & -- & 746\\
$(41, 17)$ & 8 & $(3, 1)$ & 2 & 1 & YES & YES & YES & -- & 747\\
$(41, 17)$ & 8 & $(3, 1)$ & 2 & 1 & YES & YES & YES & 534 & 748\\
$(41, 18)$ & 8 & $(3, 1)$ & 2 & 1 & YES & YES & YES & -- & 749\\
$(41, 18)$ & 8 & $(3, 1)$ & 2 & 1 & YES & YES & YES & 477 & 750\\
$(41, 13)$ & 10 & $(4, 1)$ & 3 & 1 & YES & YES & YES & NO & 751\\
$(41, 15)$ & 8 & $(4, 1)$ & 3 & 1 & YES & YES & YES & -- & 752\\
$(41, 18)$ & 8 & $(4, 1)$ & 3 & 1 & YES & YES & YES & -- & 753\\
$(41, 18)$ & 8 & $(4, 1)$ & 3 & 1 & YES & YES & YES & NO & 754\\
$(41, 15)$ & 8 & $(5, 1)$ & 4 & 1 & YES & YES & YES & -- & 755\\
$(41, 15)$ & 8 & $(5, 1)$ & 4 & 1 & YES & YES & YES & NO & 756\\
$(41, 15)$ & 8 & $(6, 1)$ & 5 & 1 & YES & YES & YES & -- & 757\\
$(41, 15)$ & 8 & $(6, 1)$ & 5 & 1 & YES & YES & YES & NO & 758\\
$(41, 11)$ & 8 & $(8, 3)$ & 4 & 1 & YES & YES & YES & NO & 759\\
$(41, 15)$ & 8 & $(11, 4)$ & 5 & 1 & YES & YES & YES & NO & 760\\
$(41, 18)$ & 8 & $(11, 5)$ & 6 & 1 & YES & YES & YES & NO & 761\\
$(41, 11)$ & 8 & $(23, 6)$ & 8 & 1 & YES & YES & YES & NO & 762\\
$(41, 15)$ & 8 & $(41, 15)$ & 8 & 41 & YES & YES & YES & NO & 763\\
$(42, 13)$ & 9 & $(2, 1)$ & 1 & 2 & YES & YES & YES & NO & 764\\
$(42, 19)$ & 9 & $(2, 1)$ & 1 & 2 & YES & YES & YES & -- & 765\\
$(42, 19)$ & 9 & $(2, 1)$ & 1 & 2 & YES & YES & YES & NO & 766\\
$(42, 13)$ & 9 & $(3, 1)$ & 2 & 3 & YES & YES & NO(2) & -- & 767\\
$(42, 13)$ & 9 & $(3, 1)$ & 2 & 3 & YES & YES & NO(2) & NO & 768\\
$(42, 19)$ & 9 & $(3, 1)$ & 2 & 3 & YES & YES & NO(2) & -- & 769\\
$(42, 19)$ & 9 & $(3, 1)$ & 2 & 3 & YES & YES & YES & NO & 770\\
$(42, 19)$ & 9 & $(4, 1)$ & 3 & 2 & YES & YES & YES & -- & 771\\
$(42, 19)$ & 9 & $(4, 1)$ & 3 & 2 & YES & YES & NO(2) & NO & 772\\
$(42, 13)$ & 9 & $(5, 1)$ & 4 & 1 & YES & YES & YES & -- & 773\\
$(42, 13)$ & 9 & $(5, 1)$ & 4 & 1 & YES & YES & YES & NO & 774\\
$(42, 19)$ & 9 & $(5, 2)$ & 3 & 1 & YES & YES & NO(2) & -- & 775\\
$(42, 19)$ & 9 & $(5, 2)$ & 3 & 1 & YES & YES & YES & NO & 776\\
$(42, 19)$ & 9 & $(6, 1)$ & 5 & 6 & YES & YES & YES & NO & 777\\
$(42, 19)$ & 9 & $(6, 1)$ & 5 & 6 & YES & YES & YES & NO & 778\\
$(42, 19)$ & 9 & $(7, 3)$ & 4 & 7 & YES & YES & YES & NO & 779\\
$(42, 19)$ & 9 & $(9, 4)$ & 5 & 3 & YES & YES & YES & NO & 780\\
$(42, 19)$ & 9 & $(42, 19)$ & 9 & 42 & YES & YES & YES & NO & 781\\
$(43, 19)$ & 9 & $(2, 1)$ & 1 & 1 & YES & YES & YES & -- & 782\\
$(43, 16)$ & 9 & $(3, 1)$ & 2 & 1 & YES & YES & YES & -- & 783\\
$(43, 19)$ & 9 & $(3, 1)$ & 2 & 1 & YES & YES & YES & NO & 784\\
$(43, 16)$ & 9 & $(4, 1)$ & 3 & 1 & YES & YES & YES & -- & 785\\
$(43, 19)$ & 9 & $(5, 1)$ & 4 & 1 & YES & YES & YES & -- & 786\\
$(43, 19)$ & 9 & $(5, 1)$ & 4 & 1 & YES & YES & YES & NO & 787\\
$(43, 19)$ & 9 & $(7, 1)$ & 6 & 1 & YES & YES & YES & NO & 788\\
$(43, 19)$ & 9 & $(7, 1)$ & 6 & 1 & YES & YES & YES & NO & 789\\
$(43, 19)$ & 9 & $(7, 3)$ & 4 & 1 & YES & YES & YES & NO & 790\\
$(43, 19)$ & 9 & $(9, 4)$ & 5 & 1 & YES & YES & YES & NO & 791\\
$(43, 16)$ & 9 & $(11, 4)$ & 5 & 1 & YES & YES & YES & NO & 792\\
$(43, 16)$ & 9 & $(35, 13)$ & 8 & 1 & YES & YES & YES & NO & 793\\
$(43, 19)$ & 9 & $(43, 19)$ & 9 & 43 & YES & YES & YES & NO & 794\\
$(44, 17)$ & 8 & $(2, 1)$ & 1 & 2 & YES & YES & YES & -- & 795\\
$(44, 19)$ & 10 & $(2, 1)$ & 1 & 2 & YES & YES & YES & NO & 796\\
$(44, 17)$ & 8 & $(3, 1)$ & 2 & 1 & YES & YES & YES & -- & 797\\
$(44, 17)$ & 8 & $(3, 1)$ & 2 & 1 & YES & YES & YES & NO & 798\\
$(44, 19)$ & 10 & $(3, 1)$ & 2 & 1 & YES & YES & YES & NO & 799\\
$(44, 17)$ & 8 & $(7, 3)$ & 4 & 1 & YES & YES & YES & NO & 800\\
$(44, 19)$ & 10 & $(9, 4)$ & 5 & 1 & YES & YES & YES & NO & 801\\
$(45, 13)$ & 10 & $(2, 1)$ & 1 & 1 & YES & YES & YES & -- & 802\\
$(45, 14)$ & 9 & $(2, 1)$ & 1 & 1 & YES & YES & YES & -- & 803\\
$(45, 14)$ & 9 & $(2, 1)$ & 1 & 1 & YES & YES & YES & NO & 804\\
$(45, 17)$ & 9 & $(2, 1)$ & 1 & 1 & YES & YES & YES & NO & 805\\
$(45, 14)$ & 9 & $(3, 1)$ & 2 & 3 & NO & YES & YES & -- & 806\\
$(45, 14)$ & 9 & $(3, 1)$ & 2 & 3 & YES & YES & YES & NO & 807\\
$(45, 17)$ & 9 & $(3, 1)$ & 2 & 3 & YES & YES & YES & -- & 808\\
$(45, 17)$ & 9 & $(4, 1)$ & 3 & 1 & YES & YES & NO(2) & -- & 809\\
$(45, 16)$ & 9 & $(5, 1)$ & 4 & 5 & YES & YES & NO(2) & -- & 810\\
$(45, 14)$ & 9 & $(6, 1)$ & 5 & 3 & YES & YES & YES & -- & 811\\
$(45, 14)$ & 9 & $(6, 1)$ & 5 & 3 & YES & YES & YES & NO & 812\\
$(45, 17)$ & 9 & $(6, 1)$ & 5 & 3 & YES & YES & NO(2) & NO & 813\\
$(45, 16)$ & 9 & $(17, 6)$ & 7 & 1 & YES & YES & NO(2) & 845 & 814\\
$(45, 17)$ & 9 & $(21, 8)$ & 6 & 3 & YES & YES & NO(2) & NO & 815\\
$(45, 17)$ & 9 & $(37, 14)$ & 8 & 1 & YES & YES & NO(2) & NO & 816\\
$(46, 19)$ & 8 & $(2, 1)$ & 1 & 2 & YES & YES & YES & NO & 817\\
$(46, 19)$ & 8 & $(3, 1)$ & 2 & 1 & YES & YES & YES & -- & 818\\
$(46, 19)$ & 8 & $(3, 1)$ & 2 & 1 & YES & YES & YES & NO & 819\\
$(46, 19)$ & 8 & $(3, 1)$ & 2 & 1 & YES & YES & YES & NO & 820\\
$(46, 13)$ & 10 & $(4, 1)$ & 3 & 2 & YES & YES & YES & -- & 821\\
$(46, 19)$ & 8 & $(4, 1)$ & 3 & 2 & YES & YES & YES & -- & 822\\
$(46, 21)$ & 10 & $(4, 1)$ & 3 & 2 & YES & YES & NO(2) & -- & 823\\
$(46, 21)$ & 10 & $(4, 1)$ & 3 & 2 & YES & YES & NO(2) & NO & 824\\
$(46, 19)$ & 8 & $(5, 2)$ & 3 & 1 & YES & YES & YES & NO & 825\\
$(46, 21)$ & 10 & $(5, 1)$ & 4 & 1 & YES & YES & NO(2) & NO & 826\\
$(46, 21)$ & 10 & $(5, 1)$ & 4 & 1 & YES & YES & NO(2) & NO & 827\\
$(46, 17)$ & 8 & $(7, 2)$ & 4 & 1 & YES & YES & NO(2) & -- & 828\\
$(46, 21)$ & 10 & $(7, 3)$ & 4 & 1 & YES & YES & NO(2) & NO & 829\\
$(46, 17)$ & 8 & $(14, 5)$ & 6 & 2 & YES & YES & NO(2) & NO & 830\\
$(46, 21)$ & 10 & $(24, 11)$ & 8 & 2 & YES & YES & NO(2) & 938 & 831\\
$(46, 21)$ & 10 & $(35, 16)$ & 9 & 1 & YES & YES & NO(2) & NO & 832\\
$(46, 13)$ & 10 & $(39, 11)$ & 9 & 1 & YES & YES & YES & NO & 833\\
$(47, 14)$ & 9 & $(2, 1)$ & 1 & 1 & YES & YES & NO(2) & -- & 834\\
$(47, 20)$ & 10 & $(2, 1)$ & 1 & 1 & NO & YES & YES & -- & 835\\
$(47, 14)$ & 9 & $(10, 3)$ & 5 & 1 & YES & YES & NO(2) & NO & 836\\
$(48, 17)$ & 9 & $(2, 1)$ & 1 & 2 & YES & YES & NO(2) & -- & 837\\
$(48, 17)$ & 9 & $(2, 1)$ & 1 & 2 & YES & YES & YES & NO & 838\\
$(48, 17)$ & 9 & $(3, 1)$ & 2 & 3 & YES & YES & YES & -- & 839\\
$(48, 17)$ & 9 & $(3, 1)$ & 2 & 3 & YES & YES & YES & NO & 840\\
$(48, 11)$ & 9 & $(4, 1)$ & 3 & 4 & NO & YES & YES & -- & 841\\
$(48, 17)$ & 9 & $(4, 1)$ & 3 & 4 & YES & YES & YES & -- & 842\\
$(48, 17)$ & 9 & $(6, 1)$ & 5 & 6 & YES & YES & YES & NO & 843\\
$(48, 17)$ & 9 & $(11, 4)$ & 5 & 1 & YES & YES & YES & 952 & 844\\
$(48, 17)$ & 9 & $(14, 5)$ & 6 & 2 & YES & YES & NO(2) & 814 & 845\\
$(48, 17)$ & 9 & $(17, 6)$ & 7 & 1 & YES & YES & YES & NO & 846\\
$(48, 17)$ & 9 & $(20, 7)$ & 8 & 4 & YES & YES & NO(2) & NO & 847\\
$(48, 17)$ & 9 & $(31, 11)$ & 8 & 1 & YES & YES & YES & NO & 848\\
$(48, 17)$ & 9 & $(48, 17)$ & 9 & 48 & YES & YES & YES & NO & 849\\
$(49, 18)$ & 8 & $(2, 1)$ & 1 & 1 & YES & YES & YES & -- & 850\\
$(49, 18)$ & 8 & $(2, 1)$ & 1 & 1 & YES & YES & YES & 556 & 851\\
$(49, 22)$ & 9 & $(2, 1)$ & 1 & 1 & YES & YES & YES & -- & 852\\
$(49, 15)$ & 9 & $(3, 1)$ & 2 & 1 & YES & YES & YES & -- & 853\\
$(49, 15)$ & 9 & $(3, 1)$ & 2 & 1 & YES & YES & YES & NO & 854\\
$(49, 19)$ & 8 & $(3, 1)$ & 2 & 1 & YES & YES & YES & -- & 855\\
$(49, 19)$ & 8 & $(3, 1)$ & 2 & 1 & YES & YES & YES & NO & 856\\
$(49, 19)$ & 8 & $(3, 1)$ & 2 & 1 & YES & YES & YES & NO & 857\\
$(49, 20)$ & 9 & $(3, 1)$ & 2 & 1 & YES & YES & YES & NO & 858\\
$(49, 22)$ & 9 & $(3, 1)$ & 2 & 1 & YES & YES & YES & NO & 859\\
$(49, 19)$ & 8 & $(4, 1)$ & 3 & 1 & YES & YES & YES & -- & 860\\
$(49, 22)$ & 9 & $(4, 1)$ & 3 & 1 & YES & YES & NO(2) & -- & 861\\
$(49, 22)$ & 9 & $(4, 1)$ & 3 & 1 & YES & YES & YES & NO & 862\\
$(49, 15)$ & 9 & $(5, 1)$ & 4 & 1 & YES & YES & NO(2) & -- & 863\\
$(49, 15)$ & 9 & $(5, 1)$ & 4 & 1 & YES & YES & NO(2) & NO & 864\\
$(49, 15)$ & 9 & $(5, 2)$ & 3 & 1 & YES & YES & YES & -- & 865\\
$(49, 20)$ & 9 & $(5, 1)$ & 4 & 1 & YES & YES & YES & -- & 866\\
$(49, 20)$ & 9 & $(5, 1)$ & 4 & 1 & YES & YES & YES & NO & 867\\
$(49, 20)$ & 9 & $(5, 2)$ & 3 & 1 & YES & YES & YES & 619 & 868\\
$(49, 18)$ & 8 & $(8, 3)$ & 4 & 1 & YES & YES & YES & NO & 869\\
$(49, 13)$ & 9 & $(9, 2)$ & 5 & 1 & YES & YES & YES & NO & 870\\
$(49, 13)$ & 9 & $(11, 3)$ & 5 & 1 & YES & YES & YES & NO & 871\\
$(49, 22)$ & 9 & $(20, 9)$ & 7 & 1 & YES & YES & YES & NO & 872\\
$(49, 11)$ & 10 & $(22, 5)$ & 7 & 1 & YES & YES & YES & NO & 873\\
$(49, 13)$ & 9 & $(23, 6)$ & 8 & 1 & YES & YES & YES & 1005 & 874\\
$(49, 9)$ & 10 & $(28, 5)$ & 8 & 7 & YES & YES & YES & NO & 875\\
$(49, 22)$ & 9 & $(29, 13)$ & 8 & 1 & YES & YES & YES & NO & 876\\
$(49, 20)$ & 9 & $(49, 20)$ & 9 & 49 & YES & YES & YES & NO & 877\\
$(50, 19)$ & 8 & $(2, 1)$ & 1 & 2 & YES & YES & YES & NO & 878\\
$(50, 23)$ & 10 & $(2, 1)$ & 1 & 2 & NO & YES & YES & -- & 879\\
$(50, 11)$ & 10 & $(7, 2)$ & 4 & 1 & YES & YES & NO(2) & NO & 880\\
$(51, 20)$ & 9 & $(2, 1)$ & 1 & 1 & YES & YES & YES & NO & 881\\
$(51, 23)$ & 9 & $(2, 1)$ & 1 & 1 & YES & YES & YES & -- & 882\\
$(51, 16)$ & 10 & $(3, 1)$ & 2 & 3 & NO & YES & NO(2) & -- & 883\\
$(51, 20)$ & 9 & $(3, 1)$ & 2 & 3 & YES & YES & NO(2) & -- & 884\\
$(51, 20)$ & 9 & $(4, 1)$ & 3 & 1 & YES & YES & NO(2) & NO & 885\\
$(51, 20)$ & 9 & $(5, 1)$ & 4 & 1 & YES & YES & YES & -- & 886\\
$(51, 20)$ & 9 & $(5, 1)$ & 4 & 1 & YES & YES & NO(2) & NO & 887\\
$(51, 20)$ & 9 & $(5, 2)$ & 3 & 1 & YES & YES & YES & 636 & 888\\
$(51, 23)$ & 9 & $(6, 1)$ & 5 & 3 & YES & YES & NO(2) & -- & 889\\
$(51, 23)$ & 9 & $(6, 1)$ & 5 & 3 & YES & YES & NO(2) & NO & 890\\
$(51, 20)$ & 9 & $(8, 3)$ & 4 & 1 & YES & YES & NO(2) & NO & 891\\
$(51, 23)$ & 9 & $(9, 4)$ & 5 & 3 & YES & YES & YES & NO & 892\\
$(51, 20)$ & 9 & $(13, 5)$ & 5 & 1 & YES & YES & NO(2) & 593 & 893\\
$(51, 23)$ & 9 & $(20, 9)$ & 7 & 1 & YES & YES & YES & NO & 894\\
$(51, 20)$ & 9 & $(51, 20)$ & 9 & 51 & YES & YES & NO(2) & NO & 895\\
$(52, 19)$ & 9 & $(3, 1)$ & 2 & 1 & YES & YES & YES & NO & 896\\
$(52, 19)$ & 9 & $(7, 1)$ & 6 & 1 & YES & YES & YES & NO & 897\\
$(52, 19)$ & 9 & $(11, 4)$ & 5 & 1 & YES & YES & YES & NO & 898\\
$(52, 19)$ & 9 & $(19, 7)$ & 6 & 1 & YES & YES & YES & NO & 899\\
$(53, 19)$ & 9 & $(3, 1)$ & 2 & 1 & YES & YES & NO(2) & -- & 900\\
$(53, 19)$ & 9 & $(3, 1)$ & 2 & 1 & YES & YES & YES & NO & 901\\
$(53, 19)$ & 9 & $(3, 1)$ & 2 & 1 & YES & YES & NO(2) & NO & 902\\
$(53, 14)$ & 9 & $(4, 1)$ & 3 & 1 & YES & YES & YES & NO & 903\\
$(53, 19)$ & 9 & $(4, 1)$ & 3 & 1 & YES & YES & NO(2) & -- & 904\\
$(53, 19)$ & 9 & $(5, 1)$ & 4 & 1 & YES & YES & NO(2) & NO & 905\\
$(53, 19)$ & 9 & $(5, 2)$ & 3 & 1 & YES & YES & NO(2) & NO & 906\\
$(53, 19)$ & 9 & $(8, 3)$ & 4 & 1 & YES & YES & NO(2) & NO & 907\\
$(53, 19)$ & 9 & $(14, 5)$ & 6 & 1 & YES & YES & YES & NO & 908\\
$(53, 19)$ & 9 & $(25, 9)$ & 7 & 1 & YES & YES & YES & 955 & 909\\
$(53, 19)$ & 9 & $(53, 19)$ & 9 & 53 & YES & YES & NO(2) & NO & 910\\
$(55, 16)$ & 9 & $(2, 1)$ & 1 & 1 & YES & YES & YES & -- & 911\\
$(55, 23)$ & 9 & $(2, 1)$ & 1 & 1 & NO & YES & YES & -- & 912\\
$(55, 16)$ & 9 & $(3, 1)$ & 2 & 1 & NO & YES & YES & -- & 913\\
$(55, 24)$ & 9 & $(3, 1)$ & 2 & 1 & YES & YES & YES & -- & 914\\
$(55, 24)$ & 9 & $(11, 5)$ & 6 & 11 & YES & YES & NO(2) & NO & 915\\
$(55, 24)$ & 9 & $(16, 7)$ & 6 & 1 & YES & YES & YES & NO & 916\\
$(56, 25)$ & 11 & $(2, 1)$ & 1 & 2 & NO & YES & YES & -- & 917\\
$(56, 13)$ & 10 & $(4, 1)$ & 3 & 4 & YES & YES & NO(2) & -- & 918\\
$(56, 13)$ & 10 & $(4, 1)$ & 3 & 4 & YES & YES & NO(2) & NO & 919\\
$(56, 15)$ & 9 & $(4, 1)$ & 3 & 4 & YES & YES & YES & -- & 920\\
$(56, 15)$ & 9 & $(4, 1)$ & 3 & 4 & YES & YES & YES & NO & 921\\
$(56, 17)$ & 9 & $(4, 1)$ & 3 & 4 & YES & YES & NO(2) & -- & 922\\
$(56, 17)$ & 9 & $(4, 1)$ & 3 & 4 & YES & YES & YES & NO & 923\\
$(56, 13)$ & 10 & $(13, 3)$ & 6 & 1 & YES & YES & NO(2) & NO & 924\\
$(56, 13)$ & 10 & $(25, 6)$ & 9 & 1 & YES & YES & NO(2) & NO & 925\\
$(57, 25)$ & 9 & $(3, 1)$ & 2 & 3 & YES & YES & YES & -- & 926\\
$(57, 25)$ & 9 & $(5, 2)$ & 3 & 1 & YES & YES & NO(2) & -- & 927\\
$(57, 25)$ & 9 & $(16, 7)$ & 6 & 1 & YES & YES & YES & NO & 928\\
$(59, 13)$ & 11 & $(2, 1)$ & 1 & 1 & YES & YES & YES & -- & 929\\
$(59, 13)$ & 11 & $(3, 1)$ & 2 & 1 & YES & YES & YES & -- & 930\\
$(59, 13)$ & 11 & $(3, 1)$ & 2 & 1 & YES & YES & NO(2) & NO & 931\\
$(59, 26)$ & 9 & $(3, 1)$ & 2 & 1 & YES & YES & NO(2) & NO & 932\\
$(59, 14)$ & 10 & $(4, 1)$ & 3 & 1 & YES & YES & NO(2) & -- & 933\\
$(59, 27)$ & 10 & $(5, 1)$ & 4 & 1 & YES & YES & NO(2) & NO & 934\\
$(59, 27)$ & 10 & $(5, 1)$ & 4 & 1 & YES & YES & NO(2) & NO & 935\\
$(59, 14)$ & 10 & $(7, 2)$ & 4 & 1 & YES & YES & YES & NO & 936\\
$(59, 26)$ & 9 & $(7, 3)$ & 4 & 1 & YES & YES & NO(2) & 656 & 937\\
$(59, 27)$ & 10 & $(11, 5)$ & 6 & 1 & YES & YES & NO(2) & 831 & 938\\
$(59, 13)$ & 11 & $(13, 3)$ & 6 & 1 & YES & YES & NO(2) & NO & 939\\
$(61, 24)$ & 10 & $(2, 1)$ & 1 & 1 & NO & YES & YES & -- & 940\\
$(61, 19)$ & 10 & $(3, 1)$ & 2 & 1 & NO & YES & YES & -- & 941\\
$(62, 27)$ & 9 & $(2, 1)$ & 1 & 2 & NO & YES & YES & -- & 942\\
$(63, 26)$ & 9 & $(2, 1)$ & 1 & 1 & NO & YES & YES & -- & 943\\
$(63, 26)$ & 9 & $(2, 1)$ & 1 & 1 & YES & YES & NO(2) & NO & 944\\
$(63, 26)$ & 9 & $(4, 1)$ & 3 & 1 & YES & YES & NO(2) & -- & 945\\
$(63, 26)$ & 9 & $(5, 2)$ & 3 & 1 & YES & YES & YES & NO & 946\\
$(64, 23)$ & 9 & $(2, 1)$ & 1 & 2 & YES & YES & NO(2) & -- & 947\\
$(64, 23)$ & 9 & $(2, 1)$ & 1 & 2 & YES & YES & YES & NO & 948\\
$(64, 27)$ & 9 & $(2, 1)$ & 1 & 2 & YES & YES & NO(2) & -- & 949\\
$(64, 27)$ & 9 & $(2, 1)$ & 1 & 2 & YES & YES & YES & 653 & 950\\
$(64, 23)$ & 9 & $(3, 1)$ & 2 & 1 & YES & YES & YES & -- & 951\\
$(64, 23)$ & 9 & $(3, 1)$ & 2 & 1 & YES & YES & YES & 844 & 952\\
$(64, 23)$ & 9 & $(5, 1)$ & 4 & 1 & YES & YES & NO(2) & NO & 953\\
$(64, 27)$ & 9 & $(5, 2)$ & 3 & 1 & YES & YES & NO(2) & NO & 954\\
$(64, 23)$ & 9 & $(14, 5)$ & 6 & 2 & YES & YES & YES & 909 & 955\\
$(64, 23)$ & 9 & $(39, 14)$ & 8 & 1 & YES & YES & YES & NO & 956\\
$(65, 24)$ & 9 & $(3, 1)$ & 2 & 1 & YES & YES & YES & -- & 957\\
$(65, 24)$ & 9 & $(11, 4)$ & 5 & 1 & YES & YES & NO(2) & NO & 958\\
$(65, 17)$ & 10 & $(23, 6)$ & 8 & 1 & YES & YES & YES & NO & 959\\
$(65, 24)$ & 9 & $(65, 24)$ & 9 & 65 & YES & YES & YES & NO & 960\\
$(66, 25)$ & 9 & $(2, 1)$ & 1 & 2 & NO & YES & YES & -- & 961\\
$(67, 16)$ & 11 & $(9, 2)$ & 5 & 1 & YES & YES & YES & 677 & 962\\
$(67, 16)$ & 11 & $(21, 5)$ & 8 & 1 & YES & YES & YES & NO & 963\\
$(68, 25)$ & 9 & $(2, 1)$ & 1 & 2 & NO & YES & YES & -- & 964\\
$(69, 19)$ & 9 & $(3, 1)$ & 2 & 3 & YES & YES & YES & NO & 965\\
$(71, 13)$ & 12 & $(2, 1)$ & 1 & 1 & YES & YES & YES & -- & 966\\
$(71, 13)$ & 12 & $(2, 1)$ & 1 & 1 & YES & YES & NO(2) & NO & 967\\
$(71, 22)$ & 10 & $(2, 1)$ & 1 & 1 & YES & YES & YES & NO & 968\\
$(71, 31)$ & 10 & $(2, 1)$ & 1 & 1 & YES & YES & NO(2) & NO & 969\\
$(71, 17)$ & 11 & $(3, 1)$ & 2 & 1 & YES & YES & YES & -- & 970\\
$(71, 17)$ & 11 & $(3, 1)$ & 2 & 1 & YES & YES & NO(2) & NO & 971\\
$(71, 13)$ & 12 & $(6, 1)$ & 5 & 1 & YES & YES & NO(2) & NO & 972\\
$(71, 17)$ & 11 & $(25, 6)$ & 9 & 1 & YES & YES & YES & NO & 973\\
$(71, 17)$ & 11 & $(29, 7)$ & 10 & 1 & YES & YES & NO(2) & NO & 974\\
$(72, 13)$ & 12 & $(2, 1)$ & 1 & 2 & YES & YES & NO(2) & -- & 975\\
$(72, 13)$ & 12 & $(2, 1)$ & 1 & 2 & YES & YES & NO(2) & NO & 976\\
$(72, 19)$ & 10 & $(2, 1)$ & 1 & 2 & YES & YES & NO(2) & -- & 977\\
$(72, 13)$ & 12 & $(3, 1)$ & 2 & 3 & YES & YES & YES & NO & 978\\
$(72, 13)$ & 12 & $(4, 1)$ & 3 & 4 & YES & YES & YES & NO & 979\\
$(72, 17)$ & 11 & $(4, 1)$ & 3 & 4 & NO & YES & NO(2) & -- & 980\\
$(72, 13)$ & 12 & $(5, 1)$ & 4 & 1 & YES & YES & YES & NO & 981\\
$(72, 13)$ & 12 & $(11, 2)$ & 6 & 1 & YES & YES & YES & NO & 982\\
$(73, 27)$ & 9 & $(3, 1)$ & 2 & 1 & YES & YES & NO(2) & NO & 983\\
$(73, 27)$ & 9 & $(4, 1)$ & 3 & 1 & YES & YES & NO(2) & -- & 984\\
$(73, 27)$ & 9 & $(8, 3)$ & 4 & 1 & YES & YES & YES & NO & 985\\
$(74, 17)$ & 11 & $(2, 1)$ & 1 & 2 & YES & YES & NO(2) & -- & 986\\
$(74, 29)$ & 10 & $(2, 1)$ & 1 & 2 & NO & YES & YES & -- & 987\\
$(74, 31)$ & 9 & $(2, 1)$ & 1 & 2 & NO & YES & YES & -- & 988\\
$(76, 13)$ & 12 & $(5, 1)$ & 4 & 1 & YES & YES & YES & NO & 989\\
$(76, 13)$ & 12 & $(35, 6)$ & 10 & 1 & YES & YES & YES & NO & 990\\
$(77, 16)$ & 11 & $(3, 1)$ & 2 & 1 & YES & YES & YES & -- & 991\\
$(77, 16)$ & 11 & $(3, 1)$ & 2 & 1 & YES & YES & NO(2) & NO & 992\\
$(77, 16)$ & 11 & $(11, 2)$ & 6 & 11 & YES & YES & NO(2) & NO & 993\\
$(77, 16)$ & 11 & $(24, 5)$ & 8 & 1 & YES & YES & YES & NO & 994\\
$(79, 17)$ & 11 & $(2, 1)$ & 1 & 1 & YES & YES & NO(2) & NO & 995\\
$(79, 30)$ & 9 & $(2, 1)$ & 1 & 1 & NO & YES & NO(2) & -- & 996\\
$(79, 31)$ & 10 & $(2, 1)$ & 1 & 1 & NO & YES & YES & -- & 997\\
$(79, 17)$ & 11 & $(4, 1)$ & 3 & 1 & YES & YES & NO(2) & NO & 998\\
$(79, 17)$ & 11 & $(14, 3)$ & 6 & 1 & YES & YES & YES & NO & 999\\
$(80, 19)$ & 11 & $(2, 1)$ & 1 & 2 & YES & YES & NO(2) & -- & 1000\\
$(82, 19)$ & 12 & $(4, 1)$ & 3 & 2 & YES & YES & NO(2) & -- & 1001\\
$(82, 19)$ & 12 & $(13, 3)$ & 6 & 1 & YES & YES & NO(2) & NO & 1002\\
$(83, 22)$ & 10 & $(3, 1)$ & 2 & 1 & YES & YES & NO(2) & NO & 1003\\
$(83, 22)$ & 10 & $(4, 1)$ & 3 & 1 & YES & YES & YES & -- & 1004\\
$(83, 22)$ & 10 & $(4, 1)$ & 3 & 1 & YES & YES & YES & 874 & 1005\\
$(83, 22)$ & 10 & $(34, 9)$ & 8 & 1 & YES & YES & NO(2) & NO & 1006\\
$(84, 37)$ & 10 & $(2, 1)$ & 1 & 2 & NO & YES & NO(2) & -- & 1007\\
$(85, 37)$ & 10 & $(2, 1)$ & 1 & 1 & NO & YES & NO(2) & -- & 1008\\
$(88, 21)$ & 12 & $(4, 1)$ & 3 & 4 & YES & YES & NO(2) & NO & 1009\\
$(88, 21)$ & 12 & $(67, 16)$ & 11 & 1 & YES & YES & NO(2) & NO & 1010\\
$(89, 27)$ & 10 & $(2, 1)$ & 1 & 1 & YES & YES & NO(2) & NO & 1011\\
$(89, 40)$ & 11 & $(2, 1)$ & 1 & 1 & NO & YES & NO(2) & -- & 1012\\
$(89, 17)$ & 12 & $(5, 1)$ & 4 & 1 & YES & YES & YES & NO & 1013\\
$(91, 19)$ & 11 & $(3, 1)$ & 2 & 1 & YES & YES & YES & -- & 1014\\
$(91, 19)$ & 11 & $(3, 1)$ & 2 & 1 & YES & YES & NO(2) & NO & 1015\\
$(91, 17)$ & 12 & $(6, 1)$ & 5 & 1 & YES & YES & NO(2) & NO & 1016\\
$(91, 17)$ & 12 & $(11, 2)$ & 6 & 1 & YES & YES & NO(2) & NO & 1017\\
$(91, 19)$ & 11 & $(24, 5)$ & 8 & 1 & YES & YES & YES & NO & 1018\\
$(91, 19)$ & 11 & $(29, 6)$ & 9 & 1 & YES & YES & NO(2) & NO & 1019\\
$(92, 19)$ & 12 & $(7, 1)$ & 6 & 1 & YES & YES & NO(2) & NO & 1020\\
$(92, 19)$ & 12 & $(29, 6)$ & 9 & 1 & YES & YES & NO(2) & NO & 1021\\
$(96, 17)$ & 12 & $(2, 1)$ & 1 & 2 & YES & YES & NO(2) & NO & 1022\\
$(97, 26)$ & 10 & $(2, 1)$ & 1 & 1 & YES & YES & NO(2) & NO & 1023\\
$(97, 26)$ & 10 & $(3, 1)$ & 2 & 1 & YES & YES & NO(2) & NO & 1024\\
$(97, 26)$ & 10 & $(4, 1)$ & 3 & 1 & YES & YES & YES & NO & 1025\\
$(99, 17)$ & 12 & $(2, 1)$ & 1 & 1 & YES & YES & YES & NO & 1026\\
$(99, 17)$ & 12 & $(35, 6)$ & 10 & 1 & YES & YES & YES & NO & 1027\\
$(101, 16)$ & 13 & $(7, 1)$ & 6 & 1 & YES & YES & YES & NO & 1028\\
$(101, 16)$ & 13 & $(19, 3)$ & 8 & 1 & YES & YES & YES & NO & 1029\\
$(120, 19)$ & 14 & $(19, 3)$ & 8 & 1 & YES & YES & NO(2) & NO & 1030\\
$(a; 2, 0, 0; 17)$ & 6 & $(2, 1)$ & 1 & 1 & YES & YES & YES & -- & 1031\\
$(a; 2, 0, 0; 17)$ & 6 & $(5, 2)$ & 3 & 1 & YES & YES & NO(2) & -- & 1032\\
$(a; 3, 0, 0; 7)$ & 7 & $(3, 1)$ & 2 & 1 & YES & YES & NO(2) & -- & 1033\\
$(a; 3, 0, 0; 7)$ & 7 & $(7, 2)$ & 4 & 7 & YES & YES & YES & -- & 1034\\
$(a; 3, 0, 1; 31)$ & 8 & $(3, 1)$ & 2 & 1 & YES & YES & YES & -- & 1035\\
$(a; 3, 1, 0; 31)$ & 8 & $(2, 1)$ & 1 & 1 & YES & YES & YES & -- & 1036\\
$(a; 3, 1, 0; 31)$ & 8 & $(3, 1)$ & 2 & 1 & YES & YES & NO(2) & -- & 1037\\
$(a; 3, 1, 0; 31)$ & 8 & $(4, 1)$ & 3 & 1 & YES & YES & YES & -- & 1038\\
$(a; 4, 0, 0; 25)$ & 8 & $(3, 1)$ & 2 & 1 & YES & YES & YES & -- & 1039\\
$(a; 4, 0, 0; 25)$ & 8 & $(7, 3)$ & 4 & 1 & YES & YES & NO(2) & -- & 1040\\
$(a; 4, 2, 0; 7)$ & 10 & $(5, 1)$ & 4 & 1 & YES & YES & NO(2) & -- & 1041\\
$(b; 0, 0, 3; 32)$ & 8 & $(2, 1)$ & 1 & 2 & YES & YES & NO(2) & -- & 1042\\
$(b; 0, 1, 0; 19)$ & 6 & $(9, 4)$ & 5 & 1 & YES & YES & NO(2) & -- & 1043\\
$(b; 0, 2, 0; 8)$ & 7 & $(4, 1)$ & 3 & 4 & YES & YES & NO(2) & -- & 1044\\
$(b; 0, 3, 0; 29)$ & 8 & $(2, 1)$ & 1 & 1 & YES & YES & YES & -- & 1045\\
$(b; 0, 3, 0; 29)$ & 8 & $(3, 1)$ & 2 & 1 & YES & YES & NO(2) & -- & 1046\\
$(b; 0, 3, 0; 29)$ & 8 & $(5, 1)$ & 4 & 1 & YES & YES & NO(2) & -- & 1047\\
$(b; 3, 0, 0; 16)$ & 8 & $(2, 1)$ & 1 & 2 & YES & YES & YES & -- & 1048\\
$(c; 0, 0, 0; 4)$ & 4 & $(15, 4)$ & 6 & 1 & YES & YES & NO(2) & -- & 1049\\
$(c; 0, 0, 0; 4)$ & 4 & $(16, 7)$ & 6 & 4 & YES & YES & NO(2) & -- & 1050\\
$(c; 0, 0, 0; 4)$ & 4 & $(20, 9)$ & 7 & 4 & YES & YES & YES & -- & 1051\\
$(c; 0, 0, 0; 4)$ & 4 & $(25, 9)$ & 7 & 1 & YES & YES & YES & -- & 1052\\
$(c; 0, 1, 0; 11)$ & 5 & $(9, 4)$ & 5 & 1 & YES & YES & YES & -- & 1053\\
$(c; 0, 1, 0; 11)$ & 5 & $(11, 4)$ & 5 & 11 & YES & YES & YES & -- & 1054\\
$(c; 0, 1, 0; 11)$ & 5 & $(11, 5)$ & 6 & 11 & YES & YES & YES & -- & 1055\\
$(c; 0, 1, 1; 5)$ & 6 & $(11, 3)$ & 5 & 1 & YES & YES & YES & -- & 1056\\
$(c; 0, 1, 1; 5)$ & 6 & $(13, 4)$ & 6 & 1 & YES & YES & YES & -- & 1057\\
$(c; 0, 2, 0; 7)$ & 6 & $(5, 1)$ & 4 & 1 & YES & YES & NO(2) & -- & 1058\\
$(c; 0, 2, 0; 7)$ & 6 & $(5, 2)$ & 3 & 1 & YES & YES & NO(2) & -- & 1059\\
$(c; 0, 2, 0; 7)$ & 6 & $(6, 1)$ & 5 & 1 & YES & YES & NO(2) & -- & 1060\\
$(c; 0, 2, 0; 7)$ & 6 & $(8, 3)$ & 4 & 1 & YES & YES & NO(2) & -- & 1061\\
$(c; 0, 2, 1; 19)$ & 7 & $(4, 1)$ & 3 & 1 & YES & YES & NO(2) & -- & 1062\\
$(c; 0, 2, 1; 19)$ & 7 & $(11, 3)$ & 5 & 1 & YES & YES & YES & -- & 1063\\
$(c; 0, 3, 0; 17)$ & 7 & $(4, 1)$ & 3 & 1 & YES & YES & YES & -- & 1064\\
$(c; 0, 3, 0; 17)$ & 7 & $(5, 1)$ & 4 & 1 & YES & YES & YES & -- & 1065\\
$(c; 0, 3, 0; 17)$ & 7 & $(5, 2)$ & 3 & 1 & YES & YES & YES & -- & 1066\\
$(c; 0, 3, 0; 17)$ & 7 & $(8, 3)$ & 4 & 1 & YES & YES & YES & -- & 1067\\
$(c; 0, 3, 1; 23)$ & 8 & $(4, 1)$ & 3 & 1 & YES & YES & YES & -- & 1068\\
$(c; 0, 3, 1; 23)$ & 8 & $(5, 1)$ & 4 & 1 & YES & YES & YES & -- & 1069\\
$(c; 0, 3, 1; 23)$ & 8 & $(6, 1)$ & 5 & 1 & YES & YES & YES & -- & 1070\\
$(c; 0, 3, 2; 29)$ & 9 & $(3, 1)$ & 2 & 1 & YES & YES & YES & -- & 1071\\
$(c; 0, 3, 2; 29)$ & 9 & $(5, 1)$ & 4 & 1 & YES & YES & YES & -- & 1072\\
$(c; 0, 4, 0; 10)$ & 8 & $(3, 1)$ & 2 & 1 & YES & YES & YES & -- & 1073\\
$(c; 0, 4, 0; 10)$ & 8 & $(4, 1)$ & 3 & 2 & YES & YES & YES & -- & 1074\\
$(c; 0, 4, 0; 10)$ & 8 & $(6, 1)$ & 5 & 2 & YES & YES & YES & -- & 1075\\
$(c; 0, 4, 1; 9)$ & 9 & $(4, 1)$ & 3 & 1 & YES & YES & YES & -- & 1076\\
$(c; 0, 4, 1; 9)$ & 9 & $(7, 1)$ & 6 & 1 & YES & YES & YES & -- & 1077\\
$(d; 0, 0, 0; 5)$ & 5 & $(11, 4)$ & 5 & 1 & YES & YES & YES & -- & 1078\\
$(d; 0, 0, 1; 14)$ & 6 & $(11, 3)$ & 5 & 1 & YES & YES & YES & -- & 1079\\
$(d; 0, 0, 2; 9)$ & 7 & $(9, 2)$ & 5 & 9 & YES & YES & NO(2) & -- & 1080\\
$(d; 0, 0, 3; 22)$ & 8 & $(4, 1)$ & 3 & 2 & YES & YES & YES & -- & 1081\\
$(d; 0, 0, 3; 22)$ & 8 & $(5, 1)$ & 4 & 1 & YES & YES & YES & -- & 1082\\
$(d; 0, 0, 3; 22)$ & 8 & $(7, 2)$ & 4 & 1 & YES & YES & YES & -- & 1083\\
$(d; 0, 0, 4; 13)$ & 9 & $(3, 1)$ & 2 & 1 & YES & YES & YES & -- & 1084\\
$(d; 0, 1, 0; 6)$ & 6 & $(5, 1)$ & 4 & 1 & YES & YES & NO(2) & -- & 1085\\
$(d; 0, 1, 0; 6)$ & 6 & $(5, 2)$ & 3 & 1 & YES & YES & NO(2) & -- & 1086\\
$(d; 0, 1, 0; 6)$ & 6 & $(6, 1)$ & 5 & 6 & YES & YES & NO(2) & -- & 1087\\
$(d; 0, 1, 0; 6)$ & 6 & $(8, 3)$ & 4 & 2 & YES & YES & NO(2) & -- & 1088\\
$(d; 0, 1, 2; 11)$ & 8 & $(4, 1)$ & 3 & 1 & YES & YES & NO(2) & -- & 1089\\
$(d; 0, 2, 0; 7)$ & 7 & $(4, 1)$ & 3 & 1 & YES & YES & YES & -- & 1090\\
$(d; 0, 2, 1; 20)$ & 8 & $(4, 1)$ & 3 & 4 & YES & YES & YES & -- & 1091\\
$(d; 0, 3, 1; 23)$ & 9 & $(3, 1)$ & 2 & 1 & YES & YES & YES & -- & 1092\\
$(d; 0, 3, 1; 23)$ & 9 & $(7, 1)$ & 6 & 1 & YES & YES & YES & -- & 1093\\
$(e; 0, 1, 0; 5)$ & 6 & $(7, 3)$ & 4 & 1 & YES & YES & NO(2) & -- & 1094\\
$(e; 0, 2, 0; 6)$ & 7 & $(7, 2)$ & 4 & 1 & YES & YES & NO(2) & -- & 1095\\
$(e; 0, 3, 0; 7)$ & 8 & $(2, 1)$ & 1 & 1 & YES & YES & YES & -- & 1096\\
$(e; 0, 3, 0; 7)$ & 8 & $(3, 1)$ & 2 & 1 & YES & YES & NO(2) & -- & 1097\\
$(e; 0, 3, 0; 7)$ & 8 & $(5, 1)$ & 4 & 1 & YES & YES & NO(2) & -- & 1098\\
$(e; 3, 0, 0; 10)$ & 8 & $(2, 1)$ & 1 & 2 & YES & YES & NO(2) & -- & 1099\\
$(f; 0, 0, 0; 6)$ & 4 & $(16, 5)$ & 7 & 2 & YES & YES & YES & -- & 1100\\
$(f; 0, 0, 0; 6)$ & 4 & $(18, 7)$ & 6 & 6 & YES & YES & YES & -- & 1101\\
$(f; 0, 0, 0; 6)$ & 4 & $(19, 5)$ & 7 & 1 & YES & YES & YES & -- & 1102\\
$(f; 0, 0, 0; 6)$ & 4 & $(19, 6)$ & 8 & 1 & YES & YES & YES & -- & 1103\\
$(f; 0, 0, 0; 6)$ & 4 & $(23, 7)$ & 7 & 1 & YES & YES & NO(2) & -- & 1104\\
$(f; 0, 0, 0; 6)$ & 4 & $(23, 9)$ & 7 & 1 & YES & YES & YES & -- & 1105\\
$(f; 0, 0, 0; 6)$ & 4 & $(24, 11)$ & 8 & 6 & YES & YES & NO(2) & -- & 1106\\
$(f; 0, 0, 0; 6)$ & 4 & $(26, 11)$ & 7 & 2 & YES & YES & YES & -- & 1107\\
$(f; 0, 0, 0; 6)$ & 4 & $(29, 13)$ & 8 & 1 & YES & YES & NO(2) & -- & 1108\\
$(f; 0, 0, 0; 6)$ & 4 & $(30, 13)$ & 8 & 6 & YES & YES & NO(2) & -- & 1109\\
$(f; 0, 0, 0; 6)$ & 4 & $(35, 8)$ & 8 & 1 & YES & YES & YES & -- & 1110\\
$(f; 0, 1, 0; 7)$ & 5 & $(10, 3)$ & 5 & 1 & YES & YES & YES & -- & 1111\\
$(f; 0, 1, 0; 7)$ & 5 & $(13, 4)$ & 6 & 1 & YES & YES & YES & -- & 1112\\
$(f; 0, 1, 0; 7)$ & 5 & $(13, 5)$ & 5 & 1 & YES & YES & YES & -- & 1113\\
$(f; 0, 1, 0; 7)$ & 5 & $(19, 4)$ & 7 & 1 & YES & YES & YES & -- & 1114\\
$(f; 0, 2, 0; 8)$ & 6 & $(11, 3)$ & 5 & 1 & YES & YES & NO(2) & -- & 1115\\
$(i; 0, 0, 0; 9)$ & 5 & $(6, 1)$ & 5 & 3 & YES & YES & YES & -- & 1116\\
$(i; 0, 0, 0; 9)$ & 5 & $(9, 4)$ & 5 & 9 & YES & YES & YES & -- & 1117\\
$(i; 0, 0, 0; 9)$ & 5 & $(10, 3)$ & 5 & 1 & YES & YES & YES & -- & 1118\\
$(i; 0, 0, 0; 9)$ & 5 & $(12, 5)$ & 5 & 3 & YES & YES & YES & -- & 1119\\
$(i; 0, 0, 0; 9)$ & 5 & $(19, 4)$ & 7 & 1 & YES & YES & YES & -- & 1120\\
$(i; 0, 0, 0; 9)$ & 5 & $(22, 5)$ & 7 & 1 & YES & YES & NO(2) & -- & 1121\\
$(i; 0, 1, 0; 12)$ & 6 & $(5, 1)$ & 4 & 1 & YES & YES & NO(2) & -- & 1122\\
$(i; 0, 1, 0; 12)$ & 6 & $(8, 3)$ & 4 & 4 & YES & YES & YES & -- & 1123\\
$(i; 0, 1, 0; 12)$ & 6 & $(13, 3)$ & 6 & 1 & YES & YES & YES & -- & 1124\\
$(i; 0, 1, 0; 12)$ & 6 & $(14, 3)$ & 6 & 2 & YES & YES & YES & -- & 1125\\
$(i; 0, 2, 0; 15)$ & 7 & $(4, 1)$ & 3 & 1 & YES & YES & YES & -- & 1126\\
$(i; 0, 3, 0; 18)$ & 8 & $(2, 1)$ & 1 & 2 & YES & YES & YES & -- & 1127\\
$(i; 0, 3, 0; 18)$ & 8 & $(3, 1)$ & 2 & 3 & YES & YES & YES & -- & 1128\\
$(i; 0, 3, 0; 18)$ & 8 & $(4, 1)$ & 3 & 2 & YES & YES & YES & -- & 1129\\
$(i; 0, 3, 0; 18)$ & 8 & $(5, 1)$ & 4 & 1 & YES & YES & YES & -- & 1130\\
$(j; 0, 0, 0; 8)$ & 5 & $(9, 4)$ & 5 & 1 & YES & YES & YES & -- & 1131\\
$(j; 0, 0, 0; 8)$ & 5 & $(11, 5)$ & 6 & 1 & YES & YES & YES & -- & 1132
\end{longtable}
\subsection{2 chains, $K^2 = 3$}
\begin{longtable}{|c|c|c|c|c|c|c|c|c|c|}
\hline
\multicolumn{10}{|c|}{2 chains, $K^2 = 3$}\\
\hline
$(n,a)$ & Length & $(n,a)$ & Length & GCD & Nef & $\mathbb Q$-ef & Obstruction 0 & WH & Index\\
\hline
\endfirsthead

\hline
$(n,a)$ & Length & $(n,a)$ & Length & GCD & Nef & $\mathbb Q$-ef & Obstruction 0 & WH & Index\\
\hline
\endhead
\hline
\endfoot

$(16, 7)$ & 6 & $(16, 7)$ & 6 & 16 & YES & YES & YES & -- & 1133\\
$(17, 3)$ & 7 & $(14, 5)$ & 6 & 1 & YES & YES & YES & -- & 1134\\
$(17, 5)$ & 6 & $(14, 3)$ & 6 & 1 & YES & YES & YES & -- & 1135\\
$(19, 5)$ & 7 & $(10, 3)$ & 5 & 1 & YES & YES & YES & -- & 1136\\
$(19, 4)$ & 7 & $(16, 7)$ & 6 & 1 & YES & YES & NO(2) & -- & 1137\\
$(19, 6)$ & 8 & $(17, 3)$ & 7 & 1 & YES & YES & YES & -- & 1138\\
$(19, 6)$ & 8 & $(17, 3)$ & 7 & 1 & YES & YES & YES & NO & 1139\\
$(19, 6)$ & 8 & $(17, 7)$ & 6 & 1 & YES & YES & YES & NO & 1140\\
$(20, 9)$ & 7 & $(13, 3)$ & 6 & 1 & YES & YES & NO(2) & -- & 1141\\
$(20, 9)$ & 7 & $(13, 3)$ & 6 & 1 & YES & YES & NO(2) & NO & 1142\\
$(20, 9)$ & 7 & $(16, 5)$ & 7 & 4 & YES & YES & YES & -- & 1143\\
$(20, 7)$ & 8 & $(18, 5)$ & 6 & 2 & YES & YES & NO(2) & -- & 1144\\
$(20, 7)$ & 8 & $(18, 5)$ & 6 & 2 & YES & YES & NO(2) & NO & 1145\\
$(20, 7)$ & 8 & $(20, 7)$ & 8 & 20 & YES & YES & YES & -- & 1146\\
$(21, 8)$ & 6 & $(9, 2)$ & 5 & 3 & YES & YES & YES & -- & 1147\\
$(21, 4)$ & 8 & $(16, 5)$ & 7 & 1 & YES & YES & YES & -- & 1148\\
$(21, 4)$ & 8 & $(16, 5)$ & 7 & 1 & YES & YES & YES & NO & 1149\\
$(21, 4)$ & 8 & $(16, 5)$ & 7 & 1 & YES & YES & YES & NO & 1150\\
$(22, 7)$ & 9 & $(18, 7)$ & 6 & 2 & YES & YES & YES & -- & 1151\\
$(23, 6)$ & 8 & $(17, 3)$ & 7 & 1 & YES & YES & YES & -- & 1152\\
$(23, 6)$ & 8 & $(17, 3)$ & 7 & 1 & YES & YES & YES & NO & 1153\\
$(23, 8)$ & 9 & $(23, 5)$ & 7 & 23 & YES & YES & YES & NO & 1154\\
$(24, 7)$ & 7 & $(19, 5)$ & 7 & 1 & YES & YES & NO(2) & -- & 1155\\
$(24, 5)$ & 8 & $(24, 5)$ & 8 & 24 & YES & YES & YES & -- & 1156\\
$(25, 11)$ & 7 & $(16, 5)$ & 7 & 1 & YES & YES & YES & -- & 1157\\
$(25, 11)$ & 7 & $(16, 5)$ & 7 & 1 & YES & YES & YES & NO & 1158\\
$(25, 9)$ & 7 & $(21, 5)$ & 8 & 1 & YES & YES & YES & -- & 1159\\
$(25, 9)$ & 7 & $(21, 5)$ & 8 & 1 & YES & YES & YES & NO & 1160\\
$(26, 7)$ & 7 & $(13, 3)$ & 6 & 13 & YES & YES & NO(2) & -- & 1161\\
$(26, 7)$ & 7 & $(14, 3)$ & 6 & 2 & YES & YES & NO(2) & -- & 1162\\
$(26, 7)$ & 7 & $(18, 5)$ & 6 & 2 & YES & YES & NO(2) & -- & 1163\\
$(26, 7)$ & 7 & $(19, 7)$ & 6 & 1 & YES & YES & NO(2) & -- & 1164\\
$(27, 11)$ & 8 & $(9, 2)$ & 5 & 9 & YES & YES & YES & -- & 1165\\
$(27, 8)$ & 7 & $(19, 7)$ & 6 & 1 & YES & YES & NO(2) & -- & 1166\\
$(27, 8)$ & 7 & $(19, 7)$ & 6 & 1 & YES & YES & NO(2) & NO & 1167\\
$(27, 11)$ & 8 & $(19, 8)$ & 6 & 1 & YES & YES & YES & -- & 1168\\
$(27, 11)$ & 8 & $(19, 8)$ & 6 & 1 & YES & YES & NO(2) & NO & 1169\\
$(28, 11)$ & 8 & $(12, 5)$ & 5 & 4 & YES & YES & YES & NO & 1170\\
$(28, 11)$ & 8 & $(17, 3)$ & 7 & 1 & YES & YES & YES & -- & 1171\\
$(28, 11)$ & 8 & $(17, 3)$ & 7 & 1 & YES & YES & YES & NO & 1172\\
$(29, 11)$ & 7 & $(13, 5)$ & 5 & 1 & YES & YES & NO(2) & -- & 1173\\
$(29, 13)$ & 8 & $(13, 4)$ & 6 & 1 & YES & YES & YES & -- & 1174\\
$(29, 13)$ & 8 & $(13, 4)$ & 6 & 1 & YES & YES & YES & NO & 1175\\
$(29, 7)$ & 10 & $(18, 7)$ & 6 & 1 & YES & YES & YES & NO & 1176\\
$(29, 8)$ & 7 & $(19, 6)$ & 8 & 1 & YES & YES & NO(2) & -- & 1177\\
$(29, 8)$ & 7 & $(19, 6)$ & 8 & 1 & YES & YES & NO(2) & NO & 1178\\
$(29, 9)$ & 8 & $(19, 6)$ & 8 & 1 & YES & YES & YES & -- & 1179\\
$(29, 9)$ & 8 & $(19, 8)$ & 6 & 1 & YES & YES & NO(2) & -- & 1180\\
$(29, 9)$ & 8 & $(19, 8)$ & 6 & 1 & YES & YES & YES & NO & 1181\\
$(29, 12)$ & 7 & $(27, 8)$ & 7 & 1 & YES & YES & NO(2) & NO & 1182\\
$(29, 8)$ & 7 & $(28, 11)$ & 8 & 1 & YES & YES & YES & -- & 1183\\
$(31, 7)$ & 8 & $(5, 2)$ & 3 & 1 & YES & YES & NO(2) & -- & 1184\\
$(31, 11)$ & 8 & $(7, 2)$ & 4 & 1 & YES & YES & NO(2) & NO & 1185\\
$(31, 14)$ & 8 & $(7, 2)$ & 4 & 1 & YES & YES & YES & -- & 1186\\
$(31, 14)$ & 8 & $(7, 2)$ & 4 & 1 & YES & YES & YES & NO & 1187\\
$(31, 14)$ & 8 & $(10, 3)$ & 5 & 1 & YES & YES & NO(2) & -- & 1188\\
$(31, 11)$ & 8 & $(11, 5)$ & 6 & 1 & YES & YES & NO(2) & -- & 1189\\
$(31, 14)$ & 8 & $(13, 5)$ & 5 & 1 & YES & YES & YES & -- & 1190\\
$(31, 14)$ & 8 & $(16, 5)$ & 7 & 1 & YES & YES & YES & -- & 1191\\
$(31, 11)$ & 8 & $(17, 4)$ & 7 & 1 & YES & YES & YES & -- & 1192\\
$(31, 11)$ & 8 & $(17, 4)$ & 7 & 1 & YES & YES & YES & NO & 1193\\
$(31, 14)$ & 8 & $(27, 11)$ & 8 & 1 & YES & YES & YES & NO & 1194\\
$(31, 5)$ & 10 & $(29, 6)$ & 9 & 1 & YES & YES & YES & -- & 1195\\
$(31, 12)$ & 7 & $(31, 12)$ & 7 & 31 & YES & YES & YES & -- & 1196\\
$(32, 13)$ & 9 & $(13, 2)$ & 7 & 1 & YES & YES & YES & -- & 1197\\
$(32, 13)$ & 9 & $(13, 2)$ & 7 & 1 & YES & YES & YES & NO & 1198\\
$(32, 13)$ & 9 & $(13, 6)$ & 7 & 1 & YES & YES & YES & -- & 1199\\
$(32, 7)$ & 8 & $(20, 7)$ & 8 & 4 & YES & YES & YES & NO & 1200\\
$(32, 13)$ & 9 & $(20, 7)$ & 8 & 4 & YES & YES & YES & NO & 1201\\
$(32, 7)$ & 8 & $(22, 7)$ & 9 & 2 & YES & YES & YES & NO & 1202\\
$(32, 13)$ & 9 & $(22, 5)$ & 7 & 2 & YES & YES & NO(2) & NO & 1203\\
$(33, 14)$ & 8 & $(9, 2)$ & 5 & 3 & YES & YES & YES & -- & 1204\\
$(33, 14)$ & 8 & $(13, 3)$ & 6 & 1 & YES & YES & YES & NO & 1205\\
$(33, 10)$ & 8 & $(15, 4)$ & 6 & 3 & YES & YES & NO(2) & -- & 1206\\
$(33, 10)$ & 8 & $(15, 4)$ & 6 & 3 & YES & YES & NO(2) & NO & 1207\\
$(33, 10)$ & 8 & $(16, 5)$ & 7 & 1 & YES & YES & NO(2) & -- & 1208\\
$(33, 10)$ & 8 & $(16, 5)$ & 7 & 1 & YES & YES & NO(2) & NO & 1209\\
$(33, 10)$ & 8 & $(17, 7)$ & 6 & 1 & YES & YES & NO(2) & NO & 1210\\
$(33, 14)$ & 8 & $(17, 7)$ & 6 & 1 & YES & YES & NO(2) & -- & 1211\\
$(33, 14)$ & 8 & $(17, 7)$ & 6 & 1 & YES & YES & NO(2) & NO & 1212\\
$(33, 13)$ & 9 & $(19, 3)$ & 8 & 1 & YES & YES & YES & NO & 1213\\
$(33, 10)$ & 8 & $(23, 10)$ & 7 & 1 & YES & YES & NO(2) & NO & 1214\\
$(34, 9)$ & 8 & $(5, 2)$ & 3 & 1 & YES & YES & YES & NO & 1215\\
$(34, 9)$ & 8 & $(13, 5)$ & 5 & 1 & YES & YES & YES & -- & 1216\\
$(34, 13)$ & 7 & $(13, 6)$ & 7 & 1 & YES & YES & NO(2) & -- & 1217\\
$(34, 15)$ & 8 & $(18, 5)$ & 6 & 2 & YES & YES & YES & -- & 1218\\
$(34, 13)$ & 7 & $(20, 7)$ & 8 & 2 & YES & YES & NO(2) & NO & 1219\\
$(34, 13)$ & 7 & $(26, 11)$ & 7 & 2 & YES & YES & YES & -- & 1220\\
$(35, 11)$ & 9 & $(7, 3)$ & 4 & 7 & YES & YES & YES & -- & 1221\\
$(35, 16)$ & 9 & $(8, 3)$ & 4 & 1 & YES & YES & YES & -- & 1222\\
$(35, 11)$ & 9 & $(10, 3)$ & 5 & 5 & YES & YES & YES & -- & 1223\\
$(35, 11)$ & 9 & $(10, 3)$ & 5 & 5 & YES & YES & YES & NO & 1224\\
$(35, 16)$ & 9 & $(11, 4)$ & 5 & 1 & YES & YES & YES & NO & 1225\\
$(35, 11)$ & 9 & $(17, 7)$ & 6 & 1 & YES & YES & NO(2) & -- & 1226\\
$(35, 8)$ & 8 & $(23, 10)$ & 7 & 1 & YES & YES & YES & -- & 1227\\
$(36, 13)$ & 8 & $(11, 4)$ & 5 & 1 & YES & YES & YES & -- & 1228\\
$(36, 13)$ & 8 & $(17, 7)$ & 6 & 1 & YES & YES & NO(2) & -- & 1229\\
$(36, 11)$ & 8 & $(20, 9)$ & 7 & 4 & YES & YES & YES & NO & 1230\\
$(36, 13)$ & 8 & $(23, 5)$ & 7 & 1 & YES & YES & NO(2) & -- & 1231\\
$(37, 16)$ & 9 & $(9, 2)$ & 5 & 1 & YES & YES & YES & -- & 1232\\
$(37, 16)$ & 9 & $(11, 2)$ & 6 & 1 & YES & YES & YES & NO & 1233\\
$(37, 11)$ & 8 & $(13, 6)$ & 7 & 1 & YES & YES & YES & -- & 1234\\
$(37, 14)$ & 8 & $(13, 4)$ & 6 & 1 & YES & YES & YES & -- & 1235\\
$(37, 14)$ & 8 & $(13, 4)$ & 6 & 1 & YES & YES & YES & NO & 1236\\
$(37, 8)$ & 8 & $(20, 7)$ & 8 & 1 & YES & YES & NO(2) & NO & 1237\\
$(37, 8)$ & 8 & $(22, 7)$ & 9 & 1 & YES & YES & NO(2) & NO & 1238\\
$(37, 14)$ & 8 & $(23, 4)$ & 8 & 1 & YES & YES & YES & NO & 1239\\
$(37, 11)$ & 8 & $(31, 12)$ & 7 & 1 & YES & YES & YES & -- & 1240\\
$(38, 9)$ & 9 & $(14, 3)$ & 6 & 2 & YES & YES & NO(2) & -- & 1241\\
$(38, 17)$ & 9 & $(16, 3)$ & 7 & 2 & YES & YES & YES & -- & 1242\\
$(38, 7)$ & 9 & $(22, 7)$ & 9 & 2 & YES & YES & NO(2) & NO & 1243\\
$(38, 7)$ & 9 & $(27, 7)$ & 9 & 1 & YES & YES & NO(2) & NO & 1244\\
$(38, 9)$ & 9 & $(31, 9)$ & 8 & 1 & YES & YES & NO(2) & NO & 1245\\
$(39, 14)$ & 8 & $(7, 2)$ & 4 & 1 & YES & YES & YES & -- & 1246\\
$(39, 14)$ & 8 & $(7, 2)$ & 4 & 1 & YES & YES & YES & NO & 1247\\
$(39, 14)$ & 8 & $(24, 7)$ & 7 & 3 & YES & YES & YES & -- & 1248\\
$(39, 14)$ & 8 & $(25, 7)$ & 7 & 1 & YES & YES & YES & -- & 1249\\
$(39, 11)$ & 9 & $(38, 7)$ & 9 & 1 & YES & YES & YES & NO & 1250\\
$(40, 11)$ & 8 & $(7, 2)$ & 4 & 1 & YES & YES & NO(2) & -- & 1251\\
$(40, 11)$ & 8 & $(9, 4)$ & 5 & 1 & YES & YES & YES & -- & 1252\\
$(40, 11)$ & 8 & $(9, 4)$ & 5 & 1 & YES & YES & YES & NO & 1253\\
$(40, 11)$ & 8 & $(9, 4)$ & 5 & 1 & YES & YES & YES & NO & 1254\\
$(40, 11)$ & 8 & $(13, 6)$ & 7 & 1 & YES & YES & YES & -- & 1255\\
$(40, 17)$ & 9 & $(13, 2)$ & 7 & 1 & YES & YES & YES & -- & 1256\\
$(40, 17)$ & 9 & $(13, 2)$ & 7 & 1 & YES & YES & YES & NO & 1257\\
$(40, 17)$ & 9 & $(13, 5)$ & 5 & 1 & YES & YES & YES & -- & 1258\\
$(40, 11)$ & 8 & $(16, 5)$ & 7 & 8 & YES & YES & YES & NO & 1259\\
$(40, 11)$ & 8 & $(22, 7)$ & 9 & 2 & YES & YES & YES & NO & 1260\\
$(40, 9)$ & 9 & $(39, 7)$ & 9 & 1 & YES & YES & NO(2) & -- & 1261\\
$(41, 11)$ & 8 & $(5, 2)$ & 3 & 1 & YES & YES & NO(2) & NO & 1262\\
$(41, 13)$ & 10 & $(7, 2)$ & 4 & 1 & YES & YES & YES & -- & 1263\\
$(41, 13)$ & 10 & $(7, 2)$ & 4 & 1 & YES & YES & YES & NO & 1264\\
$(41, 19)$ & 10 & $(7, 3)$ & 4 & 1 & YES & YES & NO(2) & -- & 1265\\
$(41, 18)$ & 8 & $(8, 3)$ & 4 & 1 & YES & YES & NO(2) & -- & 1266\\
$(41, 9)$ & 9 & $(11, 2)$ & 6 & 1 & YES & YES & NO(2) & -- & 1267\\
$(41, 9)$ & 9 & $(11, 2)$ & 6 & 1 & YES & YES & NO(2) & NO & 1268\\
$(41, 18)$ & 8 & $(18, 7)$ & 6 & 1 & YES & YES & YES & -- & 1269\\
$(41, 17)$ & 8 & $(23, 7)$ & 7 & 1 & YES & YES & YES & -- & 1270\\
$(41, 15)$ & 8 & $(24, 7)$ & 7 & 1 & YES & YES & YES & -- & 1271\\
$(41, 15)$ & 8 & $(24, 7)$ & 7 & 1 & YES & YES & YES & NO & 1272\\
$(41, 18)$ & 8 & $(25, 7)$ & 7 & 1 & YES & YES & YES & NO & 1273\\
$(41, 12)$ & 8 & $(38, 9)$ & 9 & 1 & YES & YES & NO(2) & NO & 1274\\
$(42, 19)$ & 9 & $(5, 2)$ & 3 & 1 & YES & YES & NO(2) & -- & 1275\\
$(42, 19)$ & 9 & $(16, 3)$ & 7 & 2 & YES & YES & YES & NO & 1276\\
$(42, 13)$ & 9 & $(18, 7)$ & 6 & 6 & YES & YES & NO(2) & -- & 1277\\
$(42, 5)$ & 11 & $(23, 8)$ & 9 & 1 & YES & YES & YES & NO & 1278\\
$(42, 11)$ & 9 & $(23, 7)$ & 7 & 1 & YES & YES & YES & NO & 1279\\
$(43, 15)$ & 10 & $(7, 3)$ & 4 & 1 & YES & YES & NO(2) & -- & 1280\\
$(43, 19)$ & 9 & $(7, 2)$ & 4 & 1 & YES & YES & NO(2) & -- & 1281\\
$(43, 16)$ & 9 & $(25, 9)$ & 7 & 1 & YES & YES & YES & NO & 1282\\
$(43, 13)$ & 9 & $(28, 5)$ & 8 & 1 & YES & YES & NO(2) & -- & 1283\\
$(44, 13)$ & 8 & $(13, 6)$ & 7 & 1 & YES & YES & NO(2) & -- & 1284\\
$(44, 13)$ & 8 & $(19, 5)$ & 7 & 1 & YES & YES & NO(2) & NO & 1285\\
$(44, 13)$ & 8 & $(23, 9)$ & 7 & 1 & YES & YES & YES & -- & 1286\\
$(44, 17)$ & 8 & $(24, 7)$ & 7 & 4 & YES & YES & YES & -- & 1287\\
$(45, 14)$ & 9 & $(5, 2)$ & 3 & 5 & YES & YES & NO(2) & -- & 1288\\
$(45, 16)$ & 9 & $(8, 3)$ & 4 & 1 & YES & YES & YES & -- & 1289\\
$(45, 14)$ & 9 & $(10, 3)$ & 5 & 5 & YES & YES & NO(2) & -- & 1290\\
$(45, 17)$ & 9 & $(10, 3)$ & 5 & 5 & YES & YES & YES & -- & 1291\\
$(45, 17)$ & 9 & $(10, 3)$ & 5 & 5 & YES & YES & YES & NO & 1292\\
$(45, 19)$ & 8 & $(12, 5)$ & 5 & 3 & YES & YES & NO(2) & -- & 1293\\
$(45, 19)$ & 8 & $(24, 7)$ & 7 & 3 & YES & YES & YES & -- & 1294\\
$(45, 19)$ & 8 & $(33, 14)$ & 8 & 3 & YES & YES & YES & NO & 1295\\
$(47, 18)$ & 8 & $(9, 4)$ & 5 & 1 & YES & YES & YES & -- & 1296\\
$(47, 18)$ & 8 & $(9, 4)$ & 5 & 1 & YES & YES & YES & NO & 1297\\
$(47, 13)$ & 8 & $(13, 6)$ & 7 & 1 & YES & YES & NO(2) & -- & 1298\\
$(47, 20)$ & 10 & $(13, 3)$ & 6 & 1 & YES & YES & NO(2) & -- & 1299\\
$(47, 13)$ & 8 & $(17, 6)$ & 7 & 1 & YES & YES & NO(2) & NO & 1300\\
$(47, 17)$ & 9 & $(17, 3)$ & 7 & 1 & YES & YES & NO(2) & -- & 1301\\
$(47, 13)$ & 8 & $(22, 7)$ & 9 & 1 & YES & YES & NO(2) & NO & 1302\\
$(47, 13)$ & 8 & $(23, 9)$ & 7 & 1 & YES & YES & YES & -- & 1303\\
$(47, 13)$ & 8 & $(32, 9)$ & 8 & 1 & YES & YES & NO(2) & NO & 1304\\
$(48, 17)$ & 9 & $(7, 2)$ & 4 & 1 & YES & YES & YES & -- & 1305\\
$(48, 11)$ & 9 & $(11, 3)$ & 5 & 1 & YES & YES & NO(2) & -- & 1306\\
$(48, 11)$ & 9 & $(11, 3)$ & 5 & 1 & YES & YES & NO(2) & NO & 1307\\
$(48, 17)$ & 9 & $(19, 7)$ & 6 & 1 & YES & YES & YES & 1506 & 1308\\
$(48, 17)$ & 9 & $(20, 7)$ & 8 & 4 & YES & YES & YES & NO & 1309\\
$(48, 13)$ & 9 & $(38, 7)$ & 9 & 2 & YES & YES & YES & -- & 1310\\
$(49, 13)$ & 9 & $(5, 2)$ & 3 & 1 & YES & YES & YES & -- & 1311\\
$(49, 13)$ & 9 & $(5, 2)$ & 3 & 1 & YES & YES & YES & NO & 1312\\
$(49, 20)$ & 9 & $(5, 1)$ & 4 & 1 & YES & YES & YES & -- & 1313\\
$(49, 20)$ & 9 & $(7, 2)$ & 4 & 7 & YES & YES & NO(2) & -- & 1314\\
$(49, 9)$ & 10 & $(11, 5)$ & 6 & 1 & YES & YES & YES & -- & 1315\\
$(49, 13)$ & 9 & $(11, 4)$ & 5 & 1 & YES & YES & YES & -- & 1316\\
$(49, 13)$ & 9 & $(11, 4)$ & 5 & 1 & YES & YES & YES & NO & 1317\\
$(49, 18)$ & 8 & $(23, 8)$ & 9 & 1 & YES & YES & YES & NO & 1318\\
$(49, 19)$ & 8 & $(24, 7)$ & 7 & 1 & YES & YES & YES & -- & 1319\\
$(49, 11)$ & 10 & $(25, 4)$ & 9 & 1 & YES & YES & YES & -- & 1320\\
$(49, 18)$ & 8 & $(25, 7)$ & 7 & 1 & YES & YES & YES & -- & 1321\\
$(49, 18)$ & 8 & $(25, 7)$ & 7 & 1 & YES & YES & YES & NO & 1322\\
$(49, 20)$ & 9 & $(32, 13)$ & 9 & 1 & YES & YES & YES & NO & 1323\\
$(50, 13)$ & 10 & $(13, 5)$ & 5 & 1 & YES & YES & NO(2) & -- & 1324\\
$(50, 19)$ & 8 & $(18, 7)$ & 6 & 2 & YES & YES & YES & -- & 1325\\
$(51, 14)$ & 9 & $(7, 2)$ & 4 & 1 & YES & YES & NO(2) & -- & 1326\\
$(51, 23)$ & 9 & $(7, 3)$ & 4 & 1 & YES & YES & YES & -- & 1327\\
$(51, 16)$ & 10 & $(12, 5)$ & 5 & 3 & YES & YES & YES & NO & 1328\\
$(51, 11)$ & 9 & $(18, 7)$ & 6 & 3 & YES & YES & NO(2) & NO & 1329\\
$(51, 11)$ & 9 & $(27, 10)$ & 7 & 3 & YES & YES & NO(2) & -- & 1330\\
$(52, 19)$ & 9 & $(7, 2)$ & 4 & 1 & YES & YES & NO(2) & -- & 1331\\
$(52, 23)$ & 10 & $(7, 2)$ & 4 & 1 & YES & YES & YES & -- & 1332\\
$(52, 11)$ & 9 & $(17, 7)$ & 6 & 1 & YES & YES & NO(2) & -- & 1333\\
$(52, 11)$ & 9 & $(17, 7)$ & 6 & 1 & YES & YES & NO(2) & NO & 1334\\
$(52, 15)$ & 11 & $(17, 3)$ & 7 & 1 & YES & YES & NO(2) & -- & 1335\\
$(52, 11)$ & 9 & $(25, 7)$ & 7 & 1 & YES & YES & NO(2) & -- & 1336\\
$(52, 11)$ & 9 & $(25, 7)$ & 7 & 1 & YES & YES & NO(2) & NO & 1337\\
$(52, 11)$ & 9 & $(43, 10)$ & 9 & 1 & YES & YES & NO(2) & NO & 1338\\
$(53, 19)$ & 9 & $(4, 1)$ & 3 & 1 & YES & YES & NO(2) & -- & 1339\\
$(53, 14)$ & 9 & $(5, 2)$ & 3 & 1 & YES & YES & YES & -- & 1340\\
$(53, 14)$ & 9 & $(5, 2)$ & 3 & 1 & YES & YES & YES & NO & 1341\\
$(53, 15)$ & 11 & $(5, 1)$ & 4 & 1 & YES & YES & YES & -- & 1342\\
$(53, 15)$ & 11 & $(5, 1)$ & 4 & 1 & YES & YES & YES & NO & 1343\\
$(53, 19)$ & 9 & $(5, 2)$ & 3 & 1 & YES & YES & YES & -- & 1344\\
$(53, 22)$ & 9 & $(6, 1)$ & 5 & 1 & YES & YES & YES & NO & 1345\\
$(53, 12)$ & 9 & $(7, 3)$ & 4 & 1 & YES & YES & NO(2) & -- & 1346\\
$(53, 14)$ & 9 & $(7, 2)$ & 4 & 1 & YES & YES & NO(2) & -- & 1347\\
$(53, 14)$ & 9 & $(7, 2)$ & 4 & 1 & YES & YES & NO(2) & NO & 1348\\
$(53, 14)$ & 9 & $(7, 3)$ & 4 & 1 & YES & YES & YES & -- & 1349\\
$(53, 19)$ & 9 & $(7, 3)$ & 4 & 1 & YES & YES & NO(2) & -- & 1350\\
$(53, 24)$ & 10 & $(7, 2)$ & 4 & 1 & YES & YES & YES & NO & 1351\\
$(53, 11)$ & 10 & $(9, 4)$ & 5 & 1 & YES & YES & YES & -- & 1352\\
$(53, 14)$ & 9 & $(9, 2)$ & 5 & 1 & YES & YES & YES & -- & 1353\\
$(53, 14)$ & 9 & $(9, 2)$ & 5 & 1 & YES & YES & YES & NO & 1354\\
$(53, 14)$ & 9 & $(10, 3)$ & 5 & 1 & YES & YES & YES & NO & 1355\\
$(53, 20)$ & 10 & $(11, 3)$ & 5 & 1 & YES & YES & YES & NO & 1356\\
$(53, 24)$ & 10 & $(11, 3)$ & 5 & 1 & YES & YES & NO(2) & -- & 1357\\
$(53, 14)$ & 9 & $(12, 5)$ & 5 & 1 & YES & YES & NO(2) & -- & 1358\\
$(53, 24)$ & 10 & $(17, 7)$ & 6 & 1 & YES & YES & NO(2) & NO & 1359\\
$(53, 14)$ & 9 & $(18, 5)$ & 6 & 1 & YES & YES & NO(2) & 1576 & 1360\\
$(53, 24)$ & 10 & $(19, 8)$ & 6 & 1 & YES & YES & NO(2) & NO & 1361\\
$(53, 7)$ & 11 & $(20, 7)$ & 8 & 1 & YES & YES & YES & NO & 1362\\
$(53, 22)$ & 9 & $(22, 5)$ & 7 & 1 & YES & YES & YES & -- & 1363\\
$(53, 14)$ & 9 & $(23, 5)$ & 7 & 1 & YES & YES & NO(2) & -- & 1364\\
$(53, 14)$ & 9 & $(23, 6)$ & 8 & 1 & YES & YES & YES & NO & 1365\\
$(53, 14)$ & 9 & $(23, 7)$ & 7 & 1 & YES & YES & NO(2) & NO & 1366\\
$(53, 22)$ & 9 & $(23, 5)$ & 7 & 1 & YES & YES & YES & -- & 1367\\
$(53, 14)$ & 9 & $(26, 7)$ & 7 & 1 & YES & YES & YES & NO & 1368\\
$(53, 24)$ & 10 & $(29, 13)$ & 8 & 1 & YES & YES & YES & NO & 1369\\
$(53, 15)$ & 11 & $(39, 11)$ & 9 & 1 & YES & YES & YES & 1564 & 1370\\
$(53, 7)$ & 11 & $(43, 7)$ & 12 & 1 & YES & YES & YES & NO & 1371\\
$(53, 24)$ & 10 & $(51, 23)$ & 9 & 1 & YES & YES & YES & 1651 & 1372\\
$(54, 17)$ & 10 & $(9, 4)$ & 5 & 9 & YES & YES & YES & NO & 1373\\
$(55, 23)$ & 9 & $(6, 1)$ & 5 & 1 & YES & YES & YES & -- & 1374\\
$(55, 23)$ & 9 & $(8, 3)$ & 4 & 1 & YES & YES & YES & -- & 1375\\
$(55, 23)$ & 9 & $(8, 3)$ & 4 & 1 & YES & YES & YES & NO & 1376\\
$(55, 21)$ & 8 & $(11, 5)$ & 6 & 11 & YES & YES & NO(2) & NO & 1377\\
$(55, 21)$ & 8 & $(18, 7)$ & 6 & 1 & YES & YES & YES & -- & 1378\\
$(55, 16)$ & 9 & $(21, 5)$ & 8 & 1 & YES & YES & YES & NO & 1379\\
$(56, 15)$ & 9 & $(3, 1)$ & 2 & 1 & YES & YES & YES & NO & 1380\\
$(56, 15)$ & 9 & $(13, 5)$ & 5 & 1 & YES & YES & YES & NO & 1381\\
$(56, 15)$ & 9 & $(18, 7)$ & 6 & 2 & YES & YES & YES & -- & 1382\\
$(56, 15)$ & 9 & $(18, 7)$ & 6 & 2 & YES & YES & YES & NO & 1383\\
$(57, 17)$ & 10 & $(13, 5)$ & 5 & 1 & YES & YES & YES & NO & 1384\\
$(57, 22)$ & 9 & $(23, 4)$ & 8 & 1 & YES & YES & NO(2) & NO & 1385\\
$(57, 17)$ & 10 & $(29, 9)$ & 8 & 1 & YES & YES & YES & NO & 1386\\
$(58, 17)$ & 9 & $(16, 5)$ & 7 & 2 & YES & YES & NO(2) & NO & 1387\\
$(58, 9)$ & 11 & $(17, 6)$ & 7 & 1 & YES & YES & YES & NO & 1388\\
$(58, 17)$ & 9 & $(22, 7)$ & 9 & 2 & YES & YES & NO(2) & NO & 1389\\
$(58, 9)$ & 11 & $(31, 6)$ & 10 & 1 & YES & YES & YES & NO & 1390\\
$(58, 13)$ & 11 & $(53, 12)$ & 9 & 1 & YES & YES & NO(2) & NO & 1391\\
$(59, 24)$ & 10 & $(4, 1)$ & 3 & 1 & YES & YES & YES & -- & 1392\\
$(59, 25)$ & 9 & $(4, 1)$ & 3 & 1 & YES & YES & YES & -- & 1393\\
$(59, 25)$ & 9 & $(4, 1)$ & 3 & 1 & YES & YES & YES & NO & 1394\\
$(59, 25)$ & 9 & $(5, 2)$ & 3 & 1 & YES & YES & YES & -- & 1395\\
$(59, 26)$ & 9 & $(5, 2)$ & 3 & 1 & YES & YES & NO(2) & -- & 1396\\
$(59, 24)$ & 10 & $(9, 4)$ & 5 & 1 & YES & YES & YES & NO & 1397\\
$(59, 24)$ & 10 & $(11, 2)$ & 6 & 1 & YES & YES & YES & -- & 1398\\
$(59, 25)$ & 9 & $(12, 5)$ & 5 & 1 & YES & YES & NO(2) & -- & 1399\\
$(59, 26)$ & 9 & $(12, 5)$ & 5 & 1 & YES & YES & NO(2) & NO & 1400\\
$(59, 23)$ & 9 & $(17, 5)$ & 6 & 1 & YES & YES & YES & -- & 1401\\
$(59, 23)$ & 9 & $(18, 5)$ & 6 & 1 & YES & YES & YES & -- & 1402\\
$(59, 24)$ & 10 & $(19, 8)$ & 6 & 1 & YES & YES & YES & NO & 1403\\
$(59, 26)$ & 9 & $(23, 10)$ & 7 & 1 & YES & YES & NO(2) & NO & 1404\\
$(59, 23)$ & 9 & $(33, 13)$ & 9 & 1 & YES & YES & YES & NO & 1405\\
$(59, 25)$ & 9 & $(40, 17)$ & 9 & 1 & YES & YES & YES & NO & 1406\\
$(60, 19)$ & 11 & $(7, 3)$ & 4 & 1 & YES & YES & YES & -- & 1407\\
$(60, 23)$ & 9 & $(7, 3)$ & 4 & 1 & YES & YES & YES & -- & 1408\\
$(60, 23)$ & 9 & $(7, 3)$ & 4 & 1 & YES & YES & YES & NO & 1409\\
$(60, 13)$ & 9 & $(11, 4)$ & 5 & 1 & YES & YES & YES & -- & 1410\\
$(60, 13)$ & 9 & $(11, 4)$ & 5 & 1 & YES & YES & YES & NO & 1411\\
$(60, 23)$ & 9 & $(12, 5)$ & 5 & 12 & YES & YES & YES & NO & 1412\\
$(60, 19)$ & 11 & $(54, 17)$ & 10 & 6 & YES & YES & YES & NO & 1413\\
$(61, 25)$ & 9 & $(3, 1)$ & 2 & 1 & YES & YES & NO(2) & -- & 1414\\
$(61, 25)$ & 9 & $(4, 1)$ & 3 & 1 & YES & YES & YES & -- & 1415\\
$(61, 25)$ & 9 & $(4, 1)$ & 3 & 1 & YES & YES & YES & NO & 1416\\
$(61, 18)$ & 9 & $(5, 2)$ & 3 & 1 & YES & YES & NO(2) & -- & 1417\\
$(61, 25)$ & 9 & $(5, 2)$ & 3 & 1 & YES & YES & YES & -- & 1418\\
$(61, 25)$ & 9 & $(5, 2)$ & 3 & 1 & YES & YES & NO(2) & NO & 1419\\
$(61, 25)$ & 9 & $(5, 2)$ & 3 & 1 & YES & YES & YES & NO & 1420\\
$(61, 22)$ & 9 & $(9, 4)$ & 5 & 1 & YES & YES & NO(2) & -- & 1421\\
$(61, 18)$ & 9 & $(11, 3)$ & 5 & 1 & YES & YES & NO(2) & NO & 1422\\
$(61, 16)$ & 10 & $(12, 5)$ & 5 & 1 & YES & YES & YES & NO & 1423\\
$(61, 16)$ & 10 & $(13, 5)$ & 5 & 1 & YES & YES & YES & NO & 1424\\
$(61, 23)$ & 11 & $(13, 2)$ & 7 & 1 & YES & YES & NO(2) & -- & 1425\\
$(61, 14)$ & 10 & $(16, 5)$ & 7 & 1 & YES & YES & NO(2) & NO & 1426\\
$(61, 17)$ & 9 & $(19, 5)$ & 7 & 1 & YES & YES & NO(2) & NO & 1427\\
$(61, 25)$ & 9 & $(32, 13)$ & 9 & 1 & YES & YES & YES & NO & 1428\\
$(61, 18)$ & 9 & $(42, 13)$ & 9 & 1 & YES & YES & YES & NO & 1429\\
$(63, 26)$ & 9 & $(13, 6)$ & 7 & 1 & YES & YES & YES & NO & 1430\\
$(63, 11)$ & 10 & $(19, 5)$ & 7 & 1 & YES & YES & YES & -- & 1431\\
$(64, 19)$ & 9 & $(12, 5)$ & 5 & 4 & YES & YES & NO(2) & NO & 1432\\
$(64, 15)$ & 10 & $(18, 7)$ & 6 & 2 & YES & YES & YES & -- & 1433\\
$(64, 15)$ & 10 & $(18, 7)$ & 6 & 2 & YES & YES & YES & NO & 1434\\
$(65, 23)$ & 10 & $(4, 1)$ & 3 & 1 & YES & YES & NO(2) & -- & 1435\\
$(65, 19)$ & 9 & $(7, 3)$ & 4 & 1 & YES & YES & NO(2) & -- & 1436\\
$(65, 23)$ & 10 & $(8, 3)$ & 4 & 1 & YES & YES & NO(2) & NO & 1437\\
$(65, 27)$ & 10 & $(11, 3)$ & 5 & 1 & YES & YES & YES & -- & 1438\\
$(65, 24)$ & 9 & $(20, 7)$ & 8 & 5 & YES & YES & YES & NO & 1439\\
$(66, 25)$ & 9 & $(61, 23)$ & 11 & 1 & YES & YES & NO(2) & NO & 1440\\
$(67, 21)$ & 11 & $(5, 1)$ & 4 & 1 & YES & YES & YES & -- & 1441\\
$(67, 18)$ & 9 & $(7, 2)$ & 4 & 1 & YES & YES & NO(2) & -- & 1442\\
$(67, 18)$ & 9 & $(7, 2)$ & 4 & 1 & YES & YES & NO(2) & NO & 1443\\
$(67, 20)$ & 11 & $(7, 3)$ & 4 & 1 & YES & YES & YES & NO & 1444\\
$(67, 24)$ & 10 & $(7, 2)$ & 4 & 1 & YES & YES & YES & NO & 1445\\
$(67, 26)$ & 9 & $(9, 4)$ & 5 & 1 & YES & YES & NO(2) & -- & 1446\\
$(67, 29)$ & 10 & $(44, 19)$ & 10 & 1 & YES & YES & YES & NO & 1447\\
$(67, 18)$ & 9 & $(53, 14)$ & 9 & 1 & YES & YES & NO(2) & NO & 1448\\
$(68, 19)$ & 9 & $(7, 3)$ & 4 & 1 & YES & YES & YES & -- & 1449\\
$(68, 25)$ & 9 & $(31, 11)$ & 8 & 1 & YES & YES & NO(2) & NO & 1450\\
$(69, 26)$ & 12 & $(8, 1)$ & 7 & 1 & YES & YES & YES & NO & 1451\\
$(69, 19)$ & 9 & $(9, 4)$ & 5 & 3 & YES & YES & NO(2) & -- & 1452\\
$(69, 19)$ & 9 & $(9, 4)$ & 5 & 3 & YES & YES & NO(2) & NO & 1453\\
$(69, 26)$ & 12 & $(29, 11)$ & 7 & 1 & YES & YES & YES & NO & 1454\\
$(70, 29)$ & 9 & $(17, 5)$ & 6 & 1 & YES & YES & YES & -- & 1455\\
$(70, 27)$ & 10 & $(20, 3)$ & 8 & 10 & YES & YES & NO(2) & -- & 1456\\
$(71, 26)$ & 9 & $(2, 1)$ & 1 & 1 & YES & YES & NO(2) & -- & 1457\\
$(71, 15)$ & 10 & $(3, 1)$ & 2 & 1 & YES & YES & YES & -- & 1458\\
$(71, 15)$ & 10 & $(3, 1)$ & 2 & 1 & YES & YES & YES & NO & 1459\\
$(71, 26)$ & 9 & $(4, 1)$ & 3 & 1 & YES & YES & NO(2) & -- & 1460\\
$(71, 13)$ & 12 & $(7, 3)$ & 4 & 1 & YES & YES & YES & -- & 1461\\
$(71, 17)$ & 11 & $(7, 3)$ & 4 & 1 & YES & YES & YES & -- & 1462\\
$(71, 15)$ & 10 & $(9, 2)$ & 5 & 1 & YES & YES & YES & NO & 1463\\
$(71, 27)$ & 9 & $(12, 5)$ & 5 & 1 & YES & YES & YES & -- & 1464\\
$(71, 16)$ & 10 & $(14, 3)$ & 6 & 1 & YES & YES & NO(2) & NO & 1465\\
$(71, 20)$ & 10 & $(15, 4)$ & 6 & 1 & YES & YES & NO(2) & NO & 1466\\
$(71, 19)$ & 10 & $(16, 5)$ & 7 & 1 & YES & YES & NO(2) & NO & 1467\\
$(71, 21)$ & 9 & $(17, 7)$ & 6 & 1 & YES & YES & YES & -- & 1468\\
$(71, 13)$ & 12 & $(19, 3)$ & 8 & 1 & YES & YES & YES & NO & 1469\\
$(71, 27)$ & 9 & $(45, 17)$ & 9 & 1 & YES & YES & NO(2) & NO & 1470\\
$(72, 19)$ & 10 & $(7, 2)$ & 4 & 1 & YES & YES & NO(2) & -- & 1471\\
$(73, 11)$ & 11 & $(2, 1)$ & 1 & 1 & YES & YES & YES & NO & 1472\\
$(73, 27)$ & 9 & $(5, 2)$ & 3 & 1 & YES & YES & NO(2) & -- & 1473\\
$(73, 27)$ & 9 & $(5, 2)$ & 3 & 1 & YES & YES & NO(2) & NO & 1474\\
$(73, 28)$ & 10 & $(5, 1)$ & 4 & 1 & YES & YES & NO(2) & -- & 1475\\
$(73, 11)$ & 11 & $(6, 1)$ & 5 & 1 & YES & YES & YES & -- & 1476\\
$(73, 11)$ & 11 & $(6, 1)$ & 5 & 1 & YES & YES & YES & NO & 1477\\
$(73, 11)$ & 11 & $(6, 1)$ & 5 & 1 & YES & YES & YES & NO & 1478\\
$(73, 19)$ & 11 & $(8, 3)$ & 4 & 1 & YES & YES & YES & -- & 1479\\
$(73, 14)$ & 11 & $(11, 5)$ & 6 & 1 & YES & YES & YES & NO & 1480\\
$(73, 11)$ & 11 & $(13, 6)$ & 7 & 1 & YES & YES & YES & -- & 1481\\
$(73, 31)$ & 10 & $(13, 3)$ & 6 & 1 & YES & YES & NO(2) & -- & 1482\\
$(73, 33)$ & 10 & $(13, 3)$ & 6 & 1 & YES & YES & NO(2) & NO & 1483\\
$(73, 11)$ & 11 & $(17, 6)$ & 7 & 1 & YES & YES & YES & -- & 1484\\
$(73, 19)$ & 11 & $(17, 5)$ & 6 & 1 & YES & YES & YES & NO & 1485\\
$(73, 11)$ & 11 & $(43, 7)$ & 12 & 1 & YES & YES & YES & NO & 1486\\
$(73, 11)$ & 11 & $(71, 11)$ & 12 & 1 & YES & YES & YES & NO & 1487\\
$(74, 13)$ & 11 & $(3, 1)$ & 2 & 1 & YES & YES & NO(2) & NO & 1488\\
$(74, 31)$ & 9 & $(17, 5)$ & 6 & 1 & YES & YES & YES & -- & 1489\\
$(74, 13)$ & 11 & $(31, 6)$ & 10 & 1 & YES & YES & YES & NO & 1490\\
$(74, 29)$ & 10 & $(33, 13)$ & 9 & 1 & YES & YES & YES & 1824 & 1491\\
$(75, 23)$ & 11 & $(6, 1)$ & 5 & 3 & YES & YES & NO(2) & -- & 1492\\
$(75, 29)$ & 9 & $(13, 5)$ & 5 & 1 & YES & YES & YES & -- & 1493\\
$(75, 17)$ & 10 & $(25, 7)$ & 7 & 25 & YES & YES & YES & NO & 1494\\
$(75, 17)$ & 10 & $(51, 11)$ & 9 & 3 & YES & YES & NO(2) & NO & 1495\\
$(77, 34)$ & 10 & $(3, 1)$ & 2 & 1 & YES & YES & YES & -- & 1496\\
$(77, 34)$ & 10 & $(5, 2)$ & 3 & 1 & YES & YES & YES & NO & 1497\\
$(77, 34)$ & 10 & $(7, 2)$ & 4 & 7 & YES & YES & NO(2) & -- & 1498\\
$(77, 34)$ & 10 & $(41, 18)$ & 8 & 1 & YES & YES & NO(2) & NO & 1499\\
$(79, 28)$ & 10 & $(4, 1)$ & 3 & 1 & YES & YES & YES & -- & 1500\\
$(79, 28)$ & 10 & $(4, 1)$ & 3 & 1 & YES & YES & NO(2) & NO & 1501\\
$(79, 17)$ & 11 & $(5, 2)$ & 3 & 1 & YES & YES & NO(2) & -- & 1502\\
$(79, 30)$ & 9 & $(5, 2)$ & 3 & 1 & YES & YES & NO(2) & -- & 1503\\
$(79, 33)$ & 11 & $(6, 1)$ & 5 & 1 & YES & YES & YES & NO & 1504\\
$(79, 31)$ & 10 & $(7, 3)$ & 4 & 1 & YES & YES & YES & NO & 1505\\
$(79, 28)$ & 10 & $(8, 3)$ & 4 & 1 & YES & YES & YES & 1308 & 1506\\
$(79, 30)$ & 9 & $(13, 4)$ & 6 & 1 & YES & YES & YES & -- & 1507\\
$(79, 30)$ & 9 & $(13, 4)$ & 6 & 1 & YES & YES & YES & NO & 1508\\
$(79, 23)$ & 10 & $(14, 3)$ & 6 & 1 & YES & YES & NO(2) & -- & 1509\\
$(79, 23)$ & 10 & $(17, 5)$ & 6 & 1 & YES & YES & YES & -- & 1510\\
$(79, 30)$ & 9 & $(34, 13)$ & 7 & 1 & YES & YES & NO(2) & NO & 1511\\
$(79, 30)$ & 9 & $(41, 16)$ & 8 & 1 & YES & YES & YES & 1847 & 1512\\
$(79, 33)$ & 11 & $(43, 18)$ & 8 & 1 & YES & YES & YES & NO & 1513\\
$(79, 18)$ & 10 & $(55, 13)$ & 10 & 1 & YES & YES & YES & NO & 1514\\
$(79, 14)$ & 11 & $(63, 11)$ & 10 & 1 & YES & YES & YES & NO & 1515\\
$(79, 33)$ & 11 & $(67, 28)$ & 10 & 1 & YES & YES & YES & NO & 1516\\
$(79, 33)$ & 11 & $(79, 33)$ & 11 & 79 & YES & YES & YES & NO & 1517\\
$(80, 19)$ & 11 & $(5, 1)$ & 4 & 5 & YES & YES & NO(2) & -- & 1518\\
$(80, 19)$ & 11 & $(5, 1)$ & 4 & 5 & YES & YES & NO(2) & NO & 1519\\
$(80, 31)$ & 9 & $(5, 2)$ & 3 & 5 & YES & YES & YES & -- & 1520\\
$(80, 33)$ & 10 & $(7, 2)$ & 4 & 1 & YES & YES & NO(2) & -- & 1521\\
$(80, 19)$ & 11 & $(13, 3)$ & 6 & 1 & YES & YES & NO(2) & NO & 1522\\
$(81, 35)$ & 11 & $(4, 1)$ & 3 & 1 & YES & YES & YES & -- & 1523\\
$(81, 31)$ & 9 & $(9, 4)$ & 5 & 9 & YES & YES & NO(2) & NO & 1524\\
$(81, 32)$ & 12 & $(33, 13)$ & 9 & 3 & YES & YES & YES & NO & 1525\\
$(81, 35)$ & 11 & $(44, 19)$ & 10 & 1 & YES & YES & YES & NO & 1526\\
$(82, 31)$ & 10 & $(3, 1)$ & 2 & 1 & YES & YES & YES & -- & 1527\\
$(82, 31)$ & 10 & $(5, 2)$ & 3 & 1 & YES & YES & YES & -- & 1528\\
$(82, 31)$ & 10 & $(7, 3)$ & 4 & 1 & YES & YES & YES & NO & 1529\\
$(82, 23)$ & 10 & $(12, 5)$ & 5 & 2 & YES & YES & YES & -- & 1530\\
$(82, 31)$ & 10 & $(13, 5)$ & 5 & 1 & YES & YES & YES & NO & 1531\\
$(82, 31)$ & 10 & $(82, 31)$ & 10 & 82 & YES & YES & YES & NO & 1532\\
$(83, 18)$ & 10 & $(2, 1)$ & 1 & 1 & YES & YES & YES & -- & 1533\\
$(83, 24)$ & 11 & $(2, 1)$ & 1 & 1 & YES & YES & YES & -- & 1534\\
$(83, 18)$ & 10 & $(3, 1)$ & 2 & 1 & YES & YES & NO(2) & -- & 1535\\
$(83, 18)$ & 10 & $(3, 1)$ & 2 & 1 & YES & YES & NO(2) & NO & 1536\\
$(83, 24)$ & 11 & $(3, 1)$ & 2 & 1 & YES & YES & NO(2) & -- & 1537\\
$(83, 36)$ & 10 & $(4, 1)$ & 3 & 1 & YES & YES & YES & -- & 1538\\
$(83, 36)$ & 10 & $(4, 1)$ & 3 & 1 & YES & YES & YES & NO & 1539\\
$(83, 18)$ & 10 & $(5, 2)$ & 3 & 1 & YES & YES & NO(2) & -- & 1540\\
$(83, 18)$ & 10 & $(5, 2)$ & 3 & 1 & YES & YES & NO(2) & NO & 1541\\
$(83, 29)$ & 12 & $(5, 1)$ & 4 & 1 & YES & YES & NO(2) & -- & 1542\\
$(83, 24)$ & 11 & $(10, 3)$ & 5 & 1 & YES & YES & NO(2) & NO & 1543\\
$(83, 13)$ & 11 & $(11, 5)$ & 6 & 1 & YES & YES & NO(2) & -- & 1544\\
$(83, 13)$ & 11 & $(11, 5)$ & 6 & 1 & YES & YES & NO(2) & NO & 1545\\
$(83, 29)$ & 12 & $(11, 4)$ & 5 & 1 & YES & YES & NO(2) & NO & 1546\\
$(83, 18)$ & 10 & $(13, 3)$ & 6 & 1 & YES & YES & NO(2) & NO & 1547\\
$(83, 19)$ & 10 & $(17, 7)$ & 6 & 1 & YES & YES & YES & -- & 1548\\
$(83, 18)$ & 10 & $(52, 11)$ & 9 & 1 & YES & YES & NO(2) & NO & 1549\\
$(83, 18)$ & 10 & $(83, 18)$ & 10 & 83 & YES & YES & NO(2) & NO & 1550\\
$(84, 25)$ & 10 & $(3, 1)$ & 2 & 3 & YES & YES & YES & -- & 1551\\
$(84, 25)$ & 10 & $(3, 1)$ & 2 & 3 & YES & YES & YES & NO & 1552\\
$(84, 13)$ & 13 & $(7, 2)$ & 4 & 7 & YES & YES & YES & -- & 1553\\
$(84, 13)$ & 13 & $(7, 2)$ & 4 & 7 & YES & YES & YES & NO & 1554\\
$(84, 13)$ & 13 & $(7, 3)$ & 4 & 7 & YES & YES & YES & -- & 1555\\
$(84, 13)$ & 13 & $(7, 3)$ & 4 & 7 & YES & YES & YES & NO & 1556\\
$(84, 37)$ & 10 & $(7, 2)$ & 4 & 7 & YES & YES & YES & -- & 1557\\
$(84, 25)$ & 10 & $(23, 7)$ & 7 & 1 & YES & YES & YES & NO & 1558\\
$(84, 25)$ & 10 & $(37, 11)$ & 8 & 1 & YES & YES & NO(2) & NO & 1559\\
$(85, 24)$ & 11 & $(2, 1)$ & 1 & 1 & YES & YES & YES & -- & 1560\\
$(85, 24)$ & 11 & $(2, 1)$ & 1 & 1 & YES & YES & YES & NO & 1561\\
$(85, 24)$ & 11 & $(5, 1)$ & 4 & 5 & YES & YES & YES & -- & 1562\\
$(85, 24)$ & 11 & $(5, 1)$ & 4 & 5 & YES & YES & YES & NO & 1563\\
$(85, 24)$ & 11 & $(7, 2)$ & 4 & 1 & YES & YES & YES & 1370 & 1564\\
$(85, 26)$ & 10 & $(7, 3)$ & 4 & 1 & YES & YES & NO(2) & -- & 1565\\
$(85, 33)$ & 10 & $(7, 3)$ & 4 & 1 & YES & YES & NO(2) & -- & 1566\\
$(85, 38)$ & 11 & $(7, 2)$ & 4 & 1 & YES & YES & YES & -- & 1567\\
$(85, 24)$ & 11 & $(39, 11)$ & 9 & 1 & YES & YES & NO(2) & NO & 1568\\
$(86, 27)$ & 11 & $(2, 1)$ & 1 & 2 & YES & YES & YES & -- & 1569\\
$(86, 27)$ & 11 & $(3, 1)$ & 2 & 1 & YES & YES & YES & -- & 1570\\
$(86, 27)$ & 11 & $(3, 1)$ & 2 & 1 & YES & YES & YES & NO & 1571\\
$(86, 35)$ & 11 & $(5, 2)$ & 3 & 1 & YES & YES & YES & -- & 1572\\
$(87, 23)$ & 10 & $(4, 1)$ & 3 & 1 & YES & YES & NO(2) & -- & 1573\\
$(87, 23)$ & 10 & $(4, 1)$ & 3 & 1 & YES & YES & NO(2) & NO & 1574\\
$(87, 37)$ & 11 & $(5, 2)$ & 3 & 1 & YES & YES & NO(2) & -- & 1575\\
$(87, 23)$ & 10 & $(7, 2)$ & 4 & 1 & YES & YES & NO(2) & 1360 & 1576\\
$(87, 31)$ & 12 & $(7, 1)$ & 6 & 1 & YES & YES & NO(2) & 2064 & 1577\\
$(87, 37)$ & 11 & $(7, 2)$ & 4 & 1 & YES & YES & NO(2) & NO & 1578\\
$(87, 23)$ & 10 & $(9, 4)$ & 5 & 3 & YES & YES & YES & -- & 1579\\
$(87, 20)$ & 12 & $(10, 3)$ & 5 & 1 & YES & YES & NO(2) & NO & 1580\\
$(87, 19)$ & 10 & $(11, 4)$ & 5 & 1 & YES & YES & NO(2) & NO & 1581\\
$(87, 23)$ & 10 & $(11, 3)$ & 5 & 1 & YES & YES & NO(2) & NO & 1582\\
$(87, 19)$ & 10 & $(13, 4)$ & 6 & 1 & YES & YES & NO(2) & NO & 1583\\
$(87, 37)$ & 11 & $(13, 2)$ & 7 & 1 & YES & YES & NO(2) & NO & 1584\\
$(87, 32)$ & 10 & $(17, 6)$ & 7 & 1 & YES & YES & YES & NO & 1585\\
$(87, 37)$ & 11 & $(17, 7)$ & 6 & 1 & YES & YES & NO(2) & NO & 1586\\
$(87, 23)$ & 10 & $(53, 14)$ & 9 & 1 & YES & YES & NO(2) & NO & 1587\\
$(87, 31)$ & 12 & $(59, 21)$ & 10 & 1 & YES & YES & NO(2) & 1854 & 1588\\
$(87, 37)$ & 11 & $(59, 25)$ & 9 & 1 & YES & YES & NO(2) & NO & 1589\\
$(87, 37)$ & 11 & $(73, 31)$ & 10 & 1 & YES & YES & NO(2) & NO & 1590\\
$(89, 28)$ & 11 & $(3, 1)$ & 2 & 1 & YES & YES & YES & -- & 1591\\
$(89, 28)$ & 11 & $(3, 1)$ & 2 & 1 & YES & YES & YES & NO & 1592\\
$(89, 35)$ & 11 & $(3, 1)$ & 2 & 1 & YES & YES & YES & -- & 1593\\
$(89, 27)$ & 10 & $(5, 2)$ & 3 & 1 & YES & YES & NO(2) & -- & 1594\\
$(89, 27)$ & 10 & $(5, 2)$ & 3 & 1 & YES & YES & NO(2) & NO & 1595\\
$(89, 34)$ & 9 & $(5, 2)$ & 3 & 1 & YES & YES & YES & -- & 1596\\
$(89, 26)$ & 10 & $(7, 3)$ & 4 & 1 & YES & YES & NO(2) & -- & 1597\\
$(89, 26)$ & 10 & $(7, 3)$ & 4 & 1 & YES & YES & YES & NO & 1598\\
$(89, 20)$ & 11 & $(11, 4)$ & 5 & 1 & YES & YES & NO(2) & -- & 1599\\
$(89, 34)$ & 9 & $(11, 4)$ & 5 & 1 & YES & YES & YES & NO & 1600\\
$(89, 26)$ & 10 & $(12, 5)$ & 5 & 1 & YES & YES & YES & -- & 1601\\
$(89, 20)$ & 11 & $(15, 4)$ & 6 & 1 & YES & YES & YES & NO & 1602\\
$(89, 34)$ & 9 & $(28, 11)$ & 8 & 1 & YES & YES & YES & NO & 1603\\
$(90, 19)$ & 11 & $(3, 1)$ & 2 & 3 & YES & YES & YES & -- & 1604\\
$(90, 19)$ & 11 & $(3, 1)$ & 2 & 3 & YES & YES & YES & NO & 1605\\
$(90, 19)$ & 11 & $(24, 5)$ & 8 & 6 & YES & YES & YES & NO & 1606\\
$(91, 25)$ & 10 & $(2, 1)$ & 1 & 1 & YES & YES & NO(2) & -- & 1607\\
$(91, 25)$ & 10 & $(3, 1)$ & 2 & 1 & YES & YES & NO(2) & -- & 1608\\
$(91, 41)$ & 11 & $(3, 1)$ & 2 & 1 & YES & YES & YES & NO & 1609\\
$(91, 24)$ & 11 & $(4, 1)$ & 3 & 1 & YES & YES & YES & -- & 1610\\
$(91, 25)$ & 10 & $(4, 1)$ & 3 & 1 & YES & YES & YES & -- & 1611\\
$(91, 25)$ & 10 & $(4, 1)$ & 3 & 1 & YES & YES & YES & NO & 1612\\
$(91, 25)$ & 10 & $(4, 1)$ & 3 & 1 & YES & YES & YES & NO & 1613\\
$(91, 24)$ & 11 & $(5, 1)$ & 4 & 1 & YES & YES & YES & -- & 1614\\
$(91, 24)$ & 11 & $(5, 1)$ & 4 & 1 & YES & YES & YES & NO & 1615\\
$(91, 24)$ & 11 & $(7, 2)$ & 4 & 7 & YES & YES & NO(2) & -- & 1616\\
$(91, 24)$ & 11 & $(9, 2)$ & 5 & 1 & YES & YES & YES & -- & 1617\\
$(91, 27)$ & 10 & $(9, 4)$ & 5 & 1 & YES & YES & YES & -- & 1618\\
$(91, 24)$ & 11 & $(13, 4)$ & 6 & 13 & YES & YES & YES & NO & 1619\\
$(91, 24)$ & 11 & $(14, 3)$ & 6 & 7 & YES & YES & NO(2) & NO & 1620\\
$(91, 25)$ & 10 & $(40, 11)$ & 8 & 1 & YES & YES & NO(2) & NO & 1621\\
$(91, 25)$ & 10 & $(51, 14)$ & 9 & 1 & YES & YES & NO(2) & NO & 1622\\
$(91, 24)$ & 11 & $(72, 19)$ & 10 & 1 & YES & YES & YES & NO & 1623\\
$(91, 24)$ & 11 & $(87, 23)$ & 10 & 1 & YES & YES & NO(2) & 1981 & 1624\\
$(92, 33)$ & 10 & $(3, 1)$ & 2 & 1 & YES & YES & NO(2) & -- & 1625\\
$(92, 39)$ & 10 & $(5, 2)$ & 3 & 1 & YES & YES & NO(2) & -- & 1626\\
$(92, 33)$ & 10 & $(36, 13)$ & 8 & 4 & YES & YES & NO(2) & NO & 1627\\
$(92, 33)$ & 10 & $(64, 23)$ & 9 & 4 & YES & YES & NO(2) & NO & 1628\\
$(93, 34)$ & 10 & $(3, 1)$ & 2 & 3 & YES & YES & NO(2) & -- & 1629\\
$(93, 26)$ & 10 & $(4, 1)$ & 3 & 1 & YES & YES & YES & -- & 1630\\
$(93, 26)$ & 10 & $(4, 1)$ & 3 & 1 & YES & YES & YES & NO & 1631\\
$(93, 26)$ & 10 & $(4, 1)$ & 3 & 1 & YES & YES & YES & NO & 1632\\
$(93, 29)$ & 12 & $(7, 2)$ & 4 & 1 & YES & YES & YES & NO & 1633\\
$(93, 22)$ & 11 & $(9, 4)$ & 5 & 3 & YES & YES & YES & -- & 1634\\
$(93, 22)$ & 11 & $(9, 4)$ & 5 & 3 & YES & YES & YES & NO & 1635\\
$(93, 26)$ & 10 & $(9, 4)$ & 5 & 3 & YES & YES & NO(2) & NO & 1636\\
$(93, 29)$ & 12 & $(10, 3)$ & 5 & 1 & YES & YES & YES & NO & 1637\\
$(93, 34)$ & 10 & $(10, 3)$ & 5 & 1 & YES & YES & YES & -- & 1638\\
$(93, 34)$ & 10 & $(52, 19)$ & 9 & 1 & YES & YES & NO(2) & NO & 1639\\
$(93, 34)$ & 10 & $(79, 29)$ & 9 & 1 & YES & YES & YES & NO & 1640\\
$(94, 43)$ & 11 & $(3, 1)$ & 2 & 1 & YES & YES & YES & -- & 1641\\
$(94, 43)$ & 11 & $(3, 1)$ & 2 & 1 & YES & YES & YES & NO & 1642\\
$(95, 44)$ & 12 & $(2, 1)$ & 1 & 1 & YES & YES & NO(2) & -- & 1643\\
$(95, 42)$ & 11 & $(3, 1)$ & 2 & 1 & YES & YES & YES & -- & 1644\\
$(95, 42)$ & 11 & $(4, 1)$ & 3 & 1 & YES & YES & YES & NO & 1645\\
$(95, 43)$ & 11 & $(4, 1)$ & 3 & 1 & YES & YES & YES & NO & 1646\\
$(95, 42)$ & 11 & $(5, 2)$ & 3 & 5 & YES & YES & NO(2) & -- & 1647\\
$(95, 42)$ & 11 & $(5, 2)$ & 3 & 5 & YES & YES & YES & NO & 1648\\
$(95, 36)$ & 10 & $(11, 4)$ & 5 & 1 & YES & YES & YES & NO & 1649\\
$(95, 36)$ & 10 & $(18, 7)$ & 6 & 1 & YES & YES & NO(2) & NO & 1650\\
$(95, 43)$ & 11 & $(20, 9)$ & 7 & 5 & YES & YES & YES & 1372 & 1651\\
$(95, 43)$ & 11 & $(42, 19)$ & 9 & 1 & YES & YES & YES & NO & 1652\\
$(95, 42)$ & 11 & $(52, 23)$ & 10 & 1 & YES & YES & YES & NO & 1653\\
$(95, 43)$ & 11 & $(95, 43)$ & 11 & 95 & YES & YES & YES & NO & 1654\\
$(96, 17)$ & 12 & $(5, 2)$ & 3 & 1 & YES & YES & NO(2) & -- & 1655\\
$(96, 17)$ & 12 & $(5, 2)$ & 3 & 1 & YES & YES & NO(2) & NO & 1656\\
$(96, 17)$ & 12 & $(5, 2)$ & 3 & 1 & YES & YES & NO(2) & NO & 1657\\
$(96, 17)$ & 12 & $(13, 2)$ & 7 & 1 & YES & YES & YES & NO & 1658\\
$(97, 18)$ & 11 & $(2, 1)$ & 1 & 1 & YES & YES & NO(2) & -- & 1659\\
$(97, 21)$ & 10 & $(3, 1)$ & 2 & 1 & YES & YES & NO(2) & -- & 1660\\
$(97, 21)$ & 10 & $(3, 1)$ & 2 & 1 & YES & YES & NO(2) & NO & 1661\\
$(97, 26)$ & 10 & $(5, 2)$ & 3 & 1 & YES & YES & NO(2) & -- & 1662\\
$(97, 28)$ & 12 & $(7, 2)$ & 4 & 1 & YES & YES & NO(2) & -- & 1663\\
$(97, 21)$ & 10 & $(14, 3)$ & 6 & 1 & YES & YES & NO(2) & NO & 1664\\
$(97, 30)$ & 11 & $(36, 11)$ & 8 & 1 & YES & YES & YES & NO & 1665\\
$(97, 28)$ & 12 & $(69, 20)$ & 10 & 1 & YES & YES & NO(2) & NO & 1666\\
$(98, 15)$ & 14 & $(2, 1)$ & 1 & 2 & YES & YES & YES & -- & 1667\\
$(98, 15)$ & 14 & $(2, 1)$ & 1 & 2 & YES & YES & YES & NO & 1668\\
$(98, 37)$ & 11 & $(7, 2)$ & 4 & 7 & YES & YES & NO(2) & NO & 1669\\
$(98, 43)$ & 10 & $(7, 2)$ & 4 & 7 & YES & YES & YES & -- & 1670\\
$(98, 43)$ & 10 & $(8, 3)$ & 4 & 2 & YES & YES & YES & NO & 1671\\
$(98, 27)$ & 10 & $(9, 4)$ & 5 & 1 & YES & YES & YES & -- & 1672\\
$(98, 31)$ & 13 & $(16, 5)$ & 7 & 2 & YES & YES & YES & NO & 1673\\
$(98, 27)$ & 10 & $(39, 11)$ & 9 & 1 & YES & YES & YES & NO & 1674\\
$(98, 37)$ & 11 & $(66, 25)$ & 9 & 2 & YES & YES & NO(2) & NO & 1675\\
$(98, 43)$ & 10 & $(66, 29)$ & 9 & 2 & YES & YES & YES & NO & 1676\\
$(99, 38)$ & 12 & $(5, 1)$ & 4 & 1 & YES & YES & NO(2) & -- & 1677\\
$(99, 38)$ & 12 & $(7, 1)$ & 6 & 1 & YES & YES & NO(2) & NO & 1678\\
$(99, 38)$ & 12 & $(47, 18)$ & 8 & 1 & YES & YES & NO(2) & NO & 1679\\
$(99, 38)$ & 12 & $(73, 28)$ & 10 & 1 & YES & YES & NO(2) & 1961 & 1680\\
$(100, 29)$ & 11 & $(4, 1)$ & 3 & 4 & YES & YES & YES & -- & 1681\\
$(100, 29)$ & 11 & $(4, 1)$ & 3 & 4 & YES & YES & YES & NO & 1682\\
$(100, 29)$ & 11 & $(4, 1)$ & 3 & 4 & YES & YES & YES & NO & 1683\\
$(100, 37)$ & 10 & $(7, 3)$ & 4 & 1 & YES & YES & YES & NO & 1684\\
$(100, 37)$ & 10 & $(13, 5)$ & 5 & 1 & YES & YES & YES & NO & 1685\\
$(100, 27)$ & 10 & $(25, 7)$ & 7 & 25 & YES & YES & NO(2) & NO & 1686\\
$(100, 29)$ & 11 & $(52, 15)$ & 11 & 4 & YES & YES & NO(2) & NO & 1687\\
$(100, 41)$ & 10 & $(83, 34)$ & 10 & 1 & YES & YES & NO(2) & NO & 1688\\
$(101, 24)$ & 12 & $(3, 1)$ & 2 & 1 & YES & YES & YES & NO & 1689\\
$(101, 16)$ & 13 & $(7, 2)$ & 4 & 1 & YES & YES & NO(2) & -- & 1690\\
$(101, 41)$ & 12 & $(12, 5)$ & 5 & 1 & YES & YES & YES & NO & 1691\\
$(101, 30)$ & 10 & $(23, 7)$ & 7 & 1 & YES & YES & NO(2) & NO & 1692\\
$(103, 32)$ & 11 & $(3, 1)$ & 2 & 1 & YES & YES & NO(2) & -- & 1693\\
$(103, 32)$ & 11 & $(3, 1)$ & 2 & 1 & YES & YES & NO(2) & NO & 1694\\
$(103, 47)$ & 12 & $(4, 1)$ & 3 & 1 & YES & YES & YES & NO & 1695\\
$(103, 40)$ & 11 & $(5, 2)$ & 3 & 1 & YES & YES & NO(2) & -- & 1696\\
$(103, 37)$ & 10 & $(7, 3)$ & 4 & 1 & YES & YES & NO(2) & NO & 1697\\
$(103, 39)$ & 10 & $(7, 2)$ & 4 & 1 & YES & YES & NO(2) & NO & 1698\\
$(103, 29)$ & 11 & $(17, 5)$ & 6 & 1 & YES & YES & YES & NO & 1699\\
$(103, 39)$ & 10 & $(34, 13)$ & 7 & 1 & YES & YES & NO(2) & 2003 & 1700\\
$(103, 40)$ & 11 & $(44, 17)$ & 8 & 1 & YES & YES & NO(2) & NO & 1701\\
$(103, 47)$ & 12 & $(103, 47)$ & 12 & 103 & YES & YES & YES & NO & 1702\\
$(104, 27)$ & 12 & $(3, 1)$ & 2 & 1 & YES & YES & NO(2) & -- & 1703\\
$(104, 41)$ & 12 & $(3, 1)$ & 2 & 1 & YES & YES & YES & -- & 1704\\
$(104, 47)$ & 11 & $(3, 1)$ & 2 & 1 & YES & YES & NO(2) & -- & 1705\\
$(104, 45)$ & 11 & $(5, 2)$ & 3 & 1 & YES & YES & YES & -- & 1706\\
$(104, 31)$ & 11 & $(7, 2)$ & 4 & 1 & YES & YES & NO(2) & NO & 1707\\
$(104, 45)$ & 11 & $(11, 5)$ & 6 & 1 & YES & YES & YES & NO & 1708\\
$(104, 47)$ & 11 & $(11, 5)$ & 6 & 1 & YES & YES & YES & NO & 1709\\
$(104, 47)$ & 11 & $(42, 19)$ & 9 & 2 & YES & YES & YES & 1784 & 1710\\
$(105, 41)$ & 10 & $(3, 1)$ & 2 & 3 & YES & YES & YES & NO & 1711\\
$(105, 38)$ & 11 & $(5, 2)$ & 3 & 5 & YES & YES & NO(2) & -- & 1712\\
$(105, 46)$ & 12 & $(5, 1)$ & 4 & 5 & YES & YES & YES & -- & 1713\\
$(105, 46)$ & 12 & $(5, 1)$ & 4 & 5 & YES & YES & NO(2) & NO & 1714\\
$(105, 31)$ & 10 & $(13, 4)$ & 6 & 1 & YES & YES & YES & -- & 1715\\
$(105, 41)$ & 10 & $(28, 11)$ & 8 & 7 & YES & YES & NO(2) & NO & 1716\\
$(105, 46)$ & 12 & $(73, 32)$ & 10 & 1 & YES & YES & YES & 1975 & 1717\\
$(106, 37)$ & 12 & $(2, 1)$ & 1 & 2 & YES & YES & NO(2) & NO & 1718\\
$(106, 45)$ & 11 & $(5, 2)$ & 3 & 1 & YES & YES & YES & -- & 1719\\
$(106, 41)$ & 10 & $(10, 3)$ & 5 & 2 & YES & YES & YES & -- & 1720\\
$(106, 41)$ & 10 & $(10, 3)$ & 5 & 2 & YES & YES & YES & NO & 1721\\
$(106, 41)$ & 10 & $(11, 3)$ & 5 & 1 & YES & YES & YES & -- & 1722\\
$(106, 37)$ & 12 & $(23, 8)$ & 9 & 1 & YES & YES & YES & NO & 1723\\
$(107, 25)$ & 11 & $(4, 1)$ & 3 & 1 & YES & YES & YES & -- & 1724\\
$(107, 25)$ & 11 & $(4, 1)$ & 3 & 1 & YES & YES & YES & NO & 1725\\
$(107, 47)$ & 10 & $(5, 2)$ & 3 & 1 & YES & YES & NO(2) & -- & 1726\\
$(107, 41)$ & 10 & $(8, 3)$ & 4 & 1 & YES & YES & YES & -- & 1727\\
$(107, 41)$ & 10 & $(11, 3)$ & 5 & 1 & YES & YES & YES & -- & 1728\\
$(107, 20)$ & 13 & $(13, 3)$ & 6 & 1 & YES & YES & NO(2) & NO & 1729\\
$(107, 47)$ & 10 & $(23, 10)$ & 7 & 1 & YES & YES & YES & NO & 1730\\
$(107, 41)$ & 10 & $(50, 19)$ & 8 & 1 & YES & YES & YES & NO & 1731\\
$(107, 47)$ & 10 & $(57, 25)$ & 9 & 1 & YES & YES & YES & NO & 1732\\
$(107, 41)$ & 10 & $(76, 29)$ & 9 & 1 & YES & YES & YES & 1852 & 1733\\
$(108, 41)$ & 10 & $(4, 1)$ & 3 & 4 & YES & YES & YES & -- & 1734\\
$(108, 41)$ & 10 & $(5, 2)$ & 3 & 1 & YES & YES & YES & NO & 1735\\
$(109, 30)$ & 10 & $(3, 1)$ & 2 & 1 & YES & YES & NO(2) & -- & 1736\\
$(109, 45)$ & 10 & $(3, 1)$ & 2 & 1 & YES & YES & YES & -- & 1737\\
$(109, 45)$ & 10 & $(7, 3)$ & 4 & 1 & YES & YES & YES & NO & 1738\\
$(109, 46)$ & 10 & $(10, 3)$ & 5 & 1 & YES & YES & YES & -- & 1739\\
$(109, 30)$ & 10 & $(13, 4)$ & 6 & 1 & YES & YES & NO(2) & NO & 1740\\
$(109, 50)$ & 12 & $(24, 11)$ & 8 & 1 & YES & YES & YES & NO & 1741\\
$(109, 45)$ & 10 & $(26, 11)$ & 7 & 1 & YES & YES & YES & NO & 1742\\
$(109, 46)$ & 10 & $(59, 25)$ & 9 & 1 & YES & YES & YES & NO & 1743\\
$(109, 50)$ & 12 & $(109, 50)$ & 12 & 109 & YES & YES & YES & NO & 1744\\
$(110, 29)$ & 12 & $(4, 1)$ & 3 & 2 & YES & YES & YES & -- & 1745\\
$(110, 43)$ & 11 & $(6, 1)$ & 5 & 2 & YES & YES & YES & NO & 1746\\
$(110, 29)$ & 12 & $(91, 24)$ & 11 & 1 & YES & YES & YES & NO & 1747\\
$(110, 43)$ & 11 & $(110, 43)$ & 11 & 110 & YES & YES & YES & NO & 1748\\
$(111, 34)$ & 11 & $(3, 1)$ & 2 & 3 & NO & YES & YES & -- & 1749\\
$(111, 46)$ & 10 & $(3, 1)$ & 2 & 3 & YES & YES & YES & -- & 1750\\
$(111, 46)$ & 10 & $(3, 1)$ & 2 & 3 & YES & YES & YES & NO & 1751\\
$(111, 32)$ & 13 & $(4, 1)$ & 3 & 1 & YES & YES & YES & NO & 1752\\
$(111, 46)$ & 10 & $(4, 1)$ & 3 & 1 & YES & YES & NO(2) & -- & 1753\\
$(111, 29)$ & 12 & $(5, 2)$ & 3 & 1 & YES & YES & NO(2) & -- & 1754\\
$(111, 29)$ & 12 & $(5, 2)$ & 3 & 1 & YES & YES & NO(2) & NO & 1755\\
$(111, 46)$ & 10 & $(5, 2)$ & 3 & 1 & YES & YES & NO(2) & NO & 1756\\
$(111, 29)$ & 12 & $(10, 3)$ & 5 & 1 & YES & YES & NO(2) & NO & 1757\\
$(111, 29)$ & 12 & $(11, 2)$ & 6 & 1 & YES & YES & NO(2) & -- & 1758\\
$(111, 29)$ & 12 & $(34, 9)$ & 8 & 1 & YES & YES & NO(2) & NO & 1759\\
$(112, 41)$ & 10 & $(3, 1)$ & 2 & 1 & YES & YES & YES & -- & 1760\\
$(112, 41)$ & 10 & $(19, 7)$ & 6 & 1 & YES & YES & YES & 1842 & 1761\\
$(113, 32)$ & 13 & $(2, 1)$ & 1 & 1 & YES & YES & YES & -- & 1762\\
$(113, 32)$ & 13 & $(2, 1)$ & 1 & 1 & YES & YES & YES & NO & 1763\\
$(113, 35)$ & 11 & $(2, 1)$ & 1 & 1 & YES & YES & NO(2) & NO & 1764\\
$(113, 42)$ & 11 & $(2, 1)$ & 1 & 1 & YES & YES & YES & -- & 1765\\
$(113, 35)$ & 11 & $(3, 1)$ & 2 & 1 & YES & YES & NO(2) & -- & 1766\\
$(113, 35)$ & 11 & $(3, 1)$ & 2 & 1 & YES & YES & NO(2) & NO & 1767\\
$(113, 48)$ & 11 & $(3, 1)$ & 2 & 1 & YES & YES & NO(2) & -- & 1768\\
$(113, 48)$ & 11 & $(4, 1)$ & 3 & 1 & YES & YES & NO(2) & -- & 1769\\
$(113, 24)$ & 11 & $(5, 1)$ & 4 & 1 & YES & YES & NO(2) & -- & 1770\\
$(113, 24)$ & 11 & $(5, 1)$ & 4 & 1 & YES & YES & NO(2) & NO & 1771\\
$(113, 42)$ & 11 & $(13, 5)$ & 5 & 1 & YES & YES & YES & NO & 1772\\
$(113, 35)$ & 11 & $(16, 5)$ & 7 & 1 & YES & YES & NO(2) & 1799 & 1773\\
$(113, 30)$ & 11 & $(53, 14)$ & 9 & 1 & YES & YES & NO(2) & NO & 1774\\
$(113, 32)$ & 13 & $(53, 15)$ & 11 & 1 & YES & YES & YES & NO & 1775\\
$(113, 48)$ & 11 & $(73, 31)$ & 10 & 1 & YES & YES & NO(2) & NO & 1776\\
$(113, 48)$ & 11 & $(113, 48)$ & 11 & 113 & YES & YES & NO(2) & NO & 1777\\
$(114, 53)$ & 12 & $(2, 1)$ & 1 & 2 & YES & YES & YES & -- & 1778\\
$(115, 18)$ & 12 & $(6, 1)$ & 5 & 1 & NO & YES & YES & -- & 1779\\
$(115, 44)$ & 10 & $(8, 3)$ & 4 & 1 & YES & YES & YES & -- & 1780\\
$(115, 52)$ & 11 & $(9, 4)$ & 5 & 1 & YES & YES & YES & NO & 1781\\
$(115, 26)$ & 11 & $(11, 4)$ & 5 & 1 & YES & YES & YES & -- & 1782\\
$(115, 44)$ & 10 & $(11, 4)$ & 5 & 1 & YES & YES & NO(2) & NO & 1783\\
$(115, 52)$ & 11 & $(31, 14)$ & 8 & 1 & YES & YES & YES & 1710 & 1784\\
$(116, 51)$ & 11 & $(3, 1)$ & 2 & 1 & YES & YES & YES & -- & 1785\\
$(116, 51)$ & 11 & $(3, 1)$ & 2 & 1 & YES & YES & YES & NO & 1786\\
$(116, 45)$ & 10 & $(8, 3)$ & 4 & 4 & YES & YES & YES & -- & 1787\\
$(117, 31)$ & 11 & $(10, 3)$ & 5 & 1 & YES & YES & YES & -- & 1788\\
$(117, 31)$ & 11 & $(25, 7)$ & 7 & 1 & YES & YES & YES & NO & 1789\\
$(117, 43)$ & 10 & $(109, 40)$ & 10 & 1 & YES & YES & YES & NO & 1790\\
$(118, 49)$ & 11 & $(5, 2)$ & 3 & 1 & YES & YES & NO(2) & -- & 1791\\
$(118, 49)$ & 11 & $(5, 2)$ & 3 & 1 & YES & YES & NO(2) & NO & 1792\\
$(119, 37)$ & 11 & $(2, 1)$ & 1 & 1 & YES & YES & NO(2) & -- & 1793\\
$(119, 37)$ & 11 & $(2, 1)$ & 1 & 1 & YES & YES & NO(2) & NO & 1794\\
$(119, 46)$ & 10 & $(4, 1)$ & 3 & 1 & YES & YES & YES & -- & 1795\\
$(119, 46)$ & 10 & $(5, 2)$ & 3 & 1 & YES & YES & NO(2) & NO & 1796\\
$(119, 45)$ & 11 & $(8, 3)$ & 4 & 1 & YES & YES & YES & NO & 1797\\
$(119, 46)$ & 10 & $(8, 3)$ & 4 & 1 & YES & YES & YES & -- & 1798\\
$(119, 37)$ & 11 & $(13, 4)$ & 6 & 1 & YES & YES & NO(2) & 1773 & 1799\\
$(119, 46)$ & 10 & $(13, 3)$ & 6 & 1 & YES & YES & YES & -- & 1800\\
$(119, 45)$ & 11 & $(82, 31)$ & 10 & 1 & YES & YES & NO(2) & NO & 1801\\
$(120, 43)$ & 11 & $(2, 1)$ & 1 & 2 & YES & YES & NO(2) & -- & 1802\\
$(120, 43)$ & 11 & $(3, 1)$ & 2 & 3 & YES & YES & YES & NO & 1803\\
$(120, 53)$ & 11 & $(5, 2)$ & 3 & 5 & YES & YES & NO(2) & NO & 1804\\
$(120, 49)$ & 11 & $(9, 4)$ & 5 & 3 & YES & YES & YES & NO & 1805\\
$(121, 35)$ & 12 & $(2, 1)$ & 1 & 1 & YES & YES & YES & -- & 1806\\
$(121, 35)$ & 12 & $(4, 1)$ & 3 & 1 & YES & YES & NO(2) & -- & 1807\\
$(121, 46)$ & 10 & $(7, 3)$ & 4 & 1 & YES & YES & YES & -- & 1808\\
$(121, 46)$ & 10 & $(8, 3)$ & 4 & 1 & YES & YES & YES & -- & 1809\\
$(121, 36)$ & 11 & $(10, 3)$ & 5 & 1 & YES & YES & YES & -- & 1810\\
$(121, 32)$ & 11 & $(19, 5)$ & 7 & 1 & YES & YES & YES & NO & 1811\\
$(121, 36)$ & 11 & $(71, 21)$ & 9 & 1 & YES & YES & YES & NO & 1812\\
$(122, 51)$ & 11 & $(2, 1)$ & 1 & 2 & YES & YES & YES & NO & 1813\\
$(122, 37)$ & 11 & $(8, 3)$ & 4 & 2 & YES & YES & YES & -- & 1814\\
$(122, 37)$ & 11 & $(18, 5)$ & 6 & 2 & YES & YES & YES & NO & 1815\\
$(123, 47)$ & 10 & $(18, 7)$ & 6 & 3 & YES & YES & NO(2) & NO & 1816\\
$(124, 57)$ & 12 & $(2, 1)$ & 1 & 2 & YES & YES & YES & -- & 1817\\
$(124, 37)$ & 12 & $(3, 1)$ & 2 & 1 & YES & YES & YES & NO & 1818\\
$(124, 57)$ & 12 & $(3, 1)$ & 2 & 1 & YES & YES & NO(2) & -- & 1819\\
$(124, 37)$ & 12 & $(5, 2)$ & 3 & 1 & YES & YES & NO(2) & NO & 1820\\
$(124, 57)$ & 12 & $(24, 11)$ & 8 & 4 & YES & YES & YES & NO & 1821\\
$(125, 44)$ & 12 & $(3, 1)$ & 2 & 1 & YES & YES & NO(2) & NO & 1822\\
$(125, 49)$ & 11 & $(4, 1)$ & 3 & 1 & YES & YES & NO(2) & NO & 1823\\
$(125, 49)$ & 11 & $(5, 2)$ & 3 & 5 & YES & YES & YES & 1491 & 1824\\
$(125, 26)$ & 13 & $(6, 1)$ & 5 & 1 & YES & YES & YES & NO & 1825\\
$(125, 37)$ & 11 & $(8, 3)$ & 4 & 1 & YES & YES & YES & -- & 1826\\
$(125, 49)$ & 11 & $(8, 3)$ & 4 & 1 & YES & YES & NO(2) & NO & 1827\\
$(125, 27)$ & 11 & $(9, 4)$ & 5 & 1 & YES & YES & YES & -- & 1828\\
$(125, 33)$ & 11 & $(9, 2)$ & 5 & 1 & YES & YES & YES & -- & 1829\\
$(125, 37)$ & 11 & $(14, 3)$ & 6 & 1 & YES & YES & YES & NO & 1830\\
$(125, 26)$ & 13 & $(29, 6)$ & 9 & 1 & YES & YES & YES & NO & 1831\\
$(125, 33)$ & 11 & $(91, 24)$ & 11 & 1 & YES & YES & YES & NO & 1832\\
$(125, 37)$ & 11 & $(105, 31)$ & 10 & 5 & YES & YES & YES & 2269 & 1833\\
$(126, 55)$ & 11 & $(2, 1)$ & 1 & 2 & YES & YES & YES & NO & 1834\\
$(127, 54)$ & 12 & $(2, 1)$ & 1 & 1 & YES & YES & NO(2) & -- & 1835\\
$(127, 54)$ & 12 & $(3, 1)$ & 2 & 1 & YES & YES & YES & -- & 1836\\
$(127, 56)$ & 11 & $(5, 2)$ & 3 & 1 & YES & YES & YES & -- & 1837\\
$(127, 54)$ & 12 & $(33, 14)$ & 8 & 1 & YES & YES & NO(2) & NO & 1838\\
$(127, 56)$ & 11 & $(41, 18)$ & 8 & 1 & YES & YES & YES & 2133 & 1839\\
$(128, 37)$ & 12 & $(2, 1)$ & 1 & 2 & YES & YES & YES & -- & 1840\\
$(128, 47)$ & 10 & $(8, 3)$ & 4 & 8 & YES & YES & YES & -- & 1841\\
$(128, 47)$ & 10 & $(11, 4)$ & 5 & 1 & YES & YES & YES & 1761 & 1842\\
$(128, 47)$ & 10 & $(18, 7)$ & 6 & 2 & YES & YES & YES & NO & 1843\\
$(128, 45)$ & 12 & $(20, 7)$ & 8 & 4 & YES & YES & YES & NO & 1844\\
$(129, 49)$ & 10 & $(2, 1)$ & 1 & 1 & YES & YES & NO(2) & -- & 1845\\
$(129, 59)$ & 12 & $(13, 6)$ & 7 & 1 & YES & YES & YES & NO & 1846\\
$(129, 49)$ & 10 & $(23, 9)$ & 7 & 1 & YES & YES & YES & 1512 & 1847\\
$(130, 23)$ & 14 & $(3, 1)$ & 2 & 1 & YES & YES & NO(2) & NO & 1848\\
$(130, 51)$ & 11 & $(3, 1)$ & 2 & 1 & YES & YES & NO(2) & -- & 1849\\
$(131, 50)$ & 10 & $(7, 3)$ & 4 & 1 & YES & YES & YES & -- & 1850\\
$(131, 50)$ & 10 & $(13, 3)$ & 6 & 1 & YES & YES & YES & NO & 1851\\
$(131, 50)$ & 10 & $(60, 23)$ & 9 & 1 & YES & YES & YES & 1733 & 1852\\
$(132, 59)$ & 12 & $(5, 2)$ & 3 & 1 & YES & YES & YES & NO & 1853\\
$(132, 47)$ & 12 & $(14, 5)$ & 6 & 2 & YES & YES & NO(2) & 1588 & 1854\\
$(132, 59)$ & 12 & $(20, 9)$ & 7 & 4 & YES & YES & YES & NO & 1855\\
$(132, 59)$ & 12 & $(85, 38)$ & 11 & 1 & YES & YES & YES & NO & 1856\\
$(132, 59)$ & 12 & $(132, 59)$ & 12 & 132 & YES & YES & YES & NO & 1857\\
$(134, 39)$ & 11 & $(2, 1)$ & 1 & 2 & YES & YES & NO(2) & -- & 1858\\
$(134, 39)$ & 11 & $(4, 1)$ & 3 & 2 & YES & YES & NO(2) & -- & 1859\\
$(134, 39)$ & 11 & $(4, 1)$ & 3 & 2 & YES & YES & NO(2) & NO & 1860\\
$(134, 39)$ & 11 & $(8, 3)$ & 4 & 2 & YES & YES & YES & -- & 1861\\
$(134, 49)$ & 11 & $(52, 19)$ & 9 & 2 & YES & YES & YES & 1905 & 1862\\
$(135, 26)$ & 14 & $(4, 1)$ & 3 & 1 & YES & YES & YES & NO & 1863\\
$(135, 32)$ & 12 & $(4, 1)$ & 3 & 1 & NO & YES & YES & -- & 1864\\
$(135, 32)$ & 12 & $(38, 9)$ & 9 & 1 & YES & YES & NO(2) & NO & 1865\\
$(137, 43)$ & 12 & $(3, 1)$ & 2 & 1 & NO & YES & NO(2) & -- & 1866\\
$(137, 43)$ & 12 & $(3, 1)$ & 2 & 1 & YES & YES & NO(2) & NO & 1867\\
$(137, 51)$ & 12 & $(3, 1)$ & 2 & 1 & YES & YES & NO(2) & -- & 1868\\
$(137, 63)$ & 12 & $(24, 11)$ & 8 & 1 & YES & YES & NO(2) & 1962 & 1869\\
$(138, 49)$ & 12 & $(3, 1)$ & 2 & 3 & YES & YES & YES & -- & 1870\\
$(138, 61)$ & 12 & $(5, 1)$ & 4 & 1 & YES & YES & NO(2) & -- & 1871\\
$(138, 61)$ & 12 & $(5, 1)$ & 4 & 1 & YES & YES & NO(2) & NO & 1872\\
$(138, 61)$ & 12 & $(5, 1)$ & 4 & 1 & YES & YES & NO(2) & NO & 1873\\
$(138, 31)$ & 12 & $(7, 3)$ & 4 & 1 & YES & YES & NO(2) & -- & 1874\\
$(138, 61)$ & 12 & $(25, 11)$ & 7 & 1 & YES & YES & NO(2) & NO & 1875\\
$(138, 61)$ & 12 & $(95, 42)$ & 11 & 1 & YES & YES & NO(2) & NO & 1876\\
$(138, 61)$ & 12 & $(138, 61)$ & 12 & 138 & YES & YES & NO(2) & NO & 1877\\
$(139, 39)$ & 11 & $(2, 1)$ & 1 & 1 & YES & YES & YES & -- & 1878\\
$(139, 61)$ & 11 & $(2, 1)$ & 1 & 1 & YES & YES & NO(2) & NO & 1879\\
$(139, 61)$ & 11 & $(5, 2)$ & 3 & 1 & YES & YES & YES & -- & 1880\\
$(139, 61)$ & 11 & $(12, 5)$ & 5 & 1 & YES & YES & YES & NO & 1881\\
$(140, 61)$ & 11 & $(16, 7)$ & 6 & 4 & YES & YES & NO(2) & NO & 1882\\
$(142, 59)$ & 12 & $(6, 1)$ & 5 & 2 & YES & YES & NO(2) & -- & 1883\\
$(142, 59)$ & 12 & $(7, 1)$ & 6 & 1 & YES & YES & YES & -- & 1884\\
$(142, 59)$ & 12 & $(29, 12)$ & 7 & 1 & YES & YES & NO(2) & NO & 1885\\
$(143, 54)$ & 12 & $(3, 1)$ & 2 & 1 & YES & YES & YES & NO & 1886\\
$(143, 59)$ & 11 & $(3, 1)$ & 2 & 1 & YES & YES & NO(2) & -- & 1887\\
$(143, 54)$ & 12 & $(8, 3)$ & 4 & 1 & YES & YES & YES & NO & 1888\\
$(143, 40)$ & 12 & $(29, 8)$ & 7 & 1 & YES & YES & YES & NO & 1889\\
$(143, 63)$ & 11 & $(84, 37)$ & 10 & 1 & YES & YES & YES & NO & 1890\\
$(143, 59)$ & 11 & $(143, 59)$ & 11 & 143 & YES & YES & NO(2) & NO & 1891\\
$(144, 61)$ & 11 & $(2, 1)$ & 1 & 2 & YES & YES & YES & NO & 1892\\
$(144, 43)$ & 13 & $(3, 1)$ & 2 & 3 & YES & YES & NO(2) & -- & 1893\\
$(144, 59)$ & 11 & $(3, 1)$ & 2 & 3 & YES & YES & NO(2) & -- & 1894\\
$(144, 61)$ & 11 & $(3, 1)$ & 2 & 3 & YES & YES & YES & NO & 1895\\
$(144, 55)$ & 10 & $(23, 9)$ & 7 & 1 & YES & YES & YES & NO & 1896\\
$(144, 59)$ & 11 & $(144, 59)$ & 11 & 144 & YES & YES & NO(2) & NO & 1897\\
$(144, 65)$ & 12 & $(144, 65)$ & 12 & 144 & YES & YES & YES & NO & 1898\\
$(145, 41)$ & 13 & $(2, 1)$ & 1 & 1 & YES & YES & YES & NO & 1899\\
$(145, 53)$ & 11 & $(2, 1)$ & 1 & 1 & YES & YES & YES & -- & 1900\\
$(145, 41)$ & 13 & $(3, 1)$ & 2 & 1 & YES & YES & YES & NO & 1901\\
$(145, 53)$ & 11 & $(3, 1)$ & 2 & 1 & YES & YES & NO(2) & -- & 1902\\
$(145, 53)$ & 11 & $(5, 2)$ & 3 & 5 & YES & YES & NO(2) & NO & 1903\\
$(145, 51)$ & 12 & $(20, 7)$ & 8 & 5 & YES & YES & NO(2) & 1931 & 1904\\
$(145, 53)$ & 11 & $(41, 15)$ & 8 & 1 & YES & YES & YES & 1862 & 1905\\
$(145, 41)$ & 13 & $(145, 41)$ & 13 & 145 & YES & YES & NO(2) & NO & 1906\\
$(146, 61)$ & 12 & $(67, 28)$ & 10 & 1 & YES & YES & YES & NO & 1907\\
$(147, 26)$ & 15 & $(2, 1)$ & 1 & 1 & YES & YES & YES & -- & 1908\\
$(147, 26)$ & 15 & $(2, 1)$ & 1 & 1 & YES & YES & YES & NO & 1909\\
$(147, 26)$ & 15 & $(11, 2)$ & 6 & 1 & YES & YES & YES & NO & 1910\\
$(148, 65)$ & 11 & $(4, 1)$ & 3 & 4 & YES & YES & YES & -- & 1911\\
$(148, 31)$ & 12 & $(5, 2)$ & 3 & 1 & YES & YES & NO(2) & NO & 1912\\
$(148, 31)$ & 12 & $(5, 2)$ & 3 & 1 & YES & YES & NO(2) & NO & 1913\\
$(149, 42)$ & 12 & $(2, 1)$ & 1 & 1 & YES & YES & YES & NO & 1914\\
$(149, 41)$ & 11 & $(3, 1)$ & 2 & 1 & NO & YES & YES & -- & 1915\\
$(149, 46)$ & 13 & $(4, 1)$ & 3 & 1 & YES & YES & YES & -- & 1916\\
$(149, 46)$ & 13 & $(5, 1)$ & 4 & 1 & YES & YES & YES & -- & 1917\\
$(149, 40)$ & 11 & $(7, 3)$ & 4 & 1 & YES & YES & YES & -- & 1918\\
$(149, 40)$ & 11 & $(7, 3)$ & 4 & 1 & YES & YES & YES & NO & 1919\\
$(149, 41)$ & 11 & $(7, 3)$ & 4 & 1 & YES & YES & YES & -- & 1920\\
$(149, 44)$ & 11 & $(7, 3)$ & 4 & 1 & YES & YES & YES & NO & 1921\\
$(149, 41)$ & 11 & $(13, 4)$ & 6 & 1 & YES & YES & YES & NO & 1922\\
$(149, 42)$ & 12 & $(18, 5)$ & 6 & 1 & YES & YES & NO(2) & NO & 1923\\
$(149, 46)$ & 13 & $(55, 17)$ & 10 & 1 & YES & YES & YES & NO & 1924\\
$(149, 65)$ & 11 & $(55, 24)$ & 9 & 1 & YES & YES & NO(2) & NO & 1925\\
$(149, 44)$ & 11 & $(64, 19)$ & 9 & 1 & YES & YES & YES & NO & 1926\\
$(151, 53)$ & 12 & $(2, 1)$ & 1 & 1 & YES & YES & YES & -- & 1927\\
$(151, 47)$ & 12 & $(3, 1)$ & 2 & 1 & YES & YES & NO(2) & -- & 1928\\
$(151, 47)$ & 12 & $(7, 2)$ & 4 & 1 & YES & YES & YES & NO & 1929\\
$(151, 47)$ & 12 & $(10, 3)$ & 5 & 1 & YES & YES & YES & NO & 1930\\
$(151, 53)$ & 12 & $(17, 6)$ & 7 & 1 & YES & YES & NO(2) & 1904 & 1931\\
$(152, 63)$ & 11 & $(2, 1)$ & 1 & 2 & YES & YES & NO(2) & NO & 1932\\
$(152, 63)$ & 11 & $(3, 1)$ & 2 & 1 & YES & YES & NO(2) & -- & 1933\\
$(152, 63)$ & 11 & $(3, 1)$ & 2 & 1 & YES & YES & NO(2) & NO & 1934\\
$(152, 67)$ & 11 & $(3, 1)$ & 2 & 1 & YES & YES & NO(2) & -- & 1935\\
$(152, 55)$ & 12 & $(5, 2)$ & 3 & 1 & YES & YES & YES & NO & 1936\\
$(152, 41)$ & 11 & $(7, 3)$ & 4 & 1 & YES & YES & YES & -- & 1937\\
$(152, 63)$ & 11 & $(7, 3)$ & 4 & 1 & YES & YES & NO(2) & NO & 1938\\
$(152, 67)$ & 11 & $(16, 7)$ & 6 & 8 & YES & YES & NO(2) & NO & 1939\\
$(152, 67)$ & 11 & $(152, 67)$ & 11 & 152 & YES & YES & YES & NO & 1940\\
$(153, 64)$ & 11 & $(2, 1)$ & 1 & 1 & YES & YES & NO(2) & NO & 1941\\
$(153, 64)$ & 11 & $(12, 5)$ & 5 & 3 & YES & YES & NO(2) & NO & 1942\\
$(153, 70)$ & 12 & $(13, 6)$ & 7 & 1 & YES & YES & NO(2) & NO & 1943\\
$(153, 56)$ & 11 & $(27, 10)$ & 7 & 9 & YES & YES & NO(2) & NO & 1944\\
$(154, 59)$ & 11 & $(5, 2)$ & 3 & 1 & YES & YES & YES & -- & 1945\\
$(154, 65)$ & 11 & $(5, 2)$ & 3 & 1 & YES & YES & NO(2) & -- & 1946\\
$(154, 59)$ & 11 & $(7, 2)$ & 4 & 7 & YES & YES & YES & -- & 1947\\
$(154, 59)$ & 11 & $(29, 11)$ & 7 & 1 & YES & YES & YES & NO & 1948\\
$(155, 41)$ & 12 & $(9, 2)$ & 5 & 1 & YES & YES & NO(2) & NO & 1949\\
$(155, 64)$ & 11 & $(9, 4)$ & 5 & 1 & YES & YES & YES & NO & 1950\\
$(156, 25)$ & 15 & $(4, 1)$ & 3 & 4 & YES & YES & YES & -- & 1951\\
$(157, 42)$ & 12 & $(4, 1)$ & 3 & 1 & YES & YES & YES & -- & 1952\\
$(158, 61)$ & 11 & $(9, 2)$ & 5 & 1 & YES & YES & YES & NO & 1953\\
$(159, 61)$ & 12 & $(2, 1)$ & 1 & 1 & NO & YES & YES & -- & 1954\\
$(159, 61)$ & 12 & $(2, 1)$ & 1 & 1 & YES & YES & NO(2) & NO & 1955\\
$(159, 59)$ & 11 & $(3, 1)$ & 2 & 3 & YES & YES & NO(2) & -- & 1956\\
$(159, 59)$ & 11 & $(4, 1)$ & 3 & 1 & YES & YES & NO(2) & -- & 1957\\
$(159, 61)$ & 12 & $(5, 1)$ & 4 & 1 & YES & YES & NO(2) & -- & 1958\\
$(159, 47)$ & 11 & $(7, 3)$ & 4 & 1 & YES & YES & YES & NO & 1959\\
$(159, 62)$ & 11 & $(9, 2)$ & 5 & 3 & YES & YES & YES & NO & 1960\\
$(159, 61)$ & 12 & $(13, 5)$ & 5 & 1 & YES & YES & NO(2) & 1680 & 1961\\
$(159, 73)$ & 12 & $(13, 6)$ & 7 & 1 & YES & YES & NO(2) & 1869 & 1962\\
$(159, 59)$ & 11 & $(19, 7)$ & 6 & 1 & YES & YES & NO(2) & NO & 1963\\
$(159, 37)$ & 12 & $(64, 15)$ & 10 & 1 & YES & YES & YES & NO & 1964\\
$(159, 59)$ & 11 & $(97, 36)$ & 10 & 1 & YES & YES & NO(2) & NO & 1965\\
$(161, 51)$ & 13 & $(2, 1)$ & 1 & 1 & YES & YES & YES & NO & 1966\\
$(161, 48)$ & 12 & $(3, 1)$ & 2 & 1 & YES & YES & NO(2) & NO & 1967\\
$(161, 66)$ & 11 & $(5, 2)$ & 3 & 1 & YES & YES & NO(2) & NO & 1968\\
$(162, 71)$ & 12 & $(5, 1)$ & 4 & 1 & YES & YES & YES & -- & 1969\\
$(162, 71)$ & 12 & $(5, 1)$ & 4 & 1 & YES & YES & NO(2) & NO & 1970\\
$(162, 73)$ & 12 & $(5, 1)$ & 4 & 1 & YES & YES & YES & -- & 1971\\
$(162, 73)$ & 12 & $(5, 1)$ & 4 & 1 & YES & YES & YES & NO & 1972\\
$(162, 73)$ & 12 & $(5, 1)$ & 4 & 1 & YES & YES & NO(2) & NO & 1973\\
$(162, 37)$ & 12 & $(8, 3)$ & 4 & 2 & YES & YES & YES & NO & 1974\\
$(162, 71)$ & 12 & $(16, 7)$ & 6 & 2 & YES & YES & YES & 1717 & 1975\\
$(163, 43)$ & 12 & $(3, 1)$ & 2 & 1 & YES & YES & NO(2) & -- & 1976\\
$(163, 43)$ & 12 & $(4, 1)$ & 3 & 1 & YES & YES & NO(2) & -- & 1977\\
$(163, 63)$ & 11 & $(5, 2)$ & 3 & 1 & YES & YES & YES & -- & 1978\\
$(163, 71)$ & 11 & $(7, 3)$ & 4 & 1 & YES & YES & NO(2) & NO & 1979\\
$(163, 43)$ & 12 & $(11, 3)$ & 5 & 1 & YES & YES & NO(2) & NO & 1980\\
$(163, 43)$ & 12 & $(34, 9)$ & 8 & 1 & YES & YES & NO(2) & 1624 & 1981\\
$(163, 43)$ & 12 & $(53, 14)$ & 9 & 1 & YES & YES & NO(2) & NO & 1982\\
$(163, 43)$ & 12 & $(91, 24)$ & 11 & 1 & YES & YES & NO(2) & NO & 1983\\
$(163, 63)$ & 11 & $(106, 41)$ & 10 & 1 & YES & YES & YES & 2213 & 1984\\
$(163, 44)$ & 11 & $(152, 41)$ & 11 & 1 & YES & YES & YES & NO & 1985\\
$(165, 64)$ & 11 & $(5, 2)$ & 3 & 5 & YES & YES & YES & -- & 1986\\
$(166, 63)$ & 12 & $(50, 19)$ & 8 & 2 & YES & YES & YES & NO & 1987\\
$(167, 64)$ & 11 & $(5, 1)$ & 4 & 1 & YES & YES & NO(2) & NO & 1988\\
$(167, 69)$ & 11 & $(5, 2)$ & 3 & 1 & YES & YES & YES & -- & 1989\\
$(167, 64)$ & 11 & $(60, 23)$ & 9 & 1 & YES & YES & NO(2) & NO & 1990\\
$(168, 65)$ & 12 & $(6, 1)$ & 5 & 6 & YES & YES & NO(2) & -- & 1991\\
$(168, 65)$ & 12 & $(75, 29)$ & 9 & 3 & YES & YES & NO(2) & NO & 1992\\
$(169, 62)$ & 12 & $(2, 1)$ & 1 & 1 & YES & YES & YES & NO & 1993\\
$(169, 66)$ & 11 & $(2, 1)$ & 1 & 1 & YES & YES & NO(2) & -- & 1994\\
$(169, 64)$ & 11 & $(3, 1)$ & 2 & 1 & YES & YES & NO(2) & -- & 1995\\
$(169, 64)$ & 11 & $(3, 1)$ & 2 & 1 & YES & YES & NO(2) & NO & 1996\\
$(169, 38)$ & 13 & $(5, 1)$ & 4 & 1 & YES & YES & YES & NO & 1997\\
$(169, 64)$ & 11 & $(5, 1)$ & 4 & 1 & YES & YES & NO(2) & NO & 1998\\
$(169, 66)$ & 11 & $(5, 1)$ & 4 & 1 & YES & YES & NO(2) & NO & 1999\\
$(169, 71)$ & 11 & $(5, 2)$ & 3 & 1 & YES & YES & YES & -- & 2000\\
$(169, 38)$ & 13 & $(7, 2)$ & 4 & 1 & YES & YES & NO(2) & NO & 2001\\
$(169, 70)$ & 11 & $(7, 2)$ & 4 & 1 & YES & YES & YES & -- & 2002\\
$(169, 64)$ & 11 & $(13, 5)$ & 5 & 13 & YES & YES & NO(2) & 1700 & 2003\\
$(169, 38)$ & 13 & $(49, 11)$ & 10 & 1 & YES & YES & YES & NO & 2004\\
$(169, 70)$ & 11 & $(53, 22)$ & 9 & 1 & YES & YES & YES & NO & 2005\\
$(170, 29)$ & 15 & $(2, 1)$ & 1 & 2 & YES & YES & NO(2) & -- & 2006\\
$(170, 29)$ & 15 & $(2, 1)$ & 1 & 2 & YES & YES & NO(2) & NO & 2007\\
$(171, 53)$ & 12 & $(2, 1)$ & 1 & 1 & YES & YES & NO(2) & NO & 2008\\
$(171, 71)$ & 12 & $(2, 1)$ & 1 & 1 & YES & YES & NO(2) & -- & 2009\\
$(171, 71)$ & 12 & $(3, 1)$ & 2 & 3 & YES & YES & YES & -- & 2010\\
$(171, 65)$ & 11 & $(5, 2)$ & 3 & 1 & YES & YES & YES & -- & 2011\\
$(171, 71)$ & 12 & $(5, 2)$ & 3 & 1 & YES & YES & YES & NO & 2012\\
$(171, 71)$ & 12 & $(12, 5)$ & 5 & 3 & YES & YES & YES & NO & 2013\\
$(171, 65)$ & 11 & $(18, 7)$ & 6 & 9 & YES & YES & YES & NO & 2014\\
$(171, 71)$ & 12 & $(118, 49)$ & 11 & 1 & YES & YES & NO(2) & NO & 2015\\
$(172, 71)$ & 11 & $(17, 7)$ & 6 & 1 & YES & YES & NO(2) & NO & 2016\\
$(173, 51)$ & 12 & $(2, 1)$ & 1 & 1 & YES & YES & NO(2) & -- & 2017\\
$(173, 73)$ & 11 & $(2, 1)$ & 1 & 1 & YES & YES & YES & NO & 2018\\
$(173, 78)$ & 12 & $(2, 1)$ & 1 & 1 & NO & YES & NO(2) & -- & 2019\\
$(173, 78)$ & 12 & $(2, 1)$ & 1 & 1 & YES & YES & NO(2) & NO & 2020\\
$(173, 64)$ & 11 & $(5, 2)$ & 3 & 1 & YES & YES & YES & -- & 2021\\
$(173, 51)$ & 12 & $(78, 23)$ & 10 & 1 & YES & YES & NO(2) & NO & 2022\\
$(175, 62)$ & 12 & $(2, 1)$ & 1 & 1 & YES & YES & YES & -- & 2023\\
$(175, 62)$ & 12 & $(2, 1)$ & 1 & 1 & YES & YES & NO(2) & NO & 2024\\
$(175, 62)$ & 12 & $(5, 2)$ & 3 & 5 & YES & YES & NO(2) & NO & 2025\\
$(175, 67)$ & 11 & $(5, 2)$ & 3 & 5 & YES & YES & YES & -- & 2026\\
$(175, 62)$ & 12 & $(17, 6)$ & 7 & 1 & YES & YES & YES & NO & 2027\\
$(175, 67)$ & 11 & $(18, 7)$ & 6 & 1 & YES & YES & YES & NO & 2028\\
$(175, 67)$ & 11 & $(55, 21)$ & 8 & 5 & YES & YES & YES & 2253 & 2029\\
$(176, 65)$ & 11 & $(3, 1)$ & 2 & 1 & YES & YES & NO(2) & NO & 2030\\
$(176, 65)$ & 11 & $(11, 4)$ & 5 & 11 & YES & YES & NO(2) & NO & 2031\\
$(177, 47)$ & 12 & $(4, 1)$ & 3 & 1 & YES & YES & YES & NO & 2032\\
$(177, 80)$ & 12 & $(4, 1)$ & 3 & 1 & YES & YES & NO(2) & -- & 2033\\
$(177, 47)$ & 12 & $(5, 1)$ & 4 & 1 & YES & YES & YES & -- & 2034\\
$(177, 74)$ & 12 & $(12, 5)$ & 5 & 3 & YES & YES & YES & NO & 2035\\
$(177, 46)$ & 13 & $(27, 7)$ & 9 & 3 & YES & YES & NO(2) & NO & 2036\\
$(178, 47)$ & 12 & $(2, 1)$ & 1 & 2 & YES & YES & NO(2) & -- & 2037\\
$(178, 69)$ & 11 & $(5, 2)$ & 3 & 1 & YES & YES & YES & NO & 2038\\
$(178, 47)$ & 12 & $(15, 4)$ & 6 & 1 & YES & YES & NO(2) & NO & 2039\\
$(179, 48)$ & 12 & $(3, 1)$ & 2 & 1 & NO & YES & NO(2) & -- & 2040\\
$(179, 42)$ & 13 & $(4, 1)$ & 3 & 1 & YES & YES & YES & -- & 2041\\
$(179, 76)$ & 12 & $(5, 2)$ & 3 & 1 & YES & YES & NO(2) & NO & 2042\\
$(181, 65)$ & 12 & $(2, 1)$ & 1 & 1 & YES & YES & YES & NO & 2043\\
$(181, 48)$ & 12 & $(5, 2)$ & 3 & 1 & YES & YES & NO(2) & NO & 2044\\
$(181, 70)$ & 11 & $(5, 2)$ & 3 & 1 & YES & YES & YES & -- & 2045\\
$(181, 75)$ & 11 & $(5, 2)$ & 3 & 1 & YES & YES & YES & -- & 2046\\
$(181, 41)$ & 12 & $(48, 11)$ & 9 & 1 & YES & YES & YES & NO & 2047\\
$(181, 75)$ & 11 & $(53, 22)$ & 9 & 1 & YES & YES & YES & NO & 2048\\
$(181, 41)$ & 12 & $(115, 26)$ & 11 & 1 & YES & YES & YES & NO & 2049\\
$(181, 70)$ & 11 & $(119, 46)$ & 10 & 1 & YES & YES & YES & NO & 2050\\
$(187, 50)$ & 13 & $(4, 1)$ & 3 & 1 & YES & YES & NO(2) & -- & 2051\\
$(187, 79)$ & 11 & $(17, 7)$ & 6 & 17 & YES & YES & YES & 2254 & 2052\\
$(188, 57)$ & 13 & $(2, 1)$ & 1 & 2 & YES & YES & NO(2) & NO & 2053\\
$(188, 59)$ & 13 & $(2, 1)$ & 1 & 2 & YES & YES & YES & NO & 2054\\
$(188, 73)$ & 12 & $(2, 1)$ & 1 & 2 & YES & YES & NO(2) & -- & 2055\\
$(188, 57)$ & 13 & $(10, 3)$ & 5 & 2 & YES & YES & YES & NO & 2056\\
$(188, 73)$ & 12 & $(13, 5)$ & 5 & 1 & YES & YES & NO(2) & NO & 2057\\
$(189, 50)$ & 13 & $(34, 9)$ & 8 & 1 & YES & YES & YES & NO & 2058\\
$(191, 26)$ & 17 & $(2, 1)$ & 1 & 1 & YES & YES & YES & NO & 2059\\
$(191, 50)$ & 13 & $(6, 1)$ & 5 & 1 & YES & YES & NO(2) & NO & 2060\\
$(191, 59)$ & 13 & $(13, 4)$ & 6 & 1 & YES & YES & YES & NO & 2061\\
$(191, 50)$ & 13 & $(42, 11)$ & 9 & 1 & YES & YES & YES & NO & 2062\\
$(192, 31)$ & 16 & $(2, 1)$ & 1 & 2 & YES & YES & NO(2) & -- & 2063\\
$(192, 31)$ & 16 & $(2, 1)$ & 1 & 2 & YES & YES & NO(2) & 1577 & 2064\\
$(192, 71)$ & 11 & $(2, 1)$ & 1 & 2 & YES & YES & NO(2) & NO & 2065\\
$(192, 73)$ & 11 & $(2, 1)$ & 1 & 2 & YES & YES & NO(2) & NO & 2066\\
$(194, 75)$ & 11 & $(5, 2)$ & 3 & 1 & YES & YES & YES & -- & 2067\\
$(194, 75)$ & 11 & $(106, 41)$ & 10 & 2 & YES & YES & YES & NO & 2068\\
$(196, 45)$ & 13 & $(4, 1)$ & 3 & 4 & YES & YES & YES & -- & 2069\\
$(196, 45)$ & 13 & $(35, 8)$ & 8 & 7 & YES & YES & NO(2) & NO & 2070\\
$(197, 52)$ & 12 & $(5, 2)$ & 3 & 1 & YES & YES & NO(2) & -- & 2071\\
$(197, 76)$ & 12 & $(5, 1)$ & 4 & 1 & YES & YES & NO(2) & NO & 2072\\
$(197, 43)$ & 12 & $(11, 3)$ & 5 & 1 & YES & YES & YES & NO & 2073\\
$(197, 52)$ & 12 & $(19, 5)$ & 7 & 1 & YES & YES & YES & NO & 2074\\
$(197, 76)$ & 12 & $(70, 27)$ & 10 & 1 & YES & YES & NO(2) & NO & 2075\\
$(197, 52)$ & 12 & $(91, 24)$ & 11 & 1 & YES & YES & NO(2) & NO & 2076\\
$(198, 71)$ & 12 & $(2, 1)$ & 1 & 2 & YES & YES & NO(2) & -- & 2077\\
$(198, 71)$ & 12 & $(39, 14)$ & 8 & 3 & YES & YES & NO(2) & NO & 2078\\
$(199, 78)$ & 12 & $(2, 1)$ & 1 & 1 & NO & YES & NO(2) & -- & 2079\\
$(199, 78)$ & 12 & $(2, 1)$ & 1 & 1 & YES & YES & NO(2) & NO & 2080\\
$(199, 78)$ & 12 & $(3, 1)$ & 2 & 1 & YES & YES & YES & -- & 2081\\
$(199, 78)$ & 12 & $(4, 1)$ & 3 & 1 & YES & YES & YES & NO & 2082\\
$(199, 78)$ & 12 & $(5, 1)$ & 4 & 1 & YES & YES & NO(2) & NO & 2083\\
$(199, 78)$ & 12 & $(74, 29)$ & 10 & 1 & YES & YES & NO(2) & NO & 2084\\
$(199, 78)$ & 12 & $(125, 49)$ & 11 & 1 & YES & YES & YES & NO & 2085\\
$(199, 78)$ & 12 & $(199, 78)$ & 12 & 199 & YES & YES & YES & NO & 2086\\
$(201, 59)$ & 13 & $(7, 2)$ & 4 & 1 & YES & YES & NO(2) & NO & 2087\\
$(201, 59)$ & 13 & $(92, 27)$ & 11 & 1 & YES & YES & NO(2) & NO & 2088\\
$(202, 59)$ & 12 & $(2, 1)$ & 1 & 2 & YES & YES & NO(2) & -- & 2089\\
$(202, 89)$ & 12 & $(3, 1)$ & 2 & 1 & YES & YES & NO(2) & NO & 2090\\
$(202, 89)$ & 12 & $(4, 1)$ & 3 & 2 & YES & YES & NO(2) & -- & 2091\\
$(202, 59)$ & 12 & $(5, 2)$ & 3 & 1 & YES & YES & YES & -- & 2092\\
$(202, 59)$ & 12 & $(5, 2)$ & 3 & 1 & YES & YES & YES & NO & 2093\\
$(202, 59)$ & 12 & $(17, 5)$ & 6 & 1 & YES & YES & NO(2) & NO & 2094\\
$(202, 53)$ & 13 & $(202, 53)$ & 13 & 202 & YES & YES & NO(2) & NO & 2095\\
$(203, 86)$ & 12 & $(4, 1)$ & 3 & 1 & YES & YES & YES & NO & 2096\\
$(204, 89)$ & 12 & $(2, 1)$ & 1 & 2 & NO & YES & NO(2) & -- & 2097\\
$(204, 89)$ & 12 & $(3, 1)$ & 2 & 3 & YES & YES & NO(2) & NO & 2098\\
$(205, 78)$ & 12 & $(205, 78)$ & 12 & 205 & YES & YES & YES & NO & 2099\\
$(206, 47)$ & 12 & $(83, 19)$ & 10 & 1 & YES & YES & YES & NO & 2100\\
$(207, 55)$ & 13 & $(2, 1)$ & 1 & 1 & YES & YES & YES & NO & 2101\\
$(207, 55)$ & 13 & $(3, 1)$ & 2 & 3 & YES & YES & NO(2) & NO & 2102\\
$(207, 37)$ & 15 & $(17, 3)$ & 7 & 1 & YES & YES & NO(2) & NO & 2103\\
$(207, 55)$ & 13 & $(207, 55)$ & 13 & 207 & YES & YES & YES & NO & 2104\\
$(208, 61)$ & 12 & $(9, 2)$ & 5 & 1 & YES & YES & YES & NO & 2105\\
$(208, 37)$ & 13 & $(39, 7)$ & 9 & 13 & YES & YES & NO(2) & NO & 2106\\
$(209, 82)$ & 12 & $(2, 1)$ & 1 & 1 & NO & YES & NO(2) & -- & 2107\\
$(209, 47)$ & 14 & $(4, 1)$ & 3 & 1 & YES & YES & YES & NO & 2108\\
$(209, 45)$ & 13 & $(5, 2)$ & 3 & 1 & YES & YES & NO(2) & -- & 2109\\
$(209, 56)$ & 12 & $(5, 2)$ & 3 & 1 & YES & YES & NO(2) & NO & 2110\\
$(209, 91)$ & 12 & $(5, 2)$ & 3 & 1 & YES & YES & YES & NO & 2111\\
$(209, 37)$ & 14 & $(6, 1)$ & 5 & 1 & YES & YES & YES & NO & 2112\\
$(209, 91)$ & 12 & $(9, 4)$ & 5 & 1 & YES & YES & YES & NO & 2113\\
$(209, 37)$ & 14 & $(13, 2)$ & 7 & 1 & YES & YES & NO(2) & NO & 2114\\
$(209, 37)$ & 14 & $(39, 7)$ & 9 & 1 & YES & YES & NO(2) & NO & 2115\\
$(211, 93)$ & 12 & $(9, 4)$ & 5 & 1 & YES & YES & NO(2) & NO & 2116\\
$(211, 50)$ & 14 & $(38, 9)$ & 9 & 1 & YES & YES & YES & NO & 2117\\
$(211, 50)$ & 14 & $(135, 32)$ & 12 & 1 & YES & YES & YES & 2259 & 2118\\
$(213, 38)$ & 15 & $(2, 1)$ & 1 & 1 & YES & YES & NO(2) & -- & 2119\\
$(213, 62)$ & 12 & $(9, 2)$ & 5 & 3 & YES & YES & YES & NO & 2120\\
$(215, 83)$ & 12 & $(3, 1)$ & 2 & 1 & YES & YES & YES & -- & 2121\\
$(215, 83)$ & 12 & $(3, 1)$ & 2 & 1 & YES & YES & YES & NO & 2122\\
$(215, 83)$ & 12 & $(4, 1)$ & 3 & 1 & YES & YES & YES & NO & 2123\\
$(215, 83)$ & 12 & $(18, 7)$ & 6 & 1 & YES & YES & YES & NO & 2124\\
$(218, 85)$ & 12 & $(4, 1)$ & 3 & 2 & YES & YES & YES & NO & 2125\\
$(219, 65)$ & 12 & $(5, 2)$ & 3 & 1 & YES & YES & YES & -- & 2126\\
$(219, 65)$ & 12 & $(11, 3)$ & 5 & 1 & YES & YES & YES & 2285 & 2127\\
$(219, 85)$ & 12 & $(18, 7)$ & 6 & 3 & YES & YES & NO(2) & NO & 2128\\
$(221, 58)$ & 13 & $(19, 5)$ & 7 & 1 & YES & YES & YES & NO & 2129\\
$(222, 59)$ & 13 & $(15, 4)$ & 6 & 3 & YES & YES & YES & NO & 2130\\
$(222, 85)$ & 12 & $(81, 31)$ & 9 & 3 & YES & YES & YES & NO & 2131\\
$(223, 98)$ & 12 & $(3, 1)$ & 2 & 1 & YES & YES & YES & NO & 2132\\
$(223, 98)$ & 12 & $(9, 4)$ & 5 & 1 & YES & YES & YES & 1839 & 2133\\
$(225, 98)$ & 12 & $(3, 1)$ & 2 & 3 & YES & YES & YES & -- & 2134\\
$(229, 95)$ & 12 & $(2, 1)$ & 1 & 1 & YES & YES & NO(2) & NO & 2135\\
$(229, 94)$ & 12 & $(3, 1)$ & 2 & 1 & YES & YES & YES & -- & 2136\\
$(229, 64)$ & 12 & $(5, 2)$ & 3 & 1 & YES & YES & YES & -- & 2137\\
$(229, 64)$ & 12 & $(5, 2)$ & 3 & 1 & YES & YES & YES & NO & 2138\\
$(229, 94)$ & 12 & $(229, 94)$ & 12 & 229 & YES & YES & YES & NO & 2139\\
$(231, 83)$ & 12 & $(2, 1)$ & 1 & 1 & YES & YES & YES & -- & 2140\\
$(231, 83)$ & 12 & $(3, 1)$ & 2 & 3 & YES & YES & YES & -- & 2141\\
$(231, 83)$ & 12 & $(39, 14)$ & 8 & 3 & YES & YES & YES & NO & 2142\\
$(234, 43)$ & 14 & $(6, 1)$ & 5 & 6 & YES & YES & YES & NO & 2143\\
$(237, 100)$ & 12 & $(3, 1)$ & 2 & 3 & YES & YES & YES & -- & 2144\\
$(239, 32)$ & 17 & $(2, 1)$ & 1 & 1 & YES & YES & YES & NO & 2145\\
$(239, 101)$ & 12 & $(2, 1)$ & 1 & 1 & YES & YES & YES & NO & 2146\\
$(239, 50)$ & 14 & $(5, 1)$ & 4 & 1 & YES & YES & YES & NO & 2147\\
$(241, 63)$ & 13 & $(3, 1)$ & 2 & 1 & YES & YES & NO(2) & NO & 2148\\
$(241, 89)$ & 12 & $(3, 1)$ & 2 & 1 & YES & YES & NO(2) & NO & 2149\\
$(241, 46)$ & 15 & $(4, 1)$ & 3 & 1 & YES & YES & YES & -- & 2150\\
$(242, 71)$ & 13 & $(3, 1)$ & 2 & 1 & YES & YES & YES & -- & 2151\\
$(242, 71)$ & 13 & $(5, 1)$ & 4 & 1 & YES & YES & YES & NO & 2152\\
$(243, 38)$ & 16 & $(2, 1)$ & 1 & 1 & YES & YES & NO(2) & NO & 2153\\
$(243, 46)$ & 15 & $(5, 1)$ & 4 & 1 & YES & YES & YES & NO & 2154\\
$(243, 38)$ & 16 & $(13, 2)$ & 7 & 1 & YES & YES & NO(2) & NO & 2155\\
$(244, 55)$ & 13 & $(5, 2)$ & 3 & 1 & YES & YES & YES & NO & 2156\\
$(245, 69)$ & 13 & $(5, 1)$ & 4 & 5 & YES & YES & YES & NO & 2157\\
$(245, 69)$ & 13 & $(103, 29)$ & 11 & 1 & YES & YES & YES & 2206 & 2158\\
$(246, 95)$ & 12 & $(8, 3)$ & 4 & 2 & YES & YES & YES & NO & 2159\\
$(247, 56)$ & 13 & $(5, 2)$ & 3 & 1 & YES & YES & YES & NO & 2160\\
$(253, 106)$ & 12 & $(2, 1)$ & 1 & 1 & YES & YES & YES & -- & 2161\\
$(253, 57)$ & 13 & $(5, 1)$ & 4 & 1 & YES & YES & NO(2) & NO & 2162\\
$(253, 57)$ & 13 & $(40, 9)$ & 9 & 1 & YES & YES & NO(2) & NO & 2163\\
$(255, 76)$ & 13 & $(2, 1)$ & 1 & 1 & YES & YES & YES & NO & 2164\\
$(255, 97)$ & 12 & $(163, 62)$ & 11 & 1 & YES & YES & YES & NO & 2165\\
$(256, 99)$ & 12 & $(3, 1)$ & 2 & 1 & YES & YES & YES & -- & 2166\\
$(256, 99)$ & 12 & $(3, 1)$ & 2 & 1 & YES & YES & YES & NO & 2167\\
$(256, 99)$ & 12 & $(4, 1)$ & 3 & 4 & YES & YES & YES & -- & 2168\\
$(256, 99)$ & 12 & $(4, 1)$ & 3 & 4 & YES & YES & YES & NO & 2169\\
$(256, 97)$ & 12 & $(5, 2)$ & 3 & 1 & YES & YES & YES & NO & 2170\\
$(256, 99)$ & 12 & $(106, 41)$ & 10 & 2 & YES & YES & YES & 2231 & 2171\\
$(256, 99)$ & 12 & $(181, 70)$ & 11 & 1 & YES & YES & YES & NO & 2172\\
$(256, 99)$ & 12 & $(256, 99)$ & 12 & 256 & YES & YES & YES & NO & 2173\\
$(257, 45)$ & 15 & $(3, 1)$ & 2 & 1 & YES & YES & NO(2) & -- & 2174\\
$(258, 109)$ & 12 & $(3, 1)$ & 2 & 3 & YES & YES & YES & -- & 2175\\
$(258, 109)$ & 12 & $(3, 1)$ & 2 & 3 & YES & YES & YES & NO & 2176\\
$(258, 109)$ & 12 & $(45, 19)$ & 8 & 3 & YES & YES & YES & NO & 2177\\
$(259, 76)$ & 13 & $(2, 1)$ & 1 & 1 & YES & YES & YES & NO & 2178\\
$(259, 59)$ & 13 & $(5, 1)$ & 4 & 1 & YES & YES & NO(2) & NO & 2179\\
$(261, 50)$ & 15 & $(5, 1)$ & 4 & 1 & YES & YES & YES & NO & 2180\\
$(263, 100)$ & 12 & $(4, 1)$ & 3 & 1 & YES & YES & YES & -- & 2181\\
$(263, 109)$ & 12 & $(4, 1)$ & 3 & 1 & YES & YES & YES & NO & 2182\\
$(263, 100)$ & 12 & $(6, 1)$ & 5 & 1 & YES & YES & YES & -- & 2183\\
$(263, 100)$ & 12 & $(6, 1)$ & 5 & 1 & YES & YES & YES & NO & 2184\\
$(263, 100)$ & 12 & $(6, 1)$ & 5 & 1 & YES & YES & YES & NO & 2185\\
$(263, 111)$ & 12 & $(7, 3)$ & 4 & 1 & YES & YES & YES & NO & 2186\\
$(263, 109)$ & 12 & $(17, 7)$ & 6 & 1 & YES & YES & YES & NO & 2187\\
$(265, 41)$ & 16 & $(2, 1)$ & 1 & 1 & YES & YES & YES & NO & 2188\\
$(267, 98)$ & 12 & $(3, 1)$ & 2 & 3 & YES & YES & YES & -- & 2189\\
$(267, 98)$ & 12 & $(3, 1)$ & 2 & 3 & YES & YES & YES & NO & 2190\\
$(267, 98)$ & 12 & $(8, 3)$ & 4 & 1 & YES & YES & YES & NO & 2191\\
$(268, 111)$ & 12 & $(99, 41)$ & 10 & 1 & YES & YES & YES & NO & 2192\\
$(269, 78)$ & 13 & $(2, 1)$ & 1 & 1 & YES & YES & YES & NO & 2193\\
$(269, 104)$ & 12 & $(44, 17)$ & 8 & 1 & YES & YES & YES & NO & 2194\\
$(271, 48)$ & 14 & $(3, 1)$ & 2 & 1 & YES & YES & NO(2) & NO & 2195\\
$(273, 76)$ & 13 & $(5, 1)$ & 4 & 1 & YES & YES & YES & NO & 2196\\
$(274, 115)$ & 12 & $(2, 1)$ & 1 & 2 & YES & YES & YES & -- & 2197\\
$(274, 81)$ & 12 & $(13, 4)$ & 6 & 1 & YES & YES & YES & NO & 2198\\
$(274, 43)$ & 15 & $(20, 3)$ & 8 & 2 & YES & YES & NO(2) & NO & 2199\\
$(277, 78)$ & 13 & $(2, 1)$ & 1 & 1 & YES & YES & YES & -- & 2200\\
$(277, 106)$ & 12 & $(2, 1)$ & 1 & 1 & YES & YES & YES & -- & 2201\\
$(277, 106)$ & 12 & $(2, 1)$ & 1 & 1 & YES & YES & YES & NO & 2202\\
$(277, 106)$ & 12 & $(8, 3)$ & 4 & 1 & YES & YES & YES & NO & 2203\\
$(277, 106)$ & 12 & $(13, 5)$ & 5 & 1 & YES & YES & YES & NO & 2204\\
$(277, 117)$ & 12 & $(19, 8)$ & 6 & 1 & YES & YES & YES & NO & 2205\\
$(277, 78)$ & 13 & $(71, 20)$ & 10 & 1 & YES & YES & YES & 2158 & 2206\\
$(281, 109)$ & 12 & $(2, 1)$ & 1 & 1 & YES & YES & YES & -- & 2207\\
$(281, 109)$ & 12 & $(13, 5)$ & 5 & 1 & YES & YES & YES & NO & 2208\\
$(281, 109)$ & 12 & $(116, 45)$ & 10 & 1 & YES & YES & YES & NO & 2209\\
$(282, 109)$ & 12 & $(2, 1)$ & 1 & 2 & YES & YES & YES & -- & 2210\\
$(282, 109)$ & 12 & $(4, 1)$ & 3 & 2 & YES & YES & YES & -- & 2211\\
$(282, 109)$ & 12 & $(13, 5)$ & 5 & 1 & YES & YES & YES & NO & 2212\\
$(282, 109)$ & 12 & $(31, 12)$ & 7 & 1 & YES & YES & YES & 1984 & 2213\\
$(282, 109)$ & 12 & $(119, 46)$ & 10 & 1 & YES & YES & YES & NO & 2214\\
$(283, 83)$ & 13 & $(2, 1)$ & 1 & 1 & YES & YES & YES & -- & 2215\\
$(283, 83)$ & 13 & $(2, 1)$ & 1 & 1 & YES & YES & YES & NO & 2216\\
$(283, 83)$ & 13 & $(4, 1)$ & 3 & 1 & YES & YES & YES & NO & 2217\\
$(283, 108)$ & 12 & $(6, 1)$ & 5 & 1 & YES & YES & YES & -- & 2218\\
$(283, 108)$ & 12 & $(6, 1)$ & 5 & 1 & YES & YES & YES & NO & 2219\\
$(283, 108)$ & 12 & $(6, 1)$ & 5 & 1 & YES & YES & YES & NO & 2220\\
$(283, 75)$ & 13 & $(49, 13)$ & 9 & 1 & YES & YES & YES & NO & 2221\\
$(283, 108)$ & 12 & $(131, 50)$ & 10 & 1 & YES & YES & YES & 2279 & 2222\\
$(283, 83)$ & 13 & $(283, 83)$ & 13 & 283 & YES & YES & YES & NO & 2223\\
$(286, 105)$ & 12 & $(11, 4)$ & 5 & 11 & YES & YES & YES & NO & 2224\\
$(287, 111)$ & 12 & $(2, 1)$ & 1 & 1 & YES & YES & YES & -- & 2225\\
$(287, 109)$ & 12 & $(3, 1)$ & 2 & 1 & YES & YES & YES & -- & 2226\\
$(287, 109)$ & 12 & $(3, 1)$ & 2 & 1 & YES & YES & YES & NO & 2227\\
$(287, 106)$ & 12 & $(5, 1)$ & 4 & 1 & YES & YES & YES & -- & 2228\\
$(287, 111)$ & 12 & $(5, 1)$ & 4 & 1 & YES & YES & YES & -- & 2229\\
$(287, 111)$ & 12 & $(5, 1)$ & 4 & 1 & YES & YES & YES & NO & 2230\\
$(287, 111)$ & 12 & $(75, 29)$ & 9 & 1 & YES & YES & YES & 2171 & 2231\\
$(287, 109)$ & 12 & $(79, 30)$ & 9 & 1 & YES & YES & YES & NO & 2232\\
$(287, 111)$ & 12 & $(106, 41)$ & 10 & 1 & YES & YES & YES & NO & 2233\\
$(288, 85)$ & 13 & $(4, 1)$ & 3 & 4 & YES & YES & YES & -- & 2234\\
$(288, 85)$ & 13 & $(44, 13)$ & 8 & 4 & YES & YES & YES & NO & 2235\\
$(288, 119)$ & 12 & $(121, 50)$ & 10 & 1 & YES & YES & YES & NO & 2236\\
$(289, 112)$ & 12 & $(31, 12)$ & 7 & 1 & YES & YES & YES & NO & 2237\\
$(289, 112)$ & 12 & $(49, 19)$ & 8 & 1 & YES & YES & YES & NO & 2238\\
$(291, 85)$ & 13 & $(10, 3)$ & 5 & 1 & YES & YES & YES & NO & 2239\\
$(292, 111)$ & 12 & $(2, 1)$ & 1 & 2 & YES & YES & YES & -- & 2240\\
$(292, 85)$ & 13 & $(3, 1)$ & 2 & 1 & YES & YES & YES & -- & 2241\\
$(292, 121)$ & 12 & $(3, 1)$ & 2 & 1 & YES & YES & YES & -- & 2242\\
$(292, 85)$ & 13 & $(4, 1)$ & 3 & 4 & YES & YES & YES & NO & 2243\\
$(292, 111)$ & 12 & $(5, 2)$ & 3 & 1 & YES & YES & YES & NO & 2244\\
$(292, 111)$ & 12 & $(121, 46)$ & 10 & 1 & YES & YES & YES & NO & 2245\\
$(292, 85)$ & 13 & $(134, 39)$ & 11 & 2 & YES & YES & YES & 2281 & 2246\\
$(298, 83)$ & 13 & $(3, 1)$ & 2 & 1 & YES & YES & YES & NO & 2247\\
$(298, 79)$ & 13 & $(49, 13)$ & 9 & 1 & YES & YES & YES & NO & 2248\\
$(301, 65)$ & 13 & $(5, 2)$ & 3 & 1 & YES & YES & YES & -- & 2249\\
$(301, 65)$ & 13 & $(5, 2)$ & 3 & 1 & YES & YES & YES & NO & 2250\\
$(301, 115)$ & 12 & $(5, 2)$ & 3 & 1 & YES & YES & YES & NO & 2251\\
$(301, 65)$ & 13 & $(13, 3)$ & 6 & 1 & YES & YES & YES & 2294 & 2252\\
$(301, 115)$ & 12 & $(13, 5)$ & 5 & 1 & YES & YES & YES & 2029 & 2253\\
$(303, 128)$ & 12 & $(5, 2)$ & 3 & 1 & YES & YES & YES & 2052 & 2254\\
$(303, 128)$ & 12 & $(19, 8)$ & 6 & 1 & YES & YES & YES & NO & 2255\\
$(307, 85)$ & 13 & $(4, 1)$ & 3 & 1 & YES & YES & YES & -- & 2256\\
$(307, 69)$ & 14 & $(5, 1)$ & 4 & 1 & YES & YES & NO(2) & NO & 2257\\
$(307, 69)$ & 14 & $(89, 20)$ & 11 & 1 & YES & YES & NO(2) & NO & 2258\\
$(308, 73)$ & 14 & $(38, 9)$ & 9 & 2 & YES & YES & YES & 2118 & 2259\\
$(309, 59)$ & 15 & $(4, 1)$ & 3 & 1 & YES & YES & YES & NO & 2260\\
$(313, 71)$ & 14 & $(2, 1)$ & 1 & 1 & YES & YES & NO(2) & -- & 2261\\
$(313, 71)$ & 14 & $(3, 1)$ & 2 & 1 & YES & YES & YES & NO & 2262\\
$(313, 121)$ & 12 & $(3, 1)$ & 2 & 1 & YES & YES & YES & -- & 2263\\
$(313, 86)$ & 13 & $(4, 1)$ & 3 & 1 & YES & YES & YES & NO & 2264\\
$(313, 71)$ & 14 & $(5, 1)$ & 4 & 1 & YES & YES & NO(2) & NO & 2265\\
$(313, 86)$ & 13 & $(5, 1)$ & 4 & 1 & YES & YES & YES & NO & 2266\\
$(315, 88)$ & 13 & $(2, 1)$ & 1 & 1 & YES & YES & YES & NO & 2267\\
$(321, 95)$ & 13 & $(2, 1)$ & 1 & 1 & YES & YES & YES & -- & 2268\\
$(321, 95)$ & 13 & $(17, 5)$ & 6 & 1 & YES & YES & YES & 1833 & 2269\\
$(323, 94)$ & 13 & $(2, 1)$ & 1 & 1 & YES & YES & YES & -- & 2270\\
$(323, 89)$ & 13 & $(5, 1)$ & 4 & 1 & YES & YES & YES & NO & 2271\\
$(323, 94)$ & 13 & $(134, 39)$ & 11 & 1 & YES & YES & YES & NO & 2272\\
$(325, 74)$ & 14 & $(2, 1)$ & 1 & 1 & YES & YES & YES & -- & 2273\\
$(326, 97)$ & 13 & $(121, 36)$ & 11 & 1 & YES & YES & YES & NO & 2274\\
$(327, 97)$ & 13 & $(5, 1)$ & 4 & 1 & YES & YES & YES & NO & 2275\\
$(335, 94)$ & 13 & $(2, 1)$ & 1 & 1 & YES & YES & YES & NO & 2276\\
$(335, 94)$ & 13 & $(3, 1)$ & 2 & 1 & YES & YES & YES & -- & 2277\\
$(338, 99)$ & 13 & $(2, 1)$ & 1 & 2 & YES & YES & YES & NO & 2278\\
$(338, 129)$ & 12 & $(76, 29)$ & 9 & 2 & YES & YES & YES & 2222 & 2279\\
$(347, 101)$ & 13 & $(4, 1)$ & 3 & 1 & YES & YES & YES & NO & 2280\\
$(347, 101)$ & 13 & $(79, 23)$ & 10 & 1 & YES & YES & YES & 2246 & 2281\\
$(353, 97)$ & 13 & $(2, 1)$ & 1 & 1 & YES & YES & YES & -- & 2282\\
$(353, 97)$ & 13 & $(2, 1)$ & 1 & 1 & YES & YES & YES & NO & 2283\\
$(353, 97)$ & 13 & $(7, 2)$ & 4 & 1 & YES & YES & YES & NO & 2284\\
$(355, 99)$ & 13 & $(3, 1)$ & 2 & 1 & YES & YES & YES & 2127 & 2285\\
$(359, 100)$ & 13 & $(5, 1)$ & 4 & 1 & YES & YES & YES & -- & 2286\\
$(360, 101)$ & 13 & $(2, 1)$ & 1 & 2 & YES & YES & YES & -- & 2287\\
$(360, 101)$ & 13 & $(2, 1)$ & 1 & 2 & YES & YES & YES & NO & 2288\\
$(377, 85)$ & 14 & $(3, 1)$ & 2 & 1 & YES & YES & YES & NO & 2289\\
$(377, 85)$ & 14 & $(71, 16)$ & 10 & 1 & YES & YES & YES & NO & 2290\\
$(395, 73)$ & 15 & $(3, 1)$ & 2 & 1 & YES & YES & YES & -- & 2291\\
$(407, 171)$ & 13 & $(2, 1)$ & 1 & 1 & NO & YES & YES & -- & 2292\\
$(424, 97)$ & 14 & $(3, 1)$ & 2 & 1 & YES & YES & YES & NO & 2293\\
$(437, 99)$ & 14 & $(5, 1)$ & 4 & 1 & YES & YES & YES & 2252 & 2294\\
$(451, 84)$ & 15 & $(11, 2)$ & 6 & 11 & YES & YES & YES & NO & 2295\\
$(451, 84)$ & 15 & $(27, 5)$ & 8 & 1 & YES & YES & YES & NO & 2296\\
$(495, 92)$ & 15 & $(2, 1)$ & 1 & 1 & YES & YES & YES & -- & 2297\\
$(495, 92)$ & 15 & $(2, 1)$ & 1 & 1 & YES & YES & YES & NO & 2298\\
$(a; 0, 0, 0; 3)$ & 4 & $(81, 31)$ & 9 & 3 & YES & YES & YES & -- & 2299\\
$(a; 0, 0, 0; 3)$ & 4 & $(89, 34)$ & 9 & 1 & YES & YES & YES & -- & 2300\\
$(a; 1, 0, 0; 13)$ & 5 & $(33, 14)$ & 8 & 1 & YES & YES & NO(2) & -- & 2301\\
$(a; 1, 0, 0; 13)$ & 5 & $(36, 13)$ & 8 & 1 & YES & YES & NO(2) & -- & 2302\\
$(a; 1, 0, 0; 13)$ & 5 & $(55, 21)$ & 8 & 1 & YES & YES & YES & -- & 2303\\
$(a; 1, 1, 0; 19)$ & 6 & $(19, 6)$ & 8 & 19 & YES & YES & YES & -- & 2304\\
$(a; 1, 1, 0; 19)$ & 6 & $(35, 11)$ & 9 & 1 & YES & YES & NO(2) & -- & 2305\\
$(a; 2, 1, 1; 37)$ & 8 & $(12, 5)$ & 5 & 1 & YES & YES & YES & -- & 2306\\
$(a; 3, 1, 0; 31)$ & 8 & $(13, 4)$ & 6 & 1 & YES & YES & NO(2) & -- & 2307\\
$(a; 4, 0, 0; 25)$ & 8 & $(7, 3)$ & 4 & 1 & YES & YES & YES & -- & 2308\\
$(a; 4, 0, 0; 25)$ & 8 & $(8, 3)$ & 4 & 1 & YES & YES & YES & -- & 2309\\
$(a; 4, 0, 0; 25)$ & 8 & $(16, 3)$ & 7 & 1 & YES & YES & YES & -- & 2310\\
$(a; 4, 0, 0; 25)$ & 8 & $(17, 3)$ & 7 & 1 & YES & YES & YES & -- & 2311\\
$(b; 0, 0, 3; 32)$ & 8 & $(5, 2)$ & 3 & 1 & YES & YES & NO(2) & -- & 2312\\
$(b; 0, 1, 0; 19)$ & 6 & $(31, 12)$ & 7 & 1 & YES & YES & YES & -- & 2313\\
$(b; 0, 2, 1; 34)$ & 8 & $(12, 5)$ & 5 & 2 & YES & YES & YES & -- & 2314\\
$(b; 0, 3, 0; 29)$ & 8 & $(8, 3)$ & 4 & 1 & YES & YES & NO(2) & -- & 2315\\
$(b; 0, 3, 2; 53)$ & 10 & $(4, 1)$ & 3 & 1 & YES & YES & YES & -- & 2316\\
$(b; 1, 1, 0; 27)$ & 7 & $(12, 5)$ & 5 & 3 & YES & YES & NO(2) & -- & 2317\\
$(b; 1, 2, 0; 17)$ & 8 & $(22, 5)$ & 7 & 1 & YES & YES & YES & -- & 2318\\
$(b; 1, 2, 1; 7)$ & 9 & $(11, 3)$ & 5 & 1 & YES & YES & YES & -- & 2319\\
$(b; 2, 0, 0; 26)$ & 7 & $(9, 4)$ & 5 & 1 & YES & YES & YES & -- & 2320\\
$(b; 2, 0, 0; 26)$ & 7 & $(18, 7)$ & 6 & 2 & YES & YES & YES & -- & 2321\\
$(b; 2, 1, 0; 7)$ & 8 & $(5, 2)$ & 3 & 1 & YES & YES & NO(2) & -- & 2322\\
$(b; 2, 2, 0; 44)$ & 9 & $(3, 1)$ & 2 & 1 & YES & YES & YES & -- & 2323\\
$(b; 2, 3, 0; 53)$ & 10 & $(3, 1)$ & 2 & 1 & YES & YES & YES & -- & 2324\\
$(b; 3, 0, 0; 16)$ & 8 & $(5, 2)$ & 3 & 1 & YES & YES & NO(2) & -- & 2325\\
$(b; 3, 0, 0; 16)$ & 8 & $(7, 3)$ & 4 & 1 & YES & YES & YES & -- & 2326\\
$(b; 3, 0, 0; 16)$ & 8 & $(16, 5)$ & 7 & 16 & YES & YES & NO(2) & -- & 2327\\
$(b; 3, 0, 3; 11)$ & 11 & $(5, 1)$ & 4 & 1 & YES & YES & YES & -- & 2328\\
$(b; 3, 1, 1; 63)$ & 10 & $(3, 1)$ & 2 & 3 & YES & YES & NO(2) & -- & 2329\\
$(b; 3, 1, 1; 63)$ & 10 & $(4, 1)$ & 3 & 1 & YES & YES & YES & -- & 2330\\
$(c; 0, 0, 0; 4)$ & 4 & $(34, 15)$ & 8 & 2 & YES & YES & NO(2) & -- & 2331\\
$(c; 0, 0, 0; 4)$ & 4 & $(49, 19)$ & 8 & 1 & YES & YES & NO(2) & -- & 2332\\
$(c; 0, 0, 0; 4)$ & 4 & $(61, 25)$ & 9 & 1 & YES & YES & NO(2) & -- & 2333\\
$(c; 0, 0, 0; 4)$ & 4 & $(95, 36)$ & 10 & 1 & YES & YES & YES & -- & 2334\\
$(c; 0, 1, 0; 11)$ & 5 & $(51, 16)$ & 10 & 1 & YES & YES & YES & -- & 2335\\
$(c; 0, 1, 0; 11)$ & 5 & $(61, 16)$ & 10 & 1 & YES & YES & YES & -- & 2336\\
$(c; 0, 1, 0; 11)$ & 5 & $(89, 24)$ & 10 & 1 & YES & YES & YES & -- & 2337\\
$(c; 0, 1, 1; 5)$ & 6 & $(41, 16)$ & 8 & 1 & YES & YES & YES & -- & 2338\\
$(c; 0, 1, 1; 5)$ & 6 & $(61, 17)$ & 9 & 1 & YES & YES & YES & -- & 2339\\
$(c; 0, 2, 0; 7)$ & 6 & $(12, 5)$ & 5 & 1 & YES & YES & NO(2) & -- & 2340\\
$(c; 0, 2, 0; 7)$ & 6 & $(29, 9)$ & 8 & 1 & YES & YES & YES & -- & 2341\\
$(c; 0, 2, 0; 7)$ & 6 & $(36, 11)$ & 8 & 1 & YES & YES & NO(2) & -- & 2342\\
$(c; 0, 2, 0; 7)$ & 6 & $(43, 9)$ & 9 & 1 & YES & YES & YES & -- & 2343\\
$(c; 0, 2, 0; 7)$ & 6 & $(52, 11)$ & 9 & 1 & YES & YES & NO(2) & -- & 2344\\
$(c; 0, 2, 1; 19)$ & 7 & $(27, 8)$ & 7 & 1 & YES & YES & NO(2) & -- & 2345\\
$(c; 0, 3, 0; 17)$ & 7 & $(7, 3)$ & 4 & 1 & YES & YES & NO(2) & -- & 2346\\
$(c; 0, 3, 0; 17)$ & 7 & $(19, 8)$ & 6 & 1 & YES & YES & NO(2) & -- & 2347\\
$(c; 0, 3, 1; 23)$ & 8 & $(25, 7)$ & 7 & 1 & YES & YES & YES & -- & 2348\\
$(c; 0, 3, 1; 23)$ & 8 & $(32, 7)$ & 8 & 1 & YES & YES & YES & -- & 2349\\
$(c; 0, 3, 2; 29)$ & 9 & $(7, 2)$ & 4 & 1 & YES & YES & NO(2) & -- & 2350\\
$(c; 0, 3, 3; 7)$ & 10 & $(9, 2)$ & 5 & 1 & YES & YES & YES & -- & 2351\\
$(c; 0, 4, 0; 10)$ & 8 & $(10, 3)$ & 5 & 10 & YES & YES & YES & -- & 2352\\
$(c; 0, 4, 2; 17)$ & 10 & $(11, 2)$ & 6 & 1 & YES & YES & YES & -- & 2353\\
$(d; 0, 0, 0; 5)$ & 5 & $(64, 27)$ & 9 & 1 & YES & YES & YES & -- & 2354\\
$(d; 0, 0, 0; 5)$ & 5 & $(65, 24)$ & 9 & 5 & YES & YES & YES & -- & 2355\\
$(d; 0, 0, 0; 5)$ & 5 & $(79, 24)$ & 10 & 1 & YES & YES & YES & -- & 2356\\
$(d; 0, 0, 1; 14)$ & 6 & $(44, 17)$ & 8 & 2 & YES & YES & YES & -- & 2357\\
$(d; 0, 0, 2; 9)$ & 7 & $(37, 11)$ & 8 & 1 & YES & YES & YES & -- & 2358\\
$(d; 0, 0, 3; 22)$ & 8 & $(9, 2)$ & 5 & 1 & YES & YES & NO(2) & -- & 2359\\
$(d; 0, 1, 1; 17)$ & 7 & $(37, 11)$ & 8 & 1 & YES & YES & YES & -- & 2360\\
$(d; 0, 2, 0; 7)$ & 7 & $(7, 3)$ & 4 & 7 & YES & YES & NO(2) & -- & 2361\\
$(e; 0, 1, 0; 5)$ & 6 & $(31, 12)$ & 7 & 1 & YES & YES & YES & -- & 2362\\
$(e; 0, 3, 0; 7)$ & 8 & $(8, 3)$ & 4 & 1 & YES & YES & NO(2) & -- & 2363\\
$(e; 1, 1, 0; 23)$ & 7 & $(17, 7)$ & 6 & 1 & YES & YES & NO(2) & -- & 2364\\
$(e; 1, 2, 0; 28)$ & 8 & $(18, 5)$ & 6 & 2 & YES & YES & YES & -- & 2365\\
$(e; 2, 3, 0; 45)$ & 10 & $(4, 1)$ & 3 & 1 & YES & YES & YES & -- & 2366\\
$(e; 3, 0, 0; 10)$ & 8 & $(9, 4)$ & 5 & 1 & YES & YES & NO(2) & -- & 2367\\
$(f; 0, 0, 0; 6)$ & 4 & $(29, 9)$ & 8 & 1 & YES & YES & NO(2) & -- & 2368\\
$(f; 0, 0, 0; 6)$ & 4 & $(43, 16)$ & 9 & 1 & YES & YES & YES & -- & 2369\\
$(f; 0, 0, 0; 6)$ & 4 & $(47, 20)$ & 10 & 1 & YES & YES & YES & -- & 2370\\
$(f; 0, 0, 0; 6)$ & 4 & $(55, 16)$ & 9 & 1 & YES & YES & NO(2) & -- & 2371\\
$(f; 0, 0, 0; 6)$ & 4 & $(57, 16)$ & 9 & 3 & YES & YES & YES & -- & 2372\\
$(f; 0, 0, 0; 6)$ & 4 & $(64, 19)$ & 9 & 2 & YES & YES & YES & -- & 2373\\
$(f; 0, 0, 0; 6)$ & 4 & $(65, 19)$ & 9 & 1 & YES & YES & YES & -- & 2374\\
$(f; 0, 0, 0; 6)$ & 4 & $(84, 13)$ & 13 & 6 & YES & YES & YES & -- & 2375\\
$(f; 0, 0, 0; 6)$ & 4 & $(85, 33)$ & 10 & 1 & YES & YES & NO(2) & -- & 2376\\
$(f; 0, 0, 0; 6)$ & 4 & $(131, 24)$ & 13 & 1 & YES & YES & NO(2) & -- & 2377\\
$(f; 0, 0, 0; 6)$ & 4 & $(154, 45)$ & 11 & 2 & YES & YES & YES & -- & 2378\\
$(f; 0, 1, 0; 7)$ & 5 & $(23, 10)$ & 7 & 1 & YES & YES & YES & -- & 2379\\
$(f; 0, 1, 0; 7)$ & 5 & $(27, 10)$ & 7 & 1 & YES & YES & YES & -- & 2380\\
$(g; 0, 0, 1; 26)$ & 7 & $(18, 7)$ & 6 & 2 & YES & YES & YES & -- & 2381\\
$(g; 0, 1, 0; 24)$ & 7 & $(13, 5)$ & 5 & 1 & YES & YES & NO(2) & -- & 2382\\
$(g; 0, 2, 2; 17)$ & 10 & $(2, 1)$ & 1 & 1 & YES & YES & YES & -- & 2383\\
$(g; 0, 3, 0; 34)$ & 9 & $(5, 2)$ & 3 & 1 & YES & YES & NO(2) & -- & 2384\\
$(h; 0, 3, 0; 12)$ & 8 & $(9, 4)$ & 5 & 3 & YES & YES & YES & -- & 2385\\
$(h; 0, 3, 0; 12)$ & 8 & $(14, 3)$ & 6 & 2 & YES & YES & NO(2) & -- & 2386\\
$(i; 0, 0, 0; 9)$ & 5 & $(57, 13)$ & 9 & 3 & YES & YES & NO(2) & -- & 2387\\
$(i; 0, 0, 0; 9)$ & 5 & $(58, 17)$ & 9 & 1 & YES & YES & YES & -- & 2388\\
$(i; 0, 0, 0; 9)$ & 5 & $(60, 13)$ & 9 & 3 & YES & YES & NO(2) & -- & 2389\\
$(i; 0, 0, 0; 9)$ & 5 & $(75, 17)$ & 10 & 3 & YES & YES & NO(2) & -- & 2390\\
$(i; 0, 2, 0; 15)$ & 7 & $(24, 7)$ & 7 & 3 & YES & YES & YES & -- & 2391\\
$(i; 0, 2, 0; 15)$ & 7 & $(25, 7)$ & 7 & 5 & YES & YES & YES & -- & 2392\\
$(j; 0, 0, 0; 8)$ & 5 & $(31, 11)$ & 8 & 1 & YES & YES & NO(2) & -- & 2393\\
$(j; 0, 0, 0; 8)$ & 5 & $(71, 27)$ & 9 & 1 & YES & YES & YES & -- & 2394\\
$(j; 0, 0, 0; 8)$ & 5 & $(76, 29)$ & 9 & 4 & YES & YES & YES & -- & 2395\\
$(j; 0, 1, 0; 10)$ & 6 & $(31, 14)$ & 8 & 1 & YES & YES & NO(2) & -- & 2396\\
$(j; 0, 2, 0; 12)$ & 7 & $(16, 5)$ & 7 & 4 & YES & YES & YES & -- & 2397\\
$(j; 0, 3, 0; 14)$ & 8 & $(11, 4)$ & 5 & 1 & YES & YES & YES & -- & 2398
\end{longtable}
\subsection{2 chains, $K^2 = 4$}
\begin{longtable}{|c|c|c|c|c|c|c|c|c|c|}
\hline
\multicolumn{10}{|c|}{2 chains, $K^2 = 4$}\\
\hline
$(n,a)$ & Length & $(n,a)$ & Length & GCD & Nef & $\mathbb Q$-ef & Obstruction 0 & WH & Index\\
\hline
\endfirsthead

\hline
$(n,a)$ & Length & $(n,a)$ & Length & GCD & Nef & $\mathbb Q$-ef & Obstruction 0 & WH & Index\\
\hline
\endhead
\hline
\endfoot

$(24, 11)$ & 8 & $(18, 7)$ & 6 & 6 & YES & YES & NO(3) & -- & 2399\\
$(24, 11)$ & 8 & $(24, 7)$ & 7 & 24 & YES & YES & NO(3) & -- & 2400\\
$(34, 13)$ & 7 & $(24, 5)$ & 8 & 2 & YES & YES & NO(2) & -- & 2401\\
$(36, 11)$ & 8 & $(31, 14)$ & 8 & 1 & YES & YES & NO(2) & -- & 2402\\
$(37, 10)$ & 8 & $(23, 9)$ & 7 & 1 & YES & YES & YES & NO & 2403\\
$(39, 11)$ & 9 & $(23, 5)$ & 7 & 1 & YES & YES & NO(2) & -- & 2404\\
$(41, 7)$ & 11 & $(18, 7)$ & 6 & 1 & YES & YES & NO(3) & -- & 2405\\
$(41, 7)$ & 11 & $(24, 7)$ & 7 & 1 & YES & YES & NO(3) & -- & 2406\\
$(43, 19)$ & 9 & $(33, 10)$ & 8 & 1 & YES & YES & NO(2) & -- & 2407\\
$(44, 17)$ & 8 & $(31, 12)$ & 7 & 1 & YES & YES & NO(2) & -- & 2408\\
$(44, 17)$ & 8 & $(33, 14)$ & 8 & 11 & YES & YES & NO(2) & -- & 2409\\
$(47, 20)$ & 10 & $(29, 8)$ & 7 & 1 & YES & YES & NO(2) & -- & 2410\\
$(49, 18)$ & 8 & $(33, 14)$ & 8 & 1 & YES & YES & NO(2) & -- & 2411\\
$(51, 19)$ & 10 & $(31, 7)$ & 8 & 1 & YES & YES & NO(2) & -- & 2412\\
$(51, 14)$ & 9 & $(40, 7)$ & 9 & 1 & YES & YES & NO(3) & -- & 2413\\
$(52, 23)$ & 10 & $(18, 5)$ & 6 & 2 & YES & YES & YES & -- & 2414\\
$(52, 23)$ & 10 & $(23, 5)$ & 7 & 1 & YES & YES & YES & NO & 2415\\
$(56, 23)$ & 9 & $(31, 12)$ & 7 & 1 & YES & YES & NO(2) & -- & 2416\\
$(56, 15)$ & 9 & $(44, 13)$ & 8 & 4 & YES & YES & YES & -- & 2417\\
$(57, 10)$ & 10 & $(55, 23)$ & 9 & 1 & YES & YES & NO(2) & NO & 2418\\
$(59, 26)$ & 9 & $(33, 10)$ & 8 & 1 & YES & YES & NO(2) & -- & 2419\\
$(62, 17)$ & 10 & $(26, 7)$ & 7 & 2 & YES & YES & NO(2) & -- & 2420\\
$(63, 26)$ & 9 & $(33, 7)$ & 8 & 3 & YES & YES & NO(2) & -- & 2421\\
$(64, 17)$ & 10 & $(35, 8)$ & 8 & 1 & YES & YES & NO(2) & -- & 2422\\
$(65, 19)$ & 9 & $(44, 17)$ & 8 & 1 & YES & YES & YES & -- & 2423\\
$(67, 21)$ & 11 & $(11, 4)$ & 5 & 1 & YES & YES & YES & -- & 2424\\
$(67, 20)$ & 11 & $(18, 7)$ & 6 & 1 & YES & YES & NO(2) & -- & 2425\\
$(67, 20)$ & 11 & $(32, 7)$ & 8 & 1 & YES & YES & NO(2) & NO & 2426\\
$(68, 19)$ & 9 & $(11, 4)$ & 5 & 1 & YES & YES & YES & -- & 2427\\
$(68, 19)$ & 9 & $(16, 5)$ & 7 & 4 & YES & YES & YES & -- & 2428\\
$(68, 19)$ & 9 & $(16, 5)$ & 7 & 4 & YES & YES & YES & NO & 2429\\
$(68, 19)$ & 9 & $(44, 17)$ & 8 & 4 & YES & YES & YES & -- & 2430\\
$(71, 27)$ & 9 & $(48, 11)$ & 9 & 1 & YES & YES & YES & -- & 2431\\
$(79, 21)$ & 11 & $(23, 5)$ & 7 & 1 & YES & YES & YES & -- & 2432\\
$(84, 37)$ & 10 & $(23, 7)$ & 7 & 1 & YES & YES & NO(2) & -- & 2433\\
$(84, 37)$ & 10 & $(31, 12)$ & 7 & 1 & YES & YES & NO(2) & NO & 2434\\
$(87, 19)$ & 10 & $(11, 4)$ & 5 & 1 & YES & YES & YES & -- & 2435\\
$(87, 19)$ & 10 & $(11, 4)$ & 5 & 1 & YES & YES & YES & NO & 2436\\
$(87, 20)$ & 12 & $(19, 8)$ & 6 & 1 & YES & YES & YES & NO & 2437\\
$(89, 34)$ & 9 & $(37, 11)$ & 8 & 1 & YES & YES & YES & -- & 2438\\
$(89, 26)$ & 10 & $(67, 20)$ & 11 & 1 & YES & YES & NO(2) & NO & 2439\\
$(92, 21)$ & 10 & $(43, 18)$ & 8 & 1 & YES & YES & YES & -- & 2440\\
$(97, 21)$ & 10 & $(21, 5)$ & 8 & 1 & YES & YES & NO(2) & NO & 2441\\
$(98, 41)$ & 10 & $(16, 7)$ & 6 & 2 & YES & YES & YES & -- & 2442\\
$(98, 41)$ & 10 & $(73, 31)$ & 10 & 1 & YES & YES & YES & NO & 2443\\
$(99, 29)$ & 10 & $(23, 10)$ & 7 & 1 & YES & YES & YES & -- & 2444\\
$(99, 29)$ & 10 & $(26, 11)$ & 7 & 1 & YES & YES & YES & -- & 2445\\
$(103, 37)$ & 10 & $(16, 7)$ & 6 & 1 & YES & YES & NO(2) & -- & 2446\\
$(103, 39)$ & 10 & $(53, 20)$ & 10 & 1 & YES & YES & YES & NO & 2447\\
$(106, 31)$ & 10 & $(19, 8)$ & 6 & 1 & YES & YES & YES & -- & 2448\\
$(106, 45)$ & 11 & $(49, 20)$ & 9 & 1 & YES & YES & NO(2) & NO & 2449\\
$(107, 41)$ & 10 & $(27, 8)$ & 7 & 1 & YES & YES & YES & -- & 2450\\
$(107, 47)$ & 10 & $(52, 23)$ & 10 & 1 & YES & YES & YES & NO & 2451\\
$(109, 40)$ & 10 & $(7, 2)$ & 4 & 1 & YES & YES & NO(2) & -- & 2452\\
$(110, 23)$ & 11 & $(7, 2)$ & 4 & 1 & YES & YES & NO(3) & -- & 2453\\
$(110, 23)$ & 11 & $(7, 2)$ & 4 & 1 & YES & YES & NO(3) & NO & 2454\\
$(110, 29)$ & 12 & $(16, 5)$ & 7 & 2 & YES & YES & NO(2) & -- & 2455\\
$(113, 31)$ & 11 & $(7, 2)$ & 4 & 1 & YES & YES & YES & -- & 2456\\
$(113, 31)$ & 11 & $(7, 2)$ & 4 & 1 & YES & YES & YES & NO & 2457\\
$(115, 42)$ & 11 & $(22, 5)$ & 7 & 1 & YES & YES & YES & -- & 2458\\
$(117, 49)$ & 10 & $(16, 5)$ & 7 & 1 & YES & YES & YES & NO & 2459\\
$(117, 49)$ & 10 & $(16, 7)$ & 6 & 1 & YES & YES & YES & -- & 2460\\
$(117, 31)$ & 11 & $(23, 5)$ & 7 & 1 & YES & YES & YES & -- & 2461\\
$(120, 43)$ & 11 & $(3, 1)$ & 2 & 3 & YES & YES & YES & -- & 2462\\
$(121, 32)$ & 11 & $(17, 4)$ & 7 & 1 & YES & YES & NO(2) & -- & 2463\\
$(125, 27)$ & 11 & $(11, 3)$ & 5 & 1 & YES & YES & YES & -- & 2464\\
$(125, 27)$ & 11 & $(11, 3)$ & 5 & 1 & YES & YES & YES & NO & 2465\\
$(128, 47)$ & 10 & $(115, 42)$ & 11 & 1 & YES & YES & YES & NO & 2466\\
$(129, 50)$ & 10 & $(18, 7)$ & 6 & 3 & YES & YES & YES & -- & 2467\\
$(131, 50)$ & 10 & $(18, 7)$ & 6 & 1 & YES & YES & YES & -- & 2468\\
$(131, 36)$ & 11 & $(31, 7)$ & 8 & 1 & YES & YES & YES & -- & 2469\\
$(137, 31)$ & 11 & $(56, 13)$ & 10 & 1 & YES & YES & NO(2) & NO & 2470\\
$(138, 37)$ & 11 & $(25, 7)$ & 7 & 1 & YES & YES & YES & -- & 2471\\
$(140, 53)$ & 11 & $(9, 4)$ & 5 & 1 & YES & YES & YES & -- & 2472\\
$(140, 53)$ & 11 & $(28, 11)$ & 8 & 28 & YES & YES & YES & NO & 2473\\
$(144, 61)$ & 11 & $(2, 1)$ & 1 & 2 & YES & YES & NO(2) & -- & 2474\\
$(147, 53)$ & 11 & $(103, 37)$ & 10 & 1 & YES & YES & NO(2) & NO & 2475\\
$(148, 53)$ & 12 & $(5, 2)$ & 3 & 1 & YES & YES & YES & -- & 2476\\
$(148, 53)$ & 12 & $(7, 3)$ & 4 & 1 & YES & YES & YES & NO & 2477\\
$(148, 53)$ & 12 & $(9, 2)$ & 5 & 1 & YES & YES & YES & -- & 2478\\
$(148, 53)$ & 12 & $(19, 7)$ & 6 & 1 & YES & YES & YES & NO & 2479\\
$(149, 45)$ & 12 & $(62, 19)$ & 10 & 1 & YES & YES & NO(2) & NO & 2480\\
$(152, 67)$ & 11 & $(52, 23)$ & 10 & 4 & YES & YES & YES & NO & 2481\\
$(153, 41)$ & 11 & $(12, 5)$ & 5 & 3 & YES & YES & NO(2) & -- & 2482\\
$(157, 42)$ & 12 & $(28, 5)$ & 8 & 1 & YES & YES & YES & -- & 2483\\
$(165, 64)$ & 11 & $(8, 3)$ & 4 & 1 & YES & YES & NO(2) & -- & 2484\\
$(166, 61)$ & 11 & $(18, 5)$ & 6 & 2 & YES & YES & YES & -- & 2485\\
$(173, 75)$ & 13 & $(13, 2)$ & 7 & 1 & YES & YES & NO(2) & -- & 2486\\
$(175, 48)$ & 12 & $(113, 31)$ & 11 & 1 & YES & YES & YES & NO & 2487\\
$(176, 69)$ & 12 & $(8, 3)$ & 4 & 8 & YES & YES & YES & -- & 2488\\
$(178, 63)$ & 12 & $(4, 1)$ & 3 & 2 & YES & YES & NO(2) & -- & 2489\\
$(178, 47)$ & 12 & $(27, 8)$ & 7 & 1 & YES & YES & YES & NO & 2490\\
$(178, 49)$ & 11 & $(142, 39)$ & 11 & 2 & YES & YES & NO(2) & NO & 2491\\
$(179, 75)$ & 11 & $(17, 5)$ & 6 & 1 & YES & YES & YES & -- & 2492\\
$(179, 48)$ & 12 & $(85, 23)$ & 10 & 1 & YES & YES & NO(2) & NO & 2493\\
$(183, 67)$ & 11 & $(10, 3)$ & 5 & 1 & YES & YES & NO(2) & -- & 2494\\
$(184, 51)$ & 12 & $(4, 1)$ & 3 & 4 & YES & YES & YES & -- & 2495\\
$(186, 71)$ & 11 & $(97, 37)$ & 10 & 1 & YES & YES & NO(2) & 2581 & 2496\\
$(187, 71)$ & 11 & $(13, 4)$ & 6 & 1 & YES & YES & NO(2) & -- & 2497\\
$(187, 71)$ & 11 & $(60, 23)$ & 9 & 1 & YES & YES & NO(2) & NO & 2498\\
$(189, 40)$ & 12 & $(12, 5)$ & 5 & 3 & YES & YES & NO(2) & -- & 2499\\
$(191, 59)$ & 13 & $(9, 4)$ & 5 & 1 & YES & YES & NO(2) & -- & 2500\\
$(191, 59)$ & 13 & $(9, 4)$ & 5 & 1 & YES & YES & NO(2) & NO & 2501\\
$(193, 53)$ & 12 & $(22, 5)$ & 7 & 1 & YES & YES & YES & -- & 2502\\
$(193, 53)$ & 12 & $(167, 46)$ & 11 & 1 & YES & YES & YES & NO & 2503\\
$(195, 82)$ & 12 & $(23, 10)$ & 7 & 1 & YES & YES & NO(2) & NO & 2504\\
$(205, 38)$ & 15 & $(167, 31)$ & 12 & 1 & YES & YES & NO(3) & NO & 2505\\
$(206, 45)$ & 12 & $(14, 5)$ & 6 & 2 & YES & YES & YES & -- & 2506\\
$(206, 91)$ & 13 & $(197, 87)$ & 12 & 1 & YES & YES & YES & 2605 & 2507\\
$(207, 79)$ & 11 & $(17, 5)$ & 6 & 1 & YES & YES & YES & -- & 2508\\
$(208, 95)$ & 13 & $(4, 1)$ & 3 & 4 & YES & YES & YES & -- & 2509\\
$(208, 85)$ & 13 & $(9, 2)$ & 5 & 1 & YES & YES & NO(2) & -- & 2510\\
$(208, 37)$ & 13 & $(12, 5)$ & 5 & 4 & YES & YES & NO(2) & -- & 2511\\
$(208, 37)$ & 13 & $(12, 5)$ & 5 & 4 & YES & YES & NO(2) & NO & 2512\\
$(208, 61)$ & 12 & $(18, 5)$ & 6 & 2 & YES & YES & YES & -- & 2513\\
$(208, 37)$ & 13 & $(22, 5)$ & 7 & 2 & YES & YES & NO(2) & NO & 2514\\
$(208, 37)$ & 13 & $(97, 17)$ & 11 & 1 & YES & YES & YES & NO & 2515\\
$(212, 89)$ & 11 & $(5, 2)$ & 3 & 1 & YES & YES & NO(2) & -- & 2516\\
$(212, 81)$ & 11 & $(12, 5)$ & 5 & 4 & YES & YES & YES & -- & 2517\\
$(212, 89)$ & 11 & $(26, 11)$ & 7 & 2 & YES & YES & NO(2) & NO & 2518\\
$(213, 46)$ & 12 & $(9, 4)$ & 5 & 3 & YES & YES & YES & NO & 2519\\
$(217, 58)$ & 14 & $(4, 1)$ & 3 & 1 & YES & YES & YES & -- & 2520\\
$(217, 92)$ & 12 & $(191, 81)$ & 13 & 1 & YES & YES & YES & 2597 & 2521\\
$(218, 47)$ & 13 & $(27, 5)$ & 8 & 1 & YES & YES & YES & NO & 2522\\
$(219, 67)$ & 12 & $(3, 1)$ & 2 & 3 & YES & YES & NO(2) & -- & 2523\\
$(219, 67)$ & 12 & $(3, 1)$ & 2 & 3 & YES & YES & NO(2) & NO & 2524\\
$(219, 83)$ & 12 & $(15, 4)$ & 6 & 3 & YES & YES & NO(2) & NO & 2525\\
$(220, 97)$ & 12 & $(5, 2)$ & 3 & 5 & YES & YES & YES & -- & 2526\\
$(222, 61)$ & 12 & $(8, 3)$ & 4 & 2 & YES & YES & NO(2) & -- & 2527\\
$(224, 103)$ & 13 & $(224, 103)$ & 13 & 224 & YES & YES & YES & NO & 2528\\
$(227, 60)$ & 12 & $(7, 3)$ & 4 & 1 & YES & YES & NO(2) & NO & 2529\\
$(227, 60)$ & 12 & $(91, 24)$ & 11 & 1 & YES & YES & NO(2) & NO & 2530\\
$(227, 84)$ & 12 & $(119, 44)$ & 10 & 1 & YES & YES & NO(2) & NO & 2531\\
$(229, 64)$ & 12 & $(27, 8)$ & 7 & 1 & YES & YES & YES & NO & 2532\\
$(231, 61)$ & 13 & $(11, 2)$ & 6 & 11 & YES & YES & YES & -- & 2533\\
$(232, 89)$ & 13 & $(5, 2)$ & 3 & 1 & YES & YES & YES & -- & 2534\\
$(232, 89)$ & 13 & $(7, 3)$ & 4 & 1 & YES & YES & NO(2) & -- & 2535\\
$(232, 89)$ & 13 & $(7, 3)$ & 4 & 1 & YES & YES & NO(2) & NO & 2536\\
$(233, 89)$ & 11 & $(5, 2)$ & 3 & 1 & YES & YES & NO(2) & -- & 2537\\
$(233, 103)$ & 13 & $(6, 1)$ & 5 & 1 & YES & YES & YES & -- & 2538\\
$(233, 89)$ & 11 & $(29, 11)$ & 7 & 1 & YES & YES & NO(2) & NO & 2539\\
$(233, 89)$ & 11 & $(107, 41)$ & 10 & 1 & YES & YES & YES & NO & 2540\\
$(236, 53)$ & 14 & $(22, 5)$ & 7 & 2 & YES & YES & YES & NO & 2541\\
$(236, 65)$ & 12 & $(24, 7)$ & 7 & 4 & YES & YES & YES & -- & 2542\\
$(239, 107)$ & 13 & $(4, 1)$ & 3 & 1 & YES & YES & NO(2) & -- & 2543\\
$(239, 73)$ & 14 & $(7, 1)$ & 6 & 1 & YES & YES & NO(2) & NO & 2544\\
$(239, 104)$ & 13 & $(19, 8)$ & 6 & 1 & YES & YES & NO(2) & NO & 2545\\
$(241, 88)$ & 13 & $(5, 1)$ & 4 & 1 & YES & YES & YES & NO & 2546\\
$(241, 88)$ & 13 & $(11, 2)$ & 6 & 1 & YES & YES & NO(2) & NO & 2547\\
$(243, 106)$ & 12 & $(13, 3)$ & 6 & 1 & YES & YES & YES & -- & 2548\\
$(244, 67)$ & 13 & $(142, 39)$ & 11 & 2 & YES & YES & NO(3) & 2586 & 2549\\
$(245, 107)$ & 13 & $(2, 1)$ & 1 & 1 & YES & YES & YES & -- & 2550\\
$(245, 108)$ & 12 & $(5, 2)$ & 3 & 5 & YES & YES & YES & -- & 2551\\
$(248, 91)$ & 12 & $(128, 47)$ & 10 & 8 & YES & YES & NO(2) & NO & 2552\\
$(257, 69)$ & 12 & $(7, 3)$ & 4 & 1 & YES & YES & NO(2) & -- & 2553\\
$(261, 100)$ & 12 & $(3, 1)$ & 2 & 3 & YES & YES & NO(2) & -- & 2554\\
$(261, 100)$ & 12 & $(21, 8)$ & 6 & 3 & YES & YES & NO(2) & NO & 2555\\
$(265, 104)$ & 13 & $(5, 1)$ & 4 & 5 & YES & YES & YES & NO & 2556\\
$(265, 104)$ & 13 & $(135, 53)$ & 12 & 5 & YES & YES & NO(2) & 2645 & 2557\\
$(268, 111)$ & 12 & $(5, 2)$ & 3 & 1 & YES & YES & NO(2) & -- & 2558\\
$(268, 111)$ & 12 & $(10, 3)$ & 5 & 2 & YES & YES & YES & -- & 2559\\
$(269, 71)$ & 13 & $(49, 13)$ & 9 & 1 & YES & YES & YES & NO & 2560\\
$(271, 96)$ & 14 & $(25, 9)$ & 7 & 1 & YES & YES & NO(2) & NO & 2561\\
$(273, 85)$ & 13 & $(3, 1)$ & 2 & 3 & NO & YES & NO(2) & -- & 2562\\
$(280, 103)$ & 13 & $(79, 29)$ & 9 & 1 & YES & YES & NO(2) & NO & 2563\\
$(283, 52)$ & 15 & $(125, 23)$ & 12 & 1 & YES & YES & NO(3) & NO & 2564\\
$(288, 121)$ & 12 & $(3, 1)$ & 2 & 3 & YES & YES & NO(2) & -- & 2565\\
$(288, 121)$ & 12 & $(3, 1)$ & 2 & 3 & YES & YES & NO(2) & NO & 2566\\
$(288, 121)$ & 12 & $(9, 4)$ & 5 & 9 & YES & YES & YES & NO & 2567\\
$(289, 66)$ & 13 & $(43, 10)$ & 9 & 1 & YES & YES & NO(2) & NO & 2568\\
$(292, 111)$ & 12 & $(10, 3)$ & 5 & 2 & YES & YES & YES & -- & 2569\\
$(293, 123)$ & 12 & $(7, 2)$ & 4 & 1 & YES & YES & YES & -- & 2570\\
$(298, 131)$ & 13 & $(5, 2)$ & 3 & 1 & YES & YES & NO(2) & -- & 2571\\
$(302, 117)$ & 12 & $(13, 3)$ & 6 & 1 & YES & YES & YES & -- & 2572\\
$(302, 117)$ & 12 & $(13, 3)$ & 6 & 1 & YES & YES & YES & NO & 2573\\
$(308, 87)$ & 14 & $(3, 1)$ & 2 & 1 & YES & YES & NO(3) & NO & 2574\\
$(310, 83)$ & 13 & $(7, 3)$ & 4 & 1 & YES & YES & YES & -- & 2575\\
$(313, 121)$ & 12 & $(8, 3)$ & 4 & 1 & YES & YES & YES & -- & 2576\\
$(313, 121)$ & 12 & $(13, 3)$ & 6 & 1 & YES & YES & YES & -- & 2577\\
$(314, 83)$ & 13 & $(121, 32)$ & 11 & 1 & YES & YES & NO(2) & NO & 2578\\
$(317, 121)$ & 12 & $(3, 1)$ & 2 & 1 & YES & YES & NO(2) & -- & 2579\\
$(317, 121)$ & 12 & $(3, 1)$ & 2 & 1 & YES & YES & NO(2) & NO & 2580\\
$(317, 121)$ & 12 & $(21, 8)$ & 6 & 1 & YES & YES & NO(2) & 2496 & 2581\\
$(325, 87)$ & 13 & $(157, 42)$ & 12 & 1 & YES & YES & YES & NO & 2582\\
$(332, 97)$ & 13 & $(37, 11)$ & 8 & 1 & YES & YES & YES & NO & 2583\\
$(335, 92)$ & 13 & $(4, 1)$ & 3 & 1 & YES & YES & NO(3) & NO & 2584\\
$(335, 92)$ & 13 & $(13, 3)$ & 6 & 1 & YES & YES & YES & NO & 2585\\
$(335, 92)$ & 13 & $(51, 14)$ & 9 & 1 & YES & YES & NO(3) & 2549 & 2586\\
$(335, 92)$ & 13 & $(295, 81)$ & 14 & 5 & YES & YES & YES & 2687 & 2587\\
$(336, 137)$ & 14 & $(4, 1)$ & 3 & 4 & YES & YES & NO(2) & -- & 2588\\
$(336, 137)$ & 14 & $(233, 95)$ & 13 & 1 & YES & YES & NO(2) & NO & 2589\\
$(340, 143)$ & 14 & $(5, 2)$ & 3 & 5 & YES & YES & NO(2) & -- & 2590\\
$(340, 143)$ & 14 & $(5, 2)$ & 3 & 5 & YES & YES & YES & NO & 2591\\
$(341, 90)$ & 14 & $(5, 2)$ & 3 & 1 & YES & YES & NO(2) & -- & 2592\\
$(341, 90)$ & 14 & $(269, 71)$ & 13 & 1 & YES & YES & YES & NO & 2593\\
$(347, 153)$ & 13 & $(3, 1)$ & 2 & 1 & YES & YES & YES & -- & 2594\\
$(348, 103)$ & 13 & $(5, 2)$ & 3 & 1 & YES & YES & YES & -- & 2595\\
$(348, 125)$ & 13 & $(5, 2)$ & 3 & 1 & YES & YES & YES & -- & 2596\\
$(349, 148)$ & 14 & $(92, 39)$ & 10 & 1 & YES & YES & YES & 2521 & 2597\\
$(353, 154)$ & 13 & $(243, 106)$ & 12 & 1 & YES & YES & YES & NO & 2598\\
$(355, 63)$ & 15 & $(5, 2)$ & 3 & 5 & YES & YES & YES & -- & 2599\\
$(356, 105)$ & 13 & $(5, 2)$ & 3 & 1 & YES & YES & YES & -- & 2600\\
$(363, 58)$ & 17 & $(4, 1)$ & 3 & 1 & YES & YES & YES & -- & 2601\\
$(363, 152)$ & 13 & $(4, 1)$ & 3 & 1 & YES & YES & NO(2) & -- & 2602\\
$(363, 152)$ & 13 & $(117, 49)$ & 10 & 3 & YES & YES & NO(2) & NO & 2603\\
$(367, 154)$ & 13 & $(2, 1)$ & 1 & 1 & YES & YES & NO(2) & NO & 2604\\
$(369, 163)$ & 14 & $(77, 34)$ & 10 & 1 & YES & YES & YES & 2507 & 2605\\
$(371, 132)$ & 14 & $(3, 1)$ & 2 & 1 & YES & YES & NO(2) & -- & 2606\\
$(371, 144)$ & 13 & $(3, 1)$ & 2 & 1 & YES & YES & NO(2) & -- & 2607\\
$(375, 143)$ & 14 & $(2, 1)$ & 1 & 1 & YES & YES & NO(2) & -- & 2608\\
$(375, 88)$ & 15 & $(5, 1)$ & 4 & 5 & YES & YES & YES & NO & 2609\\
$(375, 88)$ & 15 & $(11, 2)$ & 6 & 1 & YES & YES & NO(2) & NO & 2610\\
$(376, 139)$ & 13 & $(4, 1)$ & 3 & 4 & YES & YES & NO(2) & -- & 2611\\
$(376, 139)$ & 13 & $(119, 44)$ & 10 & 1 & YES & YES & NO(2) & NO & 2612\\
$(379, 165)$ & 13 & $(4, 1)$ & 3 & 1 & YES & YES & YES & NO & 2613\\
$(380, 137)$ & 13 & $(9, 2)$ & 5 & 1 & YES & YES & YES & -- & 2614\\
$(383, 140)$ & 13 & $(7, 2)$ & 4 & 1 & YES & YES & YES & -- & 2615\\
$(383, 140)$ & 13 & $(7, 3)$ & 4 & 1 & YES & YES & YES & NO & 2616\\
$(388, 113)$ & 14 & $(24, 7)$ & 7 & 4 & YES & YES & NO(2) & NO & 2617\\
$(391, 73)$ & 16 & $(5, 2)$ & 3 & 1 & YES & YES & NO(2) & -- & 2618\\
$(391, 73)$ & 16 & $(5, 2)$ & 3 & 1 & YES & YES & NO(2) & NO & 2619\\
$(393, 116)$ & 13 & $(8, 3)$ & 4 & 1 & YES & YES & NO(2) & -- & 2620\\
$(395, 122)$ & 16 & $(3, 1)$ & 2 & 1 & YES & YES & NO(2) & -- & 2621\\
$(395, 123)$ & 14 & $(16, 5)$ & 7 & 1 & YES & YES & NO(3) & NO & 2622\\
$(397, 175)$ & 13 & $(397, 175)$ & 13 & 397 & YES & YES & YES & NO & 2623\\
$(407, 71)$ & 15 & $(3, 1)$ & 2 & 1 & YES & YES & NO(3) & -- & 2624\\
$(413, 157)$ & 13 & $(7, 2)$ & 4 & 7 & YES & YES & YES & -- & 2625\\
$(415, 127)$ & 14 & $(3, 1)$ & 2 & 1 & YES & YES & NO(2) & -- & 2626\\
$(418, 111)$ & 14 & $(4, 1)$ & 3 & 2 & YES & YES & YES & -- & 2627\\
$(421, 80)$ & 16 & $(5, 2)$ & 3 & 1 & YES & YES & NO(2) & NO & 2628\\
$(421, 80)$ & 16 & $(5, 2)$ & 3 & 1 & YES & YES & NO(2) & NO & 2629\\
$(426, 97)$ & 15 & $(3, 1)$ & 2 & 3 & YES & YES & NO(2) & -- & 2630\\
$(428, 101)$ & 16 & $(4, 1)$ & 3 & 4 & YES & YES & NO(2) & -- & 2631\\
$(428, 101)$ & 16 & $(13, 3)$ & 6 & 1 & YES & YES & NO(2) & NO & 2632\\
$(432, 181)$ & 13 & $(5, 2)$ & 3 & 1 & YES & YES & YES & -- & 2633\\
$(432, 181)$ & 13 & $(8, 3)$ & 4 & 8 & YES & YES & NO(2) & NO & 2634\\
$(433, 128)$ & 13 & $(169, 50)$ & 11 & 1 & YES & YES & YES & NO & 2635\\
$(434, 115)$ & 14 & $(200, 53)$ & 12 & 2 & YES & YES & YES & 2682 & 2636\\
$(436, 115)$ & 15 & $(53, 14)$ & 9 & 1 & YES & YES & YES & NO & 2637\\
$(438, 161)$ & 13 & $(14, 5)$ & 6 & 2 & YES & YES & YES & NO & 2638\\
$(445, 172)$ & 13 & $(9, 2)$ & 5 & 1 & YES & YES & YES & -- & 2639\\
$(445, 172)$ & 13 & $(9, 2)$ & 5 & 1 & YES & YES & YES & NO & 2640\\
$(446, 173)$ & 13 & $(44, 17)$ & 8 & 2 & YES & YES & YES & NO & 2641\\
$(446, 197)$ & 14 & $(77, 34)$ & 10 & 1 & YES & YES & YES & 2680 & 2642\\
$(448, 197)$ & 15 & $(4, 1)$ & 3 & 4 & YES & YES & NO(2) & -- & 2643\\
$(448, 197)$ & 15 & $(7, 3)$ & 4 & 7 & YES & YES & NO(2) & NO & 2644\\
$(451, 177)$ & 14 & $(28, 11)$ & 8 & 1 & YES & YES & NO(2) & 2557 & 2645\\
$(461, 74)$ & 17 & $(44, 7)$ & 10 & 1 & YES & YES & NO(2) & NO & 2646\\
$(463, 98)$ & 14 & $(2, 1)$ & 1 & 1 & YES & YES & NO(2) & -- & 2647\\
$(463, 98)$ & 14 & $(2, 1)$ & 1 & 1 & YES & YES & NO(2) & NO & 2648\\
$(463, 179)$ & 13 & $(313, 121)$ & 12 & 1 & YES & YES & YES & NO & 2649\\
$(465, 197)$ & 14 & $(3, 1)$ & 2 & 3 & YES & YES & YES & -- & 2650\\
$(465, 128)$ & 13 & $(65, 18)$ & 9 & 5 & YES & YES & YES & NO & 2651\\
$(465, 197)$ & 14 & $(465, 197)$ & 14 & 465 & YES & YES & YES & NO & 2652\\
$(473, 174)$ & 14 & $(3, 1)$ & 2 & 1 & YES & YES & NO(2) & -- & 2653\\
$(473, 125)$ & 14 & $(5, 2)$ & 3 & 1 & YES & YES & YES & NO & 2654\\
$(473, 140)$ & 14 & $(44, 13)$ & 8 & 11 & YES & YES & YES & NO & 2655\\
$(473, 174)$ & 14 & $(473, 174)$ & 14 & 473 & YES & YES & NO(2) & NO & 2656\\
$(476, 109)$ & 14 & $(40, 9)$ & 9 & 4 & YES & YES & YES & NO & 2657\\
$(476, 107)$ & 15 & $(49, 11)$ & 10 & 7 & YES & YES & NO(3) & NO & 2658\\
$(476, 109)$ & 14 & $(92, 21)$ & 10 & 4 & YES & YES & YES & NO & 2659\\
$(477, 187)$ & 14 & $(28, 11)$ & 8 & 1 & YES & YES & YES & NO & 2660\\
$(478, 201)$ & 14 & $(2, 1)$ & 1 & 2 & YES & YES & NO(2) & -- & 2661\\
$(480, 133)$ & 15 & $(8, 1)$ & 7 & 8 & YES & YES & NO(2) & NO & 2662\\
$(482, 177)$ & 13 & $(4, 1)$ & 3 & 2 & YES & YES & NO(2) & -- & 2663\\
$(482, 177)$ & 13 & $(30, 11)$ & 7 & 2 & YES & YES & NO(2) & NO & 2664\\
$(485, 188)$ & 13 & $(44, 17)$ & 8 & 1 & YES & YES & YES & NO & 2665\\
$(487, 101)$ & 15 & $(4, 1)$ & 3 & 1 & YES & YES & NO(3) & NO & 2666\\
$(490, 187)$ & 13 & $(5, 2)$ & 3 & 5 & YES & YES & YES & -- & 2667\\
$(490, 187)$ & 13 & $(18, 7)$ & 6 & 2 & YES & YES & YES & NO & 2668\\
$(497, 107)$ & 15 & $(23, 5)$ & 7 & 1 & YES & YES & YES & NO & 2669\\
$(498, 209)$ & 13 & $(5, 2)$ & 3 & 1 & YES & YES & YES & -- & 2670\\
$(502, 219)$ & 14 & $(353, 154)$ & 13 & 1 & YES & YES & YES & NO & 2671\\
$(503, 113)$ & 15 & $(2, 1)$ & 1 & 1 & YES & YES & NO(3) & NO & 2672\\
$(503, 132)$ & 15 & $(2, 1)$ & 1 & 1 & YES & YES & NO(2) & -- & 2673\\
$(503, 219)$ & 14 & $(4, 1)$ & 3 & 1 & YES & YES & YES & NO & 2674\\
$(503, 132)$ & 15 & $(7, 2)$ & 4 & 1 & YES & YES & NO(2) & NO & 2675\\
$(507, 140)$ & 14 & $(3, 1)$ & 2 & 3 & YES & YES & YES & -- & 2676\\
$(507, 140)$ & 14 & $(5, 2)$ & 3 & 1 & YES & YES & YES & -- & 2677\\
$(507, 140)$ & 14 & $(76, 21)$ & 9 & 1 & YES & YES & YES & NO & 2678\\
$(514, 181)$ & 18 & $(2, 1)$ & 1 & 2 & YES & YES & NO(2) & NO & 2679\\
$(514, 227)$ & 14 & $(43, 19)$ & 9 & 1 & YES & YES & YES & 2642 & 2680\\
$(517, 142)$ & 14 & $(18, 5)$ & 6 & 1 & YES & YES & YES & NO & 2681\\
$(517, 137)$ & 14 & $(117, 31)$ & 11 & 1 & YES & YES & YES & 2636 & 2682\\
$(517, 142)$ & 14 & $(131, 36)$ & 11 & 1 & YES & YES & YES & 2750 & 2683\\
$(521, 119)$ & 15 & $(35, 8)$ & 8 & 1 & YES & YES & NO(3) & NO & 2684\\
$(537, 164)$ & 15 & $(2, 1)$ & 1 & 1 & YES & YES & NO(2) & -- & 2685\\
$(539, 123)$ & 14 & $(53, 12)$ & 9 & 1 & YES & YES & YES & NO & 2686\\
$(539, 148)$ & 15 & $(142, 39)$ & 11 & 1 & YES & YES & YES & 2587 & 2687\\
$(551, 240)$ & 14 & $(3, 1)$ & 2 & 1 & YES & YES & YES & NO & 2688\\
$(552, 199)$ & 14 & $(319, 115)$ & 13 & 1 & YES & YES & YES & NO & 2689\\
$(557, 243)$ & 14 & $(3, 1)$ & 2 & 1 & YES & YES & YES & NO & 2690\\
$(557, 243)$ & 14 & $(353, 154)$ & 13 & 1 & YES & YES & YES & NO & 2691\\
$(559, 165)$ & 14 & $(5, 2)$ & 3 & 1 & YES & YES & YES & -- & 2692\\
$(559, 165)$ & 14 & $(11, 3)$ & 5 & 1 & YES & YES & YES & NO & 2693\\
$(563, 158)$ & 15 & $(7, 2)$ & 4 & 1 & YES & YES & NO(2) & NO & 2694\\
$(583, 226)$ & 14 & $(3, 1)$ & 2 & 1 & YES & YES & NO(2) & -- & 2695\\
$(583, 173)$ & 14 & $(5, 2)$ & 3 & 1 & YES & YES & YES & -- & 2696\\
$(583, 173)$ & 14 & $(5, 2)$ & 3 & 1 & YES & YES & YES & NO & 2697\\
$(587, 256)$ & 14 & $(5, 1)$ & 4 & 1 & YES & YES & YES & NO & 2698\\
$(590, 229)$ & 14 & $(13, 5)$ & 5 & 1 & YES & YES & YES & NO & 2699\\
$(596, 165)$ & 14 & $(25, 7)$ & 7 & 1 & YES & YES & YES & NO & 2700\\
$(606, 251)$ & 14 & $(4, 1)$ & 3 & 2 & YES & YES & YES & -- & 2701\\
$(606, 251)$ & 14 & $(268, 111)$ & 12 & 2 & YES & YES & YES & 2732 & 2702\\
$(606, 251)$ & 14 & $(437, 181)$ & 13 & 1 & YES & YES & YES & NO & 2703\\
$(608, 235)$ & 14 & $(445, 172)$ & 13 & 1 & YES & YES & YES & NO & 2704\\
$(611, 256)$ & 14 & $(105, 44)$ & 10 & 1 & YES & YES & YES & NO & 2705\\
$(615, 227)$ & 14 & $(19, 7)$ & 6 & 1 & YES & YES & NO(2) & NO & 2706\\
$(623, 241)$ & 14 & $(243, 94)$ & 12 & 1 & YES & YES & YES & NO & 2707\\
$(628, 265)$ & 14 & $(3, 1)$ & 2 & 1 & YES & YES & YES & -- & 2708\\
$(628, 265)$ & 14 & $(282, 119)$ & 12 & 2 & YES & YES & YES & 2740 & 2709\\
$(634, 241)$ & 14 & $(292, 111)$ & 12 & 2 & YES & YES & YES & 2749 & 2710\\
$(635, 132)$ & 16 & $(3, 1)$ & 2 & 1 & YES & YES & NO(2) & NO & 2711\\
$(637, 263)$ & 14 & $(3, 1)$ & 2 & 1 & YES & YES & NO(2) & -- & 2712\\
$(649, 251)$ & 14 & $(2, 1)$ & 1 & 1 & YES & YES & NO(2) & -- & 2713\\
$(649, 251)$ & 14 & $(5, 1)$ & 4 & 1 & YES & YES & YES & NO & 2714\\
$(658, 241)$ & 14 & $(3, 1)$ & 2 & 1 & YES & YES & YES & -- & 2715\\
$(658, 241)$ & 14 & $(5, 2)$ & 3 & 1 & YES & YES & YES & NO & 2716\\
$(663, 275)$ & 15 & $(6, 1)$ & 5 & 3 & YES & YES & NO(2) & -- & 2717\\
$(675, 154)$ & 15 & $(31, 7)$ & 8 & 1 & YES & YES & YES & NO & 2718\\
$(680, 263)$ & 14 & $(3, 1)$ & 2 & 1 & YES & YES & YES & -- & 2719\\
$(680, 263)$ & 14 & $(3, 1)$ & 2 & 1 & YES & YES & NO(2) & NO & 2720\\
$(680, 287)$ & 14 & $(263, 111)$ & 12 & 1 & YES & YES & YES & NO & 2721\\
$(681, 154)$ & 15 & $(75, 17)$ & 10 & 3 & YES & YES & YES & NO & 2722\\
$(683, 251)$ & 14 & $(166, 61)$ & 11 & 1 & YES & YES & YES & NO & 2723\\
$(695, 288)$ & 14 & $(3, 1)$ & 2 & 1 & YES & YES & YES & -- & 2724\\
$(695, 202)$ & 15 & $(7, 2)$ & 4 & 1 & YES & YES & YES & NO & 2725\\
$(697, 266)$ & 14 & $(2, 1)$ & 1 & 1 & YES & YES & YES & -- & 2726\\
$(697, 288)$ & 14 & $(3, 1)$ & 2 & 1 & YES & YES & NO(2) & -- & 2727\\
$(697, 266)$ & 14 & $(5, 2)$ & 3 & 1 & YES & YES & YES & NO & 2728\\
$(697, 266)$ & 14 & $(131, 50)$ & 10 & 1 & YES & YES & NO(2) & NO & 2729\\
$(703, 267)$ & 14 & $(129, 49)$ & 10 & 1 & YES & YES & NO(2) & NO & 2730\\
$(705, 268)$ & 14 & $(2, 1)$ & 1 & 1 & YES & YES & NO(2) & NO & 2731\\
$(705, 292)$ & 14 & $(169, 70)$ & 11 & 1 & YES & YES & YES & 2702 & 2732\\
$(705, 268)$ & 14 & $(413, 157)$ & 13 & 1 & YES & YES & YES & NO & 2733\\
$(705, 268)$ & 14 & $(705, 268)$ & 14 & 705 & YES & YES & YES & NO & 2734\\
$(707, 274)$ & 14 & $(13, 5)$ & 5 & 1 & YES & YES & YES & NO & 2735\\
$(715, 277)$ & 14 & $(13, 5)$ & 5 & 13 & YES & YES & YES & NO & 2736\\
$(715, 277)$ & 14 & $(302, 117)$ & 12 & 1 & YES & YES & YES & NO & 2737\\
$(722, 113)$ & 18 & $(2, 1)$ & 1 & 2 & YES & YES & NO(2) & -- & 2738\\
$(722, 113)$ & 18 & $(8, 1)$ & 7 & 2 & YES & YES & NO(2) & NO & 2739\\
$(737, 311)$ & 14 & $(173, 73)$ & 11 & 1 & YES & YES & YES & 2709 & 2740\\
$(745, 313)$ & 14 & $(5, 1)$ & 4 & 5 & YES & YES & YES & -- & 2741\\
$(745, 288)$ & 14 & $(313, 121)$ & 12 & 1 & YES & YES & YES & NO & 2742\\
$(747, 169)$ & 15 & $(75, 17)$ & 10 & 3 & YES & YES & YES & NO & 2743\\
$(751, 132)$ & 17 & $(3, 1)$ & 2 & 1 & YES & YES & NO(2) & NO & 2744\\
$(751, 132)$ & 17 & $(4, 1)$ & 3 & 1 & YES & YES & NO(2) & NO & 2745\\
$(752, 287)$ & 14 & $(3, 1)$ & 2 & 1 & YES & YES & YES & -- & 2746\\
$(755, 292)$ & 14 & $(2, 1)$ & 1 & 1 & YES & YES & YES & -- & 2747\\
$(755, 312)$ & 14 & $(12, 5)$ & 5 & 1 & YES & YES & YES & NO & 2748\\
$(755, 287)$ & 14 & $(171, 65)$ & 11 & 1 & YES & YES & YES & 2710 & 2749\\
$(757, 208)$ & 15 & $(51, 14)$ & 9 & 1 & YES & YES & YES & 2683 & 2750\\
$(761, 223)$ & 15 & $(273, 80)$ & 13 & 1 & YES & YES & YES & NO & 2751\\
$(765, 317)$ & 14 & $(7, 3)$ & 4 & 1 & YES & YES & YES & NO & 2752\\
$(772, 163)$ & 16 & $(3, 1)$ & 2 & 1 & YES & YES & YES & NO & 2753\\
$(772, 163)$ & 16 & $(9, 2)$ & 5 & 1 & YES & YES & YES & NO & 2754\\
$(790, 231)$ & 15 & $(2, 1)$ & 1 & 2 & YES & YES & YES & NO & 2755\\
$(798, 143)$ & 16 & $(5, 2)$ & 3 & 1 & YES & YES & YES & -- & 2756\\
$(802, 235)$ & 15 & $(2, 1)$ & 1 & 2 & YES & YES & YES & NO & 2757\\
$(802, 215)$ & 15 & $(138, 37)$ & 11 & 2 & YES & YES & YES & NO & 2758\\
$(805, 312)$ & 14 & $(4, 1)$ & 3 & 1 & YES & YES & NO(2) & NO & 2759\\
$(809, 226)$ & 15 & $(3, 1)$ & 2 & 1 & YES & YES & YES & -- & 2760\\
$(811, 219)$ & 15 & $(4, 1)$ & 3 & 1 & YES & YES & YES & NO & 2761\\
$(835, 148)$ & 17 & $(28, 5)$ & 8 & 1 & YES & YES & YES & NO & 2762\\
$(843, 322)$ & 14 & $(3, 1)$ & 2 & 3 & YES & YES & YES & -- & 2763\\
$(843, 322)$ & 14 & $(3, 1)$ & 2 & 3 & YES & YES & YES & NO & 2764\\
$(880, 199)$ & 16 & $(199, 45)$ & 12 & 1 & YES & YES & YES & NO & 2765\\
$(883, 243)$ & 15 & $(3, 1)$ & 2 & 1 & YES & YES & YES & NO & 2766\\
$(893, 246)$ & 15 & $(5, 2)$ & 3 & 1 & YES & YES & YES & -- & 2767\\
$(893, 246)$ & 15 & $(7, 2)$ & 4 & 1 & YES & YES & YES & NO & 2768\\
$(893, 246)$ & 15 & $(236, 65)$ & 12 & 1 & YES & YES & YES & NO & 2769\\
$(901, 264)$ & 15 & $(372, 109)$ & 13 & 1 & YES & YES & YES & NO & 2770\\
$(907, 265)$ & 15 & $(2, 1)$ & 1 & 1 & YES & YES & YES & NO & 2771\\
$(908, 207)$ & 16 & $(715, 163)$ & 15 & 1 & YES & YES & YES & NO & 2772\\
$(911, 199)$ & 16 & $(206, 45)$ & 12 & 1 & YES & YES & YES & NO & 2773\\
$(923, 255)$ & 15 & $(18, 5)$ & 6 & 1 & YES & YES & YES & NO & 2774\\
$(927, 256)$ & 15 & $(2, 1)$ & 1 & 1 & YES & YES & YES & NO & 2775\\
$(937, 261)$ & 15 & $(2, 1)$ & 1 & 1 & YES & YES & YES & NO & 2776\\
$(957, 284)$ & 15 & $(17, 5)$ & 6 & 1 & YES & YES & NO(2) & NO & 2777\\
$(979, 222)$ & 16 & $(3, 1)$ & 2 & 1 & YES & YES & YES & NO & 2778\\
$(994, 227)$ & 16 & $(3, 1)$ & 2 & 1 & YES & YES & YES & NO & 2779\\
$(1013, 299)$ & 15 & $(5, 1)$ & 4 & 1 & YES & YES & NO(2) & -- & 2780\\
$(1027, 305)$ & 15 & $(4, 1)$ & 3 & 1 & YES & YES & NO(2) & -- & 2781\\
$(1027, 305)$ & 15 & $(17, 5)$ & 6 & 1 & YES & YES & YES & NO & 2782\\
$(1048, 237)$ & 16 & $(199, 45)$ & 12 & 1 & YES & YES & YES & NO & 2783\\
$(1085, 237)$ & 16 & $(4, 1)$ & 3 & 1 & YES & YES & YES & NO & 2784\\
$(1085, 237)$ & 16 & $(23, 5)$ & 7 & 1 & YES & YES & YES & NO & 2785\\
$(1117, 432)$ & 15 & $(287, 111)$ & 12 & 1 & YES & YES & YES & NO & 2786\\
$(1121, 254)$ & 16 & $(2, 1)$ & 1 & 1 & YES & YES & YES & -- & 2787\\
$(1420, 393)$ & 16 & $(271, 75)$ & 12 & 1 & YES & YES & YES & NO & 2788\\
$(a; 1, 0, 0; 13)$ & 5 & $(206, 47)$ & 12 & 1 & YES & YES & YES & -- & 2789\\
$(a; 1, 1, 0; 19)$ & 6 & $(82, 31)$ & 10 & 1 & YES & YES & NO(2) & -- & 2790\\
$(a; 2, 0, 0; 17)$ & 6 & $(73, 31)$ & 10 & 1 & YES & YES & NO(2) & -- & 2791\\
$(a; 3, 0, 0; 7)$ & 7 & $(18, 7)$ & 6 & 1 & YES & YES & NO(3) & -- & 2792\\
$(a; 3, 0, 0; 7)$ & 7 & $(24, 7)$ & 7 & 1 & YES & YES & NO(3) & -- & 2793\\
$(a; 4, 0, 1; 37)$ & 9 & $(11, 4)$ & 5 & 1 & YES & YES & YES & -- & 2794\\
$(b; 0, 0, 0; 14)$ & 5 & $(84, 37)$ & 10 & 14 & YES & YES & NO(2) & -- & 2795\\
$(b; 0, 0, 0; 14)$ & 5 & $(101, 37)$ & 10 & 1 & YES & YES & NO(2) & -- & 2796\\
$(b; 0, 0, 1; 4)$ & 6 & $(140, 41)$ & 11 & 4 & YES & YES & YES & -- & 2797\\
$(b; 0, 1, 0; 19)$ & 6 & $(44, 17)$ & 8 & 1 & YES & YES & NO(2) & -- & 2798\\
$(b; 0, 1, 0; 19)$ & 6 & $(56, 23)$ & 9 & 1 & YES & YES & NO(2) & -- & 2799\\
$(b; 0, 1, 0; 19)$ & 6 & $(89, 27)$ & 10 & 1 & YES & YES & YES & -- & 2800\\
$(b; 0, 2, 0; 8)$ & 7 & $(12, 5)$ & 5 & 4 & YES & YES & NO(2) & -- & 2801\\
$(b; 1, 3, 3; 95)$ & 12 & $(5, 2)$ & 3 & 5 & YES & YES & NO(2) & -- & 2802\\
$(b; 2, 1, 0; 7)$ & 8 & $(33, 10)$ & 8 & 1 & YES & YES & NO(2) & -- & 2803\\
$(c; 0, 0, 0; 4)$ & 4 & $(50, 23)$ & 10 & 2 & YES & YES & YES & -- & 2804\\
$(c; 0, 0, 0; 4)$ & 4 & $(61, 22)$ & 9 & 1 & YES & YES & NO(2) & -- & 2805\\
$(c; 0, 0, 0; 4)$ & 4 & $(95, 39)$ & 10 & 1 & YES & YES & NO(2) & -- & 2806\\
$(c; 0, 0, 0; 4)$ & 4 & $(97, 41)$ & 10 & 1 & YES & YES & NO(2) & -- & 2807\\
$(c; 0, 0, 0; 4)$ & 4 & $(301, 115)$ & 12 & 1 & YES & YES & YES & -- & 2808\\
$(c; 0, 1, 0; 11)$ & 5 & $(131, 47)$ & 11 & 1 & YES & YES & YES & -- & 2809\\
$(c; 0, 1, 0; 11)$ & 5 & $(165, 64)$ & 11 & 11 & YES & YES & YES & -- & 2810\\
$(c; 0, 1, 0; 11)$ & 5 & $(186, 71)$ & 11 & 1 & YES & YES & NO(2) & -- & 2811\\
$(c; 0, 1, 0; 11)$ & 5 & $(194, 75)$ & 11 & 1 & YES & YES & YES & -- & 2812\\
$(c; 0, 2, 1; 19)$ & 7 & $(53, 14)$ & 9 & 1 & YES & YES & YES & -- & 2813\\
$(c; 0, 2, 1; 19)$ & 7 & $(116, 25)$ & 11 & 1 & YES & YES & NO(2) & -- & 2814\\
$(d; 0, 0, 0; 5)$ & 5 & $(53, 19)$ & 9 & 1 & YES & YES & YES & -- & 2815\\
$(d; 0, 0, 0; 5)$ & 5 & $(165, 64)$ & 11 & 5 & YES & YES & YES & -- & 2816\\
$(d; 0, 0, 0; 5)$ & 5 & $(199, 76)$ & 11 & 1 & YES & YES & YES & -- & 2817\\
$(d; 0, 0, 0; 5)$ & 5 & $(203, 59)$ & 12 & 1 & YES & YES & YES & -- & 2818\\
$(d; 0, 0, 0; 5)$ & 5 & $(257, 76)$ & 12 & 1 & YES & YES & YES & -- & 2819\\
$(d; 0, 0, 1; 14)$ & 6 & $(60, 23)$ & 9 & 2 & YES & YES & NO(2) & -- & 2820\\
$(d; 0, 0, 1; 14)$ & 6 & $(79, 23)$ & 10 & 1 & YES & YES & NO(2) & -- & 2821\\
$(d; 0, 0, 1; 14)$ & 6 & $(94, 41)$ & 10 & 2 & YES & YES & YES & -- & 2822\\
$(d; 0, 0, 1; 14)$ & 6 & $(119, 46)$ & 10 & 7 & YES & YES & YES & -- & 2823\\
$(d; 0, 2, 1; 20)$ & 8 & $(33, 10)$ & 8 & 1 & YES & YES & NO(2) & -- & 2824\\
$(e; 0, 1, 0; 5)$ & 6 & $(105, 31)$ & 10 & 5 & YES & YES & YES & -- & 2825\\
$(e; 1, 3, 0; 33)$ & 9 & $(23, 5)$ & 7 & 1 & YES & YES & YES & -- & 2826\\
$(e; 4, 3, 0; 69)$ & 12 & $(5, 2)$ & 3 & 1 & YES & YES & YES & -- & 2827\\
$(f; 0, 0, 0; 6)$ & 4 & $(320, 57)$ & 14 & 2 & YES & YES & NO(2) & -- & 2828\\
$(g; 0, 2, 0; 29)$ & 8 & $(23, 10)$ & 7 & 1 & YES & YES & NO(2) & -- & 2829\\
$(g; 1, 0, 2; 24)$ & 9 & $(12, 5)$ & 5 & 12 & YES & YES & NO(2) & -- & 2830\\
$(g; 1, 0, 2; 24)$ & 9 & $(16, 7)$ & 6 & 8 & YES & YES & YES & -- & 2831\\
$(g; 1, 0, 2; 24)$ & 9 & $(22, 5)$ & 7 & 2 & YES & YES & NO(2) & -- & 2832\\
$(g; 2, 1, 3; 99)$ & 12 & $(4, 1)$ & 3 & 1 & YES & YES & YES & -- & 2833\\
$(g; 2, 3, 1; 19)$ & 12 & $(3, 1)$ & 2 & 1 & YES & YES & YES & -- & 2834\\
$(h; 0, 0, 0; 6)$ & 5 & $(24, 11)$ & 8 & 6 & YES & YES & YES & -- & 2835\\
$(i; 0, 0, 0; 9)$ & 5 & $(108, 29)$ & 10 & 9 & YES & YES & NO(2) & -- & 2836\\
$(i; 0, 1, 0; 12)$ & 6 & $(65, 19)$ & 9 & 1 & YES & YES & NO(2) & -- & 2837\\
$(j; 0, 1, 0; 10)$ & 6 & $(106, 45)$ & 11 & 2 & YES & YES & NO(2) & -- & 2838
\end{longtable}
\subsection{2 chains, $K^2 = 5$}
\begin{longtable}{|c|c|c|c|c|c|c|c|c|c|}
\hline
\multicolumn{10}{|c|}{2 chains, $K^2 = 5$}\\
\hline
$(n,a)$ & Length & $(n,a)$ & Length & GCD & Nef & $\mathbb Q$-ef & Obstruction 0 & WH & Index\\
\hline
\endfirsthead

\hline
$(n,a)$ & Length & $(n,a)$ & Length & GCD & Nef & $\mathbb Q$-ef & Obstruction 0 & WH & Index\\
\hline
\endhead
\hline
\endfoot

$(79, 24)$ & 10 & $(64, 27)$ & 9 & 1 & YES & YES & NO(3) & -- & 2839\\
$(251, 78)$ & 13 & $(79, 24)$ & 10 & 1 & YES & YES & NO(3) & NO & 2840\\
$(707, 254)$ & 14 & $(5, 2)$ & 3 & 1 & YES & YES & NO(3) & -- & 2841\\
$(707, 254)$ & 14 & $(142, 51)$ & 11 & 1 & YES & YES & NO(3) & NO & 2842\\
$(1192, 503)$ & 15 & $(64, 27)$ & 9 & 8 & YES & YES & NO(3) & NO & 2843\\
$(1233, 277)$ & 17 & $(129, 29)$ & 12 & 3 & YES & YES & NO(3) & NO & 2844\\
$(e; 1, 1, 0; 23)$ & 7 & $(101, 37)$ & 10 & 1 & YES & YES & NO(3) & -- & 2845\\
$(g; 0, 0, 0; 19)$ & 6 & $(119, 44)$ & 10 & 1 & YES & YES & NO(3) & -- & 2846\\
$(g; 0, 0, 1; 26)$ & 7 & $(106, 41)$ & 10 & 2 & YES & YES & NO(3) & -- & 2847\\
$(i; 0, 0, 0; 9)$ & 5 & $(351, 80)$ & 13 & 9 & YES & YES & NO(3) & -- & 2848
\end{longtable}



%%%%%%%%%%%%%%%%%%%%%%%%%%%%%%%%%%%%%%%%%%%
\section{$I_6 + I_3 + I_2 + I_1$}

Base curves:
\begin{itemize}
  \item $L_1 = x+z$.
  \item $L_2 = x+y$.
  \item $L_3 = y+z$.
  \item $x$.
  \item $y$.
  \item $z$.
  \item $C = xy + xz + yz$
  \item $L = x + y + z$
\end{itemize}
Fibration given by pencil
\[F_\lambda = L_1 L_2 L_3 + \lambda xyz\]

Nine exceptionals are as follows:
\begin{itemize}
  \item $E_1$ - $E_2$ at $z \cap x \cap L_1 = [0,1,0]$.
  \item $E_3$ - $E_4$ at $x \cap y \cap L_2 = [0,0,1]$.
  \item $E_5$ - $E_6$ at $y \cap z \cap L_3 = [1,0,0]$.
  \item $E_7$ at $y \cap L_1 = [-1,0,1]$.
  \item $E_8$ at $x \cap L_3 = [0,-1,1]$.
  \item $E_9$ at $z \cap L_2 = [-1,1,0]$.
\end{itemize}
Singular fibers are as follows:
\begin{itemize}
  \item $\lambda = \infty$: $I_6$ fiber given by $z$, $E_1$, $x$, $E_3$, $y$, $E_5$ in order.
  \item $\lambda = 0$: $I_3$ fiber given by $L_1$, $L_2$, $L_3$.
  \item $\lambda = 1$: $I_2$ fiber given by $C$ and $L$.
  \item $\lambda = -8$: $I_1$ fiber called $F_1$ with node at $[1,1,1]$.
\end{itemize}
Special curves:
\begin{itemize}
  \item $S = x + y - 2z$, double section through $[-1,1,0]$ and $[1,1,1]$
\end{itemize}
Input:
%\lstinputlisting[language=config]{Tests/6321.txt}
Result:
%\input{summary/6321_new}



%%%%%%%%%%%%%%%%%%%%%%%%%%%%%%%%%%%%%%%%%%%
\section{$2I_5 + 2I_1$}

(3886 examples from 37715968 tests)

Base curves:
\begin{itemize}
  \item $x$.
  \item $y$.
  \item $z$.
  \item $A = x + z$.
  \item $B = x + y + z$.
  \item $C = x+y$.
\end{itemize}
Fibration given by pencil
\[F_\lambda = ABC + \lambda xyz\]

Nine exceptionals are as follows:
\begin{itemize}
  \item $E_1$ - $E_2$ at $y \cap A \cap B = [-1,0,1]$.
  \item $E_3$ - $E_4$ at $x \cap y \cap C = [0,0,1]$.
  \item $E_5$ - $E_6$ at $z \cap B \cap C = [-1,1,0]$.
  \item $E_7$ - $E_8$ at $x \cap z \cap A = [0,1,0]$.
  \item $E_9$ at $x \cap B = [0,-1,1]$.
\end{itemize}
Singular fibers are as follows:
\begin{itemize}
  \item $\lambda = \infty$: $I_5$ fiber given by $x$, $E_3$, $y$, $z$, $E_7$ in order.
  \item $\lambda = 0$: $I_5$ fiber given by $A$, $C$, $E_5$, $B$, $E_1$ in order.
  \item $\lambda = \dfrac{-11 + 5\sqrt{5}}{2}$: $I_1$ fiber called $F_1$ with node at $[-1-\sqrt{5},2,2]$.
  \item $\lambda = \dfrac{-11 - 5\sqrt{5}}{2}$: $I_1$ fiber called $F_2$ with node at $[-1+\sqrt{5},2,2]$.
\end{itemize}
Special curves:
\begin{itemize}
  \item $S = 2x - (-1 -\sqrt{5})y$, double section through $[0,0,1]$ and $[-1-\sqrt{5},2,2]$.
  \item $R = y - z$, triple section through $y \cap z$, $A \cap C$ and both nodes of $I_1$'s.
  \item $Q = x^2 + x - y$, triple section through $y \cap A \cap B$ (tangent with $B$), $x \cap y \cap C$, $x \cap z \cap A$ (tangent with $z$), and both nodes of $I_1$'s.
  \item $T = y + z$, double section through $[1,0,0]$ and $[0,-1,1]$.
\end{itemize}

Input:
\lstinputlisting[language=config]{../Tests/5511.txt}
Result:
%\usepackage{longtable}
\subsection{1 chain, $K^2 = 1$}
\begin{longtable}{|c|c|c|c|c|c|c|c|}
\hline
\multicolumn{8}{|c|}{1 chain, $K^2 = 1$}\\
\hline
$(n,a)$ & Len & Nef & $\mathbb Q$-ef & Obs 0 & $\overline c_1^2 / \overline c_2$ & $(P,K)$ & Index\\
\hline
\endfirsthead

\hline
$(n,a)$ & Len & Nef & $\mathbb Q$-ef & Obs 0 & $\overline c_1^2 / \overline c_2$ & $(P,K)$ & Index\\
\hline
\endhead
\hline
\endfoot

$(13,4)$ & 6 & YES & YES & YES & $0.67$ & $(3,0)$ & 1\\
$(13,3)$ & 6 & YES & YES & YES & $0.64$ & $(1,1)$ & 2\\
$(16,5)$ & 7 & YES & YES & YES & $0.55$ & $(1,1)$ & 3\\
$(16,7)$ & 6 & YES & YES & YES & $0.83$ & $(1,1)$ & 4\\
$(17,7)$ & 6 & YES & YES & YES & $0.64$ & $(1,1)$ & 5\\
$(19,5)$ & 7 & YES & YES & YES & $0.64$ & $(1,1)$ & 6\\
$(19,8)$ & 6 & YES & YES & YES & $0.64$ & $(1,1)$ & 7\\
$(21,5)$ & 8 & YES & YES & YES & $0.85$ & $(1,1)$ & 8\\
$(24,5)$ & 8 & YES & YES & YES & $0.75$ & $(1,1)$ & 9\\
$(26,7)$ & 7 & YES & YES & YES & $0.55$ & $(1,1)$ & 10\\
$(a;1,0,0;13)$ & 5 & YES & YES & YES & $0.64$ & $(1,1)$ & 11
\end{longtable}
\subsection{1 chain, $K^2 = 2$}
\begin{longtable}{|c|c|c|c|c|c|c|c|}
\hline
\multicolumn{8}{|c|}{1 chain, $K^2 = 2$}\\
\hline
$(n,a)$ & Len & Nef & $\mathbb Q$-ef & Obs 0 & $\overline c_1^2 / \overline c_2$ & $(P,K)$ & Index\\
\hline
\endfirsthead

\hline
$(n,a)$ & Len & Nef & $\mathbb Q$-ef & Obs 0 & $\overline c_1^2 / \overline c_2$ & $(P,K)$ & Index\\
\hline
\endhead
\hline
\endfoot

$(34,13)$ & 7 & YES & YES & YES & $1.08$ & $(1,2)$ & 12\\
$(37,17)$ & 9 & YES & YES & YES & $0.89$ & $(1,2)$ & 13\\
$(37,13)$ & 9 & YES & YES & YES & $1.10$ & $(1,2)$ & 14\\
$(39,16)$ & 8 & YES & YES & YES & $1.09$ & $(1,2)$ & 15\\
$(39,14)$ & 8 & YES & YES & YES & $0.78$ & $(3,1)$ & 16\\
$(41,13)$ & 10 & YES & YES & YES & $0.89$ & $(1,2)$ & 17\\
$(41,17)$ & 8 & YES & YES & YES & $0.90$ & $(1,2)$ & 18\\
$(41,16)$ & 8 & YES & YES & YES & $1.09$ & $(1,2)$ & 19\\
$(41,15)$ & 8 & YES & YES & YES & $1.00$ & $(1,2)$ & 20\\
$(42,19)$ & 9 & YES & YES & YES & $0.89$ & $(3,1)$ & 21\\
$(43,19)$ & 9 & YES & YES & YES & $1.10$ & $(1,2)$ & 22\\
$(44,17)$ & 8 & YES & YES & YES & $1.08$ & $(1,2)$ & 23\\
$(45,19)$ & 8 & YES & YES & YES & $1.00$ & $(1,2)$ & 24\\
$(46,19)$ & 8 & YES & YES & YES & $0.90$ & $(1,2)$ & 25\\
$(48,17)$ & 9 & YES & YES & YES & $0.78$ & $(3,1)$ & 26\\
$(49,13)$ & 9 & YES & YES & YES & $0.80$ & $(3,1)$ & 27\\
$(49,15)$ & 9 & YES & YES & YES & $1.09$ & $(1,2)$ & 28\\
$(49,18)$ & 8 & YES & YES & YES & $1.08$ & $(1,2)$ & 29\\
$(49,19)$ & 8 & YES & YES & YES & $1.00$ & $(1,2)$ & 30\\
$(49,20)$ & 9 & YES & YES & YES & $1.09$ & $(1,2)$ & 31\\
$(49,22)$ & 9 & YES & YES & YES & $0.78$ & $(1,2)$ & 32\\
$(51,20)$ & 9 & YES & YES & YES & $1.00$ & $(1,2)$ & 33\\
$(51,23)$ & 9 & YES & YES & YES & $0.78$ & $(1,2)$ & 34\\
$(52,19)$ & 9 & YES & YES & YES & $1.00$ & $(1,2)$ & 35\\
$(53,19)$ & 9 & YES & YES & YES & $1.00$ & $(1,2)$ & 36\\
$(55,16)$ & 9 & YES & YES & YES & $0.80$ & $(1,2)$ & 37\\
$(59,23)$ & 9 & YES & YES & YES & $0.78$ & $(1,2)$ & 38\\
$(62,23)$ & 9 & YES & YES & YES & $0.89$ & $(1,2)$ & 39\\
$(64,23)$ & 9 & YES & YES & YES & $0.67$ & $(3,1)$ & 40\\
$(65,24)$ & 9 & YES & YES & YES & $0.90$ & $(1,2)$ & 41\\
$(67,26)$ & 9 & YES & YES & YES & $1.00$ & $(1,2)$ & 42\\
$(71,15)$ & 10 & YES & YES & YES & $1.15$ & $(1,2)$ & 43\\
$(71,27)$ & 9 & YES & YES & YES & $1.00$ & $(1,2)$ & 44\\
$(72,13)$ & 12 & YES & YES & YES & $0.67$ & $(3,1)$ & 45\\
$(75,22)$ & 10 & YES & YES & YES & $1.09$ & $(1,2)$ & 46\\
$(76,13)$ & 12 & YES & YES & YES & $0.78$ & $(1,2)$ & 47\\
$(76,29)$ & 9 & YES & YES & YES & $1.09$ & $(1,2)$ & 48\\
$(79,14)$ & 11 & YES & YES & YES & $1.08$ & $(1,2)$ & 49\\
$(79,22)$ & 10 & YES & YES & YES & $1.18$ & $(1,2)$ & 50\\
$(79,30)$ & 9 & YES & YES & YES & $1.09$ & $(1,2)$ & 51\\
$(81,31)$ & 9 & YES & YES & YES & $1.25$ & $(1,2)$ & 52\\
$(85,33)$ & 10 & YES & YES & YES & $1.00$ & $(3,1)$ & 53\\
$(89,17)$ & 12 & YES & YES & YES & $1.00$ & $(1,2)$ & 54\\
$(92,35)$ & 10 & YES & YES & YES & $1.00$ & $(1,2)$ & 55\\
$(95,36)$ & 10 & YES & YES & YES & $0.78$ & $(3,1)$ & 56\\
$(99,17)$ & 12 & YES & YES & YES & $0.78$ & $(1,2)$ & 57\\
$(101,16)$ & 13 & YES & YES & YES & $0.90$ & $(1,2)$ & 58\\
$(105,31)$ & 10 & YES & YES & YES & $1.09$ & $(1,2)$ & 59\\
$(a;3,0,1;31)$ & 8 & YES & YES & YES & $1.09$ & $(1,2)$ & 60\\
$(b;0,3,0;29)$ & 8 & YES & YES & YES & $1.00$ & $(1,2)$ & 61\\
$(b;1,1,0;27)$ & 7 & YES & YES & YES & $1.00$ & $(1,2)$ & 62\\
$(c;0,2,2;6)$ & 8 & YES & YES & YES & $1.00$ & $(1,2)$ & 63\\
$(c;0,3,1;23)$ & 8 & YES & YES & YES & $0.80$ & $(3,1)$ & 64\\
$(c;0,3,2;29)$ & 9 & YES & YES & YES & $1.00$ & $(1,2)$ & 65\\
$(d;0,1,2;11)$ & 8 & YES & YES & YES & $1.15$ & $(1,2)$ & 66\\
$(d;0,1,3;27)$ & 9 & YES & YES & YES & $0.90$ & $(1,2)$ & 67\\
$(d;0,2,2;13)$ & 9 & YES & YES & YES & $1.00$ & $(1,2)$ & 68\\
$(e;0,3,0;7)$ & 8 & YES & YES & YES & $1.00$ & $(1,2)$ & 69\\
$(i;0,3,0;18)$ & 8 & YES & YES & YES & $0.67$ & $(3,1)$ & 70
\end{longtable}
\subsection{1 chain, $K^2 = 3$}
\begin{longtable}{|c|c|c|c|c|c|c|c|}
\hline
\multicolumn{8}{|c|}{1 chain, $K^2 = 3$}\\
\hline
$(n,a)$ & Len & Nef & $\mathbb Q$-ef & Obs 0 & $\overline c_1^2 / \overline c_2$ & $(P,K)$ & Index\\
\hline
\endfirsthead

\hline
$(n,a)$ & Len & Nef & $\mathbb Q$-ef & Obs 0 & $\overline c_1^2 / \overline c_2$ & $(P,K)$ & Index\\
\hline
\endhead
\hline
\endfoot

$(67,26)$ & 9 & YES & YES & NO(2) & $1.42$ & $(1,3)$ & 71\\
$(71,21)$ & 9 & YES & YES & YES & $1.36$ & $(1,3)$ & 72\\
$(73,33)$ & 10 & YES & YES & NO(3) & $1.11$ & $(1,3)$ & 73\\
$(79,29)$ & 9 & YES & YES & NO(2) & $1.27$ & $(3,2)$ & 74\\
$(82,37)$ & 10 & YES & YES & NO(3) & $1.11$ & $(1,3)$ & 75\\
$(83,34)$ & 10 & YES & YES & NO(2) & $1.36$ & $(1,3)$ & 76\\
$(85,36)$ & 10 & YES & YES & NO(2) & $1.00$ & $(5,1)$ & 77\\
$(89,26)$ & 10 & YES & YES & YES & $1.33$ & $(1,3)$ & 78\\
$(91,27)$ & 10 & YES & YES & NO(2) & $1.27$ & $(3,2)$ & 79\\
$(93,26)$ & 10 & YES & YES & YES & $1.33$ & $(1,3)$ & 80\\
$(94,39)$ & 10 & YES & YES & NO(2) & $1.30$ & $(3,2)$ & 81\\
$(97,41)$ & 10 & YES & YES & NO(2) & $1.36$ & $(3,2)$ & 82\\
$(97,36)$ & 10 & YES & YES & YES & $1.40$ & $(1,3)$ & 83\\
$(98,41)$ & 10 & YES & YES & YES & $1.50$ & $(1,3)$ & 84\\
$(100,37)$ & 10 & YES & YES & YES & $1.33$ & $(1,3)$ & 85\\
$(100,41)$ & 10 & YES & YES & NO(2) & $1.27$ & $(1,3)$ & 86\\
$(100,31)$ & 11 & YES & YES & YES & $1.12$ & $(3,2)$ & 87\\
$(101,37)$ & 10 & YES & YES & NO(2) & $1.27$ & $(3,2)$ & 88\\
$(103,47)$ & 12 & YES & YES & YES & $1.25$ & $(1,3)$ & 89\\
$(107,41)$ & 10 & YES & YES & YES & $1.40$ & $(1,3)$ & 90\\
$(108,41)$ & 10 & YES & YES & YES & $1.33$ & $(1,3)$ & 91\\
$(111,46)$ & 10 & YES & YES & YES & $1.40$ & $(1,3)$ & 92\\
$(113,42)$ & 11 & YES & YES & YES & $1.40$ & $(1,3)$ & 93\\
$(113,49)$ & 11 & YES & YES & YES & $1.40$ & $(1,3)$ & 94\\
$(116,51)$ & 11 & YES & YES & YES & $1.33$ & $(1,3)$ & 95\\
$(128,49)$ & 10 & YES & YES & YES & $1.40$ & $(1,3)$ & 96\\
$(130,47)$ & 11 & YES & YES & YES & $1.40$ & $(1,3)$ & 97\\
$(132,47)$ & 12 & YES & YES & YES & $1.12$ & $(3,2)$ & 98\\
$(133,48)$ & 11 & YES & YES & YES & $1.45$ & $(1,3)$ & 99\\
$(147,43)$ & 11 & YES & YES & YES & $1.60$ & $(1,3)$ & 100\\
$(151,32)$ & 12 & YES & YES & YES & $1.30$ & $(1,3)$ & 101\\
$(151,62)$ & 11 & YES & YES & YES & $1.40$ & $(1,3)$ & 102\\
$(152,55)$ & 12 & YES & YES & YES & $1.40$ & $(1,3)$ & 103\\
$(160,67)$ & 11 & YES & YES & NO(2) & $1.45$ & $(1,3)$ & 104\\
$(175,41)$ & 12 & YES & YES & NO(2) & $1.20$ & $(3,2)$ & 105\\
$(192,73)$ & 11 & YES & YES & YES & $1.36$ & $(3,2)$ & 106\\
$(199,74)$ & 12 & YES & YES & YES & $1.50$ & $(1,3)$ & 107\\
$(201,37)$ & 14 & YES & YES & YES & $1.30$ & $(1,3)$ & 108\\
$(203,59)$ & 12 & YES & YES & NO(3) & $1.12$ & $(1,3)$ & 109\\
$(205,78)$ & 12 & YES & YES & YES & $1.38$ & $(1,3)$ & 110\\
$(207,79)$ & 11 & YES & YES & YES & $1.22$ & $(5,1)$ & 111\\
$(212,93)$ & 12 & YES & YES & YES & $1.38$ & $(1,3)$ & 112\\
$(215,63)$ & 12 & YES & YES & YES & $1.36$ & $(3,2)$ & 113\\
$(223,98)$ & 12 & YES & YES & YES & $1.50$ & $(1,3)$ & 114\\
$(227,88)$ & 12 & YES & YES & YES & $1.44$ & $(5,1)$ & 115\\
$(229,87)$ & 12 & YES & YES & YES & $1.30$ & $(5,1)$ & 116\\
$(231,83)$ & 12 & YES & YES & YES & $1.60$ & $(1,3)$ & 117\\
$(239,105)$ & 12 & YES & YES & YES & $1.73$ & $(1,3)$ & 118\\
$(246,73)$ & 12 & YES & YES & YES & $1.54$ & $(1,3)$ & 119\\
$(246,91)$ & 12 & YES & YES & YES & $1.55$ & $(3,2)$ & 120\\
$(246,95)$ & 12 & YES & YES & YES & $1.36$ & $(3,2)$ & 121\\
$(251,74)$ & 13 & YES & YES & YES & $1.60$ & $(1,3)$ & 122\\
$(253,106)$ & 12 & YES & YES & YES & $1.64$ & $(3,2)$ & 123\\
$(254,75)$ & 12 & YES & YES & YES & $1.45$ & $(3,2)$ & 124\\
$(256,75)$ & 12 & YES & YES & YES & $1.45$ & $(3,2)$ & 125\\
$(259,76)$ & 13 & YES & YES & YES & $1.50$ & $(1,3)$ & 126\\
$(263,78)$ & 13 & YES & YES & YES & $1.50$ & $(1,3)$ & 127\\
$(269,78)$ & 13 & YES & YES & YES & $1.60$ & $(1,3)$ & 128\\
$(269,104)$ & 12 & YES & YES & YES & $1.58$ & $(1,3)$ & 129\\
$(271,84)$ & 13 & YES & YES & YES & $1.55$ & $(1,3)$ & 130\\
$(271,112)$ & 12 & YES & YES & YES & $1.55$ & $(3,2)$ & 131\\
$(273,76)$ & 13 & YES & YES & YES & $1.60$ & $(1,3)$ & 132\\
$(274,115)$ & 12 & YES & YES & YES & $1.38$ & $(1,3)$ & 133\\
$(280,107)$ & 12 & YES & YES & YES & $1.58$ & $(3,2)$ & 134\\
$(286,105)$ & 12 & YES & YES & YES & $1.67$ & $(3,2)$ & 135\\
$(288,119)$ & 12 & YES & YES & YES & $1.44$ & $(1,3)$ & 136\\
$(292,85)$ & 13 & YES & YES & YES & $1.33$ & $(3,2)$ & 137\\
$(293,123)$ & 12 & YES & YES & YES & $1.38$ & $(1,3)$ & 138\\
$(295,87)$ & 13 & YES & YES & YES & $1.25$ & $(1,3)$ & 139\\
$(305,112)$ & 12 & YES & YES & YES & $1.50$ & $(3,2)$ & 140\\
$(307,119)$ & 12 & YES & YES & YES & $1.40$ & $(3,2)$ & 141\\
$(309,92)$ & 13 & YES & YES & YES & $1.44$ & $(1,3)$ & 142\\
$(313,86)$ & 13 & YES & YES & YES & $1.50$ & $(1,3)$ & 143\\
$(313,121)$ & 12 & YES & YES & YES & $1.33$ & $(3,2)$ & 144\\
$(317,121)$ & 12 & YES & YES & YES & $1.40$ & $(3,2)$ & 145\\
$(320,93)$ & 13 & YES & YES & YES & $1.44$ & $(1,3)$ & 146\\
$(321,94)$ & 13 & YES & YES & YES & $1.58$ & $(3,2)$ & 147\\
$(323,94)$ & 13 & YES & YES & YES & $1.50$ & $(3,2)$ & 148\\
$(325,74)$ & 14 & YES & YES & YES & $1.50$ & $(1,3)$ & 149\\
$(326,71)$ & 14 & YES & YES & YES & $1.38$ & $(1,3)$ & 150\\
$(326,99)$ & 13 & YES & YES & YES & $1.50$ & $(3,2)$ & 151\\
$(338,129)$ & 12 & YES & YES & YES & $1.45$ & $(3,2)$ & 152\\
$(339,100)$ & 13 & YES & YES & YES & $1.36$ & $(3,2)$ & 153\\
$(341,100)$ & 13 & YES & YES & YES & $1.67$ & $(3,2)$ & 154\\
$(343,131)$ & 12 & YES & YES & YES & $1.56$ & $(1,3)$ & 155\\
$(344,95)$ & 13 & YES & YES & YES & $1.44$ & $(3,2)$ & 156\\
$(353,97)$ & 13 & YES & YES & YES & $1.44$ & $(1,3)$ & 157\\
$(359,100)$ & 13 & YES & YES & YES & $1.40$ & $(5,1)$ & 158\\
$(365,108)$ & 13 & YES & YES & YES & $1.50$ & $(3,2)$ & 159\\
$(373,104)$ & 13 & YES & YES & YES & $1.50$ & $(3,2)$ & 160\\
$(376,105)$ & 13 & YES & YES & YES & $1.50$ & $(3,2)$ & 161\\
$(382,87)$ & 14 & YES & YES & YES & $1.25$ & $(3,2)$ & 162\\
$(393,116)$ & 13 & YES & YES & YES & $1.40$ & $(3,2)$ & 163\\
$(397,116)$ & 13 & YES & YES & YES & $1.40$ & $(3,2)$ & 164\\
$(398,111)$ & 13 & YES & YES & YES & $1.40$ & $(3,2)$ & 165\\
$(401,111)$ & 13 & YES & YES & YES & $1.50$ & $(3,2)$ & 166\\
$(409,121)$ & 13 & YES & YES & YES & $1.30$ & $(3,2)$ & 167\\
$(413,121)$ & 13 & YES & YES & YES & $1.30$ & $(3,2)$ & 168\\
$(464,105)$ & 14 & YES & YES & YES & $1.40$ & $(3,2)$ & 169\\
$(487,111)$ & 14 & YES & YES & YES & $1.50$ & $(3,2)$ & 170\\
$(495,92)$ & 15 & YES & YES & YES & $1.44$ & $(1,3)$ & 171\\
$(b;0,2,3;6)$ & 10 & YES & YES & YES & $1.30$ & $(1,3)$ & 172\\
$(e;3,2,0;16)$ & 10 & YES & YES & YES & $1.30$ & $(1,3)$ & 173
\end{longtable}
\subsection{1 chain, $K^2 = 4$}
\begin{longtable}{|c|c|c|c|c|c|c|c|}
\hline
\multicolumn{8}{|c|}{1 chain, $K^2 = 4$}\\
\hline
$(n,a)$ & Len & Nef & $\mathbb Q$-ef & Obs 0 & $\overline c_1^2 / \overline c_2$ & $(P,K)$ & Index\\
\hline
\endfirsthead

\hline
$(n,a)$ & Len & Nef & $\mathbb Q$-ef & Obs 0 & $\overline c_1^2 / \overline c_2$ & $(P,K)$ & Index\\
\hline
\endhead
\hline
\endfoot

$(158,61)$ & 11 & YES & YES & NO(2) & $1.75$ & $(1,4)$ & 174\\
$(202,83)$ & 12 & YES & YES & NO(2) & $1.75$ & $(1,4)$ & 175\\
$(331,119)$ & 13 & YES & YES & YES & $1.75$ & $(1,4)$ & 176\\
$(404,169)$ & 13 & YES & YES & NO(3) & $1.57$ & $(1,4)$ & 177\\
$(445,72)$ & 18 & YES & YES & YES & $1.71$ & $(1,4)$ & 178\\
$(448,171)$ & 13 & YES & YES & YES & $1.89$ & $(1,4)$ & 179\\
$(459,194)$ & 14 & YES & YES & YES & $1.75$ & $(1,4)$ & 180\\
$(487,186)$ & 13 & YES & YES & YES & $1.82$ & $(1,4)$ & 181\\
$(535,158)$ & 14 & YES & YES & YES & $1.62$ & $(5,2)$ & 182\\
$(539,159)$ & 14 & YES & YES & YES & $1.75$ & $(5,2)$ & 183\\
$(573,217)$ & 14 & YES & YES & YES & $1.57$ & $(3,3)$ & 184\\
$(577,239)$ & 14 & YES & YES & YES & $1.71$ & $(1,4)$ & 185\\
$(597,176)$ & 15 & YES & YES & YES & $1.71$ & $(3,3)$ & 186\\
$(605,183)$ & 15 & YES & YES & YES & $1.57$ & $(3,3)$ & 187\\
$(611,237)$ & 14 & YES & YES & YES & $1.89$ & $(3,3)$ & 188\\
$(622,257)$ & 14 & YES & YES & YES & $1.75$ & $(3,3)$ & 189\\
$(631,231)$ & 15 & YES & YES & YES & $2.00$ & $(1,4)$ & 190\\
$(647,246)$ & 14 & YES & YES & YES & $1.71$ & $(1,4)$ & 191\\
$(647,271)$ & 14 & YES & YES & YES & $1.71$ & $(1,4)$ & 192\\
$(649,240)$ & 14 & YES & YES & YES & $1.71$ & $(1,4)$ & 193\\
$(673,196)$ & 15 & YES & YES & YES & $1.62$ & $(3,3)$ & 194\\
$(685,253)$ & 14 & YES & YES & YES & $1.89$ & $(3,3)$ & 195\\
$(694,305)$ & 15 & YES & YES & YES & $2.00$ & $(1,4)$ & 196\\
$(697,266)$ & 14 & YES & YES & YES & $2.00$ & $(1,4)$ & 197\\
$(708,209)$ & 14 & YES & YES & YES & $1.80$ & $(1,4)$ & 198\\
$(745,288)$ & 14 & YES & YES & YES & $1.90$ & $(1,4)$ & 199\\
$(755,312)$ & 14 & YES & YES & YES & $1.90$ & $(1,4)$ & 200\\
$(780,323)$ & 15 & YES & YES & YES & $1.88$ & $(3,3)$ & 201\\
$(818,239)$ & 15 & YES & YES & NO(2) & $1.67$ & $(5,2)$ & 202\\
$(853,313)$ & 15 & YES & YES & YES & $1.89$ & $(3,3)$ & 203\\
$(875,363)$ & 15 & YES & YES & YES & $1.89$ & $(3,3)$ & 204\\
$(881,326)$ & 15 & YES & YES & YES & $1.89$ & $(3,3)$ & 205\\
$(882,337)$ & 14 & YES & YES & YES & $1.80$ & $(1,4)$ & 206\\
$(907,264)$ & 15 & YES & YES & YES & $1.90$ & $(1,4)$ & 207\\
$(941,264)$ & 15 & YES & YES & YES & $1.90$ & $(1,4)$ & 208\\
$(997,295)$ & 15 & YES & YES & YES & $1.90$ & $(1,4)$ & 209\\
$(1027,305)$ & 15 & YES & YES & YES & $1.90$ & $(1,4)$ & 210\\
$(1037,278)$ & 16 & YES & YES & YES & $1.89$ & $(3,3)$ & 211\\
$(1047,307)$ & 16 & YES & YES & YES & $1.89$ & $(3,3)$ & 212\\
$(1173,266)$ & 17 & YES & YES & YES & $1.89$ & $(3,3)$ & 213\\
$(1193,273)$ & 16 & YES & YES & NO(2) & $1.56$ & $(5,2)$ & 214\\
$(1415,593)$ & 16 & YES & YES & YES & $2.11$ & $(3,3)$ & 215\\
$(1515,443)$ & 16 & YES & YES & YES & $2.12$ & $(5,2)$ & 216\\
$(1565,436)$ & 17 & YES & YES & YES & $2.11$ & $(3,3)$ & 217\\
$(1663,487)$ & 17 & YES & YES & YES & $2.00$ & $(3,3)$ & 218\\
$(1696,473)$ & 16 & YES & YES & YES & $2.12$ & $(5,2)$ & 219\\
$(1933,438)$ & 17 & YES & YES & YES & $2.12$ & $(5,2)$ & 220\\
$(2204,503)$ & 17 & YES & YES & YES & $1.88$ & $(5,2)$ & 221\\
$(b;4,0,4;110)$ & 13 & YES & YES & YES & $1.71$ & $(1,4)$ & 222
\end{longtable}
\subsection{1 chain, $K^2 = 5$}
\begin{longtable}{|c|c|c|c|c|c|c|c|}
\hline
\multicolumn{8}{|c|}{1 chain, $K^2 = 5$}\\
\hline
$(n,a)$ & Len & Nef & $\mathbb Q$-ef & Obs 0 & $\overline c_1^2 / \overline c_2$ & $(P,K)$ & Index\\
\hline
\endfirsthead

\hline
$(n,a)$ & Len & Nef & $\mathbb Q$-ef & Obs 0 & $\overline c_1^2 / \overline c_2$ & $(P,K)$ & Index\\
\hline
\endhead
\hline
\endfoot

$(1435,403)$ & 16 & YES & YES & NO(3) & $2.12$ & $(1,5)$ & 223\\
$(1953,544)$ & 17 & YES & YES & NO(3) & $2.17$ & $(3,4)$ & 224
\end{longtable}
\subsection{2 chains, $K^2 = 1$}
\begin{longtable}{|c|c|c|c|c|c|c|c|c|c|c|c|}
\hline
\multicolumn{12}{|c|}{2 chains, $K^2 = 1$}\\
\hline
$(n,a)$ & Len & $(n,a)$ & Len & GCD & Nef & $\mathbb Q$-ef & Obs 0 & $\overline c_1^2 / \overline c_2$ & $(P,K)$ & WH & Index\\
\hline
\endfirsthead

\hline
$(n,a)$ & Len & $(n,a)$ & Len & GCD & Nef & $\mathbb Q$-ef & Obs 0 & $\overline c_1^2 / \overline c_2$ & $(P,K)$ & WH & Index\\
\hline
\endhead
\hline
\endfoot

$(6,1)$ & 5 & $(5,2)$ & 3 & 1 & YES & YES & YES & $0.80$ & $(2,1)$ & NO & 225\\
$(6,1)$ & 5 & $(5,2)$ & 3 & 1 & YES & YES & YES & $0.80$ & $(2,1)$ & NO & 226\\
$(7,3)$ & 4 & $(5,1)$ & 4 & 1 & YES & YES & YES & $0.56$ & $(4,0)$ & NO & 227\\
$(7,3)$ & 4 & $(5,1)$ & 4 & 1 & YES & YES & YES & $0.56$ & $(4,0)$ & NO & 228\\
$(7,3)$ & 4 & $(7,2)$ & 4 & 7 & YES & YES & YES & $0.82$ & $(2,1)$ & NO & 229\\
$(7,3)$ & 4 & $(7,2)$ & 4 & 7 & YES & YES & YES & $0.82$ & $(2,1)$ & -- & 230\\
$(7,3)$ & 4 & $(7,2)$ & 4 & 7 & YES & YES & YES & $0.82$ & $(2,1)$ & NO & 231\\
$(7,3)$ & 4 & $(7,3)$ & 4 & 7 & YES & YES & YES & $0.44$ & $(2,1)$ & NO & 232\\
$(8,3)$ & 4 & $(7,3)$ & 4 & 1 & YES & YES & YES & $0.82$ & $(2,1)$ & NO & 233\\
$(8,3)$ & 4 & $(7,3)$ & 4 & 1 & YES & YES & YES & $0.82$ & $(2,1)$ & -- & 234\\
$(8,3)$ & 4 & $(7,3)$ & 4 & 1 & YES & YES & YES & $0.82$ & $(2,1)$ & NO & 235\\
$(9,2)$ & 5 & $(4,1)$ & 3 & 1 & YES & YES & YES & $0.44$ & $(2,1)$ & -- & 236\\
$(9,2)$ & 5 & $(4,1)$ & 3 & 1 & YES & YES & YES & $0.56$ & $(2,1)$ & NO & 237\\
$(9,4)$ & 5 & $(4,1)$ & 3 & 1 & YES & YES & YES & $0.80$ & $(2,1)$ & NO & 238\\
$(9,4)$ & 5 & $(4,1)$ & 3 & 1 & YES & YES & YES & $0.80$ & $(2,1)$ & NO & 239\\
$(9,2)$ & 5 & $(5,1)$ & 4 & 1 & YES & YES & YES & $0.56$ & $(2,1)$ & NO & 240\\
$(9,2)$ & 5 & $(5,1)$ & 4 & 1 & YES & YES & YES & $0.56$ & $(2,1)$ & NO & 241\\
$(9,2)$ & 5 & $(5,1)$ & 4 & 1 & YES & YES & YES & $0.56$ & $(2,1)$ & -- & 242\\
$(9,4)$ & 5 & $(5,2)$ & 3 & 1 & YES & YES & YES & $0.56$ & $(2,1)$ & NO & 243\\
$(9,2)$ & 5 & $(7,3)$ & 4 & 1 & YES & YES & YES & $0.82$ & $(2,1)$ & NO & 244\\
$(9,2)$ & 5 & $(7,3)$ & 4 & 1 & YES & YES & YES & $0.82$ & $(2,1)$ & -- & 245\\
$(9,4)$ & 5 & $(7,2)$ & 4 & 1 & YES & YES & YES & $0.56$ & $(2,1)$ & NO & 246\\
$(9,4)$ & 5 & $(8,3)$ & 4 & 1 & YES & YES & YES & $0.56$ & $(2,1)$ & 294 & 247\\
$(10,3)$ & 5 & $(5,2)$ & 3 & 5 & YES & YES & YES & $0.60$ & $(2,1)$ & -- & 248\\
$(11,2)$ & 6 & $(2,1)$ & 1 & 1 & YES & YES & YES & $0.67$ & $(2,1)$ & NO & 249\\
$(11,3)$ & 5 & $(2,1)$ & 1 & 1 & YES & YES & YES & $0.60$ & $(4,0)$ & -- & 250\\
$(11,4)$ & 5 & $(3,1)$ & 2 & 1 & YES & YES & YES & $0.70$ & $(2,1)$ & -- & 251\\
$(11,4)$ & 5 & $(3,1)$ & 2 & 1 & YES & YES & YES & $0.70$ & $(2,1)$ & NO & 252\\
$(11,5)$ & 6 & $(3,1)$ & 2 & 1 & YES & YES & YES & $0.70$ & $(2,1)$ & -- & 253\\
$(11,5)$ & 6 & $(3,1)$ & 2 & 1 & YES & YES & YES & $0.70$ & $(2,1)$ & NO & 254\\
$(11,3)$ & 5 & $(4,1)$ & 3 & 1 & YES & YES & YES & $0.60$ & $(4,0)$ & NO & 255\\
$(11,3)$ & 5 & $(4,1)$ & 3 & 1 & YES & YES & YES & $0.60$ & $(4,0)$ & -- & 256\\
$(11,3)$ & 5 & $(4,1)$ & 3 & 1 & YES & YES & YES & $0.60$ & $(4,0)$ & NO & 257\\
$(11,4)$ & 5 & $(4,1)$ & 3 & 1 & YES & YES & YES & $0.82$ & $(2,1)$ & NO & 258\\
$(11,4)$ & 5 & $(4,1)$ & 3 & 1 & YES & YES & YES & $0.82$ & $(2,1)$ & -- & 259\\
$(11,5)$ & 6 & $(4,1)$ & 3 & 1 & YES & YES & YES & $0.56$ & $(2,1)$ & NO & 260\\
$(11,5)$ & 6 & $(4,1)$ & 3 & 1 & YES & YES & YES & $0.56$ & $(2,1)$ & -- & 261\\
$(11,5)$ & 6 & $(4,1)$ & 3 & 1 & YES & YES & YES & $0.80$ & $(2,1)$ & NO & 262\\
$(11,2)$ & 6 & $(5,1)$ & 4 & 1 & YES & YES & YES & $0.56$ & $(2,1)$ & NO & 263\\
$(11,2)$ & 6 & $(5,1)$ & 4 & 1 & YES & YES & YES & $0.56$ & $(2,1)$ & NO & 264\\
$(11,2)$ & 6 & $(5,1)$ & 4 & 1 & YES & YES & YES & $0.56$ & $(2,1)$ & -- & 265\\
$(11,3)$ & 5 & $(5,2)$ & 3 & 1 & YES & YES & YES & $0.70$ & $(2,1)$ & NO & 266\\
$(11,3)$ & 5 & $(5,2)$ & 3 & 1 & YES & YES & YES & $0.70$ & $(2,1)$ & -- & 267\\
$(11,4)$ & 5 & $(5,2)$ & 3 & 1 & YES & YES & YES & $0.70$ & $(2,1)$ & 288 & 268\\
$(11,4)$ & 5 & $(5,2)$ & 3 & 1 & YES & YES & YES & $0.70$ & $(2,1)$ & -- & 269\\
$(11,5)$ & 6 & $(5,1)$ & 4 & 1 & YES & YES & YES & $0.67$ & $(2,1)$ & NO & 270\\
$(11,5)$ & 6 & $(5,1)$ & 4 & 1 & YES & YES & YES & $0.67$ & $(2,1)$ & NO & 271\\
$(11,5)$ & 6 & $(5,2)$ & 3 & 1 & YES & YES & YES & $0.80$ & $(2,1)$ & NO & 272\\
$(11,5)$ & 6 & $(5,2)$ & 3 & 1 & YES & YES & YES & $0.80$ & $(2,1)$ & -- & 273\\
$(11,5)$ & 6 & $(6,1)$ & 5 & 1 & YES & YES & YES & $0.80$ & $(2,1)$ & NO & 274\\
$(11,5)$ & 6 & $(6,1)$ & 5 & 1 & YES & YES & YES & $0.80$ & $(2,1)$ & NO & 275\\
$(11,5)$ & 6 & $(7,3)$ & 4 & 1 & YES & YES & YES & $0.67$ & $(2,1)$ & 292 & 276\\
$(11,4)$ & 5 & $(8,3)$ & 4 & 1 & YES & YES & YES & $0.82$ & $(2,1)$ & NO & 277\\
$(11,2)$ & 6 & $(9,4)$ & 5 & 1 & YES & YES & YES & $0.56$ & $(2,1)$ & NO & 278\\
$(11,5)$ & 6 & $(9,4)$ & 5 & 1 & YES & YES & YES & $0.56$ & $(2,1)$ & NO & 279\\
$(11,4)$ & 5 & $(11,4)$ & 5 & 11 & YES & YES & YES & $0.70$ & $(2,1)$ & NO & 280\\
$(11,5)$ & 6 & $(11,5)$ & 6 & 11 & YES & YES & YES & $0.70$ & $(2,1)$ & NO & 281\\
$(12,5)$ & 5 & $(3,1)$ & 2 & 3 & YES & YES & YES & $0.83$ & $(2,1)$ & -- & 282\\
$(12,5)$ & 5 & $(3,1)$ & 2 & 3 & YES & YES & YES & $0.92$ & $(2,1)$ & NO & 283\\
$(12,5)$ & 5 & $(3,1)$ & 2 & 3 & YES & YES & YES & $0.92$ & $(2,1)$ & NO & 284\\
$(13,5)$ & 5 & $(2,1)$ & 1 & 1 & YES & YES & YES & $0.70$ & $(2,1)$ & NO & 285\\
$(13,5)$ & 5 & $(3,1)$ & 2 & 1 & YES & YES & YES & $0.70$ & $(2,1)$ & NO & 286\\
$(13,5)$ & 5 & $(3,1)$ & 2 & 1 & YES & YES & YES & $0.70$ & $(2,1)$ & -- & 287\\
$(13,5)$ & 5 & $(3,1)$ & 2 & 1 & YES & YES & YES & $0.70$ & $(2,1)$ & 268 & 288\\
$(13,3)$ & 6 & $(11,3)$ & 5 & 1 & YES & YES & YES & $0.60$ & $(2,1)$ & NO & 289\\
$(14,3)$ & 6 & $(5,1)$ & 4 & 1 & NO & YES & YES & $0.56$ & $(2,1)$ & -- & 290\\
$(15,4)$ & 6 & $(4,1)$ & 3 & 1 & NO & YES & YES & $0.60$ & $(4,0)$ & -- & 291\\
$(16,7)$ & 6 & $(2,1)$ & 1 & 2 & YES & YES & YES & $0.67$ & $(2,1)$ & 276 & 292\\
$(16,5)$ & 7 & $(3,1)$ & 2 & 1 & YES & YES & YES & $0.60$ & $(2,1)$ & NO & 293\\
$(16,7)$ & 6 & $(3,1)$ & 2 & 1 & YES & YES & YES & $0.56$ & $(2,1)$ & 247 & 294\\
$(16,3)$ & 7 & $(5,1)$ & 4 & 1 & NO & YES & YES & $0.56$ & $(2,1)$ & NO & 295\\
$(16,3)$ & 7 & $(5,1)$ & 4 & 1 & NO & YES & YES & $0.56$ & $(2,1)$ & -- & 296\\
$(16,7)$ & 6 & $(5,1)$ & 4 & 1 & YES & YES & YES & $0.56$ & $(2,1)$ & NO & 297\\
$(16,7)$ & 6 & $(5,1)$ & 4 & 1 & YES & YES & YES & $0.56$ & $(2,1)$ & NO & 298\\
$(16,7)$ & 6 & $(5,1)$ & 4 & 1 & YES & YES & YES & $0.82$ & $(2,1)$ & -- & 299\\
$(16,5)$ & 7 & $(7,1)$ & 6 & 1 & YES & YES & YES & $0.60$ & $(2,1)$ & NO & 300\\
$(16,7)$ & 6 & $(7,3)$ & 4 & 1 & YES & YES & YES & $0.56$ & $(2,1)$ & NO & 301\\
$(16,7)$ & 6 & $(9,4)$ & 5 & 1 & YES & YES & YES & $0.56$ & $(2,1)$ & NO & 302\\
$(16,5)$ & 7 & $(13,4)$ & 6 & 1 & YES & YES & YES & $0.60$ & $(2,1)$ & NO & 303\\
$(17,7)$ & 6 & $(2,1)$ & 1 & 1 & YES & YES & YES & $0.70$ & $(2,1)$ & NO & 304\\
$(19,8)$ & 6 & $(2,1)$ & 1 & 1 & YES & YES & YES & $0.70$ & $(2,1)$ & NO & 305\\
$(19,8)$ & 6 & $(2,1)$ & 1 & 1 & NO & YES & YES & $0.70$ & $(2,1)$ & -- & 306\\
$(19,5)$ & 7 & $(4,1)$ & 3 & 1 & YES & YES & YES & $0.70$ & $(2,1)$ & NO & 307\\
$(19,5)$ & 7 & $(7,1)$ & 6 & 1 & YES & YES & YES & $0.60$ & $(2,1)$ & NO & 308\\
$(19,4)$ & 7 & $(11,2)$ & 6 & 1 & YES & YES & YES & $0.44$ & $(2,1)$ & NO & 309\\
$(19,5)$ & 7 & $(11,3)$ & 5 & 1 & YES & YES & YES & $0.60$ & $(2,1)$ & 318 & 310\\
$(20,9)$ & 7 & $(2,1)$ & 1 & 2 & NO & YES & YES & $0.67$ & $(2,1)$ & -- & 311\\
$(21,5)$ & 8 & $(4,1)$ & 3 & 1 & YES & YES & YES & $0.44$ & $(2,1)$ & NO & 312\\
$(23,10)$ & 7 & $(2,1)$ & 1 & 1 & NO & YES & YES & $0.70$ & $(2,1)$ & -- & 313\\
$(24,5)$ & 8 & $(5,1)$ & 4 & 1 & YES & YES & YES & $0.56$ & $(2,1)$ & NO & 314\\
$(24,5)$ & 8 & $(7,1)$ & 6 & 1 & YES & YES & YES & $0.44$ & $(2,1)$ & NO & 315\\
$(24,5)$ & 8 & $(19,4)$ & 7 & 1 & YES & YES & YES & $0.44$ & $(2,1)$ & NO & 316\\
$(25,9)$ & 7 & $(2,1)$ & 1 & 1 & NO & YES & YES & $0.67$ & $(2,1)$ & -- & 317\\
$(26,7)$ & 7 & $(4,1)$ & 3 & 2 & YES & YES & YES & $0.60$ & $(2,1)$ & 310 & 318\\
$(a;1,0,0;13)$ & 5 & $(2,1)$ & 1 & 1 & YES & YES & YES & $0.70$ & $(2,1)$ & -- & 319\\
$(a;2,0,0;17)$ & 6 & $(5,1)$ & 4 & 1 & YES & YES & YES & $0.56$ & $(2,1)$ & -- & 320\\
$(c;0,1,1;5)$ & 6 & $(2,1)$ & 1 & 1 & YES & YES & YES & $0.73$ & $(2,1)$ & -- & 321\\
$(c;0,2,0;7)$ & 6 & $(2,1)$ & 1 & 1 & YES & YES & YES & $0.60$ & $(2,1)$ & -- & 322\\
$(f;0,0,0;6)$ & 4 & $(4,1)$ & 3 & 2 & YES & YES & YES & $0.44$ & $(4,0)$ & -- & 323\\
$(f;0,0,0;6)$ & 4 & $(5,2)$ & 3 & 1 & YES & YES & YES & $0.82$ & $(2,1)$ & -- & 324\\
$(f;0,0,0;6)$ & 4 & $(7,3)$ & 4 & 1 & YES & YES & YES & $0.56$ & $(2,1)$ & -- & 325\\
$(f;0,0,0;6)$ & 4 & $(9,2)$ & 5 & 3 & YES & YES & YES & $0.82$ & $(2,1)$ & -- & 326\\
$(f;0,1,0;7)$ & 5 & $(3,1)$ & 2 & 1 & YES & YES & YES & $0.70$ & $(2,1)$ & -- & 327\\
$(f;0,1,0;7)$ & 5 & $(4,1)$ & 3 & 1 & YES & YES & YES & $0.82$ & $(2,1)$ & -- & 328\\
$(f;0,1,0;7)$ & 5 & $(5,1)$ & 4 & 1 & YES & YES & YES & $0.70$ & $(2,1)$ & -- & 329\\
$(j;0,0,0;8)$ & 5 & $(3,1)$ & 2 & 1 & YES & YES & YES & $0.60$ & $(2,1)$ & -- & 330\\
$(j;0,0,0;8)$ & 5 & $(5,1)$ & 4 & 1 & YES & YES & YES & $0.60$ & $(2,1)$ & -- & 331
\end{longtable}
\subsection{2 chains, $K^2 = 2$}
\begin{longtable}{|c|c|c|c|c|c|c|c|c|c|c|c|}
\hline
\multicolumn{12}{|c|}{2 chains, $K^2 = 2$}\\
\hline
$(n,a)$ & Len & $(n,a)$ & Len & GCD & Nef & $\mathbb Q$-ef & Obs 0 & $\overline c_1^2 / \overline c_2$ & $(P,K)$ & WH & Index\\
\hline
\endfirsthead

\hline
$(n,a)$ & Len & $(n,a)$ & Len & GCD & Nef & $\mathbb Q$-ef & Obs 0 & $\overline c_1^2 / \overline c_2$ & $(P,K)$ & WH & Index\\
\hline
\endhead
\hline
\endfoot

$(11,4)$ & 5 & $(7,3)$ & 4 & 1 & YES & YES & NO(2) & $1.18$ & $(2,2)$ & -- & 332\\
$(11,3)$ & 5 & $(9,4)$ & 5 & 1 & YES & YES & YES & $0.89$ & $(4,1)$ & -- & 333\\
$(11,5)$ & 6 & $(9,2)$ & 5 & 1 & YES & YES & YES & $1.11$ & $(2,2)$ & NO & 334\\
$(11,5)$ & 6 & $(9,2)$ & 5 & 1 & YES & YES & YES & $1.11$ & $(2,2)$ & -- & 335\\
$(11,5)$ & 6 & $(9,2)$ & 5 & 1 & YES & YES & YES & $1.11$ & $(2,2)$ & NO & 336\\
$(11,5)$ & 6 & $(10,3)$ & 5 & 1 & YES & YES & YES & $0.88$ & $(4,1)$ & NO & 337\\
$(11,5)$ & 6 & $(11,3)$ & 5 & 11 & YES & YES & YES & $1.11$ & $(2,2)$ & NO & 338\\
$(11,5)$ & 6 & $(11,3)$ & 5 & 11 & YES & YES & YES & $1.11$ & $(2,2)$ & -- & 339\\
$(11,5)$ & 6 & $(11,3)$ & 5 & 11 & YES & YES & YES & $1.11$ & $(2,2)$ & NO & 340\\
$(11,5)$ & 6 & $(11,5)$ & 6 & 11 & YES & YES & YES & $1.00$ & $(2,2)$ & -- & 341\\
$(12,5)$ & 5 & $(7,3)$ & 4 & 1 & YES & YES & NO(2) & $1.09$ & $(2,2)$ & -- & 342\\
$(12,5)$ & 5 & $(9,4)$ & 5 & 3 & YES & YES & NO(2) & $1.09$ & $(2,2)$ & -- & 343\\
$(12,5)$ & 5 & $(10,3)$ & 5 & 2 & YES & YES & YES & $0.89$ & $(2,2)$ & -- & 344\\
$(12,5)$ & 5 & $(11,3)$ & 5 & 1 & YES & YES & YES & $1.00$ & $(2,2)$ & -- & 345\\
$(12,5)$ & 5 & $(11,4)$ & 5 & 1 & YES & YES & NO(2) & $1.09$ & $(2,2)$ & NO & 346\\
$(13,5)$ & 5 & $(7,3)$ & 4 & 1 & YES & YES & NO(2) & $1.09$ & $(2,2)$ & -- & 347\\
$(13,4)$ & 6 & $(8,3)$ & 4 & 1 & YES & YES & YES & $1.27$ & $(2,2)$ & NO & 348\\
$(13,3)$ & 6 & $(9,4)$ & 5 & 1 & YES & YES & YES & $0.88$ & $(4,1)$ & NO & 349\\
$(13,3)$ & 6 & $(9,4)$ & 5 & 1 & YES & YES & YES & $0.88$ & $(4,1)$ & -- & 350\\
$(13,3)$ & 6 & $(9,4)$ & 5 & 1 & YES & YES & YES & $0.88$ & $(4,1)$ & NO & 351\\
$(13,4)$ & 6 & $(9,4)$ & 5 & 1 & YES & YES & YES & $1.30$ & $(2,2)$ & NO & 352\\
$(13,5)$ & 5 & $(9,4)$ & 5 & 1 & YES & YES & NO(2) & $1.09$ & $(2,2)$ & -- & 353\\
$(13,5)$ & 5 & $(10,3)$ & 5 & 1 & YES & YES & YES & $0.89$ & $(2,2)$ & -- & 354\\
$(13,5)$ & 5 & $(12,5)$ & 5 & 1 & YES & YES & YES & $1.20$ & $(2,2)$ & -- & 355\\
$(14,5)$ & 6 & $(7,2)$ & 4 & 7 & YES & YES & YES & $1.00$ & $(4,1)$ & NO & 356\\
$(14,5)$ & 6 & $(7,2)$ & 4 & 7 & YES & YES & YES & $1.00$ & $(4,1)$ & -- & 357\\
$(14,5)$ & 6 & $(9,2)$ & 5 & 1 & YES & YES & YES & $0.75$ & $(4,1)$ & NO & 358\\
$(14,5)$ & 6 & $(9,2)$ & 5 & 1 & YES & YES & YES & $0.75$ & $(4,1)$ & -- & 359\\
$(14,5)$ & 6 & $(10,3)$ & 5 & 2 & YES & YES & YES & $0.89$ & $(2,2)$ & -- & 360\\
$(15,4)$ & 6 & $(5,1)$ & 4 & 5 & YES & YES & YES & $1.00$ & $(2,2)$ & NO & 361\\
$(15,4)$ & 6 & $(9,4)$ & 5 & 3 & YES & YES & YES & $1.00$ & $(2,2)$ & -- & 362\\
$(15,4)$ & 6 & $(12,5)$ & 5 & 3 & YES & YES & YES & $1.00$ & $(2,2)$ & -- & 363\\
$(16,5)$ & 7 & $(7,3)$ & 4 & 1 & YES & YES & YES & $1.22$ & $(2,2)$ & NO & 364\\
$(16,7)$ & 6 & $(7,3)$ & 4 & 1 & YES & YES & YES & $1.10$ & $(2,2)$ & -- & 365\\
$(16,5)$ & 7 & $(9,4)$ & 5 & 1 & YES & YES & YES & $0.88$ & $(2,2)$ & NO & 366\\
$(16,5)$ & 7 & $(9,4)$ & 5 & 1 & YES & YES & YES & $0.88$ & $(2,2)$ & -- & 367\\
$(16,7)$ & 6 & $(9,4)$ & 5 & 1 & YES & YES & YES & $1.10$ & $(2,2)$ & -- & 368\\
$(16,3)$ & 7 & $(11,5)$ & 6 & 1 & YES & YES & YES & $0.88$ & $(2,2)$ & -- & 369\\
$(16,7)$ & 6 & $(11,4)$ & 5 & 1 & YES & YES & YES & $1.10$ & $(2,2)$ & 419 & 370\\
$(16,5)$ & 7 & $(12,5)$ & 5 & 4 & YES & YES & YES & $1.11$ & $(2,2)$ & NO & 371\\
$(16,5)$ & 7 & $(12,5)$ & 5 & 4 & YES & YES & YES & $1.11$ & $(2,2)$ & -- & 372\\
$(16,7)$ & 6 & $(13,4)$ & 6 & 1 & YES & YES & YES & $0.88$ & $(2,2)$ & -- & 373\\
$(17,5)$ & 6 & $(7,3)$ & 4 & 1 & YES & YES & NO(2) & $1.09$ & $(4,1)$ & -- & 374\\
$(17,5)$ & 6 & $(7,3)$ & 4 & 1 & YES & YES & YES & $1.18$ & $(2,2)$ & NO & 375\\
$(17,5)$ & 6 & $(8,3)$ & 4 & 1 & YES & YES & YES & $1.18$ & $(2,2)$ & NO & 376\\
$(17,5)$ & 6 & $(8,3)$ & 4 & 1 & YES & YES & YES & $1.18$ & $(2,2)$ & -- & 377\\
$(17,6)$ & 7 & $(9,2)$ & 5 & 1 & YES & YES & YES & $0.75$ & $(4,1)$ & -- & 378\\
$(17,6)$ & 7 & $(9,4)$ & 5 & 1 & YES & YES & YES & $1.12$ & $(2,2)$ & NO & 379\\
$(17,6)$ & 7 & $(9,4)$ & 5 & 1 & YES & YES & YES & $1.12$ & $(2,2)$ & -- & 380\\
$(17,7)$ & 6 & $(9,2)$ & 5 & 1 & YES & YES & YES & $1.25$ & $(2,2)$ & -- & 381\\
$(17,7)$ & 6 & $(10,3)$ & 5 & 1 & YES & YES & YES & $1.30$ & $(2,2)$ & -- & 382\\
$(17,7)$ & 6 & $(10,3)$ & 5 & 1 & YES & YES & YES & $1.00$ & $(2,2)$ & NO & 383\\
$(17,3)$ & 7 & $(11,5)$ & 6 & 1 & YES & YES & YES & $0.88$ & $(4,1)$ & NO & 384\\
$(17,3)$ & 7 & $(11,5)$ & 6 & 1 & YES & YES & YES & $0.88$ & $(4,1)$ & -- & 385\\
$(17,6)$ & 7 & $(11,5)$ & 6 & 1 & YES & YES & YES & $1.12$ & $(2,2)$ & NO & 386\\
$(17,7)$ & 6 & $(11,3)$ & 5 & 1 & YES & YES & YES & $1.30$ & $(2,2)$ & -- & 387\\
$(17,7)$ & 6 & $(11,5)$ & 6 & 1 & YES & YES & YES & $0.88$ & $(4,1)$ & NO & 388\\
$(17,5)$ & 6 & $(13,5)$ & 5 & 1 & YES & YES & YES & $1.30$ & $(2,2)$ & -- & 389\\
$(17,5)$ & 6 & $(13,5)$ & 5 & 1 & YES & YES & YES & $1.30$ & $(2,2)$ & NO & 390\\
$(17,7)$ & 6 & $(17,5)$ & 6 & 17 & YES & YES & YES & $1.00$ & $(6,0)$ & -- & 391\\
$(18,5)$ & 6 & $(7,3)$ & 4 & 1 & YES & YES & YES & $1.09$ & $(2,2)$ & -- & 392\\
$(18,5)$ & 6 & $(7,3)$ & 4 & 1 & YES & YES & YES & $1.18$ & $(2,2)$ & NO & 393\\
$(18,7)$ & 6 & $(7,3)$ & 4 & 1 & YES & YES & YES & $1.00$ & $(2,2)$ & -- & 394\\
$(18,5)$ & 6 & $(8,3)$ & 4 & 2 & YES & YES & YES & $1.09$ & $(2,2)$ & -- & 395\\
$(18,5)$ & 6 & $(8,3)$ & 4 & 2 & YES & YES & YES & $1.18$ & $(2,2)$ & NO & 396\\
$(18,7)$ & 6 & $(9,2)$ & 5 & 9 & YES & YES & YES & $1.10$ & $(2,2)$ & NO & 397\\
$(18,7)$ & 6 & $(9,2)$ & 5 & 9 & YES & YES & YES & $1.10$ & $(2,2)$ & -- & 398\\
$(18,7)$ & 6 & $(9,4)$ & 5 & 9 & YES & YES & YES & $1.00$ & $(2,2)$ & -- & 399\\
$(18,7)$ & 6 & $(9,4)$ & 5 & 9 & YES & YES & YES & $0.88$ & $(4,1)$ & NO & 400\\
$(18,7)$ & 6 & $(11,3)$ & 5 & 1 & YES & YES & YES & $1.12$ & $(2,2)$ & NO & 401\\
$(18,7)$ & 6 & $(11,3)$ & 5 & 1 & YES & YES & YES & $1.12$ & $(2,2)$ & -- & 402\\
$(18,5)$ & 6 & $(13,4)$ & 6 & 1 & YES & YES & YES & $1.10$ & $(2,2)$ & NO & 403\\
$(18,5)$ & 6 & $(13,5)$ & 5 & 1 & YES & YES & YES & $1.20$ & $(2,2)$ & -- & 404\\
$(18,7)$ & 6 & $(15,4)$ & 6 & 3 & YES & YES & YES & $1.00$ & $(6,0)$ & -- & 405\\
$(18,7)$ & 6 & $(15,4)$ & 6 & 3 & YES & YES & YES & $1.22$ & $(6,0)$ & NO & 406\\
$(18,7)$ & 6 & $(16,7)$ & 6 & 2 & YES & YES & YES & $1.33$ & $(2,2)$ & -- & 407\\
$(18,5)$ & 6 & $(17,7)$ & 6 & 1 & YES & YES & YES & $0.89$ & $(6,0)$ & NO & 408\\
$(18,7)$ & 6 & $(17,4)$ & 7 & 1 & YES & YES & YES & $1.00$ & $(6,0)$ & -- & 409\\
$(18,7)$ & 6 & $(17,4)$ & 7 & 1 & YES & YES & YES & $1.22$ & $(6,0)$ & NO & 410\\
$(18,7)$ & 6 & $(18,5)$ & 6 & 18 & YES & YES & YES & $1.18$ & $(4,1)$ & -- & 411\\
$(18,7)$ & 6 & $(18,5)$ & 6 & 18 & YES & YES & YES & $1.42$ & $(4,1)$ & NO & 412\\
$(19,8)$ & 6 & $(5,2)$ & 3 & 1 & YES & YES & YES & $1.18$ & $(2,2)$ & -- & 413\\
$(19,7)$ & 6 & $(7,3)$ & 4 & 1 & YES & YES & YES & $1.10$ & $(2,2)$ & NO & 414\\
$(19,7)$ & 6 & $(7,3)$ & 4 & 1 & YES & YES & YES & $1.10$ & $(2,2)$ & -- & 415\\
$(19,8)$ & 6 & $(7,3)$ & 4 & 1 & YES & YES & YES & $1.10$ & $(2,2)$ & -- & 416\\
$(19,4)$ & 7 & $(9,4)$ & 5 & 1 & YES & YES & YES & $1.11$ & $(2,2)$ & NO & 417\\
$(19,4)$ & 7 & $(9,4)$ & 5 & 1 & YES & YES & YES & $1.11$ & $(2,2)$ & -- & 418\\
$(19,7)$ & 6 & $(9,4)$ & 5 & 1 & YES & YES & YES & $1.10$ & $(2,2)$ & 370 & 419\\
$(19,8)$ & 6 & $(9,4)$ & 5 & 1 & YES & YES & YES & $0.88$ & $(2,2)$ & -- & 420\\
$(19,7)$ & 6 & $(10,3)$ & 5 & 1 & YES & YES & YES & $0.88$ & $(2,2)$ & NO & 421\\
$(19,7)$ & 6 & $(10,3)$ & 5 & 1 & YES & YES & YES & $0.88$ & $(2,2)$ & -- & 422\\
$(19,8)$ & 6 & $(10,3)$ & 5 & 1 & YES & YES & YES & $1.10$ & $(2,2)$ & NO & 423\\
$(19,8)$ & 6 & $(10,3)$ & 5 & 1 & YES & YES & YES & $1.10$ & $(2,2)$ & -- & 424\\
$(19,4)$ & 7 & $(11,4)$ & 5 & 1 & YES & YES & YES & $1.11$ & $(2,2)$ & -- & 425\\
$(19,8)$ & 6 & $(11,4)$ & 5 & 1 & YES & YES & YES & $1.10$ & $(2,2)$ & NO & 426\\
$(19,8)$ & 6 & $(13,4)$ & 6 & 1 & YES & YES & YES & $1.12$ & $(2,2)$ & -- & 427\\
$(19,8)$ & 6 & $(13,4)$ & 6 & 1 & YES & YES & YES & $1.25$ & $(2,2)$ & NO & 428\\
$(19,7)$ & 6 & $(14,5)$ & 6 & 1 & YES & YES & YES & $0.75$ & $(4,1)$ & NO & 429\\
$(19,8)$ & 6 & $(15,4)$ & 6 & 1 & YES & YES & YES & $1.11$ & $(2,2)$ & NO & 430\\
$(19,8)$ & 6 & $(15,4)$ & 6 & 1 & YES & YES & YES & $1.33$ & $(2,2)$ & NO & 431\\
$(19,8)$ & 6 & $(15,4)$ & 6 & 1 & YES & YES & YES & $1.33$ & $(2,2)$ & -- & 432\\
$(19,4)$ & 7 & $(17,4)$ & 7 & 1 & YES & YES & YES & $1.00$ & $(2,2)$ & -- & 433\\
$(19,5)$ & 7 & $(17,3)$ & 7 & 1 & YES & YES & YES & $0.89$ & $(2,2)$ & -- & 434\\
$(19,7)$ & 6 & $(17,6)$ & 7 & 1 & YES & YES & YES & $0.75$ & $(4,1)$ & 641 & 435\\
$(19,8)$ & 6 & $(17,5)$ & 6 & 1 & YES & YES & YES & $1.00$ & $(2,2)$ & NO & 436\\
$(19,8)$ & 6 & $(17,7)$ & 6 & 1 & YES & YES & NO(2) & $1.00$ & $(4,1)$ & NO & 437\\
$(19,7)$ & 6 & $(18,7)$ & 6 & 1 & YES & YES & YES & $0.88$ & $(2,2)$ & NO & 438\\
$(19,8)$ & 6 & $(18,5)$ & 6 & 1 & YES & YES & YES & $1.12$ & $(2,2)$ & NO & 439\\
$(19,8)$ & 6 & $(18,5)$ & 6 & 1 & YES & YES & YES & $1.12$ & $(2,2)$ & -- & 440\\
$(19,4)$ & 7 & $(19,4)$ & 7 & 19 & YES & YES & YES & $1.17$ & $(2,2)$ & -- & 441\\
$(20,9)$ & 7 & $(5,2)$ & 3 & 5 & YES & YES & YES & $0.75$ & $(4,1)$ & -- & 442\\
$(20,9)$ & 7 & $(5,2)$ & 3 & 5 & YES & YES & YES & $1.00$ & $(2,2)$ & NO & 443\\
$(20,9)$ & 7 & $(8,3)$ & 4 & 4 & YES & YES & YES & $0.75$ & $(4,1)$ & NO & 444\\
$(20,9)$ & 7 & $(16,7)$ & 6 & 4 & YES & YES & YES & $0.75$ & $(4,1)$ & NO & 445\\
$(21,8)$ & 6 & $(5,2)$ & 3 & 1 & YES & YES & YES & $1.00$ & $(2,2)$ & NO & 446\\
$(21,8)$ & 6 & $(5,2)$ & 3 & 1 & YES & YES & YES & $1.00$ & $(2,2)$ & -- & 447\\
$(21,8)$ & 6 & $(7,3)$ & 4 & 7 & YES & YES & YES & $1.10$ & $(2,2)$ & NO & 448\\
$(21,8)$ & 6 & $(7,3)$ & 4 & 7 & YES & YES & YES & $1.10$ & $(2,2)$ & -- & 449\\
$(21,8)$ & 6 & $(9,4)$ & 5 & 3 & YES & YES & YES & $1.10$ & $(2,2)$ & NO & 450\\
$(21,8)$ & 6 & $(10,3)$ & 5 & 1 & YES & YES & YES & $1.45$ & $(2,2)$ & -- & 451\\
$(21,8)$ & 6 & $(10,3)$ & 5 & 1 & YES & YES & YES & $1.12$ & $(2,2)$ & NO & 452\\
$(21,8)$ & 6 & $(11,3)$ & 5 & 1 & YES & YES & YES & $1.30$ & $(2,2)$ & NO & 453\\
$(21,8)$ & 6 & $(11,3)$ & 5 & 1 & YES & YES & YES & $1.30$ & $(2,2)$ & -- & 454\\
$(21,8)$ & 6 & $(12,5)$ & 5 & 3 & YES & YES & YES & $1.11$ & $(2,2)$ & -- & 455\\
$(21,8)$ & 6 & $(13,4)$ & 6 & 1 & YES & YES & YES & $1.12$ & $(2,2)$ & -- & 456\\
$(21,8)$ & 6 & $(15,4)$ & 6 & 3 & YES & YES & YES & $1.33$ & $(2,2)$ & -- & 457\\
$(21,8)$ & 6 & $(17,4)$ & 7 & 1 & YES & YES & YES & $1.00$ & $(4,1)$ & NO & 458\\
$(21,8)$ & 6 & $(17,4)$ & 7 & 1 & YES & YES & YES & $1.00$ & $(4,1)$ & -- & 459\\
$(21,8)$ & 6 & $(17,5)$ & 6 & 1 & YES & YES & YES & $1.42$ & $(4,1)$ & -- & 460\\
$(21,8)$ & 6 & $(17,5)$ & 6 & 1 & YES & YES & YES & $1.42$ & $(4,1)$ & NO & 461\\
$(21,8)$ & 6 & $(18,5)$ & 6 & 3 & YES & YES & YES & $1.42$ & $(4,1)$ & -- & 462\\
$(21,8)$ & 6 & $(18,7)$ & 6 & 3 & YES & YES & YES & $1.10$ & $(2,2)$ & NO & 463\\
$(21,5)$ & 8 & $(21,4)$ & 8 & 21 & YES & YES & YES & $1.00$ & $(2,2)$ & NO & 464\\
$(22,9)$ & 7 & $(4,1)$ & 3 & 2 & YES & YES & YES & $1.27$ & $(2,2)$ & -- & 465\\
$(22,9)$ & 7 & $(7,2)$ & 4 & 1 & YES & YES & YES & $1.11$ & $(2,2)$ & NO & 466\\
$(22,9)$ & 7 & $(7,2)$ & 4 & 1 & YES & YES & YES & $1.11$ & $(2,2)$ & -- & 467\\
$(22,9)$ & 7 & $(9,4)$ & 5 & 1 & YES & YES & YES & $1.11$ & $(2,2)$ & NO & 468\\
$(22,9)$ & 7 & $(11,2)$ & 6 & 11 & YES & YES & YES & $1.25$ & $(2,2)$ & NO & 469\\
$(22,9)$ & 7 & $(17,4)$ & 7 & 1 & YES & YES & YES & $0.88$ & $(6,0)$ & NO & 470\\
$(22,5)$ & 7 & $(18,7)$ & 6 & 2 & YES & YES & YES & $1.27$ & $(4,1)$ & -- & 471\\
$(22,5)$ & 7 & $(18,7)$ & 6 & 2 & YES & YES & YES & $1.50$ & $(4,1)$ & NO & 472\\
$(22,9)$ & 7 & $(19,4)$ & 7 & 1 & YES & YES & YES & $0.88$ & $(6,0)$ & NO & 473\\
$(22,5)$ & 7 & $(21,8)$ & 6 & 1 & YES & YES & YES & $0.89$ & $(6,0)$ & NO & 474\\
$(23,9)$ & 7 & $(4,1)$ & 3 & 1 & YES & YES & YES & $1.11$ & $(2,2)$ & NO & 475\\
$(23,9)$ & 7 & $(4,1)$ & 3 & 1 & YES & YES & YES & $1.27$ & $(2,2)$ & -- & 476\\
$(23,9)$ & 7 & $(5,2)$ & 3 & 1 & YES & YES & YES & $1.11$ & $(2,2)$ & NO & 477\\
$(23,7)$ & 7 & $(7,3)$ & 4 & 1 & YES & YES & NO(2) & $1.00$ & $(4,1)$ & -- & 478\\
$(23,9)$ & 7 & $(7,3)$ & 4 & 1 & YES & YES & YES & $1.11$ & $(2,2)$ & NO & 479\\
$(23,9)$ & 7 & $(7,3)$ & 4 & 1 & YES & YES & YES & $1.11$ & $(2,2)$ & -- & 480\\
$(23,6)$ & 8 & $(9,4)$ & 5 & 1 & YES & YES & YES & $1.00$ & $(2,2)$ & NO & 481\\
$(23,9)$ & 7 & $(10,3)$ & 5 & 1 & YES & YES & YES & $1.12$ & $(4,1)$ & NO & 482\\
$(23,9)$ & 7 & $(10,3)$ & 5 & 1 & YES & YES & YES & $1.12$ & $(4,1)$ & -- & 483\\
$(23,9)$ & 7 & $(11,4)$ & 5 & 1 & YES & YES & YES & $1.11$ & $(2,2)$ & NO & 484\\
$(23,7)$ & 7 & $(12,5)$ & 5 & 1 & YES & YES & YES & $1.12$ & $(2,2)$ & -- & 485\\
$(23,4)$ & 8 & $(14,5)$ & 6 & 1 & YES & YES & YES & $1.00$ & $(2,2)$ & -- & 486\\
$(23,4)$ & 8 & $(14,5)$ & 6 & 1 & YES & YES & YES & $1.11$ & $(2,2)$ & NO & 487\\
$(23,5)$ & 7 & $(17,7)$ & 6 & 1 & YES & YES & YES & $1.00$ & $(2,2)$ & -- & 488\\
$(23,10)$ & 7 & $(18,5)$ & 6 & 1 & YES & YES & YES & $1.33$ & $(2,2)$ & -- & 489\\
$(23,5)$ & 7 & $(19,8)$ & 6 & 1 & YES & YES & YES & $1.22$ & $(2,2)$ & 649 & 490\\
$(23,4)$ & 8 & $(21,5)$ & 8 & 1 & YES & YES & YES & $1.00$ & $(2,2)$ & NO & 491\\
$(23,10)$ & 7 & $(23,5)$ & 7 & 23 & YES & YES & YES & $1.22$ & $(2,2)$ & -- & 492\\
$(24,11)$ & 8 & $(3,1)$ & 2 & 3 & YES & YES & YES & $1.00$ & $(2,2)$ & NO & 493\\
$(24,11)$ & 8 & $(5,1)$ & 4 & 1 & YES & YES & YES & $0.88$ & $(2,2)$ & -- & 494\\
$(24,11)$ & 8 & $(5,2)$ & 3 & 1 & YES & YES & YES & $1.00$ & $(2,2)$ & NO & 495\\
$(24,5)$ & 8 & $(9,4)$ & 5 & 3 & YES & YES & YES & $1.11$ & $(2,2)$ & -- & 496\\
$(24,7)$ & 7 & $(10,3)$ & 5 & 2 & YES & YES & YES & $1.30$ & $(2,2)$ & NO & 497\\
$(24,7)$ & 7 & $(10,3)$ & 5 & 2 & YES & YES & YES & $1.30$ & $(2,2)$ & -- & 498\\
$(24,5)$ & 8 & $(11,3)$ & 5 & 1 & YES & YES & YES & $1.00$ & $(2,2)$ & NO & 499\\
$(24,5)$ & 8 & $(11,4)$ & 5 & 1 & YES & YES & YES & $1.11$ & $(2,2)$ & -- & 500\\
$(24,7)$ & 7 & $(11,3)$ & 5 & 1 & YES & YES & YES & $1.30$ & $(2,2)$ & -- & 501\\
$(24,7)$ & 7 & $(11,3)$ & 5 & 1 & YES & YES & YES & $1.30$ & $(2,2)$ & NO & 502\\
$(24,7)$ & 7 & $(11,4)$ & 5 & 1 & YES & YES & YES & $1.00$ & $(4,1)$ & NO & 503\\
$(24,7)$ & 7 & $(12,5)$ & 5 & 12 & YES & YES & YES & $1.25$ & $(2,2)$ & NO & 504\\
$(24,7)$ & 7 & $(12,5)$ & 5 & 12 & YES & YES & YES & $1.25$ & $(2,2)$ & -- & 505\\
$(24,7)$ & 7 & $(13,5)$ & 5 & 1 & YES & YES & YES & $1.00$ & $(6,0)$ & -- & 506\\
$(24,5)$ & 8 & $(21,5)$ & 8 & 3 & YES & YES & YES & $1.00$ & $(2,2)$ & NO & 507\\
$(24,7)$ & 7 & $(23,5)$ & 7 & 1 & YES & YES & YES & $0.75$ & $(4,1)$ & NO & 508\\
$(25,9)$ & 7 & $(3,1)$ & 2 & 1 & YES & YES & YES & $0.78$ & $(4,1)$ & -- & 509\\
$(25,9)$ & 7 & $(3,1)$ & 2 & 1 & YES & YES & YES & $0.88$ & $(4,1)$ & NO & 510\\
$(25,9)$ & 7 & $(4,1)$ & 3 & 1 & YES & YES & YES & $0.88$ & $(4,1)$ & NO & 511\\
$(25,9)$ & 7 & $(4,1)$ & 3 & 1 & YES & YES & YES & $0.88$ & $(4,1)$ & -- & 512\\
$(25,9)$ & 7 & $(4,1)$ & 3 & 1 & YES & YES & YES & $0.88$ & $(4,1)$ & NO & 513\\
$(25,9)$ & 7 & $(5,2)$ & 3 & 5 & YES & YES & YES & $1.20$ & $(2,2)$ & NO & 514\\
$(25,9)$ & 7 & $(5,2)$ & 3 & 5 & YES & YES & YES & $1.20$ & $(2,2)$ & -- & 515\\
$(25,9)$ & 7 & $(7,3)$ & 4 & 1 & YES & YES & YES & $1.10$ & $(2,2)$ & NO & 516\\
$(25,9)$ & 7 & $(7,3)$ & 4 & 1 & YES & YES & YES & $0.88$ & $(2,2)$ & -- & 517\\
$(25,9)$ & 7 & $(9,4)$ & 5 & 1 & YES & YES & YES & $0.88$ & $(2,2)$ & NO & 518\\
$(25,7)$ & 7 & $(12,5)$ & 5 & 1 & YES & YES & YES & $1.25$ & $(2,2)$ & NO & 519\\
$(25,7)$ & 7 & $(12,5)$ & 5 & 1 & YES & YES & YES & $1.25$ & $(2,2)$ & -- & 520\\
$(25,7)$ & 7 & $(13,5)$ & 5 & 1 & YES & YES & YES & $1.12$ & $(2,2)$ & -- & 521\\
$(25,9)$ & 7 & $(13,3)$ & 6 & 1 & YES & YES & YES & $0.88$ & $(2,2)$ & NO & 522\\
$(25,7)$ & 7 & $(23,7)$ & 7 & 1 & YES & YES & YES & $1.11$ & $(2,2)$ & NO & 523\\
$(25,9)$ & 7 & $(25,9)$ & 7 & 25 & YES & YES & YES & $0.89$ & $(4,1)$ & NO & 524\\
$(26,11)$ & 7 & $(3,1)$ & 2 & 1 & YES & YES & NO(2) & $1.09$ & $(4,1)$ & -- & 525\\
$(26,11)$ & 7 & $(3,1)$ & 2 & 1 & YES & YES & YES & $1.27$ & $(2,2)$ & NO & 526\\
$(26,11)$ & 7 & $(4,1)$ & 3 & 2 & YES & YES & NO(2) & $1.00$ & $(4,1)$ & -- & 527\\
$(26,11)$ & 7 & $(4,1)$ & 3 & 2 & YES & YES & YES & $1.18$ & $(2,2)$ & NO & 528\\
$(26,11)$ & 7 & $(5,2)$ & 3 & 1 & YES & YES & YES & $1.00$ & $(2,2)$ & NO & 529\\
$(26,11)$ & 7 & $(5,2)$ & 3 & 1 & YES & YES & YES & $1.00$ & $(2,2)$ & -- & 530\\
$(26,11)$ & 7 & $(7,2)$ & 4 & 1 & YES & YES & YES & $1.18$ & $(2,2)$ & NO & 531\\
$(26,11)$ & 7 & $(7,2)$ & 4 & 1 & YES & YES & YES & $1.18$ & $(2,2)$ & -- & 532\\
$(26,11)$ & 7 & $(8,3)$ & 4 & 2 & YES & YES & YES & $1.12$ & $(4,1)$ & -- & 533\\
$(26,11)$ & 7 & $(8,3)$ & 4 & 2 & YES & YES & YES & $1.00$ & $(2,2)$ & 837 & 534\\
$(26,7)$ & 7 & $(12,5)$ & 5 & 2 & YES & YES & YES & $1.00$ & $(2,2)$ & -- & 535\\
$(26,11)$ & 7 & $(12,5)$ & 5 & 2 & YES & YES & NO(2) & $1.09$ & $(4,1)$ & 640 & 536\\
$(26,11)$ & 7 & $(13,3)$ & 6 & 13 & YES & YES & YES & $1.12$ & $(2,2)$ & NO & 537\\
$(26,11)$ & 7 & $(13,3)$ & 6 & 13 & YES & YES & YES & $1.11$ & $(2,2)$ & -- & 538\\
$(26,11)$ & 7 & $(14,3)$ & 6 & 2 & YES & YES & YES & $1.12$ & $(2,2)$ & NO & 539\\
$(26,11)$ & 7 & $(19,8)$ & 6 & 1 & YES & YES & NO(2) & $1.00$ & $(4,1)$ & NO & 540\\
$(26,7)$ & 7 & $(23,7)$ & 7 & 1 & YES & YES & YES & $1.12$ & $(2,2)$ & NO & 541\\
$(26,5)$ & 9 & $(26,5)$ & 9 & 26 & YES & YES & YES & $1.11$ & $(2,2)$ & NO & 542\\
$(26,11)$ & 7 & $(26,11)$ & 7 & 26 & YES & YES & YES & $1.00$ & $(2,2)$ & NO & 543\\
$(27,10)$ & 7 & $(2,1)$ & 1 & 1 & YES & YES & NO(2) & $1.17$ & $(2,2)$ & -- & 544\\
$(27,10)$ & 7 & $(2,1)$ & 1 & 1 & YES & YES & NO(2) & $1.17$ & $(2,2)$ & NO & 545\\
$(27,11)$ & 8 & $(4,1)$ & 3 & 1 & YES & YES & YES & $1.11$ & $(2,2)$ & -- & 546\\
$(27,8)$ & 7 & $(5,2)$ & 3 & 1 & YES & YES & YES & $1.00$ & $(2,2)$ & NO & 547\\
$(27,8)$ & 7 & $(5,2)$ & 3 & 1 & YES & YES & YES & $1.00$ & $(2,2)$ & -- & 548\\
$(27,11)$ & 8 & $(5,1)$ & 4 & 1 & YES & YES & YES & $1.20$ & $(2,2)$ & -- & 549\\
$(27,11)$ & 8 & $(6,1)$ & 5 & 3 & YES & YES & YES & $1.11$ & $(2,2)$ & -- & 550\\
$(27,8)$ & 7 & $(7,3)$ & 4 & 1 & YES & YES & YES & $1.00$ & $(2,2)$ & NO & 551\\
$(27,10)$ & 7 & $(7,3)$ & 4 & 1 & YES & YES & YES & $1.00$ & $(2,2)$ & NO & 552\\
$(27,10)$ & 7 & $(7,3)$ & 4 & 1 & YES & YES & YES & $1.00$ & $(2,2)$ & -- & 553\\
$(27,8)$ & 7 & $(8,3)$ & 4 & 1 & YES & YES & YES & $1.00$ & $(2,2)$ & NO & 554\\
$(27,11)$ & 8 & $(9,4)$ & 5 & 9 & YES & YES & YES & $1.11$ & $(2,2)$ & NO & 555\\
$(27,8)$ & 7 & $(12,5)$ & 5 & 3 & YES & YES & YES & $1.12$ & $(2,2)$ & -- & 556\\
$(27,8)$ & 7 & $(12,5)$ & 5 & 3 & YES & YES & YES & $1.22$ & $(2,2)$ & NO & 557\\
$(27,11)$ & 8 & $(12,5)$ & 5 & 3 & YES & YES & YES & $1.11$ & $(2,2)$ & NO & 558\\
$(27,8)$ & 7 & $(13,5)$ & 5 & 1 & YES & YES & YES & $1.18$ & $(4,1)$ & -- & 559\\
$(27,8)$ & 7 & $(13,5)$ & 5 & 1 & YES & YES & YES & $1.50$ & $(4,1)$ & NO & 560\\
$(27,11)$ & 8 & $(17,7)$ & 6 & 1 & YES & YES & YES & $1.20$ & $(2,2)$ & 755 & 561\\
$(27,11)$ & 8 & $(22,9)$ & 7 & 1 & YES & YES & YES & $1.11$ & $(2,2)$ & NO & 562\\
$(27,10)$ & 7 & $(23,5)$ & 7 & 1 & YES & YES & YES & $1.22$ & $(2,2)$ & -- & 563\\
$(27,10)$ & 7 & $(25,9)$ & 7 & 1 & YES & YES & YES & $0.88$ & $(2,2)$ & NO & 564\\
$(28,11)$ & 8 & $(4,1)$ & 3 & 4 & YES & YES & YES & $1.11$ & $(2,2)$ & -- & 565\\
$(28,11)$ & 8 & $(4,1)$ & 3 & 4 & YES & YES & YES & $1.11$ & $(2,2)$ & NO & 566\\
$(28,11)$ & 8 & $(5,1)$ & 4 & 1 & YES & YES & YES & $1.20$ & $(2,2)$ & -- & 567\\
$(28,11)$ & 8 & $(5,2)$ & 3 & 1 & YES & YES & YES & $1.11$ & $(2,2)$ & NO & 568\\
$(28,11)$ & 8 & $(6,1)$ & 5 & 2 & YES & YES & YES & $1.11$ & $(2,2)$ & -- & 569\\
$(28,11)$ & 8 & $(6,1)$ & 5 & 2 & YES & YES & YES & $1.11$ & $(2,2)$ & NO & 570\\
$(28,11)$ & 8 & $(13,3)$ & 6 & 1 & YES & YES & YES & $1.33$ & $(2,2)$ & -- & 571\\
$(28,11)$ & 8 & $(13,5)$ & 5 & 1 & YES & YES & YES & $1.11$ & $(2,2)$ & NO & 572\\
$(28,11)$ & 8 & $(14,3)$ & 6 & 14 & YES & YES & YES & $1.22$ & $(2,2)$ & -- & 573\\
$(28,11)$ & 8 & $(16,3)$ & 7 & 4 & YES & YES & YES & $1.22$ & $(2,2)$ & -- & 574\\
$(28,11)$ & 8 & $(17,3)$ & 7 & 1 & YES & YES & YES & $1.33$ & $(2,2)$ & NO & 575\\
$(28,11)$ & 8 & $(18,7)$ & 6 & 2 & YES & YES & YES & $1.20$ & $(2,2)$ & 783 & 576\\
$(28,11)$ & 8 & $(23,9)$ & 7 & 1 & YES & YES & YES & $1.11$ & $(2,2)$ & NO & 577\\
$(29,11)$ & 7 & $(3,1)$ & 2 & 1 & YES & YES & YES & $1.00$ & $(2,2)$ & -- & 578\\
$(29,12)$ & 7 & $(3,1)$ & 2 & 1 & YES & YES & YES & $1.09$ & $(2,2)$ & -- & 579\\
$(29,11)$ & 7 & $(4,1)$ & 3 & 1 & YES & YES & YES & $1.10$ & $(2,2)$ & -- & 580\\
$(29,12)$ & 7 & $(4,1)$ & 3 & 1 & YES & YES & YES & $1.10$ & $(2,2)$ & -- & 581\\
$(29,9)$ & 8 & $(5,2)$ & 3 & 1 & YES & YES & YES & $1.20$ & $(2,2)$ & NO & 582\\
$(29,9)$ & 8 & $(5,2)$ & 3 & 1 & YES & YES & YES & $1.20$ & $(2,2)$ & -- & 583\\
$(29,11)$ & 7 & $(5,2)$ & 3 & 1 & YES & YES & YES & $1.00$ & $(2,2)$ & -- & 584\\
$(29,12)$ & 7 & $(5,2)$ & 3 & 1 & YES & YES & YES & $1.30$ & $(2,2)$ & -- & 585\\
$(29,8)$ & 7 & $(7,3)$ & 4 & 1 & YES & YES & YES & $0.89$ & $(2,2)$ & -- & 586\\
$(29,8)$ & 7 & $(7,3)$ & 4 & 1 & YES & YES & YES & $1.00$ & $(2,2)$ & NO & 587\\
$(29,11)$ & 7 & $(7,2)$ & 4 & 1 & YES & YES & YES & $1.30$ & $(2,2)$ & -- & 588\\
$(29,11)$ & 7 & $(7,3)$ & 4 & 1 & YES & YES & YES & $1.11$ & $(6,0)$ & -- & 589\\
$(29,11)$ & 7 & $(7,3)$ & 4 & 1 & YES & YES & YES & $1.00$ & $(2,2)$ & 776 & 590\\
$(29,12)$ & 7 & $(7,3)$ & 4 & 1 & YES & YES & YES & $1.18$ & $(2,2)$ & NO & 591\\
$(29,8)$ & 7 & $(8,3)$ & 4 & 1 & YES & YES & YES & $1.20$ & $(2,2)$ & NO & 592\\
$(29,8)$ & 7 & $(8,3)$ & 4 & 1 & YES & YES & YES & $1.20$ & $(2,2)$ & -- & 593\\
$(29,8)$ & 7 & $(8,3)$ & 4 & 1 & YES & YES & YES & $1.00$ & $(2,2)$ & NO & 594\\
$(29,11)$ & 7 & $(10,3)$ & 5 & 1 & YES & YES & YES & $1.27$ & $(4,1)$ & -- & 595\\
$(29,12)$ & 7 & $(10,3)$ & 5 & 1 & YES & YES & YES & $1.12$ & $(2,2)$ & -- & 596\\
$(29,12)$ & 7 & $(10,3)$ & 5 & 1 & YES & YES & YES & $0.88$ & $(2,2)$ & NO & 597\\
$(29,8)$ & 7 & $(13,5)$ & 5 & 1 & YES & YES & YES & $1.27$ & $(4,1)$ & -- & 598\\
$(29,8)$ & 7 & $(13,5)$ & 5 & 1 & YES & YES & YES & $1.00$ & $(2,2)$ & NO & 599\\
$(29,11)$ & 7 & $(13,3)$ & 6 & 1 & YES & YES & YES & $1.00$ & $(6,0)$ & NO & 600\\
$(29,11)$ & 7 & $(13,3)$ & 6 & 1 & YES & YES & YES & $1.18$ & $(4,1)$ & NO & 601\\
$(29,11)$ & 7 & $(13,3)$ & 6 & 1 & YES & YES & YES & $1.18$ & $(4,1)$ & -- & 602\\
$(29,11)$ & 7 & $(13,5)$ & 5 & 1 & YES & YES & YES & $1.18$ & $(2,2)$ & 695 & 603\\
$(29,11)$ & 7 & $(14,3)$ & 6 & 1 & YES & YES & YES & $0.89$ & $(6,0)$ & NO & 604\\
$(29,11)$ & 7 & $(14,3)$ & 6 & 1 & YES & YES & YES & $1.27$ & $(4,1)$ & NO & 605\\
$(29,11)$ & 7 & $(14,3)$ & 6 & 1 & YES & YES & YES & $1.27$ & $(4,1)$ & -- & 606\\
$(29,12)$ & 7 & $(17,4)$ & 7 & 1 & YES & YES & YES & $0.88$ & $(6,0)$ & NO & 607\\
$(29,11)$ & 7 & $(21,8)$ & 6 & 1 & YES & YES & YES & $1.10$ & $(2,2)$ & NO & 608\\
$(29,12)$ & 7 & $(22,9)$ & 7 & 1 & YES & YES & YES & $1.10$ & $(2,2)$ & NO & 609\\
$(29,8)$ & 7 & $(23,5)$ & 7 & 1 & YES & YES & YES & $1.22$ & $(2,2)$ & NO & 610\\
$(29,8)$ & 7 & $(23,7)$ & 7 & 1 & YES & YES & YES & $1.12$ & $(2,2)$ & NO & 611\\
$(29,11)$ & 7 & $(29,11)$ & 7 & 29 & YES & YES & YES & $1.00$ & $(2,2)$ & NO & 612\\
$(30,11)$ & 7 & $(3,1)$ & 2 & 3 & YES & YES & YES & $1.09$ & $(2,2)$ & -- & 613\\
$(30,11)$ & 7 & $(4,1)$ & 3 & 2 & YES & YES & YES & $1.00$ & $(2,2)$ & NO & 614\\
$(30,11)$ & 7 & $(4,1)$ & 3 & 2 & YES & YES & YES & $1.00$ & $(2,2)$ & -- & 615\\
$(30,13)$ & 8 & $(4,1)$ & 3 & 2 & YES & YES & YES & $1.00$ & $(2,2)$ & -- & 616\\
$(30,11)$ & 7 & $(5,2)$ & 3 & 5 & YES & YES & YES & $1.00$ & $(2,2)$ & -- & 617\\
$(30,11)$ & 7 & $(5,2)$ & 3 & 5 & YES & YES & YES & $1.18$ & $(2,2)$ & 684 & 618\\
$(30,13)$ & 8 & $(5,1)$ & 4 & 5 & YES & YES & YES & $1.10$ & $(2,2)$ & NO & 619\\
$(30,13)$ & 8 & $(5,1)$ & 4 & 5 & YES & YES & YES & $1.10$ & $(2,2)$ & -- & 620\\
$(30,11)$ & 7 & $(7,3)$ & 4 & 1 & YES & YES & YES & $1.00$ & $(2,2)$ & 892 & 621\\
$(30,11)$ & 7 & $(7,3)$ & 4 & 1 & YES & YES & YES & $0.88$ & $(2,2)$ & -- & 622\\
$(30,11)$ & 7 & $(9,4)$ & 5 & 3 & YES & YES & YES & $0.88$ & $(2,2)$ & NO & 623\\
$(30,11)$ & 7 & $(10,3)$ & 5 & 10 & YES & YES & YES & $1.12$ & $(2,2)$ & -- & 624\\
$(30,11)$ & 7 & $(11,3)$ & 5 & 1 & YES & YES & YES & $1.12$ & $(2,2)$ & -- & 625\\
$(30,11)$ & 7 & $(30,11)$ & 7 & 30 & YES & YES & YES & $1.10$ & $(2,2)$ & NO & 626\\
$(31,13)$ & 7 & $(2,1)$ & 1 & 1 & YES & YES & YES & $1.18$ & $(2,2)$ & -- & 627\\
$(31,12)$ & 7 & $(3,1)$ & 2 & 1 & YES & YES & YES & $1.09$ & $(2,2)$ & -- & 628\\
$(31,13)$ & 7 & $(3,1)$ & 2 & 1 & YES & YES & YES & $0.89$ & $(2,2)$ & -- & 629\\
$(31,13)$ & 7 & $(3,1)$ & 2 & 1 & YES & YES & YES & $1.00$ & $(2,2)$ & NO & 630\\
$(31,13)$ & 7 & $(3,1)$ & 2 & 1 & YES & YES & YES & $1.00$ & $(2,2)$ & NO & 631\\
$(31,14)$ & 8 & $(3,1)$ & 2 & 1 & YES & YES & YES & $1.00$ & $(2,2)$ & NO & 632\\
$(31,14)$ & 8 & $(3,1)$ & 2 & 1 & YES & YES & YES & $1.00$ & $(2,2)$ & -- & 633\\
$(31,12)$ & 7 & $(4,1)$ & 3 & 1 & YES & YES & YES & $1.17$ & $(2,2)$ & -- & 634\\
$(31,11)$ & 8 & $(5,2)$ & 3 & 1 & YES & YES & YES & $0.88$ & $(4,1)$ & NO & 635\\
$(31,13)$ & 7 & $(5,2)$ & 3 & 1 & YES & YES & YES & $1.00$ & $(2,2)$ & NO & 636\\
$(31,9)$ & 8 & $(7,3)$ & 4 & 1 & YES & YES & YES & $1.22$ & $(2,2)$ & NO & 637\\
$(31,9)$ & 8 & $(7,3)$ & 4 & 1 & YES & YES & YES & $1.22$ & $(2,2)$ & -- & 638\\
$(31,13)$ & 7 & $(7,3)$ & 4 & 1 & YES & YES & YES & $1.00$ & $(2,2)$ & -- & 639\\
$(31,13)$ & 7 & $(7,3)$ & 4 & 1 & YES & YES & NO(2) & $1.09$ & $(4,1)$ & 536 & 640\\
$(31,11)$ & 8 & $(8,3)$ & 4 & 1 & YES & YES & YES & $0.75$ & $(4,1)$ & 435 & 641\\
$(31,12)$ & 7 & $(8,3)$ & 4 & 1 & YES & YES & YES & $1.18$ & $(2,2)$ & NO & 642\\
$(31,12)$ & 7 & $(9,4)$ & 5 & 1 & YES & YES & YES & $1.22$ & $(2,2)$ & -- & 643\\
$(31,9)$ & 8 & $(10,3)$ & 5 & 1 & YES & YES & YES & $1.12$ & $(2,2)$ & -- & 644\\
$(31,13)$ & 7 & $(10,3)$ & 5 & 1 & YES & YES & YES & $1.12$ & $(2,2)$ & NO & 645\\
$(31,9)$ & 8 & $(11,3)$ & 5 & 1 & YES & YES & YES & $1.12$ & $(2,2)$ & -- & 646\\
$(31,13)$ & 7 & $(11,3)$ & 5 & 1 & YES & YES & YES & $1.12$ & $(2,2)$ & NO & 647\\
$(31,13)$ & 7 & $(13,3)$ & 6 & 1 & YES & YES & YES & $1.12$ & $(2,2)$ & NO & 648\\
$(31,13)$ & 7 & $(14,3)$ & 6 & 1 & YES & YES & YES & $1.22$ & $(2,2)$ & 490 & 649\\
$(31,9)$ & 8 & $(16,3)$ & 7 & 1 & YES & YES & YES & $1.00$ & $(2,2)$ & -- & 650\\
$(31,7)$ & 8 & $(17,5)$ & 6 & 1 & YES & YES & YES & $1.22$ & $(2,2)$ & NO & 651\\
$(31,12)$ & 7 & $(18,7)$ & 6 & 1 & YES & YES & YES & $1.25$ & $(2,2)$ & NO & 652\\
$(31,7)$ & 8 & $(19,4)$ & 7 & 1 & YES & YES & YES & $1.00$ & $(2,2)$ & -- & 653\\
$(31,7)$ & 8 & $(19,7)$ & 6 & 1 & YES & YES & YES & $1.22$ & $(2,2)$ & -- & 654\\
$(31,9)$ & 8 & $(23,7)$ & 7 & 1 & YES & YES & YES & $1.11$ & $(2,2)$ & 1237 & 655\\
$(32,9)$ & 8 & $(7,3)$ & 4 & 1 & YES & YES & YES & $1.22$ & $(2,2)$ & NO & 656\\
$(32,9)$ & 8 & $(7,3)$ & 4 & 1 & YES & YES & YES & $1.22$ & $(2,2)$ & -- & 657\\
$(32,9)$ & 8 & $(10,3)$ & 5 & 2 & YES & YES & YES & $1.12$ & $(2,2)$ & -- & 658\\
$(32,9)$ & 8 & $(19,4)$ & 7 & 1 & YES & YES & YES & $0.75$ & $(6,0)$ & NO & 659\\
$(32,7)$ & 8 & $(21,5)$ & 8 & 1 & YES & YES & YES & $1.00$ & $(2,2)$ & NO & 660\\
$(33,14)$ & 8 & $(3,1)$ & 2 & 3 & YES & YES & YES & $1.00$ & $(2,2)$ & -- & 661\\
$(33,10)$ & 8 & $(4,1)$ & 3 & 1 & YES & YES & YES & $1.18$ & $(2,2)$ & NO & 662\\
$(33,10)$ & 8 & $(4,1)$ & 3 & 1 & YES & YES & YES & $1.18$ & $(2,2)$ & -- & 663\\
$(33,10)$ & 8 & $(4,1)$ & 3 & 1 & YES & YES & YES & $1.18$ & $(2,2)$ & NO & 664\\
$(33,14)$ & 8 & $(5,1)$ & 4 & 1 & YES & YES & NO(2) & $1.00$ & $(4,1)$ & NO & 665\\
$(33,14)$ & 8 & $(6,1)$ & 5 & 3 & YES & YES & YES & $1.00$ & $(2,2)$ & NO & 666\\
$(33,14)$ & 8 & $(6,1)$ & 5 & 3 & YES & YES & YES & $1.00$ & $(2,2)$ & -- & 667\\
$(33,10)$ & 8 & $(7,3)$ & 4 & 1 & YES & YES & YES & $1.18$ & $(2,2)$ & NO & 668\\
$(33,14)$ & 8 & $(7,2)$ & 4 & 1 & YES & YES & YES & $1.12$ & $(2,2)$ & -- & 669\\
$(33,10)$ & 8 & $(10,3)$ & 5 & 1 & YES & YES & YES & $1.22$ & $(2,2)$ & -- & 670\\
$(33,14)$ & 8 & $(11,2)$ & 6 & 11 & YES & YES & YES & $0.88$ & $(2,2)$ & -- & 671\\
$(33,10)$ & 8 & $(13,4)$ & 6 & 1 & YES & YES & YES & $1.10$ & $(2,2)$ & 717 & 672\\
$(33,10)$ & 8 & $(14,3)$ & 6 & 1 & YES & YES & YES & $1.11$ & $(2,2)$ & -- & 673\\
$(33,14)$ & 8 & $(19,8)$ & 6 & 1 & YES & YES & NO(2) & $1.00$ & $(4,1)$ & 860 & 674\\
$(33,14)$ & 8 & $(26,11)$ & 7 & 1 & YES & YES & YES & $1.00$ & $(2,2)$ & NO & 675\\
$(33,14)$ & 8 & $(31,13)$ & 7 & 1 & YES & YES & YES & $1.00$ & $(2,2)$ & 1017 & 676\\
$(33,14)$ & 8 & $(33,14)$ & 8 & 33 & YES & YES & YES & $1.00$ & $(2,2)$ & NO & 677\\
$(34,9)$ & 8 & $(2,1)$ & 1 & 2 & YES & YES & YES & $0.88$ & $(4,1)$ & NO & 678\\
$(34,13)$ & 7 & $(2,1)$ & 1 & 2 & YES & YES & YES & $1.00$ & $(2,2)$ & -- & 679\\
$(34,13)$ & 7 & $(2,1)$ & 1 & 2 & YES & YES & YES & $1.00$ & $(2,2)$ & NO & 680\\
$(34,9)$ & 8 & $(3,1)$ & 2 & 1 & YES & YES & YES & $0.88$ & $(4,1)$ & NO & 681\\
$(34,9)$ & 8 & $(3,1)$ & 2 & 1 & YES & YES & YES & $0.88$ & $(4,1)$ & -- & 682\\
$(34,13)$ & 7 & $(3,1)$ & 2 & 1 & YES & YES & YES & $0.89$ & $(2,2)$ & -- & 683\\
$(34,13)$ & 7 & $(3,1)$ & 2 & 1 & YES & YES & YES & $1.18$ & $(2,2)$ & 618 & 684\\
$(34,15)$ & 8 & $(3,1)$ & 2 & 1 & YES & YES & YES & $1.11$ & $(2,2)$ & NO & 685\\
$(34,15)$ & 8 & $(3,1)$ & 2 & 1 & YES & YES & YES & $1.11$ & $(2,2)$ & -- & 686\\
$(34,15)$ & 8 & $(4,1)$ & 3 & 2 & YES & YES & YES & $1.00$ & $(2,2)$ & NO & 687\\
$(34,15)$ & 8 & $(4,1)$ & 3 & 2 & YES & YES & YES & $1.00$ & $(2,2)$ & -- & 688\\
$(34,13)$ & 7 & $(5,2)$ & 3 & 1 & YES & YES & YES & $1.00$ & $(2,2)$ & NO & 689\\
$(34,9)$ & 8 & $(7,3)$ & 4 & 1 & YES & YES & YES & $1.00$ & $(2,2)$ & NO & 690\\
$(34,13)$ & 7 & $(7,3)$ & 4 & 1 & YES & YES & YES & $1.00$ & $(6,0)$ & -- & 691\\
$(34,13)$ & 7 & $(7,3)$ & 4 & 1 & YES & YES & YES & $1.20$ & $(2,2)$ & NO & 692\\
$(34,9)$ & 8 & $(8,3)$ & 4 & 2 & YES & YES & YES & $1.00$ & $(2,2)$ & NO & 693\\
$(34,13)$ & 7 & $(8,3)$ & 4 & 2 & YES & YES & YES & $1.33$ & $(4,1)$ & -- & 694\\
$(34,13)$ & 7 & $(8,3)$ & 4 & 2 & YES & YES & YES & $1.18$ & $(2,2)$ & 603 & 695\\
$(34,15)$ & 8 & $(8,3)$ & 4 & 2 & YES & YES & YES & $1.33$ & $(2,2)$ & -- & 696\\
$(34,13)$ & 7 & $(11,3)$ & 5 & 1 & YES & YES & YES & $1.27$ & $(4,1)$ & -- & 697\\
$(34,13)$ & 7 & $(11,3)$ & 5 & 1 & YES & YES & YES & $1.00$ & $(2,2)$ & NO & 698\\
$(34,15)$ & 8 & $(11,3)$ & 5 & 1 & YES & YES & YES & $1.22$ & $(2,2)$ & NO & 699\\
$(34,13)$ & 7 & $(13,3)$ & 6 & 1 & YES & YES & YES & $1.00$ & $(6,0)$ & NO & 700\\
$(34,13)$ & 7 & $(13,3)$ & 6 & 1 & YES & YES & YES & $1.25$ & $(4,1)$ & -- & 701\\
$(34,13)$ & 7 & $(13,3)$ & 6 & 1 & YES & YES & YES & $1.33$ & $(4,1)$ & NO & 702\\
$(34,13)$ & 7 & $(31,12)$ & 7 & 1 & YES & YES & YES & $1.00$ & $(6,0)$ & NO & 703\\
$(35,13)$ & 8 & $(4,1)$ & 3 & 1 & YES & YES & YES & $1.11$ & $(2,2)$ & NO & 704\\
$(35,13)$ & 8 & $(4,1)$ & 3 & 1 & YES & YES & YES & $1.11$ & $(2,2)$ & -- & 705\\
$(35,6)$ & 10 & $(5,2)$ & 3 & 5 & YES & YES & YES & $0.75$ & $(4,1)$ & NO & 706\\
$(35,6)$ & 10 & $(5,2)$ & 3 & 5 & YES & YES & YES & $0.75$ & $(4,1)$ & -- & 707\\
$(35,6)$ & 10 & $(9,2)$ & 5 & 1 & YES & YES & YES & $0.75$ & $(4,1)$ & NO & 708\\
$(35,8)$ & 8 & $(13,4)$ & 6 & 1 & YES & YES & YES & $1.00$ & $(2,2)$ & NO & 709\\
$(35,13)$ & 8 & $(14,5)$ & 6 & 7 & YES & YES & YES & $1.11$ & $(2,2)$ & NO & 710\\
$(35,8)$ & 8 & $(17,5)$ & 6 & 1 & YES & YES & YES & $1.22$ & $(2,2)$ & NO & 711\\
$(36,11)$ & 8 & $(2,1)$ & 1 & 2 & YES & YES & YES & $1.20$ & $(2,2)$ & NO & 712\\
$(36,11)$ & 8 & $(5,1)$ & 4 & 1 & YES & YES & YES & $1.10$ & $(2,2)$ & NO & 713\\
$(36,11)$ & 8 & $(5,1)$ & 4 & 1 & YES & YES & YES & $1.10$ & $(2,2)$ & -- & 714\\
$(36,11)$ & 8 & $(5,2)$ & 3 & 1 & YES & YES & YES & $1.10$ & $(2,2)$ & -- & 715\\
$(36,13)$ & 8 & $(7,3)$ & 4 & 1 & YES & YES & YES & $1.18$ & $(2,2)$ & NO & 716\\
$(36,11)$ & 8 & $(10,3)$ & 5 & 2 & YES & YES & YES & $1.10$ & $(2,2)$ & 672 & 717\\
$(36,13)$ & 8 & $(10,3)$ & 5 & 2 & YES & YES & YES & $1.22$ & $(2,2)$ & -- & 718\\
$(36,11)$ & 8 & $(13,4)$ & 6 & 1 & YES & YES & YES & $1.20$ & $(2,2)$ & NO & 719\\
$(36,13)$ & 8 & $(14,3)$ & 6 & 2 & YES & YES & YES & $1.22$ & $(2,2)$ & -- & 720\\
$(36,11)$ & 8 & $(15,4)$ & 6 & 3 & YES & YES & YES & $1.00$ & $(2,2)$ & NO & 721\\
$(36,11)$ & 8 & $(27,8)$ & 7 & 9 & YES & YES & YES & $1.00$ & $(2,2)$ & NO & 722\\
$(37,11)$ & 8 & $(2,1)$ & 1 & 1 & YES & YES & YES & $1.27$ & $(2,2)$ & NO & 723\\
$(37,11)$ & 8 & $(3,1)$ & 2 & 1 & YES & YES & YES & $1.27$ & $(2,2)$ & NO & 724\\
$(37,11)$ & 8 & $(3,1)$ & 2 & 1 & YES & YES & YES & $1.27$ & $(2,2)$ & -- & 725\\
$(37,11)$ & 8 & $(5,2)$ & 3 & 1 & YES & YES & YES & $1.00$ & $(2,2)$ & NO & 726\\
$(37,11)$ & 8 & $(5,2)$ & 3 & 1 & YES & YES & YES & $1.00$ & $(2,2)$ & -- & 727\\
$(37,16)$ & 9 & $(6,1)$ & 5 & 1 & YES & YES & YES & $1.18$ & $(2,2)$ & -- & 728\\
$(37,11)$ & 8 & $(7,3)$ & 4 & 1 & YES & YES & YES & $1.12$ & $(2,2)$ & -- & 729\\
$(37,16)$ & 9 & $(7,1)$ & 6 & 1 & YES & YES & YES & $1.11$ & $(2,2)$ & NO & 730\\
$(37,16)$ & 9 & $(7,1)$ & 6 & 1 & YES & YES & YES & $1.11$ & $(2,2)$ & NO & 731\\
$(37,11)$ & 8 & $(8,3)$ & 4 & 1 & YES & YES & YES & $1.11$ & $(6,0)$ & -- & 732\\
$(37,11)$ & 8 & $(11,3)$ & 5 & 1 & YES & YES & YES & $1.42$ & $(4,1)$ & -- & 733\\
$(37,11)$ & 8 & $(11,3)$ & 5 & 1 & YES & YES & YES & $1.12$ & $(2,2)$ & 1031 & 734\\
$(37,14)$ & 8 & $(11,2)$ & 6 & 1 & YES & YES & YES & $1.00$ & $(2,2)$ & -- & 735\\
$(37,14)$ & 8 & $(11,2)$ & 6 & 1 & YES & YES & YES & $1.33$ & $(2,2)$ & NO & 736\\
$(37,14)$ & 8 & $(11,2)$ & 6 & 1 & YES & YES & YES & $1.11$ & $(2,2)$ & NO & 737\\
$(37,10)$ & 8 & $(13,4)$ & 6 & 1 & YES & YES & YES & $0.75$ & $(6,0)$ & -- & 738\\
$(37,11)$ & 8 & $(13,3)$ & 6 & 1 & YES & YES & YES & $1.00$ & $(2,2)$ & NO & 739\\
$(37,11)$ & 8 & $(14,3)$ & 6 & 1 & YES & YES & YES & $1.00$ & $(6,0)$ & NO & 740\\
$(37,11)$ & 8 & $(15,4)$ & 6 & 1 & YES & YES & YES & $1.00$ & $(6,0)$ & NO & 741\\
$(37,11)$ & 8 & $(18,5)$ & 6 & 1 & YES & YES & YES & $1.00$ & $(6,0)$ & NO & 742\\
$(37,14)$ & 8 & $(29,11)$ & 7 & 1 & YES & YES & YES & $1.00$ & $(2,2)$ & NO & 743\\
$(37,16)$ & 9 & $(30,13)$ & 8 & 1 & YES & YES & YES & $1.18$ & $(2,2)$ & NO & 744\\
$(37,11)$ & 8 & $(31,9)$ & 8 & 1 & YES & YES & YES & $1.00$ & $(2,2)$ & NO & 745\\
$(37,10)$ & 8 & $(32,9)$ & 8 & 1 & YES & YES & YES & $0.75$ & $(6,0)$ & NO & 746\\
$(37,16)$ & 9 & $(37,16)$ & 9 & 37 & YES & YES & YES & $1.11$ & $(2,2)$ & NO & 747\\
$(38,11)$ & 9 & $(24,7)$ & 7 & 2 & YES & YES & YES & $0.89$ & $(2,2)$ & 978 & 748\\
$(39,16)$ & 8 & $(2,1)$ & 1 & 1 & YES & YES & YES & $1.20$ & $(2,2)$ & NO & 749\\
$(39,17)$ & 8 & $(2,1)$ & 1 & 1 & YES & YES & YES & $1.10$ & $(2,2)$ & -- & 750\\
$(39,14)$ & 8 & $(3,1)$ & 2 & 3 & YES & YES & YES & $0.89$ & $(2,2)$ & -- & 751\\
$(39,16)$ & 8 & $(3,1)$ & 2 & 3 & YES & YES & YES & $1.00$ & $(2,2)$ & NO & 752\\
$(39,16)$ & 8 & $(3,1)$ & 2 & 3 & YES & YES & YES & $1.00$ & $(2,2)$ & -- & 753\\
$(39,16)$ & 8 & $(5,1)$ & 4 & 1 & YES & YES & YES & $1.10$ & $(2,2)$ & -- & 754\\
$(39,16)$ & 8 & $(5,2)$ & 3 & 1 & YES & YES & YES & $1.20$ & $(2,2)$ & 561 & 755\\
$(39,16)$ & 8 & $(7,2)$ & 4 & 1 & YES & YES & YES & $1.00$ & $(2,2)$ & -- & 756\\
$(39,14)$ & 8 & $(8,3)$ & 4 & 1 & YES & YES & YES & $1.00$ & $(2,2)$ & 897 & 757\\
$(39,14)$ & 8 & $(13,3)$ & 6 & 13 & YES & YES & YES & $1.33$ & $(2,2)$ & -- & 758\\
$(39,17)$ & 8 & $(13,5)$ & 5 & 13 & YES & YES & YES & $1.22$ & $(2,2)$ & NO & 759\\
$(39,16)$ & 8 & $(19,8)$ & 6 & 1 & YES & YES & YES & $1.00$ & $(2,2)$ & NO & 760\\
$(39,17)$ & 8 & $(34,15)$ & 8 & 1 & YES & YES & YES & $1.22$ & $(2,2)$ & NO & 761\\
$(40,11)$ & 8 & $(5,2)$ & 3 & 5 & YES & YES & YES & $1.12$ & $(2,2)$ & -- & 762\\
$(40,11)$ & 8 & $(7,3)$ & 4 & 1 & YES & YES & YES & $1.00$ & $(6,0)$ & -- & 763\\
$(40,11)$ & 8 & $(10,3)$ & 5 & 10 & YES & YES & YES & $1.25$ & $(2,2)$ & 1001 & 764\\
$(40,9)$ & 9 & $(11,4)$ & 5 & 1 & YES & YES & YES & $0.88$ & $(6,0)$ & -- & 765\\
$(40,11)$ & 8 & $(17,5)$ & 6 & 1 & YES & YES & YES & $1.00$ & $(6,0)$ & NO & 766\\
$(41,15)$ & 8 & $(2,1)$ & 1 & 1 & YES & YES & YES & $1.11$ & $(2,2)$ & NO & 767\\
$(41,16)$ & 8 & $(2,1)$ & 1 & 1 & YES & YES & YES & $1.00$ & $(2,2)$ & -- & 768\\
$(41,16)$ & 8 & $(2,1)$ & 1 & 1 & YES & YES & YES & $1.00$ & $(2,2)$ & NO & 769\\
$(41,17)$ & 8 & $(2,1)$ & 1 & 1 & YES & YES & YES & $1.00$ & $(2,2)$ & NO & 770\\
$(41,17)$ & 8 & $(2,1)$ & 1 & 1 & NO & YES & NO(2) & $1.17$ & $(2,2)$ & -- & 771\\
$(41,15)$ & 8 & $(3,1)$ & 2 & 1 & YES & YES & YES & $1.00$ & $(2,2)$ & -- & 772\\
$(41,16)$ & 8 & $(3,1)$ & 2 & 1 & YES & YES & YES & $1.00$ & $(2,2)$ & NO & 773\\
$(41,16)$ & 8 & $(3,1)$ & 2 & 1 & YES & YES & YES & $1.00$ & $(2,2)$ & -- & 774\\
$(41,17)$ & 8 & $(3,1)$ & 2 & 1 & YES & YES & YES & $1.30$ & $(2,2)$ & -- & 775\\
$(41,17)$ & 8 & $(3,1)$ & 2 & 1 & YES & YES & YES & $1.00$ & $(2,2)$ & 590 & 776\\
$(41,17)$ & 8 & $(3,1)$ & 2 & 1 & YES & YES & YES & $1.00$ & $(2,2)$ & NO & 777\\
$(41,15)$ & 8 & $(4,1)$ & 3 & 1 & YES & YES & YES & $1.00$ & $(2,2)$ & -- & 778\\
$(41,17)$ & 8 & $(4,1)$ & 3 & 1 & YES & YES & YES & $1.30$ & $(2,2)$ & NO & 779\\
$(41,17)$ & 8 & $(4,1)$ & 3 & 1 & YES & YES & YES & $1.30$ & $(2,2)$ & -- & 780\\
$(41,16)$ & 8 & $(5,1)$ & 4 & 1 & YES & YES & YES & $0.88$ & $(2,2)$ & NO & 781\\
$(41,16)$ & 8 & $(5,1)$ & 4 & 1 & YES & YES & YES & $0.88$ & $(2,2)$ & -- & 782\\
$(41,16)$ & 8 & $(5,2)$ & 3 & 1 & YES & YES & YES & $1.20$ & $(2,2)$ & 576 & 783\\
$(41,11)$ & 8 & $(7,3)$ & 4 & 1 & YES & YES & YES & $1.00$ & $(2,2)$ & -- & 784\\
$(41,12)$ & 8 & $(7,3)$ & 4 & 1 & YES & YES & YES & $1.12$ & $(2,2)$ & -- & 785\\
$(41,15)$ & 8 & $(7,2)$ & 4 & 1 & YES & YES & YES & $1.00$ & $(2,2)$ & NO & 786\\
$(41,15)$ & 8 & $(7,3)$ & 4 & 1 & YES & YES & YES & $1.00$ & $(2,2)$ & NO & 787\\
$(41,16)$ & 8 & $(7,2)$ & 4 & 1 & YES & YES & YES & $1.00$ & $(2,2)$ & -- & 788\\
$(41,16)$ & 8 & $(7,3)$ & 4 & 1 & YES & YES & YES & $1.00$ & $(2,2)$ & NO & 789\\
$(41,11)$ & 8 & $(8,3)$ & 4 & 1 & YES & YES & YES & $1.22$ & $(2,2)$ & NO & 790\\
$(41,11)$ & 8 & $(8,3)$ & 4 & 1 & YES & YES & YES & $1.22$ & $(2,2)$ & -- & 791\\
$(41,12)$ & 8 & $(8,3)$ & 4 & 1 & YES & YES & YES & $1.00$ & $(6,0)$ & -- & 792\\
$(41,12)$ & 8 & $(8,3)$ & 4 & 1 & YES & YES & YES & $1.50$ & $(4,1)$ & NO & 793\\
$(41,15)$ & 8 & $(9,2)$ & 5 & 1 & YES & YES & YES & $1.00$ & $(2,2)$ & -- & 794\\
$(41,15)$ & 8 & $(9,2)$ & 5 & 1 & YES & YES & YES & $1.11$ & $(2,2)$ & NO & 795\\
$(41,12)$ & 8 & $(10,3)$ & 5 & 1 & YES & YES & YES & $1.42$ & $(4,1)$ & -- & 796\\
$(41,11)$ & 8 & $(11,3)$ & 5 & 1 & YES & YES & YES & $1.00$ & $(2,2)$ & -- & 797\\
$(41,12)$ & 8 & $(11,3)$ & 5 & 1 & YES & YES & YES & $1.33$ & $(4,1)$ & -- & 798\\
$(41,15)$ & 8 & $(11,3)$ & 5 & 1 & YES & YES & YES & $1.11$ & $(2,2)$ & NO & 799\\
$(41,17)$ & 8 & $(11,3)$ & 5 & 1 & YES & YES & YES & $1.22$ & $(2,2)$ & NO & 800\\
$(41,11)$ & 8 & $(13,4)$ & 6 & 1 & YES & YES & YES & $1.00$ & $(2,2)$ & NO & 801\\
$(41,16)$ & 8 & $(13,3)$ & 6 & 1 & YES & YES & YES & $1.22$ & $(2,2)$ & -- & 802\\
$(41,16)$ & 8 & $(18,7)$ & 6 & 1 & YES & YES & YES & $0.88$ & $(2,2)$ & NO & 803\\
$(41,17)$ & 8 & $(19,8)$ & 6 & 1 & YES & YES & YES & $1.11$ & $(2,2)$ & NO & 804\\
$(41,12)$ & 8 & $(23,7)$ & 7 & 1 & YES & YES & YES & $1.12$ & $(2,2)$ & NO & 805\\
$(41,11)$ & 8 & $(29,8)$ & 7 & 1 & YES & YES & YES & $1.00$ & $(2,2)$ & NO & 806\\
$(41,12)$ & 8 & $(37,11)$ & 8 & 1 & YES & YES & YES & $1.00$ & $(6,0)$ & NO & 807\\
$(41,15)$ & 8 & $(41,15)$ & 8 & 41 & YES & YES & YES & $1.00$ & $(2,2)$ & NO & 808\\
$(41,17)$ & 8 & $(41,17)$ & 8 & 41 & YES & YES & YES & $1.20$ & $(2,2)$ & NO & 809\\
$(42,13)$ & 9 & $(2,1)$ & 1 & 2 & YES & YES & YES & $1.20$ & $(2,2)$ & NO & 810\\
$(42,13)$ & 9 & $(5,2)$ & 3 & 1 & YES & YES & YES & $0.88$ & $(2,2)$ & -- & 811\\
$(42,19)$ & 9 & $(6,1)$ & 5 & 6 & YES & YES & YES & $0.88$ & $(4,1)$ & NO & 812\\
$(42,19)$ & 9 & $(6,1)$ & 5 & 6 & YES & YES & YES & $0.88$ & $(4,1)$ & -- & 813\\
$(42,13)$ & 9 & $(8,3)$ & 4 & 2 & YES & YES & YES & $1.33$ & $(2,2)$ & -- & 814\\
$(42,13)$ & 9 & $(18,5)$ & 6 & 6 & YES & YES & YES & $1.33$ & $(2,2)$ & NO & 815\\
$(43,16)$ & 9 & $(4,1)$ & 3 & 1 & YES & YES & YES & $1.11$ & $(2,2)$ & -- & 816\\
$(43,16)$ & 9 & $(5,1)$ & 4 & 1 & YES & YES & YES & $1.00$ & $(2,2)$ & -- & 817\\
$(43,18)$ & 8 & $(5,2)$ & 3 & 1 & YES & YES & YES & $1.00$ & $(2,2)$ & -- & 818\\
$(43,19)$ & 9 & $(5,1)$ & 4 & 1 & YES & YES & YES & $1.11$ & $(2,2)$ & NO & 819\\
$(43,19)$ & 9 & $(5,1)$ & 4 & 1 & YES & YES & YES & $1.11$ & $(2,2)$ & -- & 820\\
$(43,16)$ & 9 & $(6,1)$ & 5 & 1 & YES & YES & YES & $1.11$ & $(2,2)$ & -- & 821\\
$(43,12)$ & 8 & $(7,3)$ & 4 & 1 & YES & YES & YES & $1.33$ & $(2,2)$ & -- & 822\\
$(43,19)$ & 9 & $(7,1)$ & 6 & 1 & YES & YES & YES & $1.11$ & $(2,2)$ & NO & 823\\
$(43,19)$ & 9 & $(7,1)$ & 6 & 1 & YES & YES & YES & $1.11$ & $(2,2)$ & NO & 824\\
$(43,12)$ & 8 & $(8,3)$ & 4 & 1 & YES & YES & YES & $1.20$ & $(6,0)$ & -- & 825\\
$(43,12)$ & 8 & $(9,4)$ & 5 & 1 & YES & YES & YES & $1.33$ & $(2,2)$ & NO & 826\\
$(43,19)$ & 9 & $(9,4)$ & 5 & 1 & YES & YES & YES & $1.36$ & $(2,2)$ & NO & 827\\
$(43,12)$ & 8 & $(11,4)$ & 5 & 1 & YES & YES & YES & $1.33$ & $(2,2)$ & NO & 828\\
$(43,18)$ & 8 & $(11,2)$ & 6 & 1 & YES & YES & YES & $1.22$ & $(2,2)$ & -- & 829\\
$(43,16)$ & 9 & $(19,7)$ & 6 & 1 & YES & YES & YES & $1.11$ & $(2,2)$ & NO & 830\\
$(43,18)$ & 8 & $(26,11)$ & 7 & 1 & YES & YES & YES & $1.12$ & $(2,2)$ & 1179 & 831\\
$(43,16)$ & 9 & $(27,10)$ & 7 & 1 & YES & YES & YES & $1.00$ & $(2,2)$ & 1054 & 832\\
$(43,16)$ & 9 & $(35,13)$ & 8 & 1 & YES & YES & YES & $1.11$ & $(2,2)$ & NO & 833\\
$(43,10)$ & 9 & $(40,9)$ & 9 & 1 & YES & YES & YES & $0.75$ & $(6,0)$ & NO & 834\\
$(43,12)$ & 8 & $(40,11)$ & 8 & 1 & YES & YES & YES & $1.10$ & $(6,0)$ & NO & 835\\
$(43,19)$ & 9 & $(43,19)$ & 9 & 43 & YES & YES & YES & $1.11$ & $(2,2)$ & NO & 836\\
$(44,17)$ & 8 & $(2,1)$ & 1 & 2 & YES & YES & YES & $1.00$ & $(2,2)$ & 534 & 837\\
$(44,17)$ & 8 & $(2,1)$ & 1 & 2 & YES & YES & YES & $1.00$ & $(2,2)$ & -- & 838\\
$(44,17)$ & 8 & $(3,1)$ & 2 & 1 & YES & YES & YES & $1.00$ & $(2,2)$ & NO & 839\\
$(44,17)$ & 8 & $(3,1)$ & 2 & 1 & YES & YES & YES & $1.00$ & $(2,2)$ & -- & 840\\
$(44,17)$ & 8 & $(5,2)$ & 3 & 1 & YES & YES & YES & $1.18$ & $(2,2)$ & NO & 841\\
$(44,13)$ & 8 & $(7,3)$ & 4 & 1 & YES & YES & YES & $1.00$ & $(2,2)$ & -- & 842\\
$(44,13)$ & 8 & $(7,3)$ & 4 & 1 & YES & YES & YES & $1.00$ & $(2,2)$ & NO & 843\\
$(44,17)$ & 8 & $(7,2)$ & 4 & 1 & YES & YES & YES & $1.27$ & $(4,1)$ & -- & 844\\
$(44,17)$ & 8 & $(7,3)$ & 4 & 1 & YES & YES & YES & $1.33$ & $(2,2)$ & -- & 845\\
$(44,17)$ & 8 & $(9,2)$ & 5 & 1 & YES & YES & YES & $1.18$ & $(4,1)$ & NO & 846\\
$(44,17)$ & 8 & $(9,2)$ & 5 & 1 & YES & YES & YES & $1.18$ & $(4,1)$ & -- & 847\\
$(44,13)$ & 8 & $(13,3)$ & 6 & 1 & YES & YES & YES & $1.00$ & $(2,2)$ & NO & 848\\
$(44,13)$ & 8 & $(15,4)$ & 6 & 1 & YES & YES & YES & $1.00$ & $(2,2)$ & NO & 849\\
$(44,13)$ & 8 & $(18,5)$ & 6 & 2 & YES & YES & YES & $1.00$ & $(2,2)$ & NO & 850\\
$(44,17)$ & 8 & $(21,8)$ & 6 & 1 & YES & YES & YES & $1.00$ & $(4,1)$ & NO & 851\\
$(44,13)$ & 8 & $(23,7)$ & 7 & 1 & YES & YES & YES & $1.00$ & $(2,2)$ & NO & 852\\
$(44,13)$ & 8 & $(31,9)$ & 8 & 1 & YES & YES & YES & $1.00$ & $(2,2)$ & NO & 853\\
$(44,13)$ & 8 & $(41,12)$ & 8 & 1 & YES & YES & YES & $1.00$ & $(2,2)$ & NO & 854\\
$(45,14)$ & 9 & $(2,1)$ & 1 & 1 & YES & YES & YES & $1.11$ & $(2,2)$ & NO & 855\\
$(45,19)$ & 8 & $(2,1)$ & 1 & 1 & YES & YES & YES & $1.18$ & $(2,2)$ & -- & 856\\
$(45,14)$ & 9 & $(5,1)$ & 4 & 5 & YES & YES & YES & $1.00$ & $(2,2)$ & NO & 857\\
$(45,19)$ & 8 & $(5,1)$ & 4 & 5 & YES & YES & NO(2) & $0.90$ & $(4,1)$ & NO & 858\\
$(45,19)$ & 8 & $(5,2)$ & 3 & 5 & YES & YES & YES & $1.00$ & $(2,2)$ & -- & 859\\
$(45,19)$ & 8 & $(7,3)$ & 4 & 1 & YES & YES & NO(2) & $1.00$ & $(4,1)$ & 674 & 860\\
$(45,14)$ & 9 & $(29,9)$ & 8 & 1 & YES & YES & YES & $1.00$ & $(2,2)$ & NO & 861\\
$(46,19)$ & 8 & $(2,1)$ & 1 & 2 & YES & YES & YES & $1.00$ & $(2,2)$ & NO & 862\\
$(46,19)$ & 8 & $(3,1)$ & 2 & 1 & YES & YES & YES & $0.89$ & $(2,2)$ & -- & 863\\
$(46,19)$ & 8 & $(3,1)$ & 2 & 1 & YES & YES & YES & $1.00$ & $(2,2)$ & NO & 864\\
$(46,19)$ & 8 & $(5,2)$ & 3 & 1 & YES & YES & YES & $1.00$ & $(2,2)$ & -- & 865\\
$(46,19)$ & 8 & $(5,2)$ & 3 & 1 & YES & YES & YES & $1.00$ & $(2,2)$ & NO & 866\\
$(46,19)$ & 8 & $(7,2)$ & 4 & 1 & YES & YES & YES & $1.11$ & $(2,2)$ & NO & 867\\
$(46,19)$ & 8 & $(9,2)$ & 5 & 1 & YES & YES & YES & $1.11$ & $(2,2)$ & NO & 868\\
$(46,19)$ & 8 & $(19,8)$ & 6 & 1 & YES & YES & YES & $1.12$ & $(2,2)$ & NO & 869\\
$(47,18)$ & 8 & $(3,1)$ & 2 & 1 & YES & YES & YES & $1.30$ & $(2,2)$ & NO & 870\\
$(47,18)$ & 8 & $(3,1)$ & 2 & 1 & YES & YES & YES & $1.30$ & $(2,2)$ & -- & 871\\
$(47,18)$ & 8 & $(4,1)$ & 3 & 1 & YES & YES & YES & $1.30$ & $(2,2)$ & NO & 872\\
$(47,18)$ & 8 & $(4,1)$ & 3 & 1 & YES & YES & YES & $1.30$ & $(2,2)$ & -- & 873\\
$(47,18)$ & 8 & $(5,1)$ & 4 & 1 & YES & YES & YES & $1.20$ & $(2,2)$ & -- & 874\\
$(47,18)$ & 8 & $(5,1)$ & 4 & 1 & YES & YES & YES & $1.30$ & $(2,2)$ & NO & 875\\
$(47,18)$ & 8 & $(5,2)$ & 3 & 1 & YES & YES & YES & $1.00$ & $(6,0)$ & -- & 876\\
$(47,13)$ & 8 & $(7,3)$ & 4 & 1 & YES & YES & YES & $1.22$ & $(2,2)$ & NO & 877\\
$(47,14)$ & 9 & $(7,3)$ & 4 & 1 & YES & YES & YES & $1.11$ & $(4,1)$ & -- & 878\\
$(47,18)$ & 8 & $(7,2)$ & 4 & 1 & YES & YES & YES & $1.27$ & $(4,1)$ & -- & 879\\
$(47,18)$ & 8 & $(7,3)$ & 4 & 1 & YES & YES & YES & $1.00$ & $(6,0)$ & NO & 880\\
$(47,13)$ & 8 & $(8,3)$ & 4 & 1 & YES & YES & YES & $1.22$ & $(2,2)$ & NO & 881\\
$(47,18)$ & 8 & $(9,2)$ & 5 & 1 & YES & YES & YES & $1.27$ & $(4,1)$ & NO & 882\\
$(47,18)$ & 8 & $(9,2)$ & 5 & 1 & YES & YES & YES & $1.33$ & $(4,1)$ & -- & 883\\
$(47,18)$ & 8 & $(11,2)$ & 6 & 1 & YES & YES & YES & $1.22$ & $(2,2)$ & NO & 884\\
$(47,18)$ & 8 & $(11,2)$ & 6 & 1 & YES & YES & YES & $1.22$ & $(2,2)$ & -- & 885\\
$(47,13)$ & 8 & $(13,4)$ & 6 & 1 & YES & YES & YES & $1.22$ & $(2,2)$ & NO & 886\\
$(47,13)$ & 8 & $(17,5)$ & 6 & 1 & YES & YES & YES & $1.22$ & $(2,2)$ & NO & 887\\
$(47,18)$ & 8 & $(18,7)$ & 6 & 1 & YES & YES & YES & $1.00$ & $(6,0)$ & NO & 888\\
$(47,18)$ & 8 & $(21,8)$ & 6 & 1 & YES & YES & YES & $1.30$ & $(2,2)$ & 983 & 889\\
$(47,18)$ & 8 & $(29,11)$ & 7 & 1 & YES & YES & YES & $1.27$ & $(4,1)$ & 1293 & 890\\
$(47,18)$ & 8 & $(47,18)$ & 8 & 47 & YES & YES & YES & $1.20$ & $(2,2)$ & NO & 891\\
$(49,18)$ & 8 & $(2,1)$ & 1 & 1 & YES & YES & YES & $1.00$ & $(2,2)$ & 621 & 892\\
$(49,19)$ & 8 & $(2,1)$ & 1 & 1 & YES & YES & NO(2) & $1.00$ & $(4,1)$ & -- & 893\\
$(49,20)$ & 9 & $(2,1)$ & 1 & 1 & YES & YES & YES & $1.00$ & $(2,2)$ & NO & 894\\
$(49,15)$ & 9 & $(3,1)$ & 2 & 1 & NO & YES & YES & $1.27$ & $(2,2)$ & -- & 895\\
$(49,18)$ & 8 & $(3,1)$ & 2 & 1 & YES & YES & YES & $0.89$ & $(2,2)$ & -- & 896\\
$(49,18)$ & 8 & $(3,1)$ & 2 & 1 & YES & YES & YES & $1.00$ & $(2,2)$ & 757 & 897\\
$(49,19)$ & 8 & $(3,1)$ & 2 & 1 & YES & YES & YES & $1.20$ & $(2,2)$ & NO & 898\\
$(49,19)$ & 8 & $(3,1)$ & 2 & 1 & YES & YES & YES & $1.20$ & $(2,2)$ & -- & 899\\
$(49,20)$ & 9 & $(3,1)$ & 2 & 1 & YES & YES & YES & $1.00$ & $(2,2)$ & NO & 900\\
$(49,9)$ & 10 & $(4,1)$ & 3 & 1 & YES & YES & YES & $1.10$ & $(2,2)$ & -- & 901\\
$(49,9)$ & 10 & $(4,1)$ & 3 & 1 & YES & YES & YES & $1.20$ & $(2,2)$ & NO & 902\\
$(49,13)$ & 9 & $(5,1)$ & 4 & 1 & YES & YES & YES & $1.00$ & $(2,2)$ & NO & 903\\
$(49,13)$ & 9 & $(5,1)$ & 4 & 1 & YES & YES & YES & $0.89$ & $(2,2)$ & -- & 904\\
$(49,19)$ & 8 & $(5,2)$ & 3 & 1 & YES & YES & YES & $1.10$ & $(2,2)$ & NO & 905\\
$(49,15)$ & 9 & $(6,1)$ & 5 & 1 & YES & YES & YES & $1.10$ & $(2,2)$ & NO & 906\\
$(49,20)$ & 9 & $(6,1)$ & 5 & 1 & YES & YES & YES & $0.88$ & $(2,2)$ & NO & 907\\
$(49,18)$ & 8 & $(7,3)$ & 4 & 7 & YES & YES & YES & $1.00$ & $(2,2)$ & NO & 908\\
$(49,19)$ & 8 & $(7,2)$ & 4 & 7 & YES & YES & YES & $1.42$ & $(4,1)$ & -- & 909\\
$(49,18)$ & 8 & $(8,3)$ & 4 & 1 & YES & YES & YES & $1.00$ & $(2,2)$ & NO & 910\\
$(49,19)$ & 8 & $(8,3)$ & 4 & 1 & YES & YES & YES & $1.30$ & $(2,2)$ & 973 & 911\\
$(49,9)$ & 10 & $(9,2)$ & 5 & 1 & YES & YES & YES & $1.10$ & $(2,2)$ & 1041 & 912\\
$(49,15)$ & 9 & $(9,2)$ & 5 & 1 & YES & YES & YES & $1.22$ & $(2,2)$ & NO & 913\\
$(49,19)$ & 8 & $(9,2)$ & 5 & 1 & YES & YES & YES & $1.18$ & $(4,1)$ & NO & 914\\
$(49,18)$ & 8 & $(13,5)$ & 5 & 1 & YES & YES & YES & $1.00$ & $(2,2)$ & NO & 915\\
$(49,13)$ & 9 & $(15,4)$ & 6 & 1 & YES & YES & YES & $1.00$ & $(2,2)$ & NO & 916\\
$(49,11)$ & 10 & $(17,3)$ & 7 & 1 & YES & YES & YES & $1.33$ & $(2,2)$ & NO & 917\\
$(49,20)$ & 9 & $(17,7)$ & 6 & 1 & YES & YES & YES & $0.88$ & $(2,2)$ & NO & 918\\
$(49,13)$ & 9 & $(19,5)$ & 7 & 1 & YES & YES & YES & $1.00$ & $(2,2)$ & 955 & 919\\
$(49,9)$ & 10 & $(23,4)$ & 8 & 1 & YES & YES & YES & $0.88$ & $(2,2)$ & NO & 920\\
$(49,19)$ & 8 & $(28,11)$ & 8 & 7 & YES & YES & YES & $1.22$ & $(2,2)$ & NO & 921\\
$(49,15)$ & 9 & $(33,10)$ & 8 & 1 & YES & YES & YES & $1.22$ & $(2,2)$ & 1320 & 922\\
$(49,15)$ & 9 & $(36,11)$ & 8 & 1 & YES & YES & YES & $1.10$ & $(2,2)$ & NO & 923\\
$(49,19)$ & 8 & $(44,17)$ & 8 & 1 & YES & YES & YES & $1.18$ & $(4,1)$ & NO & 924\\
$(50,21)$ & 8 & $(2,1)$ & 1 & 2 & NO & YES & YES & $1.00$ & $(2,2)$ & -- & 925\\
$(50,19)$ & 8 & $(3,1)$ & 2 & 1 & YES & YES & YES & $1.20$ & $(2,2)$ & -- & 926\\
$(50,19)$ & 8 & $(5,2)$ & 3 & 5 & YES & YES & YES & $1.12$ & $(2,2)$ & -- & 927\\
$(50,19)$ & 8 & $(5,2)$ & 3 & 5 & YES & YES & YES & $1.33$ & $(2,2)$ & NO & 928\\
$(50,21)$ & 8 & $(5,2)$ & 3 & 5 & YES & YES & YES & $1.12$ & $(2,2)$ & -- & 929\\
$(50,19)$ & 8 & $(7,2)$ & 4 & 1 & YES & YES & YES & $1.27$ & $(4,1)$ & -- & 930\\
$(50,19)$ & 8 & $(7,3)$ & 4 & 1 & YES & YES & YES & $1.00$ & $(6,0)$ & NO & 931\\
$(50,19)$ & 8 & $(9,4)$ & 5 & 1 & YES & YES & YES & $1.33$ & $(2,2)$ & NO & 932\\
$(50,19)$ & 8 & $(13,5)$ & 5 & 1 & YES & YES & YES & $1.30$ & $(2,2)$ & NO & 933\\
$(50,21)$ & 8 & $(26,11)$ & 7 & 2 & YES & YES & YES & $1.12$ & $(2,2)$ & NO & 934\\
$(50,19)$ & 8 & $(34,13)$ & 7 & 2 & YES & YES & YES & $1.00$ & $(2,2)$ & NO & 935\\
$(51,14)$ & 9 & $(2,1)$ & 1 & 1 & YES & YES & YES & $1.27$ & $(2,2)$ & -- & 936\\
$(51,20)$ & 9 & $(2,1)$ & 1 & 1 & YES & YES & YES & $1.11$ & $(2,2)$ & -- & 937\\
$(51,20)$ & 9 & $(2,1)$ & 1 & 1 & YES & YES & YES & $1.11$ & $(2,2)$ & NO & 938\\
$(51,20)$ & 9 & $(41,16)$ & 8 & 1 & YES & YES & YES & $1.33$ & $(2,2)$ & NO & 939\\
$(53,14)$ & 9 & $(2,1)$ & 1 & 1 & YES & YES & YES & $1.00$ & $(2,2)$ & NO & 940\\
$(53,14)$ & 9 & $(2,1)$ & 1 & 1 & YES & YES & YES & $0.89$ & $(2,2)$ & -- & 941\\
$(53,19)$ & 9 & $(2,1)$ & 1 & 1 & YES & YES & YES & $1.00$ & $(2,2)$ & NO & 942\\
$(53,23)$ & 9 & $(2,1)$ & 1 & 1 & NO & YES & YES & $1.20$ & $(2,2)$ & -- & 943\\
$(53,14)$ & 9 & $(3,1)$ & 2 & 1 & YES & YES & YES & $0.89$ & $(2,2)$ & NO & 944\\
$(53,22)$ & 9 & $(4,1)$ & 3 & 1 & YES & YES & YES & $1.00$ & $(4,1)$ & NO & 945\\
$(53,22)$ & 9 & $(4,1)$ & 3 & 1 & YES & YES & YES & $1.00$ & $(4,1)$ & -- & 946\\
$(53,14)$ & 9 & $(5,1)$ & 4 & 1 & YES & YES & YES & $0.89$ & $(2,2)$ & NO & 947\\
$(53,14)$ & 9 & $(5,1)$ & 4 & 1 & YES & YES & YES & $0.89$ & $(2,2)$ & -- & 948\\
$(53,19)$ & 9 & $(5,1)$ & 4 & 1 & YES & YES & YES & $0.88$ & $(2,2)$ & -- & 949\\
$(53,19)$ & 9 & $(5,1)$ & 4 & 1 & YES & YES & YES & $1.00$ & $(2,2)$ & NO & 950\\
$(53,19)$ & 9 & $(5,2)$ & 3 & 1 & YES & YES & YES & $1.33$ & $(2,2)$ & -- & 951\\
$(53,19)$ & 9 & $(6,1)$ & 5 & 1 & YES & YES & YES & $0.88$ & $(2,2)$ & NO & 952\\
$(53,22)$ & 9 & $(7,3)$ & 4 & 1 & YES & YES & YES & $1.22$ & $(2,2)$ & NO & 953\\
$(53,19)$ & 9 & $(14,5)$ & 6 & 1 & YES & YES & YES & $1.00$ & $(2,2)$ & NO & 954\\
$(53,14)$ & 9 & $(15,4)$ & 6 & 1 & YES & YES & YES & $1.00$ & $(2,2)$ & 919 & 955\\
$(53,14)$ & 9 & $(19,5)$ & 7 & 1 & YES & YES & YES & $1.00$ & $(2,2)$ & NO & 956\\
$(53,19)$ & 9 & $(19,7)$ & 6 & 1 & YES & YES & YES & $1.22$ & $(2,2)$ & 1413 & 957\\
$(53,19)$ & 9 & $(36,13)$ & 8 & 1 & YES & YES & YES & $1.22$ & $(2,2)$ & 1375 & 958\\
$(55,16)$ & 9 & $(2,1)$ & 1 & 1 & YES & YES & YES & $0.89$ & $(2,2)$ & -- & 959\\
$(55,21)$ & 8 & $(2,1)$ & 1 & 1 & YES & YES & YES & $1.30$ & $(2,2)$ & -- & 960\\
$(55,23)$ & 9 & $(2,1)$ & 1 & 1 & YES & YES & YES & $1.22$ & $(2,2)$ & -- & 961\\
$(55,24)$ & 9 & $(2,1)$ & 1 & 1 & NO & YES & YES & $1.00$ & $(2,2)$ & -- & 962\\
$(55,16)$ & 9 & $(3,1)$ & 2 & 1 & NO & YES & YES & $0.88$ & $(4,1)$ & -- & 963\\
$(55,21)$ & 8 & $(3,1)$ & 2 & 1 & YES & YES & YES & $1.20$ & $(2,2)$ & -- & 964\\
$(55,23)$ & 9 & $(3,1)$ & 2 & 1 & YES & YES & YES & $1.22$ & $(2,2)$ & NO & 965\\
$(55,23)$ & 9 & $(3,1)$ & 2 & 1 & YES & YES & YES & $1.22$ & $(2,2)$ & -- & 966\\
$(55,16)$ & 9 & $(4,1)$ & 3 & 1 & YES & YES & YES & $0.89$ & $(2,2)$ & NO & 967\\
$(55,23)$ & 9 & $(4,1)$ & 3 & 1 & YES & YES & YES & $1.12$ & $(2,2)$ & NO & 968\\
$(55,16)$ & 9 & $(5,2)$ & 3 & 5 & YES & YES & YES & $1.22$ & $(2,2)$ & NO & 969\\
$(55,21)$ & 8 & $(5,1)$ & 4 & 5 & YES & YES & YES & $1.20$ & $(2,2)$ & -- & 970\\
$(55,21)$ & 8 & $(5,1)$ & 4 & 5 & YES & YES & YES & $1.30$ & $(2,2)$ & NO & 971\\
$(55,21)$ & 8 & $(5,2)$ & 3 & 5 & YES & YES & YES & $1.42$ & $(4,1)$ & -- & 972\\
$(55,21)$ & 8 & $(5,2)$ & 3 & 5 & YES & YES & YES & $1.30$ & $(2,2)$ & 911 & 973\\
$(55,23)$ & 9 & $(5,1)$ & 4 & 5 & YES & YES & YES & $0.88$ & $(2,2)$ & -- & 974\\
$(55,23)$ & 9 & $(6,1)$ & 5 & 1 & YES & YES & YES & $1.00$ & $(2,2)$ & NO & 975\\
$(55,23)$ & 9 & $(6,1)$ & 5 & 1 & YES & YES & YES & $1.00$ & $(2,2)$ & -- & 976\\
$(55,23)$ & 9 & $(6,1)$ & 5 & 1 & YES & YES & YES & $1.12$ & $(2,2)$ & NO & 977\\
$(55,16)$ & 9 & $(7,2)$ & 4 & 1 & YES & YES & YES & $0.89$ & $(2,2)$ & 748 & 978\\
$(55,23)$ & 9 & $(7,3)$ & 4 & 1 & YES & YES & YES & $1.22$ & $(2,2)$ & NO & 979\\
$(55,21)$ & 8 & $(8,3)$ & 4 & 1 & YES & YES & YES & $1.20$ & $(2,2)$ & NO & 980\\
$(55,21)$ & 8 & $(9,2)$ & 5 & 1 & YES & YES & YES & $1.18$ & $(4,1)$ & NO & 981\\
$(55,21)$ & 8 & $(9,2)$ & 5 & 1 & YES & YES & YES & $1.25$ & $(4,1)$ & -- & 982\\
$(55,21)$ & 8 & $(13,5)$ & 5 & 1 & YES & YES & YES & $1.30$ & $(2,2)$ & 889 & 983\\
$(55,21)$ & 8 & $(18,7)$ & 6 & 1 & YES & YES & YES & $1.18$ & $(4,1)$ & NO & 984\\
$(55,23)$ & 9 & $(19,8)$ & 6 & 1 & YES & YES & YES & $1.12$ & $(2,2)$ & NO & 985\\
$(55,21)$ & 8 & $(21,8)$ & 6 & 1 & YES & YES & YES & $1.20$ & $(2,2)$ & NO & 986\\
$(55,13)$ & 10 & $(23,5)$ & 7 & 1 & YES & YES & YES & $1.22$ & $(2,2)$ & NO & 987\\
$(55,21)$ & 8 & $(29,11)$ & 7 & 1 & YES & YES & YES & $1.20$ & $(6,0)$ & NO & 988\\
$(55,23)$ & 9 & $(31,13)$ & 7 & 1 & YES & YES & YES & $1.12$ & $(2,2)$ & 1180 & 989\\
$(55,23)$ & 9 & $(43,18)$ & 8 & 1 & YES & YES & YES & $1.00$ & $(2,2)$ & NO & 990\\
$(55,21)$ & 8 & $(47,18)$ & 8 & 1 & YES & YES & YES & $1.33$ & $(4,1)$ & NO & 991\\
$(55,23)$ & 9 & $(55,23)$ & 9 & 55 & YES & YES & YES & $1.12$ & $(2,2)$ & NO & 992\\
$(57,22)$ & 9 & $(3,1)$ & 2 & 3 & YES & YES & YES & $1.11$ & $(6,0)$ & -- & 993\\
$(57,22)$ & 9 & $(3,1)$ & 2 & 3 & YES & YES & YES & $1.22$ & $(6,0)$ & NO & 994\\
$(57,22)$ & 9 & $(4,1)$ & 3 & 1 & YES & YES & YES & $1.11$ & $(6,0)$ & NO & 995\\
$(57,22)$ & 9 & $(18,7)$ & 6 & 3 & YES & YES & YES & $1.00$ & $(6,0)$ & NO & 996\\
$(57,22)$ & 9 & $(31,12)$ & 7 & 1 & YES & YES & YES & $1.00$ & $(6,0)$ & 1197 & 997\\
$(58,17)$ & 9 & $(2,1)$ & 1 & 2 & YES & YES & YES & $1.40$ & $(2,2)$ & -- & 998\\
$(58,17)$ & 9 & $(3,1)$ & 2 & 1 & YES & YES & YES & $1.36$ & $(2,2)$ & NO & 999\\
$(58,17)$ & 9 & $(3,1)$ & 2 & 1 & YES & YES & YES & $1.36$ & $(2,2)$ & -- & 1000\\
$(58,17)$ & 9 & $(4,1)$ & 3 & 2 & YES & YES & YES & $1.25$ & $(2,2)$ & 764 & 1001\\
$(58,17)$ & 9 & $(4,1)$ & 3 & 2 & YES & YES & YES & $1.25$ & $(2,2)$ & -- & 1002\\
$(58,17)$ & 9 & $(5,2)$ & 3 & 1 & YES & YES & YES & $1.12$ & $(4,1)$ & -- & 1003\\
$(58,17)$ & 9 & $(7,2)$ & 4 & 1 & YES & YES & YES & $1.33$ & $(4,1)$ & -- & 1004\\
$(58,17)$ & 9 & $(9,2)$ & 5 & 1 & YES & YES & YES & $1.18$ & $(4,1)$ & NO & 1005\\
$(58,17)$ & 9 & $(10,3)$ & 5 & 2 & YES & YES & YES & $1.40$ & $(2,2)$ & NO & 1006\\
$(58,17)$ & 9 & $(17,5)$ & 6 & 1 & YES & YES & YES & $1.40$ & $(2,2)$ & NO & 1007\\
$(58,17)$ & 9 & $(18,5)$ & 6 & 2 & YES & YES & YES & $1.22$ & $(2,2)$ & NO & 1008\\
$(58,17)$ & 9 & $(58,17)$ & 9 & 58 & YES & YES & YES & $1.36$ & $(2,2)$ & NO & 1009\\
$(59,23)$ & 9 & $(2,1)$ & 1 & 1 & YES & YES & YES & $0.88$ & $(2,2)$ & NO & 1010\\
$(59,25)$ & 9 & $(2,1)$ & 1 & 1 & NO & YES & YES & $1.00$ & $(2,2)$ & -- & 1011\\
$(59,25)$ & 9 & $(3,1)$ & 2 & 1 & YES & YES & YES & $1.00$ & $(2,2)$ & -- & 1012\\
$(59,23)$ & 9 & $(4,1)$ & 3 & 1 & YES & YES & YES & $1.00$ & $(4,1)$ & NO & 1013\\
$(59,23)$ & 9 & $(4,1)$ & 3 & 1 & YES & YES & YES & $1.00$ & $(4,1)$ & -- & 1014\\
$(59,25)$ & 9 & $(4,1)$ & 3 & 1 & YES & YES & YES & $1.00$ & $(2,2)$ & NO & 1015\\
$(59,23)$ & 9 & $(5,2)$ & 3 & 1 & YES & YES & YES & $0.88$ & $(2,2)$ & NO & 1016\\
$(59,25)$ & 9 & $(12,5)$ & 5 & 1 & YES & YES & YES & $1.00$ & $(2,2)$ & 676 & 1017\\
$(59,23)$ & 9 & $(13,5)$ & 5 & 1 & YES & YES & YES & $1.00$ & $(4,1)$ & NO & 1018\\
$(59,23)$ & 9 & $(28,11)$ & 8 & 1 & YES & YES & YES & $1.22$ & $(2,2)$ & 1410 & 1019\\
$(59,25)$ & 9 & $(33,14)$ & 8 & 1 & YES & YES & YES & $1.00$ & $(2,2)$ & NO & 1020\\
$(59,25)$ & 9 & $(59,25)$ & 9 & 59 & YES & YES & YES & $1.00$ & $(2,2)$ & NO & 1021\\
$(60,23)$ & 9 & $(2,1)$ & 1 & 2 & YES & YES & YES & $1.22$ & $(2,2)$ & NO & 1022\\
$(60,23)$ & 9 & $(6,1)$ & 5 & 6 & YES & YES & YES & $1.00$ & $(2,2)$ & NO & 1023\\
$(60,23)$ & 9 & $(6,1)$ & 5 & 6 & YES & YES & YES & $1.00$ & $(2,2)$ & -- & 1024\\
$(60,23)$ & 9 & $(6,1)$ & 5 & 6 & YES & YES & YES & $1.12$ & $(2,2)$ & NO & 1025\\
$(60,23)$ & 9 & $(11,4)$ & 5 & 1 & YES & YES & YES & $0.88$ & $(6,0)$ & NO & 1026\\
$(60,23)$ & 9 & $(21,8)$ & 6 & 3 & YES & YES & YES & $1.12$ & $(2,2)$ & NO & 1027\\
$(60,23)$ & 9 & $(34,13)$ & 7 & 2 & YES & YES & YES & $1.22$ & $(2,2)$ & 1294 & 1028\\
$(60,23)$ & 9 & $(60,23)$ & 9 & 60 & YES & YES & YES & $1.12$ & $(2,2)$ & NO & 1029\\
$(61,18)$ & 9 & $(2,1)$ & 1 & 1 & YES & YES & YES & $1.30$ & $(2,2)$ & NO & 1030\\
$(61,17)$ & 9 & $(3,1)$ & 2 & 1 & YES & YES & YES & $1.12$ & $(2,2)$ & 734 & 1031\\
$(61,17)$ & 9 & $(3,1)$ & 2 & 1 & YES & YES & YES & $1.12$ & $(2,2)$ & -- & 1032\\
$(61,25)$ & 9 & $(3,1)$ & 2 & 1 & YES & YES & YES & $1.00$ & $(2,2)$ & -- & 1033\\
$(61,18)$ & 9 & $(4,1)$ & 3 & 1 & YES & YES & YES & $1.30$ & $(2,2)$ & NO & 1034\\
$(61,18)$ & 9 & $(4,1)$ & 3 & 1 & YES & YES & YES & $1.30$ & $(2,2)$ & -- & 1035\\
$(61,25)$ & 9 & $(4,1)$ & 3 & 1 & YES & YES & YES & $0.88$ & $(2,2)$ & NO & 1036\\
$(61,13)$ & 10 & $(5,1)$ & 4 & 1 & YES & YES & YES & $1.10$ & $(2,2)$ & NO & 1037\\
$(61,17)$ & 9 & $(5,2)$ & 3 & 1 & YES & YES & YES & $1.33$ & $(4,1)$ & -- & 1038\\
$(61,17)$ & 9 & $(5,2)$ & 3 & 1 & YES & YES & YES & $1.33$ & $(4,1)$ & NO & 1039\\
$(61,18)$ & 9 & $(5,2)$ & 3 & 1 & YES & YES & YES & $1.27$ & $(4,1)$ & -- & 1040\\
$(61,13)$ & 10 & $(6,1)$ & 5 & 1 & YES & YES & YES & $1.10$ & $(2,2)$ & 912 & 1041\\
$(61,18)$ & 9 & $(7,2)$ & 4 & 1 & YES & YES & YES & $1.27$ & $(4,1)$ & -- & 1042\\
$(61,25)$ & 9 & $(7,3)$ & 4 & 1 & YES & YES & YES & $1.00$ & $(2,2)$ & NO & 1043\\
$(61,18)$ & 9 & $(9,2)$ & 5 & 1 & YES & YES & YES & $1.27$ & $(4,1)$ & NO & 1044\\
$(61,18)$ & 9 & $(10,3)$ & 5 & 1 & YES & YES & YES & $1.30$ & $(2,2)$ & NO & 1045\\
$(61,17)$ & 9 & $(11,3)$ & 5 & 1 & YES & YES & YES & $1.30$ & $(2,2)$ & NO & 1046\\
$(61,25)$ & 9 & $(12,5)$ & 5 & 1 & YES & YES & YES & $1.22$ & $(2,2)$ & 1194 & 1047\\
$(61,13)$ & 10 & $(19,4)$ & 7 & 1 & YES & YES & YES & $1.10$ & $(2,2)$ & NO & 1048\\
$(61,25)$ & 9 & $(39,16)$ & 8 & 1 & YES & YES & YES & $0.88$ & $(2,2)$ & NO & 1049\\
$(61,25)$ & 9 & $(61,25)$ & 9 & 61 & YES & YES & YES & $0.88$ & $(2,2)$ & NO & 1050\\
$(62,23)$ & 9 & $(3,1)$ & 2 & 1 & YES & YES & YES & $1.00$ & $(2,2)$ & NO & 1051\\
$(62,23)$ & 9 & $(5,1)$ & 4 & 1 & YES & YES & YES & $0.88$ & $(2,2)$ & -- & 1052\\
$(62,19)$ & 10 & $(7,2)$ & 4 & 1 & YES & YES & YES & $1.25$ & $(2,2)$ & NO & 1053\\
$(62,23)$ & 9 & $(8,3)$ & 4 & 2 & YES & YES & YES & $1.00$ & $(2,2)$ & 832 & 1054\\
$(63,26)$ & 9 & $(2,1)$ & 1 & 1 & YES & YES & YES & $1.11$ & $(6,0)$ & -- & 1055\\
$(63,26)$ & 9 & $(3,1)$ & 2 & 3 & YES & YES & YES & $1.11$ & $(6,0)$ & NO & 1056\\
$(63,26)$ & 9 & $(3,1)$ & 2 & 3 & YES & YES & YES & $1.11$ & $(6,0)$ & -- & 1057\\
$(63,26)$ & 9 & $(3,1)$ & 2 & 3 & YES & YES & YES & $1.33$ & $(2,2)$ & NO & 1058\\
$(63,26)$ & 9 & $(4,1)$ & 3 & 1 & YES & YES & YES & $1.11$ & $(2,2)$ & -- & 1059\\
$(63,26)$ & 9 & $(4,1)$ & 3 & 1 & YES & YES & YES & $1.22$ & $(2,2)$ & NO & 1060\\
$(63,26)$ & 9 & $(5,2)$ & 3 & 1 & YES & YES & YES & $1.11$ & $(2,2)$ & NO & 1061\\
$(63,26)$ & 9 & $(12,5)$ & 5 & 3 & YES & YES & YES & $1.00$ & $(6,0)$ & NO & 1062\\
$(63,26)$ & 9 & $(22,9)$ & 7 & 1 & YES & YES & YES & $0.88$ & $(6,0)$ & NO & 1063\\
$(63,26)$ & 9 & $(29,12)$ & 7 & 1 & YES & YES & YES & $1.00$ & $(6,0)$ & 1198 & 1064\\
$(63,26)$ & 9 & $(46,19)$ & 8 & 1 & YES & YES & YES & $1.11$ & $(2,2)$ & NO & 1065\\
$(64,25)$ & 9 & $(2,1)$ & 1 & 2 & NO & YES & YES & $1.20$ & $(2,2)$ & -- & 1066\\
$(64,27)$ & 9 & $(2,1)$ & 1 & 2 & NO & YES & YES & $1.00$ & $(2,2)$ & -- & 1067\\
$(64,25)$ & 9 & $(3,1)$ & 2 & 1 & YES & YES & YES & $1.12$ & $(2,2)$ & -- & 1068\\
$(64,25)$ & 9 & $(3,1)$ & 2 & 1 & YES & YES & YES & $1.22$ & $(2,2)$ & NO & 1069\\
$(64,25)$ & 9 & $(4,1)$ & 3 & 4 & YES & YES & YES & $1.12$ & $(2,2)$ & NO & 1070\\
$(64,27)$ & 9 & $(5,2)$ & 3 & 1 & YES & YES & YES & $1.22$ & $(2,2)$ & -- & 1071\\
$(64,27)$ & 9 & $(8,3)$ & 4 & 8 & YES & YES & YES & $1.33$ & $(2,2)$ & NO & 1072\\
$(64,27)$ & 9 & $(12,5)$ & 5 & 4 & YES & YES & YES & $1.25$ & $(2,2)$ & NO & 1073\\
$(64,25)$ & 9 & $(13,5)$ & 5 & 1 & YES & YES & YES & $1.22$ & $(2,2)$ & 1276 & 1074\\
$(64,19)$ & 9 & $(24,7)$ & 7 & 8 & YES & YES & YES & $1.27$ & $(4,1)$ & NO & 1075\\
$(64,27)$ & 9 & $(26,11)$ & 7 & 2 & YES & YES & YES & $1.11$ & $(2,2)$ & 1161 & 1076\\
$(64,25)$ & 9 & $(28,11)$ & 8 & 4 & YES & YES & YES & $1.33$ & $(2,2)$ & NO & 1077\\
$(64,25)$ & 9 & $(41,16)$ & 8 & 1 & YES & YES & YES & $1.12$ & $(2,2)$ & NO & 1078\\
$(64,27)$ & 9 & $(45,19)$ & 8 & 1 & YES & YES & YES & $1.11$ & $(2,2)$ & NO & 1079\\
$(64,25)$ & 9 & $(64,25)$ & 9 & 64 & YES & YES & YES & $1.12$ & $(2,2)$ & NO & 1080\\
$(65,24)$ & 9 & $(2,1)$ & 1 & 1 & YES & YES & YES & $1.00$ & $(2,2)$ & NO & 1081\\
$(65,27)$ & 10 & $(2,1)$ & 1 & 1 & YES & YES & YES & $1.22$ & $(2,2)$ & -- & 1082\\
$(65,18)$ & 9 & $(3,1)$ & 2 & 1 & YES & YES & YES & $1.30$ & $(2,2)$ & NO & 1083\\
$(65,18)$ & 9 & $(3,1)$ & 2 & 1 & YES & YES & YES & $1.30$ & $(2,2)$ & -- & 1084\\
$(65,24)$ & 9 & $(3,1)$ & 2 & 1 & YES & YES & YES & $1.00$ & $(4,1)$ & -- & 1085\\
$(65,24)$ & 9 & $(3,1)$ & 2 & 1 & YES & YES & YES & $1.00$ & $(2,2)$ & NO & 1086\\
$(65,18)$ & 9 & $(5,2)$ & 3 & 5 & YES & YES & YES & $1.27$ & $(4,1)$ & NO & 1087\\
$(65,18)$ & 9 & $(5,2)$ & 3 & 5 & YES & YES & YES & $1.36$ & $(4,1)$ & -- & 1088\\
$(65,19)$ & 9 & $(5,2)$ & 3 & 5 & YES & YES & YES & $1.42$ & $(4,1)$ & -- & 1089\\
$(65,18)$ & 9 & $(7,2)$ & 4 & 1 & YES & YES & YES & $1.30$ & $(2,2)$ & NO & 1090\\
$(65,19)$ & 9 & $(7,2)$ & 4 & 1 & YES & YES & YES & $1.27$ & $(4,1)$ & -- & 1091\\
$(65,19)$ & 9 & $(9,2)$ & 5 & 1 & YES & YES & YES & $1.18$ & $(4,1)$ & NO & 1092\\
$(65,18)$ & 9 & $(10,3)$ & 5 & 5 & YES & YES & YES & $1.18$ & $(4,1)$ & NO & 1093\\
$(65,19)$ & 9 & $(27,8)$ & 7 & 1 & YES & YES & YES & $1.27$ & $(4,1)$ & NO & 1094\\
$(66,25)$ & 9 & $(2,1)$ & 1 & 2 & NO & YES & YES & $0.88$ & $(4,1)$ & -- & 1095\\
$(66,25)$ & 9 & $(4,1)$ & 3 & 2 & YES & YES & YES & $1.00$ & $(2,2)$ & NO & 1096\\
$(66,25)$ & 9 & $(4,1)$ & 3 & 2 & YES & YES & YES & $1.22$ & $(2,2)$ & -- & 1097\\
$(66,29)$ & 9 & $(5,2)$ & 3 & 1 & YES & YES & YES & $1.22$ & $(2,2)$ & -- & 1098\\
$(66,29)$ & 9 & $(23,10)$ & 7 & 1 & YES & YES & YES & $1.33$ & $(2,2)$ & NO & 1099\\
$(66,25)$ & 9 & $(37,14)$ & 8 & 1 & YES & YES & YES & $1.12$ & $(2,2)$ & NO & 1100\\
$(67,26)$ & 9 & $(2,1)$ & 1 & 1 & YES & YES & YES & $1.11$ & $(6,0)$ & -- & 1101\\
$(67,26)$ & 9 & $(3,1)$ & 2 & 1 & YES & YES & YES & $1.11$ & $(6,0)$ & NO & 1102\\
$(67,26)$ & 9 & $(3,1)$ & 2 & 1 & YES & YES & YES & $1.11$ & $(6,0)$ & -- & 1103\\
$(67,26)$ & 9 & $(3,1)$ & 2 & 1 & YES & YES & YES & $1.11$ & $(6,0)$ & NO & 1104\\
$(67,26)$ & 9 & $(4,1)$ & 3 & 1 & YES & YES & YES & $1.11$ & $(6,0)$ & NO & 1105\\
$(67,26)$ & 9 & $(4,1)$ & 3 & 1 & YES & YES & YES & $1.11$ & $(6,0)$ & -- & 1106\\
$(67,26)$ & 9 & $(4,1)$ & 3 & 1 & YES & YES & YES & $1.27$ & $(4,1)$ & NO & 1107\\
$(67,26)$ & 9 & $(8,3)$ & 4 & 1 & YES & YES & YES & $1.11$ & $(6,0)$ & NO & 1108\\
$(68,25)$ & 9 & $(3,1)$ & 2 & 1 & YES & YES & YES & $1.12$ & $(2,2)$ & NO & 1109\\
$(68,25)$ & 9 & $(3,1)$ & 2 & 1 & YES & YES & YES & $1.22$ & $(2,2)$ & -- & 1110\\
$(68,19)$ & 9 & $(5,2)$ & 3 & 1 & YES & YES & YES & $1.42$ & $(4,1)$ & NO & 1111\\
$(68,19)$ & 9 & $(5,2)$ & 3 & 1 & YES & YES & YES & $1.42$ & $(4,1)$ & -- & 1112\\
$(68,25)$ & 9 & $(5,1)$ & 4 & 1 & YES & YES & YES & $1.00$ & $(2,2)$ & -- & 1113\\
$(68,25)$ & 9 & $(5,2)$ & 3 & 1 & YES & YES & YES & $1.00$ & $(2,2)$ & NO & 1114\\
$(68,25)$ & 9 & $(19,7)$ & 6 & 1 & YES & YES & YES & $1.11$ & $(2,2)$ & NO & 1115\\
$(68,25)$ & 9 & $(30,11)$ & 7 & 2 & YES & YES & YES & $1.12$ & $(2,2)$ & 1263 & 1116\\
$(68,25)$ & 9 & $(68,25)$ & 9 & 68 & YES & YES & YES & $1.00$ & $(2,2)$ & NO & 1117\\
$(69,29)$ & 9 & $(2,1)$ & 1 & 1 & YES & YES & YES & $1.12$ & $(2,2)$ & -- & 1118\\
$(69,29)$ & 9 & $(3,1)$ & 2 & 3 & YES & YES & YES & $1.12$ & $(2,2)$ & NO & 1119\\
$(69,29)$ & 9 & $(3,1)$ & 2 & 3 & YES & YES & YES & $1.12$ & $(2,2)$ & -- & 1120\\
$(69,29)$ & 9 & $(3,1)$ & 2 & 3 & YES & YES & YES & $1.25$ & $(2,2)$ & NO & 1121\\
$(69,29)$ & 9 & $(5,1)$ & 4 & 1 & YES & YES & YES & $1.22$ & $(2,2)$ & -- & 1122\\
$(69,29)$ & 9 & $(6,1)$ & 5 & 3 & YES & YES & YES & $1.00$ & $(2,2)$ & NO & 1123\\
$(69,29)$ & 9 & $(6,1)$ & 5 & 3 & YES & YES & YES & $1.00$ & $(2,2)$ & -- & 1124\\
$(69,29)$ & 9 & $(7,3)$ & 4 & 1 & YES & YES & YES & $1.12$ & $(2,2)$ & NO & 1125\\
$(69,29)$ & 9 & $(12,5)$ & 5 & 3 & YES & YES & YES & $1.12$ & $(2,2)$ & NO & 1126\\
$(69,29)$ & 9 & $(19,8)$ & 6 & 1 & YES & YES & YES & $1.33$ & $(2,2)$ & NO & 1127\\
$(69,29)$ & 9 & $(31,13)$ & 7 & 1 & YES & YES & YES & $1.22$ & $(2,2)$ & 1295 & 1128\\
$(69,29)$ & 9 & $(69,29)$ & 9 & 69 & YES & YES & YES & $1.00$ & $(2,2)$ & NO & 1129\\
$(70,27)$ & 10 & $(2,1)$ & 1 & 2 & YES & YES & YES & $1.22$ & $(2,2)$ & NO & 1130\\
$(70,27)$ & 10 & $(2,1)$ & 1 & 2 & YES & YES & YES & $1.33$ & $(2,2)$ & -- & 1131\\
$(70,29)$ & 9 & $(3,1)$ & 2 & 1 & YES & YES & YES & $1.12$ & $(2,2)$ & -- & 1132\\
$(70,27)$ & 10 & $(5,1)$ & 4 & 5 & YES & YES & YES & $1.22$ & $(2,2)$ & -- & 1133\\
$(70,27)$ & 10 & $(6,1)$ & 5 & 2 & YES & YES & YES & $1.22$ & $(2,2)$ & NO & 1134\\
$(70,29)$ & 9 & $(7,3)$ & 4 & 7 & YES & YES & YES & $1.33$ & $(2,2)$ & NO & 1135\\
$(70,29)$ & 9 & $(17,7)$ & 6 & 1 & YES & YES & YES & $1.00$ & $(2,2)$ & NO & 1136\\
$(70,27)$ & 10 & $(44,17)$ & 8 & 2 & YES & YES & YES & $1.33$ & $(2,2)$ & 1417 & 1137\\
$(70,27)$ & 10 & $(57,22)$ & 9 & 1 & YES & YES & YES & $1.33$ & $(2,2)$ & NO & 1138\\
$(71,27)$ & 9 & $(2,1)$ & 1 & 1 & YES & YES & YES & $1.00$ & $(4,1)$ & -- & 1139\\
$(71,30)$ & 9 & $(2,1)$ & 1 & 1 & YES & YES & YES & $1.11$ & $(2,2)$ & -- & 1140\\
$(71,11)$ & 12 & $(3,1)$ & 2 & 1 & YES & YES & YES & $1.00$ & $(2,2)$ & NO & 1141\\
$(71,26)$ & 9 & $(3,1)$ & 2 & 1 & YES & YES & YES & $1.00$ & $(2,2)$ & -- & 1142\\
$(71,27)$ & 9 & $(3,1)$ & 2 & 1 & YES & YES & YES & $1.11$ & $(6,0)$ & NO & 1143\\
$(71,27)$ & 9 & $(3,1)$ & 2 & 1 & YES & YES & YES & $1.11$ & $(6,0)$ & -- & 1144\\
$(71,30)$ & 9 & $(3,1)$ & 2 & 1 & YES & YES & YES & $1.00$ & $(2,2)$ & -- & 1145\\
$(71,26)$ & 9 & $(4,1)$ & 3 & 1 & YES & YES & YES & $1.11$ & $(2,2)$ & -- & 1146\\
$(71,26)$ & 9 & $(4,1)$ & 3 & 1 & YES & YES & YES & $1.22$ & $(2,2)$ & NO & 1147\\
$(71,27)$ & 9 & $(4,1)$ & 3 & 1 & YES & YES & YES & $1.27$ & $(4,1)$ & -- & 1148\\
$(71,27)$ & 9 & $(4,1)$ & 3 & 1 & YES & YES & YES & $1.00$ & $(4,1)$ & NO & 1149\\
$(71,27)$ & 9 & $(4,1)$ & 3 & 1 & YES & YES & YES & $1.11$ & $(2,2)$ & NO & 1150\\
$(71,11)$ & 12 & $(5,1)$ & 4 & 1 & YES & YES & YES & $1.00$ & $(2,2)$ & NO & 1151\\
$(71,26)$ & 9 & $(5,2)$ & 3 & 1 & YES & YES & YES & $1.00$ & $(2,2)$ & NO & 1152\\
$(71,27)$ & 9 & $(5,1)$ & 4 & 1 & YES & YES & YES & $0.89$ & $(6,0)$ & -- & 1153\\
$(71,26)$ & 9 & $(7,2)$ & 4 & 1 & YES & YES & YES & $1.22$ & $(2,2)$ & NO & 1154\\
$(71,27)$ & 9 & $(7,3)$ & 4 & 1 & YES & YES & YES & $1.33$ & $(2,2)$ & NO & 1155\\
$(71,21)$ & 9 & $(9,2)$ & 5 & 1 & YES & YES & YES & $1.18$ & $(4,1)$ & NO & 1156\\
$(71,21)$ & 9 & $(11,3)$ & 5 & 1 & YES & YES & YES & $1.18$ & $(4,1)$ & NO & 1157\\
$(71,30)$ & 9 & $(12,5)$ & 5 & 1 & YES & YES & YES & $1.00$ & $(2,2)$ & 1292 & 1158\\
$(71,27)$ & 9 & $(13,5)$ & 5 & 1 & YES & YES & YES & $1.12$ & $(2,2)$ & NO & 1159\\
$(71,15)$ & 10 & $(19,4)$ & 7 & 1 & YES & YES & YES & $1.25$ & $(2,2)$ & NO & 1160\\
$(71,30)$ & 9 & $(19,8)$ & 6 & 1 & YES & YES & YES & $1.11$ & $(2,2)$ & 1076 & 1161\\
$(71,21)$ & 9 & $(24,7)$ & 7 & 1 & YES & YES & YES & $1.18$ & $(4,1)$ & NO & 1162\\
$(71,26)$ & 9 & $(27,10)$ & 7 & 1 & YES & YES & YES & $1.22$ & $(2,2)$ & NO & 1163\\
$(71,27)$ & 9 & $(29,11)$ & 7 & 1 & YES & YES & YES & $0.89$ & $(6,0)$ & 1264 & 1164\\
$(71,26)$ & 9 & $(41,15)$ & 8 & 1 & YES & YES & YES & $1.11$ & $(2,2)$ & NO & 1165\\
$(71,27)$ & 9 & $(50,19)$ & 8 & 1 & YES & YES & YES & $1.27$ & $(4,1)$ & NO & 1166\\
$(71,21)$ & 9 & $(61,18)$ & 9 & 1 & YES & YES & YES & $1.18$ & $(4,1)$ & NO & 1167\\
$(71,27)$ & 9 & $(71,27)$ & 9 & 71 & YES & YES & YES & $1.20$ & $(6,0)$ & NO & 1168\\
$(74,29)$ & 10 & $(2,1)$ & 1 & 2 & NO & YES & YES & $1.20$ & $(2,2)$ & -- & 1169\\
$(74,31)$ & 9 & $(2,1)$ & 1 & 2 & YES & YES & YES & $1.12$ & $(2,2)$ & -- & 1170\\
$(74,31)$ & 9 & $(3,1)$ & 2 & 1 & YES & YES & YES & $1.12$ & $(2,2)$ & NO & 1171\\
$(74,31)$ & 9 & $(3,1)$ & 2 & 1 & YES & YES & YES & $1.33$ & $(2,2)$ & -- & 1172\\
$(74,29)$ & 10 & $(4,1)$ & 3 & 2 & YES & YES & YES & $1.22$ & $(2,2)$ & NO & 1173\\
$(74,29)$ & 10 & $(4,1)$ & 3 & 2 & YES & YES & YES & $1.22$ & $(2,2)$ & -- & 1174\\
$(74,31)$ & 9 & $(4,1)$ & 3 & 2 & YES & YES & YES & $1.12$ & $(2,2)$ & NO & 1175\\
$(74,17)$ & 11 & $(5,2)$ & 3 & 1 & YES & YES & YES & $1.11$ & $(2,2)$ & -- & 1176\\
$(74,31)$ & 9 & $(5,1)$ & 4 & 1 & YES & YES & YES & $1.00$ & $(2,2)$ & -- & 1177\\
$(74,31)$ & 9 & $(5,2)$ & 3 & 1 & YES & YES & YES & $1.12$ & $(2,2)$ & NO & 1178\\
$(74,31)$ & 9 & $(7,3)$ & 4 & 1 & YES & YES & YES & $1.12$ & $(2,2)$ & 831 & 1179\\
$(74,31)$ & 9 & $(12,5)$ & 5 & 2 & YES & YES & YES & $1.12$ & $(2,2)$ & 989 & 1180\\
$(74,29)$ & 10 & $(13,5)$ & 5 & 1 & YES & YES & YES & $1.22$ & $(2,2)$ & NO & 1181\\
$(74,17)$ & 11 & $(14,3)$ & 6 & 2 & YES & YES & YES & $1.22$ & $(2,2)$ & NO & 1182\\
$(74,31)$ & 9 & $(19,8)$ & 6 & 1 & YES & YES & YES & $1.12$ & $(2,2)$ & NO & 1183\\
$(74,31)$ & 9 & $(31,13)$ & 7 & 1 & YES & YES & YES & $1.12$ & $(2,2)$ & NO & 1184\\
$(74,29)$ & 10 & $(51,20)$ & 9 & 1 & YES & YES & YES & $1.22$ & $(2,2)$ & NO & 1185\\
$(75,22)$ & 10 & $(2,1)$ & 1 & 1 & YES & YES & YES & $1.00$ & $(6,0)$ & NO & 1186\\
$(75,29)$ & 9 & $(2,1)$ & 1 & 1 & YES & YES & YES & $1.00$ & $(6,0)$ & -- & 1187\\
$(75,31)$ & 9 & $(2,1)$ & 1 & 1 & YES & YES & YES & $1.00$ & $(6,0)$ & -- & 1188\\
$(75,29)$ & 9 & $(3,1)$ & 2 & 3 & YES & YES & YES & $1.18$ & $(4,1)$ & -- & 1189\\
$(75,31)$ & 9 & $(3,1)$ & 2 & 3 & YES & YES & YES & $1.00$ & $(2,2)$ & -- & 1190\\
$(75,31)$ & 9 & $(3,1)$ & 2 & 3 & YES & YES & YES & $1.22$ & $(2,2)$ & NO & 1191\\
$(75,29)$ & 9 & $(4,1)$ & 3 & 1 & YES & YES & YES & $1.18$ & $(4,1)$ & NO & 1192\\
$(75,29)$ & 9 & $(4,1)$ & 3 & 1 & YES & YES & YES & $1.18$ & $(4,1)$ & -- & 1193\\
$(75,31)$ & 9 & $(5,2)$ & 3 & 5 & YES & YES & YES & $1.22$ & $(2,2)$ & 1047 & 1194\\
$(75,29)$ & 9 & $(8,3)$ & 4 & 1 & YES & YES & YES & $1.00$ & $(6,0)$ & NO & 1195\\
$(75,22)$ & 10 & $(10,3)$ & 5 & 5 & YES & YES & YES & $1.12$ & $(2,2)$ & NO & 1196\\
$(75,29)$ & 9 & $(13,5)$ & 5 & 1 & YES & YES & YES & $1.00$ & $(6,0)$ & 997 & 1197\\
$(75,31)$ & 9 & $(17,7)$ & 6 & 1 & YES & YES & YES & $1.00$ & $(6,0)$ & 1064 & 1198\\
$(75,29)$ & 9 & $(18,7)$ & 6 & 3 & YES & YES & YES & $1.27$ & $(4,1)$ & NO & 1199\\
$(75,17)$ & 10 & $(23,5)$ & 7 & 1 & YES & YES & YES & $1.22$ & $(2,2)$ & NO & 1200\\
$(75,29)$ & 9 & $(44,17)$ & 8 & 1 & YES & YES & YES & $1.18$ & $(4,1)$ & NO & 1201\\
$(75,31)$ & 9 & $(46,19)$ & 8 & 1 & YES & YES & YES & $1.11$ & $(2,2)$ & NO & 1202\\
$(75,29)$ & 9 & $(75,29)$ & 9 & 75 & YES & YES & YES & $1.27$ & $(4,1)$ & NO & 1203\\
$(76,29)$ & 9 & $(2,1)$ & 1 & 2 & YES & YES & YES & $1.00$ & $(2,2)$ & -- & 1204\\
$(76,29)$ & 9 & $(2,1)$ & 1 & 2 & YES & YES & YES & $1.25$ & $(2,2)$ & NO & 1205\\
$(76,23)$ & 10 & $(3,1)$ & 2 & 1 & YES & YES & YES & $1.00$ & $(2,2)$ & -- & 1206\\
$(76,29)$ & 9 & $(3,1)$ & 2 & 1 & YES & YES & YES & $1.36$ & $(4,1)$ & NO & 1207\\
$(76,29)$ & 9 & $(3,1)$ & 2 & 1 & YES & YES & YES & $1.11$ & $(6,0)$ & -- & 1208\\
$(76,29)$ & 9 & $(3,1)$ & 2 & 1 & YES & YES & YES & $1.12$ & $(2,2)$ & NO & 1209\\
$(76,29)$ & 9 & $(4,1)$ & 3 & 4 & YES & YES & YES & $1.18$ & $(4,1)$ & NO & 1210\\
$(76,21)$ & 9 & $(5,2)$ & 3 & 1 & YES & YES & YES & $1.33$ & $(4,1)$ & -- & 1211\\
$(76,29)$ & 9 & $(5,2)$ & 3 & 1 & YES & YES & YES & $1.00$ & $(6,0)$ & NO & 1212\\
$(76,29)$ & 9 & $(6,1)$ & 5 & 2 & YES & YES & YES & $1.00$ & $(2,2)$ & NO & 1213\\
$(76,29)$ & 9 & $(6,1)$ & 5 & 2 & YES & YES & YES & $1.00$ & $(2,2)$ & -- & 1214\\
$(76,29)$ & 9 & $(6,1)$ & 5 & 2 & YES & YES & YES & $1.12$ & $(2,2)$ & NO & 1215\\
$(76,23)$ & 10 & $(7,2)$ & 4 & 1 & YES & YES & YES & $1.00$ & $(2,2)$ & NO & 1216\\
$(76,29)$ & 9 & $(8,3)$ & 4 & 4 & YES & YES & YES & $1.11$ & $(6,0)$ & NO & 1217\\
$(76,29)$ & 9 & $(13,5)$ & 5 & 1 & YES & YES & YES & $1.12$ & $(2,2)$ & NO & 1218\\
$(76,29)$ & 9 & $(21,8)$ & 6 & 1 & YES & YES & YES & $1.12$ & $(2,2)$ & NO & 1219\\
$(76,29)$ & 9 & $(34,13)$ & 7 & 2 & YES & YES & YES & $1.42$ & $(4,1)$ & 1356 & 1220\\
$(76,29)$ & 9 & $(55,21)$ & 8 & 1 & YES & YES & YES & $1.33$ & $(4,1)$ & NO & 1221\\
$(76,29)$ & 9 & $(76,29)$ & 9 & 76 & YES & YES & YES & $1.18$ & $(4,1)$ & NO & 1222\\
$(78,23)$ & 10 & $(2,1)$ & 1 & 2 & YES & YES & YES & $1.12$ & $(2,2)$ & -- & 1223\\
$(78,23)$ & 10 & $(3,1)$ & 2 & 3 & YES & YES & YES & $1.12$ & $(2,2)$ & NO & 1224\\
$(78,23)$ & 10 & $(3,1)$ & 2 & 3 & YES & YES & YES & $1.12$ & $(2,2)$ & -- & 1225\\
$(78,23)$ & 10 & $(4,1)$ & 3 & 2 & YES & YES & YES & $1.12$ & $(2,2)$ & NO & 1226\\
$(78,23)$ & 10 & $(10,3)$ & 5 & 2 & YES & YES & YES & $1.12$ & $(2,2)$ & NO & 1227\\
$(78,23)$ & 10 & $(44,13)$ & 8 & 2 & YES & YES & YES & $1.00$ & $(2,2)$ & 1436 & 1228\\
$(79,29)$ & 9 & $(2,1)$ & 1 & 1 & YES & YES & YES & $1.12$ & $(2,2)$ & -- & 1229\\
$(79,29)$ & 9 & $(2,1)$ & 1 & 1 & YES & YES & YES & $1.00$ & $(2,2)$ & NO & 1230\\
$(79,30)$ & 9 & $(2,1)$ & 1 & 1 & YES & YES & YES & $1.00$ & $(6,0)$ & -- & 1231\\
$(79,30)$ & 9 & $(2,1)$ & 1 & 1 & YES & YES & YES & $1.11$ & $(2,2)$ & NO & 1232\\
$(79,14)$ & 11 & $(3,1)$ & 2 & 1 & YES & YES & YES & $0.89$ & $(2,2)$ & -- & 1233\\
$(79,22)$ & 10 & $(3,1)$ & 2 & 1 & YES & YES & YES & $1.11$ & $(6,0)$ & NO & 1234\\
$(79,23)$ & 10 & $(3,1)$ & 2 & 1 & YES & YES & YES & $1.12$ & $(2,2)$ & NO & 1235\\
$(79,23)$ & 10 & $(3,1)$ & 2 & 1 & YES & YES & YES & $1.12$ & $(2,2)$ & -- & 1236\\
$(79,23)$ & 10 & $(3,1)$ & 2 & 1 & YES & YES & YES & $1.11$ & $(2,2)$ & 655 & 1237\\
$(79,30)$ & 9 & $(3,1)$ & 2 & 1 & YES & YES & YES & $1.27$ & $(4,1)$ & -- & 1238\\
$(79,30)$ & 9 & $(3,1)$ & 2 & 1 & YES & YES & YES & $1.00$ & $(2,2)$ & NO & 1239\\
$(79,29)$ & 9 & $(4,1)$ & 3 & 1 & YES & YES & YES & $1.33$ & $(2,2)$ & NO & 1240\\
$(79,30)$ & 9 & $(4,1)$ & 3 & 1 & YES & YES & YES & $1.11$ & $(4,1)$ & NO & 1241\\
$(79,30)$ & 9 & $(4,1)$ & 3 & 1 & YES & YES & YES & $1.11$ & $(4,1)$ & -- & 1242\\
$(79,17)$ & 11 & $(5,2)$ & 3 & 1 & YES & YES & YES & $1.22$ & $(2,2)$ & -- & 1243\\
$(79,18)$ & 10 & $(5,2)$ & 3 & 1 & YES & YES & YES & $1.27$ & $(4,1)$ & NO & 1244\\
$(79,18)$ & 10 & $(5,2)$ & 3 & 1 & YES & YES & YES & $1.27$ & $(4,1)$ & -- & 1245\\
$(79,23)$ & 10 & $(5,1)$ & 4 & 1 & YES & YES & YES & $1.12$ & $(2,2)$ & NO & 1246\\
$(79,29)$ & 9 & $(5,1)$ & 4 & 1 & YES & YES & YES & $1.00$ & $(2,2)$ & -- & 1247\\
$(79,29)$ & 9 & $(5,2)$ & 3 & 1 & YES & YES & YES & $1.22$ & $(2,2)$ & -- & 1248\\
$(79,30)$ & 9 & $(5,1)$ & 4 & 1 & YES & YES & YES & $0.89$ & $(6,0)$ & -- & 1249\\
$(79,30)$ & 9 & $(5,1)$ & 4 & 1 & YES & YES & YES & $1.33$ & $(4,1)$ & NO & 1250\\
$(79,30)$ & 9 & $(5,1)$ & 4 & 1 & YES & YES & YES & $1.33$ & $(4,1)$ & NO & 1251\\
$(79,30)$ & 9 & $(5,2)$ & 3 & 1 & YES & YES & YES & $1.00$ & $(6,0)$ & NO & 1252\\
$(79,14)$ & 11 & $(6,1)$ & 5 & 1 & YES & YES & YES & $1.00$ & $(2,2)$ & NO & 1253\\
$(79,30)$ & 9 & $(8,3)$ & 4 & 1 & YES & YES & YES & $1.22$ & $(2,2)$ & NO & 1254\\
$(79,23)$ & 10 & $(10,3)$ & 5 & 1 & YES & YES & YES & $1.22$ & $(2,2)$ & NO & 1255\\
$(79,14)$ & 11 & $(11,2)$ & 6 & 1 & YES & YES & YES & $1.00$ & $(2,2)$ & NO & 1256\\
$(79,22)$ & 10 & $(11,3)$ & 5 & 1 & YES & YES & YES & $1.12$ & $(2,2)$ & NO & 1257\\
$(79,23)$ & 10 & $(11,3)$ & 5 & 1 & YES & YES & YES & $1.33$ & $(2,2)$ & NO & 1258\\
$(79,17)$ & 11 & $(13,3)$ & 6 & 1 & YES & YES & YES & $1.33$ & $(2,2)$ & NO & 1259\\
$(79,30)$ & 9 & $(13,5)$ & 5 & 1 & YES & YES & YES & $1.27$ & $(4,1)$ & 1352 & 1260\\
$(79,18)$ & 10 & $(14,3)$ & 6 & 1 & YES & YES & YES & $1.18$ & $(4,1)$ & NO & 1261\\
$(79,14)$ & 11 & $(17,3)$ & 7 & 1 & YES & YES & YES & $1.00$ & $(2,2)$ & NO & 1262\\
$(79,29)$ & 9 & $(19,7)$ & 6 & 1 & YES & YES & YES & $1.12$ & $(2,2)$ & 1116 & 1263\\
$(79,30)$ & 9 & $(21,8)$ & 6 & 1 & YES & YES & YES & $0.89$ & $(6,0)$ & 1164 & 1264\\
$(79,30)$ & 9 & $(29,11)$ & 7 & 1 & YES & YES & YES & $1.00$ & $(6,0)$ & NO & 1265\\
$(79,29)$ & 9 & $(30,11)$ & 7 & 1 & YES & YES & YES & $1.12$ & $(2,2)$ & NO & 1266\\
$(79,23)$ & 10 & $(31,9)$ & 8 & 1 & YES & YES & YES & $1.12$ & $(2,2)$ & 1331 & 1267\\
$(79,29)$ & 9 & $(41,15)$ & 8 & 1 & YES & YES & YES & $1.11$ & $(2,2)$ & NO & 1268\\
$(79,30)$ & 9 & $(50,19)$ & 8 & 1 & YES & YES & YES & $1.27$ & $(4,1)$ & NO & 1269\\
$(79,23)$ & 10 & $(79,23)$ & 10 & 79 & YES & YES & YES & $1.00$ & $(2,2)$ & NO & 1270\\
$(79,30)$ & 9 & $(79,30)$ & 9 & 79 & YES & YES & YES & $1.42$ & $(4,1)$ & NO & 1271\\
$(80,31)$ & 9 & $(2,1)$ & 1 & 2 & YES & YES & YES & $1.11$ & $(2,2)$ & NO & 1272\\
$(80,31)$ & 9 & $(3,1)$ & 2 & 1 & YES & YES & YES & $1.50$ & $(4,1)$ & NO & 1273\\
$(80,31)$ & 9 & $(3,1)$ & 2 & 1 & YES & YES & YES & $1.50$ & $(4,1)$ & -- & 1274\\
$(80,31)$ & 9 & $(4,1)$ & 3 & 4 & YES & YES & YES & $1.42$ & $(4,1)$ & NO & 1275\\
$(80,31)$ & 9 & $(5,2)$ & 3 & 5 & YES & YES & YES & $1.22$ & $(2,2)$ & 1074 & 1276\\
$(80,31)$ & 9 & $(8,3)$ & 4 & 8 & YES & YES & YES & $1.36$ & $(4,1)$ & NO & 1277\\
$(80,31)$ & 9 & $(13,5)$ & 5 & 1 & YES & YES & YES & $1.18$ & $(4,1)$ & NO & 1278\\
$(80,31)$ & 9 & $(49,19)$ & 8 & 1 & YES & YES & YES & $1.42$ & $(4,1)$ & NO & 1279\\
$(80,31)$ & 9 & $(80,31)$ & 9 & 80 & YES & YES & YES & $1.42$ & $(4,1)$ & NO & 1280\\
$(81,31)$ & 9 & $(2,1)$ & 1 & 1 & YES & YES & YES & $1.27$ & $(4,1)$ & -- & 1281\\
$(81,31)$ & 9 & $(2,1)$ & 1 & 1 & YES & YES & YES & $1.33$ & $(2,2)$ & NO & 1282\\
$(81,34)$ & 9 & $(2,1)$ & 1 & 1 & YES & YES & YES & $1.12$ & $(2,2)$ & -- & 1283\\
$(81,34)$ & 9 & $(3,1)$ & 2 & 3 & YES & YES & YES & $1.00$ & $(2,2)$ & NO & 1284\\
$(81,31)$ & 9 & $(4,1)$ & 3 & 1 & YES & YES & YES & $1.27$ & $(4,1)$ & NO & 1285\\
$(81,31)$ & 9 & $(4,1)$ & 3 & 1 & YES & YES & YES & $1.25$ & $(4,1)$ & -- & 1286\\
$(81,31)$ & 9 & $(5,1)$ & 4 & 1 & YES & YES & YES & $1.11$ & $(2,2)$ & -- & 1287\\
$(81,31)$ & 9 & $(5,2)$ & 3 & 1 & YES & YES & YES & $1.27$ & $(4,1)$ & NO & 1288\\
$(81,34)$ & 9 & $(5,1)$ & 4 & 1 & YES & YES & YES & $1.22$ & $(2,2)$ & NO & 1289\\
$(81,34)$ & 9 & $(5,1)$ & 4 & 1 & YES & YES & YES & $1.22$ & $(2,2)$ & -- & 1290\\
$(81,34)$ & 9 & $(5,2)$ & 3 & 1 & YES & YES & YES & $1.00$ & $(2,2)$ & NO & 1291\\
$(81,34)$ & 9 & $(7,3)$ & 4 & 1 & YES & YES & YES & $1.00$ & $(2,2)$ & 1158 & 1292\\
$(81,31)$ & 9 & $(8,3)$ & 4 & 1 & YES & YES & YES & $1.27$ & $(4,1)$ & 890 & 1293\\
$(81,31)$ & 9 & $(13,5)$ & 5 & 1 & YES & YES & YES & $1.22$ & $(2,2)$ & 1028 & 1294\\
$(81,34)$ & 9 & $(19,8)$ & 6 & 1 & YES & YES & YES & $1.22$ & $(2,2)$ & 1128 & 1295\\
$(81,34)$ & 9 & $(31,13)$ & 7 & 1 & YES & YES & YES & $1.12$ & $(2,2)$ & NO & 1296\\
$(81,31)$ & 9 & $(34,13)$ & 7 & 1 & YES & YES & YES & $1.25$ & $(4,1)$ & NO & 1297\\
$(81,31)$ & 9 & $(47,18)$ & 8 & 1 & YES & YES & YES & $1.27$ & $(4,1)$ & NO & 1298\\
$(82,23)$ & 10 & $(5,1)$ & 4 & 1 & YES & YES & YES & $1.12$ & $(2,2)$ & NO & 1299\\
$(82,23)$ & 10 & $(5,2)$ & 3 & 1 & YES & YES & YES & $1.22$ & $(2,2)$ & -- & 1300\\
$(82,23)$ & 10 & $(10,3)$ & 5 & 2 & YES & YES & YES & $1.22$ & $(2,2)$ & NO & 1301\\
$(82,23)$ & 10 & $(32,9)$ & 8 & 2 & YES & YES & YES & $1.12$ & $(2,2)$ & 1357 & 1302\\
$(82,23)$ & 10 & $(82,23)$ & 10 & 82 & YES & YES & YES & $1.12$ & $(2,2)$ & NO & 1303\\
$(83,23)$ & 10 & $(2,1)$ & 1 & 1 & YES & YES & YES & $1.12$ & $(2,2)$ & NO & 1304\\
$(83,23)$ & 10 & $(2,1)$ & 1 & 1 & YES & YES & YES & $1.12$ & $(2,2)$ & -- & 1305\\
$(83,23)$ & 10 & $(3,1)$ & 2 & 1 & YES & YES & YES & $1.12$ & $(2,2)$ & NO & 1306\\
$(83,23)$ & 10 & $(4,1)$ & 3 & 1 & YES & YES & YES & $1.00$ & $(2,2)$ & NO & 1307\\
$(83,23)$ & 10 & $(11,3)$ & 5 & 1 & YES & YES & YES & $1.12$ & $(2,2)$ & NO & 1308\\
$(84,25)$ & 10 & $(2,1)$ & 1 & 2 & YES & YES & YES & $1.00$ & $(2,2)$ & NO & 1309\\
$(84,25)$ & 10 & $(2,1)$ & 1 & 2 & YES & YES & YES & $1.22$ & $(2,2)$ & -- & 1310\\
$(84,37)$ & 10 & $(2,1)$ & 1 & 2 & YES & YES & YES & $1.33$ & $(2,2)$ & -- & 1311\\
$(84,25)$ & 10 & $(3,1)$ & 2 & 3 & YES & YES & YES & $1.00$ & $(2,2)$ & NO & 1312\\
$(84,37)$ & 10 & $(3,1)$ & 2 & 3 & YES & YES & YES & $1.33$ & $(2,2)$ & NO & 1313\\
$(84,37)$ & 10 & $(3,1)$ & 2 & 3 & YES & YES & YES & $1.33$ & $(2,2)$ & -- & 1314\\
$(84,37)$ & 10 & $(4,1)$ & 3 & 4 & YES & YES & YES & $1.33$ & $(2,2)$ & -- & 1315\\
$(84,25)$ & 10 & $(7,2)$ & 4 & 7 & YES & YES & YES & $1.00$ & $(2,2)$ & NO & 1316\\
$(84,37)$ & 10 & $(7,3)$ & 4 & 7 & YES & YES & YES & $1.44$ & $(2,2)$ & NO & 1317\\
$(85,26)$ & 10 & $(2,1)$ & 1 & 1 & YES & YES & YES & $1.22$ & $(2,2)$ & -- & 1318\\
$(85,37)$ & 10 & $(3,1)$ & 2 & 1 & YES & YES & YES & $1.33$ & $(2,2)$ & NO & 1319\\
$(85,26)$ & 10 & $(10,3)$ & 5 & 5 & YES & YES & YES & $1.22$ & $(2,2)$ & 922 & 1320\\
$(85,37)$ & 10 & $(16,7)$ & 6 & 1 & YES & YES & YES & $1.33$ & $(2,2)$ & NO & 1321\\
$(85,37)$ & 10 & $(39,17)$ & 8 & 1 & YES & YES & YES & $1.22$ & $(2,2)$ & 1418 & 1322\\
$(86,25)$ & 10 & $(2,1)$ & 1 & 2 & YES & YES & YES & $1.00$ & $(2,2)$ & NO & 1323\\
$(86,25)$ & 10 & $(2,1)$ & 1 & 2 & YES & YES & YES & $1.12$ & $(2,2)$ & -- & 1324\\
$(86,25)$ & 10 & $(3,1)$ & 2 & 1 & YES & YES & YES & $1.00$ & $(2,2)$ & NO & 1325\\
$(86,25)$ & 10 & $(4,1)$ & 3 & 2 & YES & YES & YES & $1.00$ & $(2,2)$ & NO & 1326\\
$(86,25)$ & 10 & $(5,1)$ & 4 & 1 & YES & YES & YES & $1.00$ & $(2,2)$ & -- & 1327\\
$(86,25)$ & 10 & $(5,1)$ & 4 & 1 & YES & YES & YES & $1.12$ & $(2,2)$ & NO & 1328\\
$(86,25)$ & 10 & $(10,3)$ & 5 & 2 & YES & YES & YES & $1.00$ & $(2,2)$ & NO & 1329\\
$(86,25)$ & 10 & $(17,5)$ & 6 & 1 & YES & YES & YES & $1.22$ & $(2,2)$ & 1445 & 1330\\
$(86,25)$ & 10 & $(24,7)$ & 7 & 2 & YES & YES & YES & $1.12$ & $(2,2)$ & 1267 & 1331\\
$(86,25)$ & 10 & $(31,9)$ & 8 & 1 & YES & YES & YES & $1.12$ & $(2,2)$ & NO & 1332\\
$(86,25)$ & 10 & $(86,25)$ & 10 & 86 & YES & YES & YES & $1.12$ & $(2,2)$ & NO & 1333\\
$(87,32)$ & 10 & $(4,1)$ & 3 & 1 & YES & YES & YES & $1.00$ & $(6,0)$ & -- & 1334\\
$(88,37)$ & 10 & $(2,1)$ & 1 & 2 & NO & YES & YES & $1.22$ & $(2,2)$ & -- & 1335\\
$(89,25)$ & 10 & $(2,1)$ & 1 & 1 & YES & YES & YES & $1.12$ & $(2,2)$ & -- & 1336\\
$(89,25)$ & 10 & $(2,1)$ & 1 & 1 & YES & YES & YES & $1.12$ & $(2,2)$ & NO & 1337\\
$(89,26)$ & 10 & $(2,1)$ & 1 & 1 & YES & YES & YES & $1.27$ & $(4,1)$ & -- & 1338\\
$(89,34)$ & 9 & $(2,1)$ & 1 & 1 & YES & YES & YES & $1.42$ & $(4,1)$ & -- & 1339\\
$(89,34)$ & 9 & $(2,1)$ & 1 & 1 & YES & YES & YES & $1.12$ & $(2,2)$ & NO & 1340\\
$(89,25)$ & 10 & $(3,1)$ & 2 & 1 & YES & YES & YES & $1.12$ & $(2,2)$ & NO & 1341\\
$(89,26)$ & 10 & $(3,1)$ & 2 & 1 & YES & YES & YES & $1.42$ & $(4,1)$ & -- & 1342\\
$(89,26)$ & 10 & $(3,1)$ & 2 & 1 & YES & YES & YES & $1.12$ & $(2,2)$ & NO & 1343\\
$(89,26)$ & 10 & $(3,1)$ & 2 & 1 & YES & YES & YES & $1.33$ & $(2,2)$ & NO & 1344\\
$(89,34)$ & 9 & $(3,1)$ & 2 & 1 & YES & YES & YES & $1.36$ & $(4,1)$ & -- & 1345\\
$(89,34)$ & 9 & $(3,1)$ & 2 & 1 & YES & YES & YES & $1.42$ & $(4,1)$ & NO & 1346\\
$(89,25)$ & 10 & $(4,1)$ & 3 & 1 & YES & YES & YES & $0.88$ & $(2,2)$ & NO & 1347\\
$(89,26)$ & 10 & $(4,1)$ & 3 & 1 & YES & YES & YES & $1.27$ & $(4,1)$ & NO & 1348\\
$(89,25)$ & 10 & $(5,1)$ & 4 & 1 & YES & YES & YES & $1.12$ & $(2,2)$ & NO & 1349\\
$(89,34)$ & 9 & $(5,2)$ & 3 & 1 & YES & YES & YES & $1.50$ & $(4,1)$ & NO & 1350\\
$(89,17)$ & 12 & $(6,1)$ & 5 & 1 & YES & YES & YES & $0.88$ & $(2,2)$ & NO & 1351\\
$(89,34)$ & 9 & $(8,3)$ & 4 & 1 & YES & YES & YES & $1.27$ & $(4,1)$ & 1260 & 1352\\
$(89,34)$ & 9 & $(13,5)$ & 5 & 1 & YES & YES & YES & $1.27$ & $(4,1)$ & NO & 1353\\
$(89,26)$ & 10 & $(17,5)$ & 6 & 1 & YES & YES & YES & $1.18$ & $(4,1)$ & NO & 1354\\
$(89,17)$ & 12 & $(21,4)$ & 8 & 1 & YES & YES & YES & $0.88$ & $(2,2)$ & NO & 1355\\
$(89,34)$ & 9 & $(21,8)$ & 6 & 1 & YES & YES & YES & $1.42$ & $(4,1)$ & 1220 & 1356\\
$(89,25)$ & 10 & $(25,7)$ & 7 & 1 & YES & YES & YES & $1.12$ & $(2,2)$ & 1302 & 1357\\
$(89,25)$ & 10 & $(32,9)$ & 8 & 1 & YES & YES & YES & $1.12$ & $(2,2)$ & NO & 1358\\
$(89,27)$ & 10 & $(33,10)$ & 8 & 1 & YES & YES & YES & $1.22$ & $(2,2)$ & NO & 1359\\
$(89,26)$ & 10 & $(41,12)$ & 8 & 1 & YES & YES & YES & $1.42$ & $(4,1)$ & 1447 & 1360\\
$(89,25)$ & 10 & $(57,16)$ & 9 & 1 & YES & YES & YES & $1.33$ & $(2,2)$ & NO & 1361\\
$(89,26)$ & 10 & $(65,19)$ & 9 & 1 & YES & YES & YES & $1.18$ & $(4,1)$ & NO & 1362\\
$(89,26)$ & 10 & $(89,26)$ & 10 & 89 & YES & YES & YES & $1.00$ & $(2,2)$ & NO & 1363\\
$(91,27)$ & 10 & $(2,1)$ & 1 & 1 & YES & YES & YES & $1.11$ & $(2,2)$ & -- & 1364\\
$(91,27)$ & 10 & $(3,1)$ & 2 & 1 & YES & YES & YES & $1.42$ & $(4,1)$ & -- & 1365\\
$(91,25)$ & 10 & $(4,1)$ & 3 & 1 & YES & YES & YES & $1.00$ & $(2,2)$ & NO & 1366\\
$(91,27)$ & 10 & $(4,1)$ & 3 & 1 & YES & YES & YES & $1.33$ & $(4,1)$ & NO & 1367\\
$(91,27)$ & 10 & $(7,2)$ & 4 & 7 & YES & YES & YES & $1.27$ & $(4,1)$ & NO & 1368\\
$(91,27)$ & 10 & $(17,5)$ & 6 & 1 & YES & YES & YES & $1.27$ & $(4,1)$ & NO & 1369\\
$(91,27)$ & 10 & $(37,11)$ & 8 & 1 & YES & YES & YES & $1.42$ & $(4,1)$ & 1419 & 1370\\
$(91,27)$ & 10 & $(91,27)$ & 10 & 91 & YES & YES & YES & $1.27$ & $(4,1)$ & NO & 1371\\
$(92,33)$ & 10 & $(2,1)$ & 1 & 2 & YES & YES & YES & $1.22$ & $(2,2)$ & -- & 1372\\
$(92,33)$ & 10 & $(4,1)$ & 3 & 4 & YES & YES & YES & $1.22$ & $(2,2)$ & -- & 1373\\
$(92,35)$ & 10 & $(8,3)$ & 4 & 4 & YES & YES & YES & $1.00$ & $(4,1)$ & NO & 1374\\
$(92,33)$ & 10 & $(11,4)$ & 5 & 1 & YES & YES & YES & $1.22$ & $(2,2)$ & 958 & 1375\\
$(92,33)$ & 10 & $(39,14)$ & 8 & 1 & YES & YES & YES & $1.22$ & $(2,2)$ & NO & 1376\\
$(93,26)$ & 10 & $(2,1)$ & 1 & 1 & YES & YES & YES & $1.27$ & $(4,1)$ & NO & 1377\\
$(93,26)$ & 10 & $(2,1)$ & 1 & 1 & YES & YES & YES & $1.36$ & $(4,1)$ & -- & 1378\\
$(93,26)$ & 10 & $(3,1)$ & 2 & 3 & YES & YES & YES & $1.42$ & $(4,1)$ & -- & 1379\\
$(93,26)$ & 10 & $(5,1)$ & 4 & 1 & YES & YES & YES & $1.42$ & $(4,1)$ & NO & 1380\\
$(93,26)$ & 10 & $(11,3)$ & 5 & 1 & YES & YES & YES & $1.42$ & $(4,1)$ & NO & 1381\\
$(93,26)$ & 10 & $(18,5)$ & 6 & 3 & YES & YES & YES & $1.27$ & $(4,1)$ & NO & 1382\\
$(93,26)$ & 10 & $(93,26)$ & 10 & 93 & YES & YES & YES & $1.42$ & $(4,1)$ & NO & 1383\\
$(94,41)$ & 10 & $(3,1)$ & 2 & 1 & YES & YES & YES & $1.33$ & $(2,2)$ & NO & 1384\\
$(96,17)$ & 12 & $(5,2)$ & 3 & 1 & YES & YES & YES & $1.11$ & $(2,2)$ & -- & 1385\\
$(96,17)$ & 12 & $(16,3)$ & 7 & 16 & YES & YES & YES & $1.22$ & $(2,2)$ & NO & 1386\\
$(97,26)$ & 10 & $(5,2)$ & 3 & 1 & YES & YES & YES & $1.22$ & $(2,2)$ & NO & 1387\\
$(97,35)$ & 10 & $(5,1)$ & 4 & 1 & YES & YES & YES & $1.22$ & $(2,2)$ & -- & 1388\\
$(97,35)$ & 10 & $(36,13)$ & 8 & 1 & YES & YES & YES & $1.22$ & $(2,2)$ & NO & 1389\\
$(98,27)$ & 10 & $(2,1)$ & 1 & 2 & YES & YES & YES & $1.42$ & $(4,1)$ & -- & 1390\\
$(98,27)$ & 10 & $(2,1)$ & 1 & 2 & YES & YES & YES & $1.42$ & $(4,1)$ & NO & 1391\\
$(98,29)$ & 10 & $(2,1)$ & 1 & 2 & YES & YES & YES & $1.36$ & $(4,1)$ & -- & 1392\\
$(98,29)$ & 10 & $(2,1)$ & 1 & 2 & YES & YES & YES & $1.50$ & $(4,1)$ & NO & 1393\\
$(98,29)$ & 10 & $(3,1)$ & 2 & 1 & YES & YES & YES & $1.33$ & $(4,1)$ & -- & 1394\\
$(98,29)$ & 10 & $(4,1)$ & 3 & 2 & YES & YES & YES & $1.18$ & $(4,1)$ & NO & 1395\\
$(98,27)$ & 10 & $(7,2)$ & 4 & 7 & YES & YES & YES & $1.42$ & $(4,1)$ & NO & 1396\\
$(98,29)$ & 10 & $(7,2)$ & 4 & 7 & YES & YES & YES & $1.36$ & $(4,1)$ & NO & 1397\\
$(98,29)$ & 10 & $(17,5)$ & 6 & 1 & YES & YES & YES & $1.18$ & $(4,1)$ & NO & 1398\\
$(99,29)$ & 10 & $(2,1)$ & 1 & 1 & YES & YES & YES & $1.18$ & $(4,1)$ & NO & 1399\\
$(99,29)$ & 10 & $(2,1)$ & 1 & 1 & YES & YES & YES & $1.42$ & $(4,1)$ & -- & 1400\\
$(99,41)$ & 10 & $(3,1)$ & 2 & 3 & YES & YES & YES & $1.22$ & $(2,2)$ & NO & 1401\\
$(99,41)$ & 10 & $(3,1)$ & 2 & 3 & YES & YES & YES & $1.22$ & $(2,2)$ & -- & 1402\\
$(99,29)$ & 10 & $(4,1)$ & 3 & 1 & YES & YES & YES & $1.18$ & $(4,1)$ & NO & 1403\\
$(99,29)$ & 10 & $(10,3)$ & 5 & 1 & YES & YES & YES & $1.18$ & $(4,1)$ & NO & 1404\\
$(99,29)$ & 10 & $(24,7)$ & 7 & 3 & YES & YES & YES & $1.33$ & $(4,1)$ & NO & 1405\\
$(99,29)$ & 10 & $(41,12)$ & 8 & 1 & YES & YES & YES & $1.18$ & $(4,1)$ & NO & 1406\\
$(99,29)$ & 10 & $(58,17)$ & 9 & 1 & YES & YES & YES & $1.27$ & $(4,1)$ & NO & 1407\\
$(100,31)$ & 11 & $(2,1)$ & 1 & 2 & YES & YES & YES & $1.33$ & $(2,2)$ & NO & 1408\\
$(100,37)$ & 10 & $(3,1)$ & 2 & 1 & YES & YES & YES & $1.33$ & $(2,2)$ & NO & 1409\\
$(100,39)$ & 10 & $(5,2)$ & 3 & 5 & YES & YES & YES & $1.22$ & $(2,2)$ & 1019 & 1410\\
$(100,19)$ & 12 & $(11,2)$ & 6 & 1 & YES & YES & YES & $1.00$ & $(2,2)$ & NO & 1411\\
$(101,30)$ & 10 & $(2,1)$ & 1 & 1 & YES & YES & YES & $1.42$ & $(4,1)$ & -- & 1412\\
$(101,37)$ & 10 & $(3,1)$ & 2 & 1 & YES & YES & YES & $1.22$ & $(2,2)$ & 957 & 1413\\
$(101,39)$ & 10 & $(5,1)$ & 4 & 1 & YES & YES & YES & $1.11$ & $(2,2)$ & -- & 1414\\
$(101,39)$ & 10 & $(5,1)$ & 4 & 1 & YES & YES & YES & $1.22$ & $(2,2)$ & NO & 1415\\
$(101,16)$ & 13 & $(6,1)$ & 5 & 1 & YES & YES & YES & $1.18$ & $(2,2)$ & NO & 1416\\
$(101,39)$ & 10 & $(13,5)$ & 5 & 1 & YES & YES & YES & $1.33$ & $(2,2)$ & 1137 & 1417\\
$(101,44)$ & 10 & $(23,10)$ & 7 & 1 & YES & YES & YES & $1.22$ & $(2,2)$ & 1322 & 1418\\
$(101,30)$ & 10 & $(27,8)$ & 7 & 1 & YES & YES & YES & $1.42$ & $(4,1)$ & 1370 & 1419\\
$(102,43)$ & 11 & $(2,1)$ & 1 & 2 & NO & YES & YES & $1.22$ & $(2,2)$ & -- & 1420\\
$(102,23)$ & 11 & $(14,3)$ & 6 & 2 & YES & YES & YES & $1.22$ & $(2,2)$ & NO & 1421\\
$(103,30)$ & 11 & $(2,1)$ & 1 & 1 & YES & YES & YES & $1.00$ & $(4,1)$ & -- & 1422\\
$(104,29)$ & 10 & $(2,1)$ & 1 & 2 & YES & YES & YES & $1.33$ & $(4,1)$ & NO & 1423\\
$(104,29)$ & 10 & $(11,3)$ & 5 & 1 & YES & YES & YES & $1.33$ & $(4,1)$ & NO & 1424\\
$(105,29)$ & 10 & $(2,1)$ & 1 & 1 & YES & YES & YES & $1.42$ & $(4,1)$ & NO & 1425\\
$(105,31)$ & 10 & $(2,1)$ & 1 & 1 & YES & YES & YES & $1.18$ & $(4,1)$ & -- & 1426\\
$(105,31)$ & 10 & $(2,1)$ & 1 & 1 & YES & YES & YES & $1.00$ & $(6,0)$ & NO & 1427\\
$(105,29)$ & 10 & $(3,1)$ & 2 & 3 & YES & YES & YES & $1.18$ & $(4,1)$ & -- & 1428\\
$(105,29)$ & 10 & $(3,1)$ & 2 & 3 & YES & YES & YES & $1.27$ & $(4,1)$ & NO & 1429\\
$(105,31)$ & 10 & $(3,1)$ & 2 & 3 & YES & YES & YES & $1.27$ & $(4,1)$ & -- & 1430\\
$(105,31)$ & 10 & $(4,1)$ & 3 & 1 & YES & YES & YES & $1.00$ & $(2,2)$ & NO & 1431\\
$(105,29)$ & 10 & $(7,2)$ & 4 & 7 & YES & YES & YES & $1.27$ & $(4,1)$ & NO & 1432\\
$(105,31)$ & 10 & $(7,2)$ & 4 & 7 & YES & YES & YES & $1.00$ & $(6,0)$ & NO & 1433\\
$(105,23)$ & 11 & $(13,3)$ & 6 & 1 & YES & YES & YES & $1.33$ & $(2,2)$ & NO & 1434\\
$(105,23)$ & 11 & $(14,3)$ & 6 & 7 & YES & YES & YES & $1.00$ & $(2,2)$ & NO & 1435\\
$(105,31)$ & 10 & $(17,5)$ & 6 & 1 & YES & YES & YES & $1.00$ & $(2,2)$ & 1228 & 1436\\
$(105,23)$ & 11 & $(23,5)$ & 7 & 1 & YES & YES & YES & $1.00$ & $(2,2)$ & NO & 1437\\
$(105,31)$ & 10 & $(105,31)$ & 10 & 105 & YES & YES & YES & $1.42$ & $(4,1)$ & NO & 1438\\
$(106,31)$ & 10 & $(2,1)$ & 1 & 2 & YES & YES & YES & $1.33$ & $(4,1)$ & -- & 1439\\
$(106,41)$ & 10 & $(2,1)$ & 1 & 2 & YES & YES & YES & $1.22$ & $(2,2)$ & -- & 1440\\
$(106,41)$ & 10 & $(2,1)$ & 1 & 2 & YES & YES & YES & $1.22$ & $(2,2)$ & NO & 1441\\
$(106,31)$ & 10 & $(3,1)$ & 2 & 1 & YES & YES & YES & $1.33$ & $(4,1)$ & -- & 1442\\
$(106,31)$ & 10 & $(4,1)$ & 3 & 2 & YES & YES & YES & $1.42$ & $(4,1)$ & NO & 1443\\
$(106,31)$ & 10 & $(5,1)$ & 4 & 1 & YES & YES & YES & $1.27$ & $(4,1)$ & NO & 1444\\
$(106,31)$ & 10 & $(7,2)$ & 4 & 1 & YES & YES & YES & $1.22$ & $(2,2)$ & 1330 & 1445\\
$(106,31)$ & 10 & $(17,5)$ & 6 & 1 & YES & YES & YES & $1.33$ & $(4,1)$ & NO & 1446\\
$(106,31)$ & 10 & $(24,7)$ & 7 & 2 & YES & YES & YES & $1.42$ & $(4,1)$ & 1360 & 1447\\
$(106,31)$ & 10 & $(41,12)$ & 8 & 1 & YES & YES & YES & $1.33$ & $(4,1)$ & NO & 1448\\
$(106,31)$ & 10 & $(65,19)$ & 9 & 1 & YES & YES & YES & $1.27$ & $(4,1)$ & NO & 1449\\
$(107,25)$ & 11 & $(13,3)$ & 6 & 1 & YES & YES & YES & $1.22$ & $(2,2)$ & NO & 1450\\
$(109,30)$ & 10 & $(2,1)$ & 1 & 1 & YES & YES & YES & $1.42$ & $(4,1)$ & -- & 1451\\
$(109,30)$ & 10 & $(2,1)$ & 1 & 1 & YES & YES & YES & $1.42$ & $(4,1)$ & NO & 1452\\
$(109,30)$ & 10 & $(7,2)$ & 4 & 1 & YES & YES & YES & $1.42$ & $(4,1)$ & NO & 1453\\
$(109,30)$ & 10 & $(11,3)$ & 5 & 1 & YES & YES & YES & $1.00$ & $(2,2)$ & NO & 1454\\
$(111,31)$ & 10 & $(2,1)$ & 1 & 1 & YES & YES & YES & $1.33$ & $(4,1)$ & NO & 1455\\
$(111,43)$ & 10 & $(2,1)$ & 1 & 1 & NO & YES & YES & $1.30$ & $(2,2)$ & -- & 1456\\
$(111,25)$ & 11 & $(3,1)$ & 2 & 3 & YES & YES & YES & $1.00$ & $(2,2)$ & NO & 1457\\
$(111,31)$ & 10 & $(3,1)$ & 2 & 3 & YES & YES & YES & $1.42$ & $(4,1)$ & NO & 1458\\
$(111,41)$ & 10 & $(3,1)$ & 2 & 3 & YES & YES & YES & $1.22$ & $(2,2)$ & NO & 1459\\
$(111,31)$ & 10 & $(18,5)$ & 6 & 3 & YES & YES & YES & $1.33$ & $(4,1)$ & NO & 1460\\
$(112,31)$ & 10 & $(2,1)$ & 1 & 2 & YES & YES & YES & $1.33$ & $(4,1)$ & -- & 1461\\
$(112,47)$ & 10 & $(2,1)$ & 1 & 2 & NO & YES & YES & $1.25$ & $(2,2)$ & -- & 1462\\
$(113,24)$ & 11 & $(19,4)$ & 7 & 1 & YES & YES & YES & $1.11$ & $(2,2)$ & NO & 1463\\
$(115,26)$ & 11 & $(3,1)$ & 2 & 1 & YES & YES & YES & $1.33$ & $(4,1)$ & -- & 1464\\
$(115,26)$ & 11 & $(3,1)$ & 2 & 1 & YES & YES & YES & $1.42$ & $(4,1)$ & NO & 1465\\
$(115,26)$ & 11 & $(3,1)$ & 2 & 1 & YES & YES & YES & $1.42$ & $(4,1)$ & NO & 1466\\
$(115,26)$ & 11 & $(9,2)$ & 5 & 1 & YES & YES & YES & $1.11$ & $(2,2)$ & NO & 1467\\
$(115,26)$ & 11 & $(31,7)$ & 8 & 1 & YES & YES & YES & $1.11$ & $(2,2)$ & NO & 1468\\
$(116,49)$ & 10 & $(2,1)$ & 1 & 2 & NO & YES & YES & $1.22$ & $(2,2)$ & -- & 1469\\
$(117,49)$ & 10 & $(2,1)$ & 1 & 1 & NO & YES & YES & $1.33$ & $(2,2)$ & -- & 1470\\
$(117,31)$ & 11 & $(3,1)$ & 2 & 3 & YES & YES & YES & $1.33$ & $(2,2)$ & NO & 1471\\
$(118,27)$ & 11 & $(2,1)$ & 1 & 2 & YES & YES & YES & $1.42$ & $(4,1)$ & NO & 1472\\
$(118,27)$ & 11 & $(9,2)$ & 5 & 1 & YES & YES & YES & $1.18$ & $(4,1)$ & NO & 1473\\
$(119,50)$ & 10 & $(2,1)$ & 1 & 1 & NO & YES & YES & $1.33$ & $(2,2)$ & -- & 1474\\
$(119,27)$ & 12 & $(5,1)$ & 4 & 1 & YES & YES & YES & $1.33$ & $(2,2)$ & NO & 1475\\
$(119,22)$ & 12 & $(16,3)$ & 7 & 1 & YES & YES & YES & $1.22$ & $(2,2)$ & NO & 1476\\
$(124,23)$ & 12 & $(2,1)$ & 1 & 2 & YES & YES & YES & $1.00$ & $(2,2)$ & -- & 1477\\
$(124,23)$ & 12 & $(2,1)$ & 1 & 2 & YES & YES & YES & $1.12$ & $(2,2)$ & NO & 1478\\
$(124,27)$ & 12 & $(4,1)$ & 3 & 4 & YES & YES & YES & $1.22$ & $(2,2)$ & NO & 1479\\
$(124,23)$ & 12 & $(5,1)$ & 4 & 1 & YES & YES & YES & $0.88$ & $(2,2)$ & NO & 1480\\
$(124,23)$ & 12 & $(6,1)$ & 5 & 2 & YES & YES & YES & $1.00$ & $(2,2)$ & NO & 1481\\
$(124,23)$ & 12 & $(11,2)$ & 6 & 1 & YES & YES & YES & $1.00$ & $(2,2)$ & NO & 1482\\
$(124,23)$ & 12 & $(16,3)$ & 7 & 4 & YES & YES & YES & $1.00$ & $(2,2)$ & NO & 1483\\
$(127,29)$ & 11 & $(9,2)$ & 5 & 1 & YES & YES & YES & $1.33$ & $(4,1)$ & NO & 1484\\
$(129,23)$ & 12 & $(2,1)$ & 1 & 1 & YES & YES & YES & $1.22$ & $(2,2)$ & -- & 1485\\
$(129,23)$ & 12 & $(2,1)$ & 1 & 1 & YES & YES & YES & $1.33$ & $(2,2)$ & NO & 1486\\
$(129,23)$ & 12 & $(5,1)$ & 4 & 1 & YES & YES & YES & $1.22$ & $(2,2)$ & NO & 1487\\
$(134,29)$ & 11 & $(2,1)$ & 1 & 2 & YES & YES & YES & $1.33$ & $(4,1)$ & -- & 1488\\
$(134,29)$ & 11 & $(2,1)$ & 1 & 2 & YES & YES & YES & $1.42$ & $(4,1)$ & NO & 1489\\
$(148,31)$ & 12 & $(2,1)$ & 1 & 2 & YES & YES & YES & $1.33$ & $(2,2)$ & NO & 1490\\
$(148,35)$ & 12 & $(2,1)$ & 1 & 2 & YES & YES & YES & $1.33$ & $(2,2)$ & NO & 1491\\
$(148,31)$ & 12 & $(4,1)$ & 3 & 4 & YES & YES & YES & $1.22$ & $(2,2)$ & NO & 1492\\
$(149,34)$ & 11 & $(2,1)$ & 1 & 1 & YES & YES & YES & $1.33$ & $(4,1)$ & NO & 1493\\
$(149,34)$ & 11 & $(2,1)$ & 1 & 1 & YES & YES & YES & $1.25$ & $(4,1)$ & -- & 1494\\
$(149,34)$ & 11 & $(9,2)$ & 5 & 1 & YES & YES & YES & $1.33$ & $(4,1)$ & NO & 1495\\
$(149,34)$ & 11 & $(22,5)$ & 7 & 1 & YES & YES & YES & $1.33$ & $(4,1)$ & NO & 1496\\
$(151,27)$ & 13 & $(5,1)$ & 4 & 1 & YES & YES & YES & $1.22$ & $(2,2)$ & NO & 1497\\
$(154,65)$ & 11 & $(2,1)$ & 1 & 2 & NO & YES & YES & $1.22$ & $(2,2)$ & -- & 1498\\
$(156,29)$ & 12 & $(5,1)$ & 4 & 1 & YES & YES & YES & $1.11$ & $(2,2)$ & NO & 1499\\
$(a;1,0,0;13)$ & 5 & $(11,3)$ & 5 & 1 & YES & YES & YES & $1.33$ & $(2,2)$ & -- & 1500\\
$(a;1,1,1;4)$ & 7 & $(5,2)$ & 3 & 1 & YES & YES & YES & $1.27$ & $(4,1)$ & -- & 1501\\
$(a;2,0,1;25)$ & 7 & $(4,1)$ & 3 & 1 & YES & YES & YES & $1.17$ & $(2,2)$ & -- & 1502\\
$(a;2,1,1;37)$ & 8 & $(3,1)$ & 2 & 1 & YES & YES & YES & $1.12$ & $(2,2)$ & -- & 1503\\
$(a;2,1,1;37)$ & 8 & $(5,2)$ & 3 & 1 & YES & YES & YES & $1.22$ & $(2,2)$ & -- & 1504\\
$(a;3,0,0;7)$ & 7 & $(3,1)$ & 2 & 1 & YES & YES & YES & $0.88$ & $(4,1)$ & -- & 1505\\
$(a;3,0,1;31)$ & 8 & $(2,1)$ & 1 & 1 & YES & YES & YES & $1.00$ & $(2,2)$ & -- & 1506\\
$(a;3,0,1;31)$ & 8 & $(5,1)$ & 4 & 1 & YES & YES & YES & $0.88$ & $(2,2)$ & -- & 1507\\
$(b;0,0,0;14)$ & 5 & $(10,3)$ & 5 & 2 & YES & YES & YES & $1.00$ & $(2,2)$ & -- & 1508\\
$(b;0,0,1;4)$ & 6 & $(7,3)$ & 4 & 1 & YES & YES & YES & $1.00$ & $(6,0)$ & -- & 1509\\
$(b;0,1,0;19)$ & 6 & $(11,3)$ & 5 & 1 & YES & YES & YES & $1.00$ & $(6,0)$ & -- & 1510\\
$(b;0,1,1;27)$ & 7 & $(5,2)$ & 3 & 1 & YES & YES & YES & $1.27$ & $(4,1)$ & -- & 1511\\
$(b;0,1,1;27)$ & 7 & $(7,3)$ & 4 & 1 & YES & YES & YES & $1.22$ & $(2,2)$ & -- & 1512\\
$(b;0,1,3;43)$ & 9 & $(5,1)$ & 4 & 1 & YES & YES & YES & $1.22$ & $(2,2)$ & -- & 1513\\
$(b;0,2,0;8)$ & 7 & $(3,1)$ & 2 & 1 & YES & YES & YES & $0.89$ & $(2,2)$ & -- & 1514\\
$(b;0,2,1;34)$ & 8 & $(5,2)$ & 3 & 1 & YES & YES & YES & $1.22$ & $(2,2)$ & -- & 1515\\
$(b;0,3,0;29)$ & 8 & $(2,1)$ & 1 & 1 & YES & YES & YES & $1.11$ & $(2,2)$ & -- & 1516\\
$(b;0,3,0;29)$ & 8 & $(11,2)$ & 6 & 1 & YES & YES & YES & $1.22$ & $(2,2)$ & -- & 1517\\
$(b;1,0,0;5)$ & 6 & $(7,3)$ & 4 & 1 & YES & YES & YES & $1.00$ & $(2,2)$ & -- & 1518\\
$(b;1,0,0;5)$ & 6 & $(13,4)$ & 6 & 1 & YES & YES & YES & $1.00$ & $(2,2)$ & -- & 1519\\
$(b;1,0,1;29)$ & 7 & $(5,2)$ & 3 & 1 & YES & YES & YES & $1.27$ & $(4,1)$ & -- & 1520\\
$(b;1,0,1;29)$ & 7 & $(10,3)$ & 5 & 1 & YES & YES & YES & $1.27$ & $(4,1)$ & -- & 1521\\
$(b;1,1,0;27)$ & 7 & $(5,2)$ & 3 & 1 & YES & YES & YES & $1.11$ & $(2,2)$ & -- & 1522\\
$(b;1,1,0;27)$ & 7 & $(13,3)$ & 6 & 1 & YES & YES & YES & $1.22$ & $(2,2)$ & -- & 1523\\
$(b;1,1,1;39)$ & 8 & $(2,1)$ & 1 & 1 & YES & YES & YES & $1.27$ & $(4,1)$ & -- & 1524\\
$(b;1,1,1;39)$ & 8 & $(3,1)$ & 2 & 3 & YES & YES & YES & $1.27$ & $(4,1)$ & -- & 1525\\
$(b;1,2,0;17)$ & 8 & $(3,1)$ & 2 & 1 & YES & YES & YES & $0.88$ & $(2,2)$ & -- & 1526\\
$(c;0,0,0;4)$ & 4 & $(18,7)$ & 6 & 2 & YES & YES & YES & $1.20$ & $(2,2)$ & -- & 1527\\
$(c;0,0,0;4)$ & 4 & $(22,9)$ & 7 & 2 & YES & YES & YES & $1.12$ & $(2,2)$ & -- & 1528\\
$(c;0,0,0;4)$ & 4 & $(26,11)$ & 7 & 2 & YES & YES & YES & $1.25$ & $(2,2)$ & -- & 1529\\
$(c;0,0,0;4)$ & 4 & $(29,11)$ & 7 & 1 & YES & YES & YES & $1.12$ & $(2,2)$ & -- & 1530\\
$(c;0,0,0;4)$ & 4 & $(29,12)$ & 7 & 1 & YES & YES & YES & $1.00$ & $(2,2)$ & -- & 1531\\
$(c;0,0,0;4)$ & 4 & $(31,9)$ & 8 & 1 & YES & YES & YES & $1.22$ & $(2,2)$ & -- & 1532\\
$(c;0,0,0;4)$ & 4 & $(31,12)$ & 7 & 1 & YES & YES & YES & $1.42$ & $(4,1)$ & -- & 1533\\
$(c;0,0,0;4)$ & 4 & $(34,13)$ & 7 & 2 & YES & YES & YES & $1.42$ & $(4,1)$ & -- & 1534\\
$(c;0,1,0;11)$ & 5 & $(9,4)$ & 5 & 1 & YES & YES & YES & $1.31$ & $(2,2)$ & -- & 1535\\
$(c;0,1,0;11)$ & 5 & $(13,5)$ & 5 & 1 & YES & YES & YES & $1.12$ & $(2,2)$ & -- & 1536\\
$(c;0,1,0;11)$ & 5 & $(17,7)$ & 6 & 1 & YES & YES & YES & $1.12$ & $(2,2)$ & -- & 1537\\
$(c;0,1,0;11)$ & 5 & $(18,7)$ & 6 & 1 & YES & YES & YES & $1.11$ & $(6,0)$ & -- & 1538\\
$(c;0,1,0;11)$ & 5 & $(19,7)$ & 6 & 1 & YES & YES & YES & $1.22$ & $(2,2)$ & -- & 1539\\
$(c;0,1,0;11)$ & 5 & $(21,8)$ & 6 & 1 & YES & YES & YES & $1.27$ & $(4,1)$ & -- & 1540\\
$(c;0,1,0;11)$ & 5 & $(24,7)$ & 7 & 1 & YES & YES & YES & $1.11$ & $(6,0)$ & -- & 1541\\
$(c;0,1,0;11)$ & 5 & $(47,11)$ & 9 & 1 & YES & YES & YES & $1.22$ & $(2,2)$ & -- & 1542\\
$(c;0,1,1;5)$ & 6 & $(13,5)$ & 5 & 1 & YES & YES & YES & $1.27$ & $(4,1)$ & -- & 1543\\
$(c;0,1,1;5)$ & 6 & $(17,5)$ & 6 & 1 & YES & YES & YES & $1.22$ & $(2,2)$ & -- & 1544\\
$(c;0,2,0;7)$ & 6 & $(5,2)$ & 3 & 1 & YES & YES & YES & $1.10$ & $(2,2)$ & -- & 1545\\
$(c;0,2,0;7)$ & 6 & $(7,3)$ & 4 & 7 & YES & YES & YES & $1.10$ & $(2,2)$ & -- & 1546\\
$(c;0,2,0;7)$ & 6 & $(8,3)$ & 4 & 1 & YES & YES & YES & $0.75$ & $(4,1)$ & -- & 1547\\
$(c;0,2,0;7)$ & 6 & $(9,2)$ & 5 & 1 & YES & YES & YES & $1.10$ & $(2,2)$ & -- & 1548\\
$(c;0,2,0;7)$ & 6 & $(9,4)$ & 5 & 1 & YES & YES & YES & $0.88$ & $(2,2)$ & -- & 1549\\
$(c;0,2,0;7)$ & 6 & $(13,4)$ & 6 & 1 & YES & YES & YES & $1.12$ & $(2,2)$ & -- & 1550\\
$(c;0,2,0;7)$ & 6 & $(13,5)$ & 5 & 1 & YES & YES & YES & $1.11$ & $(2,2)$ & -- & 1551\\
$(c;0,2,0;7)$ & 6 & $(15,4)$ & 6 & 1 & YES & YES & YES & $1.11$ & $(2,2)$ & -- & 1552\\
$(c;0,2,0;7)$ & 6 & $(17,4)$ & 7 & 1 & YES & YES & YES & $1.11$ & $(2,2)$ & -- & 1553\\
$(c;0,2,0;7)$ & 6 & $(17,5)$ & 6 & 1 & YES & YES & YES & $1.12$ & $(2,2)$ & -- & 1554\\
$(c;0,2,0;7)$ & 6 & $(18,5)$ & 6 & 1 & YES & YES & YES & $1.12$ & $(2,2)$ & -- & 1555\\
$(c;0,2,0;7)$ & 6 & $(22,5)$ & 7 & 1 & YES & YES & YES & $1.12$ & $(2,2)$ & -- & 1556\\
$(c;0,2,1;19)$ & 7 & $(3,1)$ & 2 & 1 & YES & YES & YES & $1.09$ & $(2,2)$ & -- & 1557\\
$(c;0,2,1;19)$ & 7 & $(4,1)$ & 3 & 1 & YES & YES & YES & $1.10$ & $(2,2)$ & -- & 1558\\
$(c;0,2,1;19)$ & 7 & $(9,2)$ & 5 & 1 & YES & YES & YES & $1.00$ & $(2,2)$ & -- & 1559\\
$(c;0,2,1;19)$ & 7 & $(10,3)$ & 5 & 1 & YES & YES & YES & $0.88$ & $(2,2)$ & -- & 1560\\
$(c;0,2,1;19)$ & 7 & $(17,4)$ & 7 & 1 & YES & YES & YES & $0.75$ & $(6,0)$ & -- & 1561\\
$(c;0,2,2;6)$ & 8 & $(3,1)$ & 2 & 3 & YES & YES & YES & $0.88$ & $(2,2)$ & -- & 1562\\
$(c;0,2,2;6)$ & 8 & $(5,1)$ & 4 & 1 & YES & YES & YES & $1.10$ & $(2,2)$ & -- & 1563\\
$(c;0,2,2;6)$ & 8 & $(7,2)$ & 4 & 1 & YES & YES & YES & $1.00$ & $(2,2)$ & -- & 1564\\
$(c;0,3,1;23)$ & 8 & $(2,1)$ & 1 & 1 & YES & YES & YES & $1.00$ & $(2,2)$ & -- & 1565\\
$(c;0,3,1;23)$ & 8 & $(3,1)$ & 2 & 1 & YES & YES & YES & $1.00$ & $(2,2)$ & -- & 1566\\
$(c;0,3,1;23)$ & 8 & $(5,1)$ & 4 & 1 & YES & YES & YES & $1.00$ & $(2,2)$ & -- & 1567\\
$(c;0,3,1;23)$ & 8 & $(6,1)$ & 5 & 1 & YES & YES & YES & $1.00$ & $(2,2)$ & -- & 1568\\
$(c;0,3,2;29)$ & 9 & $(3,1)$ & 2 & 1 & YES & YES & YES & $0.88$ & $(2,2)$ & -- & 1569\\
$(c;0,3,2;29)$ & 9 & $(6,1)$ & 5 & 1 & YES & YES & YES & $0.88$ & $(2,2)$ & -- & 1570\\
$(d;0,0,0;5)$ & 5 & $(9,4)$ & 5 & 1 & YES & YES & YES & $1.00$ & $(2,2)$ & -- & 1571\\
$(d;0,0,0;5)$ & 5 & $(13,5)$ & 5 & 1 & YES & YES & YES & $1.30$ & $(2,2)$ & -- & 1572\\
$(d;0,0,0;5)$ & 5 & $(17,5)$ & 6 & 1 & YES & YES & YES & $1.30$ & $(2,2)$ & -- & 1573\\
$(d;0,0,0;5)$ & 5 & $(17,7)$ & 6 & 1 & YES & YES & YES & $1.12$ & $(2,2)$ & -- & 1574\\
$(d;0,0,0;5)$ & 5 & $(18,7)$ & 6 & 1 & YES & YES & YES & $1.11$ & $(6,0)$ & -- & 1575\\
$(d;0,0,0;5)$ & 5 & $(19,8)$ & 6 & 1 & YES & YES & YES & $1.22$ & $(2,2)$ & -- & 1576\\
$(d;0,0,0;5)$ & 5 & $(21,8)$ & 6 & 1 & YES & YES & YES & $1.27$ & $(4,1)$ & -- & 1577\\
$(d;0,0,0;5)$ & 5 & $(24,7)$ & 7 & 1 & YES & YES & YES & $1.11$ & $(6,0)$ & -- & 1578\\
$(d;0,0,0;5)$ & 5 & $(29,8)$ & 7 & 1 & YES & YES & YES & $1.22$ & $(2,2)$ & -- & 1579\\
$(d;0,0,1;14)$ & 6 & $(12,5)$ & 5 & 2 & YES & YES & YES & $1.11$ & $(2,2)$ & -- & 1580\\
$(d;0,0,1;14)$ & 6 & $(13,4)$ & 6 & 1 & YES & YES & YES & $1.00$ & $(4,1)$ & -- & 1581\\
$(d;0,0,1;14)$ & 6 & $(13,5)$ & 5 & 1 & YES & YES & YES & $1.11$ & $(4,1)$ & -- & 1582\\
$(d;0,0,2;9)$ & 7 & $(3,1)$ & 2 & 3 & YES & YES & YES & $1.09$ & $(2,2)$ & -- & 1583\\
$(d;0,0,3;22)$ & 8 & $(2,1)$ & 1 & 2 & YES & YES & YES & $1.00$ & $(2,2)$ & -- & 1584\\
$(d;0,0,3;22)$ & 8 & $(6,1)$ & 5 & 2 & YES & YES & YES & $1.00$ & $(2,2)$ & -- & 1585\\
$(d;0,1,0;6)$ & 6 & $(9,2)$ & 5 & 3 & YES & YES & YES & $1.10$ & $(2,2)$ & -- & 1586\\
$(d;0,1,0;6)$ & 6 & $(12,5)$ & 5 & 6 & YES & YES & YES & $1.11$ & $(2,2)$ & -- & 1587\\
$(d;0,1,0;6)$ & 6 & $(13,4)$ & 6 & 1 & YES & YES & YES & $1.12$ & $(2,2)$ & -- & 1588\\
$(d;0,1,0;6)$ & 6 & $(13,5)$ & 5 & 1 & YES & YES & YES & $1.11$ & $(2,2)$ & -- & 1589\\
$(d;0,1,0;6)$ & 6 & $(15,4)$ & 6 & 3 & YES & YES & YES & $1.33$ & $(2,2)$ & -- & 1590\\
$(d;0,1,2;11)$ & 8 & $(2,1)$ & 1 & 1 & YES & YES & YES & $0.88$ & $(2,2)$ & -- & 1591\\
$(d;0,1,2;11)$ & 8 & $(5,1)$ & 4 & 1 & YES & YES & YES & $1.10$ & $(2,2)$ & -- & 1592\\
$(d;0,1,2;11)$ & 8 & $(7,2)$ & 4 & 1 & YES & YES & YES & $1.00$ & $(2,2)$ & -- & 1593\\
$(d;0,1,3;27)$ & 9 & $(2,1)$ & 1 & 1 & YES & YES & YES & $1.00$ & $(2,2)$ & -- & 1594\\
$(d;0,2,2;13)$ & 9 & $(2,1)$ & 1 & 1 & YES & YES & YES & $0.88$ & $(2,2)$ & -- & 1595\\
$(d;0,2,2;13)$ & 9 & $(6,1)$ & 5 & 1 & YES & YES & YES & $0.88$ & $(2,2)$ & -- & 1596\\
$(e;0,0,0;4)$ & 5 & $(7,3)$ & 4 & 1 & YES & YES & YES & $1.33$ & $(2,2)$ & -- & 1597\\
$(e;0,0,0;4)$ & 5 & $(10,3)$ & 5 & 2 & YES & YES & YES & $1.45$ & $(2,2)$ & -- & 1598\\
$(e;0,0,0;4)$ & 5 & $(17,5)$ & 6 & 1 & YES & YES & YES & $1.11$ & $(4,1)$ & -- & 1599\\
$(e;0,1,0;5)$ & 6 & $(3,1)$ & 2 & 1 & YES & YES & YES & $1.10$ & $(2,2)$ & -- & 1600\\
$(e;0,3,0;7)$ & 8 & $(2,1)$ & 1 & 1 & YES & YES & YES & $1.11$ & $(2,2)$ & -- & 1601\\
$(e;0,3,0;7)$ & 8 & $(6,1)$ & 5 & 1 & YES & YES & YES & $1.00$ & $(2,2)$ & -- & 1602\\
$(e;0,3,0;7)$ & 8 & $(11,2)$ & 6 & 1 & YES & YES & YES & $1.22$ & $(2,2)$ & -- & 1603\\
$(e;1,0,0;18)$ & 6 & $(7,3)$ & 4 & 1 & YES & YES & YES & $1.22$ & $(2,2)$ & -- & 1604\\
$(e;1,0,0;18)$ & 6 & $(8,3)$ & 4 & 2 & YES & YES & YES & $1.20$ & $(6,0)$ & -- & 1605\\
$(e;1,0,0;18)$ & 6 & $(10,3)$ & 5 & 2 & YES & YES & YES & $1.00$ & $(2,2)$ & -- & 1606\\
$(e;1,1,0;23)$ & 7 & $(5,2)$ & 3 & 1 & YES & YES & YES & $1.33$ & $(4,1)$ & -- & 1607\\
$(e;1,1,0;23)$ & 7 & $(7,3)$ & 4 & 1 & YES & YES & YES & $1.22$ & $(2,2)$ & -- & 1608\\
$(e;2,0,0;24)$ & 7 & $(2,1)$ & 1 & 2 & YES & YES & NO(2) & $0.90$ & $(4,1)$ & -- & 1609\\
$(f;0,0,0;6)$ & 4 & $(11,4)$ & 5 & 1 & YES & YES & NO(2) & $1.09$ & $(2,2)$ & -- & 1610\\
$(f;0,0,0;6)$ & 4 & $(12,5)$ & 5 & 6 & YES & YES & YES & $1.10$ & $(2,2)$ & -- & 1611\\
$(f;0,0,0;6)$ & 4 & $(13,4)$ & 6 & 1 & YES & YES & YES & $1.11$ & $(2,2)$ & -- & 1612\\
$(f;0,0,0;6)$ & 4 & $(16,5)$ & 7 & 2 & YES & YES & YES & $1.11$ & $(2,2)$ & -- & 1613\\
$(f;0,0,0;6)$ & 4 & $(18,7)$ & 6 & 6 & YES & YES & YES & $0.88$ & $(2,2)$ & -- & 1614\\
$(f;0,0,0;6)$ & 4 & $(27,10)$ & 7 & 3 & YES & YES & YES & $0.88$ & $(2,2)$ & -- & 1615\\
$(f;0,0,0;6)$ & 4 & $(29,11)$ & 7 & 1 & YES & YES & YES & $1.11$ & $(6,0)$ & -- & 1616\\
$(f;0,0,0;6)$ & 4 & $(40,11)$ & 8 & 2 & YES & YES & YES & $1.00$ & $(2,2)$ & -- & 1617\\
$(f;0,0,0;6)$ & 4 & $(44,17)$ & 8 & 2 & YES & YES & YES & $1.33$ & $(2,2)$ & -- & 1618\\
$(f;0,1,0;7)$ & 5 & $(10,3)$ & 5 & 1 & YES & YES & YES & $0.88$ & $(4,1)$ & -- & 1619\\
$(g;0,0,0;19)$ & 6 & $(7,3)$ & 4 & 1 & YES & YES & YES & $1.00$ & $(2,2)$ & -- & 1620\\
$(g;0,0,0;19)$ & 6 & $(8,3)$ & 4 & 1 & YES & YES & YES & $1.22$ & $(2,2)$ & -- & 1621\\
$(g;0,0,0;19)$ & 6 & $(13,4)$ & 6 & 1 & YES & YES & YES & $1.00$ & $(2,2)$ & -- & 1622\\
$(g;0,0,1;26)$ & 7 & $(5,2)$ & 3 & 1 & YES & YES & YES & $1.42$ & $(4,1)$ & -- & 1623\\
$(g;0,0,2;11)$ & 8 & $(2,1)$ & 1 & 1 & YES & YES & YES & $1.00$ & $(2,2)$ & -- & 1624\\
$(g;0,0,2;11)$ & 8 & $(3,1)$ & 2 & 1 & YES & YES & YES & $1.00$ & $(2,2)$ & -- & 1625\\
$(g;0,0,2;11)$ & 8 & $(5,1)$ & 4 & 1 & YES & YES & YES & $1.00$ & $(2,2)$ & -- & 1626\\
$(g;0,0,2;11)$ & 8 & $(11,2)$ & 6 & 11 & YES & YES & YES & $1.22$ & $(2,2)$ & -- & 1627\\
$(g;0,1,0;24)$ & 7 & $(5,2)$ & 3 & 1 & YES & YES & YES & $1.42$ & $(4,1)$ & -- & 1628\\
$(g;0,1,0;24)$ & 7 & $(13,3)$ & 6 & 1 & YES & YES & YES & $1.27$ & $(4,1)$ & -- & 1629\\
$(g;0,2,0;29)$ & 8 & $(2,1)$ & 1 & 1 & YES & YES & YES & $1.00$ & $(2,2)$ & -- & 1630\\
$(g;0,2,0;29)$ & 8 & $(5,1)$ & 4 & 1 & YES & YES & YES & $1.00$ & $(2,2)$ & -- & 1631\\
$(g;1,0,0;7)$ & 7 & $(5,2)$ & 3 & 1 & YES & YES & YES & $1.18$ & $(4,1)$ & -- & 1632\\
$(g;1,0,1;38)$ & 8 & $(2,1)$ & 1 & 2 & YES & YES & YES & $1.33$ & $(4,1)$ & -- & 1633\\
$(g;1,0,1;38)$ & 8 & $(4,1)$ & 3 & 2 & YES & YES & YES & $1.33$ & $(4,1)$ & -- & 1634\\
$(g;1,1,0;9)$ & 8 & $(2,1)$ & 1 & 1 & YES & YES & YES & $1.33$ & $(4,1)$ & -- & 1635\\
$(h;0,0,0;6)$ & 5 & $(8,3)$ & 4 & 2 & YES & YES & YES & $1.22$ & $(2,2)$ & -- & 1636\\
$(h;0,0,0;6)$ & 5 & $(10,3)$ & 5 & 2 & YES & YES & YES & $1.30$ & $(2,2)$ & -- & 1637\\
$(h;0,0,0;6)$ & 5 & $(12,5)$ & 5 & 6 & YES & YES & YES & $1.00$ & $(2,2)$ & -- & 1638\\
$(h;0,0,0;6)$ & 5 & $(13,5)$ & 5 & 1 & YES & YES & YES & $1.22$ & $(2,2)$ & -- & 1639\\
$(h;0,1,0;8)$ & 6 & $(7,3)$ & 4 & 1 & YES & YES & YES & $1.00$ & $(2,2)$ & -- & 1640\\
$(h;0,1,0;8)$ & 6 & $(13,4)$ & 6 & 1 & YES & YES & YES & $1.00$ & $(2,2)$ & -- & 1641\\
$(i;0,0,0;9)$ & 5 & $(5,2)$ & 3 & 1 & YES & YES & NO(2) & $1.23$ & $(2,2)$ & -- & 1642\\
$(i;0,0,0;9)$ & 5 & $(7,2)$ & 4 & 1 & YES & YES & YES & $1.10$ & $(2,2)$ & -- & 1643\\
$(i;0,0,0;9)$ & 5 & $(8,3)$ & 4 & 1 & YES & YES & YES & $1.10$ & $(2,2)$ & -- & 1644\\
$(i;0,0,0;9)$ & 5 & $(10,3)$ & 5 & 1 & YES & YES & YES & $0.89$ & $(2,2)$ & -- & 1645\\
$(i;0,0,0;9)$ & 5 & $(17,5)$ & 6 & 1 & YES & YES & YES & $1.00$ & $(2,2)$ & -- & 1646\\
$(i;0,0,0;9)$ & 5 & $(18,5)$ & 6 & 9 & YES & YES & YES & $1.00$ & $(2,2)$ & -- & 1647\\
$(i;0,0,0;9)$ & 5 & $(19,4)$ & 7 & 1 & YES & YES & YES & $1.11$ & $(2,2)$ & -- & 1648\\
$(i;0,1,0;12)$ & 6 & $(4,1)$ & 3 & 4 & YES & YES & YES & $1.10$ & $(2,2)$ & -- & 1649\\
$(i;0,1,0;12)$ & 6 & $(7,3)$ & 4 & 1 & YES & YES & YES & $0.88$ & $(2,2)$ & -- & 1650\\
$(i;0,1,0;12)$ & 6 & $(10,3)$ & 5 & 2 & YES & YES & YES & $1.12$ & $(2,2)$ & -- & 1651\\
$(i;0,1,0;12)$ & 6 & $(11,3)$ & 5 & 1 & YES & YES & YES & $1.12$ & $(2,2)$ & -- & 1652\\
$(i;0,2,0;15)$ & 7 & $(3,1)$ & 2 & 3 & YES & YES & YES & $0.89$ & $(2,2)$ & -- & 1653\\
$(i;0,2,0;15)$ & 7 & $(13,3)$ & 6 & 1 & YES & YES & YES & $1.22$ & $(2,2)$ & -- & 1654\\
$(j;0,0,0;8)$ & 5 & $(8,3)$ & 4 & 8 & YES & YES & YES & $1.25$ & $(2,2)$ & -- & 1655\\
$(j;0,0,0;8)$ & 5 & $(9,4)$ & 5 & 1 & YES & YES & YES & $1.11$ & $(2,2)$ & -- & 1656\\
$(j;0,0,0;8)$ & 5 & $(10,3)$ & 5 & 2 & YES & YES & YES & $0.89$ & $(2,2)$ & -- & 1657\\
$(j;0,0,0;8)$ & 5 & $(11,4)$ & 5 & 1 & YES & YES & YES & $0.88$ & $(2,2)$ & -- & 1658\\
$(j;0,0,0;8)$ & 5 & $(17,7)$ & 6 & 1 & YES & YES & YES & $0.88$ & $(2,2)$ & -- & 1659\\
$(j;0,0,0;8)$ & 5 & $(23,7)$ & 7 & 1 & YES & YES & YES & $1.22$ & $(2,2)$ & -- & 1660\\
$(j;0,0,0;8)$ & 5 & $(24,7)$ & 7 & 8 & YES & YES & YES & $1.12$ & $(2,2)$ & -- & 1661\\
$(j;0,1,0;10)$ & 6 & $(9,4)$ & 5 & 1 & YES & YES & YES & $0.88$ & $(2,2)$ & -- & 1662\\
$(j;0,1,0;10)$ & 6 & $(11,4)$ & 5 & 1 & YES & YES & YES & $0.88$ & $(2,2)$ & -- & 1663\\
$(j;0,1,0;10)$ & 6 & $(18,7)$ & 6 & 2 & YES & YES & YES & $1.22$ & $(2,2)$ & -- & 1664
\end{longtable}
\subsection{2 chains, $K^2 = 3$}
\begin{longtable}{|c|c|c|c|c|c|c|c|c|c|c|c|}
\hline
\multicolumn{12}{|c|}{2 chains, $K^2 = 3$}\\
\hline
$(n,a)$ & Len & $(n,a)$ & Len & GCD & Nef & $\mathbb Q$-ef & Obs 0 & $\overline c_1^2 / \overline c_2$ & $(P,K)$ & WH & Index\\
\hline
\endfirsthead

\hline
$(n,a)$ & Len & $(n,a)$ & Len & GCD & Nef & $\mathbb Q$-ef & Obs 0 & $\overline c_1^2 / \overline c_2$ & $(P,K)$ & WH & Index\\
\hline
\endhead
\hline
\endfoot

$(16,7)$ & 6 & $(14,5)$ & 6 & 2 & YES & YES & YES & $1.38$ & $(4,2)$ & -- & 1665\\
$(18,7)$ & 6 & $(11,4)$ & 5 & 1 & YES & YES & NO(2) & $1.50$ & $(2,3)$ & NO & 1666\\
$(19,8)$ & 6 & $(12,5)$ & 5 & 1 & YES & YES & YES & $1.50$ & $(2,3)$ & -- & 1667\\
$(22,9)$ & 7 & $(11,3)$ & 5 & 11 & YES & YES & NO(2) & $1.55$ & $(2,3)$ & NO & 1668\\
$(22,9)$ & 7 & $(11,3)$ & 5 & 11 & YES & YES & NO(2) & $1.55$ & $(2,3)$ & -- & 1669\\
$(23,9)$ & 7 & $(16,5)$ & 7 & 1 & YES & YES & YES & $1.57$ & $(2,3)$ & NO & 1670\\
$(23,9)$ & 7 & $(16,5)$ & 7 & 1 & YES & YES & YES & $1.57$ & $(2,3)$ & -- & 1671\\
$(23,10)$ & 7 & $(18,7)$ & 6 & 1 & YES & YES & NO(2) & $1.50$ & $(2,3)$ & NO & 1672\\
$(25,9)$ & 7 & $(21,5)$ & 8 & 1 & YES & YES & YES & $1.50$ & $(2,3)$ & NO & 1673\\
$(25,9)$ & 7 & $(21,5)$ & 8 & 1 & YES & YES & YES & $1.50$ & $(2,3)$ & -- & 1674\\
$(25,7)$ & 7 & $(23,10)$ & 7 & 1 & YES & YES & NO(2) & $1.73$ & $(2,3)$ & -- & 1675\\
$(26,11)$ & 7 & $(7,3)$ & 4 & 1 & YES & YES & NO(2) & $1.50$ & $(2,3)$ & -- & 1676\\
$(26,11)$ & 7 & $(9,4)$ & 5 & 1 & YES & YES & NO(2) & $1.50$ & $(2,3)$ & -- & 1677\\
$(26,11)$ & 7 & $(24,5)$ & 8 & 2 & YES & YES & YES & $1.50$ & $(2,3)$ & -- & 1678\\
$(27,8)$ & 7 & $(10,3)$ & 5 & 1 & YES & YES & NO(2) & $1.40$ & $(4,2)$ & -- & 1679\\
$(27,10)$ & 7 & $(11,5)$ & 6 & 1 & YES & YES & YES & $1.29$ & $(4,2)$ & -- & 1680\\
$(27,11)$ & 8 & $(13,4)$ & 6 & 1 & YES & YES & YES & $1.70$ & $(2,3)$ & NO & 1681\\
$(27,11)$ & 8 & $(13,4)$ & 6 & 1 & YES & YES & YES & $1.70$ & $(2,3)$ & -- & 1682\\
$(27,8)$ & 7 & $(21,8)$ & 6 & 3 & YES & YES & YES & $1.38$ & $(6,1)$ & NO & 1683\\
$(27,8)$ & 7 & $(21,8)$ & 6 & 3 & YES & YES & YES & $1.38$ & $(6,1)$ & -- & 1684\\
$(27,10)$ & 7 & $(21,8)$ & 6 & 3 & YES & YES & YES & $1.43$ & $(4,2)$ & -- & 1685\\
$(28,11)$ & 8 & $(27,8)$ & 7 & 1 & YES & YES & YES & $1.57$ & $(2,3)$ & -- & 1686\\
$(29,13)$ & 8 & $(14,5)$ & 6 & 1 & YES & YES & YES & $1.29$ & $(4,2)$ & NO & 1687\\
$(29,12)$ & 7 & $(16,5)$ & 7 & 1 & YES & YES & YES & $1.50$ & $(2,3)$ & -- & 1688\\
$(29,12)$ & 7 & $(17,5)$ & 6 & 1 & YES & YES & YES & $1.38$ & $(6,1)$ & -- & 1689\\
$(29,8)$ & 7 & $(21,8)$ & 6 & 1 & YES & YES & YES & $1.38$ & $(6,1)$ & -- & 1690\\
$(29,8)$ & 7 & $(24,7)$ & 7 & 1 & YES & YES & YES & $1.29$ & $(8,0)$ & NO & 1691\\
$(29,8)$ & 7 & $(24,7)$ & 7 & 1 & YES & YES & YES & $1.29$ & $(8,0)$ & -- & 1692\\
$(29,12)$ & 7 & $(27,10)$ & 7 & 1 & YES & YES & YES & $1.50$ & $(6,1)$ & -- & 1693\\
$(29,8)$ & 7 & $(28,11)$ & 8 & 1 & YES & YES & YES & $1.71$ & $(2,3)$ & -- & 1694\\
$(29,12)$ & 7 & $(29,8)$ & 7 & 29 & YES & YES & YES & $1.43$ & $(2,3)$ & -- & 1695\\
$(29,12)$ & 7 & $(29,11)$ & 7 & 29 & YES & YES & YES & $1.60$ & $(2,3)$ & -- & 1696\\
$(30,13)$ & 8 & $(9,4)$ & 5 & 3 & YES & YES & NO(2) & $1.50$ & $(2,3)$ & -- & 1697\\
$(30,11)$ & 7 & $(25,7)$ & 7 & 5 & YES & YES & YES & $1.57$ & $(2,3)$ & -- & 1698\\
$(30,11)$ & 7 & $(25,7)$ & 7 & 5 & YES & YES & YES & $1.43$ & $(4,2)$ & NO & 1699\\
$(31,7)$ & 8 & $(13,4)$ & 6 & 1 & YES & YES & YES & $1.38$ & $(2,3)$ & -- & 1700\\
$(31,9)$ & 8 & $(17,4)$ & 7 & 1 & YES & YES & YES & $1.43$ & $(4,2)$ & NO & 1701\\
$(31,9)$ & 8 & $(17,4)$ & 7 & 1 & YES & YES & YES & $1.43$ & $(4,2)$ & -- & 1702\\
$(31,9)$ & 8 & $(24,7)$ & 7 & 1 & YES & YES & YES & $1.50$ & $(6,1)$ & -- & 1703\\
$(31,13)$ & 7 & $(24,7)$ & 7 & 1 & YES & YES & YES & $1.57$ & $(2,3)$ & -- & 1704\\
$(31,13)$ & 7 & $(25,7)$ & 7 & 1 & YES & YES & YES & $1.57$ & $(2,3)$ & -- & 1705\\
$(31,7)$ & 8 & $(26,11)$ & 7 & 1 & YES & YES & YES & $1.56$ & $(2,3)$ & -- & 1706\\
$(31,12)$ & 7 & $(26,11)$ & 7 & 1 & YES & YES & YES & $1.67$ & $(4,2)$ & -- & 1707\\
$(31,12)$ & 7 & $(27,8)$ & 7 & 1 & YES & YES & YES & $1.43$ & $(2,3)$ & NO & 1708\\
$(31,12)$ & 7 & $(28,11)$ & 8 & 1 & YES & YES & YES & $1.62$ & $(2,3)$ & -- & 1709\\
$(31,9)$ & 8 & $(29,11)$ & 7 & 1 & YES & YES & YES & $1.60$ & $(2,3)$ & -- & 1710\\
$(32,9)$ & 8 & $(24,7)$ & 7 & 8 & YES & YES & YES & $1.50$ & $(6,1)$ & -- & 1711\\
$(33,10)$ & 8 & $(9,4)$ & 5 & 3 & YES & YES & YES & $1.29$ & $(4,2)$ & NO & 1712\\
$(33,10)$ & 8 & $(9,4)$ & 5 & 3 & YES & YES & YES & $1.29$ & $(4,2)$ & -- & 1713\\
$(33,14)$ & 8 & $(23,4)$ & 8 & 1 & YES & YES & YES & $1.43$ & $(2,3)$ & -- & 1714\\
$(33,10)$ & 8 & $(24,7)$ & 7 & 3 & YES & YES & YES & $1.70$ & $(4,2)$ & -- & 1715\\
$(33,10)$ & 8 & $(25,7)$ & 7 & 1 & YES & YES & YES & $1.43$ & $(2,3)$ & NO & 1716\\
$(33,10)$ & 8 & $(31,12)$ & 7 & 1 & YES & YES & YES & $1.56$ & $(4,2)$ & -- & 1717\\
$(34,13)$ & 7 & $(12,5)$ & 5 & 2 & YES & YES & YES & $1.38$ & $(6,1)$ & -- & 1718\\
$(34,13)$ & 7 & $(17,5)$ & 6 & 17 & YES & YES & YES & $1.50$ & $(6,1)$ & -- & 1719\\
$(34,13)$ & 7 & $(17,5)$ & 6 & 17 & YES & YES & YES & $1.67$ & $(6,1)$ & NO & 1720\\
$(34,13)$ & 7 & $(23,9)$ & 7 & 1 & YES & YES & YES & $1.57$ & $(2,3)$ & -- & 1721\\
$(34,13)$ & 7 & $(25,7)$ & 7 & 1 & YES & YES & YES & $1.38$ & $(4,2)$ & NO & 1722\\
$(34,13)$ & 7 & $(25,7)$ & 7 & 1 & YES & YES & YES & $1.83$ & $(2,3)$ & -- & 1723\\
$(34,13)$ & 7 & $(31,12)$ & 7 & 1 & YES & YES & YES & $1.67$ & $(4,2)$ & -- & 1724\\
$(34,15)$ & 8 & $(32,7)$ & 8 & 2 & YES & YES & YES & $1.71$ & $(2,3)$ & NO & 1725\\
$(35,13)$ & 8 & $(9,4)$ & 5 & 1 & YES & YES & NO(2) & $1.55$ & $(2,3)$ & NO & 1726\\
$(35,13)$ & 8 & $(31,7)$ & 8 & 1 & YES & YES & YES & $1.75$ & $(2,3)$ & -- & 1727\\
$(35,13)$ & 8 & $(35,8)$ & 8 & 35 & YES & YES & YES & $1.56$ & $(4,2)$ & -- & 1728\\
$(36,13)$ & 8 & $(9,4)$ & 5 & 9 & YES & YES & YES & $1.29$ & $(4,2)$ & NO & 1729\\
$(36,13)$ & 8 & $(9,4)$ & 5 & 9 & YES & YES & YES & $1.29$ & $(4,2)$ & -- & 1730\\
$(36,13)$ & 8 & $(11,5)$ & 6 & 1 & YES & YES & YES & $1.29$ & $(4,2)$ & NO & 1731\\
$(36,11)$ & 8 & $(24,7)$ & 7 & 12 & YES & YES & YES & $1.75$ & $(2,3)$ & NO & 1732\\
$(36,11)$ & 8 & $(24,7)$ & 7 & 12 & YES & YES & YES & $1.75$ & $(2,3)$ & -- & 1733\\
$(36,11)$ & 8 & $(25,7)$ & 7 & 1 & YES & YES & YES & $1.75$ & $(2,3)$ & -- & 1734\\
$(36,11)$ & 8 & $(25,7)$ & 7 & 1 & YES & YES & YES & $1.75$ & $(2,3)$ & NO & 1735\\
$(37,11)$ & 8 & $(7,3)$ & 4 & 1 & YES & YES & NO(2) & $1.44$ & $(4,2)$ & NO & 1736\\
$(37,14)$ & 8 & $(9,4)$ & 5 & 1 & YES & YES & NO(2) & $1.55$ & $(2,3)$ & NO & 1737\\
$(37,11)$ & 8 & $(17,7)$ & 6 & 1 & YES & YES & YES & $1.75$ & $(2,3)$ & -- & 1738\\
$(37,14)$ & 8 & $(17,5)$ & 6 & 1 & YES & YES & YES & $1.62$ & $(2,3)$ & NO & 1739\\
$(37,14)$ & 8 & $(17,5)$ & 6 & 1 & YES & YES & YES & $1.62$ & $(2,3)$ & -- & 1740\\
$(37,14)$ & 8 & $(31,7)$ & 8 & 1 & YES & YES & YES & $1.56$ & $(4,2)$ & -- & 1741\\
$(37,14)$ & 8 & $(32,7)$ & 8 & 1 & YES & YES & YES & $1.57$ & $(4,2)$ & NO & 1742\\
$(37,14)$ & 8 & $(32,7)$ & 8 & 1 & YES & YES & YES & $1.67$ & $(4,2)$ & -- & 1743\\
$(37,8)$ & 8 & $(35,13)$ & 8 & 1 & YES & YES & YES & $1.62$ & $(4,2)$ & NO & 1744\\
$(38,9)$ & 9 & $(11,4)$ & 5 & 1 & YES & YES & YES & $1.56$ & $(2,3)$ & NO & 1745\\
$(38,9)$ & 9 & $(11,4)$ & 5 & 1 & YES & YES & YES & $1.56$ & $(2,3)$ & -- & 1746\\
$(39,14)$ & 8 & $(5,2)$ & 3 & 1 & YES & YES & YES & $1.29$ & $(4,2)$ & -- & 1747\\
$(39,16)$ & 8 & $(17,5)$ & 6 & 1 & YES & YES & YES & $1.38$ & $(6,1)$ & -- & 1748\\
$(39,16)$ & 8 & $(21,5)$ & 8 & 3 & YES & YES & YES & $1.50$ & $(2,3)$ & NO & 1749\\
$(39,16)$ & 8 & $(21,8)$ & 6 & 3 & YES & YES & YES & $1.62$ & $(4,2)$ & -- & 1750\\
$(39,14)$ & 8 & $(24,7)$ & 7 & 3 & YES & YES & YES & $1.57$ & $(2,3)$ & -- & 1751\\
$(39,14)$ & 8 & $(31,12)$ & 7 & 1 & YES & YES & YES & $1.57$ & $(2,3)$ & 1919 & 1752\\
$(39,7)$ & 9 & $(38,11)$ & 9 & 1 & YES & YES & YES & $1.62$ & $(6,1)$ & NO & 1753\\
$(39,11)$ & 9 & $(38,7)$ & 9 & 1 & YES & YES & YES & $1.57$ & $(2,3)$ & NO & 1754\\
$(40,11)$ & 8 & $(17,4)$ & 7 & 1 & YES & YES & YES & $1.57$ & $(2,3)$ & -- & 1755\\
$(40,11)$ & 8 & $(17,5)$ & 6 & 1 & YES & YES & YES & $1.50$ & $(2,3)$ & -- & 1756\\
$(40,9)$ & 9 & $(18,7)$ & 6 & 2 & YES & YES & YES & $1.62$ & $(2,3)$ & NO & 1757\\
$(40,9)$ & 9 & $(18,7)$ & 6 & 2 & YES & YES & YES & $1.62$ & $(2,3)$ & -- & 1758\\
$(40,9)$ & 9 & $(21,8)$ & 6 & 1 & YES & YES & YES & $1.70$ & $(4,2)$ & -- & 1759\\
$(40,9)$ & 9 & $(21,8)$ & 6 & 1 & YES & YES & YES & $1.82$ & $(4,2)$ & NO & 1760\\
$(40,11)$ & 8 & $(23,10)$ & 7 & 1 & YES & YES & YES & $1.50$ & $(4,2)$ & -- & 1761\\
$(40,11)$ & 8 & $(23,10)$ & 7 & 1 & YES & YES & YES & $1.75$ & $(4,2)$ & NO & 1762\\
$(40,9)$ & 9 & $(24,7)$ & 7 & 8 & YES & YES & YES & $1.70$ & $(4,2)$ & -- & 1763\\
$(40,11)$ & 8 & $(27,10)$ & 7 & 1 & YES & YES & YES & $1.80$ & $(2,3)$ & -- & 1764\\
$(40,11)$ & 8 & $(31,9)$ & 8 & 1 & YES & YES & YES & $1.70$ & $(2,3)$ & -- & 1765\\
$(40,11)$ & 8 & $(32,9)$ & 8 & 8 & YES & YES & YES & $1.70$ & $(2,3)$ & -- & 1766\\
$(41,16)$ & 8 & $(9,4)$ & 5 & 1 & YES & YES & YES & $1.44$ & $(2,3)$ & -- & 1767\\
$(41,17)$ & 8 & $(17,5)$ & 6 & 1 & YES & YES & YES & $1.73$ & $(4,2)$ & -- & 1768\\
$(41,16)$ & 8 & $(21,8)$ & 6 & 1 & YES & YES & YES & $1.67$ & $(4,2)$ & -- & 1769\\
$(41,17)$ & 8 & $(22,5)$ & 7 & 1 & YES & YES & YES & $1.75$ & $(2,3)$ & NO & 1770\\
$(41,12)$ & 8 & $(23,9)$ & 7 & 1 & YES & YES & YES & $1.67$ & $(4,2)$ & -- & 1771\\
$(41,12)$ & 8 & $(29,12)$ & 7 & 1 & YES & YES & YES & $1.67$ & $(4,2)$ & -- & 1772\\
$(41,17)$ & 8 & $(29,8)$ & 7 & 1 & YES & YES & YES & $1.67$ & $(4,2)$ & NO & 1773\\
$(41,17)$ & 8 & $(29,8)$ & 7 & 1 & YES & YES & YES & $1.67$ & $(4,2)$ & -- & 1774\\
$(41,12)$ & 8 & $(31,12)$ & 7 & 1 & YES & YES & YES & $1.67$ & $(4,2)$ & -- & 1775\\
$(41,17)$ & 8 & $(31,7)$ & 8 & 1 & YES & YES & YES & $1.43$ & $(4,2)$ & NO & 1776\\
$(43,18)$ & 8 & $(15,4)$ & 6 & 1 & YES & YES & YES & $1.62$ & $(6,1)$ & -- & 1777\\
$(43,18)$ & 8 & $(17,4)$ & 7 & 1 & YES & YES & YES & $1.62$ & $(6,1)$ & -- & 1778\\
$(43,12)$ & 8 & $(18,5)$ & 6 & 1 & YES & YES & YES & $1.62$ & $(2,3)$ & NO & 1779\\
$(43,12)$ & 8 & $(18,5)$ & 6 & 1 & YES & YES & YES & $1.62$ & $(2,3)$ & -- & 1780\\
$(43,12)$ & 8 & $(21,8)$ & 6 & 1 & YES & YES & YES & $1.43$ & $(4,2)$ & NO & 1781\\
$(43,12)$ & 8 & $(21,8)$ & 6 & 1 & YES & YES & YES & $1.56$ & $(6,1)$ & -- & 1782\\
$(43,13)$ & 9 & $(21,8)$ & 6 & 1 & YES & YES & YES & $1.62$ & $(4,2)$ & -- & 1783\\
$(43,16)$ & 9 & $(25,9)$ & 7 & 1 & YES & YES & YES & $1.50$ & $(2,3)$ & NO & 1784\\
$(43,10)$ & 9 & $(26,11)$ & 7 & 1 & YES & YES & YES & $1.67$ & $(2,3)$ & NO & 1785\\
$(43,13)$ & 9 & $(37,8)$ & 8 & 1 & YES & YES & YES & $1.62$ & $(4,2)$ & NO & 1786\\
$(44,13)$ & 8 & $(13,5)$ & 5 & 1 & YES & YES & YES & $1.56$ & $(6,1)$ & NO & 1787\\
$(44,13)$ & 8 & $(13,5)$ & 5 & 1 & YES & YES & YES & $1.56$ & $(6,1)$ & -- & 1788\\
$(44,17)$ & 8 & $(19,5)$ & 7 & 1 & YES & YES & YES & $1.57$ & $(4,2)$ & NO & 1789\\
$(44,17)$ & 8 & $(19,5)$ & 7 & 1 & YES & YES & YES & $1.57$ & $(4,2)$ & -- & 1790\\
$(44,17)$ & 8 & $(21,5)$ & 8 & 1 & YES & YES & YES & $1.57$ & $(4,2)$ & NO & 1791\\
$(44,17)$ & 8 & $(21,5)$ & 8 & 1 & YES & YES & YES & $1.57$ & $(4,2)$ & -- & 1792\\
$(44,13)$ & 8 & $(24,7)$ & 7 & 4 & YES & YES & YES & $1.70$ & $(2,3)$ & -- & 1793\\
$(44,13)$ & 8 & $(43,10)$ & 9 & 1 & YES & YES & YES & $1.43$ & $(4,2)$ & NO & 1794\\
$(45,17)$ & 9 & $(6,1)$ & 5 & 3 & YES & YES & YES & $1.57$ & $(2,3)$ & -- & 1795\\
$(45,17)$ & 9 & $(7,3)$ & 4 & 1 & YES & YES & YES & $1.62$ & $(2,3)$ & NO & 1796\\
$(45,17)$ & 9 & $(12,5)$ & 5 & 3 & YES & YES & YES & $1.50$ & $(2,3)$ & -- & 1797\\
$(45,19)$ & 8 & $(29,11)$ & 7 & 1 & YES & YES & YES & $1.75$ & $(2,3)$ & NO & 1798\\
$(46,17)$ & 8 & $(17,7)$ & 6 & 1 & YES & YES & YES & $1.67$ & $(4,2)$ & -- & 1799\\
$(46,19)$ & 8 & $(24,7)$ & 7 & 2 & YES & YES & YES & $1.67$ & $(4,2)$ & -- & 1800\\
$(46,17)$ & 8 & $(26,11)$ & 7 & 2 & YES & YES & YES & $1.73$ & $(2,3)$ & NO & 1801\\
$(46,17)$ & 8 & $(31,13)$ & 7 & 1 & YES & YES & YES & $1.67$ & $(4,2)$ & NO & 1802\\
$(46,17)$ & 8 & $(44,17)$ & 8 & 2 & YES & YES & YES & $1.73$ & $(2,3)$ & NO & 1803\\
$(47,13)$ & 8 & $(17,4)$ & 7 & 1 & YES & YES & YES & $1.57$ & $(2,3)$ & -- & 1804\\
$(47,13)$ & 8 & $(17,7)$ & 6 & 1 & YES & YES & YES & $1.75$ & $(2,3)$ & NO & 1805\\
$(47,18)$ & 8 & $(17,5)$ & 6 & 1 & YES & YES & YES & $1.60$ & $(2,3)$ & -- & 1806\\
$(47,18)$ & 8 & $(18,5)$ & 6 & 1 & YES & YES & YES & $1.60$ & $(2,3)$ & -- & 1807\\
$(47,18)$ & 8 & $(18,7)$ & 6 & 1 & YES & YES & YES & $1.70$ & $(2,3)$ & -- & 1808\\
$(47,13)$ & 8 & $(21,8)$ & 6 & 1 & YES & YES & YES & $1.60$ & $(2,3)$ & -- & 1809\\
$(47,13)$ & 8 & $(23,9)$ & 7 & 1 & YES & YES & YES & $1.67$ & $(4,2)$ & -- & 1810\\
$(47,13)$ & 8 & $(23,9)$ & 7 & 1 & YES & YES & YES & $1.67$ & $(4,2)$ & NO & 1811\\
$(47,10)$ & 9 & $(31,9)$ & 8 & 1 & YES & YES & YES & $1.83$ & $(2,3)$ & NO & 1812\\
$(47,14)$ & 9 & $(38,7)$ & 9 & 1 & YES & YES & YES & $1.43$ & $(4,2)$ & NO & 1813\\
$(48,13)$ & 9 & $(11,3)$ & 5 & 1 & YES & YES & YES & $1.82$ & $(2,3)$ & -- & 1814\\
$(48,11)$ & 9 & $(27,10)$ & 7 & 3 & YES & YES & YES & $1.50$ & $(4,2)$ & -- & 1815\\
$(48,13)$ & 9 & $(32,7)$ & 8 & 16 & YES & YES & YES & $1.56$ & $(4,2)$ & -- & 1816\\
$(48,11)$ & 9 & $(41,11)$ & 8 & 1 & YES & YES & YES & $1.44$ & $(4,2)$ & -- & 1817\\
$(49,15)$ & 9 & $(5,2)$ & 3 & 1 & YES & YES & NO(2) & $1.55$ & $(2,3)$ & -- & 1818\\
$(49,15)$ & 9 & $(7,2)$ & 4 & 7 & YES & YES & YES & $1.50$ & $(2,3)$ & NO & 1819\\
$(49,15)$ & 9 & $(7,2)$ & 4 & 7 & YES & YES & YES & $1.50$ & $(2,3)$ & -- & 1820\\
$(49,15)$ & 9 & $(13,5)$ & 5 & 1 & YES & YES & YES & $1.38$ & $(4,2)$ & -- & 1821\\
$(49,18)$ & 8 & $(13,5)$ & 5 & 1 & YES & YES & YES & $1.43$ & $(2,3)$ & -- & 1822\\
$(49,19)$ & 8 & $(16,5)$ & 7 & 1 & YES & YES & YES & $1.57$ & $(2,3)$ & NO & 1823\\
$(49,19)$ & 8 & $(16,5)$ & 7 & 1 & YES & YES & YES & $1.57$ & $(2,3)$ & -- & 1824\\
$(49,19)$ & 8 & $(17,6)$ & 7 & 1 & YES & YES & YES & $1.43$ & $(4,2)$ & -- & 1825\\
$(49,19)$ & 8 & $(18,7)$ & 6 & 1 & YES & YES & YES & $1.67$ & $(4,2)$ & -- & 1826\\
$(49,19)$ & 8 & $(23,7)$ & 7 & 1 & YES & YES & YES & $1.50$ & $(4,2)$ & -- & 1827\\
$(49,19)$ & 8 & $(25,7)$ & 7 & 1 & YES & YES & YES & $1.56$ & $(4,2)$ & -- & 1828\\
$(49,18)$ & 8 & $(31,12)$ & 7 & 1 & YES & YES & YES & $1.43$ & $(2,3)$ & NO & 1829\\
$(49,13)$ & 9 & $(37,11)$ & 8 & 1 & YES & YES & YES & $1.43$ & $(4,2)$ & 2076 & 1830\\
$(49,15)$ & 9 & $(41,12)$ & 8 & 1 & YES & YES & YES & $1.38$ & $(4,2)$ & NO & 1831\\
$(49,13)$ & 9 & $(44,13)$ & 8 & 1 & YES & YES & YES & $1.43$ & $(4,2)$ & NO & 1832\\
$(50,21)$ & 8 & $(2,1)$ & 1 & 2 & YES & YES & NO(2) & $1.45$ & $(2,3)$ & -- & 1833\\
$(50,19)$ & 8 & $(12,5)$ & 5 & 2 & YES & YES & YES & $1.60$ & $(4,2)$ & -- & 1834\\
$(50,19)$ & 8 & $(17,6)$ & 7 & 1 & YES & YES & YES & $1.57$ & $(4,2)$ & -- & 1835\\
$(50,19)$ & 8 & $(18,7)$ & 6 & 2 & YES & YES & YES & $1.73$ & $(2,3)$ & -- & 1836\\
$(50,21)$ & 8 & $(18,7)$ & 6 & 2 & YES & YES & YES & $1.80$ & $(2,3)$ & -- & 1837\\
$(50,19)$ & 8 & $(24,7)$ & 7 & 2 & YES & YES & YES & $1.60$ & $(2,3)$ & -- & 1838\\
$(51,16)$ & 10 & $(5,1)$ & 4 & 1 & YES & YES & YES & $1.43$ & $(2,3)$ & NO & 1839\\
$(51,16)$ & 10 & $(5,1)$ & 4 & 1 & YES & YES & YES & $1.43$ & $(2,3)$ & -- & 1840\\
$(51,16)$ & 10 & $(5,1)$ & 4 & 1 & YES & YES & YES & $1.43$ & $(2,3)$ & NO & 1841\\
$(51,11)$ & 9 & $(22,9)$ & 7 & 1 & YES & YES & YES & $1.92$ & $(2,3)$ & -- & 1842\\
$(51,11)$ & 9 & $(23,9)$ & 7 & 1 & YES & YES & YES & $1.62$ & $(4,2)$ & NO & 1843\\
$(51,11)$ & 9 & $(23,9)$ & 7 & 1 & YES & YES & YES & $1.62$ & $(4,2)$ & -- & 1844\\
$(52,19)$ & 9 & $(16,3)$ & 7 & 4 & YES & YES & YES & $1.29$ & $(2,3)$ & -- & 1845\\
$(53,16)$ & 10 & $(9,4)$ & 5 & 1 & YES & YES & YES & $1.43$ & $(2,3)$ & -- & 1846\\
$(53,19)$ & 9 & $(18,7)$ & 6 & 1 & YES & YES & YES & $1.75$ & $(2,3)$ & -- & 1847\\
$(53,12)$ & 9 & $(21,8)$ & 6 & 1 & YES & YES & YES & $1.56$ & $(2,3)$ & NO & 1848\\
$(53,20)$ & 10 & $(49,19)$ & 8 & 1 & YES & YES & YES & $1.43$ & $(4,2)$ & NO & 1849\\
$(55,23)$ & 9 & $(9,4)$ & 5 & 1 & YES & YES & YES & $1.43$ & $(2,3)$ & -- & 1850\\
$(55,21)$ & 8 & $(10,3)$ & 5 & 5 & YES & YES & YES & $1.38$ & $(6,1)$ & -- & 1851\\
$(55,21)$ & 8 & $(11,3)$ & 5 & 11 & YES & YES & YES & $1.38$ & $(6,1)$ & -- & 1852\\
$(55,21)$ & 8 & $(13,5)$ & 5 & 1 & YES & YES & YES & $1.50$ & $(4,2)$ & -- & 1853\\
$(55,21)$ & 8 & $(17,7)$ & 6 & 1 & YES & YES & YES & $1.67$ & $(4,2)$ & -- & 1854\\
$(55,16)$ & 9 & $(18,7)$ & 6 & 1 & YES & YES & YES & $1.73$ & $(2,3)$ & -- & 1855\\
$(55,21)$ & 8 & $(18,5)$ & 6 & 1 & YES & YES & YES & $1.60$ & $(2,3)$ & -- & 1856\\
$(55,21)$ & 8 & $(18,7)$ & 6 & 1 & YES & YES & YES & $1.73$ & $(2,3)$ & -- & 1857\\
$(55,23)$ & 9 & $(18,7)$ & 6 & 1 & YES & YES & YES & $1.43$ & $(2,3)$ & NO & 1858\\
$(55,24)$ & 9 & $(18,7)$ & 6 & 1 & YES & YES & YES & $1.62$ & $(2,3)$ & -- & 1859\\
$(55,13)$ & 10 & $(21,8)$ & 6 & 1 & YES & YES & YES & $1.62$ & $(4,2)$ & -- & 1860\\
$(55,21)$ & 8 & $(25,7)$ & 7 & 5 & YES & YES & YES & $1.29$ & $(4,2)$ & NO & 1861\\
$(56,13)$ & 10 & $(18,7)$ & 6 & 2 & YES & YES & YES & $1.73$ & $(2,3)$ & NO & 1862\\
$(56,17)$ & 9 & $(29,8)$ & 7 & 1 & YES & YES & YES & $1.43$ & $(2,3)$ & NO & 1863\\
$(56,13)$ & 10 & $(51,11)$ & 9 & 1 & YES & YES & YES & $1.82$ & $(2,3)$ & NO & 1864\\
$(57,16)$ & 9 & $(19,7)$ & 6 & 19 & YES & YES & YES & $1.50$ & $(4,2)$ & -- & 1865\\
$(57,22)$ & 9 & $(23,5)$ & 7 & 1 & YES & YES & YES & $1.44$ & $(4,2)$ & -- & 1866\\
$(57,13)$ & 9 & $(30,11)$ & 7 & 3 & YES & YES & YES & $1.56$ & $(4,2)$ & -- & 1867\\
$(58,17)$ & 9 & $(11,3)$ & 5 & 1 & YES & YES & YES & $1.29$ & $(8,0)$ & -- & 1868\\
$(58,17)$ & 9 & $(13,3)$ & 6 & 1 & YES & YES & YES & $1.43$ & $(8,0)$ & -- & 1869\\
$(58,17)$ & 9 & $(17,7)$ & 6 & 1 & YES & YES & YES & $1.43$ & $(4,2)$ & -- & 1870\\
$(58,17)$ & 9 & $(19,7)$ & 6 & 1 & YES & YES & YES & $1.70$ & $(2,3)$ & -- & 1871\\
$(58,21)$ & 10 & $(39,14)$ & 8 & 1 & YES & YES & YES & $1.50$ & $(2,3)$ & NO & 1872\\
$(58,17)$ & 9 & $(40,11)$ & 8 & 2 & YES & YES & YES & $1.62$ & $(4,2)$ & NO & 1873\\
$(59,23)$ & 9 & $(12,5)$ & 5 & 1 & YES & YES & YES & $1.43$ & $(2,3)$ & -- & 1874\\
$(59,11)$ & 10 & $(32,9)$ & 8 & 1 & YES & YES & YES & $1.56$ & $(2,3)$ & NO & 1875\\
$(59,18)$ & 9 & $(40,11)$ & 8 & 1 & YES & YES & YES & $1.67$ & $(2,3)$ & NO & 1876\\
$(59,25)$ & 9 & $(55,23)$ & 9 & 1 & YES & YES & YES & $1.57$ & $(2,3)$ & NO & 1877\\
$(60,23)$ & 9 & $(4,1)$ & 3 & 4 & YES & YES & YES & $1.60$ & $(2,3)$ & -- & 1878\\
$(60,23)$ & 9 & $(10,3)$ & 5 & 10 & YES & YES & YES & $1.75$ & $(2,3)$ & -- & 1879\\
$(60,23)$ & 9 & $(13,5)$ & 5 & 1 & YES & YES & YES & $1.78$ & $(2,3)$ & -- & 1880\\
$(60,11)$ & 11 & $(14,5)$ & 6 & 2 & YES & YES & YES & $1.50$ & $(2,3)$ & -- & 1881\\
$(60,23)$ & 9 & $(18,5)$ & 6 & 6 & YES & YES & YES & $1.70$ & $(2,3)$ & -- & 1882\\
$(60,13)$ & 9 & $(23,9)$ & 7 & 1 & YES & YES & YES & $1.50$ & $(4,2)$ & NO & 1883\\
$(60,23)$ & 9 & $(27,5)$ & 8 & 3 & YES & YES & YES & $1.80$ & $(2,3)$ & NO & 1884\\
$(60,13)$ & 9 & $(31,9)$ & 8 & 1 & YES & YES & YES & $1.50$ & $(4,2)$ & NO & 1885\\
$(61,25)$ & 9 & $(2,1)$ & 1 & 1 & YES & YES & NO(2) & $1.50$ & $(2,3)$ & -- & 1886\\
$(61,25)$ & 9 & $(3,1)$ & 2 & 1 & YES & YES & NO(2) & $1.50$ & $(2,3)$ & NO & 1887\\
$(61,25)$ & 9 & $(3,1)$ & 2 & 1 & YES & YES & NO(2) & $1.50$ & $(2,3)$ & -- & 1888\\
$(61,25)$ & 9 & $(5,1)$ & 4 & 1 & YES & YES & NO(2) & $1.40$ & $(2,3)$ & -- & 1889\\
$(61,25)$ & 9 & $(7,3)$ & 4 & 1 & YES & YES & YES & $1.29$ & $(2,3)$ & -- & 1890\\
$(61,17)$ & 9 & $(9,4)$ & 5 & 1 & YES & YES & NO(2) & $1.73$ & $(2,3)$ & -- & 1891\\
$(61,18)$ & 9 & $(10,3)$ & 5 & 1 & YES & YES & YES & $1.50$ & $(6,1)$ & -- & 1892\\
$(61,18)$ & 9 & $(10,3)$ & 5 & 1 & YES & YES & YES & $1.50$ & $(6,1)$ & NO & 1893\\
$(61,22)$ & 9 & $(10,3)$ & 5 & 1 & YES & YES & YES & $1.75$ & $(2,3)$ & -- & 1894\\
$(61,25)$ & 9 & $(10,3)$ & 5 & 1 & YES & YES & YES & $1.29$ & $(4,2)$ & -- & 1895\\
$(61,25)$ & 9 & $(10,3)$ & 5 & 1 & YES & YES & YES & $1.29$ & $(2,3)$ & NO & 1896\\
$(61,17)$ & 9 & $(12,5)$ & 5 & 1 & YES & YES & YES & $1.64$ & $(4,2)$ & -- & 1897\\
$(61,17)$ & 9 & $(13,4)$ & 6 & 1 & YES & YES & YES & $1.73$ & $(4,2)$ & -- & 1898\\
$(61,18)$ & 9 & $(13,5)$ & 5 & 1 & YES & YES & YES & $1.73$ & $(2,3)$ & -- & 1899\\
$(61,25)$ & 9 & $(13,4)$ & 6 & 1 & YES & YES & YES & $1.57$ & $(2,3)$ & -- & 1900\\
$(61,17)$ & 9 & $(17,7)$ & 6 & 1 & YES & YES & YES & $1.29$ & $(4,2)$ & -- & 1901\\
$(61,18)$ & 9 & $(17,5)$ & 6 & 1 & YES & YES & YES & $1.70$ & $(2,3)$ & -- & 1902\\
$(61,17)$ & 9 & $(19,8)$ & 6 & 1 & YES & YES & YES & $1.62$ & $(4,2)$ & -- & 1903\\
$(61,17)$ & 9 & $(21,8)$ & 6 & 1 & YES & YES & YES & $1.50$ & $(4,2)$ & -- & 1904\\
$(61,25)$ & 9 & $(22,9)$ & 7 & 1 & YES & YES & NO(2) & $1.50$ & $(2,3)$ & NO & 1905\\
$(61,18)$ & 9 & $(33,7)$ & 8 & 1 & YES & YES & YES & $1.50$ & $(4,2)$ & -- & 1906\\
$(61,17)$ & 9 & $(37,11)$ & 8 & 1 & YES & YES & YES & $1.43$ & $(4,2)$ & NO & 1907\\
$(61,14)$ & 10 & $(47,10)$ & 9 & 1 & YES & YES & YES & $1.83$ & $(2,3)$ & NO & 1908\\
$(61,14)$ & 10 & $(51,11)$ & 9 & 1 & YES & YES & YES & $1.83$ & $(2,3)$ & NO & 1909\\
$(62,27)$ & 9 & $(15,4)$ & 6 & 1 & YES & YES & YES & $1.43$ & $(4,2)$ & -- & 1910\\
$(63,26)$ & 9 & $(10,3)$ & 5 & 1 & YES & YES & YES & $1.43$ & $(2,3)$ & -- & 1911\\
$(64,25)$ & 9 & $(2,1)$ & 1 & 2 & YES & YES & NO(2) & $1.50$ & $(2,3)$ & -- & 1912\\
$(64,27)$ & 9 & $(2,1)$ & 1 & 2 & YES & YES & NO(2) & $1.45$ & $(2,3)$ & -- & 1913\\
$(64,25)$ & 9 & $(3,1)$ & 2 & 1 & YES & YES & NO(2) & $1.50$ & $(2,3)$ & NO & 1914\\
$(64,25)$ & 9 & $(3,1)$ & 2 & 1 & YES & YES & NO(2) & $1.50$ & $(2,3)$ & -- & 1915\\
$(64,25)$ & 9 & $(5,1)$ & 4 & 1 & YES & YES & NO(2) & $1.40$ & $(2,3)$ & -- & 1916\\
$(64,23)$ & 9 & $(10,3)$ & 5 & 2 & YES & YES & YES & $1.57$ & $(2,3)$ & -- & 1917\\
$(64,19)$ & 9 & $(18,7)$ & 6 & 2 & YES & YES & YES & $1.80$ & $(2,3)$ & -- & 1918\\
$(64,23)$ & 9 & $(18,7)$ & 6 & 2 & YES & YES & YES & $1.57$ & $(2,3)$ & 1752 & 1919\\
$(64,27)$ & 9 & $(18,5)$ & 6 & 2 & YES & YES & YES & $1.67$ & $(4,2)$ & -- & 1920\\
$(64,19)$ & 9 & $(23,7)$ & 7 & 1 & YES & YES & YES & $1.70$ & $(2,3)$ & -- & 1921\\
$(64,19)$ & 9 & $(24,7)$ & 7 & 8 & YES & YES & YES & $1.60$ & $(2,3)$ & -- & 1922\\
$(64,25)$ & 9 & $(34,13)$ & 7 & 2 & YES & YES & YES & $1.43$ & $(4,2)$ & NO & 1923\\
$(65,19)$ & 9 & $(10,3)$ & 5 & 5 & YES & YES & YES & $1.50$ & $(6,1)$ & -- & 1924\\
$(65,19)$ & 9 & $(11,4)$ & 5 & 1 & YES & YES & YES & $1.73$ & $(2,3)$ & -- & 1925\\
$(65,19)$ & 9 & $(13,3)$ & 6 & 13 & YES & YES & YES & $1.43$ & $(8,0)$ & -- & 1926\\
$(65,24)$ & 9 & $(13,5)$ & 5 & 13 & YES & YES & YES & $1.70$ & $(2,3)$ & -- & 1927\\
$(65,18)$ & 9 & $(17,7)$ & 6 & 1 & YES & YES & YES & $1.43$ & $(4,2)$ & NO & 1928\\
$(65,18)$ & 9 & $(18,7)$ & 6 & 1 & YES & YES & YES & $1.43$ & $(4,2)$ & NO & 1929\\
$(65,19)$ & 9 & $(18,7)$ & 6 & 1 & YES & YES & YES & $1.73$ & $(2,3)$ & -- & 1930\\
$(65,18)$ & 9 & $(21,8)$ & 6 & 1 & YES & YES & YES & $1.67$ & $(2,3)$ & NO & 1931\\
$(65,14)$ & 10 & $(31,7)$ & 8 & 1 & YES & YES & YES & $1.38$ & $(2,3)$ & NO & 1932\\
$(65,24)$ & 9 & $(53,19)$ & 9 & 1 & YES & YES & YES & $1.75$ & $(2,3)$ & NO & 1933\\
$(66,25)$ & 9 & $(10,3)$ & 5 & 2 & YES & YES & YES & $1.50$ & $(2,3)$ & NO & 1934\\
$(66,25)$ & 9 & $(10,3)$ & 5 & 2 & YES & YES & YES & $1.50$ & $(2,3)$ & -- & 1935\\
$(66,25)$ & 9 & $(13,5)$ & 5 & 1 & YES & YES & YES & $1.78$ & $(4,2)$ & -- & 1936\\
$(66,25)$ & 9 & $(22,5)$ & 7 & 22 & YES & YES & YES & $1.56$ & $(4,2)$ & -- & 1937\\
$(67,28)$ & 10 & $(6,1)$ & 5 & 1 & YES & YES & YES & $1.38$ & $(2,3)$ & NO & 1938\\
$(67,28)$ & 10 & $(6,1)$ & 5 & 1 & YES & YES & YES & $1.38$ & $(2,3)$ & -- & 1939\\
$(67,28)$ & 10 & $(7,3)$ & 4 & 1 & YES & YES & YES & $1.50$ & $(2,3)$ & -- & 1940\\
$(67,28)$ & 10 & $(13,5)$ & 5 & 1 & YES & YES & YES & $1.50$ & $(2,3)$ & NO & 1941\\
$(67,26)$ & 9 & $(30,11)$ & 7 & 1 & YES & YES & YES & $1.70$ & $(2,3)$ & NO & 1942\\
$(67,26)$ & 9 & $(50,19)$ & 8 & 1 & YES & YES & YES & $1.70$ & $(2,3)$ & NO & 1943\\
$(68,19)$ & 9 & $(10,3)$ & 5 & 2 & YES & YES & YES & $1.62$ & $(2,3)$ & -- & 1944\\
$(68,25)$ & 9 & $(11,3)$ & 5 & 1 & YES & YES & YES & $1.73$ & $(4,2)$ & -- & 1945\\
$(68,19)$ & 9 & $(17,7)$ & 6 & 17 & YES & YES & YES & $1.80$ & $(2,3)$ & -- & 1946\\
$(69,29)$ & 9 & $(23,5)$ & 7 & 23 & YES & YES & YES & $1.70$ & $(2,3)$ & -- & 1947\\
$(69,19)$ & 9 & $(24,7)$ & 7 & 3 & YES & YES & YES & $1.60$ & $(2,3)$ & -- & 1948\\
$(69,13)$ & 11 & $(60,11)$ & 11 & 3 & YES & YES & YES & $1.50$ & $(2,3)$ & NO & 1949\\
$(70,29)$ & 9 & $(13,4)$ & 6 & 1 & YES & YES & YES & $1.50$ & $(4,2)$ & -- & 1950\\
$(70,29)$ & 9 & $(13,5)$ & 5 & 1 & YES & YES & YES & $1.78$ & $(4,2)$ & -- & 1951\\
$(70,29)$ & 9 & $(15,4)$ & 6 & 5 & YES & YES & YES & $1.75$ & $(4,2)$ & -- & 1952\\
$(70,29)$ & 9 & $(17,5)$ & 6 & 1 & YES & YES & YES & $1.78$ & $(4,2)$ & -- & 1953\\
$(71,21)$ & 9 & $(2,1)$ & 1 & 1 & YES & YES & NO(2) & $1.40$ & $(4,2)$ & NO & 1954\\
$(71,26)$ & 9 & $(4,1)$ & 3 & 1 & YES & YES & NO(2) & $1.22$ & $(4,2)$ & -- & 1955\\
$(71,30)$ & 9 & $(5,1)$ & 4 & 1 & YES & YES & NO(3) & $1.30$ & $(2,3)$ & NO & 1956\\
$(71,21)$ & 9 & $(10,3)$ & 5 & 1 & YES & YES & NO(2) & $1.40$ & $(4,2)$ & NO & 1957\\
$(71,22)$ & 10 & $(10,3)$ & 5 & 1 & YES & YES & YES & $1.57$ & $(2,3)$ & -- & 1958\\
$(71,27)$ & 9 & $(10,3)$ & 5 & 1 & YES & YES & YES & $1.75$ & $(2,3)$ & NO & 1959\\
$(71,27)$ & 9 & $(10,3)$ & 5 & 1 & YES & YES & YES & $1.75$ & $(2,3)$ & -- & 1960\\
$(71,21)$ & 9 & $(13,5)$ & 5 & 1 & YES & YES & YES & $1.70$ & $(2,3)$ & -- & 1961\\
$(71,27)$ & 9 & $(13,5)$ & 5 & 1 & YES & YES & YES & $1.70$ & $(2,3)$ & -- & 1962\\
$(71,30)$ & 9 & $(14,5)$ & 6 & 1 & YES & YES & YES & $1.57$ & $(2,3)$ & NO & 1963\\
$(71,30)$ & 9 & $(17,5)$ & 6 & 1 & YES & YES & YES & $1.56$ & $(4,2)$ & -- & 1964\\
$(71,27)$ & 9 & $(18,5)$ & 6 & 1 & YES & YES & YES & $1.70$ & $(2,3)$ & -- & 1965\\
$(71,27)$ & 9 & $(23,10)$ & 7 & 1 & YES & YES & YES & $1.62$ & $(2,3)$ & NO & 1966\\
$(71,19)$ & 10 & $(31,9)$ & 8 & 1 & YES & YES & YES & $1.29$ & $(6,1)$ & NO & 1967\\
$(71,26)$ & 9 & $(41,15)$ & 8 & 1 & YES & YES & NO(2) & $1.33$ & $(4,2)$ & NO & 1968\\
$(73,27)$ & 9 & $(19,8)$ & 6 & 1 & YES & YES & YES & $1.50$ & $(4,2)$ & NO & 1969\\
$(73,27)$ & 9 & $(22,5)$ & 7 & 1 & YES & YES & YES & $1.38$ & $(4,2)$ & NO & 1970\\
$(73,26)$ & 11 & $(59,21)$ & 10 & 1 & YES & YES & YES & $1.29$ & $(4,2)$ & NO & 1971\\
$(74,29)$ & 10 & $(4,1)$ & 3 & 2 & YES & YES & YES & $1.29$ & $(4,2)$ & NO & 1972\\
$(74,29)$ & 10 & $(4,1)$ & 3 & 2 & YES & YES & YES & $1.29$ & $(4,2)$ & -- & 1973\\
$(74,31)$ & 9 & $(13,5)$ & 5 & 1 & YES & YES & YES & $1.70$ & $(2,3)$ & -- & 1974\\
$(74,31)$ & 9 & $(17,4)$ & 7 & 1 & YES & YES & YES & $1.57$ & $(2,3)$ & NO & 1975\\
$(75,22)$ & 10 & $(7,3)$ & 4 & 1 & YES & YES & YES & $1.64$ & $(2,3)$ & -- & 1976\\
$(75,22)$ & 10 & $(11,3)$ & 5 & 1 & YES & YES & YES & $1.83$ & $(2,3)$ & -- & 1977\\
$(75,29)$ & 9 & $(13,5)$ & 5 & 1 & YES & YES & YES & $1.56$ & $(4,2)$ & -- & 1978\\
$(75,29)$ & 9 & $(14,5)$ & 6 & 1 & YES & YES & YES & $1.62$ & $(2,3)$ & -- & 1979\\
$(75,17)$ & 10 & $(17,7)$ & 6 & 1 & YES & YES & YES & $1.43$ & $(4,2)$ & NO & 1980\\
$(75,29)$ & 9 & $(18,5)$ & 6 & 3 & YES & YES & YES & $1.70$ & $(2,3)$ & -- & 1981\\
$(75,22)$ & 10 & $(19,4)$ & 7 & 1 & YES & YES & YES & $1.83$ & $(2,3)$ & NO & 1982\\
$(75,22)$ & 10 & $(27,5)$ & 8 & 3 & YES & YES & YES & $1.50$ & $(4,2)$ & NO & 1983\\
$(75,22)$ & 10 & $(27,5)$ & 8 & 3 & YES & YES & YES & $1.50$ & $(4,2)$ & -- & 1984\\
$(76,29)$ & 9 & $(7,2)$ & 4 & 1 & YES & YES & YES & $1.50$ & $(6,1)$ & NO & 1985\\
$(76,29)$ & 9 & $(7,2)$ & 4 & 1 & YES & YES & YES & $1.50$ & $(6,1)$ & -- & 1986\\
$(76,21)$ & 9 & $(8,3)$ & 4 & 4 & YES & YES & YES & $1.62$ & $(2,3)$ & -- & 1987\\
$(76,21)$ & 9 & $(11,4)$ & 5 & 1 & YES & YES & YES & $1.50$ & $(4,2)$ & NO & 1988\\
$(76,21)$ & 9 & $(11,4)$ & 5 & 1 & YES & YES & YES & $1.50$ & $(4,2)$ & -- & 1989\\
$(76,21)$ & 9 & $(13,3)$ & 6 & 1 & YES & YES & YES & $1.62$ & $(2,3)$ & NO & 1990\\
$(76,21)$ & 9 & $(13,3)$ & 6 & 1 & YES & YES & YES & $1.62$ & $(2,3)$ & -- & 1991\\
$(76,29)$ & 9 & $(41,16)$ & 8 & 1 & YES & YES & YES & $1.43$ & $(4,2)$ & NO & 1992\\
$(76,29)$ & 9 & $(60,23)$ & 9 & 4 & YES & YES & YES & $1.75$ & $(2,3)$ & NO & 1993\\
$(78,23)$ & 10 & $(4,1)$ & 3 & 2 & YES & YES & YES & $1.56$ & $(4,2)$ & -- & 1994\\
$(78,29)$ & 10 & $(5,1)$ & 4 & 1 & YES & YES & YES & $1.44$ & $(2,3)$ & -- & 1995\\
$(78,23)$ & 10 & $(10,3)$ & 5 & 2 & YES & YES & YES & $1.62$ & $(2,3)$ & NO & 1996\\
$(78,29)$ & 10 & $(10,3)$ & 5 & 2 & YES & YES & YES & $1.50$ & $(6,1)$ & -- & 1997\\
$(78,29)$ & 10 & $(11,4)$ & 5 & 1 & YES & YES & YES & $1.56$ & $(2,3)$ & NO & 1998\\
$(79,29)$ & 9 & $(2,1)$ & 1 & 1 & YES & YES & NO(2) & $1.40$ & $(4,2)$ & NO & 1999\\
$(79,30)$ & 9 & $(9,4)$ & 5 & 1 & YES & YES & YES & $1.43$ & $(2,3)$ & -- & 2000\\
$(79,18)$ & 10 & $(10,3)$ & 5 & 1 & YES & YES & YES & $1.50$ & $(6,1)$ & -- & 2001\\
$(79,29)$ & 9 & $(10,3)$ & 5 & 1 & YES & YES & YES & $1.73$ & $(4,2)$ & -- & 2002\\
$(79,18)$ & 10 & $(11,4)$ & 5 & 1 & YES & YES & YES & $1.73$ & $(2,3)$ & NO & 2003\\
$(79,29)$ & 9 & $(11,4)$ & 5 & 1 & YES & YES & YES & $1.43$ & $(2,3)$ & -- & 2004\\
$(79,22)$ & 10 & $(13,5)$ & 5 & 1 & YES & YES & YES & $1.62$ & $(4,2)$ & NO & 2005\\
$(79,23)$ & 10 & $(13,5)$ & 5 & 1 & YES & YES & YES & $1.70$ & $(2,3)$ & -- & 2006\\
$(79,29)$ & 9 & $(13,4)$ & 6 & 1 & YES & YES & YES & $1.43$ & $(2,3)$ & -- & 2007\\
$(79,30)$ & 9 & $(13,3)$ & 6 & 1 & YES & YES & YES & $1.56$ & $(2,3)$ & NO & 2008\\
$(79,29)$ & 9 & $(17,7)$ & 6 & 1 & YES & YES & YES & $1.43$ & $(2,3)$ & NO & 2009\\
$(79,30)$ & 9 & $(17,5)$ & 6 & 1 & YES & YES & YES & $1.80$ & $(2,3)$ & -- & 2010\\
$(79,30)$ & 9 & $(17,7)$ & 6 & 1 & YES & YES & YES & $1.43$ & $(2,3)$ & NO & 2011\\
$(79,24)$ & 10 & $(18,5)$ & 6 & 1 & YES & YES & YES & $1.56$ & $(4,2)$ & -- & 2012\\
$(79,30)$ & 9 & $(19,8)$ & 6 & 1 & YES & YES & YES & $1.50$ & $(4,2)$ & NO & 2013\\
$(79,18)$ & 10 & $(21,8)$ & 6 & 1 & YES & YES & YES & $1.56$ & $(2,3)$ & NO & 2014\\
$(79,29)$ & 9 & $(23,9)$ & 7 & 1 & YES & YES & YES & $1.43$ & $(2,3)$ & NO & 2015\\
$(79,30)$ & 9 & $(28,11)$ & 8 & 1 & YES & YES & YES & $1.43$ & $(2,3)$ & NO & 2016\\
$(79,18)$ & 10 & $(55,13)$ & 10 & 1 & YES & YES & YES & $1.71$ & $(2,3)$ & NO & 2017\\
$(79,30)$ & 9 & $(60,23)$ & 9 & 1 & YES & YES & YES & $1.67$ & $(2,3)$ & NO & 2018\\
$(80,31)$ & 9 & $(7,2)$ & 4 & 1 & YES & YES & NO(2) & $1.44$ & $(4,2)$ & NO & 2019\\
$(80,31)$ & 9 & $(7,2)$ & 4 & 1 & YES & YES & NO(2) & $1.44$ & $(4,2)$ & -- & 2020\\
$(80,31)$ & 9 & $(8,3)$ & 4 & 8 & YES & YES & YES & $1.62$ & $(2,3)$ & -- & 2021\\
$(80,31)$ & 9 & $(19,7)$ & 6 & 1 & YES & YES & YES & $1.62$ & $(2,3)$ & NO & 2022\\
$(80,33)$ & 10 & $(70,29)$ & 9 & 10 & YES & YES & YES & $1.43$ & $(2,3)$ & 2680 & 2023\\
$(81,31)$ & 9 & $(7,3)$ & 4 & 1 & YES & YES & YES & $1.64$ & $(2,3)$ & -- & 2024\\
$(81,34)$ & 9 & $(7,3)$ & 4 & 1 & YES & YES & YES & $1.38$ & $(6,1)$ & -- & 2025\\
$(81,31)$ & 9 & $(8,3)$ & 4 & 1 & YES & YES & YES & $1.62$ & $(2,3)$ & -- & 2026\\
$(81,31)$ & 9 & $(10,3)$ & 5 & 1 & YES & YES & YES & $1.60$ & $(2,3)$ & -- & 2027\\
$(81,31)$ & 9 & $(12,5)$ & 5 & 3 & YES & YES & YES & $1.67$ & $(4,2)$ & -- & 2028\\
$(81,31)$ & 9 & $(13,3)$ & 6 & 1 & YES & YES & YES & $1.62$ & $(2,3)$ & -- & 2029\\
$(82,31)$ & 10 & $(5,2)$ & 3 & 1 & YES & YES & YES & $1.50$ & $(2,3)$ & -- & 2030\\
$(82,23)$ & 10 & $(13,5)$ & 5 & 1 & YES & YES & YES & $1.80$ & $(2,3)$ & -- & 2031\\
$(82,25)$ & 10 & $(23,5)$ & 7 & 1 & YES & YES & YES & $1.70$ & $(2,3)$ & NO & 2032\\
$(83,36)$ & 10 & $(2,1)$ & 1 & 1 & YES & YES & YES & $1.56$ & $(2,3)$ & -- & 2033\\
$(83,36)$ & 10 & $(5,1)$ & 4 & 1 & YES & YES & YES & $1.44$ & $(2,3)$ & -- & 2034\\
$(83,18)$ & 10 & $(14,5)$ & 6 & 1 & YES & YES & YES & $1.50$ & $(6,1)$ & NO & 2035\\
$(83,18)$ & 10 & $(16,5)$ & 7 & 1 & YES & YES & YES & $1.50$ & $(6,1)$ & NO & 2036\\
$(83,19)$ & 10 & $(17,7)$ & 6 & 1 & YES & YES & YES & $1.56$ & $(4,2)$ & -- & 2037\\
$(84,25)$ & 10 & $(2,1)$ & 1 & 2 & YES & YES & NO(2) & $1.44$ & $(4,2)$ & NO & 2038\\
$(84,25)$ & 10 & $(13,4)$ & 6 & 1 & YES & YES & YES & $1.38$ & $(2,3)$ & NO & 2039\\
$(84,19)$ & 10 & $(17,7)$ & 6 & 1 & YES & YES & YES & $1.62$ & $(4,2)$ & -- & 2040\\
$(84,25)$ & 10 & $(37,11)$ & 8 & 1 & YES & YES & YES & $1.44$ & $(2,3)$ & NO & 2041\\
$(85,33)$ & 10 & $(13,3)$ & 6 & 1 & YES & YES & YES & $1.67$ & $(4,2)$ & -- & 2042\\
$(86,25)$ & 10 & $(7,3)$ & 4 & 1 & YES & YES & YES & $1.43$ & $(2,3)$ & NO & 2043\\
$(86,31)$ & 10 & $(7,2)$ & 4 & 1 & YES & YES & YES & $1.75$ & $(2,3)$ & -- & 2044\\
$(86,25)$ & 10 & $(13,5)$ & 5 & 1 & YES & YES & YES & $1.70$ & $(2,3)$ & -- & 2045\\
$(89,26)$ & 10 & $(2,1)$ & 1 & 1 & YES & YES & YES & $1.50$ & $(2,3)$ & NO & 2046\\
$(89,26)$ & 10 & $(3,1)$ & 2 & 1 & YES & YES & NO(2) & $1.50$ & $(4,2)$ & NO & 2047\\
$(89,26)$ & 10 & $(3,1)$ & 2 & 1 & YES & YES & NO(2) & $1.50$ & $(4,2)$ & -- & 2048\\
$(89,26)$ & 10 & $(4,1)$ & 3 & 1 & YES & YES & YES & $1.38$ & $(2,3)$ & -- & 2049\\
$(89,25)$ & 10 & $(7,3)$ & 4 & 1 & YES & YES & YES & $1.43$ & $(2,3)$ & NO & 2050\\
$(89,34)$ & 9 & $(7,3)$ & 4 & 1 & YES & YES & YES & $1.73$ & $(2,3)$ & -- & 2051\\
$(89,39)$ & 11 & $(7,1)$ & 6 & 1 & YES & YES & YES & $1.50$ & $(2,3)$ & NO & 2052\\
$(89,39)$ & 11 & $(7,1)$ & 6 & 1 & YES & YES & YES & $1.50$ & $(2,3)$ & NO & 2053\\
$(89,25)$ & 10 & $(8,3)$ & 4 & 1 & YES & YES & YES & $1.29$ & $(2,3)$ & -- & 2054\\
$(89,25)$ & 10 & $(8,3)$ & 4 & 1 & YES & YES & YES & $1.43$ & $(2,3)$ & NO & 2055\\
$(89,26)$ & 10 & $(9,4)$ & 5 & 1 & YES & YES & YES & $1.43$ & $(2,3)$ & NO & 2056\\
$(89,32)$ & 10 & $(10,3)$ & 5 & 1 & YES & YES & YES & $1.50$ & $(6,1)$ & -- & 2057\\
$(89,34)$ & 9 & $(10,3)$ & 5 & 1 & YES & YES & YES & $1.67$ & $(4,2)$ & -- & 2058\\
$(89,34)$ & 9 & $(10,3)$ & 5 & 1 & YES & YES & YES & $1.67$ & $(4,2)$ & NO & 2059\\
$(89,34)$ & 9 & $(11,3)$ & 5 & 1 & YES & YES & YES & $1.70$ & $(2,3)$ & -- & 2060\\
$(89,34)$ & 9 & $(12,5)$ & 5 & 1 & YES & YES & YES & $1.67$ & $(4,2)$ & -- & 2061\\
$(89,24)$ & 10 & $(13,5)$ & 5 & 1 & YES & YES & YES & $1.57$ & $(4,2)$ & -- & 2062\\
$(89,24)$ & 10 & $(18,5)$ & 6 & 1 & YES & YES & YES & $1.56$ & $(4,2)$ & -- & 2063\\
$(89,24)$ & 10 & $(24,7)$ & 7 & 1 & YES & YES & YES & $1.82$ & $(2,3)$ & NO & 2064\\
$(89,34)$ & 9 & $(28,11)$ & 8 & 1 & YES & YES & YES & $1.57$ & $(2,3)$ & NO & 2065\\
$(89,34)$ & 9 & $(37,14)$ & 8 & 1 & YES & YES & YES & $1.56$ & $(2,3)$ & NO & 2066\\
$(89,25)$ & 10 & $(61,17)$ & 9 & 1 & YES & YES & YES & $1.43$ & $(2,3)$ & NO & 2067\\
$(89,26)$ & 10 & $(64,19)$ & 9 & 1 & YES & YES & YES & $1.80$ & $(2,3)$ & NO & 2068\\
$(89,34)$ & 9 & $(81,31)$ & 9 & 1 & YES & YES & YES & $1.75$ & $(2,3)$ & NO & 2069\\
$(90,37)$ & 11 & $(5,1)$ & 4 & 5 & YES & YES & YES & $1.44$ & $(2,3)$ & -- & 2070\\
$(91,27)$ & 10 & $(2,1)$ & 1 & 1 & YES & YES & NO(2) & $1.40$ & $(4,2)$ & NO & 2071\\
$(91,40)$ & 10 & $(5,2)$ & 3 & 1 & YES & YES & YES & $1.71$ & $(2,3)$ & -- & 2072\\
$(91,27)$ & 10 & $(9,4)$ & 5 & 1 & YES & YES & YES & $1.29$ & $(6,1)$ & -- & 2073\\
$(91,27)$ & 10 & $(12,5)$ & 5 & 1 & YES & YES & YES & $1.43$ & $(4,2)$ & -- & 2074\\
$(91,25)$ & 10 & $(13,5)$ & 5 & 13 & YES & YES & YES & $1.78$ & $(4,2)$ & -- & 2075\\
$(91,27)$ & 10 & $(19,5)$ & 7 & 1 & YES & YES & YES & $1.43$ & $(4,2)$ & 1830 & 2076\\
$(93,26)$ & 10 & $(2,1)$ & 1 & 1 & YES & YES & YES & $1.50$ & $(2,3)$ & NO & 2077\\
$(93,26)$ & 10 & $(6,1)$ & 5 & 3 & YES & YES & YES & $1.38$ & $(2,3)$ & NO & 2078\\
$(93,26)$ & 10 & $(6,1)$ & 5 & 3 & YES & YES & YES & $1.38$ & $(2,3)$ & -- & 2079\\
$(93,26)$ & 10 & $(7,2)$ & 4 & 1 & YES & YES & YES & $1.50$ & $(2,3)$ & NO & 2080\\
$(93,26)$ & 10 & $(25,7)$ & 7 & 1 & YES & YES & YES & $1.50$ & $(2,3)$ & NO & 2081\\
$(93,34)$ & 10 & $(41,15)$ & 8 & 1 & YES & YES & NO(2) & $1.33$ & $(4,2)$ & NO & 2082\\
$(93,26)$ & 10 & $(47,13)$ & 8 & 1 & YES & YES & YES & $1.83$ & $(2,3)$ & NO & 2083\\
$(94,39)$ & 10 & $(5,1)$ & 4 & 1 & YES & YES & YES & $1.44$ & $(2,3)$ & -- & 2084\\
$(94,39)$ & 10 & $(8,3)$ & 4 & 2 & YES & YES & YES & $1.67$ & $(4,2)$ & -- & 2085\\
$(94,39)$ & 10 & $(11,3)$ & 5 & 1 & YES & YES & YES & $1.67$ & $(4,2)$ & -- & 2086\\
$(94,39)$ & 10 & $(11,3)$ & 5 & 1 & YES & YES & YES & $1.67$ & $(4,2)$ & NO & 2087\\
$(95,39)$ & 10 & $(2,1)$ & 1 & 1 & YES & YES & YES & $1.44$ & $(2,3)$ & NO & 2088\\
$(95,37)$ & 11 & $(6,1)$ & 5 & 1 & YES & YES & YES & $1.44$ & $(2,3)$ & NO & 2089\\
$(95,36)$ & 10 & $(10,3)$ & 5 & 5 & YES & YES & YES & $1.67$ & $(4,2)$ & -- & 2090\\
$(97,41)$ & 10 & $(2,1)$ & 1 & 1 & YES & YES & YES & $1.60$ & $(2,3)$ & -- & 2091\\
$(97,22)$ & 11 & $(7,3)$ & 4 & 1 & YES & YES & YES & $1.73$ & $(2,3)$ & NO & 2092\\
$(97,36)$ & 10 & $(7,3)$ & 4 & 1 & YES & YES & YES & $1.43$ & $(2,3)$ & NO & 2093\\
$(97,22)$ & 11 & $(11,4)$ & 5 & 1 & YES & YES & YES & $1.50$ & $(4,2)$ & NO & 2094\\
$(97,22)$ & 11 & $(11,4)$ & 5 & 1 & YES & YES & YES & $1.82$ & $(2,3)$ & -- & 2095\\
$(97,37)$ & 10 & $(17,7)$ & 6 & 1 & YES & YES & YES & $1.29$ & $(6,1)$ & NO & 2096\\
$(97,37)$ & 10 & $(18,7)$ & 6 & 1 & YES & YES & YES & $1.62$ & $(2,3)$ & NO & 2097\\
$(97,41)$ & 10 & $(43,18)$ & 8 & 1 & YES & YES & YES & $1.57$ & $(2,3)$ & NO & 2098\\
$(98,29)$ & 10 & $(8,3)$ & 4 & 2 & YES & YES & YES & $1.75$ & $(2,3)$ & -- & 2099\\
$(98,27)$ & 10 & $(9,4)$ & 5 & 1 & YES & YES & YES & $1.29$ & $(6,1)$ & -- & 2100\\
$(98,27)$ & 10 & $(9,4)$ & 5 & 1 & YES & YES & YES & $1.29$ & $(6,1)$ & NO & 2101\\
$(98,27)$ & 10 & $(11,4)$ & 5 & 1 & YES & YES & YES & $1.50$ & $(4,2)$ & -- & 2102\\
$(98,27)$ & 10 & $(24,7)$ & 7 & 2 & YES & YES & YES & $1.29$ & $(6,1)$ & NO & 2103\\
$(98,27)$ & 10 & $(39,11)$ & 9 & 1 & YES & YES & YES & $1.57$ & $(2,3)$ & NO & 2104\\
$(98,27)$ & 10 & $(47,13)$ & 8 & 1 & YES & YES & YES & $1.57$ & $(2,3)$ & NO & 2105\\
$(99,41)$ & 10 & $(7,3)$ & 4 & 1 & YES & YES & YES & $1.75$ & $(2,3)$ & -- & 2106\\
$(99,29)$ & 10 & $(8,3)$ & 4 & 1 & YES & YES & YES & $1.56$ & $(2,3)$ & -- & 2107\\
$(99,41)$ & 10 & $(8,3)$ & 4 & 1 & YES & YES & YES & $1.70$ & $(2,3)$ & -- & 2108\\
$(99,29)$ & 10 & $(10,3)$ & 5 & 1 & YES & YES & YES & $1.60$ & $(2,3)$ & -- & 2109\\
$(99,41)$ & 10 & $(11,3)$ & 5 & 11 & YES & YES & YES & $1.62$ & $(4,2)$ & -- & 2110\\
$(99,41)$ & 10 & $(11,3)$ & 5 & 11 & YES & YES & YES & $1.70$ & $(2,3)$ & NO & 2111\\
$(99,29)$ & 10 & $(13,4)$ & 6 & 1 & YES & YES & YES & $1.56$ & $(4,2)$ & -- & 2112\\
$(99,29)$ & 10 & $(89,26)$ & 10 & 1 & YES & YES & YES & $1.56$ & $(2,3)$ & NO & 2113\\
$(100,29)$ & 11 & $(7,3)$ & 4 & 1 & YES & YES & YES & $1.43$ & $(2,3)$ & NO & 2114\\
$(100,29)$ & 11 & $(8,3)$ & 4 & 4 & YES & YES & YES & $1.43$ & $(2,3)$ & 2389 & 2115\\
$(100,39)$ & 10 & $(10,3)$ & 5 & 10 & YES & YES & YES & $1.44$ & $(4,2)$ & -- & 2116\\
$(100,39)$ & 10 & $(11,3)$ & 5 & 1 & YES & YES & YES & $1.56$ & $(4,2)$ & -- & 2117\\
$(100,29)$ & 11 & $(13,3)$ & 6 & 1 & YES & YES & YES & $1.50$ & $(6,1)$ & -- & 2118\\
$(100,27)$ & 10 & $(22,5)$ & 7 & 2 & YES & YES & YES & $1.44$ & $(4,2)$ & -- & 2119\\
$(100,29)$ & 11 & $(58,17)$ & 9 & 2 & YES & YES & YES & $1.50$ & $(6,1)$ & NO & 2120\\
$(101,39)$ & 10 & $(5,1)$ & 4 & 1 & YES & YES & YES & $1.44$ & $(2,3)$ & -- & 2121\\
$(101,37)$ & 10 & $(7,3)$ & 4 & 1 & YES & YES & YES & $1.57$ & $(2,3)$ & -- & 2122\\
$(101,39)$ & 10 & $(7,3)$ & 4 & 1 & YES & YES & YES & $1.56$ & $(4,2)$ & -- & 2123\\
$(101,37)$ & 10 & $(21,8)$ & 6 & 1 & YES & YES & YES & $1.57$ & $(2,3)$ & NO & 2124\\
$(102,31)$ & 11 & $(10,3)$ & 5 & 2 & YES & YES & YES & $1.78$ & $(4,2)$ & -- & 2125\\
$(103,39)$ & 10 & $(5,1)$ & 4 & 1 & YES & YES & YES & $1.29$ & $(2,3)$ & -- & 2126\\
$(103,39)$ & 10 & $(5,2)$ & 3 & 1 & YES & YES & YES & $1.43$ & $(2,3)$ & NO & 2127\\
$(103,39)$ & 10 & $(7,3)$ & 4 & 1 & YES & YES & YES & $1.43$ & $(2,3)$ & NO & 2128\\
$(103,29)$ & 11 & $(10,3)$ & 5 & 1 & YES & YES & YES & $1.57$ & $(2,3)$ & -- & 2129\\
$(103,29)$ & 11 & $(11,3)$ & 5 & 1 & YES & YES & YES & $1.62$ & $(6,1)$ & -- & 2130\\
$(103,30)$ & 11 & $(14,3)$ & 6 & 1 & YES & YES & YES & $1.67$ & $(4,2)$ & -- & 2131\\
$(103,40)$ & 11 & $(75,29)$ & 9 & 1 & YES & YES & YES & $1.75$ & $(2,3)$ & NO & 2132\\
$(104,43)$ & 10 & $(5,2)$ & 3 & 1 & YES & YES & YES & $1.43$ & $(6,1)$ & -- & 2133\\
$(104,29)$ & 10 & $(13,4)$ & 6 & 13 & YES & YES & YES & $1.67$ & $(4,2)$ & -- & 2134\\
$(104,43)$ & 10 & $(63,26)$ & 9 & 1 & YES & YES & YES & $1.43$ & $(2,3)$ & 2564 & 2135\\
$(105,38)$ & 11 & $(4,1)$ & 3 & 1 & YES & YES & YES & $1.38$ & $(2,3)$ & -- & 2136\\
$(105,43)$ & 11 & $(5,1)$ & 4 & 5 & YES & YES & YES & $1.29$ & $(2,3)$ & -- & 2137\\
$(105,44)$ & 10 & $(5,2)$ & 3 & 5 & YES & YES & YES & $1.83$ & $(2,3)$ & -- & 2138\\
$(105,44)$ & 10 & $(7,2)$ & 4 & 7 & YES & YES & YES & $1.73$ & $(4,2)$ & NO & 2139\\
$(105,31)$ & 10 & $(8,3)$ & 4 & 1 & YES & YES & YES & $1.64$ & $(2,3)$ & -- & 2140\\
$(105,29)$ & 10 & $(9,4)$ & 5 & 3 & YES & YES & YES & $1.50$ & $(6,1)$ & NO & 2141\\
$(105,44)$ & 10 & $(9,4)$ & 5 & 3 & YES & YES & NO(2) & $1.64$ & $(2,3)$ & NO & 2142\\
$(105,29)$ & 10 & $(11,4)$ & 5 & 1 & YES & YES & YES & $1.50$ & $(6,1)$ & NO & 2143\\
$(105,29)$ & 10 & $(11,4)$ & 5 & 1 & YES & YES & YES & $1.70$ & $(2,3)$ & -- & 2144\\
$(105,44)$ & 10 & $(11,3)$ & 5 & 1 & YES & YES & YES & $1.70$ & $(2,3)$ & -- & 2145\\
$(105,29)$ & 10 & $(12,5)$ & 5 & 3 & YES & YES & YES & $1.70$ & $(2,3)$ & -- & 2146\\
$(105,29)$ & 10 & $(16,5)$ & 7 & 1 & YES & YES & YES & $1.50$ & $(6,1)$ & NO & 2147\\
$(105,29)$ & 10 & $(24,7)$ & 7 & 3 & YES & YES & YES & $1.50$ & $(6,1)$ & NO & 2148\\
$(105,38)$ & 11 & $(58,21)$ & 10 & 1 & YES & YES & YES & $1.50$ & $(2,3)$ & NO & 2149\\
$(105,29)$ & 10 & $(68,19)$ & 9 & 1 & YES & YES & YES & $1.70$ & $(2,3)$ & NO & 2150\\
$(105,43)$ & 11 & $(83,34)$ & 10 & 1 & YES & YES & YES & $1.43$ & $(2,3)$ & NO & 2151\\
$(105,31)$ & 10 & $(98,29)$ & 10 & 7 & YES & YES & YES & $1.75$ & $(2,3)$ & NO & 2152\\
$(106,31)$ & 10 & $(5,2)$ & 3 & 1 & YES & YES & YES & $1.43$ & $(8,0)$ & -- & 2153\\
$(106,41)$ & 10 & $(5,2)$ & 3 & 1 & YES & YES & YES & $1.60$ & $(4,2)$ & -- & 2154\\
$(106,41)$ & 10 & $(7,3)$ & 4 & 1 & YES & YES & YES & $1.70$ & $(2,3)$ & -- & 2155\\
$(106,23)$ & 11 & $(8,3)$ & 4 & 2 & YES & YES & YES & $1.73$ & $(2,3)$ & NO & 2156\\
$(106,31)$ & 10 & $(8,3)$ & 4 & 2 & YES & YES & YES & $1.75$ & $(2,3)$ & -- & 2157\\
$(106,41)$ & 10 & $(8,3)$ & 4 & 2 & YES & YES & YES & $1.70$ & $(2,3)$ & -- & 2158\\
$(106,23)$ & 11 & $(9,4)$ & 5 & 1 & YES & YES & YES & $1.57$ & $(4,2)$ & -- & 2159\\
$(106,23)$ & 11 & $(10,3)$ & 5 & 2 & YES & YES & YES & $1.83$ & $(2,3)$ & NO & 2160\\
$(106,41)$ & 10 & $(10,3)$ & 5 & 2 & YES & YES & YES & $1.67$ & $(4,2)$ & -- & 2161\\
$(106,41)$ & 10 & $(11,3)$ & 5 & 1 & YES & YES & YES & $1.56$ & $(4,2)$ & NO & 2162\\
$(106,41)$ & 10 & $(11,3)$ & 5 & 1 & YES & YES & YES & $1.56$ & $(4,2)$ & -- & 2163\\
$(106,41)$ & 10 & $(11,3)$ & 5 & 1 & YES & YES & YES & $1.60$ & $(2,3)$ & NO & 2164\\
$(106,31)$ & 10 & $(58,17)$ & 9 & 2 & YES & YES & YES & $1.43$ & $(8,0)$ & NO & 2165\\
$(106,41)$ & 10 & $(101,39)$ & 10 & 1 & YES & YES & YES & $1.60$ & $(2,3)$ & NO & 2166\\
$(107,41)$ & 10 & $(7,3)$ & 4 & 1 & YES & YES & YES & $1.50$ & $(4,2)$ & -- & 2167\\
$(107,41)$ & 10 & $(11,3)$ & 5 & 1 & YES & YES & YES & $1.67$ & $(4,2)$ & -- & 2168\\
$(107,41)$ & 10 & $(29,11)$ & 7 & 1 & YES & YES & YES & $1.38$ & $(4,2)$ & NO & 2169\\
$(107,41)$ & 10 & $(81,31)$ & 9 & 1 & YES & YES & YES & $1.62$ & $(2,3)$ & NO & 2170\\
$(107,44)$ & 12 & $(90,37)$ & 11 & 1 & YES & YES & YES & $1.50$ & $(2,3)$ & NO & 2171\\
$(108,41)$ & 10 & $(5,2)$ & 3 & 1 & YES & YES & YES & $1.57$ & $(2,3)$ & -- & 2172\\
$(108,41)$ & 10 & $(7,3)$ & 4 & 1 & YES & YES & YES & $1.43$ & $(2,3)$ & -- & 2173\\
$(108,41)$ & 10 & $(10,3)$ & 5 & 2 & YES & YES & YES & $1.70$ & $(2,3)$ & -- & 2174\\
$(108,41)$ & 10 & $(34,13)$ & 7 & 2 & YES & YES & YES & $1.57$ & $(2,3)$ & 2599 & 2175\\
$(109,40)$ & 10 & $(5,2)$ & 3 & 1 & YES & YES & NO(2) & $1.60$ & $(2,3)$ & NO & 2176\\
$(109,40)$ & 10 & $(8,3)$ & 4 & 1 & YES & YES & YES & $1.43$ & $(4,2)$ & -- & 2177\\
$(109,45)$ & 10 & $(10,3)$ & 5 & 1 & YES & YES & YES & $1.50$ & $(4,2)$ & -- & 2178\\
$(109,46)$ & 10 & $(10,3)$ & 5 & 1 & YES & YES & YES & $1.44$ & $(4,2)$ & -- & 2179\\
$(109,40)$ & 10 & $(18,7)$ & 6 & 1 & YES & YES & YES & $1.43$ & $(4,2)$ & NO & 2180\\
$(109,45)$ & 10 & $(26,11)$ & 7 & 1 & YES & YES & YES & $1.67$ & $(4,2)$ & NO & 2181\\
$(109,45)$ & 10 & $(31,13)$ & 7 & 1 & YES & YES & YES & $1.67$ & $(4,2)$ & NO & 2182\\
$(110,43)$ & 11 & $(6,1)$ & 5 & 2 & YES & YES & YES & $1.29$ & $(2,3)$ & NO & 2183\\
$(110,43)$ & 11 & $(110,43)$ & 11 & 110 & YES & YES & YES & $1.43$ & $(2,3)$ & NO & 2184\\
$(111,41)$ & 10 & $(3,1)$ & 2 & 3 & YES & YES & NO(2) & $1.73$ & $(2,3)$ & -- & 2185\\
$(111,46)$ & 10 & $(3,1)$ & 2 & 3 & YES & YES & YES & $1.38$ & $(6,1)$ & -- & 2186\\
$(111,41)$ & 10 & $(10,3)$ & 5 & 1 & YES & YES & YES & $1.50$ & $(4,2)$ & -- & 2187\\
$(111,43)$ & 10 & $(14,3)$ & 6 & 1 & YES & YES & YES & $1.70$ & $(2,3)$ & NO & 2188\\
$(111,46)$ & 10 & $(17,7)$ & 6 & 1 & YES & YES & YES & $1.38$ & $(6,1)$ & 2250 & 2189\\
$(111,41)$ & 10 & $(27,10)$ & 7 & 3 & YES & YES & NO(2) & $1.64$ & $(2,3)$ & NO & 2190\\
$(112,47)$ & 10 & $(5,2)$ & 3 & 1 & YES & YES & YES & $1.75$ & $(4,2)$ & -- & 2191\\
$(112,47)$ & 10 & $(7,2)$ & 4 & 7 & YES & YES & YES & $1.75$ & $(2,3)$ & -- & 2192\\
$(112,41)$ & 10 & $(8,3)$ & 4 & 8 & YES & YES & YES & $1.67$ & $(4,2)$ & -- & 2193\\
$(112,47)$ & 10 & $(11,3)$ & 5 & 1 & YES & YES & YES & $1.60$ & $(2,3)$ & NO & 2194\\
$(112,41)$ & 10 & $(13,3)$ & 6 & 1 & YES & YES & YES & $1.67$ & $(4,2)$ & -- & 2195\\
$(112,47)$ & 10 & $(17,7)$ & 6 & 1 & YES & YES & YES & $1.43$ & $(4,2)$ & NO & 2196\\
$(112,47)$ & 10 & $(26,11)$ & 7 & 2 & YES & YES & YES & $1.56$ & $(2,3)$ & 2407 & 2197\\
$(112,47)$ & 10 & $(43,18)$ & 8 & 1 & YES & YES & YES & $1.62$ & $(6,1)$ & NO & 2198\\
$(112,47)$ & 10 & $(69,29)$ & 9 & 1 & YES & YES & YES & $1.62$ & $(4,2)$ & 2631 & 2199\\
$(113,42)$ & 11 & $(5,2)$ & 3 & 1 & YES & YES & YES & $1.43$ & $(2,3)$ & NO & 2200\\
$(113,49)$ & 11 & $(6,1)$ & 5 & 1 & YES & YES & YES & $1.44$ & $(2,3)$ & NO & 2201\\
$(113,42)$ & 11 & $(7,3)$ & 4 & 1 & YES & YES & YES & $1.43$ & $(2,3)$ & NO & 2202\\
$(113,44)$ & 12 & $(113,44)$ & 12 & 113 & YES & YES & YES & $1.50$ & $(2,3)$ & NO & 2203\\
$(115,34)$ & 10 & $(5,2)$ & 3 & 5 & YES & YES & YES & $1.43$ & $(8,0)$ & -- & 2204\\
$(115,44)$ & 10 & $(5,2)$ & 3 & 5 & YES & YES & YES & $1.83$ & $(2,3)$ & -- & 2205\\
$(115,31)$ & 11 & $(8,3)$ & 4 & 1 & YES & YES & YES & $1.71$ & $(2,3)$ & -- & 2206\\
$(115,44)$ & 10 & $(8,3)$ & 4 & 1 & YES & YES & YES & $1.29$ & $(4,2)$ & -- & 2207\\
$(115,26)$ & 11 & $(9,4)$ & 5 & 1 & YES & YES & YES & $1.62$ & $(6,1)$ & -- & 2208\\
$(115,44)$ & 10 & $(9,4)$ & 5 & 1 & YES & YES & YES & $1.29$ & $(6,1)$ & NO & 2209\\
$(115,44)$ & 10 & $(10,3)$ & 5 & 5 & YES & YES & YES & $1.67$ & $(4,2)$ & -- & 2210\\
$(115,34)$ & 10 & $(24,7)$ & 7 & 1 & YES & YES & YES & $1.43$ & $(8,0)$ & NO & 2211\\
$(115,44)$ & 10 & $(55,21)$ & 8 & 5 & YES & YES & YES & $1.83$ & $(2,3)$ & NO & 2212\\
$(115,26)$ & 11 & $(79,18)$ & 10 & 1 & YES & YES & YES & $1.50$ & $(6,1)$ & NO & 2213\\
$(115,47)$ & 12 & $(93,38)$ & 11 & 1 & YES & YES & YES & $1.43$ & $(2,3)$ & NO & 2214\\
$(115,44)$ & 10 & $(107,41)$ & 10 & 1 & YES & YES & YES & $1.67$ & $(4,2)$ & NO & 2215\\
$(116,49)$ & 10 & $(10,3)$ & 5 & 2 & YES & YES & YES & $1.67$ & $(4,2)$ & -- & 2216\\
$(116,49)$ & 10 & $(11,3)$ & 5 & 1 & YES & YES & YES & $1.56$ & $(4,2)$ & NO & 2217\\
$(116,51)$ & 11 & $(25,11)$ & 7 & 1 & YES & YES & YES & $1.50$ & $(2,3)$ & NO & 2218\\
$(116,49)$ & 10 & $(29,12)$ & 7 & 29 & YES & YES & YES & $1.67$ & $(4,2)$ & NO & 2219\\
$(116,49)$ & 10 & $(43,18)$ & 8 & 1 & YES & YES & YES & $1.67$ & $(4,2)$ & NO & 2220\\
$(116,51)$ & 11 & $(116,51)$ & 11 & 116 & YES & YES & YES & $1.38$ & $(2,3)$ & NO & 2221\\
$(117,49)$ & 10 & $(5,2)$ & 3 & 1 & YES & YES & YES & $1.60$ & $(4,2)$ & -- & 2222\\
$(117,31)$ & 11 & $(29,8)$ & 7 & 1 & YES & YES & YES & $1.71$ & $(2,3)$ & NO & 2223\\
$(118,45)$ & 11 & $(6,1)$ & 5 & 2 & YES & YES & YES & $1.43$ & $(2,3)$ & NO & 2224\\
$(118,45)$ & 11 & $(6,1)$ & 5 & 2 & YES & YES & YES & $1.43$ & $(2,3)$ & -- & 2225\\
$(118,27)$ & 11 & $(11,4)$ & 5 & 1 & YES & YES & YES & $1.62$ & $(4,2)$ & -- & 2226\\
$(118,27)$ & 11 & $(32,7)$ & 8 & 2 & YES & YES & YES & $1.43$ & $(4,2)$ & NO & 2227\\
$(119,44)$ & 10 & $(2,1)$ & 1 & 1 & YES & YES & NO(2) & $1.64$ & $(2,3)$ & -- & 2228\\
$(119,45)$ & 11 & $(5,2)$ & 3 & 1 & YES & YES & YES & $1.43$ & $(2,3)$ & NO & 2229\\
$(119,46)$ & 10 & $(5,2)$ & 3 & 1 & YES & YES & YES & $1.56$ & $(2,3)$ & -- & 2230\\
$(119,26)$ & 11 & $(8,3)$ & 4 & 1 & YES & YES & YES & $1.56$ & $(2,3)$ & NO & 2231\\
$(119,44)$ & 10 & $(8,3)$ & 4 & 1 & YES & YES & YES & $1.56$ & $(4,2)$ & -- & 2232\\
$(119,26)$ & 11 & $(10,3)$ & 5 & 1 & YES & YES & YES & $1.56$ & $(2,3)$ & NO & 2233\\
$(119,46)$ & 10 & $(10,3)$ & 5 & 1 & YES & YES & YES & $1.50$ & $(4,2)$ & -- & 2234\\
$(119,50)$ & 10 & $(10,3)$ & 5 & 1 & YES & YES & YES & $1.60$ & $(2,3)$ & -- & 2235\\
$(119,44)$ & 10 & $(13,3)$ & 6 & 1 & YES & YES & YES & $1.44$ & $(4,2)$ & NO & 2236\\
$(119,46)$ & 10 & $(13,3)$ & 6 & 1 & YES & YES & YES & $1.56$ & $(4,2)$ & -- & 2237\\
$(119,46)$ & 10 & $(21,8)$ & 6 & 7 & YES & YES & YES & $1.60$ & $(4,2)$ & NO & 2238\\
$(119,45)$ & 11 & $(31,12)$ & 7 & 1 & YES & YES & YES & $1.75$ & $(2,3)$ & NO & 2239\\
$(119,45)$ & 11 & $(34,13)$ & 7 & 17 & YES & YES & YES & $1.57$ & $(2,3)$ & NO & 2240\\
$(119,44)$ & 10 & $(41,15)$ & 8 & 1 & YES & YES & YES & $1.56$ & $(4,2)$ & NO & 2241\\
$(119,46)$ & 10 & $(41,16)$ & 8 & 1 & YES & YES & YES & $1.50$ & $(4,2)$ & NO & 2242\\
$(119,50)$ & 10 & $(74,31)$ & 9 & 1 & YES & YES & YES & $1.70$ & $(2,3)$ & NO & 2243\\
$(119,44)$ & 10 & $(111,41)$ & 10 & 1 & YES & YES & YES & $1.75$ & $(2,3)$ & NO & 2244\\
$(120,47)$ & 12 & $(120,47)$ & 12 & 120 & YES & YES & YES & $1.43$ & $(2,3)$ & NO & 2245\\
$(121,50)$ & 10 & $(2,1)$ & 1 & 1 & NO & YES & NO(2) & $1.40$ & $(4,2)$ & -- & 2246\\
$(121,50)$ & 10 & $(3,1)$ & 2 & 1 & YES & YES & YES & $1.38$ & $(6,1)$ & -- & 2247\\
$(121,46)$ & 10 & $(5,2)$ & 3 & 1 & YES & YES & YES & $1.75$ & $(2,3)$ & -- & 2248\\
$(121,46)$ & 10 & $(8,3)$ & 4 & 1 & YES & YES & YES & $1.78$ & $(4,2)$ & -- & 2249\\
$(121,50)$ & 10 & $(12,5)$ & 5 & 1 & YES & YES & YES & $1.38$ & $(6,1)$ & 2189 & 2250\\
$(121,50)$ & 10 & $(13,3)$ & 6 & 1 & YES & YES & YES & $1.56$ & $(4,2)$ & NO & 2251\\
$(121,32)$ & 11 & $(34,9)$ & 8 & 1 & YES & YES & YES & $1.38$ & $(2,3)$ & NO & 2252\\
$(121,46)$ & 10 & $(66,25)$ & 9 & 11 & YES & YES & YES & $1.67$ & $(4,2)$ & NO & 2253\\
$(121,46)$ & 10 & $(79,30)$ & 9 & 1 & YES & YES & YES & $1.56$ & $(2,3)$ & NO & 2254\\
$(121,46)$ & 10 & $(92,35)$ & 10 & 1 & YES & YES & YES & $1.38$ & $(4,2)$ & NO & 2255\\
$(122,51)$ & 11 & $(2,1)$ & 1 & 2 & YES & YES & YES & $1.50$ & $(2,3)$ & NO & 2256\\
$(122,37)$ & 11 & $(3,1)$ & 2 & 1 & NO & YES & NO(2) & $1.50$ & $(4,2)$ & -- & 2257\\
$(122,51)$ & 11 & $(5,2)$ & 3 & 1 & YES & YES & YES & $1.50$ & $(2,3)$ & NO & 2258\\
$(122,37)$ & 11 & $(7,2)$ & 4 & 1 & YES & YES & YES & $1.60$ & $(4,2)$ & -- & 2259\\
$(122,37)$ & 11 & $(7,3)$ & 4 & 1 & YES & YES & YES & $1.67$ & $(4,2)$ & -- & 2260\\
$(122,33)$ & 11 & $(8,3)$ & 4 & 2 & YES & YES & YES & $1.56$ & $(4,2)$ & -- & 2261\\
$(122,37)$ & 11 & $(102,31)$ & 11 & 2 & YES & YES & YES & $1.67$ & $(4,2)$ & NO & 2262\\
$(123,47)$ & 10 & $(2,1)$ & 1 & 1 & YES & YES & YES & $1.50$ & $(2,3)$ & NO & 2263\\
$(123,47)$ & 10 & $(4,1)$ & 3 & 1 & YES & YES & YES & $1.50$ & $(6,1)$ & NO & 2264\\
$(123,47)$ & 10 & $(4,1)$ & 3 & 1 & YES & YES & YES & $1.50$ & $(6,1)$ & -- & 2265\\
$(123,47)$ & 10 & $(5,2)$ & 3 & 1 & YES & YES & YES & $1.62$ & $(4,2)$ & -- & 2266\\
$(123,52)$ & 11 & $(5,1)$ & 4 & 1 & YES & YES & YES & $1.29$ & $(4,2)$ & -- & 2267\\
$(123,52)$ & 11 & $(6,1)$ & 5 & 3 & YES & YES & YES & $1.43$ & $(4,2)$ & NO & 2268\\
$(123,52)$ & 11 & $(6,1)$ & 5 & 3 & YES & YES & YES & $1.43$ & $(4,2)$ & -- & 2269\\
$(123,47)$ & 10 & $(7,2)$ & 4 & 1 & YES & YES & YES & $1.60$ & $(2,3)$ & -- & 2270\\
$(123,47)$ & 10 & $(8,3)$ & 4 & 1 & YES & YES & YES & $1.56$ & $(4,2)$ & -- & 2271\\
$(123,47)$ & 10 & $(9,4)$ & 5 & 3 & YES & YES & YES & $1.50$ & $(6,1)$ & NO & 2272\\
$(123,47)$ & 10 & $(11,4)$ & 5 & 1 & YES & YES & YES & $1.83$ & $(2,3)$ & NO & 2273\\
$(123,47)$ & 10 & $(37,14)$ & 8 & 1 & YES & YES & YES & $1.67$ & $(2,3)$ & NO & 2274\\
$(123,47)$ & 10 & $(47,18)$ & 8 & 1 & YES & YES & YES & $1.75$ & $(2,3)$ & NO & 2275\\
$(123,47)$ & 10 & $(76,29)$ & 9 & 1 & YES & YES & YES & $1.56$ & $(6,1)$ & 2718 & 2276\\
$(123,52)$ & 11 & $(97,41)$ & 10 & 1 & YES & YES & YES & $1.29$ & $(4,2)$ & NO & 2277\\
$(123,47)$ & 10 & $(123,47)$ & 10 & 123 & YES & YES & YES & $1.38$ & $(6,1)$ & NO & 2278\\
$(123,52)$ & 11 & $(123,52)$ & 11 & 123 & YES & YES & YES & $1.43$ & $(4,2)$ & NO & 2279\\
$(124,23)$ & 12 & $(7,3)$ & 4 & 1 & YES & YES & YES & $1.50$ & $(6,1)$ & NO & 2280\\
$(125,53)$ & 11 & $(2,1)$ & 1 & 1 & YES & YES & YES & $1.43$ & $(2,3)$ & -- & 2281\\
$(125,53)$ & 11 & $(6,1)$ & 5 & 1 & YES & YES & YES & $1.29$ & $(2,3)$ & NO & 2282\\
$(125,37)$ & 11 & $(11,3)$ & 5 & 1 & YES & YES & YES & $1.50$ & $(6,1)$ & NO & 2283\\
$(125,53)$ & 11 & $(33,14)$ & 8 & 1 & YES & YES & YES & $1.43$ & $(2,3)$ & NO & 2284\\
$(127,35)$ & 11 & $(3,1)$ & 2 & 1 & YES & YES & YES & $1.71$ & $(2,3)$ & -- & 2285\\
$(127,29)$ & 11 & $(33,7)$ & 8 & 1 & YES & YES & YES & $1.50$ & $(4,2)$ & NO & 2286\\
$(127,35)$ & 11 & $(40,11)$ & 8 & 1 & YES & YES & YES & $1.57$ & $(2,3)$ & NO & 2287\\
$(127,29)$ & 11 & $(84,19)$ & 10 & 1 & YES & YES & YES & $1.50$ & $(4,2)$ & NO & 2288\\
$(128,49)$ & 10 & $(3,1)$ & 2 & 1 & YES & YES & YES & $1.73$ & $(4,2)$ & -- & 2289\\
$(128,49)$ & 10 & $(5,2)$ & 3 & 1 & YES & YES & YES & $1.50$ & $(4,2)$ & -- & 2290\\
$(128,53)$ & 11 & $(5,2)$ & 3 & 1 & YES & YES & YES & $1.62$ & $(6,1)$ & -- & 2291\\
$(128,49)$ & 10 & $(8,3)$ & 4 & 8 & YES & YES & YES & $1.78$ & $(4,2)$ & NO & 2292\\
$(128,49)$ & 10 & $(8,3)$ & 4 & 8 & YES & YES & YES & $1.78$ & $(4,2)$ & -- & 2293\\
$(128,47)$ & 10 & $(13,5)$ & 5 & 1 & YES & YES & YES & $1.56$ & $(6,1)$ & NO & 2294\\
$(128,49)$ & 10 & $(21,8)$ & 6 & 1 & YES & YES & YES & $1.38$ & $(6,1)$ & 2347 & 2295\\
$(128,47)$ & 10 & $(35,13)$ & 8 & 1 & YES & YES & YES & $1.62$ & $(4,2)$ & NO & 2296\\
$(128,49)$ & 10 & $(55,21)$ & 8 & 1 & YES & YES & YES & $1.75$ & $(2,3)$ & NO & 2297\\
$(128,49)$ & 10 & $(76,29)$ & 9 & 4 & YES & YES & YES & $1.67$ & $(4,2)$ & NO & 2298\\
$(128,49)$ & 10 & $(128,49)$ & 10 & 128 & YES & YES & YES & $1.64$ & $(4,2)$ & NO & 2299\\
$(129,50)$ & 10 & $(2,1)$ & 1 & 1 & NO & YES & NO(2) & $1.40$ & $(4,2)$ & -- & 2300\\
$(129,56)$ & 11 & $(2,1)$ & 1 & 1 & NO & YES & YES & $1.44$ & $(2,3)$ & -- & 2301\\
$(129,53)$ & 11 & $(5,1)$ & 4 & 1 & YES & YES & YES & $1.44$ & $(2,3)$ & -- & 2302\\
$(129,50)$ & 10 & $(8,3)$ & 4 & 1 & YES & YES & YES & $1.56$ & $(6,1)$ & 2343 & 2303\\
$(129,49)$ & 10 & $(11,3)$ & 5 & 1 & YES & YES & YES & $1.56$ & $(4,2)$ & NO & 2304\\
$(129,49)$ & 10 & $(37,14)$ & 8 & 1 & YES & YES & YES & $1.56$ & $(2,3)$ & NO & 2305\\
$(131,50)$ & 10 & $(3,1)$ & 2 & 1 & YES & YES & YES & $1.38$ & $(6,1)$ & -- & 2306\\
$(131,48)$ & 11 & $(5,2)$ & 3 & 1 & YES & YES & YES & $1.62$ & $(4,2)$ & -- & 2307\\
$(131,55)$ & 10 & $(5,2)$ & 3 & 1 & YES & YES & YES & $1.50$ & $(4,2)$ & -- & 2308\\
$(131,50)$ & 10 & $(7,2)$ & 4 & 1 & YES & YES & YES & $1.67$ & $(4,2)$ & -- & 2309\\
$(131,50)$ & 10 & $(8,3)$ & 4 & 1 & YES & YES & YES & $1.50$ & $(4,2)$ & -- & 2310\\
$(131,50)$ & 10 & $(10,3)$ & 5 & 1 & YES & YES & YES & $1.60$ & $(2,3)$ & -- & 2311\\
$(131,48)$ & 11 & $(13,5)$ & 5 & 1 & YES & YES & YES & $1.62$ & $(4,2)$ & NO & 2312\\
$(131,50)$ & 10 & $(34,13)$ & 7 & 1 & YES & YES & YES & $1.38$ & $(6,1)$ & NO & 2313\\
$(131,50)$ & 10 & $(123,47)$ & 10 & 1 & YES & YES & YES & $1.38$ & $(4,2)$ & NO & 2314\\
$(133,39)$ & 11 & $(8,3)$ & 4 & 1 & YES & YES & YES & $1.80$ & $(2,3)$ & -- & 2315\\
$(133,58)$ & 11 & $(13,5)$ & 5 & 1 & YES & YES & YES & $1.62$ & $(2,3)$ & NO & 2316\\
$(133,31)$ & 12 & $(23,5)$ & 7 & 1 & YES & YES & YES & $1.44$ & $(4,2)$ & NO & 2317\\
$(134,39)$ & 11 & $(8,3)$ & 4 & 2 & YES & YES & YES & $1.80$ & $(2,3)$ & -- & 2318\\
$(134,37)$ & 11 & $(112,31)$ & 10 & 2 & YES & YES & YES & $1.60$ & $(2,3)$ & 3200 & 2319\\
$(135,56)$ & 11 & $(5,2)$ & 3 & 5 & YES & YES & YES & $1.78$ & $(4,2)$ & -- & 2320\\
$(135,56)$ & 11 & $(7,2)$ & 4 & 1 & YES & YES & YES & $1.67$ & $(4,2)$ & NO & 2321\\
$(136,57)$ & 11 & $(43,18)$ & 8 & 1 & YES & YES & YES & $1.62$ & $(6,1)$ & NO & 2322\\
$(137,37)$ & 11 & $(3,1)$ & 2 & 1 & YES & YES & YES & $1.57$ & $(2,3)$ & -- & 2323\\
$(137,37)$ & 11 & $(7,3)$ & 4 & 1 & YES & YES & YES & $1.50$ & $(4,2)$ & -- & 2324\\
$(137,37)$ & 11 & $(11,3)$ & 5 & 1 & YES & YES & YES & $1.67$ & $(4,2)$ & -- & 2325\\
$(137,37)$ & 11 & $(56,15)$ & 9 & 1 & YES & YES & YES & $1.50$ & $(4,2)$ & NO & 2326\\
$(139,57)$ & 11 & $(2,1)$ & 1 & 1 & YES & YES & YES & $1.50$ & $(2,3)$ & -- & 2327\\
$(139,51)$ & 11 & $(68,25)$ & 9 & 1 & YES & YES & YES & $1.67$ & $(2,3)$ & NO & 2328\\
$(140,41)$ & 11 & $(3,1)$ & 2 & 1 & YES & YES & YES & $1.57$ & $(8,0)$ & -- & 2329\\
$(140,53)$ & 11 & $(3,1)$ & 2 & 1 & YES & YES & YES & $1.29$ & $(4,2)$ & -- & 2330\\
$(140,61)$ & 11 & $(5,2)$ & 3 & 5 & YES & YES & YES & $1.57$ & $(2,3)$ & NO & 2331\\
$(140,41)$ & 11 & $(7,3)$ & 4 & 7 & YES & YES & YES & $1.70$ & $(2,3)$ & -- & 2332\\
$(140,41)$ & 11 & $(8,3)$ & 4 & 4 & YES & YES & YES & $1.50$ & $(4,2)$ & -- & 2333\\
$(140,41)$ & 11 & $(44,13)$ & 8 & 4 & YES & YES & YES & $1.43$ & $(4,2)$ & NO & 2334\\
$(140,41)$ & 11 & $(58,17)$ & 9 & 2 & YES & YES & YES & $1.29$ & $(8,0)$ & 2426 & 2335\\
$(140,41)$ & 11 & $(140,41)$ & 11 & 140 & YES & YES & YES & $1.29$ & $(8,0)$ & NO & 2336\\
$(141,59)$ & 11 & $(26,11)$ & 7 & 1 & YES & YES & YES & $1.67$ & $(2,3)$ & NO & 2337\\
$(142,51)$ & 11 & $(3,1)$ & 2 & 1 & YES & YES & YES & $1.71$ & $(2,3)$ & -- & 2338\\
$(142,55)$ & 11 & $(44,17)$ & 8 & 2 & YES & YES & YES & $1.57$ & $(4,2)$ & NO & 2339\\
$(144,55)$ & 10 & $(2,1)$ & 1 & 2 & YES & YES & YES & $1.38$ & $(6,1)$ & -- & 2340\\
$(144,55)$ & 10 & $(3,1)$ & 2 & 3 & YES & YES & YES & $1.38$ & $(6,1)$ & -- & 2341\\
$(144,55)$ & 10 & $(3,1)$ & 2 & 3 & YES & YES & YES & $1.56$ & $(6,1)$ & NO & 2342\\
$(144,55)$ & 10 & $(5,2)$ & 3 & 1 & YES & YES & YES & $1.56$ & $(6,1)$ & 2303 & 2343\\
$(144,55)$ & 10 & $(5,2)$ & 3 & 1 & YES & YES & YES & $1.56$ & $(6,1)$ & -- & 2344\\
$(144,55)$ & 10 & $(8,3)$ & 4 & 8 & YES & YES & YES & $1.70$ & $(2,3)$ & -- & 2345\\
$(144,55)$ & 10 & $(11,4)$ & 5 & 1 & YES & YES & YES & $1.50$ & $(4,2)$ & NO & 2346\\
$(144,55)$ & 10 & $(13,5)$ & 5 & 1 & YES & YES & YES & $1.38$ & $(6,1)$ & 2295 & 2347\\
$(144,55)$ & 10 & $(21,8)$ & 6 & 3 & YES & YES & YES & $1.38$ & $(6,1)$ & NO & 2348\\
$(144,55)$ & 10 & $(23,9)$ & 7 & 1 & YES & YES & YES & $1.50$ & $(4,2)$ & NO & 2349\\
$(144,55)$ & 10 & $(55,21)$ & 8 & 1 & YES & YES & YES & $1.38$ & $(6,1)$ & NO & 2350\\
$(144,55)$ & 10 & $(60,23)$ & 9 & 12 & YES & YES & YES & $1.70$ & $(2,3)$ & NO & 2351\\
$(144,55)$ & 10 & $(97,37)$ & 10 & 1 & YES & YES & YES & $1.70$ & $(2,3)$ & NO & 2352\\
$(145,53)$ & 11 & $(5,1)$ & 4 & 5 & YES & YES & YES & $1.29$ & $(2,3)$ & -- & 2353\\
$(145,56)$ & 11 & $(7,2)$ & 4 & 1 & YES & YES & YES & $1.56$ & $(4,2)$ & NO & 2354\\
$(145,53)$ & 11 & $(8,3)$ & 4 & 1 & YES & YES & YES & $1.43$ & $(2,3)$ & NO & 2355\\
$(145,43)$ & 12 & $(11,3)$ & 5 & 1 & YES & YES & YES & $1.43$ & $(4,2)$ & NO & 2356\\
$(145,53)$ & 11 & $(52,19)$ & 9 & 1 & YES & YES & YES & $1.43$ & $(2,3)$ & NO & 2357\\
$(145,44)$ & 11 & $(122,37)$ & 11 & 1 & YES & YES & YES & $1.60$ & $(4,2)$ & NO & 2358\\
$(146,57)$ & 11 & $(4,1)$ & 3 & 2 & YES & YES & YES & $1.38$ & $(4,2)$ & NO & 2359\\
$(146,57)$ & 11 & $(8,3)$ & 4 & 2 & YES & YES & YES & $1.38$ & $(4,2)$ & NO & 2360\\
$(147,43)$ & 11 & $(3,1)$ & 2 & 3 & YES & YES & YES & $1.62$ & $(6,1)$ & NO & 2361\\
$(147,43)$ & 11 & $(3,1)$ & 2 & 3 & YES & YES & YES & $1.62$ & $(6,1)$ & -- & 2362\\
$(147,41)$ & 11 & $(7,3)$ & 4 & 7 & YES & YES & YES & $1.70$ & $(2,3)$ & NO & 2363\\
$(147,41)$ & 11 & $(7,3)$ & 4 & 7 & YES & YES & YES & $1.70$ & $(2,3)$ & -- & 2364\\
$(147,43)$ & 11 & $(11,3)$ & 5 & 1 & YES & YES & YES & $1.73$ & $(4,2)$ & NO & 2365\\
$(147,43)$ & 11 & $(13,4)$ & 6 & 1 & YES & YES & YES & $1.70$ & $(4,2)$ & NO & 2366\\
$(147,43)$ & 11 & $(14,3)$ & 6 & 7 & YES & YES & YES & $1.70$ & $(2,3)$ & NO & 2367\\
$(147,43)$ & 11 & $(23,7)$ & 7 & 1 & YES & YES & YES & $1.67$ & $(4,2)$ & NO & 2368\\
$(147,43)$ & 11 & $(31,9)$ & 8 & 1 & YES & YES & YES & $1.73$ & $(4,2)$ & 2613 & 2369\\
$(147,41)$ & 11 & $(93,26)$ & 10 & 3 & YES & YES & YES & $1.70$ & $(2,3)$ & NO & 2370\\
$(148,65)$ & 11 & $(5,2)$ & 3 & 1 & YES & YES & YES & $1.57$ & $(2,3)$ & -- & 2371\\
$(148,65)$ & 11 & $(34,15)$ & 8 & 2 & YES & YES & YES & $1.71$ & $(2,3)$ & 2672 & 2372\\
$(149,40)$ & 11 & $(3,1)$ & 2 & 1 & YES & YES & YES & $1.57$ & $(2,3)$ & NO & 2373\\
$(149,40)$ & 11 & $(3,1)$ & 2 & 1 & YES & YES & YES & $1.57$ & $(2,3)$ & -- & 2374\\
$(149,44)$ & 11 & $(3,1)$ & 2 & 1 & YES & YES & YES & $1.43$ & $(4,2)$ & -- & 2375\\
$(149,44)$ & 11 & $(8,3)$ & 4 & 1 & YES & YES & YES & $1.70$ & $(2,3)$ & -- & 2376\\
$(149,41)$ & 11 & $(11,3)$ & 5 & 1 & YES & YES & YES & $1.56$ & $(2,3)$ & -- & 2377\\
$(149,41)$ & 11 & $(13,4)$ & 6 & 1 & YES & YES & YES & $1.70$ & $(2,3)$ & NO & 2378\\
$(149,41)$ & 11 & $(32,9)$ & 8 & 1 & YES & YES & YES & $1.56$ & $(2,3)$ & NO & 2379\\
$(149,44)$ & 11 & $(47,14)$ & 9 & 1 & YES & YES & YES & $1.43$ & $(4,2)$ & NO & 2380\\
$(149,44)$ & 11 & $(64,19)$ & 9 & 1 & YES & YES & YES & $1.70$ & $(2,3)$ & NO & 2381\\
$(151,62)$ & 11 & $(3,1)$ & 2 & 1 & YES & YES & YES & $1.38$ & $(6,1)$ & -- & 2382\\
$(151,34)$ & 12 & $(5,2)$ & 3 & 1 & YES & YES & YES & $1.29$ & $(4,2)$ & NO & 2383\\
$(151,62)$ & 11 & $(5,2)$ & 3 & 1 & YES & YES & YES & $1.67$ & $(4,2)$ & -- & 2384\\
$(151,62)$ & 11 & $(9,2)$ & 5 & 1 & YES & YES & YES & $1.67$ & $(4,2)$ & -- & 2385\\
$(151,62)$ & 11 & $(22,9)$ & 7 & 1 & YES & YES & YES & $1.38$ & $(6,1)$ & 2457 & 2386\\
$(152,59)$ & 11 & $(2,1)$ & 1 & 2 & NO & YES & NO(2) & $1.44$ & $(4,2)$ & -- & 2387\\
$(152,59)$ & 11 & $(3,1)$ & 2 & 1 & YES & YES & YES & $1.50$ & $(4,2)$ & -- & 2388\\
$(152,55)$ & 12 & $(4,1)$ & 3 & 4 & YES & YES & YES & $1.43$ & $(2,3)$ & 2115 & 2389\\
$(152,63)$ & 11 & $(5,2)$ & 3 & 1 & YES & YES & YES & $1.56$ & $(4,2)$ & -- & 2390\\
$(152,41)$ & 11 & $(7,3)$ & 4 & 1 & YES & YES & YES & $1.38$ & $(4,2)$ & -- & 2391\\
$(152,55)$ & 12 & $(8,3)$ & 4 & 8 & YES & YES & YES & $1.43$ & $(2,3)$ & NO & 2392\\
$(152,63)$ & 11 & $(8,3)$ & 4 & 8 & YES & YES & YES & $1.56$ & $(4,2)$ & NO & 2393\\
$(152,45)$ & 12 & $(11,3)$ & 5 & 1 & YES & YES & YES & $1.43$ & $(4,2)$ & NO & 2394\\
$(152,41)$ & 11 & $(13,4)$ & 6 & 1 & YES & YES & YES & $1.38$ & $(4,2)$ & NO & 2395\\
$(152,45)$ & 12 & $(24,7)$ & 7 & 8 & YES & YES & YES & $1.43$ & $(4,2)$ & NO & 2396\\
$(152,63)$ & 11 & $(111,46)$ & 10 & 1 & YES & YES & YES & $1.38$ & $(4,2)$ & NO & 2397\\
$(153,64)$ & 11 & $(7,3)$ & 4 & 1 & YES & YES & YES & $1.14$ & $(4,2)$ & NO & 2398\\
$(153,35)$ & 12 & $(31,7)$ & 8 & 1 & YES & YES & YES & $1.50$ & $(4,2)$ & NO & 2399\\
$(154,59)$ & 11 & $(2,1)$ & 1 & 2 & YES & YES & YES & $1.50$ & $(2,3)$ & -- & 2400\\
$(154,65)$ & 11 & $(3,1)$ & 2 & 1 & YES & YES & YES & $1.56$ & $(2,3)$ & -- & 2401\\
$(154,45)$ & 11 & $(4,1)$ & 3 & 2 & YES & YES & YES & $1.43$ & $(8,0)$ & -- & 2402\\
$(154,59)$ & 11 & $(5,2)$ & 3 & 1 & YES & YES & YES & $1.80$ & $(2,3)$ & -- & 2403\\
$(154,65)$ & 11 & $(5,2)$ & 3 & 1 & YES & YES & YES & $1.43$ & $(4,2)$ & -- & 2404\\
$(154,59)$ & 11 & $(7,2)$ & 4 & 7 & YES & YES & YES & $1.67$ & $(4,2)$ & -- & 2405\\
$(154,45)$ & 11 & $(10,3)$ & 5 & 2 & YES & YES & YES & $1.70$ & $(2,3)$ & -- & 2406\\
$(154,65)$ & 11 & $(12,5)$ & 5 & 2 & YES & YES & YES & $1.56$ & $(2,3)$ & 2197 & 2407\\
$(154,65)$ & 11 & $(17,7)$ & 6 & 1 & YES & YES & YES & $1.43$ & $(4,2)$ & NO & 2408\\
$(154,59)$ & 11 & $(107,41)$ & 10 & 1 & YES & YES & YES & $1.62$ & $(2,3)$ & NO & 2409\\
$(154,45)$ & 11 & $(147,43)$ & 11 & 7 & YES & YES & YES & $1.70$ & $(2,3)$ & NO & 2410\\
$(155,48)$ & 12 & $(3,1)$ & 2 & 1 & YES & YES & YES & $1.57$ & $(2,3)$ & -- & 2411\\
$(155,64)$ & 11 & $(5,2)$ & 3 & 5 & YES & YES & YES & $1.50$ & $(4,2)$ & -- & 2412\\
$(155,64)$ & 11 & $(9,4)$ & 5 & 1 & YES & YES & YES & $1.29$ & $(6,1)$ & NO & 2413\\
$(155,48)$ & 12 & $(71,22)$ & 10 & 1 & YES & YES & YES & $1.57$ & $(2,3)$ & 2593 & 2414\\
$(156,43)$ & 12 & $(5,2)$ & 3 & 1 & YES & YES & YES & $1.62$ & $(4,2)$ & -- & 2415\\
$(156,43)$ & 12 & $(10,3)$ & 5 & 2 & YES & YES & YES & $1.62$ & $(4,2)$ & NO & 2416\\
$(157,69)$ & 11 & $(2,1)$ & 1 & 1 & YES & YES & YES & $1.57$ & $(2,3)$ & -- & 2417\\
$(157,46)$ & 11 & $(3,1)$ & 2 & 1 & YES & YES & YES & $1.29$ & $(8,0)$ & NO & 2418\\
$(157,46)$ & 11 & $(3,1)$ & 2 & 1 & YES & YES & YES & $1.29$ & $(8,0)$ & -- & 2419\\
$(157,46)$ & 11 & $(5,2)$ & 3 & 1 & YES & YES & YES & $1.67$ & $(2,3)$ & -- & 2420\\
$(157,58)$ & 11 & $(5,2)$ & 3 & 1 & YES & YES & YES & $1.70$ & $(2,3)$ & -- & 2421\\
$(157,28)$ & 13 & $(6,1)$ & 5 & 1 & YES & YES & YES & $1.38$ & $(2,3)$ & NO & 2422\\
$(157,58)$ & 11 & $(7,3)$ & 4 & 1 & YES & YES & YES & $1.57$ & $(2,3)$ & NO & 2423\\
$(157,60)$ & 11 & $(7,3)$ & 4 & 1 & YES & YES & YES & $1.43$ & $(4,2)$ & NO & 2424\\
$(157,58)$ & 11 & $(25,9)$ & 7 & 1 & YES & YES & YES & $1.75$ & $(2,3)$ & NO & 2425\\
$(157,46)$ & 11 & $(41,12)$ & 8 & 1 & YES & YES & YES & $1.29$ & $(8,0)$ & 2335 & 2426\\
$(157,65)$ & 12 & $(128,53)$ & 11 & 1 & YES & YES & YES & $1.50$ & $(6,1)$ & NO & 2427\\
$(157,65)$ & 12 & $(157,65)$ & 12 & 157 & YES & YES & YES & $1.50$ & $(6,1)$ & NO & 2428\\
$(158,57)$ & 11 & $(4,1)$ & 3 & 2 & YES & YES & YES & $1.80$ & $(2,3)$ & NO & 2429\\
$(158,57)$ & 11 & $(4,1)$ & 3 & 2 & YES & YES & YES & $1.80$ & $(2,3)$ & -- & 2430\\
$(158,61)$ & 11 & $(7,2)$ & 4 & 1 & YES & YES & YES & $1.78$ & $(4,2)$ & NO & 2431\\
$(158,61)$ & 11 & $(8,3)$ & 4 & 2 & YES & YES & YES & $1.29$ & $(4,2)$ & NO & 2432\\
$(158,61)$ & 11 & $(75,29)$ & 9 & 1 & YES & YES & YES & $1.67$ & $(4,2)$ & NO & 2433\\
$(158,57)$ & 11 & $(158,57)$ & 11 & 158 & YES & YES & YES & $1.75$ & $(2,3)$ & NO & 2434\\
$(158,61)$ & 11 & $(158,61)$ & 11 & 158 & YES & YES & YES & $1.38$ & $(4,2)$ & NO & 2435\\
$(159,44)$ & 11 & $(3,1)$ & 2 & 3 & YES & YES & YES & $1.43$ & $(4,2)$ & NO & 2436\\
$(159,44)$ & 11 & $(3,1)$ & 2 & 3 & YES & YES & YES & $1.43$ & $(4,2)$ & -- & 2437\\
$(159,47)$ & 11 & $(5,2)$ & 3 & 1 & YES & YES & YES & $1.82$ & $(4,2)$ & NO & 2438\\
$(159,47)$ & 11 & $(5,2)$ & 3 & 1 & YES & YES & YES & $1.82$ & $(4,2)$ & -- & 2439\\
$(159,44)$ & 11 & $(7,3)$ & 4 & 1 & YES & YES & YES & $1.67$ & $(4,2)$ & NO & 2440\\
$(159,44)$ & 11 & $(7,3)$ & 4 & 1 & YES & YES & YES & $1.70$ & $(2,3)$ & -- & 2441\\
$(159,47)$ & 11 & $(7,2)$ & 4 & 1 & YES & YES & YES & $1.50$ & $(6,1)$ & NO & 2442\\
$(159,47)$ & 11 & $(7,2)$ & 4 & 1 & YES & YES & YES & $1.70$ & $(2,3)$ & -- & 2443\\
$(159,62)$ & 11 & $(7,2)$ & 4 & 1 & YES & YES & YES & $1.56$ & $(4,2)$ & -- & 2444\\
$(159,37)$ & 12 & $(8,3)$ & 4 & 1 & YES & YES & YES & $1.50$ & $(4,2)$ & -- & 2445\\
$(159,37)$ & 12 & $(8,3)$ & 4 & 1 & YES & YES & YES & $1.44$ & $(4,2)$ & NO & 2446\\
$(159,47)$ & 11 & $(13,4)$ & 6 & 1 & YES & YES & YES & $1.82$ & $(4,2)$ & NO & 2447\\
$(159,44)$ & 11 & $(17,5)$ & 6 & 1 & YES & YES & YES & $1.67$ & $(4,2)$ & NO & 2448\\
$(159,44)$ & 11 & $(105,29)$ & 10 & 3 & YES & YES & YES & $1.60$ & $(2,3)$ & NO & 2449\\
$(160,67)$ & 11 & $(3,1)$ & 2 & 1 & YES & YES & YES & $1.57$ & $(2,3)$ & NO & 2450\\
$(160,67)$ & 11 & $(3,1)$ & 2 & 1 & YES & YES & YES & $1.57$ & $(2,3)$ & -- & 2451\\
$(160,67)$ & 11 & $(5,2)$ & 3 & 5 & YES & YES & YES & $1.43$ & $(4,2)$ & -- & 2452\\
$(160,67)$ & 11 & $(5,2)$ & 3 & 5 & YES & YES & YES & $1.43$ & $(2,3)$ & NO & 2453\\
$(160,67)$ & 11 & $(9,4)$ & 5 & 1 & YES & YES & YES & $1.62$ & $(6,1)$ & NO & 2454\\
$(161,68)$ & 11 & $(2,1)$ & 1 & 1 & YES & YES & YES & $1.56$ & $(2,3)$ & -- & 2455\\
$(161,66)$ & 11 & $(3,1)$ & 2 & 1 & YES & YES & YES & $1.38$ & $(6,1)$ & -- & 2456\\
$(161,66)$ & 11 & $(17,7)$ & 6 & 1 & YES & YES & YES & $1.38$ & $(6,1)$ & 2386 & 2457\\
$(162,49)$ & 12 & $(2,1)$ & 1 & 2 & YES & YES & YES & $1.50$ & $(2,3)$ & -- & 2458\\
$(162,49)$ & 12 & $(2,1)$ & 1 & 2 & YES & YES & YES & $1.50$ & $(2,3)$ & NO & 2459\\
$(163,63)$ & 11 & $(2,1)$ & 1 & 1 & NO & YES & YES & $1.50$ & $(2,3)$ & -- & 2460\\
$(163,62)$ & 11 & $(4,1)$ & 3 & 1 & YES & YES & YES & $1.55$ & $(4,2)$ & -- & 2461\\
$(163,62)$ & 11 & $(4,1)$ & 3 & 1 & YES & YES & YES & $1.64$ & $(4,2)$ & NO & 2462\\
$(163,45)$ & 12 & $(5,2)$ & 3 & 1 & YES & YES & YES & $1.43$ & $(4,2)$ & -- & 2463\\
$(163,71)$ & 11 & $(5,2)$ & 3 & 1 & YES & YES & YES & $1.43$ & $(2,3)$ & NO & 2464\\
$(163,62)$ & 11 & $(7,3)$ & 4 & 1 & YES & YES & YES & $1.56$ & $(4,2)$ & NO & 2465\\
$(163,63)$ & 11 & $(7,2)$ & 4 & 1 & YES & YES & YES & $1.44$ & $(4,2)$ & -- & 2466\\
$(164,45)$ & 12 & $(25,7)$ & 7 & 1 & YES & YES & YES & $1.57$ & $(2,3)$ & NO & 2467\\
$(165,64)$ & 11 & $(2,1)$ & 1 & 1 & YES & YES & YES & $1.29$ & $(6,1)$ & -- & 2468\\
$(165,61)$ & 11 & $(3,1)$ & 2 & 3 & YES & YES & YES & $1.75$ & $(4,2)$ & -- & 2469\\
$(165,61)$ & 11 & $(3,1)$ & 2 & 3 & YES & YES & YES & $1.75$ & $(4,2)$ & NO & 2470\\
$(165,64)$ & 11 & $(3,1)$ & 2 & 3 & YES & YES & YES & $1.29$ & $(6,1)$ & NO & 2471\\
$(165,64)$ & 11 & $(3,1)$ & 2 & 3 & YES & YES & YES & $1.29$ & $(6,1)$ & -- & 2472\\
$(165,61)$ & 11 & $(4,1)$ & 3 & 1 & YES & YES & YES & $1.73$ & $(4,2)$ & NO & 2473\\
$(165,61)$ & 11 & $(5,2)$ & 3 & 5 & YES & YES & YES & $1.70$ & $(2,3)$ & -- & 2474\\
$(165,46)$ & 11 & $(7,3)$ & 4 & 1 & YES & YES & YES & $1.70$ & $(2,3)$ & -- & 2475\\
$(166,61)$ & 11 & $(3,1)$ & 2 & 1 & YES & YES & YES & $1.57$ & $(2,3)$ & -- & 2476\\
$(166,49)$ & 11 & $(71,21)$ & 9 & 1 & YES & YES & YES & $1.70$ & $(2,3)$ & NO & 2477\\
$(166,61)$ & 11 & $(166,61)$ & 11 & 166 & YES & YES & YES & $1.60$ & $(4,2)$ & NO & 2478\\
$(167,64)$ & 11 & $(2,1)$ & 1 & 1 & YES & YES & YES & $1.29$ & $(4,2)$ & -- & 2479\\
$(167,69)$ & 11 & $(2,1)$ & 1 & 1 & YES & YES & YES & $1.56$ & $(2,3)$ & -- & 2480\\
$(167,69)$ & 11 & $(3,1)$ & 2 & 1 & YES & YES & YES & $1.75$ & $(2,3)$ & NO & 2481\\
$(167,69)$ & 11 & $(3,1)$ & 2 & 1 & YES & YES & YES & $1.75$ & $(2,3)$ & -- & 2482\\
$(167,51)$ & 12 & $(5,2)$ & 3 & 1 & YES & YES & YES & $1.57$ & $(2,3)$ & -- & 2483\\
$(167,69)$ & 11 & $(5,2)$ & 3 & 1 & YES & YES & YES & $1.62$ & $(4,2)$ & -- & 2484\\
$(167,69)$ & 11 & $(7,2)$ & 4 & 1 & YES & YES & YES & $1.67$ & $(4,2)$ & -- & 2485\\
$(167,64)$ & 11 & $(8,3)$ & 4 & 1 & YES & YES & YES & $1.43$ & $(2,3)$ & NO & 2486\\
$(167,46)$ & 11 & $(18,5)$ & 6 & 1 & YES & YES & YES & $1.62$ & $(2,3)$ & NO & 2487\\
$(167,69)$ & 11 & $(22,9)$ & 7 & 1 & YES & YES & YES & $1.92$ & $(2,3)$ & NO & 2488\\
$(167,69)$ & 11 & $(41,17)$ & 8 & 1 & YES & YES & YES & $1.62$ & $(4,2)$ & 2836 & 2489\\
$(168,71)$ & 11 & $(2,1)$ & 1 & 2 & YES & YES & YES & $1.38$ & $(4,2)$ & -- & 2490\\
$(168,71)$ & 11 & $(3,1)$ & 2 & 3 & YES & YES & YES & $1.38$ & $(4,2)$ & -- & 2491\\
$(168,65)$ & 12 & $(4,1)$ & 3 & 4 & YES & YES & YES & $1.57$ & $(4,2)$ & NO & 2492\\
$(168,65)$ & 12 & $(44,17)$ & 8 & 4 & YES & YES & YES & $1.57$ & $(4,2)$ & NO & 2493\\
$(168,65)$ & 12 & $(75,29)$ & 9 & 3 & YES & YES & YES & $1.62$ & $(4,2)$ & NO & 2494\\
$(168,71)$ & 11 & $(168,71)$ & 11 & 168 & YES & YES & YES & $1.50$ & $(4,2)$ & NO & 2495\\
$(169,64)$ & 11 & $(2,1)$ & 1 & 1 & YES & YES & YES & $1.29$ & $(2,3)$ & NO & 2496\\
$(169,71)$ & 11 & $(2,1)$ & 1 & 1 & YES & YES & YES & $1.50$ & $(6,1)$ & -- & 2497\\
$(169,70)$ & 11 & $(3,1)$ & 2 & 1 & YES & YES & YES & $1.73$ & $(4,2)$ & NO & 2498\\
$(169,70)$ & 11 & $(3,1)$ & 2 & 1 & YES & YES & YES & $1.75$ & $(2,3)$ & -- & 2499\\
$(169,50)$ & 11 & $(5,2)$ & 3 & 1 & YES & YES & YES & $1.62$ & $(2,3)$ & -- & 2500\\
$(169,70)$ & 11 & $(5,2)$ & 3 & 1 & YES & YES & YES & $1.56$ & $(4,2)$ & -- & 2501\\
$(169,50)$ & 11 & $(7,3)$ & 4 & 1 & YES & YES & YES & $1.67$ & $(4,2)$ & -- & 2502\\
$(169,70)$ & 11 & $(7,2)$ & 4 & 1 & YES & YES & YES & $1.67$ & $(4,2)$ & -- & 2503\\
$(169,71)$ & 11 & $(8,3)$ & 4 & 1 & YES & YES & YES & $1.29$ & $(4,2)$ & NO & 2504\\
$(169,71)$ & 11 & $(17,7)$ & 6 & 1 & YES & YES & YES & $1.29$ & $(4,2)$ & NO & 2505\\
$(169,71)$ & 11 & $(31,13)$ & 7 & 1 & YES & YES & YES & $1.83$ & $(2,3)$ & NO & 2506\\
$(169,38)$ & 13 & $(40,9)$ & 9 & 1 & YES & YES & YES & $1.38$ & $(2,3)$ & NO & 2507\\
$(169,50)$ & 11 & $(61,18)$ & 9 & 1 & YES & YES & YES & $1.70$ & $(2,3)$ & NO & 2508\\
$(170,47)$ & 11 & $(5,2)$ & 3 & 5 & YES & YES & YES & $1.56$ & $(4,2)$ & NO & 2509\\
$(170,47)$ & 11 & $(7,3)$ & 4 & 1 & YES & YES & YES & $1.56$ & $(4,2)$ & NO & 2510\\
$(170,47)$ & 11 & $(7,3)$ & 4 & 1 & YES & YES & YES & $1.60$ & $(2,3)$ & -- & 2511\\
$(170,47)$ & 11 & $(8,3)$ & 4 & 2 & YES & YES & YES & $1.56$ & $(4,2)$ & NO & 2512\\
$(171,50)$ & 11 & $(2,1)$ & 1 & 1 & YES & YES & YES & $1.62$ & $(2,3)$ & NO & 2513\\
$(171,65)$ & 11 & $(2,1)$ & 1 & 1 & YES & YES & YES & $1.50$ & $(2,3)$ & NO & 2514\\
$(171,65)$ & 11 & $(3,1)$ & 2 & 3 & YES & YES & YES & $1.67$ & $(2,3)$ & NO & 2515\\
$(171,65)$ & 11 & $(3,1)$ & 2 & 3 & YES & YES & YES & $1.67$ & $(2,3)$ & -- & 2516\\
$(171,65)$ & 11 & $(5,2)$ & 3 & 1 & YES & YES & YES & $1.29$ & $(4,2)$ & -- & 2517\\
$(171,50)$ & 11 & $(7,3)$ & 4 & 1 & YES & YES & YES & $1.50$ & $(4,2)$ & -- & 2518\\
$(171,65)$ & 11 & $(7,3)$ & 4 & 1 & YES & YES & YES & $1.43$ & $(4,2)$ & 2831 & 2519\\
$(171,65)$ & 11 & $(9,4)$ & 5 & 9 & YES & YES & YES & $1.75$ & $(2,3)$ & NO & 2520\\
$(171,50)$ & 11 & $(13,3)$ & 6 & 1 & YES & YES & YES & $1.56$ & $(4,2)$ & NO & 2521\\
$(171,65)$ & 11 & $(37,14)$ & 8 & 1 & YES & YES & YES & $1.43$ & $(4,2)$ & NO & 2522\\
$(172,71)$ & 11 & $(3,1)$ & 2 & 1 & YES & YES & YES & $1.43$ & $(2,3)$ & NO & 2523\\
$(172,75)$ & 12 & $(3,1)$ & 2 & 1 & YES & YES & YES & $1.62$ & $(6,1)$ & -- & 2524\\
$(172,75)$ & 12 & $(5,2)$ & 3 & 1 & YES & YES & YES & $1.62$ & $(6,1)$ & NO & 2525\\
$(172,71)$ & 11 & $(29,12)$ & 7 & 1 & YES & YES & YES & $1.43$ & $(2,3)$ & 2648 & 2526\\
$(172,63)$ & 11 & $(112,41)$ & 10 & 4 & YES & YES & YES & $1.67$ & $(4,2)$ & NO & 2527\\
$(173,64)$ & 11 & $(5,2)$ & 3 & 1 & YES & YES & YES & $1.56$ & $(4,2)$ & -- & 2528\\
$(173,73)$ & 11 & $(5,2)$ & 3 & 1 & YES & YES & YES & $1.67$ & $(4,2)$ & -- & 2529\\
$(173,66)$ & 11 & $(7,3)$ & 4 & 1 & YES & YES & YES & $1.50$ & $(4,2)$ & NO & 2530\\
$(173,66)$ & 11 & $(9,2)$ & 5 & 1 & YES & YES & YES & $1.67$ & $(4,2)$ & NO & 2531\\
$(173,64)$ & 11 & $(11,4)$ & 5 & 1 & YES & YES & YES & $1.50$ & $(4,2)$ & NO & 2532\\
$(173,64)$ & 11 & $(119,44)$ & 10 & 1 & YES & YES & YES & $1.44$ & $(4,2)$ & NO & 2533\\
$(173,73)$ & 11 & $(154,65)$ & 11 & 1 & YES & YES & YES & $1.56$ & $(4,2)$ & NO & 2534\\
$(173,64)$ & 11 & $(173,64)$ & 11 & 173 & YES & YES & YES & $1.50$ & $(4,2)$ & NO & 2535\\
$(175,67)$ & 11 & $(2,1)$ & 1 & 1 & NO & YES & YES & $1.50$ & $(2,3)$ & -- & 2536\\
$(175,67)$ & 11 & $(5,2)$ & 3 & 5 & YES & YES & YES & $1.78$ & $(4,2)$ & NO & 2537\\
$(175,67)$ & 11 & $(5,2)$ & 3 & 5 & YES & YES & YES & $1.78$ & $(4,2)$ & -- & 2538\\
$(175,67)$ & 11 & $(13,5)$ & 5 & 1 & YES & YES & YES & $1.38$ & $(4,2)$ & NO & 2539\\
$(175,67)$ & 11 & $(115,44)$ & 10 & 5 & YES & YES & YES & $1.67$ & $(4,2)$ & 3123 & 2540\\
$(176,65)$ & 11 & $(5,2)$ & 3 & 1 & YES & YES & YES & $1.56$ & $(4,2)$ & -- & 2541\\
$(176,65)$ & 11 & $(7,3)$ & 4 & 1 & YES & YES & YES & $1.67$ & $(4,2)$ & NO & 2542\\
$(177,65)$ & 11 & $(2,1)$ & 1 & 1 & YES & YES & YES & $1.57$ & $(2,3)$ & -- & 2543\\
$(177,65)$ & 11 & $(3,1)$ & 2 & 3 & YES & YES & YES & $1.43$ & $(6,1)$ & -- & 2544\\
$(177,65)$ & 11 & $(5,2)$ & 3 & 1 & YES & YES & YES & $1.57$ & $(4,2)$ & -- & 2545\\
$(177,49)$ & 11 & $(7,3)$ & 4 & 1 & YES & YES & YES & $1.56$ & $(4,2)$ & NO & 2546\\
$(177,49)$ & 11 & $(17,5)$ & 6 & 1 & YES & YES & YES & $1.56$ & $(4,2)$ & NO & 2547\\
$(177,65)$ & 11 & $(27,10)$ & 7 & 3 & YES & YES & YES & $1.50$ & $(4,2)$ & NO & 2548\\
$(177,65)$ & 11 & $(30,11)$ & 7 & 3 & YES & YES & YES & $1.57$ & $(2,3)$ & NO & 2549\\
$(178,69)$ & 11 & $(2,1)$ & 1 & 2 & YES & YES & YES & $1.29$ & $(6,1)$ & -- & 2550\\
$(178,69)$ & 11 & $(3,1)$ & 2 & 1 & YES & YES & YES & $1.29$ & $(4,2)$ & NO & 2551\\
$(178,69)$ & 11 & $(3,1)$ & 2 & 1 & YES & YES & YES & $1.29$ & $(4,2)$ & -- & 2552\\
$(178,69)$ & 11 & $(3,1)$ & 2 & 1 & YES & YES & YES & $1.83$ & $(2,3)$ & NO & 2553\\
$(178,69)$ & 11 & $(5,2)$ & 3 & 1 & YES & YES & YES & $1.70$ & $(2,3)$ & -- & 2554\\
$(178,69)$ & 11 & $(13,5)$ & 5 & 1 & YES & YES & YES & $1.62$ & $(4,2)$ & NO & 2555\\
$(178,69)$ & 11 & $(21,8)$ & 6 & 1 & YES & YES & YES & $1.70$ & $(2,3)$ & NO & 2556\\
$(178,69)$ & 11 & $(23,9)$ & 7 & 1 & YES & YES & YES & $1.62$ & $(4,2)$ & NO & 2557\\
$(179,75)$ & 11 & $(2,1)$ & 1 & 1 & YES & YES & YES & $1.75$ & $(2,3)$ & -- & 2558\\
$(179,50)$ & 11 & $(3,1)$ & 2 & 1 & YES & YES & YES & $1.62$ & $(2,3)$ & -- & 2559\\
$(179,74)$ & 11 & $(3,1)$ & 2 & 1 & YES & YES & YES & $1.43$ & $(2,3)$ & NO & 2560\\
$(179,75)$ & 11 & $(3,1)$ & 2 & 1 & YES & YES & YES & $1.62$ & $(4,2)$ & -- & 2561\\
$(179,75)$ & 11 & $(3,1)$ & 2 & 1 & YES & YES & YES & $1.73$ & $(4,2)$ & NO & 2562\\
$(179,78)$ & 12 & $(3,1)$ & 2 & 1 & YES & YES & YES & $1.57$ & $(4,2)$ & -- & 2563\\
$(179,74)$ & 11 & $(17,7)$ & 6 & 1 & YES & YES & YES & $1.43$ & $(2,3)$ & 2135 & 2564\\
$(179,74)$ & 11 & $(121,50)$ & 10 & 1 & YES & YES & YES & $1.56$ & $(4,2)$ & NO & 2565\\
$(179,75)$ & 11 & $(179,75)$ & 11 & 179 & YES & YES & YES & $1.73$ & $(4,2)$ & NO & 2566\\
$(180,41)$ & 12 & $(7,3)$ & 4 & 1 & YES & YES & YES & $1.38$ & $(4,2)$ & -- & 2567\\
$(180,41)$ & 12 & $(8,3)$ & 4 & 4 & YES & YES & YES & $1.50$ & $(4,2)$ & -- & 2568\\
$(181,50)$ & 11 & $(2,1)$ & 1 & 1 & YES & YES & YES & $1.62$ & $(2,3)$ & -- & 2569\\
$(181,65)$ & 12 & $(2,1)$ & 1 & 1 & YES & YES & YES & $1.50$ & $(2,3)$ & NO & 2570\\
$(181,75)$ & 11 & $(2,1)$ & 1 & 1 & YES & YES & YES & $1.60$ & $(4,2)$ & -- & 2571\\
$(181,76)$ & 11 & $(2,1)$ & 1 & 1 & YES & YES & YES & $1.75$ & $(2,3)$ & -- & 2572\\
$(181,53)$ & 12 & $(3,1)$ & 2 & 1 & YES & YES & YES & $1.43$ & $(4,2)$ & -- & 2573\\
$(181,53)$ & 12 & $(3,1)$ & 2 & 1 & YES & YES & YES & $1.62$ & $(4,2)$ & NO & 2574\\
$(181,70)$ & 11 & $(5,2)$ & 3 & 1 & YES & YES & YES & $1.44$ & $(4,2)$ & -- & 2575\\
$(181,75)$ & 11 & $(5,2)$ & 3 & 1 & YES & YES & YES & $1.56$ & $(4,2)$ & -- & 2576\\
$(181,76)$ & 11 & $(5,2)$ & 3 & 1 & YES & YES & YES & $1.70$ & $(2,3)$ & -- & 2577\\
$(181,53)$ & 12 & $(11,3)$ & 5 & 1 & YES & YES & YES & $1.75$ & $(2,3)$ & NO & 2578\\
$(181,55)$ & 12 & $(11,3)$ & 5 & 1 & YES & YES & YES & $1.67$ & $(4,2)$ & NO & 2579\\
$(181,76)$ & 11 & $(19,8)$ & 6 & 1 & YES & YES & YES & $1.57$ & $(2,3)$ & NO & 2580\\
$(181,53)$ & 12 & $(24,7)$ & 7 & 1 & YES & YES & YES & $1.43$ & $(6,1)$ & NO & 2581\\
$(181,76)$ & 11 & $(31,13)$ & 7 & 1 & YES & YES & YES & $1.57$ & $(2,3)$ & NO & 2582\\
$(181,50)$ & 11 & $(76,21)$ & 9 & 1 & YES & YES & YES & $1.62$ & $(2,3)$ & NO & 2583\\
$(181,41)$ & 12 & $(115,26)$ & 11 & 1 & YES & YES & YES & $1.50$ & $(4,2)$ & NO & 2584\\
$(181,70)$ & 11 & $(119,46)$ & 10 & 1 & YES & YES & YES & $1.56$ & $(4,2)$ & NO & 2585\\
$(182,71)$ & 12 & $(3,1)$ & 2 & 1 & YES & YES & YES & $1.71$ & $(2,3)$ & -- & 2586\\
$(182,71)$ & 12 & $(3,1)$ & 2 & 1 & YES & YES & YES & $1.71$ & $(2,3)$ & NO & 2587\\
$(183,71)$ & 11 & $(2,1)$ & 1 & 1 & YES & YES & YES & $1.56$ & $(2,3)$ & -- & 2588\\
$(183,71)$ & 11 & $(4,1)$ & 3 & 1 & YES & YES & YES & $1.67$ & $(2,3)$ & NO & 2589\\
$(183,71)$ & 11 & $(8,3)$ & 4 & 1 & YES & YES & YES & $1.67$ & $(2,3)$ & NO & 2590\\
$(183,71)$ & 11 & $(85,33)$ & 10 & 1 & YES & YES & YES & $1.67$ & $(4,2)$ & NO & 2591\\
$(184,71)$ & 12 & $(4,1)$ & 3 & 4 & YES & YES & YES & $1.29$ & $(6,1)$ & NO & 2592\\
$(184,57)$ & 12 & $(42,13)$ & 9 & 2 & YES & YES & YES & $1.57$ & $(2,3)$ & 2414 & 2593\\
$(184,77)$ & 12 & $(184,77)$ & 12 & 184 & YES & YES & YES & $1.57$ & $(2,3)$ & NO & 2594\\
$(186,71)$ & 11 & $(2,1)$ & 1 & 2 & YES & YES & YES & $1.67$ & $(2,3)$ & -- & 2595\\
$(186,71)$ & 11 & $(4,1)$ & 3 & 2 & YES & YES & YES & $1.83$ & $(2,3)$ & NO & 2596\\
$(186,71)$ & 11 & $(4,1)$ & 3 & 2 & YES & YES & YES & $1.83$ & $(2,3)$ & -- & 2597\\
$(186,71)$ & 11 & $(7,3)$ & 4 & 1 & YES & YES & YES & $1.43$ & $(4,2)$ & NO & 2598\\
$(186,71)$ & 11 & $(8,3)$ & 4 & 2 & YES & YES & YES & $1.57$ & $(2,3)$ & 2175 & 2599\\
$(186,71)$ & 11 & $(34,13)$ & 7 & 2 & YES & YES & YES & $1.83$ & $(2,3)$ & NO & 2600\\
$(187,71)$ & 11 & $(2,1)$ & 1 & 1 & YES & YES & YES & $1.43$ & $(2,3)$ & NO & 2601\\
$(187,71)$ & 11 & $(3,1)$ & 2 & 1 & YES & YES & YES & $1.64$ & $(2,3)$ & -- & 2602\\
$(187,71)$ & 11 & $(4,1)$ & 3 & 1 & YES & YES & YES & $1.56$ & $(2,3)$ & NO & 2603\\
$(187,71)$ & 11 & $(18,7)$ & 6 & 1 & YES & YES & YES & $1.80$ & $(2,3)$ & NO & 2604\\
$(187,71)$ & 11 & $(50,19)$ & 8 & 1 & YES & YES & YES & $1.73$ & $(2,3)$ & NO & 2605\\
$(188,69)$ & 11 & $(3,1)$ & 2 & 1 & YES & YES & YES & $1.38$ & $(4,2)$ & -- & 2606\\
$(188,79)$ & 11 & $(7,2)$ & 4 & 1 & YES & YES & YES & $1.70$ & $(2,3)$ & NO & 2607\\
$(188,79)$ & 11 & $(8,3)$ & 4 & 4 & YES & YES & YES & $1.80$ & $(2,3)$ & NO & 2608\\
$(188,79)$ & 11 & $(17,7)$ & 6 & 1 & YES & YES & YES & $1.80$ & $(2,3)$ & NO & 2609\\
$(188,79)$ & 11 & $(43,18)$ & 8 & 1 & YES & YES & YES & $1.70$ & $(2,3)$ & 3134 & 2610\\
$(189,73)$ & 12 & $(5,1)$ & 4 & 1 & YES & YES & YES & $1.56$ & $(4,2)$ & -- & 2611\\
$(189,73)$ & 12 & $(5,1)$ & 4 & 1 & YES & YES & YES & $1.56$ & $(4,2)$ & NO & 2612\\
$(189,55)$ & 12 & $(17,5)$ & 6 & 1 & YES & YES & YES & $1.73$ & $(4,2)$ & 2369 & 2613\\
$(189,55)$ & 12 & $(38,11)$ & 9 & 1 & YES & YES & YES & $1.50$ & $(6,1)$ & NO & 2614\\
$(189,83)$ & 12 & $(66,29)$ & 9 & 3 & YES & YES & YES & $1.57$ & $(4,2)$ & NO & 2615\\
$(191,80)$ & 11 & $(2,1)$ & 1 & 1 & YES & YES & YES & $1.70$ & $(4,2)$ & -- & 2616\\
$(191,71)$ & 12 & $(3,1)$ & 2 & 1 & YES & YES & YES & $1.62$ & $(6,1)$ & -- & 2617\\
$(191,56)$ & 12 & $(5,2)$ & 3 & 1 & YES & YES & YES & $1.67$ & $(4,2)$ & NO & 2618\\
$(191,74)$ & 11 & $(5,2)$ & 3 & 1 & YES & YES & YES & $1.56$ & $(4,2)$ & -- & 2619\\
$(191,58)$ & 12 & $(33,10)$ & 8 & 1 & YES & YES & YES & $1.73$ & $(4,2)$ & NO & 2620\\
$(191,56)$ & 12 & $(75,22)$ & 10 & 1 & YES & YES & YES & $1.83$ & $(2,3)$ & 2699 & 2621\\
$(192,73)$ & 11 & $(2,1)$ & 1 & 2 & YES & YES & YES & $1.50$ & $(4,2)$ & NO & 2622\\
$(192,71)$ & 11 & $(3,1)$ & 2 & 3 & YES & YES & YES & $1.50$ & $(4,2)$ & NO & 2623\\
$(192,71)$ & 11 & $(3,1)$ & 2 & 3 & YES & YES & YES & $1.50$ & $(4,2)$ & -- & 2624\\
$(192,73)$ & 11 & $(4,1)$ & 3 & 4 & YES & YES & YES & $1.67$ & $(2,3)$ & NO & 2625\\
$(192,73)$ & 11 & $(8,3)$ & 4 & 8 & YES & YES & YES & $1.43$ & $(2,3)$ & NO & 2626\\
$(192,73)$ & 11 & $(21,8)$ & 6 & 3 & YES & YES & YES & $1.29$ & $(4,2)$ & NO & 2627\\
$(192,73)$ & 11 & $(192,73)$ & 11 & 192 & YES & YES & YES & $1.75$ & $(2,3)$ & NO & 2628\\
$(193,81)$ & 11 & $(2,1)$ & 1 & 1 & YES & YES & YES & $1.75$ & $(2,3)$ & -- & 2629\\
$(193,81)$ & 11 & $(8,3)$ & 4 & 1 & YES & YES & YES & $1.50$ & $(4,2)$ & NO & 2630\\
$(193,81)$ & 11 & $(19,8)$ & 6 & 1 & YES & YES & YES & $1.62$ & $(4,2)$ & 2199 & 2631\\
$(193,80)$ & 12 & $(70,29)$ & 9 & 1 & YES & YES & YES & $1.50$ & $(4,2)$ & NO & 2632\\
$(193,81)$ & 11 & $(81,34)$ & 9 & 1 & YES & YES & YES & $1.56$ & $(2,3)$ & NO & 2633\\
$(193,81)$ & 11 & $(131,55)$ & 10 & 1 & YES & YES & YES & $1.60$ & $(2,3)$ & NO & 2634\\
$(194,75)$ & 11 & $(2,1)$ & 1 & 2 & YES & YES & YES & $1.56$ & $(2,3)$ & -- & 2635\\
$(194,75)$ & 11 & $(4,1)$ & 3 & 2 & YES & YES & YES & $1.44$ & $(4,2)$ & NO & 2636\\
$(194,75)$ & 11 & $(5,2)$ & 3 & 1 & YES & YES & YES & $1.56$ & $(4,2)$ & -- & 2637\\
$(194,75)$ & 11 & $(8,3)$ & 4 & 2 & YES & YES & YES & $1.64$ & $(2,3)$ & NO & 2638\\
$(194,75)$ & 11 & $(57,22)$ & 9 & 1 & YES & YES & YES & $1.56$ & $(4,2)$ & NO & 2639\\
$(194,75)$ & 11 & $(106,41)$ & 10 & 2 & YES & YES & YES & $1.56$ & $(4,2)$ & NO & 2640\\
$(194,75)$ & 11 & $(119,46)$ & 10 & 1 & YES & YES & YES & $1.44$ & $(4,2)$ & NO & 2641\\
$(196,75)$ & 11 & $(2,1)$ & 1 & 2 & YES & YES & YES & $1.56$ & $(2,3)$ & -- & 2642\\
$(196,75)$ & 11 & $(3,1)$ & 2 & 1 & YES & YES & YES & $1.60$ & $(2,3)$ & -- & 2643\\
$(196,75)$ & 11 & $(3,1)$ & 2 & 1 & YES & YES & YES & $1.64$ & $(4,2)$ & NO & 2644\\
$(196,75)$ & 11 & $(4,1)$ & 3 & 4 & YES & YES & YES & $1.62$ & $(2,3)$ & NO & 2645\\
$(196,75)$ & 11 & $(4,1)$ & 3 & 4 & YES & YES & YES & $1.62$ & $(2,3)$ & -- & 2646\\
$(196,75)$ & 11 & $(8,3)$ & 4 & 4 & YES & YES & YES & $1.56$ & $(2,3)$ & NO & 2647\\
$(196,81)$ & 11 & $(17,7)$ & 6 & 1 & YES & YES & YES & $1.43$ & $(2,3)$ & 2526 & 2648\\
$(196,75)$ & 11 & $(21,8)$ & 6 & 7 & YES & YES & YES & $1.75$ & $(2,3)$ & NO & 2649\\
$(196,81)$ & 11 & $(22,9)$ & 7 & 2 & YES & YES & YES & $1.50$ & $(4,2)$ & NO & 2650\\
$(196,55)$ & 12 & $(29,8)$ & 7 & 1 & YES & YES & YES & $1.67$ & $(4,2)$ & NO & 2651\\
$(196,81)$ & 11 & $(41,17)$ & 8 & 1 & YES & YES & YES & $1.67$ & $(4,2)$ & NO & 2652\\
$(196,75)$ & 11 & $(81,31)$ & 9 & 1 & YES & YES & YES & $1.62$ & $(2,3)$ & NO & 2653\\
$(197,76)$ & 12 & $(4,1)$ & 3 & 1 & YES & YES & YES & $1.43$ & $(2,3)$ & NO & 2654\\
$(197,61)$ & 13 & $(13,4)$ & 6 & 1 & YES & YES & YES & $1.50$ & $(2,3)$ & NO & 2655\\
$(197,43)$ & 12 & $(33,7)$ & 8 & 1 & YES & YES & YES & $1.56$ & $(4,2)$ & NO & 2656\\
$(198,71)$ & 12 & $(3,1)$ & 2 & 3 & YES & YES & YES & $1.57$ & $(2,3)$ & -- & 2657\\
$(199,76)$ & 11 & $(2,1)$ & 1 & 1 & YES & YES & YES & $1.75$ & $(2,3)$ & -- & 2658\\
$(199,76)$ & 11 & $(3,1)$ & 2 & 1 & YES & YES & YES & $1.75$ & $(2,3)$ & -- & 2659\\
$(199,55)$ & 11 & $(5,2)$ & 3 & 1 & YES & YES & YES & $1.60$ & $(2,3)$ & -- & 2660\\
$(199,76)$ & 11 & $(5,2)$ & 3 & 1 & YES & YES & YES & $1.60$ & $(2,3)$ & -- & 2661\\
$(199,76)$ & 11 & $(13,5)$ & 5 & 1 & YES & YES & YES & $1.73$ & $(2,3)$ & NO & 2662\\
$(199,76)$ & 11 & $(34,13)$ & 7 & 1 & YES & YES & YES & $1.83$ & $(2,3)$ & NO & 2663\\
$(199,74)$ & 12 & $(78,29)$ & 10 & 1 & YES & YES & YES & $1.50$ & $(6,1)$ & NO & 2664\\
$(199,76)$ & 11 & $(89,34)$ & 9 & 1 & YES & YES & YES & $1.70$ & $(2,3)$ & 2805 & 2665\\
$(200,59)$ & 12 & $(4,1)$ & 3 & 4 & YES & YES & YES & $1.73$ & $(4,2)$ & -- & 2666\\
$(200,61)$ & 12 & $(36,11)$ & 8 & 4 & YES & YES & YES & $1.75$ & $(2,3)$ & NO & 2667\\
$(201,37)$ & 14 & $(2,1)$ & 1 & 1 & YES & YES & YES & $1.44$ & $(2,3)$ & -- & 2668\\
$(201,77)$ & 12 & $(5,1)$ & 4 & 1 & YES & YES & YES & $1.50$ & $(4,2)$ & NO & 2669\\
$(201,61)$ & 12 & $(33,10)$ & 8 & 3 & YES & YES & YES & $1.60$ & $(4,2)$ & NO & 2670\\
$(201,83)$ & 12 & $(155,64)$ & 11 & 1 & YES & YES & YES & $1.50$ & $(4,2)$ & NO & 2671\\
$(202,89)$ & 12 & $(16,7)$ & 6 & 2 & YES & YES & YES & $1.71$ & $(2,3)$ & 2372 & 2672\\
$(202,59)$ & 12 & $(89,26)$ & 10 & 1 & YES & YES & YES & $1.14$ & $(4,2)$ & NO & 2673\\
$(203,57)$ & 12 & $(2,1)$ & 1 & 1 & YES & YES & YES & $1.50$ & $(6,1)$ & -- & 2674\\
$(203,75)$ & 12 & $(3,1)$ & 2 & 1 & YES & YES & YES & $1.82$ & $(2,3)$ & -- & 2675\\
$(203,75)$ & 12 & $(11,4)$ & 5 & 1 & YES & YES & YES & $1.82$ & $(2,3)$ & NO & 2676\\
$(204,89)$ & 12 & $(3,1)$ & 2 & 3 & YES & YES & YES & $1.75$ & $(4,2)$ & NO & 2677\\
$(204,89)$ & 12 & $(3,1)$ & 2 & 3 & YES & YES & YES & $1.75$ & $(4,2)$ & -- & 2678\\
$(205,78)$ & 12 & $(5,2)$ & 3 & 5 & YES & YES & YES & $1.43$ & $(4,2)$ & NO & 2679\\
$(206,85)$ & 12 & $(12,5)$ & 5 & 2 & YES & YES & YES & $1.43$ & $(2,3)$ & 2023 & 2680\\
$(206,47)$ & 12 & $(19,4)$ & 7 & 1 & YES & YES & YES & $1.56$ & $(4,2)$ & NO & 2681\\
$(207,76)$ & 11 & $(2,1)$ & 1 & 1 & YES & YES & YES & $1.56$ & $(6,1)$ & -- & 2682\\
$(207,76)$ & 11 & $(3,1)$ & 2 & 3 & YES & YES & YES & $1.38$ & $(4,2)$ & -- & 2683\\
$(207,85)$ & 12 & $(3,1)$ & 2 & 3 & YES & YES & YES & $1.75$ & $(2,3)$ & NO & 2684\\
$(207,85)$ & 12 & $(3,1)$ & 2 & 3 & YES & YES & YES & $1.75$ & $(2,3)$ & -- & 2685\\
$(207,79)$ & 11 & $(4,1)$ & 3 & 1 & YES & YES & YES & $1.78$ & $(4,2)$ & -- & 2686\\
$(207,79)$ & 11 & $(4,1)$ & 3 & 1 & YES & YES & YES & $1.62$ & $(4,2)$ & NO & 2687\\
$(207,79)$ & 11 & $(7,2)$ & 4 & 1 & YES & YES & YES & $1.62$ & $(4,2)$ & NO & 2688\\
$(207,79)$ & 11 & $(34,13)$ & 7 & 1 & YES & YES & YES & $1.60$ & $(2,3)$ & 2804 & 2689\\
$(207,79)$ & 11 & $(47,18)$ & 8 & 1 & YES & YES & YES & $1.50$ & $(4,2)$ & 3194 & 2690\\
$(207,79)$ & 11 & $(97,37)$ & 10 & 1 & YES & YES & YES & $1.70$ & $(2,3)$ & NO & 2691\\
$(207,79)$ & 11 & $(131,50)$ & 10 & 1 & YES & YES & YES & $1.60$ & $(2,3)$ & NO & 2692\\
$(207,85)$ & 12 & $(151,62)$ & 11 & 1 & YES & YES & YES & $1.67$ & $(4,2)$ & NO & 2693\\
$(207,79)$ & 11 & $(207,79)$ & 11 & 207 & YES & YES & YES & $1.60$ & $(2,3)$ & NO & 2694\\
$(207,85)$ & 12 & $(207,85)$ & 12 & 207 & YES & YES & YES & $1.50$ & $(4,2)$ & NO & 2695\\
$(208,79)$ & 11 & $(2,1)$ & 1 & 2 & YES & YES & YES & $1.64$ & $(2,3)$ & -- & 2696\\
$(208,79)$ & 11 & $(3,1)$ & 2 & 1 & YES & YES & YES & $1.75$ & $(2,3)$ & -- & 2697\\
$(208,79)$ & 11 & $(37,14)$ & 8 & 1 & YES & YES & YES & $1.50$ & $(4,2)$ & NO & 2698\\
$(208,61)$ & 12 & $(58,17)$ & 9 & 2 & YES & YES & YES & $1.83$ & $(2,3)$ & 2621 & 2699\\
$(209,80)$ & 11 & $(2,1)$ & 1 & 1 & YES & YES & YES & $1.70$ & $(2,3)$ & -- & 2700\\
$(209,80)$ & 11 & $(3,1)$ & 2 & 1 & YES & YES & YES & $1.60$ & $(2,3)$ & -- & 2701\\
$(209,81)$ & 11 & $(5,2)$ & 3 & 1 & YES & YES & YES & $1.56$ & $(4,2)$ & -- & 2702\\
$(209,81)$ & 11 & $(13,5)$ & 5 & 1 & YES & YES & YES & $1.50$ & $(4,2)$ & NO & 2703\\
$(209,80)$ & 11 & $(21,8)$ & 6 & 1 & YES & YES & YES & $1.70$ & $(2,3)$ & NO & 2704\\
$(209,80)$ & 11 & $(34,13)$ & 7 & 1 & YES & YES & YES & $1.70$ & $(2,3)$ & NO & 2705\\
$(211,89)$ & 12 & $(2,1)$ & 1 & 1 & YES & YES & YES & $1.71$ & $(2,3)$ & -- & 2706\\
$(211,78)$ & 12 & $(46,17)$ & 8 & 1 & YES & YES & YES & $1.50$ & $(6,1)$ & NO & 2707\\
$(212,81)$ & 11 & $(2,1)$ & 1 & 2 & YES & YES & YES & $1.55$ & $(2,3)$ & -- & 2708\\
$(212,93)$ & 12 & $(2,1)$ & 1 & 2 & YES & YES & YES & $1.57$ & $(2,3)$ & -- & 2709\\
$(212,81)$ & 11 & $(3,1)$ & 2 & 1 & YES & YES & YES & $1.60$ & $(2,3)$ & -- & 2710\\
$(212,81)$ & 11 & $(3,1)$ & 2 & 1 & YES & YES & YES & $1.62$ & $(2,3)$ & NO & 2711\\
$(212,81)$ & 11 & $(4,1)$ & 3 & 4 & YES & YES & YES & $1.56$ & $(6,1)$ & NO & 2712\\
$(212,81)$ & 11 & $(4,1)$ & 3 & 4 & YES & YES & YES & $1.56$ & $(6,1)$ & -- & 2713\\
$(212,93)$ & 12 & $(4,1)$ & 3 & 4 & YES & YES & YES & $1.43$ & $(4,2)$ & -- & 2714\\
$(212,89)$ & 11 & $(5,2)$ & 3 & 1 & YES & YES & YES & $1.70$ & $(2,3)$ & -- & 2715\\
$(212,81)$ & 11 & $(7,3)$ & 4 & 1 & YES & YES & YES & $1.67$ & $(4,2)$ & NO & 2716\\
$(212,93)$ & 12 & $(7,3)$ & 4 & 1 & YES & YES & YES & $1.43$ & $(2,3)$ & NO & 2717\\
$(212,81)$ & 11 & $(21,8)$ & 6 & 1 & YES & YES & YES & $1.56$ & $(6,1)$ & 2276 & 2718\\
$(212,63)$ & 13 & $(27,8)$ & 7 & 1 & YES & YES & YES & $1.57$ & $(2,3)$ & NO & 2719\\
$(212,89)$ & 11 & $(112,47)$ & 10 & 4 & YES & YES & YES & $1.60$ & $(2,3)$ & NO & 2720\\
$(212,81)$ & 11 & $(123,47)$ & 10 & 1 & YES & YES & YES & $1.60$ & $(2,3)$ & NO & 2721\\
$(212,81)$ & 11 & $(212,81)$ & 11 & 212 & YES & YES & YES & $1.60$ & $(2,3)$ & NO & 2722\\
$(213,59)$ & 12 & $(3,1)$ & 2 & 3 & YES & YES & YES & $1.62$ & $(4,2)$ & -- & 2723\\
$(213,65)$ & 12 & $(3,1)$ & 2 & 3 & YES & YES & YES & $1.75$ & $(2,3)$ & NO & 2724\\
$(213,65)$ & 12 & $(3,1)$ & 2 & 3 & YES & YES & YES & $1.75$ & $(2,3)$ & -- & 2725\\
$(213,59)$ & 12 & $(5,2)$ & 3 & 1 & YES & YES & YES & $1.80$ & $(2,3)$ & NO & 2726\\
$(213,62)$ & 12 & $(7,2)$ & 4 & 1 & YES & YES & YES & $1.43$ & $(2,3)$ & NO & 2727\\
$(213,59)$ & 12 & $(10,3)$ & 5 & 1 & YES & YES & YES & $1.50$ & $(4,2)$ & NO & 2728\\
$(213,88)$ & 12 & $(29,12)$ & 7 & 1 & YES & YES & YES & $1.62$ & $(4,2)$ & NO & 2729\\
$(213,65)$ & 12 & $(36,11)$ & 8 & 3 & YES & YES & YES & $1.75$ & $(2,3)$ & NO & 2730\\
$(213,88)$ & 12 & $(167,69)$ & 11 & 1 & YES & YES & YES & $1.50$ & $(4,2)$ & NO & 2731\\
$(214,79)$ & 12 & $(3,1)$ & 2 & 1 & YES & YES & YES & $1.71$ & $(2,3)$ & -- & 2732\\
$(214,79)$ & 12 & $(4,1)$ & 3 & 2 & YES & YES & YES & $1.56$ & $(4,2)$ & -- & 2733\\
$(214,79)$ & 12 & $(27,10)$ & 7 & 1 & YES & YES & YES & $1.71$ & $(2,3)$ & NO & 2734\\
$(214,79)$ & 12 & $(46,17)$ & 8 & 2 & YES & YES & YES & $1.67$ & $(4,2)$ & NO & 2735\\
$(215,83)$ & 12 & $(4,1)$ & 3 & 1 & YES & YES & YES & $1.50$ & $(4,2)$ & NO & 2736\\
$(215,83)$ & 12 & $(4,1)$ & 3 & 1 & YES & YES & YES & $1.50$ & $(4,2)$ & -- & 2737\\
$(215,82)$ & 12 & $(6,1)$ & 5 & 1 & YES & YES & YES & $1.50$ & $(6,1)$ & -- & 2738\\
$(215,63)$ & 12 & $(7,2)$ & 4 & 1 & YES & YES & YES & $1.50$ & $(4,2)$ & NO & 2739\\
$(215,63)$ & 12 & $(11,2)$ & 6 & 1 & YES & YES & YES & $1.56$ & $(4,2)$ & NO & 2740\\
$(215,79)$ & 12 & $(11,4)$ & 5 & 1 & YES & YES & YES & $1.50$ & $(4,2)$ & NO & 2741\\
$(215,51)$ & 13 & $(17,4)$ & 7 & 1 & YES & YES & YES & $1.57$ & $(2,3)$ & NO & 2742\\
$(215,63)$ & 12 & $(24,7)$ & 7 & 1 & YES & YES & YES & $1.67$ & $(2,3)$ & NO & 2743\\
$(215,83)$ & 12 & $(31,12)$ & 7 & 1 & YES & YES & YES & $1.50$ & $(4,2)$ & NO & 2744\\
$(215,82)$ & 12 & $(97,37)$ & 10 & 1 & YES & YES & YES & $1.62$ & $(6,1)$ & NO & 2745\\
$(215,58)$ & 12 & $(100,27)$ & 10 & 5 & YES & YES & YES & $1.67$ & $(4,2)$ & NO & 2746\\
$(215,83)$ & 12 & $(101,39)$ & 10 & 1 & YES & YES & YES & $1.38$ & $(4,2)$ & 2948 & 2747\\
$(217,60)$ & 12 & $(2,1)$ & 1 & 1 & YES & YES & YES & $1.75$ & $(2,3)$ & -- & 2748\\
$(217,60)$ & 12 & $(5,2)$ & 3 & 1 & YES & YES & YES & $1.50$ & $(4,2)$ & NO & 2749\\
$(217,60)$ & 12 & $(5,2)$ & 3 & 1 & YES & YES & YES & $1.50$ & $(4,2)$ & -- & 2750\\
$(217,60)$ & 12 & $(10,3)$ & 5 & 1 & YES & YES & YES & $1.50$ & $(4,2)$ & NO & 2751\\
$(217,78)$ & 12 & $(39,14)$ & 8 & 1 & YES & YES & YES & $1.57$ & $(2,3)$ & NO & 2752\\
$(217,90)$ & 13 & $(217,90)$ & 13 & 217 & YES & YES & YES & $1.29$ & $(6,1)$ & NO & 2753\\
$(218,49)$ & 13 & $(3,1)$ & 2 & 1 & YES & YES & YES & $1.43$ & $(2,3)$ & NO & 2754\\
$(218,85)$ & 12 & $(3,1)$ & 2 & 1 & YES & YES & YES & $1.56$ & $(4,2)$ & -- & 2755\\
$(218,85)$ & 12 & $(100,39)$ & 10 & 2 & YES & YES & YES & $1.44$ & $(4,2)$ & 2949 & 2756\\
$(218,85)$ & 12 & $(218,85)$ & 12 & 218 & YES & YES & YES & $1.67$ & $(4,2)$ & NO & 2757\\
$(219,79)$ & 12 & $(2,1)$ & 1 & 1 & YES & YES & YES & $1.43$ & $(6,1)$ & -- & 2758\\
$(219,64)$ & 12 & $(3,1)$ & 2 & 3 & YES & YES & YES & $1.67$ & $(4,2)$ & -- & 2759\\
$(219,79)$ & 12 & $(3,1)$ & 2 & 3 & YES & YES & YES & $1.57$ & $(4,2)$ & -- & 2760\\
$(219,79)$ & 12 & $(4,1)$ & 3 & 1 & YES & YES & YES & $1.50$ & $(6,1)$ & NO & 2761\\
$(219,85)$ & 12 & $(4,1)$ & 3 & 1 & YES & YES & YES & $1.29$ & $(4,2)$ & NO & 2762\\
$(219,61)$ & 12 & $(5,2)$ & 3 & 1 & YES & YES & YES & $1.67$ & $(4,2)$ & NO & 2763\\
$(219,65)$ & 12 & $(5,2)$ & 3 & 1 & YES & YES & YES & $1.56$ & $(4,2)$ & NO & 2764\\
$(219,65)$ & 12 & $(5,2)$ & 3 & 1 & YES & YES & YES & $1.56$ & $(4,2)$ & -- & 2765\\
$(219,65)$ & 12 & $(11,3)$ & 5 & 1 & YES & YES & YES & $1.56$ & $(4,2)$ & 3212 & 2766\\
$(219,79)$ & 12 & $(14,5)$ & 6 & 1 & YES & YES & YES & $1.50$ & $(6,1)$ & NO & 2767\\
$(219,79)$ & 12 & $(25,9)$ & 7 & 1 & YES & YES & YES & $1.43$ & $(6,1)$ & NO & 2768\\
$(219,64)$ & 12 & $(41,12)$ & 8 & 1 & YES & YES & YES & $1.67$ & $(2,3)$ & NO & 2769\\
$(221,84)$ & 12 & $(3,1)$ & 2 & 1 & YES & YES & YES & $1.50$ & $(4,2)$ & NO & 2770\\
$(221,84)$ & 12 & $(8,3)$ & 4 & 1 & YES & YES & YES & $1.71$ & $(2,3)$ & NO & 2771\\
$(222,65)$ & 13 & $(2,1)$ & 1 & 2 & YES & YES & YES & $1.75$ & $(4,2)$ & NO & 2772\\
$(222,65)$ & 13 & $(24,7)$ & 7 & 6 & YES & YES & YES & $1.57$ & $(4,2)$ & NO & 2773\\
$(222,85)$ & 12 & $(34,13)$ & 7 & 2 & YES & YES & YES & $1.43$ & $(4,2)$ & NO & 2774\\
$(225,98)$ & 12 & $(3,1)$ & 2 & 3 & YES & YES & YES & $1.50$ & $(6,1)$ & -- & 2775\\
$(226,83)$ & 12 & $(2,1)$ & 1 & 2 & YES & YES & YES & $1.57$ & $(2,3)$ & NO & 2776\\
$(226,63)$ & 12 & $(3,1)$ & 2 & 1 & YES & YES & YES & $1.67$ & $(4,2)$ & -- & 2777\\
$(226,61)$ & 12 & $(5,2)$ & 3 & 1 & YES & YES & YES & $1.67$ & $(4,2)$ & -- & 2778\\
$(226,69)$ & 12 & $(17,5)$ & 6 & 1 & YES & YES & YES & $1.70$ & $(2,3)$ & NO & 2779\\
$(227,66)$ & 12 & $(2,1)$ & 1 & 1 & YES & YES & YES & $1.75$ & $(2,3)$ & -- & 2780\\
$(227,86)$ & 12 & $(2,1)$ & 1 & 1 & YES & YES & YES & $1.71$ & $(2,3)$ & -- & 2781\\
$(227,94)$ & 12 & $(2,1)$ & 1 & 1 & YES & YES & YES & $1.75$ & $(2,3)$ & -- & 2782\\
$(227,86)$ & 12 & $(3,1)$ & 2 & 1 & YES & YES & YES & $1.62$ & $(4,2)$ & -- & 2783\\
$(227,86)$ & 12 & $(3,1)$ & 2 & 1 & YES & YES & YES & $1.75$ & $(4,2)$ & NO & 2784\\
$(227,86)$ & 12 & $(4,1)$ & 3 & 1 & YES & YES & YES & $1.57$ & $(4,2)$ & NO & 2785\\
$(227,52)$ & 13 & $(5,1)$ & 4 & 1 & YES & YES & YES & $1.50$ & $(4,2)$ & NO & 2786\\
$(227,52)$ & 13 & $(5,2)$ & 3 & 1 & YES & YES & YES & $1.67$ & $(4,2)$ & -- & 2787\\
$(227,88)$ & 12 & $(5,2)$ & 3 & 1 & YES & YES & YES & $1.57$ & $(2,3)$ & NO & 2788\\
$(227,86)$ & 12 & $(13,5)$ & 5 & 1 & YES & YES & YES & $1.57$ & $(4,2)$ & NO & 2789\\
$(227,86)$ & 12 & $(66,25)$ & 9 & 1 & YES & YES & YES & $1.71$ & $(2,3)$ & NO & 2790\\
$(227,86)$ & 12 & $(95,36)$ & 10 & 1 & YES & YES & YES & $1.62$ & $(4,2)$ & 2925 & 2791\\
$(227,94)$ & 12 & $(99,41)$ & 10 & 1 & YES & YES & YES & $1.75$ & $(2,3)$ & NO & 2792\\
$(227,86)$ & 12 & $(227,86)$ & 12 & 227 & YES & YES & YES & $1.62$ & $(4,2)$ & NO & 2793\\
$(229,95)$ & 12 & $(2,1)$ & 1 & 1 & YES & YES & YES & $1.78$ & $(4,2)$ & -- & 2794\\
$(229,95)$ & 12 & $(3,1)$ & 2 & 1 & YES & YES & YES & $1.56$ & $(4,2)$ & NO & 2795\\
$(229,63)$ & 13 & $(4,1)$ & 3 & 1 & YES & YES & YES & $1.71$ & $(2,3)$ & -- & 2796\\
$(229,64)$ & 12 & $(5,2)$ & 3 & 1 & YES & YES & YES & $1.67$ & $(4,2)$ & NO & 2797\\
$(229,63)$ & 13 & $(29,8)$ & 7 & 1 & YES & YES & YES & $1.71$ & $(2,3)$ & NO & 2798\\
$(231,83)$ & 12 & $(3,1)$ & 2 & 3 & YES & YES & YES & $1.62$ & $(6,1)$ & -- & 2799\\
$(231,83)$ & 12 & $(11,4)$ & 5 & 11 & YES & YES & YES & $1.50$ & $(6,1)$ & NO & 2800\\
$(233,89)$ & 11 & $(2,1)$ & 1 & 1 & YES & YES & YES & $1.67$ & $(2,3)$ & -- & 2801\\
$(233,89)$ & 11 & $(3,1)$ & 2 & 1 & YES & YES & YES & $1.60$ & $(2,3)$ & -- & 2802\\
$(233,89)$ & 11 & $(13,5)$ & 5 & 1 & YES & YES & YES & $1.70$ & $(2,3)$ & NO & 2803\\
$(233,89)$ & 11 & $(21,8)$ & 6 & 1 & YES & YES & YES & $1.60$ & $(2,3)$ & 2689 & 2804\\
$(233,89)$ & 11 & $(55,21)$ & 8 & 1 & YES & YES & YES & $1.70$ & $(2,3)$ & 2665 & 2805\\
$(234,43)$ & 14 & $(2,1)$ & 1 & 2 & YES & YES & YES & $1.29$ & $(2,3)$ & -- & 2806\\
$(234,71)$ & 12 & $(2,1)$ & 1 & 2 & YES & YES & YES & $1.73$ & $(4,2)$ & -- & 2807\\
$(234,53)$ & 13 & $(5,2)$ & 3 & 1 & YES & YES & YES & $1.56$ & $(4,2)$ & -- & 2808\\
$(234,43)$ & 14 & $(6,1)$ & 5 & 6 & YES & YES & YES & $1.29$ & $(2,3)$ & NO & 2809\\
$(234,53)$ & 13 & $(35,8)$ & 8 & 1 & YES & YES & YES & $1.56$ & $(4,2)$ & 2890 & 2810\\
$(234,71)$ & 12 & $(79,24)$ & 10 & 1 & YES & YES & YES & $1.56$ & $(4,2)$ & NO & 2811\\
$(235,66)$ & 12 & $(2,1)$ & 1 & 1 & YES & YES & YES & $1.75$ & $(2,3)$ & -- & 2812\\
$(235,97)$ & 12 & $(2,1)$ & 1 & 1 & YES & YES & YES & $1.57$ & $(2,3)$ & NO & 2813\\
$(236,69)$ & 12 & $(2,1)$ & 1 & 2 & YES & YES & YES & $1.60$ & $(2,3)$ & -- & 2814\\
$(236,69)$ & 12 & $(3,1)$ & 2 & 1 & YES & YES & YES & $1.70$ & $(4,2)$ & NO & 2815\\
$(236,69)$ & 12 & $(3,1)$ & 2 & 1 & YES & YES & YES & $1.70$ & $(4,2)$ & -- & 2816\\
$(236,69)$ & 12 & $(5,1)$ & 4 & 1 & YES & YES & YES & $1.70$ & $(2,3)$ & NO & 2817\\
$(236,69)$ & 12 & $(17,5)$ & 6 & 1 & YES & YES & YES & $1.60$ & $(2,3)$ & NO & 2818\\
$(236,69)$ & 12 & $(41,12)$ & 8 & 1 & YES & YES & YES & $1.60$ & $(2,3)$ & NO & 2819\\
$(237,100)$ & 12 & $(3,1)$ & 2 & 3 & YES & YES & YES & $1.56$ & $(4,2)$ & -- & 2820\\
$(237,64)$ & 12 & $(5,2)$ & 3 & 1 & YES & YES & YES & $1.56$ & $(4,2)$ & NO & 2821\\
$(237,100)$ & 12 & $(109,46)$ & 10 & 1 & YES & YES & YES & $1.44$ & $(4,2)$ & 3046 & 2822\\
$(238,69)$ & 13 & $(2,1)$ & 1 & 2 & YES & YES & YES & $1.62$ & $(6,1)$ & -- & 2823\\
$(238,69)$ & 13 & $(5,1)$ & 4 & 1 & YES & YES & YES & $1.62$ & $(6,1)$ & NO & 2824\\
$(238,69)$ & 13 & $(10,3)$ & 5 & 2 & YES & YES & YES & $1.50$ & $(6,1)$ & NO & 2825\\
$(238,69)$ & 13 & $(31,9)$ & 8 & 1 & YES & YES & YES & $1.62$ & $(6,1)$ & NO & 2826\\
$(239,99)$ & 12 & $(2,1)$ & 1 & 1 & YES & YES & YES & $1.70$ & $(2,3)$ & -- & 2827\\
$(239,70)$ & 12 & $(3,1)$ & 2 & 1 & YES & YES & YES & $1.56$ & $(4,2)$ & NO & 2828\\
$(239,70)$ & 12 & $(3,1)$ & 2 & 1 & YES & YES & YES & $1.56$ & $(4,2)$ & -- & 2829\\
$(239,99)$ & 12 & $(3,1)$ & 2 & 1 & YES & YES & YES & $1.75$ & $(4,2)$ & -- & 2830\\
$(239,99)$ & 12 & $(3,1)$ & 2 & 1 & YES & YES & YES & $1.43$ & $(4,2)$ & 2519 & 2831\\
$(239,101)$ & 12 & $(5,2)$ & 3 & 1 & YES & YES & YES & $1.43$ & $(4,2)$ & NO & 2832\\
$(239,99)$ & 12 & $(7,3)$ & 4 & 1 & YES & YES & YES & $1.62$ & $(4,2)$ & NO & 2833\\
$(239,101)$ & 12 & $(12,5)$ & 5 & 1 & YES & YES & YES & $1.43$ & $(4,2)$ & NO & 2834\\
$(239,70)$ & 12 & $(13,4)$ & 6 & 1 & YES & YES & YES & $1.44$ & $(4,2)$ & NO & 2835\\
$(239,99)$ & 12 & $(17,7)$ & 6 & 1 & YES & YES & YES & $1.62$ & $(4,2)$ & 2489 & 2836\\
$(239,67)$ & 13 & $(18,5)$ & 6 & 1 & YES & YES & YES & $1.43$ & $(4,2)$ & NO & 2837\\
$(239,71)$ & 12 & $(24,7)$ & 7 & 1 & YES & YES & YES & $1.70$ & $(2,3)$ & NO & 2838\\
$(239,99)$ & 12 & $(41,17)$ & 8 & 1 & YES & YES & YES & $1.62$ & $(4,2)$ & NO & 2839\\
$(239,99)$ & 12 & $(239,99)$ & 12 & 239 & YES & YES & YES & $1.62$ & $(4,2)$ & NO & 2840\\
$(240,71)$ & 12 & $(3,1)$ & 2 & 3 & NO & YES & YES & $1.43$ & $(4,2)$ & -- & 2841\\
$(240,71)$ & 12 & $(44,13)$ & 8 & 4 & YES & YES & YES & $1.70$ & $(2,3)$ & NO & 2842\\
$(241,89)$ & 12 & $(2,1)$ & 1 & 1 & YES & YES & YES & $1.80$ & $(2,3)$ & -- & 2843\\
$(241,101)$ & 12 & $(3,1)$ & 2 & 1 & YES & YES & YES & $1.50$ & $(4,2)$ & -- & 2844\\
$(241,94)$ & 12 & $(8,3)$ & 4 & 1 & YES & YES & YES & $1.62$ & $(4,2)$ & NO & 2845\\
$(241,94)$ & 12 & $(13,5)$ & 5 & 1 & YES & YES & YES & $1.62$ & $(4,2)$ & NO & 2846\\
$(241,89)$ & 12 & $(46,17)$ & 8 & 1 & YES & YES & YES & $1.70$ & $(2,3)$ & NO & 2847\\
$(241,89)$ & 12 & $(111,41)$ & 10 & 1 & YES & YES & YES & $1.50$ & $(4,2)$ & 3069 & 2848\\
$(241,89)$ & 12 & $(176,65)$ & 11 & 1 & YES & YES & YES & $1.67$ & $(4,2)$ & NO & 2849\\
$(242,45)$ & 14 & $(2,1)$ & 1 & 2 & YES & YES & YES & $1.43$ & $(2,3)$ & -- & 2850\\
$(242,71)$ & 13 & $(2,1)$ & 1 & 2 & YES & YES & YES & $1.29$ & $(6,1)$ & -- & 2851\\
$(242,71)$ & 13 & $(2,1)$ & 1 & 2 & YES & YES & YES & $1.29$ & $(6,1)$ & NO & 2852\\
$(242,71)$ & 13 & $(4,1)$ & 3 & 2 & YES & YES & YES & $1.29$ & $(6,1)$ & NO & 2853\\
$(242,65)$ & 12 & $(5,2)$ & 3 & 1 & YES & YES & YES & $1.56$ & $(4,2)$ & NO & 2854\\
$(242,71)$ & 13 & $(10,3)$ & 5 & 2 & YES & YES & YES & $1.50$ & $(6,1)$ & NO & 2855\\
$(242,71)$ & 13 & $(92,27)$ & 11 & 2 & YES & YES & YES & $1.62$ & $(6,1)$ & 2950 & 2856\\
$(243,71)$ & 12 & $(2,1)$ & 1 & 1 & YES & YES & YES & $1.56$ & $(2,3)$ & -- & 2857\\
$(243,94)$ & 12 & $(2,1)$ & 1 & 1 & YES & YES & YES & $1.70$ & $(2,3)$ & -- & 2858\\
$(243,94)$ & 12 & $(3,1)$ & 2 & 3 & YES & YES & YES & $1.56$ & $(4,2)$ & NO & 2859\\
$(243,53)$ & 13 & $(19,4)$ & 7 & 1 & YES & YES & YES & $1.56$ & $(4,2)$ & NO & 2860\\
$(243,53)$ & 13 & $(37,8)$ & 8 & 1 & YES & YES & YES & $1.56$ & $(4,2)$ & 2967 & 2861\\
$(243,71)$ & 12 & $(89,26)$ & 10 & 1 & YES & YES & YES & $1.56$ & $(2,3)$ & NO & 2862\\
$(243,94)$ & 12 & $(106,41)$ & 10 & 1 & YES & YES & YES & $1.60$ & $(2,3)$ & NO & 2863\\
$(243,94)$ & 12 & $(243,94)$ & 12 & 243 & YES & YES & YES & $1.50$ & $(4,2)$ & NO & 2864\\
$(245,69)$ & 13 & $(2,1)$ & 1 & 1 & YES & YES & YES & $1.62$ & $(6,1)$ & -- & 2865\\
$(245,69)$ & 13 & $(4,1)$ & 3 & 1 & YES & YES & YES & $1.62$ & $(6,1)$ & NO & 2866\\
$(245,69)$ & 13 & $(5,1)$ & 4 & 5 & YES & YES & YES & $1.62$ & $(6,1)$ & NO & 2867\\
$(245,69)$ & 13 & $(32,9)$ & 8 & 1 & YES & YES & YES & $1.62$ & $(6,1)$ & NO & 2868\\
$(245,69)$ & 13 & $(103,29)$ & 11 & 1 & YES & YES & YES & $1.62$ & $(6,1)$ & 3031 & 2869\\
$(246,73)$ & 12 & $(2,1)$ & 1 & 2 & YES & YES & YES & $1.67$ & $(2,3)$ & NO & 2870\\
$(246,91)$ & 12 & $(2,1)$ & 1 & 2 & YES & YES & YES & $1.44$ & $(4,2)$ & -- & 2871\\
$(246,91)$ & 12 & $(2,1)$ & 1 & 2 & YES & YES & YES & $1.56$ & $(4,2)$ & NO & 2872\\
$(246,95)$ & 12 & $(2,1)$ & 1 & 2 & YES & YES & YES & $1.56$ & $(4,2)$ & -- & 2873\\
$(246,91)$ & 12 & $(3,1)$ & 2 & 3 & YES & YES & YES & $1.70$ & $(2,3)$ & -- & 2874\\
$(246,95)$ & 12 & $(3,1)$ & 2 & 3 & YES & YES & YES & $1.67$ & $(4,2)$ & -- & 2875\\
$(246,95)$ & 12 & $(3,1)$ & 2 & 3 & YES & YES & YES & $1.56$ & $(4,2)$ & NO & 2876\\
$(246,101)$ & 12 & $(3,1)$ & 2 & 3 & YES & YES & YES & $1.50$ & $(4,2)$ & NO & 2877\\
$(246,91)$ & 12 & $(11,4)$ & 5 & 1 & YES & YES & YES & $1.50$ & $(4,2)$ & NO & 2878\\
$(246,91)$ & 12 & $(19,7)$ & 6 & 1 & YES & YES & YES & $1.70$ & $(2,3)$ & NO & 2879\\
$(246,101)$ & 12 & $(39,16)$ & 8 & 3 & YES & YES & YES & $1.62$ & $(4,2)$ & NO & 2880\\
$(246,95)$ & 12 & $(57,22)$ & 9 & 3 & YES & YES & YES & $1.38$ & $(4,2)$ & NO & 2881\\
$(246,95)$ & 12 & $(101,39)$ & 10 & 1 & YES & YES & YES & $1.60$ & $(2,3)$ & NO & 2882\\
$(246,91)$ & 12 & $(173,64)$ & 11 & 1 & YES & YES & YES & $1.56$ & $(4,2)$ & NO & 2883\\
$(247,69)$ & 12 & $(2,1)$ & 1 & 1 & YES & YES & YES & $1.60$ & $(2,3)$ & -- & 2884\\
$(247,69)$ & 12 & $(3,1)$ & 2 & 1 & YES & YES & YES & $1.70$ & $(2,3)$ & -- & 2885\\
$(247,69)$ & 12 & $(5,1)$ & 4 & 1 & YES & YES & YES & $1.70$ & $(2,3)$ & NO & 2886\\
$(247,69)$ & 12 & $(18,5)$ & 6 & 1 & YES & YES & YES & $1.60$ & $(2,3)$ & NO & 2887\\
$(249,95)$ & 12 & $(3,1)$ & 2 & 3 & YES & YES & YES & $1.67$ & $(2,3)$ & -- & 2888\\
$(249,58)$ & 13 & $(5,2)$ & 3 & 1 & YES & YES & YES & $1.44$ & $(4,2)$ & -- & 2889\\
$(250,57)$ & 13 & $(31,7)$ & 8 & 1 & YES & YES & YES & $1.56$ & $(4,2)$ & 2810 & 2890\\
$(251,74)$ & 13 & $(2,1)$ & 1 & 1 & YES & YES & YES & $1.62$ & $(6,1)$ & -- & 2891\\
$(251,104)$ & 12 & $(2,1)$ & 1 & 1 & YES & YES & YES & $1.62$ & $(4,2)$ & -- & 2892\\
$(251,74)$ & 13 & $(4,1)$ & 3 & 1 & YES & YES & YES & $1.57$ & $(4,2)$ & NO & 2893\\
$(251,46)$ & 15 & $(5,1)$ & 4 & 1 & YES & YES & YES & $1.50$ & $(2,3)$ & NO & 2894\\
$(251,104)$ & 12 & $(7,3)$ & 4 & 1 & YES & YES & YES & $1.78$ & $(4,2)$ & NO & 2895\\
$(251,74)$ & 13 & $(10,3)$ & 5 & 1 & YES & YES & YES & $1.57$ & $(2,3)$ & NO & 2896\\
$(251,104)$ & 12 & $(29,12)$ & 7 & 1 & YES & YES & YES & $1.75$ & $(2,3)$ & NO & 2897\\
$(251,74)$ & 13 & $(44,13)$ & 8 & 1 & YES & YES & YES & $1.43$ & $(4,2)$ & NO & 2898\\
$(253,60)$ & 13 & $(2,1)$ & 1 & 1 & YES & YES & YES & $1.57$ & $(2,3)$ & -- & 2899\\
$(253,106)$ & 12 & $(2,1)$ & 1 & 1 & YES & YES & YES & $1.43$ & $(4,2)$ & -- & 2900\\
$(253,98)$ & 12 & $(3,1)$ & 2 & 1 & YES & YES & YES & $1.50$ & $(4,2)$ & NO & 2901\\
$(253,106)$ & 12 & $(5,2)$ & 3 & 1 & YES & YES & YES & $1.38$ & $(6,1)$ & NO & 2902\\
$(253,106)$ & 12 & $(105,44)$ & 10 & 1 & YES & YES & YES & $1.70$ & $(2,3)$ & 3057 & 2903\\
$(254,105)$ & 12 & $(2,1)$ & 1 & 2 & YES & YES & YES & $1.29$ & $(4,2)$ & -- & 2904\\
$(254,75)$ & 12 & $(3,1)$ & 2 & 1 & YES & YES & YES & $1.60$ & $(2,3)$ & -- & 2905\\
$(254,71)$ & 12 & $(10,3)$ & 5 & 2 & YES & YES & YES & $1.67$ & $(4,2)$ & NO & 2906\\
$(254,75)$ & 12 & $(13,4)$ & 6 & 1 & YES & YES & YES & $1.56$ & $(4,2)$ & NO & 2907\\
$(254,75)$ & 12 & $(27,8)$ & 7 & 1 & YES & YES & YES & $1.60$ & $(2,3)$ & NO & 2908\\
$(255,71)$ & 13 & $(3,1)$ & 2 & 3 & YES & YES & YES & $1.29$ & $(6,1)$ & NO & 2909\\
$(255,71)$ & 13 & $(11,3)$ & 5 & 1 & YES & YES & YES & $1.62$ & $(6,1)$ & NO & 2910\\
$(255,71)$ & 13 & $(97,27)$ & 11 & 1 & YES & YES & YES & $1.50$ & $(6,1)$ & 3008 & 2911\\
$(255,76)$ & 13 & $(104,31)$ & 11 & 1 & YES & YES & YES & $1.57$ & $(2,3)$ & NO & 2912\\
$(256,75)$ & 12 & $(2,1)$ & 1 & 2 & YES & YES & YES & $1.44$ & $(4,2)$ & -- & 2913\\
$(256,75)$ & 12 & $(2,1)$ & 1 & 2 & YES & YES & YES & $1.56$ & $(4,2)$ & NO & 2914\\
$(256,99)$ & 12 & $(2,1)$ & 1 & 2 & YES & YES & YES & $1.70$ & $(2,3)$ & -- & 2915\\
$(256,75)$ & 12 & $(3,1)$ & 2 & 1 & YES & YES & YES & $1.44$ & $(4,2)$ & -- & 2916\\
$(256,97)$ & 12 & $(3,1)$ & 2 & 1 & YES & YES & YES & $1.67$ & $(4,2)$ & -- & 2917\\
$(256,99)$ & 12 & $(3,1)$ & 2 & 1 & YES & YES & YES & $1.56$ & $(4,2)$ & -- & 2918\\
$(256,99)$ & 12 & $(3,1)$ & 2 & 1 & YES & YES & YES & $1.67$ & $(4,2)$ & NO & 2919\\
$(256,99)$ & 12 & $(3,1)$ & 2 & 1 & YES & YES & YES & $1.70$ & $(2,3)$ & NO & 2920\\
$(256,99)$ & 12 & $(4,1)$ & 3 & 4 & YES & YES & YES & $1.56$ & $(4,2)$ & -- & 2921\\
$(256,99)$ & 12 & $(4,1)$ & 3 & 4 & YES & YES & YES & $1.56$ & $(4,2)$ & NO & 2922\\
$(256,75)$ & 12 & $(24,7)$ & 7 & 8 & YES & YES & YES & $1.60$ & $(2,3)$ & NO & 2923\\
$(256,99)$ & 12 & $(31,12)$ & 7 & 1 & YES & YES & YES & $1.70$ & $(2,3)$ & NO & 2924\\
$(256,97)$ & 12 & $(66,25)$ & 9 & 2 & YES & YES & YES & $1.62$ & $(4,2)$ & 2791 & 2925\\
$(256,99)$ & 12 & $(75,29)$ & 9 & 1 & YES & YES & YES & $1.70$ & $(2,3)$ & NO & 2926\\
$(256,75)$ & 12 & $(99,29)$ & 10 & 1 & YES & YES & YES & $1.60$ & $(2,3)$ & NO & 2927\\
$(256,99)$ & 12 & $(106,41)$ & 10 & 2 & YES & YES & YES & $1.56$ & $(4,2)$ & 3070 & 2928\\
$(256,99)$ & 12 & $(181,70)$ & 11 & 1 & YES & YES & YES & $1.67$ & $(4,2)$ & NO & 2929\\
$(256,99)$ & 12 & $(256,99)$ & 12 & 256 & YES & YES & YES & $1.56$ & $(4,2)$ & NO & 2930\\
$(257,108)$ & 12 & $(2,1)$ & 1 & 1 & YES & YES & YES & $1.80$ & $(2,3)$ & -- & 2931\\
$(257,76)$ & 12 & $(3,1)$ & 2 & 1 & YES & YES & YES & $1.70$ & $(2,3)$ & -- & 2932\\
$(257,108)$ & 12 & $(3,1)$ & 2 & 1 & YES & YES & YES & $1.80$ & $(2,3)$ & NO & 2933\\
$(257,108)$ & 12 & $(3,1)$ & 2 & 1 & YES & YES & YES & $1.80$ & $(2,3)$ & -- & 2934\\
$(257,76)$ & 12 & $(7,2)$ & 4 & 1 & YES & YES & YES & $1.70$ & $(2,3)$ & NO & 2935\\
$(257,76)$ & 12 & $(17,5)$ & 6 & 1 & YES & YES & YES & $1.70$ & $(2,3)$ & NO & 2936\\
$(257,108)$ & 12 & $(50,21)$ & 8 & 1 & YES & YES & YES & $1.80$ & $(2,3)$ & NO & 2937\\
$(257,59)$ & 14 & $(74,17)$ & 11 & 1 & YES & YES & YES & $1.57$ & $(4,2)$ & NO & 2938\\
$(257,108)$ & 12 & $(119,50)$ & 10 & 1 & YES & YES & YES & $1.70$ & $(2,3)$ & 3136 & 2939\\
$(258,109)$ & 12 & $(4,1)$ & 3 & 2 & YES & YES & YES & $1.62$ & $(4,2)$ & -- & 2940\\
$(258,109)$ & 12 & $(4,1)$ & 3 & 2 & YES & YES & YES & $1.62$ & $(4,2)$ & NO & 2941\\
$(258,109)$ & 12 & $(116,49)$ & 10 & 2 & YES & YES & YES & $1.56$ & $(4,2)$ & 3125 & 2942\\
$(259,76)$ & 13 & $(2,1)$ & 1 & 1 & YES & YES & YES & $1.62$ & $(6,1)$ & -- & 2943\\
$(259,100)$ & 12 & $(3,1)$ & 2 & 1 & YES & YES & YES & $1.67$ & $(4,2)$ & NO & 2944\\
$(259,100)$ & 12 & $(3,1)$ & 2 & 1 & YES & YES & YES & $1.67$ & $(4,2)$ & -- & 2945\\
$(259,100)$ & 12 & $(3,1)$ & 2 & 1 & YES & YES & YES & $1.78$ & $(4,2)$ & NO & 2946\\
$(259,76)$ & 13 & $(4,1)$ & 3 & 1 & YES & YES & YES & $1.50$ & $(6,1)$ & NO & 2947\\
$(259,100)$ & 12 & $(57,22)$ & 9 & 1 & YES & YES & YES & $1.38$ & $(4,2)$ & 2747 & 2948\\
$(259,101)$ & 12 & $(59,23)$ & 9 & 1 & YES & YES & YES & $1.44$ & $(4,2)$ & 2756 & 2949\\
$(259,76)$ & 13 & $(75,22)$ & 10 & 1 & YES & YES & YES & $1.62$ & $(6,1)$ & 2856 & 2950\\
$(259,101)$ & 12 & $(100,39)$ & 10 & 1 & YES & YES & YES & $1.56$ & $(4,2)$ & NO & 2951\\
$(259,100)$ & 12 & $(158,61)$ & 11 & 1 & YES & YES & YES & $1.67$ & $(4,2)$ & NO & 2952\\
$(259,101)$ & 12 & $(159,62)$ & 11 & 1 & YES & YES & YES & $1.44$ & $(4,2)$ & NO & 2953\\
$(259,101)$ & 12 & $(259,101)$ & 12 & 259 & YES & YES & YES & $1.56$ & $(4,2)$ & NO & 2954\\
$(261,100)$ & 12 & $(2,1)$ & 1 & 1 & YES & YES & YES & $1.67$ & $(4,2)$ & -- & 2955\\
$(261,100)$ & 12 & $(3,1)$ & 2 & 3 & YES & YES & YES & $1.70$ & $(2,3)$ & -- & 2956\\
$(261,100)$ & 12 & $(4,1)$ & 3 & 1 & YES & YES & YES & $1.56$ & $(4,2)$ & -- & 2957\\
$(261,100)$ & 12 & $(60,23)$ & 9 & 3 & YES & YES & YES & $1.70$ & $(2,3)$ & NO & 2958\\
$(261,100)$ & 12 & $(107,41)$ & 10 & 1 & YES & YES & YES & $1.50$ & $(4,2)$ & NO & 2959\\
$(263,78)$ & 13 & $(2,1)$ & 1 & 1 & YES & YES & YES & $1.50$ & $(6,1)$ & -- & 2960\\
$(263,109)$ & 12 & $(2,1)$ & 1 & 1 & YES & YES & YES & $1.50$ & $(4,2)$ & -- & 2961\\
$(263,109)$ & 12 & $(3,1)$ & 2 & 1 & YES & YES & YES & $1.44$ & $(4,2)$ & -- & 2962\\
$(263,111)$ & 12 & $(3,1)$ & 2 & 1 & YES & YES & YES & $1.67$ & $(4,2)$ & -- & 2963\\
$(263,60)$ & 13 & $(5,2)$ & 3 & 1 & YES & YES & YES & $1.67$ & $(4,2)$ & -- & 2964\\
$(263,78)$ & 13 & $(7,2)$ & 4 & 1 & YES & YES & YES & $1.50$ & $(6,1)$ & NO & 2965\\
$(263,109)$ & 12 & $(17,7)$ & 6 & 1 & YES & YES & YES & $1.56$ & $(4,2)$ & NO & 2966\\
$(263,57)$ & 13 & $(32,7)$ & 8 & 1 & YES & YES & YES & $1.56$ & $(4,2)$ & 2861 & 2967\\
$(263,71)$ & 12 & $(89,24)$ & 10 & 1 & YES & YES & YES & $1.67$ & $(4,2)$ & NO & 2968\\
$(263,78)$ & 13 & $(118,35)$ & 11 & 1 & YES & YES & YES & $1.57$ & $(2,3)$ & NO & 2969\\
$(263,71)$ & 12 & $(137,37)$ & 11 & 1 & YES & YES & YES & $1.56$ & $(4,2)$ & NO & 2970\\
$(263,111)$ & 12 & $(263,111)$ & 12 & 263 & YES & YES & YES & $1.56$ & $(4,2)$ & NO & 2971\\
$(264,109)$ & 12 & $(109,45)$ & 10 & 1 & YES & YES & YES & $1.56$ & $(4,2)$ & NO & 2972\\
$(265,98)$ & 12 & $(11,4)$ & 5 & 1 & YES & YES & YES & $1.50$ & $(4,2)$ & NO & 2973\\
$(265,97)$ & 12 & $(112,41)$ & 10 & 1 & YES & YES & YES & $1.67$ & $(4,2)$ & NO & 2974\\
$(266,101)$ & 12 & $(2,1)$ & 1 & 2 & YES & YES & YES & $1.70$ & $(2,3)$ & -- & 2975\\
$(266,101)$ & 12 & $(2,1)$ & 1 & 2 & YES & YES & YES & $1.80$ & $(2,3)$ & NO & 2976\\
$(267,74)$ & 13 & $(2,1)$ & 1 & 1 & YES & YES & YES & $1.50$ & $(6,1)$ & NO & 2977\\
$(267,74)$ & 13 & $(3,1)$ & 2 & 3 & YES & YES & YES & $1.43$ & $(4,2)$ & NO & 2978\\
$(267,98)$ & 12 & $(3,1)$ & 2 & 3 & YES & YES & YES & $1.67$ & $(4,2)$ & -- & 2979\\
$(267,98)$ & 12 & $(8,3)$ & 4 & 1 & YES & YES & YES & $1.67$ & $(4,2)$ & NO & 2980\\
$(267,98)$ & 12 & $(19,7)$ & 6 & 1 & YES & YES & YES & $1.67$ & $(4,2)$ & NO & 2981\\
$(267,98)$ & 12 & $(30,11)$ & 7 & 3 & YES & YES & YES & $1.57$ & $(2,3)$ & NO & 2982\\
$(268,111)$ & 12 & $(2,1)$ & 1 & 2 & YES & YES & YES & $1.50$ & $(4,2)$ & -- & 2983\\
$(268,111)$ & 12 & $(3,1)$ & 2 & 1 & YES & YES & YES & $1.67$ & $(4,2)$ & -- & 2984\\
$(268,111)$ & 12 & $(3,1)$ & 2 & 1 & YES & YES & YES & $1.67$ & $(4,2)$ & NO & 2985\\
$(268,111)$ & 12 & $(4,1)$ & 3 & 4 & YES & YES & YES & $1.44$ & $(4,2)$ & -- & 2986\\
$(268,99)$ & 12 & $(5,2)$ & 3 & 1 & YES & YES & YES & $1.50$ & $(4,2)$ & NO & 2987\\
$(268,111)$ & 12 & $(7,3)$ & 4 & 1 & YES & YES & YES & $1.80$ & $(2,3)$ & NO & 2988\\
$(268,111)$ & 12 & $(17,7)$ & 6 & 1 & YES & YES & YES & $1.62$ & $(4,2)$ & NO & 2989\\
$(268,111)$ & 12 & $(29,12)$ & 7 & 1 & YES & YES & YES & $1.50$ & $(4,2)$ & NO & 2990\\
$(268,111)$ & 12 & $(41,17)$ & 8 & 1 & YES & YES & YES & $1.62$ & $(4,2)$ & 3098 & 2991\\
$(268,111)$ & 12 & $(268,111)$ & 12 & 268 & YES & YES & YES & $1.56$ & $(4,2)$ & NO & 2992\\
$(269,78)$ & 13 & $(2,1)$ & 1 & 1 & YES & YES & YES & $1.62$ & $(6,1)$ & -- & 2993\\
$(269,78)$ & 13 & $(2,1)$ & 1 & 1 & YES & YES & YES & $1.57$ & $(2,3)$ & NO & 2994\\
$(269,104)$ & 12 & $(3,1)$ & 2 & 1 & YES & YES & YES & $1.43$ & $(4,2)$ & NO & 2995\\
$(269,104)$ & 12 & $(3,1)$ & 2 & 1 & YES & YES & YES & $1.50$ & $(4,2)$ & -- & 2996\\
$(269,78)$ & 13 & $(5,1)$ & 4 & 1 & YES & YES & YES & $1.62$ & $(6,1)$ & NO & 2997\\
$(269,104)$ & 12 & $(8,3)$ & 4 & 1 & YES & YES & YES & $1.70$ & $(2,3)$ & NO & 2998\\
$(271,105)$ & 12 & $(3,1)$ & 2 & 1 & YES & YES & YES & $1.56$ & $(4,2)$ & -- & 2999\\
$(271,112)$ & 12 & $(3,1)$ & 2 & 1 & YES & YES & YES & $1.29$ & $(4,2)$ & -- & 3000\\
$(271,112)$ & 12 & $(46,19)$ & 8 & 1 & YES & YES & YES & $1.78$ & $(4,2)$ & NO & 3001\\
$(273,76)$ & 13 & $(2,1)$ & 1 & 1 & YES & YES & YES & $1.50$ & $(6,1)$ & NO & 3002\\
$(273,106)$ & 13 & $(2,1)$ & 1 & 1 & YES & YES & YES & $1.75$ & $(2,3)$ & -- & 3003\\
$(273,76)$ & 13 & $(3,1)$ & 2 & 3 & YES & YES & YES & $1.62$ & $(6,1)$ & NO & 3004\\
$(273,100)$ & 12 & $(5,2)$ & 3 & 1 & YES & YES & YES & $1.56$ & $(4,2)$ & NO & 3005\\
$(273,106)$ & 13 & $(13,5)$ & 5 & 13 & YES & YES & YES & $1.75$ & $(2,3)$ & NO & 3006\\
$(273,80)$ & 13 & $(41,12)$ & 8 & 1 & YES & YES & YES & $1.50$ & $(4,2)$ & NO & 3007\\
$(273,76)$ & 13 & $(79,22)$ & 10 & 1 & YES & YES & YES & $1.50$ & $(6,1)$ & 2911 & 3008\\
$(273,80)$ & 13 & $(99,29)$ & 10 & 3 & YES & YES & YES & $1.56$ & $(4,2)$ & NO & 3009\\
$(273,80)$ & 13 & $(215,63)$ & 12 & 1 & YES & YES & YES & $1.67$ & $(4,2)$ & NO & 3010\\
$(274,81)$ & 12 & $(2,1)$ & 1 & 2 & YES & YES & YES & $1.60$ & $(2,3)$ & -- & 3011\\
$(274,115)$ & 12 & $(2,1)$ & 1 & 2 & YES & YES & YES & $1.70$ & $(2,3)$ & -- & 3012\\
$(274,81)$ & 12 & $(3,1)$ & 2 & 1 & YES & YES & YES & $1.70$ & $(2,3)$ & -- & 3013\\
$(274,81)$ & 12 & $(3,1)$ & 2 & 1 & YES & YES & YES & $1.70$ & $(2,3)$ & NO & 3014\\
$(274,105)$ & 12 & $(3,1)$ & 2 & 1 & YES & YES & YES & $1.70$ & $(2,3)$ & -- & 3015\\
$(274,115)$ & 12 & $(3,1)$ & 2 & 1 & YES & YES & YES & $1.70$ & $(2,3)$ & NO & 3016\\
$(274,115)$ & 12 & $(3,1)$ & 2 & 1 & YES & YES & YES & $1.70$ & $(2,3)$ & -- & 3017\\
$(274,81)$ & 12 & $(11,3)$ & 5 & 1 & YES & YES & YES & $1.60$ & $(2,3)$ & NO & 3018\\
$(274,115)$ & 12 & $(19,8)$ & 6 & 1 & YES & YES & YES & $1.70$ & $(2,3)$ & NO & 3019\\
$(274,81)$ & 12 & $(24,7)$ & 7 & 2 & YES & YES & YES & $1.60$ & $(2,3)$ & NO & 3020\\
$(275,76)$ & 12 & $(2,1)$ & 1 & 1 & YES & YES & YES & $1.60$ & $(2,3)$ & -- & 3021\\
$(275,76)$ & 12 & $(2,1)$ & 1 & 1 & YES & YES & YES & $1.70$ & $(2,3)$ & NO & 3022\\
$(275,76)$ & 12 & $(7,2)$ & 4 & 1 & YES & YES & YES & $1.70$ & $(2,3)$ & NO & 3023\\
$(277,76)$ & 13 & $(2,1)$ & 1 & 1 & YES & YES & YES & $1.57$ & $(2,3)$ & NO & 3024\\
$(277,78)$ & 13 & $(2,1)$ & 1 & 1 & YES & YES & YES & $1.50$ & $(6,1)$ & NO & 3025\\
$(277,81)$ & 12 & $(2,1)$ & 1 & 1 & YES & YES & YES & $1.60$ & $(2,3)$ & -- & 3026\\
$(277,106)$ & 12 & $(3,1)$ & 2 & 1 & YES & YES & YES & $1.50$ & $(4,2)$ & -- & 3027\\
$(277,117)$ & 12 & $(4,1)$ & 3 & 1 & YES & YES & YES & $1.67$ & $(4,2)$ & -- & 3028\\
$(277,60)$ & 13 & $(5,2)$ & 3 & 1 & YES & YES & YES & $1.56$ & $(4,2)$ & -- & 3029\\
$(277,76)$ & 13 & $(7,2)$ & 4 & 1 & YES & YES & YES & $1.71$ & $(2,3)$ & NO & 3030\\
$(277,78)$ & 13 & $(71,20)$ & 10 & 1 & YES & YES & YES & $1.62$ & $(6,1)$ & 2869 & 3031\\
$(277,78)$ & 13 & $(103,29)$ & 11 & 1 & YES & YES & YES & $1.57$ & $(2,3)$ & NO & 3032\\
$(277,81)$ & 12 & $(106,31)$ & 10 & 1 & YES & YES & YES & $1.70$ & $(2,3)$ & NO & 3033\\
$(277,117)$ & 12 & $(116,49)$ & 10 & 1 & YES & YES & YES & $1.56$ & $(4,2)$ & NO & 3034\\
$(277,117)$ & 12 & $(277,117)$ & 12 & 277 & YES & YES & YES & $1.56$ & $(4,2)$ & NO & 3035\\
$(280,107)$ & 12 & $(5,1)$ & 4 & 5 & YES & YES & YES & $1.60$ & $(2,3)$ & -- & 3036\\
$(281,64)$ & 13 & $(2,1)$ & 1 & 1 & YES & YES & YES & $1.43$ & $(2,3)$ & NO & 3037\\
$(281,109)$ & 12 & $(13,5)$ & 5 & 1 & YES & YES & YES & $1.67$ & $(4,2)$ & NO & 3038\\
$(281,109)$ & 12 & $(116,45)$ & 10 & 1 & YES & YES & YES & $1.67$ & $(4,2)$ & NO & 3039\\
$(282,109)$ & 12 & $(2,1)$ & 1 & 2 & YES & YES & YES & $1.67$ & $(4,2)$ & -- & 3040\\
$(282,119)$ & 12 & $(3,1)$ & 2 & 3 & YES & YES & YES & $1.56$ & $(4,2)$ & -- & 3041\\
$(282,109)$ & 12 & $(4,1)$ & 3 & 2 & YES & YES & YES & $1.56$ & $(4,2)$ & -- & 3042\\
$(282,119)$ & 12 & $(5,1)$ & 4 & 1 & YES & YES & YES & $1.44$ & $(4,2)$ & NO & 3043\\
$(282,119)$ & 12 & $(5,2)$ & 3 & 1 & YES & YES & YES & $1.50$ & $(4,2)$ & NO & 3044\\
$(282,119)$ & 12 & $(45,19)$ & 8 & 3 & YES & YES & YES & $1.67$ & $(4,2)$ & NO & 3045\\
$(282,119)$ & 12 & $(64,27)$ & 9 & 2 & YES & YES & YES & $1.44$ & $(4,2)$ & 2822 & 3046\\
$(282,109)$ & 12 & $(119,46)$ & 10 & 1 & YES & YES & YES & $1.67$ & $(4,2)$ & NO & 3047\\
$(282,119)$ & 12 & $(173,73)$ & 11 & 1 & YES & YES & YES & $1.56$ & $(4,2)$ & NO & 3048\\
$(283,108)$ & 12 & $(2,1)$ & 1 & 1 & YES & YES & YES & $1.70$ & $(2,3)$ & -- & 3049\\
$(283,83)$ & 13 & $(3,1)$ & 2 & 1 & YES & YES & YES & $1.56$ & $(4,2)$ & -- & 3050\\
$(283,108)$ & 12 & $(4,1)$ & 3 & 1 & YES & YES & YES & $1.70$ & $(2,3)$ & -- & 3051\\
$(283,108)$ & 12 & $(6,1)$ & 5 & 1 & YES & YES & YES & $1.44$ & $(4,2)$ & NO & 3052\\
$(283,108)$ & 12 & $(13,5)$ & 5 & 1 & YES & YES & YES & $1.70$ & $(2,3)$ & NO & 3053\\
$(283,104)$ & 12 & $(30,11)$ & 7 & 1 & YES & YES & YES & $1.67$ & $(4,2)$ & NO & 3054\\
$(283,108)$ & 12 & $(55,21)$ & 8 & 1 & YES & YES & YES & $1.70$ & $(2,3)$ & NO & 3055\\
$(283,83)$ & 13 & $(133,39)$ & 11 & 1 & YES & YES & YES & $1.62$ & $(4,2)$ & 3195 & 3056\\
$(284,119)$ & 12 & $(74,31)$ & 9 & 2 & YES & YES & YES & $1.70$ & $(2,3)$ & 2903 & 3057\\
$(284,105)$ & 12 & $(284,105)$ & 12 & 284 & YES & YES & YES & $1.56$ & $(4,2)$ & NO & 3058\\
$(286,105)$ & 12 & $(2,1)$ & 1 & 2 & YES & YES & YES & $1.78$ & $(4,2)$ & -- & 3059\\
$(287,106)$ & 12 & $(2,1)$ & 1 & 1 & YES & YES & YES & $1.50$ & $(4,2)$ & -- & 3060\\
$(287,109)$ & 12 & $(2,1)$ & 1 & 1 & YES & YES & YES & $1.43$ & $(4,2)$ & -- & 3061\\
$(287,111)$ & 12 & $(2,1)$ & 1 & 1 & YES & YES & YES & $1.67$ & $(4,2)$ & -- & 3062\\
$(287,109)$ & 12 & $(3,1)$ & 2 & 1 & YES & YES & YES & $1.70$ & $(2,3)$ & -- & 3063\\
$(287,106)$ & 12 & $(5,2)$ & 3 & 1 & YES & YES & YES & $1.50$ & $(4,2)$ & NO & 3064\\
$(287,111)$ & 12 & $(5,1)$ & 4 & 1 & YES & YES & YES & $1.56$ & $(4,2)$ & NO & 3065\\
$(287,111)$ & 12 & $(5,1)$ & 4 & 1 & YES & YES & YES & $1.56$ & $(4,2)$ & -- & 3066\\
$(287,53)$ & 14 & $(7,2)$ & 4 & 7 & YES & YES & YES & $1.67$ & $(4,2)$ & NO & 3067\\
$(287,109)$ & 12 & $(21,8)$ & 6 & 7 & YES & YES & YES & $1.70$ & $(2,3)$ & NO & 3068\\
$(287,106)$ & 12 & $(65,24)$ & 9 & 1 & YES & YES & YES & $1.50$ & $(4,2)$ & 2848 & 3069\\
$(287,111)$ & 12 & $(75,29)$ & 9 & 1 & YES & YES & YES & $1.56$ & $(4,2)$ & 2928 & 3070\\
$(287,80)$ & 13 & $(104,29)$ & 10 & 1 & YES & YES & YES & $1.67$ & $(4,2)$ & NO & 3071\\
$(287,111)$ & 12 & $(106,41)$ & 10 & 1 & YES & YES & YES & $1.67$ & $(4,2)$ & NO & 3072\\
$(287,111)$ & 12 & $(181,70)$ & 11 & 1 & YES & YES & YES & $1.56$ & $(4,2)$ & NO & 3073\\
$(288,85)$ & 13 & $(2,1)$ & 1 & 2 & YES & YES & YES & $1.78$ & $(2,3)$ & -- & 3074\\
$(288,119)$ & 12 & $(3,1)$ & 2 & 3 & YES & YES & YES & $1.50$ & $(4,2)$ & -- & 3075\\
$(288,119)$ & 12 & $(3,1)$ & 2 & 3 & YES & YES & YES & $1.67$ & $(4,2)$ & NO & 3076\\
$(288,119)$ & 12 & $(4,1)$ & 3 & 4 & YES & YES & YES & $1.56$ & $(4,2)$ & NO & 3077\\
$(288,121)$ & 12 & $(12,5)$ & 5 & 12 & YES & YES & YES & $1.70$ & $(2,3)$ & NO & 3078\\
$(288,85)$ & 13 & $(166,49)$ & 11 & 2 & YES & YES & YES & $1.56$ & $(4,2)$ & 3230 & 3079\\
$(288,119)$ & 12 & $(167,69)$ & 11 & 1 & YES & YES & YES & $1.67$ & $(4,2)$ & NO & 3080\\
$(288,119)$ & 12 & $(288,119)$ & 12 & 288 & YES & YES & YES & $1.56$ & $(4,2)$ & NO & 3081\\
$(289,80)$ & 12 & $(2,1)$ & 1 & 1 & YES & YES & YES & $1.60$ & $(2,3)$ & NO & 3082\\
$(289,84)$ & 13 & $(3,1)$ & 2 & 1 & YES & YES & YES & $1.29$ & $(4,2)$ & -- & 3083\\
$(289,112)$ & 12 & $(13,5)$ & 5 & 1 & YES & YES & YES & $1.62$ & $(4,2)$ & NO & 3084\\
$(289,112)$ & 12 & $(49,19)$ & 8 & 1 & YES & YES & YES & $1.44$ & $(4,2)$ & NO & 3085\\
$(290,81)$ & 12 & $(2,1)$ & 1 & 2 & YES & YES & YES & $1.60$ & $(2,3)$ & -- & 3086\\
$(290,81)$ & 12 & $(2,1)$ & 1 & 2 & YES & YES & YES & $1.70$ & $(2,3)$ & NO & 3087\\
$(290,111)$ & 12 & $(8,3)$ & 4 & 2 & YES & YES & YES & $1.70$ & $(2,3)$ & NO & 3088\\
$(290,81)$ & 12 & $(18,5)$ & 6 & 2 & YES & YES & YES & $1.60$ & $(2,3)$ & NO & 3089\\
$(290,111)$ & 12 & $(34,13)$ & 7 & 2 & YES & YES & YES & $1.70$ & $(2,3)$ & NO & 3090\\
$(291,85)$ & 13 & $(3,1)$ & 2 & 3 & YES & YES & YES & $1.67$ & $(4,2)$ & -- & 3091\\
$(291,85)$ & 13 & $(4,1)$ & 3 & 1 & YES & YES & YES & $1.44$ & $(4,2)$ & -- & 3092\\
$(291,85)$ & 13 & $(10,3)$ & 5 & 1 & YES & YES & YES & $1.78$ & $(4,2)$ & NO & 3093\\
$(291,85)$ & 13 & $(65,19)$ & 9 & 1 & YES & YES & YES & $1.56$ & $(4,2)$ & NO & 3094\\
$(292,111)$ & 12 & $(2,1)$ & 1 & 2 & YES & YES & YES & $1.67$ & $(4,2)$ & -- & 3095\\
$(292,111)$ & 12 & $(3,1)$ & 2 & 1 & YES & YES & YES & $1.70$ & $(2,3)$ & -- & 3096\\
$(292,121)$ & 12 & $(3,1)$ & 2 & 1 & YES & YES & YES & $1.62$ & $(4,2)$ & -- & 3097\\
$(292,121)$ & 12 & $(29,12)$ & 7 & 1 & YES & YES & YES & $1.62$ & $(4,2)$ & 2991 & 3098\\
$(292,85)$ & 13 & $(31,9)$ & 8 & 1 & YES & YES & YES & $1.70$ & $(2,3)$ & NO & 3099\\
$(292,85)$ & 13 & $(55,16)$ & 9 & 1 & YES & YES & YES & $1.56$ & $(2,3)$ & NO & 3100\\
$(292,111)$ & 12 & $(121,46)$ & 10 & 1 & YES & YES & YES & $1.56$ & $(4,2)$ & NO & 3101\\
$(293,123)$ & 12 & $(2,1)$ & 1 & 1 & YES & YES & YES & $1.43$ & $(4,2)$ & -- & 3102\\
$(293,79)$ & 13 & $(3,1)$ & 2 & 1 & YES & YES & YES & $1.75$ & $(2,3)$ & NO & 3103\\
$(293,123)$ & 12 & $(7,3)$ & 4 & 1 & YES & YES & YES & $1.43$ & $(4,2)$ & NO & 3104\\
$(295,108)$ & 12 & $(4,1)$ & 3 & 1 & YES & YES & YES & $1.56$ & $(4,2)$ & -- & 3105\\
$(295,87)$ & 13 & $(61,18)$ & 9 & 1 & YES & YES & YES & $1.29$ & $(4,2)$ & NO & 3106\\
$(297,83)$ & 13 & $(2,1)$ & 1 & 1 & YES & YES & YES & $1.50$ & $(4,2)$ & -- & 3107\\
$(297,83)$ & 13 & $(3,1)$ & 2 & 3 & YES & YES & YES & $1.70$ & $(2,3)$ & -- & 3108\\
$(297,83)$ & 13 & $(18,5)$ & 6 & 9 & YES & YES & YES & $1.70$ & $(2,3)$ & NO & 3109\\
$(298,83)$ & 13 & $(3,1)$ & 2 & 1 & YES & YES & YES & $1.67$ & $(4,2)$ & -- & 3110\\
$(298,83)$ & 13 & $(298,83)$ & 13 & 298 & YES & YES & YES & $1.62$ & $(4,2)$ & NO & 3111\\
$(301,115)$ & 12 & $(2,1)$ & 1 & 1 & YES & YES & YES & $1.60$ & $(2,3)$ & -- & 3112\\
$(301,65)$ & 13 & $(5,2)$ & 3 & 1 & YES & YES & YES & $1.56$ & $(4,2)$ & -- & 3113\\
$(301,65)$ & 13 & $(5,2)$ & 3 & 1 & YES & YES & YES & $1.67$ & $(4,2)$ & NO & 3114\\
$(301,115)$ & 12 & $(5,2)$ & 3 & 1 & YES & YES & YES & $1.44$ & $(4,2)$ & NO & 3115\\
$(301,65)$ & 13 & $(13,3)$ & 6 & 1 & YES & YES & YES & $1.56$ & $(4,2)$ & 3245 & 3116\\
$(301,115)$ & 12 & $(21,8)$ & 6 & 7 & YES & YES & YES & $1.50$ & $(4,2)$ & NO & 3117\\
$(301,88)$ & 13 & $(41,12)$ & 8 & 1 & YES & YES & YES & $1.67$ & $(4,2)$ & NO & 3118\\
$(303,85)$ & 13 & $(2,1)$ & 1 & 1 & YES & YES & YES & $1.56$ & $(4,2)$ & -- & 3119\\
$(303,85)$ & 13 & $(3,1)$ & 2 & 3 & YES & YES & YES & $1.70$ & $(2,3)$ & -- & 3120\\
$(303,85)$ & 13 & $(11,3)$ & 5 & 1 & YES & YES & YES & $1.67$ & $(4,2)$ & NO & 3121\\
$(303,85)$ & 13 & $(32,9)$ & 8 & 1 & YES & YES & YES & $1.70$ & $(2,3)$ & NO & 3122\\
$(303,116)$ & 12 & $(34,13)$ & 7 & 1 & YES & YES & YES & $1.67$ & $(4,2)$ & 2540 & 3123\\
$(303,85)$ & 13 & $(57,16)$ & 9 & 3 & YES & YES & YES & $1.56$ & $(4,2)$ & NO & 3124\\
$(303,128)$ & 12 & $(71,30)$ & 9 & 1 & YES & YES & YES & $1.56$ & $(4,2)$ & 2942 & 3125\\
$(304,85)$ & 13 & $(3,1)$ & 2 & 1 & YES & YES & YES & $1.50$ & $(4,2)$ & -- & 3126\\
$(304,85)$ & 13 & $(11,3)$ & 5 & 1 & YES & YES & YES & $1.62$ & $(4,2)$ & NO & 3127\\
$(305,84)$ & 13 & $(2,1)$ & 1 & 1 & YES & YES & YES & $1.50$ & $(4,2)$ & -- & 3128\\
$(305,118)$ & 13 & $(137,53)$ & 11 & 1 & YES & YES & YES & $1.75$ & $(2,3)$ & NO & 3129\\
$(307,119)$ & 12 & $(2,1)$ & 1 & 1 & YES & YES & YES & $1.70$ & $(2,3)$ & -- & 3130\\
$(307,129)$ & 12 & $(2,1)$ & 1 & 1 & YES & YES & YES & $1.78$ & $(4,2)$ & -- & 3131\\
$(307,69)$ & 14 & $(3,1)$ & 2 & 1 & YES & YES & YES & $1.50$ & $(6,1)$ & NO & 3132\\
$(307,119)$ & 12 & $(3,1)$ & 2 & 1 & YES & YES & YES & $1.67$ & $(4,2)$ & -- & 3133\\
$(307,129)$ & 12 & $(12,5)$ & 5 & 1 & YES & YES & YES & $1.70$ & $(2,3)$ & 2610 & 3134\\
$(307,85)$ & 13 & $(47,13)$ & 8 & 1 & YES & YES & YES & $1.67$ & $(4,2)$ & NO & 3135\\
$(307,129)$ & 12 & $(69,29)$ & 9 & 1 & YES & YES & YES & $1.70$ & $(2,3)$ & 2939 & 3136\\
$(313,86)$ & 13 & $(2,1)$ & 1 & 1 & YES & YES & YES & $1.57$ & $(4,2)$ & NO & 3137\\
$(313,121)$ & 12 & $(2,1)$ & 1 & 1 & NO & YES & YES & $1.62$ & $(2,3)$ & -- & 3138\\
$(313,86)$ & 13 & $(3,1)$ & 2 & 1 & YES & YES & YES & $1.50$ & $(4,2)$ & NO & 3139\\
$(313,86)$ & 13 & $(3,1)$ & 2 & 1 & YES & YES & YES & $1.67$ & $(4,2)$ & -- & 3140\\
$(313,86)$ & 13 & $(3,1)$ & 2 & 1 & YES & YES & YES & $1.67$ & $(4,2)$ & NO & 3141\\
$(313,121)$ & 12 & $(3,1)$ & 2 & 1 & YES & YES & YES & $1.56$ & $(4,2)$ & -- & 3142\\
$(313,121)$ & 12 & $(13,5)$ & 5 & 1 & YES & YES & YES & $1.50$ & $(4,2)$ & NO & 3143\\
$(313,86)$ & 13 & $(18,5)$ & 6 & 1 & YES & YES & YES & $1.56$ & $(4,2)$ & NO & 3144\\
$(313,91)$ & 13 & $(55,16)$ & 9 & 1 & YES & YES & YES & $1.67$ & $(2,3)$ & NO & 3145\\
$(313,119)$ & 12 & $(121,46)$ & 10 & 1 & YES & YES & YES & $1.70$ & $(2,3)$ & NO & 3146\\
$(315,88)$ & 13 & $(3,1)$ & 2 & 3 & YES & YES & YES & $1.70$ & $(2,3)$ & -- & 3147\\
$(315,88)$ & 13 & $(18,5)$ & 6 & 9 & YES & YES & YES & $1.70$ & $(2,3)$ & NO & 3148\\
$(317,121)$ & 12 & $(5,1)$ & 4 & 1 & YES & YES & YES & $1.38$ & $(4,2)$ & -- & 3149\\
$(317,121)$ & 12 & $(5,2)$ & 3 & 1 & YES & YES & YES & $1.70$ & $(2,3)$ & NO & 3150\\
$(317,121)$ & 12 & $(13,5)$ & 5 & 1 & YES & YES & YES & $1.67$ & $(4,2)$ & NO & 3151\\
$(321,94)$ & 13 & $(2,1)$ & 1 & 1 & YES & YES & YES & $1.60$ & $(2,3)$ & -- & 3152\\
$(321,95)$ & 13 & $(2,1)$ & 1 & 1 & YES & YES & YES & $1.78$ & $(4,2)$ & -- & 3153\\
$(321,94)$ & 13 & $(3,1)$ & 2 & 3 & YES & YES & YES & $1.50$ & $(4,2)$ & -- & 3154\\
$(321,95)$ & 13 & $(4,1)$ & 3 & 1 & YES & YES & YES & $1.70$ & $(2,3)$ & NO & 3155\\
$(321,94)$ & 13 & $(140,41)$ & 11 & 1 & YES & YES & YES & $1.50$ & $(4,2)$ & NO & 3156\\
$(322,73)$ & 14 & $(5,1)$ & 4 & 1 & YES & YES & YES & $1.56$ & $(4,2)$ & NO & 3157\\
$(323,60)$ & 14 & $(2,1)$ & 1 & 1 & YES & YES & YES & $1.75$ & $(2,3)$ & -- & 3158\\
$(323,60)$ & 14 & $(2,1)$ & 1 & 1 & YES & YES & YES & $1.83$ & $(2,3)$ & NO & 3159\\
$(323,94)$ & 13 & $(2,1)$ & 1 & 1 & YES & YES & YES & $1.50$ & $(4,2)$ & -- & 3160\\
$(323,98)$ & 13 & $(2,1)$ & 1 & 1 & YES & YES & YES & $1.78$ & $(4,2)$ & NO & 3161\\
$(323,98)$ & 13 & $(3,1)$ & 2 & 1 & NO & YES & YES & $1.70$ & $(4,2)$ & -- & 3162\\
$(323,94)$ & 13 & $(17,5)$ & 6 & 17 & YES & YES & YES & $1.44$ & $(4,2)$ & NO & 3163\\
$(323,98)$ & 13 & $(23,7)$ & 7 & 1 & YES & YES & YES & $1.67$ & $(4,2)$ & NO & 3164\\
$(323,94)$ & 13 & $(31,9)$ & 8 & 1 & YES & YES & YES & $1.70$ & $(2,3)$ & NO & 3165\\
$(323,98)$ & 13 & $(56,17)$ & 9 & 1 & YES & YES & YES & $1.67$ & $(4,2)$ & NO & 3166\\
$(323,89)$ & 13 & $(98,27)$ & 10 & 1 & YES & YES & YES & $1.50$ & $(4,2)$ & NO & 3167\\
$(323,94)$ & 13 & $(134,39)$ & 11 & 1 & YES & YES & YES & $1.80$ & $(2,3)$ & NO & 3168\\
$(324,95)$ & 13 & $(10,3)$ & 5 & 2 & YES & YES & YES & $1.67$ & $(4,2)$ & NO & 3169\\
$(324,95)$ & 13 & $(75,22)$ & 10 & 3 & YES & YES & YES & $1.50$ & $(4,2)$ & NO & 3170\\
$(325,74)$ & 14 & $(5,1)$ & 4 & 5 & YES & YES & YES & $1.50$ & $(6,1)$ & NO & 3171\\
$(326,99)$ & 13 & $(2,1)$ & 1 & 2 & YES & YES & YES & $1.56$ & $(4,2)$ & NO & 3172\\
$(326,97)$ & 13 & $(3,1)$ & 2 & 1 & YES & YES & YES & $1.44$ & $(4,2)$ & -- & 3173\\
$(326,99)$ & 13 & $(3,1)$ & 2 & 1 & YES & YES & YES & $1.56$ & $(4,2)$ & -- & 3174\\
$(326,71)$ & 14 & $(4,1)$ & 3 & 2 & YES & YES & YES & $1.43$ & $(4,2)$ & NO & 3175\\
$(326,99)$ & 13 & $(79,24)$ & 10 & 1 & YES & YES & YES & $1.67$ & $(4,2)$ & NO & 3176\\
$(326,97)$ & 13 & $(326,97)$ & 13 & 326 & YES & YES & YES & $1.62$ & $(4,2)$ & NO & 3177\\
$(333,101)$ & 13 & $(2,1)$ & 1 & 1 & YES & YES & YES & $1.50$ & $(4,2)$ & NO & 3178\\
$(333,101)$ & 13 & $(2,1)$ & 1 & 1 & YES & YES & YES & $1.70$ & $(2,3)$ & -- & 3179\\
$(333,92)$ & 13 & $(3,1)$ & 2 & 3 & YES & YES & YES & $1.60$ & $(2,3)$ & -- & 3180\\
$(333,76)$ & 13 & $(9,2)$ & 5 & 9 & YES & YES & YES & $1.60$ & $(2,3)$ & NO & 3181\\
$(333,76)$ & 13 & $(22,5)$ & 7 & 1 & YES & YES & YES & $1.60$ & $(2,3)$ & NO & 3182\\
$(333,101)$ & 13 & $(23,7)$ & 7 & 1 & YES & YES & YES & $1.50$ & $(4,2)$ & NO & 3183\\
$(335,73)$ & 14 & $(2,1)$ & 1 & 1 & YES & YES & YES & $1.56$ & $(4,2)$ & -- & 3184\\
$(335,73)$ & 14 & $(2,1)$ & 1 & 1 & YES & YES & YES & $1.67$ & $(4,2)$ & NO & 3185\\
$(335,73)$ & 14 & $(3,1)$ & 2 & 1 & YES & YES & YES & $1.67$ & $(4,2)$ & NO & 3186\\
$(335,73)$ & 14 & $(3,1)$ & 2 & 1 & YES & YES & YES & $1.67$ & $(4,2)$ & -- & 3187\\
$(335,76)$ & 14 & $(3,1)$ & 2 & 1 & YES & YES & YES & $1.57$ & $(2,3)$ & NO & 3188\\
$(337,98)$ & 13 & $(24,7)$ & 7 & 1 & YES & YES & YES & $1.70$ & $(2,3)$ & NO & 3189\\
$(337,91)$ & 13 & $(137,37)$ & 11 & 1 & YES & YES & YES & $1.56$ & $(4,2)$ & 3225 & 3190\\
$(338,99)$ & 13 & $(2,1)$ & 1 & 2 & YES & YES & YES & $1.56$ & $(4,2)$ & NO & 3191\\
$(338,129)$ & 12 & $(2,1)$ & 1 & 2 & YES & YES & YES & $1.67$ & $(4,2)$ & -- & 3192\\
$(338,77)$ & 14 & $(5,1)$ & 4 & 1 & YES & YES & YES & $1.50$ & $(4,2)$ & NO & 3193\\
$(338,129)$ & 12 & $(13,5)$ & 5 & 13 & YES & YES & YES & $1.50$ & $(4,2)$ & 2690 & 3194\\
$(341,100)$ & 13 & $(75,22)$ & 10 & 1 & YES & YES & YES & $1.62$ & $(4,2)$ & 3056 & 3195\\
$(341,100)$ & 13 & $(133,39)$ & 11 & 1 & YES & YES & YES & $1.70$ & $(2,3)$ & NO & 3196\\
$(342,101)$ & 13 & $(193,57)$ & 12 & 1 & YES & YES & YES & $1.44$ & $(4,2)$ & NO & 3197\\
$(344,95)$ & 13 & $(2,1)$ & 1 & 2 & YES & YES & YES & $1.62$ & $(4,2)$ & -- & 3198\\
$(344,95)$ & 13 & $(2,1)$ & 1 & 2 & YES & YES & YES & $1.70$ & $(2,3)$ & NO & 3199\\
$(344,95)$ & 13 & $(18,5)$ & 6 & 2 & YES & YES & YES & $1.60$ & $(2,3)$ & 2319 & 3200\\
$(347,93)$ & 13 & $(3,1)$ & 2 & 1 & YES & YES & YES & $1.56$ & $(4,2)$ & NO & 3201\\
$(347,101)$ & 13 & $(3,1)$ & 2 & 1 & YES & YES & YES & $1.50$ & $(4,2)$ & -- & 3202\\
$(347,93)$ & 13 & $(4,1)$ & 3 & 1 & YES & YES & YES & $1.56$ & $(4,2)$ & -- & 3203\\
$(347,101)$ & 13 & $(4,1)$ & 3 & 1 & YES & YES & YES & $1.56$ & $(4,2)$ & NO & 3204\\
$(347,93)$ & 13 & $(41,11)$ & 8 & 1 & YES & YES & YES & $1.56$ & $(4,2)$ & NO & 3205\\
$(347,101)$ & 13 & $(134,39)$ & 11 & 1 & YES & YES & YES & $1.56$ & $(4,2)$ & NO & 3206\\
$(349,135)$ & 13 & $(31,12)$ & 7 & 1 & YES & YES & YES & $1.75$ & $(2,3)$ & NO & 3207\\
$(353,97)$ & 13 & $(3,1)$ & 2 & 1 & YES & YES & YES & $1.67$ & $(4,2)$ & NO & 3208\\
$(353,97)$ & 13 & $(3,1)$ & 2 & 1 & YES & YES & YES & $1.56$ & $(4,2)$ & -- & 3209\\
$(353,97)$ & 13 & $(7,2)$ & 4 & 1 & YES & YES & YES & $1.56$ & $(4,2)$ & NO & 3210\\
$(355,77)$ & 14 & $(2,1)$ & 1 & 1 & YES & YES & YES & $1.67$ & $(4,2)$ & NO & 3211\\
$(355,99)$ & 13 & $(3,1)$ & 2 & 1 & YES & YES & YES & $1.56$ & $(4,2)$ & 2766 & 3212\\
$(355,77)$ & 14 & $(14,3)$ & 6 & 1 & YES & YES & YES & $1.67$ & $(4,2)$ & NO & 3213\\
$(359,100)$ & 13 & $(2,1)$ & 1 & 1 & YES & YES & YES & $1.50$ & $(4,2)$ & -- & 3214\\
$(359,57)$ & 16 & $(3,1)$ & 2 & 1 & YES & YES & YES & $1.43$ & $(4,2)$ & -- & 3215\\
$(359,100)$ & 13 & $(61,17)$ & 9 & 1 & YES & YES & YES & $1.50$ & $(4,2)$ & NO & 3216\\
$(360,101)$ & 13 & $(2,1)$ & 1 & 2 & YES & YES & YES & $1.56$ & $(4,2)$ & NO & 3217\\
$(360,101)$ & 13 & $(57,16)$ & 9 & 3 & YES & YES & YES & $1.50$ & $(4,2)$ & NO & 3218\\
$(366,83)$ & 14 & $(3,1)$ & 2 & 3 & YES & YES & YES & $1.67$ & $(4,2)$ & -- & 3219\\
$(367,99)$ & 13 & $(3,1)$ & 2 & 1 & YES & YES & YES & $1.67$ & $(4,2)$ & -- & 3220\\
$(367,112)$ & 13 & $(23,7)$ & 7 & 1 & YES & YES & YES & $1.70$ & $(2,3)$ & NO & 3221\\
$(367,99)$ & 13 & $(89,24)$ & 10 & 1 & YES & YES & YES & $1.56$ & $(4,2)$ & NO & 3222\\
$(372,109)$ & 13 & $(4,1)$ & 3 & 4 & YES & YES & YES & $1.56$ & $(4,2)$ & -- & 3223\\
$(372,109)$ & 13 & $(17,5)$ & 6 & 1 & YES & YES & YES & $1.67$ & $(4,2)$ & NO & 3224\\
$(374,101)$ & 13 & $(100,27)$ & 10 & 2 & YES & YES & YES & $1.56$ & $(4,2)$ & 3190 & 3225\\
$(383,106)$ & 13 & $(18,5)$ & 6 & 1 & YES & YES & YES & $1.56$ & $(4,2)$ & NO & 3226\\
$(389,89)$ & 14 & $(2,1)$ & 1 & 1 & YES & YES & YES & $1.44$ & $(4,2)$ & NO & 3227\\
$(389,89)$ & 14 & $(83,19)$ & 10 & 1 & YES & YES & YES & $1.56$ & $(4,2)$ & NO & 3228\\
$(393,116)$ & 13 & $(2,1)$ & 1 & 1 & YES & YES & YES & $1.70$ & $(2,3)$ & NO & 3229\\
$(393,116)$ & 13 & $(61,18)$ & 9 & 1 & YES & YES & YES & $1.56$ & $(4,2)$ & 3079 & 3230\\
$(393,116)$ & 13 & $(166,49)$ & 11 & 1 & YES & YES & YES & $1.70$ & $(2,3)$ & NO & 3231\\
$(394,165)$ & 13 & $(2,1)$ & 1 & 2 & NO & YES & YES & $1.80$ & $(2,3)$ & -- & 3232\\
$(397,75)$ & 15 & $(3,1)$ & 2 & 1 & YES & YES & YES & $1.38$ & $(4,2)$ & -- & 3233\\
$(398,111)$ & 13 & $(2,1)$ & 1 & 2 & YES & YES & YES & $1.44$ & $(4,2)$ & -- & 3234\\
$(398,111)$ & 13 & $(7,2)$ & 4 & 1 & YES & YES & YES & $1.56$ & $(4,2)$ & NO & 3235\\
$(403,87)$ & 14 & $(4,1)$ & 3 & 1 & YES & YES & YES & $1.44$ & $(4,2)$ & -- & 3236\\
$(407,119)$ & 13 & $(17,5)$ & 6 & 1 & YES & YES & YES & $1.60$ & $(2,3)$ & NO & 3237\\
$(419,89)$ & 14 & $(2,1)$ & 1 & 1 & YES & YES & YES & $1.56$ & $(4,2)$ & -- & 3238\\
$(419,89)$ & 14 & $(2,1)$ & 1 & 1 & YES & YES & YES & $1.67$ & $(4,2)$ & NO & 3239\\
$(423,97)$ & 14 & $(2,1)$ & 1 & 1 & YES & YES & YES & $1.50$ & $(4,2)$ & NO & 3240\\
$(423,97)$ & 14 & $(2,1)$ & 1 & 1 & YES & YES & YES & $1.44$ & $(4,2)$ & -- & 3241\\
$(423,97)$ & 14 & $(3,1)$ & 2 & 3 & YES & YES & YES & $1.44$ & $(4,2)$ & -- & 3242\\
$(424,97)$ & 14 & $(3,1)$ & 2 & 1 & YES & YES & YES & $1.56$ & $(4,2)$ & -- & 3243\\
$(424,97)$ & 14 & $(48,11)$ & 9 & 8 & YES & YES & YES & $1.44$ & $(4,2)$ & NO & 3244\\
$(437,99)$ & 14 & $(5,1)$ & 4 & 1 & YES & YES & YES & $1.56$ & $(4,2)$ & 3116 & 3245\\
$(451,84)$ & 15 & $(2,1)$ & 1 & 1 & YES & YES & YES & $1.50$ & $(4,2)$ & -- & 3246\\
$(451,84)$ & 15 & $(2,1)$ & 1 & 1 & YES & YES & YES & $1.62$ & $(4,2)$ & NO & 3247\\
$(451,84)$ & 15 & $(3,1)$ & 2 & 1 & YES & YES & YES & $1.44$ & $(4,2)$ & NO & 3248\\
$(461,98)$ & 14 & $(4,1)$ & 3 & 1 & YES & YES & YES & $1.44$ & $(4,2)$ & NO & 3249\\
$(461,98)$ & 14 & $(33,7)$ & 8 & 1 & YES & YES & YES & $1.56$ & $(4,2)$ & NO & 3250\\
$(466,109)$ & 14 & $(13,3)$ & 6 & 1 & YES & YES & YES & $1.56$ & $(4,2)$ & NO & 3251\\
$(466,109)$ & 14 & $(30,7)$ & 8 & 2 & YES & YES & YES & $1.56$ & $(4,2)$ & NO & 3252\\
$(469,107)$ & 14 & $(22,5)$ & 7 & 1 & YES & YES & YES & $1.56$ & $(4,2)$ & NO & 3253\\
$(477,88)$ & 15 & $(27,5)$ & 8 & 9 & YES & YES & YES & $1.60$ & $(2,3)$ & NO & 3254\\
$(495,92)$ & 15 & $(3,1)$ & 2 & 3 & YES & YES & YES & $1.50$ & $(4,2)$ & -- & 3255\\
$(495,92)$ & 15 & $(11,2)$ & 6 & 11 & YES & YES & YES & $1.56$ & $(4,2)$ & NO & 3256\\
$(495,92)$ & 15 & $(27,5)$ & 8 & 9 & YES & YES & YES & $1.56$ & $(4,2)$ & NO & 3257\\
$(522,119)$ & 14 & $(22,5)$ & 7 & 2 & YES & YES & YES & $1.44$ & $(4,2)$ & NO & 3258\\
$(522,119)$ & 14 & $(57,13)$ & 9 & 3 & YES & YES & YES & $1.44$ & $(4,2)$ & NO & 3259\\
$(a;0,0,0;3)$ & 4 & $(65,19)$ & 9 & 1 & YES & YES & YES & $1.73$ & $(2,3)$ & -- & 3260\\
$(a;0,0,0;3)$ & 4 & $(76,29)$ & 9 & 1 & YES & YES & YES & $1.67$ & $(2,3)$ & -- & 3261\\
$(a;0,0,0;3)$ & 4 & $(79,18)$ & 10 & 1 & YES & YES & YES & $1.73$ & $(2,3)$ & -- & 3262\\
$(a;0,0,0;3)$ & 4 & $(89,24)$ & 10 & 1 & YES & YES & YES & $1.70$ & $(2,3)$ & -- & 3263\\
$(a;0,0,0;3)$ & 4 & $(101,22)$ & 11 & 1 & YES & YES & YES & $1.80$ & $(2,3)$ & -- & 3264\\
$(a;0,0,0;3)$ & 4 & $(101,23)$ & 11 & 1 & YES & YES & YES & $1.56$ & $(4,2)$ & -- & 3265\\
$(a;1,0,0;13)$ & 5 & $(46,19)$ & 8 & 1 & YES & YES & YES & $1.67$ & $(4,2)$ & -- & 3266\\
$(a;1,0,0;13)$ & 5 & $(55,23)$ & 9 & 1 & YES & YES & YES & $1.29$ & $(6,1)$ & -- & 3267\\
$(a;1,0,0;13)$ & 5 & $(61,17)$ & 9 & 1 & YES & YES & YES & $1.80$ & $(2,3)$ & -- & 3268\\
$(a;1,0,0;13)$ & 5 & $(89,27)$ & 10 & 1 & YES & YES & YES & $1.67$ & $(4,2)$ & -- & 3269\\
$(a;1,1,0;19)$ & 6 & $(29,11)$ & 7 & 1 & YES & YES & YES & $1.60$ & $(2,3)$ & -- & 3270\\
$(a;1,1,0;19)$ & 6 & $(31,12)$ & 7 & 1 & YES & YES & YES & $1.67$ & $(4,2)$ & -- & 3271\\
$(a;1,1,0;19)$ & 6 & $(37,14)$ & 8 & 1 & YES & YES & YES & $1.29$ & $(6,1)$ & -- & 3272\\
$(a;1,1,1;4)$ & 7 & $(12,5)$ & 5 & 4 & YES & YES & YES & $1.64$ & $(4,2)$ & -- & 3273\\
$(a;2,0,0;17)$ & 6 & $(58,17)$ & 9 & 1 & YES & YES & YES & $1.75$ & $(2,3)$ & -- & 3274\\
$(a;2,0,0;17)$ & 6 & $(79,18)$ & 10 & 1 & YES & YES & YES & $1.62$ & $(4,2)$ & -- & 3275\\
$(a;2,1,1;37)$ & 8 & $(13,5)$ & 5 & 1 & YES & YES & YES & $1.70$ & $(2,3)$ & -- & 3276\\
$(a;2,1,1;37)$ & 8 & $(16,5)$ & 7 & 1 & YES & YES & YES & $1.29$ & $(6,1)$ & -- & 3277\\
$(a;3,3,0;17)$ & 10 & $(2,1)$ & 1 & 1 & YES & YES & YES & $1.14$ & $(4,2)$ & -- & 3278\\
$(a;3,3,0;17)$ & 10 & $(5,1)$ & 4 & 1 & YES & YES & YES & $1.14$ & $(4,2)$ & -- & 3279\\
$(b;0,0,0;14)$ & 5 & $(25,7)$ & 7 & 1 & YES & YES & YES & $1.43$ & $(2,3)$ & -- & 3280\\
$(b;0,0,0;14)$ & 5 & $(29,12)$ & 7 & 1 & YES & YES & YES & $1.43$ & $(2,3)$ & -- & 3281\\
$(b;0,0,0;14)$ & 5 & $(31,12)$ & 7 & 1 & YES & YES & YES & $1.60$ & $(4,2)$ & -- & 3282\\
$(b;0,0,0;14)$ & 5 & $(40,9)$ & 9 & 2 & YES & YES & YES & $1.70$ & $(4,2)$ & -- & 3283\\
$(b;0,0,0;14)$ & 5 & $(44,17)$ & 8 & 2 & YES & YES & YES & $1.43$ & $(2,3)$ & -- & 3284\\
$(b;0,0,1;4)$ & 6 & $(17,7)$ & 6 & 1 & YES & YES & YES & $1.62$ & $(2,3)$ & -- & 3285\\
$(b;0,0,1;4)$ & 6 & $(23,9)$ & 7 & 1 & YES & YES & YES & $1.29$ & $(4,2)$ & -- & 3286\\
$(b;0,0,1;4)$ & 6 & $(26,11)$ & 7 & 2 & YES & YES & YES & $1.67$ & $(2,3)$ & -- & 3287\\
$(b;0,0,1;4)$ & 6 & $(31,12)$ & 7 & 1 & YES & YES & YES & $1.67$ & $(4,2)$ & -- & 3288\\
$(b;0,1,0;19)$ & 6 & $(24,7)$ & 7 & 1 & YES & YES & YES & $1.38$ & $(4,2)$ & -- & 3289\\
$(b;0,1,0;19)$ & 6 & $(26,11)$ & 7 & 1 & YES & YES & YES & $1.73$ & $(2,3)$ & -- & 3290\\
$(b;0,1,0;19)$ & 6 & $(29,12)$ & 7 & 1 & YES & YES & YES & $1.60$ & $(2,3)$ & -- & 3291\\
$(b;0,1,0;19)$ & 6 & $(31,9)$ & 8 & 1 & YES & YES & YES & $1.60$ & $(2,3)$ & -- & 3292\\
$(b;0,1,0;19)$ & 6 & $(42,13)$ & 9 & 1 & YES & YES & YES & $1.75$ & $(2,3)$ & -- & 3293\\
$(b;0,1,1;27)$ & 7 & $(12,5)$ & 5 & 3 & YES & YES & YES & $1.64$ & $(4,2)$ & -- & 3294\\
$(b;0,1,1;27)$ & 7 & $(17,5)$ & 6 & 1 & YES & YES & YES & $1.56$ & $(2,3)$ & -- & 3295\\
$(b;0,1,1;27)$ & 7 & $(17,7)$ & 6 & 1 & YES & YES & YES & $1.80$ & $(2,3)$ & -- & 3296\\
$(b;0,1,1;27)$ & 7 & $(24,7)$ & 7 & 3 & YES & YES & YES & $1.29$ & $(4,2)$ & -- & 3297\\
$(b;0,1,2;5)$ & 8 & $(13,5)$ & 5 & 1 & YES & YES & YES & $1.50$ & $(4,2)$ & -- & 3298\\
$(b;0,2,0;8)$ & 7 & $(7,2)$ & 4 & 1 & YES & YES & YES & $1.44$ & $(2,3)$ & -- & 3299\\
$(b;0,2,0;8)$ & 7 & $(18,7)$ & 6 & 2 & YES & YES & YES & $1.64$ & $(2,3)$ & -- & 3300\\
$(b;0,2,0;8)$ & 7 & $(21,8)$ & 6 & 1 & YES & YES & YES & $1.67$ & $(4,2)$ & -- & 3301\\
$(b;0,2,0;8)$ & 7 & $(25,7)$ & 7 & 1 & YES & YES & YES & $1.43$ & $(4,2)$ & -- & 3302\\
$(b;0,2,0;8)$ & 7 & $(27,8)$ & 7 & 1 & YES & YES & YES & $1.67$ & $(4,2)$ & -- & 3303\\
$(b;0,2,0;8)$ & 7 & $(31,7)$ & 8 & 1 & YES & YES & YES & $1.73$ & $(2,3)$ & -- & 3304\\
$(b;0,2,0;8)$ & 7 & $(35,8)$ & 8 & 1 & YES & YES & YES & $1.50$ & $(4,2)$ & -- & 3305\\
$(b;0,2,1;34)$ & 8 & $(13,5)$ & 5 & 1 & YES & YES & YES & $1.50$ & $(4,2)$ & -- & 3306\\
$(b;0,2,1;34)$ & 8 & $(17,5)$ & 6 & 17 & YES & YES & YES & $1.70$ & $(2,3)$ & -- & 3307\\
$(b;0,3,2;53)$ & 10 & $(6,1)$ & 5 & 1 & YES & YES & YES & $1.38$ & $(2,3)$ & -- & 3308\\
$(b;1,0,1;29)$ & 7 & $(13,4)$ & 6 & 1 & YES & YES & YES & $1.50$ & $(6,1)$ & -- & 3309\\
$(b;1,1,0;27)$ & 7 & $(17,7)$ & 6 & 1 & YES & YES & YES & $1.75$ & $(2,3)$ & -- & 3310\\
$(b;1,1,1;39)$ & 8 & $(7,3)$ & 4 & 1 & YES & YES & YES & $1.73$ & $(2,3)$ & -- & 3311\\
$(b;1,1,1;39)$ & 8 & $(10,3)$ & 5 & 1 & YES & YES & YES & $1.83$ & $(2,3)$ & -- & 3312\\
$(b;1,1,1;39)$ & 8 & $(11,4)$ & 5 & 1 & YES & YES & YES & $1.43$ & $(4,2)$ & -- & 3313\\
$(b;1,1,1;39)$ & 8 & $(13,5)$ & 5 & 13 & YES & YES & YES & $1.70$ & $(2,3)$ & -- & 3314\\
$(b;1,1,2;51)$ & 9 & $(5,2)$ & 3 & 1 & YES & YES & YES & $1.50$ & $(4,2)$ & -- & 3315\\
$(b;1,1,2;51)$ & 9 & $(7,3)$ & 4 & 1 & YES & YES & YES & $1.50$ & $(4,2)$ & -- & 3316\\
$(b;1,2,0;17)$ & 8 & $(13,4)$ & 6 & 1 & YES & YES & YES & $1.57$ & $(2,3)$ & -- & 3317\\
$(b;2,0,1;38)$ & 8 & $(13,5)$ & 5 & 1 & YES & YES & YES & $1.70$ & $(2,3)$ & -- & 3318\\
$(b;2,0,1;38)$ & 8 & $(17,5)$ & 6 & 1 & YES & YES & YES & $1.70$ & $(2,3)$ & -- & 3319\\
$(c;0,0,0;4)$ & 4 & $(47,18)$ & 8 & 1 & YES & YES & YES & $1.62$ & $(6,1)$ & -- & 3320\\
$(c;0,0,0;4)$ & 4 & $(49,19)$ & 8 & 1 & YES & YES & YES & $1.38$ & $(6,1)$ & -- & 3321\\
$(c;0,0,0;4)$ & 4 & $(57,22)$ & 9 & 1 & YES & YES & YES & $1.83$ & $(2,3)$ & -- & 3322\\
$(c;0,0,0;4)$ & 4 & $(58,17)$ & 9 & 2 & YES & YES & YES & $1.75$ & $(2,3)$ & -- & 3323\\
$(c;0,0,0;4)$ & 4 & $(61,17)$ & 9 & 1 & YES & YES & YES & $1.62$ & $(2,3)$ & -- & 3324\\
$(c;0,0,0;4)$ & 4 & $(69,29)$ & 9 & 1 & YES & YES & YES & $1.71$ & $(2,3)$ & -- & 3325\\
$(c;0,0,0;4)$ & 4 & $(75,31)$ & 9 & 1 & YES & YES & YES & $1.62$ & $(4,2)$ & -- & 3326\\
$(c;0,0,0;4)$ & 4 & $(76,29)$ & 9 & 4 & YES & YES & YES & $1.71$ & $(2,3)$ & -- & 3327\\
$(c;0,0,0;4)$ & 4 & $(79,22)$ & 10 & 1 & YES & YES & YES & $1.83$ & $(2,3)$ & -- & 3328\\
$(c;0,0,0;4)$ & 4 & $(82,23)$ & 10 & 2 & YES & YES & YES & $1.56$ & $(2,3)$ & -- & 3329\\
$(c;0,0,0;4)$ & 4 & $(92,35)$ & 10 & 4 & YES & YES & YES & $1.78$ & $(2,3)$ & -- & 3330\\
$(c;0,0,0;4)$ & 4 & $(95,36)$ & 10 & 1 & YES & YES & YES & $1.67$ & $(4,2)$ & -- & 3331\\
$(c;0,0,0;4)$ & 4 & $(99,41)$ & 10 & 1 & YES & YES & YES & $1.56$ & $(4,2)$ & -- & 3332\\
$(c;0,0,0;4)$ & 4 & $(106,31)$ & 10 & 2 & YES & YES & YES & $1.60$ & $(2,3)$ & -- & 3333\\
$(c;0,0,0;4)$ & 4 & $(108,41)$ & 10 & 4 & YES & YES & YES & $1.57$ & $(4,2)$ & -- & 3334\\
$(c;0,1,0;11)$ & 5 & $(17,7)$ & 6 & 1 & YES & YES & NO(2) & $1.45$ & $(2,3)$ & -- & 3335\\
$(c;0,1,0;11)$ & 5 & $(45,19)$ & 8 & 1 & YES & YES & YES & $1.29$ & $(4,2)$ & -- & 3336\\
$(c;0,1,0;11)$ & 5 & $(56,23)$ & 9 & 1 & YES & YES & YES & $1.67$ & $(2,3)$ & -- & 3337\\
$(c;0,1,0;11)$ & 5 & $(58,17)$ & 9 & 1 & YES & YES & YES & $1.67$ & $(2,3)$ & -- & 3338\\
$(c;0,1,0;11)$ & 5 & $(61,17)$ & 9 & 1 & YES & YES & YES & $1.67$ & $(2,3)$ & -- & 3339\\
$(c;0,1,0;11)$ & 5 & $(64,27)$ & 9 & 1 & YES & YES & YES & $1.67$ & $(4,2)$ & -- & 3340\\
$(c;0,1,0;11)$ & 5 & $(65,24)$ & 9 & 1 & YES & YES & YES & $1.50$ & $(4,2)$ & -- & 3341\\
$(c;0,1,0;11)$ & 5 & $(70,29)$ & 9 & 1 & YES & YES & YES & $1.62$ & $(4,2)$ & -- & 3342\\
$(c;0,1,0;11)$ & 5 & $(79,22)$ & 10 & 1 & YES & YES & YES & $1.44$ & $(4,2)$ & -- & 3343\\
$(c;0,1,0;11)$ & 5 & $(79,24)$ & 10 & 1 & YES & YES & YES & $1.62$ & $(4,2)$ & -- & 3344\\
$(c;0,1,0;11)$ & 5 & $(99,29)$ & 10 & 11 & YES & YES & YES & $1.70$ & $(2,3)$ & -- & 3345\\
$(c;0,1,1;5)$ & 6 & $(30,11)$ & 7 & 5 & YES & YES & YES & $1.64$ & $(4,2)$ & -- & 3346\\
$(c;0,1,1;5)$ & 6 & $(41,17)$ & 8 & 1 & YES & YES & YES & $1.62$ & $(4,2)$ & -- & 3347\\
$(c;0,2,0;7)$ & 6 & $(26,11)$ & 7 & 1 & YES & YES & YES & $1.43$ & $(4,2)$ & -- & 3348\\
$(c;0,2,0;7)$ & 6 & $(37,11)$ & 8 & 1 & YES & YES & YES & $1.57$ & $(2,3)$ & -- & 3349\\
$(c;0,2,0;7)$ & 6 & $(48,11)$ & 9 & 1 & YES & YES & YES & $1.43$ & $(4,2)$ & -- & 3350\\
$(c;0,2,1;19)$ & 7 & $(16,5)$ & 7 & 1 & YES & YES & YES & $1.50$ & $(2,3)$ & -- & 3351\\
$(c;0,2,1;19)$ & 7 & $(41,12)$ & 8 & 1 & YES & YES & YES & $1.50$ & $(4,2)$ & -- & 3352\\
$(c;0,2,2;6)$ & 8 & $(21,5)$ & 8 & 3 & YES & YES & YES & $1.50$ & $(2,3)$ & -- & 3353\\
$(c;0,3,0;17)$ & 7 & $(16,5)$ & 7 & 1 & YES & YES & YES & $1.50$ & $(2,3)$ & -- & 3354\\
$(c;0,3,0;17)$ & 7 & $(24,5)$ & 8 & 1 & YES & YES & YES & $1.50$ & $(2,3)$ & -- & 3355\\
$(d;0,0,0;5)$ & 5 & $(63,26)$ & 9 & 1 & YES & YES & YES & $1.67$ & $(4,2)$ & -- & 3356\\
$(d;0,0,0;5)$ & 5 & $(64,27)$ & 9 & 1 & YES & YES & YES & $1.67$ & $(4,2)$ & -- & 3357\\
$(d;0,0,0;5)$ & 5 & $(65,24)$ & 9 & 5 & YES & YES & YES & $1.56$ & $(4,2)$ & -- & 3358\\
$(d;0,0,0;5)$ & 5 & $(70,29)$ & 9 & 5 & YES & YES & YES & $1.67$ & $(4,2)$ & -- & 3359\\
$(d;0,0,0;5)$ & 5 & $(75,31)$ & 9 & 5 & YES & YES & YES & $1.56$ & $(4,2)$ & -- & 3360\\
$(d;0,0,0;5)$ & 5 & $(79,24)$ & 10 & 1 & YES & YES & YES & $1.56$ & $(4,2)$ & -- & 3361\\
$(d;0,0,0;5)$ & 5 & $(104,29)$ & 10 & 1 & YES & YES & YES & $1.67$ & $(4,2)$ & -- & 3362\\
$(d;0,0,1;14)$ & 6 & $(23,9)$ & 7 & 1 & YES & YES & YES & $1.62$ & $(2,3)$ & -- & 3363\\
$(d;0,0,1;14)$ & 6 & $(39,16)$ & 8 & 1 & YES & YES & YES & $1.38$ & $(4,2)$ & -- & 3364\\
$(d;0,0,1;14)$ & 6 & $(41,17)$ & 8 & 1 & YES & YES & YES & $1.62$ & $(4,2)$ & -- & 3365\\
$(d;0,0,1;14)$ & 6 & $(46,17)$ & 8 & 2 & YES & YES & YES & $1.56$ & $(4,2)$ & -- & 3366\\
$(d;0,0,2;9)$ & 7 & $(7,3)$ & 4 & 1 & YES & YES & NO(2) & $1.40$ & $(2,3)$ & -- & 3367\\
$(d;0,0,2;9)$ & 7 & $(16,5)$ & 7 & 1 & YES & YES & YES & $1.50$ & $(2,3)$ & -- & 3368\\
$(d;0,1,0;6)$ & 6 & $(41,12)$ & 8 & 1 & YES & YES & YES & $1.56$ & $(6,1)$ & -- & 3369\\
$(d;0,1,0;6)$ & 6 & $(43,12)$ & 8 & 1 & YES & YES & YES & $1.56$ & $(6,1)$ & -- & 3370\\
$(d;0,1,1;17)$ & 7 & $(34,13)$ & 7 & 17 & YES & YES & YES & $1.56$ & $(4,2)$ & -- & 3371\\
$(d;0,1,1;17)$ & 7 & $(41,12)$ & 8 & 1 & YES & YES & YES & $1.67$ & $(4,2)$ & -- & 3372\\
$(d;0,1,2;11)$ & 8 & $(9,4)$ & 5 & 1 & YES & YES & YES & $1.44$ & $(2,3)$ & -- & 3373\\
$(e;1,0,0;18)$ & 6 & $(12,5)$ & 5 & 6 & YES & YES & YES & $1.38$ & $(6,1)$ & -- & 3374\\
$(e;1,0,0;18)$ & 6 & $(17,7)$ & 6 & 1 & YES & YES & YES & $1.62$ & $(2,3)$ & -- & 3375\\
$(e;1,0,0;18)$ & 6 & $(21,8)$ & 6 & 3 & YES & YES & YES & $1.56$ & $(6,1)$ & -- & 3376\\
$(e;1,0,0;18)$ & 6 & $(23,9)$ & 7 & 1 & YES & YES & YES & $1.78$ & $(2,3)$ & -- & 3377\\
$(e;1,0,0;18)$ & 6 & $(24,7)$ & 7 & 6 & YES & YES & YES & $1.83$ & $(2,3)$ & -- & 3378\\
$(e;1,0,0;18)$ & 6 & $(33,10)$ & 8 & 3 & YES & YES & YES & $1.67$ & $(4,2)$ & -- & 3379\\
$(e;1,1,0;23)$ & 7 & $(12,5)$ & 5 & 1 & YES & YES & YES & $1.64$ & $(2,3)$ & -- & 3380\\
$(e;1,1,0;23)$ & 7 & $(13,5)$ & 5 & 1 & YES & YES & YES & $1.56$ & $(2,3)$ & -- & 3381\\
$(e;1,2,0;28)$ & 8 & $(13,4)$ & 6 & 1 & YES & YES & YES & $1.29$ & $(4,2)$ & -- & 3382\\
$(e;1,2,0;28)$ & 8 & $(13,5)$ & 5 & 1 & YES & YES & YES & $1.62$ & $(4,2)$ & -- & 3383\\
$(e;2,0,0;24)$ & 7 & $(13,5)$ & 5 & 1 & YES & YES & YES & $1.75$ & $(2,3)$ & -- & 3384\\
$(e;2,0,0;24)$ & 7 & $(17,5)$ & 6 & 1 & YES & YES & YES & $1.43$ & $(2,3)$ & -- & 3385\\
$(e;2,0,0;24)$ & 7 & $(18,7)$ & 6 & 6 & YES & YES & YES & $1.43$ & $(4,2)$ & -- & 3386\\
$(e;2,0,0;24)$ & 7 & $(21,8)$ & 6 & 3 & YES & YES & YES & $1.67$ & $(4,2)$ & -- & 3387\\
$(e;2,3,0;45)$ & 10 & $(6,1)$ & 5 & 3 & YES & YES & YES & $1.38$ & $(2,3)$ & -- & 3388\\
$(f;0,0,0;6)$ & 4 & $(22,9)$ & 7 & 2 & YES & YES & NO(2) & $1.55$ & $(2,3)$ & -- & 3389\\
$(f;0,0,0;6)$ & 4 & $(23,9)$ & 7 & 1 & YES & YES & NO(2) & $1.55$ & $(2,3)$ & -- & 3390\\
$(f;0,0,0;6)$ & 4 & $(26,11)$ & 7 & 2 & YES & YES & NO(2) & $1.40$ & $(2,3)$ & -- & 3391\\
$(f;0,0,0;6)$ & 4 & $(30,11)$ & 7 & 6 & YES & YES & NO(2) & $1.40$ & $(2,3)$ & -- & 3392\\
$(f;0,0,0;6)$ & 4 & $(37,11)$ & 8 & 1 & YES & YES & YES & $1.44$ & $(2,3)$ & -- & 3393\\
$(f;0,0,0;6)$ & 4 & $(37,16)$ & 9 & 1 & YES & YES & YES & $1.50$ & $(2,3)$ & -- & 3394\\
$(f;0,0,0;6)$ & 4 & $(41,15)$ & 8 & 1 & YES & YES & YES & $1.44$ & $(2,3)$ & -- & 3395\\
$(f;0,0,0;6)$ & 4 & $(45,16)$ & 9 & 3 & YES & YES & YES & $1.29$ & $(4,2)$ & -- & 3396\\
$(f;0,0,0;6)$ & 4 & $(45,17)$ & 9 & 3 & YES & YES & YES & $1.50$ & $(2,3)$ & -- & 3397\\
$(f;0,0,0;6)$ & 4 & $(69,29)$ & 9 & 3 & YES & YES & YES & $1.50$ & $(6,1)$ & -- & 3398\\
$(f;0,0,0;6)$ & 4 & $(80,33)$ & 10 & 2 & YES & YES & YES & $1.43$ & $(6,1)$ & -- & 3399\\
$(f;0,0,0;6)$ & 4 & $(89,25)$ & 10 & 1 & YES & YES & YES & $1.14$ & $(4,2)$ & -- & 3400\\
$(f;0,0,0;6)$ & 4 & $(91,27)$ & 10 & 1 & YES & YES & YES & $1.62$ & $(6,1)$ & -- & 3401\\
$(f;0,0,0;6)$ & 4 & $(97,37)$ & 10 & 1 & YES & YES & YES & $1.43$ & $(4,2)$ & -- & 3402\\
$(f;0,0,0;6)$ & 4 & $(98,27)$ & 10 & 2 & YES & YES & YES & $1.71$ & $(2,3)$ & -- & 3403\\
$(f;0,0,0;6)$ & 4 & $(106,41)$ & 10 & 2 & YES & YES & YES & $1.70$ & $(2,3)$ & -- & 3404\\
$(f;0,0,0;6)$ & 4 & $(111,46)$ & 10 & 3 & YES & YES & YES & $1.67$ & $(4,2)$ & -- & 3405\\
$(f;0,0,0;6)$ & 4 & $(123,47)$ & 10 & 3 & YES & YES & YES & $1.56$ & $(4,2)$ & -- & 3406\\
$(f;0,0,0;6)$ & 4 & $(124,23)$ & 12 & 2 & YES & YES & YES & $1.38$ & $(6,1)$ & -- & 3407\\
$(f;0,0,0;6)$ & 4 & $(140,39)$ & 11 & 2 & YES & YES & YES & $1.38$ & $(4,2)$ & -- & 3408\\
$(f;0,0,0;6)$ & 4 & $(140,41)$ & 11 & 2 & YES & YES & YES & $1.60$ & $(2,3)$ & -- & 3409\\
$(f;0,1,0;7)$ & 5 & $(19,4)$ & 7 & 1 & YES & YES & YES & $1.43$ & $(2,3)$ & -- & 3410\\
$(f;0,1,0;7)$ & 5 & $(24,11)$ & 8 & 1 & YES & YES & YES & $1.43$ & $(2,3)$ & -- & 3411\\
$(f;0,1,0;7)$ & 5 & $(29,7)$ & 10 & 1 & YES & YES & YES & $1.43$ & $(2,3)$ & -- & 3412\\
$(g;0,0,0;19)$ & 6 & $(12,5)$ & 5 & 1 & YES & YES & YES & $1.64$ & $(2,3)$ & -- & 3413\\
$(g;0,0,0;19)$ & 6 & $(17,7)$ & 6 & 1 & YES & YES & YES & $1.83$ & $(2,3)$ & -- & 3414\\
$(g;0,0,0;19)$ & 6 & $(21,8)$ & 6 & 1 & YES & YES & YES & $1.70$ & $(2,3)$ & -- & 3415\\
$(g;0,0,0;19)$ & 6 & $(23,9)$ & 7 & 1 & YES & YES & YES & $1.29$ & $(4,2)$ & -- & 3416\\
$(g;0,0,0;19)$ & 6 & $(23,10)$ & 7 & 1 & YES & YES & YES & $1.50$ & $(6,1)$ & -- & 3417\\
$(g;0,0,0;19)$ & 6 & $(24,7)$ & 7 & 1 & YES & YES & YES & $1.70$ & $(2,3)$ & -- & 3418\\
$(g;0,0,1;26)$ & 7 & $(13,5)$ & 5 & 13 & YES & YES & YES & $1.56$ & $(2,3)$ & -- & 3419\\
$(g;0,0,1;26)$ & 7 & $(17,5)$ & 6 & 1 & YES & YES & YES & $1.56$ & $(2,3)$ & -- & 3420\\
$(g;0,0,1;26)$ & 7 & $(17,7)$ & 6 & 1 & YES & YES & YES & $1.78$ & $(4,2)$ & -- & 3421\\
$(g;0,0,2;11)$ & 8 & $(10,3)$ & 5 & 1 & YES & YES & YES & $1.75$ & $(2,3)$ & -- & 3422\\
$(g;0,0,2;11)$ & 8 & $(11,3)$ & 5 & 11 & YES & YES & YES & $1.75$ & $(2,3)$ & -- & 3423\\
$(g;0,0,2;11)$ & 8 & $(13,5)$ & 5 & 1 & YES & YES & YES & $1.70$ & $(2,3)$ & -- & 3424\\
$(g;0,1,0;24)$ & 7 & $(9,4)$ & 5 & 3 & YES & YES & NO(2) & $1.50$ & $(2,3)$ & -- & 3425\\
$(g;0,1,0;24)$ & 7 & $(11,4)$ & 5 & 1 & YES & YES & YES & $1.57$ & $(2,3)$ & -- & 3426\\
$(g;0,1,0;24)$ & 7 & $(13,5)$ & 5 & 1 & YES & YES & YES & $1.83$ & $(2,3)$ & -- & 3427\\
$(g;0,1,0;24)$ & 7 & $(17,5)$ & 6 & 1 & YES & YES & YES & $1.73$ & $(2,3)$ & -- & 3428\\
$(g;0,1,1;33)$ & 8 & $(8,3)$ & 4 & 1 & YES & YES & YES & $1.56$ & $(2,3)$ & -- & 3429\\
$(g;0,1,1;33)$ & 8 & $(10,3)$ & 5 & 1 & YES & YES & YES & $1.56$ & $(2,3)$ & -- & 3430\\
$(g;0,2,0;29)$ & 8 & $(10,3)$ & 5 & 1 & YES & YES & YES & $1.75$ & $(2,3)$ & -- & 3431\\
$(g;0,2,2;17)$ & 10 & $(5,1)$ & 4 & 1 & YES & YES & YES & $1.29$ & $(2,3)$ & -- & 3432\\
$(g;1,0,1;38)$ & 8 & $(16,5)$ & 7 & 2 & YES & YES & YES & $1.43$ & $(4,2)$ & -- & 3433\\
$(g;1,1,0;9)$ & 8 & $(7,3)$ & 4 & 1 & YES & YES & YES & $1.64$ & $(2,3)$ & -- & 3434\\
$(g;1,1,0;9)$ & 8 & $(13,5)$ & 5 & 1 & YES & YES & YES & $1.50$ & $(4,2)$ & -- & 3435\\
$(g;3,1,0;30)$ & 10 & $(2,1)$ & 1 & 2 & YES & YES & YES & $1.43$ & $(2,3)$ & -- & 3436\\
$(h;0,0,0;6)$ & 5 & $(21,8)$ & 6 & 3 & YES & YES & YES & $1.38$ & $(6,1)$ & -- & 3437\\
$(h;0,0,0;6)$ & 5 & $(27,10)$ & 7 & 3 & YES & YES & YES & $1.50$ & $(4,2)$ & -- & 3438\\
$(h;0,0,0;6)$ & 5 & $(31,12)$ & 7 & 1 & YES & YES & YES & $1.75$ & $(2,3)$ & -- & 3439\\
$(h;0,0,0;6)$ & 5 & $(37,14)$ & 8 & 1 & YES & YES & YES & $1.43$ & $(4,2)$ & -- & 3440\\
$(h;0,1,0;8)$ & 6 & $(12,5)$ & 5 & 4 & YES & YES & YES & $1.64$ & $(2,3)$ & -- & 3441\\
$(h;0,1,0;8)$ & 6 & $(17,7)$ & 6 & 1 & YES & YES & YES & $1.43$ & $(2,3)$ & -- & 3442\\
$(h;0,1,0;8)$ & 6 & $(21,8)$ & 6 & 1 & YES & YES & YES & $1.70$ & $(2,3)$ & -- & 3443\\
$(h;0,1,0;8)$ & 6 & $(23,9)$ & 7 & 1 & YES & YES & YES & $1.29$ & $(4,2)$ & -- & 3444\\
$(h;0,1,0;8)$ & 6 & $(24,7)$ & 7 & 8 & YES & YES & YES & $1.70$ & $(2,3)$ & -- & 3445\\
$(h;0,2,0;10)$ & 7 & $(13,5)$ & 5 & 1 & YES & YES & YES & $1.83$ & $(2,3)$ & -- & 3446\\
$(h;0,2,0;10)$ & 7 & $(18,7)$ & 6 & 2 & YES & YES & YES & $1.43$ & $(4,2)$ & -- & 3447\\
$(h;0,2,0;10)$ & 7 & $(24,7)$ & 7 & 2 & YES & YES & YES & $1.43$ & $(4,2)$ & -- & 3448\\
$(i;0,0,0;9)$ & 5 & $(12,5)$ & 5 & 3 & YES & YES & NO(2) & $1.40$ & $(2,3)$ & -- & 3449\\
$(i;0,0,0;9)$ & 5 & $(16,7)$ & 6 & 1 & YES & YES & YES & $1.44$ & $(2,3)$ & -- & 3450\\
$(i;0,0,0;9)$ & 5 & $(26,11)$ & 7 & 1 & YES & YES & YES & $1.50$ & $(2,3)$ & -- & 3451\\
$(i;0,0,0;9)$ & 5 & $(35,13)$ & 8 & 1 & YES & YES & YES & $1.50$ & $(4,2)$ & -- & 3452\\
$(i;0,0,0;9)$ & 5 & $(43,12)$ & 8 & 1 & YES & YES & YES & $1.29$ & $(4,2)$ & -- & 3453\\
$(i;0,1,0;12)$ & 6 & $(13,4)$ & 6 & 1 & YES & YES & YES & $1.50$ & $(2,3)$ & -- & 3454\\
$(i;0,1,0;12)$ & 6 & $(33,10)$ & 8 & 3 & YES & YES & YES & $1.57$ & $(2,3)$ & -- & 3455\\
$(i;0,2,0;15)$ & 7 & $(9,4)$ & 5 & 3 & YES & YES & YES & $1.50$ & $(2,3)$ & -- & 3456\\
$(i;0,2,0;15)$ & 7 & $(24,7)$ & 7 & 3 & YES & YES & YES & $1.71$ & $(2,3)$ & -- & 3457\\
$(j;0,0,0;8)$ & 5 & $(32,13)$ & 9 & 8 & YES & YES & YES & $1.50$ & $(2,3)$ & -- & 3458\\
$(j;0,0,0;8)$ & 5 & $(40,17)$ & 9 & 8 & YES & YES & YES & $1.50$ & $(2,3)$ & -- & 3459\\
$(j;0,0,0;8)$ & 5 & $(75,29)$ & 9 & 1 & YES & YES & YES & $1.67$ & $(4,2)$ & -- & 3460\\
$(j;0,0,0;8)$ & 5 & $(76,29)$ & 9 & 4 & YES & YES & YES & $1.38$ & $(4,2)$ & -- & 3461\\
$(j;0,0,0;8)$ & 5 & $(89,26)$ & 10 & 1 & YES & YES & YES & $1.70$ & $(2,3)$ & -- & 3462\\
$(j;0,1,0;10)$ & 6 & $(27,11)$ & 8 & 1 & YES & YES & YES & $1.50$ & $(2,3)$ & -- & 3463\\
$(j;0,1,0;10)$ & 6 & $(37,11)$ & 8 & 1 & YES & YES & YES & $1.67$ & $(2,3)$ & -- & 3464\\
$(j;0,1,0;10)$ & 6 & $(43,13)$ & 9 & 1 & YES & YES & YES & $1.57$ & $(4,2)$ & -- & 3465
\end{longtable}
\subsection{2 chains, $K^2 = 4$}
\begin{longtable}{|c|c|c|c|c|c|c|c|c|c|c|c|}
\hline
\multicolumn{12}{|c|}{2 chains, $K^2 = 4$}\\
\hline
$(n,a)$ & Len & $(n,a)$ & Len & GCD & Nef & $\mathbb Q$-ef & Obs 0 & $\overline c_1^2 / \overline c_2$ & $(P,K)$ & WH & Index\\
\hline
\endfirsthead

\hline
$(n,a)$ & Len & $(n,a)$ & Len & GCD & Nef & $\mathbb Q$-ef & Obs 0 & $\overline c_1^2 / \overline c_2$ & $(P,K)$ & WH & Index\\
\hline
\endhead
\hline
\endfoot

$(29,9)$ & 8 & $(25,9)$ & 7 & 1 & YES & YES & YES & $1.67$ & $(4,3)$ & -- & 3466\\
$(39,14)$ & 8 & $(12,5)$ & 5 & 3 & YES & YES & YES & $1.83$ & $(4,3)$ & -- & 3467\\
$(45,19)$ & 8 & $(44,13)$ & 8 & 1 & YES & YES & NO(2) & $2.36$ & $(2,4)$ & -- & 3468\\
$(49,19)$ & 8 & $(40,11)$ & 8 & 1 & YES & YES & NO(2) & $2.00$ & $(2,4)$ & -- & 3469\\
$(56,15)$ & 9 & $(43,18)$ & 8 & 1 & YES & YES & NO(2) & $2.00$ & $(4,3)$ & -- & 3470\\
$(57,16)$ & 9 & $(41,12)$ & 8 & 1 & YES & YES & YES & $2.00$ & $(2,4)$ & -- & 3471\\
$(58,17)$ & 9 & $(50,19)$ & 8 & 2 & YES & YES & YES & $2.14$ & $(2,4)$ & -- & 3472\\
$(61,17)$ & 9 & $(55,21)$ & 8 & 1 & YES & YES & YES & $2.00$ & $(2,4)$ & -- & 3473\\
$(63,26)$ & 9 & $(35,8)$ & 8 & 7 & YES & YES & YES & $2.00$ & $(2,4)$ & -- & 3474\\
$(64,27)$ & 9 & $(40,11)$ & 8 & 8 & YES & YES & YES & $2.00$ & $(2,4)$ & NO & 3475\\
$(64,19)$ & 9 & $(45,19)$ & 8 & 1 & YES & YES & YES & $2.25$ & $(6,2)$ & -- & 3476\\
$(65,27)$ & 10 & $(34,13)$ & 7 & 1 & YES & YES & YES & $2.00$ & $(4,3)$ & -- & 3477\\
$(65,18)$ & 9 & $(46,17)$ & 8 & 1 & YES & YES & YES & $2.11$ & $(2,4)$ & -- & 3478\\
$(69,29)$ & 9 & $(40,11)$ & 8 & 1 & YES & YES & YES & $2.10$ & $(2,4)$ & -- & 3479\\
$(71,30)$ & 9 & $(27,8)$ & 7 & 1 & YES & YES & NO(2) & $2.27$ & $(2,4)$ & -- & 3480\\
$(71,21)$ & 9 & $(44,17)$ & 8 & 1 & YES & YES & NO(2) & $1.89$ & $(4,3)$ & -- & 3481\\
$(76,21)$ & 9 & $(44,17)$ & 8 & 4 & YES & YES & YES & $2.00$ & $(2,4)$ & -- & 3482\\
$(79,24)$ & 10 & $(19,8)$ & 6 & 1 & YES & YES & YES & $2.00$ & $(2,4)$ & -- & 3483\\
$(79,30)$ & 9 & $(23,9)$ & 7 & 1 & YES & YES & YES & $1.83$ & $(4,3)$ & -- & 3484\\
$(80,31)$ & 9 & $(37,11)$ & 8 & 1 & YES & YES & YES & $2.12$ & $(6,2)$ & -- & 3485\\
$(83,23)$ & 10 & $(32,7)$ & 8 & 1 & YES & YES & YES & $1.86$ & $(4,3)$ & NO & 3486\\
$(89,25)$ & 10 & $(19,8)$ & 6 & 1 & YES & YES & NO(3) & $1.83$ & $(2,4)$ & -- & 3487\\
$(91,27)$ & 10 & $(27,10)$ & 7 & 1 & YES & YES & NO(2) & $2.00$ & $(4,3)$ & -- & 3488\\
$(92,35)$ & 10 & $(29,8)$ & 7 & 1 & YES & YES & YES & $2.00$ & $(2,4)$ & -- & 3489\\
$(95,36)$ & 10 & $(24,7)$ & 7 & 1 & YES & YES & YES & $2.12$ & $(2,4)$ & -- & 3490\\
$(97,37)$ & 10 & $(32,7)$ & 8 & 1 & YES & YES & YES & $2.00$ & $(2,4)$ & NO & 3491\\
$(98,41)$ & 10 & $(18,7)$ & 6 & 2 & YES & YES & YES & $1.83$ & $(4,3)$ & -- & 3492\\
$(98,27)$ & 10 & $(22,9)$ & 7 & 2 & YES & YES & YES & $2.11$ & $(2,4)$ & -- & 3493\\
$(98,27)$ & 10 & $(26,11)$ & 7 & 2 & YES & YES & YES & $2.00$ & $(2,4)$ & NO & 3494\\
$(98,27)$ & 10 & $(44,17)$ & 8 & 2 & YES & YES & YES & $2.14$ & $(2,4)$ & NO & 3495\\
$(98,27)$ & 10 & $(61,18)$ & 9 & 1 & YES & YES & YES & $2.00$ & $(2,4)$ & NO & 3496\\
$(100,37)$ & 10 & $(31,7)$ & 8 & 1 & YES & YES & NO(2) & $2.00$ & $(2,4)$ & NO & 3497\\
$(101,30)$ & 10 & $(18,7)$ & 6 & 1 & YES & YES & NO(2) & $1.75$ & $(6,2)$ & -- & 3498\\
$(101,39)$ & 10 & $(18,7)$ & 6 & 1 & YES & YES & YES & $1.83$ & $(4,3)$ & -- & 3499\\
$(106,41)$ & 10 & $(13,5)$ & 5 & 1 & YES & YES & YES & $1.83$ & $(4,3)$ & -- & 3500\\
$(108,41)$ & 10 & $(17,4)$ & 7 & 1 & YES & YES & YES & $2.00$ & $(2,4)$ & NO & 3501\\
$(109,45)$ & 10 & $(25,7)$ & 7 & 1 & YES & YES & NO(2) & $2.12$ & $(4,3)$ & NO & 3502\\
$(109,30)$ & 10 & $(32,9)$ & 8 & 1 & YES & YES & YES & $2.00$ & $(2,4)$ & -- & 3503\\
$(111,43)$ & 10 & $(25,7)$ & 7 & 1 & YES & YES & YES & $2.14$ & $(2,4)$ & -- & 3504\\
$(112,31)$ & 10 & $(21,8)$ & 6 & 7 & YES & YES & YES & $2.00$ & $(2,4)$ & NO & 3505\\
$(112,31)$ & 10 & $(32,9)$ & 8 & 16 & YES & YES & YES & $2.00$ & $(2,4)$ & -- & 3506\\
$(112,47)$ & 10 & $(56,23)$ & 9 & 56 & YES & YES & NO(2) & $2.20$ & $(2,4)$ & NO & 3507\\
$(113,49)$ & 11 & $(13,4)$ & 6 & 1 & YES & YES & YES & $1.83$ & $(4,3)$ & -- & 3508\\
$(119,46)$ & 10 & $(18,5)$ & 6 & 1 & YES & YES & YES & $2.00$ & $(2,4)$ & -- & 3509\\
$(121,37)$ & 11 & $(12,5)$ & 5 & 1 & YES & YES & YES & $1.86$ & $(4,3)$ & -- & 3510\\
$(121,37)$ & 11 & $(29,8)$ & 7 & 1 & YES & YES & YES & $2.38$ & $(6,2)$ & -- & 3511\\
$(121,37)$ & 11 & $(44,13)$ & 8 & 11 & YES & YES & YES & $1.86$ & $(4,3)$ & NO & 3512\\
$(124,23)$ & 12 & $(21,8)$ & 6 & 1 & YES & YES & YES & $1.88$ & $(2,4)$ & -- & 3513\\
$(127,29)$ & 11 & $(37,11)$ & 8 & 1 & YES & YES & YES & $2.00$ & $(2,4)$ & NO & 3514\\
$(129,50)$ & 10 & $(25,7)$ & 7 & 1 & YES & YES & YES & $2.14$ & $(2,4)$ & -- & 3515\\
$(131,50)$ & 10 & $(10,3)$ & 5 & 1 & YES & YES & NO(2) & $2.00$ & $(2,4)$ & -- & 3516\\
$(131,55)$ & 10 & $(63,26)$ & 9 & 1 & YES & YES & NO(2) & $2.10$ & $(2,4)$ & NO & 3517\\
$(134,39)$ & 11 & $(29,8)$ & 7 & 1 & YES & YES & YES & $2.00$ & $(2,4)$ & -- & 3518\\
$(137,37)$ & 11 & $(37,11)$ & 8 & 1 & YES & YES & NO(2) & $2.12$ & $(4,3)$ & NO & 3519\\
$(149,41)$ & 11 & $(10,3)$ & 5 & 1 & YES & YES & YES & $1.83$ & $(4,3)$ & -- & 3520\\
$(149,44)$ & 11 & $(13,5)$ & 5 & 1 & YES & YES & YES & $2.00$ & $(2,4)$ & -- & 3521\\
$(153,56)$ & 11 & $(13,5)$ & 5 & 1 & YES & YES & YES & $2.00$ & $(4,3)$ & -- & 3522\\
$(154,45)$ & 11 & $(10,3)$ & 5 & 2 & YES & YES & YES & $2.00$ & $(4,3)$ & -- & 3523\\
$(157,46)$ & 11 & $(17,7)$ & 6 & 1 & YES & YES & NO(2) & $2.00$ & $(4,3)$ & NO & 3524\\
$(163,44)$ & 11 & $(17,7)$ & 6 & 1 & YES & YES & YES & $2.00$ & $(2,4)$ & -- & 3525\\
$(163,44)$ & 11 & $(33,10)$ & 8 & 1 & YES & YES & YES & $2.00$ & $(2,4)$ & NO & 3526\\
$(166,61)$ & 11 & $(18,7)$ & 6 & 2 & YES & YES & YES & $2.00$ & $(4,3)$ & -- & 3527\\
$(166,61)$ & 11 & $(44,17)$ & 8 & 2 & YES & YES & YES & $2.00$ & $(4,3)$ & NO & 3528\\
$(169,50)$ & 11 & $(23,7)$ & 7 & 1 & YES & YES & YES & $2.00$ & $(2,4)$ & -- & 3529\\
$(170,47)$ & 11 & $(44,13)$ & 8 & 2 & YES & YES & YES & $2.00$ & $(2,4)$ & NO & 3530\\
$(170,47)$ & 11 & $(89,25)$ & 10 & 1 & YES & YES & YES & $2.00$ & $(2,4)$ & NO & 3531\\
$(171,50)$ & 11 & $(17,7)$ & 6 & 1 & YES & YES & NO(2) & $1.88$ & $(4,3)$ & NO & 3532\\
$(189,55)$ & 12 & $(64,19)$ & 9 & 1 & YES & YES & NO(2) & $2.00$ & $(4,3)$ & NO & 3533\\
$(194,75)$ & 11 & $(13,4)$ & 6 & 1 & YES & YES & NO(2) & $2.00$ & $(4,3)$ & -- & 3534\\
$(203,60)$ & 12 & $(12,5)$ & 5 & 1 & YES & YES & YES & $1.86$ & $(4,3)$ & -- & 3535\\
$(214,79)$ & 12 & $(10,3)$ & 5 & 2 & YES & YES & YES & $2.00$ & $(4,3)$ & -- & 3536\\
$(227,87)$ & 12 & $(5,1)$ & 4 & 1 & YES & YES & YES & $1.83$ & $(2,4)$ & -- & 3537\\
$(234,89)$ & 12 & $(7,2)$ & 4 & 1 & YES & YES & NO(2) & $1.91$ & $(2,4)$ & -- & 3538\\
$(235,97)$ & 12 & $(10,3)$ & 5 & 5 & YES & YES & YES & $2.12$ & $(2,4)$ & -- & 3539\\
$(236,65)$ & 12 & $(24,7)$ & 7 & 4 & YES & YES & YES & $2.00$ & $(8,1)$ & -- & 3540\\
$(237,100)$ & 12 & $(10,3)$ & 5 & 1 & YES & YES & NO(2) & $1.86$ & $(6,2)$ & -- & 3541\\
$(242,65)$ & 12 & $(13,4)$ & 6 & 1 & YES & YES & YES & $2.12$ & $(2,4)$ & -- & 3542\\
$(242,65)$ & 12 & $(24,7)$ & 7 & 2 & YES & YES & YES & $2.12$ & $(2,4)$ & NO & 3543\\
$(246,73)$ & 12 & $(10,3)$ & 5 & 2 & YES & YES & YES & $2.00$ & $(2,4)$ & -- & 3544\\
$(253,106)$ & 12 & $(7,3)$ & 4 & 1 & YES & YES & YES & $2.00$ & $(2,4)$ & -- & 3545\\
$(253,68)$ & 12 & $(22,5)$ & 7 & 11 & YES & YES & YES & $2.12$ & $(6,2)$ & -- & 3546\\
$(254,105)$ & 12 & $(26,11)$ & 7 & 2 & YES & YES & YES & $2.00$ & $(2,4)$ & NO & 3547\\
$(257,108)$ & 12 & $(11,3)$ & 5 & 1 & YES & YES & YES & $2.12$ & $(6,2)$ & -- & 3548\\
$(265,112)$ & 12 & $(11,3)$ & 5 & 1 & YES & YES & NO(2) & $1.88$ & $(6,2)$ & NO & 3549\\
$(266,101)$ & 12 & $(44,17)$ & 8 & 2 & YES & YES & YES & $2.00$ & $(4,3)$ & NO & 3550\\
$(274,115)$ & 12 & $(22,9)$ & 7 & 2 & YES & YES & YES & $2.11$ & $(2,4)$ & NO & 3551\\
$(277,116)$ & 12 & $(10,3)$ & 5 & 1 & YES & YES & YES & $2.11$ & $(2,4)$ & NO & 3552\\
$(277,116)$ & 12 & $(179,75)$ & 11 & 1 & YES & YES & YES & $2.11$ & $(2,4)$ & NO & 3553\\
$(292,85)$ & 13 & $(8,3)$ & 4 & 4 & YES & YES & YES & $1.88$ & $(4,3)$ & -- & 3554\\
$(292,111)$ & 12 & $(8,3)$ & 4 & 4 & YES & YES & YES & $1.86$ & $(4,3)$ & -- & 3555\\
$(292,111)$ & 12 & $(263,100)$ & 12 & 1 & YES & YES & YES & $1.86$ & $(4,3)$ & NO & 3556\\
$(295,112)$ & 12 & $(11,3)$ & 5 & 1 & YES & YES & NO(2) & $1.88$ & $(6,2)$ & NO & 3557\\
$(298,123)$ & 13 & $(5,2)$ & 3 & 1 & YES & YES & YES & $1.83$ & $(4,3)$ & -- & 3558\\
$(301,115)$ & 12 & $(8,3)$ & 4 & 1 & YES & YES & YES & $2.00$ & $(2,4)$ & -- & 3559\\
$(303,116)$ & 12 & $(10,3)$ & 5 & 1 & YES & YES & YES & $2.14$ & $(2,4)$ & -- & 3560\\
$(304,85)$ & 13 & $(11,4)$ & 5 & 1 & YES & YES & YES & $2.00$ & $(4,3)$ & -- & 3561\\
$(312,131)$ & 12 & $(17,7)$ & 6 & 1 & YES & YES & NO(2) & $1.89$ & $(4,3)$ & NO & 3562\\
$(313,121)$ & 12 & $(5,2)$ & 3 & 1 & YES & YES & YES & $2.00$ & $(2,4)$ & -- & 3563\\
$(313,91)$ & 13 & $(10,3)$ & 5 & 1 & YES & YES & YES & $2.00$ & $(2,4)$ & -- & 3564\\
$(313,91)$ & 13 & $(37,11)$ & 8 & 1 & YES & YES & YES & $2.00$ & $(2,4)$ & NO & 3565\\
$(313,91)$ & 13 & $(44,13)$ & 8 & 1 & YES & YES & YES & $2.00$ & $(2,4)$ & NO & 3566\\
$(317,131)$ & 12 & $(5,2)$ & 3 & 1 & YES & YES & NO(2) & $1.89$ & $(4,3)$ & -- & 3567\\
$(317,89)$ & 14 & $(7,1)$ & 6 & 1 & YES & YES & NO(3) & $1.83$ & $(2,4)$ & NO & 3568\\
$(317,131)$ & 12 & $(9,2)$ & 5 & 1 & YES & YES & NO(2) & $2.18$ & $(2,4)$ & NO & 3569\\
$(317,131)$ & 12 & $(167,69)$ & 11 & 1 & YES & YES & NO(2) & $1.89$ & $(4,3)$ & NO & 3570\\
$(321,95)$ & 13 & $(5,2)$ & 3 & 1 & YES & YES & YES & $1.88$ & $(2,4)$ & NO & 3571\\
$(323,134)$ & 13 & $(7,2)$ & 4 & 1 & YES & YES & YES & $2.00$ & $(4,3)$ & NO & 3572\\
$(324,91)$ & 13 & $(203,57)$ & 12 & 1 & YES & YES & YES & $1.88$ & $(2,4)$ & 3670 & 3573\\
$(326,99)$ & 13 & $(7,3)$ & 4 & 1 & YES & YES & NO(2) & $1.89$ & $(4,3)$ & -- & 3574\\
$(326,99)$ & 13 & $(25,7)$ & 7 & 1 & YES & YES & YES & $2.14$ & $(2,4)$ & NO & 3575\\
$(332,97)$ & 13 & $(3,1)$ & 2 & 1 & YES & YES & YES & $2.00$ & $(4,3)$ & -- & 3576\\
$(332,97)$ & 13 & $(16,3)$ & 7 & 4 & YES & YES & NO(2) & $1.75$ & $(6,2)$ & NO & 3577\\
$(332,97)$ & 13 & $(41,12)$ & 8 & 1 & YES & YES & YES & $2.00$ & $(4,3)$ & NO & 3578\\
$(333,101)$ & 13 & $(201,61)$ & 12 & 3 & YES & YES & NO(2) & $1.88$ & $(6,2)$ & NO & 3579\\
$(337,100)$ & 13 & $(5,2)$ & 3 & 1 & YES & YES & NO(2) & $1.75$ & $(6,2)$ & -- & 3580\\
$(337,100)$ & 13 & $(101,30)$ & 10 & 1 & YES & YES & NO(2) & $1.75$ & $(6,2)$ & 3669 & 3581\\
$(338,129)$ & 12 & $(7,3)$ & 4 & 1 & YES & YES & YES & $2.11$ & $(2,4)$ & -- & 3582\\
$(338,129)$ & 12 & $(131,50)$ & 10 & 1 & YES & YES & NO(2) & $2.00$ & $(2,4)$ & NO & 3583\\
$(346,131)$ & 13 & $(34,13)$ & 7 & 2 & YES & YES & YES & $1.83$ & $(4,3)$ & NO & 3584\\
$(347,134)$ & 13 & $(7,2)$ & 4 & 1 & YES & YES & YES & $2.00$ & $(4,3)$ & NO & 3585\\
$(356,139)$ & 13 & $(4,1)$ & 3 & 4 & YES & YES & YES & $1.83$ & $(4,3)$ & NO & 3586\\
$(356,139)$ & 13 & $(4,1)$ & 3 & 4 & YES & YES & YES & $1.83$ & $(4,3)$ & -- & 3587\\
$(356,139)$ & 13 & $(8,3)$ & 4 & 4 & YES & YES & YES & $1.83$ & $(4,3)$ & NO & 3588\\
$(361,151)$ & 13 & $(2,1)$ & 1 & 1 & YES & YES & NO(3) & $1.83$ & $(2,4)$ & NO & 3589\\
$(363,100)$ & 13 & $(13,4)$ & 6 & 1 & YES & YES & YES & $2.00$ & $(4,3)$ & NO & 3590\\
$(365,108)$ & 13 & $(2,1)$ & 1 & 1 & YES & YES & YES & $1.83$ & $(4,3)$ & -- & 3591\\
$(365,108)$ & 13 & $(7,2)$ & 4 & 1 & YES & YES & YES & $2.00$ & $(2,4)$ & -- & 3592\\
$(365,108)$ & 13 & $(61,18)$ & 9 & 1 & YES & YES & YES & $2.00$ & $(2,4)$ & NO & 3593\\
$(383,112)$ & 13 & $(2,1)$ & 1 & 1 & YES & YES & YES & $2.00$ & $(4,3)$ & -- & 3594\\
$(383,161)$ & 13 & $(157,66)$ & 11 & 1 & YES & YES & YES & $2.00$ & $(2,4)$ & NO & 3595\\
$(385,167)$ & 14 & $(30,13)$ & 8 & 5 & YES & YES & YES & $2.00$ & $(2,4)$ & NO & 3596\\
$(391,108)$ & 13 & $(13,4)$ & 6 & 1 & YES & YES & NO(2) & $1.88$ & $(6,2)$ & NO & 3597\\
$(397,116)$ & 13 & $(37,11)$ & 8 & 1 & YES & YES & YES & $2.12$ & $(6,2)$ & NO & 3598\\
$(397,116)$ & 13 & $(154,45)$ & 11 & 1 & YES & YES & YES & $2.00$ & $(4,3)$ & NO & 3599\\
$(398,111)$ & 13 & $(40,11)$ & 8 & 2 & YES & YES & YES & $2.00$ & $(2,4)$ & NO & 3600\\
$(400,117)$ & 13 & $(7,3)$ & 4 & 1 & YES & YES & YES & $2.11$ & $(2,4)$ & -- & 3601\\
$(401,155)$ & 13 & $(3,1)$ & 2 & 1 & YES & YES & YES & $1.88$ & $(4,3)$ & -- & 3602\\
$(401,155)$ & 13 & $(5,2)$ & 3 & 1 & YES & YES & YES & $2.00$ & $(4,3)$ & -- & 3603\\
$(401,155)$ & 13 & $(19,7)$ & 6 & 1 & YES & YES & YES & $2.00$ & $(4,3)$ & NO & 3604\\
$(402,175)$ & 14 & $(4,1)$ & 3 & 2 & YES & YES & YES & $1.83$ & $(4,3)$ & -- & 3605\\
$(402,175)$ & 14 & $(7,3)$ & 4 & 1 & YES & YES & YES & $2.00$ & $(2,4)$ & NO & 3606\\
$(403,153)$ & 13 & $(108,41)$ & 10 & 1 & YES & YES & YES & $2.00$ & $(2,4)$ & NO & 3607\\
$(407,112)$ & 13 & $(10,3)$ & 5 & 1 & YES & YES & YES & $2.12$ & $(6,2)$ & -- & 3608\\
$(407,171)$ & 13 & $(19,8)$ & 6 & 1 & YES & YES & YES & $2.00$ & $(2,4)$ & NO & 3609\\
$(407,112)$ & 13 & $(167,46)$ & 11 & 1 & YES & YES & YES & $2.12$ & $(6,2)$ & NO & 3610\\
$(407,119)$ & 13 & $(383,112)$ & 13 & 1 & YES & YES & YES & $2.00$ & $(2,4)$ & NO & 3611\\
$(409,121)$ & 13 & $(365,108)$ & 13 & 1 & YES & YES & YES & $2.25$ & $(6,2)$ & NO & 3612\\
$(422,183)$ & 14 & $(113,49)$ & 11 & 1 & YES & YES & YES & $1.83$ & $(4,3)$ & NO & 3613\\
$(424,155)$ & 14 & $(13,5)$ & 5 & 1 & YES & YES & YES & $2.00$ & $(6,2)$ & NO & 3614\\
$(431,128)$ & 13 & $(394,117)$ & 13 & 1 & YES & YES & YES & $2.00$ & $(2,4)$ & NO & 3615\\
$(433,128)$ & 13 & $(3,1)$ & 2 & 1 & YES & YES & YES & $2.00$ & $(2,4)$ & NO & 3616\\
$(433,128)$ & 13 & $(3,1)$ & 2 & 1 & YES & YES & YES & $2.00$ & $(2,4)$ & -- & 3617\\
$(433,131)$ & 14 & $(4,1)$ & 3 & 1 & YES & YES & YES & $1.71$ & $(4,3)$ & -- & 3618\\
$(435,182)$ & 14 & $(5,2)$ & 3 & 5 & YES & YES & YES & $2.17$ & $(4,3)$ & -- & 3619\\
$(437,100)$ & 14 & $(10,3)$ & 5 & 1 & YES & YES & NO(2) & $1.71$ & $(6,2)$ & NO & 3620\\
$(437,183)$ & 13 & $(26,11)$ & 7 & 1 & YES & YES & YES & $2.12$ & $(6,2)$ & NO & 3621\\
$(437,181)$ & 13 & $(128,53)$ & 11 & 1 & YES & YES & YES & $2.00$ & $(4,3)$ & NO & 3622\\
$(438,181)$ & 13 & $(196,81)$ & 11 & 2 & YES & YES & NO(2) & $2.36$ & $(2,4)$ & 3658 & 3623\\
$(438,181)$ & 13 & $(317,131)$ & 12 & 1 & YES & YES & NO(2) & $2.27$ & $(2,4)$ & NO & 3624\\
$(438,185)$ & 13 & $(438,185)$ & 13 & 438 & YES & YES & NO(2) & $2.27$ & $(2,4)$ & NO & 3625\\
$(441,169)$ & 13 & $(5,1)$ & 4 & 1 & YES & YES & YES & $1.88$ & $(2,4)$ & -- & 3626\\
$(448,173)$ & 14 & $(347,134)$ & 13 & 1 & YES & YES & YES & $1.86$ & $(4,3)$ & NO & 3627\\
$(455,188)$ & 13 & $(5,2)$ & 3 & 5 & YES & YES & YES & $2.25$ & $(4,3)$ & -- & 3628\\
$(459,179)$ & 14 & $(218,85)$ & 12 & 1 & YES & YES & YES & $2.14$ & $(4,3)$ & NO & 3629\\
$(463,176)$ & 13 & $(3,1)$ & 2 & 1 & YES & YES & NO(2) & $1.75$ & $(6,2)$ & -- & 3630\\
$(463,171)$ & 13 & $(4,1)$ & 3 & 1 & YES & YES & NO(2) & $1.88$ & $(6,2)$ & NO & 3631\\
$(463,171)$ & 13 & $(4,1)$ & 3 & 1 & YES & YES & NO(2) & $1.88$ & $(6,2)$ & -- & 3632\\
$(463,170)$ & 13 & $(5,2)$ & 3 & 1 & YES & YES & NO(2) & $2.00$ & $(4,3)$ & -- & 3633\\
$(467,181)$ & 13 & $(5,2)$ & 3 & 1 & YES & YES & NO(2) & $2.00$ & $(4,3)$ & -- & 3634\\
$(467,181)$ & 13 & $(49,19)$ & 8 & 1 & YES & YES & NO(2) & $2.00$ & $(2,4)$ & NO & 3635\\
$(467,196)$ & 13 & $(193,81)$ & 11 & 1 & YES & YES & YES & $2.00$ & $(2,4)$ & NO & 3636\\
$(467,193)$ & 13 & $(271,112)$ & 12 & 1 & YES & YES & NO(2) & $2.18$ & $(2,4)$ & NO & 3637\\
$(474,131)$ & 13 & $(7,3)$ & 4 & 1 & YES & YES & YES & $2.00$ & $(2,4)$ & -- & 3638\\
$(474,131)$ & 13 & $(32,9)$ & 8 & 2 & YES & YES & YES & $2.00$ & $(2,4)$ & NO & 3639\\
$(477,131)$ & 14 & $(5,2)$ & 3 & 1 & YES & YES & YES & $2.00$ & $(4,3)$ & -- & 3640\\
$(481,140)$ & 14 & $(7,2)$ & 4 & 1 & YES & YES & NO(2) & $1.88$ & $(6,2)$ & -- & 3641\\
$(484,89)$ & 16 & $(484,89)$ & 16 & 484 & YES & YES & NO(3) & $1.83$ & $(2,4)$ & NO & 3642\\
$(485,188)$ & 13 & $(4,1)$ & 3 & 1 & YES & YES & YES & $2.00$ & $(2,4)$ & NO & 3643\\
$(485,188)$ & 13 & $(485,188)$ & 13 & 485 & YES & YES & NO(2) & $1.89$ & $(4,3)$ & NO & 3644\\
$(487,186)$ & 13 & $(13,5)$ & 5 & 1 & YES & YES & YES & $2.00$ & $(2,4)$ & NO & 3645\\
$(487,136)$ & 14 & $(29,8)$ & 7 & 1 & YES & YES & YES & $2.00$ & $(2,4)$ & NO & 3646\\
$(490,207)$ & 13 & $(3,1)$ & 2 & 1 & YES & YES & NO(2) & $2.27$ & $(2,4)$ & -- & 3647\\
$(490,207)$ & 13 & $(4,1)$ & 3 & 2 & YES & YES & NO(2) & $2.27$ & $(2,4)$ & -- & 3648\\
$(493,207)$ & 13 & $(5,2)$ & 3 & 1 & YES & YES & YES & $2.00$ & $(6,2)$ & -- & 3649\\
$(495,137)$ & 14 & $(5,2)$ & 3 & 5 & YES & YES & YES & $2.00$ & $(2,4)$ & -- & 3650\\
$(499,139)$ & 14 & $(5,2)$ & 3 & 1 & YES & YES & YES & $2.00$ & $(2,4)$ & NO & 3651\\
$(505,212)$ & 13 & $(26,11)$ & 7 & 1 & YES & YES & YES & $2.12$ & $(6,2)$ & NO & 3652\\
$(507,196)$ & 13 & $(5,1)$ & 4 & 1 & YES & YES & YES & $2.00$ & $(2,4)$ & NO & 3653\\
$(507,196)$ & 13 & $(5,1)$ & 4 & 1 & YES & YES & YES & $2.00$ & $(2,4)$ & -- & 3654\\
$(513,215)$ & 14 & $(4,1)$ & 3 & 1 & YES & YES & NO(2) & $1.88$ & $(4,3)$ & NO & 3655\\
$(513,155)$ & 15 & $(43,13)$ & 9 & 1 & YES & YES & YES & $1.83$ & $(4,3)$ & NO & 3656\\
$(513,215)$ & 14 & $(43,18)$ & 8 & 1 & YES & YES & NO(2) & $2.00$ & $(4,3)$ & NO & 3657\\
$(513,212)$ & 13 & $(121,50)$ & 10 & 1 & YES & YES & NO(2) & $2.36$ & $(2,4)$ & 3623 & 3658\\
$(517,144)$ & 14 & $(140,39)$ & 11 & 1 & YES & YES & YES & $2.00$ & $(2,4)$ & NO & 3659\\
$(519,140)$ & 14 & $(241,65)$ & 12 & 1 & YES & YES & YES & $2.00$ & $(4,3)$ & NO & 3660\\
$(522,119)$ & 14 & $(5,2)$ & 3 & 1 & YES & YES & NO(2) & $1.89$ & $(4,3)$ & NO & 3661\\
$(522,119)$ & 14 & $(5,2)$ & 3 & 1 & YES & YES & NO(2) & $2.00$ & $(4,3)$ & -- & 3662\\
$(536,207)$ & 14 & $(158,61)$ & 11 & 2 & YES & YES & YES & $2.14$ & $(2,4)$ & 3795 & 3663\\
$(548,225)$ & 14 & $(4,1)$ & 3 & 4 & YES & YES & YES & $2.11$ & $(2,4)$ & NO & 3664\\
$(551,161)$ & 14 & $(2,1)$ & 1 & 1 & YES & YES & NO(2) & $1.89$ & $(4,3)$ & -- & 3665\\
$(559,157)$ & 14 & $(2,1)$ & 1 & 1 & YES & YES & YES & $1.88$ & $(2,4)$ & -- & 3666\\
$(559,165)$ & 14 & $(2,1)$ & 1 & 1 & YES & YES & NO(2) & $1.89$ & $(4,3)$ & -- & 3667\\
$(559,214)$ & 14 & $(5,2)$ & 3 & 1 & YES & YES & NO(2) & $2.00$ & $(4,3)$ & NO & 3668\\
$(559,166)$ & 14 & $(27,8)$ & 7 & 1 & YES & YES & NO(2) & $1.75$ & $(6,2)$ & 3581 & 3669\\
$(559,157)$ & 14 & $(57,16)$ & 9 & 1 & YES & YES & YES & $1.88$ & $(2,4)$ & 3573 & 3670\\
$(565,219)$ & 14 & $(4,1)$ & 3 & 1 & YES & YES & YES & $2.14$ & $(2,4)$ & NO & 3671\\
$(565,128)$ & 15 & $(35,8)$ & 8 & 5 & YES & YES & YES & $2.00$ & $(4,3)$ & NO & 3672\\
$(577,239)$ & 14 & $(2,1)$ & 1 & 1 & YES & YES & YES & $1.83$ & $(4,3)$ & -- & 3673\\
$(577,169)$ & 14 & $(5,2)$ & 3 & 1 & YES & YES & YES & $2.00$ & $(4,3)$ & NO & 3674\\
$(577,213)$ & 14 & $(5,1)$ & 4 & 1 & YES & YES & YES & $2.00$ & $(4,3)$ & NO & 3675\\
$(577,239)$ & 14 & $(12,5)$ & 5 & 1 & YES & YES & YES & $1.83$ & $(4,3)$ & NO & 3676\\
$(577,213)$ & 14 & $(214,79)$ & 12 & 1 & YES & YES & YES & $2.00$ & $(4,3)$ & NO & 3677\\
$(579,239)$ & 14 & $(3,1)$ & 2 & 3 & YES & YES & NO(2) & $2.12$ & $(4,3)$ & NO & 3678\\
$(579,221)$ & 14 & $(186,71)$ & 11 & 3 & YES & YES & YES & $2.14$ & $(2,4)$ & NO & 3679\\
$(582,215)$ & 14 & $(11,4)$ & 5 & 1 & YES & YES & YES & $2.00$ & $(4,3)$ & NO & 3680\\
$(582,215)$ & 14 & $(19,7)$ & 6 & 1 & YES & YES & YES & $2.00$ & $(4,3)$ & NO & 3681\\
$(582,223)$ & 15 & $(34,13)$ & 7 & 2 & YES & YES & YES & $2.00$ & $(4,3)$ & NO & 3682\\
$(583,246)$ & 14 & $(2,1)$ & 1 & 1 & YES & YES & NO(2) & $2.00$ & $(2,4)$ & -- & 3683\\
$(592,173)$ & 14 & $(10,3)$ & 5 & 2 & YES & YES & YES & $2.00$ & $(2,4)$ & NO & 3684\\
$(592,173)$ & 14 & $(41,12)$ & 8 & 1 & YES & YES & YES & $2.00$ & $(2,4)$ & NO & 3685\\
$(592,175)$ & 14 & $(433,128)$ & 13 & 1 & YES & YES & YES & $2.00$ & $(2,4)$ & NO & 3686\\
$(595,227)$ & 14 & $(4,1)$ & 3 & 1 & YES & YES & YES & $2.00$ & $(2,4)$ & NO & 3687\\
$(597,250)$ & 14 & $(437,183)$ & 13 & 1 & YES & YES & YES & $2.14$ & $(6,2)$ & NO & 3688\\
$(599,165)$ & 14 & $(18,5)$ & 6 & 1 & YES & YES & NO(2) & $1.89$ & $(4,3)$ & NO & 3689\\
$(601,137)$ & 15 & $(31,7)$ & 8 & 1 & YES & YES & YES & $2.00$ & $(4,3)$ & NO & 3690\\
$(613,237)$ & 14 & $(5,1)$ & 4 & 1 & YES & YES & YES & $2.00$ & $(4,3)$ & NO & 3691\\
$(613,234)$ & 14 & $(131,50)$ & 10 & 1 & YES & YES & YES & $2.00$ & $(2,4)$ & NO & 3692\\
$(613,234)$ & 14 & $(613,234)$ & 14 & 613 & YES & YES & YES & $2.00$ & $(2,4)$ & NO & 3693\\
$(617,182)$ & 15 & $(617,182)$ & 15 & 617 & YES & YES & YES & $2.12$ & $(2,4)$ & NO & 3694\\
$(625,258)$ & 14 & $(2,1)$ & 1 & 1 & YES & YES & YES & $2.00$ & $(2,4)$ & -- & 3695\\
$(626,263)$ & 14 & $(69,29)$ & 9 & 1 & YES & YES & YES & $2.10$ & $(2,4)$ & NO & 3696\\
$(631,231)$ & 15 & $(4,1)$ & 3 & 1 & YES & YES & YES & $2.17$ & $(4,3)$ & -- & 3697\\
$(631,234)$ & 14 & $(89,33)$ & 10 & 1 & YES & YES & NO(2) & $2.00$ & $(4,3)$ & NO & 3698\\
$(632,137)$ & 15 & $(19,4)$ & 7 & 1 & YES & YES & YES & $2.00$ & $(4,3)$ & NO & 3699\\
$(633,266)$ & 14 & $(257,108)$ & 12 & 1 & YES & YES & YES & $2.12$ & $(6,2)$ & 3733 & 3700\\
$(633,266)$ & 14 & $(445,187)$ & 13 & 1 & YES & YES & YES & $2.25$ & $(6,2)$ & NO & 3701\\
$(640,243)$ & 14 & $(5,2)$ & 3 & 5 & YES & YES & YES & $1.83$ & $(4,3)$ & NO & 3702\\
$(641,146)$ & 15 & $(9,2)$ & 5 & 1 & YES & YES & YES & $2.14$ & $(6,2)$ & -- & 3703\\
$(642,265)$ & 14 & $(4,1)$ & 3 & 2 & YES & YES & YES & $1.86$ & $(6,2)$ & -- & 3704\\
$(642,265)$ & 14 & $(642,265)$ & 14 & 642 & YES & YES & YES & $1.86$ & $(6,2)$ & NO & 3705\\
$(647,246)$ & 14 & $(2,1)$ & 1 & 1 & YES & YES & YES & $1.83$ & $(4,3)$ & -- & 3706\\
$(647,271)$ & 14 & $(2,1)$ & 1 & 1 & YES & YES & YES & $1.83$ & $(4,3)$ & -- & 3707\\
$(649,240)$ & 14 & $(2,1)$ & 1 & 1 & YES & YES & YES & $1.83$ & $(4,3)$ & -- & 3708\\
$(650,283)$ & 15 & $(3,1)$ & 2 & 1 & YES & YES & YES & $2.00$ & $(4,3)$ & -- & 3709\\
$(653,253)$ & 14 & $(3,1)$ & 2 & 1 & YES & YES & YES & $2.14$ & $(2,4)$ & -- & 3710\\
$(653,250)$ & 14 & $(6,1)$ & 5 & 1 & YES & YES & NO(2) & $1.75$ & $(6,2)$ & NO & 3711\\
$(659,184)$ & 15 & $(25,7)$ & 7 & 1 & YES & YES & YES & $2.11$ & $(2,4)$ & NO & 3712\\
$(663,196)$ & 14 & $(389,115)$ & 13 & 1 & YES & YES & YES & $2.00$ & $(2,4)$ & NO & 3713\\
$(664,185)$ & 15 & $(5,2)$ & 3 & 1 & YES & YES & YES & $2.00$ & $(4,3)$ & -- & 3714\\
$(665,258)$ & 14 & $(3,1)$ & 2 & 1 & YES & YES & NO(2) & $1.75$ & $(6,2)$ & -- & 3715\\
$(665,258)$ & 14 & $(67,26)$ & 9 & 1 & YES & YES & NO(2) & $2.00$ & $(4,3)$ & NO & 3716\\
$(665,258)$ & 14 & $(116,45)$ & 10 & 1 & YES & YES & NO(2) & $1.75$ & $(6,2)$ & NO & 3717\\
$(674,283)$ & 14 & $(2,1)$ & 1 & 2 & YES & YES & YES & $2.00$ & $(2,4)$ & -- & 3718\\
$(674,283)$ & 14 & $(131,55)$ & 10 & 1 & YES & YES & YES & $2.00$ & $(2,4)$ & NO & 3719\\
$(683,287)$ & 14 & $(3,1)$ & 2 & 1 & YES & YES & YES & $2.00$ & $(2,4)$ & NO & 3720\\
$(691,254)$ & 14 & $(3,1)$ & 2 & 1 & YES & YES & YES & $2.25$ & $(6,2)$ & -- & 3721\\
$(691,264)$ & 14 & $(301,115)$ & 12 & 1 & YES & YES & YES & $2.00$ & $(2,4)$ & NO & 3722\\
$(691,254)$ & 14 & $(691,254)$ & 14 & 691 & YES & YES & YES & $2.38$ & $(6,2)$ & NO & 3723\\
$(694,305)$ & 15 & $(3,1)$ & 2 & 1 & YES & YES & YES & $2.17$ & $(4,3)$ & -- & 3724\\
$(697,266)$ & 14 & $(34,13)$ & 7 & 17 & YES & YES & YES & $2.00$ & $(2,4)$ & NO & 3725\\
$(698,265)$ & 14 & $(3,1)$ & 2 & 1 & YES & YES & YES & $2.25$ & $(4,3)$ & -- & 3726\\
$(698,265)$ & 14 & $(13,5)$ & 5 & 1 & YES & YES & YES & $2.12$ & $(4,3)$ & NO & 3727\\
$(698,295)$ & 14 & $(265,112)$ & 12 & 1 & YES & YES & NO(2) & $1.88$ & $(6,2)$ & NO & 3728\\
$(701,204)$ & 15 & $(2,1)$ & 1 & 1 & YES & YES & YES & $2.00$ & $(4,3)$ & NO & 3729\\
$(701,207)$ & 15 & $(4,1)$ & 3 & 1 & YES & YES & YES & $2.12$ & $(6,2)$ & -- & 3730\\
$(701,207)$ & 15 & $(403,119)$ & 13 & 1 & YES & YES & YES & $2.12$ & $(6,2)$ & 3813 & 3731\\
$(702,295)$ & 14 & $(2,1)$ & 1 & 2 & YES & YES & YES & $2.25$ & $(6,2)$ & -- & 3732\\
$(702,295)$ & 14 & $(188,79)$ & 11 & 2 & YES & YES & YES & $2.12$ & $(6,2)$ & 3700 & 3733\\
$(702,295)$ & 14 & $(702,295)$ & 14 & 702 & YES & YES & YES & $1.86$ & $(6,2)$ & NO & 3734\\
$(703,267)$ & 14 & $(13,5)$ & 5 & 1 & YES & YES & YES & $2.00$ & $(2,4)$ & NO & 3735\\
$(707,274)$ & 14 & $(129,50)$ & 10 & 1 & YES & YES & YES & $2.14$ & $(2,4)$ & NO & 3736\\
$(709,293)$ & 14 & $(2,1)$ & 1 & 1 & YES & YES & YES & $2.00$ & $(2,4)$ & -- & 3737\\
$(714,299)$ & 14 & $(3,1)$ & 2 & 3 & YES & YES & YES & $2.00$ & $(6,2)$ & -- & 3738\\
$(714,299)$ & 14 & $(437,183)$ & 13 & 1 & YES & YES & YES & $2.25$ & $(6,2)$ & NO & 3739\\
$(717,212)$ & 14 & $(3,1)$ & 2 & 3 & YES & YES & YES & $2.00$ & $(2,4)$ & NO & 3740\\
$(717,212)$ & 14 & $(3,1)$ & 2 & 3 & YES & YES & YES & $2.00$ & $(2,4)$ & -- & 3741\\
$(718,213)$ & 15 & $(91,27)$ & 10 & 1 & YES & YES & NO(2) & $2.00$ & $(4,3)$ & NO & 3742\\
$(729,212)$ & 15 & $(4,1)$ & 3 & 1 & YES & YES & YES & $2.00$ & $(6,2)$ & -- & 3743\\
$(729,212)$ & 15 & $(533,155)$ & 14 & 1 & YES & YES & YES & $2.00$ & $(6,2)$ & NO & 3744\\
$(734,281)$ & 14 & $(5,1)$ & 4 & 1 & YES & YES & YES & $1.86$ & $(4,3)$ & -- & 3745\\
$(734,303)$ & 14 & $(5,1)$ & 4 & 1 & YES & YES & YES & $2.00$ & $(2,4)$ & -- & 3746\\
$(741,283)$ & 14 & $(4,1)$ & 3 & 1 & YES & YES & YES & $2.14$ & $(2,4)$ & -- & 3747\\
$(752,287)$ & 14 & $(3,1)$ & 2 & 1 & YES & YES & YES & $2.14$ & $(2,4)$ & -- & 3748\\
$(752,219)$ & 15 & $(4,1)$ & 3 & 4 & YES & YES & YES & $1.86$ & $(4,3)$ & NO & 3749\\
$(752,287)$ & 14 & $(131,50)$ & 10 & 1 & YES & YES & YES & $2.14$ & $(2,4)$ & NO & 3750\\
$(753,286)$ & 14 & $(2,1)$ & 1 & 1 & YES & YES & YES & $2.00$ & $(2,4)$ & -- & 3751\\
$(753,328)$ & 15 & $(62,27)$ & 9 & 1 & YES & YES & YES & $2.17$ & $(4,3)$ & NO & 3752\\
$(753,220)$ & 15 & $(332,97)$ & 13 & 1 & YES & YES & NO(2) & $1.88$ & $(6,2)$ & NO & 3753\\
$(755,229)$ & 15 & $(5,1)$ & 4 & 5 & YES & YES & NO(2) & $1.86$ & $(6,2)$ & -- & 3754\\
$(755,292)$ & 14 & $(44,17)$ & 8 & 1 & YES & YES & YES & $2.00$ & $(2,4)$ & 3782 & 3755\\
$(755,229)$ & 15 & $(755,229)$ & 15 & 755 & YES & YES & NO(2) & $2.00$ & $(4,3)$ & NO & 3756\\
$(761,223)$ & 15 & $(3,1)$ & 2 & 1 & YES & YES & YES & $2.00$ & $(6,2)$ & -- & 3757\\
$(761,223)$ & 15 & $(157,46)$ & 11 & 1 & YES & YES & YES & $2.12$ & $(6,2)$ & 3815 & 3758\\
$(761,226)$ & 15 & $(431,128)$ & 13 & 1 & YES & YES & YES & $2.00$ & $(2,4)$ & 3824 & 3759\\
$(767,322)$ & 14 & $(5,2)$ & 3 & 1 & YES & YES & YES & $2.12$ & $(2,4)$ & NO & 3760\\
$(767,223)$ & 15 & $(141,41)$ & 11 & 1 & YES & YES & YES & $2.12$ & $(2,4)$ & NO & 3761\\
$(775,143)$ & 16 & $(2,1)$ & 1 & 1 & YES & YES & YES & $1.88$ & $(2,4)$ & -- & 3762\\
$(775,143)$ & 16 & $(2,1)$ & 1 & 1 & YES & YES & YES & $2.00$ & $(2,4)$ & NO & 3763\\
$(777,214)$ & 15 & $(2,1)$ & 1 & 1 & YES & YES & NO(2) & $2.00$ & $(4,3)$ & -- & 3764\\
$(777,295)$ & 14 & $(4,1)$ & 3 & 1 & YES & YES & NO(2) & $1.86$ & $(6,2)$ & -- & 3765\\
$(777,295)$ & 14 & $(295,112)$ & 12 & 1 & YES & YES & NO(2) & $1.75$ & $(6,2)$ & NO & 3766\\
$(780,227)$ & 15 & $(2,1)$ & 1 & 2 & YES & YES & YES & $2.11$ & $(2,4)$ & -- & 3767\\
$(781,215)$ & 15 & $(29,8)$ & 7 & 1 & YES & YES & YES & $2.00$ & $(2,4)$ & NO & 3768\\
$(784,229)$ & 15 & $(4,1)$ & 3 & 4 & YES & YES & YES & $2.00$ & $(4,3)$ & NO & 3769\\
$(788,301)$ & 14 & $(2,1)$ & 1 & 2 & YES & YES & YES & $2.14$ & $(2,4)$ & -- & 3770\\
$(788,291)$ & 15 & $(5,2)$ & 3 & 1 & YES & YES & YES & $2.00$ & $(4,3)$ & NO & 3771\\
$(788,301)$ & 14 & $(8,3)$ & 4 & 4 & YES & YES & YES & $2.00$ & $(2,4)$ & NO & 3772\\
$(790,217)$ & 15 & $(2,1)$ & 1 & 2 & YES & YES & YES & $2.00$ & $(4,3)$ & -- & 3773\\
$(790,217)$ & 15 & $(3,1)$ & 2 & 1 & YES & YES & YES & $1.83$ & $(4,3)$ & NO & 3774\\
$(793,242)$ & 15 & $(3,1)$ & 2 & 1 & YES & YES & YES & $2.00$ & $(2,4)$ & -- & 3775\\
$(797,219)$ & 15 & $(3,1)$ & 2 & 1 & YES & YES & NO(2) & $1.88$ & $(6,2)$ & -- & 3776\\
$(797,219)$ & 15 & $(131,36)$ & 11 & 1 & YES & YES & NO(2) & $2.00$ & $(4,3)$ & NO & 3777\\
$(802,225)$ & 15 & $(2,1)$ & 1 & 2 & YES & YES & YES & $2.11$ & $(2,4)$ & -- & 3778\\
$(802,337)$ & 14 & $(2,1)$ & 1 & 2 & YES & YES & YES & $2.14$ & $(2,4)$ & -- & 3779\\
$(803,305)$ & 14 & $(5,1)$ & 4 & 1 & YES & YES & YES & $2.00$ & $(2,4)$ & -- & 3780\\
$(808,185)$ & 15 & $(2,1)$ & 1 & 2 & YES & YES & NO(2) & $1.78$ & $(4,3)$ & -- & 3781\\
$(820,317)$ & 14 & $(31,12)$ & 7 & 1 & YES & YES & YES & $2.00$ & $(2,4)$ & 3755 & 3782\\
$(820,317)$ & 14 & $(44,17)$ & 8 & 4 & YES & YES & NO(2) & $1.89$ & $(4,3)$ & NO & 3783\\
$(822,239)$ & 15 & $(2,1)$ & 1 & 2 & YES & YES & YES & $2.12$ & $(2,4)$ & -- & 3784\\
$(822,239)$ & 15 & $(86,25)$ & 10 & 2 & YES & YES & YES & $2.12$ & $(2,4)$ & NO & 3785\\
$(830,253)$ & 16 & $(10,3)$ & 5 & 10 & YES & YES & YES & $2.17$ & $(4,3)$ & NO & 3786\\
$(833,246)$ & 15 & $(2,1)$ & 1 & 1 & YES & YES & YES & $2.00$ & $(2,4)$ & -- & 3787\\
$(833,253)$ & 15 & $(56,17)$ & 9 & 7 & YES & YES & YES & $2.00$ & $(4,3)$ & NO & 3788\\
$(833,246)$ & 15 & $(342,101)$ & 13 & 1 & YES & YES & YES & $2.00$ & $(2,4)$ & NO & 3789\\
$(852,229)$ & 15 & $(346,93)$ & 13 & 2 & YES & YES & YES & $2.12$ & $(6,2)$ & 3812 & 3790\\
$(860,263)$ & 15 & $(3,1)$ & 2 & 1 & YES & YES & YES & $2.12$ & $(6,2)$ & -- & 3791\\
$(863,256)$ & 15 & $(5,1)$ & 4 & 1 & YES & YES & YES & $2.00$ & $(2,4)$ & -- & 3792\\
$(877,266)$ & 15 & $(2,1)$ & 1 & 1 & YES & YES & NO(2) & $2.00$ & $(6,2)$ & -- & 3793\\
$(878,339)$ & 15 & $(5,2)$ & 3 & 1 & YES & YES & YES & $2.29$ & $(2,4)$ & NO & 3794\\
$(878,339)$ & 15 & $(44,17)$ & 8 & 2 & YES & YES & YES & $2.14$ & $(2,4)$ & 3663 & 3795\\
$(882,337)$ & 14 & $(5,2)$ & 3 & 1 & YES & YES & YES & $2.14$ & $(2,4)$ & NO & 3796\\
$(889,246)$ & 15 & $(2,1)$ & 1 & 1 & YES & YES & YES & $2.00$ & $(2,4)$ & -- & 3797\\
$(893,246)$ & 15 & $(5,2)$ & 3 & 1 & YES & YES & YES & $2.17$ & $(8,1)$ & -- & 3798\\
$(893,246)$ & 15 & $(236,65)$ & 12 & 1 & YES & YES & YES & $2.17$ & $(8,1)$ & NO & 3799\\
$(903,274)$ & 15 & $(56,17)$ & 9 & 7 & YES & YES & YES & $2.11$ & $(2,4)$ & NO & 3800\\
$(907,264)$ & 15 & $(2,1)$ & 1 & 1 & YES & YES & YES & $2.00$ & $(2,4)$ & NO & 3801\\
$(913,207)$ & 16 & $(13,3)$ & 6 & 1 & YES & YES & YES & $2.11$ & $(2,4)$ & NO & 3802\\
$(915,338)$ & 15 & $(3,1)$ & 2 & 3 & YES & YES & YES & $2.14$ & $(4,3)$ & -- & 3803\\
$(920,273)$ & 15 & $(64,19)$ & 9 & 8 & YES & YES & NO(2) & $1.89$ & $(4,3)$ & NO & 3804\\
$(928,353)$ & 15 & $(5,2)$ & 3 & 1 & YES & YES & YES & $2.14$ & $(4,3)$ & NO & 3805\\
$(932,283)$ & 16 & $(79,24)$ & 10 & 1 & YES & YES & YES & $2.29$ & $(2,4)$ & NO & 3806\\
$(935,259)$ & 15 & $(11,3)$ & 5 & 11 & YES & YES & YES & $2.00$ & $(2,4)$ & NO & 3807\\
$(943,215)$ & 16 & $(2,1)$ & 1 & 1 & YES & YES & NO(2) & $2.00$ & $(4,3)$ & NO & 3808\\
$(943,215)$ & 16 & $(943,215)$ & 16 & 943 & YES & YES & NO(2) & $1.88$ & $(6,2)$ & NO & 3809\\
$(944,261)$ & 15 & $(29,8)$ & 7 & 1 & YES & YES & YES & $2.00$ & $(2,4)$ & NO & 3810\\
$(945,254)$ & 15 & $(4,1)$ & 3 & 1 & YES & YES & YES & $2.00$ & $(6,2)$ & -- & 3811\\
$(945,254)$ & 15 & $(253,68)$ & 12 & 1 & YES & YES & YES & $2.12$ & $(6,2)$ & 3790 & 3812\\
$(955,282)$ & 15 & $(149,44)$ & 11 & 1 & YES & YES & YES & $2.12$ & $(6,2)$ & 3731 & 3813\\
$(957,284)$ & 15 & $(10,3)$ & 5 & 1 & YES & YES & YES & $2.00$ & $(4,3)$ & NO & 3814\\
$(959,281)$ & 15 & $(58,17)$ & 9 & 1 & YES & YES & YES & $2.12$ & $(6,2)$ & 3758 & 3815\\
$(965,282)$ & 15 & $(7,2)$ & 4 & 1 & YES & YES & YES & $2.12$ & $(6,2)$ & NO & 3816\\
$(985,407)$ & 15 & $(2,1)$ & 1 & 1 & YES & YES & YES & $2.00$ & $(4,3)$ & -- & 3817\\
$(987,292)$ & 15 & $(17,5)$ & 6 & 1 & YES & YES & YES & $2.00$ & $(2,4)$ & NO & 3818\\
$(992,277)$ & 15 & $(11,3)$ & 5 & 1 & YES & YES & YES & $2.00$ & $(2,4)$ & NO & 3819\\
$(997,295)$ & 15 & $(4,1)$ & 3 & 1 & YES & YES & YES & $2.14$ & $(2,4)$ & NO & 3820\\
$(997,295)$ & 15 & $(365,108)$ & 13 & 1 & YES & YES & YES & $2.14$ & $(2,4)$ & NO & 3821\\
$(1024,283)$ & 15 & $(7,2)$ & 4 & 1 & YES & YES & YES & $2.00$ & $(2,4)$ & NO & 3822\\
$(1025,303)$ & 15 & $(2,1)$ & 1 & 1 & YES & YES & YES & $2.14$ & $(2,4)$ & -- & 3823\\
$(1027,305)$ & 15 & $(165,49)$ & 11 & 1 & YES & YES & YES & $2.00$ & $(2,4)$ & 3759 & 3824\\
$(1042,403)$ & 15 & $(5,2)$ & 3 & 1 & YES & YES & YES & $2.14$ & $(4,3)$ & NO & 3825\\
$(1055,242)$ & 16 & $(4,1)$ & 3 & 1 & YES & YES & YES & $2.12$ & $(6,2)$ & -- & 3826\\
$(1055,242)$ & 16 & $(22,5)$ & 7 & 1 & YES & YES & YES & $2.00$ & $(6,2)$ & NO & 3827\\
$(1096,303)$ & 15 & $(7,2)$ & 4 & 1 & YES & YES & YES & $2.00$ & $(2,4)$ & NO & 3828\\
$(1117,432)$ & 15 & $(287,111)$ & 12 & 1 & YES & YES & YES & $2.00$ & $(8,1)$ & NO & 3829\\
$(1149,206)$ & 17 & $(3,1)$ & 2 & 3 & YES & YES & YES & $2.12$ & $(6,2)$ & NO & 3830\\
$(1149,206)$ & 17 & $(4,1)$ & 3 & 1 & YES & YES & YES & $2.25$ & $(6,2)$ & NO & 3831\\
$(1420,393)$ & 16 & $(271,75)$ & 12 & 1 & YES & YES & YES & $2.00$ & $(8,1)$ & NO & 3832\\
$(a;0,0,0;3)$ & 4 & $(290,81)$ & 12 & 1 & YES & YES & YES & $2.00$ & $(2,4)$ & -- & 3833\\
$(a;1,0,0;13)$ & 5 & $(140,41)$ & 11 & 1 & YES & YES & NO(2) & $2.00$ & $(4,3)$ & -- & 3834\\
$(b;0,0,0;14)$ & 5 & $(112,47)$ & 10 & 14 & YES & YES & YES & $2.00$ & $(4,3)$ & -- & 3835\\
$(b;0,0,0;14)$ & 5 & $(123,47)$ & 10 & 1 & YES & YES & YES & $2.14$ & $(2,4)$ & -- & 3836\\
$(b;0,0,0;14)$ & 5 & $(124,23)$ & 12 & 2 & YES & YES & YES & $1.88$ & $(2,4)$ & -- & 3837\\
$(b;0,0,0;14)$ & 5 & $(145,56)$ & 11 & 1 & YES & YES & YES & $2.29$ & $(2,4)$ & -- & 3838\\
$(b;0,0,1;4)$ & 6 & $(65,19)$ & 9 & 1 & YES & YES & YES & $1.89$ & $(2,4)$ & -- & 3839\\
$(b;0,0,1;4)$ & 6 & $(105,31)$ & 10 & 1 & YES & YES & YES & $2.00$ & $(2,4)$ & -- & 3840\\
$(b;0,0,1;4)$ & 6 & $(140,41)$ & 11 & 4 & YES & YES & YES & $2.00$ & $(8,1)$ & -- & 3841\\
$(b;0,0,2;26)$ & 7 & $(40,11)$ & 8 & 2 & YES & YES & YES & $1.86$ & $(4,3)$ & -- & 3842\\
$(b;0,0,2;26)$ & 7 & $(79,24)$ & 10 & 1 & YES & YES & YES & $2.14$ & $(2,4)$ & -- & 3843\\
$(b;0,1,0;19)$ & 6 & $(95,29)$ & 10 & 19 & YES & YES & YES & $2.25$ & $(6,2)$ & -- & 3844\\
$(b;0,1,0;19)$ & 6 & $(98,29)$ & 10 & 1 & YES & YES & NO(2) & $2.00$ & $(4,3)$ & -- & 3845\\
$(b;0,1,1;27)$ & 7 & $(41,17)$ & 8 & 1 & YES & YES & YES & $1.83$ & $(4,3)$ & -- & 3846\\
$(b;0,1,1;27)$ & 7 & $(56,13)$ & 10 & 1 & YES & YES & YES & $1.88$ & $(4,3)$ & -- & 3847\\
$(b;0,1,1;27)$ & 7 & $(59,18)$ & 9 & 1 & YES & YES & YES & $2.38$ & $(6,2)$ & -- & 3848\\
$(b;1,0,1;29)$ & 7 & $(41,17)$ & 8 & 1 & YES & YES & YES & $2.00$ & $(2,4)$ & -- & 3849\\
$(b;1,1,0;27)$ & 7 & $(64,19)$ & 9 & 1 & YES & YES & YES & $2.25$ & $(6,2)$ & -- & 3850\\
$(b;2,0,1;38)$ & 8 & $(17,7)$ & 6 & 1 & YES & YES & YES & $1.83$ & $(4,3)$ & -- & 3851\\
$(c;0,0,0;4)$ & 4 & $(167,69)$ & 11 & 1 & YES & YES & YES & $2.00$ & $(2,4)$ & -- & 3852\\
$(c;0,0,0;4)$ & 4 & $(256,99)$ & 12 & 4 & YES & YES & NO(2) & $2.00$ & $(4,3)$ & -- & 3853\\
$(c;0,1,0;11)$ & 5 & $(116,49)$ & 10 & 1 & YES & YES & NO(2) & $2.27$ & $(2,4)$ & -- & 3854\\
$(c;0,1,0;11)$ & 5 & $(140,41)$ & 11 & 1 & YES & YES & YES & $2.00$ & $(4,3)$ & -- & 3855\\
$(c;0,1,0;11)$ & 5 & $(149,44)$ & 11 & 1 & YES & YES & YES & $1.83$ & $(4,3)$ & -- & 3856\\
$(c;0,1,0;11)$ & 5 & $(169,50)$ & 11 & 1 & YES & YES & NO(2) & $1.71$ & $(6,2)$ & -- & 3857\\
$(c;0,1,0;11)$ & 5 & $(169,70)$ & 11 & 1 & YES & YES & YES & $2.25$ & $(4,3)$ & -- & 3858\\
$(c;0,1,0;11)$ & 5 & $(186,71)$ & 11 & 1 & YES & YES & YES & $2.14$ & $(2,4)$ & -- & 3859\\
$(c;0,2,0;7)$ & 6 & $(89,25)$ & 10 & 1 & YES & YES & NO(3) & $1.83$ & $(2,4)$ & -- & 3860\\
$(c;0,2,0;7)$ & 6 & $(124,47)$ & 11 & 1 & YES & YES & YES & $2.17$ & $(4,3)$ & -- & 3861\\
$(c;0,2,0;7)$ & 6 & $(154,45)$ & 11 & 7 & YES & YES & YES & $2.14$ & $(6,2)$ & -- & 3862\\
$(c;0,2,1;19)$ & 7 & $(41,18)$ & 8 & 1 & YES & YES & NO(3) & $1.83$ & $(2,4)$ & -- & 3863\\
$(d;0,0,0;5)$ & 5 & $(49,20)$ & 9 & 1 & YES & YES & YES & $1.86$ & $(2,4)$ & -- & 3864\\
$(e;0,0,0;4)$ & 5 & $(89,26)$ & 10 & 1 & YES & YES & NO(2) & $1.89$ & $(4,3)$ & -- & 3865\\
$(e;0,0,0;4)$ & 5 & $(134,37)$ & 11 & 2 & YES & YES & YES & $2.14$ & $(2,4)$ & -- & 3866\\
$(e;0,1,0;5)$ & 6 & $(71,27)$ & 9 & 1 & YES & YES & YES & $2.00$ & $(4,3)$ & -- & 3867\\
$(e;1,0,0;18)$ & 6 & $(50,21)$ & 8 & 2 & YES & YES & NO(2) & $2.27$ & $(2,4)$ & -- & 3868\\
$(e;1,0,0;18)$ & 6 & $(56,23)$ & 9 & 2 & YES & YES & YES & $2.00$ & $(2,4)$ & -- & 3869\\
$(e;1,1,0;23)$ & 7 & $(61,18)$ & 9 & 1 & YES & YES & NO(2) & $1.88$ & $(4,3)$ & -- & 3870\\
$(e;2,1,0;31)$ & 8 & $(58,17)$ & 9 & 1 & YES & YES & YES & $2.00$ & $(4,3)$ & -- & 3871\\
$(f;0,0,0;6)$ & 4 & $(215,64)$ & 12 & 1 & YES & YES & YES & $2.27$ & $(2,4)$ & -- & 3872\\
$(f;0,0,0;6)$ & 4 & $(246,95)$ & 12 & 6 & YES & YES & YES & $1.86$ & $(4,3)$ & -- & 3873\\
$(g;0,0,0;19)$ & 6 & $(26,11)$ & 7 & 1 & YES & YES & NO(2) & $2.00$ & $(2,4)$ & -- & 3874\\
$(g;0,0,1;26)$ & 7 & $(41,17)$ & 8 & 1 & YES & YES & YES & $2.00$ & $(2,4)$ & -- & 3875\\
$(g;0,0,2;11)$ & 8 & $(13,5)$ & 5 & 1 & YES & YES & NO(3) & $1.83$ & $(2,4)$ & -- & 3876\\
$(g;0,0,2;11)$ & 8 & $(40,11)$ & 8 & 1 & YES & YES & YES & $2.00$ & $(2,4)$ & -- & 3877\\
$(h;0,0,0;6)$ & 5 & $(108,41)$ & 10 & 6 & YES & YES & YES & $2.00$ & $(4,3)$ & -- & 3878\\
$(h;0,0,0;6)$ & 5 & $(119,46)$ & 10 & 1 & YES & YES & YES & $2.25$ & $(6,2)$ & -- & 3879\\
$(h;0,1,0;8)$ & 6 & $(26,11)$ & 7 & 2 & YES & YES & NO(2) & $2.00$ & $(2,4)$ & -- & 3880\\
$(h;0,1,0;8)$ & 6 & $(44,17)$ & 8 & 4 & YES & YES & YES & $1.83$ & $(4,3)$ & -- & 3881\\
$(h;0,1,0;8)$ & 6 & $(69,29)$ & 9 & 1 & YES & YES & YES & $2.00$ & $(2,4)$ & -- & 3882\\
$(h;0,1,0;8)$ & 6 & $(71,27)$ & 9 & 1 & YES & YES & YES & $1.83$ & $(4,3)$ & -- & 3883\\
$(i;0,0,0;9)$ & 5 & $(166,61)$ & 11 & 1 & YES & YES & YES & $2.29$ & $(2,4)$ & -- & 3884\\
$(j;0,0,0;8)$ & 5 & $(208,79)$ & 11 & 8 & YES & YES & YES & $2.00$ & $(2,4)$ & -- & 3885
\end{longtable}
\subsection{2 chains, $K^2 = 5$}
\begin{longtable}{|c|c|c|c|c|c|c|c|c|c|c|c|}
\hline
\multicolumn{12}{|c|}{2 chains, $K^2 = 5$}\\
\hline
$(n,a)$ & Len & $(n,a)$ & Len & GCD & Nef & $\mathbb Q$-ef & Obs 0 & $\overline c_1^2 / \overline c_2$ & $(P,K)$ & WH & Index\\
\hline
\endfirsthead

\hline
$(n,a)$ & Len & $(n,a)$ & Len & GCD & Nef & $\mathbb Q$-ef & Obs 0 & $\overline c_1^2 / \overline c_2$ & $(P,K)$ & WH & Index\\
\hline
\endhead
\hline
\endfoot

$(b;0,0,0;14)$ & 5 & $(167,69)$ & 11 & 1 & YES & YES & NO(3) & $2.38$ & $(2,5)$ & -- & 3886
\end{longtable}





%%%%%%%%%%%%%%%%%%%%%%%%%%%%%%%%%%%%%%%%%%%
\section{$2I_4 + 2I_2$}


Base curves:
\begin{itemize}
  \item $L_x = x$.
  \item $L_y = y$.
  \item $L_z = z$.
  \item $A = x - z$.
  \item $B = x + y + z$.
  \item $C = x - y + z$.
  \item $Q_1 = (x+z)^2 - y(x-z)$.
  \item $L_1 = x + y - z$.
  \item $Q_2 = (x+z)^2 + y(x-z)$.
  \item $L_2 = x - y - z$.
\end{itemize}
Fibration given by pencil
\[F_\lambda = ABC + \lambda L_xL_yL_z\]

Nine exceptionals are as follows:
\begin{itemize}
  \item $E_1$ - $E_2$ at $L_x \cap L_z \cap A = [0,1,0]$.
  \item $E_3$ - $E_4$ at $L_y \cap B \cap C = [-1,0,1]$.
  \item $E_5$ at $L_y \cap A = [1,0,1]$.
  \item $E_6$ at $L_x \cap C = [0,1,1]$.
  \item $E_7$ at $L_x \cap B = [0,-1,1]$.
  \item $E_8$ at $L_z \cap C = [1,1,0]$.
  \item $E_9$ at $L_z \cap B = [-1,1,0]$.
\end{itemize}
Singular fibers are as follows:
\begin{itemize}
  \item $\lambda = \infty$: $I_4$ fiber given by $L_z$, $L_x$, $L_y$, $E_1$ in order.
  \item $\lambda = 0$: $I_4$ fiber given by $B$, $A$, $C$, $E_3$ in order.
  \item $\lambda = 4$: $I_2$ fiber given by $Q_1$, $L_1$ with nodes at $B_1 = [-i,1+i,1]$ and $T_1 = [i,1-i,1]$.
  \item $\lambda = -4$: $I_2$ fiber given by $Q_2$, $L_2$ with nodes at $B_2 = [-i,-1-i,1]$ and $T_2 = [i,-1+i,1]$.
\end{itemize}
Special curves:
\begin{itemize}
  \item $H = x+z$, a section through $[0,1,0]$ and $[-1,0,1]$.
  \item $N = (2+i)x + iz + iy$, a double section through $[0,-1,1]$ and $T_1$.
  \item $BT = x + iy + z$, a double section through $B_1, T_2$ and $[-1,0,1]$.
  \item $TB = x - iy + z$, a double section through $T_1, B_2$ and $[-1,0,1]$.
  \item $BB = x + iz$, a double section through $B_1, B_2$ and $[0,1,0]$.
  \item $BT = x - iz$, a double section through $T_1, T_2$ and $[0,1,0]$.
\end{itemize}

Input:
%\lstinputlisting[language=config]{Tests/4422.txt}
Result:
%\input{summary/4422_new}




%%%%%%%%%%%%%%%%%%%%%%%%%%%%%%%%%%%%%%%%%%%
\section{$4I_3$}

Hesse configuration. Let $\zeta$ be a primitive third root of unity. Base curves:
\begin{itemize}
  \item $L_x = x$.
  \item $L_y = y$.
  \item $L_z = z$.
  \item $L_{i,j} = X + \zeta^i Y + \zeta^j z$
\end{itemize}
Fibration given by pencil
\[F_\lambda = L_x L_y L_z + \lambda L_{0,1} L_{1,0} L_{2,2}\]

Singular fibers are as follows:
\begin{itemize}
  \item $I_3$ fiber given by $L_x$, $L_y$, $L_z$.
  \item $I_3$ fiber given by $L_{0,1}$, $L_{1,0}$, $L_{2,2}$.
  \item $I_3$ fiber given by $L_{0,2}$, $L_{1,1}$, $L_{2,0}$.
  \item $I_3$ fiber given by $L_{0,0}$, $L_{1,2}$, $L_{2,1}$.
\end{itemize}
Special curves:

Input:
%\lstinputlisting[language=config]{Tests/3333.txt}
Result:
%\input{summary/3333_new}

%%%%%%%%%%%%%%%%%%%%%%%%%%%%%%%%%%%%%%%%%%%
\section{$II^* + 2I_1$}

Base curves:
\begin{itemize}
  \item $A = z$.
  \item $F_1 = y^2z - x^3 - x^2z$.
  \item $F_2 = y^2z - x^3 - x^2z +\frac{4}{27}z^3$.
\end{itemize}
Pencil given by
\[F_\lambda = y^2z - x^3 - x^2z - \lambda z^3\]

All nine blowups are done at $[0,1,0]$.

Singular fibers are as follows:
\begin{itemize}
  \item $\lambda = \infty$: $II^*$ fiber given by $A$ and $E_1$ - $E_8$
  \item $\lambda = 0$: $I_1$ fiber given by $F_1$ with node at $[0,0,1]$.
  \item $\lambda = -4/27$: $I_1$ fiber given by $F_2$ with node at $[-2,0,3]$.
\end{itemize}

Special curves:
\begin{itemize}
  \item $R_1 = x$, double section through $[0,1,0]$ and $[0,0,1]$.
  \item $R_2 = 3x + 2z$, double section through $[0,1,0]$ and $[-2,0,3]$.
  \item $T = y$, triple section through $[0,0,1]$ and $[-2,0,3]$.
\end{itemize}
Input:
%\lstinputlisting[language=config]{Tests/IIs11.txt}
Result:
%\input{summary/IIs11_new}






%%%%%%%%%%%%%%%%%%%%%%%%%%%%%%%%%%%%%%%%%%%
\section{$I_4^* + 2I_1$}

Base curves:
\begin{itemize}
  \item $A = z$.
  \item $B = y$.
\end{itemize}
Pencil given by
\[F_\lambda = x^2y + z^3 + y^2z + \lambda yz^2\]

Nine exceptionals are as follows:
\begin{itemize}
  \item $E_1$ - $E_5$ at $A \cap B = [1,0,0]$.
  \item $E_6$ - $E_9$ at $A \cap x = [0,1,0]$.
\end{itemize}
Singular fibers are as follows:
\begin{itemize}
  \item $\lambda = 0$: $I_4^*$ fiber given by $A$, $B$ and $E_1$ - $E_4$, and $E_5$ - $E_8$.
  \item $\lambda = 2$: $I_1$ fiber called $F_1$ with node at $[0,-1,1]$.
  \item $\lambda = -2$: $I_1$ fiber called $F_2$ with node at $[0,1,1]$.
\end{itemize}

Special curves:
\begin{itemize}
  \item $H = x$, double section through $[0,1,1]$ and $[0,-1,1]$ and $[0,0,1]$.
  \item $V = y + z$, double section through $[1,0,0]$ and $[0,-1,1]$.
  \item $V = y - z$, double section through $[1,0,0]$ and $[0,1,1]$.
\end{itemize}
Input:
%\lstinputlisting[language=config]{Tests/4s11.txt}
Result:
%\input{summary/4s11_new}

%%%%%%%%%%%%%%%%%%%%%%%%%%%%%%%%%%%%%%%%%%%
\section{$III^* + I_2 + I_1$}
Input:
%\lstinputlisting[language=config]{Tests/IIIs21.txt}
Result:
%\input{summary/IIIs21_new}



%%%%%%%%%%%%%%%%%%%%%%%%%%%%%%%%%%%%%%%%%%%
\section{$IV^* + IV$}
Input:
%\lstinputlisting[language=config]{Tests/IVsIV.txt}
Result:
%\input{summary/IVsIV_new}


%%%%%%%%%%%%%%%%%%%%%%%%%%%%%%%%%%%%%%%%%%%
\section{$IV^* + I_3 + I_1$}
Input:
%\lstinputlisting[language=config]{Tests/IVs31.txt}
Result:
%\input{summary/IVs31_new}

%%%%%%%%%%%%%%%%%%%%%%%%%%%%%%%%%%%%%%%%%%%
\section{$I_2^* + 2I_2$}
Input:
%\lstinputlisting[language=config]{Tests/2s22.txt}
Result:
%\input{summary/2s22_new}

%%%%%%%%%%%%%%%%%%%%%%%%%%%%%%%%%%%%%%%%%%%
\section{$I_1^* + I_4 + I_1$}
Input:
%\lstinputlisting[language=config]{Tests/1s41.txt}
Result:
%\input{summary/1s41_new}

%%%%%%%%%%%%%%%%%%%%%%%%%%%%%%%%%%%%%%%%%%%
\section{$2I_0^*$}
Input:
%\lstinputlisting[language=config]{Tests/0s0s.txt}
Result:
%%\usepackage{longtable}



%%%%%%%%%%%%%%%%%%%%%%%%%%%%%%%%%%%%%%%%%%%
\section{Extra: $3IV$}

Dual Hesse configuration

$R_{16}$ is a section through $S_1$ and $S_6$. Same for $R_{24}$ and $R_{35}$. These three sections are concurrent.

$Q_1$, $Q_2$ and $Q_3$ are conics through 5 special points. Each pair of them share 4 of those points.


Input:
%\lstinputlisting[language=config]{Tests/IV.txt}
Result:
%\input{summary/3IV_new}
\end{document}